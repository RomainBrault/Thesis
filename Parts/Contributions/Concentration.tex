%!TEX root = ../../ThesisRomainbrault.tex

\section{Convergence with high probability of the \acpdfstring{ORFF} estimator}
\label{sec:consistency_of_the_ORFF_estimator}
We are now interested in a non-asymptotic analysis of the \ac{ORFF}
approximation of shift-invariant $\mathcal{Y}$-Mercer kernels on \acs{LCA}
group $\mathcal{X}$ endowed with the operation group $\groupop$ where
$\mathcal{X}$ is a Banach space (The more general case where $\mathcal{X}$ is
a Polish space is discussed in the appendix \cref{subsec:concentration_proof}).
For a given $D$, we study how close is the
approximation $\tilde{K}(x,z)=\tildePhi{1:D}(x)^*\tildePhi{1:D}(z)$ to the
target kernel $K(x,z)$ for any $x,z$ in $\mathcal{X}$.
\paragraph{}
If $A\in\mathcal{L}_+(\mathcal{Y})$ we denote
$\norm{A}_{\mathcal{Y},\mathcal{Y}}$ its operator norm, which amounts to the
square root of the largest eigenvalue of $A$ when $\mathcal{Y}=\mathbb{R}^p$ is
finite dimensional. For $x$ and $z$ in some non-empty compact $\mathcal{C}
\subset \mathbb{R}^d$, we consider: $F(x \groupop \inv{z})
=\tilde{K}(x,z)-K(x,z)$ and study how the uniform norm
\begin{dmath}\label{eq:norm_inf}
    \norm{\tilde{K}-K}_{\mathcal{C}\times\mathcal{C}}
    = \sup_{(x,z)\in\mathcal{C}\times\mathcal{C}}
    \norm{\tilde{K}(x,z)-K(x,z)}_{\mathcal{Y},\mathcal{Y}}
\end{dmath}
behaves according to $D$. All along this document we denote $\delta=x\groupop
z^{-1}$ for all $x$ and $z\in\mathcal{X}$. \Cref{fig:approximation_error}
empirically shows convergence of three different \acs{OVK} approximations for
$x,z$ sampled from the compact $[-1,1]^4$ and using an increasing number of
sample points $D$. The log-log plot shows that all three kernels have the same
convergence rate, up to a multiplicative factor.
\begin{pycode}[approximation]
sys.path.append('./src/')
import approximation

err = approximation.main()
\end{pycode}

\begin{figure}[!ht]
    \centering
    \pyc{print(r'\centering\resizebox{\textwidth}{!}{%% Creator: Matplotlib, PGF backend
%%
%% To include the figure in your LaTeX document, write
%%   \input{<filename>.pgf}
%%
%% Make sure the required packages are loaded in your preamble
%%   \usepackage{pgf}
%%
%% Figures using additional raster images can only be included by \input if
%% they are in the same directory as the main LaTeX file. For loading figures
%% from other directories you can use the `import` package
%%   \usepackage{import}
%% and then include the figures with
%%   \import{<path to file>}{<filename>.pgf}
%%
%% Matplotlib used the following preamble
%%   \usepackage{fontspec}
%%   \setmainfont{DejaVu Serif}
%%   \setsansfont{DejaVu Sans}
%%   \setmonofont{DejaVu Sans Mono}
%%
\begingroup%
\makeatletter%
\begin{pgfpicture}%
\pgfpathrectangle{\pgfpointorigin}{\pgfqpoint{6.223380in}{4.622826in}}%
\pgfusepath{use as bounding box, clip}%
\begin{pgfscope}%
\pgfsetbuttcap%
\pgfsetmiterjoin%
\definecolor{currentfill}{rgb}{1.000000,1.000000,1.000000}%
\pgfsetfillcolor{currentfill}%
\pgfsetlinewidth{0.000000pt}%
\definecolor{currentstroke}{rgb}{1.000000,1.000000,1.000000}%
\pgfsetstrokecolor{currentstroke}%
\pgfsetdash{}{0pt}%
\pgfpathmoveto{\pgfqpoint{0.000000in}{0.000000in}}%
\pgfpathlineto{\pgfqpoint{6.223380in}{0.000000in}}%
\pgfpathlineto{\pgfqpoint{6.223380in}{4.622826in}}%
\pgfpathlineto{\pgfqpoint{0.000000in}{4.622826in}}%
\pgfpathclose%
\pgfusepath{fill}%
\end{pgfscope}%
\begin{pgfscope}%
\pgfsetbuttcap%
\pgfsetmiterjoin%
\definecolor{currentfill}{rgb}{1.000000,1.000000,1.000000}%
\pgfsetfillcolor{currentfill}%
\pgfsetlinewidth{0.000000pt}%
\definecolor{currentstroke}{rgb}{0.000000,0.000000,0.000000}%
\pgfsetstrokecolor{currentstroke}%
\pgfsetstrokeopacity{0.000000}%
\pgfsetdash{}{0pt}%
\pgfpathmoveto{\pgfqpoint{0.710919in}{0.521603in}}%
\pgfpathlineto{\pgfqpoint{6.088380in}{0.521603in}}%
\pgfpathlineto{\pgfqpoint{6.088380in}{4.487826in}}%
\pgfpathlineto{\pgfqpoint{0.710919in}{4.487826in}}%
\pgfpathclose%
\pgfusepath{fill}%
\end{pgfscope}%
\begin{pgfscope}%
\pgfsetbuttcap%
\pgfsetroundjoin%
\definecolor{currentfill}{rgb}{0.000000,0.000000,0.000000}%
\pgfsetfillcolor{currentfill}%
\pgfsetlinewidth{0.803000pt}%
\definecolor{currentstroke}{rgb}{0.000000,0.000000,0.000000}%
\pgfsetstrokecolor{currentstroke}%
\pgfsetdash{}{0pt}%
\pgfsys@defobject{currentmarker}{\pgfqpoint{0.000000in}{-0.048611in}}{\pgfqpoint{0.000000in}{0.000000in}}{%
\pgfpathmoveto{\pgfqpoint{0.000000in}{0.000000in}}%
\pgfpathlineto{\pgfqpoint{0.000000in}{-0.048611in}}%
\pgfusepath{stroke,fill}%
}%
\begin{pgfscope}%
\pgfsys@transformshift{0.955349in}{0.521603in}%
\pgfsys@useobject{currentmarker}{}%
\end{pgfscope}%
\end{pgfscope}%
\begin{pgfscope}%
\pgftext[x=0.955349in,y=0.424381in,,top]{\sffamily\fontsize{10.000000}{12.000000}\selectfont \(\displaystyle {10^{0}}\)}%
\end{pgfscope}%
\begin{pgfscope}%
\pgfsetbuttcap%
\pgfsetroundjoin%
\definecolor{currentfill}{rgb}{0.000000,0.000000,0.000000}%
\pgfsetfillcolor{currentfill}%
\pgfsetlinewidth{0.803000pt}%
\definecolor{currentstroke}{rgb}{0.000000,0.000000,0.000000}%
\pgfsetstrokecolor{currentstroke}%
\pgfsetdash{}{0pt}%
\pgfsys@defobject{currentmarker}{\pgfqpoint{0.000000in}{-0.048611in}}{\pgfqpoint{0.000000in}{0.000000in}}{%
\pgfpathmoveto{\pgfqpoint{0.000000in}{0.000000in}}%
\pgfpathlineto{\pgfqpoint{0.000000in}{-0.048611in}}%
\pgfusepath{stroke,fill}%
}%
\begin{pgfscope}%
\pgfsys@transformshift{1.933069in}{0.521603in}%
\pgfsys@useobject{currentmarker}{}%
\end{pgfscope}%
\end{pgfscope}%
\begin{pgfscope}%
\pgftext[x=1.933069in,y=0.424381in,,top]{\sffamily\fontsize{10.000000}{12.000000}\selectfont \(\displaystyle {10^{1}}\)}%
\end{pgfscope}%
\begin{pgfscope}%
\pgfsetbuttcap%
\pgfsetroundjoin%
\definecolor{currentfill}{rgb}{0.000000,0.000000,0.000000}%
\pgfsetfillcolor{currentfill}%
\pgfsetlinewidth{0.803000pt}%
\definecolor{currentstroke}{rgb}{0.000000,0.000000,0.000000}%
\pgfsetstrokecolor{currentstroke}%
\pgfsetdash{}{0pt}%
\pgfsys@defobject{currentmarker}{\pgfqpoint{0.000000in}{-0.048611in}}{\pgfqpoint{0.000000in}{0.000000in}}{%
\pgfpathmoveto{\pgfqpoint{0.000000in}{0.000000in}}%
\pgfpathlineto{\pgfqpoint{0.000000in}{-0.048611in}}%
\pgfusepath{stroke,fill}%
}%
\begin{pgfscope}%
\pgfsys@transformshift{2.910789in}{0.521603in}%
\pgfsys@useobject{currentmarker}{}%
\end{pgfscope}%
\end{pgfscope}%
\begin{pgfscope}%
\pgftext[x=2.910789in,y=0.424381in,,top]{\sffamily\fontsize{10.000000}{12.000000}\selectfont \(\displaystyle {10^{2}}\)}%
\end{pgfscope}%
\begin{pgfscope}%
\pgfsetbuttcap%
\pgfsetroundjoin%
\definecolor{currentfill}{rgb}{0.000000,0.000000,0.000000}%
\pgfsetfillcolor{currentfill}%
\pgfsetlinewidth{0.803000pt}%
\definecolor{currentstroke}{rgb}{0.000000,0.000000,0.000000}%
\pgfsetstrokecolor{currentstroke}%
\pgfsetdash{}{0pt}%
\pgfsys@defobject{currentmarker}{\pgfqpoint{0.000000in}{-0.048611in}}{\pgfqpoint{0.000000in}{0.000000in}}{%
\pgfpathmoveto{\pgfqpoint{0.000000in}{0.000000in}}%
\pgfpathlineto{\pgfqpoint{0.000000in}{-0.048611in}}%
\pgfusepath{stroke,fill}%
}%
\begin{pgfscope}%
\pgfsys@transformshift{3.888509in}{0.521603in}%
\pgfsys@useobject{currentmarker}{}%
\end{pgfscope}%
\end{pgfscope}%
\begin{pgfscope}%
\pgftext[x=3.888509in,y=0.424381in,,top]{\sffamily\fontsize{10.000000}{12.000000}\selectfont \(\displaystyle {10^{3}}\)}%
\end{pgfscope}%
\begin{pgfscope}%
\pgfsetbuttcap%
\pgfsetroundjoin%
\definecolor{currentfill}{rgb}{0.000000,0.000000,0.000000}%
\pgfsetfillcolor{currentfill}%
\pgfsetlinewidth{0.803000pt}%
\definecolor{currentstroke}{rgb}{0.000000,0.000000,0.000000}%
\pgfsetstrokecolor{currentstroke}%
\pgfsetdash{}{0pt}%
\pgfsys@defobject{currentmarker}{\pgfqpoint{0.000000in}{-0.048611in}}{\pgfqpoint{0.000000in}{0.000000in}}{%
\pgfpathmoveto{\pgfqpoint{0.000000in}{0.000000in}}%
\pgfpathlineto{\pgfqpoint{0.000000in}{-0.048611in}}%
\pgfusepath{stroke,fill}%
}%
\begin{pgfscope}%
\pgfsys@transformshift{4.866230in}{0.521603in}%
\pgfsys@useobject{currentmarker}{}%
\end{pgfscope}%
\end{pgfscope}%
\begin{pgfscope}%
\pgftext[x=4.866230in,y=0.424381in,,top]{\sffamily\fontsize{10.000000}{12.000000}\selectfont \(\displaystyle {10^{4}}\)}%
\end{pgfscope}%
\begin{pgfscope}%
\pgfsetbuttcap%
\pgfsetroundjoin%
\definecolor{currentfill}{rgb}{0.000000,0.000000,0.000000}%
\pgfsetfillcolor{currentfill}%
\pgfsetlinewidth{0.803000pt}%
\definecolor{currentstroke}{rgb}{0.000000,0.000000,0.000000}%
\pgfsetstrokecolor{currentstroke}%
\pgfsetdash{}{0pt}%
\pgfsys@defobject{currentmarker}{\pgfqpoint{0.000000in}{-0.048611in}}{\pgfqpoint{0.000000in}{0.000000in}}{%
\pgfpathmoveto{\pgfqpoint{0.000000in}{0.000000in}}%
\pgfpathlineto{\pgfqpoint{0.000000in}{-0.048611in}}%
\pgfusepath{stroke,fill}%
}%
\begin{pgfscope}%
\pgfsys@transformshift{5.843950in}{0.521603in}%
\pgfsys@useobject{currentmarker}{}%
\end{pgfscope}%
\end{pgfscope}%
\begin{pgfscope}%
\pgftext[x=5.843950in,y=0.424381in,,top]{\sffamily\fontsize{10.000000}{12.000000}\selectfont \(\displaystyle {10^{5}}\)}%
\end{pgfscope}%
\begin{pgfscope}%
\pgfsetbuttcap%
\pgfsetroundjoin%
\definecolor{currentfill}{rgb}{0.000000,0.000000,0.000000}%
\pgfsetfillcolor{currentfill}%
\pgfsetlinewidth{0.602250pt}%
\definecolor{currentstroke}{rgb}{0.000000,0.000000,0.000000}%
\pgfsetstrokecolor{currentstroke}%
\pgfsetdash{}{0pt}%
\pgfsys@defobject{currentmarker}{\pgfqpoint{0.000000in}{-0.027778in}}{\pgfqpoint{0.000000in}{0.000000in}}{%
\pgfpathmoveto{\pgfqpoint{0.000000in}{0.000000in}}%
\pgfpathlineto{\pgfqpoint{0.000000in}{-0.027778in}}%
\pgfusepath{stroke,fill}%
}%
\begin{pgfscope}%
\pgfsys@transformshift{0.738443in}{0.521603in}%
\pgfsys@useobject{currentmarker}{}%
\end{pgfscope}%
\end{pgfscope}%
\begin{pgfscope}%
\pgfsetbuttcap%
\pgfsetroundjoin%
\definecolor{currentfill}{rgb}{0.000000,0.000000,0.000000}%
\pgfsetfillcolor{currentfill}%
\pgfsetlinewidth{0.602250pt}%
\definecolor{currentstroke}{rgb}{0.000000,0.000000,0.000000}%
\pgfsetstrokecolor{currentstroke}%
\pgfsetdash{}{0pt}%
\pgfsys@defobject{currentmarker}{\pgfqpoint{0.000000in}{-0.027778in}}{\pgfqpoint{0.000000in}{0.000000in}}{%
\pgfpathmoveto{\pgfqpoint{0.000000in}{0.000000in}}%
\pgfpathlineto{\pgfqpoint{0.000000in}{-0.027778in}}%
\pgfusepath{stroke,fill}%
}%
\begin{pgfscope}%
\pgfsys@transformshift{0.803898in}{0.521603in}%
\pgfsys@useobject{currentmarker}{}%
\end{pgfscope}%
\end{pgfscope}%
\begin{pgfscope}%
\pgfsetbuttcap%
\pgfsetroundjoin%
\definecolor{currentfill}{rgb}{0.000000,0.000000,0.000000}%
\pgfsetfillcolor{currentfill}%
\pgfsetlinewidth{0.602250pt}%
\definecolor{currentstroke}{rgb}{0.000000,0.000000,0.000000}%
\pgfsetstrokecolor{currentstroke}%
\pgfsetdash{}{0pt}%
\pgfsys@defobject{currentmarker}{\pgfqpoint{0.000000in}{-0.027778in}}{\pgfqpoint{0.000000in}{0.000000in}}{%
\pgfpathmoveto{\pgfqpoint{0.000000in}{0.000000in}}%
\pgfpathlineto{\pgfqpoint{0.000000in}{-0.027778in}}%
\pgfusepath{stroke,fill}%
}%
\begin{pgfscope}%
\pgfsys@transformshift{0.860598in}{0.521603in}%
\pgfsys@useobject{currentmarker}{}%
\end{pgfscope}%
\end{pgfscope}%
\begin{pgfscope}%
\pgfsetbuttcap%
\pgfsetroundjoin%
\definecolor{currentfill}{rgb}{0.000000,0.000000,0.000000}%
\pgfsetfillcolor{currentfill}%
\pgfsetlinewidth{0.602250pt}%
\definecolor{currentstroke}{rgb}{0.000000,0.000000,0.000000}%
\pgfsetstrokecolor{currentstroke}%
\pgfsetdash{}{0pt}%
\pgfsys@defobject{currentmarker}{\pgfqpoint{0.000000in}{-0.027778in}}{\pgfqpoint{0.000000in}{0.000000in}}{%
\pgfpathmoveto{\pgfqpoint{0.000000in}{0.000000in}}%
\pgfpathlineto{\pgfqpoint{0.000000in}{-0.027778in}}%
\pgfusepath{stroke,fill}%
}%
\begin{pgfscope}%
\pgfsys@transformshift{0.910611in}{0.521603in}%
\pgfsys@useobject{currentmarker}{}%
\end{pgfscope}%
\end{pgfscope}%
\begin{pgfscope}%
\pgfsetbuttcap%
\pgfsetroundjoin%
\definecolor{currentfill}{rgb}{0.000000,0.000000,0.000000}%
\pgfsetfillcolor{currentfill}%
\pgfsetlinewidth{0.602250pt}%
\definecolor{currentstroke}{rgb}{0.000000,0.000000,0.000000}%
\pgfsetstrokecolor{currentstroke}%
\pgfsetdash{}{0pt}%
\pgfsys@defobject{currentmarker}{\pgfqpoint{0.000000in}{-0.027778in}}{\pgfqpoint{0.000000in}{0.000000in}}{%
\pgfpathmoveto{\pgfqpoint{0.000000in}{0.000000in}}%
\pgfpathlineto{\pgfqpoint{0.000000in}{-0.027778in}}%
\pgfusepath{stroke,fill}%
}%
\begin{pgfscope}%
\pgfsys@transformshift{1.249672in}{0.521603in}%
\pgfsys@useobject{currentmarker}{}%
\end{pgfscope}%
\end{pgfscope}%
\begin{pgfscope}%
\pgfsetbuttcap%
\pgfsetroundjoin%
\definecolor{currentfill}{rgb}{0.000000,0.000000,0.000000}%
\pgfsetfillcolor{currentfill}%
\pgfsetlinewidth{0.602250pt}%
\definecolor{currentstroke}{rgb}{0.000000,0.000000,0.000000}%
\pgfsetstrokecolor{currentstroke}%
\pgfsetdash{}{0pt}%
\pgfsys@defobject{currentmarker}{\pgfqpoint{0.000000in}{-0.027778in}}{\pgfqpoint{0.000000in}{0.000000in}}{%
\pgfpathmoveto{\pgfqpoint{0.000000in}{0.000000in}}%
\pgfpathlineto{\pgfqpoint{0.000000in}{-0.027778in}}%
\pgfusepath{stroke,fill}%
}%
\begin{pgfscope}%
\pgfsys@transformshift{1.421840in}{0.521603in}%
\pgfsys@useobject{currentmarker}{}%
\end{pgfscope}%
\end{pgfscope}%
\begin{pgfscope}%
\pgfsetbuttcap%
\pgfsetroundjoin%
\definecolor{currentfill}{rgb}{0.000000,0.000000,0.000000}%
\pgfsetfillcolor{currentfill}%
\pgfsetlinewidth{0.602250pt}%
\definecolor{currentstroke}{rgb}{0.000000,0.000000,0.000000}%
\pgfsetstrokecolor{currentstroke}%
\pgfsetdash{}{0pt}%
\pgfsys@defobject{currentmarker}{\pgfqpoint{0.000000in}{-0.027778in}}{\pgfqpoint{0.000000in}{0.000000in}}{%
\pgfpathmoveto{\pgfqpoint{0.000000in}{0.000000in}}%
\pgfpathlineto{\pgfqpoint{0.000000in}{-0.027778in}}%
\pgfusepath{stroke,fill}%
}%
\begin{pgfscope}%
\pgfsys@transformshift{1.543995in}{0.521603in}%
\pgfsys@useobject{currentmarker}{}%
\end{pgfscope}%
\end{pgfscope}%
\begin{pgfscope}%
\pgfsetbuttcap%
\pgfsetroundjoin%
\definecolor{currentfill}{rgb}{0.000000,0.000000,0.000000}%
\pgfsetfillcolor{currentfill}%
\pgfsetlinewidth{0.602250pt}%
\definecolor{currentstroke}{rgb}{0.000000,0.000000,0.000000}%
\pgfsetstrokecolor{currentstroke}%
\pgfsetdash{}{0pt}%
\pgfsys@defobject{currentmarker}{\pgfqpoint{0.000000in}{-0.027778in}}{\pgfqpoint{0.000000in}{0.000000in}}{%
\pgfpathmoveto{\pgfqpoint{0.000000in}{0.000000in}}%
\pgfpathlineto{\pgfqpoint{0.000000in}{-0.027778in}}%
\pgfusepath{stroke,fill}%
}%
\begin{pgfscope}%
\pgfsys@transformshift{1.638746in}{0.521603in}%
\pgfsys@useobject{currentmarker}{}%
\end{pgfscope}%
\end{pgfscope}%
\begin{pgfscope}%
\pgfsetbuttcap%
\pgfsetroundjoin%
\definecolor{currentfill}{rgb}{0.000000,0.000000,0.000000}%
\pgfsetfillcolor{currentfill}%
\pgfsetlinewidth{0.602250pt}%
\definecolor{currentstroke}{rgb}{0.000000,0.000000,0.000000}%
\pgfsetstrokecolor{currentstroke}%
\pgfsetdash{}{0pt}%
\pgfsys@defobject{currentmarker}{\pgfqpoint{0.000000in}{-0.027778in}}{\pgfqpoint{0.000000in}{0.000000in}}{%
\pgfpathmoveto{\pgfqpoint{0.000000in}{0.000000in}}%
\pgfpathlineto{\pgfqpoint{0.000000in}{-0.027778in}}%
\pgfusepath{stroke,fill}%
}%
\begin{pgfscope}%
\pgfsys@transformshift{1.716163in}{0.521603in}%
\pgfsys@useobject{currentmarker}{}%
\end{pgfscope}%
\end{pgfscope}%
\begin{pgfscope}%
\pgfsetbuttcap%
\pgfsetroundjoin%
\definecolor{currentfill}{rgb}{0.000000,0.000000,0.000000}%
\pgfsetfillcolor{currentfill}%
\pgfsetlinewidth{0.602250pt}%
\definecolor{currentstroke}{rgb}{0.000000,0.000000,0.000000}%
\pgfsetstrokecolor{currentstroke}%
\pgfsetdash{}{0pt}%
\pgfsys@defobject{currentmarker}{\pgfqpoint{0.000000in}{-0.027778in}}{\pgfqpoint{0.000000in}{0.000000in}}{%
\pgfpathmoveto{\pgfqpoint{0.000000in}{0.000000in}}%
\pgfpathlineto{\pgfqpoint{0.000000in}{-0.027778in}}%
\pgfusepath{stroke,fill}%
}%
\begin{pgfscope}%
\pgfsys@transformshift{1.781619in}{0.521603in}%
\pgfsys@useobject{currentmarker}{}%
\end{pgfscope}%
\end{pgfscope}%
\begin{pgfscope}%
\pgfsetbuttcap%
\pgfsetroundjoin%
\definecolor{currentfill}{rgb}{0.000000,0.000000,0.000000}%
\pgfsetfillcolor{currentfill}%
\pgfsetlinewidth{0.602250pt}%
\definecolor{currentstroke}{rgb}{0.000000,0.000000,0.000000}%
\pgfsetstrokecolor{currentstroke}%
\pgfsetdash{}{0pt}%
\pgfsys@defobject{currentmarker}{\pgfqpoint{0.000000in}{-0.027778in}}{\pgfqpoint{0.000000in}{0.000000in}}{%
\pgfpathmoveto{\pgfqpoint{0.000000in}{0.000000in}}%
\pgfpathlineto{\pgfqpoint{0.000000in}{-0.027778in}}%
\pgfusepath{stroke,fill}%
}%
\begin{pgfscope}%
\pgfsys@transformshift{1.838318in}{0.521603in}%
\pgfsys@useobject{currentmarker}{}%
\end{pgfscope}%
\end{pgfscope}%
\begin{pgfscope}%
\pgfsetbuttcap%
\pgfsetroundjoin%
\definecolor{currentfill}{rgb}{0.000000,0.000000,0.000000}%
\pgfsetfillcolor{currentfill}%
\pgfsetlinewidth{0.602250pt}%
\definecolor{currentstroke}{rgb}{0.000000,0.000000,0.000000}%
\pgfsetstrokecolor{currentstroke}%
\pgfsetdash{}{0pt}%
\pgfsys@defobject{currentmarker}{\pgfqpoint{0.000000in}{-0.027778in}}{\pgfqpoint{0.000000in}{0.000000in}}{%
\pgfpathmoveto{\pgfqpoint{0.000000in}{0.000000in}}%
\pgfpathlineto{\pgfqpoint{0.000000in}{-0.027778in}}%
\pgfusepath{stroke,fill}%
}%
\begin{pgfscope}%
\pgfsys@transformshift{1.888331in}{0.521603in}%
\pgfsys@useobject{currentmarker}{}%
\end{pgfscope}%
\end{pgfscope}%
\begin{pgfscope}%
\pgfsetbuttcap%
\pgfsetroundjoin%
\definecolor{currentfill}{rgb}{0.000000,0.000000,0.000000}%
\pgfsetfillcolor{currentfill}%
\pgfsetlinewidth{0.602250pt}%
\definecolor{currentstroke}{rgb}{0.000000,0.000000,0.000000}%
\pgfsetstrokecolor{currentstroke}%
\pgfsetdash{}{0pt}%
\pgfsys@defobject{currentmarker}{\pgfqpoint{0.000000in}{-0.027778in}}{\pgfqpoint{0.000000in}{0.000000in}}{%
\pgfpathmoveto{\pgfqpoint{0.000000in}{0.000000in}}%
\pgfpathlineto{\pgfqpoint{0.000000in}{-0.027778in}}%
\pgfusepath{stroke,fill}%
}%
\begin{pgfscope}%
\pgfsys@transformshift{2.227392in}{0.521603in}%
\pgfsys@useobject{currentmarker}{}%
\end{pgfscope}%
\end{pgfscope}%
\begin{pgfscope}%
\pgfsetbuttcap%
\pgfsetroundjoin%
\definecolor{currentfill}{rgb}{0.000000,0.000000,0.000000}%
\pgfsetfillcolor{currentfill}%
\pgfsetlinewidth{0.602250pt}%
\definecolor{currentstroke}{rgb}{0.000000,0.000000,0.000000}%
\pgfsetstrokecolor{currentstroke}%
\pgfsetdash{}{0pt}%
\pgfsys@defobject{currentmarker}{\pgfqpoint{0.000000in}{-0.027778in}}{\pgfqpoint{0.000000in}{0.000000in}}{%
\pgfpathmoveto{\pgfqpoint{0.000000in}{0.000000in}}%
\pgfpathlineto{\pgfqpoint{0.000000in}{-0.027778in}}%
\pgfusepath{stroke,fill}%
}%
\begin{pgfscope}%
\pgfsys@transformshift{2.399560in}{0.521603in}%
\pgfsys@useobject{currentmarker}{}%
\end{pgfscope}%
\end{pgfscope}%
\begin{pgfscope}%
\pgfsetbuttcap%
\pgfsetroundjoin%
\definecolor{currentfill}{rgb}{0.000000,0.000000,0.000000}%
\pgfsetfillcolor{currentfill}%
\pgfsetlinewidth{0.602250pt}%
\definecolor{currentstroke}{rgb}{0.000000,0.000000,0.000000}%
\pgfsetstrokecolor{currentstroke}%
\pgfsetdash{}{0pt}%
\pgfsys@defobject{currentmarker}{\pgfqpoint{0.000000in}{-0.027778in}}{\pgfqpoint{0.000000in}{0.000000in}}{%
\pgfpathmoveto{\pgfqpoint{0.000000in}{0.000000in}}%
\pgfpathlineto{\pgfqpoint{0.000000in}{-0.027778in}}%
\pgfusepath{stroke,fill}%
}%
\begin{pgfscope}%
\pgfsys@transformshift{2.521715in}{0.521603in}%
\pgfsys@useobject{currentmarker}{}%
\end{pgfscope}%
\end{pgfscope}%
\begin{pgfscope}%
\pgfsetbuttcap%
\pgfsetroundjoin%
\definecolor{currentfill}{rgb}{0.000000,0.000000,0.000000}%
\pgfsetfillcolor{currentfill}%
\pgfsetlinewidth{0.602250pt}%
\definecolor{currentstroke}{rgb}{0.000000,0.000000,0.000000}%
\pgfsetstrokecolor{currentstroke}%
\pgfsetdash{}{0pt}%
\pgfsys@defobject{currentmarker}{\pgfqpoint{0.000000in}{-0.027778in}}{\pgfqpoint{0.000000in}{0.000000in}}{%
\pgfpathmoveto{\pgfqpoint{0.000000in}{0.000000in}}%
\pgfpathlineto{\pgfqpoint{0.000000in}{-0.027778in}}%
\pgfusepath{stroke,fill}%
}%
\begin{pgfscope}%
\pgfsys@transformshift{2.616466in}{0.521603in}%
\pgfsys@useobject{currentmarker}{}%
\end{pgfscope}%
\end{pgfscope}%
\begin{pgfscope}%
\pgfsetbuttcap%
\pgfsetroundjoin%
\definecolor{currentfill}{rgb}{0.000000,0.000000,0.000000}%
\pgfsetfillcolor{currentfill}%
\pgfsetlinewidth{0.602250pt}%
\definecolor{currentstroke}{rgb}{0.000000,0.000000,0.000000}%
\pgfsetstrokecolor{currentstroke}%
\pgfsetdash{}{0pt}%
\pgfsys@defobject{currentmarker}{\pgfqpoint{0.000000in}{-0.027778in}}{\pgfqpoint{0.000000in}{0.000000in}}{%
\pgfpathmoveto{\pgfqpoint{0.000000in}{0.000000in}}%
\pgfpathlineto{\pgfqpoint{0.000000in}{-0.027778in}}%
\pgfusepath{stroke,fill}%
}%
\begin{pgfscope}%
\pgfsys@transformshift{2.693883in}{0.521603in}%
\pgfsys@useobject{currentmarker}{}%
\end{pgfscope}%
\end{pgfscope}%
\begin{pgfscope}%
\pgfsetbuttcap%
\pgfsetroundjoin%
\definecolor{currentfill}{rgb}{0.000000,0.000000,0.000000}%
\pgfsetfillcolor{currentfill}%
\pgfsetlinewidth{0.602250pt}%
\definecolor{currentstroke}{rgb}{0.000000,0.000000,0.000000}%
\pgfsetstrokecolor{currentstroke}%
\pgfsetdash{}{0pt}%
\pgfsys@defobject{currentmarker}{\pgfqpoint{0.000000in}{-0.027778in}}{\pgfqpoint{0.000000in}{0.000000in}}{%
\pgfpathmoveto{\pgfqpoint{0.000000in}{0.000000in}}%
\pgfpathlineto{\pgfqpoint{0.000000in}{-0.027778in}}%
\pgfusepath{stroke,fill}%
}%
\begin{pgfscope}%
\pgfsys@transformshift{2.759339in}{0.521603in}%
\pgfsys@useobject{currentmarker}{}%
\end{pgfscope}%
\end{pgfscope}%
\begin{pgfscope}%
\pgfsetbuttcap%
\pgfsetroundjoin%
\definecolor{currentfill}{rgb}{0.000000,0.000000,0.000000}%
\pgfsetfillcolor{currentfill}%
\pgfsetlinewidth{0.602250pt}%
\definecolor{currentstroke}{rgb}{0.000000,0.000000,0.000000}%
\pgfsetstrokecolor{currentstroke}%
\pgfsetdash{}{0pt}%
\pgfsys@defobject{currentmarker}{\pgfqpoint{0.000000in}{-0.027778in}}{\pgfqpoint{0.000000in}{0.000000in}}{%
\pgfpathmoveto{\pgfqpoint{0.000000in}{0.000000in}}%
\pgfpathlineto{\pgfqpoint{0.000000in}{-0.027778in}}%
\pgfusepath{stroke,fill}%
}%
\begin{pgfscope}%
\pgfsys@transformshift{2.816038in}{0.521603in}%
\pgfsys@useobject{currentmarker}{}%
\end{pgfscope}%
\end{pgfscope}%
\begin{pgfscope}%
\pgfsetbuttcap%
\pgfsetroundjoin%
\definecolor{currentfill}{rgb}{0.000000,0.000000,0.000000}%
\pgfsetfillcolor{currentfill}%
\pgfsetlinewidth{0.602250pt}%
\definecolor{currentstroke}{rgb}{0.000000,0.000000,0.000000}%
\pgfsetstrokecolor{currentstroke}%
\pgfsetdash{}{0pt}%
\pgfsys@defobject{currentmarker}{\pgfqpoint{0.000000in}{-0.027778in}}{\pgfqpoint{0.000000in}{0.000000in}}{%
\pgfpathmoveto{\pgfqpoint{0.000000in}{0.000000in}}%
\pgfpathlineto{\pgfqpoint{0.000000in}{-0.027778in}}%
\pgfusepath{stroke,fill}%
}%
\begin{pgfscope}%
\pgfsys@transformshift{2.866051in}{0.521603in}%
\pgfsys@useobject{currentmarker}{}%
\end{pgfscope}%
\end{pgfscope}%
\begin{pgfscope}%
\pgfsetbuttcap%
\pgfsetroundjoin%
\definecolor{currentfill}{rgb}{0.000000,0.000000,0.000000}%
\pgfsetfillcolor{currentfill}%
\pgfsetlinewidth{0.602250pt}%
\definecolor{currentstroke}{rgb}{0.000000,0.000000,0.000000}%
\pgfsetstrokecolor{currentstroke}%
\pgfsetdash{}{0pt}%
\pgfsys@defobject{currentmarker}{\pgfqpoint{0.000000in}{-0.027778in}}{\pgfqpoint{0.000000in}{0.000000in}}{%
\pgfpathmoveto{\pgfqpoint{0.000000in}{0.000000in}}%
\pgfpathlineto{\pgfqpoint{0.000000in}{-0.027778in}}%
\pgfusepath{stroke,fill}%
}%
\begin{pgfscope}%
\pgfsys@transformshift{3.205112in}{0.521603in}%
\pgfsys@useobject{currentmarker}{}%
\end{pgfscope}%
\end{pgfscope}%
\begin{pgfscope}%
\pgfsetbuttcap%
\pgfsetroundjoin%
\definecolor{currentfill}{rgb}{0.000000,0.000000,0.000000}%
\pgfsetfillcolor{currentfill}%
\pgfsetlinewidth{0.602250pt}%
\definecolor{currentstroke}{rgb}{0.000000,0.000000,0.000000}%
\pgfsetstrokecolor{currentstroke}%
\pgfsetdash{}{0pt}%
\pgfsys@defobject{currentmarker}{\pgfqpoint{0.000000in}{-0.027778in}}{\pgfqpoint{0.000000in}{0.000000in}}{%
\pgfpathmoveto{\pgfqpoint{0.000000in}{0.000000in}}%
\pgfpathlineto{\pgfqpoint{0.000000in}{-0.027778in}}%
\pgfusepath{stroke,fill}%
}%
\begin{pgfscope}%
\pgfsys@transformshift{3.377280in}{0.521603in}%
\pgfsys@useobject{currentmarker}{}%
\end{pgfscope}%
\end{pgfscope}%
\begin{pgfscope}%
\pgfsetbuttcap%
\pgfsetroundjoin%
\definecolor{currentfill}{rgb}{0.000000,0.000000,0.000000}%
\pgfsetfillcolor{currentfill}%
\pgfsetlinewidth{0.602250pt}%
\definecolor{currentstroke}{rgb}{0.000000,0.000000,0.000000}%
\pgfsetstrokecolor{currentstroke}%
\pgfsetdash{}{0pt}%
\pgfsys@defobject{currentmarker}{\pgfqpoint{0.000000in}{-0.027778in}}{\pgfqpoint{0.000000in}{0.000000in}}{%
\pgfpathmoveto{\pgfqpoint{0.000000in}{0.000000in}}%
\pgfpathlineto{\pgfqpoint{0.000000in}{-0.027778in}}%
\pgfusepath{stroke,fill}%
}%
\begin{pgfscope}%
\pgfsys@transformshift{3.499436in}{0.521603in}%
\pgfsys@useobject{currentmarker}{}%
\end{pgfscope}%
\end{pgfscope}%
\begin{pgfscope}%
\pgfsetbuttcap%
\pgfsetroundjoin%
\definecolor{currentfill}{rgb}{0.000000,0.000000,0.000000}%
\pgfsetfillcolor{currentfill}%
\pgfsetlinewidth{0.602250pt}%
\definecolor{currentstroke}{rgb}{0.000000,0.000000,0.000000}%
\pgfsetstrokecolor{currentstroke}%
\pgfsetdash{}{0pt}%
\pgfsys@defobject{currentmarker}{\pgfqpoint{0.000000in}{-0.027778in}}{\pgfqpoint{0.000000in}{0.000000in}}{%
\pgfpathmoveto{\pgfqpoint{0.000000in}{0.000000in}}%
\pgfpathlineto{\pgfqpoint{0.000000in}{-0.027778in}}%
\pgfusepath{stroke,fill}%
}%
\begin{pgfscope}%
\pgfsys@transformshift{3.594186in}{0.521603in}%
\pgfsys@useobject{currentmarker}{}%
\end{pgfscope}%
\end{pgfscope}%
\begin{pgfscope}%
\pgfsetbuttcap%
\pgfsetroundjoin%
\definecolor{currentfill}{rgb}{0.000000,0.000000,0.000000}%
\pgfsetfillcolor{currentfill}%
\pgfsetlinewidth{0.602250pt}%
\definecolor{currentstroke}{rgb}{0.000000,0.000000,0.000000}%
\pgfsetstrokecolor{currentstroke}%
\pgfsetdash{}{0pt}%
\pgfsys@defobject{currentmarker}{\pgfqpoint{0.000000in}{-0.027778in}}{\pgfqpoint{0.000000in}{0.000000in}}{%
\pgfpathmoveto{\pgfqpoint{0.000000in}{0.000000in}}%
\pgfpathlineto{\pgfqpoint{0.000000in}{-0.027778in}}%
\pgfusepath{stroke,fill}%
}%
\begin{pgfscope}%
\pgfsys@transformshift{3.671603in}{0.521603in}%
\pgfsys@useobject{currentmarker}{}%
\end{pgfscope}%
\end{pgfscope}%
\begin{pgfscope}%
\pgfsetbuttcap%
\pgfsetroundjoin%
\definecolor{currentfill}{rgb}{0.000000,0.000000,0.000000}%
\pgfsetfillcolor{currentfill}%
\pgfsetlinewidth{0.602250pt}%
\definecolor{currentstroke}{rgb}{0.000000,0.000000,0.000000}%
\pgfsetstrokecolor{currentstroke}%
\pgfsetdash{}{0pt}%
\pgfsys@defobject{currentmarker}{\pgfqpoint{0.000000in}{-0.027778in}}{\pgfqpoint{0.000000in}{0.000000in}}{%
\pgfpathmoveto{\pgfqpoint{0.000000in}{0.000000in}}%
\pgfpathlineto{\pgfqpoint{0.000000in}{-0.027778in}}%
\pgfusepath{stroke,fill}%
}%
\begin{pgfscope}%
\pgfsys@transformshift{3.737059in}{0.521603in}%
\pgfsys@useobject{currentmarker}{}%
\end{pgfscope}%
\end{pgfscope}%
\begin{pgfscope}%
\pgfsetbuttcap%
\pgfsetroundjoin%
\definecolor{currentfill}{rgb}{0.000000,0.000000,0.000000}%
\pgfsetfillcolor{currentfill}%
\pgfsetlinewidth{0.602250pt}%
\definecolor{currentstroke}{rgb}{0.000000,0.000000,0.000000}%
\pgfsetstrokecolor{currentstroke}%
\pgfsetdash{}{0pt}%
\pgfsys@defobject{currentmarker}{\pgfqpoint{0.000000in}{-0.027778in}}{\pgfqpoint{0.000000in}{0.000000in}}{%
\pgfpathmoveto{\pgfqpoint{0.000000in}{0.000000in}}%
\pgfpathlineto{\pgfqpoint{0.000000in}{-0.027778in}}%
\pgfusepath{stroke,fill}%
}%
\begin{pgfscope}%
\pgfsys@transformshift{3.793759in}{0.521603in}%
\pgfsys@useobject{currentmarker}{}%
\end{pgfscope}%
\end{pgfscope}%
\begin{pgfscope}%
\pgfsetbuttcap%
\pgfsetroundjoin%
\definecolor{currentfill}{rgb}{0.000000,0.000000,0.000000}%
\pgfsetfillcolor{currentfill}%
\pgfsetlinewidth{0.602250pt}%
\definecolor{currentstroke}{rgb}{0.000000,0.000000,0.000000}%
\pgfsetstrokecolor{currentstroke}%
\pgfsetdash{}{0pt}%
\pgfsys@defobject{currentmarker}{\pgfqpoint{0.000000in}{-0.027778in}}{\pgfqpoint{0.000000in}{0.000000in}}{%
\pgfpathmoveto{\pgfqpoint{0.000000in}{0.000000in}}%
\pgfpathlineto{\pgfqpoint{0.000000in}{-0.027778in}}%
\pgfusepath{stroke,fill}%
}%
\begin{pgfscope}%
\pgfsys@transformshift{3.843771in}{0.521603in}%
\pgfsys@useobject{currentmarker}{}%
\end{pgfscope}%
\end{pgfscope}%
\begin{pgfscope}%
\pgfsetbuttcap%
\pgfsetroundjoin%
\definecolor{currentfill}{rgb}{0.000000,0.000000,0.000000}%
\pgfsetfillcolor{currentfill}%
\pgfsetlinewidth{0.602250pt}%
\definecolor{currentstroke}{rgb}{0.000000,0.000000,0.000000}%
\pgfsetstrokecolor{currentstroke}%
\pgfsetdash{}{0pt}%
\pgfsys@defobject{currentmarker}{\pgfqpoint{0.000000in}{-0.027778in}}{\pgfqpoint{0.000000in}{0.000000in}}{%
\pgfpathmoveto{\pgfqpoint{0.000000in}{0.000000in}}%
\pgfpathlineto{\pgfqpoint{0.000000in}{-0.027778in}}%
\pgfusepath{stroke,fill}%
}%
\begin{pgfscope}%
\pgfsys@transformshift{4.182833in}{0.521603in}%
\pgfsys@useobject{currentmarker}{}%
\end{pgfscope}%
\end{pgfscope}%
\begin{pgfscope}%
\pgfsetbuttcap%
\pgfsetroundjoin%
\definecolor{currentfill}{rgb}{0.000000,0.000000,0.000000}%
\pgfsetfillcolor{currentfill}%
\pgfsetlinewidth{0.602250pt}%
\definecolor{currentstroke}{rgb}{0.000000,0.000000,0.000000}%
\pgfsetstrokecolor{currentstroke}%
\pgfsetdash{}{0pt}%
\pgfsys@defobject{currentmarker}{\pgfqpoint{0.000000in}{-0.027778in}}{\pgfqpoint{0.000000in}{0.000000in}}{%
\pgfpathmoveto{\pgfqpoint{0.000000in}{0.000000in}}%
\pgfpathlineto{\pgfqpoint{0.000000in}{-0.027778in}}%
\pgfusepath{stroke,fill}%
}%
\begin{pgfscope}%
\pgfsys@transformshift{4.355000in}{0.521603in}%
\pgfsys@useobject{currentmarker}{}%
\end{pgfscope}%
\end{pgfscope}%
\begin{pgfscope}%
\pgfsetbuttcap%
\pgfsetroundjoin%
\definecolor{currentfill}{rgb}{0.000000,0.000000,0.000000}%
\pgfsetfillcolor{currentfill}%
\pgfsetlinewidth{0.602250pt}%
\definecolor{currentstroke}{rgb}{0.000000,0.000000,0.000000}%
\pgfsetstrokecolor{currentstroke}%
\pgfsetdash{}{0pt}%
\pgfsys@defobject{currentmarker}{\pgfqpoint{0.000000in}{-0.027778in}}{\pgfqpoint{0.000000in}{0.000000in}}{%
\pgfpathmoveto{\pgfqpoint{0.000000in}{0.000000in}}%
\pgfpathlineto{\pgfqpoint{0.000000in}{-0.027778in}}%
\pgfusepath{stroke,fill}%
}%
\begin{pgfscope}%
\pgfsys@transformshift{4.477156in}{0.521603in}%
\pgfsys@useobject{currentmarker}{}%
\end{pgfscope}%
\end{pgfscope}%
\begin{pgfscope}%
\pgfsetbuttcap%
\pgfsetroundjoin%
\definecolor{currentfill}{rgb}{0.000000,0.000000,0.000000}%
\pgfsetfillcolor{currentfill}%
\pgfsetlinewidth{0.602250pt}%
\definecolor{currentstroke}{rgb}{0.000000,0.000000,0.000000}%
\pgfsetstrokecolor{currentstroke}%
\pgfsetdash{}{0pt}%
\pgfsys@defobject{currentmarker}{\pgfqpoint{0.000000in}{-0.027778in}}{\pgfqpoint{0.000000in}{0.000000in}}{%
\pgfpathmoveto{\pgfqpoint{0.000000in}{0.000000in}}%
\pgfpathlineto{\pgfqpoint{0.000000in}{-0.027778in}}%
\pgfusepath{stroke,fill}%
}%
\begin{pgfscope}%
\pgfsys@transformshift{4.571906in}{0.521603in}%
\pgfsys@useobject{currentmarker}{}%
\end{pgfscope}%
\end{pgfscope}%
\begin{pgfscope}%
\pgfsetbuttcap%
\pgfsetroundjoin%
\definecolor{currentfill}{rgb}{0.000000,0.000000,0.000000}%
\pgfsetfillcolor{currentfill}%
\pgfsetlinewidth{0.602250pt}%
\definecolor{currentstroke}{rgb}{0.000000,0.000000,0.000000}%
\pgfsetstrokecolor{currentstroke}%
\pgfsetdash{}{0pt}%
\pgfsys@defobject{currentmarker}{\pgfqpoint{0.000000in}{-0.027778in}}{\pgfqpoint{0.000000in}{0.000000in}}{%
\pgfpathmoveto{\pgfqpoint{0.000000in}{0.000000in}}%
\pgfpathlineto{\pgfqpoint{0.000000in}{-0.027778in}}%
\pgfusepath{stroke,fill}%
}%
\begin{pgfscope}%
\pgfsys@transformshift{4.649324in}{0.521603in}%
\pgfsys@useobject{currentmarker}{}%
\end{pgfscope}%
\end{pgfscope}%
\begin{pgfscope}%
\pgfsetbuttcap%
\pgfsetroundjoin%
\definecolor{currentfill}{rgb}{0.000000,0.000000,0.000000}%
\pgfsetfillcolor{currentfill}%
\pgfsetlinewidth{0.602250pt}%
\definecolor{currentstroke}{rgb}{0.000000,0.000000,0.000000}%
\pgfsetstrokecolor{currentstroke}%
\pgfsetdash{}{0pt}%
\pgfsys@defobject{currentmarker}{\pgfqpoint{0.000000in}{-0.027778in}}{\pgfqpoint{0.000000in}{0.000000in}}{%
\pgfpathmoveto{\pgfqpoint{0.000000in}{0.000000in}}%
\pgfpathlineto{\pgfqpoint{0.000000in}{-0.027778in}}%
\pgfusepath{stroke,fill}%
}%
\begin{pgfscope}%
\pgfsys@transformshift{4.714779in}{0.521603in}%
\pgfsys@useobject{currentmarker}{}%
\end{pgfscope}%
\end{pgfscope}%
\begin{pgfscope}%
\pgfsetbuttcap%
\pgfsetroundjoin%
\definecolor{currentfill}{rgb}{0.000000,0.000000,0.000000}%
\pgfsetfillcolor{currentfill}%
\pgfsetlinewidth{0.602250pt}%
\definecolor{currentstroke}{rgb}{0.000000,0.000000,0.000000}%
\pgfsetstrokecolor{currentstroke}%
\pgfsetdash{}{0pt}%
\pgfsys@defobject{currentmarker}{\pgfqpoint{0.000000in}{-0.027778in}}{\pgfqpoint{0.000000in}{0.000000in}}{%
\pgfpathmoveto{\pgfqpoint{0.000000in}{0.000000in}}%
\pgfpathlineto{\pgfqpoint{0.000000in}{-0.027778in}}%
\pgfusepath{stroke,fill}%
}%
\begin{pgfscope}%
\pgfsys@transformshift{4.771479in}{0.521603in}%
\pgfsys@useobject{currentmarker}{}%
\end{pgfscope}%
\end{pgfscope}%
\begin{pgfscope}%
\pgfsetbuttcap%
\pgfsetroundjoin%
\definecolor{currentfill}{rgb}{0.000000,0.000000,0.000000}%
\pgfsetfillcolor{currentfill}%
\pgfsetlinewidth{0.602250pt}%
\definecolor{currentstroke}{rgb}{0.000000,0.000000,0.000000}%
\pgfsetstrokecolor{currentstroke}%
\pgfsetdash{}{0pt}%
\pgfsys@defobject{currentmarker}{\pgfqpoint{0.000000in}{-0.027778in}}{\pgfqpoint{0.000000in}{0.000000in}}{%
\pgfpathmoveto{\pgfqpoint{0.000000in}{0.000000in}}%
\pgfpathlineto{\pgfqpoint{0.000000in}{-0.027778in}}%
\pgfusepath{stroke,fill}%
}%
\begin{pgfscope}%
\pgfsys@transformshift{4.821492in}{0.521603in}%
\pgfsys@useobject{currentmarker}{}%
\end{pgfscope}%
\end{pgfscope}%
\begin{pgfscope}%
\pgfsetbuttcap%
\pgfsetroundjoin%
\definecolor{currentfill}{rgb}{0.000000,0.000000,0.000000}%
\pgfsetfillcolor{currentfill}%
\pgfsetlinewidth{0.602250pt}%
\definecolor{currentstroke}{rgb}{0.000000,0.000000,0.000000}%
\pgfsetstrokecolor{currentstroke}%
\pgfsetdash{}{0pt}%
\pgfsys@defobject{currentmarker}{\pgfqpoint{0.000000in}{-0.027778in}}{\pgfqpoint{0.000000in}{0.000000in}}{%
\pgfpathmoveto{\pgfqpoint{0.000000in}{0.000000in}}%
\pgfpathlineto{\pgfqpoint{0.000000in}{-0.027778in}}%
\pgfusepath{stroke,fill}%
}%
\begin{pgfscope}%
\pgfsys@transformshift{5.160553in}{0.521603in}%
\pgfsys@useobject{currentmarker}{}%
\end{pgfscope}%
\end{pgfscope}%
\begin{pgfscope}%
\pgfsetbuttcap%
\pgfsetroundjoin%
\definecolor{currentfill}{rgb}{0.000000,0.000000,0.000000}%
\pgfsetfillcolor{currentfill}%
\pgfsetlinewidth{0.602250pt}%
\definecolor{currentstroke}{rgb}{0.000000,0.000000,0.000000}%
\pgfsetstrokecolor{currentstroke}%
\pgfsetdash{}{0pt}%
\pgfsys@defobject{currentmarker}{\pgfqpoint{0.000000in}{-0.027778in}}{\pgfqpoint{0.000000in}{0.000000in}}{%
\pgfpathmoveto{\pgfqpoint{0.000000in}{0.000000in}}%
\pgfpathlineto{\pgfqpoint{0.000000in}{-0.027778in}}%
\pgfusepath{stroke,fill}%
}%
\begin{pgfscope}%
\pgfsys@transformshift{5.332721in}{0.521603in}%
\pgfsys@useobject{currentmarker}{}%
\end{pgfscope}%
\end{pgfscope}%
\begin{pgfscope}%
\pgfsetbuttcap%
\pgfsetroundjoin%
\definecolor{currentfill}{rgb}{0.000000,0.000000,0.000000}%
\pgfsetfillcolor{currentfill}%
\pgfsetlinewidth{0.602250pt}%
\definecolor{currentstroke}{rgb}{0.000000,0.000000,0.000000}%
\pgfsetstrokecolor{currentstroke}%
\pgfsetdash{}{0pt}%
\pgfsys@defobject{currentmarker}{\pgfqpoint{0.000000in}{-0.027778in}}{\pgfqpoint{0.000000in}{0.000000in}}{%
\pgfpathmoveto{\pgfqpoint{0.000000in}{0.000000in}}%
\pgfpathlineto{\pgfqpoint{0.000000in}{-0.027778in}}%
\pgfusepath{stroke,fill}%
}%
\begin{pgfscope}%
\pgfsys@transformshift{5.454876in}{0.521603in}%
\pgfsys@useobject{currentmarker}{}%
\end{pgfscope}%
\end{pgfscope}%
\begin{pgfscope}%
\pgfsetbuttcap%
\pgfsetroundjoin%
\definecolor{currentfill}{rgb}{0.000000,0.000000,0.000000}%
\pgfsetfillcolor{currentfill}%
\pgfsetlinewidth{0.602250pt}%
\definecolor{currentstroke}{rgb}{0.000000,0.000000,0.000000}%
\pgfsetstrokecolor{currentstroke}%
\pgfsetdash{}{0pt}%
\pgfsys@defobject{currentmarker}{\pgfqpoint{0.000000in}{-0.027778in}}{\pgfqpoint{0.000000in}{0.000000in}}{%
\pgfpathmoveto{\pgfqpoint{0.000000in}{0.000000in}}%
\pgfpathlineto{\pgfqpoint{0.000000in}{-0.027778in}}%
\pgfusepath{stroke,fill}%
}%
\begin{pgfscope}%
\pgfsys@transformshift{5.549627in}{0.521603in}%
\pgfsys@useobject{currentmarker}{}%
\end{pgfscope}%
\end{pgfscope}%
\begin{pgfscope}%
\pgfsetbuttcap%
\pgfsetroundjoin%
\definecolor{currentfill}{rgb}{0.000000,0.000000,0.000000}%
\pgfsetfillcolor{currentfill}%
\pgfsetlinewidth{0.602250pt}%
\definecolor{currentstroke}{rgb}{0.000000,0.000000,0.000000}%
\pgfsetstrokecolor{currentstroke}%
\pgfsetdash{}{0pt}%
\pgfsys@defobject{currentmarker}{\pgfqpoint{0.000000in}{-0.027778in}}{\pgfqpoint{0.000000in}{0.000000in}}{%
\pgfpathmoveto{\pgfqpoint{0.000000in}{0.000000in}}%
\pgfpathlineto{\pgfqpoint{0.000000in}{-0.027778in}}%
\pgfusepath{stroke,fill}%
}%
\begin{pgfscope}%
\pgfsys@transformshift{5.627044in}{0.521603in}%
\pgfsys@useobject{currentmarker}{}%
\end{pgfscope}%
\end{pgfscope}%
\begin{pgfscope}%
\pgfsetbuttcap%
\pgfsetroundjoin%
\definecolor{currentfill}{rgb}{0.000000,0.000000,0.000000}%
\pgfsetfillcolor{currentfill}%
\pgfsetlinewidth{0.602250pt}%
\definecolor{currentstroke}{rgb}{0.000000,0.000000,0.000000}%
\pgfsetstrokecolor{currentstroke}%
\pgfsetdash{}{0pt}%
\pgfsys@defobject{currentmarker}{\pgfqpoint{0.000000in}{-0.027778in}}{\pgfqpoint{0.000000in}{0.000000in}}{%
\pgfpathmoveto{\pgfqpoint{0.000000in}{0.000000in}}%
\pgfpathlineto{\pgfqpoint{0.000000in}{-0.027778in}}%
\pgfusepath{stroke,fill}%
}%
\begin{pgfscope}%
\pgfsys@transformshift{5.692499in}{0.521603in}%
\pgfsys@useobject{currentmarker}{}%
\end{pgfscope}%
\end{pgfscope}%
\begin{pgfscope}%
\pgfsetbuttcap%
\pgfsetroundjoin%
\definecolor{currentfill}{rgb}{0.000000,0.000000,0.000000}%
\pgfsetfillcolor{currentfill}%
\pgfsetlinewidth{0.602250pt}%
\definecolor{currentstroke}{rgb}{0.000000,0.000000,0.000000}%
\pgfsetstrokecolor{currentstroke}%
\pgfsetdash{}{0pt}%
\pgfsys@defobject{currentmarker}{\pgfqpoint{0.000000in}{-0.027778in}}{\pgfqpoint{0.000000in}{0.000000in}}{%
\pgfpathmoveto{\pgfqpoint{0.000000in}{0.000000in}}%
\pgfpathlineto{\pgfqpoint{0.000000in}{-0.027778in}}%
\pgfusepath{stroke,fill}%
}%
\begin{pgfscope}%
\pgfsys@transformshift{5.749199in}{0.521603in}%
\pgfsys@useobject{currentmarker}{}%
\end{pgfscope}%
\end{pgfscope}%
\begin{pgfscope}%
\pgfsetbuttcap%
\pgfsetroundjoin%
\definecolor{currentfill}{rgb}{0.000000,0.000000,0.000000}%
\pgfsetfillcolor{currentfill}%
\pgfsetlinewidth{0.602250pt}%
\definecolor{currentstroke}{rgb}{0.000000,0.000000,0.000000}%
\pgfsetstrokecolor{currentstroke}%
\pgfsetdash{}{0pt}%
\pgfsys@defobject{currentmarker}{\pgfqpoint{0.000000in}{-0.027778in}}{\pgfqpoint{0.000000in}{0.000000in}}{%
\pgfpathmoveto{\pgfqpoint{0.000000in}{0.000000in}}%
\pgfpathlineto{\pgfqpoint{0.000000in}{-0.027778in}}%
\pgfusepath{stroke,fill}%
}%
\begin{pgfscope}%
\pgfsys@transformshift{5.799212in}{0.521603in}%
\pgfsys@useobject{currentmarker}{}%
\end{pgfscope}%
\end{pgfscope}%
\begin{pgfscope}%
\pgftext[x=3.399649in,y=0.234413in,,top]{\sffamily\fontsize{10.000000}{12.000000}\selectfont \(\displaystyle D\) (Number of features)}%
\end{pgfscope}%
\begin{pgfscope}%
\pgfsetbuttcap%
\pgfsetroundjoin%
\definecolor{currentfill}{rgb}{0.000000,0.000000,0.000000}%
\pgfsetfillcolor{currentfill}%
\pgfsetlinewidth{0.803000pt}%
\definecolor{currentstroke}{rgb}{0.000000,0.000000,0.000000}%
\pgfsetstrokecolor{currentstroke}%
\pgfsetdash{}{0pt}%
\pgfsys@defobject{currentmarker}{\pgfqpoint{-0.048611in}{0.000000in}}{\pgfqpoint{0.000000in}{0.000000in}}{%
\pgfpathmoveto{\pgfqpoint{0.000000in}{0.000000in}}%
\pgfpathlineto{\pgfqpoint{-0.048611in}{0.000000in}}%
\pgfusepath{stroke,fill}%
}%
\begin{pgfscope}%
\pgfsys@transformshift{0.710919in}{0.787342in}%
\pgfsys@useobject{currentmarker}{}%
\end{pgfscope}%
\end{pgfscope}%
\begin{pgfscope}%
\pgftext[x=0.325694in,y=0.734580in,left,base]{\sffamily\fontsize{10.000000}{12.000000}\selectfont \(\displaystyle {10^{-2}}\)}%
\end{pgfscope}%
\begin{pgfscope}%
\pgfsetbuttcap%
\pgfsetroundjoin%
\definecolor{currentfill}{rgb}{0.000000,0.000000,0.000000}%
\pgfsetfillcolor{currentfill}%
\pgfsetlinewidth{0.803000pt}%
\definecolor{currentstroke}{rgb}{0.000000,0.000000,0.000000}%
\pgfsetstrokecolor{currentstroke}%
\pgfsetdash{}{0pt}%
\pgfsys@defobject{currentmarker}{\pgfqpoint{-0.048611in}{0.000000in}}{\pgfqpoint{0.000000in}{0.000000in}}{%
\pgfpathmoveto{\pgfqpoint{0.000000in}{0.000000in}}%
\pgfpathlineto{\pgfqpoint{-0.048611in}{0.000000in}}%
\pgfusepath{stroke,fill}%
}%
\begin{pgfscope}%
\pgfsys@transformshift{0.710919in}{1.857334in}%
\pgfsys@useobject{currentmarker}{}%
\end{pgfscope}%
\end{pgfscope}%
\begin{pgfscope}%
\pgftext[x=0.325694in,y=1.804572in,left,base]{\sffamily\fontsize{10.000000}{12.000000}\selectfont \(\displaystyle {10^{-1}}\)}%
\end{pgfscope}%
\begin{pgfscope}%
\pgfsetbuttcap%
\pgfsetroundjoin%
\definecolor{currentfill}{rgb}{0.000000,0.000000,0.000000}%
\pgfsetfillcolor{currentfill}%
\pgfsetlinewidth{0.803000pt}%
\definecolor{currentstroke}{rgb}{0.000000,0.000000,0.000000}%
\pgfsetstrokecolor{currentstroke}%
\pgfsetdash{}{0pt}%
\pgfsys@defobject{currentmarker}{\pgfqpoint{-0.048611in}{0.000000in}}{\pgfqpoint{0.000000in}{0.000000in}}{%
\pgfpathmoveto{\pgfqpoint{0.000000in}{0.000000in}}%
\pgfpathlineto{\pgfqpoint{-0.048611in}{0.000000in}}%
\pgfusepath{stroke,fill}%
}%
\begin{pgfscope}%
\pgfsys@transformshift{0.710919in}{2.927326in}%
\pgfsys@useobject{currentmarker}{}%
\end{pgfscope}%
\end{pgfscope}%
\begin{pgfscope}%
\pgftext[x=0.412500in,y=2.874564in,left,base]{\sffamily\fontsize{10.000000}{12.000000}\selectfont \(\displaystyle {10^{0}}\)}%
\end{pgfscope}%
\begin{pgfscope}%
\pgfsetbuttcap%
\pgfsetroundjoin%
\definecolor{currentfill}{rgb}{0.000000,0.000000,0.000000}%
\pgfsetfillcolor{currentfill}%
\pgfsetlinewidth{0.803000pt}%
\definecolor{currentstroke}{rgb}{0.000000,0.000000,0.000000}%
\pgfsetstrokecolor{currentstroke}%
\pgfsetdash{}{0pt}%
\pgfsys@defobject{currentmarker}{\pgfqpoint{-0.048611in}{0.000000in}}{\pgfqpoint{0.000000in}{0.000000in}}{%
\pgfpathmoveto{\pgfqpoint{0.000000in}{0.000000in}}%
\pgfpathlineto{\pgfqpoint{-0.048611in}{0.000000in}}%
\pgfusepath{stroke,fill}%
}%
\begin{pgfscope}%
\pgfsys@transformshift{0.710919in}{3.997317in}%
\pgfsys@useobject{currentmarker}{}%
\end{pgfscope}%
\end{pgfscope}%
\begin{pgfscope}%
\pgftext[x=0.412500in,y=3.944556in,left,base]{\sffamily\fontsize{10.000000}{12.000000}\selectfont \(\displaystyle {10^{1}}\)}%
\end{pgfscope}%
\begin{pgfscope}%
\pgfsetbuttcap%
\pgfsetroundjoin%
\definecolor{currentfill}{rgb}{0.000000,0.000000,0.000000}%
\pgfsetfillcolor{currentfill}%
\pgfsetlinewidth{0.602250pt}%
\definecolor{currentstroke}{rgb}{0.000000,0.000000,0.000000}%
\pgfsetstrokecolor{currentstroke}%
\pgfsetdash{}{0pt}%
\pgfsys@defobject{currentmarker}{\pgfqpoint{-0.027778in}{0.000000in}}{\pgfqpoint{0.000000in}{0.000000in}}{%
\pgfpathmoveto{\pgfqpoint{0.000000in}{0.000000in}}%
\pgfpathlineto{\pgfqpoint{-0.027778in}{0.000000in}}%
\pgfusepath{stroke,fill}%
}%
\begin{pgfscope}%
\pgfsys@transformshift{0.710919in}{0.549965in}%
\pgfsys@useobject{currentmarker}{}%
\end{pgfscope}%
\end{pgfscope}%
\begin{pgfscope}%
\pgfsetbuttcap%
\pgfsetroundjoin%
\definecolor{currentfill}{rgb}{0.000000,0.000000,0.000000}%
\pgfsetfillcolor{currentfill}%
\pgfsetlinewidth{0.602250pt}%
\definecolor{currentstroke}{rgb}{0.000000,0.000000,0.000000}%
\pgfsetstrokecolor{currentstroke}%
\pgfsetdash{}{0pt}%
\pgfsys@defobject{currentmarker}{\pgfqpoint{-0.027778in}{0.000000in}}{\pgfqpoint{0.000000in}{0.000000in}}{%
\pgfpathmoveto{\pgfqpoint{0.000000in}{0.000000in}}%
\pgfpathlineto{\pgfqpoint{-0.027778in}{0.000000in}}%
\pgfusepath{stroke,fill}%
}%
\begin{pgfscope}%
\pgfsys@transformshift{0.710919in}{0.621598in}%
\pgfsys@useobject{currentmarker}{}%
\end{pgfscope}%
\end{pgfscope}%
\begin{pgfscope}%
\pgfsetbuttcap%
\pgfsetroundjoin%
\definecolor{currentfill}{rgb}{0.000000,0.000000,0.000000}%
\pgfsetfillcolor{currentfill}%
\pgfsetlinewidth{0.602250pt}%
\definecolor{currentstroke}{rgb}{0.000000,0.000000,0.000000}%
\pgfsetstrokecolor{currentstroke}%
\pgfsetdash{}{0pt}%
\pgfsys@defobject{currentmarker}{\pgfqpoint{-0.027778in}{0.000000in}}{\pgfqpoint{0.000000in}{0.000000in}}{%
\pgfpathmoveto{\pgfqpoint{0.000000in}{0.000000in}}%
\pgfpathlineto{\pgfqpoint{-0.027778in}{0.000000in}}%
\pgfusepath{stroke,fill}%
}%
\begin{pgfscope}%
\pgfsys@transformshift{0.710919in}{0.683649in}%
\pgfsys@useobject{currentmarker}{}%
\end{pgfscope}%
\end{pgfscope}%
\begin{pgfscope}%
\pgfsetbuttcap%
\pgfsetroundjoin%
\definecolor{currentfill}{rgb}{0.000000,0.000000,0.000000}%
\pgfsetfillcolor{currentfill}%
\pgfsetlinewidth{0.602250pt}%
\definecolor{currentstroke}{rgb}{0.000000,0.000000,0.000000}%
\pgfsetstrokecolor{currentstroke}%
\pgfsetdash{}{0pt}%
\pgfsys@defobject{currentmarker}{\pgfqpoint{-0.027778in}{0.000000in}}{\pgfqpoint{0.000000in}{0.000000in}}{%
\pgfpathmoveto{\pgfqpoint{0.000000in}{0.000000in}}%
\pgfpathlineto{\pgfqpoint{-0.027778in}{0.000000in}}%
\pgfusepath{stroke,fill}%
}%
\begin{pgfscope}%
\pgfsys@transformshift{0.710919in}{0.738381in}%
\pgfsys@useobject{currentmarker}{}%
\end{pgfscope}%
\end{pgfscope}%
\begin{pgfscope}%
\pgfsetbuttcap%
\pgfsetroundjoin%
\definecolor{currentfill}{rgb}{0.000000,0.000000,0.000000}%
\pgfsetfillcolor{currentfill}%
\pgfsetlinewidth{0.602250pt}%
\definecolor{currentstroke}{rgb}{0.000000,0.000000,0.000000}%
\pgfsetstrokecolor{currentstroke}%
\pgfsetdash{}{0pt}%
\pgfsys@defobject{currentmarker}{\pgfqpoint{-0.027778in}{0.000000in}}{\pgfqpoint{0.000000in}{0.000000in}}{%
\pgfpathmoveto{\pgfqpoint{0.000000in}{0.000000in}}%
\pgfpathlineto{\pgfqpoint{-0.027778in}{0.000000in}}%
\pgfusepath{stroke,fill}%
}%
\begin{pgfscope}%
\pgfsys@transformshift{0.710919in}{1.109441in}%
\pgfsys@useobject{currentmarker}{}%
\end{pgfscope}%
\end{pgfscope}%
\begin{pgfscope}%
\pgfsetbuttcap%
\pgfsetroundjoin%
\definecolor{currentfill}{rgb}{0.000000,0.000000,0.000000}%
\pgfsetfillcolor{currentfill}%
\pgfsetlinewidth{0.602250pt}%
\definecolor{currentstroke}{rgb}{0.000000,0.000000,0.000000}%
\pgfsetstrokecolor{currentstroke}%
\pgfsetdash{}{0pt}%
\pgfsys@defobject{currentmarker}{\pgfqpoint{-0.027778in}{0.000000in}}{\pgfqpoint{0.000000in}{0.000000in}}{%
\pgfpathmoveto{\pgfqpoint{0.000000in}{0.000000in}}%
\pgfpathlineto{\pgfqpoint{-0.027778in}{0.000000in}}%
\pgfusepath{stroke,fill}%
}%
\begin{pgfscope}%
\pgfsys@transformshift{0.710919in}{1.297857in}%
\pgfsys@useobject{currentmarker}{}%
\end{pgfscope}%
\end{pgfscope}%
\begin{pgfscope}%
\pgfsetbuttcap%
\pgfsetroundjoin%
\definecolor{currentfill}{rgb}{0.000000,0.000000,0.000000}%
\pgfsetfillcolor{currentfill}%
\pgfsetlinewidth{0.602250pt}%
\definecolor{currentstroke}{rgb}{0.000000,0.000000,0.000000}%
\pgfsetstrokecolor{currentstroke}%
\pgfsetdash{}{0pt}%
\pgfsys@defobject{currentmarker}{\pgfqpoint{-0.027778in}{0.000000in}}{\pgfqpoint{0.000000in}{0.000000in}}{%
\pgfpathmoveto{\pgfqpoint{0.000000in}{0.000000in}}%
\pgfpathlineto{\pgfqpoint{-0.027778in}{0.000000in}}%
\pgfusepath{stroke,fill}%
}%
\begin{pgfscope}%
\pgfsys@transformshift{0.710919in}{1.431541in}%
\pgfsys@useobject{currentmarker}{}%
\end{pgfscope}%
\end{pgfscope}%
\begin{pgfscope}%
\pgfsetbuttcap%
\pgfsetroundjoin%
\definecolor{currentfill}{rgb}{0.000000,0.000000,0.000000}%
\pgfsetfillcolor{currentfill}%
\pgfsetlinewidth{0.602250pt}%
\definecolor{currentstroke}{rgb}{0.000000,0.000000,0.000000}%
\pgfsetstrokecolor{currentstroke}%
\pgfsetdash{}{0pt}%
\pgfsys@defobject{currentmarker}{\pgfqpoint{-0.027778in}{0.000000in}}{\pgfqpoint{0.000000in}{0.000000in}}{%
\pgfpathmoveto{\pgfqpoint{0.000000in}{0.000000in}}%
\pgfpathlineto{\pgfqpoint{-0.027778in}{0.000000in}}%
\pgfusepath{stroke,fill}%
}%
\begin{pgfscope}%
\pgfsys@transformshift{0.710919in}{1.535234in}%
\pgfsys@useobject{currentmarker}{}%
\end{pgfscope}%
\end{pgfscope}%
\begin{pgfscope}%
\pgfsetbuttcap%
\pgfsetroundjoin%
\definecolor{currentfill}{rgb}{0.000000,0.000000,0.000000}%
\pgfsetfillcolor{currentfill}%
\pgfsetlinewidth{0.602250pt}%
\definecolor{currentstroke}{rgb}{0.000000,0.000000,0.000000}%
\pgfsetstrokecolor{currentstroke}%
\pgfsetdash{}{0pt}%
\pgfsys@defobject{currentmarker}{\pgfqpoint{-0.027778in}{0.000000in}}{\pgfqpoint{0.000000in}{0.000000in}}{%
\pgfpathmoveto{\pgfqpoint{0.000000in}{0.000000in}}%
\pgfpathlineto{\pgfqpoint{-0.027778in}{0.000000in}}%
\pgfusepath{stroke,fill}%
}%
\begin{pgfscope}%
\pgfsys@transformshift{0.710919in}{1.619957in}%
\pgfsys@useobject{currentmarker}{}%
\end{pgfscope}%
\end{pgfscope}%
\begin{pgfscope}%
\pgfsetbuttcap%
\pgfsetroundjoin%
\definecolor{currentfill}{rgb}{0.000000,0.000000,0.000000}%
\pgfsetfillcolor{currentfill}%
\pgfsetlinewidth{0.602250pt}%
\definecolor{currentstroke}{rgb}{0.000000,0.000000,0.000000}%
\pgfsetstrokecolor{currentstroke}%
\pgfsetdash{}{0pt}%
\pgfsys@defobject{currentmarker}{\pgfqpoint{-0.027778in}{0.000000in}}{\pgfqpoint{0.000000in}{0.000000in}}{%
\pgfpathmoveto{\pgfqpoint{0.000000in}{0.000000in}}%
\pgfpathlineto{\pgfqpoint{-0.027778in}{0.000000in}}%
\pgfusepath{stroke,fill}%
}%
\begin{pgfscope}%
\pgfsys@transformshift{0.710919in}{1.691590in}%
\pgfsys@useobject{currentmarker}{}%
\end{pgfscope}%
\end{pgfscope}%
\begin{pgfscope}%
\pgfsetbuttcap%
\pgfsetroundjoin%
\definecolor{currentfill}{rgb}{0.000000,0.000000,0.000000}%
\pgfsetfillcolor{currentfill}%
\pgfsetlinewidth{0.602250pt}%
\definecolor{currentstroke}{rgb}{0.000000,0.000000,0.000000}%
\pgfsetstrokecolor{currentstroke}%
\pgfsetdash{}{0pt}%
\pgfsys@defobject{currentmarker}{\pgfqpoint{-0.027778in}{0.000000in}}{\pgfqpoint{0.000000in}{0.000000in}}{%
\pgfpathmoveto{\pgfqpoint{0.000000in}{0.000000in}}%
\pgfpathlineto{\pgfqpoint{-0.027778in}{0.000000in}}%
\pgfusepath{stroke,fill}%
}%
\begin{pgfscope}%
\pgfsys@transformshift{0.710919in}{1.753641in}%
\pgfsys@useobject{currentmarker}{}%
\end{pgfscope}%
\end{pgfscope}%
\begin{pgfscope}%
\pgfsetbuttcap%
\pgfsetroundjoin%
\definecolor{currentfill}{rgb}{0.000000,0.000000,0.000000}%
\pgfsetfillcolor{currentfill}%
\pgfsetlinewidth{0.602250pt}%
\definecolor{currentstroke}{rgb}{0.000000,0.000000,0.000000}%
\pgfsetstrokecolor{currentstroke}%
\pgfsetdash{}{0pt}%
\pgfsys@defobject{currentmarker}{\pgfqpoint{-0.027778in}{0.000000in}}{\pgfqpoint{0.000000in}{0.000000in}}{%
\pgfpathmoveto{\pgfqpoint{0.000000in}{0.000000in}}%
\pgfpathlineto{\pgfqpoint{-0.027778in}{0.000000in}}%
\pgfusepath{stroke,fill}%
}%
\begin{pgfscope}%
\pgfsys@transformshift{0.710919in}{1.808373in}%
\pgfsys@useobject{currentmarker}{}%
\end{pgfscope}%
\end{pgfscope}%
\begin{pgfscope}%
\pgfsetbuttcap%
\pgfsetroundjoin%
\definecolor{currentfill}{rgb}{0.000000,0.000000,0.000000}%
\pgfsetfillcolor{currentfill}%
\pgfsetlinewidth{0.602250pt}%
\definecolor{currentstroke}{rgb}{0.000000,0.000000,0.000000}%
\pgfsetstrokecolor{currentstroke}%
\pgfsetdash{}{0pt}%
\pgfsys@defobject{currentmarker}{\pgfqpoint{-0.027778in}{0.000000in}}{\pgfqpoint{0.000000in}{0.000000in}}{%
\pgfpathmoveto{\pgfqpoint{0.000000in}{0.000000in}}%
\pgfpathlineto{\pgfqpoint{-0.027778in}{0.000000in}}%
\pgfusepath{stroke,fill}%
}%
\begin{pgfscope}%
\pgfsys@transformshift{0.710919in}{2.179433in}%
\pgfsys@useobject{currentmarker}{}%
\end{pgfscope}%
\end{pgfscope}%
\begin{pgfscope}%
\pgfsetbuttcap%
\pgfsetroundjoin%
\definecolor{currentfill}{rgb}{0.000000,0.000000,0.000000}%
\pgfsetfillcolor{currentfill}%
\pgfsetlinewidth{0.602250pt}%
\definecolor{currentstroke}{rgb}{0.000000,0.000000,0.000000}%
\pgfsetstrokecolor{currentstroke}%
\pgfsetdash{}{0pt}%
\pgfsys@defobject{currentmarker}{\pgfqpoint{-0.027778in}{0.000000in}}{\pgfqpoint{0.000000in}{0.000000in}}{%
\pgfpathmoveto{\pgfqpoint{0.000000in}{0.000000in}}%
\pgfpathlineto{\pgfqpoint{-0.027778in}{0.000000in}}%
\pgfusepath{stroke,fill}%
}%
\begin{pgfscope}%
\pgfsys@transformshift{0.710919in}{2.367849in}%
\pgfsys@useobject{currentmarker}{}%
\end{pgfscope}%
\end{pgfscope}%
\begin{pgfscope}%
\pgfsetbuttcap%
\pgfsetroundjoin%
\definecolor{currentfill}{rgb}{0.000000,0.000000,0.000000}%
\pgfsetfillcolor{currentfill}%
\pgfsetlinewidth{0.602250pt}%
\definecolor{currentstroke}{rgb}{0.000000,0.000000,0.000000}%
\pgfsetstrokecolor{currentstroke}%
\pgfsetdash{}{0pt}%
\pgfsys@defobject{currentmarker}{\pgfqpoint{-0.027778in}{0.000000in}}{\pgfqpoint{0.000000in}{0.000000in}}{%
\pgfpathmoveto{\pgfqpoint{0.000000in}{0.000000in}}%
\pgfpathlineto{\pgfqpoint{-0.027778in}{0.000000in}}%
\pgfusepath{stroke,fill}%
}%
\begin{pgfscope}%
\pgfsys@transformshift{0.710919in}{2.501533in}%
\pgfsys@useobject{currentmarker}{}%
\end{pgfscope}%
\end{pgfscope}%
\begin{pgfscope}%
\pgfsetbuttcap%
\pgfsetroundjoin%
\definecolor{currentfill}{rgb}{0.000000,0.000000,0.000000}%
\pgfsetfillcolor{currentfill}%
\pgfsetlinewidth{0.602250pt}%
\definecolor{currentstroke}{rgb}{0.000000,0.000000,0.000000}%
\pgfsetstrokecolor{currentstroke}%
\pgfsetdash{}{0pt}%
\pgfsys@defobject{currentmarker}{\pgfqpoint{-0.027778in}{0.000000in}}{\pgfqpoint{0.000000in}{0.000000in}}{%
\pgfpathmoveto{\pgfqpoint{0.000000in}{0.000000in}}%
\pgfpathlineto{\pgfqpoint{-0.027778in}{0.000000in}}%
\pgfusepath{stroke,fill}%
}%
\begin{pgfscope}%
\pgfsys@transformshift{0.710919in}{2.605226in}%
\pgfsys@useobject{currentmarker}{}%
\end{pgfscope}%
\end{pgfscope}%
\begin{pgfscope}%
\pgfsetbuttcap%
\pgfsetroundjoin%
\definecolor{currentfill}{rgb}{0.000000,0.000000,0.000000}%
\pgfsetfillcolor{currentfill}%
\pgfsetlinewidth{0.602250pt}%
\definecolor{currentstroke}{rgb}{0.000000,0.000000,0.000000}%
\pgfsetstrokecolor{currentstroke}%
\pgfsetdash{}{0pt}%
\pgfsys@defobject{currentmarker}{\pgfqpoint{-0.027778in}{0.000000in}}{\pgfqpoint{0.000000in}{0.000000in}}{%
\pgfpathmoveto{\pgfqpoint{0.000000in}{0.000000in}}%
\pgfpathlineto{\pgfqpoint{-0.027778in}{0.000000in}}%
\pgfusepath{stroke,fill}%
}%
\begin{pgfscope}%
\pgfsys@transformshift{0.710919in}{2.689949in}%
\pgfsys@useobject{currentmarker}{}%
\end{pgfscope}%
\end{pgfscope}%
\begin{pgfscope}%
\pgfsetbuttcap%
\pgfsetroundjoin%
\definecolor{currentfill}{rgb}{0.000000,0.000000,0.000000}%
\pgfsetfillcolor{currentfill}%
\pgfsetlinewidth{0.602250pt}%
\definecolor{currentstroke}{rgb}{0.000000,0.000000,0.000000}%
\pgfsetstrokecolor{currentstroke}%
\pgfsetdash{}{0pt}%
\pgfsys@defobject{currentmarker}{\pgfqpoint{-0.027778in}{0.000000in}}{\pgfqpoint{0.000000in}{0.000000in}}{%
\pgfpathmoveto{\pgfqpoint{0.000000in}{0.000000in}}%
\pgfpathlineto{\pgfqpoint{-0.027778in}{0.000000in}}%
\pgfusepath{stroke,fill}%
}%
\begin{pgfscope}%
\pgfsys@transformshift{0.710919in}{2.761582in}%
\pgfsys@useobject{currentmarker}{}%
\end{pgfscope}%
\end{pgfscope}%
\begin{pgfscope}%
\pgfsetbuttcap%
\pgfsetroundjoin%
\definecolor{currentfill}{rgb}{0.000000,0.000000,0.000000}%
\pgfsetfillcolor{currentfill}%
\pgfsetlinewidth{0.602250pt}%
\definecolor{currentstroke}{rgb}{0.000000,0.000000,0.000000}%
\pgfsetstrokecolor{currentstroke}%
\pgfsetdash{}{0pt}%
\pgfsys@defobject{currentmarker}{\pgfqpoint{-0.027778in}{0.000000in}}{\pgfqpoint{0.000000in}{0.000000in}}{%
\pgfpathmoveto{\pgfqpoint{0.000000in}{0.000000in}}%
\pgfpathlineto{\pgfqpoint{-0.027778in}{0.000000in}}%
\pgfusepath{stroke,fill}%
}%
\begin{pgfscope}%
\pgfsys@transformshift{0.710919in}{2.823633in}%
\pgfsys@useobject{currentmarker}{}%
\end{pgfscope}%
\end{pgfscope}%
\begin{pgfscope}%
\pgfsetbuttcap%
\pgfsetroundjoin%
\definecolor{currentfill}{rgb}{0.000000,0.000000,0.000000}%
\pgfsetfillcolor{currentfill}%
\pgfsetlinewidth{0.602250pt}%
\definecolor{currentstroke}{rgb}{0.000000,0.000000,0.000000}%
\pgfsetstrokecolor{currentstroke}%
\pgfsetdash{}{0pt}%
\pgfsys@defobject{currentmarker}{\pgfqpoint{-0.027778in}{0.000000in}}{\pgfqpoint{0.000000in}{0.000000in}}{%
\pgfpathmoveto{\pgfqpoint{0.000000in}{0.000000in}}%
\pgfpathlineto{\pgfqpoint{-0.027778in}{0.000000in}}%
\pgfusepath{stroke,fill}%
}%
\begin{pgfscope}%
\pgfsys@transformshift{0.710919in}{2.878365in}%
\pgfsys@useobject{currentmarker}{}%
\end{pgfscope}%
\end{pgfscope}%
\begin{pgfscope}%
\pgfsetbuttcap%
\pgfsetroundjoin%
\definecolor{currentfill}{rgb}{0.000000,0.000000,0.000000}%
\pgfsetfillcolor{currentfill}%
\pgfsetlinewidth{0.602250pt}%
\definecolor{currentstroke}{rgb}{0.000000,0.000000,0.000000}%
\pgfsetstrokecolor{currentstroke}%
\pgfsetdash{}{0pt}%
\pgfsys@defobject{currentmarker}{\pgfqpoint{-0.027778in}{0.000000in}}{\pgfqpoint{0.000000in}{0.000000in}}{%
\pgfpathmoveto{\pgfqpoint{0.000000in}{0.000000in}}%
\pgfpathlineto{\pgfqpoint{-0.027778in}{0.000000in}}%
\pgfusepath{stroke,fill}%
}%
\begin{pgfscope}%
\pgfsys@transformshift{0.710919in}{3.249425in}%
\pgfsys@useobject{currentmarker}{}%
\end{pgfscope}%
\end{pgfscope}%
\begin{pgfscope}%
\pgfsetbuttcap%
\pgfsetroundjoin%
\definecolor{currentfill}{rgb}{0.000000,0.000000,0.000000}%
\pgfsetfillcolor{currentfill}%
\pgfsetlinewidth{0.602250pt}%
\definecolor{currentstroke}{rgb}{0.000000,0.000000,0.000000}%
\pgfsetstrokecolor{currentstroke}%
\pgfsetdash{}{0pt}%
\pgfsys@defobject{currentmarker}{\pgfqpoint{-0.027778in}{0.000000in}}{\pgfqpoint{0.000000in}{0.000000in}}{%
\pgfpathmoveto{\pgfqpoint{0.000000in}{0.000000in}}%
\pgfpathlineto{\pgfqpoint{-0.027778in}{0.000000in}}%
\pgfusepath{stroke,fill}%
}%
\begin{pgfscope}%
\pgfsys@transformshift{0.710919in}{3.437841in}%
\pgfsys@useobject{currentmarker}{}%
\end{pgfscope}%
\end{pgfscope}%
\begin{pgfscope}%
\pgfsetbuttcap%
\pgfsetroundjoin%
\definecolor{currentfill}{rgb}{0.000000,0.000000,0.000000}%
\pgfsetfillcolor{currentfill}%
\pgfsetlinewidth{0.602250pt}%
\definecolor{currentstroke}{rgb}{0.000000,0.000000,0.000000}%
\pgfsetstrokecolor{currentstroke}%
\pgfsetdash{}{0pt}%
\pgfsys@defobject{currentmarker}{\pgfqpoint{-0.027778in}{0.000000in}}{\pgfqpoint{0.000000in}{0.000000in}}{%
\pgfpathmoveto{\pgfqpoint{0.000000in}{0.000000in}}%
\pgfpathlineto{\pgfqpoint{-0.027778in}{0.000000in}}%
\pgfusepath{stroke,fill}%
}%
\begin{pgfscope}%
\pgfsys@transformshift{0.710919in}{3.571525in}%
\pgfsys@useobject{currentmarker}{}%
\end{pgfscope}%
\end{pgfscope}%
\begin{pgfscope}%
\pgfsetbuttcap%
\pgfsetroundjoin%
\definecolor{currentfill}{rgb}{0.000000,0.000000,0.000000}%
\pgfsetfillcolor{currentfill}%
\pgfsetlinewidth{0.602250pt}%
\definecolor{currentstroke}{rgb}{0.000000,0.000000,0.000000}%
\pgfsetstrokecolor{currentstroke}%
\pgfsetdash{}{0pt}%
\pgfsys@defobject{currentmarker}{\pgfqpoint{-0.027778in}{0.000000in}}{\pgfqpoint{0.000000in}{0.000000in}}{%
\pgfpathmoveto{\pgfqpoint{0.000000in}{0.000000in}}%
\pgfpathlineto{\pgfqpoint{-0.027778in}{0.000000in}}%
\pgfusepath{stroke,fill}%
}%
\begin{pgfscope}%
\pgfsys@transformshift{0.710919in}{3.675218in}%
\pgfsys@useobject{currentmarker}{}%
\end{pgfscope}%
\end{pgfscope}%
\begin{pgfscope}%
\pgfsetbuttcap%
\pgfsetroundjoin%
\definecolor{currentfill}{rgb}{0.000000,0.000000,0.000000}%
\pgfsetfillcolor{currentfill}%
\pgfsetlinewidth{0.602250pt}%
\definecolor{currentstroke}{rgb}{0.000000,0.000000,0.000000}%
\pgfsetstrokecolor{currentstroke}%
\pgfsetdash{}{0pt}%
\pgfsys@defobject{currentmarker}{\pgfqpoint{-0.027778in}{0.000000in}}{\pgfqpoint{0.000000in}{0.000000in}}{%
\pgfpathmoveto{\pgfqpoint{0.000000in}{0.000000in}}%
\pgfpathlineto{\pgfqpoint{-0.027778in}{0.000000in}}%
\pgfusepath{stroke,fill}%
}%
\begin{pgfscope}%
\pgfsys@transformshift{0.710919in}{3.759941in}%
\pgfsys@useobject{currentmarker}{}%
\end{pgfscope}%
\end{pgfscope}%
\begin{pgfscope}%
\pgfsetbuttcap%
\pgfsetroundjoin%
\definecolor{currentfill}{rgb}{0.000000,0.000000,0.000000}%
\pgfsetfillcolor{currentfill}%
\pgfsetlinewidth{0.602250pt}%
\definecolor{currentstroke}{rgb}{0.000000,0.000000,0.000000}%
\pgfsetstrokecolor{currentstroke}%
\pgfsetdash{}{0pt}%
\pgfsys@defobject{currentmarker}{\pgfqpoint{-0.027778in}{0.000000in}}{\pgfqpoint{0.000000in}{0.000000in}}{%
\pgfpathmoveto{\pgfqpoint{0.000000in}{0.000000in}}%
\pgfpathlineto{\pgfqpoint{-0.027778in}{0.000000in}}%
\pgfusepath{stroke,fill}%
}%
\begin{pgfscope}%
\pgfsys@transformshift{0.710919in}{3.831574in}%
\pgfsys@useobject{currentmarker}{}%
\end{pgfscope}%
\end{pgfscope}%
\begin{pgfscope}%
\pgfsetbuttcap%
\pgfsetroundjoin%
\definecolor{currentfill}{rgb}{0.000000,0.000000,0.000000}%
\pgfsetfillcolor{currentfill}%
\pgfsetlinewidth{0.602250pt}%
\definecolor{currentstroke}{rgb}{0.000000,0.000000,0.000000}%
\pgfsetstrokecolor{currentstroke}%
\pgfsetdash{}{0pt}%
\pgfsys@defobject{currentmarker}{\pgfqpoint{-0.027778in}{0.000000in}}{\pgfqpoint{0.000000in}{0.000000in}}{%
\pgfpathmoveto{\pgfqpoint{0.000000in}{0.000000in}}%
\pgfpathlineto{\pgfqpoint{-0.027778in}{0.000000in}}%
\pgfusepath{stroke,fill}%
}%
\begin{pgfscope}%
\pgfsys@transformshift{0.710919in}{3.893625in}%
\pgfsys@useobject{currentmarker}{}%
\end{pgfscope}%
\end{pgfscope}%
\begin{pgfscope}%
\pgfsetbuttcap%
\pgfsetroundjoin%
\definecolor{currentfill}{rgb}{0.000000,0.000000,0.000000}%
\pgfsetfillcolor{currentfill}%
\pgfsetlinewidth{0.602250pt}%
\definecolor{currentstroke}{rgb}{0.000000,0.000000,0.000000}%
\pgfsetstrokecolor{currentstroke}%
\pgfsetdash{}{0pt}%
\pgfsys@defobject{currentmarker}{\pgfqpoint{-0.027778in}{0.000000in}}{\pgfqpoint{0.000000in}{0.000000in}}{%
\pgfpathmoveto{\pgfqpoint{0.000000in}{0.000000in}}%
\pgfpathlineto{\pgfqpoint{-0.027778in}{0.000000in}}%
\pgfusepath{stroke,fill}%
}%
\begin{pgfscope}%
\pgfsys@transformshift{0.710919in}{3.948357in}%
\pgfsys@useobject{currentmarker}{}%
\end{pgfscope}%
\end{pgfscope}%
\begin{pgfscope}%
\pgfsetbuttcap%
\pgfsetroundjoin%
\definecolor{currentfill}{rgb}{0.000000,0.000000,0.000000}%
\pgfsetfillcolor{currentfill}%
\pgfsetlinewidth{0.602250pt}%
\definecolor{currentstroke}{rgb}{0.000000,0.000000,0.000000}%
\pgfsetstrokecolor{currentstroke}%
\pgfsetdash{}{0pt}%
\pgfsys@defobject{currentmarker}{\pgfqpoint{-0.027778in}{0.000000in}}{\pgfqpoint{0.000000in}{0.000000in}}{%
\pgfpathmoveto{\pgfqpoint{0.000000in}{0.000000in}}%
\pgfpathlineto{\pgfqpoint{-0.027778in}{0.000000in}}%
\pgfusepath{stroke,fill}%
}%
\begin{pgfscope}%
\pgfsys@transformshift{0.710919in}{4.319417in}%
\pgfsys@useobject{currentmarker}{}%
\end{pgfscope}%
\end{pgfscope}%
\begin{pgfscope}%
\pgftext[x=0.270139in,y=2.504714in,,bottom,rotate=90.000000]{\sffamily\fontsize{10.000000}{12.000000}\selectfont \(\displaystyle ||\widetilde{K}-K||_{\infty}\) (Error)}%
\end{pgfscope}%
\begin{pgfscope}%
\pgfpathrectangle{\pgfqpoint{0.710919in}{0.521603in}}{\pgfqpoint{5.377460in}{3.966222in}} %
\pgfusepath{clip}%
\pgfsetbuttcap%
\pgfsetroundjoin%
\pgfsetlinewidth{1.505625pt}%
\definecolor{currentstroke}{rgb}{0.121569,0.466667,0.705882}%
\pgfsetstrokecolor{currentstroke}%
\pgfsetdash{}{0pt}%
\pgfpathmoveto{\pgfqpoint{0.955349in}{3.338181in}}%
\pgfpathlineto{\pgfqpoint{0.955349in}{3.358524in}}%
\pgfusepath{stroke}%
\end{pgfscope}%
\begin{pgfscope}%
\pgfpathrectangle{\pgfqpoint{0.710919in}{0.521603in}}{\pgfqpoint{5.377460in}{3.966222in}} %
\pgfusepath{clip}%
\pgfsetbuttcap%
\pgfsetroundjoin%
\pgfsetlinewidth{1.505625pt}%
\definecolor{currentstroke}{rgb}{0.121569,0.466667,0.705882}%
\pgfsetstrokecolor{currentstroke}%
\pgfsetdash{}{0pt}%
\pgfpathmoveto{\pgfqpoint{1.421840in}{3.129864in}}%
\pgfpathlineto{\pgfqpoint{1.421840in}{3.157661in}}%
\pgfusepath{stroke}%
\end{pgfscope}%
\begin{pgfscope}%
\pgfpathrectangle{\pgfqpoint{0.710919in}{0.521603in}}{\pgfqpoint{5.377460in}{3.966222in}} %
\pgfusepath{clip}%
\pgfsetbuttcap%
\pgfsetroundjoin%
\pgfsetlinewidth{1.505625pt}%
\definecolor{currentstroke}{rgb}{0.121569,0.466667,0.705882}%
\pgfsetstrokecolor{currentstroke}%
\pgfsetdash{}{0pt}%
\pgfpathmoveto{\pgfqpoint{2.010486in}{2.856557in}}%
\pgfpathlineto{\pgfqpoint{2.010486in}{2.880982in}}%
\pgfusepath{stroke}%
\end{pgfscope}%
\begin{pgfscope}%
\pgfpathrectangle{\pgfqpoint{0.710919in}{0.521603in}}{\pgfqpoint{5.377460in}{3.966222in}} %
\pgfusepath{clip}%
\pgfsetbuttcap%
\pgfsetroundjoin%
\pgfsetlinewidth{1.505625pt}%
\definecolor{currentstroke}{rgb}{0.121569,0.466667,0.705882}%
\pgfsetstrokecolor{currentstroke}%
\pgfsetdash{}{0pt}%
\pgfpathmoveto{\pgfqpoint{2.581061in}{2.532529in}}%
\pgfpathlineto{\pgfqpoint{2.581061in}{2.599312in}}%
\pgfusepath{stroke}%
\end{pgfscope}%
\begin{pgfscope}%
\pgfpathrectangle{\pgfqpoint{0.710919in}{0.521603in}}{\pgfqpoint{5.377460in}{3.966222in}} %
\pgfusepath{clip}%
\pgfsetbuttcap%
\pgfsetroundjoin%
\pgfsetlinewidth{1.505625pt}%
\definecolor{currentstroke}{rgb}{0.121569,0.466667,0.705882}%
\pgfsetstrokecolor{currentstroke}%
\pgfsetdash{}{0pt}%
\pgfpathmoveto{\pgfqpoint{3.125993in}{2.126826in}}%
\pgfpathlineto{\pgfqpoint{3.125993in}{2.205354in}}%
\pgfusepath{stroke}%
\end{pgfscope}%
\begin{pgfscope}%
\pgfpathrectangle{\pgfqpoint{0.710919in}{0.521603in}}{\pgfqpoint{5.377460in}{3.966222in}} %
\pgfusepath{clip}%
\pgfsetbuttcap%
\pgfsetroundjoin%
\pgfsetlinewidth{1.505625pt}%
\definecolor{currentstroke}{rgb}{0.121569,0.466667,0.705882}%
\pgfsetstrokecolor{currentstroke}%
\pgfsetdash{}{0pt}%
\pgfpathmoveto{\pgfqpoint{3.670895in}{1.879701in}}%
\pgfpathlineto{\pgfqpoint{3.670895in}{1.907167in}}%
\pgfusepath{stroke}%
\end{pgfscope}%
\begin{pgfscope}%
\pgfpathrectangle{\pgfqpoint{0.710919in}{0.521603in}}{\pgfqpoint{5.377460in}{3.966222in}} %
\pgfusepath{clip}%
\pgfsetbuttcap%
\pgfsetroundjoin%
\pgfsetlinewidth{1.505625pt}%
\definecolor{currentstroke}{rgb}{0.121569,0.466667,0.705882}%
\pgfsetstrokecolor{currentstroke}%
\pgfsetdash{}{0pt}%
\pgfpathmoveto{\pgfqpoint{4.214330in}{1.623400in}}%
\pgfpathlineto{\pgfqpoint{4.214330in}{1.664892in}}%
\pgfusepath{stroke}%
\end{pgfscope}%
\begin{pgfscope}%
\pgfpathrectangle{\pgfqpoint{0.710919in}{0.521603in}}{\pgfqpoint{5.377460in}{3.966222in}} %
\pgfusepath{clip}%
\pgfsetbuttcap%
\pgfsetroundjoin%
\pgfsetlinewidth{1.505625pt}%
\definecolor{currentstroke}{rgb}{0.121569,0.466667,0.705882}%
\pgfsetstrokecolor{currentstroke}%
\pgfsetdash{}{0pt}%
\pgfpathmoveto{\pgfqpoint{4.757559in}{1.429693in}}%
\pgfpathlineto{\pgfqpoint{4.757559in}{1.489518in}}%
\pgfusepath{stroke}%
\end{pgfscope}%
\begin{pgfscope}%
\pgfpathrectangle{\pgfqpoint{0.710919in}{0.521603in}}{\pgfqpoint{5.377460in}{3.966222in}} %
\pgfusepath{clip}%
\pgfsetbuttcap%
\pgfsetroundjoin%
\pgfsetlinewidth{1.505625pt}%
\definecolor{currentstroke}{rgb}{0.121569,0.466667,0.705882}%
\pgfsetstrokecolor{currentstroke}%
\pgfsetdash{}{0pt}%
\pgfpathmoveto{\pgfqpoint{5.300763in}{0.967314in}}%
\pgfpathlineto{\pgfqpoint{5.300763in}{1.020584in}}%
\pgfusepath{stroke}%
\end{pgfscope}%
\begin{pgfscope}%
\pgfpathrectangle{\pgfqpoint{0.710919in}{0.521603in}}{\pgfqpoint{5.377460in}{3.966222in}} %
\pgfusepath{clip}%
\pgfsetbuttcap%
\pgfsetroundjoin%
\pgfsetlinewidth{1.505625pt}%
\definecolor{currentstroke}{rgb}{0.121569,0.466667,0.705882}%
\pgfsetstrokecolor{currentstroke}%
\pgfsetdash{}{0pt}%
\pgfpathmoveto{\pgfqpoint{5.843950in}{0.701886in}}%
\pgfpathlineto{\pgfqpoint{5.843950in}{0.744250in}}%
\pgfusepath{stroke}%
\end{pgfscope}%
\begin{pgfscope}%
\pgfpathrectangle{\pgfqpoint{0.710919in}{0.521603in}}{\pgfqpoint{5.377460in}{3.966222in}} %
\pgfusepath{clip}%
\pgfsetbuttcap%
\pgfsetroundjoin%
\pgfsetlinewidth{1.505625pt}%
\definecolor{currentstroke}{rgb}{1.000000,0.498039,0.054902}%
\pgfsetstrokecolor{currentstroke}%
\pgfsetdash{}{0pt}%
\pgfpathmoveto{\pgfqpoint{0.955349in}{4.280798in}}%
\pgfpathlineto{\pgfqpoint{0.955349in}{4.286142in}}%
\pgfusepath{stroke}%
\end{pgfscope}%
\begin{pgfscope}%
\pgfpathrectangle{\pgfqpoint{0.710919in}{0.521603in}}{\pgfqpoint{5.377460in}{3.966222in}} %
\pgfusepath{clip}%
\pgfsetbuttcap%
\pgfsetroundjoin%
\pgfsetlinewidth{1.505625pt}%
\definecolor{currentstroke}{rgb}{1.000000,0.498039,0.054902}%
\pgfsetstrokecolor{currentstroke}%
\pgfsetdash{}{0pt}%
\pgfpathmoveto{\pgfqpoint{1.421840in}{3.838338in}}%
\pgfpathlineto{\pgfqpoint{1.421840in}{3.869105in}}%
\pgfusepath{stroke}%
\end{pgfscope}%
\begin{pgfscope}%
\pgfpathrectangle{\pgfqpoint{0.710919in}{0.521603in}}{\pgfqpoint{5.377460in}{3.966222in}} %
\pgfusepath{clip}%
\pgfsetbuttcap%
\pgfsetroundjoin%
\pgfsetlinewidth{1.505625pt}%
\definecolor{currentstroke}{rgb}{1.000000,0.498039,0.054902}%
\pgfsetstrokecolor{currentstroke}%
\pgfsetdash{}{0pt}%
\pgfpathmoveto{\pgfqpoint{2.010486in}{3.401051in}}%
\pgfpathlineto{\pgfqpoint{2.010486in}{3.448620in}}%
\pgfusepath{stroke}%
\end{pgfscope}%
\begin{pgfscope}%
\pgfpathrectangle{\pgfqpoint{0.710919in}{0.521603in}}{\pgfqpoint{5.377460in}{3.966222in}} %
\pgfusepath{clip}%
\pgfsetbuttcap%
\pgfsetroundjoin%
\pgfsetlinewidth{1.505625pt}%
\definecolor{currentstroke}{rgb}{1.000000,0.498039,0.054902}%
\pgfsetstrokecolor{currentstroke}%
\pgfsetdash{}{0pt}%
\pgfpathmoveto{\pgfqpoint{2.581061in}{2.996009in}}%
\pgfpathlineto{\pgfqpoint{2.581061in}{3.041984in}}%
\pgfusepath{stroke}%
\end{pgfscope}%
\begin{pgfscope}%
\pgfpathrectangle{\pgfqpoint{0.710919in}{0.521603in}}{\pgfqpoint{5.377460in}{3.966222in}} %
\pgfusepath{clip}%
\pgfsetbuttcap%
\pgfsetroundjoin%
\pgfsetlinewidth{1.505625pt}%
\definecolor{currentstroke}{rgb}{1.000000,0.498039,0.054902}%
\pgfsetstrokecolor{currentstroke}%
\pgfsetdash{}{0pt}%
\pgfpathmoveto{\pgfqpoint{3.125993in}{2.636805in}}%
\pgfpathlineto{\pgfqpoint{3.125993in}{2.682027in}}%
\pgfusepath{stroke}%
\end{pgfscope}%
\begin{pgfscope}%
\pgfpathrectangle{\pgfqpoint{0.710919in}{0.521603in}}{\pgfqpoint{5.377460in}{3.966222in}} %
\pgfusepath{clip}%
\pgfsetbuttcap%
\pgfsetroundjoin%
\pgfsetlinewidth{1.505625pt}%
\definecolor{currentstroke}{rgb}{1.000000,0.498039,0.054902}%
\pgfsetstrokecolor{currentstroke}%
\pgfsetdash{}{0pt}%
\pgfpathmoveto{\pgfqpoint{3.670895in}{2.300230in}}%
\pgfpathlineto{\pgfqpoint{3.670895in}{2.356052in}}%
\pgfusepath{stroke}%
\end{pgfscope}%
\begin{pgfscope}%
\pgfpathrectangle{\pgfqpoint{0.710919in}{0.521603in}}{\pgfqpoint{5.377460in}{3.966222in}} %
\pgfusepath{clip}%
\pgfsetbuttcap%
\pgfsetroundjoin%
\pgfsetlinewidth{1.505625pt}%
\definecolor{currentstroke}{rgb}{1.000000,0.498039,0.054902}%
\pgfsetstrokecolor{currentstroke}%
\pgfsetdash{}{0pt}%
\pgfpathmoveto{\pgfqpoint{4.214330in}{2.032357in}}%
\pgfpathlineto{\pgfqpoint{4.214330in}{2.078358in}}%
\pgfusepath{stroke}%
\end{pgfscope}%
\begin{pgfscope}%
\pgfpathrectangle{\pgfqpoint{0.710919in}{0.521603in}}{\pgfqpoint{5.377460in}{3.966222in}} %
\pgfusepath{clip}%
\pgfsetbuttcap%
\pgfsetroundjoin%
\pgfsetlinewidth{1.505625pt}%
\definecolor{currentstroke}{rgb}{1.000000,0.498039,0.054902}%
\pgfsetstrokecolor{currentstroke}%
\pgfsetdash{}{0pt}%
\pgfpathmoveto{\pgfqpoint{4.757559in}{1.792560in}}%
\pgfpathlineto{\pgfqpoint{4.757559in}{1.848227in}}%
\pgfusepath{stroke}%
\end{pgfscope}%
\begin{pgfscope}%
\pgfpathrectangle{\pgfqpoint{0.710919in}{0.521603in}}{\pgfqpoint{5.377460in}{3.966222in}} %
\pgfusepath{clip}%
\pgfsetbuttcap%
\pgfsetroundjoin%
\pgfsetlinewidth{1.505625pt}%
\definecolor{currentstroke}{rgb}{1.000000,0.498039,0.054902}%
\pgfsetstrokecolor{currentstroke}%
\pgfsetdash{}{0pt}%
\pgfpathmoveto{\pgfqpoint{5.300763in}{1.459432in}}%
\pgfpathlineto{\pgfqpoint{5.300763in}{1.491469in}}%
\pgfusepath{stroke}%
\end{pgfscope}%
\begin{pgfscope}%
\pgfpathrectangle{\pgfqpoint{0.710919in}{0.521603in}}{\pgfqpoint{5.377460in}{3.966222in}} %
\pgfusepath{clip}%
\pgfsetbuttcap%
\pgfsetroundjoin%
\pgfsetlinewidth{1.505625pt}%
\definecolor{currentstroke}{rgb}{1.000000,0.498039,0.054902}%
\pgfsetstrokecolor{currentstroke}%
\pgfsetdash{}{0pt}%
\pgfpathmoveto{\pgfqpoint{5.843950in}{1.150620in}}%
\pgfpathlineto{\pgfqpoint{5.843950in}{1.189502in}}%
\pgfusepath{stroke}%
\end{pgfscope}%
\begin{pgfscope}%
\pgfpathrectangle{\pgfqpoint{0.710919in}{0.521603in}}{\pgfqpoint{5.377460in}{3.966222in}} %
\pgfusepath{clip}%
\pgfsetbuttcap%
\pgfsetroundjoin%
\pgfsetlinewidth{1.505625pt}%
\definecolor{currentstroke}{rgb}{0.172549,0.627451,0.172549}%
\pgfsetstrokecolor{currentstroke}%
\pgfsetdash{}{0pt}%
\pgfpathmoveto{\pgfqpoint{0.955349in}{4.296674in}}%
\pgfpathlineto{\pgfqpoint{0.955349in}{4.307543in}}%
\pgfusepath{stroke}%
\end{pgfscope}%
\begin{pgfscope}%
\pgfpathrectangle{\pgfqpoint{0.710919in}{0.521603in}}{\pgfqpoint{5.377460in}{3.966222in}} %
\pgfusepath{clip}%
\pgfsetbuttcap%
\pgfsetroundjoin%
\pgfsetlinewidth{1.505625pt}%
\definecolor{currentstroke}{rgb}{0.172549,0.627451,0.172549}%
\pgfsetstrokecolor{currentstroke}%
\pgfsetdash{}{0pt}%
\pgfpathmoveto{\pgfqpoint{1.421840in}{3.962589in}}%
\pgfpathlineto{\pgfqpoint{1.421840in}{3.997601in}}%
\pgfusepath{stroke}%
\end{pgfscope}%
\begin{pgfscope}%
\pgfpathrectangle{\pgfqpoint{0.710919in}{0.521603in}}{\pgfqpoint{5.377460in}{3.966222in}} %
\pgfusepath{clip}%
\pgfsetbuttcap%
\pgfsetroundjoin%
\pgfsetlinewidth{1.505625pt}%
\definecolor{currentstroke}{rgb}{0.172549,0.627451,0.172549}%
\pgfsetstrokecolor{currentstroke}%
\pgfsetdash{}{0pt}%
\pgfpathmoveto{\pgfqpoint{2.010486in}{3.621794in}}%
\pgfpathlineto{\pgfqpoint{2.010486in}{3.689168in}}%
\pgfusepath{stroke}%
\end{pgfscope}%
\begin{pgfscope}%
\pgfpathrectangle{\pgfqpoint{0.710919in}{0.521603in}}{\pgfqpoint{5.377460in}{3.966222in}} %
\pgfusepath{clip}%
\pgfsetbuttcap%
\pgfsetroundjoin%
\pgfsetlinewidth{1.505625pt}%
\definecolor{currentstroke}{rgb}{0.172549,0.627451,0.172549}%
\pgfsetstrokecolor{currentstroke}%
\pgfsetdash{}{0pt}%
\pgfpathmoveto{\pgfqpoint{2.581061in}{3.191402in}}%
\pgfpathlineto{\pgfqpoint{2.581061in}{3.240412in}}%
\pgfusepath{stroke}%
\end{pgfscope}%
\begin{pgfscope}%
\pgfpathrectangle{\pgfqpoint{0.710919in}{0.521603in}}{\pgfqpoint{5.377460in}{3.966222in}} %
\pgfusepath{clip}%
\pgfsetbuttcap%
\pgfsetroundjoin%
\pgfsetlinewidth{1.505625pt}%
\definecolor{currentstroke}{rgb}{0.172549,0.627451,0.172549}%
\pgfsetstrokecolor{currentstroke}%
\pgfsetdash{}{0pt}%
\pgfpathmoveto{\pgfqpoint{3.125993in}{2.929306in}}%
\pgfpathlineto{\pgfqpoint{3.125993in}{2.995311in}}%
\pgfusepath{stroke}%
\end{pgfscope}%
\begin{pgfscope}%
\pgfpathrectangle{\pgfqpoint{0.710919in}{0.521603in}}{\pgfqpoint{5.377460in}{3.966222in}} %
\pgfusepath{clip}%
\pgfsetbuttcap%
\pgfsetroundjoin%
\pgfsetlinewidth{1.505625pt}%
\definecolor{currentstroke}{rgb}{0.172549,0.627451,0.172549}%
\pgfsetstrokecolor{currentstroke}%
\pgfsetdash{}{0pt}%
\pgfpathmoveto{\pgfqpoint{3.670895in}{2.620531in}}%
\pgfpathlineto{\pgfqpoint{3.670895in}{2.676448in}}%
\pgfusepath{stroke}%
\end{pgfscope}%
\begin{pgfscope}%
\pgfpathrectangle{\pgfqpoint{0.710919in}{0.521603in}}{\pgfqpoint{5.377460in}{3.966222in}} %
\pgfusepath{clip}%
\pgfsetbuttcap%
\pgfsetroundjoin%
\pgfsetlinewidth{1.505625pt}%
\definecolor{currentstroke}{rgb}{0.172549,0.627451,0.172549}%
\pgfsetstrokecolor{currentstroke}%
\pgfsetdash{}{0pt}%
\pgfpathmoveto{\pgfqpoint{4.214330in}{2.354348in}}%
\pgfpathlineto{\pgfqpoint{4.214330in}{2.420163in}}%
\pgfusepath{stroke}%
\end{pgfscope}%
\begin{pgfscope}%
\pgfpathrectangle{\pgfqpoint{0.710919in}{0.521603in}}{\pgfqpoint{5.377460in}{3.966222in}} %
\pgfusepath{clip}%
\pgfsetbuttcap%
\pgfsetroundjoin%
\pgfsetlinewidth{1.505625pt}%
\definecolor{currentstroke}{rgb}{0.172549,0.627451,0.172549}%
\pgfsetstrokecolor{currentstroke}%
\pgfsetdash{}{0pt}%
\pgfpathmoveto{\pgfqpoint{4.757559in}{2.104636in}}%
\pgfpathlineto{\pgfqpoint{4.757559in}{2.158913in}}%
\pgfusepath{stroke}%
\end{pgfscope}%
\begin{pgfscope}%
\pgfpathrectangle{\pgfqpoint{0.710919in}{0.521603in}}{\pgfqpoint{5.377460in}{3.966222in}} %
\pgfusepath{clip}%
\pgfsetbuttcap%
\pgfsetroundjoin%
\pgfsetlinewidth{1.505625pt}%
\definecolor{currentstroke}{rgb}{0.172549,0.627451,0.172549}%
\pgfsetstrokecolor{currentstroke}%
\pgfsetdash{}{0pt}%
\pgfpathmoveto{\pgfqpoint{5.300763in}{1.745958in}}%
\pgfpathlineto{\pgfqpoint{5.300763in}{1.810720in}}%
\pgfusepath{stroke}%
\end{pgfscope}%
\begin{pgfscope}%
\pgfpathrectangle{\pgfqpoint{0.710919in}{0.521603in}}{\pgfqpoint{5.377460in}{3.966222in}} %
\pgfusepath{clip}%
\pgfsetbuttcap%
\pgfsetroundjoin%
\pgfsetlinewidth{1.505625pt}%
\definecolor{currentstroke}{rgb}{0.172549,0.627451,0.172549}%
\pgfsetstrokecolor{currentstroke}%
\pgfsetdash{}{0pt}%
\pgfpathmoveto{\pgfqpoint{5.843950in}{1.476844in}}%
\pgfpathlineto{\pgfqpoint{5.843950in}{1.526957in}}%
\pgfusepath{stroke}%
\end{pgfscope}%
\begin{pgfscope}%
\pgfpathrectangle{\pgfqpoint{0.710919in}{0.521603in}}{\pgfqpoint{5.377460in}{3.966222in}} %
\pgfusepath{clip}%
\pgfsetrectcap%
\pgfsetroundjoin%
\pgfsetlinewidth{1.505625pt}%
\definecolor{currentstroke}{rgb}{0.121569,0.466667,0.705882}%
\pgfsetstrokecolor{currentstroke}%
\pgfsetdash{}{0pt}%
\pgfpathmoveto{\pgfqpoint{0.955349in}{3.348464in}}%
\pgfpathlineto{\pgfqpoint{1.421840in}{3.143970in}}%
\pgfpathlineto{\pgfqpoint{2.010486in}{2.868930in}}%
\pgfpathlineto{\pgfqpoint{2.581061in}{2.567119in}}%
\pgfpathlineto{\pgfqpoint{3.125993in}{2.167747in}}%
\pgfpathlineto{\pgfqpoint{3.670895in}{1.893637in}}%
\pgfpathlineto{\pgfqpoint{4.214330in}{1.644609in}}%
\pgfpathlineto{\pgfqpoint{4.757559in}{1.460567in}}%
\pgfpathlineto{\pgfqpoint{5.300763in}{0.994712in}}%
\pgfpathlineto{\pgfqpoint{5.843950in}{0.723550in}}%
\pgfusepath{stroke}%
\end{pgfscope}%
\begin{pgfscope}%
\pgfpathrectangle{\pgfqpoint{0.710919in}{0.521603in}}{\pgfqpoint{5.377460in}{3.966222in}} %
\pgfusepath{clip}%
\pgfsetrectcap%
\pgfsetroundjoin%
\pgfsetlinewidth{1.505625pt}%
\definecolor{currentstroke}{rgb}{1.000000,0.498039,0.054902}%
\pgfsetstrokecolor{currentstroke}%
\pgfsetdash{}{0pt}%
\pgfpathmoveto{\pgfqpoint{0.955349in}{4.283478in}}%
\pgfpathlineto{\pgfqpoint{1.421840in}{3.853976in}}%
\pgfpathlineto{\pgfqpoint{2.010486in}{3.425444in}}%
\pgfpathlineto{\pgfqpoint{2.581061in}{3.019565in}}%
\pgfpathlineto{\pgfqpoint{3.125993in}{2.659966in}}%
\pgfpathlineto{\pgfqpoint{3.670895in}{2.328978in}}%
\pgfpathlineto{\pgfqpoint{4.214330in}{2.055927in}}%
\pgfpathlineto{\pgfqpoint{4.757559in}{1.821226in}}%
\pgfpathlineto{\pgfqpoint{5.300763in}{1.475726in}}%
\pgfpathlineto{\pgfqpoint{5.843950in}{1.170468in}}%
\pgfusepath{stroke}%
\end{pgfscope}%
\begin{pgfscope}%
\pgfpathrectangle{\pgfqpoint{0.710919in}{0.521603in}}{\pgfqpoint{5.377460in}{3.966222in}} %
\pgfusepath{clip}%
\pgfsetrectcap%
\pgfsetroundjoin%
\pgfsetlinewidth{1.505625pt}%
\definecolor{currentstroke}{rgb}{0.172549,0.627451,0.172549}%
\pgfsetstrokecolor{currentstroke}%
\pgfsetdash{}{0pt}%
\pgfpathmoveto{\pgfqpoint{0.955349in}{4.302140in}}%
\pgfpathlineto{\pgfqpoint{1.421840in}{3.980425in}}%
\pgfpathlineto{\pgfqpoint{2.010486in}{3.656701in}}%
\pgfpathlineto{\pgfqpoint{2.581061in}{3.216553in}}%
\pgfpathlineto{\pgfqpoint{3.125993in}{2.963479in}}%
\pgfpathlineto{\pgfqpoint{3.670895in}{2.649330in}}%
\pgfpathlineto{\pgfqpoint{4.214330in}{2.388420in}}%
\pgfpathlineto{\pgfqpoint{4.757559in}{2.132567in}}%
\pgfpathlineto{\pgfqpoint{5.300763in}{1.779466in}}%
\pgfpathlineto{\pgfqpoint{5.843950in}{1.502576in}}%
\pgfusepath{stroke}%
\end{pgfscope}%
\begin{pgfscope}%
\pgfsetrectcap%
\pgfsetmiterjoin%
\pgfsetlinewidth{0.803000pt}%
\definecolor{currentstroke}{rgb}{0.000000,0.000000,0.000000}%
\pgfsetstrokecolor{currentstroke}%
\pgfsetdash{}{0pt}%
\pgfpathmoveto{\pgfqpoint{0.710919in}{0.521603in}}%
\pgfpathlineto{\pgfqpoint{0.710919in}{4.487826in}}%
\pgfusepath{stroke}%
\end{pgfscope}%
\begin{pgfscope}%
\pgfsetrectcap%
\pgfsetmiterjoin%
\pgfsetlinewidth{0.803000pt}%
\definecolor{currentstroke}{rgb}{0.000000,0.000000,0.000000}%
\pgfsetstrokecolor{currentstroke}%
\pgfsetdash{}{0pt}%
\pgfpathmoveto{\pgfqpoint{6.088380in}{0.521603in}}%
\pgfpathlineto{\pgfqpoint{6.088380in}{4.487826in}}%
\pgfusepath{stroke}%
\end{pgfscope}%
\begin{pgfscope}%
\pgfsetrectcap%
\pgfsetmiterjoin%
\pgfsetlinewidth{0.803000pt}%
\definecolor{currentstroke}{rgb}{0.000000,0.000000,0.000000}%
\pgfsetstrokecolor{currentstroke}%
\pgfsetdash{}{0pt}%
\pgfpathmoveto{\pgfqpoint{0.710919in}{0.521603in}}%
\pgfpathlineto{\pgfqpoint{6.088380in}{0.521603in}}%
\pgfusepath{stroke}%
\end{pgfscope}%
\begin{pgfscope}%
\pgfsetrectcap%
\pgfsetmiterjoin%
\pgfsetlinewidth{0.803000pt}%
\definecolor{currentstroke}{rgb}{0.000000,0.000000,0.000000}%
\pgfsetstrokecolor{currentstroke}%
\pgfsetdash{}{0pt}%
\pgfpathmoveto{\pgfqpoint{0.710919in}{4.487826in}}%
\pgfpathlineto{\pgfqpoint{6.088380in}{4.487826in}}%
\pgfusepath{stroke}%
\end{pgfscope}%
\begin{pgfscope}%
\pgfsetbuttcap%
\pgfsetmiterjoin%
\definecolor{currentfill}{rgb}{1.000000,1.000000,1.000000}%
\pgfsetfillcolor{currentfill}%
\pgfsetfillopacity{0.800000}%
\pgfsetlinewidth{1.003750pt}%
\definecolor{currentstroke}{rgb}{0.800000,0.800000,0.800000}%
\pgfsetstrokecolor{currentstroke}%
\pgfsetstrokeopacity{0.800000}%
\pgfsetdash{}{0pt}%
\pgfpathmoveto{\pgfqpoint{3.768894in}{3.765143in}}%
\pgfpathlineto{\pgfqpoint{5.991157in}{3.765143in}}%
\pgfpathquadraticcurveto{\pgfqpoint{6.018935in}{3.765143in}}{\pgfqpoint{6.018935in}{3.792921in}}%
\pgfpathlineto{\pgfqpoint{6.018935in}{4.390603in}}%
\pgfpathquadraticcurveto{\pgfqpoint{6.018935in}{4.418381in}}{\pgfqpoint{5.991157in}{4.418381in}}%
\pgfpathlineto{\pgfqpoint{3.768894in}{4.418381in}}%
\pgfpathquadraticcurveto{\pgfqpoint{3.741117in}{4.418381in}}{\pgfqpoint{3.741117in}{4.390603in}}%
\pgfpathlineto{\pgfqpoint{3.741117in}{3.792921in}}%
\pgfpathquadraticcurveto{\pgfqpoint{3.741117in}{3.765143in}}{\pgfqpoint{3.768894in}{3.765143in}}%
\pgfpathclose%
\pgfusepath{stroke,fill}%
\end{pgfscope}%
\begin{pgfscope}%
\pgfsetbuttcap%
\pgfsetroundjoin%
\pgfsetlinewidth{1.505625pt}%
\definecolor{currentstroke}{rgb}{0.121569,0.466667,0.705882}%
\pgfsetstrokecolor{currentstroke}%
\pgfsetdash{}{0pt}%
\pgfpathmoveto{\pgfqpoint{3.935561in}{4.236469in}}%
\pgfpathlineto{\pgfqpoint{3.935561in}{4.375358in}}%
\pgfusepath{stroke}%
\end{pgfscope}%
\begin{pgfscope}%
\pgfsetrectcap%
\pgfsetroundjoin%
\pgfsetlinewidth{1.505625pt}%
\definecolor{currentstroke}{rgb}{0.121569,0.466667,0.705882}%
\pgfsetstrokecolor{currentstroke}%
\pgfsetdash{}{0pt}%
\pgfpathmoveto{\pgfqpoint{3.796672in}{4.305914in}}%
\pgfpathlineto{\pgfqpoint{4.074450in}{4.305914in}}%
\pgfusepath{stroke}%
\end{pgfscope}%
\begin{pgfscope}%
\pgftext[x=4.185561in,y=4.257302in,left,base]{\sffamily\fontsize{10.000000}{12.000000}\selectfont Gaussian decomposable}%
\end{pgfscope}%
\begin{pgfscope}%
\pgfsetbuttcap%
\pgfsetroundjoin%
\pgfsetlinewidth{1.505625pt}%
\definecolor{currentstroke}{rgb}{1.000000,0.498039,0.054902}%
\pgfsetstrokecolor{currentstroke}%
\pgfsetdash{}{0pt}%
\pgfpathmoveto{\pgfqpoint{3.935561in}{4.032612in}}%
\pgfpathlineto{\pgfqpoint{3.935561in}{4.171501in}}%
\pgfusepath{stroke}%
\end{pgfscope}%
\begin{pgfscope}%
\pgfsetrectcap%
\pgfsetroundjoin%
\pgfsetlinewidth{1.505625pt}%
\definecolor{currentstroke}{rgb}{1.000000,0.498039,0.054902}%
\pgfsetstrokecolor{currentstroke}%
\pgfsetdash{}{0pt}%
\pgfpathmoveto{\pgfqpoint{3.796672in}{4.102056in}}%
\pgfpathlineto{\pgfqpoint{4.074450in}{4.102056in}}%
\pgfusepath{stroke}%
\end{pgfscope}%
\begin{pgfscope}%
\pgftext[x=4.185561in,y=4.053445in,left,base]{\sffamily\fontsize{10.000000}{12.000000}\selectfont Gaussian curl-free}%
\end{pgfscope}%
\begin{pgfscope}%
\pgfsetbuttcap%
\pgfsetroundjoin%
\pgfsetlinewidth{1.505625pt}%
\definecolor{currentstroke}{rgb}{0.172549,0.627451,0.172549}%
\pgfsetstrokecolor{currentstroke}%
\pgfsetdash{}{0pt}%
\pgfpathmoveto{\pgfqpoint{3.935561in}{3.828755in}}%
\pgfpathlineto{\pgfqpoint{3.935561in}{3.967644in}}%
\pgfusepath{stroke}%
\end{pgfscope}%
\begin{pgfscope}%
\pgfsetrectcap%
\pgfsetroundjoin%
\pgfsetlinewidth{1.505625pt}%
\definecolor{currentstroke}{rgb}{0.172549,0.627451,0.172549}%
\pgfsetstrokecolor{currentstroke}%
\pgfsetdash{}{0pt}%
\pgfpathmoveto{\pgfqpoint{3.796672in}{3.898199in}}%
\pgfpathlineto{\pgfqpoint{4.074450in}{3.898199in}}%
\pgfusepath{stroke}%
\end{pgfscope}%
\begin{pgfscope}%
\pgftext[x=4.185561in,y=3.849588in,left,base]{\sffamily\fontsize{10.000000}{12.000000}\selectfont Gaussian divergence-free}%
\end{pgfscope}%
\end{pgfpicture}%
\makeatother%
\endgroup%
}')}
    \caption[\acs{ORFF} reconstruction error]{Error reconstructing the target
    operator-valued kernel $K$ with \acs{ORFF}
    approximation $\tilde{K}$ for the decomposable, curl-free and
    divergence-free kernel.}
    \label{fig:approximation_error}
\end{figure}
\paragraph{}
In order to bound the error with high probability, we turn to concentration
inequalities devoted to random matrices~\citep{Boucheron}. The concentration
phenomenon can be summarized in the following sentence of
\citet{ledoux2005concentration}. \say{A random variable that depends (in a
smooth way) on the influence of many random variables (but not too much on any
of them) is essentially constant}.
\paragraph{}
A typical application is the study of the deviation of the empirical mean of
\acl{iid} random variables to their expectation. This means that given an error
$\epsilon$ between the kernel approximation $\tildeK{\omega}$ and the true
kernel $K$, if we are given enough samples to construct $\tildeK{\omega}$, the
probability of measuring an error greater than $\epsilon$ is essentially zero
(it drops at an exponential rate with respect to the number of samples $D$). To
measure the error between the kernel approximation and the true kernel at a
given point many metrics are possible. \acs{eg} any matrix norm such as the
Hilbert-Schmidt norm, trace norm, the operator norm or Schatten norms. In this
work we focus on measuring the error in terms of operator norm. For all $x$,
$z\in\mathcal{X}$ we look for a bound on
\begin{dmath*}
    \probability_{\rho} \Set{(\omega_j)_{j=1}^D | \norm{\tildeK{\omega}(x, z) -
    K(x, z)}_{\mathcal{Y}, \mathcal{Y}} \ge \epsilon }
    =
    \probability_{\rho} \Set{(\omega_j)_{j=1}^D | \sup_{0\neq y\in\mathcal{Y}}
    \frac{\norm{(\tildeK{\omega}(x, z) - K(x,
    z))y}_{\mathcal{Y}}}{\norm{y}_{\mathcal{Y}}} \ge \epsilon}
\end{dmath*}
In other words, given any vector $y\in\mathcal{Y}$ we study how the residual
operator $\tildeK{\omega} - K$ is able to send $y$ to zero. We believe that
this way of measuring the \say{error} to be more intuitive. Moreover, on
contrary to an error measure with the Hilbert-Schmidt norm, the operator norm
error does not grows linearly with the dimension of the output space as the
Hilbert-Schmidt norm does. On the other hand the Hilbert-schmidt norm makes the
studied random variables Hilbert space valued, for which it is much easier to
derive concentration inequalities \citep{smale2007learning, pinelis1994optimum,
naor2012banach}. Note that in the scalar case ($A(\omega)= 1$) the
Hilbert-Schmidt norm error and the operator norm are the same and measure the
deviation between $\tildeK{\omega}$ and $K$ as the absolute value of their
difference.
\paragraph{}
A raw concentration inequality of the kernel estimator gives the error on one
point. If one is interesting in bounding the maximum error over $N$ points,
applying a union bound on all the point would yield a bound that grows linearly
with $N$. This would suggest that when the number of points increase, even if
all of them are concentrated in a small subset of $\mathcal{X}$, we should draw
increasingly more features to have an error below $\epsilon$ with high
probability. However if we restrict ourselves to study the error on a compact
subset of $\mathcal{X}$ (and in practice data points lies often in a closed
bounded subset of $\mathbb{R}^d$), we can cover this compact subset by a finite
number of closed balls and apply the concentration inequality and the union
bound only on the center of each ball. Then if the function
$\norm{\tildeK{\omega}_e-K_e}$ is smooth enough on each ball (\acs{ie}
Lipschitz) we can guarantee with high probability that the error between the
centers of the balls will not be too high. Eventually we obtain a bound in the
worst case scenario on all the points in a subset $\mathcal{C}$ of
$\mathcal{X}$. This bound depends on the covering number
$\mathcal{N}(\mathcal{C}, r)$ of $\mathcal{X}$ with ball of radius $r$. When
$\mathcal{X}$ is a Banach space, the covering number is proportional to the
diameter of the diameter of $\mathcal{C}\subseteq\mathcal{X}$.
\paragraph{}
Prior to the presentation of general results, we briefly recall the uniform
convergence of \acs{RFF} approximation for a scalar shift invariant kernel on
the additive \acs{LCA} group $\mathbb{R}^d$ and introduce a direct corollary
about decomposable shift-invariant \acs{OVK} on the \acs{LCA} group
$(\mathbb{R}^d, +)$.
\subsection{Random Fourier Features in the scalar case and decomposable OVK}
\citet{Rahimi2007} proved the uniform convergence of \acf{RFF} approximation
for a scalar shift-invariant kernel on the \acs{LCA} group $\mathbb{R}^d$
endowed with the group operation $\groupop=+$. In the case of the
shift-invariant decomposable \acs{OVK}, an upper bound on the error can be
obtained as a direct consequence of the result in the scalar case obtained
by~\citet{Rahimi2007} and other authors~\citep{sutherland2015, sriper2015}.

\begin{theorem}[Uniform error bound for \ac{RFF},~\citet{Rahimi2007}]
    \label{rff-scalar-bound}
    Let $\mathcal{C}$ be a compact of subset of $\mathbb{R}^d$ of diameter
    $\abs{\mathcal{C}}$. Let $k$ be a shift invariant kernel, differentiable
    with a bounded second derivative and $\probability_{\rho}$ its normalized
    \acl{IFT} such that it defines a probability measure. Let
    \begin{dmath*}
        \widetilde{k}=\sum_{j=1}^D\cos{\inner{\cdot, \omega_j}}
        \hiderel{\approx} k(x,z) \enskip\text{and}\enskip
        \sigma^2\hiderel{=}\expectation_{\rho}
        \norm{\omega}^2_2.
    \end{dmath*}
    Then we have
    \begin{dmath*}
        \probability_{\rho}\Set{(\omega_j)_{j=1}^D |
        \norm{\tilde{k}-k}_{\mathcal {C}\times\mathcal{C}}\ge \epsilon } \le
        2^8\left( \frac{\sigma \abs{\mathcal{C}}}{\epsilon} \right)^2\exp\left(
        -\frac{\epsilon^2D}{4(d+2)} \right)
    \end{dmath*}
\end{theorem}
From \cref{rff-scalar-bound}, we can deduce the following corollary about the
uniform convergence of the \acs{ORFF} approximation of the decomposable kernel.
We recall that for a given pair $x$, $z$ in $\mathcal{C}$, $\tilde{K}(x,z)=
\tildePhi{\omega}(x)^* \tildePhi{\omega}(z)=\Gamma\tilde{k}(x,z)$ and
$K_0(x-z)=\Gamma \expectation_{\dual{\Haar},\rho}[\tilde{k}(x,z)]$.
\begin{corollary}[Uniform error bound for decomposable \acs{ORFF}]
    \label{c:dec-bound}
    Let $\mathcal{C}$ be a compact of subset of $\mathbb{R}^d$ of diameter
    $\abs{\mathcal{C}}$. Let $K$ be a decomposable kernel built from a positive
    operator self-adjoint $\Gamma$, and $k$ a shift invariant kernel with bounded
    second derivative such that
    \begin{dmath*}
        \widetilde{K}=\sum_{j=1}^D\cos{\inner{\cdot, \omega_j}}\Gamma
        \hiderel{\approx} K \enskip\text{and}\enskip
        \sigma^2\hiderel{=}\expectation_{\rho}
        \norm{\omega}^2_2.
    \end{dmath*}
    Then
    \begin{dmath*}
        \probability_{%
        \rho}\Set{(\omega_j)_{j=1}^D|\norm{\widetilde{K}-K}_{\mathcal{C} \times
        \mathcal{C}}\ge \epsilon } \le 2^8\left( \frac{\sigma
        \norm{\Gamma}_{\mathcal{Y},\mathcal{Y}} \abs{\mathcal{C}}}{\epsilon}
        \right)^2\exp\left( -\frac{\epsilon^2D}{4\norm{\Gamma}_2^2(d+2)} \right)
    \end{dmath*}
\end{corollary}
\begin{proof}
    The proof directly extends \cref{rff-scalar-bound} given
    by~\cite{Rahimi2007}. Let $\tilde{k}$ the Random Fourier approximation for
    the scalar-valued kernel $k$. Since
    \begin{dmath*}
        \sup_{(x,z)\in\mathcal{C}\times\mathcal{C}}\norm{\widetilde{K}(x,z) -
        K(x,z)}_{\mathcal{Y},\mathcal{Y}} =
        \sup_{(x,z)\in\mathcal{C}\times\mathcal{C}}
        \norm{\Gamma}_{\mathcal{Y},\mathcal{Y}} \abs{\widetilde{K}(x,z) - k(x,z)},
    \end{dmath*}
    taking $\epsilon' = \norm{\Gamma}_{\mathcal{Y},\mathcal{Y}} \epsilon$ gives the
    following result for all positive $\epsilon'$:
    \begin{dmath*}
        \probability_{%
        \rho}\Set{(\omega_j)_{j=1}^D|\sup_{x,z\in\mathcal{C}}\norm{\Gamma
        \left(\widetilde{k}(x,z)-k(x,z)\right)}_{\mathcal{Y},\mathcal{Y}}\ge
        \epsilon' } \le 2^8\left( \frac{\sigma
        \norm{\Gamma}_{\mathcal{Y},\mathcal{Y}} \abs{\mathcal{C}}}{\epsilon'}
        \right)^2\exp\left( -\frac{\left(\epsilon'\right)^2D}{4
        \norm{\Gamma}_{\mathcal{Y}, \mathcal{Y}}^2(d + 2)} \right)
    \end{dmath*}
    which concludes the proof.
\end{proof}
Please note that a similar corollary could have been obtained for the recent
result of~\citet{sutherland2015} who refined the bound proposed by Rahimi and
Recht by using a Bernstein concentration inequality instead of the Hoeffding
inequality. More recently~\citet{sriper2015} showed an optimal bound for
\acl{RFF}. The improvement of~\citet{sriper2015} is mainly in the constant
factors where the bound does not depend linearly on the diameter
$\abs{\mathcal{C}}$ of $\mathcal{C}$ but exhibit a logarithmic dependency
$\log\left(\abs{\mathcal{C}}\right)$, hence requiring significantly less
random features to reach a desired uniform error with high probability. Moreover,
\citet{sutherland2015} also considered a bound on the expected max error
$\expectation_{\dual{\Haar}, \rho} \norm{\widetilde{K}-K}_{\infty}$, which is
obtained using Dudley's entropy integral~\citep{dudley1967sizes, Boucheron} as
a bound on the supremum of an empirical process by the covering number of the
indexing set. This useful theorem is also part of the proof of
\citet{sriper2015}.

\subsection{Uniform convergence of \acpdfstring{ORFF} approximation on
\acpdfstring{LCA} groups}
In this analysis, we assume that $\mathcal{Y}$ is finite dimensional, in
\cref{remark:infinite_dimension}, we discuss how the proof could be extended to
infinite dimensional output Hilbert spaces. We propose a bound for \acl{ORFF}
approximation in the general case. It relies on two main ideas:
\begin{enumerate}
    \item a matrix-Bernstein concentration inequality for random matrices need
    to be used instead of concentration inequality for scalar random variables,
    \item a general theorem valid for random matrices with bounded norms such
    as decomposable kernel \acs{ORFF} approximation as well as un\-bound\-ed
    norms such as the \acs{ORFF} approximation we proposed for curl and
    divergence-free kernels that behave as subexponential random variables.
\end{enumerate}
Before introducing the new theorem, we give the definition of the Orlicz norm
which gives a proxy-bound on the norm of subexponential random variables.
\begin{definition}[Orlicz norm~\citep{van1996weak}]
    Let $\psi:\mathbb{R}_+\to\mathbb{R}_+$ be a non-decreasing convex function
    with $\psi(0)=0$. For a random variable $X$ on a measured space
    $(\Omega,\mathcal{T} (\Omega),\mu)$, the quantity
    \begin{dmath*}
        \norm{X}_{\psi} \hiderel{=} \inf \Set{C > 0  |
        \expectation_{\mu}[\psi\left( \abs{X}/C \right)]\le 1}.
    \end{dmath*}
    is called the Orlicz norm of $X$.
\end{definition}
Here, the function $\psi$ is chosen as $\psi(u)=\psi_{\alpha}(u)$ where
$\psi_{\alpha}(u) \colonequals e^{u^{\alpha}}-1$. When $\alpha=1$, a random
variable with finite Orlicz norm is called a \emph{subexponential variable}
because its tails decrease at an exponential rate. Let $X$ be a self-adjoint
random operator. Given a scalar-valued measure $\mu$, we call \emph{variance}
of an operator $X$ the quantity $\variance_{\mu}[X]=\expectation_
{\mu}[X-\expectation_{\mu}[X]]^2$. With this convention if $X$ is a $p\times
p$ Hermitian matrix,
\begin{dmath*}
    \variance_{\mu}[X]_{\ell m}=\sum_{r=1}^p\covariance{X_{\ell r}, X_{rm}}.
\end{dmath*}
Among the possible concentration inequalities adapted to random operators
\citep{tropp2015introduction, minsker2011some, ledoux2013probability,
pinelis1994optimum, koltchinskii2013remark}, we focus on the results of
\citet{tropp2015introduction, minsker2011some}, for their robustness to high or
potentially infinite dimension of the output space $\mathcal{Y}$. To guarantee
a good scaling with the dimension of $\mathcal{Y}$ we introduce the notion of
intrinsic dimension (or effective rank) of an operator.
\begin{definition}
    Let $A$ be a trace class operator acting on a Hilbert space
    $\mathcal{Y}$. We call intrinsic dimension the quantity
    \begin{dmath*}
        \intdim(A) = \frac{\Tr\left[A\right]}{\norm{A}_{\mathcal{Y},
        \mathcal{Y}}}.
    \end{dmath*}
\end{definition}
Indeed the bound proposed in our first publication at \acs{ACML}
\citep{brault2016random} based on \citet{koltchinskii2013remark} depends on $p$
while the present bound depends on the intrinsic dimension of the variance of
$A(\omega)$ which is always smaller than $p$ when the operator $A(\omega)$ is
Hilbert-Schmidt ($p\le\infty$).
\begin{corollary}
    \label{corr:unbounded_consistency}
    Let $K:\mathcal{X}\times\mathcal{X}\to\mathcal{L}(\mathcal{Y})$ be a
    shift-invariant $\mathcal{Y}$-Mercer kernel, where $\mathcal{Y}$ is a
    finite dimensional Hilbert space of dimension $p$ and $\mathcal{X}$ a
    finite dimensional Banach space of dimension $d$. Moreover, let
    $\mathcal{C}$ be a closed ball of $\mathcal{X}$ centred at the origin of
    diameter $\abs{\mathcal{C}}$,
    $A:\dual{\mathcal{X}}\to\mathcal{L}(\mathcal{Y})$ and
    $\probability_{\dual{\Haar},\rho}$ a pair such that
    \begin{dmath*}
        \tilde{K}_e = \sum_{j=1}^D \cos{\pairing{\cdot,\omega_j}}A(\omega_j)
        \hiderel{\approx}
        K_e\condition{$\omega_j\sim\probability_{\dual{\Haar}, \rho}$
        \acs{iid}.}.
    \end{dmath*}
    Let $\mathcal{D}_{\mathcal{C}}=\mathcal{C}\groupop\mathcal{C}^{-1}$ and
    \begin{dmath*}
        V(\delta) \succcurlyeq \variance_{\dual{\Haar},\rho}
        \tilde{K}_e(\delta) \condition{for all
        $\delta\in\mathcal{D}_{\mathcal{C}}$}
    \end{dmath*}
    and $H_\omega$ be the Lipschitz constant of the function $h: x\mapsto
    \pairing{x,\omega}$. If the three following constants exist
    \begin{dmath*}
        m \ge \int_{\dual{\mathcal{X}}} H_\omega
        \norm{A(\omega)}_{\mathcal{Y},\mathcal{Y}} d\probability_{\dual{\Haar},
        \rho}(\omega) \hiderel{<} \infty
    \end{dmath*}
    and
    \begin{dmath*}
        u \ge 4\left(\norm{\norm{A(\omega)}_{\mathcal{Y},\mathcal{Y}}}_{\psi_1}
        + \sup_{\delta\in\mathcal{D}_{\mathcal{C}}}
        \norm{K_e(\delta)}_{\mathcal{Y},\mathcal{Y}}\right) \hiderel{<} \infty
    \end{dmath*}
    and
    \begin{dmath*}
        v \ge \sup_{\delta\in\mathcal{D}_{\mathcal{C}}} D
        \norm{V(\delta)}_{\mathcal{Y}, \mathcal{Y}} \hiderel{<} \infty.
    \end{dmath*}
    Define $p_{int}\ge \sup_{\delta\in\mathcal{D}_{\mathcal{C}}}
    \intdim(V(\delta))$, then for all $0 < \epsilon \le m \abs{C}$,
    \begin{dmath*}
        \probability_{\dual{\Haar,\rho}}\Set{(\omega_j)_{j=1}^D |
        \norm{\tilde{K}-K}_{\mathcal{C}\times\mathcal{C}} \ge \epsilon}
        \le 8\sqrt{2} \left( \frac{m\abs{\mathcal{C}}}{\epsilon}
        \right)
        {\left(p_{int}r_{v/D}(\epsilon)\right)}^{\frac{1}{d + 1}}
        \begin{cases}
            \exp\left(-D\frac{\epsilon^2}{8
            v(d+1)\left(1 + \frac{1}{p}\right)}
            \right) \condition{$\epsilon \le
            \frac{v}{u}\frac{1+1/p}{K(v,
            p)}$} \\
            \exp\left(-D\frac{\epsilon}{8u(d+1)K(v,
            p)}\right)\condition{otherwise,}
        \end{cases}
    \end{dmath*}
    where $K(v, p)=\log\left(16 \sqrt{2}
    p\right)+\log\left(\frac{u^2}{v}\right) $ and $r_{v/D}(\epsilon)=1 +
    \frac{3}{\epsilon^2\log^2(1 + D \epsilon / v)}$.
\end{corollary}
\begin{sproof}
    In the following, let $F(\delta)=F(x \groupop
    \inv{z})=\tilde{K}(x,z)-K(x,z)$.  Let
    $\mathcal{D}_{\mathcal{C}}=\mathcal{C}\groupop \mathcal{C}^{-1} =
    \Set{x\groupop \inv{z} | x,z\in\mathcal{C}}$. Since $\mathcal{C}$ is
    supposed compact, so is $\mathcal{D}_{\mathcal{C}}$. Its diameter is at
    most $2\abs{\mathcal{C}}$ where $\abs{\mathcal{C}}$ is the diameter of
    $\mathcal{C}$. Since $\mathcal{C}$ is supposed to be a closed ball of a
    Bochner space tt is then possible to find an $\epsilon$-net covering
    $\mathcal {C}_{\Delta}$ with at most $T=(4\abs{\mathcal{C}}/r)^d$ balls of
    radius $r$ \citep{cucker2001mathematical}. We call $\delta_i$ for
    $i\inrange{1}{T}$ the center of the $i$-th ball, called \emph{anchors} of
    the $\epsilon$-net. Denote $L_{F}$ the Lipschitz constant of $F$. We
    introduce the following lemma proved in~\cite{Rahimi2007}.
    \begin{lemma}
        \label{lm:error_decomposition_main}
        For all $\delta \in \mathcal{D}_{\mathcal{C}}$, if
        \begin{dmath}
            L_{F}\le\frac{\epsilon}{2r}
            \label{condition1_main}
        \end{dmath}
        and
        \begin{dmath}
            \norm{F(\delta_i)}_{\mathcal{Y},\mathcal{Y}}
            \le\frac{\epsilon}{2}\condition{for all $i\in\mathbb{N}^*_T$}
            \label{condition2_main}
        \end{dmath}
        then $\norm{F(\delta)}_{\mathcal{Y},\mathcal{Y}} \leq \epsilon$.
    \end{lemma}
    To apply the lemma, we must check assumptions \cref{condition1_main} and
    \cref{condition2_main}.
    \begin{sproof}[\cref{condition1_main}]
        \begin{lemma}
            \label{lm:LipschitzK_main}
            Let $H_\omega \in \mathbb{R}_+$ be the Lipschitz constant of
            $h_\omega(\cdot)$ and assume that
            \begin{dmath*}
                \int_{\dual{\mathcal{X}}} H_\omega
                \norm{A(\omega)}_{\mathcal{Y}, \mathcal{Y}}d
                \probability_{\dual{\Haar}, \rho}(\omega) < \infty.
            \end{dmath*}
            Then the operator-valued function
            $K_e:\mathcal{X}\to\mathcal{L}(\mathcal{Y})$ is Lipschitz with
            \begin{dmath}
                \norm{K_e(x) - K_e(z)}_{\mathcal{Y},\mathcal{Y}}\le
                d_{\mathcal{X}}(x,z) \int_{\dual{\mathcal{X}}} H_\omega
                \norm{A(\omega)}_{\mathcal{Y},\mathcal{Y}}
                d\probability_{\dual{\Haar}, \rho}(\omega).
            \end{dmath}
        \end{lemma}
        In the same way, considering
        $\tilde{K}_e(\delta)=\frac{1}{D}\sum_{j=1}^D\cos
        h_{\omega_j}(\delta)A(\omega_j)$, where
        $\omega_j\sim\probability_{\dual{\Haar},\rho}$, we can show that
        $\tilde{K}_e$ is Lipschitz with
        \begin{dmath*}
            \norm{\tilde{K}_e(x) - \tilde{K}_e(z)}_{\mathcal{Y},\mathcal{Y}}
            \le d_{\mathcal{X}}(x,z)\frac{1}{D}\sum_{j=1}^DH_{\omega_j}
            \norm{A(\omega_j)}_{\mathcal{Y},\mathcal{Y}}.
        \end{dmath*}
        Taking the expectation yields
        \begin{dmath*}
            \expectation_{\dual{\Haar},\rho}\left[ L_F \right] = 2
            \int_{\dual{\mathcal{X}}} H_\omega
            \norm{A(\omega)}_{\mathcal{Y},\mathcal{Y}}
            d\probability_{\dual{\Haar}, \rho}
        \end{dmath*}
        Thus by Markov's inequality,
        \begin{dmath}
            \probability_{\dual{\Haar},\rho}\set{(\omega_j)_{j=1}^D | L_F \ge
            \epsilon} \le \frac{\expectation_{\dual{\Haar},\rho}\left[ L_F
            \right]}{\epsilon} \le \frac{2}{\epsilon} \int_{\dual{\mathcal{X}}}
            H_\omega \norm{A(\omega)}_{\mathcal{Y},\mathcal{Y}}
            d\probability_{\dual{\Haar},\rho}.
            \label{eq:Lipschitz_constant_main}
        \end{dmath}
    \end{sproof}
    \begin{sproof}[\cref{condition2_main}]
        To obtain a bound on the anchors we apply theorem 4
        of~\citet{koltchinskii2013remark}.  We suppose the existence of the two
        constants
        \begin{dmath*}
            v_i=D\variance_{\dual{\Haar},\rho}\left[ \tilde{K}(\delta_i)
            \right]
        \end{dmath*}
        and
        \begin{dmath*}
            u_i=4\left(\norm{\norm{A(\omega)}_{\mathcal{Y},
            \mathcal{Y}}}_{\psi_1} + \norm{K_e(\delta_i)}_{\mathcal{Y},
            \mathcal{Y}}\right)
        \end{dmath*}
        Then $\forall i\in\Set{1, \hdots, T}$,
        \begin{dmath*}
            \probability_{\dual{\Haar,\rho}}\Set{(\omega_j)_{j=1}^D |
            \norm{F(\delta_i)}_{\mathcal{Y}, \mathcal{Y}} \ge \epsilon } \le
            \begin{cases}
                4 \intdim(v_i)\exp\left(-D\frac{\epsilon^2}{2
                \norm{v_i}_{\mathcal{Y},\mathcal{Y}}\left(1 +
                \frac{1}{p}\right)} \right) r_{v_i/D}(\epsilon)
                \condition{$\epsilon \le
                \frac{\norm{v_i}_{\mathcal{Y},\mathcal{Y}}}{2u_i}\frac{1 + 1 /
                p}{K(v_i, p)}$} \\
                4 \intdim(v_i)\exp\left(-D\frac{\epsilon}{4u_iK(v_i,
                p)}\right)r_{v_i/D}(\epsilon) \condition{otherwise.}
            \end{cases}
            \label{eq:anchor_bound_sproof}
        \end{dmath*}
        where
        \begin{dmath*}
            K(v_i,p)=\log\left(16 \sqrt{2}
            p\right)+\log\left(\frac{u_i^2}{\norm{v_i}_{\mathcal{Y},
            \mathcal{Y}}}\right)
        \end{dmath*}
        and
        \begin{dmath*}
            r_{v_i/D}= 1 + \frac{3}{\epsilon^2\log^2(1 + D \epsilon /
            \norm{v_i}_{\mathcal{Y},\mathcal{Y}})}.
        \end{dmath*}
    \end{sproof}
    \emph{Combining \cref{condition1_main} and \cref{condition2_main}}.  Now
    applying the lemma and taking the union bound over the centers of the
    $\epsilon$-net yields
    \begin{dmath*}
        \probability_{\dual{\Haar,\rho}}\Set{(\omega_j)_{j=1}^D |
        \norm{\tilde{K}-K}_{\mathcal{C}\times\mathcal{C}} \ge \epsilon}
        \le 4\left(\frac{r m}{\epsilon} + p_{int} \left(\frac{2\abs{C}}{r}
        \right)^d r_{v/D}(\epsilon) \\
        \begin{cases}
            \exp\left(-D\frac{\epsilon^2}{8
            v\left(1 + \frac{1}{p}\right)}
            \right) \condition{$\epsilon \le
            \frac{v}{u}\frac{1+1/p}{K(v,
            p)}$} \\
            \exp\left(-D\frac{\epsilon}{8uK(v,
            p)}\right)\condition{otherwise.}
        \end{cases}\right)
    \end{dmath*}
    The right hand side of the equation has the form $ar+br^{-d}$ with
    \begin{dmath*}
        a = \frac{m}{\epsilon}
    \end{dmath*}
    and
    \begin{dmath*}
        b =  p_{int} {\left(2 \abs{\mathcal{C}}\right)}^d r_{v/D}(\epsilon)
        \begin{cases}
            \exp\left(-D\frac{\epsilon^2}{8
            v\left(1 + \frac{1}{p}\right)}
            \right) \condition{$\epsilon \le
            \frac{v}{u}\frac{1+1/p}{K(v,
            p)}$} \\
            \exp\left(-D\frac{\epsilon}{8uK(v,
            p)}\right)\condition{otherwise.}
        \end{cases}
    \end{dmath*}
    Following \cite{Rahimi2007, sutherland2015, minh2016operator}, we optimize
    over $r$.  It is a convex continuous function on $\mathbb{R}_+$ and achieve
    the minimum value
    \begin{dmath*}
        r_*=a^{\frac{d}{d + 1}}b^{\frac{1}{d + 1}}\left( d^{\frac{1}{d + 1}} +
        d^{-\frac{d}{d+1}} \right),
    \end{dmath*}
    hence
    \begin{dmath*}
        \probability_{\dual{\Haar,\rho}}\Set{(\omega_j)_{j=1}^D |
        \norm{\tilde{K}-K}_{\mathcal{C}\times\mathcal{C}} \ge \epsilon}
        \le 8\sqrt{2} \left( \frac{m\abs{\mathcal{C}}}{\epsilon}
        \right)
        {\left(p_{int}r_{v/D}(\epsilon)\right)}^{\frac{1}{d + 1}}
        \begin{cases}
            \exp\left(-D\frac{\epsilon^2}{8
            v(d+1)\left(1 + \frac{1}{p}\right)}
            \right) \condition{$\epsilon \le
            \frac{v}{u}\frac{1+1/p}{K(v,
            p)}$} \\
            \exp\left(-D\frac{\epsilon}{8u(d+1)K(v,
            p)}\right)\condition{otherwise,}
        \end{cases}
    \end{dmath*}
\end{sproof}
We give a comprehensive full proof of the theorem in
\cref{subsec:concentration_proof}. It follows the usual scheme derived
in~\citet{Rahimi2007} and~\citet{sutherland2015} and involves Bernstein
concentration inequality for unbounded symmetric matrices
(\cref{th:Bernstein3}).

\subsection{Dealing with infinite dimensional operators}
\label{remark:infinite_dimension}
We studied the concentration of \acsp{ORFF} under the assumption that
$\mathcal{Y}$ is finite dimensional. Indeed a $d$ term characterizing the
dimension of the input space $\mathcal{X}$ appears in the bound proposed in
\cref{corr:unbounded_consistency}, and when $d$ tends to infinity, the
exponential part goes to zero so that the probability is bounded by a
constant greater than one. Unfortunately, considering unbounded random
operators \citet{minsker2011some} doesn't give any tighter solution.
\paragraph{}
In our first bound presented at \acs{ACML}, we presented a bound based on a
matrix concentration inequality for unbounded random variable. Compared to this
previous bound, \cref{corr:unbounded_consistency} does not depend on the
dimensionality $p$ of the output space $\mathcal{Y}$ but on the intrinsic
dimension of the operator $A(\omega)$. However to remove the dependency in $p$
in the exponential part, we must turn our attention to operator concentration
inequalities for bounded random variable. To the best of our knowledge we are
not aware of concentration inequalities working for \say{unbounded} operator-
valued random variables. Following the same proof than
\cref{corr:unbounded_consistency} we obtain
\begin{corollary}
    \label{corr:bounded_infinite_dim_consistency}
    Let $K:\mathcal{X}\times\mathcal{X}\to\mathcal{L}(\mathcal{Y})$ be a
    shift-invariant $\mathcal{Y}$-Mercer kernel, where $\mathcal{Y}$ is a
    Hilbert space and $\mathcal{X}$ a finite dimensional Banach space of
    dimension $D$. Moreover, let $\mathcal{C}$ be a closed ball of
    $\mathcal{X}$ centered at the origin of diameter $\abs{\mathcal{C}}$,
    subset of $\mathcal{X}$, $A:\dual{\mathcal{X}}\to\mathcal{L}(\mathcal{Y})$
    and $\probability_{\dual{\Haar},\rho}$ a pair such that
    \begin{dmath*}
        \tilde{K}_e = \sum_{j=1}^D \cos{\pairing{\cdot,\omega_j}}A (\omega_j)
        \hiderel{\approx}
        K_e\condition{$\omega_j\sim\probability_{\dual{\Haar}, \rho}$
        \acs{iid}.}
    \end{dmath*}
    where $A(\omega_j)$ is a Hilbert-Schmidt operator for all $j \in
    \mathbb{N}^*_D$. Let $\mathcal{D}_{\mathcal{C}}=\mathcal{C} \groupop
    \mathcal{C}^{-1}$ and
    \begin{dmath*}
        V (\delta) \succcurlyeq\variance_{\dual{\Haar},\rho}
        \tilde{K}_e (\delta) \condition{for all
        $\delta\in\mathcal{D}_{\mathcal{C}}$}
    \end{dmath*}
    and $H_\omega$ be the Lipschitz constant of the function $h: x\mapsto
    \pairing{x,\omega}$. If the three following constants exists
    \begin{dmath*}
        m \ge\int_{\dual{\mathcal{X}}} H_{\omega}
        \norm{A (\omega)}_{\mathcal{Y},\mathcal{Y}}
        d\probability_{\dual{\Haar}, \rho}(\omega) \hiderel{<} \infty{}
    \end{dmath*}
    and
    \begin{dmath*}
        u \ge\esssup_{\omega\in\dual{\mathcal{X}}}
        \norm{A (\omega)}_{\mathcal{Y}, \mathcal{Y}} +
        \sup_{\delta\in\mathcal{D}_{\mathcal{C}}}
        \norm{K_e (\delta)}_{\mathcal{Y}, \mathcal{Y}} \hiderel{<} \infty{}
    \end{dmath*}
    and
    \begin{dmath*}
        v \ge\sup_{\delta\in\mathcal{D}_{\mathcal{C}}} D
        \norm{V (\delta)}_{\mathcal{Y}, \mathcal{Y}} \hiderel{<} \infty.
    \end{dmath*}
    define $p_{int} \ge \sup_{\delta\in\mathcal{D}_{\mathcal{C}}}
    \intdim\left(V(\delta)\right)$ then for all $\sqrt{\frac{v}{D}} +
    \frac{u}{3D} < \epsilon < m\abs{\mathcal{C}}$,
    \begin{dmath*}
        \probability_{\dual{\Haar,\rho}}\Set{(\omega_j)_{j=1}^D |
        \sup_{\delta\in\mathcal{D}_{\mathcal{C}}}
        \norm{F (\delta)}_{\mathcal{Y}, \mathcal{Y}} \ge\epsilon} \le~8\sqrt{2}
        \left(\frac{m\abs{\mathcal{C}}}{\epsilon}\right) p_{int}^{\frac{1}{d +
        1}} \exp\left(-D\psi_{v,d,u} (\epsilon) \right)
    \end{dmath*}
    where $\psi_{v,d,u}(\epsilon)=\frac{\epsilon^2}{2(d+1)(v + u
    \epsilon / 3)}$.
\end{corollary}
Again a full comprehensive proof is given in \cref{subsec:concentration_proof}
of the appendix. Notice that in this result, The dimension
$p=\dim{\mathcal{Y}}$ does not appear. Only the intrinsic dimension of the
variance of the estimator. Moreover when $d$ is large, the term
$p_{int}^{\frac{1}{d + 1}}$ goes to one, so that the impact of the intrinsic
dimension on the bound is limited.

\subsection{Variance of the \acpdfstring{ORFF} approximation}
We now provide a bound on the norm of the variance of $\tilde{K}$, required to
apply \cref{corr:unbounded_consistency,corr:bounded_infinite_dim_consistency}.
This is an extension of the proof of \citet{sutherland2015} to the
operator-valued case, and we recover their results in the scalar case when
$A(\omega)=1$. An illustration of the bound is provided in
\cref{fig:approximation_error_var} for the decomposable and the curl-free
\acs{OVK}.
\begin{proposition}[Bounding the \emph{variance} of $\tilde{K}$]
    \label{pr:variance_bound}
    Let $K$ be a shift invariant $\mathcal{Y}$-Mercer
    kernel on a second countable \ac{LCA} topological space $\mathcal{X}$. Let
    $A:\dual{\mathcal{X}}\to\mathcal{L}(\mathcal{Y})$ and
    $\probability_{\dual{\Haar},\rho}$ a pair such that
    \begin{dmath*}
        \tilde{K}_e = \sum_{j=1}^D \cos{\pairing{\cdot,\omega_j}}A (\omega_j)
        \hiderel{\approx}
        K_e\condition{$\omega_j\sim\probability_{\dual{\Haar}, \rho}$
        \acs{iid}.}
    \end{dmath*}
    Then,
    \begin{dmath*}
        \variance_{\dual{\Haar}, \rho} \left[ \tilde{K}_e(\delta) \right]
        \preccurlyeq \frac{1}{2D} \left( \left( K_e(2\delta) + K_e(e) \right)
        \expectation_{\dual{\Haar}, \rho}\left[ A(\omega) \right] -
        2 K_e(\delta)^2 + \variance_{\dual{\Haar}, \rho}\left[
        A(\omega) \right]\right)
    \end{dmath*}
\end{proposition}
\begin{proof}
    It relies on the \ac{iid}~property of the random vectors $\omega_j$ and
    trigonometric identities (see the proof in \cref{pr:variance_bound} of the
    appendix).
\end{proof}
\begin{figure}
    \centering
    \includegraphics[width=\textwidth]{./gfx/variance_dec.tikz}
    \caption[decomposable ORFF variance bound]{Comparison between an empirical
    bound on the norm of the variance of the decomposable ORFF obtained and the
    theoretical bound proposed in \cref{pr:variance_bound} versus $D$
    \label{fig:approximation_error_var}}
\end{figure}
\begin{figure}
    \centering
    \includegraphics[width=\textwidth]{./gfx/variance_curl.tikz}
    \caption[Curl-free ORFF variance bound]{Comparison between an empirical
    bound on the norm of the variance of the curl-free ORFF obtained and the
    theoretical bound proposed in \cref{pr:variance_bound} versus $D$
    \label{fig:approximation_error_var}}
\end{figure}
\subsection{Application on decomposable, curl-free and divergence-free
\acpdfstring{OVK}}
First, the two following examples discuss the form of $H_\omega$ for the
additive group and the skewed-multiplicative group. Here we view
$\mathcal{X}=\mathbb{R}^d$ as a Banach space endowed with the Euclidean norm.
Thus the Lipschitz constant $H_{\omega}$ is bounded by the supremum of the norm
of the gradient of $h_{\omega}$.
\begin{example}[Additive group]
    On the additive group, $h_\omega(\delta)=\inner{\omega, \delta}$. Hence
    $H_\omega=\norm{\omega}_2$.
\end{example}
\begin{example}[Skewed-multiplicative group]
    On the skewed multiplicative group, $h_\omega(\delta)=\inner{\omega,
    \log(\delta+c)}$. Therefore
    \begin{dmath*}
        \sup_{\delta\in\mathcal{C}}\norm{\nabla
        h_\omega(\delta)}_2 = \sup_{\delta\in\mathcal{C}}\norm{\omega/(\delta +
        c)}_2.
    \end{dmath*}
    Eventually $\mathcal{C}$ is compact subset of $\mathcal{X}$ and finite
    dimensional thus $\mathcal{C}$ is closed and bounded. Thus
    $H_\omega=\norm{\omega}_2/(\min_{\delta\in\mathcal{C}} \norm{\delta}_2+c)$.
\end{example}
Now we compute upper bounds on the norm of the variance and Orlicz norm of the
three \acsp{ORFF} we took as examples.
\subsubsection{Decomposable kernel}
notice that in the case of the Gaussian decomposable kernel, \acs{ie}
$A(\omega)=A$, $e=0$, $K_0(\delta)= Ak_0(\delta)$, $k_0(\delta) \geq 0$ and
$k_0(\delta)=1$, then we have
\begin{equation*}
    D\norm{\variance_\mu \left[ \tilde{K}_0(\delta)
    \right]}_{\mathcal{Y},\mathcal{Y}}\leq
    (1+k_0(2\delta))\norm{A}_{\mathcal{Y},\mathcal{Y}}/2 + k_0(\delta)^2.
\end{equation*}
\subsubsection{Curl-free and divergence-free kernels:}
recall that in this case $p=d$. For the (Gaussian) curl-free kernel,
$A(\omega)=\omega\omega^*$ where $\omega\in\mathbb{R}^d\sim\mathcal{N}(0,
\sigma^{-2}I_d)$ thus $\expectation_\mu [A(\omega)] = I_d/\sigma^2$ and
$\variance_{\mu}[A(\omega)]=(d+1)I_d/\sigma^4$. Hence,
\begin{equation*}
    D\norm{\variance_\mu \left[ \tilde{K}_0(\delta)
    \right]}_{\mathcal{Y},\mathcal{Y}} \leq
    \frac{1}{2}\norm{\frac{1}{\sigma^2}K_0(2\delta)-2
    K_0(\delta)^2}_{\mathcal{Y},\mathcal{Y}} + \frac{(d+1)}{\sigma^4}.
\end{equation*}
This bound is illustrated by \cref{fig:approximation_error} B, for a given
datapoint. Eventually for the Gaussian divergence-free kernel,
$A(\omega)=I\norm{\omega}_2^2-\omega\omega^*$, thus $\expectation_\mu
[A(\omega)] = I_d(d-1)/\sigma^2$ and $
\variance_{\mu}[A(\omega)]=d(4d-3)I_d/\sigma^4$. Hence,
\begin{equation*}
    D\norm{\variance_\mu \left[ \tilde{K}_0(\delta)
    \right]}_{\mathcal{Y},\mathcal{Y}} \leq
    \frac{1}{2}\norm{\frac{(d-1)}{\sigma^2}K_0(2\delta)-2
    K_0(\delta)^2}_{\mathcal{Y}, \mathcal{Y}}+ \frac{d(4d-3)}{\sigma^4}.
\end{equation*}
To conclude, we ensure that the random variable $\norm{A(\omega)}_{\mathcal{Y},
\mathcal{Y}}$ has a finite Orlicz norm with $\psi=\psi_1$ in these three cases.
\subsubsection{Computing the Orlicz norm}
for a random variable with strictly monotonic moment generating function (MGF),
one can characterize its inverse $\psi_1$ Orlicz norm by taking the functional
inverse of the MGF evaluated at 2 (see \cref{lm:orlicz_mgf} of the
appendix). In other words
$\norm{X}_{\psi_1}^{-1}=\MGF(x)^{-1}_X(2)$. For the Gaussian curl-free and
divergence-free kernel,
\begin{dmath*}
    \norm{A^{div}(\omega)}_{\mathcal{Y},\mathcal{Y}} =
    \norm{A^{curl}(\omega)}_{\mathcal{Y},\mathcal{Y}} \hiderel{=}
    \norm{\omega}_{2}^2,
\end{dmath*}
where$\omega\sim\mathcal{N}(0,I_d/\sigma^2)$, hence $\norm{A(\omega)}_2\sim
\Gamma(p/2,2/\sigma^2)$. The MGF of this gamma distribution is
$\MGF(x)(t)=(1-2t/\sigma^2)^{-(p/2)}$. Eventually
\begin{equation*}
    \norm{\norm{A^{div}(\omega)}_{\mathcal{Y},\mathcal{Y}}}_{\psi_1}^{-1} =
    \norm{\norm{A^{curl}(\omega)}_{\mathcal{Y},\mathcal{Y}}}_{\psi_1}^{-1} =
    \frac{\sigma^2}{2}\left(1-4^{-\frac{1}{p}}\right).
\end{equation*}

\section{Conclusions}
In this chapter we have seen how to bound $\norm{\widetilde{K} - K}$ in the
operator norm with high probability
(\cref{sec:consistency_of_the_ORFF_estimator}). We studied the case of
unbounded finite dimensional \acsp{OVK} and bounded potentially infinite
dimensional \acsp{OVK}. The current lack of concentration inequalities working
for both unbounded and infinite dimensional with the operator norm (Ba\-nach
space) in the literature provides us to unify these bounds.

\chapterend
% \clearpage
