\section{Learning with ORFF}
\label{sec:learning_with_operator-valued_random-fourier_features}
We now turn our attention to learning function with an ORFF model that approximate an OVK model.
\subsection{Warm-up: supervised regression}
Let $\seq{s} = (x_i,y_i)_{i=1}^N\in\left(\mathcal{X}\times\mathcal{Y}\right)^N$ be a sequence of training samples. Given a local loss function $L: \mathcal{X}\times\mathcal{F}\times\mathcal{Y}\to \overline{\mathbb{R}}$ such that $L$ is proper, convex and lower semi-continous in $f$, we are interested in finding a \emph{vector-valued function} $f_{\seq{s}}:\mathcal{X}\to\mathcal{Y}$, that lives in a \acs{vv-RKHS} and minimize a tradeoff between a data fitting term $L$ and a regularization term to prevent from overfitting. Namely finding $f_{\seq{s}}\in\mathcal{H}_K$ such that
\begin{dmath}
f_{\seq{s}} = \argmin_{f\in\mathcal{H}_K}  \frac{1}{N}\displaystyle\sum_{i=1}^NL(x_i, f, y_i) + \frac{\lambda}{2}\norm{f}^2_{K},
\label{eq:learning_rkhs}
\end{dmath}
where $\lambda\in\mathbb{R}_+$ is a regularization\mpar{Tychonov regularization.} parameter. We call the quantity
\begin{dmath*}
\mathcal{R}(f)=\frac{1}{N}\displaystyle\sum_{i=1}^NL(x_i, f, y_i) \condition{$\forall f\in\mathcal{H}_K$, $\forall \seq{s}\in\left(\mathcal{X}\times\mathcal{Y}\right)^N$.}
\end{dmath*}
The empirical risk of the model $f\in\mathcal{H}_K$. A common choice of data fitting term for regression is $L:(x_i, f, y_i) \mapsto \norm{f(x_i)-y_i}_{\mathcal{Y}}^2$.
We introduce a corollary from Mazur and Schauder proposed in 1936 (see \citet{kurdila2006convex, gorniewicz1999topological}) showing that \cref{eq:learning_rkhs} --and \cref{eq:learning_rkhs_gen}-- attains a unique mimimizer.
\begin{theorem}[Mazur-Schauder]
\label{cor:unique_minimizer}
Let $\mathcal{H}$ be a Hilbert space and $J:\mathcal{H}\to \overline{\mathbb{R}}$ be a proper, convex, lower semi-continuous and coercive function. Then $J$ is bounded from below and attains a minimizer. Moreover if $J$ is strictly convex the minimizer is unique.
\end{theorem}
This is easily verified for Ridge regression. Define
\begin{dmath}
\label{eq:ridge}
J_\lambda(f)=\frac{1}{N}\sum_{i=1}^N\norm{f(x_i)-y_i}_{\mathcal{Y}}^2+\frac{\lambda}{2}\norm{f}_K^2,
\end{dmath}
where $f\in\mathcal{H}_K$ and $\lambda\in\mathbb{R}_{>0}$. $J_\lambda$ is continuous\mpar{Reminder, if $f\in\mathcal{H}_k, \text{ev}_x:f\mapsto f(x)$ is continuous, see \cref{pr:unique_rkhs}.} and strictly convex. Additionally $J_\lambda$ is coercive since $\norm{f}_K$ is coercive, $\lambda\in\mathbb{R}_{>0}$, and all the summands of $J_\lambda$ are positive. Hence for all positive $\lambda$, $f_{\seq{s}}=\argmin_{f\in\mathcal{H}_K}J_\lambda(f)$ exists, is unique and attained.
\begin{remark}
\label{rk:rkhs_bound}
We condider the optimization problem proposed in \cref{eq:ridge} where $L:(x_i, f, y_i) \mapsto \norm{f(x_i)-y_i}_{\mathcal{Y}}^2$. If given a training sample $\seq{s}$, we have
\begin{dmath*}
\frac{1}{N}\sum_{i=1}^N\norm{y_i}_{\mathcal{Y}}^2 \le \sigma_y^2,
\end{dmath*}
then $\lambda\norm{f_{\seq{s}}}_K\le 2\sigma_y^2$. Indeed, since $\mathcal{H}_K$ is a Hilbert space, $0\in\mathcal{H}_K$, thus
\begin{dmath*}
\frac{\lambda}{2}\norm{f_{\seq{s}}}^2_{K} \le \frac{1}{N}\displaystyle\sum_{i=1}^NL(x_i, f_{\seq{s}}, y_i) + \frac{\lambda}{2}\norm{f_{\seq{s}}}^2_{K}
\le \frac{1}{N}\displaystyle\sum_{i=1}^NL(x_i, 0, y_i) \hiderel{\le} \sigma_y^2 \condition{by optimality of $f_{\seq{s}}$.}
\end{dmath*}
Since for all $x\in\mathcal{X}$, $\norm{f(x)}_{\mathcal{Y}}\le \sqrt{\norm{K(x, x)}_{\mathcal{Y},\mathcal{Y}}}\norm{f}_{K}$, the maximum value that the solution $\norm{f_{\seq{s}}(x)}_{\mathcal{Y}}$ of \cref{eq:ridge} can reach is $2\sqrt{\norm{K(x, x)}}\frac{\sigma_y^2}{\lambda}$. Thus when solving a Ridge regression problem, given a shift-invariant kernel $K_e$, one should choose
\begin{dmath*}
0 \hiderel{<} \lambda \hiderel{\le} 2\frac{\sqrt{\norm{K_e(e)}}\sigma_y^2}{C}.
\end{dmath*}
with $C\in\mathbb{R}_{>0}$ to have a chance to fit all the $y_i$ with norm $\norm{y_i}_{\mathcal{Y}} \le C$ in the train set.
\end{remark}
\subsection{Semi-supervised regression}
Regression in \acl{vv-RKHS} has been well studied \citep{Alvarez2012, Minh_icml13,minh2016unifying,sangnier2016joint,kadri2015operator,Micchelli2005,Brouard2016_jmlr}, and a cornerstone of learning in \acs{vv-RKHS} is the representer theorem\mpar{Sometimes referred to as minimal norm interpolation theorem.}, which allows to replace the search of a minimizer in a infinite dimensional \acs{vv-RKHS} by a finite number of paramaters $(u_i)_{i=1}^N$, $u_i\in\mathcal{Y}$. We present here the very genreal form of \citet{minh2016unifying}. This framework encompass Vector-valued Manifold Regularization \citep{belkin2006manifold,Brouard2011,minh2013unifying} and Co-regularized Multi-view Learning \citep{brefeld2006efficient,sindhwani2008rkhs,rosenberg2009kernel,sun2011multi}.
\paragraph{}
In the following we suppose we are given a cost function $c:\mathcal{Y}\times\mathcal{Y}\to\overline{\mathbb{R}}$, such that $c(f(x),y)$ returns the error of the prediction $f(x)$ \wrt~the ground truth $y$. A loss function of a model $f$ with respect to an example $(x,y)\in\mathcal{X}\times\mathcal{Y}$ can be naturally defined from a cost function as $L(x,f,y)=c(f(x),y)$. Conceptually the function $c$ evaluate the quality of the prediction versus its ground truth $y\in\mathcal{Y}$ while the loss function $L$ evaluate the quality of the model $f$ at a training point $(x,y)\in\mathcal{X}\times\mathcal{Y}$. Moreover we suppose that we are given a training sample $\seq{u}=(x_i)_{i=N}^{N+U}\in\mathcal{X}^U$ of unlabelled exemple. We note $\seq{z}\in\left(\mathcal{X}\times\mathcal{Y}\right)^N\times\mathcal{X}^U$ the sequence $\seq{z}=\seq{s}\seq{u}$ concatenating both labeled ($\seq{s}$) and unlabelled ($\seq{u}$) training examples.
\begin{theorem}[Representer \citep{minh2016unifying}]
\label{th:representer}
Let $K$ be a $\mathcal{U}$-Mercer \acl{OVK} and $\mathcal{H}_K$ its corresponding $\mathcal{U}$-Reproducing Kernel Hilbert space.
\paragraph{}
Let $V:\mathcal{U}\to\mathcal{Y}$ be a bounded linear operator and let $c:\mathcal{Y}\times\mathcal{Y}\to\overline{\mathbb{R}}$ be a cost function such that $L(x, f, y)=c(Vf(x), y)$ is a proper convex lower semi-continuous function in $f$ for all $x\in\mathcal{X}$ and all $y\in\mathcal{Y}$.
\paragraph{}
Eventually let $\lambda_K\in\mathbb{R}_{>0}$ and $\lambda_M \in \mathbb{R}_+$ be two regularization hyperparameters and $(M_{ik})_{i,k=1}^{N+U}$ be a sequence of data dependent bounded linear operators in $\mathcal{L}(\mathcal{U})$, such that
\begin{dmath*}
\sum_{i,j=1}^{N+U} \inner{u_i, M_{ik}u_k} \ge 0 \condition{$\forall (u_i)_{i=1}^{N+U}\in\mathcal{U}^{N+U}$ and $M_{ik}=M_{ki}^*$}.
\end{dmath*}
The solution $f_{\seq{z}}\in\mathcal{H}_K$ of the regularized optimization problem
\begin{dmath}
f_{\seq{z}} = \argmin_{f\in\mathcal{H}_K} \frac{1}{N}\displaystyle\sum_{i=1}^N c(Vf(x_i), y_i) + \frac{\lambda_K}{2}\norm{f}^2_{K} \\ + \frac{\lambda_M}{2}\sum_{i,k=1}^{N+U}\inner{f(x_i), M_{ik}f(x_k)}_{\mathcal{U}}
\label{eq:learning_rkhs_gen}
\end{dmath}
has the form $f_{\seq{z}}=\sum_{j=1}^{N+U}K(\cdot,x_j)u_{\seq{z},j}$ where $u_{\seq{z},j}\in\mathcal{U}$ and
\begin{dmath}
    \label{eq:argmin_u}
    u_{\seq{z}} = \argmin_{u\in\Vect_{i=1}^{N+U}\mathcal{U}}\frac{1}{N}\displaystyle\sum_{i=1}^N c\left(V\sum_{k=1}^{N+U}K(x_i,x_j)u_j, y_i\right) + \frac{\lambda_K}{2}\sum_{k=1}^{N+U}u_i^\adjoint K(x_i,x_k)u_k \\ +
    \frac{\lambda_M}{2}\sum_{i,k=1}^{N+U} \inner*{\sum_{j=1}^{N+U}K(x_i,x_j)u_j, M_{ik}\sum_{j=1}^{N+U}K(x_k,x_j)u_j}_{\mathcal{U}}
\end{dmath}
\end{theorem}
The first representer theorem was first introduced by \citet{Wahba90} in the case where $\mathcal{Y}=\mathbb{R}$. The extension to an arbitrary Hilbert space $\mathcal{Y}$ has been proved by many authors in different forms \citep{Brouard2011,kadri2015operator,Micchelli2005}. The idea behind the representer theorem is that eventhough we minimize over the whole space $\mathcal{H}_K$, when $\lambda_K>0$, the solution of \cref{eq:learning_rkhs_gen} falls inevitably into the set $\mathcal{H}_{K, \seq{z}}=\Set{\sum_{j=1}^{N+U}K_{x_j}u_j| \forall (u_i)_{i=1}^{N+U} \in\mathcal{U}^{N+U}}$. Therefore the result can be expressed as a finite linear combination of basis functions of the form $K(\cdot,x_k)$. Remark that we can perform the kernel expansion of $f_{\seq{z}}=\sum_{j=1}^{N+U}K(\cdot,x_j)u_{\seq{z},j}$ eventhough $\lambda_K=0$. However $f_{\seq{z}}$ is no longer the solution of \cref{eq:learning_rkhs_gen} over the whole space $\mathcal{H}_K$ but a projection on the subspace $\mathcal{H}_{K, \seq{z}}$. While this is in general not a problem for practical applications, it might raise issues for further theoretical investigations. In particular, it makes it difficult to perform theoretical comparison the \say{exact} solution of \cref{eq:learning_rkhs_gen} with respect to the \acs{ORFF} approximation solution given in \cref{cr:orff_representer}.
\paragraph{}
We present here the proof of the generic formulation proposed by \citet{minh2016unifying}. In the mean time we clarify some elements of the proof. Indeed the existence of a global minimizer is not trivial and we must invoke the Mazur-Schauder theorem. Moreover the coercivity of the objective function required by the Mazur-Schauder theorem is not obvious when we do not require the cost function to take only positive values. However a corollary of Hahn-Banach theorem linking strong convexity to coercivity gives the solution.
\begin{proof}
Since $f(x)=K_x^*f$ (see \cref{eq:reproducing_prop}), the optimization problem reads
\begin{dmath*}
f_{\seq{z}} = \argmin_{f\in\mathcal{H}_K} \frac{1}{N}\displaystyle\sum_{i=1}^N c(VK_{x_i}^\adjoint f, y_i) + \frac{\lambda_K}{2}\norm{f}^2_{K} \\ + \frac{\lambda_M}{2}\sum_{i,k=1}^{N+U}\inner{K_{x_i}^\adjoint f, M_{ik}K_{x_k}^\adjoint f}_{\mathcal{U}}
\end{dmath*}
Let $W_{V,\seq{s}}:\mathcal{H}_K\to\Vect_{i=1}^N\mathcal{Y}$ be a linear operator defined as
\begin{dmath*}
W_{V,\seq{s}}f = \Vect_{i=1}^N CK_{x_i}^\adjoint f,
\end{dmath*}
with $VK_{x_i}^\adjoint:\mathcal{H}_K\to\mathcal{Y}$ and $K_{x_i}V^\adjoint:\mathcal{Y}\to\mathcal{H}_K$. Let $Y=\vect_{i=1}^Ny_i\in\mathcal{Y}^N$. We have
\begin{dmath*}
\inner{Y, W_{V,\seq{s}}f}_{\Vect_{i=1}^N\mathcal{Y}}=\sum_{i=1}^N\inner{y_i, VK_{x_i}^\adjoint f}_{\mathcal{Y}}
\hiderel{=}\sum_{i=1}^N\inner{K_{x_i}V^\adjoint y_i, f}_{\mathcal{H}_K}.
\end{dmath*}
Thus the adjoint operator $W_{V,\seq{s}}^\adjoint:\Vect_{i=1}^N\mathcal{Y}\to\mathcal{H}_K$ is
\begin{dmath*}
W_{V,\seq{s}}^\adjoint Y=\sum_{i=1}^NK_{x_i}V^\adjoint y_i,
\end{dmath*}
and the operator $W_{V,\seq{s}}^*W_{V,\seq{s}}:\mathcal{H}_K\to\mathcal{H}_K$ is
\begin{dmath*}
W_{V,\seq{s}}^\adjoint W_{V,\seq{s}}f = \sum_{i=1}^NK_{x_i}V^\adjoint VK_{x_i}^\adjoint f
\end{dmath*}
where $V^\adjoint V\in\mathcal{L}(\mathcal{U})$. Let
\begin{dmath*}
J_{\lambda_K}(f) = \underbrace{\frac{1}{N}\displaystyle\sum_{i=1}^N c(Vf(x_i), y_i)}_{=J_c} + \frac{\lambda_K}{2}\norm{f}^2_{K} \\ + \underbrace{\frac{\lambda_M}{2}\sum_{i,k=1}^{N+U}\inner{f(x_i), M_{ik}f(x_k)}_{\mathcal{U}}}_{=J_M}
\end{dmath*}
To ensure that $J_{\lambda_K}$ has a global minimizer we need the following technical lemma (which is a consequence of the Hahn-Banach theorem for lower-semicontimuous functional, see \citet{kurdila2006convex}).
\begin{lemma}
\label{lm:strongly_convex_is_coercive}
Let $J$ be a proper, convex, lower semi-continuous functional, defined on a Hilbert space $\mathcal{H}$. If $J$ is strongly convex, then $J$ is coercive.
\end{lemma}
\begin{proof}
Consider the convex function function $G(f)\colonequals J(f)-\lambda\norm{f}^2$, for some $\lambda>0$. Since $J$ is by assumption proper, lower semi-continuous and strongly convex with parameter $\lambda$, $G$ is proper, lower semi-continuous and convex. Thus Hahn-Banach theorem apply, stating that $G$ is bounded by below by an affine functional. \Ie~there exists $f_0$ and $f_1\in\mathcal{H}$ such that
\begin{dmath*}
G(f)\ge G(f_0) + \inner{f - f_0, f_1} \condition{for all $f\in\mathcal{H}$.}
\end{dmath*}
Then substitute the definition of $G$ to obtain
\begin{dmath*}
J(f)\ge J(f_0) + \lambda\left(\norm{f}-\norm{f_0}\right) + \inner{f - f_0, f_1}.
\end{dmath*}
By the Cauchy-Schwartz inequality, $\inner{f, f_1}\ge - \norm{f}\norm{f_1}$, thus
\begin{dmath*}
J(f)\ge J(f_0) + \lambda\left(\norm{f}-\norm{f_0}\right)  - \norm{f}\norm{f_1} - \inner{f_0, f_1},
\end{dmath*}
which tends to infinity as $f$ tends to infinity. Hence $J$ is coercive
\end{proof}
for all $f\in\mathcal{H}_K$. Since $c$ is proper, lower semi-continuous and convex by assumption, thus the term $J_c$ is also proper, lower semi-continuous and convex. Moreover the term $J_M$ is always positive for any $f\in\mathcal{H}_K$ and $\frac{\lambda_K}{2}\norm{f}^2_{K}$ is strongly convex. Thus $J_{\lambda_K}$ is strongly convex. Apply \cref{lm:strongly_convex_is_coercive} to obtain the coercivity of $J_{\lambda_K}$, and then \cref{cor:unique_minimizer} to show that $J_{\lambda_K}$ has a unique minimizer and is attained. Then let $\mathcal{H}_{K, \seq{z}}=\Set{\sum_{j=1}^{N+U}K_{x_j}u_j| \forall (u_i)_{i=1}^{N+U} \in\mathcal{U}^{N+U}}$. For $f\in\mathcal{H}_{K, \seq{z}}^\perp$\mpar{$\mathcal{H}_{K, \seq{z}}^\perp\oplus\mathcal{H}_{K, \seq{z}}=\mathcal{H_K}$.}, the operator $W_{V,\seq{s}}$ satisfies
\begin{dmath*}
\inner{Y, W_{V,\seq{s}}f}_{\Vect_{i=1}^N\mathcal{Y}} = \inner{\underbrace{f}_{\in\mathcal{H}_{K, \seq{z}}^\perp}, \underbrace{\sum_{i=1}^{N+U}K_{x_i}V^\adjoint y_i}_{\in\mathcal{H}_{K, \seq{z}}}}_{\mathcal{H}_K} \hiderel{=} 0
\end{dmath*}
for all sequences $(y_i)_{i=1}^N$, since $V^\adjoint y_i\in\mathcal{U}$. Hence,
\begin{dmath}
\label{eq:null1}
(Vf(x_i))_{i=1}^{N}=0
\end{dmath}
In the same way,
\begin{dmath*}
\sum_{i=1}^{N+U}\inner{K_{x_i}^* f, u_i}_{\mathcal{U}} \hiderel{=} \inner{\underbrace{f}_{\in\mathcal{H}_{K, \seq{z}}^\perp}, \underbrace{\sum_{j=1}^{N+U}K_{x_j}u_j}_{\in\mathcal{H}_{K, \seq{z}}}}_{\mathcal{H}_K} \hiderel{=} 0.
\end{dmath*}
for all sequences $(u_i)_{i=1}^{N+U}\in\mathcal{U}^{N+U}$. As a result,
\begin{dmath}
\label{eq:null2}
(f(x_i))_{i=1}^{U+N}=0.
\end{dmath}
Now for an arbitrary $f\in\mathcal{H_K}$, consider the orthogonal decomposition $f=f^{\perp}+f^{\parallel}$, where $f^{\perp}\in\mathcal{H}_{K, \seq{z}}^\perp$ and $f^{\parallel}\in\mathcal{H}_{K, \seq{z}}$. Then since $\norm{f^{\perp}+f^{\parallel}}_{\mathcal{H}_K}^2=\norm{f^{\perp}}_{\mathcal{H}_K}^2+\norm{f^{\parallel}}_{\mathcal{H}_K}^2$, \cref{eq:null1} and \cref{eq:null2} shows that if $\lambda_K\ge 0$, clearly then
\begin{dmath*}
J_{\lambda_K}(f)=J_{\lambda_K}\left(f^{\perp}+f^{\parallel}\right) \hiderel{\ge} J_{\lambda_K}\left(f^{\parallel}\right)
\end{dmath*}
The last inequality holds only when $\norm{f^{\perp}}_{\mathcal{H}_K}=0$, that is when $f^{\perp}=0$. As a result since the minimizer of $J_{\lambda_K}$is unique and attained, it must lies in $\mathcal{H}_{K, \seq{z}}$.
\end{proof}
The representer theorem show that minimizing a functional in a \acs{vv-RKHS} yields a solution which depends on all the points in the training set. Assuming that for all $x_i$, $x\in\mathcal{X}$ and for all $u_i\in\mathcal{Y}$ it takes time $O(P)$, to compute $K(x_i, x)u_i$, making a prediction using the representer theorem take $O(2P)$. Obviously If $\mathcal{Y}=\mathbb{R}^p$, Then $P=O(p^2)$ thus making a prediction cost $O(2p^2)$ operations.
\paragraph{}
Instead learning a model $f$ that depends on all the points of the training set, we would like to learn a parametric model of the form
$\tildef{\omega}(x)=\tildePhi{\omega}(x)^\adjoint \theta$, where $\theta$ lives in some redescription space $\tildeH{\omega}$. We are interested in finding a parameter vector $\theta_{\seq{z}}$ such that
\begin{dmath}
\label{eq:argmin_applied}
\theta_{\seq{z}}=\argmin_{\theta\in \tildeH{\omega}} \frac{1}{N}\sum_{i=1}^Nc\left(V\tildePhi{\omega}(x_i)^\adjoint \theta, y_i\right) + \frac{\lambda_K}{2}\norm{\theta}^2_{\tildeH{\omega}} \\ + \frac{\lambda_M}{2}\sum_{i,k=1}^{N+U}\inner{\theta, \tildePhi{\omega}(x_i)M_{ik}\tildePhi{\omega}(x_k)^\adjoint \theta}_{\tildeH{\omega}}
\end{dmath}

\begin{corollary}[\acs{ORFF} representer]
\label{cr:orff_representer}
Let $\tildeK{\omega}$ be an \acl{OVK} such that for all $x$, $z\in\mathcal{X}$, $\tildePhi{\omega}(x)^\adjoint \tildePhi{\omega}(z) = \widetilde{K}(x,z)$ where $\widetilde{K}$ is a $\mathcal{U}$-Mercer \acs{OVK} and $\mathcal{H}_{\tildeK{\omega}}$ its corresponding $\mathcal{U}$-Reproducing kernel Hilbert space.
\paragraph{}
Let $V:\mathcal{U}\to\mathcal{Y}$ be a bounded linear operator and let $c:\mathcal{Y}\times\mathcal{Y}\to\overline{\mathbb{R}}$ be a cost function such that $L(x, \widetilde{f}, y)=c(V\widetilde{f}(x), y)$ is a proper convex lower semi-continuous function in $\widetilde{f}\in\mathcal{H}_{\tildeK{\omega}}$ for all $x\in\mathcal{X}$ and all $y\in\mathcal{Y}$.
\paragraph{}
Eventually let $\lambda_K\in\mathbb{R}_{>0}$ and $\lambda_M \in \mathbb{R}_+$ be two regularization hyperparameters and $(M_{ik})_{i,k=1}^{N+U}$ be a sequence of data dependent bounded linear operators in $\mathcal{L}(\mathcal{U})$, such that
\begin{dmath*}
\sum_{i,j=1}^{N+U} \inner{u_i, M_{ik}u_k} \ge 0 \condition{$\forall (u_i)_{i=1}^{N+U}\in\mathcal{U}^{N+U}$ and $M_{ik}=M_{ki}^*$}.
\end{dmath*}
The solution $f_{\seq{z}}\in\mathcal{H}_{\tildeK{\omega}}$ of the regularized optimization problem
\begin{dmath}
\label{eq:argmin_RKHS_rand}
\widetilde{f}_{\seq{z}} = \argmin_{\widetilde{f}\in\mathcal{H}_{\tildeK{\omega}}} \frac{1}{N}\displaystyle\sum_{i=1}^N c\left(V\widetilde{f}(x_i), y_i\right) + \frac{\lambda_K}{2}\norm{\widetilde{f}}^2_{\tildeK{\omega}} \\ + \frac{\lambda_M}{2}\sum_{i,k=1}^{N+U}\inner{\widetilde{f}(x_i), M_{ik}\widetilde{f}(x_k)}_{\mathcal{U}}
\end{dmath}
has the form $\widetilde{f}_{\seq{z}}=\tildePhi{\omega}(\cdot)^\adjoint\theta_{\seq{z}}$, where $\theta_{\seq{z}}\in (\Ker \tildeW{\omega})^{\perp}$ and
\begin{dmath}
\theta_{\seq{z}}=\argmin_{\theta\in \tildeH{\omega}} \frac{1}{N}\sum_{i=1}^Nc\left(V\tildePhi{\omega}(x_i)^\adjoint \theta, y_i\right) + \frac{\lambda_K}{2}\norm{\theta}^2_{\tildeH{\omega}} \\ + \frac{\lambda_M}{2}\sum_{i,k=1}^{N+U}\inner{\theta, \tildePhi{\omega}(x_i)M_{ik}\tildePhi{\omega}(x_k)^\adjoint \theta}_{\tildeH{\omega}}.
\end{dmath}
\end{corollary}
\begin{proof}
Since $\tildeK{\omega}$ is an operator-valued kernel, from \cref{th:representer}, \cref{eq:argmin_RKHS_rand} has a solution of the form
\begin{dmath*}
\widetilde{f}_{\seq{z}} = \sum_{i=1}^{N+U} \tildeK{\omega}(\cdot, x_i)u_i \condition{$u_i \hiderel{\in} \mathcal{U}$, $x_i \hiderel{\in}\mathcal{X}$}
= \sum_{i=1}^N \tildePhi{\omega}(\cdot)^\adjoint \tildePhi{\omega}(x_i)u_i
\hiderel{=} \tildePhi{\omega}(\cdot)^\adjoint \underbrace{\left(\sum_{i=1}^{N+U}\tildePhi{\omega}(x_i)u_i\right)}_{=\theta\in \left(\Ker \tildeW{\omega}\right)^\perp\subset \tildeH{\omega}}.
\end{dmath*}
Let
\begin{dmath*}
\label{eq:argmin_theta}
\theta_{\seq{z}}=\argmin_{\theta\in\left(\Ker \tildeW{\omega}\right)^\perp} \frac{1}{N}\sum_{i=1}^Nc\left(V\tildePhi{\omega}(x_i)^\adjoint \theta, y_i\right) + \frac{\lambda_K}{2}\norm{\tildePhi{\omega}(\cdot)^\adjoint\theta}^2_{\tildeK{\omega}} \\ + \frac{\lambda_M}{2}\sum_{i,k=1}^{N+U}\inner*{\tildePhi{\omega}(x_i)^\adjoint \theta, M_{ik}\tildePhi{\omega}(x_k)^\adjoint \theta}_{\mathcal{U}}.
\end{dmath*}
Since $\theta\in(\Ker \tildeW{\omega})^\perp$ and $W$ is an isometry from $(\Ker \tildeW{\omega})^\perp\subset \tildeH{\omega}$ onto $\mathcal{H}_{\tildeK{\omega}}$, we have $\norm{\tildePhi{\omega}(\cdot)^\adjoint\theta}^2_{\tildeK{\omega}} = \norm{\theta}^2_{\tildeH{\omega}}$. Hence
\begin{dmath*}
\theta_{\seq{z}}=\argmin_{\theta\in\left(\Ker \tildeW{\omega}\right)^\perp} \frac{1}{N}\sum_{i=1}^Nc\left(V\tildePhi{\omega}(x_i)^\adjoint \theta, y_i\right) + \frac{\lambda_K}{2}\norm{\theta}^2_{\tildeH{\omega}} \\ + \frac{\lambda_M}{2}\sum_{i,k=1}^{N+U}\inner{\tildePhi{\omega}(x_i)^\adjoint \theta, M_{ik}\tildePhi{\omega}(x_k)^\adjoint \theta}_{\mathcal{U}}.
\end{dmath*}
Finding a minimizer $\theta_{\seq{z}}$ over $\left(\Ker \tildeW{\omega}\right)^\perp$ is not the same than finding a minimizer over $\tildeH{\omega}$. Although in both cases Mazur-Schauder's theorem guarantee that the respective minimizers are unique, they might not be the same. Since $\tildeW{\omega}$ is bounded, $\Ker \tildeW{\omega}$ is closed, so that we can perform the decomposition $\tildeH{\omega}=\left(\Ker \tildeW{\omega}\right)^\perp\oplus \left(\Ker \tildeW{\omega}\right)$. Then clearly by linearity of $W$ and the fact that for all $\theta^{\parallel}\in\Ker \tildeW{\omega}$, $\tildeW{\omega}\theta^{\parallel}=0$, if $\lambda > 0$ we have
\begin{dmath*}
\theta_{\seq{z}}=\argmin_{\theta\in\tildeH{\omega}} \frac{1}{N}\sum_{i=1}^Nc\left(V\tildePhi{\omega}(x_i)^\adjoint \theta, y_i\right) + \frac{\lambda_K}{2}\norm{\theta}^2_{\tildeH{\omega}} \\ + \frac{\lambda_M}{2}\sum_{i,k=1}^{N+U}\inner*{\tildePhi{\omega}(x_i)^\adjoint \theta, M_{ik}\tildePhi{\omega}(x_k)^\adjoint \theta}_{\mathcal{U}}
\end{dmath*}
Thus
\begin{dmath*}
\theta_{\seq{z}}=\argmin_{\substack{\theta^{\perp}\in\left(\Ker \tildeW{\omega}\right)^\perp, \\ \theta^{\parallel}\in\Ker \tildeW{\omega}}} \frac{1}{N}\sum_{i=1}^Nc\left(V\left(\tildeW{\omega}\theta^{\perp}\right)(x)+\underbrace{V\left(\tildeW{\omega}\theta^{\parallel}\right)(x)}_{=0\enskip\text{for all}\enskip \theta^{\parallel} }, y_i\right) \\ + \frac{\lambda_K}{2}\norm{\theta^\perp}^2_{\tildeH{\omega}} + \underbrace{\frac{\lambda}{2}\norm{\theta^{\parallel}}^2_{\tildeH{\omega}}}_{=0 \enskip\text{only if}\enskip \theta^{\parallel}=0} \\ + \frac{\lambda_M}{2}\sum_{i,k=1}^{N+U}\inner*{\tildePhi{\omega}(x_i)^\adjoint \theta^{\perp}, M_{ik}\left(\tildeW{\omega}\theta^{\perp}\right)(x_k)}_{\mathcal{U}} \\
+ \frac{\lambda_M}{2}\sum_{i,k=1}^{N+U}\inner*{\underbrace{\left(\tildeW{\omega}\theta^{\parallel}\right)(x_i)}_{=0\enskip\text{for all}\enskip \theta^{\parallel} }, M_{ik}\left(\tildeW{\omega}\theta^{\perp}\right)(x_k)}_{\mathcal{U}}
\\ + \frac{\lambda_M}{2}\sum_{i,k=1}^{N+U}\inner*{\left(\tildeW{\omega}\theta^{\perp}\right)(x_i), M_{ik}\underbrace{\left(\tildeW{\omega}\theta^{\parallel}\right)(x_k)}_{=0\enskip\text{for all}\enskip \theta^{\parallel} }}_{\mathcal{U}} \\ + \frac{\lambda_M}{2}\sum_{i,k=1}^{N+U}\inner*{\underbrace{\left(\tildeW{\omega}\theta^{\parallel}\right)(x_i)}_{=0\enskip\text{for all}\enskip \theta^{\parallel} }, M_{ik}\underbrace{\left(\tildeW{\omega}\theta^{\parallel}\right)(x_k)}_{=0\enskip\text{for all}\enskip \theta^{\parallel} }}_{\mathcal{U}}.
\end{dmath*}
Thus
\begin{dmath*}
\theta_{\seq{z}}=\argmin_{\theta^{\perp}\in\left(\Ker \tildeW{\omega}\right)^\perp}
\frac{1}{N}\sum_{i=1}^Nc\left(V\left(\tildeW{\omega}\theta^{\perp}\right)(x), y_i\right) + \frac{\lambda_K}{2}\norm{\theta^\perp}^2_{\tildeH{\omega}} \\ + \frac{\lambda_M}{2}\sum_{i,k=1}^{N+U}\inner*{\tildePhi{\omega}(x_i)^\adjoint \theta^{\perp}, M_{ik}\left(\tildeW{\omega}\theta^{\perp}\right)(x_k)}_{\mathcal{U}}.
\end{dmath*}
Hence minimizing over $\left(\Ker \tildeW{\omega}\right)^\perp$ or $\tildeH{\omega}$ is the same when $\lambda_K > 0$. Eventually,
% Eventually for any outcome of $\omega_j \sim \probability_{\dual{\Haar},\rho}$ \iid,
\begin{dmath*}
\theta_{\seq{z}}=\argmin_{\theta\in\tildeH{\omega}} \frac{1}{N}\sum_{i=1}^Nc\left(V\tildePhi{\omega}(x_i)^\adjoint \theta, y_i\right) + \frac{\lambda_K}{2}\norm{\theta}^2_{\tildeH{\omega}} \\ + \frac{\lambda_M}{2}\sum_{i,k=1}^{N+U}\inner*{\tildePhi{\omega}(x_i)^\adjoint\theta, M_{ik}\tildePhi{\omega}(x_k)^\adjoint \theta}_{\mathcal{U}}
=\argmin_{\theta \in \tildeH{\omega}} \frac{1}{N}\sum_{i=1}^Nc\left(V\tildePhi{\omega}(x_i)^\adjoint\theta, y_i\right) + \frac{\lambda_K}{2}\norm{\theta}^2_{\tildeH{\omega}} \\ + \frac{\lambda_M}{2}\sum_{i,k=1}^{N+U}\inner*{\theta, \tildePhi{\omega}(x_i)M_{ik}\tildePhi{\omega}(x_k)^\adjoint \theta}_{\tildeH{\omega}}. \qed
\end{dmath*}
\end{proof}
This shows that when $\lambda_K>0$ the solution of \cref{eq:argmin_u} with the approximated kernel $K(x,z) \approx \tildeK{\omega}(x,z) = \tildePhi{\omega}(x)^\adjoint\tildePhi{\omega}(z)$ is the same than the solution of \cref{eq:argmin_theta} up to a transformation. Namely, if $u_{\seq{z}}$ is the solution of \cref{eq:argmin_u}, $\theta_{\seq{z}}$ is the solution of \cref{eq:argmin_theta} and $\lambda_K>0$ we have
\begin{dmath*}
\theta_{\seq{z}} = \sum_{j=1}^{N+U} \tildePhi{\omega}(x_j) u_{\seq{z}} \hiderel{\in}\tildeH{\omega}.
\end{dmath*}
If $\lambda_K=0$ we can still find a solution $u_{\seq{z}}$ of \cref{eq:argmin_u}. By construction of the kernel expansion, we have $u_{\seq{z}}\in(\Ker W)^\bot$. However looking at the proof of \cref{cr:orff_representer} we see that $\theta_{\seq{z}}$ might \emph{not} g belong to $(\Ker W)^\bot$. Then we can compute a residual vector
\begin{dmath*}
r_{\seq{z}} = \theta_{\seq{z}} - \sum_{j=1}^{N+U} \tildePhi{\omega}(x_j) u_{\seq{z}} \hiderel{\in}\tildeH{\omega}.
\end{dmath*}
Then if $r_{\seq{z}}=0$, it means that $\lambda_K$ is large enough for both reprensenter theorem and \acs{ORFF} representer theorem to apply. If $r_{\seq{z}}\neq0$ but $\tildePhi{\omega}(\cdot)^\adjoint r_{\seq{z}} = 0$ means that both $\theta_{\seq{z}}$ and $\sum_{j=1}^{N+U} \tildePhi{\omega}(x_j) u_{\seq{z}}$ are in $(\Ker W)^\bot$, thus the representer theorem fails to find the \say{true} solution over the whole space $\mathcal{H}_K$ but returns a projection onto $\mathcal{H}_{\tildeK{\omega},\seq{z}}$ of the solution. If $r_{\seq{z}} \neq 0$ and $\tildePhi{\omega}(\cdot)^\adjoint r_{\seq{z}} \neq 0$ means that $\theta_{\seq{z}}$ is \emph{not} in $(\Ker W)^\bot$, thus the \acs{ORFF} representer theorem fails to apply. This remark is illustrated by \cref{fig:representer}.

\section{Solution of the empirical risk minimization}
% We illustrate the ORFF representer theorem (\cref{cr:orff_representer}) on two experiment involving scalar valued kernels.

\subsection{Gradient methods}
\label{subsec:gradient_methods}
In order to find a solution to \cref{eq:argmin_theta}, we turn our attention to gradient descent methods. In the following we let
\begin{dmath}
\label{eq:cost_functional}
J_{\lambda_K}(\theta) = \frac{1}{N}\sum_{i=1}^Nc\left(V\tildePhi{\omega}(x_i)^\adjoint \theta, y_i\right) + \frac{\lambda_K}{2}\norm{\theta}^2_{\tildeH{\omega}} \\ + \frac{\lambda_M}{2}\sum_{i,k=1}^{N+U}\inner{\theta, \tildePhi{\omega}(x_i)M_{ik}\tildePhi{\omega}(x_k)^\adjoint \theta}_{\tildeH{\omega}}.
\end{dmath}
Since the solution of \cref{eq:argmin_theta} is unique when $\lambda_K>0$, a sufficient and necessary condition is that the gradient of $J_{\lambda_K}$ at the minimizer $\theta_{\seq{z}}$ is zero. We use the Frechet derivative, the strongest notion of derivative in Banach spaces\mpar{Here we view the Hilbert space $\mathcal{H}$ (feature space) as a reflexive Banach space.} \cite{conway2013course, kurdila2006convex} wich directly generalizes the notion of gradient to Banach spaces. A function $f:\mathcal{H}_0\to\mathcal{H}_1$ is call Frechet differentiable at $\theta_0\in \mathcal{H}_0$ if there exist a bounded linear operator $A:\mathcal{H}_0\to \mathcal{H}_1$ such that
\begin{dmath*}
\lim_{h\to 0} \frac{\norm{f(\theta_0+h)-f(\theta_0)-Ah}_{\mathcal{H}_1}}{\norm{h}_{\mathcal{H}_0}}=0
\end{dmath*}
We write
\begin{dmath*}
(D_Ff)(\theta_0)\hiderel{=}\derivativeat{f(\theta)}{\theta}{\theta_0}\hiderel{=}A
\end{dmath*}
and call it Frechet derivative of $f$ with respect to $\theta$ at $\theta_0$. With mild abuse of notation we write $\derivative{f(\theta_0)}{\theta_0} = \derivativeat{f(\theta)}{\theta}{\theta_0}$. The Frechet derivative is an unbounded linear operator. The chain rule is valid in this context. Namely if a function $f:\mathcal{H}_0\to\mathcal{H}_1$ is Frechet differentiable at $\theta$ and $g:\mathcal{H}_1\to \mathcal{H}_2$ is Frechet differentiable at $f(\theta)$ then $g\circ f$ is Frechet differentiable at $\theta$ and
\begin{dmath*}
\lderivative{(g\circ f)(\theta)}{\theta} = \derivative{g(f(\theta))}{f(\theta)} \circ \derivative{f(\theta)}{\theta}.
\end{dmath*}
If $f:\mathcal{H}\to\mathbb{R}$ we define the gradient of $f$ as
\begin{dmath*}
\nabla_{\theta} f(\theta) = \left(\derivative{f(\theta)}{\theta}\right)^\adjoint
\end{dmath*}
and in the same way, for a function $f:\mathcal{H}_0\to\mathcal{H}_1$ the jacobian of $f$ as
\begin{dmath*}
\jacobian_{\theta} f(\theta) = \left(\derivative{f(\theta)}{\theta}\right)^\adjoint.
\end{dmath*}
In this context if $f:\mathcal{H}_0\to\mathcal{H}_1$ and $g:\mathcal{H}_1\to\mathbb{R}$ the chain rule reads
\begin{dmath*}
\nabla_{\theta} (g\circ f)(\theta) = \jacobian_{\theta}f(\theta) \circ \nabla_{f(\theta)}g(f(\theta)).
\end{dmath*}
We consider that $\mathcal{U}$ and $\mathcal{U}'$ are both isometrically isomorphic to some $\ell^2$ such that $\nabla_{\theta} \norm{\theta}^2 = 2\theta$. By linearity and applying the chaine rule to \cref{eq:argmin_theta} and since $M_{ik}^\adjoint = M_{ki}$ for all $i$, $k\in\mathbb{N}_{N+U}$, we have
\begin{dgroup*}
\begin{dmath*}
\nabla_{\theta}c\left(V\tildePhi{\omega}(x_i)^\adjoint \theta, y_i\right)= \tildePhi{\omega}(x_i)V^\adjoint \left(\lderivativeat{c\left(y, y_i\right)}{y}{V\tildePhi{\omega}(x_i)^\adjoint \theta}\right)^\adjoint,
\end{dmath*}
\begin{dmath*}
\nabla_{\theta}\inner*{\tildePhi{\omega}(x_i)^\adjoint \theta, M_{ik}\tildePhi{\omega}(x_k)^\adjoint \theta}_{\mathcal{U}}=\tildePhi{\omega}(x_i)\left(M_{ik}+M_{ki}\right)\tildePhi{\omega}(x_k)^\adjoint \theta,
\end{dmath*}
\begin{dmath*}
\nabla_{\theta}\norm{\theta}^2_{\tildeH{\omega}}=2\theta.
\end{dmath*}
\end{dgroup*}
Provided that $c(y,y_i)$ is Frechet differentiable \wrt~$y$, for all $y$ and $y_i\in\mathcal{Y}$ we have
\begin{dmath*}
\nabla_{\theta} J_{\lambda_K}(\theta) = \frac{1}{N}\sum_{i=1}^N \tildePhi{\omega}(x_i)V^\adjoint \left(\lderivativeat{c\left(y, y_i\right)}{y}{V\tildePhi{\omega}(x_i)^\adjoint \theta}\right)^\adjoint + \lambda_K\theta + \lambda_M\sum_{i,k=1}^{N+U}\tildePhi{\omega}(x_i)M_{ik}\tildePhi{\omega}(x_k)^\adjoint \theta
\end{dmath*}
Therefore after factorization, considering $\lambda_K > 0$,
\begin{dmath*}
\nabla_{\theta} J_{\lambda_K}(\theta) = \frac{1}{N}\sum_{i=1}^N \tildePhi{\omega}(x_i)V^\adjoint \left(\lderivativeat{c\left(y, y_i\right)}{y}{V\tildePhi{\omega}(x_i)^\adjoint \theta}\right)^\adjoint + \lambda_K\left(I_{\tildeH{\omega}} + \frac{\lambda_M}{\lambda_K}\sum_{i,k=1}^{N+U}\tildePhi{\omega}(x_i)M_{ik}\tildePhi{\omega}(x_k)^\adjoint \right)\theta
\end{dmath*}
We note the quantity $\mathbf{\widetilde{M}}_{\left(\lambda_K,\lambda_M\right)}=I_{\tildeH{\omega}} + \frac{\lambda_M}{\lambda_K}\sum_{i,k=1}^{N+U}\tildePhi{\omega}(x_i)M_{ik}\tildePhi{\omega}(x_k)^\adjoint \in \mathcal{L}(\Vect_{j=1}^D\mathcal{U}')$ so that
\begin{dmath*}
\nabla_{\theta} J_{\lambda_K}(\theta) = \frac{1}{N}\sum_{i=1}^N \tildePhi{\omega}(x_i)V^\adjoint \left(\lderivativeat{c\left(y, y_i\right)}{y}{V\tildePhi{\omega}(x_i)^\adjoint \theta}\right)^\adjoint + \lambda_K\mathbf{\widetilde{M}}_{\left(\lambda_K,\lambda_M\right)}\theta
\end{dmath*}
\begin{example}[Naive close form for the squared error cost]
Consider the cost function defined for all $y$, $y'\in\mathcal{Y}$ by $c(y,y')=\norm{y-y}_2^2$. Then
\begin{dmath*}
\left(\lderivativeat{c\left(y, y_i\right)}{y}{V\tildePhi{\omega}(x_i)^\adjoint \theta}\right)^\adjoint = 2\left(V\tildePhi{\omega}(x_i)^\adjoint \theta-y_i\right).
\end{dmath*}
Thus, since the optimal solution $\theta_{\seq{z}}$ verifies $\nabla_{\theta_{\seq{z}}} J_{\lambda_K}(\theta_{\seq{z}}) = 0$ we have
\begin{dmath*}
\frac{1}{N}\sum_{i=1}^N \tildePhi{\omega}(x_i)V^\adjoint\left(V\tildePhi{\omega}(x_i)^\adjoint \theta_{\seq{z}}-y_i\right) + \lambda_K\mathbf{\widetilde{M}}_{\left(\lambda_K,\lambda_M\right)}\theta_{\seq{z}}= 0.
\end{dmath*}
Therefore,
\begin{dmath*}
\left(\frac{1}{N}\sum_{i=1}^N \tildePhi{\omega}(x_i)V^\adjoint V\tildePhi{\omega}(x_i)^\adjoint + \lambda_K\mathbf{\widetilde{M}}_{\left(\lambda_K,\lambda_M\right)}\right)\theta_{\seq{z}} = \frac{1}{N}\sum_{i=1}^N \tildePhi{\omega}(x_i)V^\adjoint y_i.
\end{dmath*}
Suppose that $\mathcal{Y}\subseteq\mathbb{R}^p$, $V:\mathcal{U}\to\mathcal{Y}$ where $\mathcal{U}\subseteq\mathbb{R}^u$ and for all $x\in\mathcal{X}$, $\tildePhi{\omega}(x): \mathbb{R}^{r}\to\mathbb{R}^u$. From this we can derive \cref{alg:close_form} which return the close form solution of \cref{eq:cost_functional} for $c(y,y')=\norm{y-y'}_2^2$.
\end{example}
\paragraph{Complexity analysis:}
\Cref{alg:close_form} constitute our first step toward large-scale learning with operator-valued kernels. We can easily compute the time complexity of \cref{alg:close_form} when all the operators act on finite dimensional Hilbert spaces. Suppose that $u=\dim(\mathcal{U})<+\infty$ and $u'=\dim(\mathcal{U}')<+\infty$ and for all $x\in\mathcal{X}$, $\tildePhi{\omega}(x):\mathcal{U}'\to\tildeH{\omega}$ where $r=\dim(\tildeH{\omega})<+\infty$ is the dimension of the redescription space $\tildeH{\omega}=\mathbb{R}^{r}$. Since $u$, $u'$, and $r<+\infty$, we view the operators $\tildePhi{\omega}(x)$, $V$ and $\mathbf{\widetilde{M}}_{\left(\lambda_K,\lambda_M\right)}$ as matrices. Computing $V^\adjoint V$ cost $O_t(u^2p)$. Step 1 costs $O_t(r^2u + ru^2)$. Steps 5 (optional) has the same cost except that the sum is done over all pair of $N+U$ points thus it costs $O_t((N+U)^2(r^2u + r u^2))$. Steps 7 costs $O_t(N(ru + up))$. For step 8, the naive inversion of the operator costs $O_t(r^3)$. Eventually the overall complexity of \cref{alg:close_form} is
\begin{dmath*}
O_t\left(ru(r + u) \begin{cases} (N+U)^2 & \text{if $\lambda_M > 0$} \\ N & \text{if $\lambda_M = 0$} \end{cases}+ r^3 + Nu(r+p)
\right),
\end{dmath*}
while the space complexity is $O_s(r^2)$.
\afterpage{%
\begin{center}
\begin{algorithm2e}[H]
    \label{alg:close_form}
    \SetAlgoLined
    \Input{\begin{itemize}
    \item $\seq{s}=(x_i, y_i)_{i=1}^N\in\left(\mathcal{X}\times\mathbb{R}^p\right)^N$ a sequence of supervised training points,
    \item $\seq{u}=(x_i)_{i=N+1}^{N+U}\in\mathcal{X}^{U}$ a sequence of unsupervised training points,
    \item $\tildePhi{\omega}(x_i) \in \left(\mathbb{R}^u, \mathbb{R}^{r}\right)$ a feature map defined for all $x_i\in\mathcal{X}$,
    \item $(M_{ik})_{i,k=1}^{N+U}$ a sequence of data dependent operators (see \cref{cr:orff_representer}),
    \item $V \in \mathcal{L}\left(\mathbb{R}^u, \to \mathbb{R}^p\right)$ a combination operator,
    \item $\lambda_K \in\mathbb{R}_{>0}$ the Tychonov regularization term,
    \item $\lambda_M \in\mathbb{R}_+$ the manifold regularization term.
    \end{itemize}}
    \Output{A model
    \begin{dmath*}
    h:\begin{cases} \mathcal{X} \to \mathbb{R}^p \\ x\mapsto\tildePhi{\omega}(x)^T\theta_{\seq{z}},\end{cases}
    \end{dmath*}
    such that $\theta_{\seq{z}}$ minimize \cref{eq:cost_functional}, where $c(y,y')=\norm{y-y'}_2^2$.}
    $\mathbf{P} \gets \frac{1}{N}\sum_{i=1}^N \tildePhi{\omega}(x_i)V^T V\tildePhi{\omega}(x_i)^T \in\mathcal{L}(\mathbb{R}^{r}, \mathbb{R}^{r})  $\;
    \uIf{$\lambda_M = 0$}{
        $\mathbf{\widetilde{M}}_{\left(\lambda_K,\lambda_M\right)} \gets I_{r} \in\mathcal{L}(\mathbb{R}^{r}, \mathbb{R}^{r})$\;
    }
    \Else{
        $\mathbf{\widetilde{M}}_{\left(\lambda_K,\lambda_M\right)} \gets \left(I_{r} + \frac{\lambda_M}{\lambda_K}\sum_{i,k=1}^{N+U}\tildePhi{\omega}(x_i)M_{ik}\tildePhi{\omega}(x_k)^T \right) \in\mathcal{L}(\mathbb{R}^{r}, \mathbb{R}^{r}) $\;
    }
    $\mathbf{Y} \gets \frac{1}{N}\sum_{i=1}^N \tildePhi{\omega}(x_i)V^T y_i \in \mathbb{R}^{r} $\;
    $\theta_{\seq{z}} \gets \left(P + \lambda_K\mathbf{\widetilde{M}}_{\left(\lambda_K,\lambda_M\right)}\right)^{-1}Y \in \mathbb{R}^{r} $\;
    \Return $h: x \mapsto \tildePhi{\omega}(x)^T\theta_{\seq{z}}$\;
    \caption{Naive close form for the squared error cost.}
\end{algorithm2e}
\end{center}}
This complexity is to compare with the kernelized solution proposed by \citet{minh2016unifying}. Let
\begin{dmath*}
\mathbf{K}:\begin{cases}
\mathcal{U}^{N+U} \to \mathcal{U}^{N+U} \\
u\mapsto\Vect_{i=1}^{N+U}\sum_{j=1}^{N+U}K(x_i, x_j)u_j
\end{cases}
\end{dmath*}
and
\begin{dmath*}
\mathbf{M}:\begin{cases}
\mathcal{U}^{N+U} \to \mathcal{U}^{N+U} \\
u\mapsto\Vect_{i=1}^{N+U}\sum_{k=1}^{N+U}M_{ik}u_k.
\end{cases}
\end{dmath*}
When $\mathcal{U}=\mathbb{R}$,
\begin{dmath*}
    \mathbf{K}=\begin{pmatrix} K(x_1, x_1) & \hdots & K(x_1, x_{N+U})
    \\ \vdots & \ddots & \vdots \\  K(x_{N+U}, x_1) & \hdots & K(x_{N+U}, x_{N+U}) \end{pmatrix}
\end{dmath*}
is called the Gram matrix of $K$. When $\mathcal{U}=\mathbb{R}^p$, $\mathbf{K}$ is a matrix-valued Gram matrix of size $u(N+U)\times u(N+U)$ where each entry $\mathbf{K}_{ij}\in\mathcal{L}(\mathbb{R}^u)$. When $\mathcal{U}=\mathbb{R}^u$, $\mathbf{M}$ can also be seen as a matrix-valued matrix where each entry is $M_{ik}\in\mathcal{L}(\mathbb{R}^u)$. We also introduce the operators $\mathbf{C}^T \mathbf{C}\colonequals I_{N+U} \otimes (C^T C)$ and
\begin{dmath*}
\mathbf{P}:\begin{cases}
\mathcal{U}^{N+U} \to \mathcal{U}^{N+U} \\
u\mapsto \left(\Vect_{j=1}^Nu_j\right) \oplus \left(\Vect_{j=N+1}^{N+U}0\right)
\end{cases}
\end{dmath*}
The operator $\mathbf{P}$ is a projection that sets all the terms $u_j$, $N < j \le N + U$ of $u$ to zero. When $\mathcal{U}=\mathbb{R}^u$ it can also be seen as the block matrix of size $u(N+U) \times u(N + U)$ and
\begin{dmath*}
    \mathbf{P}=\begin{pmatrix}  & & & 0 & \hdots & 0 \\ & I_u \otimes I_{N} & & \vdots & \ddots & \vdots \\ & & & 0 & \hdots & 0 \\
    0 & \hdots & 0 & 0 & \hdots & 0 \\
    \vdots & \ddots & \vdots & \vdots & \ddots & \vdots \\
    0 & \hdots & 0 & 0 & \hdots & 0
    \end{pmatrix}
\end{dmath*}
Then the equivalent kernelized solution $u_{\seq{z}}$ of \cref{th:representer} is given by \cite{minh2016unifying}
\begin{dmath*}
\left(\frac{1}{N}\mathbf{C}^T \mathbf{C} \mathbf{P} \mathbf{K} + \lambda_M \mathbf{M} \mathbf{K} + \lambda_K I_{\Vect_{i=1}^{N+U}\mathcal{U}}\right)u_{\seq{z}}=\left(\Vect_{i=1}^N C^T y_i\right) \oplus \left(\Vect_{i=N+1}^{N+U} 0 \right).
\end{dmath*}
which has time complexity $O_t(((N+U)u)^3+ Nup)$ and space complexity $O_s(((N+U)u)^2)$. Hence \cref{alg:close_form} is better that its kernelized counterpart when $r=2Du'$ is small compare to $(N+U)u$. Roughly speaking it is better to use \cref{alg:close_form} when the number of features, $r$, required is small compared to the number of training point. Notice that computing the data dependent norm (manifold regularization) is expensive. Indeed when $\lambda_M=0$, \cref{alg:close_form} has a linear complexity with respect to the number of supervised training points $N$ while the complexity becomes quatratic in the number of supervised and unsupervised training points $N+U$ when $\lambda_M>0$. Moreover suppose that $\lambda_M=0$, $\mathcal{U}=\mathbb{R}^p$ and $\mathcal{U}'=\mathbb{R}^{p}$ and the combination operator is $V=I_{p}$. Then the complexity of \cref{alg:close_form} boils down to
\begin{dmath*}
O_t(p^3(ND^2+D^3)),
\end{dmath*}
which is annoying. Indeed learning $p$ independent models with scalar Random Fourier Features would cost $O_t(p(ND^2+D^3))$, meaning that learning vector-valued function has increase the (expected) complexity from $p$ to $p^3$. However in some cases we can drastically reduce the complexity by viewing the feature-maps as linear operators rather than matrices.

\afterpage{%
\begin{landscape}
\begin{figure}[tb]
\centering
\resizebox{\textheight}{!}{%
%% Creator: Matplotlib, PGF backend
%%
%% To include the figure in your LaTeX document, write
%%   \input{<filename>.pgf}
%%
%% Make sure the required packages are loaded in your preamble
%%   \usepackage{pgf}
%%
%% Figures using additional raster images can only be included by \input if
%% they are in the same directory as the main LaTeX file. For loading figures
%% from other directories you can use the `import` package
%%   \usepackage{import}
%% and then include the figures with
%%   \import{<path to file>}{<filename>.pgf}
%%
%% Matplotlib used the following preamble
%%   \usepackage{fontspec}
%%   \setmainfont{Times New Roman}
%%   \setsansfont{Lucida Grande}
%%   \setmonofont{Andale Mono}
%%
\begingroup%
\makeatletter%
\begin{pgfpicture}%
\pgfpathrectangle{\pgfpointorigin}{\pgfqpoint{16.000000in}{6.000000in}}%
\pgfusepath{use as bounding box, clip}%
\begin{pgfscope}%
\pgfsetbuttcap%
\pgfsetmiterjoin%
\definecolor{currentfill}{rgb}{1.000000,1.000000,1.000000}%
\pgfsetfillcolor{currentfill}%
\pgfsetlinewidth{0.000000pt}%
\definecolor{currentstroke}{rgb}{1.000000,1.000000,1.000000}%
\pgfsetstrokecolor{currentstroke}%
\pgfsetdash{}{0pt}%
\pgfpathmoveto{\pgfqpoint{0.000000in}{0.000000in}}%
\pgfpathlineto{\pgfqpoint{16.000000in}{0.000000in}}%
\pgfpathlineto{\pgfqpoint{16.000000in}{6.000000in}}%
\pgfpathlineto{\pgfqpoint{0.000000in}{6.000000in}}%
\pgfpathclose%
\pgfusepath{fill}%
\end{pgfscope}%
\begin{pgfscope}%
\pgfsetbuttcap%
\pgfsetmiterjoin%
\definecolor{currentfill}{rgb}{1.000000,1.000000,1.000000}%
\pgfsetfillcolor{currentfill}%
\pgfsetlinewidth{0.000000pt}%
\definecolor{currentstroke}{rgb}{0.000000,0.000000,0.000000}%
\pgfsetstrokecolor{currentstroke}%
\pgfsetstrokeopacity{0.000000}%
\pgfsetdash{}{0pt}%
\pgfpathmoveto{\pgfqpoint{2.000000in}{4.326316in}}%
\pgfpathlineto{\pgfqpoint{6.376471in}{4.326316in}}%
\pgfpathlineto{\pgfqpoint{6.376471in}{5.280000in}}%
\pgfpathlineto{\pgfqpoint{2.000000in}{5.280000in}}%
\pgfpathclose%
\pgfusepath{fill}%
\end{pgfscope}%
\begin{pgfscope}%
\pgfpathrectangle{\pgfqpoint{2.000000in}{4.326316in}}{\pgfqpoint{4.376471in}{0.953684in}} %
\pgfusepath{clip}%
\pgfsetbuttcap%
\pgfsetroundjoin%
\definecolor{currentfill}{rgb}{1.000000,0.000000,0.000000}%
\pgfsetfillcolor{currentfill}%
\pgfsetlinewidth{2.007500pt}%
\definecolor{currentstroke}{rgb}{1.000000,0.000000,0.000000}%
\pgfsetstrokecolor{currentstroke}%
\pgfsetdash{}{0pt}%
\pgfpathmoveto{\pgfqpoint{4.755120in}{4.682823in}}%
\pgfpathlineto{\pgfqpoint{4.838454in}{4.682823in}}%
\pgfpathmoveto{\pgfqpoint{4.796787in}{4.641156in}}%
\pgfpathlineto{\pgfqpoint{4.796787in}{4.724490in}}%
\pgfusepath{stroke,fill}%
\end{pgfscope}%
\begin{pgfscope}%
\pgfpathrectangle{\pgfqpoint{2.000000in}{4.326316in}}{\pgfqpoint{4.376471in}{0.953684in}} %
\pgfusepath{clip}%
\pgfsetbuttcap%
\pgfsetroundjoin%
\definecolor{currentfill}{rgb}{1.000000,0.000000,0.000000}%
\pgfsetfillcolor{currentfill}%
\pgfsetlinewidth{2.007500pt}%
\definecolor{currentstroke}{rgb}{1.000000,0.000000,0.000000}%
\pgfsetstrokecolor{currentstroke}%
\pgfsetdash{}{0pt}%
\pgfpathmoveto{\pgfqpoint{5.337632in}{4.769016in}}%
\pgfpathlineto{\pgfqpoint{5.420965in}{4.769016in}}%
\pgfpathmoveto{\pgfqpoint{5.379298in}{4.727350in}}%
\pgfpathlineto{\pgfqpoint{5.379298in}{4.810683in}}%
\pgfusepath{stroke,fill}%
\end{pgfscope}%
\begin{pgfscope}%
\pgfpathrectangle{\pgfqpoint{2.000000in}{4.326316in}}{\pgfqpoint{4.376471in}{0.953684in}} %
\pgfusepath{clip}%
\pgfsetbuttcap%
\pgfsetroundjoin%
\definecolor{currentfill}{rgb}{1.000000,0.000000,0.000000}%
\pgfsetfillcolor{currentfill}%
\pgfsetlinewidth{2.007500pt}%
\definecolor{currentstroke}{rgb}{1.000000,0.000000,0.000000}%
\pgfsetstrokecolor{currentstroke}%
\pgfsetdash{}{0pt}%
\pgfpathmoveto{\pgfqpoint{4.944008in}{4.825923in}}%
\pgfpathlineto{\pgfqpoint{5.027342in}{4.825923in}}%
\pgfpathmoveto{\pgfqpoint{4.985675in}{4.784256in}}%
\pgfpathlineto{\pgfqpoint{4.985675in}{4.867589in}}%
\pgfusepath{stroke,fill}%
\end{pgfscope}%
\begin{pgfscope}%
\pgfpathrectangle{\pgfqpoint{2.000000in}{4.326316in}}{\pgfqpoint{4.376471in}{0.953684in}} %
\pgfusepath{clip}%
\pgfsetbuttcap%
\pgfsetroundjoin%
\definecolor{currentfill}{rgb}{1.000000,0.000000,0.000000}%
\pgfsetfillcolor{currentfill}%
\pgfsetlinewidth{2.007500pt}%
\definecolor{currentstroke}{rgb}{1.000000,0.000000,0.000000}%
\pgfsetstrokecolor{currentstroke}%
\pgfsetdash{}{0pt}%
\pgfpathmoveto{\pgfqpoint{4.741360in}{4.798976in}}%
\pgfpathlineto{\pgfqpoint{4.824693in}{4.798976in}}%
\pgfpathmoveto{\pgfqpoint{4.783026in}{4.757309in}}%
\pgfpathlineto{\pgfqpoint{4.783026in}{4.840643in}}%
\pgfusepath{stroke,fill}%
\end{pgfscope}%
\begin{pgfscope}%
\pgfpathrectangle{\pgfqpoint{2.000000in}{4.326316in}}{\pgfqpoint{4.376471in}{0.953684in}} %
\pgfusepath{clip}%
\pgfsetbuttcap%
\pgfsetroundjoin%
\definecolor{currentfill}{rgb}{1.000000,0.000000,0.000000}%
\pgfsetfillcolor{currentfill}%
\pgfsetlinewidth{2.007500pt}%
\definecolor{currentstroke}{rgb}{1.000000,0.000000,0.000000}%
\pgfsetstrokecolor{currentstroke}%
\pgfsetdash{}{0pt}%
\pgfpathmoveto{\pgfqpoint{4.316918in}{4.370264in}}%
\pgfpathlineto{\pgfqpoint{4.400251in}{4.370264in}}%
\pgfpathmoveto{\pgfqpoint{4.358584in}{4.328597in}}%
\pgfpathlineto{\pgfqpoint{4.358584in}{4.411930in}}%
\pgfusepath{stroke,fill}%
\end{pgfscope}%
\begin{pgfscope}%
\pgfpathrectangle{\pgfqpoint{2.000000in}{4.326316in}}{\pgfqpoint{4.376471in}{0.953684in}} %
\pgfusepath{clip}%
\pgfsetbuttcap%
\pgfsetroundjoin%
\definecolor{currentfill}{rgb}{1.000000,0.000000,0.000000}%
\pgfsetfillcolor{currentfill}%
\pgfsetlinewidth{2.007500pt}%
\definecolor{currentstroke}{rgb}{1.000000,0.000000,0.000000}%
\pgfsetstrokecolor{currentstroke}%
\pgfsetdash{}{0pt}%
\pgfpathmoveto{\pgfqpoint{5.095017in}{4.839095in}}%
\pgfpathlineto{\pgfqpoint{5.178350in}{4.839095in}}%
\pgfpathmoveto{\pgfqpoint{5.136683in}{4.797428in}}%
\pgfpathlineto{\pgfqpoint{5.136683in}{4.880762in}}%
\pgfusepath{stroke,fill}%
\end{pgfscope}%
\begin{pgfscope}%
\pgfpathrectangle{\pgfqpoint{2.000000in}{4.326316in}}{\pgfqpoint{4.376471in}{0.953684in}} %
\pgfusepath{clip}%
\pgfsetbuttcap%
\pgfsetroundjoin%
\definecolor{currentfill}{rgb}{1.000000,0.000000,0.000000}%
\pgfsetfillcolor{currentfill}%
\pgfsetlinewidth{2.007500pt}%
\definecolor{currentstroke}{rgb}{1.000000,0.000000,0.000000}%
\pgfsetstrokecolor{currentstroke}%
\pgfsetdash{}{0pt}%
\pgfpathmoveto{\pgfqpoint{4.365697in}{4.341372in}}%
\pgfpathlineto{\pgfqpoint{4.449031in}{4.341372in}}%
\pgfpathmoveto{\pgfqpoint{4.407364in}{4.299706in}}%
\pgfpathlineto{\pgfqpoint{4.407364in}{4.383039in}}%
\pgfusepath{stroke,fill}%
\end{pgfscope}%
\begin{pgfscope}%
\pgfpathrectangle{\pgfqpoint{2.000000in}{4.326316in}}{\pgfqpoint{4.376471in}{0.953684in}} %
\pgfusepath{clip}%
\pgfsetbuttcap%
\pgfsetroundjoin%
\definecolor{currentfill}{rgb}{1.000000,0.000000,0.000000}%
\pgfsetfillcolor{currentfill}%
\pgfsetlinewidth{2.007500pt}%
\definecolor{currentstroke}{rgb}{1.000000,0.000000,0.000000}%
\pgfsetstrokecolor{currentstroke}%
\pgfsetdash{}{0pt}%
\pgfpathmoveto{\pgfqpoint{5.955882in}{5.105426in}}%
\pgfpathlineto{\pgfqpoint{6.039215in}{5.105426in}}%
\pgfpathmoveto{\pgfqpoint{5.997549in}{5.063759in}}%
\pgfpathlineto{\pgfqpoint{5.997549in}{5.147092in}}%
\pgfusepath{stroke,fill}%
\end{pgfscope}%
\begin{pgfscope}%
\pgfpathrectangle{\pgfqpoint{2.000000in}{4.326316in}}{\pgfqpoint{4.376471in}{0.953684in}} %
\pgfusepath{clip}%
\pgfsetbuttcap%
\pgfsetroundjoin%
\definecolor{currentfill}{rgb}{1.000000,0.000000,0.000000}%
\pgfsetfillcolor{currentfill}%
\pgfsetlinewidth{2.007500pt}%
\definecolor{currentstroke}{rgb}{1.000000,0.000000,0.000000}%
\pgfsetstrokecolor{currentstroke}%
\pgfsetdash{}{0pt}%
\pgfpathmoveto{\pgfqpoint{6.207581in}{4.966690in}}%
\pgfpathlineto{\pgfqpoint{6.290914in}{4.966690in}}%
\pgfpathmoveto{\pgfqpoint{6.249248in}{4.925023in}}%
\pgfpathlineto{\pgfqpoint{6.249248in}{5.008357in}}%
\pgfusepath{stroke,fill}%
\end{pgfscope}%
\begin{pgfscope}%
\pgfpathrectangle{\pgfqpoint{2.000000in}{4.326316in}}{\pgfqpoint{4.376471in}{0.953684in}} %
\pgfusepath{clip}%
\pgfsetbuttcap%
\pgfsetroundjoin%
\definecolor{currentfill}{rgb}{1.000000,0.000000,0.000000}%
\pgfsetfillcolor{currentfill}%
\pgfsetlinewidth{2.007500pt}%
\definecolor{currentstroke}{rgb}{1.000000,0.000000,0.000000}%
\pgfsetstrokecolor{currentstroke}%
\pgfsetdash{}{0pt}%
\pgfpathmoveto{\pgfqpoint{4.176124in}{4.600015in}}%
\pgfpathlineto{\pgfqpoint{4.259457in}{4.600015in}}%
\pgfpathmoveto{\pgfqpoint{4.217791in}{4.558348in}}%
\pgfpathlineto{\pgfqpoint{4.217791in}{4.641682in}}%
\pgfusepath{stroke,fill}%
\end{pgfscope}%
\begin{pgfscope}%
\pgfpathrectangle{\pgfqpoint{2.000000in}{4.326316in}}{\pgfqpoint{4.376471in}{0.953684in}} %
\pgfusepath{clip}%
\pgfsetbuttcap%
\pgfsetroundjoin%
\definecolor{currentfill}{rgb}{1.000000,0.000000,0.000000}%
\pgfsetfillcolor{currentfill}%
\pgfsetlinewidth{2.007500pt}%
\definecolor{currentstroke}{rgb}{1.000000,0.000000,0.000000}%
\pgfsetstrokecolor{currentstroke}%
\pgfsetdash{}{0pt}%
\pgfpathmoveto{\pgfqpoint{5.605597in}{4.682287in}}%
\pgfpathlineto{\pgfqpoint{5.688930in}{4.682287in}}%
\pgfpathmoveto{\pgfqpoint{5.647263in}{4.640621in}}%
\pgfpathlineto{\pgfqpoint{5.647263in}{4.723954in}}%
\pgfusepath{stroke,fill}%
\end{pgfscope}%
\begin{pgfscope}%
\pgfpathrectangle{\pgfqpoint{2.000000in}{4.326316in}}{\pgfqpoint{4.376471in}{0.953684in}} %
\pgfusepath{clip}%
\pgfsetbuttcap%
\pgfsetroundjoin%
\definecolor{currentfill}{rgb}{1.000000,0.000000,0.000000}%
\pgfsetfillcolor{currentfill}%
\pgfsetlinewidth{2.007500pt}%
\definecolor{currentstroke}{rgb}{1.000000,0.000000,0.000000}%
\pgfsetstrokecolor{currentstroke}%
\pgfsetdash{}{0pt}%
\pgfpathmoveto{\pgfqpoint{4.685382in}{4.676522in}}%
\pgfpathlineto{\pgfqpoint{4.768715in}{4.676522in}}%
\pgfpathmoveto{\pgfqpoint{4.727049in}{4.634855in}}%
\pgfpathlineto{\pgfqpoint{4.727049in}{4.718189in}}%
\pgfusepath{stroke,fill}%
\end{pgfscope}%
\begin{pgfscope}%
\pgfpathrectangle{\pgfqpoint{2.000000in}{4.326316in}}{\pgfqpoint{4.376471in}{0.953684in}} %
\pgfusepath{clip}%
\pgfsetbuttcap%
\pgfsetroundjoin%
\definecolor{currentfill}{rgb}{1.000000,0.000000,0.000000}%
\pgfsetfillcolor{currentfill}%
\pgfsetlinewidth{2.007500pt}%
\definecolor{currentstroke}{rgb}{1.000000,0.000000,0.000000}%
\pgfsetstrokecolor{currentstroke}%
\pgfsetdash{}{0pt}%
\pgfpathmoveto{\pgfqpoint{4.822452in}{4.882830in}}%
\pgfpathlineto{\pgfqpoint{4.905785in}{4.882830in}}%
\pgfpathmoveto{\pgfqpoint{4.864118in}{4.841164in}}%
\pgfpathlineto{\pgfqpoint{4.864118in}{4.924497in}}%
\pgfusepath{stroke,fill}%
\end{pgfscope}%
\begin{pgfscope}%
\pgfpathrectangle{\pgfqpoint{2.000000in}{4.326316in}}{\pgfqpoint{4.376471in}{0.953684in}} %
\pgfusepath{clip}%
\pgfsetbuttcap%
\pgfsetroundjoin%
\definecolor{currentfill}{rgb}{1.000000,0.000000,0.000000}%
\pgfsetfillcolor{currentfill}%
\pgfsetlinewidth{2.007500pt}%
\definecolor{currentstroke}{rgb}{1.000000,0.000000,0.000000}%
\pgfsetstrokecolor{currentstroke}%
\pgfsetdash{}{0pt}%
\pgfpathmoveto{\pgfqpoint{6.074305in}{5.171612in}}%
\pgfpathlineto{\pgfqpoint{6.157638in}{5.171612in}}%
\pgfpathmoveto{\pgfqpoint{6.115971in}{5.129946in}}%
\pgfpathlineto{\pgfqpoint{6.115971in}{5.213279in}}%
\pgfusepath{stroke,fill}%
\end{pgfscope}%
\begin{pgfscope}%
\pgfpathrectangle{\pgfqpoint{2.000000in}{4.326316in}}{\pgfqpoint{4.376471in}{0.953684in}} %
\pgfusepath{clip}%
\pgfsetbuttcap%
\pgfsetroundjoin%
\definecolor{currentfill}{rgb}{1.000000,0.000000,0.000000}%
\pgfsetfillcolor{currentfill}%
\pgfsetlinewidth{2.007500pt}%
\definecolor{currentstroke}{rgb}{1.000000,0.000000,0.000000}%
\pgfsetstrokecolor{currentstroke}%
\pgfsetdash{}{0pt}%
\pgfpathmoveto{\pgfqpoint{3.082337in}{4.834359in}}%
\pgfpathlineto{\pgfqpoint{3.165671in}{4.834359in}}%
\pgfpathmoveto{\pgfqpoint{3.124004in}{4.792692in}}%
\pgfpathlineto{\pgfqpoint{3.124004in}{4.876025in}}%
\pgfusepath{stroke,fill}%
\end{pgfscope}%
\begin{pgfscope}%
\pgfpathrectangle{\pgfqpoint{2.000000in}{4.326316in}}{\pgfqpoint{4.376471in}{0.953684in}} %
\pgfusepath{clip}%
\pgfsetbuttcap%
\pgfsetroundjoin%
\definecolor{currentfill}{rgb}{1.000000,0.000000,0.000000}%
\pgfsetfillcolor{currentfill}%
\pgfsetlinewidth{2.007500pt}%
\definecolor{currentstroke}{rgb}{1.000000,0.000000,0.000000}%
\pgfsetstrokecolor{currentstroke}%
\pgfsetdash{}{0pt}%
\pgfpathmoveto{\pgfqpoint{3.138683in}{4.802207in}}%
\pgfpathlineto{\pgfqpoint{3.222016in}{4.802207in}}%
\pgfpathmoveto{\pgfqpoint{3.180349in}{4.760540in}}%
\pgfpathlineto{\pgfqpoint{3.180349in}{4.843874in}}%
\pgfusepath{stroke,fill}%
\end{pgfscope}%
\begin{pgfscope}%
\pgfpathrectangle{\pgfqpoint{2.000000in}{4.326316in}}{\pgfqpoint{4.376471in}{0.953684in}} %
\pgfusepath{clip}%
\pgfsetbuttcap%
\pgfsetroundjoin%
\definecolor{currentfill}{rgb}{1.000000,0.000000,0.000000}%
\pgfsetfillcolor{currentfill}%
\pgfsetlinewidth{2.007500pt}%
\definecolor{currentstroke}{rgb}{1.000000,0.000000,0.000000}%
\pgfsetstrokecolor{currentstroke}%
\pgfsetdash{}{0pt}%
\pgfpathmoveto{\pgfqpoint{2.904416in}{5.020595in}}%
\pgfpathlineto{\pgfqpoint{2.987749in}{5.020595in}}%
\pgfpathmoveto{\pgfqpoint{2.946082in}{4.978928in}}%
\pgfpathlineto{\pgfqpoint{2.946082in}{5.062262in}}%
\pgfusepath{stroke,fill}%
\end{pgfscope}%
\begin{pgfscope}%
\pgfpathrectangle{\pgfqpoint{2.000000in}{4.326316in}}{\pgfqpoint{4.376471in}{0.953684in}} %
\pgfusepath{clip}%
\pgfsetbuttcap%
\pgfsetroundjoin%
\definecolor{currentfill}{rgb}{1.000000,0.000000,0.000000}%
\pgfsetfillcolor{currentfill}%
\pgfsetlinewidth{2.007500pt}%
\definecolor{currentstroke}{rgb}{1.000000,0.000000,0.000000}%
\pgfsetstrokecolor{currentstroke}%
\pgfsetdash{}{0pt}%
\pgfpathmoveto{\pgfqpoint{5.748776in}{4.895466in}}%
\pgfpathlineto{\pgfqpoint{5.832110in}{4.895466in}}%
\pgfpathmoveto{\pgfqpoint{5.790443in}{4.853799in}}%
\pgfpathlineto{\pgfqpoint{5.790443in}{4.937133in}}%
\pgfusepath{stroke,fill}%
\end{pgfscope}%
\begin{pgfscope}%
\pgfpathrectangle{\pgfqpoint{2.000000in}{4.326316in}}{\pgfqpoint{4.376471in}{0.953684in}} %
\pgfusepath{clip}%
\pgfsetbuttcap%
\pgfsetroundjoin%
\definecolor{currentfill}{rgb}{1.000000,0.000000,0.000000}%
\pgfsetfillcolor{currentfill}%
\pgfsetlinewidth{2.007500pt}%
\definecolor{currentstroke}{rgb}{1.000000,0.000000,0.000000}%
\pgfsetstrokecolor{currentstroke}%
\pgfsetdash{}{0pt}%
\pgfpathmoveto{\pgfqpoint{5.558092in}{4.609106in}}%
\pgfpathlineto{\pgfqpoint{5.641425in}{4.609106in}}%
\pgfpathmoveto{\pgfqpoint{5.599758in}{4.567440in}}%
\pgfpathlineto{\pgfqpoint{5.599758in}{4.650773in}}%
\pgfusepath{stroke,fill}%
\end{pgfscope}%
\begin{pgfscope}%
\pgfpathrectangle{\pgfqpoint{2.000000in}{4.326316in}}{\pgfqpoint{4.376471in}{0.953684in}} %
\pgfusepath{clip}%
\pgfsetbuttcap%
\pgfsetroundjoin%
\definecolor{currentfill}{rgb}{1.000000,0.000000,0.000000}%
\pgfsetfillcolor{currentfill}%
\pgfsetlinewidth{2.007500pt}%
\definecolor{currentstroke}{rgb}{1.000000,0.000000,0.000000}%
\pgfsetstrokecolor{currentstroke}%
\pgfsetdash{}{0pt}%
\pgfpathmoveto{\pgfqpoint{5.879694in}{5.044422in}}%
\pgfpathlineto{\pgfqpoint{5.963027in}{5.044422in}}%
\pgfpathmoveto{\pgfqpoint{5.921360in}{5.002755in}}%
\pgfpathlineto{\pgfqpoint{5.921360in}{5.086089in}}%
\pgfusepath{stroke,fill}%
\end{pgfscope}%
\begin{pgfscope}%
\pgfpathrectangle{\pgfqpoint{2.000000in}{4.326316in}}{\pgfqpoint{4.376471in}{0.953684in}} %
\pgfusepath{clip}%
\pgfsetbuttcap%
\pgfsetroundjoin%
\definecolor{currentfill}{rgb}{1.000000,0.000000,0.000000}%
\pgfsetfillcolor{currentfill}%
\pgfsetlinewidth{2.007500pt}%
\definecolor{currentstroke}{rgb}{1.000000,0.000000,0.000000}%
\pgfsetstrokecolor{currentstroke}%
\pgfsetdash{}{0pt}%
\pgfpathmoveto{\pgfqpoint{6.259943in}{4.938060in}}%
\pgfpathlineto{\pgfqpoint{6.343276in}{4.938060in}}%
\pgfpathmoveto{\pgfqpoint{6.301610in}{4.896393in}}%
\pgfpathlineto{\pgfqpoint{6.301610in}{4.979727in}}%
\pgfusepath{stroke,fill}%
\end{pgfscope}%
\begin{pgfscope}%
\pgfpathrectangle{\pgfqpoint{2.000000in}{4.326316in}}{\pgfqpoint{4.376471in}{0.953684in}} %
\pgfusepath{clip}%
\pgfsetbuttcap%
\pgfsetroundjoin%
\definecolor{currentfill}{rgb}{1.000000,0.000000,0.000000}%
\pgfsetfillcolor{currentfill}%
\pgfsetlinewidth{2.007500pt}%
\definecolor{currentstroke}{rgb}{1.000000,0.000000,0.000000}%
\pgfsetstrokecolor{currentstroke}%
\pgfsetdash{}{0pt}%
\pgfpathmoveto{\pgfqpoint{5.631623in}{4.595988in}}%
\pgfpathlineto{\pgfqpoint{5.714956in}{4.595988in}}%
\pgfpathmoveto{\pgfqpoint{5.673289in}{4.554322in}}%
\pgfpathlineto{\pgfqpoint{5.673289in}{4.637655in}}%
\pgfusepath{stroke,fill}%
\end{pgfscope}%
\begin{pgfscope}%
\pgfpathrectangle{\pgfqpoint{2.000000in}{4.326316in}}{\pgfqpoint{4.376471in}{0.953684in}} %
\pgfusepath{clip}%
\pgfsetbuttcap%
\pgfsetroundjoin%
\definecolor{currentfill}{rgb}{1.000000,0.000000,0.000000}%
\pgfsetfillcolor{currentfill}%
\pgfsetlinewidth{2.007500pt}%
\definecolor{currentstroke}{rgb}{1.000000,0.000000,0.000000}%
\pgfsetstrokecolor{currentstroke}%
\pgfsetdash{}{0pt}%
\pgfpathmoveto{\pgfqpoint{4.449348in}{4.565562in}}%
\pgfpathlineto{\pgfqpoint{4.532681in}{4.565562in}}%
\pgfpathmoveto{\pgfqpoint{4.491015in}{4.523895in}}%
\pgfpathlineto{\pgfqpoint{4.491015in}{4.607229in}}%
\pgfusepath{stroke,fill}%
\end{pgfscope}%
\begin{pgfscope}%
\pgfpathrectangle{\pgfqpoint{2.000000in}{4.326316in}}{\pgfqpoint{4.376471in}{0.953684in}} %
\pgfusepath{clip}%
\pgfsetbuttcap%
\pgfsetroundjoin%
\definecolor{currentfill}{rgb}{1.000000,0.000000,0.000000}%
\pgfsetfillcolor{currentfill}%
\pgfsetlinewidth{2.007500pt}%
\definecolor{currentstroke}{rgb}{1.000000,0.000000,0.000000}%
\pgfsetstrokecolor{currentstroke}%
\pgfsetdash{}{0pt}%
\pgfpathmoveto{\pgfqpoint{5.566398in}{4.645335in}}%
\pgfpathlineto{\pgfqpoint{5.649731in}{4.645335in}}%
\pgfpathmoveto{\pgfqpoint{5.608065in}{4.603668in}}%
\pgfpathlineto{\pgfqpoint{5.608065in}{4.687002in}}%
\pgfusepath{stroke,fill}%
\end{pgfscope}%
\begin{pgfscope}%
\pgfpathrectangle{\pgfqpoint{2.000000in}{4.326316in}}{\pgfqpoint{4.376471in}{0.953684in}} %
\pgfusepath{clip}%
\pgfsetbuttcap%
\pgfsetroundjoin%
\definecolor{currentfill}{rgb}{1.000000,0.000000,0.000000}%
\pgfsetfillcolor{currentfill}%
\pgfsetlinewidth{2.007500pt}%
\definecolor{currentstroke}{rgb}{1.000000,0.000000,0.000000}%
\pgfsetstrokecolor{currentstroke}%
\pgfsetdash{}{0pt}%
\pgfpathmoveto{\pgfqpoint{3.247727in}{4.559691in}}%
\pgfpathlineto{\pgfqpoint{3.331060in}{4.559691in}}%
\pgfpathmoveto{\pgfqpoint{3.289394in}{4.518024in}}%
\pgfpathlineto{\pgfqpoint{3.289394in}{4.601358in}}%
\pgfusepath{stroke,fill}%
\end{pgfscope}%
\begin{pgfscope}%
\pgfpathrectangle{\pgfqpoint{2.000000in}{4.326316in}}{\pgfqpoint{4.376471in}{0.953684in}} %
\pgfusepath{clip}%
\pgfsetbuttcap%
\pgfsetroundjoin%
\definecolor{currentfill}{rgb}{1.000000,0.000000,0.000000}%
\pgfsetfillcolor{currentfill}%
\pgfsetlinewidth{2.007500pt}%
\definecolor{currentstroke}{rgb}{1.000000,0.000000,0.000000}%
\pgfsetstrokecolor{currentstroke}%
\pgfsetdash{}{0pt}%
\pgfpathmoveto{\pgfqpoint{5.074104in}{4.827892in}}%
\pgfpathlineto{\pgfqpoint{5.157437in}{4.827892in}}%
\pgfpathmoveto{\pgfqpoint{5.115771in}{4.786225in}}%
\pgfpathlineto{\pgfqpoint{5.115771in}{4.869559in}}%
\pgfusepath{stroke,fill}%
\end{pgfscope}%
\begin{pgfscope}%
\pgfpathrectangle{\pgfqpoint{2.000000in}{4.326316in}}{\pgfqpoint{4.376471in}{0.953684in}} %
\pgfusepath{clip}%
\pgfsetbuttcap%
\pgfsetroundjoin%
\definecolor{currentfill}{rgb}{1.000000,0.000000,0.000000}%
\pgfsetfillcolor{currentfill}%
\pgfsetlinewidth{2.007500pt}%
\definecolor{currentstroke}{rgb}{1.000000,0.000000,0.000000}%
\pgfsetstrokecolor{currentstroke}%
\pgfsetdash{}{0pt}%
\pgfpathmoveto{\pgfqpoint{3.335533in}{4.646756in}}%
\pgfpathlineto{\pgfqpoint{3.418866in}{4.646756in}}%
\pgfpathmoveto{\pgfqpoint{3.377199in}{4.605089in}}%
\pgfpathlineto{\pgfqpoint{3.377199in}{4.688423in}}%
\pgfusepath{stroke,fill}%
\end{pgfscope}%
\begin{pgfscope}%
\pgfpathrectangle{\pgfqpoint{2.000000in}{4.326316in}}{\pgfqpoint{4.376471in}{0.953684in}} %
\pgfusepath{clip}%
\pgfsetbuttcap%
\pgfsetroundjoin%
\definecolor{currentfill}{rgb}{1.000000,0.000000,0.000000}%
\pgfsetfillcolor{currentfill}%
\pgfsetlinewidth{2.007500pt}%
\definecolor{currentstroke}{rgb}{1.000000,0.000000,0.000000}%
\pgfsetstrokecolor{currentstroke}%
\pgfsetdash{}{0pt}%
\pgfpathmoveto{\pgfqpoint{6.141080in}{5.104076in}}%
\pgfpathlineto{\pgfqpoint{6.224413in}{5.104076in}}%
\pgfpathmoveto{\pgfqpoint{6.182747in}{5.062410in}}%
\pgfpathlineto{\pgfqpoint{6.182747in}{5.145743in}}%
\pgfusepath{stroke,fill}%
\end{pgfscope}%
\begin{pgfscope}%
\pgfpathrectangle{\pgfqpoint{2.000000in}{4.326316in}}{\pgfqpoint{4.376471in}{0.953684in}} %
\pgfusepath{clip}%
\pgfsetbuttcap%
\pgfsetroundjoin%
\definecolor{currentfill}{rgb}{1.000000,0.000000,0.000000}%
\pgfsetfillcolor{currentfill}%
\pgfsetlinewidth{2.007500pt}%
\definecolor{currentstroke}{rgb}{1.000000,0.000000,0.000000}%
\pgfsetstrokecolor{currentstroke}%
\pgfsetdash{}{0pt}%
\pgfpathmoveto{\pgfqpoint{4.660711in}{4.841217in}}%
\pgfpathlineto{\pgfqpoint{4.744044in}{4.841217in}}%
\pgfpathmoveto{\pgfqpoint{4.702377in}{4.799551in}}%
\pgfpathlineto{\pgfqpoint{4.702377in}{4.882884in}}%
\pgfusepath{stroke,fill}%
\end{pgfscope}%
\begin{pgfscope}%
\pgfpathrectangle{\pgfqpoint{2.000000in}{4.326316in}}{\pgfqpoint{4.376471in}{0.953684in}} %
\pgfusepath{clip}%
\pgfsetbuttcap%
\pgfsetroundjoin%
\definecolor{currentfill}{rgb}{1.000000,0.000000,0.000000}%
\pgfsetfillcolor{currentfill}%
\pgfsetlinewidth{2.007500pt}%
\definecolor{currentstroke}{rgb}{1.000000,0.000000,0.000000}%
\pgfsetstrokecolor{currentstroke}%
\pgfsetdash{}{0pt}%
\pgfpathmoveto{\pgfqpoint{4.285432in}{4.511486in}}%
\pgfpathlineto{\pgfqpoint{4.368765in}{4.511486in}}%
\pgfpathmoveto{\pgfqpoint{4.327099in}{4.469819in}}%
\pgfpathlineto{\pgfqpoint{4.327099in}{4.553152in}}%
\pgfusepath{stroke,fill}%
\end{pgfscope}%
\begin{pgfscope}%
\pgfpathrectangle{\pgfqpoint{2.000000in}{4.326316in}}{\pgfqpoint{4.376471in}{0.953684in}} %
\pgfusepath{clip}%
\pgfsetbuttcap%
\pgfsetroundjoin%
\definecolor{currentfill}{rgb}{1.000000,0.000000,0.000000}%
\pgfsetfillcolor{currentfill}%
\pgfsetlinewidth{2.007500pt}%
\definecolor{currentstroke}{rgb}{1.000000,0.000000,0.000000}%
\pgfsetstrokecolor{currentstroke}%
\pgfsetdash{}{0pt}%
\pgfpathmoveto{\pgfqpoint{3.759883in}{5.083932in}}%
\pgfpathlineto{\pgfqpoint{3.843217in}{5.083932in}}%
\pgfpathmoveto{\pgfqpoint{3.801550in}{5.042265in}}%
\pgfpathlineto{\pgfqpoint{3.801550in}{5.125598in}}%
\pgfusepath{stroke,fill}%
\end{pgfscope}%
\begin{pgfscope}%
\pgfpathrectangle{\pgfqpoint{2.000000in}{4.326316in}}{\pgfqpoint{4.376471in}{0.953684in}} %
\pgfusepath{clip}%
\pgfsetbuttcap%
\pgfsetroundjoin%
\definecolor{currentfill}{rgb}{1.000000,0.000000,0.000000}%
\pgfsetfillcolor{currentfill}%
\pgfsetlinewidth{2.007500pt}%
\definecolor{currentstroke}{rgb}{1.000000,0.000000,0.000000}%
\pgfsetstrokecolor{currentstroke}%
\pgfsetdash{}{0pt}%
\pgfpathmoveto{\pgfqpoint{5.544356in}{4.538491in}}%
\pgfpathlineto{\pgfqpoint{5.627690in}{4.538491in}}%
\pgfpathmoveto{\pgfqpoint{5.586023in}{4.496824in}}%
\pgfpathlineto{\pgfqpoint{5.586023in}{4.580158in}}%
\pgfusepath{stroke,fill}%
\end{pgfscope}%
\begin{pgfscope}%
\pgfpathrectangle{\pgfqpoint{2.000000in}{4.326316in}}{\pgfqpoint{4.376471in}{0.953684in}} %
\pgfusepath{clip}%
\pgfsetbuttcap%
\pgfsetroundjoin%
\definecolor{currentfill}{rgb}{1.000000,0.000000,0.000000}%
\pgfsetfillcolor{currentfill}%
\pgfsetlinewidth{2.007500pt}%
\definecolor{currentstroke}{rgb}{1.000000,0.000000,0.000000}%
\pgfsetstrokecolor{currentstroke}%
\pgfsetdash{}{0pt}%
\pgfpathmoveto{\pgfqpoint{4.430690in}{4.574221in}}%
\pgfpathlineto{\pgfqpoint{4.514024in}{4.574221in}}%
\pgfpathmoveto{\pgfqpoint{4.472357in}{4.532554in}}%
\pgfpathlineto{\pgfqpoint{4.472357in}{4.615887in}}%
\pgfusepath{stroke,fill}%
\end{pgfscope}%
\begin{pgfscope}%
\pgfpathrectangle{\pgfqpoint{2.000000in}{4.326316in}}{\pgfqpoint{4.376471in}{0.953684in}} %
\pgfusepath{clip}%
\pgfsetbuttcap%
\pgfsetroundjoin%
\definecolor{currentfill}{rgb}{1.000000,0.000000,0.000000}%
\pgfsetfillcolor{currentfill}%
\pgfsetlinewidth{2.007500pt}%
\definecolor{currentstroke}{rgb}{1.000000,0.000000,0.000000}%
\pgfsetstrokecolor{currentstroke}%
\pgfsetdash{}{0pt}%
\pgfpathmoveto{\pgfqpoint{4.823815in}{4.818000in}}%
\pgfpathlineto{\pgfqpoint{4.907148in}{4.818000in}}%
\pgfpathmoveto{\pgfqpoint{4.865482in}{4.776334in}}%
\pgfpathlineto{\pgfqpoint{4.865482in}{4.859667in}}%
\pgfusepath{stroke,fill}%
\end{pgfscope}%
\begin{pgfscope}%
\pgfpathrectangle{\pgfqpoint{2.000000in}{4.326316in}}{\pgfqpoint{4.376471in}{0.953684in}} %
\pgfusepath{clip}%
\pgfsetbuttcap%
\pgfsetroundjoin%
\definecolor{currentfill}{rgb}{1.000000,0.000000,0.000000}%
\pgfsetfillcolor{currentfill}%
\pgfsetlinewidth{2.007500pt}%
\definecolor{currentstroke}{rgb}{1.000000,0.000000,0.000000}%
\pgfsetstrokecolor{currentstroke}%
\pgfsetdash{}{0pt}%
\pgfpathmoveto{\pgfqpoint{2.899414in}{5.004865in}}%
\pgfpathlineto{\pgfqpoint{2.982747in}{5.004865in}}%
\pgfpathmoveto{\pgfqpoint{2.941081in}{4.963198in}}%
\pgfpathlineto{\pgfqpoint{2.941081in}{5.046532in}}%
\pgfusepath{stroke,fill}%
\end{pgfscope}%
\begin{pgfscope}%
\pgfpathrectangle{\pgfqpoint{2.000000in}{4.326316in}}{\pgfqpoint{4.376471in}{0.953684in}} %
\pgfusepath{clip}%
\pgfsetbuttcap%
\pgfsetroundjoin%
\definecolor{currentfill}{rgb}{1.000000,0.000000,0.000000}%
\pgfsetfillcolor{currentfill}%
\pgfsetlinewidth{2.007500pt}%
\definecolor{currentstroke}{rgb}{1.000000,0.000000,0.000000}%
\pgfsetstrokecolor{currentstroke}%
\pgfsetdash{}{0pt}%
\pgfpathmoveto{\pgfqpoint{4.996078in}{4.789418in}}%
\pgfpathlineto{\pgfqpoint{5.079412in}{4.789418in}}%
\pgfpathmoveto{\pgfqpoint{5.037745in}{4.747752in}}%
\pgfpathlineto{\pgfqpoint{5.037745in}{4.831085in}}%
\pgfusepath{stroke,fill}%
\end{pgfscope}%
\begin{pgfscope}%
\pgfpathrectangle{\pgfqpoint{2.000000in}{4.326316in}}{\pgfqpoint{4.376471in}{0.953684in}} %
\pgfusepath{clip}%
\pgfsetbuttcap%
\pgfsetroundjoin%
\definecolor{currentfill}{rgb}{1.000000,0.000000,0.000000}%
\pgfsetfillcolor{currentfill}%
\pgfsetlinewidth{2.007500pt}%
\definecolor{currentstroke}{rgb}{1.000000,0.000000,0.000000}%
\pgfsetstrokecolor{currentstroke}%
\pgfsetdash{}{0pt}%
\pgfpathmoveto{\pgfqpoint{4.976683in}{4.828564in}}%
\pgfpathlineto{\pgfqpoint{5.060016in}{4.828564in}}%
\pgfpathmoveto{\pgfqpoint{5.018349in}{4.786898in}}%
\pgfpathlineto{\pgfqpoint{5.018349in}{4.870231in}}%
\pgfusepath{stroke,fill}%
\end{pgfscope}%
\begin{pgfscope}%
\pgfpathrectangle{\pgfqpoint{2.000000in}{4.326316in}}{\pgfqpoint{4.376471in}{0.953684in}} %
\pgfusepath{clip}%
\pgfsetbuttcap%
\pgfsetroundjoin%
\definecolor{currentfill}{rgb}{1.000000,0.000000,0.000000}%
\pgfsetfillcolor{currentfill}%
\pgfsetlinewidth{2.007500pt}%
\definecolor{currentstroke}{rgb}{1.000000,0.000000,0.000000}%
\pgfsetstrokecolor{currentstroke}%
\pgfsetdash{}{0pt}%
\pgfpathmoveto{\pgfqpoint{4.993622in}{4.854191in}}%
\pgfpathlineto{\pgfqpoint{5.076956in}{4.854191in}}%
\pgfpathmoveto{\pgfqpoint{5.035289in}{4.812524in}}%
\pgfpathlineto{\pgfqpoint{5.035289in}{4.895857in}}%
\pgfusepath{stroke,fill}%
\end{pgfscope}%
\begin{pgfscope}%
\pgfpathrectangle{\pgfqpoint{2.000000in}{4.326316in}}{\pgfqpoint{4.376471in}{0.953684in}} %
\pgfusepath{clip}%
\pgfsetbuttcap%
\pgfsetroundjoin%
\definecolor{currentfill}{rgb}{1.000000,0.000000,0.000000}%
\pgfsetfillcolor{currentfill}%
\pgfsetlinewidth{2.007500pt}%
\definecolor{currentstroke}{rgb}{1.000000,0.000000,0.000000}%
\pgfsetstrokecolor{currentstroke}%
\pgfsetdash{}{0pt}%
\pgfpathmoveto{\pgfqpoint{6.137856in}{4.985005in}}%
\pgfpathlineto{\pgfqpoint{6.221189in}{4.985005in}}%
\pgfpathmoveto{\pgfqpoint{6.179523in}{4.943338in}}%
\pgfpathlineto{\pgfqpoint{6.179523in}{5.026671in}}%
\pgfusepath{stroke,fill}%
\end{pgfscope}%
\begin{pgfscope}%
\pgfpathrectangle{\pgfqpoint{2.000000in}{4.326316in}}{\pgfqpoint{4.376471in}{0.953684in}} %
\pgfusepath{clip}%
\pgfsetbuttcap%
\pgfsetroundjoin%
\definecolor{currentfill}{rgb}{1.000000,0.000000,0.000000}%
\pgfsetfillcolor{currentfill}%
\pgfsetlinewidth{2.007500pt}%
\definecolor{currentstroke}{rgb}{1.000000,0.000000,0.000000}%
\pgfsetstrokecolor{currentstroke}%
\pgfsetdash{}{0pt}%
\pgfpathmoveto{\pgfqpoint{5.220801in}{4.731635in}}%
\pgfpathlineto{\pgfqpoint{5.304134in}{4.731635in}}%
\pgfpathmoveto{\pgfqpoint{5.262467in}{4.689968in}}%
\pgfpathlineto{\pgfqpoint{5.262467in}{4.773301in}}%
\pgfusepath{stroke,fill}%
\end{pgfscope}%
\begin{pgfscope}%
\pgfpathrectangle{\pgfqpoint{2.000000in}{4.326316in}}{\pgfqpoint{4.376471in}{0.953684in}} %
\pgfusepath{clip}%
\pgfsetbuttcap%
\pgfsetroundjoin%
\definecolor{currentfill}{rgb}{1.000000,0.000000,0.000000}%
\pgfsetfillcolor{currentfill}%
\pgfsetlinewidth{2.007500pt}%
\definecolor{currentstroke}{rgb}{1.000000,0.000000,0.000000}%
\pgfsetstrokecolor{currentstroke}%
\pgfsetdash{}{0pt}%
\pgfpathmoveto{\pgfqpoint{4.092328in}{4.696869in}}%
\pgfpathlineto{\pgfqpoint{4.175661in}{4.696869in}}%
\pgfpathmoveto{\pgfqpoint{4.133995in}{4.655202in}}%
\pgfpathlineto{\pgfqpoint{4.133995in}{4.738536in}}%
\pgfusepath{stroke,fill}%
\end{pgfscope}%
\begin{pgfscope}%
\pgfpathrectangle{\pgfqpoint{2.000000in}{4.326316in}}{\pgfqpoint{4.376471in}{0.953684in}} %
\pgfusepath{clip}%
\pgfsetbuttcap%
\pgfsetroundjoin%
\definecolor{currentfill}{rgb}{1.000000,0.000000,0.000000}%
\pgfsetfillcolor{currentfill}%
\pgfsetlinewidth{2.007500pt}%
\definecolor{currentstroke}{rgb}{1.000000,0.000000,0.000000}%
\pgfsetstrokecolor{currentstroke}%
\pgfsetdash{}{0pt}%
\pgfpathmoveto{\pgfqpoint{4.363753in}{4.348509in}}%
\pgfpathlineto{\pgfqpoint{4.447087in}{4.348509in}}%
\pgfpathmoveto{\pgfqpoint{4.405420in}{4.306843in}}%
\pgfpathlineto{\pgfqpoint{4.405420in}{4.390176in}}%
\pgfusepath{stroke,fill}%
\end{pgfscope}%
\begin{pgfscope}%
\pgfpathrectangle{\pgfqpoint{2.000000in}{4.326316in}}{\pgfqpoint{4.376471in}{0.953684in}} %
\pgfusepath{clip}%
\pgfsetbuttcap%
\pgfsetroundjoin%
\definecolor{currentfill}{rgb}{1.000000,0.000000,0.000000}%
\pgfsetfillcolor{currentfill}%
\pgfsetlinewidth{2.007500pt}%
\definecolor{currentstroke}{rgb}{1.000000,0.000000,0.000000}%
\pgfsetstrokecolor{currentstroke}%
\pgfsetdash{}{0pt}%
\pgfpathmoveto{\pgfqpoint{5.276157in}{4.749830in}}%
\pgfpathlineto{\pgfqpoint{5.359491in}{4.749830in}}%
\pgfpathmoveto{\pgfqpoint{5.317824in}{4.708163in}}%
\pgfpathlineto{\pgfqpoint{5.317824in}{4.791496in}}%
\pgfusepath{stroke,fill}%
\end{pgfscope}%
\begin{pgfscope}%
\pgfpathrectangle{\pgfqpoint{2.000000in}{4.326316in}}{\pgfqpoint{4.376471in}{0.953684in}} %
\pgfusepath{clip}%
\pgfsetbuttcap%
\pgfsetroundjoin%
\definecolor{currentfill}{rgb}{1.000000,0.000000,0.000000}%
\pgfsetfillcolor{currentfill}%
\pgfsetlinewidth{2.007500pt}%
\definecolor{currentstroke}{rgb}{1.000000,0.000000,0.000000}%
\pgfsetstrokecolor{currentstroke}%
\pgfsetdash{}{0pt}%
\pgfpathmoveto{\pgfqpoint{3.044487in}{5.062756in}}%
\pgfpathlineto{\pgfqpoint{3.127821in}{5.062756in}}%
\pgfpathmoveto{\pgfqpoint{3.086154in}{5.021090in}}%
\pgfpathlineto{\pgfqpoint{3.086154in}{5.104423in}}%
\pgfusepath{stroke,fill}%
\end{pgfscope}%
\begin{pgfscope}%
\pgfpathrectangle{\pgfqpoint{2.000000in}{4.326316in}}{\pgfqpoint{4.376471in}{0.953684in}} %
\pgfusepath{clip}%
\pgfsetbuttcap%
\pgfsetroundjoin%
\definecolor{currentfill}{rgb}{1.000000,0.000000,0.000000}%
\pgfsetfillcolor{currentfill}%
\pgfsetlinewidth{2.007500pt}%
\definecolor{currentstroke}{rgb}{1.000000,0.000000,0.000000}%
\pgfsetstrokecolor{currentstroke}%
\pgfsetdash{}{0pt}%
\pgfpathmoveto{\pgfqpoint{5.168095in}{4.807002in}}%
\pgfpathlineto{\pgfqpoint{5.251429in}{4.807002in}}%
\pgfpathmoveto{\pgfqpoint{5.209762in}{4.765335in}}%
\pgfpathlineto{\pgfqpoint{5.209762in}{4.848669in}}%
\pgfusepath{stroke,fill}%
\end{pgfscope}%
\begin{pgfscope}%
\pgfpathrectangle{\pgfqpoint{2.000000in}{4.326316in}}{\pgfqpoint{4.376471in}{0.953684in}} %
\pgfusepath{clip}%
\pgfsetbuttcap%
\pgfsetroundjoin%
\definecolor{currentfill}{rgb}{1.000000,0.000000,0.000000}%
\pgfsetfillcolor{currentfill}%
\pgfsetlinewidth{2.007500pt}%
\definecolor{currentstroke}{rgb}{1.000000,0.000000,0.000000}%
\pgfsetstrokecolor{currentstroke}%
\pgfsetdash{}{0pt}%
\pgfpathmoveto{\pgfqpoint{5.181649in}{4.659136in}}%
\pgfpathlineto{\pgfqpoint{5.264982in}{4.659136in}}%
\pgfpathmoveto{\pgfqpoint{5.223316in}{4.617469in}}%
\pgfpathlineto{\pgfqpoint{5.223316in}{4.700803in}}%
\pgfusepath{stroke,fill}%
\end{pgfscope}%
\begin{pgfscope}%
\pgfpathrectangle{\pgfqpoint{2.000000in}{4.326316in}}{\pgfqpoint{4.376471in}{0.953684in}} %
\pgfusepath{clip}%
\pgfsetbuttcap%
\pgfsetroundjoin%
\definecolor{currentfill}{rgb}{1.000000,0.000000,0.000000}%
\pgfsetfillcolor{currentfill}%
\pgfsetlinewidth{2.007500pt}%
\definecolor{currentstroke}{rgb}{1.000000,0.000000,0.000000}%
\pgfsetstrokecolor{currentstroke}%
\pgfsetdash{}{0pt}%
\pgfpathmoveto{\pgfqpoint{3.570214in}{4.804551in}}%
\pgfpathlineto{\pgfqpoint{3.653547in}{4.804551in}}%
\pgfpathmoveto{\pgfqpoint{3.611881in}{4.762885in}}%
\pgfpathlineto{\pgfqpoint{3.611881in}{4.846218in}}%
\pgfusepath{stroke,fill}%
\end{pgfscope}%
\begin{pgfscope}%
\pgfpathrectangle{\pgfqpoint{2.000000in}{4.326316in}}{\pgfqpoint{4.376471in}{0.953684in}} %
\pgfusepath{clip}%
\pgfsetbuttcap%
\pgfsetroundjoin%
\definecolor{currentfill}{rgb}{1.000000,0.000000,0.000000}%
\pgfsetfillcolor{currentfill}%
\pgfsetlinewidth{2.007500pt}%
\definecolor{currentstroke}{rgb}{1.000000,0.000000,0.000000}%
\pgfsetstrokecolor{currentstroke}%
\pgfsetdash{}{0pt}%
\pgfpathmoveto{\pgfqpoint{3.285021in}{4.827174in}}%
\pgfpathlineto{\pgfqpoint{3.368355in}{4.827174in}}%
\pgfpathmoveto{\pgfqpoint{3.326688in}{4.785507in}}%
\pgfpathlineto{\pgfqpoint{3.326688in}{4.868840in}}%
\pgfusepath{stroke,fill}%
\end{pgfscope}%
\begin{pgfscope}%
\pgfpathrectangle{\pgfqpoint{2.000000in}{4.326316in}}{\pgfqpoint{4.376471in}{0.953684in}} %
\pgfusepath{clip}%
\pgfsetbuttcap%
\pgfsetroundjoin%
\definecolor{currentfill}{rgb}{1.000000,0.000000,0.000000}%
\pgfsetfillcolor{currentfill}%
\pgfsetlinewidth{2.007500pt}%
\definecolor{currentstroke}{rgb}{1.000000,0.000000,0.000000}%
\pgfsetstrokecolor{currentstroke}%
\pgfsetdash{}{0pt}%
\pgfpathmoveto{\pgfqpoint{3.937998in}{4.801866in}}%
\pgfpathlineto{\pgfqpoint{4.021331in}{4.801866in}}%
\pgfpathmoveto{\pgfqpoint{3.979664in}{4.760199in}}%
\pgfpathlineto{\pgfqpoint{3.979664in}{4.843532in}}%
\pgfusepath{stroke,fill}%
\end{pgfscope}%
\begin{pgfscope}%
\pgfpathrectangle{\pgfqpoint{2.000000in}{4.326316in}}{\pgfqpoint{4.376471in}{0.953684in}} %
\pgfusepath{clip}%
\pgfsetbuttcap%
\pgfsetroundjoin%
\definecolor{currentfill}{rgb}{1.000000,0.000000,0.000000}%
\pgfsetfillcolor{currentfill}%
\pgfsetlinewidth{2.007500pt}%
\definecolor{currentstroke}{rgb}{1.000000,0.000000,0.000000}%
\pgfsetstrokecolor{currentstroke}%
\pgfsetdash{}{0pt}%
\pgfpathmoveto{\pgfqpoint{4.107043in}{4.754657in}}%
\pgfpathlineto{\pgfqpoint{4.190376in}{4.754657in}}%
\pgfpathmoveto{\pgfqpoint{4.148710in}{4.712991in}}%
\pgfpathlineto{\pgfqpoint{4.148710in}{4.796324in}}%
\pgfusepath{stroke,fill}%
\end{pgfscope}%
\begin{pgfscope}%
\pgfpathrectangle{\pgfqpoint{2.000000in}{4.326316in}}{\pgfqpoint{4.376471in}{0.953684in}} %
\pgfusepath{clip}%
\pgfsetbuttcap%
\pgfsetroundjoin%
\definecolor{currentfill}{rgb}{0.000000,0.000000,0.000000}%
\pgfsetfillcolor{currentfill}%
\pgfsetlinewidth{1.003750pt}%
\definecolor{currentstroke}{rgb}{0.000000,0.000000,0.000000}%
\pgfsetstrokecolor{currentstroke}%
\pgfsetdash{}{0pt}%
\pgfsys@defobject{currentmarker}{\pgfqpoint{-0.020833in}{-0.020833in}}{\pgfqpoint{0.020833in}{0.020833in}}{%
\pgfpathmoveto{\pgfqpoint{0.000000in}{-0.020833in}}%
\pgfpathcurveto{\pgfqpoint{0.005525in}{-0.020833in}}{\pgfqpoint{0.010825in}{-0.018638in}}{\pgfqpoint{0.014731in}{-0.014731in}}%
\pgfpathcurveto{\pgfqpoint{0.018638in}{-0.010825in}}{\pgfqpoint{0.020833in}{-0.005525in}}{\pgfqpoint{0.020833in}{0.000000in}}%
\pgfpathcurveto{\pgfqpoint{0.020833in}{0.005525in}}{\pgfqpoint{0.018638in}{0.010825in}}{\pgfqpoint{0.014731in}{0.014731in}}%
\pgfpathcurveto{\pgfqpoint{0.010825in}{0.018638in}}{\pgfqpoint{0.005525in}{0.020833in}}{\pgfqpoint{0.000000in}{0.020833in}}%
\pgfpathcurveto{\pgfqpoint{-0.005525in}{0.020833in}}{\pgfqpoint{-0.010825in}{0.018638in}}{\pgfqpoint{-0.014731in}{0.014731in}}%
\pgfpathcurveto{\pgfqpoint{-0.018638in}{0.010825in}}{\pgfqpoint{-0.020833in}{0.005525in}}{\pgfqpoint{-0.020833in}{0.000000in}}%
\pgfpathcurveto{\pgfqpoint{-0.020833in}{-0.005525in}}{\pgfqpoint{-0.018638in}{-0.010825in}}{\pgfqpoint{-0.014731in}{-0.014731in}}%
\pgfpathcurveto{\pgfqpoint{-0.010825in}{-0.018638in}}{\pgfqpoint{-0.005525in}{-0.020833in}}{\pgfqpoint{0.000000in}{-0.020833in}}%
\pgfpathclose%
\pgfusepath{stroke,fill}%
}%
\begin{pgfscope}%
\pgfsys@transformshift{2.875294in}{5.210095in}%
\pgfsys@useobject{currentmarker}{}%
\end{pgfscope}%
\begin{pgfscope}%
\pgfsys@transformshift{2.892888in}{5.261206in}%
\pgfsys@useobject{currentmarker}{}%
\end{pgfscope}%
\begin{pgfscope}%
\pgfsys@transformshift{2.910482in}{5.173532in}%
\pgfsys@useobject{currentmarker}{}%
\end{pgfscope}%
\begin{pgfscope}%
\pgfsys@transformshift{2.928076in}{5.181614in}%
\pgfsys@useobject{currentmarker}{}%
\end{pgfscope}%
\begin{pgfscope}%
\pgfsys@transformshift{2.945670in}{5.085295in}%
\pgfsys@useobject{currentmarker}{}%
\end{pgfscope}%
\begin{pgfscope}%
\pgfsys@transformshift{2.963263in}{5.234228in}%
\pgfsys@useobject{currentmarker}{}%
\end{pgfscope}%
\begin{pgfscope}%
\pgfsys@transformshift{2.980857in}{5.041923in}%
\pgfsys@useobject{currentmarker}{}%
\end{pgfscope}%
\begin{pgfscope}%
\pgfsys@transformshift{2.998451in}{5.041885in}%
\pgfsys@useobject{currentmarker}{}%
\end{pgfscope}%
\begin{pgfscope}%
\pgfsys@transformshift{3.016045in}{5.161999in}%
\pgfsys@useobject{currentmarker}{}%
\end{pgfscope}%
\begin{pgfscope}%
\pgfsys@transformshift{3.033639in}{4.814541in}%
\pgfsys@useobject{currentmarker}{}%
\end{pgfscope}%
\begin{pgfscope}%
\pgfsys@transformshift{3.051233in}{4.796454in}%
\pgfsys@useobject{currentmarker}{}%
\end{pgfscope}%
\begin{pgfscope}%
\pgfsys@transformshift{3.068826in}{4.994120in}%
\pgfsys@useobject{currentmarker}{}%
\end{pgfscope}%
\begin{pgfscope}%
\pgfsys@transformshift{3.086420in}{4.757524in}%
\pgfsys@useobject{currentmarker}{}%
\end{pgfscope}%
\begin{pgfscope}%
\pgfsys@transformshift{3.104014in}{5.044657in}%
\pgfsys@useobject{currentmarker}{}%
\end{pgfscope}%
\begin{pgfscope}%
\pgfsys@transformshift{3.121608in}{4.789412in}%
\pgfsys@useobject{currentmarker}{}%
\end{pgfscope}%
\begin{pgfscope}%
\pgfsys@transformshift{3.139202in}{4.736740in}%
\pgfsys@useobject{currentmarker}{}%
\end{pgfscope}%
\begin{pgfscope}%
\pgfsys@transformshift{3.156796in}{4.984204in}%
\pgfsys@useobject{currentmarker}{}%
\end{pgfscope}%
\begin{pgfscope}%
\pgfsys@transformshift{3.174390in}{4.924236in}%
\pgfsys@useobject{currentmarker}{}%
\end{pgfscope}%
\begin{pgfscope}%
\pgfsys@transformshift{3.191983in}{4.948566in}%
\pgfsys@useobject{currentmarker}{}%
\end{pgfscope}%
\begin{pgfscope}%
\pgfsys@transformshift{3.209577in}{4.840900in}%
\pgfsys@useobject{currentmarker}{}%
\end{pgfscope}%
\begin{pgfscope}%
\pgfsys@transformshift{3.227171in}{4.655218in}%
\pgfsys@useobject{currentmarker}{}%
\end{pgfscope}%
\begin{pgfscope}%
\pgfsys@transformshift{3.244765in}{4.922473in}%
\pgfsys@useobject{currentmarker}{}%
\end{pgfscope}%
\begin{pgfscope}%
\pgfsys@transformshift{3.262359in}{4.700133in}%
\pgfsys@useobject{currentmarker}{}%
\end{pgfscope}%
\begin{pgfscope}%
\pgfsys@transformshift{3.279953in}{4.802590in}%
\pgfsys@useobject{currentmarker}{}%
\end{pgfscope}%
\begin{pgfscope}%
\pgfsys@transformshift{3.297547in}{4.815147in}%
\pgfsys@useobject{currentmarker}{}%
\end{pgfscope}%
\begin{pgfscope}%
\pgfsys@transformshift{3.315140in}{4.705834in}%
\pgfsys@useobject{currentmarker}{}%
\end{pgfscope}%
\begin{pgfscope}%
\pgfsys@transformshift{3.332734in}{4.784377in}%
\pgfsys@useobject{currentmarker}{}%
\end{pgfscope}%
\begin{pgfscope}%
\pgfsys@transformshift{3.350328in}{4.818998in}%
\pgfsys@useobject{currentmarker}{}%
\end{pgfscope}%
\begin{pgfscope}%
\pgfsys@transformshift{3.367922in}{4.770561in}%
\pgfsys@useobject{currentmarker}{}%
\end{pgfscope}%
\begin{pgfscope}%
\pgfsys@transformshift{3.385516in}{4.631385in}%
\pgfsys@useobject{currentmarker}{}%
\end{pgfscope}%
\begin{pgfscope}%
\pgfsys@transformshift{3.403110in}{4.779188in}%
\pgfsys@useobject{currentmarker}{}%
\end{pgfscope}%
\begin{pgfscope}%
\pgfsys@transformshift{3.420704in}{4.891659in}%
\pgfsys@useobject{currentmarker}{}%
\end{pgfscope}%
\begin{pgfscope}%
\pgfsys@transformshift{3.438297in}{4.702440in}%
\pgfsys@useobject{currentmarker}{}%
\end{pgfscope}%
\begin{pgfscope}%
\pgfsys@transformshift{3.455891in}{4.769165in}%
\pgfsys@useobject{currentmarker}{}%
\end{pgfscope}%
\begin{pgfscope}%
\pgfsys@transformshift{3.473485in}{4.754253in}%
\pgfsys@useobject{currentmarker}{}%
\end{pgfscope}%
\begin{pgfscope}%
\pgfsys@transformshift{3.491079in}{4.995316in}%
\pgfsys@useobject{currentmarker}{}%
\end{pgfscope}%
\begin{pgfscope}%
\pgfsys@transformshift{3.508673in}{4.892995in}%
\pgfsys@useobject{currentmarker}{}%
\end{pgfscope}%
\begin{pgfscope}%
\pgfsys@transformshift{3.526267in}{4.881565in}%
\pgfsys@useobject{currentmarker}{}%
\end{pgfscope}%
\begin{pgfscope}%
\pgfsys@transformshift{3.543860in}{4.779557in}%
\pgfsys@useobject{currentmarker}{}%
\end{pgfscope}%
\begin{pgfscope}%
\pgfsys@transformshift{3.561454in}{4.924444in}%
\pgfsys@useobject{currentmarker}{}%
\end{pgfscope}%
\begin{pgfscope}%
\pgfsys@transformshift{3.579048in}{4.818392in}%
\pgfsys@useobject{currentmarker}{}%
\end{pgfscope}%
\begin{pgfscope}%
\pgfsys@transformshift{3.596642in}{4.902533in}%
\pgfsys@useobject{currentmarker}{}%
\end{pgfscope}%
\begin{pgfscope}%
\pgfsys@transformshift{3.614236in}{4.849486in}%
\pgfsys@useobject{currentmarker}{}%
\end{pgfscope}%
\begin{pgfscope}%
\pgfsys@transformshift{3.631830in}{4.992244in}%
\pgfsys@useobject{currentmarker}{}%
\end{pgfscope}%
\begin{pgfscope}%
\pgfsys@transformshift{3.649424in}{4.993659in}%
\pgfsys@useobject{currentmarker}{}%
\end{pgfscope}%
\begin{pgfscope}%
\pgfsys@transformshift{3.667017in}{4.925800in}%
\pgfsys@useobject{currentmarker}{}%
\end{pgfscope}%
\begin{pgfscope}%
\pgfsys@transformshift{3.684611in}{4.994580in}%
\pgfsys@useobject{currentmarker}{}%
\end{pgfscope}%
\begin{pgfscope}%
\pgfsys@transformshift{3.702205in}{4.853902in}%
\pgfsys@useobject{currentmarker}{}%
\end{pgfscope}%
\begin{pgfscope}%
\pgfsys@transformshift{3.719799in}{4.819979in}%
\pgfsys@useobject{currentmarker}{}%
\end{pgfscope}%
\begin{pgfscope}%
\pgfsys@transformshift{3.737393in}{5.015695in}%
\pgfsys@useobject{currentmarker}{}%
\end{pgfscope}%
\begin{pgfscope}%
\pgfsys@transformshift{3.754987in}{4.990730in}%
\pgfsys@useobject{currentmarker}{}%
\end{pgfscope}%
\begin{pgfscope}%
\pgfsys@transformshift{3.772581in}{5.037529in}%
\pgfsys@useobject{currentmarker}{}%
\end{pgfscope}%
\begin{pgfscope}%
\pgfsys@transformshift{3.790174in}{5.209562in}%
\pgfsys@useobject{currentmarker}{}%
\end{pgfscope}%
\begin{pgfscope}%
\pgfsys@transformshift{3.807768in}{5.063094in}%
\pgfsys@useobject{currentmarker}{}%
\end{pgfscope}%
\begin{pgfscope}%
\pgfsys@transformshift{3.825362in}{4.873113in}%
\pgfsys@useobject{currentmarker}{}%
\end{pgfscope}%
\begin{pgfscope}%
\pgfsys@transformshift{3.842956in}{5.067384in}%
\pgfsys@useobject{currentmarker}{}%
\end{pgfscope}%
\begin{pgfscope}%
\pgfsys@transformshift{3.860550in}{4.816447in}%
\pgfsys@useobject{currentmarker}{}%
\end{pgfscope}%
\begin{pgfscope}%
\pgfsys@transformshift{3.878144in}{4.890263in}%
\pgfsys@useobject{currentmarker}{}%
\end{pgfscope}%
\begin{pgfscope}%
\pgfsys@transformshift{3.895738in}{4.916551in}%
\pgfsys@useobject{currentmarker}{}%
\end{pgfscope}%
\begin{pgfscope}%
\pgfsys@transformshift{3.913331in}{5.079111in}%
\pgfsys@useobject{currentmarker}{}%
\end{pgfscope}%
\begin{pgfscope}%
\pgfsys@transformshift{3.930925in}{4.818967in}%
\pgfsys@useobject{currentmarker}{}%
\end{pgfscope}%
\begin{pgfscope}%
\pgfsys@transformshift{3.948519in}{4.793504in}%
\pgfsys@useobject{currentmarker}{}%
\end{pgfscope}%
\begin{pgfscope}%
\pgfsys@transformshift{3.966113in}{4.847222in}%
\pgfsys@useobject{currentmarker}{}%
\end{pgfscope}%
\begin{pgfscope}%
\pgfsys@transformshift{3.983707in}{4.771393in}%
\pgfsys@useobject{currentmarker}{}%
\end{pgfscope}%
\begin{pgfscope}%
\pgfsys@transformshift{4.001301in}{4.928600in}%
\pgfsys@useobject{currentmarker}{}%
\end{pgfscope}%
\begin{pgfscope}%
\pgfsys@transformshift{4.018894in}{4.688020in}%
\pgfsys@useobject{currentmarker}{}%
\end{pgfscope}%
\begin{pgfscope}%
\pgfsys@transformshift{4.036488in}{4.659374in}%
\pgfsys@useobject{currentmarker}{}%
\end{pgfscope}%
\begin{pgfscope}%
\pgfsys@transformshift{4.054082in}{4.707568in}%
\pgfsys@useobject{currentmarker}{}%
\end{pgfscope}%
\begin{pgfscope}%
\pgfsys@transformshift{4.071676in}{4.679113in}%
\pgfsys@useobject{currentmarker}{}%
\end{pgfscope}%
\begin{pgfscope}%
\pgfsys@transformshift{4.089270in}{4.897831in}%
\pgfsys@useobject{currentmarker}{}%
\end{pgfscope}%
\begin{pgfscope}%
\pgfsys@transformshift{4.106864in}{4.778256in}%
\pgfsys@useobject{currentmarker}{}%
\end{pgfscope}%
\begin{pgfscope}%
\pgfsys@transformshift{4.124458in}{4.670860in}%
\pgfsys@useobject{currentmarker}{}%
\end{pgfscope}%
\begin{pgfscope}%
\pgfsys@transformshift{4.142051in}{4.519279in}%
\pgfsys@useobject{currentmarker}{}%
\end{pgfscope}%
\begin{pgfscope}%
\pgfsys@transformshift{4.159645in}{4.704570in}%
\pgfsys@useobject{currentmarker}{}%
\end{pgfscope}%
\begin{pgfscope}%
\pgfsys@transformshift{4.177239in}{4.502002in}%
\pgfsys@useobject{currentmarker}{}%
\end{pgfscope}%
\begin{pgfscope}%
\pgfsys@transformshift{4.194833in}{4.429759in}%
\pgfsys@useobject{currentmarker}{}%
\end{pgfscope}%
\begin{pgfscope}%
\pgfsys@transformshift{4.212427in}{4.684435in}%
\pgfsys@useobject{currentmarker}{}%
\end{pgfscope}%
\begin{pgfscope}%
\pgfsys@transformshift{4.230021in}{4.582583in}%
\pgfsys@useobject{currentmarker}{}%
\end{pgfscope}%
\begin{pgfscope}%
\pgfsys@transformshift{4.247615in}{4.628887in}%
\pgfsys@useobject{currentmarker}{}%
\end{pgfscope}%
\begin{pgfscope}%
\pgfsys@transformshift{4.265208in}{4.557126in}%
\pgfsys@useobject{currentmarker}{}%
\end{pgfscope}%
\begin{pgfscope}%
\pgfsys@transformshift{4.282802in}{4.600489in}%
\pgfsys@useobject{currentmarker}{}%
\end{pgfscope}%
\begin{pgfscope}%
\pgfsys@transformshift{4.300396in}{4.442537in}%
\pgfsys@useobject{currentmarker}{}%
\end{pgfscope}%
\begin{pgfscope}%
\pgfsys@transformshift{4.317990in}{4.398312in}%
\pgfsys@useobject{currentmarker}{}%
\end{pgfscope}%
\begin{pgfscope}%
\pgfsys@transformshift{4.335584in}{4.564662in}%
\pgfsys@useobject{currentmarker}{}%
\end{pgfscope}%
\begin{pgfscope}%
\pgfsys@transformshift{4.353178in}{4.415111in}%
\pgfsys@useobject{currentmarker}{}%
\end{pgfscope}%
\begin{pgfscope}%
\pgfsys@transformshift{4.370772in}{4.426479in}%
\pgfsys@useobject{currentmarker}{}%
\end{pgfscope}%
\begin{pgfscope}%
\pgfsys@transformshift{4.388365in}{4.451857in}%
\pgfsys@useobject{currentmarker}{}%
\end{pgfscope}%
\begin{pgfscope}%
\pgfsys@transformshift{4.405959in}{4.503017in}%
\pgfsys@useobject{currentmarker}{}%
\end{pgfscope}%
\begin{pgfscope}%
\pgfsys@transformshift{4.423553in}{4.472271in}%
\pgfsys@useobject{currentmarker}{}%
\end{pgfscope}%
\begin{pgfscope}%
\pgfsys@transformshift{4.441147in}{4.378919in}%
\pgfsys@useobject{currentmarker}{}%
\end{pgfscope}%
\begin{pgfscope}%
\pgfsys@transformshift{4.458741in}{4.461469in}%
\pgfsys@useobject{currentmarker}{}%
\end{pgfscope}%
\begin{pgfscope}%
\pgfsys@transformshift{4.476335in}{4.316143in}%
\pgfsys@useobject{currentmarker}{}%
\end{pgfscope}%
\begin{pgfscope}%
\pgfsys@transformshift{4.493928in}{4.612339in}%
\pgfsys@useobject{currentmarker}{}%
\end{pgfscope}%
\begin{pgfscope}%
\pgfsys@transformshift{4.511522in}{4.405734in}%
\pgfsys@useobject{currentmarker}{}%
\end{pgfscope}%
\begin{pgfscope}%
\pgfsys@transformshift{4.529116in}{4.471132in}%
\pgfsys@useobject{currentmarker}{}%
\end{pgfscope}%
\begin{pgfscope}%
\pgfsys@transformshift{4.546710in}{4.603117in}%
\pgfsys@useobject{currentmarker}{}%
\end{pgfscope}%
\begin{pgfscope}%
\pgfsys@transformshift{4.564304in}{4.542513in}%
\pgfsys@useobject{currentmarker}{}%
\end{pgfscope}%
\begin{pgfscope}%
\pgfsys@transformshift{4.581898in}{4.788087in}%
\pgfsys@useobject{currentmarker}{}%
\end{pgfscope}%
\begin{pgfscope}%
\pgfsys@transformshift{4.599492in}{4.525763in}%
\pgfsys@useobject{currentmarker}{}%
\end{pgfscope}%
\begin{pgfscope}%
\pgfsys@transformshift{4.617085in}{4.700507in}%
\pgfsys@useobject{currentmarker}{}%
\end{pgfscope}%
\begin{pgfscope}%
\pgfsys@transformshift{4.634679in}{4.690035in}%
\pgfsys@useobject{currentmarker}{}%
\end{pgfscope}%
\begin{pgfscope}%
\pgfsys@transformshift{4.652273in}{4.597836in}%
\pgfsys@useobject{currentmarker}{}%
\end{pgfscope}%
\begin{pgfscope}%
\pgfsys@transformshift{4.669867in}{4.785579in}%
\pgfsys@useobject{currentmarker}{}%
\end{pgfscope}%
\begin{pgfscope}%
\pgfsys@transformshift{4.687461in}{4.735854in}%
\pgfsys@useobject{currentmarker}{}%
\end{pgfscope}%
\begin{pgfscope}%
\pgfsys@transformshift{4.705055in}{4.848257in}%
\pgfsys@useobject{currentmarker}{}%
\end{pgfscope}%
\begin{pgfscope}%
\pgfsys@transformshift{4.722649in}{4.871288in}%
\pgfsys@useobject{currentmarker}{}%
\end{pgfscope}%
\begin{pgfscope}%
\pgfsys@transformshift{4.740242in}{5.021301in}%
\pgfsys@useobject{currentmarker}{}%
\end{pgfscope}%
\begin{pgfscope}%
\pgfsys@transformshift{4.757836in}{4.954956in}%
\pgfsys@useobject{currentmarker}{}%
\end{pgfscope}%
\begin{pgfscope}%
\pgfsys@transformshift{4.775430in}{4.799998in}%
\pgfsys@useobject{currentmarker}{}%
\end{pgfscope}%
\begin{pgfscope}%
\pgfsys@transformshift{4.793024in}{4.825952in}%
\pgfsys@useobject{currentmarker}{}%
\end{pgfscope}%
\begin{pgfscope}%
\pgfsys@transformshift{4.810618in}{4.970478in}%
\pgfsys@useobject{currentmarker}{}%
\end{pgfscope}%
\begin{pgfscope}%
\pgfsys@transformshift{4.828212in}{4.936169in}%
\pgfsys@useobject{currentmarker}{}%
\end{pgfscope}%
\begin{pgfscope}%
\pgfsys@transformshift{4.845805in}{4.942749in}%
\pgfsys@useobject{currentmarker}{}%
\end{pgfscope}%
\begin{pgfscope}%
\pgfsys@transformshift{4.863399in}{4.724782in}%
\pgfsys@useobject{currentmarker}{}%
\end{pgfscope}%
\begin{pgfscope}%
\pgfsys@transformshift{4.880993in}{4.887375in}%
\pgfsys@useobject{currentmarker}{}%
\end{pgfscope}%
\begin{pgfscope}%
\pgfsys@transformshift{4.898587in}{4.818940in}%
\pgfsys@useobject{currentmarker}{}%
\end{pgfscope}%
\begin{pgfscope}%
\pgfsys@transformshift{4.916181in}{4.920607in}%
\pgfsys@useobject{currentmarker}{}%
\end{pgfscope}%
\begin{pgfscope}%
\pgfsys@transformshift{4.933775in}{4.881672in}%
\pgfsys@useobject{currentmarker}{}%
\end{pgfscope}%
\begin{pgfscope}%
\pgfsys@transformshift{4.951369in}{4.978576in}%
\pgfsys@useobject{currentmarker}{}%
\end{pgfscope}%
\begin{pgfscope}%
\pgfsys@transformshift{4.968962in}{4.914563in}%
\pgfsys@useobject{currentmarker}{}%
\end{pgfscope}%
\begin{pgfscope}%
\pgfsys@transformshift{4.986556in}{4.954282in}%
\pgfsys@useobject{currentmarker}{}%
\end{pgfscope}%
\begin{pgfscope}%
\pgfsys@transformshift{5.004150in}{4.821260in}%
\pgfsys@useobject{currentmarker}{}%
\end{pgfscope}%
\begin{pgfscope}%
\pgfsys@transformshift{5.021744in}{4.763613in}%
\pgfsys@useobject{currentmarker}{}%
\end{pgfscope}%
\begin{pgfscope}%
\pgfsys@transformshift{5.039338in}{4.805078in}%
\pgfsys@useobject{currentmarker}{}%
\end{pgfscope}%
\begin{pgfscope}%
\pgfsys@transformshift{5.056932in}{4.831169in}%
\pgfsys@useobject{currentmarker}{}%
\end{pgfscope}%
\begin{pgfscope}%
\pgfsys@transformshift{5.074526in}{4.856393in}%
\pgfsys@useobject{currentmarker}{}%
\end{pgfscope}%
\begin{pgfscope}%
\pgfsys@transformshift{5.092119in}{5.028023in}%
\pgfsys@useobject{currentmarker}{}%
\end{pgfscope}%
\begin{pgfscope}%
\pgfsys@transformshift{5.109713in}{4.783388in}%
\pgfsys@useobject{currentmarker}{}%
\end{pgfscope}%
\begin{pgfscope}%
\pgfsys@transformshift{5.127307in}{4.675969in}%
\pgfsys@useobject{currentmarker}{}%
\end{pgfscope}%
\begin{pgfscope}%
\pgfsys@transformshift{5.144901in}{4.719445in}%
\pgfsys@useobject{currentmarker}{}%
\end{pgfscope}%
\begin{pgfscope}%
\pgfsys@transformshift{5.162495in}{4.690369in}%
\pgfsys@useobject{currentmarker}{}%
\end{pgfscope}%
\begin{pgfscope}%
\pgfsys@transformshift{5.180089in}{4.766790in}%
\pgfsys@useobject{currentmarker}{}%
\end{pgfscope}%
\begin{pgfscope}%
\pgfsys@transformshift{5.197683in}{4.548557in}%
\pgfsys@useobject{currentmarker}{}%
\end{pgfscope}%
\begin{pgfscope}%
\pgfsys@transformshift{5.215276in}{4.690901in}%
\pgfsys@useobject{currentmarker}{}%
\end{pgfscope}%
\begin{pgfscope}%
\pgfsys@transformshift{5.232870in}{4.683687in}%
\pgfsys@useobject{currentmarker}{}%
\end{pgfscope}%
\begin{pgfscope}%
\pgfsys@transformshift{5.250464in}{4.675448in}%
\pgfsys@useobject{currentmarker}{}%
\end{pgfscope}%
\begin{pgfscope}%
\pgfsys@transformshift{5.268058in}{4.578198in}%
\pgfsys@useobject{currentmarker}{}%
\end{pgfscope}%
\begin{pgfscope}%
\pgfsys@transformshift{5.285652in}{4.600143in}%
\pgfsys@useobject{currentmarker}{}%
\end{pgfscope}%
\begin{pgfscope}%
\pgfsys@transformshift{5.303246in}{4.469808in}%
\pgfsys@useobject{currentmarker}{}%
\end{pgfscope}%
\begin{pgfscope}%
\pgfsys@transformshift{5.320839in}{4.551183in}%
\pgfsys@useobject{currentmarker}{}%
\end{pgfscope}%
\begin{pgfscope}%
\pgfsys@transformshift{5.338433in}{4.536732in}%
\pgfsys@useobject{currentmarker}{}%
\end{pgfscope}%
\begin{pgfscope}%
\pgfsys@transformshift{5.356027in}{4.624149in}%
\pgfsys@useobject{currentmarker}{}%
\end{pgfscope}%
\begin{pgfscope}%
\pgfsys@transformshift{5.373621in}{4.461915in}%
\pgfsys@useobject{currentmarker}{}%
\end{pgfscope}%
\begin{pgfscope}%
\pgfsys@transformshift{5.391215in}{4.650228in}%
\pgfsys@useobject{currentmarker}{}%
\end{pgfscope}%
\begin{pgfscope}%
\pgfsys@transformshift{5.408809in}{4.718906in}%
\pgfsys@useobject{currentmarker}{}%
\end{pgfscope}%
\begin{pgfscope}%
\pgfsys@transformshift{5.426403in}{4.364651in}%
\pgfsys@useobject{currentmarker}{}%
\end{pgfscope}%
\begin{pgfscope}%
\pgfsys@transformshift{5.443996in}{4.614547in}%
\pgfsys@useobject{currentmarker}{}%
\end{pgfscope}%
\begin{pgfscope}%
\pgfsys@transformshift{5.461590in}{4.643444in}%
\pgfsys@useobject{currentmarker}{}%
\end{pgfscope}%
\begin{pgfscope}%
\pgfsys@transformshift{5.479184in}{4.518952in}%
\pgfsys@useobject{currentmarker}{}%
\end{pgfscope}%
\begin{pgfscope}%
\pgfsys@transformshift{5.496778in}{4.550949in}%
\pgfsys@useobject{currentmarker}{}%
\end{pgfscope}%
\begin{pgfscope}%
\pgfsys@transformshift{5.514372in}{4.587369in}%
\pgfsys@useobject{currentmarker}{}%
\end{pgfscope}%
\begin{pgfscope}%
\pgfsys@transformshift{5.531966in}{4.583096in}%
\pgfsys@useobject{currentmarker}{}%
\end{pgfscope}%
\begin{pgfscope}%
\pgfsys@transformshift{5.549560in}{4.595964in}%
\pgfsys@useobject{currentmarker}{}%
\end{pgfscope}%
\begin{pgfscope}%
\pgfsys@transformshift{5.567153in}{4.475923in}%
\pgfsys@useobject{currentmarker}{}%
\end{pgfscope}%
\begin{pgfscope}%
\pgfsys@transformshift{5.584747in}{4.774312in}%
\pgfsys@useobject{currentmarker}{}%
\end{pgfscope}%
\begin{pgfscope}%
\pgfsys@transformshift{5.602341in}{4.786033in}%
\pgfsys@useobject{currentmarker}{}%
\end{pgfscope}%
\begin{pgfscope}%
\pgfsys@transformshift{5.619935in}{4.618304in}%
\pgfsys@useobject{currentmarker}{}%
\end{pgfscope}%
\begin{pgfscope}%
\pgfsys@transformshift{5.637529in}{4.574972in}%
\pgfsys@useobject{currentmarker}{}%
\end{pgfscope}%
\begin{pgfscope}%
\pgfsys@transformshift{5.655123in}{4.795007in}%
\pgfsys@useobject{currentmarker}{}%
\end{pgfscope}%
\begin{pgfscope}%
\pgfsys@transformshift{5.672717in}{4.709505in}%
\pgfsys@useobject{currentmarker}{}%
\end{pgfscope}%
\begin{pgfscope}%
\pgfsys@transformshift{5.690310in}{4.804947in}%
\pgfsys@useobject{currentmarker}{}%
\end{pgfscope}%
\begin{pgfscope}%
\pgfsys@transformshift{5.707904in}{4.783747in}%
\pgfsys@useobject{currentmarker}{}%
\end{pgfscope}%
\begin{pgfscope}%
\pgfsys@transformshift{5.725498in}{4.909079in}%
\pgfsys@useobject{currentmarker}{}%
\end{pgfscope}%
\begin{pgfscope}%
\pgfsys@transformshift{5.743092in}{4.934392in}%
\pgfsys@useobject{currentmarker}{}%
\end{pgfscope}%
\begin{pgfscope}%
\pgfsys@transformshift{5.760686in}{4.818168in}%
\pgfsys@useobject{currentmarker}{}%
\end{pgfscope}%
\begin{pgfscope}%
\pgfsys@transformshift{5.778280in}{4.777384in}%
\pgfsys@useobject{currentmarker}{}%
\end{pgfscope}%
\begin{pgfscope}%
\pgfsys@transformshift{5.795873in}{4.781605in}%
\pgfsys@useobject{currentmarker}{}%
\end{pgfscope}%
\begin{pgfscope}%
\pgfsys@transformshift{5.813467in}{5.022841in}%
\pgfsys@useobject{currentmarker}{}%
\end{pgfscope}%
\begin{pgfscope}%
\pgfsys@transformshift{5.831061in}{4.866566in}%
\pgfsys@useobject{currentmarker}{}%
\end{pgfscope}%
\begin{pgfscope}%
\pgfsys@transformshift{5.848655in}{4.955732in}%
\pgfsys@useobject{currentmarker}{}%
\end{pgfscope}%
\begin{pgfscope}%
\pgfsys@transformshift{5.866249in}{4.967038in}%
\pgfsys@useobject{currentmarker}{}%
\end{pgfscope}%
\begin{pgfscope}%
\pgfsys@transformshift{5.883843in}{5.039890in}%
\pgfsys@useobject{currentmarker}{}%
\end{pgfscope}%
\begin{pgfscope}%
\pgfsys@transformshift{5.901437in}{4.870099in}%
\pgfsys@useobject{currentmarker}{}%
\end{pgfscope}%
\begin{pgfscope}%
\pgfsys@transformshift{5.919030in}{5.096938in}%
\pgfsys@useobject{currentmarker}{}%
\end{pgfscope}%
\begin{pgfscope}%
\pgfsys@transformshift{5.936624in}{5.144249in}%
\pgfsys@useobject{currentmarker}{}%
\end{pgfscope}%
\begin{pgfscope}%
\pgfsys@transformshift{5.954218in}{5.113218in}%
\pgfsys@useobject{currentmarker}{}%
\end{pgfscope}%
\begin{pgfscope}%
\pgfsys@transformshift{5.971812in}{5.083924in}%
\pgfsys@useobject{currentmarker}{}%
\end{pgfscope}%
\begin{pgfscope}%
\pgfsys@transformshift{5.989406in}{5.132962in}%
\pgfsys@useobject{currentmarker}{}%
\end{pgfscope}%
\begin{pgfscope}%
\pgfsys@transformshift{6.007000in}{5.169484in}%
\pgfsys@useobject{currentmarker}{}%
\end{pgfscope}%
\begin{pgfscope}%
\pgfsys@transformshift{6.024594in}{4.859040in}%
\pgfsys@useobject{currentmarker}{}%
\end{pgfscope}%
\begin{pgfscope}%
\pgfsys@transformshift{6.042187in}{5.331506in}%
\pgfsys@useobject{currentmarker}{}%
\end{pgfscope}%
\begin{pgfscope}%
\pgfsys@transformshift{6.059781in}{5.176812in}%
\pgfsys@useobject{currentmarker}{}%
\end{pgfscope}%
\begin{pgfscope}%
\pgfsys@transformshift{6.077375in}{5.072293in}%
\pgfsys@useobject{currentmarker}{}%
\end{pgfscope}%
\begin{pgfscope}%
\pgfsys@transformshift{6.094969in}{5.095575in}%
\pgfsys@useobject{currentmarker}{}%
\end{pgfscope}%
\begin{pgfscope}%
\pgfsys@transformshift{6.112563in}{5.179130in}%
\pgfsys@useobject{currentmarker}{}%
\end{pgfscope}%
\begin{pgfscope}%
\pgfsys@transformshift{6.130157in}{5.112685in}%
\pgfsys@useobject{currentmarker}{}%
\end{pgfscope}%
\begin{pgfscope}%
\pgfsys@transformshift{6.147751in}{4.915222in}%
\pgfsys@useobject{currentmarker}{}%
\end{pgfscope}%
\begin{pgfscope}%
\pgfsys@transformshift{6.165344in}{5.313522in}%
\pgfsys@useobject{currentmarker}{}%
\end{pgfscope}%
\begin{pgfscope}%
\pgfsys@transformshift{6.182938in}{5.087825in}%
\pgfsys@useobject{currentmarker}{}%
\end{pgfscope}%
\begin{pgfscope}%
\pgfsys@transformshift{6.200532in}{5.189593in}%
\pgfsys@useobject{currentmarker}{}%
\end{pgfscope}%
\begin{pgfscope}%
\pgfsys@transformshift{6.218126in}{5.008182in}%
\pgfsys@useobject{currentmarker}{}%
\end{pgfscope}%
\begin{pgfscope}%
\pgfsys@transformshift{6.235720in}{5.217676in}%
\pgfsys@useobject{currentmarker}{}%
\end{pgfscope}%
\begin{pgfscope}%
\pgfsys@transformshift{6.253314in}{5.081238in}%
\pgfsys@useobject{currentmarker}{}%
\end{pgfscope}%
\begin{pgfscope}%
\pgfsys@transformshift{6.270907in}{5.100842in}%
\pgfsys@useobject{currentmarker}{}%
\end{pgfscope}%
\begin{pgfscope}%
\pgfsys@transformshift{6.288501in}{4.924112in}%
\pgfsys@useobject{currentmarker}{}%
\end{pgfscope}%
\begin{pgfscope}%
\pgfsys@transformshift{6.306095in}{5.136084in}%
\pgfsys@useobject{currentmarker}{}%
\end{pgfscope}%
\begin{pgfscope}%
\pgfsys@transformshift{6.323689in}{5.072537in}%
\pgfsys@useobject{currentmarker}{}%
\end{pgfscope}%
\begin{pgfscope}%
\pgfsys@transformshift{6.341283in}{5.122202in}%
\pgfsys@useobject{currentmarker}{}%
\end{pgfscope}%
\begin{pgfscope}%
\pgfsys@transformshift{6.358877in}{4.920157in}%
\pgfsys@useobject{currentmarker}{}%
\end{pgfscope}%
\begin{pgfscope}%
\pgfsys@transformshift{6.376471in}{4.925441in}%
\pgfsys@useobject{currentmarker}{}%
\end{pgfscope}%
\end{pgfscope}%
\begin{pgfscope}%
\pgfsetbuttcap%
\pgfsetroundjoin%
\definecolor{currentfill}{rgb}{0.000000,0.000000,0.000000}%
\pgfsetfillcolor{currentfill}%
\pgfsetlinewidth{0.803000pt}%
\definecolor{currentstroke}{rgb}{0.000000,0.000000,0.000000}%
\pgfsetstrokecolor{currentstroke}%
\pgfsetdash{}{0pt}%
\pgfsys@defobject{currentmarker}{\pgfqpoint{0.000000in}{-0.048611in}}{\pgfqpoint{0.000000in}{0.000000in}}{%
\pgfpathmoveto{\pgfqpoint{0.000000in}{0.000000in}}%
\pgfpathlineto{\pgfqpoint{0.000000in}{-0.048611in}}%
\pgfusepath{stroke,fill}%
}%
\begin{pgfscope}%
\pgfsys@transformshift{2.000000in}{4.326316in}%
\pgfsys@useobject{currentmarker}{}%
\end{pgfscope}%
\end{pgfscope}%
\begin{pgfscope}%
\pgfsetbuttcap%
\pgfsetroundjoin%
\definecolor{currentfill}{rgb}{0.000000,0.000000,0.000000}%
\pgfsetfillcolor{currentfill}%
\pgfsetlinewidth{0.803000pt}%
\definecolor{currentstroke}{rgb}{0.000000,0.000000,0.000000}%
\pgfsetstrokecolor{currentstroke}%
\pgfsetdash{}{0pt}%
\pgfsys@defobject{currentmarker}{\pgfqpoint{0.000000in}{-0.048611in}}{\pgfqpoint{0.000000in}{0.000000in}}{%
\pgfpathmoveto{\pgfqpoint{0.000000in}{0.000000in}}%
\pgfpathlineto{\pgfqpoint{0.000000in}{-0.048611in}}%
\pgfusepath{stroke,fill}%
}%
\begin{pgfscope}%
\pgfsys@transformshift{2.875294in}{4.326316in}%
\pgfsys@useobject{currentmarker}{}%
\end{pgfscope}%
\end{pgfscope}%
\begin{pgfscope}%
\pgfsetbuttcap%
\pgfsetroundjoin%
\definecolor{currentfill}{rgb}{0.000000,0.000000,0.000000}%
\pgfsetfillcolor{currentfill}%
\pgfsetlinewidth{0.803000pt}%
\definecolor{currentstroke}{rgb}{0.000000,0.000000,0.000000}%
\pgfsetstrokecolor{currentstroke}%
\pgfsetdash{}{0pt}%
\pgfsys@defobject{currentmarker}{\pgfqpoint{0.000000in}{-0.048611in}}{\pgfqpoint{0.000000in}{0.000000in}}{%
\pgfpathmoveto{\pgfqpoint{0.000000in}{0.000000in}}%
\pgfpathlineto{\pgfqpoint{0.000000in}{-0.048611in}}%
\pgfusepath{stroke,fill}%
}%
\begin{pgfscope}%
\pgfsys@transformshift{3.750588in}{4.326316in}%
\pgfsys@useobject{currentmarker}{}%
\end{pgfscope}%
\end{pgfscope}%
\begin{pgfscope}%
\pgfsetbuttcap%
\pgfsetroundjoin%
\definecolor{currentfill}{rgb}{0.000000,0.000000,0.000000}%
\pgfsetfillcolor{currentfill}%
\pgfsetlinewidth{0.803000pt}%
\definecolor{currentstroke}{rgb}{0.000000,0.000000,0.000000}%
\pgfsetstrokecolor{currentstroke}%
\pgfsetdash{}{0pt}%
\pgfsys@defobject{currentmarker}{\pgfqpoint{0.000000in}{-0.048611in}}{\pgfqpoint{0.000000in}{0.000000in}}{%
\pgfpathmoveto{\pgfqpoint{0.000000in}{0.000000in}}%
\pgfpathlineto{\pgfqpoint{0.000000in}{-0.048611in}}%
\pgfusepath{stroke,fill}%
}%
\begin{pgfscope}%
\pgfsys@transformshift{4.625882in}{4.326316in}%
\pgfsys@useobject{currentmarker}{}%
\end{pgfscope}%
\end{pgfscope}%
\begin{pgfscope}%
\pgfsetbuttcap%
\pgfsetroundjoin%
\definecolor{currentfill}{rgb}{0.000000,0.000000,0.000000}%
\pgfsetfillcolor{currentfill}%
\pgfsetlinewidth{0.803000pt}%
\definecolor{currentstroke}{rgb}{0.000000,0.000000,0.000000}%
\pgfsetstrokecolor{currentstroke}%
\pgfsetdash{}{0pt}%
\pgfsys@defobject{currentmarker}{\pgfqpoint{0.000000in}{-0.048611in}}{\pgfqpoint{0.000000in}{0.000000in}}{%
\pgfpathmoveto{\pgfqpoint{0.000000in}{0.000000in}}%
\pgfpathlineto{\pgfqpoint{0.000000in}{-0.048611in}}%
\pgfusepath{stroke,fill}%
}%
\begin{pgfscope}%
\pgfsys@transformshift{5.501176in}{4.326316in}%
\pgfsys@useobject{currentmarker}{}%
\end{pgfscope}%
\end{pgfscope}%
\begin{pgfscope}%
\pgfsetbuttcap%
\pgfsetroundjoin%
\definecolor{currentfill}{rgb}{0.000000,0.000000,0.000000}%
\pgfsetfillcolor{currentfill}%
\pgfsetlinewidth{0.803000pt}%
\definecolor{currentstroke}{rgb}{0.000000,0.000000,0.000000}%
\pgfsetstrokecolor{currentstroke}%
\pgfsetdash{}{0pt}%
\pgfsys@defobject{currentmarker}{\pgfqpoint{0.000000in}{-0.048611in}}{\pgfqpoint{0.000000in}{0.000000in}}{%
\pgfpathmoveto{\pgfqpoint{0.000000in}{0.000000in}}%
\pgfpathlineto{\pgfqpoint{0.000000in}{-0.048611in}}%
\pgfusepath{stroke,fill}%
}%
\begin{pgfscope}%
\pgfsys@transformshift{6.376471in}{4.326316in}%
\pgfsys@useobject{currentmarker}{}%
\end{pgfscope}%
\end{pgfscope}%
\begin{pgfscope}%
\pgfsetbuttcap%
\pgfsetroundjoin%
\definecolor{currentfill}{rgb}{0.000000,0.000000,0.000000}%
\pgfsetfillcolor{currentfill}%
\pgfsetlinewidth{0.803000pt}%
\definecolor{currentstroke}{rgb}{0.000000,0.000000,0.000000}%
\pgfsetstrokecolor{currentstroke}%
\pgfsetdash{}{0pt}%
\pgfsys@defobject{currentmarker}{\pgfqpoint{-0.048611in}{0.000000in}}{\pgfqpoint{0.000000in}{0.000000in}}{%
\pgfpathmoveto{\pgfqpoint{0.000000in}{0.000000in}}%
\pgfpathlineto{\pgfqpoint{-0.048611in}{0.000000in}}%
\pgfusepath{stroke,fill}%
}%
\begin{pgfscope}%
\pgfsys@transformshift{2.000000in}{4.683947in}%
\pgfsys@useobject{currentmarker}{}%
\end{pgfscope}%
\end{pgfscope}%
\begin{pgfscope}%
\pgftext[x=1.833333in,y=4.635730in,left,base]{\rmfamily\fontsize{10.000000}{12.000000}\selectfont \(\displaystyle 0\)}%
\end{pgfscope}%
\begin{pgfscope}%
\pgfsetbuttcap%
\pgfsetroundjoin%
\definecolor{currentfill}{rgb}{0.000000,0.000000,0.000000}%
\pgfsetfillcolor{currentfill}%
\pgfsetlinewidth{0.803000pt}%
\definecolor{currentstroke}{rgb}{0.000000,0.000000,0.000000}%
\pgfsetstrokecolor{currentstroke}%
\pgfsetdash{}{0pt}%
\pgfsys@defobject{currentmarker}{\pgfqpoint{-0.048611in}{0.000000in}}{\pgfqpoint{0.000000in}{0.000000in}}{%
\pgfpathmoveto{\pgfqpoint{0.000000in}{0.000000in}}%
\pgfpathlineto{\pgfqpoint{-0.048611in}{0.000000in}}%
\pgfusepath{stroke,fill}%
}%
\begin{pgfscope}%
\pgfsys@transformshift{2.000000in}{5.081316in}%
\pgfsys@useobject{currentmarker}{}%
\end{pgfscope}%
\end{pgfscope}%
\begin{pgfscope}%
\pgftext[x=1.833333in,y=5.033098in,left,base]{\rmfamily\fontsize{10.000000}{12.000000}\selectfont \(\displaystyle 2\)}%
\end{pgfscope}%
\begin{pgfscope}%
\pgftext[x=1.777777in,y=4.803158in,,bottom,rotate=90.000000]{\rmfamily\fontsize{10.000000}{12.000000}\selectfont y}%
\end{pgfscope}%
\begin{pgfscope}%
\pgfpathrectangle{\pgfqpoint{2.000000in}{4.326316in}}{\pgfqpoint{4.376471in}{0.953684in}} %
\pgfusepath{clip}%
\pgfsetrectcap%
\pgfsetroundjoin%
\pgfsetlinewidth{1.505625pt}%
\definecolor{currentstroke}{rgb}{0.121569,0.466667,0.705882}%
\pgfsetstrokecolor{currentstroke}%
\pgfsetdash{}{0pt}%
\pgfpathmoveto{\pgfqpoint{2.875294in}{5.176896in}}%
\pgfpathlineto{\pgfqpoint{3.033639in}{4.926852in}}%
\pgfpathlineto{\pgfqpoint{3.068826in}{4.878864in}}%
\pgfpathlineto{\pgfqpoint{3.104014in}{4.836349in}}%
\pgfpathlineto{\pgfqpoint{3.139202in}{4.800205in}}%
\pgfpathlineto{\pgfqpoint{3.174390in}{4.771098in}}%
\pgfpathlineto{\pgfqpoint{3.191983in}{4.759322in}}%
\pgfpathlineto{\pgfqpoint{3.209577in}{4.749439in}}%
\pgfpathlineto{\pgfqpoint{3.227171in}{4.741460in}}%
\pgfpathlineto{\pgfqpoint{3.244765in}{4.735380in}}%
\pgfpathlineto{\pgfqpoint{3.262359in}{4.731177in}}%
\pgfpathlineto{\pgfqpoint{3.279953in}{4.728815in}}%
\pgfpathlineto{\pgfqpoint{3.297547in}{4.728240in}}%
\pgfpathlineto{\pgfqpoint{3.315140in}{4.729387in}}%
\pgfpathlineto{\pgfqpoint{3.350328in}{4.736507in}}%
\pgfpathlineto{\pgfqpoint{3.385516in}{4.749378in}}%
\pgfpathlineto{\pgfqpoint{3.420704in}{4.767024in}}%
\pgfpathlineto{\pgfqpoint{3.455891in}{4.788329in}}%
\pgfpathlineto{\pgfqpoint{3.508673in}{4.824463in}}%
\pgfpathlineto{\pgfqpoint{3.596642in}{4.885170in}}%
\pgfpathlineto{\pgfqpoint{3.631830in}{4.906006in}}%
\pgfpathlineto{\pgfqpoint{3.667017in}{4.923204in}}%
\pgfpathlineto{\pgfqpoint{3.702205in}{4.935843in}}%
\pgfpathlineto{\pgfqpoint{3.737393in}{4.943181in}}%
\pgfpathlineto{\pgfqpoint{3.772581in}{4.944681in}}%
\pgfpathlineto{\pgfqpoint{3.790174in}{4.943133in}}%
\pgfpathlineto{\pgfqpoint{3.825362in}{4.935350in}}%
\pgfpathlineto{\pgfqpoint{3.860550in}{4.921364in}}%
\pgfpathlineto{\pgfqpoint{3.895738in}{4.901431in}}%
\pgfpathlineto{\pgfqpoint{3.930925in}{4.876023in}}%
\pgfpathlineto{\pgfqpoint{3.966113in}{4.845814in}}%
\pgfpathlineto{\pgfqpoint{4.001301in}{4.811652in}}%
\pgfpathlineto{\pgfqpoint{4.054082in}{4.755208in}}%
\pgfpathlineto{\pgfqpoint{4.177239in}{4.619416in}}%
\pgfpathlineto{\pgfqpoint{4.212427in}{4.584902in}}%
\pgfpathlineto{\pgfqpoint{4.247615in}{4.554313in}}%
\pgfpathlineto{\pgfqpoint{4.282802in}{4.528572in}}%
\pgfpathlineto{\pgfqpoint{4.317990in}{4.508446in}}%
\pgfpathlineto{\pgfqpoint{4.353178in}{4.494513in}}%
\pgfpathlineto{\pgfqpoint{4.370772in}{4.489995in}}%
\pgfpathlineto{\pgfqpoint{4.388365in}{4.487148in}}%
\pgfpathlineto{\pgfqpoint{4.405959in}{4.485987in}}%
\pgfpathlineto{\pgfqpoint{4.423553in}{4.486509in}}%
\pgfpathlineto{\pgfqpoint{4.441147in}{4.488699in}}%
\pgfpathlineto{\pgfqpoint{4.458741in}{4.492528in}}%
\pgfpathlineto{\pgfqpoint{4.493928in}{4.504920in}}%
\pgfpathlineto{\pgfqpoint{4.529116in}{4.523188in}}%
\pgfpathlineto{\pgfqpoint{4.564304in}{4.546645in}}%
\pgfpathlineto{\pgfqpoint{4.599492in}{4.574431in}}%
\pgfpathlineto{\pgfqpoint{4.652273in}{4.622022in}}%
\pgfpathlineto{\pgfqpoint{4.793024in}{4.756298in}}%
\pgfpathlineto{\pgfqpoint{4.828212in}{4.785291in}}%
\pgfpathlineto{\pgfqpoint{4.863399in}{4.810391in}}%
\pgfpathlineto{\pgfqpoint{4.898587in}{4.830788in}}%
\pgfpathlineto{\pgfqpoint{4.933775in}{4.845855in}}%
\pgfpathlineto{\pgfqpoint{4.968962in}{4.855163in}}%
\pgfpathlineto{\pgfqpoint{5.004150in}{4.858503in}}%
\pgfpathlineto{\pgfqpoint{5.039338in}{4.855890in}}%
\pgfpathlineto{\pgfqpoint{5.074526in}{4.847567in}}%
\pgfpathlineto{\pgfqpoint{5.109713in}{4.833997in}}%
\pgfpathlineto{\pgfqpoint{5.144901in}{4.815845in}}%
\pgfpathlineto{\pgfqpoint{5.180089in}{4.793959in}}%
\pgfpathlineto{\pgfqpoint{5.232870in}{4.756357in}}%
\pgfpathlineto{\pgfqpoint{5.338433in}{4.678458in}}%
\pgfpathlineto{\pgfqpoint{5.373621in}{4.656181in}}%
\pgfpathlineto{\pgfqpoint{5.408809in}{4.637694in}}%
\pgfpathlineto{\pgfqpoint{5.443996in}{4.624018in}}%
\pgfpathlineto{\pgfqpoint{5.479184in}{4.616015in}}%
\pgfpathlineto{\pgfqpoint{5.496778in}{4.614359in}}%
\pgfpathlineto{\pgfqpoint{5.514372in}{4.614356in}}%
\pgfpathlineto{\pgfqpoint{5.531966in}{4.616056in}}%
\pgfpathlineto{\pgfqpoint{5.549560in}{4.619494in}}%
\pgfpathlineto{\pgfqpoint{5.567153in}{4.624686in}}%
\pgfpathlineto{\pgfqpoint{5.584747in}{4.631637in}}%
\pgfpathlineto{\pgfqpoint{5.619935in}{4.650738in}}%
\pgfpathlineto{\pgfqpoint{5.655123in}{4.676489in}}%
\pgfpathlineto{\pgfqpoint{5.690310in}{4.708319in}}%
\pgfpathlineto{\pgfqpoint{5.725498in}{4.745411in}}%
\pgfpathlineto{\pgfqpoint{5.760686in}{4.786716in}}%
\pgfpathlineto{\pgfqpoint{5.813467in}{4.853798in}}%
\pgfpathlineto{\pgfqpoint{5.901437in}{4.966852in}}%
\pgfpathlineto{\pgfqpoint{5.936624in}{5.007802in}}%
\pgfpathlineto{\pgfqpoint{5.971812in}{5.043857in}}%
\pgfpathlineto{\pgfqpoint{6.007000in}{5.073459in}}%
\pgfpathlineto{\pgfqpoint{6.024594in}{5.085385in}}%
\pgfpathlineto{\pgfqpoint{6.042187in}{5.095172in}}%
\pgfpathlineto{\pgfqpoint{6.059781in}{5.102668in}}%
\pgfpathlineto{\pgfqpoint{6.077375in}{5.107733in}}%
\pgfpathlineto{\pgfqpoint{6.094969in}{5.110245in}}%
\pgfpathlineto{\pgfqpoint{6.112563in}{5.110094in}}%
\pgfpathlineto{\pgfqpoint{6.130157in}{5.107190in}}%
\pgfpathlineto{\pgfqpoint{6.147751in}{5.101459in}}%
\pgfpathlineto{\pgfqpoint{6.165344in}{5.092846in}}%
\pgfpathlineto{\pgfqpoint{6.182938in}{5.081315in}}%
\pgfpathlineto{\pgfqpoint{6.200532in}{5.066849in}}%
\pgfpathlineto{\pgfqpoint{6.218126in}{5.049453in}}%
\pgfpathlineto{\pgfqpoint{6.235720in}{5.029151in}}%
\pgfpathlineto{\pgfqpoint{6.253314in}{5.005987in}}%
\pgfpathlineto{\pgfqpoint{6.288501in}{4.951350in}}%
\pgfpathlineto{\pgfqpoint{6.323689in}{4.886298in}}%
\pgfpathlineto{\pgfqpoint{6.358877in}{4.811885in}}%
\pgfpathlineto{\pgfqpoint{6.376471in}{4.771576in}}%
\pgfpathlineto{\pgfqpoint{6.376471in}{4.771576in}}%
\pgfusepath{stroke}%
\end{pgfscope}%
\begin{pgfscope}%
\pgfpathrectangle{\pgfqpoint{2.000000in}{4.326316in}}{\pgfqpoint{4.376471in}{0.953684in}} %
\pgfusepath{clip}%
\pgfsetbuttcap%
\pgfsetroundjoin%
\pgfsetlinewidth{1.505625pt}%
\definecolor{currentstroke}{rgb}{1.000000,0.498039,0.054902}%
\pgfsetstrokecolor{currentstroke}%
\pgfsetdash{{9.600000pt}{2.400000pt}{1.600000pt}{2.400000pt}}{0.000000pt}%
\pgfpathmoveto{\pgfqpoint{2.875294in}{5.176896in}}%
\pgfpathlineto{\pgfqpoint{3.033639in}{4.926852in}}%
\pgfpathlineto{\pgfqpoint{3.068826in}{4.878864in}}%
\pgfpathlineto{\pgfqpoint{3.104014in}{4.836349in}}%
\pgfpathlineto{\pgfqpoint{3.139202in}{4.800205in}}%
\pgfpathlineto{\pgfqpoint{3.174390in}{4.771098in}}%
\pgfpathlineto{\pgfqpoint{3.191983in}{4.759322in}}%
\pgfpathlineto{\pgfqpoint{3.209577in}{4.749439in}}%
\pgfpathlineto{\pgfqpoint{3.227171in}{4.741460in}}%
\pgfpathlineto{\pgfqpoint{3.244765in}{4.735380in}}%
\pgfpathlineto{\pgfqpoint{3.262359in}{4.731177in}}%
\pgfpathlineto{\pgfqpoint{3.279953in}{4.728815in}}%
\pgfpathlineto{\pgfqpoint{3.297547in}{4.728240in}}%
\pgfpathlineto{\pgfqpoint{3.315140in}{4.729387in}}%
\pgfpathlineto{\pgfqpoint{3.350328in}{4.736507in}}%
\pgfpathlineto{\pgfqpoint{3.385516in}{4.749378in}}%
\pgfpathlineto{\pgfqpoint{3.420704in}{4.767024in}}%
\pgfpathlineto{\pgfqpoint{3.455891in}{4.788329in}}%
\pgfpathlineto{\pgfqpoint{3.508673in}{4.824463in}}%
\pgfpathlineto{\pgfqpoint{3.596642in}{4.885170in}}%
\pgfpathlineto{\pgfqpoint{3.631830in}{4.906006in}}%
\pgfpathlineto{\pgfqpoint{3.667017in}{4.923204in}}%
\pgfpathlineto{\pgfqpoint{3.702205in}{4.935843in}}%
\pgfpathlineto{\pgfqpoint{3.737393in}{4.943181in}}%
\pgfpathlineto{\pgfqpoint{3.772581in}{4.944681in}}%
\pgfpathlineto{\pgfqpoint{3.790174in}{4.943133in}}%
\pgfpathlineto{\pgfqpoint{3.825362in}{4.935350in}}%
\pgfpathlineto{\pgfqpoint{3.860550in}{4.921364in}}%
\pgfpathlineto{\pgfqpoint{3.895738in}{4.901431in}}%
\pgfpathlineto{\pgfqpoint{3.930925in}{4.876023in}}%
\pgfpathlineto{\pgfqpoint{3.966113in}{4.845814in}}%
\pgfpathlineto{\pgfqpoint{4.001301in}{4.811652in}}%
\pgfpathlineto{\pgfqpoint{4.054082in}{4.755208in}}%
\pgfpathlineto{\pgfqpoint{4.177239in}{4.619416in}}%
\pgfpathlineto{\pgfqpoint{4.212427in}{4.584902in}}%
\pgfpathlineto{\pgfqpoint{4.247615in}{4.554313in}}%
\pgfpathlineto{\pgfqpoint{4.282802in}{4.528572in}}%
\pgfpathlineto{\pgfqpoint{4.317990in}{4.508446in}}%
\pgfpathlineto{\pgfqpoint{4.353178in}{4.494513in}}%
\pgfpathlineto{\pgfqpoint{4.370772in}{4.489995in}}%
\pgfpathlineto{\pgfqpoint{4.388365in}{4.487148in}}%
\pgfpathlineto{\pgfqpoint{4.405959in}{4.485987in}}%
\pgfpathlineto{\pgfqpoint{4.423553in}{4.486509in}}%
\pgfpathlineto{\pgfqpoint{4.441147in}{4.488699in}}%
\pgfpathlineto{\pgfqpoint{4.458741in}{4.492528in}}%
\pgfpathlineto{\pgfqpoint{4.493928in}{4.504920in}}%
\pgfpathlineto{\pgfqpoint{4.529116in}{4.523188in}}%
\pgfpathlineto{\pgfqpoint{4.564304in}{4.546645in}}%
\pgfpathlineto{\pgfqpoint{4.599492in}{4.574431in}}%
\pgfpathlineto{\pgfqpoint{4.652273in}{4.622022in}}%
\pgfpathlineto{\pgfqpoint{4.793024in}{4.756298in}}%
\pgfpathlineto{\pgfqpoint{4.828212in}{4.785291in}}%
\pgfpathlineto{\pgfqpoint{4.863399in}{4.810391in}}%
\pgfpathlineto{\pgfqpoint{4.898587in}{4.830788in}}%
\pgfpathlineto{\pgfqpoint{4.933775in}{4.845855in}}%
\pgfpathlineto{\pgfqpoint{4.968962in}{4.855163in}}%
\pgfpathlineto{\pgfqpoint{5.004150in}{4.858503in}}%
\pgfpathlineto{\pgfqpoint{5.039338in}{4.855890in}}%
\pgfpathlineto{\pgfqpoint{5.074526in}{4.847567in}}%
\pgfpathlineto{\pgfqpoint{5.109713in}{4.833997in}}%
\pgfpathlineto{\pgfqpoint{5.144901in}{4.815845in}}%
\pgfpathlineto{\pgfqpoint{5.180089in}{4.793959in}}%
\pgfpathlineto{\pgfqpoint{5.232870in}{4.756357in}}%
\pgfpathlineto{\pgfqpoint{5.338433in}{4.678458in}}%
\pgfpathlineto{\pgfqpoint{5.373621in}{4.656181in}}%
\pgfpathlineto{\pgfqpoint{5.408809in}{4.637694in}}%
\pgfpathlineto{\pgfqpoint{5.443996in}{4.624018in}}%
\pgfpathlineto{\pgfqpoint{5.479184in}{4.616015in}}%
\pgfpathlineto{\pgfqpoint{5.496778in}{4.614359in}}%
\pgfpathlineto{\pgfqpoint{5.514372in}{4.614356in}}%
\pgfpathlineto{\pgfqpoint{5.531966in}{4.616056in}}%
\pgfpathlineto{\pgfqpoint{5.549560in}{4.619494in}}%
\pgfpathlineto{\pgfqpoint{5.567153in}{4.624686in}}%
\pgfpathlineto{\pgfqpoint{5.584747in}{4.631637in}}%
\pgfpathlineto{\pgfqpoint{5.619935in}{4.650738in}}%
\pgfpathlineto{\pgfqpoint{5.655123in}{4.676489in}}%
\pgfpathlineto{\pgfqpoint{5.690310in}{4.708319in}}%
\pgfpathlineto{\pgfqpoint{5.725498in}{4.745411in}}%
\pgfpathlineto{\pgfqpoint{5.760686in}{4.786716in}}%
\pgfpathlineto{\pgfqpoint{5.813467in}{4.853798in}}%
\pgfpathlineto{\pgfqpoint{5.901437in}{4.966852in}}%
\pgfpathlineto{\pgfqpoint{5.936624in}{5.007802in}}%
\pgfpathlineto{\pgfqpoint{5.971812in}{5.043857in}}%
\pgfpathlineto{\pgfqpoint{6.007000in}{5.073459in}}%
\pgfpathlineto{\pgfqpoint{6.024594in}{5.085385in}}%
\pgfpathlineto{\pgfqpoint{6.042187in}{5.095172in}}%
\pgfpathlineto{\pgfqpoint{6.059781in}{5.102668in}}%
\pgfpathlineto{\pgfqpoint{6.077375in}{5.107733in}}%
\pgfpathlineto{\pgfqpoint{6.094969in}{5.110245in}}%
\pgfpathlineto{\pgfqpoint{6.112563in}{5.110094in}}%
\pgfpathlineto{\pgfqpoint{6.130157in}{5.107190in}}%
\pgfpathlineto{\pgfqpoint{6.147751in}{5.101459in}}%
\pgfpathlineto{\pgfqpoint{6.165344in}{5.092846in}}%
\pgfpathlineto{\pgfqpoint{6.182938in}{5.081315in}}%
\pgfpathlineto{\pgfqpoint{6.200532in}{5.066849in}}%
\pgfpathlineto{\pgfqpoint{6.218126in}{5.049453in}}%
\pgfpathlineto{\pgfqpoint{6.235720in}{5.029151in}}%
\pgfpathlineto{\pgfqpoint{6.253314in}{5.005987in}}%
\pgfpathlineto{\pgfqpoint{6.288501in}{4.951350in}}%
\pgfpathlineto{\pgfqpoint{6.323689in}{4.886298in}}%
\pgfpathlineto{\pgfqpoint{6.358877in}{4.811885in}}%
\pgfpathlineto{\pgfqpoint{6.376471in}{4.771576in}}%
\pgfpathlineto{\pgfqpoint{6.376471in}{4.771576in}}%
\pgfusepath{stroke}%
\end{pgfscope}%
\begin{pgfscope}%
\pgfsetrectcap%
\pgfsetmiterjoin%
\pgfsetlinewidth{0.803000pt}%
\definecolor{currentstroke}{rgb}{0.000000,0.000000,0.000000}%
\pgfsetstrokecolor{currentstroke}%
\pgfsetdash{}{0pt}%
\pgfpathmoveto{\pgfqpoint{2.000000in}{4.326316in}}%
\pgfpathlineto{\pgfqpoint{2.000000in}{5.280000in}}%
\pgfusepath{stroke}%
\end{pgfscope}%
\begin{pgfscope}%
\pgfsetrectcap%
\pgfsetmiterjoin%
\pgfsetlinewidth{0.803000pt}%
\definecolor{currentstroke}{rgb}{0.000000,0.000000,0.000000}%
\pgfsetstrokecolor{currentstroke}%
\pgfsetdash{}{0pt}%
\pgfpathmoveto{\pgfqpoint{6.376471in}{4.326316in}}%
\pgfpathlineto{\pgfqpoint{6.376471in}{5.280000in}}%
\pgfusepath{stroke}%
\end{pgfscope}%
\begin{pgfscope}%
\pgfsetrectcap%
\pgfsetmiterjoin%
\pgfsetlinewidth{0.803000pt}%
\definecolor{currentstroke}{rgb}{0.000000,0.000000,0.000000}%
\pgfsetstrokecolor{currentstroke}%
\pgfsetdash{}{0pt}%
\pgfpathmoveto{\pgfqpoint{2.000000in}{4.326316in}}%
\pgfpathlineto{\pgfqpoint{6.376471in}{4.326316in}}%
\pgfusepath{stroke}%
\end{pgfscope}%
\begin{pgfscope}%
\pgfsetrectcap%
\pgfsetmiterjoin%
\pgfsetlinewidth{0.803000pt}%
\definecolor{currentstroke}{rgb}{0.000000,0.000000,0.000000}%
\pgfsetstrokecolor{currentstroke}%
\pgfsetdash{}{0pt}%
\pgfpathmoveto{\pgfqpoint{2.000000in}{5.280000in}}%
\pgfpathlineto{\pgfqpoint{6.376471in}{5.280000in}}%
\pgfusepath{stroke}%
\end{pgfscope}%
\begin{pgfscope}%
\pgfsetbuttcap%
\pgfsetmiterjoin%
\definecolor{currentfill}{rgb}{1.000000,1.000000,1.000000}%
\pgfsetfillcolor{currentfill}%
\pgfsetfillopacity{0.800000}%
\pgfsetlinewidth{1.003750pt}%
\definecolor{currentstroke}{rgb}{0.800000,0.800000,0.800000}%
\pgfsetstrokecolor{currentstroke}%
\pgfsetstrokeopacity{0.800000}%
\pgfsetdash{}{0pt}%
\pgfpathmoveto{\pgfqpoint{2.097222in}{4.395760in}}%
\pgfpathlineto{\pgfqpoint{2.796183in}{4.395760in}}%
\pgfpathquadraticcurveto{\pgfqpoint{2.823961in}{4.395760in}}{\pgfqpoint{2.823961in}{4.423538in}}%
\pgfpathlineto{\pgfqpoint{2.823961in}{5.212112in}}%
\pgfpathquadraticcurveto{\pgfqpoint{2.823961in}{5.239890in}}{\pgfqpoint{2.796183in}{5.239890in}}%
\pgfpathlineto{\pgfqpoint{2.097222in}{5.239890in}}%
\pgfpathquadraticcurveto{\pgfqpoint{2.069444in}{5.239890in}}{\pgfqpoint{2.069444in}{5.212112in}}%
\pgfpathlineto{\pgfqpoint{2.069444in}{4.423538in}}%
\pgfpathquadraticcurveto{\pgfqpoint{2.069444in}{4.395760in}}{\pgfqpoint{2.097222in}{4.395760in}}%
\pgfpathclose%
\pgfusepath{stroke,fill}%
\end{pgfscope}%
\begin{pgfscope}%
\pgfsetrectcap%
\pgfsetroundjoin%
\pgfsetlinewidth{1.505625pt}%
\definecolor{currentstroke}{rgb}{0.121569,0.466667,0.705882}%
\pgfsetstrokecolor{currentstroke}%
\pgfsetdash{}{0pt}%
\pgfpathmoveto{\pgfqpoint{2.125000in}{5.127232in}}%
\pgfpathlineto{\pgfqpoint{2.402778in}{5.127232in}}%
\pgfusepath{stroke}%
\end{pgfscope}%
\begin{pgfscope}%
\pgftext[x=2.513889in,y=5.078621in,left,base]{\rmfamily\fontsize{10.000000}{12.000000}\selectfont \(\displaystyle \widetilde{\Phi}^* \theta\)}%
\end{pgfscope}%
\begin{pgfscope}%
\pgfsetbuttcap%
\pgfsetroundjoin%
\pgfsetlinewidth{1.505625pt}%
\definecolor{currentstroke}{rgb}{1.000000,0.498039,0.054902}%
\pgfsetstrokecolor{currentstroke}%
\pgfsetdash{{9.600000pt}{2.400000pt}{1.600000pt}{2.400000pt}}{0.000000pt}%
\pgfpathmoveto{\pgfqpoint{2.125000in}{4.922371in}}%
\pgfpathlineto{\pgfqpoint{2.402778in}{4.922371in}}%
\pgfusepath{stroke}%
\end{pgfscope}%
\begin{pgfscope}%
\pgftext[x=2.513889in,y=4.873760in,left,base]{\rmfamily\fontsize{10.000000}{12.000000}\selectfont \(\displaystyle \widetilde{K}u\)}%
\end{pgfscope}%
\begin{pgfscope}%
\pgfsetbuttcap%
\pgfsetroundjoin%
\definecolor{currentfill}{rgb}{1.000000,0.000000,0.000000}%
\pgfsetfillcolor{currentfill}%
\pgfsetlinewidth{2.007500pt}%
\definecolor{currentstroke}{rgb}{1.000000,0.000000,0.000000}%
\pgfsetstrokecolor{currentstroke}%
\pgfsetdash{}{0pt}%
\pgfpathmoveto{\pgfqpoint{2.222222in}{4.713848in}}%
\pgfpathlineto{\pgfqpoint{2.305556in}{4.713848in}}%
\pgfpathmoveto{\pgfqpoint{2.263889in}{4.672181in}}%
\pgfpathlineto{\pgfqpoint{2.263889in}{4.755515in}}%
\pgfusepath{stroke,fill}%
\end{pgfscope}%
\begin{pgfscope}%
\pgftext[x=2.513889in,y=4.677390in,left,base]{\rmfamily\fontsize{10.000000}{12.000000}\selectfont train}%
\end{pgfscope}%
\begin{pgfscope}%
\pgfsetbuttcap%
\pgfsetroundjoin%
\definecolor{currentfill}{rgb}{0.000000,0.000000,0.000000}%
\pgfsetfillcolor{currentfill}%
\pgfsetlinewidth{1.003750pt}%
\definecolor{currentstroke}{rgb}{0.000000,0.000000,0.000000}%
\pgfsetstrokecolor{currentstroke}%
\pgfsetdash{}{0pt}%
\pgfsys@defobject{currentmarker}{\pgfqpoint{-0.020833in}{-0.020833in}}{\pgfqpoint{0.020833in}{0.020833in}}{%
\pgfpathmoveto{\pgfqpoint{0.000000in}{-0.020833in}}%
\pgfpathcurveto{\pgfqpoint{0.005525in}{-0.020833in}}{\pgfqpoint{0.010825in}{-0.018638in}}{\pgfqpoint{0.014731in}{-0.014731in}}%
\pgfpathcurveto{\pgfqpoint{0.018638in}{-0.010825in}}{\pgfqpoint{0.020833in}{-0.005525in}}{\pgfqpoint{0.020833in}{0.000000in}}%
\pgfpathcurveto{\pgfqpoint{0.020833in}{0.005525in}}{\pgfqpoint{0.018638in}{0.010825in}}{\pgfqpoint{0.014731in}{0.014731in}}%
\pgfpathcurveto{\pgfqpoint{0.010825in}{0.018638in}}{\pgfqpoint{0.005525in}{0.020833in}}{\pgfqpoint{0.000000in}{0.020833in}}%
\pgfpathcurveto{\pgfqpoint{-0.005525in}{0.020833in}}{\pgfqpoint{-0.010825in}{0.018638in}}{\pgfqpoint{-0.014731in}{0.014731in}}%
\pgfpathcurveto{\pgfqpoint{-0.018638in}{0.010825in}}{\pgfqpoint{-0.020833in}{0.005525in}}{\pgfqpoint{-0.020833in}{0.000000in}}%
\pgfpathcurveto{\pgfqpoint{-0.020833in}{-0.005525in}}{\pgfqpoint{-0.018638in}{-0.010825in}}{\pgfqpoint{-0.014731in}{-0.014731in}}%
\pgfpathcurveto{\pgfqpoint{-0.010825in}{-0.018638in}}{\pgfqpoint{-0.005525in}{-0.020833in}}{\pgfqpoint{0.000000in}{-0.020833in}}%
\pgfpathclose%
\pgfusepath{stroke,fill}%
}%
\begin{pgfscope}%
\pgfsys@transformshift{2.263889in}{4.517478in}%
\pgfsys@useobject{currentmarker}{}%
\end{pgfscope}%
\end{pgfscope}%
\begin{pgfscope}%
\pgftext[x=2.513889in,y=4.481019in,left,base]{\rmfamily\fontsize{10.000000}{12.000000}\selectfont test}%
\end{pgfscope}%
\begin{pgfscope}%
\pgfsetbuttcap%
\pgfsetmiterjoin%
\definecolor{currentfill}{rgb}{1.000000,1.000000,1.000000}%
\pgfsetfillcolor{currentfill}%
\pgfsetlinewidth{0.000000pt}%
\definecolor{currentstroke}{rgb}{0.000000,0.000000,0.000000}%
\pgfsetstrokecolor{currentstroke}%
\pgfsetstrokeopacity{0.000000}%
\pgfsetdash{}{0pt}%
\pgfpathmoveto{\pgfqpoint{7.105882in}{4.326316in}}%
\pgfpathlineto{\pgfqpoint{11.482353in}{4.326316in}}%
\pgfpathlineto{\pgfqpoint{11.482353in}{5.280000in}}%
\pgfpathlineto{\pgfqpoint{7.105882in}{5.280000in}}%
\pgfpathclose%
\pgfusepath{fill}%
\end{pgfscope}%
\begin{pgfscope}%
\pgfpathrectangle{\pgfqpoint{7.105882in}{4.326316in}}{\pgfqpoint{4.376471in}{0.953684in}} %
\pgfusepath{clip}%
\pgfsetbuttcap%
\pgfsetroundjoin%
\definecolor{currentfill}{rgb}{1.000000,0.000000,0.000000}%
\pgfsetfillcolor{currentfill}%
\pgfsetlinewidth{2.007500pt}%
\definecolor{currentstroke}{rgb}{1.000000,0.000000,0.000000}%
\pgfsetstrokecolor{currentstroke}%
\pgfsetdash{}{0pt}%
\pgfpathmoveto{\pgfqpoint{9.861003in}{4.682823in}}%
\pgfpathlineto{\pgfqpoint{9.944336in}{4.682823in}}%
\pgfpathmoveto{\pgfqpoint{9.902669in}{4.641156in}}%
\pgfpathlineto{\pgfqpoint{9.902669in}{4.724490in}}%
\pgfusepath{stroke,fill}%
\end{pgfscope}%
\begin{pgfscope}%
\pgfpathrectangle{\pgfqpoint{7.105882in}{4.326316in}}{\pgfqpoint{4.376471in}{0.953684in}} %
\pgfusepath{clip}%
\pgfsetbuttcap%
\pgfsetroundjoin%
\definecolor{currentfill}{rgb}{1.000000,0.000000,0.000000}%
\pgfsetfillcolor{currentfill}%
\pgfsetlinewidth{2.007500pt}%
\definecolor{currentstroke}{rgb}{1.000000,0.000000,0.000000}%
\pgfsetstrokecolor{currentstroke}%
\pgfsetdash{}{0pt}%
\pgfpathmoveto{\pgfqpoint{10.443514in}{4.769016in}}%
\pgfpathlineto{\pgfqpoint{10.526847in}{4.769016in}}%
\pgfpathmoveto{\pgfqpoint{10.485181in}{4.727350in}}%
\pgfpathlineto{\pgfqpoint{10.485181in}{4.810683in}}%
\pgfusepath{stroke,fill}%
\end{pgfscope}%
\begin{pgfscope}%
\pgfpathrectangle{\pgfqpoint{7.105882in}{4.326316in}}{\pgfqpoint{4.376471in}{0.953684in}} %
\pgfusepath{clip}%
\pgfsetbuttcap%
\pgfsetroundjoin%
\definecolor{currentfill}{rgb}{1.000000,0.000000,0.000000}%
\pgfsetfillcolor{currentfill}%
\pgfsetlinewidth{2.007500pt}%
\definecolor{currentstroke}{rgb}{1.000000,0.000000,0.000000}%
\pgfsetstrokecolor{currentstroke}%
\pgfsetdash{}{0pt}%
\pgfpathmoveto{\pgfqpoint{10.049891in}{4.825923in}}%
\pgfpathlineto{\pgfqpoint{10.133224in}{4.825923in}}%
\pgfpathmoveto{\pgfqpoint{10.091557in}{4.784256in}}%
\pgfpathlineto{\pgfqpoint{10.091557in}{4.867589in}}%
\pgfusepath{stroke,fill}%
\end{pgfscope}%
\begin{pgfscope}%
\pgfpathrectangle{\pgfqpoint{7.105882in}{4.326316in}}{\pgfqpoint{4.376471in}{0.953684in}} %
\pgfusepath{clip}%
\pgfsetbuttcap%
\pgfsetroundjoin%
\definecolor{currentfill}{rgb}{1.000000,0.000000,0.000000}%
\pgfsetfillcolor{currentfill}%
\pgfsetlinewidth{2.007500pt}%
\definecolor{currentstroke}{rgb}{1.000000,0.000000,0.000000}%
\pgfsetstrokecolor{currentstroke}%
\pgfsetdash{}{0pt}%
\pgfpathmoveto{\pgfqpoint{9.847242in}{4.798976in}}%
\pgfpathlineto{\pgfqpoint{9.930575in}{4.798976in}}%
\pgfpathmoveto{\pgfqpoint{9.888909in}{4.757309in}}%
\pgfpathlineto{\pgfqpoint{9.888909in}{4.840643in}}%
\pgfusepath{stroke,fill}%
\end{pgfscope}%
\begin{pgfscope}%
\pgfpathrectangle{\pgfqpoint{7.105882in}{4.326316in}}{\pgfqpoint{4.376471in}{0.953684in}} %
\pgfusepath{clip}%
\pgfsetbuttcap%
\pgfsetroundjoin%
\definecolor{currentfill}{rgb}{1.000000,0.000000,0.000000}%
\pgfsetfillcolor{currentfill}%
\pgfsetlinewidth{2.007500pt}%
\definecolor{currentstroke}{rgb}{1.000000,0.000000,0.000000}%
\pgfsetstrokecolor{currentstroke}%
\pgfsetdash{}{0pt}%
\pgfpathmoveto{\pgfqpoint{9.422800in}{4.370264in}}%
\pgfpathlineto{\pgfqpoint{9.506133in}{4.370264in}}%
\pgfpathmoveto{\pgfqpoint{9.464467in}{4.328597in}}%
\pgfpathlineto{\pgfqpoint{9.464467in}{4.411930in}}%
\pgfusepath{stroke,fill}%
\end{pgfscope}%
\begin{pgfscope}%
\pgfpathrectangle{\pgfqpoint{7.105882in}{4.326316in}}{\pgfqpoint{4.376471in}{0.953684in}} %
\pgfusepath{clip}%
\pgfsetbuttcap%
\pgfsetroundjoin%
\definecolor{currentfill}{rgb}{1.000000,0.000000,0.000000}%
\pgfsetfillcolor{currentfill}%
\pgfsetlinewidth{2.007500pt}%
\definecolor{currentstroke}{rgb}{1.000000,0.000000,0.000000}%
\pgfsetstrokecolor{currentstroke}%
\pgfsetdash{}{0pt}%
\pgfpathmoveto{\pgfqpoint{10.200899in}{4.839095in}}%
\pgfpathlineto{\pgfqpoint{10.284232in}{4.839095in}}%
\pgfpathmoveto{\pgfqpoint{10.242566in}{4.797428in}}%
\pgfpathlineto{\pgfqpoint{10.242566in}{4.880762in}}%
\pgfusepath{stroke,fill}%
\end{pgfscope}%
\begin{pgfscope}%
\pgfpathrectangle{\pgfqpoint{7.105882in}{4.326316in}}{\pgfqpoint{4.376471in}{0.953684in}} %
\pgfusepath{clip}%
\pgfsetbuttcap%
\pgfsetroundjoin%
\definecolor{currentfill}{rgb}{1.000000,0.000000,0.000000}%
\pgfsetfillcolor{currentfill}%
\pgfsetlinewidth{2.007500pt}%
\definecolor{currentstroke}{rgb}{1.000000,0.000000,0.000000}%
\pgfsetstrokecolor{currentstroke}%
\pgfsetdash{}{0pt}%
\pgfpathmoveto{\pgfqpoint{9.471580in}{4.341372in}}%
\pgfpathlineto{\pgfqpoint{9.554913in}{4.341372in}}%
\pgfpathmoveto{\pgfqpoint{9.513247in}{4.299706in}}%
\pgfpathlineto{\pgfqpoint{9.513247in}{4.383039in}}%
\pgfusepath{stroke,fill}%
\end{pgfscope}%
\begin{pgfscope}%
\pgfpathrectangle{\pgfqpoint{7.105882in}{4.326316in}}{\pgfqpoint{4.376471in}{0.953684in}} %
\pgfusepath{clip}%
\pgfsetbuttcap%
\pgfsetroundjoin%
\definecolor{currentfill}{rgb}{1.000000,0.000000,0.000000}%
\pgfsetfillcolor{currentfill}%
\pgfsetlinewidth{2.007500pt}%
\definecolor{currentstroke}{rgb}{1.000000,0.000000,0.000000}%
\pgfsetstrokecolor{currentstroke}%
\pgfsetdash{}{0pt}%
\pgfpathmoveto{\pgfqpoint{11.061764in}{5.105426in}}%
\pgfpathlineto{\pgfqpoint{11.145098in}{5.105426in}}%
\pgfpathmoveto{\pgfqpoint{11.103431in}{5.063759in}}%
\pgfpathlineto{\pgfqpoint{11.103431in}{5.147092in}}%
\pgfusepath{stroke,fill}%
\end{pgfscope}%
\begin{pgfscope}%
\pgfpathrectangle{\pgfqpoint{7.105882in}{4.326316in}}{\pgfqpoint{4.376471in}{0.953684in}} %
\pgfusepath{clip}%
\pgfsetbuttcap%
\pgfsetroundjoin%
\definecolor{currentfill}{rgb}{1.000000,0.000000,0.000000}%
\pgfsetfillcolor{currentfill}%
\pgfsetlinewidth{2.007500pt}%
\definecolor{currentstroke}{rgb}{1.000000,0.000000,0.000000}%
\pgfsetstrokecolor{currentstroke}%
\pgfsetdash{}{0pt}%
\pgfpathmoveto{\pgfqpoint{11.313463in}{4.966690in}}%
\pgfpathlineto{\pgfqpoint{11.396797in}{4.966690in}}%
\pgfpathmoveto{\pgfqpoint{11.355130in}{4.925023in}}%
\pgfpathlineto{\pgfqpoint{11.355130in}{5.008357in}}%
\pgfusepath{stroke,fill}%
\end{pgfscope}%
\begin{pgfscope}%
\pgfpathrectangle{\pgfqpoint{7.105882in}{4.326316in}}{\pgfqpoint{4.376471in}{0.953684in}} %
\pgfusepath{clip}%
\pgfsetbuttcap%
\pgfsetroundjoin%
\definecolor{currentfill}{rgb}{1.000000,0.000000,0.000000}%
\pgfsetfillcolor{currentfill}%
\pgfsetlinewidth{2.007500pt}%
\definecolor{currentstroke}{rgb}{1.000000,0.000000,0.000000}%
\pgfsetstrokecolor{currentstroke}%
\pgfsetdash{}{0pt}%
\pgfpathmoveto{\pgfqpoint{9.282006in}{4.600015in}}%
\pgfpathlineto{\pgfqpoint{9.365340in}{4.600015in}}%
\pgfpathmoveto{\pgfqpoint{9.323673in}{4.558348in}}%
\pgfpathlineto{\pgfqpoint{9.323673in}{4.641682in}}%
\pgfusepath{stroke,fill}%
\end{pgfscope}%
\begin{pgfscope}%
\pgfpathrectangle{\pgfqpoint{7.105882in}{4.326316in}}{\pgfqpoint{4.376471in}{0.953684in}} %
\pgfusepath{clip}%
\pgfsetbuttcap%
\pgfsetroundjoin%
\definecolor{currentfill}{rgb}{1.000000,0.000000,0.000000}%
\pgfsetfillcolor{currentfill}%
\pgfsetlinewidth{2.007500pt}%
\definecolor{currentstroke}{rgb}{1.000000,0.000000,0.000000}%
\pgfsetstrokecolor{currentstroke}%
\pgfsetdash{}{0pt}%
\pgfpathmoveto{\pgfqpoint{10.711479in}{4.682287in}}%
\pgfpathlineto{\pgfqpoint{10.794812in}{4.682287in}}%
\pgfpathmoveto{\pgfqpoint{10.753146in}{4.640621in}}%
\pgfpathlineto{\pgfqpoint{10.753146in}{4.723954in}}%
\pgfusepath{stroke,fill}%
\end{pgfscope}%
\begin{pgfscope}%
\pgfpathrectangle{\pgfqpoint{7.105882in}{4.326316in}}{\pgfqpoint{4.376471in}{0.953684in}} %
\pgfusepath{clip}%
\pgfsetbuttcap%
\pgfsetroundjoin%
\definecolor{currentfill}{rgb}{1.000000,0.000000,0.000000}%
\pgfsetfillcolor{currentfill}%
\pgfsetlinewidth{2.007500pt}%
\definecolor{currentstroke}{rgb}{1.000000,0.000000,0.000000}%
\pgfsetstrokecolor{currentstroke}%
\pgfsetdash{}{0pt}%
\pgfpathmoveto{\pgfqpoint{9.791264in}{4.676522in}}%
\pgfpathlineto{\pgfqpoint{9.874598in}{4.676522in}}%
\pgfpathmoveto{\pgfqpoint{9.832931in}{4.634855in}}%
\pgfpathlineto{\pgfqpoint{9.832931in}{4.718189in}}%
\pgfusepath{stroke,fill}%
\end{pgfscope}%
\begin{pgfscope}%
\pgfpathrectangle{\pgfqpoint{7.105882in}{4.326316in}}{\pgfqpoint{4.376471in}{0.953684in}} %
\pgfusepath{clip}%
\pgfsetbuttcap%
\pgfsetroundjoin%
\definecolor{currentfill}{rgb}{1.000000,0.000000,0.000000}%
\pgfsetfillcolor{currentfill}%
\pgfsetlinewidth{2.007500pt}%
\definecolor{currentstroke}{rgb}{1.000000,0.000000,0.000000}%
\pgfsetstrokecolor{currentstroke}%
\pgfsetdash{}{0pt}%
\pgfpathmoveto{\pgfqpoint{9.928334in}{4.882830in}}%
\pgfpathlineto{\pgfqpoint{10.011667in}{4.882830in}}%
\pgfpathmoveto{\pgfqpoint{9.970001in}{4.841164in}}%
\pgfpathlineto{\pgfqpoint{9.970001in}{4.924497in}}%
\pgfusepath{stroke,fill}%
\end{pgfscope}%
\begin{pgfscope}%
\pgfpathrectangle{\pgfqpoint{7.105882in}{4.326316in}}{\pgfqpoint{4.376471in}{0.953684in}} %
\pgfusepath{clip}%
\pgfsetbuttcap%
\pgfsetroundjoin%
\definecolor{currentfill}{rgb}{1.000000,0.000000,0.000000}%
\pgfsetfillcolor{currentfill}%
\pgfsetlinewidth{2.007500pt}%
\definecolor{currentstroke}{rgb}{1.000000,0.000000,0.000000}%
\pgfsetstrokecolor{currentstroke}%
\pgfsetdash{}{0pt}%
\pgfpathmoveto{\pgfqpoint{11.180187in}{5.171612in}}%
\pgfpathlineto{\pgfqpoint{11.263520in}{5.171612in}}%
\pgfpathmoveto{\pgfqpoint{11.221854in}{5.129946in}}%
\pgfpathlineto{\pgfqpoint{11.221854in}{5.213279in}}%
\pgfusepath{stroke,fill}%
\end{pgfscope}%
\begin{pgfscope}%
\pgfpathrectangle{\pgfqpoint{7.105882in}{4.326316in}}{\pgfqpoint{4.376471in}{0.953684in}} %
\pgfusepath{clip}%
\pgfsetbuttcap%
\pgfsetroundjoin%
\definecolor{currentfill}{rgb}{1.000000,0.000000,0.000000}%
\pgfsetfillcolor{currentfill}%
\pgfsetlinewidth{2.007500pt}%
\definecolor{currentstroke}{rgb}{1.000000,0.000000,0.000000}%
\pgfsetstrokecolor{currentstroke}%
\pgfsetdash{}{0pt}%
\pgfpathmoveto{\pgfqpoint{8.188220in}{4.834359in}}%
\pgfpathlineto{\pgfqpoint{8.271553in}{4.834359in}}%
\pgfpathmoveto{\pgfqpoint{8.229886in}{4.792692in}}%
\pgfpathlineto{\pgfqpoint{8.229886in}{4.876025in}}%
\pgfusepath{stroke,fill}%
\end{pgfscope}%
\begin{pgfscope}%
\pgfpathrectangle{\pgfqpoint{7.105882in}{4.326316in}}{\pgfqpoint{4.376471in}{0.953684in}} %
\pgfusepath{clip}%
\pgfsetbuttcap%
\pgfsetroundjoin%
\definecolor{currentfill}{rgb}{1.000000,0.000000,0.000000}%
\pgfsetfillcolor{currentfill}%
\pgfsetlinewidth{2.007500pt}%
\definecolor{currentstroke}{rgb}{1.000000,0.000000,0.000000}%
\pgfsetstrokecolor{currentstroke}%
\pgfsetdash{}{0pt}%
\pgfpathmoveto{\pgfqpoint{8.244565in}{4.802207in}}%
\pgfpathlineto{\pgfqpoint{8.327898in}{4.802207in}}%
\pgfpathmoveto{\pgfqpoint{8.286232in}{4.760540in}}%
\pgfpathlineto{\pgfqpoint{8.286232in}{4.843874in}}%
\pgfusepath{stroke,fill}%
\end{pgfscope}%
\begin{pgfscope}%
\pgfpathrectangle{\pgfqpoint{7.105882in}{4.326316in}}{\pgfqpoint{4.376471in}{0.953684in}} %
\pgfusepath{clip}%
\pgfsetbuttcap%
\pgfsetroundjoin%
\definecolor{currentfill}{rgb}{1.000000,0.000000,0.000000}%
\pgfsetfillcolor{currentfill}%
\pgfsetlinewidth{2.007500pt}%
\definecolor{currentstroke}{rgb}{1.000000,0.000000,0.000000}%
\pgfsetstrokecolor{currentstroke}%
\pgfsetdash{}{0pt}%
\pgfpathmoveto{\pgfqpoint{8.010298in}{5.020595in}}%
\pgfpathlineto{\pgfqpoint{8.093631in}{5.020595in}}%
\pgfpathmoveto{\pgfqpoint{8.051965in}{4.978928in}}%
\pgfpathlineto{\pgfqpoint{8.051965in}{5.062262in}}%
\pgfusepath{stroke,fill}%
\end{pgfscope}%
\begin{pgfscope}%
\pgfpathrectangle{\pgfqpoint{7.105882in}{4.326316in}}{\pgfqpoint{4.376471in}{0.953684in}} %
\pgfusepath{clip}%
\pgfsetbuttcap%
\pgfsetroundjoin%
\definecolor{currentfill}{rgb}{1.000000,0.000000,0.000000}%
\pgfsetfillcolor{currentfill}%
\pgfsetlinewidth{2.007500pt}%
\definecolor{currentstroke}{rgb}{1.000000,0.000000,0.000000}%
\pgfsetstrokecolor{currentstroke}%
\pgfsetdash{}{0pt}%
\pgfpathmoveto{\pgfqpoint{10.854659in}{4.895466in}}%
\pgfpathlineto{\pgfqpoint{10.937992in}{4.895466in}}%
\pgfpathmoveto{\pgfqpoint{10.896325in}{4.853799in}}%
\pgfpathlineto{\pgfqpoint{10.896325in}{4.937133in}}%
\pgfusepath{stroke,fill}%
\end{pgfscope}%
\begin{pgfscope}%
\pgfpathrectangle{\pgfqpoint{7.105882in}{4.326316in}}{\pgfqpoint{4.376471in}{0.953684in}} %
\pgfusepath{clip}%
\pgfsetbuttcap%
\pgfsetroundjoin%
\definecolor{currentfill}{rgb}{1.000000,0.000000,0.000000}%
\pgfsetfillcolor{currentfill}%
\pgfsetlinewidth{2.007500pt}%
\definecolor{currentstroke}{rgb}{1.000000,0.000000,0.000000}%
\pgfsetstrokecolor{currentstroke}%
\pgfsetdash{}{0pt}%
\pgfpathmoveto{\pgfqpoint{10.663974in}{4.609106in}}%
\pgfpathlineto{\pgfqpoint{10.747307in}{4.609106in}}%
\pgfpathmoveto{\pgfqpoint{10.705641in}{4.567440in}}%
\pgfpathlineto{\pgfqpoint{10.705641in}{4.650773in}}%
\pgfusepath{stroke,fill}%
\end{pgfscope}%
\begin{pgfscope}%
\pgfpathrectangle{\pgfqpoint{7.105882in}{4.326316in}}{\pgfqpoint{4.376471in}{0.953684in}} %
\pgfusepath{clip}%
\pgfsetbuttcap%
\pgfsetroundjoin%
\definecolor{currentfill}{rgb}{1.000000,0.000000,0.000000}%
\pgfsetfillcolor{currentfill}%
\pgfsetlinewidth{2.007500pt}%
\definecolor{currentstroke}{rgb}{1.000000,0.000000,0.000000}%
\pgfsetstrokecolor{currentstroke}%
\pgfsetdash{}{0pt}%
\pgfpathmoveto{\pgfqpoint{10.985576in}{5.044422in}}%
\pgfpathlineto{\pgfqpoint{11.068909in}{5.044422in}}%
\pgfpathmoveto{\pgfqpoint{11.027243in}{5.002755in}}%
\pgfpathlineto{\pgfqpoint{11.027243in}{5.086089in}}%
\pgfusepath{stroke,fill}%
\end{pgfscope}%
\begin{pgfscope}%
\pgfpathrectangle{\pgfqpoint{7.105882in}{4.326316in}}{\pgfqpoint{4.376471in}{0.953684in}} %
\pgfusepath{clip}%
\pgfsetbuttcap%
\pgfsetroundjoin%
\definecolor{currentfill}{rgb}{1.000000,0.000000,0.000000}%
\pgfsetfillcolor{currentfill}%
\pgfsetlinewidth{2.007500pt}%
\definecolor{currentstroke}{rgb}{1.000000,0.000000,0.000000}%
\pgfsetstrokecolor{currentstroke}%
\pgfsetdash{}{0pt}%
\pgfpathmoveto{\pgfqpoint{11.365825in}{4.938060in}}%
\pgfpathlineto{\pgfqpoint{11.449159in}{4.938060in}}%
\pgfpathmoveto{\pgfqpoint{11.407492in}{4.896393in}}%
\pgfpathlineto{\pgfqpoint{11.407492in}{4.979727in}}%
\pgfusepath{stroke,fill}%
\end{pgfscope}%
\begin{pgfscope}%
\pgfpathrectangle{\pgfqpoint{7.105882in}{4.326316in}}{\pgfqpoint{4.376471in}{0.953684in}} %
\pgfusepath{clip}%
\pgfsetbuttcap%
\pgfsetroundjoin%
\definecolor{currentfill}{rgb}{1.000000,0.000000,0.000000}%
\pgfsetfillcolor{currentfill}%
\pgfsetlinewidth{2.007500pt}%
\definecolor{currentstroke}{rgb}{1.000000,0.000000,0.000000}%
\pgfsetstrokecolor{currentstroke}%
\pgfsetdash{}{0pt}%
\pgfpathmoveto{\pgfqpoint{10.737505in}{4.595988in}}%
\pgfpathlineto{\pgfqpoint{10.820838in}{4.595988in}}%
\pgfpathmoveto{\pgfqpoint{10.779172in}{4.554322in}}%
\pgfpathlineto{\pgfqpoint{10.779172in}{4.637655in}}%
\pgfusepath{stroke,fill}%
\end{pgfscope}%
\begin{pgfscope}%
\pgfpathrectangle{\pgfqpoint{7.105882in}{4.326316in}}{\pgfqpoint{4.376471in}{0.953684in}} %
\pgfusepath{clip}%
\pgfsetbuttcap%
\pgfsetroundjoin%
\definecolor{currentfill}{rgb}{1.000000,0.000000,0.000000}%
\pgfsetfillcolor{currentfill}%
\pgfsetlinewidth{2.007500pt}%
\definecolor{currentstroke}{rgb}{1.000000,0.000000,0.000000}%
\pgfsetstrokecolor{currentstroke}%
\pgfsetdash{}{0pt}%
\pgfpathmoveto{\pgfqpoint{9.555230in}{4.565562in}}%
\pgfpathlineto{\pgfqpoint{9.638564in}{4.565562in}}%
\pgfpathmoveto{\pgfqpoint{9.596897in}{4.523895in}}%
\pgfpathlineto{\pgfqpoint{9.596897in}{4.607229in}}%
\pgfusepath{stroke,fill}%
\end{pgfscope}%
\begin{pgfscope}%
\pgfpathrectangle{\pgfqpoint{7.105882in}{4.326316in}}{\pgfqpoint{4.376471in}{0.953684in}} %
\pgfusepath{clip}%
\pgfsetbuttcap%
\pgfsetroundjoin%
\definecolor{currentfill}{rgb}{1.000000,0.000000,0.000000}%
\pgfsetfillcolor{currentfill}%
\pgfsetlinewidth{2.007500pt}%
\definecolor{currentstroke}{rgb}{1.000000,0.000000,0.000000}%
\pgfsetstrokecolor{currentstroke}%
\pgfsetdash{}{0pt}%
\pgfpathmoveto{\pgfqpoint{10.672280in}{4.645335in}}%
\pgfpathlineto{\pgfqpoint{10.755614in}{4.645335in}}%
\pgfpathmoveto{\pgfqpoint{10.713947in}{4.603668in}}%
\pgfpathlineto{\pgfqpoint{10.713947in}{4.687002in}}%
\pgfusepath{stroke,fill}%
\end{pgfscope}%
\begin{pgfscope}%
\pgfpathrectangle{\pgfqpoint{7.105882in}{4.326316in}}{\pgfqpoint{4.376471in}{0.953684in}} %
\pgfusepath{clip}%
\pgfsetbuttcap%
\pgfsetroundjoin%
\definecolor{currentfill}{rgb}{1.000000,0.000000,0.000000}%
\pgfsetfillcolor{currentfill}%
\pgfsetlinewidth{2.007500pt}%
\definecolor{currentstroke}{rgb}{1.000000,0.000000,0.000000}%
\pgfsetstrokecolor{currentstroke}%
\pgfsetdash{}{0pt}%
\pgfpathmoveto{\pgfqpoint{8.353609in}{4.559691in}}%
\pgfpathlineto{\pgfqpoint{8.436943in}{4.559691in}}%
\pgfpathmoveto{\pgfqpoint{8.395276in}{4.518024in}}%
\pgfpathlineto{\pgfqpoint{8.395276in}{4.601358in}}%
\pgfusepath{stroke,fill}%
\end{pgfscope}%
\begin{pgfscope}%
\pgfpathrectangle{\pgfqpoint{7.105882in}{4.326316in}}{\pgfqpoint{4.376471in}{0.953684in}} %
\pgfusepath{clip}%
\pgfsetbuttcap%
\pgfsetroundjoin%
\definecolor{currentfill}{rgb}{1.000000,0.000000,0.000000}%
\pgfsetfillcolor{currentfill}%
\pgfsetlinewidth{2.007500pt}%
\definecolor{currentstroke}{rgb}{1.000000,0.000000,0.000000}%
\pgfsetstrokecolor{currentstroke}%
\pgfsetdash{}{0pt}%
\pgfpathmoveto{\pgfqpoint{10.179986in}{4.827892in}}%
\pgfpathlineto{\pgfqpoint{10.263320in}{4.827892in}}%
\pgfpathmoveto{\pgfqpoint{10.221653in}{4.786225in}}%
\pgfpathlineto{\pgfqpoint{10.221653in}{4.869559in}}%
\pgfusepath{stroke,fill}%
\end{pgfscope}%
\begin{pgfscope}%
\pgfpathrectangle{\pgfqpoint{7.105882in}{4.326316in}}{\pgfqpoint{4.376471in}{0.953684in}} %
\pgfusepath{clip}%
\pgfsetbuttcap%
\pgfsetroundjoin%
\definecolor{currentfill}{rgb}{1.000000,0.000000,0.000000}%
\pgfsetfillcolor{currentfill}%
\pgfsetlinewidth{2.007500pt}%
\definecolor{currentstroke}{rgb}{1.000000,0.000000,0.000000}%
\pgfsetstrokecolor{currentstroke}%
\pgfsetdash{}{0pt}%
\pgfpathmoveto{\pgfqpoint{8.441415in}{4.646756in}}%
\pgfpathlineto{\pgfqpoint{8.524748in}{4.646756in}}%
\pgfpathmoveto{\pgfqpoint{8.483082in}{4.605089in}}%
\pgfpathlineto{\pgfqpoint{8.483082in}{4.688423in}}%
\pgfusepath{stroke,fill}%
\end{pgfscope}%
\begin{pgfscope}%
\pgfpathrectangle{\pgfqpoint{7.105882in}{4.326316in}}{\pgfqpoint{4.376471in}{0.953684in}} %
\pgfusepath{clip}%
\pgfsetbuttcap%
\pgfsetroundjoin%
\definecolor{currentfill}{rgb}{1.000000,0.000000,0.000000}%
\pgfsetfillcolor{currentfill}%
\pgfsetlinewidth{2.007500pt}%
\definecolor{currentstroke}{rgb}{1.000000,0.000000,0.000000}%
\pgfsetstrokecolor{currentstroke}%
\pgfsetdash{}{0pt}%
\pgfpathmoveto{\pgfqpoint{11.246962in}{5.104076in}}%
\pgfpathlineto{\pgfqpoint{11.330296in}{5.104076in}}%
\pgfpathmoveto{\pgfqpoint{11.288629in}{5.062410in}}%
\pgfpathlineto{\pgfqpoint{11.288629in}{5.145743in}}%
\pgfusepath{stroke,fill}%
\end{pgfscope}%
\begin{pgfscope}%
\pgfpathrectangle{\pgfqpoint{7.105882in}{4.326316in}}{\pgfqpoint{4.376471in}{0.953684in}} %
\pgfusepath{clip}%
\pgfsetbuttcap%
\pgfsetroundjoin%
\definecolor{currentfill}{rgb}{1.000000,0.000000,0.000000}%
\pgfsetfillcolor{currentfill}%
\pgfsetlinewidth{2.007500pt}%
\definecolor{currentstroke}{rgb}{1.000000,0.000000,0.000000}%
\pgfsetstrokecolor{currentstroke}%
\pgfsetdash{}{0pt}%
\pgfpathmoveto{\pgfqpoint{9.766593in}{4.841217in}}%
\pgfpathlineto{\pgfqpoint{9.849926in}{4.841217in}}%
\pgfpathmoveto{\pgfqpoint{9.808260in}{4.799551in}}%
\pgfpathlineto{\pgfqpoint{9.808260in}{4.882884in}}%
\pgfusepath{stroke,fill}%
\end{pgfscope}%
\begin{pgfscope}%
\pgfpathrectangle{\pgfqpoint{7.105882in}{4.326316in}}{\pgfqpoint{4.376471in}{0.953684in}} %
\pgfusepath{clip}%
\pgfsetbuttcap%
\pgfsetroundjoin%
\definecolor{currentfill}{rgb}{1.000000,0.000000,0.000000}%
\pgfsetfillcolor{currentfill}%
\pgfsetlinewidth{2.007500pt}%
\definecolor{currentstroke}{rgb}{1.000000,0.000000,0.000000}%
\pgfsetstrokecolor{currentstroke}%
\pgfsetdash{}{0pt}%
\pgfpathmoveto{\pgfqpoint{9.391314in}{4.511486in}}%
\pgfpathlineto{\pgfqpoint{9.474648in}{4.511486in}}%
\pgfpathmoveto{\pgfqpoint{9.432981in}{4.469819in}}%
\pgfpathlineto{\pgfqpoint{9.432981in}{4.553152in}}%
\pgfusepath{stroke,fill}%
\end{pgfscope}%
\begin{pgfscope}%
\pgfpathrectangle{\pgfqpoint{7.105882in}{4.326316in}}{\pgfqpoint{4.376471in}{0.953684in}} %
\pgfusepath{clip}%
\pgfsetbuttcap%
\pgfsetroundjoin%
\definecolor{currentfill}{rgb}{1.000000,0.000000,0.000000}%
\pgfsetfillcolor{currentfill}%
\pgfsetlinewidth{2.007500pt}%
\definecolor{currentstroke}{rgb}{1.000000,0.000000,0.000000}%
\pgfsetstrokecolor{currentstroke}%
\pgfsetdash{}{0pt}%
\pgfpathmoveto{\pgfqpoint{8.865766in}{5.083932in}}%
\pgfpathlineto{\pgfqpoint{8.949099in}{5.083932in}}%
\pgfpathmoveto{\pgfqpoint{8.907432in}{5.042265in}}%
\pgfpathlineto{\pgfqpoint{8.907432in}{5.125598in}}%
\pgfusepath{stroke,fill}%
\end{pgfscope}%
\begin{pgfscope}%
\pgfpathrectangle{\pgfqpoint{7.105882in}{4.326316in}}{\pgfqpoint{4.376471in}{0.953684in}} %
\pgfusepath{clip}%
\pgfsetbuttcap%
\pgfsetroundjoin%
\definecolor{currentfill}{rgb}{1.000000,0.000000,0.000000}%
\pgfsetfillcolor{currentfill}%
\pgfsetlinewidth{2.007500pt}%
\definecolor{currentstroke}{rgb}{1.000000,0.000000,0.000000}%
\pgfsetstrokecolor{currentstroke}%
\pgfsetdash{}{0pt}%
\pgfpathmoveto{\pgfqpoint{10.650239in}{4.538491in}}%
\pgfpathlineto{\pgfqpoint{10.733572in}{4.538491in}}%
\pgfpathmoveto{\pgfqpoint{10.691905in}{4.496824in}}%
\pgfpathlineto{\pgfqpoint{10.691905in}{4.580158in}}%
\pgfusepath{stroke,fill}%
\end{pgfscope}%
\begin{pgfscope}%
\pgfpathrectangle{\pgfqpoint{7.105882in}{4.326316in}}{\pgfqpoint{4.376471in}{0.953684in}} %
\pgfusepath{clip}%
\pgfsetbuttcap%
\pgfsetroundjoin%
\definecolor{currentfill}{rgb}{1.000000,0.000000,0.000000}%
\pgfsetfillcolor{currentfill}%
\pgfsetlinewidth{2.007500pt}%
\definecolor{currentstroke}{rgb}{1.000000,0.000000,0.000000}%
\pgfsetstrokecolor{currentstroke}%
\pgfsetdash{}{0pt}%
\pgfpathmoveto{\pgfqpoint{9.536573in}{4.574221in}}%
\pgfpathlineto{\pgfqpoint{9.619906in}{4.574221in}}%
\pgfpathmoveto{\pgfqpoint{9.578239in}{4.532554in}}%
\pgfpathlineto{\pgfqpoint{9.578239in}{4.615887in}}%
\pgfusepath{stroke,fill}%
\end{pgfscope}%
\begin{pgfscope}%
\pgfpathrectangle{\pgfqpoint{7.105882in}{4.326316in}}{\pgfqpoint{4.376471in}{0.953684in}} %
\pgfusepath{clip}%
\pgfsetbuttcap%
\pgfsetroundjoin%
\definecolor{currentfill}{rgb}{1.000000,0.000000,0.000000}%
\pgfsetfillcolor{currentfill}%
\pgfsetlinewidth{2.007500pt}%
\definecolor{currentstroke}{rgb}{1.000000,0.000000,0.000000}%
\pgfsetstrokecolor{currentstroke}%
\pgfsetdash{}{0pt}%
\pgfpathmoveto{\pgfqpoint{9.929697in}{4.818000in}}%
\pgfpathlineto{\pgfqpoint{10.013031in}{4.818000in}}%
\pgfpathmoveto{\pgfqpoint{9.971364in}{4.776334in}}%
\pgfpathlineto{\pgfqpoint{9.971364in}{4.859667in}}%
\pgfusepath{stroke,fill}%
\end{pgfscope}%
\begin{pgfscope}%
\pgfpathrectangle{\pgfqpoint{7.105882in}{4.326316in}}{\pgfqpoint{4.376471in}{0.953684in}} %
\pgfusepath{clip}%
\pgfsetbuttcap%
\pgfsetroundjoin%
\definecolor{currentfill}{rgb}{1.000000,0.000000,0.000000}%
\pgfsetfillcolor{currentfill}%
\pgfsetlinewidth{2.007500pt}%
\definecolor{currentstroke}{rgb}{1.000000,0.000000,0.000000}%
\pgfsetstrokecolor{currentstroke}%
\pgfsetdash{}{0pt}%
\pgfpathmoveto{\pgfqpoint{8.005296in}{5.004865in}}%
\pgfpathlineto{\pgfqpoint{8.088630in}{5.004865in}}%
\pgfpathmoveto{\pgfqpoint{8.046963in}{4.963198in}}%
\pgfpathlineto{\pgfqpoint{8.046963in}{5.046532in}}%
\pgfusepath{stroke,fill}%
\end{pgfscope}%
\begin{pgfscope}%
\pgfpathrectangle{\pgfqpoint{7.105882in}{4.326316in}}{\pgfqpoint{4.376471in}{0.953684in}} %
\pgfusepath{clip}%
\pgfsetbuttcap%
\pgfsetroundjoin%
\definecolor{currentfill}{rgb}{1.000000,0.000000,0.000000}%
\pgfsetfillcolor{currentfill}%
\pgfsetlinewidth{2.007500pt}%
\definecolor{currentstroke}{rgb}{1.000000,0.000000,0.000000}%
\pgfsetstrokecolor{currentstroke}%
\pgfsetdash{}{0pt}%
\pgfpathmoveto{\pgfqpoint{10.101961in}{4.789418in}}%
\pgfpathlineto{\pgfqpoint{10.185294in}{4.789418in}}%
\pgfpathmoveto{\pgfqpoint{10.143627in}{4.747752in}}%
\pgfpathlineto{\pgfqpoint{10.143627in}{4.831085in}}%
\pgfusepath{stroke,fill}%
\end{pgfscope}%
\begin{pgfscope}%
\pgfpathrectangle{\pgfqpoint{7.105882in}{4.326316in}}{\pgfqpoint{4.376471in}{0.953684in}} %
\pgfusepath{clip}%
\pgfsetbuttcap%
\pgfsetroundjoin%
\definecolor{currentfill}{rgb}{1.000000,0.000000,0.000000}%
\pgfsetfillcolor{currentfill}%
\pgfsetlinewidth{2.007500pt}%
\definecolor{currentstroke}{rgb}{1.000000,0.000000,0.000000}%
\pgfsetstrokecolor{currentstroke}%
\pgfsetdash{}{0pt}%
\pgfpathmoveto{\pgfqpoint{10.082565in}{4.828564in}}%
\pgfpathlineto{\pgfqpoint{10.165898in}{4.828564in}}%
\pgfpathmoveto{\pgfqpoint{10.124232in}{4.786898in}}%
\pgfpathlineto{\pgfqpoint{10.124232in}{4.870231in}}%
\pgfusepath{stroke,fill}%
\end{pgfscope}%
\begin{pgfscope}%
\pgfpathrectangle{\pgfqpoint{7.105882in}{4.326316in}}{\pgfqpoint{4.376471in}{0.953684in}} %
\pgfusepath{clip}%
\pgfsetbuttcap%
\pgfsetroundjoin%
\definecolor{currentfill}{rgb}{1.000000,0.000000,0.000000}%
\pgfsetfillcolor{currentfill}%
\pgfsetlinewidth{2.007500pt}%
\definecolor{currentstroke}{rgb}{1.000000,0.000000,0.000000}%
\pgfsetstrokecolor{currentstroke}%
\pgfsetdash{}{0pt}%
\pgfpathmoveto{\pgfqpoint{10.099505in}{4.854191in}}%
\pgfpathlineto{\pgfqpoint{10.182838in}{4.854191in}}%
\pgfpathmoveto{\pgfqpoint{10.141171in}{4.812524in}}%
\pgfpathlineto{\pgfqpoint{10.141171in}{4.895857in}}%
\pgfusepath{stroke,fill}%
\end{pgfscope}%
\begin{pgfscope}%
\pgfpathrectangle{\pgfqpoint{7.105882in}{4.326316in}}{\pgfqpoint{4.376471in}{0.953684in}} %
\pgfusepath{clip}%
\pgfsetbuttcap%
\pgfsetroundjoin%
\definecolor{currentfill}{rgb}{1.000000,0.000000,0.000000}%
\pgfsetfillcolor{currentfill}%
\pgfsetlinewidth{2.007500pt}%
\definecolor{currentstroke}{rgb}{1.000000,0.000000,0.000000}%
\pgfsetstrokecolor{currentstroke}%
\pgfsetdash{}{0pt}%
\pgfpathmoveto{\pgfqpoint{11.243738in}{4.985005in}}%
\pgfpathlineto{\pgfqpoint{11.327072in}{4.985005in}}%
\pgfpathmoveto{\pgfqpoint{11.285405in}{4.943338in}}%
\pgfpathlineto{\pgfqpoint{11.285405in}{5.026671in}}%
\pgfusepath{stroke,fill}%
\end{pgfscope}%
\begin{pgfscope}%
\pgfpathrectangle{\pgfqpoint{7.105882in}{4.326316in}}{\pgfqpoint{4.376471in}{0.953684in}} %
\pgfusepath{clip}%
\pgfsetbuttcap%
\pgfsetroundjoin%
\definecolor{currentfill}{rgb}{1.000000,0.000000,0.000000}%
\pgfsetfillcolor{currentfill}%
\pgfsetlinewidth{2.007500pt}%
\definecolor{currentstroke}{rgb}{1.000000,0.000000,0.000000}%
\pgfsetstrokecolor{currentstroke}%
\pgfsetdash{}{0pt}%
\pgfpathmoveto{\pgfqpoint{10.326683in}{4.731635in}}%
\pgfpathlineto{\pgfqpoint{10.410016in}{4.731635in}}%
\pgfpathmoveto{\pgfqpoint{10.368350in}{4.689968in}}%
\pgfpathlineto{\pgfqpoint{10.368350in}{4.773301in}}%
\pgfusepath{stroke,fill}%
\end{pgfscope}%
\begin{pgfscope}%
\pgfpathrectangle{\pgfqpoint{7.105882in}{4.326316in}}{\pgfqpoint{4.376471in}{0.953684in}} %
\pgfusepath{clip}%
\pgfsetbuttcap%
\pgfsetroundjoin%
\definecolor{currentfill}{rgb}{1.000000,0.000000,0.000000}%
\pgfsetfillcolor{currentfill}%
\pgfsetlinewidth{2.007500pt}%
\definecolor{currentstroke}{rgb}{1.000000,0.000000,0.000000}%
\pgfsetstrokecolor{currentstroke}%
\pgfsetdash{}{0pt}%
\pgfpathmoveto{\pgfqpoint{9.198210in}{4.696869in}}%
\pgfpathlineto{\pgfqpoint{9.281544in}{4.696869in}}%
\pgfpathmoveto{\pgfqpoint{9.239877in}{4.655202in}}%
\pgfpathlineto{\pgfqpoint{9.239877in}{4.738536in}}%
\pgfusepath{stroke,fill}%
\end{pgfscope}%
\begin{pgfscope}%
\pgfpathrectangle{\pgfqpoint{7.105882in}{4.326316in}}{\pgfqpoint{4.376471in}{0.953684in}} %
\pgfusepath{clip}%
\pgfsetbuttcap%
\pgfsetroundjoin%
\definecolor{currentfill}{rgb}{1.000000,0.000000,0.000000}%
\pgfsetfillcolor{currentfill}%
\pgfsetlinewidth{2.007500pt}%
\definecolor{currentstroke}{rgb}{1.000000,0.000000,0.000000}%
\pgfsetstrokecolor{currentstroke}%
\pgfsetdash{}{0pt}%
\pgfpathmoveto{\pgfqpoint{9.469636in}{4.348509in}}%
\pgfpathlineto{\pgfqpoint{9.552969in}{4.348509in}}%
\pgfpathmoveto{\pgfqpoint{9.511302in}{4.306843in}}%
\pgfpathlineto{\pgfqpoint{9.511302in}{4.390176in}}%
\pgfusepath{stroke,fill}%
\end{pgfscope}%
\begin{pgfscope}%
\pgfpathrectangle{\pgfqpoint{7.105882in}{4.326316in}}{\pgfqpoint{4.376471in}{0.953684in}} %
\pgfusepath{clip}%
\pgfsetbuttcap%
\pgfsetroundjoin%
\definecolor{currentfill}{rgb}{1.000000,0.000000,0.000000}%
\pgfsetfillcolor{currentfill}%
\pgfsetlinewidth{2.007500pt}%
\definecolor{currentstroke}{rgb}{1.000000,0.000000,0.000000}%
\pgfsetstrokecolor{currentstroke}%
\pgfsetdash{}{0pt}%
\pgfpathmoveto{\pgfqpoint{10.382040in}{4.749830in}}%
\pgfpathlineto{\pgfqpoint{10.465373in}{4.749830in}}%
\pgfpathmoveto{\pgfqpoint{10.423706in}{4.708163in}}%
\pgfpathlineto{\pgfqpoint{10.423706in}{4.791496in}}%
\pgfusepath{stroke,fill}%
\end{pgfscope}%
\begin{pgfscope}%
\pgfpathrectangle{\pgfqpoint{7.105882in}{4.326316in}}{\pgfqpoint{4.376471in}{0.953684in}} %
\pgfusepath{clip}%
\pgfsetbuttcap%
\pgfsetroundjoin%
\definecolor{currentfill}{rgb}{1.000000,0.000000,0.000000}%
\pgfsetfillcolor{currentfill}%
\pgfsetlinewidth{2.007500pt}%
\definecolor{currentstroke}{rgb}{1.000000,0.000000,0.000000}%
\pgfsetstrokecolor{currentstroke}%
\pgfsetdash{}{0pt}%
\pgfpathmoveto{\pgfqpoint{8.150370in}{5.062756in}}%
\pgfpathlineto{\pgfqpoint{8.233703in}{5.062756in}}%
\pgfpathmoveto{\pgfqpoint{8.192036in}{5.021090in}}%
\pgfpathlineto{\pgfqpoint{8.192036in}{5.104423in}}%
\pgfusepath{stroke,fill}%
\end{pgfscope}%
\begin{pgfscope}%
\pgfpathrectangle{\pgfqpoint{7.105882in}{4.326316in}}{\pgfqpoint{4.376471in}{0.953684in}} %
\pgfusepath{clip}%
\pgfsetbuttcap%
\pgfsetroundjoin%
\definecolor{currentfill}{rgb}{1.000000,0.000000,0.000000}%
\pgfsetfillcolor{currentfill}%
\pgfsetlinewidth{2.007500pt}%
\definecolor{currentstroke}{rgb}{1.000000,0.000000,0.000000}%
\pgfsetstrokecolor{currentstroke}%
\pgfsetdash{}{0pt}%
\pgfpathmoveto{\pgfqpoint{10.273978in}{4.807002in}}%
\pgfpathlineto{\pgfqpoint{10.357311in}{4.807002in}}%
\pgfpathmoveto{\pgfqpoint{10.315644in}{4.765335in}}%
\pgfpathlineto{\pgfqpoint{10.315644in}{4.848669in}}%
\pgfusepath{stroke,fill}%
\end{pgfscope}%
\begin{pgfscope}%
\pgfpathrectangle{\pgfqpoint{7.105882in}{4.326316in}}{\pgfqpoint{4.376471in}{0.953684in}} %
\pgfusepath{clip}%
\pgfsetbuttcap%
\pgfsetroundjoin%
\definecolor{currentfill}{rgb}{1.000000,0.000000,0.000000}%
\pgfsetfillcolor{currentfill}%
\pgfsetlinewidth{2.007500pt}%
\definecolor{currentstroke}{rgb}{1.000000,0.000000,0.000000}%
\pgfsetstrokecolor{currentstroke}%
\pgfsetdash{}{0pt}%
\pgfpathmoveto{\pgfqpoint{10.287531in}{4.659136in}}%
\pgfpathlineto{\pgfqpoint{10.370865in}{4.659136in}}%
\pgfpathmoveto{\pgfqpoint{10.329198in}{4.617469in}}%
\pgfpathlineto{\pgfqpoint{10.329198in}{4.700803in}}%
\pgfusepath{stroke,fill}%
\end{pgfscope}%
\begin{pgfscope}%
\pgfpathrectangle{\pgfqpoint{7.105882in}{4.326316in}}{\pgfqpoint{4.376471in}{0.953684in}} %
\pgfusepath{clip}%
\pgfsetbuttcap%
\pgfsetroundjoin%
\definecolor{currentfill}{rgb}{1.000000,0.000000,0.000000}%
\pgfsetfillcolor{currentfill}%
\pgfsetlinewidth{2.007500pt}%
\definecolor{currentstroke}{rgb}{1.000000,0.000000,0.000000}%
\pgfsetstrokecolor{currentstroke}%
\pgfsetdash{}{0pt}%
\pgfpathmoveto{\pgfqpoint{8.676096in}{4.804551in}}%
\pgfpathlineto{\pgfqpoint{8.759430in}{4.804551in}}%
\pgfpathmoveto{\pgfqpoint{8.717763in}{4.762885in}}%
\pgfpathlineto{\pgfqpoint{8.717763in}{4.846218in}}%
\pgfusepath{stroke,fill}%
\end{pgfscope}%
\begin{pgfscope}%
\pgfpathrectangle{\pgfqpoint{7.105882in}{4.326316in}}{\pgfqpoint{4.376471in}{0.953684in}} %
\pgfusepath{clip}%
\pgfsetbuttcap%
\pgfsetroundjoin%
\definecolor{currentfill}{rgb}{1.000000,0.000000,0.000000}%
\pgfsetfillcolor{currentfill}%
\pgfsetlinewidth{2.007500pt}%
\definecolor{currentstroke}{rgb}{1.000000,0.000000,0.000000}%
\pgfsetstrokecolor{currentstroke}%
\pgfsetdash{}{0pt}%
\pgfpathmoveto{\pgfqpoint{8.390904in}{4.827174in}}%
\pgfpathlineto{\pgfqpoint{8.474237in}{4.827174in}}%
\pgfpathmoveto{\pgfqpoint{8.432570in}{4.785507in}}%
\pgfpathlineto{\pgfqpoint{8.432570in}{4.868840in}}%
\pgfusepath{stroke,fill}%
\end{pgfscope}%
\begin{pgfscope}%
\pgfpathrectangle{\pgfqpoint{7.105882in}{4.326316in}}{\pgfqpoint{4.376471in}{0.953684in}} %
\pgfusepath{clip}%
\pgfsetbuttcap%
\pgfsetroundjoin%
\definecolor{currentfill}{rgb}{1.000000,0.000000,0.000000}%
\pgfsetfillcolor{currentfill}%
\pgfsetlinewidth{2.007500pt}%
\definecolor{currentstroke}{rgb}{1.000000,0.000000,0.000000}%
\pgfsetstrokecolor{currentstroke}%
\pgfsetdash{}{0pt}%
\pgfpathmoveto{\pgfqpoint{9.043880in}{4.801866in}}%
\pgfpathlineto{\pgfqpoint{9.127213in}{4.801866in}}%
\pgfpathmoveto{\pgfqpoint{9.085547in}{4.760199in}}%
\pgfpathlineto{\pgfqpoint{9.085547in}{4.843532in}}%
\pgfusepath{stroke,fill}%
\end{pgfscope}%
\begin{pgfscope}%
\pgfpathrectangle{\pgfqpoint{7.105882in}{4.326316in}}{\pgfqpoint{4.376471in}{0.953684in}} %
\pgfusepath{clip}%
\pgfsetbuttcap%
\pgfsetroundjoin%
\definecolor{currentfill}{rgb}{1.000000,0.000000,0.000000}%
\pgfsetfillcolor{currentfill}%
\pgfsetlinewidth{2.007500pt}%
\definecolor{currentstroke}{rgb}{1.000000,0.000000,0.000000}%
\pgfsetstrokecolor{currentstroke}%
\pgfsetdash{}{0pt}%
\pgfpathmoveto{\pgfqpoint{9.212925in}{4.754657in}}%
\pgfpathlineto{\pgfqpoint{9.296259in}{4.754657in}}%
\pgfpathmoveto{\pgfqpoint{9.254592in}{4.712991in}}%
\pgfpathlineto{\pgfqpoint{9.254592in}{4.796324in}}%
\pgfusepath{stroke,fill}%
\end{pgfscope}%
\begin{pgfscope}%
\pgfpathrectangle{\pgfqpoint{7.105882in}{4.326316in}}{\pgfqpoint{4.376471in}{0.953684in}} %
\pgfusepath{clip}%
\pgfsetbuttcap%
\pgfsetroundjoin%
\definecolor{currentfill}{rgb}{0.000000,0.000000,0.000000}%
\pgfsetfillcolor{currentfill}%
\pgfsetlinewidth{1.003750pt}%
\definecolor{currentstroke}{rgb}{0.000000,0.000000,0.000000}%
\pgfsetstrokecolor{currentstroke}%
\pgfsetdash{}{0pt}%
\pgfsys@defobject{currentmarker}{\pgfqpoint{-0.020833in}{-0.020833in}}{\pgfqpoint{0.020833in}{0.020833in}}{%
\pgfpathmoveto{\pgfqpoint{0.000000in}{-0.020833in}}%
\pgfpathcurveto{\pgfqpoint{0.005525in}{-0.020833in}}{\pgfqpoint{0.010825in}{-0.018638in}}{\pgfqpoint{0.014731in}{-0.014731in}}%
\pgfpathcurveto{\pgfqpoint{0.018638in}{-0.010825in}}{\pgfqpoint{0.020833in}{-0.005525in}}{\pgfqpoint{0.020833in}{0.000000in}}%
\pgfpathcurveto{\pgfqpoint{0.020833in}{0.005525in}}{\pgfqpoint{0.018638in}{0.010825in}}{\pgfqpoint{0.014731in}{0.014731in}}%
\pgfpathcurveto{\pgfqpoint{0.010825in}{0.018638in}}{\pgfqpoint{0.005525in}{0.020833in}}{\pgfqpoint{0.000000in}{0.020833in}}%
\pgfpathcurveto{\pgfqpoint{-0.005525in}{0.020833in}}{\pgfqpoint{-0.010825in}{0.018638in}}{\pgfqpoint{-0.014731in}{0.014731in}}%
\pgfpathcurveto{\pgfqpoint{-0.018638in}{0.010825in}}{\pgfqpoint{-0.020833in}{0.005525in}}{\pgfqpoint{-0.020833in}{0.000000in}}%
\pgfpathcurveto{\pgfqpoint{-0.020833in}{-0.005525in}}{\pgfqpoint{-0.018638in}{-0.010825in}}{\pgfqpoint{-0.014731in}{-0.014731in}}%
\pgfpathcurveto{\pgfqpoint{-0.010825in}{-0.018638in}}{\pgfqpoint{-0.005525in}{-0.020833in}}{\pgfqpoint{0.000000in}{-0.020833in}}%
\pgfpathclose%
\pgfusepath{stroke,fill}%
}%
\begin{pgfscope}%
\pgfsys@transformshift{7.981176in}{5.210095in}%
\pgfsys@useobject{currentmarker}{}%
\end{pgfscope}%
\begin{pgfscope}%
\pgfsys@transformshift{7.998770in}{5.261206in}%
\pgfsys@useobject{currentmarker}{}%
\end{pgfscope}%
\begin{pgfscope}%
\pgfsys@transformshift{8.016364in}{5.173532in}%
\pgfsys@useobject{currentmarker}{}%
\end{pgfscope}%
\begin{pgfscope}%
\pgfsys@transformshift{8.033958in}{5.181614in}%
\pgfsys@useobject{currentmarker}{}%
\end{pgfscope}%
\begin{pgfscope}%
\pgfsys@transformshift{8.051552in}{5.085295in}%
\pgfsys@useobject{currentmarker}{}%
\end{pgfscope}%
\begin{pgfscope}%
\pgfsys@transformshift{8.069146in}{5.234228in}%
\pgfsys@useobject{currentmarker}{}%
\end{pgfscope}%
\begin{pgfscope}%
\pgfsys@transformshift{8.086740in}{5.041923in}%
\pgfsys@useobject{currentmarker}{}%
\end{pgfscope}%
\begin{pgfscope}%
\pgfsys@transformshift{8.104333in}{5.041885in}%
\pgfsys@useobject{currentmarker}{}%
\end{pgfscope}%
\begin{pgfscope}%
\pgfsys@transformshift{8.121927in}{5.161999in}%
\pgfsys@useobject{currentmarker}{}%
\end{pgfscope}%
\begin{pgfscope}%
\pgfsys@transformshift{8.139521in}{4.814541in}%
\pgfsys@useobject{currentmarker}{}%
\end{pgfscope}%
\begin{pgfscope}%
\pgfsys@transformshift{8.157115in}{4.796454in}%
\pgfsys@useobject{currentmarker}{}%
\end{pgfscope}%
\begin{pgfscope}%
\pgfsys@transformshift{8.174709in}{4.994120in}%
\pgfsys@useobject{currentmarker}{}%
\end{pgfscope}%
\begin{pgfscope}%
\pgfsys@transformshift{8.192303in}{4.757524in}%
\pgfsys@useobject{currentmarker}{}%
\end{pgfscope}%
\begin{pgfscope}%
\pgfsys@transformshift{8.209897in}{5.044657in}%
\pgfsys@useobject{currentmarker}{}%
\end{pgfscope}%
\begin{pgfscope}%
\pgfsys@transformshift{8.227490in}{4.789412in}%
\pgfsys@useobject{currentmarker}{}%
\end{pgfscope}%
\begin{pgfscope}%
\pgfsys@transformshift{8.245084in}{4.736740in}%
\pgfsys@useobject{currentmarker}{}%
\end{pgfscope}%
\begin{pgfscope}%
\pgfsys@transformshift{8.262678in}{4.984204in}%
\pgfsys@useobject{currentmarker}{}%
\end{pgfscope}%
\begin{pgfscope}%
\pgfsys@transformshift{8.280272in}{4.924236in}%
\pgfsys@useobject{currentmarker}{}%
\end{pgfscope}%
\begin{pgfscope}%
\pgfsys@transformshift{8.297866in}{4.948566in}%
\pgfsys@useobject{currentmarker}{}%
\end{pgfscope}%
\begin{pgfscope}%
\pgfsys@transformshift{8.315460in}{4.840900in}%
\pgfsys@useobject{currentmarker}{}%
\end{pgfscope}%
\begin{pgfscope}%
\pgfsys@transformshift{8.333054in}{4.655218in}%
\pgfsys@useobject{currentmarker}{}%
\end{pgfscope}%
\begin{pgfscope}%
\pgfsys@transformshift{8.350647in}{4.922473in}%
\pgfsys@useobject{currentmarker}{}%
\end{pgfscope}%
\begin{pgfscope}%
\pgfsys@transformshift{8.368241in}{4.700133in}%
\pgfsys@useobject{currentmarker}{}%
\end{pgfscope}%
\begin{pgfscope}%
\pgfsys@transformshift{8.385835in}{4.802590in}%
\pgfsys@useobject{currentmarker}{}%
\end{pgfscope}%
\begin{pgfscope}%
\pgfsys@transformshift{8.403429in}{4.815147in}%
\pgfsys@useobject{currentmarker}{}%
\end{pgfscope}%
\begin{pgfscope}%
\pgfsys@transformshift{8.421023in}{4.705834in}%
\pgfsys@useobject{currentmarker}{}%
\end{pgfscope}%
\begin{pgfscope}%
\pgfsys@transformshift{8.438617in}{4.784377in}%
\pgfsys@useobject{currentmarker}{}%
\end{pgfscope}%
\begin{pgfscope}%
\pgfsys@transformshift{8.456210in}{4.818998in}%
\pgfsys@useobject{currentmarker}{}%
\end{pgfscope}%
\begin{pgfscope}%
\pgfsys@transformshift{8.473804in}{4.770561in}%
\pgfsys@useobject{currentmarker}{}%
\end{pgfscope}%
\begin{pgfscope}%
\pgfsys@transformshift{8.491398in}{4.631385in}%
\pgfsys@useobject{currentmarker}{}%
\end{pgfscope}%
\begin{pgfscope}%
\pgfsys@transformshift{8.508992in}{4.779188in}%
\pgfsys@useobject{currentmarker}{}%
\end{pgfscope}%
\begin{pgfscope}%
\pgfsys@transformshift{8.526586in}{4.891659in}%
\pgfsys@useobject{currentmarker}{}%
\end{pgfscope}%
\begin{pgfscope}%
\pgfsys@transformshift{8.544180in}{4.702440in}%
\pgfsys@useobject{currentmarker}{}%
\end{pgfscope}%
\begin{pgfscope}%
\pgfsys@transformshift{8.561774in}{4.769165in}%
\pgfsys@useobject{currentmarker}{}%
\end{pgfscope}%
\begin{pgfscope}%
\pgfsys@transformshift{8.579367in}{4.754253in}%
\pgfsys@useobject{currentmarker}{}%
\end{pgfscope}%
\begin{pgfscope}%
\pgfsys@transformshift{8.596961in}{4.995316in}%
\pgfsys@useobject{currentmarker}{}%
\end{pgfscope}%
\begin{pgfscope}%
\pgfsys@transformshift{8.614555in}{4.892995in}%
\pgfsys@useobject{currentmarker}{}%
\end{pgfscope}%
\begin{pgfscope}%
\pgfsys@transformshift{8.632149in}{4.881565in}%
\pgfsys@useobject{currentmarker}{}%
\end{pgfscope}%
\begin{pgfscope}%
\pgfsys@transformshift{8.649743in}{4.779557in}%
\pgfsys@useobject{currentmarker}{}%
\end{pgfscope}%
\begin{pgfscope}%
\pgfsys@transformshift{8.667337in}{4.924444in}%
\pgfsys@useobject{currentmarker}{}%
\end{pgfscope}%
\begin{pgfscope}%
\pgfsys@transformshift{8.684931in}{4.818392in}%
\pgfsys@useobject{currentmarker}{}%
\end{pgfscope}%
\begin{pgfscope}%
\pgfsys@transformshift{8.702524in}{4.902533in}%
\pgfsys@useobject{currentmarker}{}%
\end{pgfscope}%
\begin{pgfscope}%
\pgfsys@transformshift{8.720118in}{4.849486in}%
\pgfsys@useobject{currentmarker}{}%
\end{pgfscope}%
\begin{pgfscope}%
\pgfsys@transformshift{8.737712in}{4.992244in}%
\pgfsys@useobject{currentmarker}{}%
\end{pgfscope}%
\begin{pgfscope}%
\pgfsys@transformshift{8.755306in}{4.993659in}%
\pgfsys@useobject{currentmarker}{}%
\end{pgfscope}%
\begin{pgfscope}%
\pgfsys@transformshift{8.772900in}{4.925800in}%
\pgfsys@useobject{currentmarker}{}%
\end{pgfscope}%
\begin{pgfscope}%
\pgfsys@transformshift{8.790494in}{4.994580in}%
\pgfsys@useobject{currentmarker}{}%
\end{pgfscope}%
\begin{pgfscope}%
\pgfsys@transformshift{8.808087in}{4.853902in}%
\pgfsys@useobject{currentmarker}{}%
\end{pgfscope}%
\begin{pgfscope}%
\pgfsys@transformshift{8.825681in}{4.819979in}%
\pgfsys@useobject{currentmarker}{}%
\end{pgfscope}%
\begin{pgfscope}%
\pgfsys@transformshift{8.843275in}{5.015695in}%
\pgfsys@useobject{currentmarker}{}%
\end{pgfscope}%
\begin{pgfscope}%
\pgfsys@transformshift{8.860869in}{4.990730in}%
\pgfsys@useobject{currentmarker}{}%
\end{pgfscope}%
\begin{pgfscope}%
\pgfsys@transformshift{8.878463in}{5.037529in}%
\pgfsys@useobject{currentmarker}{}%
\end{pgfscope}%
\begin{pgfscope}%
\pgfsys@transformshift{8.896057in}{5.209562in}%
\pgfsys@useobject{currentmarker}{}%
\end{pgfscope}%
\begin{pgfscope}%
\pgfsys@transformshift{8.913651in}{5.063094in}%
\pgfsys@useobject{currentmarker}{}%
\end{pgfscope}%
\begin{pgfscope}%
\pgfsys@transformshift{8.931244in}{4.873113in}%
\pgfsys@useobject{currentmarker}{}%
\end{pgfscope}%
\begin{pgfscope}%
\pgfsys@transformshift{8.948838in}{5.067384in}%
\pgfsys@useobject{currentmarker}{}%
\end{pgfscope}%
\begin{pgfscope}%
\pgfsys@transformshift{8.966432in}{4.816447in}%
\pgfsys@useobject{currentmarker}{}%
\end{pgfscope}%
\begin{pgfscope}%
\pgfsys@transformshift{8.984026in}{4.890263in}%
\pgfsys@useobject{currentmarker}{}%
\end{pgfscope}%
\begin{pgfscope}%
\pgfsys@transformshift{9.001620in}{4.916551in}%
\pgfsys@useobject{currentmarker}{}%
\end{pgfscope}%
\begin{pgfscope}%
\pgfsys@transformshift{9.019214in}{5.079111in}%
\pgfsys@useobject{currentmarker}{}%
\end{pgfscope}%
\begin{pgfscope}%
\pgfsys@transformshift{9.036808in}{4.818967in}%
\pgfsys@useobject{currentmarker}{}%
\end{pgfscope}%
\begin{pgfscope}%
\pgfsys@transformshift{9.054401in}{4.793504in}%
\pgfsys@useobject{currentmarker}{}%
\end{pgfscope}%
\begin{pgfscope}%
\pgfsys@transformshift{9.071995in}{4.847222in}%
\pgfsys@useobject{currentmarker}{}%
\end{pgfscope}%
\begin{pgfscope}%
\pgfsys@transformshift{9.089589in}{4.771393in}%
\pgfsys@useobject{currentmarker}{}%
\end{pgfscope}%
\begin{pgfscope}%
\pgfsys@transformshift{9.107183in}{4.928600in}%
\pgfsys@useobject{currentmarker}{}%
\end{pgfscope}%
\begin{pgfscope}%
\pgfsys@transformshift{9.124777in}{4.688020in}%
\pgfsys@useobject{currentmarker}{}%
\end{pgfscope}%
\begin{pgfscope}%
\pgfsys@transformshift{9.142371in}{4.659374in}%
\pgfsys@useobject{currentmarker}{}%
\end{pgfscope}%
\begin{pgfscope}%
\pgfsys@transformshift{9.159965in}{4.707568in}%
\pgfsys@useobject{currentmarker}{}%
\end{pgfscope}%
\begin{pgfscope}%
\pgfsys@transformshift{9.177558in}{4.679113in}%
\pgfsys@useobject{currentmarker}{}%
\end{pgfscope}%
\begin{pgfscope}%
\pgfsys@transformshift{9.195152in}{4.897831in}%
\pgfsys@useobject{currentmarker}{}%
\end{pgfscope}%
\begin{pgfscope}%
\pgfsys@transformshift{9.212746in}{4.778256in}%
\pgfsys@useobject{currentmarker}{}%
\end{pgfscope}%
\begin{pgfscope}%
\pgfsys@transformshift{9.230340in}{4.670860in}%
\pgfsys@useobject{currentmarker}{}%
\end{pgfscope}%
\begin{pgfscope}%
\pgfsys@transformshift{9.247934in}{4.519279in}%
\pgfsys@useobject{currentmarker}{}%
\end{pgfscope}%
\begin{pgfscope}%
\pgfsys@transformshift{9.265528in}{4.704570in}%
\pgfsys@useobject{currentmarker}{}%
\end{pgfscope}%
\begin{pgfscope}%
\pgfsys@transformshift{9.283121in}{4.502002in}%
\pgfsys@useobject{currentmarker}{}%
\end{pgfscope}%
\begin{pgfscope}%
\pgfsys@transformshift{9.300715in}{4.429759in}%
\pgfsys@useobject{currentmarker}{}%
\end{pgfscope}%
\begin{pgfscope}%
\pgfsys@transformshift{9.318309in}{4.684435in}%
\pgfsys@useobject{currentmarker}{}%
\end{pgfscope}%
\begin{pgfscope}%
\pgfsys@transformshift{9.335903in}{4.582583in}%
\pgfsys@useobject{currentmarker}{}%
\end{pgfscope}%
\begin{pgfscope}%
\pgfsys@transformshift{9.353497in}{4.628887in}%
\pgfsys@useobject{currentmarker}{}%
\end{pgfscope}%
\begin{pgfscope}%
\pgfsys@transformshift{9.371091in}{4.557126in}%
\pgfsys@useobject{currentmarker}{}%
\end{pgfscope}%
\begin{pgfscope}%
\pgfsys@transformshift{9.388685in}{4.600489in}%
\pgfsys@useobject{currentmarker}{}%
\end{pgfscope}%
\begin{pgfscope}%
\pgfsys@transformshift{9.406278in}{4.442537in}%
\pgfsys@useobject{currentmarker}{}%
\end{pgfscope}%
\begin{pgfscope}%
\pgfsys@transformshift{9.423872in}{4.398312in}%
\pgfsys@useobject{currentmarker}{}%
\end{pgfscope}%
\begin{pgfscope}%
\pgfsys@transformshift{9.441466in}{4.564662in}%
\pgfsys@useobject{currentmarker}{}%
\end{pgfscope}%
\begin{pgfscope}%
\pgfsys@transformshift{9.459060in}{4.415111in}%
\pgfsys@useobject{currentmarker}{}%
\end{pgfscope}%
\begin{pgfscope}%
\pgfsys@transformshift{9.476654in}{4.426479in}%
\pgfsys@useobject{currentmarker}{}%
\end{pgfscope}%
\begin{pgfscope}%
\pgfsys@transformshift{9.494248in}{4.451857in}%
\pgfsys@useobject{currentmarker}{}%
\end{pgfscope}%
\begin{pgfscope}%
\pgfsys@transformshift{9.511842in}{4.503017in}%
\pgfsys@useobject{currentmarker}{}%
\end{pgfscope}%
\begin{pgfscope}%
\pgfsys@transformshift{9.529435in}{4.472271in}%
\pgfsys@useobject{currentmarker}{}%
\end{pgfscope}%
\begin{pgfscope}%
\pgfsys@transformshift{9.547029in}{4.378919in}%
\pgfsys@useobject{currentmarker}{}%
\end{pgfscope}%
\begin{pgfscope}%
\pgfsys@transformshift{9.564623in}{4.461469in}%
\pgfsys@useobject{currentmarker}{}%
\end{pgfscope}%
\begin{pgfscope}%
\pgfsys@transformshift{9.582217in}{4.316143in}%
\pgfsys@useobject{currentmarker}{}%
\end{pgfscope}%
\begin{pgfscope}%
\pgfsys@transformshift{9.599811in}{4.612339in}%
\pgfsys@useobject{currentmarker}{}%
\end{pgfscope}%
\begin{pgfscope}%
\pgfsys@transformshift{9.617405in}{4.405734in}%
\pgfsys@useobject{currentmarker}{}%
\end{pgfscope}%
\begin{pgfscope}%
\pgfsys@transformshift{9.634999in}{4.471132in}%
\pgfsys@useobject{currentmarker}{}%
\end{pgfscope}%
\begin{pgfscope}%
\pgfsys@transformshift{9.652592in}{4.603117in}%
\pgfsys@useobject{currentmarker}{}%
\end{pgfscope}%
\begin{pgfscope}%
\pgfsys@transformshift{9.670186in}{4.542513in}%
\pgfsys@useobject{currentmarker}{}%
\end{pgfscope}%
\begin{pgfscope}%
\pgfsys@transformshift{9.687780in}{4.788087in}%
\pgfsys@useobject{currentmarker}{}%
\end{pgfscope}%
\begin{pgfscope}%
\pgfsys@transformshift{9.705374in}{4.525763in}%
\pgfsys@useobject{currentmarker}{}%
\end{pgfscope}%
\begin{pgfscope}%
\pgfsys@transformshift{9.722968in}{4.700507in}%
\pgfsys@useobject{currentmarker}{}%
\end{pgfscope}%
\begin{pgfscope}%
\pgfsys@transformshift{9.740562in}{4.690035in}%
\pgfsys@useobject{currentmarker}{}%
\end{pgfscope}%
\begin{pgfscope}%
\pgfsys@transformshift{9.758155in}{4.597836in}%
\pgfsys@useobject{currentmarker}{}%
\end{pgfscope}%
\begin{pgfscope}%
\pgfsys@transformshift{9.775749in}{4.785579in}%
\pgfsys@useobject{currentmarker}{}%
\end{pgfscope}%
\begin{pgfscope}%
\pgfsys@transformshift{9.793343in}{4.735854in}%
\pgfsys@useobject{currentmarker}{}%
\end{pgfscope}%
\begin{pgfscope}%
\pgfsys@transformshift{9.810937in}{4.848257in}%
\pgfsys@useobject{currentmarker}{}%
\end{pgfscope}%
\begin{pgfscope}%
\pgfsys@transformshift{9.828531in}{4.871288in}%
\pgfsys@useobject{currentmarker}{}%
\end{pgfscope}%
\begin{pgfscope}%
\pgfsys@transformshift{9.846125in}{5.021301in}%
\pgfsys@useobject{currentmarker}{}%
\end{pgfscope}%
\begin{pgfscope}%
\pgfsys@transformshift{9.863719in}{4.954956in}%
\pgfsys@useobject{currentmarker}{}%
\end{pgfscope}%
\begin{pgfscope}%
\pgfsys@transformshift{9.881312in}{4.799998in}%
\pgfsys@useobject{currentmarker}{}%
\end{pgfscope}%
\begin{pgfscope}%
\pgfsys@transformshift{9.898906in}{4.825952in}%
\pgfsys@useobject{currentmarker}{}%
\end{pgfscope}%
\begin{pgfscope}%
\pgfsys@transformshift{9.916500in}{4.970478in}%
\pgfsys@useobject{currentmarker}{}%
\end{pgfscope}%
\begin{pgfscope}%
\pgfsys@transformshift{9.934094in}{4.936169in}%
\pgfsys@useobject{currentmarker}{}%
\end{pgfscope}%
\begin{pgfscope}%
\pgfsys@transformshift{9.951688in}{4.942749in}%
\pgfsys@useobject{currentmarker}{}%
\end{pgfscope}%
\begin{pgfscope}%
\pgfsys@transformshift{9.969282in}{4.724782in}%
\pgfsys@useobject{currentmarker}{}%
\end{pgfscope}%
\begin{pgfscope}%
\pgfsys@transformshift{9.986876in}{4.887375in}%
\pgfsys@useobject{currentmarker}{}%
\end{pgfscope}%
\begin{pgfscope}%
\pgfsys@transformshift{10.004469in}{4.818940in}%
\pgfsys@useobject{currentmarker}{}%
\end{pgfscope}%
\begin{pgfscope}%
\pgfsys@transformshift{10.022063in}{4.920607in}%
\pgfsys@useobject{currentmarker}{}%
\end{pgfscope}%
\begin{pgfscope}%
\pgfsys@transformshift{10.039657in}{4.881672in}%
\pgfsys@useobject{currentmarker}{}%
\end{pgfscope}%
\begin{pgfscope}%
\pgfsys@transformshift{10.057251in}{4.978576in}%
\pgfsys@useobject{currentmarker}{}%
\end{pgfscope}%
\begin{pgfscope}%
\pgfsys@transformshift{10.074845in}{4.914563in}%
\pgfsys@useobject{currentmarker}{}%
\end{pgfscope}%
\begin{pgfscope}%
\pgfsys@transformshift{10.092439in}{4.954282in}%
\pgfsys@useobject{currentmarker}{}%
\end{pgfscope}%
\begin{pgfscope}%
\pgfsys@transformshift{10.110033in}{4.821260in}%
\pgfsys@useobject{currentmarker}{}%
\end{pgfscope}%
\begin{pgfscope}%
\pgfsys@transformshift{10.127626in}{4.763613in}%
\pgfsys@useobject{currentmarker}{}%
\end{pgfscope}%
\begin{pgfscope}%
\pgfsys@transformshift{10.145220in}{4.805078in}%
\pgfsys@useobject{currentmarker}{}%
\end{pgfscope}%
\begin{pgfscope}%
\pgfsys@transformshift{10.162814in}{4.831169in}%
\pgfsys@useobject{currentmarker}{}%
\end{pgfscope}%
\begin{pgfscope}%
\pgfsys@transformshift{10.180408in}{4.856393in}%
\pgfsys@useobject{currentmarker}{}%
\end{pgfscope}%
\begin{pgfscope}%
\pgfsys@transformshift{10.198002in}{5.028023in}%
\pgfsys@useobject{currentmarker}{}%
\end{pgfscope}%
\begin{pgfscope}%
\pgfsys@transformshift{10.215596in}{4.783388in}%
\pgfsys@useobject{currentmarker}{}%
\end{pgfscope}%
\begin{pgfscope}%
\pgfsys@transformshift{10.233189in}{4.675969in}%
\pgfsys@useobject{currentmarker}{}%
\end{pgfscope}%
\begin{pgfscope}%
\pgfsys@transformshift{10.250783in}{4.719445in}%
\pgfsys@useobject{currentmarker}{}%
\end{pgfscope}%
\begin{pgfscope}%
\pgfsys@transformshift{10.268377in}{4.690369in}%
\pgfsys@useobject{currentmarker}{}%
\end{pgfscope}%
\begin{pgfscope}%
\pgfsys@transformshift{10.285971in}{4.766790in}%
\pgfsys@useobject{currentmarker}{}%
\end{pgfscope}%
\begin{pgfscope}%
\pgfsys@transformshift{10.303565in}{4.548557in}%
\pgfsys@useobject{currentmarker}{}%
\end{pgfscope}%
\begin{pgfscope}%
\pgfsys@transformshift{10.321159in}{4.690901in}%
\pgfsys@useobject{currentmarker}{}%
\end{pgfscope}%
\begin{pgfscope}%
\pgfsys@transformshift{10.338753in}{4.683687in}%
\pgfsys@useobject{currentmarker}{}%
\end{pgfscope}%
\begin{pgfscope}%
\pgfsys@transformshift{10.356346in}{4.675448in}%
\pgfsys@useobject{currentmarker}{}%
\end{pgfscope}%
\begin{pgfscope}%
\pgfsys@transformshift{10.373940in}{4.578198in}%
\pgfsys@useobject{currentmarker}{}%
\end{pgfscope}%
\begin{pgfscope}%
\pgfsys@transformshift{10.391534in}{4.600143in}%
\pgfsys@useobject{currentmarker}{}%
\end{pgfscope}%
\begin{pgfscope}%
\pgfsys@transformshift{10.409128in}{4.469808in}%
\pgfsys@useobject{currentmarker}{}%
\end{pgfscope}%
\begin{pgfscope}%
\pgfsys@transformshift{10.426722in}{4.551183in}%
\pgfsys@useobject{currentmarker}{}%
\end{pgfscope}%
\begin{pgfscope}%
\pgfsys@transformshift{10.444316in}{4.536732in}%
\pgfsys@useobject{currentmarker}{}%
\end{pgfscope}%
\begin{pgfscope}%
\pgfsys@transformshift{10.461910in}{4.624149in}%
\pgfsys@useobject{currentmarker}{}%
\end{pgfscope}%
\begin{pgfscope}%
\pgfsys@transformshift{10.479503in}{4.461915in}%
\pgfsys@useobject{currentmarker}{}%
\end{pgfscope}%
\begin{pgfscope}%
\pgfsys@transformshift{10.497097in}{4.650228in}%
\pgfsys@useobject{currentmarker}{}%
\end{pgfscope}%
\begin{pgfscope}%
\pgfsys@transformshift{10.514691in}{4.718906in}%
\pgfsys@useobject{currentmarker}{}%
\end{pgfscope}%
\begin{pgfscope}%
\pgfsys@transformshift{10.532285in}{4.364651in}%
\pgfsys@useobject{currentmarker}{}%
\end{pgfscope}%
\begin{pgfscope}%
\pgfsys@transformshift{10.549879in}{4.614547in}%
\pgfsys@useobject{currentmarker}{}%
\end{pgfscope}%
\begin{pgfscope}%
\pgfsys@transformshift{10.567473in}{4.643444in}%
\pgfsys@useobject{currentmarker}{}%
\end{pgfscope}%
\begin{pgfscope}%
\pgfsys@transformshift{10.585067in}{4.518952in}%
\pgfsys@useobject{currentmarker}{}%
\end{pgfscope}%
\begin{pgfscope}%
\pgfsys@transformshift{10.602660in}{4.550949in}%
\pgfsys@useobject{currentmarker}{}%
\end{pgfscope}%
\begin{pgfscope}%
\pgfsys@transformshift{10.620254in}{4.587369in}%
\pgfsys@useobject{currentmarker}{}%
\end{pgfscope}%
\begin{pgfscope}%
\pgfsys@transformshift{10.637848in}{4.583096in}%
\pgfsys@useobject{currentmarker}{}%
\end{pgfscope}%
\begin{pgfscope}%
\pgfsys@transformshift{10.655442in}{4.595964in}%
\pgfsys@useobject{currentmarker}{}%
\end{pgfscope}%
\begin{pgfscope}%
\pgfsys@transformshift{10.673036in}{4.475923in}%
\pgfsys@useobject{currentmarker}{}%
\end{pgfscope}%
\begin{pgfscope}%
\pgfsys@transformshift{10.690630in}{4.774312in}%
\pgfsys@useobject{currentmarker}{}%
\end{pgfscope}%
\begin{pgfscope}%
\pgfsys@transformshift{10.708223in}{4.786033in}%
\pgfsys@useobject{currentmarker}{}%
\end{pgfscope}%
\begin{pgfscope}%
\pgfsys@transformshift{10.725817in}{4.618304in}%
\pgfsys@useobject{currentmarker}{}%
\end{pgfscope}%
\begin{pgfscope}%
\pgfsys@transformshift{10.743411in}{4.574972in}%
\pgfsys@useobject{currentmarker}{}%
\end{pgfscope}%
\begin{pgfscope}%
\pgfsys@transformshift{10.761005in}{4.795007in}%
\pgfsys@useobject{currentmarker}{}%
\end{pgfscope}%
\begin{pgfscope}%
\pgfsys@transformshift{10.778599in}{4.709505in}%
\pgfsys@useobject{currentmarker}{}%
\end{pgfscope}%
\begin{pgfscope}%
\pgfsys@transformshift{10.796193in}{4.804947in}%
\pgfsys@useobject{currentmarker}{}%
\end{pgfscope}%
\begin{pgfscope}%
\pgfsys@transformshift{10.813787in}{4.783747in}%
\pgfsys@useobject{currentmarker}{}%
\end{pgfscope}%
\begin{pgfscope}%
\pgfsys@transformshift{10.831380in}{4.909079in}%
\pgfsys@useobject{currentmarker}{}%
\end{pgfscope}%
\begin{pgfscope}%
\pgfsys@transformshift{10.848974in}{4.934392in}%
\pgfsys@useobject{currentmarker}{}%
\end{pgfscope}%
\begin{pgfscope}%
\pgfsys@transformshift{10.866568in}{4.818168in}%
\pgfsys@useobject{currentmarker}{}%
\end{pgfscope}%
\begin{pgfscope}%
\pgfsys@transformshift{10.884162in}{4.777384in}%
\pgfsys@useobject{currentmarker}{}%
\end{pgfscope}%
\begin{pgfscope}%
\pgfsys@transformshift{10.901756in}{4.781605in}%
\pgfsys@useobject{currentmarker}{}%
\end{pgfscope}%
\begin{pgfscope}%
\pgfsys@transformshift{10.919350in}{5.022841in}%
\pgfsys@useobject{currentmarker}{}%
\end{pgfscope}%
\begin{pgfscope}%
\pgfsys@transformshift{10.936944in}{4.866566in}%
\pgfsys@useobject{currentmarker}{}%
\end{pgfscope}%
\begin{pgfscope}%
\pgfsys@transformshift{10.954537in}{4.955732in}%
\pgfsys@useobject{currentmarker}{}%
\end{pgfscope}%
\begin{pgfscope}%
\pgfsys@transformshift{10.972131in}{4.967038in}%
\pgfsys@useobject{currentmarker}{}%
\end{pgfscope}%
\begin{pgfscope}%
\pgfsys@transformshift{10.989725in}{5.039890in}%
\pgfsys@useobject{currentmarker}{}%
\end{pgfscope}%
\begin{pgfscope}%
\pgfsys@transformshift{11.007319in}{4.870099in}%
\pgfsys@useobject{currentmarker}{}%
\end{pgfscope}%
\begin{pgfscope}%
\pgfsys@transformshift{11.024913in}{5.096938in}%
\pgfsys@useobject{currentmarker}{}%
\end{pgfscope}%
\begin{pgfscope}%
\pgfsys@transformshift{11.042507in}{5.144249in}%
\pgfsys@useobject{currentmarker}{}%
\end{pgfscope}%
\begin{pgfscope}%
\pgfsys@transformshift{11.060101in}{5.113218in}%
\pgfsys@useobject{currentmarker}{}%
\end{pgfscope}%
\begin{pgfscope}%
\pgfsys@transformshift{11.077694in}{5.083924in}%
\pgfsys@useobject{currentmarker}{}%
\end{pgfscope}%
\begin{pgfscope}%
\pgfsys@transformshift{11.095288in}{5.132962in}%
\pgfsys@useobject{currentmarker}{}%
\end{pgfscope}%
\begin{pgfscope}%
\pgfsys@transformshift{11.112882in}{5.169484in}%
\pgfsys@useobject{currentmarker}{}%
\end{pgfscope}%
\begin{pgfscope}%
\pgfsys@transformshift{11.130476in}{4.859040in}%
\pgfsys@useobject{currentmarker}{}%
\end{pgfscope}%
\begin{pgfscope}%
\pgfsys@transformshift{11.148070in}{5.331506in}%
\pgfsys@useobject{currentmarker}{}%
\end{pgfscope}%
\begin{pgfscope}%
\pgfsys@transformshift{11.165664in}{5.176812in}%
\pgfsys@useobject{currentmarker}{}%
\end{pgfscope}%
\begin{pgfscope}%
\pgfsys@transformshift{11.183257in}{5.072293in}%
\pgfsys@useobject{currentmarker}{}%
\end{pgfscope}%
\begin{pgfscope}%
\pgfsys@transformshift{11.200851in}{5.095575in}%
\pgfsys@useobject{currentmarker}{}%
\end{pgfscope}%
\begin{pgfscope}%
\pgfsys@transformshift{11.218445in}{5.179130in}%
\pgfsys@useobject{currentmarker}{}%
\end{pgfscope}%
\begin{pgfscope}%
\pgfsys@transformshift{11.236039in}{5.112685in}%
\pgfsys@useobject{currentmarker}{}%
\end{pgfscope}%
\begin{pgfscope}%
\pgfsys@transformshift{11.253633in}{4.915222in}%
\pgfsys@useobject{currentmarker}{}%
\end{pgfscope}%
\begin{pgfscope}%
\pgfsys@transformshift{11.271227in}{5.313522in}%
\pgfsys@useobject{currentmarker}{}%
\end{pgfscope}%
\begin{pgfscope}%
\pgfsys@transformshift{11.288821in}{5.087825in}%
\pgfsys@useobject{currentmarker}{}%
\end{pgfscope}%
\begin{pgfscope}%
\pgfsys@transformshift{11.306414in}{5.189593in}%
\pgfsys@useobject{currentmarker}{}%
\end{pgfscope}%
\begin{pgfscope}%
\pgfsys@transformshift{11.324008in}{5.008182in}%
\pgfsys@useobject{currentmarker}{}%
\end{pgfscope}%
\begin{pgfscope}%
\pgfsys@transformshift{11.341602in}{5.217676in}%
\pgfsys@useobject{currentmarker}{}%
\end{pgfscope}%
\begin{pgfscope}%
\pgfsys@transformshift{11.359196in}{5.081238in}%
\pgfsys@useobject{currentmarker}{}%
\end{pgfscope}%
\begin{pgfscope}%
\pgfsys@transformshift{11.376790in}{5.100842in}%
\pgfsys@useobject{currentmarker}{}%
\end{pgfscope}%
\begin{pgfscope}%
\pgfsys@transformshift{11.394384in}{4.924112in}%
\pgfsys@useobject{currentmarker}{}%
\end{pgfscope}%
\begin{pgfscope}%
\pgfsys@transformshift{11.411978in}{5.136084in}%
\pgfsys@useobject{currentmarker}{}%
\end{pgfscope}%
\begin{pgfscope}%
\pgfsys@transformshift{11.429571in}{5.072537in}%
\pgfsys@useobject{currentmarker}{}%
\end{pgfscope}%
\begin{pgfscope}%
\pgfsys@transformshift{11.447165in}{5.122202in}%
\pgfsys@useobject{currentmarker}{}%
\end{pgfscope}%
\begin{pgfscope}%
\pgfsys@transformshift{11.464759in}{4.920157in}%
\pgfsys@useobject{currentmarker}{}%
\end{pgfscope}%
\begin{pgfscope}%
\pgfsys@transformshift{11.482353in}{4.925441in}%
\pgfsys@useobject{currentmarker}{}%
\end{pgfscope}%
\end{pgfscope}%
\begin{pgfscope}%
\pgfsetbuttcap%
\pgfsetroundjoin%
\definecolor{currentfill}{rgb}{0.000000,0.000000,0.000000}%
\pgfsetfillcolor{currentfill}%
\pgfsetlinewidth{0.803000pt}%
\definecolor{currentstroke}{rgb}{0.000000,0.000000,0.000000}%
\pgfsetstrokecolor{currentstroke}%
\pgfsetdash{}{0pt}%
\pgfsys@defobject{currentmarker}{\pgfqpoint{0.000000in}{-0.048611in}}{\pgfqpoint{0.000000in}{0.000000in}}{%
\pgfpathmoveto{\pgfqpoint{0.000000in}{0.000000in}}%
\pgfpathlineto{\pgfqpoint{0.000000in}{-0.048611in}}%
\pgfusepath{stroke,fill}%
}%
\begin{pgfscope}%
\pgfsys@transformshift{7.105882in}{4.326316in}%
\pgfsys@useobject{currentmarker}{}%
\end{pgfscope}%
\end{pgfscope}%
\begin{pgfscope}%
\pgfsetbuttcap%
\pgfsetroundjoin%
\definecolor{currentfill}{rgb}{0.000000,0.000000,0.000000}%
\pgfsetfillcolor{currentfill}%
\pgfsetlinewidth{0.803000pt}%
\definecolor{currentstroke}{rgb}{0.000000,0.000000,0.000000}%
\pgfsetstrokecolor{currentstroke}%
\pgfsetdash{}{0pt}%
\pgfsys@defobject{currentmarker}{\pgfqpoint{0.000000in}{-0.048611in}}{\pgfqpoint{0.000000in}{0.000000in}}{%
\pgfpathmoveto{\pgfqpoint{0.000000in}{0.000000in}}%
\pgfpathlineto{\pgfqpoint{0.000000in}{-0.048611in}}%
\pgfusepath{stroke,fill}%
}%
\begin{pgfscope}%
\pgfsys@transformshift{7.981176in}{4.326316in}%
\pgfsys@useobject{currentmarker}{}%
\end{pgfscope}%
\end{pgfscope}%
\begin{pgfscope}%
\pgfsetbuttcap%
\pgfsetroundjoin%
\definecolor{currentfill}{rgb}{0.000000,0.000000,0.000000}%
\pgfsetfillcolor{currentfill}%
\pgfsetlinewidth{0.803000pt}%
\definecolor{currentstroke}{rgb}{0.000000,0.000000,0.000000}%
\pgfsetstrokecolor{currentstroke}%
\pgfsetdash{}{0pt}%
\pgfsys@defobject{currentmarker}{\pgfqpoint{0.000000in}{-0.048611in}}{\pgfqpoint{0.000000in}{0.000000in}}{%
\pgfpathmoveto{\pgfqpoint{0.000000in}{0.000000in}}%
\pgfpathlineto{\pgfqpoint{0.000000in}{-0.048611in}}%
\pgfusepath{stroke,fill}%
}%
\begin{pgfscope}%
\pgfsys@transformshift{8.856471in}{4.326316in}%
\pgfsys@useobject{currentmarker}{}%
\end{pgfscope}%
\end{pgfscope}%
\begin{pgfscope}%
\pgfsetbuttcap%
\pgfsetroundjoin%
\definecolor{currentfill}{rgb}{0.000000,0.000000,0.000000}%
\pgfsetfillcolor{currentfill}%
\pgfsetlinewidth{0.803000pt}%
\definecolor{currentstroke}{rgb}{0.000000,0.000000,0.000000}%
\pgfsetstrokecolor{currentstroke}%
\pgfsetdash{}{0pt}%
\pgfsys@defobject{currentmarker}{\pgfqpoint{0.000000in}{-0.048611in}}{\pgfqpoint{0.000000in}{0.000000in}}{%
\pgfpathmoveto{\pgfqpoint{0.000000in}{0.000000in}}%
\pgfpathlineto{\pgfqpoint{0.000000in}{-0.048611in}}%
\pgfusepath{stroke,fill}%
}%
\begin{pgfscope}%
\pgfsys@transformshift{9.731765in}{4.326316in}%
\pgfsys@useobject{currentmarker}{}%
\end{pgfscope}%
\end{pgfscope}%
\begin{pgfscope}%
\pgfsetbuttcap%
\pgfsetroundjoin%
\definecolor{currentfill}{rgb}{0.000000,0.000000,0.000000}%
\pgfsetfillcolor{currentfill}%
\pgfsetlinewidth{0.803000pt}%
\definecolor{currentstroke}{rgb}{0.000000,0.000000,0.000000}%
\pgfsetstrokecolor{currentstroke}%
\pgfsetdash{}{0pt}%
\pgfsys@defobject{currentmarker}{\pgfqpoint{0.000000in}{-0.048611in}}{\pgfqpoint{0.000000in}{0.000000in}}{%
\pgfpathmoveto{\pgfqpoint{0.000000in}{0.000000in}}%
\pgfpathlineto{\pgfqpoint{0.000000in}{-0.048611in}}%
\pgfusepath{stroke,fill}%
}%
\begin{pgfscope}%
\pgfsys@transformshift{10.607059in}{4.326316in}%
\pgfsys@useobject{currentmarker}{}%
\end{pgfscope}%
\end{pgfscope}%
\begin{pgfscope}%
\pgfsetbuttcap%
\pgfsetroundjoin%
\definecolor{currentfill}{rgb}{0.000000,0.000000,0.000000}%
\pgfsetfillcolor{currentfill}%
\pgfsetlinewidth{0.803000pt}%
\definecolor{currentstroke}{rgb}{0.000000,0.000000,0.000000}%
\pgfsetstrokecolor{currentstroke}%
\pgfsetdash{}{0pt}%
\pgfsys@defobject{currentmarker}{\pgfqpoint{0.000000in}{-0.048611in}}{\pgfqpoint{0.000000in}{0.000000in}}{%
\pgfpathmoveto{\pgfqpoint{0.000000in}{0.000000in}}%
\pgfpathlineto{\pgfqpoint{0.000000in}{-0.048611in}}%
\pgfusepath{stroke,fill}%
}%
\begin{pgfscope}%
\pgfsys@transformshift{11.482353in}{4.326316in}%
\pgfsys@useobject{currentmarker}{}%
\end{pgfscope}%
\end{pgfscope}%
\begin{pgfscope}%
\pgfsetbuttcap%
\pgfsetroundjoin%
\definecolor{currentfill}{rgb}{0.000000,0.000000,0.000000}%
\pgfsetfillcolor{currentfill}%
\pgfsetlinewidth{0.803000pt}%
\definecolor{currentstroke}{rgb}{0.000000,0.000000,0.000000}%
\pgfsetstrokecolor{currentstroke}%
\pgfsetdash{}{0pt}%
\pgfsys@defobject{currentmarker}{\pgfqpoint{-0.048611in}{0.000000in}}{\pgfqpoint{0.000000in}{0.000000in}}{%
\pgfpathmoveto{\pgfqpoint{0.000000in}{0.000000in}}%
\pgfpathlineto{\pgfqpoint{-0.048611in}{0.000000in}}%
\pgfusepath{stroke,fill}%
}%
\begin{pgfscope}%
\pgfsys@transformshift{7.105882in}{4.683947in}%
\pgfsys@useobject{currentmarker}{}%
\end{pgfscope}%
\end{pgfscope}%
\begin{pgfscope}%
\pgftext[x=6.939215in,y=4.635730in,left,base]{\rmfamily\fontsize{10.000000}{12.000000}\selectfont \(\displaystyle 0\)}%
\end{pgfscope}%
\begin{pgfscope}%
\pgfsetbuttcap%
\pgfsetroundjoin%
\definecolor{currentfill}{rgb}{0.000000,0.000000,0.000000}%
\pgfsetfillcolor{currentfill}%
\pgfsetlinewidth{0.803000pt}%
\definecolor{currentstroke}{rgb}{0.000000,0.000000,0.000000}%
\pgfsetstrokecolor{currentstroke}%
\pgfsetdash{}{0pt}%
\pgfsys@defobject{currentmarker}{\pgfqpoint{-0.048611in}{0.000000in}}{\pgfqpoint{0.000000in}{0.000000in}}{%
\pgfpathmoveto{\pgfqpoint{0.000000in}{0.000000in}}%
\pgfpathlineto{\pgfqpoint{-0.048611in}{0.000000in}}%
\pgfusepath{stroke,fill}%
}%
\begin{pgfscope}%
\pgfsys@transformshift{7.105882in}{5.081316in}%
\pgfsys@useobject{currentmarker}{}%
\end{pgfscope}%
\end{pgfscope}%
\begin{pgfscope}%
\pgftext[x=6.939215in,y=5.033098in,left,base]{\rmfamily\fontsize{10.000000}{12.000000}\selectfont \(\displaystyle 2\)}%
\end{pgfscope}%
\begin{pgfscope}%
\pgfpathrectangle{\pgfqpoint{7.105882in}{4.326316in}}{\pgfqpoint{4.376471in}{0.953684in}} %
\pgfusepath{clip}%
\pgfsetrectcap%
\pgfsetroundjoin%
\pgfsetlinewidth{1.505625pt}%
\definecolor{currentstroke}{rgb}{0.121569,0.466667,0.705882}%
\pgfsetstrokecolor{currentstroke}%
\pgfsetdash{}{0pt}%
\pgfpathmoveto{\pgfqpoint{7.981176in}{4.683947in}}%
\pgfpathlineto{\pgfqpoint{11.482353in}{4.683947in}}%
\pgfpathlineto{\pgfqpoint{11.482353in}{4.683947in}}%
\pgfusepath{stroke}%
\end{pgfscope}%
\begin{pgfscope}%
\pgfsetrectcap%
\pgfsetmiterjoin%
\pgfsetlinewidth{0.803000pt}%
\definecolor{currentstroke}{rgb}{0.000000,0.000000,0.000000}%
\pgfsetstrokecolor{currentstroke}%
\pgfsetdash{}{0pt}%
\pgfpathmoveto{\pgfqpoint{7.105882in}{4.326316in}}%
\pgfpathlineto{\pgfqpoint{7.105882in}{5.280000in}}%
\pgfusepath{stroke}%
\end{pgfscope}%
\begin{pgfscope}%
\pgfsetrectcap%
\pgfsetmiterjoin%
\pgfsetlinewidth{0.803000pt}%
\definecolor{currentstroke}{rgb}{0.000000,0.000000,0.000000}%
\pgfsetstrokecolor{currentstroke}%
\pgfsetdash{}{0pt}%
\pgfpathmoveto{\pgfqpoint{11.482353in}{4.326316in}}%
\pgfpathlineto{\pgfqpoint{11.482353in}{5.280000in}}%
\pgfusepath{stroke}%
\end{pgfscope}%
\begin{pgfscope}%
\pgfsetrectcap%
\pgfsetmiterjoin%
\pgfsetlinewidth{0.803000pt}%
\definecolor{currentstroke}{rgb}{0.000000,0.000000,0.000000}%
\pgfsetstrokecolor{currentstroke}%
\pgfsetdash{}{0pt}%
\pgfpathmoveto{\pgfqpoint{7.105882in}{4.326316in}}%
\pgfpathlineto{\pgfqpoint{11.482353in}{4.326316in}}%
\pgfusepath{stroke}%
\end{pgfscope}%
\begin{pgfscope}%
\pgfsetrectcap%
\pgfsetmiterjoin%
\pgfsetlinewidth{0.803000pt}%
\definecolor{currentstroke}{rgb}{0.000000,0.000000,0.000000}%
\pgfsetstrokecolor{currentstroke}%
\pgfsetdash{}{0pt}%
\pgfpathmoveto{\pgfqpoint{7.105882in}{5.280000in}}%
\pgfpathlineto{\pgfqpoint{11.482353in}{5.280000in}}%
\pgfusepath{stroke}%
\end{pgfscope}%
\begin{pgfscope}%
\pgfsetbuttcap%
\pgfsetmiterjoin%
\definecolor{currentfill}{rgb}{1.000000,1.000000,1.000000}%
\pgfsetfillcolor{currentfill}%
\pgfsetfillopacity{0.800000}%
\pgfsetlinewidth{1.003750pt}%
\definecolor{currentstroke}{rgb}{0.800000,0.800000,0.800000}%
\pgfsetstrokecolor{currentstroke}%
\pgfsetstrokeopacity{0.800000}%
\pgfsetdash{}{0pt}%
\pgfpathmoveto{\pgfqpoint{7.203105in}{4.395760in}}%
\pgfpathlineto{\pgfqpoint{7.944617in}{4.395760in}}%
\pgfpathquadraticcurveto{\pgfqpoint{7.972395in}{4.395760in}}{\pgfqpoint{7.972395in}{4.423538in}}%
\pgfpathlineto{\pgfqpoint{7.972395in}{5.007251in}}%
\pgfpathquadraticcurveto{\pgfqpoint{7.972395in}{5.035029in}}{\pgfqpoint{7.944617in}{5.035029in}}%
\pgfpathlineto{\pgfqpoint{7.203105in}{5.035029in}}%
\pgfpathquadraticcurveto{\pgfqpoint{7.175327in}{5.035029in}}{\pgfqpoint{7.175327in}{5.007251in}}%
\pgfpathlineto{\pgfqpoint{7.175327in}{4.423538in}}%
\pgfpathquadraticcurveto{\pgfqpoint{7.175327in}{4.395760in}}{\pgfqpoint{7.203105in}{4.395760in}}%
\pgfpathclose%
\pgfusepath{stroke,fill}%
\end{pgfscope}%
\begin{pgfscope}%
\pgfsetrectcap%
\pgfsetroundjoin%
\pgfsetlinewidth{1.505625pt}%
\definecolor{currentstroke}{rgb}{0.121569,0.466667,0.705882}%
\pgfsetstrokecolor{currentstroke}%
\pgfsetdash{}{0pt}%
\pgfpathmoveto{\pgfqpoint{7.230882in}{4.922371in}}%
\pgfpathlineto{\pgfqpoint{7.508660in}{4.922371in}}%
\pgfusepath{stroke}%
\end{pgfscope}%
\begin{pgfscope}%
\pgftext[x=7.619771in,y=4.873760in,left,base]{\rmfamily\fontsize{10.000000}{12.000000}\selectfont \(\displaystyle \widetilde{\Phi}^* \theta^{\parallel}\)}%
\end{pgfscope}%
\begin{pgfscope}%
\pgfsetbuttcap%
\pgfsetroundjoin%
\definecolor{currentfill}{rgb}{1.000000,0.000000,0.000000}%
\pgfsetfillcolor{currentfill}%
\pgfsetlinewidth{2.007500pt}%
\definecolor{currentstroke}{rgb}{1.000000,0.000000,0.000000}%
\pgfsetstrokecolor{currentstroke}%
\pgfsetdash{}{0pt}%
\pgfpathmoveto{\pgfqpoint{7.328105in}{4.713848in}}%
\pgfpathlineto{\pgfqpoint{7.411438in}{4.713848in}}%
\pgfpathmoveto{\pgfqpoint{7.369771in}{4.672181in}}%
\pgfpathlineto{\pgfqpoint{7.369771in}{4.755515in}}%
\pgfusepath{stroke,fill}%
\end{pgfscope}%
\begin{pgfscope}%
\pgftext[x=7.619771in,y=4.677390in,left,base]{\rmfamily\fontsize{10.000000}{12.000000}\selectfont train}%
\end{pgfscope}%
\begin{pgfscope}%
\pgfsetbuttcap%
\pgfsetroundjoin%
\definecolor{currentfill}{rgb}{0.000000,0.000000,0.000000}%
\pgfsetfillcolor{currentfill}%
\pgfsetlinewidth{1.003750pt}%
\definecolor{currentstroke}{rgb}{0.000000,0.000000,0.000000}%
\pgfsetstrokecolor{currentstroke}%
\pgfsetdash{}{0pt}%
\pgfsys@defobject{currentmarker}{\pgfqpoint{-0.020833in}{-0.020833in}}{\pgfqpoint{0.020833in}{0.020833in}}{%
\pgfpathmoveto{\pgfqpoint{0.000000in}{-0.020833in}}%
\pgfpathcurveto{\pgfqpoint{0.005525in}{-0.020833in}}{\pgfqpoint{0.010825in}{-0.018638in}}{\pgfqpoint{0.014731in}{-0.014731in}}%
\pgfpathcurveto{\pgfqpoint{0.018638in}{-0.010825in}}{\pgfqpoint{0.020833in}{-0.005525in}}{\pgfqpoint{0.020833in}{0.000000in}}%
\pgfpathcurveto{\pgfqpoint{0.020833in}{0.005525in}}{\pgfqpoint{0.018638in}{0.010825in}}{\pgfqpoint{0.014731in}{0.014731in}}%
\pgfpathcurveto{\pgfqpoint{0.010825in}{0.018638in}}{\pgfqpoint{0.005525in}{0.020833in}}{\pgfqpoint{0.000000in}{0.020833in}}%
\pgfpathcurveto{\pgfqpoint{-0.005525in}{0.020833in}}{\pgfqpoint{-0.010825in}{0.018638in}}{\pgfqpoint{-0.014731in}{0.014731in}}%
\pgfpathcurveto{\pgfqpoint{-0.018638in}{0.010825in}}{\pgfqpoint{-0.020833in}{0.005525in}}{\pgfqpoint{-0.020833in}{0.000000in}}%
\pgfpathcurveto{\pgfqpoint{-0.020833in}{-0.005525in}}{\pgfqpoint{-0.018638in}{-0.010825in}}{\pgfqpoint{-0.014731in}{-0.014731in}}%
\pgfpathcurveto{\pgfqpoint{-0.010825in}{-0.018638in}}{\pgfqpoint{-0.005525in}{-0.020833in}}{\pgfqpoint{0.000000in}{-0.020833in}}%
\pgfpathclose%
\pgfusepath{stroke,fill}%
}%
\begin{pgfscope}%
\pgfsys@transformshift{7.369771in}{4.517478in}%
\pgfsys@useobject{currentmarker}{}%
\end{pgfscope}%
\end{pgfscope}%
\begin{pgfscope}%
\pgftext[x=7.619771in,y=4.481019in,left,base]{\rmfamily\fontsize{10.000000}{12.000000}\selectfont test}%
\end{pgfscope}%
\begin{pgfscope}%
\pgfsetbuttcap%
\pgfsetmiterjoin%
\definecolor{currentfill}{rgb}{1.000000,1.000000,1.000000}%
\pgfsetfillcolor{currentfill}%
\pgfsetlinewidth{0.000000pt}%
\definecolor{currentstroke}{rgb}{0.000000,0.000000,0.000000}%
\pgfsetstrokecolor{currentstroke}%
\pgfsetstrokeopacity{0.000000}%
\pgfsetdash{}{0pt}%
\pgfpathmoveto{\pgfqpoint{12.211765in}{4.326316in}}%
\pgfpathlineto{\pgfqpoint{14.400000in}{4.326316in}}%
\pgfpathlineto{\pgfqpoint{14.400000in}{5.280000in}}%
\pgfpathlineto{\pgfqpoint{12.211765in}{5.280000in}}%
\pgfpathclose%
\pgfusepath{fill}%
\end{pgfscope}%
\begin{pgfscope}%
\pgfpathrectangle{\pgfqpoint{12.211765in}{4.326316in}}{\pgfqpoint{2.188235in}{0.953684in}} %
\pgfusepath{clip}%
\pgfsetbuttcap%
\pgfsetmiterjoin%
\definecolor{currentfill}{rgb}{0.121569,0.466667,0.705882}%
\pgfsetfillcolor{currentfill}%
\pgfsetlinewidth{0.000000pt}%
\definecolor{currentstroke}{rgb}{0.000000,0.000000,0.000000}%
\pgfsetstrokecolor{currentstroke}%
\pgfsetstrokeopacity{0.000000}%
\pgfsetdash{}{0pt}%
\pgfpathmoveto{\pgfqpoint{-23.809001in}{4.369665in}}%
\pgfpathlineto{\pgfqpoint{12.638302in}{4.369665in}}%
\pgfpathlineto{\pgfqpoint{12.638302in}{4.376615in}}%
\pgfpathlineto{\pgfqpoint{-23.809001in}{4.376615in}}%
\pgfpathclose%
\pgfusepath{fill}%
\end{pgfscope}%
\begin{pgfscope}%
\pgfpathrectangle{\pgfqpoint{12.211765in}{4.326316in}}{\pgfqpoint{2.188235in}{0.953684in}} %
\pgfusepath{clip}%
\pgfsetbuttcap%
\pgfsetmiterjoin%
\definecolor{currentfill}{rgb}{0.121569,0.466667,0.705882}%
\pgfsetfillcolor{currentfill}%
\pgfsetlinewidth{0.000000pt}%
\definecolor{currentstroke}{rgb}{0.000000,0.000000,0.000000}%
\pgfsetstrokecolor{currentstroke}%
\pgfsetstrokeopacity{0.000000}%
\pgfsetdash{}{0pt}%
\pgfpathmoveto{\pgfqpoint{-23.809001in}{4.378352in}}%
\pgfpathlineto{\pgfqpoint{12.670291in}{4.378352in}}%
\pgfpathlineto{\pgfqpoint{12.670291in}{4.385302in}}%
\pgfpathlineto{\pgfqpoint{-23.809001in}{4.385302in}}%
\pgfpathclose%
\pgfusepath{fill}%
\end{pgfscope}%
\begin{pgfscope}%
\pgfpathrectangle{\pgfqpoint{12.211765in}{4.326316in}}{\pgfqpoint{2.188235in}{0.953684in}} %
\pgfusepath{clip}%
\pgfsetbuttcap%
\pgfsetmiterjoin%
\definecolor{currentfill}{rgb}{0.121569,0.466667,0.705882}%
\pgfsetfillcolor{currentfill}%
\pgfsetlinewidth{0.000000pt}%
\definecolor{currentstroke}{rgb}{0.000000,0.000000,0.000000}%
\pgfsetstrokecolor{currentstroke}%
\pgfsetstrokeopacity{0.000000}%
\pgfsetdash{}{0pt}%
\pgfpathmoveto{\pgfqpoint{-23.809001in}{4.387040in}}%
\pgfpathlineto{\pgfqpoint{12.657891in}{4.387040in}}%
\pgfpathlineto{\pgfqpoint{12.657891in}{4.393989in}}%
\pgfpathlineto{\pgfqpoint{-23.809001in}{4.393989in}}%
\pgfpathclose%
\pgfusepath{fill}%
\end{pgfscope}%
\begin{pgfscope}%
\pgfpathrectangle{\pgfqpoint{12.211765in}{4.326316in}}{\pgfqpoint{2.188235in}{0.953684in}} %
\pgfusepath{clip}%
\pgfsetbuttcap%
\pgfsetmiterjoin%
\definecolor{currentfill}{rgb}{0.121569,0.466667,0.705882}%
\pgfsetfillcolor{currentfill}%
\pgfsetlinewidth{0.000000pt}%
\definecolor{currentstroke}{rgb}{0.000000,0.000000,0.000000}%
\pgfsetstrokecolor{currentstroke}%
\pgfsetstrokeopacity{0.000000}%
\pgfsetdash{}{0pt}%
\pgfpathmoveto{\pgfqpoint{-23.809001in}{4.395727in}}%
\pgfpathlineto{\pgfqpoint{12.771244in}{4.395727in}}%
\pgfpathlineto{\pgfqpoint{12.771244in}{4.402677in}}%
\pgfpathlineto{\pgfqpoint{-23.809001in}{4.402677in}}%
\pgfpathclose%
\pgfusepath{fill}%
\end{pgfscope}%
\begin{pgfscope}%
\pgfpathrectangle{\pgfqpoint{12.211765in}{4.326316in}}{\pgfqpoint{2.188235in}{0.953684in}} %
\pgfusepath{clip}%
\pgfsetbuttcap%
\pgfsetmiterjoin%
\definecolor{currentfill}{rgb}{0.121569,0.466667,0.705882}%
\pgfsetfillcolor{currentfill}%
\pgfsetlinewidth{0.000000pt}%
\definecolor{currentstroke}{rgb}{0.000000,0.000000,0.000000}%
\pgfsetstrokecolor{currentstroke}%
\pgfsetstrokeopacity{0.000000}%
\pgfsetdash{}{0pt}%
\pgfpathmoveto{\pgfqpoint{-23.809001in}{4.404414in}}%
\pgfpathlineto{\pgfqpoint{12.723953in}{4.404414in}}%
\pgfpathlineto{\pgfqpoint{12.723953in}{4.411364in}}%
\pgfpathlineto{\pgfqpoint{-23.809001in}{4.411364in}}%
\pgfpathclose%
\pgfusepath{fill}%
\end{pgfscope}%
\begin{pgfscope}%
\pgfpathrectangle{\pgfqpoint{12.211765in}{4.326316in}}{\pgfqpoint{2.188235in}{0.953684in}} %
\pgfusepath{clip}%
\pgfsetbuttcap%
\pgfsetmiterjoin%
\definecolor{currentfill}{rgb}{0.121569,0.466667,0.705882}%
\pgfsetfillcolor{currentfill}%
\pgfsetlinewidth{0.000000pt}%
\definecolor{currentstroke}{rgb}{0.000000,0.000000,0.000000}%
\pgfsetstrokecolor{currentstroke}%
\pgfsetstrokeopacity{0.000000}%
\pgfsetdash{}{0pt}%
\pgfpathmoveto{\pgfqpoint{-23.809001in}{4.413101in}}%
\pgfpathlineto{\pgfqpoint{12.600497in}{4.413101in}}%
\pgfpathlineto{\pgfqpoint{12.600497in}{4.420051in}}%
\pgfpathlineto{\pgfqpoint{-23.809001in}{4.420051in}}%
\pgfpathclose%
\pgfusepath{fill}%
\end{pgfscope}%
\begin{pgfscope}%
\pgfpathrectangle{\pgfqpoint{12.211765in}{4.326316in}}{\pgfqpoint{2.188235in}{0.953684in}} %
\pgfusepath{clip}%
\pgfsetbuttcap%
\pgfsetmiterjoin%
\definecolor{currentfill}{rgb}{0.121569,0.466667,0.705882}%
\pgfsetfillcolor{currentfill}%
\pgfsetlinewidth{0.000000pt}%
\definecolor{currentstroke}{rgb}{0.000000,0.000000,0.000000}%
\pgfsetstrokecolor{currentstroke}%
\pgfsetstrokeopacity{0.000000}%
\pgfsetdash{}{0pt}%
\pgfpathmoveto{\pgfqpoint{-23.809001in}{4.421788in}}%
\pgfpathlineto{\pgfqpoint{12.623777in}{4.421788in}}%
\pgfpathlineto{\pgfqpoint{12.623777in}{4.428738in}}%
\pgfpathlineto{\pgfqpoint{-23.809001in}{4.428738in}}%
\pgfpathclose%
\pgfusepath{fill}%
\end{pgfscope}%
\begin{pgfscope}%
\pgfpathrectangle{\pgfqpoint{12.211765in}{4.326316in}}{\pgfqpoint{2.188235in}{0.953684in}} %
\pgfusepath{clip}%
\pgfsetbuttcap%
\pgfsetmiterjoin%
\definecolor{currentfill}{rgb}{0.121569,0.466667,0.705882}%
\pgfsetfillcolor{currentfill}%
\pgfsetlinewidth{0.000000pt}%
\definecolor{currentstroke}{rgb}{0.000000,0.000000,0.000000}%
\pgfsetstrokecolor{currentstroke}%
\pgfsetstrokeopacity{0.000000}%
\pgfsetdash{}{0pt}%
\pgfpathmoveto{\pgfqpoint{-23.809001in}{4.430476in}}%
\pgfpathlineto{\pgfqpoint{12.602927in}{4.430476in}}%
\pgfpathlineto{\pgfqpoint{12.602927in}{4.437425in}}%
\pgfpathlineto{\pgfqpoint{-23.809001in}{4.437425in}}%
\pgfpathclose%
\pgfusepath{fill}%
\end{pgfscope}%
\begin{pgfscope}%
\pgfpathrectangle{\pgfqpoint{12.211765in}{4.326316in}}{\pgfqpoint{2.188235in}{0.953684in}} %
\pgfusepath{clip}%
\pgfsetbuttcap%
\pgfsetmiterjoin%
\definecolor{currentfill}{rgb}{0.121569,0.466667,0.705882}%
\pgfsetfillcolor{currentfill}%
\pgfsetlinewidth{0.000000pt}%
\definecolor{currentstroke}{rgb}{0.000000,0.000000,0.000000}%
\pgfsetstrokecolor{currentstroke}%
\pgfsetstrokeopacity{0.000000}%
\pgfsetdash{}{0pt}%
\pgfpathmoveto{\pgfqpoint{-23.809001in}{4.439163in}}%
\pgfpathlineto{\pgfqpoint{12.688182in}{4.439163in}}%
\pgfpathlineto{\pgfqpoint{12.688182in}{4.446113in}}%
\pgfpathlineto{\pgfqpoint{-23.809001in}{4.446113in}}%
\pgfpathclose%
\pgfusepath{fill}%
\end{pgfscope}%
\begin{pgfscope}%
\pgfpathrectangle{\pgfqpoint{12.211765in}{4.326316in}}{\pgfqpoint{2.188235in}{0.953684in}} %
\pgfusepath{clip}%
\pgfsetbuttcap%
\pgfsetmiterjoin%
\definecolor{currentfill}{rgb}{0.121569,0.466667,0.705882}%
\pgfsetfillcolor{currentfill}%
\pgfsetlinewidth{0.000000pt}%
\definecolor{currentstroke}{rgb}{0.000000,0.000000,0.000000}%
\pgfsetstrokecolor{currentstroke}%
\pgfsetstrokeopacity{0.000000}%
\pgfsetdash{}{0pt}%
\pgfpathmoveto{\pgfqpoint{-23.809001in}{4.447850in}}%
\pgfpathlineto{\pgfqpoint{12.623983in}{4.447850in}}%
\pgfpathlineto{\pgfqpoint{12.623983in}{4.454800in}}%
\pgfpathlineto{\pgfqpoint{-23.809001in}{4.454800in}}%
\pgfpathclose%
\pgfusepath{fill}%
\end{pgfscope}%
\begin{pgfscope}%
\pgfpathrectangle{\pgfqpoint{12.211765in}{4.326316in}}{\pgfqpoint{2.188235in}{0.953684in}} %
\pgfusepath{clip}%
\pgfsetbuttcap%
\pgfsetmiterjoin%
\definecolor{currentfill}{rgb}{0.121569,0.466667,0.705882}%
\pgfsetfillcolor{currentfill}%
\pgfsetlinewidth{0.000000pt}%
\definecolor{currentstroke}{rgb}{0.000000,0.000000,0.000000}%
\pgfsetstrokecolor{currentstroke}%
\pgfsetstrokeopacity{0.000000}%
\pgfsetdash{}{0pt}%
\pgfpathmoveto{\pgfqpoint{-23.809001in}{4.456537in}}%
\pgfpathlineto{\pgfqpoint{12.672205in}{4.456537in}}%
\pgfpathlineto{\pgfqpoint{12.672205in}{4.463487in}}%
\pgfpathlineto{\pgfqpoint{-23.809001in}{4.463487in}}%
\pgfpathclose%
\pgfusepath{fill}%
\end{pgfscope}%
\begin{pgfscope}%
\pgfpathrectangle{\pgfqpoint{12.211765in}{4.326316in}}{\pgfqpoint{2.188235in}{0.953684in}} %
\pgfusepath{clip}%
\pgfsetbuttcap%
\pgfsetmiterjoin%
\definecolor{currentfill}{rgb}{0.121569,0.466667,0.705882}%
\pgfsetfillcolor{currentfill}%
\pgfsetlinewidth{0.000000pt}%
\definecolor{currentstroke}{rgb}{0.000000,0.000000,0.000000}%
\pgfsetstrokecolor{currentstroke}%
\pgfsetstrokeopacity{0.000000}%
\pgfsetdash{}{0pt}%
\pgfpathmoveto{\pgfqpoint{-23.809001in}{4.465225in}}%
\pgfpathlineto{\pgfqpoint{12.577466in}{4.465225in}}%
\pgfpathlineto{\pgfqpoint{12.577466in}{4.472174in}}%
\pgfpathlineto{\pgfqpoint{-23.809001in}{4.472174in}}%
\pgfpathclose%
\pgfusepath{fill}%
\end{pgfscope}%
\begin{pgfscope}%
\pgfpathrectangle{\pgfqpoint{12.211765in}{4.326316in}}{\pgfqpoint{2.188235in}{0.953684in}} %
\pgfusepath{clip}%
\pgfsetbuttcap%
\pgfsetmiterjoin%
\definecolor{currentfill}{rgb}{0.121569,0.466667,0.705882}%
\pgfsetfillcolor{currentfill}%
\pgfsetlinewidth{0.000000pt}%
\definecolor{currentstroke}{rgb}{0.000000,0.000000,0.000000}%
\pgfsetstrokecolor{currentstroke}%
\pgfsetstrokeopacity{0.000000}%
\pgfsetdash{}{0pt}%
\pgfpathmoveto{\pgfqpoint{-23.809001in}{4.473912in}}%
\pgfpathlineto{\pgfqpoint{12.591751in}{4.473912in}}%
\pgfpathlineto{\pgfqpoint{12.591751in}{4.480862in}}%
\pgfpathlineto{\pgfqpoint{-23.809001in}{4.480862in}}%
\pgfpathclose%
\pgfusepath{fill}%
\end{pgfscope}%
\begin{pgfscope}%
\pgfpathrectangle{\pgfqpoint{12.211765in}{4.326316in}}{\pgfqpoint{2.188235in}{0.953684in}} %
\pgfusepath{clip}%
\pgfsetbuttcap%
\pgfsetmiterjoin%
\definecolor{currentfill}{rgb}{0.121569,0.466667,0.705882}%
\pgfsetfillcolor{currentfill}%
\pgfsetlinewidth{0.000000pt}%
\definecolor{currentstroke}{rgb}{0.000000,0.000000,0.000000}%
\pgfsetstrokecolor{currentstroke}%
\pgfsetstrokeopacity{0.000000}%
\pgfsetdash{}{0pt}%
\pgfpathmoveto{\pgfqpoint{-23.809001in}{4.482599in}}%
\pgfpathlineto{\pgfqpoint{12.597080in}{4.482599in}}%
\pgfpathlineto{\pgfqpoint{12.597080in}{4.489549in}}%
\pgfpathlineto{\pgfqpoint{-23.809001in}{4.489549in}}%
\pgfpathclose%
\pgfusepath{fill}%
\end{pgfscope}%
\begin{pgfscope}%
\pgfpathrectangle{\pgfqpoint{12.211765in}{4.326316in}}{\pgfqpoint{2.188235in}{0.953684in}} %
\pgfusepath{clip}%
\pgfsetbuttcap%
\pgfsetmiterjoin%
\definecolor{currentfill}{rgb}{0.121569,0.466667,0.705882}%
\pgfsetfillcolor{currentfill}%
\pgfsetlinewidth{0.000000pt}%
\definecolor{currentstroke}{rgb}{0.000000,0.000000,0.000000}%
\pgfsetstrokecolor{currentstroke}%
\pgfsetstrokeopacity{0.000000}%
\pgfsetdash{}{0pt}%
\pgfpathmoveto{\pgfqpoint{-23.809001in}{4.491286in}}%
\pgfpathlineto{\pgfqpoint{12.644233in}{4.491286in}}%
\pgfpathlineto{\pgfqpoint{12.644233in}{4.498236in}}%
\pgfpathlineto{\pgfqpoint{-23.809001in}{4.498236in}}%
\pgfpathclose%
\pgfusepath{fill}%
\end{pgfscope}%
\begin{pgfscope}%
\pgfpathrectangle{\pgfqpoint{12.211765in}{4.326316in}}{\pgfqpoint{2.188235in}{0.953684in}} %
\pgfusepath{clip}%
\pgfsetbuttcap%
\pgfsetmiterjoin%
\definecolor{currentfill}{rgb}{0.121569,0.466667,0.705882}%
\pgfsetfillcolor{currentfill}%
\pgfsetlinewidth{0.000000pt}%
\definecolor{currentstroke}{rgb}{0.000000,0.000000,0.000000}%
\pgfsetstrokecolor{currentstroke}%
\pgfsetstrokeopacity{0.000000}%
\pgfsetdash{}{0pt}%
\pgfpathmoveto{\pgfqpoint{-23.809001in}{4.499974in}}%
\pgfpathlineto{\pgfqpoint{12.686162in}{4.499974in}}%
\pgfpathlineto{\pgfqpoint{12.686162in}{4.506923in}}%
\pgfpathlineto{\pgfqpoint{-23.809001in}{4.506923in}}%
\pgfpathclose%
\pgfusepath{fill}%
\end{pgfscope}%
\begin{pgfscope}%
\pgfpathrectangle{\pgfqpoint{12.211765in}{4.326316in}}{\pgfqpoint{2.188235in}{0.953684in}} %
\pgfusepath{clip}%
\pgfsetbuttcap%
\pgfsetmiterjoin%
\definecolor{currentfill}{rgb}{0.121569,0.466667,0.705882}%
\pgfsetfillcolor{currentfill}%
\pgfsetlinewidth{0.000000pt}%
\definecolor{currentstroke}{rgb}{0.000000,0.000000,0.000000}%
\pgfsetstrokecolor{currentstroke}%
\pgfsetstrokeopacity{0.000000}%
\pgfsetdash{}{0pt}%
\pgfpathmoveto{\pgfqpoint{-23.809001in}{4.508661in}}%
\pgfpathlineto{\pgfqpoint{12.645010in}{4.508661in}}%
\pgfpathlineto{\pgfqpoint{12.645010in}{4.515611in}}%
\pgfpathlineto{\pgfqpoint{-23.809001in}{4.515611in}}%
\pgfpathclose%
\pgfusepath{fill}%
\end{pgfscope}%
\begin{pgfscope}%
\pgfpathrectangle{\pgfqpoint{12.211765in}{4.326316in}}{\pgfqpoint{2.188235in}{0.953684in}} %
\pgfusepath{clip}%
\pgfsetbuttcap%
\pgfsetmiterjoin%
\definecolor{currentfill}{rgb}{0.121569,0.466667,0.705882}%
\pgfsetfillcolor{currentfill}%
\pgfsetlinewidth{0.000000pt}%
\definecolor{currentstroke}{rgb}{0.000000,0.000000,0.000000}%
\pgfsetstrokecolor{currentstroke}%
\pgfsetstrokeopacity{0.000000}%
\pgfsetdash{}{0pt}%
\pgfpathmoveto{\pgfqpoint{-23.809001in}{4.517348in}}%
\pgfpathlineto{\pgfqpoint{12.593481in}{4.517348in}}%
\pgfpathlineto{\pgfqpoint{12.593481in}{4.524298in}}%
\pgfpathlineto{\pgfqpoint{-23.809001in}{4.524298in}}%
\pgfpathclose%
\pgfusepath{fill}%
\end{pgfscope}%
\begin{pgfscope}%
\pgfpathrectangle{\pgfqpoint{12.211765in}{4.326316in}}{\pgfqpoint{2.188235in}{0.953684in}} %
\pgfusepath{clip}%
\pgfsetbuttcap%
\pgfsetmiterjoin%
\definecolor{currentfill}{rgb}{0.121569,0.466667,0.705882}%
\pgfsetfillcolor{currentfill}%
\pgfsetlinewidth{0.000000pt}%
\definecolor{currentstroke}{rgb}{0.000000,0.000000,0.000000}%
\pgfsetstrokecolor{currentstroke}%
\pgfsetstrokeopacity{0.000000}%
\pgfsetdash{}{0pt}%
\pgfpathmoveto{\pgfqpoint{-23.809001in}{4.526035in}}%
\pgfpathlineto{\pgfqpoint{12.611665in}{4.526035in}}%
\pgfpathlineto{\pgfqpoint{12.611665in}{4.532985in}}%
\pgfpathlineto{\pgfqpoint{-23.809001in}{4.532985in}}%
\pgfpathclose%
\pgfusepath{fill}%
\end{pgfscope}%
\begin{pgfscope}%
\pgfpathrectangle{\pgfqpoint{12.211765in}{4.326316in}}{\pgfqpoint{2.188235in}{0.953684in}} %
\pgfusepath{clip}%
\pgfsetbuttcap%
\pgfsetmiterjoin%
\definecolor{currentfill}{rgb}{0.121569,0.466667,0.705882}%
\pgfsetfillcolor{currentfill}%
\pgfsetlinewidth{0.000000pt}%
\definecolor{currentstroke}{rgb}{0.000000,0.000000,0.000000}%
\pgfsetstrokecolor{currentstroke}%
\pgfsetstrokeopacity{0.000000}%
\pgfsetdash{}{0pt}%
\pgfpathmoveto{\pgfqpoint{-23.809001in}{4.534722in}}%
\pgfpathlineto{\pgfqpoint{12.574762in}{4.534722in}}%
\pgfpathlineto{\pgfqpoint{12.574762in}{4.541672in}}%
\pgfpathlineto{\pgfqpoint{-23.809001in}{4.541672in}}%
\pgfpathclose%
\pgfusepath{fill}%
\end{pgfscope}%
\begin{pgfscope}%
\pgfpathrectangle{\pgfqpoint{12.211765in}{4.326316in}}{\pgfqpoint{2.188235in}{0.953684in}} %
\pgfusepath{clip}%
\pgfsetbuttcap%
\pgfsetmiterjoin%
\definecolor{currentfill}{rgb}{0.121569,0.466667,0.705882}%
\pgfsetfillcolor{currentfill}%
\pgfsetlinewidth{0.000000pt}%
\definecolor{currentstroke}{rgb}{0.000000,0.000000,0.000000}%
\pgfsetstrokecolor{currentstroke}%
\pgfsetstrokeopacity{0.000000}%
\pgfsetdash{}{0pt}%
\pgfpathmoveto{\pgfqpoint{-23.809001in}{4.543410in}}%
\pgfpathlineto{\pgfqpoint{12.704792in}{4.543410in}}%
\pgfpathlineto{\pgfqpoint{12.704792in}{4.550359in}}%
\pgfpathlineto{\pgfqpoint{-23.809001in}{4.550359in}}%
\pgfpathclose%
\pgfusepath{fill}%
\end{pgfscope}%
\begin{pgfscope}%
\pgfpathrectangle{\pgfqpoint{12.211765in}{4.326316in}}{\pgfqpoint{2.188235in}{0.953684in}} %
\pgfusepath{clip}%
\pgfsetbuttcap%
\pgfsetmiterjoin%
\definecolor{currentfill}{rgb}{0.121569,0.466667,0.705882}%
\pgfsetfillcolor{currentfill}%
\pgfsetlinewidth{0.000000pt}%
\definecolor{currentstroke}{rgb}{0.000000,0.000000,0.000000}%
\pgfsetstrokecolor{currentstroke}%
\pgfsetstrokeopacity{0.000000}%
\pgfsetdash{}{0pt}%
\pgfpathmoveto{\pgfqpoint{-23.809001in}{4.552097in}}%
\pgfpathlineto{\pgfqpoint{12.654904in}{4.552097in}}%
\pgfpathlineto{\pgfqpoint{12.654904in}{4.559047in}}%
\pgfpathlineto{\pgfqpoint{-23.809001in}{4.559047in}}%
\pgfpathclose%
\pgfusepath{fill}%
\end{pgfscope}%
\begin{pgfscope}%
\pgfpathrectangle{\pgfqpoint{12.211765in}{4.326316in}}{\pgfqpoint{2.188235in}{0.953684in}} %
\pgfusepath{clip}%
\pgfsetbuttcap%
\pgfsetmiterjoin%
\definecolor{currentfill}{rgb}{0.121569,0.466667,0.705882}%
\pgfsetfillcolor{currentfill}%
\pgfsetlinewidth{0.000000pt}%
\definecolor{currentstroke}{rgb}{0.000000,0.000000,0.000000}%
\pgfsetstrokecolor{currentstroke}%
\pgfsetstrokeopacity{0.000000}%
\pgfsetdash{}{0pt}%
\pgfpathmoveto{\pgfqpoint{-23.809001in}{4.560784in}}%
\pgfpathlineto{\pgfqpoint{12.581065in}{4.560784in}}%
\pgfpathlineto{\pgfqpoint{12.581065in}{4.567734in}}%
\pgfpathlineto{\pgfqpoint{-23.809001in}{4.567734in}}%
\pgfpathclose%
\pgfusepath{fill}%
\end{pgfscope}%
\begin{pgfscope}%
\pgfpathrectangle{\pgfqpoint{12.211765in}{4.326316in}}{\pgfqpoint{2.188235in}{0.953684in}} %
\pgfusepath{clip}%
\pgfsetbuttcap%
\pgfsetmiterjoin%
\definecolor{currentfill}{rgb}{0.121569,0.466667,0.705882}%
\pgfsetfillcolor{currentfill}%
\pgfsetlinewidth{0.000000pt}%
\definecolor{currentstroke}{rgb}{0.000000,0.000000,0.000000}%
\pgfsetstrokecolor{currentstroke}%
\pgfsetstrokeopacity{0.000000}%
\pgfsetdash{}{0pt}%
\pgfpathmoveto{\pgfqpoint{-23.809001in}{4.569471in}}%
\pgfpathlineto{\pgfqpoint{12.593928in}{4.569471in}}%
\pgfpathlineto{\pgfqpoint{12.593928in}{4.576421in}}%
\pgfpathlineto{\pgfqpoint{-23.809001in}{4.576421in}}%
\pgfpathclose%
\pgfusepath{fill}%
\end{pgfscope}%
\begin{pgfscope}%
\pgfpathrectangle{\pgfqpoint{12.211765in}{4.326316in}}{\pgfqpoint{2.188235in}{0.953684in}} %
\pgfusepath{clip}%
\pgfsetbuttcap%
\pgfsetmiterjoin%
\definecolor{currentfill}{rgb}{0.121569,0.466667,0.705882}%
\pgfsetfillcolor{currentfill}%
\pgfsetlinewidth{0.000000pt}%
\definecolor{currentstroke}{rgb}{0.000000,0.000000,0.000000}%
\pgfsetstrokecolor{currentstroke}%
\pgfsetstrokeopacity{0.000000}%
\pgfsetdash{}{0pt}%
\pgfpathmoveto{\pgfqpoint{-23.809001in}{4.578159in}}%
\pgfpathlineto{\pgfqpoint{12.480214in}{4.578159in}}%
\pgfpathlineto{\pgfqpoint{12.480214in}{4.585108in}}%
\pgfpathlineto{\pgfqpoint{-23.809001in}{4.585108in}}%
\pgfpathclose%
\pgfusepath{fill}%
\end{pgfscope}%
\begin{pgfscope}%
\pgfpathrectangle{\pgfqpoint{12.211765in}{4.326316in}}{\pgfqpoint{2.188235in}{0.953684in}} %
\pgfusepath{clip}%
\pgfsetbuttcap%
\pgfsetmiterjoin%
\definecolor{currentfill}{rgb}{0.121569,0.466667,0.705882}%
\pgfsetfillcolor{currentfill}%
\pgfsetlinewidth{0.000000pt}%
\definecolor{currentstroke}{rgb}{0.000000,0.000000,0.000000}%
\pgfsetstrokecolor{currentstroke}%
\pgfsetstrokeopacity{0.000000}%
\pgfsetdash{}{0pt}%
\pgfpathmoveto{\pgfqpoint{-23.809001in}{4.586846in}}%
\pgfpathlineto{\pgfqpoint{12.603714in}{4.586846in}}%
\pgfpathlineto{\pgfqpoint{12.603714in}{4.593796in}}%
\pgfpathlineto{\pgfqpoint{-23.809001in}{4.593796in}}%
\pgfpathclose%
\pgfusepath{fill}%
\end{pgfscope}%
\begin{pgfscope}%
\pgfpathrectangle{\pgfqpoint{12.211765in}{4.326316in}}{\pgfqpoint{2.188235in}{0.953684in}} %
\pgfusepath{clip}%
\pgfsetbuttcap%
\pgfsetmiterjoin%
\definecolor{currentfill}{rgb}{0.121569,0.466667,0.705882}%
\pgfsetfillcolor{currentfill}%
\pgfsetlinewidth{0.000000pt}%
\definecolor{currentstroke}{rgb}{0.000000,0.000000,0.000000}%
\pgfsetstrokecolor{currentstroke}%
\pgfsetstrokeopacity{0.000000}%
\pgfsetdash{}{0pt}%
\pgfpathmoveto{\pgfqpoint{-23.809001in}{4.595533in}}%
\pgfpathlineto{\pgfqpoint{12.571917in}{4.595533in}}%
\pgfpathlineto{\pgfqpoint{12.571917in}{4.602483in}}%
\pgfpathlineto{\pgfqpoint{-23.809001in}{4.602483in}}%
\pgfpathclose%
\pgfusepath{fill}%
\end{pgfscope}%
\begin{pgfscope}%
\pgfpathrectangle{\pgfqpoint{12.211765in}{4.326316in}}{\pgfqpoint{2.188235in}{0.953684in}} %
\pgfusepath{clip}%
\pgfsetbuttcap%
\pgfsetmiterjoin%
\definecolor{currentfill}{rgb}{0.121569,0.466667,0.705882}%
\pgfsetfillcolor{currentfill}%
\pgfsetlinewidth{0.000000pt}%
\definecolor{currentstroke}{rgb}{0.000000,0.000000,0.000000}%
\pgfsetstrokecolor{currentstroke}%
\pgfsetstrokeopacity{0.000000}%
\pgfsetdash{}{0pt}%
\pgfpathmoveto{\pgfqpoint{-23.809001in}{4.604220in}}%
\pgfpathlineto{\pgfqpoint{12.667052in}{4.604220in}}%
\pgfpathlineto{\pgfqpoint{12.667052in}{4.611170in}}%
\pgfpathlineto{\pgfqpoint{-23.809001in}{4.611170in}}%
\pgfpathclose%
\pgfusepath{fill}%
\end{pgfscope}%
\begin{pgfscope}%
\pgfpathrectangle{\pgfqpoint{12.211765in}{4.326316in}}{\pgfqpoint{2.188235in}{0.953684in}} %
\pgfusepath{clip}%
\pgfsetbuttcap%
\pgfsetmiterjoin%
\definecolor{currentfill}{rgb}{0.121569,0.466667,0.705882}%
\pgfsetfillcolor{currentfill}%
\pgfsetlinewidth{0.000000pt}%
\definecolor{currentstroke}{rgb}{0.000000,0.000000,0.000000}%
\pgfsetstrokecolor{currentstroke}%
\pgfsetstrokeopacity{0.000000}%
\pgfsetdash{}{0pt}%
\pgfpathmoveto{\pgfqpoint{-23.809001in}{4.612908in}}%
\pgfpathlineto{\pgfqpoint{12.421929in}{4.612908in}}%
\pgfpathlineto{\pgfqpoint{12.421929in}{4.619857in}}%
\pgfpathlineto{\pgfqpoint{-23.809001in}{4.619857in}}%
\pgfpathclose%
\pgfusepath{fill}%
\end{pgfscope}%
\begin{pgfscope}%
\pgfpathrectangle{\pgfqpoint{12.211765in}{4.326316in}}{\pgfqpoint{2.188235in}{0.953684in}} %
\pgfusepath{clip}%
\pgfsetbuttcap%
\pgfsetmiterjoin%
\definecolor{currentfill}{rgb}{0.121569,0.466667,0.705882}%
\pgfsetfillcolor{currentfill}%
\pgfsetlinewidth{0.000000pt}%
\definecolor{currentstroke}{rgb}{0.000000,0.000000,0.000000}%
\pgfsetstrokecolor{currentstroke}%
\pgfsetstrokeopacity{0.000000}%
\pgfsetdash{}{0pt}%
\pgfpathmoveto{\pgfqpoint{-23.809001in}{4.621595in}}%
\pgfpathlineto{\pgfqpoint{12.672610in}{4.621595in}}%
\pgfpathlineto{\pgfqpoint{12.672610in}{4.628545in}}%
\pgfpathlineto{\pgfqpoint{-23.809001in}{4.628545in}}%
\pgfpathclose%
\pgfusepath{fill}%
\end{pgfscope}%
\begin{pgfscope}%
\pgfpathrectangle{\pgfqpoint{12.211765in}{4.326316in}}{\pgfqpoint{2.188235in}{0.953684in}} %
\pgfusepath{clip}%
\pgfsetbuttcap%
\pgfsetmiterjoin%
\definecolor{currentfill}{rgb}{0.121569,0.466667,0.705882}%
\pgfsetfillcolor{currentfill}%
\pgfsetlinewidth{0.000000pt}%
\definecolor{currentstroke}{rgb}{0.000000,0.000000,0.000000}%
\pgfsetstrokecolor{currentstroke}%
\pgfsetstrokeopacity{0.000000}%
\pgfsetdash{}{0pt}%
\pgfpathmoveto{\pgfqpoint{-23.809001in}{4.630282in}}%
\pgfpathlineto{\pgfqpoint{12.674831in}{4.630282in}}%
\pgfpathlineto{\pgfqpoint{12.674831in}{4.637232in}}%
\pgfpathlineto{\pgfqpoint{-23.809001in}{4.637232in}}%
\pgfpathclose%
\pgfusepath{fill}%
\end{pgfscope}%
\begin{pgfscope}%
\pgfpathrectangle{\pgfqpoint{12.211765in}{4.326316in}}{\pgfqpoint{2.188235in}{0.953684in}} %
\pgfusepath{clip}%
\pgfsetbuttcap%
\pgfsetmiterjoin%
\definecolor{currentfill}{rgb}{0.121569,0.466667,0.705882}%
\pgfsetfillcolor{currentfill}%
\pgfsetlinewidth{0.000000pt}%
\definecolor{currentstroke}{rgb}{0.000000,0.000000,0.000000}%
\pgfsetstrokecolor{currentstroke}%
\pgfsetstrokeopacity{0.000000}%
\pgfsetdash{}{0pt}%
\pgfpathmoveto{\pgfqpoint{-23.809001in}{4.638969in}}%
\pgfpathlineto{\pgfqpoint{12.602053in}{4.638969in}}%
\pgfpathlineto{\pgfqpoint{12.602053in}{4.645919in}}%
\pgfpathlineto{\pgfqpoint{-23.809001in}{4.645919in}}%
\pgfpathclose%
\pgfusepath{fill}%
\end{pgfscope}%
\begin{pgfscope}%
\pgfpathrectangle{\pgfqpoint{12.211765in}{4.326316in}}{\pgfqpoint{2.188235in}{0.953684in}} %
\pgfusepath{clip}%
\pgfsetbuttcap%
\pgfsetmiterjoin%
\definecolor{currentfill}{rgb}{0.121569,0.466667,0.705882}%
\pgfsetfillcolor{currentfill}%
\pgfsetlinewidth{0.000000pt}%
\definecolor{currentstroke}{rgb}{0.000000,0.000000,0.000000}%
\pgfsetstrokecolor{currentstroke}%
\pgfsetstrokeopacity{0.000000}%
\pgfsetdash{}{0pt}%
\pgfpathmoveto{\pgfqpoint{-23.809001in}{4.647656in}}%
\pgfpathlineto{\pgfqpoint{12.573229in}{4.647656in}}%
\pgfpathlineto{\pgfqpoint{12.573229in}{4.654606in}}%
\pgfpathlineto{\pgfqpoint{-23.809001in}{4.654606in}}%
\pgfpathclose%
\pgfusepath{fill}%
\end{pgfscope}%
\begin{pgfscope}%
\pgfpathrectangle{\pgfqpoint{12.211765in}{4.326316in}}{\pgfqpoint{2.188235in}{0.953684in}} %
\pgfusepath{clip}%
\pgfsetbuttcap%
\pgfsetmiterjoin%
\definecolor{currentfill}{rgb}{0.121569,0.466667,0.705882}%
\pgfsetfillcolor{currentfill}%
\pgfsetlinewidth{0.000000pt}%
\definecolor{currentstroke}{rgb}{0.000000,0.000000,0.000000}%
\pgfsetstrokecolor{currentstroke}%
\pgfsetstrokeopacity{0.000000}%
\pgfsetdash{}{0pt}%
\pgfpathmoveto{\pgfqpoint{-23.809001in}{4.656344in}}%
\pgfpathlineto{\pgfqpoint{12.599200in}{4.656344in}}%
\pgfpathlineto{\pgfqpoint{12.599200in}{4.663293in}}%
\pgfpathlineto{\pgfqpoint{-23.809001in}{4.663293in}}%
\pgfpathclose%
\pgfusepath{fill}%
\end{pgfscope}%
\begin{pgfscope}%
\pgfpathrectangle{\pgfqpoint{12.211765in}{4.326316in}}{\pgfqpoint{2.188235in}{0.953684in}} %
\pgfusepath{clip}%
\pgfsetbuttcap%
\pgfsetmiterjoin%
\definecolor{currentfill}{rgb}{0.121569,0.466667,0.705882}%
\pgfsetfillcolor{currentfill}%
\pgfsetlinewidth{0.000000pt}%
\definecolor{currentstroke}{rgb}{0.000000,0.000000,0.000000}%
\pgfsetstrokecolor{currentstroke}%
\pgfsetstrokeopacity{0.000000}%
\pgfsetdash{}{0pt}%
\pgfpathmoveto{\pgfqpoint{-23.809001in}{4.665031in}}%
\pgfpathlineto{\pgfqpoint{12.654537in}{4.665031in}}%
\pgfpathlineto{\pgfqpoint{12.654537in}{4.671981in}}%
\pgfpathlineto{\pgfqpoint{-23.809001in}{4.671981in}}%
\pgfpathclose%
\pgfusepath{fill}%
\end{pgfscope}%
\begin{pgfscope}%
\pgfpathrectangle{\pgfqpoint{12.211765in}{4.326316in}}{\pgfqpoint{2.188235in}{0.953684in}} %
\pgfusepath{clip}%
\pgfsetbuttcap%
\pgfsetmiterjoin%
\definecolor{currentfill}{rgb}{0.121569,0.466667,0.705882}%
\pgfsetfillcolor{currentfill}%
\pgfsetlinewidth{0.000000pt}%
\definecolor{currentstroke}{rgb}{0.000000,0.000000,0.000000}%
\pgfsetstrokecolor{currentstroke}%
\pgfsetstrokeopacity{0.000000}%
\pgfsetdash{}{0pt}%
\pgfpathmoveto{\pgfqpoint{-23.809001in}{4.673718in}}%
\pgfpathlineto{\pgfqpoint{12.583159in}{4.673718in}}%
\pgfpathlineto{\pgfqpoint{12.583159in}{4.680668in}}%
\pgfpathlineto{\pgfqpoint{-23.809001in}{4.680668in}}%
\pgfpathclose%
\pgfusepath{fill}%
\end{pgfscope}%
\begin{pgfscope}%
\pgfpathrectangle{\pgfqpoint{12.211765in}{4.326316in}}{\pgfqpoint{2.188235in}{0.953684in}} %
\pgfusepath{clip}%
\pgfsetbuttcap%
\pgfsetmiterjoin%
\definecolor{currentfill}{rgb}{0.121569,0.466667,0.705882}%
\pgfsetfillcolor{currentfill}%
\pgfsetlinewidth{0.000000pt}%
\definecolor{currentstroke}{rgb}{0.000000,0.000000,0.000000}%
\pgfsetstrokecolor{currentstroke}%
\pgfsetstrokeopacity{0.000000}%
\pgfsetdash{}{0pt}%
\pgfpathmoveto{\pgfqpoint{-23.809001in}{4.682405in}}%
\pgfpathlineto{\pgfqpoint{12.590156in}{4.682405in}}%
\pgfpathlineto{\pgfqpoint{12.590156in}{4.689355in}}%
\pgfpathlineto{\pgfqpoint{-23.809001in}{4.689355in}}%
\pgfpathclose%
\pgfusepath{fill}%
\end{pgfscope}%
\begin{pgfscope}%
\pgfpathrectangle{\pgfqpoint{12.211765in}{4.326316in}}{\pgfqpoint{2.188235in}{0.953684in}} %
\pgfusepath{clip}%
\pgfsetbuttcap%
\pgfsetmiterjoin%
\definecolor{currentfill}{rgb}{0.121569,0.466667,0.705882}%
\pgfsetfillcolor{currentfill}%
\pgfsetlinewidth{0.000000pt}%
\definecolor{currentstroke}{rgb}{0.000000,0.000000,0.000000}%
\pgfsetstrokecolor{currentstroke}%
\pgfsetstrokeopacity{0.000000}%
\pgfsetdash{}{0pt}%
\pgfpathmoveto{\pgfqpoint{-23.809001in}{4.691093in}}%
\pgfpathlineto{\pgfqpoint{12.692942in}{4.691093in}}%
\pgfpathlineto{\pgfqpoint{12.692942in}{4.698042in}}%
\pgfpathlineto{\pgfqpoint{-23.809001in}{4.698042in}}%
\pgfpathclose%
\pgfusepath{fill}%
\end{pgfscope}%
\begin{pgfscope}%
\pgfpathrectangle{\pgfqpoint{12.211765in}{4.326316in}}{\pgfqpoint{2.188235in}{0.953684in}} %
\pgfusepath{clip}%
\pgfsetbuttcap%
\pgfsetmiterjoin%
\definecolor{currentfill}{rgb}{0.121569,0.466667,0.705882}%
\pgfsetfillcolor{currentfill}%
\pgfsetlinewidth{0.000000pt}%
\definecolor{currentstroke}{rgb}{0.000000,0.000000,0.000000}%
\pgfsetstrokecolor{currentstroke}%
\pgfsetstrokeopacity{0.000000}%
\pgfsetdash{}{0pt}%
\pgfpathmoveto{\pgfqpoint{-23.809001in}{4.699780in}}%
\pgfpathlineto{\pgfqpoint{12.638658in}{4.699780in}}%
\pgfpathlineto{\pgfqpoint{12.638658in}{4.706730in}}%
\pgfpathlineto{\pgfqpoint{-23.809001in}{4.706730in}}%
\pgfpathclose%
\pgfusepath{fill}%
\end{pgfscope}%
\begin{pgfscope}%
\pgfpathrectangle{\pgfqpoint{12.211765in}{4.326316in}}{\pgfqpoint{2.188235in}{0.953684in}} %
\pgfusepath{clip}%
\pgfsetbuttcap%
\pgfsetmiterjoin%
\definecolor{currentfill}{rgb}{0.121569,0.466667,0.705882}%
\pgfsetfillcolor{currentfill}%
\pgfsetlinewidth{0.000000pt}%
\definecolor{currentstroke}{rgb}{0.000000,0.000000,0.000000}%
\pgfsetstrokecolor{currentstroke}%
\pgfsetstrokeopacity{0.000000}%
\pgfsetdash{}{0pt}%
\pgfpathmoveto{\pgfqpoint{-23.809001in}{4.708467in}}%
\pgfpathlineto{\pgfqpoint{12.678445in}{4.708467in}}%
\pgfpathlineto{\pgfqpoint{12.678445in}{4.715417in}}%
\pgfpathlineto{\pgfqpoint{-23.809001in}{4.715417in}}%
\pgfpathclose%
\pgfusepath{fill}%
\end{pgfscope}%
\begin{pgfscope}%
\pgfpathrectangle{\pgfqpoint{12.211765in}{4.326316in}}{\pgfqpoint{2.188235in}{0.953684in}} %
\pgfusepath{clip}%
\pgfsetbuttcap%
\pgfsetmiterjoin%
\definecolor{currentfill}{rgb}{0.121569,0.466667,0.705882}%
\pgfsetfillcolor{currentfill}%
\pgfsetlinewidth{0.000000pt}%
\definecolor{currentstroke}{rgb}{0.000000,0.000000,0.000000}%
\pgfsetstrokecolor{currentstroke}%
\pgfsetstrokeopacity{0.000000}%
\pgfsetdash{}{0pt}%
\pgfpathmoveto{\pgfqpoint{-23.809001in}{4.717154in}}%
\pgfpathlineto{\pgfqpoint{12.670622in}{4.717154in}}%
\pgfpathlineto{\pgfqpoint{12.670622in}{4.724104in}}%
\pgfpathlineto{\pgfqpoint{-23.809001in}{4.724104in}}%
\pgfpathclose%
\pgfusepath{fill}%
\end{pgfscope}%
\begin{pgfscope}%
\pgfpathrectangle{\pgfqpoint{12.211765in}{4.326316in}}{\pgfqpoint{2.188235in}{0.953684in}} %
\pgfusepath{clip}%
\pgfsetbuttcap%
\pgfsetmiterjoin%
\definecolor{currentfill}{rgb}{0.121569,0.466667,0.705882}%
\pgfsetfillcolor{currentfill}%
\pgfsetlinewidth{0.000000pt}%
\definecolor{currentstroke}{rgb}{0.000000,0.000000,0.000000}%
\pgfsetstrokecolor{currentstroke}%
\pgfsetstrokeopacity{0.000000}%
\pgfsetdash{}{0pt}%
\pgfpathmoveto{\pgfqpoint{-23.809001in}{4.725842in}}%
\pgfpathlineto{\pgfqpoint{12.692722in}{4.725842in}}%
\pgfpathlineto{\pgfqpoint{12.692722in}{4.732791in}}%
\pgfpathlineto{\pgfqpoint{-23.809001in}{4.732791in}}%
\pgfpathclose%
\pgfusepath{fill}%
\end{pgfscope}%
\begin{pgfscope}%
\pgfpathrectangle{\pgfqpoint{12.211765in}{4.326316in}}{\pgfqpoint{2.188235in}{0.953684in}} %
\pgfusepath{clip}%
\pgfsetbuttcap%
\pgfsetmiterjoin%
\definecolor{currentfill}{rgb}{0.121569,0.466667,0.705882}%
\pgfsetfillcolor{currentfill}%
\pgfsetlinewidth{0.000000pt}%
\definecolor{currentstroke}{rgb}{0.000000,0.000000,0.000000}%
\pgfsetstrokecolor{currentstroke}%
\pgfsetstrokeopacity{0.000000}%
\pgfsetdash{}{0pt}%
\pgfpathmoveto{\pgfqpoint{-23.809001in}{4.734529in}}%
\pgfpathlineto{\pgfqpoint{12.625002in}{4.734529in}}%
\pgfpathlineto{\pgfqpoint{12.625002in}{4.741479in}}%
\pgfpathlineto{\pgfqpoint{-23.809001in}{4.741479in}}%
\pgfpathclose%
\pgfusepath{fill}%
\end{pgfscope}%
\begin{pgfscope}%
\pgfpathrectangle{\pgfqpoint{12.211765in}{4.326316in}}{\pgfqpoint{2.188235in}{0.953684in}} %
\pgfusepath{clip}%
\pgfsetbuttcap%
\pgfsetmiterjoin%
\definecolor{currentfill}{rgb}{0.121569,0.466667,0.705882}%
\pgfsetfillcolor{currentfill}%
\pgfsetlinewidth{0.000000pt}%
\definecolor{currentstroke}{rgb}{0.000000,0.000000,0.000000}%
\pgfsetstrokecolor{currentstroke}%
\pgfsetstrokeopacity{0.000000}%
\pgfsetdash{}{0pt}%
\pgfpathmoveto{\pgfqpoint{-23.809001in}{4.743216in}}%
\pgfpathlineto{\pgfqpoint{12.584017in}{4.743216in}}%
\pgfpathlineto{\pgfqpoint{12.584017in}{4.750166in}}%
\pgfpathlineto{\pgfqpoint{-23.809001in}{4.750166in}}%
\pgfpathclose%
\pgfusepath{fill}%
\end{pgfscope}%
\begin{pgfscope}%
\pgfpathrectangle{\pgfqpoint{12.211765in}{4.326316in}}{\pgfqpoint{2.188235in}{0.953684in}} %
\pgfusepath{clip}%
\pgfsetbuttcap%
\pgfsetmiterjoin%
\definecolor{currentfill}{rgb}{0.121569,0.466667,0.705882}%
\pgfsetfillcolor{currentfill}%
\pgfsetlinewidth{0.000000pt}%
\definecolor{currentstroke}{rgb}{0.000000,0.000000,0.000000}%
\pgfsetstrokecolor{currentstroke}%
\pgfsetstrokeopacity{0.000000}%
\pgfsetdash{}{0pt}%
\pgfpathmoveto{\pgfqpoint{-23.809001in}{4.751903in}}%
\pgfpathlineto{\pgfqpoint{12.594809in}{4.751903in}}%
\pgfpathlineto{\pgfqpoint{12.594809in}{4.758853in}}%
\pgfpathlineto{\pgfqpoint{-23.809001in}{4.758853in}}%
\pgfpathclose%
\pgfusepath{fill}%
\end{pgfscope}%
\begin{pgfscope}%
\pgfpathrectangle{\pgfqpoint{12.211765in}{4.326316in}}{\pgfqpoint{2.188235in}{0.953684in}} %
\pgfusepath{clip}%
\pgfsetbuttcap%
\pgfsetmiterjoin%
\definecolor{currentfill}{rgb}{0.121569,0.466667,0.705882}%
\pgfsetfillcolor{currentfill}%
\pgfsetlinewidth{0.000000pt}%
\definecolor{currentstroke}{rgb}{0.000000,0.000000,0.000000}%
\pgfsetstrokecolor{currentstroke}%
\pgfsetstrokeopacity{0.000000}%
\pgfsetdash{}{0pt}%
\pgfpathmoveto{\pgfqpoint{-23.809001in}{4.760590in}}%
\pgfpathlineto{\pgfqpoint{12.569232in}{4.760590in}}%
\pgfpathlineto{\pgfqpoint{12.569232in}{4.767540in}}%
\pgfpathlineto{\pgfqpoint{-23.809001in}{4.767540in}}%
\pgfpathclose%
\pgfusepath{fill}%
\end{pgfscope}%
\begin{pgfscope}%
\pgfpathrectangle{\pgfqpoint{12.211765in}{4.326316in}}{\pgfqpoint{2.188235in}{0.953684in}} %
\pgfusepath{clip}%
\pgfsetbuttcap%
\pgfsetmiterjoin%
\definecolor{currentfill}{rgb}{0.121569,0.466667,0.705882}%
\pgfsetfillcolor{currentfill}%
\pgfsetlinewidth{0.000000pt}%
\definecolor{currentstroke}{rgb}{0.000000,0.000000,0.000000}%
\pgfsetstrokecolor{currentstroke}%
\pgfsetstrokeopacity{0.000000}%
\pgfsetdash{}{0pt}%
\pgfpathmoveto{\pgfqpoint{-23.809001in}{4.769278in}}%
\pgfpathlineto{\pgfqpoint{12.564543in}{4.769278in}}%
\pgfpathlineto{\pgfqpoint{12.564543in}{4.776227in}}%
\pgfpathlineto{\pgfqpoint{-23.809001in}{4.776227in}}%
\pgfpathclose%
\pgfusepath{fill}%
\end{pgfscope}%
\begin{pgfscope}%
\pgfpathrectangle{\pgfqpoint{12.211765in}{4.326316in}}{\pgfqpoint{2.188235in}{0.953684in}} %
\pgfusepath{clip}%
\pgfsetbuttcap%
\pgfsetmiterjoin%
\definecolor{currentfill}{rgb}{0.121569,0.466667,0.705882}%
\pgfsetfillcolor{currentfill}%
\pgfsetlinewidth{0.000000pt}%
\definecolor{currentstroke}{rgb}{0.000000,0.000000,0.000000}%
\pgfsetstrokecolor{currentstroke}%
\pgfsetstrokeopacity{0.000000}%
\pgfsetdash{}{0pt}%
\pgfpathmoveto{\pgfqpoint{-23.809001in}{4.777965in}}%
\pgfpathlineto{\pgfqpoint{12.688762in}{4.777965in}}%
\pgfpathlineto{\pgfqpoint{12.688762in}{4.784915in}}%
\pgfpathlineto{\pgfqpoint{-23.809001in}{4.784915in}}%
\pgfpathclose%
\pgfusepath{fill}%
\end{pgfscope}%
\begin{pgfscope}%
\pgfpathrectangle{\pgfqpoint{12.211765in}{4.326316in}}{\pgfqpoint{2.188235in}{0.953684in}} %
\pgfusepath{clip}%
\pgfsetbuttcap%
\pgfsetmiterjoin%
\definecolor{currentfill}{rgb}{0.121569,0.466667,0.705882}%
\pgfsetfillcolor{currentfill}%
\pgfsetlinewidth{0.000000pt}%
\definecolor{currentstroke}{rgb}{0.000000,0.000000,0.000000}%
\pgfsetstrokecolor{currentstroke}%
\pgfsetstrokeopacity{0.000000}%
\pgfsetdash{}{0pt}%
\pgfpathmoveto{\pgfqpoint{-23.809001in}{4.786652in}}%
\pgfpathlineto{\pgfqpoint{12.645325in}{4.786652in}}%
\pgfpathlineto{\pgfqpoint{12.645325in}{4.793602in}}%
\pgfpathlineto{\pgfqpoint{-23.809001in}{4.793602in}}%
\pgfpathclose%
\pgfusepath{fill}%
\end{pgfscope}%
\begin{pgfscope}%
\pgfpathrectangle{\pgfqpoint{12.211765in}{4.326316in}}{\pgfqpoint{2.188235in}{0.953684in}} %
\pgfusepath{clip}%
\pgfsetbuttcap%
\pgfsetmiterjoin%
\definecolor{currentfill}{rgb}{0.121569,0.466667,0.705882}%
\pgfsetfillcolor{currentfill}%
\pgfsetlinewidth{0.000000pt}%
\definecolor{currentstroke}{rgb}{0.000000,0.000000,0.000000}%
\pgfsetstrokecolor{currentstroke}%
\pgfsetstrokeopacity{0.000000}%
\pgfsetdash{}{0pt}%
\pgfpathmoveto{\pgfqpoint{-23.809001in}{4.795339in}}%
\pgfpathlineto{\pgfqpoint{12.609466in}{4.795339in}}%
\pgfpathlineto{\pgfqpoint{12.609466in}{4.802289in}}%
\pgfpathlineto{\pgfqpoint{-23.809001in}{4.802289in}}%
\pgfpathclose%
\pgfusepath{fill}%
\end{pgfscope}%
\begin{pgfscope}%
\pgfpathrectangle{\pgfqpoint{12.211765in}{4.326316in}}{\pgfqpoint{2.188235in}{0.953684in}} %
\pgfusepath{clip}%
\pgfsetbuttcap%
\pgfsetmiterjoin%
\definecolor{currentfill}{rgb}{0.121569,0.466667,0.705882}%
\pgfsetfillcolor{currentfill}%
\pgfsetlinewidth{0.000000pt}%
\definecolor{currentstroke}{rgb}{0.000000,0.000000,0.000000}%
\pgfsetstrokecolor{currentstroke}%
\pgfsetstrokeopacity{0.000000}%
\pgfsetdash{}{0pt}%
\pgfpathmoveto{\pgfqpoint{-23.809001in}{4.804027in}}%
\pgfpathlineto{\pgfqpoint{12.600929in}{4.804027in}}%
\pgfpathlineto{\pgfqpoint{12.600929in}{4.810976in}}%
\pgfpathlineto{\pgfqpoint{-23.809001in}{4.810976in}}%
\pgfpathclose%
\pgfusepath{fill}%
\end{pgfscope}%
\begin{pgfscope}%
\pgfpathrectangle{\pgfqpoint{12.211765in}{4.326316in}}{\pgfqpoint{2.188235in}{0.953684in}} %
\pgfusepath{clip}%
\pgfsetbuttcap%
\pgfsetmiterjoin%
\definecolor{currentfill}{rgb}{0.121569,0.466667,0.705882}%
\pgfsetfillcolor{currentfill}%
\pgfsetlinewidth{0.000000pt}%
\definecolor{currentstroke}{rgb}{0.000000,0.000000,0.000000}%
\pgfsetstrokecolor{currentstroke}%
\pgfsetstrokeopacity{0.000000}%
\pgfsetdash{}{0pt}%
\pgfpathmoveto{\pgfqpoint{-23.809001in}{4.812714in}}%
\pgfpathlineto{\pgfqpoint{12.624495in}{4.812714in}}%
\pgfpathlineto{\pgfqpoint{12.624495in}{4.819664in}}%
\pgfpathlineto{\pgfqpoint{-23.809001in}{4.819664in}}%
\pgfpathclose%
\pgfusepath{fill}%
\end{pgfscope}%
\begin{pgfscope}%
\pgfpathrectangle{\pgfqpoint{12.211765in}{4.326316in}}{\pgfqpoint{2.188235in}{0.953684in}} %
\pgfusepath{clip}%
\pgfsetbuttcap%
\pgfsetmiterjoin%
\definecolor{currentfill}{rgb}{0.121569,0.466667,0.705882}%
\pgfsetfillcolor{currentfill}%
\pgfsetlinewidth{0.000000pt}%
\definecolor{currentstroke}{rgb}{0.000000,0.000000,0.000000}%
\pgfsetstrokecolor{currentstroke}%
\pgfsetstrokeopacity{0.000000}%
\pgfsetdash{}{0pt}%
\pgfpathmoveto{\pgfqpoint{-23.809001in}{4.821401in}}%
\pgfpathlineto{\pgfqpoint{12.563004in}{4.821401in}}%
\pgfpathlineto{\pgfqpoint{12.563004in}{4.828351in}}%
\pgfpathlineto{\pgfqpoint{-23.809001in}{4.828351in}}%
\pgfpathclose%
\pgfusepath{fill}%
\end{pgfscope}%
\begin{pgfscope}%
\pgfpathrectangle{\pgfqpoint{12.211765in}{4.326316in}}{\pgfqpoint{2.188235in}{0.953684in}} %
\pgfusepath{clip}%
\pgfsetbuttcap%
\pgfsetmiterjoin%
\definecolor{currentfill}{rgb}{0.121569,0.466667,0.705882}%
\pgfsetfillcolor{currentfill}%
\pgfsetlinewidth{0.000000pt}%
\definecolor{currentstroke}{rgb}{0.000000,0.000000,0.000000}%
\pgfsetstrokecolor{currentstroke}%
\pgfsetstrokeopacity{0.000000}%
\pgfsetdash{}{0pt}%
\pgfpathmoveto{\pgfqpoint{-23.809001in}{4.830088in}}%
\pgfpathlineto{\pgfqpoint{12.579700in}{4.830088in}}%
\pgfpathlineto{\pgfqpoint{12.579700in}{4.837038in}}%
\pgfpathlineto{\pgfqpoint{-23.809001in}{4.837038in}}%
\pgfpathclose%
\pgfusepath{fill}%
\end{pgfscope}%
\begin{pgfscope}%
\pgfpathrectangle{\pgfqpoint{12.211765in}{4.326316in}}{\pgfqpoint{2.188235in}{0.953684in}} %
\pgfusepath{clip}%
\pgfsetbuttcap%
\pgfsetmiterjoin%
\definecolor{currentfill}{rgb}{0.121569,0.466667,0.705882}%
\pgfsetfillcolor{currentfill}%
\pgfsetlinewidth{0.000000pt}%
\definecolor{currentstroke}{rgb}{0.000000,0.000000,0.000000}%
\pgfsetstrokecolor{currentstroke}%
\pgfsetstrokeopacity{0.000000}%
\pgfsetdash{}{0pt}%
\pgfpathmoveto{\pgfqpoint{-23.809001in}{4.838776in}}%
\pgfpathlineto{\pgfqpoint{12.554995in}{4.838776in}}%
\pgfpathlineto{\pgfqpoint{12.554995in}{4.845725in}}%
\pgfpathlineto{\pgfqpoint{-23.809001in}{4.845725in}}%
\pgfpathclose%
\pgfusepath{fill}%
\end{pgfscope}%
\begin{pgfscope}%
\pgfpathrectangle{\pgfqpoint{12.211765in}{4.326316in}}{\pgfqpoint{2.188235in}{0.953684in}} %
\pgfusepath{clip}%
\pgfsetbuttcap%
\pgfsetmiterjoin%
\definecolor{currentfill}{rgb}{0.121569,0.466667,0.705882}%
\pgfsetfillcolor{currentfill}%
\pgfsetlinewidth{0.000000pt}%
\definecolor{currentstroke}{rgb}{0.000000,0.000000,0.000000}%
\pgfsetstrokecolor{currentstroke}%
\pgfsetstrokeopacity{0.000000}%
\pgfsetdash{}{0pt}%
\pgfpathmoveto{\pgfqpoint{-23.809001in}{4.847463in}}%
\pgfpathlineto{\pgfqpoint{12.513707in}{4.847463in}}%
\pgfpathlineto{\pgfqpoint{12.513707in}{4.854413in}}%
\pgfpathlineto{\pgfqpoint{-23.809001in}{4.854413in}}%
\pgfpathclose%
\pgfusepath{fill}%
\end{pgfscope}%
\begin{pgfscope}%
\pgfpathrectangle{\pgfqpoint{12.211765in}{4.326316in}}{\pgfqpoint{2.188235in}{0.953684in}} %
\pgfusepath{clip}%
\pgfsetbuttcap%
\pgfsetmiterjoin%
\definecolor{currentfill}{rgb}{0.121569,0.466667,0.705882}%
\pgfsetfillcolor{currentfill}%
\pgfsetlinewidth{0.000000pt}%
\definecolor{currentstroke}{rgb}{0.000000,0.000000,0.000000}%
\pgfsetstrokecolor{currentstroke}%
\pgfsetstrokeopacity{0.000000}%
\pgfsetdash{}{0pt}%
\pgfpathmoveto{\pgfqpoint{-23.809001in}{4.856150in}}%
\pgfpathlineto{\pgfqpoint{12.555668in}{4.856150in}}%
\pgfpathlineto{\pgfqpoint{12.555668in}{4.863100in}}%
\pgfpathlineto{\pgfqpoint{-23.809001in}{4.863100in}}%
\pgfpathclose%
\pgfusepath{fill}%
\end{pgfscope}%
\begin{pgfscope}%
\pgfpathrectangle{\pgfqpoint{12.211765in}{4.326316in}}{\pgfqpoint{2.188235in}{0.953684in}} %
\pgfusepath{clip}%
\pgfsetbuttcap%
\pgfsetmiterjoin%
\definecolor{currentfill}{rgb}{0.121569,0.466667,0.705882}%
\pgfsetfillcolor{currentfill}%
\pgfsetlinewidth{0.000000pt}%
\definecolor{currentstroke}{rgb}{0.000000,0.000000,0.000000}%
\pgfsetstrokecolor{currentstroke}%
\pgfsetstrokeopacity{0.000000}%
\pgfsetdash{}{0pt}%
\pgfpathmoveto{\pgfqpoint{-23.809001in}{4.864837in}}%
\pgfpathlineto{\pgfqpoint{12.476574in}{4.864837in}}%
\pgfpathlineto{\pgfqpoint{12.476574in}{4.871787in}}%
\pgfpathlineto{\pgfqpoint{-23.809001in}{4.871787in}}%
\pgfpathclose%
\pgfusepath{fill}%
\end{pgfscope}%
\begin{pgfscope}%
\pgfpathrectangle{\pgfqpoint{12.211765in}{4.326316in}}{\pgfqpoint{2.188235in}{0.953684in}} %
\pgfusepath{clip}%
\pgfsetbuttcap%
\pgfsetmiterjoin%
\definecolor{currentfill}{rgb}{0.121569,0.466667,0.705882}%
\pgfsetfillcolor{currentfill}%
\pgfsetlinewidth{0.000000pt}%
\definecolor{currentstroke}{rgb}{0.000000,0.000000,0.000000}%
\pgfsetstrokecolor{currentstroke}%
\pgfsetstrokeopacity{0.000000}%
\pgfsetdash{}{0pt}%
\pgfpathmoveto{\pgfqpoint{-23.809001in}{4.873524in}}%
\pgfpathlineto{\pgfqpoint{12.632231in}{4.873524in}}%
\pgfpathlineto{\pgfqpoint{12.632231in}{4.880474in}}%
\pgfpathlineto{\pgfqpoint{-23.809001in}{4.880474in}}%
\pgfpathclose%
\pgfusepath{fill}%
\end{pgfscope}%
\begin{pgfscope}%
\pgfpathrectangle{\pgfqpoint{12.211765in}{4.326316in}}{\pgfqpoint{2.188235in}{0.953684in}} %
\pgfusepath{clip}%
\pgfsetbuttcap%
\pgfsetmiterjoin%
\definecolor{currentfill}{rgb}{0.121569,0.466667,0.705882}%
\pgfsetfillcolor{currentfill}%
\pgfsetlinewidth{0.000000pt}%
\definecolor{currentstroke}{rgb}{0.000000,0.000000,0.000000}%
\pgfsetstrokecolor{currentstroke}%
\pgfsetstrokeopacity{0.000000}%
\pgfsetdash{}{0pt}%
\pgfpathmoveto{\pgfqpoint{-23.809001in}{4.882212in}}%
\pgfpathlineto{\pgfqpoint{12.652294in}{4.882212in}}%
\pgfpathlineto{\pgfqpoint{12.652294in}{4.889161in}}%
\pgfpathlineto{\pgfqpoint{-23.809001in}{4.889161in}}%
\pgfpathclose%
\pgfusepath{fill}%
\end{pgfscope}%
\begin{pgfscope}%
\pgfpathrectangle{\pgfqpoint{12.211765in}{4.326316in}}{\pgfqpoint{2.188235in}{0.953684in}} %
\pgfusepath{clip}%
\pgfsetbuttcap%
\pgfsetmiterjoin%
\definecolor{currentfill}{rgb}{0.121569,0.466667,0.705882}%
\pgfsetfillcolor{currentfill}%
\pgfsetlinewidth{0.000000pt}%
\definecolor{currentstroke}{rgb}{0.000000,0.000000,0.000000}%
\pgfsetstrokecolor{currentstroke}%
\pgfsetstrokeopacity{0.000000}%
\pgfsetdash{}{0pt}%
\pgfpathmoveto{\pgfqpoint{-23.809001in}{4.890899in}}%
\pgfpathlineto{\pgfqpoint{12.562848in}{4.890899in}}%
\pgfpathlineto{\pgfqpoint{12.562848in}{4.897849in}}%
\pgfpathlineto{\pgfqpoint{-23.809001in}{4.897849in}}%
\pgfpathclose%
\pgfusepath{fill}%
\end{pgfscope}%
\begin{pgfscope}%
\pgfpathrectangle{\pgfqpoint{12.211765in}{4.326316in}}{\pgfqpoint{2.188235in}{0.953684in}} %
\pgfusepath{clip}%
\pgfsetbuttcap%
\pgfsetmiterjoin%
\definecolor{currentfill}{rgb}{0.121569,0.466667,0.705882}%
\pgfsetfillcolor{currentfill}%
\pgfsetlinewidth{0.000000pt}%
\definecolor{currentstroke}{rgb}{0.000000,0.000000,0.000000}%
\pgfsetstrokecolor{currentstroke}%
\pgfsetstrokeopacity{0.000000}%
\pgfsetdash{}{0pt}%
\pgfpathmoveto{\pgfqpoint{-23.809001in}{4.899586in}}%
\pgfpathlineto{\pgfqpoint{12.566774in}{4.899586in}}%
\pgfpathlineto{\pgfqpoint{12.566774in}{4.906536in}}%
\pgfpathlineto{\pgfqpoint{-23.809001in}{4.906536in}}%
\pgfpathclose%
\pgfusepath{fill}%
\end{pgfscope}%
\begin{pgfscope}%
\pgfpathrectangle{\pgfqpoint{12.211765in}{4.326316in}}{\pgfqpoint{2.188235in}{0.953684in}} %
\pgfusepath{clip}%
\pgfsetbuttcap%
\pgfsetmiterjoin%
\definecolor{currentfill}{rgb}{0.121569,0.466667,0.705882}%
\pgfsetfillcolor{currentfill}%
\pgfsetlinewidth{0.000000pt}%
\definecolor{currentstroke}{rgb}{0.000000,0.000000,0.000000}%
\pgfsetstrokecolor{currentstroke}%
\pgfsetstrokeopacity{0.000000}%
\pgfsetdash{}{0pt}%
\pgfpathmoveto{\pgfqpoint{-23.809001in}{4.908273in}}%
\pgfpathlineto{\pgfqpoint{12.317018in}{4.908273in}}%
\pgfpathlineto{\pgfqpoint{12.317018in}{4.915223in}}%
\pgfpathlineto{\pgfqpoint{-23.809001in}{4.915223in}}%
\pgfpathclose%
\pgfusepath{fill}%
\end{pgfscope}%
\begin{pgfscope}%
\pgfpathrectangle{\pgfqpoint{12.211765in}{4.326316in}}{\pgfqpoint{2.188235in}{0.953684in}} %
\pgfusepath{clip}%
\pgfsetbuttcap%
\pgfsetmiterjoin%
\definecolor{currentfill}{rgb}{0.121569,0.466667,0.705882}%
\pgfsetfillcolor{currentfill}%
\pgfsetlinewidth{0.000000pt}%
\definecolor{currentstroke}{rgb}{0.000000,0.000000,0.000000}%
\pgfsetstrokecolor{currentstroke}%
\pgfsetstrokeopacity{0.000000}%
\pgfsetdash{}{0pt}%
\pgfpathmoveto{\pgfqpoint{-23.809001in}{4.916961in}}%
\pgfpathlineto{\pgfqpoint{12.565892in}{4.916961in}}%
\pgfpathlineto{\pgfqpoint{12.565892in}{4.923910in}}%
\pgfpathlineto{\pgfqpoint{-23.809001in}{4.923910in}}%
\pgfpathclose%
\pgfusepath{fill}%
\end{pgfscope}%
\begin{pgfscope}%
\pgfpathrectangle{\pgfqpoint{12.211765in}{4.326316in}}{\pgfqpoint{2.188235in}{0.953684in}} %
\pgfusepath{clip}%
\pgfsetbuttcap%
\pgfsetmiterjoin%
\definecolor{currentfill}{rgb}{0.121569,0.466667,0.705882}%
\pgfsetfillcolor{currentfill}%
\pgfsetlinewidth{0.000000pt}%
\definecolor{currentstroke}{rgb}{0.000000,0.000000,0.000000}%
\pgfsetstrokecolor{currentstroke}%
\pgfsetstrokeopacity{0.000000}%
\pgfsetdash{}{0pt}%
\pgfpathmoveto{\pgfqpoint{-23.809001in}{4.925648in}}%
\pgfpathlineto{\pgfqpoint{12.595677in}{4.925648in}}%
\pgfpathlineto{\pgfqpoint{12.595677in}{4.932598in}}%
\pgfpathlineto{\pgfqpoint{-23.809001in}{4.932598in}}%
\pgfpathclose%
\pgfusepath{fill}%
\end{pgfscope}%
\begin{pgfscope}%
\pgfpathrectangle{\pgfqpoint{12.211765in}{4.326316in}}{\pgfqpoint{2.188235in}{0.953684in}} %
\pgfusepath{clip}%
\pgfsetbuttcap%
\pgfsetmiterjoin%
\definecolor{currentfill}{rgb}{0.121569,0.466667,0.705882}%
\pgfsetfillcolor{currentfill}%
\pgfsetlinewidth{0.000000pt}%
\definecolor{currentstroke}{rgb}{0.000000,0.000000,0.000000}%
\pgfsetstrokecolor{currentstroke}%
\pgfsetstrokeopacity{0.000000}%
\pgfsetdash{}{0pt}%
\pgfpathmoveto{\pgfqpoint{-23.809001in}{4.934335in}}%
\pgfpathlineto{\pgfqpoint{12.608246in}{4.934335in}}%
\pgfpathlineto{\pgfqpoint{12.608246in}{4.941285in}}%
\pgfpathlineto{\pgfqpoint{-23.809001in}{4.941285in}}%
\pgfpathclose%
\pgfusepath{fill}%
\end{pgfscope}%
\begin{pgfscope}%
\pgfpathrectangle{\pgfqpoint{12.211765in}{4.326316in}}{\pgfqpoint{2.188235in}{0.953684in}} %
\pgfusepath{clip}%
\pgfsetbuttcap%
\pgfsetmiterjoin%
\definecolor{currentfill}{rgb}{0.121569,0.466667,0.705882}%
\pgfsetfillcolor{currentfill}%
\pgfsetlinewidth{0.000000pt}%
\definecolor{currentstroke}{rgb}{0.000000,0.000000,0.000000}%
\pgfsetstrokecolor{currentstroke}%
\pgfsetstrokeopacity{0.000000}%
\pgfsetdash{}{0pt}%
\pgfpathmoveto{\pgfqpoint{-23.809001in}{4.943022in}}%
\pgfpathlineto{\pgfqpoint{12.658050in}{4.943022in}}%
\pgfpathlineto{\pgfqpoint{12.658050in}{4.949972in}}%
\pgfpathlineto{\pgfqpoint{-23.809001in}{4.949972in}}%
\pgfpathclose%
\pgfusepath{fill}%
\end{pgfscope}%
\begin{pgfscope}%
\pgfpathrectangle{\pgfqpoint{12.211765in}{4.326316in}}{\pgfqpoint{2.188235in}{0.953684in}} %
\pgfusepath{clip}%
\pgfsetbuttcap%
\pgfsetmiterjoin%
\definecolor{currentfill}{rgb}{0.121569,0.466667,0.705882}%
\pgfsetfillcolor{currentfill}%
\pgfsetlinewidth{0.000000pt}%
\definecolor{currentstroke}{rgb}{0.000000,0.000000,0.000000}%
\pgfsetstrokecolor{currentstroke}%
\pgfsetstrokeopacity{0.000000}%
\pgfsetdash{}{0pt}%
\pgfpathmoveto{\pgfqpoint{-23.809001in}{4.951710in}}%
\pgfpathlineto{\pgfqpoint{12.604965in}{4.951710in}}%
\pgfpathlineto{\pgfqpoint{12.604965in}{4.958659in}}%
\pgfpathlineto{\pgfqpoint{-23.809001in}{4.958659in}}%
\pgfpathclose%
\pgfusepath{fill}%
\end{pgfscope}%
\begin{pgfscope}%
\pgfpathrectangle{\pgfqpoint{12.211765in}{4.326316in}}{\pgfqpoint{2.188235in}{0.953684in}} %
\pgfusepath{clip}%
\pgfsetbuttcap%
\pgfsetmiterjoin%
\definecolor{currentfill}{rgb}{0.121569,0.466667,0.705882}%
\pgfsetfillcolor{currentfill}%
\pgfsetlinewidth{0.000000pt}%
\definecolor{currentstroke}{rgb}{0.000000,0.000000,0.000000}%
\pgfsetstrokecolor{currentstroke}%
\pgfsetstrokeopacity{0.000000}%
\pgfsetdash{}{0pt}%
\pgfpathmoveto{\pgfqpoint{-23.809001in}{4.960397in}}%
\pgfpathlineto{\pgfqpoint{12.565595in}{4.960397in}}%
\pgfpathlineto{\pgfqpoint{12.565595in}{4.967347in}}%
\pgfpathlineto{\pgfqpoint{-23.809001in}{4.967347in}}%
\pgfpathclose%
\pgfusepath{fill}%
\end{pgfscope}%
\begin{pgfscope}%
\pgfpathrectangle{\pgfqpoint{12.211765in}{4.326316in}}{\pgfqpoint{2.188235in}{0.953684in}} %
\pgfusepath{clip}%
\pgfsetbuttcap%
\pgfsetmiterjoin%
\definecolor{currentfill}{rgb}{0.121569,0.466667,0.705882}%
\pgfsetfillcolor{currentfill}%
\pgfsetlinewidth{0.000000pt}%
\definecolor{currentstroke}{rgb}{0.000000,0.000000,0.000000}%
\pgfsetstrokecolor{currentstroke}%
\pgfsetstrokeopacity{0.000000}%
\pgfsetdash{}{0pt}%
\pgfpathmoveto{\pgfqpoint{-23.809001in}{4.969084in}}%
\pgfpathlineto{\pgfqpoint{12.592304in}{4.969084in}}%
\pgfpathlineto{\pgfqpoint{12.592304in}{4.976034in}}%
\pgfpathlineto{\pgfqpoint{-23.809001in}{4.976034in}}%
\pgfpathclose%
\pgfusepath{fill}%
\end{pgfscope}%
\begin{pgfscope}%
\pgfpathrectangle{\pgfqpoint{12.211765in}{4.326316in}}{\pgfqpoint{2.188235in}{0.953684in}} %
\pgfusepath{clip}%
\pgfsetbuttcap%
\pgfsetmiterjoin%
\definecolor{currentfill}{rgb}{0.121569,0.466667,0.705882}%
\pgfsetfillcolor{currentfill}%
\pgfsetlinewidth{0.000000pt}%
\definecolor{currentstroke}{rgb}{0.000000,0.000000,0.000000}%
\pgfsetstrokecolor{currentstroke}%
\pgfsetstrokeopacity{0.000000}%
\pgfsetdash{}{0pt}%
\pgfpathmoveto{\pgfqpoint{-23.809001in}{4.977771in}}%
\pgfpathlineto{\pgfqpoint{12.592120in}{4.977771in}}%
\pgfpathlineto{\pgfqpoint{12.592120in}{4.984721in}}%
\pgfpathlineto{\pgfqpoint{-23.809001in}{4.984721in}}%
\pgfpathclose%
\pgfusepath{fill}%
\end{pgfscope}%
\begin{pgfscope}%
\pgfpathrectangle{\pgfqpoint{12.211765in}{4.326316in}}{\pgfqpoint{2.188235in}{0.953684in}} %
\pgfusepath{clip}%
\pgfsetbuttcap%
\pgfsetmiterjoin%
\definecolor{currentfill}{rgb}{0.121569,0.466667,0.705882}%
\pgfsetfillcolor{currentfill}%
\pgfsetlinewidth{0.000000pt}%
\definecolor{currentstroke}{rgb}{0.000000,0.000000,0.000000}%
\pgfsetstrokecolor{currentstroke}%
\pgfsetstrokeopacity{0.000000}%
\pgfsetdash{}{0pt}%
\pgfpathmoveto{\pgfqpoint{-23.809001in}{4.986458in}}%
\pgfpathlineto{\pgfqpoint{12.641146in}{4.986458in}}%
\pgfpathlineto{\pgfqpoint{12.641146in}{4.993408in}}%
\pgfpathlineto{\pgfqpoint{-23.809001in}{4.993408in}}%
\pgfpathclose%
\pgfusepath{fill}%
\end{pgfscope}%
\begin{pgfscope}%
\pgfpathrectangle{\pgfqpoint{12.211765in}{4.326316in}}{\pgfqpoint{2.188235in}{0.953684in}} %
\pgfusepath{clip}%
\pgfsetbuttcap%
\pgfsetmiterjoin%
\definecolor{currentfill}{rgb}{0.121569,0.466667,0.705882}%
\pgfsetfillcolor{currentfill}%
\pgfsetlinewidth{0.000000pt}%
\definecolor{currentstroke}{rgb}{0.000000,0.000000,0.000000}%
\pgfsetstrokecolor{currentstroke}%
\pgfsetstrokeopacity{0.000000}%
\pgfsetdash{}{0pt}%
\pgfpathmoveto{\pgfqpoint{-23.809001in}{4.995146in}}%
\pgfpathlineto{\pgfqpoint{12.609733in}{4.995146in}}%
\pgfpathlineto{\pgfqpoint{12.609733in}{5.002095in}}%
\pgfpathlineto{\pgfqpoint{-23.809001in}{5.002095in}}%
\pgfpathclose%
\pgfusepath{fill}%
\end{pgfscope}%
\begin{pgfscope}%
\pgfpathrectangle{\pgfqpoint{12.211765in}{4.326316in}}{\pgfqpoint{2.188235in}{0.953684in}} %
\pgfusepath{clip}%
\pgfsetbuttcap%
\pgfsetmiterjoin%
\definecolor{currentfill}{rgb}{0.121569,0.466667,0.705882}%
\pgfsetfillcolor{currentfill}%
\pgfsetlinewidth{0.000000pt}%
\definecolor{currentstroke}{rgb}{0.000000,0.000000,0.000000}%
\pgfsetstrokecolor{currentstroke}%
\pgfsetstrokeopacity{0.000000}%
\pgfsetdash{}{0pt}%
\pgfpathmoveto{\pgfqpoint{-23.809001in}{5.003833in}}%
\pgfpathlineto{\pgfqpoint{12.599583in}{5.003833in}}%
\pgfpathlineto{\pgfqpoint{12.599583in}{5.010783in}}%
\pgfpathlineto{\pgfqpoint{-23.809001in}{5.010783in}}%
\pgfpathclose%
\pgfusepath{fill}%
\end{pgfscope}%
\begin{pgfscope}%
\pgfpathrectangle{\pgfqpoint{12.211765in}{4.326316in}}{\pgfqpoint{2.188235in}{0.953684in}} %
\pgfusepath{clip}%
\pgfsetbuttcap%
\pgfsetmiterjoin%
\definecolor{currentfill}{rgb}{0.121569,0.466667,0.705882}%
\pgfsetfillcolor{currentfill}%
\pgfsetlinewidth{0.000000pt}%
\definecolor{currentstroke}{rgb}{0.000000,0.000000,0.000000}%
\pgfsetstrokecolor{currentstroke}%
\pgfsetstrokeopacity{0.000000}%
\pgfsetdash{}{0pt}%
\pgfpathmoveto{\pgfqpoint{-23.809001in}{5.012520in}}%
\pgfpathlineto{\pgfqpoint{12.341538in}{5.012520in}}%
\pgfpathlineto{\pgfqpoint{12.341538in}{5.019470in}}%
\pgfpathlineto{\pgfqpoint{-23.809001in}{5.019470in}}%
\pgfpathclose%
\pgfusepath{fill}%
\end{pgfscope}%
\begin{pgfscope}%
\pgfpathrectangle{\pgfqpoint{12.211765in}{4.326316in}}{\pgfqpoint{2.188235in}{0.953684in}} %
\pgfusepath{clip}%
\pgfsetbuttcap%
\pgfsetmiterjoin%
\definecolor{currentfill}{rgb}{0.121569,0.466667,0.705882}%
\pgfsetfillcolor{currentfill}%
\pgfsetlinewidth{0.000000pt}%
\definecolor{currentstroke}{rgb}{0.000000,0.000000,0.000000}%
\pgfsetstrokecolor{currentstroke}%
\pgfsetstrokeopacity{0.000000}%
\pgfsetdash{}{0pt}%
\pgfpathmoveto{\pgfqpoint{-23.809001in}{5.021207in}}%
\pgfpathlineto{\pgfqpoint{12.613345in}{5.021207in}}%
\pgfpathlineto{\pgfqpoint{12.613345in}{5.028157in}}%
\pgfpathlineto{\pgfqpoint{-23.809001in}{5.028157in}}%
\pgfpathclose%
\pgfusepath{fill}%
\end{pgfscope}%
\begin{pgfscope}%
\pgfpathrectangle{\pgfqpoint{12.211765in}{4.326316in}}{\pgfqpoint{2.188235in}{0.953684in}} %
\pgfusepath{clip}%
\pgfsetbuttcap%
\pgfsetmiterjoin%
\definecolor{currentfill}{rgb}{0.121569,0.466667,0.705882}%
\pgfsetfillcolor{currentfill}%
\pgfsetlinewidth{0.000000pt}%
\definecolor{currentstroke}{rgb}{0.000000,0.000000,0.000000}%
\pgfsetstrokecolor{currentstroke}%
\pgfsetstrokeopacity{0.000000}%
\pgfsetdash{}{0pt}%
\pgfpathmoveto{\pgfqpoint{-23.809001in}{5.029895in}}%
\pgfpathlineto{\pgfqpoint{12.500821in}{5.029895in}}%
\pgfpathlineto{\pgfqpoint{12.500821in}{5.036844in}}%
\pgfpathlineto{\pgfqpoint{-23.809001in}{5.036844in}}%
\pgfpathclose%
\pgfusepath{fill}%
\end{pgfscope}%
\begin{pgfscope}%
\pgfpathrectangle{\pgfqpoint{12.211765in}{4.326316in}}{\pgfqpoint{2.188235in}{0.953684in}} %
\pgfusepath{clip}%
\pgfsetbuttcap%
\pgfsetmiterjoin%
\definecolor{currentfill}{rgb}{0.121569,0.466667,0.705882}%
\pgfsetfillcolor{currentfill}%
\pgfsetlinewidth{0.000000pt}%
\definecolor{currentstroke}{rgb}{0.000000,0.000000,0.000000}%
\pgfsetstrokecolor{currentstroke}%
\pgfsetstrokeopacity{0.000000}%
\pgfsetdash{}{0pt}%
\pgfpathmoveto{\pgfqpoint{-23.809001in}{5.038582in}}%
\pgfpathlineto{\pgfqpoint{12.548281in}{5.038582in}}%
\pgfpathlineto{\pgfqpoint{12.548281in}{5.045532in}}%
\pgfpathlineto{\pgfqpoint{-23.809001in}{5.045532in}}%
\pgfpathclose%
\pgfusepath{fill}%
\end{pgfscope}%
\begin{pgfscope}%
\pgfpathrectangle{\pgfqpoint{12.211765in}{4.326316in}}{\pgfqpoint{2.188235in}{0.953684in}} %
\pgfusepath{clip}%
\pgfsetbuttcap%
\pgfsetmiterjoin%
\definecolor{currentfill}{rgb}{0.121569,0.466667,0.705882}%
\pgfsetfillcolor{currentfill}%
\pgfsetlinewidth{0.000000pt}%
\definecolor{currentstroke}{rgb}{0.000000,0.000000,0.000000}%
\pgfsetstrokecolor{currentstroke}%
\pgfsetstrokeopacity{0.000000}%
\pgfsetdash{}{0pt}%
\pgfpathmoveto{\pgfqpoint{-23.809001in}{5.047269in}}%
\pgfpathlineto{\pgfqpoint{12.545104in}{5.047269in}}%
\pgfpathlineto{\pgfqpoint{12.545104in}{5.054219in}}%
\pgfpathlineto{\pgfqpoint{-23.809001in}{5.054219in}}%
\pgfpathclose%
\pgfusepath{fill}%
\end{pgfscope}%
\begin{pgfscope}%
\pgfpathrectangle{\pgfqpoint{12.211765in}{4.326316in}}{\pgfqpoint{2.188235in}{0.953684in}} %
\pgfusepath{clip}%
\pgfsetbuttcap%
\pgfsetmiterjoin%
\definecolor{currentfill}{rgb}{0.121569,0.466667,0.705882}%
\pgfsetfillcolor{currentfill}%
\pgfsetlinewidth{0.000000pt}%
\definecolor{currentstroke}{rgb}{0.000000,0.000000,0.000000}%
\pgfsetstrokecolor{currentstroke}%
\pgfsetstrokeopacity{0.000000}%
\pgfsetdash{}{0pt}%
\pgfpathmoveto{\pgfqpoint{-23.809001in}{5.055956in}}%
\pgfpathlineto{\pgfqpoint{12.582094in}{5.055956in}}%
\pgfpathlineto{\pgfqpoint{12.582094in}{5.062906in}}%
\pgfpathlineto{\pgfqpoint{-23.809001in}{5.062906in}}%
\pgfpathclose%
\pgfusepath{fill}%
\end{pgfscope}%
\begin{pgfscope}%
\pgfpathrectangle{\pgfqpoint{12.211765in}{4.326316in}}{\pgfqpoint{2.188235in}{0.953684in}} %
\pgfusepath{clip}%
\pgfsetbuttcap%
\pgfsetmiterjoin%
\definecolor{currentfill}{rgb}{0.121569,0.466667,0.705882}%
\pgfsetfillcolor{currentfill}%
\pgfsetlinewidth{0.000000pt}%
\definecolor{currentstroke}{rgb}{0.000000,0.000000,0.000000}%
\pgfsetstrokecolor{currentstroke}%
\pgfsetstrokeopacity{0.000000}%
\pgfsetdash{}{0pt}%
\pgfpathmoveto{\pgfqpoint{-23.809001in}{5.064644in}}%
\pgfpathlineto{\pgfqpoint{12.586311in}{5.064644in}}%
\pgfpathlineto{\pgfqpoint{12.586311in}{5.071593in}}%
\pgfpathlineto{\pgfqpoint{-23.809001in}{5.071593in}}%
\pgfpathclose%
\pgfusepath{fill}%
\end{pgfscope}%
\begin{pgfscope}%
\pgfpathrectangle{\pgfqpoint{12.211765in}{4.326316in}}{\pgfqpoint{2.188235in}{0.953684in}} %
\pgfusepath{clip}%
\pgfsetbuttcap%
\pgfsetmiterjoin%
\definecolor{currentfill}{rgb}{0.121569,0.466667,0.705882}%
\pgfsetfillcolor{currentfill}%
\pgfsetlinewidth{0.000000pt}%
\definecolor{currentstroke}{rgb}{0.000000,0.000000,0.000000}%
\pgfsetstrokecolor{currentstroke}%
\pgfsetstrokeopacity{0.000000}%
\pgfsetdash{}{0pt}%
\pgfpathmoveto{\pgfqpoint{-23.809001in}{5.073331in}}%
\pgfpathlineto{\pgfqpoint{12.586004in}{5.073331in}}%
\pgfpathlineto{\pgfqpoint{12.586004in}{5.080281in}}%
\pgfpathlineto{\pgfqpoint{-23.809001in}{5.080281in}}%
\pgfpathclose%
\pgfusepath{fill}%
\end{pgfscope}%
\begin{pgfscope}%
\pgfpathrectangle{\pgfqpoint{12.211765in}{4.326316in}}{\pgfqpoint{2.188235in}{0.953684in}} %
\pgfusepath{clip}%
\pgfsetbuttcap%
\pgfsetmiterjoin%
\definecolor{currentfill}{rgb}{0.121569,0.466667,0.705882}%
\pgfsetfillcolor{currentfill}%
\pgfsetlinewidth{0.000000pt}%
\definecolor{currentstroke}{rgb}{0.000000,0.000000,0.000000}%
\pgfsetstrokecolor{currentstroke}%
\pgfsetstrokeopacity{0.000000}%
\pgfsetdash{}{0pt}%
\pgfpathmoveto{\pgfqpoint{-23.809001in}{5.082018in}}%
\pgfpathlineto{\pgfqpoint{12.608315in}{5.082018in}}%
\pgfpathlineto{\pgfqpoint{12.608315in}{5.088968in}}%
\pgfpathlineto{\pgfqpoint{-23.809001in}{5.088968in}}%
\pgfpathclose%
\pgfusepath{fill}%
\end{pgfscope}%
\begin{pgfscope}%
\pgfpathrectangle{\pgfqpoint{12.211765in}{4.326316in}}{\pgfqpoint{2.188235in}{0.953684in}} %
\pgfusepath{clip}%
\pgfsetbuttcap%
\pgfsetmiterjoin%
\definecolor{currentfill}{rgb}{0.121569,0.466667,0.705882}%
\pgfsetfillcolor{currentfill}%
\pgfsetlinewidth{0.000000pt}%
\definecolor{currentstroke}{rgb}{0.000000,0.000000,0.000000}%
\pgfsetstrokecolor{currentstroke}%
\pgfsetstrokeopacity{0.000000}%
\pgfsetdash{}{0pt}%
\pgfpathmoveto{\pgfqpoint{-23.809001in}{5.090705in}}%
\pgfpathlineto{\pgfqpoint{12.540336in}{5.090705in}}%
\pgfpathlineto{\pgfqpoint{12.540336in}{5.097655in}}%
\pgfpathlineto{\pgfqpoint{-23.809001in}{5.097655in}}%
\pgfpathclose%
\pgfusepath{fill}%
\end{pgfscope}%
\begin{pgfscope}%
\pgfpathrectangle{\pgfqpoint{12.211765in}{4.326316in}}{\pgfqpoint{2.188235in}{0.953684in}} %
\pgfusepath{clip}%
\pgfsetbuttcap%
\pgfsetmiterjoin%
\definecolor{currentfill}{rgb}{0.121569,0.466667,0.705882}%
\pgfsetfillcolor{currentfill}%
\pgfsetlinewidth{0.000000pt}%
\definecolor{currentstroke}{rgb}{0.000000,0.000000,0.000000}%
\pgfsetstrokecolor{currentstroke}%
\pgfsetstrokeopacity{0.000000}%
\pgfsetdash{}{0pt}%
\pgfpathmoveto{\pgfqpoint{-23.809001in}{5.099392in}}%
\pgfpathlineto{\pgfqpoint{12.602927in}{5.099392in}}%
\pgfpathlineto{\pgfqpoint{12.602927in}{5.106342in}}%
\pgfpathlineto{\pgfqpoint{-23.809001in}{5.106342in}}%
\pgfpathclose%
\pgfusepath{fill}%
\end{pgfscope}%
\begin{pgfscope}%
\pgfpathrectangle{\pgfqpoint{12.211765in}{4.326316in}}{\pgfqpoint{2.188235in}{0.953684in}} %
\pgfusepath{clip}%
\pgfsetbuttcap%
\pgfsetmiterjoin%
\definecolor{currentfill}{rgb}{0.121569,0.466667,0.705882}%
\pgfsetfillcolor{currentfill}%
\pgfsetlinewidth{0.000000pt}%
\definecolor{currentstroke}{rgb}{0.000000,0.000000,0.000000}%
\pgfsetstrokecolor{currentstroke}%
\pgfsetstrokeopacity{0.000000}%
\pgfsetdash{}{0pt}%
\pgfpathmoveto{\pgfqpoint{-23.809001in}{5.108080in}}%
\pgfpathlineto{\pgfqpoint{12.625605in}{5.108080in}}%
\pgfpathlineto{\pgfqpoint{12.625605in}{5.115029in}}%
\pgfpathlineto{\pgfqpoint{-23.809001in}{5.115029in}}%
\pgfpathclose%
\pgfusepath{fill}%
\end{pgfscope}%
\begin{pgfscope}%
\pgfpathrectangle{\pgfqpoint{12.211765in}{4.326316in}}{\pgfqpoint{2.188235in}{0.953684in}} %
\pgfusepath{clip}%
\pgfsetbuttcap%
\pgfsetmiterjoin%
\definecolor{currentfill}{rgb}{0.121569,0.466667,0.705882}%
\pgfsetfillcolor{currentfill}%
\pgfsetlinewidth{0.000000pt}%
\definecolor{currentstroke}{rgb}{0.000000,0.000000,0.000000}%
\pgfsetstrokecolor{currentstroke}%
\pgfsetstrokeopacity{0.000000}%
\pgfsetdash{}{0pt}%
\pgfpathmoveto{\pgfqpoint{-23.809001in}{5.116767in}}%
\pgfpathlineto{\pgfqpoint{12.532220in}{5.116767in}}%
\pgfpathlineto{\pgfqpoint{12.532220in}{5.123717in}}%
\pgfpathlineto{\pgfqpoint{-23.809001in}{5.123717in}}%
\pgfpathclose%
\pgfusepath{fill}%
\end{pgfscope}%
\begin{pgfscope}%
\pgfpathrectangle{\pgfqpoint{12.211765in}{4.326316in}}{\pgfqpoint{2.188235in}{0.953684in}} %
\pgfusepath{clip}%
\pgfsetbuttcap%
\pgfsetmiterjoin%
\definecolor{currentfill}{rgb}{0.121569,0.466667,0.705882}%
\pgfsetfillcolor{currentfill}%
\pgfsetlinewidth{0.000000pt}%
\definecolor{currentstroke}{rgb}{0.000000,0.000000,0.000000}%
\pgfsetstrokecolor{currentstroke}%
\pgfsetstrokeopacity{0.000000}%
\pgfsetdash{}{0pt}%
\pgfpathmoveto{\pgfqpoint{-23.809001in}{5.125454in}}%
\pgfpathlineto{\pgfqpoint{12.584336in}{5.125454in}}%
\pgfpathlineto{\pgfqpoint{12.584336in}{5.132404in}}%
\pgfpathlineto{\pgfqpoint{-23.809001in}{5.132404in}}%
\pgfpathclose%
\pgfusepath{fill}%
\end{pgfscope}%
\begin{pgfscope}%
\pgfpathrectangle{\pgfqpoint{12.211765in}{4.326316in}}{\pgfqpoint{2.188235in}{0.953684in}} %
\pgfusepath{clip}%
\pgfsetbuttcap%
\pgfsetmiterjoin%
\definecolor{currentfill}{rgb}{0.121569,0.466667,0.705882}%
\pgfsetfillcolor{currentfill}%
\pgfsetlinewidth{0.000000pt}%
\definecolor{currentstroke}{rgb}{0.000000,0.000000,0.000000}%
\pgfsetstrokecolor{currentstroke}%
\pgfsetstrokeopacity{0.000000}%
\pgfsetdash{}{0pt}%
\pgfpathmoveto{\pgfqpoint{-23.809001in}{5.134141in}}%
\pgfpathlineto{\pgfqpoint{12.633560in}{5.134141in}}%
\pgfpathlineto{\pgfqpoint{12.633560in}{5.141091in}}%
\pgfpathlineto{\pgfqpoint{-23.809001in}{5.141091in}}%
\pgfpathclose%
\pgfusepath{fill}%
\end{pgfscope}%
\begin{pgfscope}%
\pgfpathrectangle{\pgfqpoint{12.211765in}{4.326316in}}{\pgfqpoint{2.188235in}{0.953684in}} %
\pgfusepath{clip}%
\pgfsetbuttcap%
\pgfsetmiterjoin%
\definecolor{currentfill}{rgb}{0.121569,0.466667,0.705882}%
\pgfsetfillcolor{currentfill}%
\pgfsetlinewidth{0.000000pt}%
\definecolor{currentstroke}{rgb}{0.000000,0.000000,0.000000}%
\pgfsetstrokecolor{currentstroke}%
\pgfsetstrokeopacity{0.000000}%
\pgfsetdash{}{0pt}%
\pgfpathmoveto{\pgfqpoint{-23.809001in}{5.142829in}}%
\pgfpathlineto{\pgfqpoint{12.610264in}{5.142829in}}%
\pgfpathlineto{\pgfqpoint{12.610264in}{5.149778in}}%
\pgfpathlineto{\pgfqpoint{-23.809001in}{5.149778in}}%
\pgfpathclose%
\pgfusepath{fill}%
\end{pgfscope}%
\begin{pgfscope}%
\pgfpathrectangle{\pgfqpoint{12.211765in}{4.326316in}}{\pgfqpoint{2.188235in}{0.953684in}} %
\pgfusepath{clip}%
\pgfsetbuttcap%
\pgfsetmiterjoin%
\definecolor{currentfill}{rgb}{0.121569,0.466667,0.705882}%
\pgfsetfillcolor{currentfill}%
\pgfsetlinewidth{0.000000pt}%
\definecolor{currentstroke}{rgb}{0.000000,0.000000,0.000000}%
\pgfsetstrokecolor{currentstroke}%
\pgfsetstrokeopacity{0.000000}%
\pgfsetdash{}{0pt}%
\pgfpathmoveto{\pgfqpoint{-23.809001in}{5.151516in}}%
\pgfpathlineto{\pgfqpoint{12.538233in}{5.151516in}}%
\pgfpathlineto{\pgfqpoint{12.538233in}{5.158466in}}%
\pgfpathlineto{\pgfqpoint{-23.809001in}{5.158466in}}%
\pgfpathclose%
\pgfusepath{fill}%
\end{pgfscope}%
\begin{pgfscope}%
\pgfpathrectangle{\pgfqpoint{12.211765in}{4.326316in}}{\pgfqpoint{2.188235in}{0.953684in}} %
\pgfusepath{clip}%
\pgfsetbuttcap%
\pgfsetmiterjoin%
\definecolor{currentfill}{rgb}{0.121569,0.466667,0.705882}%
\pgfsetfillcolor{currentfill}%
\pgfsetlinewidth{0.000000pt}%
\definecolor{currentstroke}{rgb}{0.000000,0.000000,0.000000}%
\pgfsetstrokecolor{currentstroke}%
\pgfsetstrokeopacity{0.000000}%
\pgfsetdash{}{0pt}%
\pgfpathmoveto{\pgfqpoint{-23.809001in}{5.160203in}}%
\pgfpathlineto{\pgfqpoint{12.595935in}{5.160203in}}%
\pgfpathlineto{\pgfqpoint{12.595935in}{5.167153in}}%
\pgfpathlineto{\pgfqpoint{-23.809001in}{5.167153in}}%
\pgfpathclose%
\pgfusepath{fill}%
\end{pgfscope}%
\begin{pgfscope}%
\pgfpathrectangle{\pgfqpoint{12.211765in}{4.326316in}}{\pgfqpoint{2.188235in}{0.953684in}} %
\pgfusepath{clip}%
\pgfsetbuttcap%
\pgfsetmiterjoin%
\definecolor{currentfill}{rgb}{0.121569,0.466667,0.705882}%
\pgfsetfillcolor{currentfill}%
\pgfsetlinewidth{0.000000pt}%
\definecolor{currentstroke}{rgb}{0.000000,0.000000,0.000000}%
\pgfsetstrokecolor{currentstroke}%
\pgfsetstrokeopacity{0.000000}%
\pgfsetdash{}{0pt}%
\pgfpathmoveto{\pgfqpoint{-23.809001in}{5.168890in}}%
\pgfpathlineto{\pgfqpoint{12.507427in}{5.168890in}}%
\pgfpathlineto{\pgfqpoint{12.507427in}{5.175840in}}%
\pgfpathlineto{\pgfqpoint{-23.809001in}{5.175840in}}%
\pgfpathclose%
\pgfusepath{fill}%
\end{pgfscope}%
\begin{pgfscope}%
\pgfpathrectangle{\pgfqpoint{12.211765in}{4.326316in}}{\pgfqpoint{2.188235in}{0.953684in}} %
\pgfusepath{clip}%
\pgfsetbuttcap%
\pgfsetmiterjoin%
\definecolor{currentfill}{rgb}{0.121569,0.466667,0.705882}%
\pgfsetfillcolor{currentfill}%
\pgfsetlinewidth{0.000000pt}%
\definecolor{currentstroke}{rgb}{0.000000,0.000000,0.000000}%
\pgfsetstrokecolor{currentstroke}%
\pgfsetstrokeopacity{0.000000}%
\pgfsetdash{}{0pt}%
\pgfpathmoveto{\pgfqpoint{-23.809001in}{5.177578in}}%
\pgfpathlineto{\pgfqpoint{12.598693in}{5.177578in}}%
\pgfpathlineto{\pgfqpoint{12.598693in}{5.184527in}}%
\pgfpathlineto{\pgfqpoint{-23.809001in}{5.184527in}}%
\pgfpathclose%
\pgfusepath{fill}%
\end{pgfscope}%
\begin{pgfscope}%
\pgfpathrectangle{\pgfqpoint{12.211765in}{4.326316in}}{\pgfqpoint{2.188235in}{0.953684in}} %
\pgfusepath{clip}%
\pgfsetbuttcap%
\pgfsetmiterjoin%
\definecolor{currentfill}{rgb}{0.121569,0.466667,0.705882}%
\pgfsetfillcolor{currentfill}%
\pgfsetlinewidth{0.000000pt}%
\definecolor{currentstroke}{rgb}{0.000000,0.000000,0.000000}%
\pgfsetstrokecolor{currentstroke}%
\pgfsetstrokeopacity{0.000000}%
\pgfsetdash{}{0pt}%
\pgfpathmoveto{\pgfqpoint{-23.809001in}{5.186265in}}%
\pgfpathlineto{\pgfqpoint{12.652048in}{5.186265in}}%
\pgfpathlineto{\pgfqpoint{12.652048in}{5.193215in}}%
\pgfpathlineto{\pgfqpoint{-23.809001in}{5.193215in}}%
\pgfpathclose%
\pgfusepath{fill}%
\end{pgfscope}%
\begin{pgfscope}%
\pgfpathrectangle{\pgfqpoint{12.211765in}{4.326316in}}{\pgfqpoint{2.188235in}{0.953684in}} %
\pgfusepath{clip}%
\pgfsetbuttcap%
\pgfsetmiterjoin%
\definecolor{currentfill}{rgb}{0.121569,0.466667,0.705882}%
\pgfsetfillcolor{currentfill}%
\pgfsetlinewidth{0.000000pt}%
\definecolor{currentstroke}{rgb}{0.000000,0.000000,0.000000}%
\pgfsetstrokecolor{currentstroke}%
\pgfsetstrokeopacity{0.000000}%
\pgfsetdash{}{0pt}%
\pgfpathmoveto{\pgfqpoint{-23.809001in}{5.194952in}}%
\pgfpathlineto{\pgfqpoint{12.619371in}{5.194952in}}%
\pgfpathlineto{\pgfqpoint{12.619371in}{5.201902in}}%
\pgfpathlineto{\pgfqpoint{-23.809001in}{5.201902in}}%
\pgfpathclose%
\pgfusepath{fill}%
\end{pgfscope}%
\begin{pgfscope}%
\pgfpathrectangle{\pgfqpoint{12.211765in}{4.326316in}}{\pgfqpoint{2.188235in}{0.953684in}} %
\pgfusepath{clip}%
\pgfsetbuttcap%
\pgfsetmiterjoin%
\definecolor{currentfill}{rgb}{0.121569,0.466667,0.705882}%
\pgfsetfillcolor{currentfill}%
\pgfsetlinewidth{0.000000pt}%
\definecolor{currentstroke}{rgb}{0.000000,0.000000,0.000000}%
\pgfsetstrokecolor{currentstroke}%
\pgfsetstrokeopacity{0.000000}%
\pgfsetdash{}{0pt}%
\pgfpathmoveto{\pgfqpoint{-23.809001in}{5.203639in}}%
\pgfpathlineto{\pgfqpoint{12.590061in}{5.203639in}}%
\pgfpathlineto{\pgfqpoint{12.590061in}{5.210589in}}%
\pgfpathlineto{\pgfqpoint{-23.809001in}{5.210589in}}%
\pgfpathclose%
\pgfusepath{fill}%
\end{pgfscope}%
\begin{pgfscope}%
\pgfpathrectangle{\pgfqpoint{12.211765in}{4.326316in}}{\pgfqpoint{2.188235in}{0.953684in}} %
\pgfusepath{clip}%
\pgfsetbuttcap%
\pgfsetmiterjoin%
\definecolor{currentfill}{rgb}{0.121569,0.466667,0.705882}%
\pgfsetfillcolor{currentfill}%
\pgfsetlinewidth{0.000000pt}%
\definecolor{currentstroke}{rgb}{0.000000,0.000000,0.000000}%
\pgfsetstrokecolor{currentstroke}%
\pgfsetstrokeopacity{0.000000}%
\pgfsetdash{}{0pt}%
\pgfpathmoveto{\pgfqpoint{-23.809001in}{5.212326in}}%
\pgfpathlineto{\pgfqpoint{12.484285in}{5.212326in}}%
\pgfpathlineto{\pgfqpoint{12.484285in}{5.219276in}}%
\pgfpathlineto{\pgfqpoint{-23.809001in}{5.219276in}}%
\pgfpathclose%
\pgfusepath{fill}%
\end{pgfscope}%
\begin{pgfscope}%
\pgfpathrectangle{\pgfqpoint{12.211765in}{4.326316in}}{\pgfqpoint{2.188235in}{0.953684in}} %
\pgfusepath{clip}%
\pgfsetbuttcap%
\pgfsetmiterjoin%
\definecolor{currentfill}{rgb}{0.121569,0.466667,0.705882}%
\pgfsetfillcolor{currentfill}%
\pgfsetlinewidth{0.000000pt}%
\definecolor{currentstroke}{rgb}{0.000000,0.000000,0.000000}%
\pgfsetstrokecolor{currentstroke}%
\pgfsetstrokeopacity{0.000000}%
\pgfsetdash{}{0pt}%
\pgfpathmoveto{\pgfqpoint{-23.809001in}{5.221014in}}%
\pgfpathlineto{\pgfqpoint{12.577944in}{5.221014in}}%
\pgfpathlineto{\pgfqpoint{12.577944in}{5.227963in}}%
\pgfpathlineto{\pgfqpoint{-23.809001in}{5.227963in}}%
\pgfpathclose%
\pgfusepath{fill}%
\end{pgfscope}%
\begin{pgfscope}%
\pgfpathrectangle{\pgfqpoint{12.211765in}{4.326316in}}{\pgfqpoint{2.188235in}{0.953684in}} %
\pgfusepath{clip}%
\pgfsetbuttcap%
\pgfsetmiterjoin%
\definecolor{currentfill}{rgb}{0.121569,0.466667,0.705882}%
\pgfsetfillcolor{currentfill}%
\pgfsetlinewidth{0.000000pt}%
\definecolor{currentstroke}{rgb}{0.000000,0.000000,0.000000}%
\pgfsetstrokecolor{currentstroke}%
\pgfsetstrokeopacity{0.000000}%
\pgfsetdash{}{0pt}%
\pgfpathmoveto{\pgfqpoint{-23.809001in}{5.229701in}}%
\pgfpathlineto{\pgfqpoint{12.555668in}{5.229701in}}%
\pgfpathlineto{\pgfqpoint{12.555668in}{5.236651in}}%
\pgfpathlineto{\pgfqpoint{-23.809001in}{5.236651in}}%
\pgfpathclose%
\pgfusepath{fill}%
\end{pgfscope}%
\begin{pgfscope}%
\pgfsetbuttcap%
\pgfsetroundjoin%
\definecolor{currentfill}{rgb}{0.000000,0.000000,0.000000}%
\pgfsetfillcolor{currentfill}%
\pgfsetlinewidth{0.803000pt}%
\definecolor{currentstroke}{rgb}{0.000000,0.000000,0.000000}%
\pgfsetstrokecolor{currentstroke}%
\pgfsetdash{}{0pt}%
\pgfsys@defobject{currentmarker}{\pgfqpoint{0.000000in}{-0.048611in}}{\pgfqpoint{0.000000in}{0.000000in}}{%
\pgfpathmoveto{\pgfqpoint{0.000000in}{0.000000in}}%
\pgfpathlineto{\pgfqpoint{0.000000in}{-0.048611in}}%
\pgfusepath{stroke,fill}%
}%
\begin{pgfscope}%
\pgfsys@transformshift{12.624272in}{4.326316in}%
\pgfsys@useobject{currentmarker}{}%
\end{pgfscope}%
\end{pgfscope}%
\begin{pgfscope}%
\pgfsetbuttcap%
\pgfsetroundjoin%
\definecolor{currentfill}{rgb}{0.000000,0.000000,0.000000}%
\pgfsetfillcolor{currentfill}%
\pgfsetlinewidth{0.803000pt}%
\definecolor{currentstroke}{rgb}{0.000000,0.000000,0.000000}%
\pgfsetstrokecolor{currentstroke}%
\pgfsetdash{}{0pt}%
\pgfsys@defobject{currentmarker}{\pgfqpoint{0.000000in}{-0.048611in}}{\pgfqpoint{0.000000in}{0.000000in}}{%
\pgfpathmoveto{\pgfqpoint{0.000000in}{0.000000in}}%
\pgfpathlineto{\pgfqpoint{0.000000in}{-0.048611in}}%
\pgfusepath{stroke,fill}%
}%
\begin{pgfscope}%
\pgfsys@transformshift{13.130289in}{4.326316in}%
\pgfsys@useobject{currentmarker}{}%
\end{pgfscope}%
\end{pgfscope}%
\begin{pgfscope}%
\pgfsetbuttcap%
\pgfsetroundjoin%
\definecolor{currentfill}{rgb}{0.000000,0.000000,0.000000}%
\pgfsetfillcolor{currentfill}%
\pgfsetlinewidth{0.803000pt}%
\definecolor{currentstroke}{rgb}{0.000000,0.000000,0.000000}%
\pgfsetstrokecolor{currentstroke}%
\pgfsetdash{}{0pt}%
\pgfsys@defobject{currentmarker}{\pgfqpoint{0.000000in}{-0.048611in}}{\pgfqpoint{0.000000in}{0.000000in}}{%
\pgfpathmoveto{\pgfqpoint{0.000000in}{0.000000in}}%
\pgfpathlineto{\pgfqpoint{0.000000in}{-0.048611in}}%
\pgfusepath{stroke,fill}%
}%
\begin{pgfscope}%
\pgfsys@transformshift{13.636307in}{4.326316in}%
\pgfsys@useobject{currentmarker}{}%
\end{pgfscope}%
\end{pgfscope}%
\begin{pgfscope}%
\pgfsetbuttcap%
\pgfsetroundjoin%
\definecolor{currentfill}{rgb}{0.000000,0.000000,0.000000}%
\pgfsetfillcolor{currentfill}%
\pgfsetlinewidth{0.803000pt}%
\definecolor{currentstroke}{rgb}{0.000000,0.000000,0.000000}%
\pgfsetstrokecolor{currentstroke}%
\pgfsetdash{}{0pt}%
\pgfsys@defobject{currentmarker}{\pgfqpoint{0.000000in}{-0.048611in}}{\pgfqpoint{0.000000in}{0.000000in}}{%
\pgfpathmoveto{\pgfqpoint{0.000000in}{0.000000in}}%
\pgfpathlineto{\pgfqpoint{0.000000in}{-0.048611in}}%
\pgfusepath{stroke,fill}%
}%
\begin{pgfscope}%
\pgfsys@transformshift{14.142325in}{4.326316in}%
\pgfsys@useobject{currentmarker}{}%
\end{pgfscope}%
\end{pgfscope}%
\begin{pgfscope}%
\pgfsetbuttcap%
\pgfsetroundjoin%
\definecolor{currentfill}{rgb}{0.000000,0.000000,0.000000}%
\pgfsetfillcolor{currentfill}%
\pgfsetlinewidth{0.803000pt}%
\definecolor{currentstroke}{rgb}{0.000000,0.000000,0.000000}%
\pgfsetstrokecolor{currentstroke}%
\pgfsetdash{}{0pt}%
\pgfsys@defobject{currentmarker}{\pgfqpoint{-0.048611in}{0.000000in}}{\pgfqpoint{0.000000in}{0.000000in}}{%
\pgfpathmoveto{\pgfqpoint{0.000000in}{0.000000in}}%
\pgfpathlineto{\pgfqpoint{-0.048611in}{0.000000in}}%
\pgfusepath{stroke,fill}%
}%
\begin{pgfscope}%
\pgfsys@transformshift{12.211765in}{4.373140in}%
\pgfsys@useobject{currentmarker}{}%
\end{pgfscope}%
\end{pgfscope}%
\begin{pgfscope}%
\pgftext[x=12.045098in,y=4.324922in,left,base]{\rmfamily\fontsize{10.000000}{12.000000}\selectfont \(\displaystyle 0\)}%
\end{pgfscope}%
\begin{pgfscope}%
\pgfsetbuttcap%
\pgfsetroundjoin%
\definecolor{currentfill}{rgb}{0.000000,0.000000,0.000000}%
\pgfsetfillcolor{currentfill}%
\pgfsetlinewidth{0.803000pt}%
\definecolor{currentstroke}{rgb}{0.000000,0.000000,0.000000}%
\pgfsetstrokecolor{currentstroke}%
\pgfsetdash{}{0pt}%
\pgfsys@defobject{currentmarker}{\pgfqpoint{-0.048611in}{0.000000in}}{\pgfqpoint{0.000000in}{0.000000in}}{%
\pgfpathmoveto{\pgfqpoint{0.000000in}{0.000000in}}%
\pgfpathlineto{\pgfqpoint{-0.048611in}{0.000000in}}%
\pgfusepath{stroke,fill}%
}%
\begin{pgfscope}%
\pgfsys@transformshift{12.211765in}{4.807502in}%
\pgfsys@useobject{currentmarker}{}%
\end{pgfscope}%
\end{pgfscope}%
\begin{pgfscope}%
\pgftext[x=11.975653in,y=4.759284in,left,base]{\rmfamily\fontsize{10.000000}{12.000000}\selectfont \(\displaystyle 50\)}%
\end{pgfscope}%
\begin{pgfscope}%
\pgfsetbuttcap%
\pgfsetroundjoin%
\definecolor{currentfill}{rgb}{0.000000,0.000000,0.000000}%
\pgfsetfillcolor{currentfill}%
\pgfsetlinewidth{0.803000pt}%
\definecolor{currentstroke}{rgb}{0.000000,0.000000,0.000000}%
\pgfsetstrokecolor{currentstroke}%
\pgfsetdash{}{0pt}%
\pgfsys@defobject{currentmarker}{\pgfqpoint{-0.048611in}{0.000000in}}{\pgfqpoint{0.000000in}{0.000000in}}{%
\pgfpathmoveto{\pgfqpoint{0.000000in}{0.000000in}}%
\pgfpathlineto{\pgfqpoint{-0.048611in}{0.000000in}}%
\pgfusepath{stroke,fill}%
}%
\begin{pgfscope}%
\pgfsys@transformshift{12.211765in}{5.241863in}%
\pgfsys@useobject{currentmarker}{}%
\end{pgfscope}%
\end{pgfscope}%
\begin{pgfscope}%
\pgftext[x=11.906208in,y=5.193645in,left,base]{\rmfamily\fontsize{10.000000}{12.000000}\selectfont \(\displaystyle 100\)}%
\end{pgfscope}%
\begin{pgfscope}%
\pgftext[x=11.850653in,y=4.803158in,,bottom,rotate=90.000000]{\rmfamily\fontsize{10.000000}{12.000000}\selectfont \(\displaystyle j\)}%
\end{pgfscope}%
\begin{pgfscope}%
\pgfsetrectcap%
\pgfsetmiterjoin%
\pgfsetlinewidth{0.803000pt}%
\definecolor{currentstroke}{rgb}{0.000000,0.000000,0.000000}%
\pgfsetstrokecolor{currentstroke}%
\pgfsetdash{}{0pt}%
\pgfpathmoveto{\pgfqpoint{12.211765in}{4.326316in}}%
\pgfpathlineto{\pgfqpoint{12.211765in}{5.280000in}}%
\pgfusepath{stroke}%
\end{pgfscope}%
\begin{pgfscope}%
\pgfsetrectcap%
\pgfsetmiterjoin%
\pgfsetlinewidth{0.803000pt}%
\definecolor{currentstroke}{rgb}{0.000000,0.000000,0.000000}%
\pgfsetstrokecolor{currentstroke}%
\pgfsetdash{}{0pt}%
\pgfpathmoveto{\pgfqpoint{14.400000in}{4.326316in}}%
\pgfpathlineto{\pgfqpoint{14.400000in}{5.280000in}}%
\pgfusepath{stroke}%
\end{pgfscope}%
\begin{pgfscope}%
\pgfsetrectcap%
\pgfsetmiterjoin%
\pgfsetlinewidth{0.803000pt}%
\definecolor{currentstroke}{rgb}{0.000000,0.000000,0.000000}%
\pgfsetstrokecolor{currentstroke}%
\pgfsetdash{}{0pt}%
\pgfpathmoveto{\pgfqpoint{12.211765in}{4.326316in}}%
\pgfpathlineto{\pgfqpoint{14.400000in}{4.326316in}}%
\pgfusepath{stroke}%
\end{pgfscope}%
\begin{pgfscope}%
\pgfsetrectcap%
\pgfsetmiterjoin%
\pgfsetlinewidth{0.803000pt}%
\definecolor{currentstroke}{rgb}{0.000000,0.000000,0.000000}%
\pgfsetstrokecolor{currentstroke}%
\pgfsetdash{}{0pt}%
\pgfpathmoveto{\pgfqpoint{12.211765in}{5.280000in}}%
\pgfpathlineto{\pgfqpoint{14.400000in}{5.280000in}}%
\pgfusepath{stroke}%
\end{pgfscope}%
\begin{pgfscope}%
\pgfsetbuttcap%
\pgfsetmiterjoin%
\definecolor{currentfill}{rgb}{1.000000,1.000000,1.000000}%
\pgfsetfillcolor{currentfill}%
\pgfsetlinewidth{0.000000pt}%
\definecolor{currentstroke}{rgb}{0.000000,0.000000,0.000000}%
\pgfsetstrokecolor{currentstroke}%
\pgfsetstrokeopacity{0.000000}%
\pgfsetdash{}{0pt}%
\pgfpathmoveto{\pgfqpoint{2.000000in}{3.134211in}}%
\pgfpathlineto{\pgfqpoint{6.376471in}{3.134211in}}%
\pgfpathlineto{\pgfqpoint{6.376471in}{4.087895in}}%
\pgfpathlineto{\pgfqpoint{2.000000in}{4.087895in}}%
\pgfpathclose%
\pgfusepath{fill}%
\end{pgfscope}%
\begin{pgfscope}%
\pgfpathrectangle{\pgfqpoint{2.000000in}{3.134211in}}{\pgfqpoint{4.376471in}{0.953684in}} %
\pgfusepath{clip}%
\pgfsetbuttcap%
\pgfsetroundjoin%
\definecolor{currentfill}{rgb}{1.000000,0.000000,0.000000}%
\pgfsetfillcolor{currentfill}%
\pgfsetlinewidth{2.007500pt}%
\definecolor{currentstroke}{rgb}{1.000000,0.000000,0.000000}%
\pgfsetstrokecolor{currentstroke}%
\pgfsetdash{}{0pt}%
\pgfpathmoveto{\pgfqpoint{4.755120in}{3.490718in}}%
\pgfpathlineto{\pgfqpoint{4.838454in}{3.490718in}}%
\pgfpathmoveto{\pgfqpoint{4.796787in}{3.449051in}}%
\pgfpathlineto{\pgfqpoint{4.796787in}{3.532384in}}%
\pgfusepath{stroke,fill}%
\end{pgfscope}%
\begin{pgfscope}%
\pgfpathrectangle{\pgfqpoint{2.000000in}{3.134211in}}{\pgfqpoint{4.376471in}{0.953684in}} %
\pgfusepath{clip}%
\pgfsetbuttcap%
\pgfsetroundjoin%
\definecolor{currentfill}{rgb}{1.000000,0.000000,0.000000}%
\pgfsetfillcolor{currentfill}%
\pgfsetlinewidth{2.007500pt}%
\definecolor{currentstroke}{rgb}{1.000000,0.000000,0.000000}%
\pgfsetstrokecolor{currentstroke}%
\pgfsetdash{}{0pt}%
\pgfpathmoveto{\pgfqpoint{5.337632in}{3.576911in}}%
\pgfpathlineto{\pgfqpoint{5.420965in}{3.576911in}}%
\pgfpathmoveto{\pgfqpoint{5.379298in}{3.535244in}}%
\pgfpathlineto{\pgfqpoint{5.379298in}{3.618578in}}%
\pgfusepath{stroke,fill}%
\end{pgfscope}%
\begin{pgfscope}%
\pgfpathrectangle{\pgfqpoint{2.000000in}{3.134211in}}{\pgfqpoint{4.376471in}{0.953684in}} %
\pgfusepath{clip}%
\pgfsetbuttcap%
\pgfsetroundjoin%
\definecolor{currentfill}{rgb}{1.000000,0.000000,0.000000}%
\pgfsetfillcolor{currentfill}%
\pgfsetlinewidth{2.007500pt}%
\definecolor{currentstroke}{rgb}{1.000000,0.000000,0.000000}%
\pgfsetstrokecolor{currentstroke}%
\pgfsetdash{}{0pt}%
\pgfpathmoveto{\pgfqpoint{4.944008in}{3.633817in}}%
\pgfpathlineto{\pgfqpoint{5.027342in}{3.633817in}}%
\pgfpathmoveto{\pgfqpoint{4.985675in}{3.592151in}}%
\pgfpathlineto{\pgfqpoint{4.985675in}{3.675484in}}%
\pgfusepath{stroke,fill}%
\end{pgfscope}%
\begin{pgfscope}%
\pgfpathrectangle{\pgfqpoint{2.000000in}{3.134211in}}{\pgfqpoint{4.376471in}{0.953684in}} %
\pgfusepath{clip}%
\pgfsetbuttcap%
\pgfsetroundjoin%
\definecolor{currentfill}{rgb}{1.000000,0.000000,0.000000}%
\pgfsetfillcolor{currentfill}%
\pgfsetlinewidth{2.007500pt}%
\definecolor{currentstroke}{rgb}{1.000000,0.000000,0.000000}%
\pgfsetstrokecolor{currentstroke}%
\pgfsetdash{}{0pt}%
\pgfpathmoveto{\pgfqpoint{4.741360in}{3.606871in}}%
\pgfpathlineto{\pgfqpoint{4.824693in}{3.606871in}}%
\pgfpathmoveto{\pgfqpoint{4.783026in}{3.565204in}}%
\pgfpathlineto{\pgfqpoint{4.783026in}{3.648537in}}%
\pgfusepath{stroke,fill}%
\end{pgfscope}%
\begin{pgfscope}%
\pgfpathrectangle{\pgfqpoint{2.000000in}{3.134211in}}{\pgfqpoint{4.376471in}{0.953684in}} %
\pgfusepath{clip}%
\pgfsetbuttcap%
\pgfsetroundjoin%
\definecolor{currentfill}{rgb}{1.000000,0.000000,0.000000}%
\pgfsetfillcolor{currentfill}%
\pgfsetlinewidth{2.007500pt}%
\definecolor{currentstroke}{rgb}{1.000000,0.000000,0.000000}%
\pgfsetstrokecolor{currentstroke}%
\pgfsetdash{}{0pt}%
\pgfpathmoveto{\pgfqpoint{4.316918in}{3.178158in}}%
\pgfpathlineto{\pgfqpoint{4.400251in}{3.178158in}}%
\pgfpathmoveto{\pgfqpoint{4.358584in}{3.136492in}}%
\pgfpathlineto{\pgfqpoint{4.358584in}{3.219825in}}%
\pgfusepath{stroke,fill}%
\end{pgfscope}%
\begin{pgfscope}%
\pgfpathrectangle{\pgfqpoint{2.000000in}{3.134211in}}{\pgfqpoint{4.376471in}{0.953684in}} %
\pgfusepath{clip}%
\pgfsetbuttcap%
\pgfsetroundjoin%
\definecolor{currentfill}{rgb}{1.000000,0.000000,0.000000}%
\pgfsetfillcolor{currentfill}%
\pgfsetlinewidth{2.007500pt}%
\definecolor{currentstroke}{rgb}{1.000000,0.000000,0.000000}%
\pgfsetstrokecolor{currentstroke}%
\pgfsetdash{}{0pt}%
\pgfpathmoveto{\pgfqpoint{5.095017in}{3.646990in}}%
\pgfpathlineto{\pgfqpoint{5.178350in}{3.646990in}}%
\pgfpathmoveto{\pgfqpoint{5.136683in}{3.605323in}}%
\pgfpathlineto{\pgfqpoint{5.136683in}{3.688656in}}%
\pgfusepath{stroke,fill}%
\end{pgfscope}%
\begin{pgfscope}%
\pgfpathrectangle{\pgfqpoint{2.000000in}{3.134211in}}{\pgfqpoint{4.376471in}{0.953684in}} %
\pgfusepath{clip}%
\pgfsetbuttcap%
\pgfsetroundjoin%
\definecolor{currentfill}{rgb}{1.000000,0.000000,0.000000}%
\pgfsetfillcolor{currentfill}%
\pgfsetlinewidth{2.007500pt}%
\definecolor{currentstroke}{rgb}{1.000000,0.000000,0.000000}%
\pgfsetstrokecolor{currentstroke}%
\pgfsetdash{}{0pt}%
\pgfpathmoveto{\pgfqpoint{4.365697in}{3.149267in}}%
\pgfpathlineto{\pgfqpoint{4.449031in}{3.149267in}}%
\pgfpathmoveto{\pgfqpoint{4.407364in}{3.107600in}}%
\pgfpathlineto{\pgfqpoint{4.407364in}{3.190934in}}%
\pgfusepath{stroke,fill}%
\end{pgfscope}%
\begin{pgfscope}%
\pgfpathrectangle{\pgfqpoint{2.000000in}{3.134211in}}{\pgfqpoint{4.376471in}{0.953684in}} %
\pgfusepath{clip}%
\pgfsetbuttcap%
\pgfsetroundjoin%
\definecolor{currentfill}{rgb}{1.000000,0.000000,0.000000}%
\pgfsetfillcolor{currentfill}%
\pgfsetlinewidth{2.007500pt}%
\definecolor{currentstroke}{rgb}{1.000000,0.000000,0.000000}%
\pgfsetstrokecolor{currentstroke}%
\pgfsetdash{}{0pt}%
\pgfpathmoveto{\pgfqpoint{5.955882in}{3.913320in}}%
\pgfpathlineto{\pgfqpoint{6.039215in}{3.913320in}}%
\pgfpathmoveto{\pgfqpoint{5.997549in}{3.871654in}}%
\pgfpathlineto{\pgfqpoint{5.997549in}{3.954987in}}%
\pgfusepath{stroke,fill}%
\end{pgfscope}%
\begin{pgfscope}%
\pgfpathrectangle{\pgfqpoint{2.000000in}{3.134211in}}{\pgfqpoint{4.376471in}{0.953684in}} %
\pgfusepath{clip}%
\pgfsetbuttcap%
\pgfsetroundjoin%
\definecolor{currentfill}{rgb}{1.000000,0.000000,0.000000}%
\pgfsetfillcolor{currentfill}%
\pgfsetlinewidth{2.007500pt}%
\definecolor{currentstroke}{rgb}{1.000000,0.000000,0.000000}%
\pgfsetstrokecolor{currentstroke}%
\pgfsetdash{}{0pt}%
\pgfpathmoveto{\pgfqpoint{6.207581in}{3.774585in}}%
\pgfpathlineto{\pgfqpoint{6.290914in}{3.774585in}}%
\pgfpathmoveto{\pgfqpoint{6.249248in}{3.732918in}}%
\pgfpathlineto{\pgfqpoint{6.249248in}{3.816251in}}%
\pgfusepath{stroke,fill}%
\end{pgfscope}%
\begin{pgfscope}%
\pgfpathrectangle{\pgfqpoint{2.000000in}{3.134211in}}{\pgfqpoint{4.376471in}{0.953684in}} %
\pgfusepath{clip}%
\pgfsetbuttcap%
\pgfsetroundjoin%
\definecolor{currentfill}{rgb}{1.000000,0.000000,0.000000}%
\pgfsetfillcolor{currentfill}%
\pgfsetlinewidth{2.007500pt}%
\definecolor{currentstroke}{rgb}{1.000000,0.000000,0.000000}%
\pgfsetstrokecolor{currentstroke}%
\pgfsetdash{}{0pt}%
\pgfpathmoveto{\pgfqpoint{4.176124in}{3.407910in}}%
\pgfpathlineto{\pgfqpoint{4.259457in}{3.407910in}}%
\pgfpathmoveto{\pgfqpoint{4.217791in}{3.366243in}}%
\pgfpathlineto{\pgfqpoint{4.217791in}{3.449577in}}%
\pgfusepath{stroke,fill}%
\end{pgfscope}%
\begin{pgfscope}%
\pgfpathrectangle{\pgfqpoint{2.000000in}{3.134211in}}{\pgfqpoint{4.376471in}{0.953684in}} %
\pgfusepath{clip}%
\pgfsetbuttcap%
\pgfsetroundjoin%
\definecolor{currentfill}{rgb}{1.000000,0.000000,0.000000}%
\pgfsetfillcolor{currentfill}%
\pgfsetlinewidth{2.007500pt}%
\definecolor{currentstroke}{rgb}{1.000000,0.000000,0.000000}%
\pgfsetstrokecolor{currentstroke}%
\pgfsetdash{}{0pt}%
\pgfpathmoveto{\pgfqpoint{5.605597in}{3.490182in}}%
\pgfpathlineto{\pgfqpoint{5.688930in}{3.490182in}}%
\pgfpathmoveto{\pgfqpoint{5.647263in}{3.448515in}}%
\pgfpathlineto{\pgfqpoint{5.647263in}{3.531849in}}%
\pgfusepath{stroke,fill}%
\end{pgfscope}%
\begin{pgfscope}%
\pgfpathrectangle{\pgfqpoint{2.000000in}{3.134211in}}{\pgfqpoint{4.376471in}{0.953684in}} %
\pgfusepath{clip}%
\pgfsetbuttcap%
\pgfsetroundjoin%
\definecolor{currentfill}{rgb}{1.000000,0.000000,0.000000}%
\pgfsetfillcolor{currentfill}%
\pgfsetlinewidth{2.007500pt}%
\definecolor{currentstroke}{rgb}{1.000000,0.000000,0.000000}%
\pgfsetstrokecolor{currentstroke}%
\pgfsetdash{}{0pt}%
\pgfpathmoveto{\pgfqpoint{4.685382in}{3.484417in}}%
\pgfpathlineto{\pgfqpoint{4.768715in}{3.484417in}}%
\pgfpathmoveto{\pgfqpoint{4.727049in}{3.442750in}}%
\pgfpathlineto{\pgfqpoint{4.727049in}{3.526084in}}%
\pgfusepath{stroke,fill}%
\end{pgfscope}%
\begin{pgfscope}%
\pgfpathrectangle{\pgfqpoint{2.000000in}{3.134211in}}{\pgfqpoint{4.376471in}{0.953684in}} %
\pgfusepath{clip}%
\pgfsetbuttcap%
\pgfsetroundjoin%
\definecolor{currentfill}{rgb}{1.000000,0.000000,0.000000}%
\pgfsetfillcolor{currentfill}%
\pgfsetlinewidth{2.007500pt}%
\definecolor{currentstroke}{rgb}{1.000000,0.000000,0.000000}%
\pgfsetstrokecolor{currentstroke}%
\pgfsetdash{}{0pt}%
\pgfpathmoveto{\pgfqpoint{4.822452in}{3.690725in}}%
\pgfpathlineto{\pgfqpoint{4.905785in}{3.690725in}}%
\pgfpathmoveto{\pgfqpoint{4.864118in}{3.649058in}}%
\pgfpathlineto{\pgfqpoint{4.864118in}{3.732392in}}%
\pgfusepath{stroke,fill}%
\end{pgfscope}%
\begin{pgfscope}%
\pgfpathrectangle{\pgfqpoint{2.000000in}{3.134211in}}{\pgfqpoint{4.376471in}{0.953684in}} %
\pgfusepath{clip}%
\pgfsetbuttcap%
\pgfsetroundjoin%
\definecolor{currentfill}{rgb}{1.000000,0.000000,0.000000}%
\pgfsetfillcolor{currentfill}%
\pgfsetlinewidth{2.007500pt}%
\definecolor{currentstroke}{rgb}{1.000000,0.000000,0.000000}%
\pgfsetstrokecolor{currentstroke}%
\pgfsetdash{}{0pt}%
\pgfpathmoveto{\pgfqpoint{6.074305in}{3.979507in}}%
\pgfpathlineto{\pgfqpoint{6.157638in}{3.979507in}}%
\pgfpathmoveto{\pgfqpoint{6.115971in}{3.937840in}}%
\pgfpathlineto{\pgfqpoint{6.115971in}{4.021174in}}%
\pgfusepath{stroke,fill}%
\end{pgfscope}%
\begin{pgfscope}%
\pgfpathrectangle{\pgfqpoint{2.000000in}{3.134211in}}{\pgfqpoint{4.376471in}{0.953684in}} %
\pgfusepath{clip}%
\pgfsetbuttcap%
\pgfsetroundjoin%
\definecolor{currentfill}{rgb}{1.000000,0.000000,0.000000}%
\pgfsetfillcolor{currentfill}%
\pgfsetlinewidth{2.007500pt}%
\definecolor{currentstroke}{rgb}{1.000000,0.000000,0.000000}%
\pgfsetstrokecolor{currentstroke}%
\pgfsetdash{}{0pt}%
\pgfpathmoveto{\pgfqpoint{3.082337in}{3.642253in}}%
\pgfpathlineto{\pgfqpoint{3.165671in}{3.642253in}}%
\pgfpathmoveto{\pgfqpoint{3.124004in}{3.600587in}}%
\pgfpathlineto{\pgfqpoint{3.124004in}{3.683920in}}%
\pgfusepath{stroke,fill}%
\end{pgfscope}%
\begin{pgfscope}%
\pgfpathrectangle{\pgfqpoint{2.000000in}{3.134211in}}{\pgfqpoint{4.376471in}{0.953684in}} %
\pgfusepath{clip}%
\pgfsetbuttcap%
\pgfsetroundjoin%
\definecolor{currentfill}{rgb}{1.000000,0.000000,0.000000}%
\pgfsetfillcolor{currentfill}%
\pgfsetlinewidth{2.007500pt}%
\definecolor{currentstroke}{rgb}{1.000000,0.000000,0.000000}%
\pgfsetstrokecolor{currentstroke}%
\pgfsetdash{}{0pt}%
\pgfpathmoveto{\pgfqpoint{3.138683in}{3.610102in}}%
\pgfpathlineto{\pgfqpoint{3.222016in}{3.610102in}}%
\pgfpathmoveto{\pgfqpoint{3.180349in}{3.568435in}}%
\pgfpathlineto{\pgfqpoint{3.180349in}{3.651769in}}%
\pgfusepath{stroke,fill}%
\end{pgfscope}%
\begin{pgfscope}%
\pgfpathrectangle{\pgfqpoint{2.000000in}{3.134211in}}{\pgfqpoint{4.376471in}{0.953684in}} %
\pgfusepath{clip}%
\pgfsetbuttcap%
\pgfsetroundjoin%
\definecolor{currentfill}{rgb}{1.000000,0.000000,0.000000}%
\pgfsetfillcolor{currentfill}%
\pgfsetlinewidth{2.007500pt}%
\definecolor{currentstroke}{rgb}{1.000000,0.000000,0.000000}%
\pgfsetstrokecolor{currentstroke}%
\pgfsetdash{}{0pt}%
\pgfpathmoveto{\pgfqpoint{2.904416in}{3.828490in}}%
\pgfpathlineto{\pgfqpoint{2.987749in}{3.828490in}}%
\pgfpathmoveto{\pgfqpoint{2.946082in}{3.786823in}}%
\pgfpathlineto{\pgfqpoint{2.946082in}{3.870157in}}%
\pgfusepath{stroke,fill}%
\end{pgfscope}%
\begin{pgfscope}%
\pgfpathrectangle{\pgfqpoint{2.000000in}{3.134211in}}{\pgfqpoint{4.376471in}{0.953684in}} %
\pgfusepath{clip}%
\pgfsetbuttcap%
\pgfsetroundjoin%
\definecolor{currentfill}{rgb}{1.000000,0.000000,0.000000}%
\pgfsetfillcolor{currentfill}%
\pgfsetlinewidth{2.007500pt}%
\definecolor{currentstroke}{rgb}{1.000000,0.000000,0.000000}%
\pgfsetstrokecolor{currentstroke}%
\pgfsetdash{}{0pt}%
\pgfpathmoveto{\pgfqpoint{5.748776in}{3.703361in}}%
\pgfpathlineto{\pgfqpoint{5.832110in}{3.703361in}}%
\pgfpathmoveto{\pgfqpoint{5.790443in}{3.661694in}}%
\pgfpathlineto{\pgfqpoint{5.790443in}{3.745027in}}%
\pgfusepath{stroke,fill}%
\end{pgfscope}%
\begin{pgfscope}%
\pgfpathrectangle{\pgfqpoint{2.000000in}{3.134211in}}{\pgfqpoint{4.376471in}{0.953684in}} %
\pgfusepath{clip}%
\pgfsetbuttcap%
\pgfsetroundjoin%
\definecolor{currentfill}{rgb}{1.000000,0.000000,0.000000}%
\pgfsetfillcolor{currentfill}%
\pgfsetlinewidth{2.007500pt}%
\definecolor{currentstroke}{rgb}{1.000000,0.000000,0.000000}%
\pgfsetstrokecolor{currentstroke}%
\pgfsetdash{}{0pt}%
\pgfpathmoveto{\pgfqpoint{5.558092in}{3.417001in}}%
\pgfpathlineto{\pgfqpoint{5.641425in}{3.417001in}}%
\pgfpathmoveto{\pgfqpoint{5.599758in}{3.375334in}}%
\pgfpathlineto{\pgfqpoint{5.599758in}{3.458668in}}%
\pgfusepath{stroke,fill}%
\end{pgfscope}%
\begin{pgfscope}%
\pgfpathrectangle{\pgfqpoint{2.000000in}{3.134211in}}{\pgfqpoint{4.376471in}{0.953684in}} %
\pgfusepath{clip}%
\pgfsetbuttcap%
\pgfsetroundjoin%
\definecolor{currentfill}{rgb}{1.000000,0.000000,0.000000}%
\pgfsetfillcolor{currentfill}%
\pgfsetlinewidth{2.007500pt}%
\definecolor{currentstroke}{rgb}{1.000000,0.000000,0.000000}%
\pgfsetstrokecolor{currentstroke}%
\pgfsetdash{}{0pt}%
\pgfpathmoveto{\pgfqpoint{5.879694in}{3.852317in}}%
\pgfpathlineto{\pgfqpoint{5.963027in}{3.852317in}}%
\pgfpathmoveto{\pgfqpoint{5.921360in}{3.810650in}}%
\pgfpathlineto{\pgfqpoint{5.921360in}{3.893984in}}%
\pgfusepath{stroke,fill}%
\end{pgfscope}%
\begin{pgfscope}%
\pgfpathrectangle{\pgfqpoint{2.000000in}{3.134211in}}{\pgfqpoint{4.376471in}{0.953684in}} %
\pgfusepath{clip}%
\pgfsetbuttcap%
\pgfsetroundjoin%
\definecolor{currentfill}{rgb}{1.000000,0.000000,0.000000}%
\pgfsetfillcolor{currentfill}%
\pgfsetlinewidth{2.007500pt}%
\definecolor{currentstroke}{rgb}{1.000000,0.000000,0.000000}%
\pgfsetstrokecolor{currentstroke}%
\pgfsetdash{}{0pt}%
\pgfpathmoveto{\pgfqpoint{6.259943in}{3.745955in}}%
\pgfpathlineto{\pgfqpoint{6.343276in}{3.745955in}}%
\pgfpathmoveto{\pgfqpoint{6.301610in}{3.704288in}}%
\pgfpathlineto{\pgfqpoint{6.301610in}{3.787621in}}%
\pgfusepath{stroke,fill}%
\end{pgfscope}%
\begin{pgfscope}%
\pgfpathrectangle{\pgfqpoint{2.000000in}{3.134211in}}{\pgfqpoint{4.376471in}{0.953684in}} %
\pgfusepath{clip}%
\pgfsetbuttcap%
\pgfsetroundjoin%
\definecolor{currentfill}{rgb}{1.000000,0.000000,0.000000}%
\pgfsetfillcolor{currentfill}%
\pgfsetlinewidth{2.007500pt}%
\definecolor{currentstroke}{rgb}{1.000000,0.000000,0.000000}%
\pgfsetstrokecolor{currentstroke}%
\pgfsetdash{}{0pt}%
\pgfpathmoveto{\pgfqpoint{5.631623in}{3.403883in}}%
\pgfpathlineto{\pgfqpoint{5.714956in}{3.403883in}}%
\pgfpathmoveto{\pgfqpoint{5.673289in}{3.362216in}}%
\pgfpathlineto{\pgfqpoint{5.673289in}{3.445550in}}%
\pgfusepath{stroke,fill}%
\end{pgfscope}%
\begin{pgfscope}%
\pgfpathrectangle{\pgfqpoint{2.000000in}{3.134211in}}{\pgfqpoint{4.376471in}{0.953684in}} %
\pgfusepath{clip}%
\pgfsetbuttcap%
\pgfsetroundjoin%
\definecolor{currentfill}{rgb}{1.000000,0.000000,0.000000}%
\pgfsetfillcolor{currentfill}%
\pgfsetlinewidth{2.007500pt}%
\definecolor{currentstroke}{rgb}{1.000000,0.000000,0.000000}%
\pgfsetstrokecolor{currentstroke}%
\pgfsetdash{}{0pt}%
\pgfpathmoveto{\pgfqpoint{4.449348in}{3.373457in}}%
\pgfpathlineto{\pgfqpoint{4.532681in}{3.373457in}}%
\pgfpathmoveto{\pgfqpoint{4.491015in}{3.331790in}}%
\pgfpathlineto{\pgfqpoint{4.491015in}{3.415123in}}%
\pgfusepath{stroke,fill}%
\end{pgfscope}%
\begin{pgfscope}%
\pgfpathrectangle{\pgfqpoint{2.000000in}{3.134211in}}{\pgfqpoint{4.376471in}{0.953684in}} %
\pgfusepath{clip}%
\pgfsetbuttcap%
\pgfsetroundjoin%
\definecolor{currentfill}{rgb}{1.000000,0.000000,0.000000}%
\pgfsetfillcolor{currentfill}%
\pgfsetlinewidth{2.007500pt}%
\definecolor{currentstroke}{rgb}{1.000000,0.000000,0.000000}%
\pgfsetstrokecolor{currentstroke}%
\pgfsetdash{}{0pt}%
\pgfpathmoveto{\pgfqpoint{5.566398in}{3.453230in}}%
\pgfpathlineto{\pgfqpoint{5.649731in}{3.453230in}}%
\pgfpathmoveto{\pgfqpoint{5.608065in}{3.411563in}}%
\pgfpathlineto{\pgfqpoint{5.608065in}{3.494896in}}%
\pgfusepath{stroke,fill}%
\end{pgfscope}%
\begin{pgfscope}%
\pgfpathrectangle{\pgfqpoint{2.000000in}{3.134211in}}{\pgfqpoint{4.376471in}{0.953684in}} %
\pgfusepath{clip}%
\pgfsetbuttcap%
\pgfsetroundjoin%
\definecolor{currentfill}{rgb}{1.000000,0.000000,0.000000}%
\pgfsetfillcolor{currentfill}%
\pgfsetlinewidth{2.007500pt}%
\definecolor{currentstroke}{rgb}{1.000000,0.000000,0.000000}%
\pgfsetstrokecolor{currentstroke}%
\pgfsetdash{}{0pt}%
\pgfpathmoveto{\pgfqpoint{3.247727in}{3.367586in}}%
\pgfpathlineto{\pgfqpoint{3.331060in}{3.367586in}}%
\pgfpathmoveto{\pgfqpoint{3.289394in}{3.325919in}}%
\pgfpathlineto{\pgfqpoint{3.289394in}{3.409252in}}%
\pgfusepath{stroke,fill}%
\end{pgfscope}%
\begin{pgfscope}%
\pgfpathrectangle{\pgfqpoint{2.000000in}{3.134211in}}{\pgfqpoint{4.376471in}{0.953684in}} %
\pgfusepath{clip}%
\pgfsetbuttcap%
\pgfsetroundjoin%
\definecolor{currentfill}{rgb}{1.000000,0.000000,0.000000}%
\pgfsetfillcolor{currentfill}%
\pgfsetlinewidth{2.007500pt}%
\definecolor{currentstroke}{rgb}{1.000000,0.000000,0.000000}%
\pgfsetstrokecolor{currentstroke}%
\pgfsetdash{}{0pt}%
\pgfpathmoveto{\pgfqpoint{5.074104in}{3.635787in}}%
\pgfpathlineto{\pgfqpoint{5.157437in}{3.635787in}}%
\pgfpathmoveto{\pgfqpoint{5.115771in}{3.594120in}}%
\pgfpathlineto{\pgfqpoint{5.115771in}{3.677454in}}%
\pgfusepath{stroke,fill}%
\end{pgfscope}%
\begin{pgfscope}%
\pgfpathrectangle{\pgfqpoint{2.000000in}{3.134211in}}{\pgfqpoint{4.376471in}{0.953684in}} %
\pgfusepath{clip}%
\pgfsetbuttcap%
\pgfsetroundjoin%
\definecolor{currentfill}{rgb}{1.000000,0.000000,0.000000}%
\pgfsetfillcolor{currentfill}%
\pgfsetlinewidth{2.007500pt}%
\definecolor{currentstroke}{rgb}{1.000000,0.000000,0.000000}%
\pgfsetstrokecolor{currentstroke}%
\pgfsetdash{}{0pt}%
\pgfpathmoveto{\pgfqpoint{3.335533in}{3.454651in}}%
\pgfpathlineto{\pgfqpoint{3.418866in}{3.454651in}}%
\pgfpathmoveto{\pgfqpoint{3.377199in}{3.412984in}}%
\pgfpathlineto{\pgfqpoint{3.377199in}{3.496317in}}%
\pgfusepath{stroke,fill}%
\end{pgfscope}%
\begin{pgfscope}%
\pgfpathrectangle{\pgfqpoint{2.000000in}{3.134211in}}{\pgfqpoint{4.376471in}{0.953684in}} %
\pgfusepath{clip}%
\pgfsetbuttcap%
\pgfsetroundjoin%
\definecolor{currentfill}{rgb}{1.000000,0.000000,0.000000}%
\pgfsetfillcolor{currentfill}%
\pgfsetlinewidth{2.007500pt}%
\definecolor{currentstroke}{rgb}{1.000000,0.000000,0.000000}%
\pgfsetstrokecolor{currentstroke}%
\pgfsetdash{}{0pt}%
\pgfpathmoveto{\pgfqpoint{6.141080in}{3.911971in}}%
\pgfpathlineto{\pgfqpoint{6.224413in}{3.911971in}}%
\pgfpathmoveto{\pgfqpoint{6.182747in}{3.870304in}}%
\pgfpathlineto{\pgfqpoint{6.182747in}{3.953638in}}%
\pgfusepath{stroke,fill}%
\end{pgfscope}%
\begin{pgfscope}%
\pgfpathrectangle{\pgfqpoint{2.000000in}{3.134211in}}{\pgfqpoint{4.376471in}{0.953684in}} %
\pgfusepath{clip}%
\pgfsetbuttcap%
\pgfsetroundjoin%
\definecolor{currentfill}{rgb}{1.000000,0.000000,0.000000}%
\pgfsetfillcolor{currentfill}%
\pgfsetlinewidth{2.007500pt}%
\definecolor{currentstroke}{rgb}{1.000000,0.000000,0.000000}%
\pgfsetstrokecolor{currentstroke}%
\pgfsetdash{}{0pt}%
\pgfpathmoveto{\pgfqpoint{4.660711in}{3.649112in}}%
\pgfpathlineto{\pgfqpoint{4.744044in}{3.649112in}}%
\pgfpathmoveto{\pgfqpoint{4.702377in}{3.607445in}}%
\pgfpathlineto{\pgfqpoint{4.702377in}{3.690779in}}%
\pgfusepath{stroke,fill}%
\end{pgfscope}%
\begin{pgfscope}%
\pgfpathrectangle{\pgfqpoint{2.000000in}{3.134211in}}{\pgfqpoint{4.376471in}{0.953684in}} %
\pgfusepath{clip}%
\pgfsetbuttcap%
\pgfsetroundjoin%
\definecolor{currentfill}{rgb}{1.000000,0.000000,0.000000}%
\pgfsetfillcolor{currentfill}%
\pgfsetlinewidth{2.007500pt}%
\definecolor{currentstroke}{rgb}{1.000000,0.000000,0.000000}%
\pgfsetstrokecolor{currentstroke}%
\pgfsetdash{}{0pt}%
\pgfpathmoveto{\pgfqpoint{4.285432in}{3.319380in}}%
\pgfpathlineto{\pgfqpoint{4.368765in}{3.319380in}}%
\pgfpathmoveto{\pgfqpoint{4.327099in}{3.277714in}}%
\pgfpathlineto{\pgfqpoint{4.327099in}{3.361047in}}%
\pgfusepath{stroke,fill}%
\end{pgfscope}%
\begin{pgfscope}%
\pgfpathrectangle{\pgfqpoint{2.000000in}{3.134211in}}{\pgfqpoint{4.376471in}{0.953684in}} %
\pgfusepath{clip}%
\pgfsetbuttcap%
\pgfsetroundjoin%
\definecolor{currentfill}{rgb}{1.000000,0.000000,0.000000}%
\pgfsetfillcolor{currentfill}%
\pgfsetlinewidth{2.007500pt}%
\definecolor{currentstroke}{rgb}{1.000000,0.000000,0.000000}%
\pgfsetstrokecolor{currentstroke}%
\pgfsetdash{}{0pt}%
\pgfpathmoveto{\pgfqpoint{3.759883in}{3.891827in}}%
\pgfpathlineto{\pgfqpoint{3.843217in}{3.891827in}}%
\pgfpathmoveto{\pgfqpoint{3.801550in}{3.850160in}}%
\pgfpathlineto{\pgfqpoint{3.801550in}{3.933493in}}%
\pgfusepath{stroke,fill}%
\end{pgfscope}%
\begin{pgfscope}%
\pgfpathrectangle{\pgfqpoint{2.000000in}{3.134211in}}{\pgfqpoint{4.376471in}{0.953684in}} %
\pgfusepath{clip}%
\pgfsetbuttcap%
\pgfsetroundjoin%
\definecolor{currentfill}{rgb}{1.000000,0.000000,0.000000}%
\pgfsetfillcolor{currentfill}%
\pgfsetlinewidth{2.007500pt}%
\definecolor{currentstroke}{rgb}{1.000000,0.000000,0.000000}%
\pgfsetstrokecolor{currentstroke}%
\pgfsetdash{}{0pt}%
\pgfpathmoveto{\pgfqpoint{5.544356in}{3.346386in}}%
\pgfpathlineto{\pgfqpoint{5.627690in}{3.346386in}}%
\pgfpathmoveto{\pgfqpoint{5.586023in}{3.304719in}}%
\pgfpathlineto{\pgfqpoint{5.586023in}{3.388052in}}%
\pgfusepath{stroke,fill}%
\end{pgfscope}%
\begin{pgfscope}%
\pgfpathrectangle{\pgfqpoint{2.000000in}{3.134211in}}{\pgfqpoint{4.376471in}{0.953684in}} %
\pgfusepath{clip}%
\pgfsetbuttcap%
\pgfsetroundjoin%
\definecolor{currentfill}{rgb}{1.000000,0.000000,0.000000}%
\pgfsetfillcolor{currentfill}%
\pgfsetlinewidth{2.007500pt}%
\definecolor{currentstroke}{rgb}{1.000000,0.000000,0.000000}%
\pgfsetstrokecolor{currentstroke}%
\pgfsetdash{}{0pt}%
\pgfpathmoveto{\pgfqpoint{4.430690in}{3.382115in}}%
\pgfpathlineto{\pgfqpoint{4.514024in}{3.382115in}}%
\pgfpathmoveto{\pgfqpoint{4.472357in}{3.340449in}}%
\pgfpathlineto{\pgfqpoint{4.472357in}{3.423782in}}%
\pgfusepath{stroke,fill}%
\end{pgfscope}%
\begin{pgfscope}%
\pgfpathrectangle{\pgfqpoint{2.000000in}{3.134211in}}{\pgfqpoint{4.376471in}{0.953684in}} %
\pgfusepath{clip}%
\pgfsetbuttcap%
\pgfsetroundjoin%
\definecolor{currentfill}{rgb}{1.000000,0.000000,0.000000}%
\pgfsetfillcolor{currentfill}%
\pgfsetlinewidth{2.007500pt}%
\definecolor{currentstroke}{rgb}{1.000000,0.000000,0.000000}%
\pgfsetstrokecolor{currentstroke}%
\pgfsetdash{}{0pt}%
\pgfpathmoveto{\pgfqpoint{4.823815in}{3.625895in}}%
\pgfpathlineto{\pgfqpoint{4.907148in}{3.625895in}}%
\pgfpathmoveto{\pgfqpoint{4.865482in}{3.584229in}}%
\pgfpathlineto{\pgfqpoint{4.865482in}{3.667562in}}%
\pgfusepath{stroke,fill}%
\end{pgfscope}%
\begin{pgfscope}%
\pgfpathrectangle{\pgfqpoint{2.000000in}{3.134211in}}{\pgfqpoint{4.376471in}{0.953684in}} %
\pgfusepath{clip}%
\pgfsetbuttcap%
\pgfsetroundjoin%
\definecolor{currentfill}{rgb}{1.000000,0.000000,0.000000}%
\pgfsetfillcolor{currentfill}%
\pgfsetlinewidth{2.007500pt}%
\definecolor{currentstroke}{rgb}{1.000000,0.000000,0.000000}%
\pgfsetstrokecolor{currentstroke}%
\pgfsetdash{}{0pt}%
\pgfpathmoveto{\pgfqpoint{2.899414in}{3.812760in}}%
\pgfpathlineto{\pgfqpoint{2.982747in}{3.812760in}}%
\pgfpathmoveto{\pgfqpoint{2.941081in}{3.771093in}}%
\pgfpathlineto{\pgfqpoint{2.941081in}{3.854426in}}%
\pgfusepath{stroke,fill}%
\end{pgfscope}%
\begin{pgfscope}%
\pgfpathrectangle{\pgfqpoint{2.000000in}{3.134211in}}{\pgfqpoint{4.376471in}{0.953684in}} %
\pgfusepath{clip}%
\pgfsetbuttcap%
\pgfsetroundjoin%
\definecolor{currentfill}{rgb}{1.000000,0.000000,0.000000}%
\pgfsetfillcolor{currentfill}%
\pgfsetlinewidth{2.007500pt}%
\definecolor{currentstroke}{rgb}{1.000000,0.000000,0.000000}%
\pgfsetstrokecolor{currentstroke}%
\pgfsetdash{}{0pt}%
\pgfpathmoveto{\pgfqpoint{4.996078in}{3.597313in}}%
\pgfpathlineto{\pgfqpoint{5.079412in}{3.597313in}}%
\pgfpathmoveto{\pgfqpoint{5.037745in}{3.555646in}}%
\pgfpathlineto{\pgfqpoint{5.037745in}{3.638980in}}%
\pgfusepath{stroke,fill}%
\end{pgfscope}%
\begin{pgfscope}%
\pgfpathrectangle{\pgfqpoint{2.000000in}{3.134211in}}{\pgfqpoint{4.376471in}{0.953684in}} %
\pgfusepath{clip}%
\pgfsetbuttcap%
\pgfsetroundjoin%
\definecolor{currentfill}{rgb}{1.000000,0.000000,0.000000}%
\pgfsetfillcolor{currentfill}%
\pgfsetlinewidth{2.007500pt}%
\definecolor{currentstroke}{rgb}{1.000000,0.000000,0.000000}%
\pgfsetstrokecolor{currentstroke}%
\pgfsetdash{}{0pt}%
\pgfpathmoveto{\pgfqpoint{4.976683in}{3.636459in}}%
\pgfpathlineto{\pgfqpoint{5.060016in}{3.636459in}}%
\pgfpathmoveto{\pgfqpoint{5.018349in}{3.594793in}}%
\pgfpathlineto{\pgfqpoint{5.018349in}{3.678126in}}%
\pgfusepath{stroke,fill}%
\end{pgfscope}%
\begin{pgfscope}%
\pgfpathrectangle{\pgfqpoint{2.000000in}{3.134211in}}{\pgfqpoint{4.376471in}{0.953684in}} %
\pgfusepath{clip}%
\pgfsetbuttcap%
\pgfsetroundjoin%
\definecolor{currentfill}{rgb}{1.000000,0.000000,0.000000}%
\pgfsetfillcolor{currentfill}%
\pgfsetlinewidth{2.007500pt}%
\definecolor{currentstroke}{rgb}{1.000000,0.000000,0.000000}%
\pgfsetstrokecolor{currentstroke}%
\pgfsetdash{}{0pt}%
\pgfpathmoveto{\pgfqpoint{4.993622in}{3.662085in}}%
\pgfpathlineto{\pgfqpoint{5.076956in}{3.662085in}}%
\pgfpathmoveto{\pgfqpoint{5.035289in}{3.620419in}}%
\pgfpathlineto{\pgfqpoint{5.035289in}{3.703752in}}%
\pgfusepath{stroke,fill}%
\end{pgfscope}%
\begin{pgfscope}%
\pgfpathrectangle{\pgfqpoint{2.000000in}{3.134211in}}{\pgfqpoint{4.376471in}{0.953684in}} %
\pgfusepath{clip}%
\pgfsetbuttcap%
\pgfsetroundjoin%
\definecolor{currentfill}{rgb}{1.000000,0.000000,0.000000}%
\pgfsetfillcolor{currentfill}%
\pgfsetlinewidth{2.007500pt}%
\definecolor{currentstroke}{rgb}{1.000000,0.000000,0.000000}%
\pgfsetstrokecolor{currentstroke}%
\pgfsetdash{}{0pt}%
\pgfpathmoveto{\pgfqpoint{6.137856in}{3.792899in}}%
\pgfpathlineto{\pgfqpoint{6.221189in}{3.792899in}}%
\pgfpathmoveto{\pgfqpoint{6.179523in}{3.751233in}}%
\pgfpathlineto{\pgfqpoint{6.179523in}{3.834566in}}%
\pgfusepath{stroke,fill}%
\end{pgfscope}%
\begin{pgfscope}%
\pgfpathrectangle{\pgfqpoint{2.000000in}{3.134211in}}{\pgfqpoint{4.376471in}{0.953684in}} %
\pgfusepath{clip}%
\pgfsetbuttcap%
\pgfsetroundjoin%
\definecolor{currentfill}{rgb}{1.000000,0.000000,0.000000}%
\pgfsetfillcolor{currentfill}%
\pgfsetlinewidth{2.007500pt}%
\definecolor{currentstroke}{rgb}{1.000000,0.000000,0.000000}%
\pgfsetstrokecolor{currentstroke}%
\pgfsetdash{}{0pt}%
\pgfpathmoveto{\pgfqpoint{5.220801in}{3.539529in}}%
\pgfpathlineto{\pgfqpoint{5.304134in}{3.539529in}}%
\pgfpathmoveto{\pgfqpoint{5.262467in}{3.497863in}}%
\pgfpathlineto{\pgfqpoint{5.262467in}{3.581196in}}%
\pgfusepath{stroke,fill}%
\end{pgfscope}%
\begin{pgfscope}%
\pgfpathrectangle{\pgfqpoint{2.000000in}{3.134211in}}{\pgfqpoint{4.376471in}{0.953684in}} %
\pgfusepath{clip}%
\pgfsetbuttcap%
\pgfsetroundjoin%
\definecolor{currentfill}{rgb}{1.000000,0.000000,0.000000}%
\pgfsetfillcolor{currentfill}%
\pgfsetlinewidth{2.007500pt}%
\definecolor{currentstroke}{rgb}{1.000000,0.000000,0.000000}%
\pgfsetstrokecolor{currentstroke}%
\pgfsetdash{}{0pt}%
\pgfpathmoveto{\pgfqpoint{4.092328in}{3.504764in}}%
\pgfpathlineto{\pgfqpoint{4.175661in}{3.504764in}}%
\pgfpathmoveto{\pgfqpoint{4.133995in}{3.463097in}}%
\pgfpathlineto{\pgfqpoint{4.133995in}{3.546431in}}%
\pgfusepath{stroke,fill}%
\end{pgfscope}%
\begin{pgfscope}%
\pgfpathrectangle{\pgfqpoint{2.000000in}{3.134211in}}{\pgfqpoint{4.376471in}{0.953684in}} %
\pgfusepath{clip}%
\pgfsetbuttcap%
\pgfsetroundjoin%
\definecolor{currentfill}{rgb}{1.000000,0.000000,0.000000}%
\pgfsetfillcolor{currentfill}%
\pgfsetlinewidth{2.007500pt}%
\definecolor{currentstroke}{rgb}{1.000000,0.000000,0.000000}%
\pgfsetstrokecolor{currentstroke}%
\pgfsetdash{}{0pt}%
\pgfpathmoveto{\pgfqpoint{4.363753in}{3.156404in}}%
\pgfpathlineto{\pgfqpoint{4.447087in}{3.156404in}}%
\pgfpathmoveto{\pgfqpoint{4.405420in}{3.114738in}}%
\pgfpathlineto{\pgfqpoint{4.405420in}{3.198071in}}%
\pgfusepath{stroke,fill}%
\end{pgfscope}%
\begin{pgfscope}%
\pgfpathrectangle{\pgfqpoint{2.000000in}{3.134211in}}{\pgfqpoint{4.376471in}{0.953684in}} %
\pgfusepath{clip}%
\pgfsetbuttcap%
\pgfsetroundjoin%
\definecolor{currentfill}{rgb}{1.000000,0.000000,0.000000}%
\pgfsetfillcolor{currentfill}%
\pgfsetlinewidth{2.007500pt}%
\definecolor{currentstroke}{rgb}{1.000000,0.000000,0.000000}%
\pgfsetstrokecolor{currentstroke}%
\pgfsetdash{}{0pt}%
\pgfpathmoveto{\pgfqpoint{5.276157in}{3.557724in}}%
\pgfpathlineto{\pgfqpoint{5.359491in}{3.557724in}}%
\pgfpathmoveto{\pgfqpoint{5.317824in}{3.516058in}}%
\pgfpathlineto{\pgfqpoint{5.317824in}{3.599391in}}%
\pgfusepath{stroke,fill}%
\end{pgfscope}%
\begin{pgfscope}%
\pgfpathrectangle{\pgfqpoint{2.000000in}{3.134211in}}{\pgfqpoint{4.376471in}{0.953684in}} %
\pgfusepath{clip}%
\pgfsetbuttcap%
\pgfsetroundjoin%
\definecolor{currentfill}{rgb}{1.000000,0.000000,0.000000}%
\pgfsetfillcolor{currentfill}%
\pgfsetlinewidth{2.007500pt}%
\definecolor{currentstroke}{rgb}{1.000000,0.000000,0.000000}%
\pgfsetstrokecolor{currentstroke}%
\pgfsetdash{}{0pt}%
\pgfpathmoveto{\pgfqpoint{3.044487in}{3.870651in}}%
\pgfpathlineto{\pgfqpoint{3.127821in}{3.870651in}}%
\pgfpathmoveto{\pgfqpoint{3.086154in}{3.828985in}}%
\pgfpathlineto{\pgfqpoint{3.086154in}{3.912318in}}%
\pgfusepath{stroke,fill}%
\end{pgfscope}%
\begin{pgfscope}%
\pgfpathrectangle{\pgfqpoint{2.000000in}{3.134211in}}{\pgfqpoint{4.376471in}{0.953684in}} %
\pgfusepath{clip}%
\pgfsetbuttcap%
\pgfsetroundjoin%
\definecolor{currentfill}{rgb}{1.000000,0.000000,0.000000}%
\pgfsetfillcolor{currentfill}%
\pgfsetlinewidth{2.007500pt}%
\definecolor{currentstroke}{rgb}{1.000000,0.000000,0.000000}%
\pgfsetstrokecolor{currentstroke}%
\pgfsetdash{}{0pt}%
\pgfpathmoveto{\pgfqpoint{5.168095in}{3.614897in}}%
\pgfpathlineto{\pgfqpoint{5.251429in}{3.614897in}}%
\pgfpathmoveto{\pgfqpoint{5.209762in}{3.573230in}}%
\pgfpathlineto{\pgfqpoint{5.209762in}{3.656563in}}%
\pgfusepath{stroke,fill}%
\end{pgfscope}%
\begin{pgfscope}%
\pgfpathrectangle{\pgfqpoint{2.000000in}{3.134211in}}{\pgfqpoint{4.376471in}{0.953684in}} %
\pgfusepath{clip}%
\pgfsetbuttcap%
\pgfsetroundjoin%
\definecolor{currentfill}{rgb}{1.000000,0.000000,0.000000}%
\pgfsetfillcolor{currentfill}%
\pgfsetlinewidth{2.007500pt}%
\definecolor{currentstroke}{rgb}{1.000000,0.000000,0.000000}%
\pgfsetstrokecolor{currentstroke}%
\pgfsetdash{}{0pt}%
\pgfpathmoveto{\pgfqpoint{5.181649in}{3.467031in}}%
\pgfpathlineto{\pgfqpoint{5.264982in}{3.467031in}}%
\pgfpathmoveto{\pgfqpoint{5.223316in}{3.425364in}}%
\pgfpathlineto{\pgfqpoint{5.223316in}{3.508697in}}%
\pgfusepath{stroke,fill}%
\end{pgfscope}%
\begin{pgfscope}%
\pgfpathrectangle{\pgfqpoint{2.000000in}{3.134211in}}{\pgfqpoint{4.376471in}{0.953684in}} %
\pgfusepath{clip}%
\pgfsetbuttcap%
\pgfsetroundjoin%
\definecolor{currentfill}{rgb}{1.000000,0.000000,0.000000}%
\pgfsetfillcolor{currentfill}%
\pgfsetlinewidth{2.007500pt}%
\definecolor{currentstroke}{rgb}{1.000000,0.000000,0.000000}%
\pgfsetstrokecolor{currentstroke}%
\pgfsetdash{}{0pt}%
\pgfpathmoveto{\pgfqpoint{3.570214in}{3.612446in}}%
\pgfpathlineto{\pgfqpoint{3.653547in}{3.612446in}}%
\pgfpathmoveto{\pgfqpoint{3.611881in}{3.570779in}}%
\pgfpathlineto{\pgfqpoint{3.611881in}{3.654113in}}%
\pgfusepath{stroke,fill}%
\end{pgfscope}%
\begin{pgfscope}%
\pgfpathrectangle{\pgfqpoint{2.000000in}{3.134211in}}{\pgfqpoint{4.376471in}{0.953684in}} %
\pgfusepath{clip}%
\pgfsetbuttcap%
\pgfsetroundjoin%
\definecolor{currentfill}{rgb}{1.000000,0.000000,0.000000}%
\pgfsetfillcolor{currentfill}%
\pgfsetlinewidth{2.007500pt}%
\definecolor{currentstroke}{rgb}{1.000000,0.000000,0.000000}%
\pgfsetstrokecolor{currentstroke}%
\pgfsetdash{}{0pt}%
\pgfpathmoveto{\pgfqpoint{3.285021in}{3.635068in}}%
\pgfpathlineto{\pgfqpoint{3.368355in}{3.635068in}}%
\pgfpathmoveto{\pgfqpoint{3.326688in}{3.593402in}}%
\pgfpathlineto{\pgfqpoint{3.326688in}{3.676735in}}%
\pgfusepath{stroke,fill}%
\end{pgfscope}%
\begin{pgfscope}%
\pgfpathrectangle{\pgfqpoint{2.000000in}{3.134211in}}{\pgfqpoint{4.376471in}{0.953684in}} %
\pgfusepath{clip}%
\pgfsetbuttcap%
\pgfsetroundjoin%
\definecolor{currentfill}{rgb}{1.000000,0.000000,0.000000}%
\pgfsetfillcolor{currentfill}%
\pgfsetlinewidth{2.007500pt}%
\definecolor{currentstroke}{rgb}{1.000000,0.000000,0.000000}%
\pgfsetstrokecolor{currentstroke}%
\pgfsetdash{}{0pt}%
\pgfpathmoveto{\pgfqpoint{3.937998in}{3.609760in}}%
\pgfpathlineto{\pgfqpoint{4.021331in}{3.609760in}}%
\pgfpathmoveto{\pgfqpoint{3.979664in}{3.568094in}}%
\pgfpathlineto{\pgfqpoint{3.979664in}{3.651427in}}%
\pgfusepath{stroke,fill}%
\end{pgfscope}%
\begin{pgfscope}%
\pgfpathrectangle{\pgfqpoint{2.000000in}{3.134211in}}{\pgfqpoint{4.376471in}{0.953684in}} %
\pgfusepath{clip}%
\pgfsetbuttcap%
\pgfsetroundjoin%
\definecolor{currentfill}{rgb}{1.000000,0.000000,0.000000}%
\pgfsetfillcolor{currentfill}%
\pgfsetlinewidth{2.007500pt}%
\definecolor{currentstroke}{rgb}{1.000000,0.000000,0.000000}%
\pgfsetstrokecolor{currentstroke}%
\pgfsetdash{}{0pt}%
\pgfpathmoveto{\pgfqpoint{4.107043in}{3.562552in}}%
\pgfpathlineto{\pgfqpoint{4.190376in}{3.562552in}}%
\pgfpathmoveto{\pgfqpoint{4.148710in}{3.520885in}}%
\pgfpathlineto{\pgfqpoint{4.148710in}{3.604219in}}%
\pgfusepath{stroke,fill}%
\end{pgfscope}%
\begin{pgfscope}%
\pgfpathrectangle{\pgfqpoint{2.000000in}{3.134211in}}{\pgfqpoint{4.376471in}{0.953684in}} %
\pgfusepath{clip}%
\pgfsetbuttcap%
\pgfsetroundjoin%
\definecolor{currentfill}{rgb}{0.000000,0.000000,0.000000}%
\pgfsetfillcolor{currentfill}%
\pgfsetlinewidth{1.003750pt}%
\definecolor{currentstroke}{rgb}{0.000000,0.000000,0.000000}%
\pgfsetstrokecolor{currentstroke}%
\pgfsetdash{}{0pt}%
\pgfsys@defobject{currentmarker}{\pgfqpoint{-0.020833in}{-0.020833in}}{\pgfqpoint{0.020833in}{0.020833in}}{%
\pgfpathmoveto{\pgfqpoint{0.000000in}{-0.020833in}}%
\pgfpathcurveto{\pgfqpoint{0.005525in}{-0.020833in}}{\pgfqpoint{0.010825in}{-0.018638in}}{\pgfqpoint{0.014731in}{-0.014731in}}%
\pgfpathcurveto{\pgfqpoint{0.018638in}{-0.010825in}}{\pgfqpoint{0.020833in}{-0.005525in}}{\pgfqpoint{0.020833in}{0.000000in}}%
\pgfpathcurveto{\pgfqpoint{0.020833in}{0.005525in}}{\pgfqpoint{0.018638in}{0.010825in}}{\pgfqpoint{0.014731in}{0.014731in}}%
\pgfpathcurveto{\pgfqpoint{0.010825in}{0.018638in}}{\pgfqpoint{0.005525in}{0.020833in}}{\pgfqpoint{0.000000in}{0.020833in}}%
\pgfpathcurveto{\pgfqpoint{-0.005525in}{0.020833in}}{\pgfqpoint{-0.010825in}{0.018638in}}{\pgfqpoint{-0.014731in}{0.014731in}}%
\pgfpathcurveto{\pgfqpoint{-0.018638in}{0.010825in}}{\pgfqpoint{-0.020833in}{0.005525in}}{\pgfqpoint{-0.020833in}{0.000000in}}%
\pgfpathcurveto{\pgfqpoint{-0.020833in}{-0.005525in}}{\pgfqpoint{-0.018638in}{-0.010825in}}{\pgfqpoint{-0.014731in}{-0.014731in}}%
\pgfpathcurveto{\pgfqpoint{-0.010825in}{-0.018638in}}{\pgfqpoint{-0.005525in}{-0.020833in}}{\pgfqpoint{0.000000in}{-0.020833in}}%
\pgfpathclose%
\pgfusepath{stroke,fill}%
}%
\begin{pgfscope}%
\pgfsys@transformshift{2.875294in}{4.017989in}%
\pgfsys@useobject{currentmarker}{}%
\end{pgfscope}%
\begin{pgfscope}%
\pgfsys@transformshift{2.892888in}{4.069101in}%
\pgfsys@useobject{currentmarker}{}%
\end{pgfscope}%
\begin{pgfscope}%
\pgfsys@transformshift{2.910482in}{3.981426in}%
\pgfsys@useobject{currentmarker}{}%
\end{pgfscope}%
\begin{pgfscope}%
\pgfsys@transformshift{2.928076in}{3.989509in}%
\pgfsys@useobject{currentmarker}{}%
\end{pgfscope}%
\begin{pgfscope}%
\pgfsys@transformshift{2.945670in}{3.893189in}%
\pgfsys@useobject{currentmarker}{}%
\end{pgfscope}%
\begin{pgfscope}%
\pgfsys@transformshift{2.963263in}{4.042123in}%
\pgfsys@useobject{currentmarker}{}%
\end{pgfscope}%
\begin{pgfscope}%
\pgfsys@transformshift{2.980857in}{3.849818in}%
\pgfsys@useobject{currentmarker}{}%
\end{pgfscope}%
\begin{pgfscope}%
\pgfsys@transformshift{2.998451in}{3.849780in}%
\pgfsys@useobject{currentmarker}{}%
\end{pgfscope}%
\begin{pgfscope}%
\pgfsys@transformshift{3.016045in}{3.969894in}%
\pgfsys@useobject{currentmarker}{}%
\end{pgfscope}%
\begin{pgfscope}%
\pgfsys@transformshift{3.033639in}{3.622436in}%
\pgfsys@useobject{currentmarker}{}%
\end{pgfscope}%
\begin{pgfscope}%
\pgfsys@transformshift{3.051233in}{3.604348in}%
\pgfsys@useobject{currentmarker}{}%
\end{pgfscope}%
\begin{pgfscope}%
\pgfsys@transformshift{3.068826in}{3.802015in}%
\pgfsys@useobject{currentmarker}{}%
\end{pgfscope}%
\begin{pgfscope}%
\pgfsys@transformshift{3.086420in}{3.565419in}%
\pgfsys@useobject{currentmarker}{}%
\end{pgfscope}%
\begin{pgfscope}%
\pgfsys@transformshift{3.104014in}{3.852552in}%
\pgfsys@useobject{currentmarker}{}%
\end{pgfscope}%
\begin{pgfscope}%
\pgfsys@transformshift{3.121608in}{3.597307in}%
\pgfsys@useobject{currentmarker}{}%
\end{pgfscope}%
\begin{pgfscope}%
\pgfsys@transformshift{3.139202in}{3.544634in}%
\pgfsys@useobject{currentmarker}{}%
\end{pgfscope}%
\begin{pgfscope}%
\pgfsys@transformshift{3.156796in}{3.792099in}%
\pgfsys@useobject{currentmarker}{}%
\end{pgfscope}%
\begin{pgfscope}%
\pgfsys@transformshift{3.174390in}{3.732131in}%
\pgfsys@useobject{currentmarker}{}%
\end{pgfscope}%
\begin{pgfscope}%
\pgfsys@transformshift{3.191983in}{3.756461in}%
\pgfsys@useobject{currentmarker}{}%
\end{pgfscope}%
\begin{pgfscope}%
\pgfsys@transformshift{3.209577in}{3.648795in}%
\pgfsys@useobject{currentmarker}{}%
\end{pgfscope}%
\begin{pgfscope}%
\pgfsys@transformshift{3.227171in}{3.463113in}%
\pgfsys@useobject{currentmarker}{}%
\end{pgfscope}%
\begin{pgfscope}%
\pgfsys@transformshift{3.244765in}{3.730368in}%
\pgfsys@useobject{currentmarker}{}%
\end{pgfscope}%
\begin{pgfscope}%
\pgfsys@transformshift{3.262359in}{3.508028in}%
\pgfsys@useobject{currentmarker}{}%
\end{pgfscope}%
\begin{pgfscope}%
\pgfsys@transformshift{3.279953in}{3.610485in}%
\pgfsys@useobject{currentmarker}{}%
\end{pgfscope}%
\begin{pgfscope}%
\pgfsys@transformshift{3.297547in}{3.623042in}%
\pgfsys@useobject{currentmarker}{}%
\end{pgfscope}%
\begin{pgfscope}%
\pgfsys@transformshift{3.315140in}{3.513729in}%
\pgfsys@useobject{currentmarker}{}%
\end{pgfscope}%
\begin{pgfscope}%
\pgfsys@transformshift{3.332734in}{3.592272in}%
\pgfsys@useobject{currentmarker}{}%
\end{pgfscope}%
\begin{pgfscope}%
\pgfsys@transformshift{3.350328in}{3.626893in}%
\pgfsys@useobject{currentmarker}{}%
\end{pgfscope}%
\begin{pgfscope}%
\pgfsys@transformshift{3.367922in}{3.578455in}%
\pgfsys@useobject{currentmarker}{}%
\end{pgfscope}%
\begin{pgfscope}%
\pgfsys@transformshift{3.385516in}{3.439280in}%
\pgfsys@useobject{currentmarker}{}%
\end{pgfscope}%
\begin{pgfscope}%
\pgfsys@transformshift{3.403110in}{3.587082in}%
\pgfsys@useobject{currentmarker}{}%
\end{pgfscope}%
\begin{pgfscope}%
\pgfsys@transformshift{3.420704in}{3.699554in}%
\pgfsys@useobject{currentmarker}{}%
\end{pgfscope}%
\begin{pgfscope}%
\pgfsys@transformshift{3.438297in}{3.510335in}%
\pgfsys@useobject{currentmarker}{}%
\end{pgfscope}%
\begin{pgfscope}%
\pgfsys@transformshift{3.455891in}{3.577060in}%
\pgfsys@useobject{currentmarker}{}%
\end{pgfscope}%
\begin{pgfscope}%
\pgfsys@transformshift{3.473485in}{3.562148in}%
\pgfsys@useobject{currentmarker}{}%
\end{pgfscope}%
\begin{pgfscope}%
\pgfsys@transformshift{3.491079in}{3.803211in}%
\pgfsys@useobject{currentmarker}{}%
\end{pgfscope}%
\begin{pgfscope}%
\pgfsys@transformshift{3.508673in}{3.700889in}%
\pgfsys@useobject{currentmarker}{}%
\end{pgfscope}%
\begin{pgfscope}%
\pgfsys@transformshift{3.526267in}{3.689460in}%
\pgfsys@useobject{currentmarker}{}%
\end{pgfscope}%
\begin{pgfscope}%
\pgfsys@transformshift{3.543860in}{3.587452in}%
\pgfsys@useobject{currentmarker}{}%
\end{pgfscope}%
\begin{pgfscope}%
\pgfsys@transformshift{3.561454in}{3.732339in}%
\pgfsys@useobject{currentmarker}{}%
\end{pgfscope}%
\begin{pgfscope}%
\pgfsys@transformshift{3.579048in}{3.626287in}%
\pgfsys@useobject{currentmarker}{}%
\end{pgfscope}%
\begin{pgfscope}%
\pgfsys@transformshift{3.596642in}{3.710428in}%
\pgfsys@useobject{currentmarker}{}%
\end{pgfscope}%
\begin{pgfscope}%
\pgfsys@transformshift{3.614236in}{3.657381in}%
\pgfsys@useobject{currentmarker}{}%
\end{pgfscope}%
\begin{pgfscope}%
\pgfsys@transformshift{3.631830in}{3.800139in}%
\pgfsys@useobject{currentmarker}{}%
\end{pgfscope}%
\begin{pgfscope}%
\pgfsys@transformshift{3.649424in}{3.801553in}%
\pgfsys@useobject{currentmarker}{}%
\end{pgfscope}%
\begin{pgfscope}%
\pgfsys@transformshift{3.667017in}{3.733694in}%
\pgfsys@useobject{currentmarker}{}%
\end{pgfscope}%
\begin{pgfscope}%
\pgfsys@transformshift{3.684611in}{3.802475in}%
\pgfsys@useobject{currentmarker}{}%
\end{pgfscope}%
\begin{pgfscope}%
\pgfsys@transformshift{3.702205in}{3.661797in}%
\pgfsys@useobject{currentmarker}{}%
\end{pgfscope}%
\begin{pgfscope}%
\pgfsys@transformshift{3.719799in}{3.627874in}%
\pgfsys@useobject{currentmarker}{}%
\end{pgfscope}%
\begin{pgfscope}%
\pgfsys@transformshift{3.737393in}{3.823590in}%
\pgfsys@useobject{currentmarker}{}%
\end{pgfscope}%
\begin{pgfscope}%
\pgfsys@transformshift{3.754987in}{3.798625in}%
\pgfsys@useobject{currentmarker}{}%
\end{pgfscope}%
\begin{pgfscope}%
\pgfsys@transformshift{3.772581in}{3.845424in}%
\pgfsys@useobject{currentmarker}{}%
\end{pgfscope}%
\begin{pgfscope}%
\pgfsys@transformshift{3.790174in}{4.017457in}%
\pgfsys@useobject{currentmarker}{}%
\end{pgfscope}%
\begin{pgfscope}%
\pgfsys@transformshift{3.807768in}{3.870988in}%
\pgfsys@useobject{currentmarker}{}%
\end{pgfscope}%
\begin{pgfscope}%
\pgfsys@transformshift{3.825362in}{3.681008in}%
\pgfsys@useobject{currentmarker}{}%
\end{pgfscope}%
\begin{pgfscope}%
\pgfsys@transformshift{3.842956in}{3.875279in}%
\pgfsys@useobject{currentmarker}{}%
\end{pgfscope}%
\begin{pgfscope}%
\pgfsys@transformshift{3.860550in}{3.624342in}%
\pgfsys@useobject{currentmarker}{}%
\end{pgfscope}%
\begin{pgfscope}%
\pgfsys@transformshift{3.878144in}{3.698158in}%
\pgfsys@useobject{currentmarker}{}%
\end{pgfscope}%
\begin{pgfscope}%
\pgfsys@transformshift{3.895738in}{3.724445in}%
\pgfsys@useobject{currentmarker}{}%
\end{pgfscope}%
\begin{pgfscope}%
\pgfsys@transformshift{3.913331in}{3.887006in}%
\pgfsys@useobject{currentmarker}{}%
\end{pgfscope}%
\begin{pgfscope}%
\pgfsys@transformshift{3.930925in}{3.626862in}%
\pgfsys@useobject{currentmarker}{}%
\end{pgfscope}%
\begin{pgfscope}%
\pgfsys@transformshift{3.948519in}{3.601399in}%
\pgfsys@useobject{currentmarker}{}%
\end{pgfscope}%
\begin{pgfscope}%
\pgfsys@transformshift{3.966113in}{3.655117in}%
\pgfsys@useobject{currentmarker}{}%
\end{pgfscope}%
\begin{pgfscope}%
\pgfsys@transformshift{3.983707in}{3.579287in}%
\pgfsys@useobject{currentmarker}{}%
\end{pgfscope}%
\begin{pgfscope}%
\pgfsys@transformshift{4.001301in}{3.736495in}%
\pgfsys@useobject{currentmarker}{}%
\end{pgfscope}%
\begin{pgfscope}%
\pgfsys@transformshift{4.018894in}{3.495914in}%
\pgfsys@useobject{currentmarker}{}%
\end{pgfscope}%
\begin{pgfscope}%
\pgfsys@transformshift{4.036488in}{3.467269in}%
\pgfsys@useobject{currentmarker}{}%
\end{pgfscope}%
\begin{pgfscope}%
\pgfsys@transformshift{4.054082in}{3.515463in}%
\pgfsys@useobject{currentmarker}{}%
\end{pgfscope}%
\begin{pgfscope}%
\pgfsys@transformshift{4.071676in}{3.487008in}%
\pgfsys@useobject{currentmarker}{}%
\end{pgfscope}%
\begin{pgfscope}%
\pgfsys@transformshift{4.089270in}{3.705725in}%
\pgfsys@useobject{currentmarker}{}%
\end{pgfscope}%
\begin{pgfscope}%
\pgfsys@transformshift{4.106864in}{3.586150in}%
\pgfsys@useobject{currentmarker}{}%
\end{pgfscope}%
\begin{pgfscope}%
\pgfsys@transformshift{4.124458in}{3.478755in}%
\pgfsys@useobject{currentmarker}{}%
\end{pgfscope}%
\begin{pgfscope}%
\pgfsys@transformshift{4.142051in}{3.327173in}%
\pgfsys@useobject{currentmarker}{}%
\end{pgfscope}%
\begin{pgfscope}%
\pgfsys@transformshift{4.159645in}{3.512465in}%
\pgfsys@useobject{currentmarker}{}%
\end{pgfscope}%
\begin{pgfscope}%
\pgfsys@transformshift{4.177239in}{3.309897in}%
\pgfsys@useobject{currentmarker}{}%
\end{pgfscope}%
\begin{pgfscope}%
\pgfsys@transformshift{4.194833in}{3.237654in}%
\pgfsys@useobject{currentmarker}{}%
\end{pgfscope}%
\begin{pgfscope}%
\pgfsys@transformshift{4.212427in}{3.492329in}%
\pgfsys@useobject{currentmarker}{}%
\end{pgfscope}%
\begin{pgfscope}%
\pgfsys@transformshift{4.230021in}{3.390477in}%
\pgfsys@useobject{currentmarker}{}%
\end{pgfscope}%
\begin{pgfscope}%
\pgfsys@transformshift{4.247615in}{3.436782in}%
\pgfsys@useobject{currentmarker}{}%
\end{pgfscope}%
\begin{pgfscope}%
\pgfsys@transformshift{4.265208in}{3.365021in}%
\pgfsys@useobject{currentmarker}{}%
\end{pgfscope}%
\begin{pgfscope}%
\pgfsys@transformshift{4.282802in}{3.408384in}%
\pgfsys@useobject{currentmarker}{}%
\end{pgfscope}%
\begin{pgfscope}%
\pgfsys@transformshift{4.300396in}{3.250432in}%
\pgfsys@useobject{currentmarker}{}%
\end{pgfscope}%
\begin{pgfscope}%
\pgfsys@transformshift{4.317990in}{3.206207in}%
\pgfsys@useobject{currentmarker}{}%
\end{pgfscope}%
\begin{pgfscope}%
\pgfsys@transformshift{4.335584in}{3.372556in}%
\pgfsys@useobject{currentmarker}{}%
\end{pgfscope}%
\begin{pgfscope}%
\pgfsys@transformshift{4.353178in}{3.223005in}%
\pgfsys@useobject{currentmarker}{}%
\end{pgfscope}%
\begin{pgfscope}%
\pgfsys@transformshift{4.370772in}{3.234374in}%
\pgfsys@useobject{currentmarker}{}%
\end{pgfscope}%
\begin{pgfscope}%
\pgfsys@transformshift{4.388365in}{3.259752in}%
\pgfsys@useobject{currentmarker}{}%
\end{pgfscope}%
\begin{pgfscope}%
\pgfsys@transformshift{4.405959in}{3.310912in}%
\pgfsys@useobject{currentmarker}{}%
\end{pgfscope}%
\begin{pgfscope}%
\pgfsys@transformshift{4.423553in}{3.280166in}%
\pgfsys@useobject{currentmarker}{}%
\end{pgfscope}%
\begin{pgfscope}%
\pgfsys@transformshift{4.441147in}{3.186814in}%
\pgfsys@useobject{currentmarker}{}%
\end{pgfscope}%
\begin{pgfscope}%
\pgfsys@transformshift{4.458741in}{3.269363in}%
\pgfsys@useobject{currentmarker}{}%
\end{pgfscope}%
\begin{pgfscope}%
\pgfsys@transformshift{4.476335in}{3.124038in}%
\pgfsys@useobject{currentmarker}{}%
\end{pgfscope}%
\begin{pgfscope}%
\pgfsys@transformshift{4.493928in}{3.420234in}%
\pgfsys@useobject{currentmarker}{}%
\end{pgfscope}%
\begin{pgfscope}%
\pgfsys@transformshift{4.511522in}{3.213629in}%
\pgfsys@useobject{currentmarker}{}%
\end{pgfscope}%
\begin{pgfscope}%
\pgfsys@transformshift{4.529116in}{3.279027in}%
\pgfsys@useobject{currentmarker}{}%
\end{pgfscope}%
\begin{pgfscope}%
\pgfsys@transformshift{4.546710in}{3.411012in}%
\pgfsys@useobject{currentmarker}{}%
\end{pgfscope}%
\begin{pgfscope}%
\pgfsys@transformshift{4.564304in}{3.350408in}%
\pgfsys@useobject{currentmarker}{}%
\end{pgfscope}%
\begin{pgfscope}%
\pgfsys@transformshift{4.581898in}{3.595982in}%
\pgfsys@useobject{currentmarker}{}%
\end{pgfscope}%
\begin{pgfscope}%
\pgfsys@transformshift{4.599492in}{3.333658in}%
\pgfsys@useobject{currentmarker}{}%
\end{pgfscope}%
\begin{pgfscope}%
\pgfsys@transformshift{4.617085in}{3.508402in}%
\pgfsys@useobject{currentmarker}{}%
\end{pgfscope}%
\begin{pgfscope}%
\pgfsys@transformshift{4.634679in}{3.497930in}%
\pgfsys@useobject{currentmarker}{}%
\end{pgfscope}%
\begin{pgfscope}%
\pgfsys@transformshift{4.652273in}{3.405731in}%
\pgfsys@useobject{currentmarker}{}%
\end{pgfscope}%
\begin{pgfscope}%
\pgfsys@transformshift{4.669867in}{3.593473in}%
\pgfsys@useobject{currentmarker}{}%
\end{pgfscope}%
\begin{pgfscope}%
\pgfsys@transformshift{4.687461in}{3.543749in}%
\pgfsys@useobject{currentmarker}{}%
\end{pgfscope}%
\begin{pgfscope}%
\pgfsys@transformshift{4.705055in}{3.656151in}%
\pgfsys@useobject{currentmarker}{}%
\end{pgfscope}%
\begin{pgfscope}%
\pgfsys@transformshift{4.722649in}{3.679182in}%
\pgfsys@useobject{currentmarker}{}%
\end{pgfscope}%
\begin{pgfscope}%
\pgfsys@transformshift{4.740242in}{3.829195in}%
\pgfsys@useobject{currentmarker}{}%
\end{pgfscope}%
\begin{pgfscope}%
\pgfsys@transformshift{4.757836in}{3.762851in}%
\pgfsys@useobject{currentmarker}{}%
\end{pgfscope}%
\begin{pgfscope}%
\pgfsys@transformshift{4.775430in}{3.607893in}%
\pgfsys@useobject{currentmarker}{}%
\end{pgfscope}%
\begin{pgfscope}%
\pgfsys@transformshift{4.793024in}{3.633847in}%
\pgfsys@useobject{currentmarker}{}%
\end{pgfscope}%
\begin{pgfscope}%
\pgfsys@transformshift{4.810618in}{3.778372in}%
\pgfsys@useobject{currentmarker}{}%
\end{pgfscope}%
\begin{pgfscope}%
\pgfsys@transformshift{4.828212in}{3.744063in}%
\pgfsys@useobject{currentmarker}{}%
\end{pgfscope}%
\begin{pgfscope}%
\pgfsys@transformshift{4.845805in}{3.750644in}%
\pgfsys@useobject{currentmarker}{}%
\end{pgfscope}%
\begin{pgfscope}%
\pgfsys@transformshift{4.863399in}{3.532677in}%
\pgfsys@useobject{currentmarker}{}%
\end{pgfscope}%
\begin{pgfscope}%
\pgfsys@transformshift{4.880993in}{3.695270in}%
\pgfsys@useobject{currentmarker}{}%
\end{pgfscope}%
\begin{pgfscope}%
\pgfsys@transformshift{4.898587in}{3.626835in}%
\pgfsys@useobject{currentmarker}{}%
\end{pgfscope}%
\begin{pgfscope}%
\pgfsys@transformshift{4.916181in}{3.728502in}%
\pgfsys@useobject{currentmarker}{}%
\end{pgfscope}%
\begin{pgfscope}%
\pgfsys@transformshift{4.933775in}{3.689567in}%
\pgfsys@useobject{currentmarker}{}%
\end{pgfscope}%
\begin{pgfscope}%
\pgfsys@transformshift{4.951369in}{3.786471in}%
\pgfsys@useobject{currentmarker}{}%
\end{pgfscope}%
\begin{pgfscope}%
\pgfsys@transformshift{4.968962in}{3.722457in}%
\pgfsys@useobject{currentmarker}{}%
\end{pgfscope}%
\begin{pgfscope}%
\pgfsys@transformshift{4.986556in}{3.762177in}%
\pgfsys@useobject{currentmarker}{}%
\end{pgfscope}%
\begin{pgfscope}%
\pgfsys@transformshift{5.004150in}{3.629154in}%
\pgfsys@useobject{currentmarker}{}%
\end{pgfscope}%
\begin{pgfscope}%
\pgfsys@transformshift{5.021744in}{3.571508in}%
\pgfsys@useobject{currentmarker}{}%
\end{pgfscope}%
\begin{pgfscope}%
\pgfsys@transformshift{5.039338in}{3.612973in}%
\pgfsys@useobject{currentmarker}{}%
\end{pgfscope}%
\begin{pgfscope}%
\pgfsys@transformshift{5.056932in}{3.639064in}%
\pgfsys@useobject{currentmarker}{}%
\end{pgfscope}%
\begin{pgfscope}%
\pgfsys@transformshift{5.074526in}{3.664288in}%
\pgfsys@useobject{currentmarker}{}%
\end{pgfscope}%
\begin{pgfscope}%
\pgfsys@transformshift{5.092119in}{3.835917in}%
\pgfsys@useobject{currentmarker}{}%
\end{pgfscope}%
\begin{pgfscope}%
\pgfsys@transformshift{5.109713in}{3.591283in}%
\pgfsys@useobject{currentmarker}{}%
\end{pgfscope}%
\begin{pgfscope}%
\pgfsys@transformshift{5.127307in}{3.483864in}%
\pgfsys@useobject{currentmarker}{}%
\end{pgfscope}%
\begin{pgfscope}%
\pgfsys@transformshift{5.144901in}{3.527339in}%
\pgfsys@useobject{currentmarker}{}%
\end{pgfscope}%
\begin{pgfscope}%
\pgfsys@transformshift{5.162495in}{3.498263in}%
\pgfsys@useobject{currentmarker}{}%
\end{pgfscope}%
\begin{pgfscope}%
\pgfsys@transformshift{5.180089in}{3.574685in}%
\pgfsys@useobject{currentmarker}{}%
\end{pgfscope}%
\begin{pgfscope}%
\pgfsys@transformshift{5.197683in}{3.356452in}%
\pgfsys@useobject{currentmarker}{}%
\end{pgfscope}%
\begin{pgfscope}%
\pgfsys@transformshift{5.215276in}{3.498796in}%
\pgfsys@useobject{currentmarker}{}%
\end{pgfscope}%
\begin{pgfscope}%
\pgfsys@transformshift{5.232870in}{3.491581in}%
\pgfsys@useobject{currentmarker}{}%
\end{pgfscope}%
\begin{pgfscope}%
\pgfsys@transformshift{5.250464in}{3.483343in}%
\pgfsys@useobject{currentmarker}{}%
\end{pgfscope}%
\begin{pgfscope}%
\pgfsys@transformshift{5.268058in}{3.386093in}%
\pgfsys@useobject{currentmarker}{}%
\end{pgfscope}%
\begin{pgfscope}%
\pgfsys@transformshift{5.285652in}{3.408037in}%
\pgfsys@useobject{currentmarker}{}%
\end{pgfscope}%
\begin{pgfscope}%
\pgfsys@transformshift{5.303246in}{3.277703in}%
\pgfsys@useobject{currentmarker}{}%
\end{pgfscope}%
\begin{pgfscope}%
\pgfsys@transformshift{5.320839in}{3.359077in}%
\pgfsys@useobject{currentmarker}{}%
\end{pgfscope}%
\begin{pgfscope}%
\pgfsys@transformshift{5.338433in}{3.344626in}%
\pgfsys@useobject{currentmarker}{}%
\end{pgfscope}%
\begin{pgfscope}%
\pgfsys@transformshift{5.356027in}{3.432044in}%
\pgfsys@useobject{currentmarker}{}%
\end{pgfscope}%
\begin{pgfscope}%
\pgfsys@transformshift{5.373621in}{3.269810in}%
\pgfsys@useobject{currentmarker}{}%
\end{pgfscope}%
\begin{pgfscope}%
\pgfsys@transformshift{5.391215in}{3.458122in}%
\pgfsys@useobject{currentmarker}{}%
\end{pgfscope}%
\begin{pgfscope}%
\pgfsys@transformshift{5.408809in}{3.526801in}%
\pgfsys@useobject{currentmarker}{}%
\end{pgfscope}%
\begin{pgfscope}%
\pgfsys@transformshift{5.426403in}{3.172546in}%
\pgfsys@useobject{currentmarker}{}%
\end{pgfscope}%
\begin{pgfscope}%
\pgfsys@transformshift{5.443996in}{3.422441in}%
\pgfsys@useobject{currentmarker}{}%
\end{pgfscope}%
\begin{pgfscope}%
\pgfsys@transformshift{5.461590in}{3.451338in}%
\pgfsys@useobject{currentmarker}{}%
\end{pgfscope}%
\begin{pgfscope}%
\pgfsys@transformshift{5.479184in}{3.326847in}%
\pgfsys@useobject{currentmarker}{}%
\end{pgfscope}%
\begin{pgfscope}%
\pgfsys@transformshift{5.496778in}{3.358844in}%
\pgfsys@useobject{currentmarker}{}%
\end{pgfscope}%
\begin{pgfscope}%
\pgfsys@transformshift{5.514372in}{3.395263in}%
\pgfsys@useobject{currentmarker}{}%
\end{pgfscope}%
\begin{pgfscope}%
\pgfsys@transformshift{5.531966in}{3.390990in}%
\pgfsys@useobject{currentmarker}{}%
\end{pgfscope}%
\begin{pgfscope}%
\pgfsys@transformshift{5.549560in}{3.403859in}%
\pgfsys@useobject{currentmarker}{}%
\end{pgfscope}%
\begin{pgfscope}%
\pgfsys@transformshift{5.567153in}{3.283817in}%
\pgfsys@useobject{currentmarker}{}%
\end{pgfscope}%
\begin{pgfscope}%
\pgfsys@transformshift{5.584747in}{3.582207in}%
\pgfsys@useobject{currentmarker}{}%
\end{pgfscope}%
\begin{pgfscope}%
\pgfsys@transformshift{5.602341in}{3.593927in}%
\pgfsys@useobject{currentmarker}{}%
\end{pgfscope}%
\begin{pgfscope}%
\pgfsys@transformshift{5.619935in}{3.426199in}%
\pgfsys@useobject{currentmarker}{}%
\end{pgfscope}%
\begin{pgfscope}%
\pgfsys@transformshift{5.637529in}{3.382867in}%
\pgfsys@useobject{currentmarker}{}%
\end{pgfscope}%
\begin{pgfscope}%
\pgfsys@transformshift{5.655123in}{3.602901in}%
\pgfsys@useobject{currentmarker}{}%
\end{pgfscope}%
\begin{pgfscope}%
\pgfsys@transformshift{5.672717in}{3.517399in}%
\pgfsys@useobject{currentmarker}{}%
\end{pgfscope}%
\begin{pgfscope}%
\pgfsys@transformshift{5.690310in}{3.612842in}%
\pgfsys@useobject{currentmarker}{}%
\end{pgfscope}%
\begin{pgfscope}%
\pgfsys@transformshift{5.707904in}{3.591642in}%
\pgfsys@useobject{currentmarker}{}%
\end{pgfscope}%
\begin{pgfscope}%
\pgfsys@transformshift{5.725498in}{3.716974in}%
\pgfsys@useobject{currentmarker}{}%
\end{pgfscope}%
\begin{pgfscope}%
\pgfsys@transformshift{5.743092in}{3.742286in}%
\pgfsys@useobject{currentmarker}{}%
\end{pgfscope}%
\begin{pgfscope}%
\pgfsys@transformshift{5.760686in}{3.626062in}%
\pgfsys@useobject{currentmarker}{}%
\end{pgfscope}%
\begin{pgfscope}%
\pgfsys@transformshift{5.778280in}{3.585278in}%
\pgfsys@useobject{currentmarker}{}%
\end{pgfscope}%
\begin{pgfscope}%
\pgfsys@transformshift{5.795873in}{3.589500in}%
\pgfsys@useobject{currentmarker}{}%
\end{pgfscope}%
\begin{pgfscope}%
\pgfsys@transformshift{5.813467in}{3.830736in}%
\pgfsys@useobject{currentmarker}{}%
\end{pgfscope}%
\begin{pgfscope}%
\pgfsys@transformshift{5.831061in}{3.674460in}%
\pgfsys@useobject{currentmarker}{}%
\end{pgfscope}%
\begin{pgfscope}%
\pgfsys@transformshift{5.848655in}{3.763627in}%
\pgfsys@useobject{currentmarker}{}%
\end{pgfscope}%
\begin{pgfscope}%
\pgfsys@transformshift{5.866249in}{3.774932in}%
\pgfsys@useobject{currentmarker}{}%
\end{pgfscope}%
\begin{pgfscope}%
\pgfsys@transformshift{5.883843in}{3.847784in}%
\pgfsys@useobject{currentmarker}{}%
\end{pgfscope}%
\begin{pgfscope}%
\pgfsys@transformshift{5.901437in}{3.677994in}%
\pgfsys@useobject{currentmarker}{}%
\end{pgfscope}%
\begin{pgfscope}%
\pgfsys@transformshift{5.919030in}{3.904833in}%
\pgfsys@useobject{currentmarker}{}%
\end{pgfscope}%
\begin{pgfscope}%
\pgfsys@transformshift{5.936624in}{3.952144in}%
\pgfsys@useobject{currentmarker}{}%
\end{pgfscope}%
\begin{pgfscope}%
\pgfsys@transformshift{5.954218in}{3.921113in}%
\pgfsys@useobject{currentmarker}{}%
\end{pgfscope}%
\begin{pgfscope}%
\pgfsys@transformshift{5.971812in}{3.891819in}%
\pgfsys@useobject{currentmarker}{}%
\end{pgfscope}%
\begin{pgfscope}%
\pgfsys@transformshift{5.989406in}{3.940856in}%
\pgfsys@useobject{currentmarker}{}%
\end{pgfscope}%
\begin{pgfscope}%
\pgfsys@transformshift{6.007000in}{3.977379in}%
\pgfsys@useobject{currentmarker}{}%
\end{pgfscope}%
\begin{pgfscope}%
\pgfsys@transformshift{6.024594in}{3.666935in}%
\pgfsys@useobject{currentmarker}{}%
\end{pgfscope}%
\begin{pgfscope}%
\pgfsys@transformshift{6.042187in}{4.139401in}%
\pgfsys@useobject{currentmarker}{}%
\end{pgfscope}%
\begin{pgfscope}%
\pgfsys@transformshift{6.059781in}{3.984706in}%
\pgfsys@useobject{currentmarker}{}%
\end{pgfscope}%
\begin{pgfscope}%
\pgfsys@transformshift{6.077375in}{3.880187in}%
\pgfsys@useobject{currentmarker}{}%
\end{pgfscope}%
\begin{pgfscope}%
\pgfsys@transformshift{6.094969in}{3.903470in}%
\pgfsys@useobject{currentmarker}{}%
\end{pgfscope}%
\begin{pgfscope}%
\pgfsys@transformshift{6.112563in}{3.987025in}%
\pgfsys@useobject{currentmarker}{}%
\end{pgfscope}%
\begin{pgfscope}%
\pgfsys@transformshift{6.130157in}{3.920579in}%
\pgfsys@useobject{currentmarker}{}%
\end{pgfscope}%
\begin{pgfscope}%
\pgfsys@transformshift{6.147751in}{3.723117in}%
\pgfsys@useobject{currentmarker}{}%
\end{pgfscope}%
\begin{pgfscope}%
\pgfsys@transformshift{6.165344in}{4.121417in}%
\pgfsys@useobject{currentmarker}{}%
\end{pgfscope}%
\begin{pgfscope}%
\pgfsys@transformshift{6.182938in}{3.895719in}%
\pgfsys@useobject{currentmarker}{}%
\end{pgfscope}%
\begin{pgfscope}%
\pgfsys@transformshift{6.200532in}{3.997487in}%
\pgfsys@useobject{currentmarker}{}%
\end{pgfscope}%
\begin{pgfscope}%
\pgfsys@transformshift{6.218126in}{3.816077in}%
\pgfsys@useobject{currentmarker}{}%
\end{pgfscope}%
\begin{pgfscope}%
\pgfsys@transformshift{6.235720in}{4.025571in}%
\pgfsys@useobject{currentmarker}{}%
\end{pgfscope}%
\begin{pgfscope}%
\pgfsys@transformshift{6.253314in}{3.889133in}%
\pgfsys@useobject{currentmarker}{}%
\end{pgfscope}%
\begin{pgfscope}%
\pgfsys@transformshift{6.270907in}{3.908737in}%
\pgfsys@useobject{currentmarker}{}%
\end{pgfscope}%
\begin{pgfscope}%
\pgfsys@transformshift{6.288501in}{3.732006in}%
\pgfsys@useobject{currentmarker}{}%
\end{pgfscope}%
\begin{pgfscope}%
\pgfsys@transformshift{6.306095in}{3.943979in}%
\pgfsys@useobject{currentmarker}{}%
\end{pgfscope}%
\begin{pgfscope}%
\pgfsys@transformshift{6.323689in}{3.880432in}%
\pgfsys@useobject{currentmarker}{}%
\end{pgfscope}%
\begin{pgfscope}%
\pgfsys@transformshift{6.341283in}{3.930097in}%
\pgfsys@useobject{currentmarker}{}%
\end{pgfscope}%
\begin{pgfscope}%
\pgfsys@transformshift{6.358877in}{3.728051in}%
\pgfsys@useobject{currentmarker}{}%
\end{pgfscope}%
\begin{pgfscope}%
\pgfsys@transformshift{6.376471in}{3.733336in}%
\pgfsys@useobject{currentmarker}{}%
\end{pgfscope}%
\end{pgfscope}%
\begin{pgfscope}%
\pgfsetbuttcap%
\pgfsetroundjoin%
\definecolor{currentfill}{rgb}{0.000000,0.000000,0.000000}%
\pgfsetfillcolor{currentfill}%
\pgfsetlinewidth{0.803000pt}%
\definecolor{currentstroke}{rgb}{0.000000,0.000000,0.000000}%
\pgfsetstrokecolor{currentstroke}%
\pgfsetdash{}{0pt}%
\pgfsys@defobject{currentmarker}{\pgfqpoint{0.000000in}{-0.048611in}}{\pgfqpoint{0.000000in}{0.000000in}}{%
\pgfpathmoveto{\pgfqpoint{0.000000in}{0.000000in}}%
\pgfpathlineto{\pgfqpoint{0.000000in}{-0.048611in}}%
\pgfusepath{stroke,fill}%
}%
\begin{pgfscope}%
\pgfsys@transformshift{2.000000in}{3.134211in}%
\pgfsys@useobject{currentmarker}{}%
\end{pgfscope}%
\end{pgfscope}%
\begin{pgfscope}%
\pgfsetbuttcap%
\pgfsetroundjoin%
\definecolor{currentfill}{rgb}{0.000000,0.000000,0.000000}%
\pgfsetfillcolor{currentfill}%
\pgfsetlinewidth{0.803000pt}%
\definecolor{currentstroke}{rgb}{0.000000,0.000000,0.000000}%
\pgfsetstrokecolor{currentstroke}%
\pgfsetdash{}{0pt}%
\pgfsys@defobject{currentmarker}{\pgfqpoint{0.000000in}{-0.048611in}}{\pgfqpoint{0.000000in}{0.000000in}}{%
\pgfpathmoveto{\pgfqpoint{0.000000in}{0.000000in}}%
\pgfpathlineto{\pgfqpoint{0.000000in}{-0.048611in}}%
\pgfusepath{stroke,fill}%
}%
\begin{pgfscope}%
\pgfsys@transformshift{2.875294in}{3.134211in}%
\pgfsys@useobject{currentmarker}{}%
\end{pgfscope}%
\end{pgfscope}%
\begin{pgfscope}%
\pgfsetbuttcap%
\pgfsetroundjoin%
\definecolor{currentfill}{rgb}{0.000000,0.000000,0.000000}%
\pgfsetfillcolor{currentfill}%
\pgfsetlinewidth{0.803000pt}%
\definecolor{currentstroke}{rgb}{0.000000,0.000000,0.000000}%
\pgfsetstrokecolor{currentstroke}%
\pgfsetdash{}{0pt}%
\pgfsys@defobject{currentmarker}{\pgfqpoint{0.000000in}{-0.048611in}}{\pgfqpoint{0.000000in}{0.000000in}}{%
\pgfpathmoveto{\pgfqpoint{0.000000in}{0.000000in}}%
\pgfpathlineto{\pgfqpoint{0.000000in}{-0.048611in}}%
\pgfusepath{stroke,fill}%
}%
\begin{pgfscope}%
\pgfsys@transformshift{3.750588in}{3.134211in}%
\pgfsys@useobject{currentmarker}{}%
\end{pgfscope}%
\end{pgfscope}%
\begin{pgfscope}%
\pgfsetbuttcap%
\pgfsetroundjoin%
\definecolor{currentfill}{rgb}{0.000000,0.000000,0.000000}%
\pgfsetfillcolor{currentfill}%
\pgfsetlinewidth{0.803000pt}%
\definecolor{currentstroke}{rgb}{0.000000,0.000000,0.000000}%
\pgfsetstrokecolor{currentstroke}%
\pgfsetdash{}{0pt}%
\pgfsys@defobject{currentmarker}{\pgfqpoint{0.000000in}{-0.048611in}}{\pgfqpoint{0.000000in}{0.000000in}}{%
\pgfpathmoveto{\pgfqpoint{0.000000in}{0.000000in}}%
\pgfpathlineto{\pgfqpoint{0.000000in}{-0.048611in}}%
\pgfusepath{stroke,fill}%
}%
\begin{pgfscope}%
\pgfsys@transformshift{4.625882in}{3.134211in}%
\pgfsys@useobject{currentmarker}{}%
\end{pgfscope}%
\end{pgfscope}%
\begin{pgfscope}%
\pgfsetbuttcap%
\pgfsetroundjoin%
\definecolor{currentfill}{rgb}{0.000000,0.000000,0.000000}%
\pgfsetfillcolor{currentfill}%
\pgfsetlinewidth{0.803000pt}%
\definecolor{currentstroke}{rgb}{0.000000,0.000000,0.000000}%
\pgfsetstrokecolor{currentstroke}%
\pgfsetdash{}{0pt}%
\pgfsys@defobject{currentmarker}{\pgfqpoint{0.000000in}{-0.048611in}}{\pgfqpoint{0.000000in}{0.000000in}}{%
\pgfpathmoveto{\pgfqpoint{0.000000in}{0.000000in}}%
\pgfpathlineto{\pgfqpoint{0.000000in}{-0.048611in}}%
\pgfusepath{stroke,fill}%
}%
\begin{pgfscope}%
\pgfsys@transformshift{5.501176in}{3.134211in}%
\pgfsys@useobject{currentmarker}{}%
\end{pgfscope}%
\end{pgfscope}%
\begin{pgfscope}%
\pgfsetbuttcap%
\pgfsetroundjoin%
\definecolor{currentfill}{rgb}{0.000000,0.000000,0.000000}%
\pgfsetfillcolor{currentfill}%
\pgfsetlinewidth{0.803000pt}%
\definecolor{currentstroke}{rgb}{0.000000,0.000000,0.000000}%
\pgfsetstrokecolor{currentstroke}%
\pgfsetdash{}{0pt}%
\pgfsys@defobject{currentmarker}{\pgfqpoint{0.000000in}{-0.048611in}}{\pgfqpoint{0.000000in}{0.000000in}}{%
\pgfpathmoveto{\pgfqpoint{0.000000in}{0.000000in}}%
\pgfpathlineto{\pgfqpoint{0.000000in}{-0.048611in}}%
\pgfusepath{stroke,fill}%
}%
\begin{pgfscope}%
\pgfsys@transformshift{6.376471in}{3.134211in}%
\pgfsys@useobject{currentmarker}{}%
\end{pgfscope}%
\end{pgfscope}%
\begin{pgfscope}%
\pgfsetbuttcap%
\pgfsetroundjoin%
\definecolor{currentfill}{rgb}{0.000000,0.000000,0.000000}%
\pgfsetfillcolor{currentfill}%
\pgfsetlinewidth{0.803000pt}%
\definecolor{currentstroke}{rgb}{0.000000,0.000000,0.000000}%
\pgfsetstrokecolor{currentstroke}%
\pgfsetdash{}{0pt}%
\pgfsys@defobject{currentmarker}{\pgfqpoint{-0.048611in}{0.000000in}}{\pgfqpoint{0.000000in}{0.000000in}}{%
\pgfpathmoveto{\pgfqpoint{0.000000in}{0.000000in}}%
\pgfpathlineto{\pgfqpoint{-0.048611in}{0.000000in}}%
\pgfusepath{stroke,fill}%
}%
\begin{pgfscope}%
\pgfsys@transformshift{2.000000in}{3.491842in}%
\pgfsys@useobject{currentmarker}{}%
\end{pgfscope}%
\end{pgfscope}%
\begin{pgfscope}%
\pgftext[x=1.833333in,y=3.443624in,left,base]{\rmfamily\fontsize{10.000000}{12.000000}\selectfont \(\displaystyle 0\)}%
\end{pgfscope}%
\begin{pgfscope}%
\pgfsetbuttcap%
\pgfsetroundjoin%
\definecolor{currentfill}{rgb}{0.000000,0.000000,0.000000}%
\pgfsetfillcolor{currentfill}%
\pgfsetlinewidth{0.803000pt}%
\definecolor{currentstroke}{rgb}{0.000000,0.000000,0.000000}%
\pgfsetstrokecolor{currentstroke}%
\pgfsetdash{}{0pt}%
\pgfsys@defobject{currentmarker}{\pgfqpoint{-0.048611in}{0.000000in}}{\pgfqpoint{0.000000in}{0.000000in}}{%
\pgfpathmoveto{\pgfqpoint{0.000000in}{0.000000in}}%
\pgfpathlineto{\pgfqpoint{-0.048611in}{0.000000in}}%
\pgfusepath{stroke,fill}%
}%
\begin{pgfscope}%
\pgfsys@transformshift{2.000000in}{3.889211in}%
\pgfsys@useobject{currentmarker}{}%
\end{pgfscope}%
\end{pgfscope}%
\begin{pgfscope}%
\pgftext[x=1.833333in,y=3.840993in,left,base]{\rmfamily\fontsize{10.000000}{12.000000}\selectfont \(\displaystyle 2\)}%
\end{pgfscope}%
\begin{pgfscope}%
\pgftext[x=1.777777in,y=3.611053in,,bottom,rotate=90.000000]{\rmfamily\fontsize{10.000000}{12.000000}\selectfont y}%
\end{pgfscope}%
\begin{pgfscope}%
\pgfpathrectangle{\pgfqpoint{2.000000in}{3.134211in}}{\pgfqpoint{4.376471in}{0.953684in}} %
\pgfusepath{clip}%
\pgfsetrectcap%
\pgfsetroundjoin%
\pgfsetlinewidth{1.505625pt}%
\definecolor{currentstroke}{rgb}{0.121569,0.466667,0.705882}%
\pgfsetstrokecolor{currentstroke}%
\pgfsetdash{}{0pt}%
\pgfpathmoveto{\pgfqpoint{2.875294in}{3.694293in}}%
\pgfpathlineto{\pgfqpoint{2.892888in}{3.745126in}}%
\pgfpathlineto{\pgfqpoint{2.910482in}{3.784647in}}%
\pgfpathlineto{\pgfqpoint{2.928076in}{3.813804in}}%
\pgfpathlineto{\pgfqpoint{2.945670in}{3.833545in}}%
\pgfpathlineto{\pgfqpoint{2.963263in}{3.844812in}}%
\pgfpathlineto{\pgfqpoint{2.980857in}{3.848532in}}%
\pgfpathlineto{\pgfqpoint{2.998451in}{3.845619in}}%
\pgfpathlineto{\pgfqpoint{3.016045in}{3.836960in}}%
\pgfpathlineto{\pgfqpoint{3.033639in}{3.823417in}}%
\pgfpathlineto{\pgfqpoint{3.051233in}{3.805820in}}%
\pgfpathlineto{\pgfqpoint{3.068826in}{3.784962in}}%
\pgfpathlineto{\pgfqpoint{3.104014in}{3.736435in}}%
\pgfpathlineto{\pgfqpoint{3.191983in}{3.605257in}}%
\pgfpathlineto{\pgfqpoint{3.227171in}{3.559649in}}%
\pgfpathlineto{\pgfqpoint{3.244765in}{3.539848in}}%
\pgfpathlineto{\pgfqpoint{3.262359in}{3.522402in}}%
\pgfpathlineto{\pgfqpoint{3.279953in}{3.507508in}}%
\pgfpathlineto{\pgfqpoint{3.297547in}{3.495316in}}%
\pgfpathlineto{\pgfqpoint{3.315140in}{3.485925in}}%
\pgfpathlineto{\pgfqpoint{3.332734in}{3.479388in}}%
\pgfpathlineto{\pgfqpoint{3.350328in}{3.475714in}}%
\pgfpathlineto{\pgfqpoint{3.367922in}{3.474869in}}%
\pgfpathlineto{\pgfqpoint{3.385516in}{3.476780in}}%
\pgfpathlineto{\pgfqpoint{3.403110in}{3.481339in}}%
\pgfpathlineto{\pgfqpoint{3.420704in}{3.488403in}}%
\pgfpathlineto{\pgfqpoint{3.438297in}{3.497801in}}%
\pgfpathlineto{\pgfqpoint{3.455891in}{3.509335in}}%
\pgfpathlineto{\pgfqpoint{3.491079in}{3.537907in}}%
\pgfpathlineto{\pgfqpoint{3.526267in}{3.572145in}}%
\pgfpathlineto{\pgfqpoint{3.579048in}{3.629359in}}%
\pgfpathlineto{\pgfqpoint{3.631830in}{3.686886in}}%
\pgfpathlineto{\pgfqpoint{3.667017in}{3.721837in}}%
\pgfpathlineto{\pgfqpoint{3.702205in}{3.751835in}}%
\pgfpathlineto{\pgfqpoint{3.737393in}{3.775279in}}%
\pgfpathlineto{\pgfqpoint{3.754987in}{3.784133in}}%
\pgfpathlineto{\pgfqpoint{3.772581in}{3.790902in}}%
\pgfpathlineto{\pgfqpoint{3.790174in}{3.795490in}}%
\pgfpathlineto{\pgfqpoint{3.807768in}{3.797822in}}%
\pgfpathlineto{\pgfqpoint{3.825362in}{3.797851in}}%
\pgfpathlineto{\pgfqpoint{3.842956in}{3.795554in}}%
\pgfpathlineto{\pgfqpoint{3.860550in}{3.790935in}}%
\pgfpathlineto{\pgfqpoint{3.878144in}{3.784023in}}%
\pgfpathlineto{\pgfqpoint{3.895738in}{3.774869in}}%
\pgfpathlineto{\pgfqpoint{3.913331in}{3.763550in}}%
\pgfpathlineto{\pgfqpoint{3.930925in}{3.750163in}}%
\pgfpathlineto{\pgfqpoint{3.966113in}{3.717679in}}%
\pgfpathlineto{\pgfqpoint{4.001301in}{3.678590in}}%
\pgfpathlineto{\pgfqpoint{4.036488in}{3.634316in}}%
\pgfpathlineto{\pgfqpoint{4.089270in}{3.561745in}}%
\pgfpathlineto{\pgfqpoint{4.177239in}{3.438976in}}%
\pgfpathlineto{\pgfqpoint{4.212427in}{3.394285in}}%
\pgfpathlineto{\pgfqpoint{4.247615in}{3.354476in}}%
\pgfpathlineto{\pgfqpoint{4.282802in}{3.320862in}}%
\pgfpathlineto{\pgfqpoint{4.317990in}{3.294501in}}%
\pgfpathlineto{\pgfqpoint{4.335584in}{3.284295in}}%
\pgfpathlineto{\pgfqpoint{4.353178in}{3.276168in}}%
\pgfpathlineto{\pgfqpoint{4.370772in}{3.270170in}}%
\pgfpathlineto{\pgfqpoint{4.388365in}{3.266329in}}%
\pgfpathlineto{\pgfqpoint{4.405959in}{3.264652in}}%
\pgfpathlineto{\pgfqpoint{4.423553in}{3.265128in}}%
\pgfpathlineto{\pgfqpoint{4.441147in}{3.267724in}}%
\pgfpathlineto{\pgfqpoint{4.458741in}{3.272390in}}%
\pgfpathlineto{\pgfqpoint{4.476335in}{3.279057in}}%
\pgfpathlineto{\pgfqpoint{4.493928in}{3.287638in}}%
\pgfpathlineto{\pgfqpoint{4.529116in}{3.310110in}}%
\pgfpathlineto{\pgfqpoint{4.564304in}{3.338800in}}%
\pgfpathlineto{\pgfqpoint{4.599492in}{3.372496in}}%
\pgfpathlineto{\pgfqpoint{4.652273in}{3.429409in}}%
\pgfpathlineto{\pgfqpoint{4.757836in}{3.547749in}}%
\pgfpathlineto{\pgfqpoint{4.793024in}{3.583184in}}%
\pgfpathlineto{\pgfqpoint{4.828212in}{3.614448in}}%
\pgfpathlineto{\pgfqpoint{4.863399in}{3.640498in}}%
\pgfpathlineto{\pgfqpoint{4.898587in}{3.660514in}}%
\pgfpathlineto{\pgfqpoint{4.916181in}{3.668070in}}%
\pgfpathlineto{\pgfqpoint{4.933775in}{3.673925in}}%
\pgfpathlineto{\pgfqpoint{4.951369in}{3.678048in}}%
\pgfpathlineto{\pgfqpoint{4.968962in}{3.680428in}}%
\pgfpathlineto{\pgfqpoint{4.986556in}{3.681070in}}%
\pgfpathlineto{\pgfqpoint{5.004150in}{3.679995in}}%
\pgfpathlineto{\pgfqpoint{5.021744in}{3.677244in}}%
\pgfpathlineto{\pgfqpoint{5.056932in}{3.666945in}}%
\pgfpathlineto{\pgfqpoint{5.092119in}{3.650798in}}%
\pgfpathlineto{\pgfqpoint{5.127307in}{3.629648in}}%
\pgfpathlineto{\pgfqpoint{5.162495in}{3.604533in}}%
\pgfpathlineto{\pgfqpoint{5.215276in}{3.562048in}}%
\pgfpathlineto{\pgfqpoint{5.303246in}{3.489400in}}%
\pgfpathlineto{\pgfqpoint{5.338433in}{3.463599in}}%
\pgfpathlineto{\pgfqpoint{5.373621in}{3.441541in}}%
\pgfpathlineto{\pgfqpoint{5.408809in}{3.424319in}}%
\pgfpathlineto{\pgfqpoint{5.443996in}{3.412852in}}%
\pgfpathlineto{\pgfqpoint{5.461590in}{3.409511in}}%
\pgfpathlineto{\pgfqpoint{5.479184in}{3.407862in}}%
\pgfpathlineto{\pgfqpoint{5.496778in}{3.407958in}}%
\pgfpathlineto{\pgfqpoint{5.514372in}{3.409840in}}%
\pgfpathlineto{\pgfqpoint{5.531966in}{3.413530in}}%
\pgfpathlineto{\pgfqpoint{5.549560in}{3.419035in}}%
\pgfpathlineto{\pgfqpoint{5.567153in}{3.426348in}}%
\pgfpathlineto{\pgfqpoint{5.584747in}{3.435444in}}%
\pgfpathlineto{\pgfqpoint{5.619935in}{3.458809in}}%
\pgfpathlineto{\pgfqpoint{5.655123in}{3.488625in}}%
\pgfpathlineto{\pgfqpoint{5.690310in}{3.524156in}}%
\pgfpathlineto{\pgfqpoint{5.725498in}{3.564451in}}%
\pgfpathlineto{\pgfqpoint{5.778280in}{3.631303in}}%
\pgfpathlineto{\pgfqpoint{5.901437in}{3.792955in}}%
\pgfpathlineto{\pgfqpoint{5.936624in}{3.833759in}}%
\pgfpathlineto{\pgfqpoint{5.971812in}{3.869315in}}%
\pgfpathlineto{\pgfqpoint{6.007000in}{3.898107in}}%
\pgfpathlineto{\pgfqpoint{6.024594in}{3.909512in}}%
\pgfpathlineto{\pgfqpoint{6.042187in}{3.918689in}}%
\pgfpathlineto{\pgfqpoint{6.059781in}{3.925471in}}%
\pgfpathlineto{\pgfqpoint{6.077375in}{3.929697in}}%
\pgfpathlineto{\pgfqpoint{6.094969in}{3.931213in}}%
\pgfpathlineto{\pgfqpoint{6.112563in}{3.929870in}}%
\pgfpathlineto{\pgfqpoint{6.130157in}{3.925527in}}%
\pgfpathlineto{\pgfqpoint{6.147751in}{3.918051in}}%
\pgfpathlineto{\pgfqpoint{6.165344in}{3.907315in}}%
\pgfpathlineto{\pgfqpoint{6.182938in}{3.893198in}}%
\pgfpathlineto{\pgfqpoint{6.200532in}{3.875588in}}%
\pgfpathlineto{\pgfqpoint{6.218126in}{3.854379in}}%
\pgfpathlineto{\pgfqpoint{6.235720in}{3.829470in}}%
\pgfpathlineto{\pgfqpoint{6.253314in}{3.800768in}}%
\pgfpathlineto{\pgfqpoint{6.270907in}{3.768184in}}%
\pgfpathlineto{\pgfqpoint{6.288501in}{3.731636in}}%
\pgfpathlineto{\pgfqpoint{6.306095in}{3.691047in}}%
\pgfpathlineto{\pgfqpoint{6.341283in}{3.597459in}}%
\pgfpathlineto{\pgfqpoint{6.376471in}{3.486891in}}%
\pgfpathlineto{\pgfqpoint{6.376471in}{3.486891in}}%
\pgfusepath{stroke}%
\end{pgfscope}%
\begin{pgfscope}%
\pgfpathrectangle{\pgfqpoint{2.000000in}{3.134211in}}{\pgfqpoint{4.376471in}{0.953684in}} %
\pgfusepath{clip}%
\pgfsetbuttcap%
\pgfsetroundjoin%
\pgfsetlinewidth{1.505625pt}%
\definecolor{currentstroke}{rgb}{1.000000,0.498039,0.054902}%
\pgfsetstrokecolor{currentstroke}%
\pgfsetdash{{9.600000pt}{2.400000pt}{1.600000pt}{2.400000pt}}{0.000000pt}%
\pgfpathmoveto{\pgfqpoint{2.875294in}{3.694293in}}%
\pgfpathlineto{\pgfqpoint{2.892888in}{3.745126in}}%
\pgfpathlineto{\pgfqpoint{2.910482in}{3.784647in}}%
\pgfpathlineto{\pgfqpoint{2.928076in}{3.813804in}}%
\pgfpathlineto{\pgfqpoint{2.945670in}{3.833545in}}%
\pgfpathlineto{\pgfqpoint{2.963263in}{3.844812in}}%
\pgfpathlineto{\pgfqpoint{2.980857in}{3.848532in}}%
\pgfpathlineto{\pgfqpoint{2.998451in}{3.845619in}}%
\pgfpathlineto{\pgfqpoint{3.016045in}{3.836960in}}%
\pgfpathlineto{\pgfqpoint{3.033639in}{3.823417in}}%
\pgfpathlineto{\pgfqpoint{3.051233in}{3.805820in}}%
\pgfpathlineto{\pgfqpoint{3.068826in}{3.784962in}}%
\pgfpathlineto{\pgfqpoint{3.104014in}{3.736435in}}%
\pgfpathlineto{\pgfqpoint{3.191983in}{3.605257in}}%
\pgfpathlineto{\pgfqpoint{3.227171in}{3.559649in}}%
\pgfpathlineto{\pgfqpoint{3.244765in}{3.539848in}}%
\pgfpathlineto{\pgfqpoint{3.262359in}{3.522402in}}%
\pgfpathlineto{\pgfqpoint{3.279953in}{3.507508in}}%
\pgfpathlineto{\pgfqpoint{3.297547in}{3.495316in}}%
\pgfpathlineto{\pgfqpoint{3.315140in}{3.485925in}}%
\pgfpathlineto{\pgfqpoint{3.332734in}{3.479388in}}%
\pgfpathlineto{\pgfqpoint{3.350328in}{3.475714in}}%
\pgfpathlineto{\pgfqpoint{3.367922in}{3.474869in}}%
\pgfpathlineto{\pgfqpoint{3.385516in}{3.476780in}}%
\pgfpathlineto{\pgfqpoint{3.403110in}{3.481339in}}%
\pgfpathlineto{\pgfqpoint{3.420704in}{3.488403in}}%
\pgfpathlineto{\pgfqpoint{3.438297in}{3.497801in}}%
\pgfpathlineto{\pgfqpoint{3.455891in}{3.509335in}}%
\pgfpathlineto{\pgfqpoint{3.491079in}{3.537907in}}%
\pgfpathlineto{\pgfqpoint{3.526267in}{3.572145in}}%
\pgfpathlineto{\pgfqpoint{3.579048in}{3.629359in}}%
\pgfpathlineto{\pgfqpoint{3.631830in}{3.686886in}}%
\pgfpathlineto{\pgfqpoint{3.667017in}{3.721837in}}%
\pgfpathlineto{\pgfqpoint{3.702205in}{3.751835in}}%
\pgfpathlineto{\pgfqpoint{3.737393in}{3.775279in}}%
\pgfpathlineto{\pgfqpoint{3.754987in}{3.784133in}}%
\pgfpathlineto{\pgfqpoint{3.772581in}{3.790902in}}%
\pgfpathlineto{\pgfqpoint{3.790174in}{3.795490in}}%
\pgfpathlineto{\pgfqpoint{3.807768in}{3.797822in}}%
\pgfpathlineto{\pgfqpoint{3.825362in}{3.797851in}}%
\pgfpathlineto{\pgfqpoint{3.842956in}{3.795554in}}%
\pgfpathlineto{\pgfqpoint{3.860550in}{3.790935in}}%
\pgfpathlineto{\pgfqpoint{3.878144in}{3.784023in}}%
\pgfpathlineto{\pgfqpoint{3.895738in}{3.774869in}}%
\pgfpathlineto{\pgfqpoint{3.913331in}{3.763550in}}%
\pgfpathlineto{\pgfqpoint{3.930925in}{3.750163in}}%
\pgfpathlineto{\pgfqpoint{3.966113in}{3.717679in}}%
\pgfpathlineto{\pgfqpoint{4.001301in}{3.678590in}}%
\pgfpathlineto{\pgfqpoint{4.036488in}{3.634316in}}%
\pgfpathlineto{\pgfqpoint{4.089270in}{3.561745in}}%
\pgfpathlineto{\pgfqpoint{4.177239in}{3.438976in}}%
\pgfpathlineto{\pgfqpoint{4.212427in}{3.394285in}}%
\pgfpathlineto{\pgfqpoint{4.247615in}{3.354476in}}%
\pgfpathlineto{\pgfqpoint{4.282802in}{3.320862in}}%
\pgfpathlineto{\pgfqpoint{4.317990in}{3.294501in}}%
\pgfpathlineto{\pgfqpoint{4.335584in}{3.284295in}}%
\pgfpathlineto{\pgfqpoint{4.353178in}{3.276168in}}%
\pgfpathlineto{\pgfqpoint{4.370772in}{3.270170in}}%
\pgfpathlineto{\pgfqpoint{4.388365in}{3.266329in}}%
\pgfpathlineto{\pgfqpoint{4.405959in}{3.264652in}}%
\pgfpathlineto{\pgfqpoint{4.423553in}{3.265128in}}%
\pgfpathlineto{\pgfqpoint{4.441147in}{3.267724in}}%
\pgfpathlineto{\pgfqpoint{4.458741in}{3.272390in}}%
\pgfpathlineto{\pgfqpoint{4.476335in}{3.279057in}}%
\pgfpathlineto{\pgfqpoint{4.493928in}{3.287638in}}%
\pgfpathlineto{\pgfqpoint{4.529116in}{3.310110in}}%
\pgfpathlineto{\pgfqpoint{4.564304in}{3.338800in}}%
\pgfpathlineto{\pgfqpoint{4.599492in}{3.372496in}}%
\pgfpathlineto{\pgfqpoint{4.652273in}{3.429409in}}%
\pgfpathlineto{\pgfqpoint{4.757836in}{3.547749in}}%
\pgfpathlineto{\pgfqpoint{4.793024in}{3.583184in}}%
\pgfpathlineto{\pgfqpoint{4.828212in}{3.614448in}}%
\pgfpathlineto{\pgfqpoint{4.863399in}{3.640498in}}%
\pgfpathlineto{\pgfqpoint{4.898587in}{3.660514in}}%
\pgfpathlineto{\pgfqpoint{4.916181in}{3.668070in}}%
\pgfpathlineto{\pgfqpoint{4.933775in}{3.673925in}}%
\pgfpathlineto{\pgfqpoint{4.951369in}{3.678048in}}%
\pgfpathlineto{\pgfqpoint{4.968962in}{3.680428in}}%
\pgfpathlineto{\pgfqpoint{4.986556in}{3.681070in}}%
\pgfpathlineto{\pgfqpoint{5.004150in}{3.679995in}}%
\pgfpathlineto{\pgfqpoint{5.021744in}{3.677244in}}%
\pgfpathlineto{\pgfqpoint{5.056932in}{3.666945in}}%
\pgfpathlineto{\pgfqpoint{5.092119in}{3.650798in}}%
\pgfpathlineto{\pgfqpoint{5.127307in}{3.629648in}}%
\pgfpathlineto{\pgfqpoint{5.162495in}{3.604533in}}%
\pgfpathlineto{\pgfqpoint{5.215276in}{3.562048in}}%
\pgfpathlineto{\pgfqpoint{5.303246in}{3.489400in}}%
\pgfpathlineto{\pgfqpoint{5.338433in}{3.463599in}}%
\pgfpathlineto{\pgfqpoint{5.373621in}{3.441541in}}%
\pgfpathlineto{\pgfqpoint{5.408809in}{3.424319in}}%
\pgfpathlineto{\pgfqpoint{5.443996in}{3.412852in}}%
\pgfpathlineto{\pgfqpoint{5.461590in}{3.409511in}}%
\pgfpathlineto{\pgfqpoint{5.479184in}{3.407862in}}%
\pgfpathlineto{\pgfqpoint{5.496778in}{3.407958in}}%
\pgfpathlineto{\pgfqpoint{5.514372in}{3.409840in}}%
\pgfpathlineto{\pgfqpoint{5.531966in}{3.413530in}}%
\pgfpathlineto{\pgfqpoint{5.549560in}{3.419035in}}%
\pgfpathlineto{\pgfqpoint{5.567153in}{3.426348in}}%
\pgfpathlineto{\pgfqpoint{5.584747in}{3.435444in}}%
\pgfpathlineto{\pgfqpoint{5.619935in}{3.458809in}}%
\pgfpathlineto{\pgfqpoint{5.655123in}{3.488625in}}%
\pgfpathlineto{\pgfqpoint{5.690310in}{3.524156in}}%
\pgfpathlineto{\pgfqpoint{5.725498in}{3.564451in}}%
\pgfpathlineto{\pgfqpoint{5.778280in}{3.631303in}}%
\pgfpathlineto{\pgfqpoint{5.901437in}{3.792955in}}%
\pgfpathlineto{\pgfqpoint{5.936624in}{3.833759in}}%
\pgfpathlineto{\pgfqpoint{5.971812in}{3.869315in}}%
\pgfpathlineto{\pgfqpoint{6.007000in}{3.898107in}}%
\pgfpathlineto{\pgfqpoint{6.024594in}{3.909512in}}%
\pgfpathlineto{\pgfqpoint{6.042187in}{3.918689in}}%
\pgfpathlineto{\pgfqpoint{6.059781in}{3.925471in}}%
\pgfpathlineto{\pgfqpoint{6.077375in}{3.929697in}}%
\pgfpathlineto{\pgfqpoint{6.094969in}{3.931213in}}%
\pgfpathlineto{\pgfqpoint{6.112563in}{3.929870in}}%
\pgfpathlineto{\pgfqpoint{6.130157in}{3.925527in}}%
\pgfpathlineto{\pgfqpoint{6.147751in}{3.918051in}}%
\pgfpathlineto{\pgfqpoint{6.165344in}{3.907315in}}%
\pgfpathlineto{\pgfqpoint{6.182938in}{3.893198in}}%
\pgfpathlineto{\pgfqpoint{6.200532in}{3.875588in}}%
\pgfpathlineto{\pgfqpoint{6.218126in}{3.854379in}}%
\pgfpathlineto{\pgfqpoint{6.235720in}{3.829470in}}%
\pgfpathlineto{\pgfqpoint{6.253314in}{3.800768in}}%
\pgfpathlineto{\pgfqpoint{6.270907in}{3.768184in}}%
\pgfpathlineto{\pgfqpoint{6.288501in}{3.731636in}}%
\pgfpathlineto{\pgfqpoint{6.306095in}{3.691047in}}%
\pgfpathlineto{\pgfqpoint{6.341283in}{3.597459in}}%
\pgfpathlineto{\pgfqpoint{6.376471in}{3.486891in}}%
\pgfpathlineto{\pgfqpoint{6.376471in}{3.486891in}}%
\pgfusepath{stroke}%
\end{pgfscope}%
\begin{pgfscope}%
\pgfsetrectcap%
\pgfsetmiterjoin%
\pgfsetlinewidth{0.803000pt}%
\definecolor{currentstroke}{rgb}{0.000000,0.000000,0.000000}%
\pgfsetstrokecolor{currentstroke}%
\pgfsetdash{}{0pt}%
\pgfpathmoveto{\pgfqpoint{2.000000in}{3.134211in}}%
\pgfpathlineto{\pgfqpoint{2.000000in}{4.087895in}}%
\pgfusepath{stroke}%
\end{pgfscope}%
\begin{pgfscope}%
\pgfsetrectcap%
\pgfsetmiterjoin%
\pgfsetlinewidth{0.803000pt}%
\definecolor{currentstroke}{rgb}{0.000000,0.000000,0.000000}%
\pgfsetstrokecolor{currentstroke}%
\pgfsetdash{}{0pt}%
\pgfpathmoveto{\pgfqpoint{6.376471in}{3.134211in}}%
\pgfpathlineto{\pgfqpoint{6.376471in}{4.087895in}}%
\pgfusepath{stroke}%
\end{pgfscope}%
\begin{pgfscope}%
\pgfsetrectcap%
\pgfsetmiterjoin%
\pgfsetlinewidth{0.803000pt}%
\definecolor{currentstroke}{rgb}{0.000000,0.000000,0.000000}%
\pgfsetstrokecolor{currentstroke}%
\pgfsetdash{}{0pt}%
\pgfpathmoveto{\pgfqpoint{2.000000in}{3.134211in}}%
\pgfpathlineto{\pgfqpoint{6.376471in}{3.134211in}}%
\pgfusepath{stroke}%
\end{pgfscope}%
\begin{pgfscope}%
\pgfsetrectcap%
\pgfsetmiterjoin%
\pgfsetlinewidth{0.803000pt}%
\definecolor{currentstroke}{rgb}{0.000000,0.000000,0.000000}%
\pgfsetstrokecolor{currentstroke}%
\pgfsetdash{}{0pt}%
\pgfpathmoveto{\pgfqpoint{2.000000in}{4.087895in}}%
\pgfpathlineto{\pgfqpoint{6.376471in}{4.087895in}}%
\pgfusepath{stroke}%
\end{pgfscope}%
\begin{pgfscope}%
\pgfsetbuttcap%
\pgfsetmiterjoin%
\definecolor{currentfill}{rgb}{1.000000,1.000000,1.000000}%
\pgfsetfillcolor{currentfill}%
\pgfsetfillopacity{0.800000}%
\pgfsetlinewidth{1.003750pt}%
\definecolor{currentstroke}{rgb}{0.800000,0.800000,0.800000}%
\pgfsetstrokecolor{currentstroke}%
\pgfsetstrokeopacity{0.800000}%
\pgfsetdash{}{0pt}%
\pgfpathmoveto{\pgfqpoint{2.097222in}{3.203655in}}%
\pgfpathlineto{\pgfqpoint{2.796183in}{3.203655in}}%
\pgfpathquadraticcurveto{\pgfqpoint{2.823961in}{3.203655in}}{\pgfqpoint{2.823961in}{3.231433in}}%
\pgfpathlineto{\pgfqpoint{2.823961in}{4.020007in}}%
\pgfpathquadraticcurveto{\pgfqpoint{2.823961in}{4.047784in}}{\pgfqpoint{2.796183in}{4.047784in}}%
\pgfpathlineto{\pgfqpoint{2.097222in}{4.047784in}}%
\pgfpathquadraticcurveto{\pgfqpoint{2.069444in}{4.047784in}}{\pgfqpoint{2.069444in}{4.020007in}}%
\pgfpathlineto{\pgfqpoint{2.069444in}{3.231433in}}%
\pgfpathquadraticcurveto{\pgfqpoint{2.069444in}{3.203655in}}{\pgfqpoint{2.097222in}{3.203655in}}%
\pgfpathclose%
\pgfusepath{stroke,fill}%
\end{pgfscope}%
\begin{pgfscope}%
\pgfsetrectcap%
\pgfsetroundjoin%
\pgfsetlinewidth{1.505625pt}%
\definecolor{currentstroke}{rgb}{0.121569,0.466667,0.705882}%
\pgfsetstrokecolor{currentstroke}%
\pgfsetdash{}{0pt}%
\pgfpathmoveto{\pgfqpoint{2.125000in}{3.935127in}}%
\pgfpathlineto{\pgfqpoint{2.402778in}{3.935127in}}%
\pgfusepath{stroke}%
\end{pgfscope}%
\begin{pgfscope}%
\pgftext[x=2.513889in,y=3.886516in,left,base]{\rmfamily\fontsize{10.000000}{12.000000}\selectfont \(\displaystyle \widetilde{\Phi}^* \theta\)}%
\end{pgfscope}%
\begin{pgfscope}%
\pgfsetbuttcap%
\pgfsetroundjoin%
\pgfsetlinewidth{1.505625pt}%
\definecolor{currentstroke}{rgb}{1.000000,0.498039,0.054902}%
\pgfsetstrokecolor{currentstroke}%
\pgfsetdash{{9.600000pt}{2.400000pt}{1.600000pt}{2.400000pt}}{0.000000pt}%
\pgfpathmoveto{\pgfqpoint{2.125000in}{3.730266in}}%
\pgfpathlineto{\pgfqpoint{2.402778in}{3.730266in}}%
\pgfusepath{stroke}%
\end{pgfscope}%
\begin{pgfscope}%
\pgftext[x=2.513889in,y=3.681655in,left,base]{\rmfamily\fontsize{10.000000}{12.000000}\selectfont \(\displaystyle \widetilde{K}u\)}%
\end{pgfscope}%
\begin{pgfscope}%
\pgfsetbuttcap%
\pgfsetroundjoin%
\definecolor{currentfill}{rgb}{1.000000,0.000000,0.000000}%
\pgfsetfillcolor{currentfill}%
\pgfsetlinewidth{2.007500pt}%
\definecolor{currentstroke}{rgb}{1.000000,0.000000,0.000000}%
\pgfsetstrokecolor{currentstroke}%
\pgfsetdash{}{0pt}%
\pgfpathmoveto{\pgfqpoint{2.222222in}{3.521743in}}%
\pgfpathlineto{\pgfqpoint{2.305556in}{3.521743in}}%
\pgfpathmoveto{\pgfqpoint{2.263889in}{3.480076in}}%
\pgfpathlineto{\pgfqpoint{2.263889in}{3.563409in}}%
\pgfusepath{stroke,fill}%
\end{pgfscope}%
\begin{pgfscope}%
\pgftext[x=2.513889in,y=3.485284in,left,base]{\rmfamily\fontsize{10.000000}{12.000000}\selectfont train}%
\end{pgfscope}%
\begin{pgfscope}%
\pgfsetbuttcap%
\pgfsetroundjoin%
\definecolor{currentfill}{rgb}{0.000000,0.000000,0.000000}%
\pgfsetfillcolor{currentfill}%
\pgfsetlinewidth{1.003750pt}%
\definecolor{currentstroke}{rgb}{0.000000,0.000000,0.000000}%
\pgfsetstrokecolor{currentstroke}%
\pgfsetdash{}{0pt}%
\pgfsys@defobject{currentmarker}{\pgfqpoint{-0.020833in}{-0.020833in}}{\pgfqpoint{0.020833in}{0.020833in}}{%
\pgfpathmoveto{\pgfqpoint{0.000000in}{-0.020833in}}%
\pgfpathcurveto{\pgfqpoint{0.005525in}{-0.020833in}}{\pgfqpoint{0.010825in}{-0.018638in}}{\pgfqpoint{0.014731in}{-0.014731in}}%
\pgfpathcurveto{\pgfqpoint{0.018638in}{-0.010825in}}{\pgfqpoint{0.020833in}{-0.005525in}}{\pgfqpoint{0.020833in}{0.000000in}}%
\pgfpathcurveto{\pgfqpoint{0.020833in}{0.005525in}}{\pgfqpoint{0.018638in}{0.010825in}}{\pgfqpoint{0.014731in}{0.014731in}}%
\pgfpathcurveto{\pgfqpoint{0.010825in}{0.018638in}}{\pgfqpoint{0.005525in}{0.020833in}}{\pgfqpoint{0.000000in}{0.020833in}}%
\pgfpathcurveto{\pgfqpoint{-0.005525in}{0.020833in}}{\pgfqpoint{-0.010825in}{0.018638in}}{\pgfqpoint{-0.014731in}{0.014731in}}%
\pgfpathcurveto{\pgfqpoint{-0.018638in}{0.010825in}}{\pgfqpoint{-0.020833in}{0.005525in}}{\pgfqpoint{-0.020833in}{0.000000in}}%
\pgfpathcurveto{\pgfqpoint{-0.020833in}{-0.005525in}}{\pgfqpoint{-0.018638in}{-0.010825in}}{\pgfqpoint{-0.014731in}{-0.014731in}}%
\pgfpathcurveto{\pgfqpoint{-0.010825in}{-0.018638in}}{\pgfqpoint{-0.005525in}{-0.020833in}}{\pgfqpoint{0.000000in}{-0.020833in}}%
\pgfpathclose%
\pgfusepath{stroke,fill}%
}%
\begin{pgfscope}%
\pgfsys@transformshift{2.263889in}{3.325372in}%
\pgfsys@useobject{currentmarker}{}%
\end{pgfscope}%
\end{pgfscope}%
\begin{pgfscope}%
\pgftext[x=2.513889in,y=3.288914in,left,base]{\rmfamily\fontsize{10.000000}{12.000000}\selectfont test}%
\end{pgfscope}%
\begin{pgfscope}%
\pgfsetbuttcap%
\pgfsetmiterjoin%
\definecolor{currentfill}{rgb}{1.000000,1.000000,1.000000}%
\pgfsetfillcolor{currentfill}%
\pgfsetlinewidth{0.000000pt}%
\definecolor{currentstroke}{rgb}{0.000000,0.000000,0.000000}%
\pgfsetstrokecolor{currentstroke}%
\pgfsetstrokeopacity{0.000000}%
\pgfsetdash{}{0pt}%
\pgfpathmoveto{\pgfqpoint{7.105882in}{3.134211in}}%
\pgfpathlineto{\pgfqpoint{11.482353in}{3.134211in}}%
\pgfpathlineto{\pgfqpoint{11.482353in}{4.087895in}}%
\pgfpathlineto{\pgfqpoint{7.105882in}{4.087895in}}%
\pgfpathclose%
\pgfusepath{fill}%
\end{pgfscope}%
\begin{pgfscope}%
\pgfpathrectangle{\pgfqpoint{7.105882in}{3.134211in}}{\pgfqpoint{4.376471in}{0.953684in}} %
\pgfusepath{clip}%
\pgfsetbuttcap%
\pgfsetroundjoin%
\definecolor{currentfill}{rgb}{1.000000,0.000000,0.000000}%
\pgfsetfillcolor{currentfill}%
\pgfsetlinewidth{2.007500pt}%
\definecolor{currentstroke}{rgb}{1.000000,0.000000,0.000000}%
\pgfsetstrokecolor{currentstroke}%
\pgfsetdash{}{0pt}%
\pgfpathmoveto{\pgfqpoint{9.861003in}{3.490718in}}%
\pgfpathlineto{\pgfqpoint{9.944336in}{3.490718in}}%
\pgfpathmoveto{\pgfqpoint{9.902669in}{3.449051in}}%
\pgfpathlineto{\pgfqpoint{9.902669in}{3.532384in}}%
\pgfusepath{stroke,fill}%
\end{pgfscope}%
\begin{pgfscope}%
\pgfpathrectangle{\pgfqpoint{7.105882in}{3.134211in}}{\pgfqpoint{4.376471in}{0.953684in}} %
\pgfusepath{clip}%
\pgfsetbuttcap%
\pgfsetroundjoin%
\definecolor{currentfill}{rgb}{1.000000,0.000000,0.000000}%
\pgfsetfillcolor{currentfill}%
\pgfsetlinewidth{2.007500pt}%
\definecolor{currentstroke}{rgb}{1.000000,0.000000,0.000000}%
\pgfsetstrokecolor{currentstroke}%
\pgfsetdash{}{0pt}%
\pgfpathmoveto{\pgfqpoint{10.443514in}{3.576911in}}%
\pgfpathlineto{\pgfqpoint{10.526847in}{3.576911in}}%
\pgfpathmoveto{\pgfqpoint{10.485181in}{3.535244in}}%
\pgfpathlineto{\pgfqpoint{10.485181in}{3.618578in}}%
\pgfusepath{stroke,fill}%
\end{pgfscope}%
\begin{pgfscope}%
\pgfpathrectangle{\pgfqpoint{7.105882in}{3.134211in}}{\pgfqpoint{4.376471in}{0.953684in}} %
\pgfusepath{clip}%
\pgfsetbuttcap%
\pgfsetroundjoin%
\definecolor{currentfill}{rgb}{1.000000,0.000000,0.000000}%
\pgfsetfillcolor{currentfill}%
\pgfsetlinewidth{2.007500pt}%
\definecolor{currentstroke}{rgb}{1.000000,0.000000,0.000000}%
\pgfsetstrokecolor{currentstroke}%
\pgfsetdash{}{0pt}%
\pgfpathmoveto{\pgfqpoint{10.049891in}{3.633817in}}%
\pgfpathlineto{\pgfqpoint{10.133224in}{3.633817in}}%
\pgfpathmoveto{\pgfqpoint{10.091557in}{3.592151in}}%
\pgfpathlineto{\pgfqpoint{10.091557in}{3.675484in}}%
\pgfusepath{stroke,fill}%
\end{pgfscope}%
\begin{pgfscope}%
\pgfpathrectangle{\pgfqpoint{7.105882in}{3.134211in}}{\pgfqpoint{4.376471in}{0.953684in}} %
\pgfusepath{clip}%
\pgfsetbuttcap%
\pgfsetroundjoin%
\definecolor{currentfill}{rgb}{1.000000,0.000000,0.000000}%
\pgfsetfillcolor{currentfill}%
\pgfsetlinewidth{2.007500pt}%
\definecolor{currentstroke}{rgb}{1.000000,0.000000,0.000000}%
\pgfsetstrokecolor{currentstroke}%
\pgfsetdash{}{0pt}%
\pgfpathmoveto{\pgfqpoint{9.847242in}{3.606871in}}%
\pgfpathlineto{\pgfqpoint{9.930575in}{3.606871in}}%
\pgfpathmoveto{\pgfqpoint{9.888909in}{3.565204in}}%
\pgfpathlineto{\pgfqpoint{9.888909in}{3.648537in}}%
\pgfusepath{stroke,fill}%
\end{pgfscope}%
\begin{pgfscope}%
\pgfpathrectangle{\pgfqpoint{7.105882in}{3.134211in}}{\pgfqpoint{4.376471in}{0.953684in}} %
\pgfusepath{clip}%
\pgfsetbuttcap%
\pgfsetroundjoin%
\definecolor{currentfill}{rgb}{1.000000,0.000000,0.000000}%
\pgfsetfillcolor{currentfill}%
\pgfsetlinewidth{2.007500pt}%
\definecolor{currentstroke}{rgb}{1.000000,0.000000,0.000000}%
\pgfsetstrokecolor{currentstroke}%
\pgfsetdash{}{0pt}%
\pgfpathmoveto{\pgfqpoint{9.422800in}{3.178158in}}%
\pgfpathlineto{\pgfqpoint{9.506133in}{3.178158in}}%
\pgfpathmoveto{\pgfqpoint{9.464467in}{3.136492in}}%
\pgfpathlineto{\pgfqpoint{9.464467in}{3.219825in}}%
\pgfusepath{stroke,fill}%
\end{pgfscope}%
\begin{pgfscope}%
\pgfpathrectangle{\pgfqpoint{7.105882in}{3.134211in}}{\pgfqpoint{4.376471in}{0.953684in}} %
\pgfusepath{clip}%
\pgfsetbuttcap%
\pgfsetroundjoin%
\definecolor{currentfill}{rgb}{1.000000,0.000000,0.000000}%
\pgfsetfillcolor{currentfill}%
\pgfsetlinewidth{2.007500pt}%
\definecolor{currentstroke}{rgb}{1.000000,0.000000,0.000000}%
\pgfsetstrokecolor{currentstroke}%
\pgfsetdash{}{0pt}%
\pgfpathmoveto{\pgfqpoint{10.200899in}{3.646990in}}%
\pgfpathlineto{\pgfqpoint{10.284232in}{3.646990in}}%
\pgfpathmoveto{\pgfqpoint{10.242566in}{3.605323in}}%
\pgfpathlineto{\pgfqpoint{10.242566in}{3.688656in}}%
\pgfusepath{stroke,fill}%
\end{pgfscope}%
\begin{pgfscope}%
\pgfpathrectangle{\pgfqpoint{7.105882in}{3.134211in}}{\pgfqpoint{4.376471in}{0.953684in}} %
\pgfusepath{clip}%
\pgfsetbuttcap%
\pgfsetroundjoin%
\definecolor{currentfill}{rgb}{1.000000,0.000000,0.000000}%
\pgfsetfillcolor{currentfill}%
\pgfsetlinewidth{2.007500pt}%
\definecolor{currentstroke}{rgb}{1.000000,0.000000,0.000000}%
\pgfsetstrokecolor{currentstroke}%
\pgfsetdash{}{0pt}%
\pgfpathmoveto{\pgfqpoint{9.471580in}{3.149267in}}%
\pgfpathlineto{\pgfqpoint{9.554913in}{3.149267in}}%
\pgfpathmoveto{\pgfqpoint{9.513247in}{3.107600in}}%
\pgfpathlineto{\pgfqpoint{9.513247in}{3.190934in}}%
\pgfusepath{stroke,fill}%
\end{pgfscope}%
\begin{pgfscope}%
\pgfpathrectangle{\pgfqpoint{7.105882in}{3.134211in}}{\pgfqpoint{4.376471in}{0.953684in}} %
\pgfusepath{clip}%
\pgfsetbuttcap%
\pgfsetroundjoin%
\definecolor{currentfill}{rgb}{1.000000,0.000000,0.000000}%
\pgfsetfillcolor{currentfill}%
\pgfsetlinewidth{2.007500pt}%
\definecolor{currentstroke}{rgb}{1.000000,0.000000,0.000000}%
\pgfsetstrokecolor{currentstroke}%
\pgfsetdash{}{0pt}%
\pgfpathmoveto{\pgfqpoint{11.061764in}{3.913320in}}%
\pgfpathlineto{\pgfqpoint{11.145098in}{3.913320in}}%
\pgfpathmoveto{\pgfqpoint{11.103431in}{3.871654in}}%
\pgfpathlineto{\pgfqpoint{11.103431in}{3.954987in}}%
\pgfusepath{stroke,fill}%
\end{pgfscope}%
\begin{pgfscope}%
\pgfpathrectangle{\pgfqpoint{7.105882in}{3.134211in}}{\pgfqpoint{4.376471in}{0.953684in}} %
\pgfusepath{clip}%
\pgfsetbuttcap%
\pgfsetroundjoin%
\definecolor{currentfill}{rgb}{1.000000,0.000000,0.000000}%
\pgfsetfillcolor{currentfill}%
\pgfsetlinewidth{2.007500pt}%
\definecolor{currentstroke}{rgb}{1.000000,0.000000,0.000000}%
\pgfsetstrokecolor{currentstroke}%
\pgfsetdash{}{0pt}%
\pgfpathmoveto{\pgfqpoint{11.313463in}{3.774585in}}%
\pgfpathlineto{\pgfqpoint{11.396797in}{3.774585in}}%
\pgfpathmoveto{\pgfqpoint{11.355130in}{3.732918in}}%
\pgfpathlineto{\pgfqpoint{11.355130in}{3.816251in}}%
\pgfusepath{stroke,fill}%
\end{pgfscope}%
\begin{pgfscope}%
\pgfpathrectangle{\pgfqpoint{7.105882in}{3.134211in}}{\pgfqpoint{4.376471in}{0.953684in}} %
\pgfusepath{clip}%
\pgfsetbuttcap%
\pgfsetroundjoin%
\definecolor{currentfill}{rgb}{1.000000,0.000000,0.000000}%
\pgfsetfillcolor{currentfill}%
\pgfsetlinewidth{2.007500pt}%
\definecolor{currentstroke}{rgb}{1.000000,0.000000,0.000000}%
\pgfsetstrokecolor{currentstroke}%
\pgfsetdash{}{0pt}%
\pgfpathmoveto{\pgfqpoint{9.282006in}{3.407910in}}%
\pgfpathlineto{\pgfqpoint{9.365340in}{3.407910in}}%
\pgfpathmoveto{\pgfqpoint{9.323673in}{3.366243in}}%
\pgfpathlineto{\pgfqpoint{9.323673in}{3.449577in}}%
\pgfusepath{stroke,fill}%
\end{pgfscope}%
\begin{pgfscope}%
\pgfpathrectangle{\pgfqpoint{7.105882in}{3.134211in}}{\pgfqpoint{4.376471in}{0.953684in}} %
\pgfusepath{clip}%
\pgfsetbuttcap%
\pgfsetroundjoin%
\definecolor{currentfill}{rgb}{1.000000,0.000000,0.000000}%
\pgfsetfillcolor{currentfill}%
\pgfsetlinewidth{2.007500pt}%
\definecolor{currentstroke}{rgb}{1.000000,0.000000,0.000000}%
\pgfsetstrokecolor{currentstroke}%
\pgfsetdash{}{0pt}%
\pgfpathmoveto{\pgfqpoint{10.711479in}{3.490182in}}%
\pgfpathlineto{\pgfqpoint{10.794812in}{3.490182in}}%
\pgfpathmoveto{\pgfqpoint{10.753146in}{3.448515in}}%
\pgfpathlineto{\pgfqpoint{10.753146in}{3.531849in}}%
\pgfusepath{stroke,fill}%
\end{pgfscope}%
\begin{pgfscope}%
\pgfpathrectangle{\pgfqpoint{7.105882in}{3.134211in}}{\pgfqpoint{4.376471in}{0.953684in}} %
\pgfusepath{clip}%
\pgfsetbuttcap%
\pgfsetroundjoin%
\definecolor{currentfill}{rgb}{1.000000,0.000000,0.000000}%
\pgfsetfillcolor{currentfill}%
\pgfsetlinewidth{2.007500pt}%
\definecolor{currentstroke}{rgb}{1.000000,0.000000,0.000000}%
\pgfsetstrokecolor{currentstroke}%
\pgfsetdash{}{0pt}%
\pgfpathmoveto{\pgfqpoint{9.791264in}{3.484417in}}%
\pgfpathlineto{\pgfqpoint{9.874598in}{3.484417in}}%
\pgfpathmoveto{\pgfqpoint{9.832931in}{3.442750in}}%
\pgfpathlineto{\pgfqpoint{9.832931in}{3.526084in}}%
\pgfusepath{stroke,fill}%
\end{pgfscope}%
\begin{pgfscope}%
\pgfpathrectangle{\pgfqpoint{7.105882in}{3.134211in}}{\pgfqpoint{4.376471in}{0.953684in}} %
\pgfusepath{clip}%
\pgfsetbuttcap%
\pgfsetroundjoin%
\definecolor{currentfill}{rgb}{1.000000,0.000000,0.000000}%
\pgfsetfillcolor{currentfill}%
\pgfsetlinewidth{2.007500pt}%
\definecolor{currentstroke}{rgb}{1.000000,0.000000,0.000000}%
\pgfsetstrokecolor{currentstroke}%
\pgfsetdash{}{0pt}%
\pgfpathmoveto{\pgfqpoint{9.928334in}{3.690725in}}%
\pgfpathlineto{\pgfqpoint{10.011667in}{3.690725in}}%
\pgfpathmoveto{\pgfqpoint{9.970001in}{3.649058in}}%
\pgfpathlineto{\pgfqpoint{9.970001in}{3.732392in}}%
\pgfusepath{stroke,fill}%
\end{pgfscope}%
\begin{pgfscope}%
\pgfpathrectangle{\pgfqpoint{7.105882in}{3.134211in}}{\pgfqpoint{4.376471in}{0.953684in}} %
\pgfusepath{clip}%
\pgfsetbuttcap%
\pgfsetroundjoin%
\definecolor{currentfill}{rgb}{1.000000,0.000000,0.000000}%
\pgfsetfillcolor{currentfill}%
\pgfsetlinewidth{2.007500pt}%
\definecolor{currentstroke}{rgb}{1.000000,0.000000,0.000000}%
\pgfsetstrokecolor{currentstroke}%
\pgfsetdash{}{0pt}%
\pgfpathmoveto{\pgfqpoint{11.180187in}{3.979507in}}%
\pgfpathlineto{\pgfqpoint{11.263520in}{3.979507in}}%
\pgfpathmoveto{\pgfqpoint{11.221854in}{3.937840in}}%
\pgfpathlineto{\pgfqpoint{11.221854in}{4.021174in}}%
\pgfusepath{stroke,fill}%
\end{pgfscope}%
\begin{pgfscope}%
\pgfpathrectangle{\pgfqpoint{7.105882in}{3.134211in}}{\pgfqpoint{4.376471in}{0.953684in}} %
\pgfusepath{clip}%
\pgfsetbuttcap%
\pgfsetroundjoin%
\definecolor{currentfill}{rgb}{1.000000,0.000000,0.000000}%
\pgfsetfillcolor{currentfill}%
\pgfsetlinewidth{2.007500pt}%
\definecolor{currentstroke}{rgb}{1.000000,0.000000,0.000000}%
\pgfsetstrokecolor{currentstroke}%
\pgfsetdash{}{0pt}%
\pgfpathmoveto{\pgfqpoint{8.188220in}{3.642253in}}%
\pgfpathlineto{\pgfqpoint{8.271553in}{3.642253in}}%
\pgfpathmoveto{\pgfqpoint{8.229886in}{3.600587in}}%
\pgfpathlineto{\pgfqpoint{8.229886in}{3.683920in}}%
\pgfusepath{stroke,fill}%
\end{pgfscope}%
\begin{pgfscope}%
\pgfpathrectangle{\pgfqpoint{7.105882in}{3.134211in}}{\pgfqpoint{4.376471in}{0.953684in}} %
\pgfusepath{clip}%
\pgfsetbuttcap%
\pgfsetroundjoin%
\definecolor{currentfill}{rgb}{1.000000,0.000000,0.000000}%
\pgfsetfillcolor{currentfill}%
\pgfsetlinewidth{2.007500pt}%
\definecolor{currentstroke}{rgb}{1.000000,0.000000,0.000000}%
\pgfsetstrokecolor{currentstroke}%
\pgfsetdash{}{0pt}%
\pgfpathmoveto{\pgfqpoint{8.244565in}{3.610102in}}%
\pgfpathlineto{\pgfqpoint{8.327898in}{3.610102in}}%
\pgfpathmoveto{\pgfqpoint{8.286232in}{3.568435in}}%
\pgfpathlineto{\pgfqpoint{8.286232in}{3.651769in}}%
\pgfusepath{stroke,fill}%
\end{pgfscope}%
\begin{pgfscope}%
\pgfpathrectangle{\pgfqpoint{7.105882in}{3.134211in}}{\pgfqpoint{4.376471in}{0.953684in}} %
\pgfusepath{clip}%
\pgfsetbuttcap%
\pgfsetroundjoin%
\definecolor{currentfill}{rgb}{1.000000,0.000000,0.000000}%
\pgfsetfillcolor{currentfill}%
\pgfsetlinewidth{2.007500pt}%
\definecolor{currentstroke}{rgb}{1.000000,0.000000,0.000000}%
\pgfsetstrokecolor{currentstroke}%
\pgfsetdash{}{0pt}%
\pgfpathmoveto{\pgfqpoint{8.010298in}{3.828490in}}%
\pgfpathlineto{\pgfqpoint{8.093631in}{3.828490in}}%
\pgfpathmoveto{\pgfqpoint{8.051965in}{3.786823in}}%
\pgfpathlineto{\pgfqpoint{8.051965in}{3.870157in}}%
\pgfusepath{stroke,fill}%
\end{pgfscope}%
\begin{pgfscope}%
\pgfpathrectangle{\pgfqpoint{7.105882in}{3.134211in}}{\pgfqpoint{4.376471in}{0.953684in}} %
\pgfusepath{clip}%
\pgfsetbuttcap%
\pgfsetroundjoin%
\definecolor{currentfill}{rgb}{1.000000,0.000000,0.000000}%
\pgfsetfillcolor{currentfill}%
\pgfsetlinewidth{2.007500pt}%
\definecolor{currentstroke}{rgb}{1.000000,0.000000,0.000000}%
\pgfsetstrokecolor{currentstroke}%
\pgfsetdash{}{0pt}%
\pgfpathmoveto{\pgfqpoint{10.854659in}{3.703361in}}%
\pgfpathlineto{\pgfqpoint{10.937992in}{3.703361in}}%
\pgfpathmoveto{\pgfqpoint{10.896325in}{3.661694in}}%
\pgfpathlineto{\pgfqpoint{10.896325in}{3.745027in}}%
\pgfusepath{stroke,fill}%
\end{pgfscope}%
\begin{pgfscope}%
\pgfpathrectangle{\pgfqpoint{7.105882in}{3.134211in}}{\pgfqpoint{4.376471in}{0.953684in}} %
\pgfusepath{clip}%
\pgfsetbuttcap%
\pgfsetroundjoin%
\definecolor{currentfill}{rgb}{1.000000,0.000000,0.000000}%
\pgfsetfillcolor{currentfill}%
\pgfsetlinewidth{2.007500pt}%
\definecolor{currentstroke}{rgb}{1.000000,0.000000,0.000000}%
\pgfsetstrokecolor{currentstroke}%
\pgfsetdash{}{0pt}%
\pgfpathmoveto{\pgfqpoint{10.663974in}{3.417001in}}%
\pgfpathlineto{\pgfqpoint{10.747307in}{3.417001in}}%
\pgfpathmoveto{\pgfqpoint{10.705641in}{3.375334in}}%
\pgfpathlineto{\pgfqpoint{10.705641in}{3.458668in}}%
\pgfusepath{stroke,fill}%
\end{pgfscope}%
\begin{pgfscope}%
\pgfpathrectangle{\pgfqpoint{7.105882in}{3.134211in}}{\pgfqpoint{4.376471in}{0.953684in}} %
\pgfusepath{clip}%
\pgfsetbuttcap%
\pgfsetroundjoin%
\definecolor{currentfill}{rgb}{1.000000,0.000000,0.000000}%
\pgfsetfillcolor{currentfill}%
\pgfsetlinewidth{2.007500pt}%
\definecolor{currentstroke}{rgb}{1.000000,0.000000,0.000000}%
\pgfsetstrokecolor{currentstroke}%
\pgfsetdash{}{0pt}%
\pgfpathmoveto{\pgfqpoint{10.985576in}{3.852317in}}%
\pgfpathlineto{\pgfqpoint{11.068909in}{3.852317in}}%
\pgfpathmoveto{\pgfqpoint{11.027243in}{3.810650in}}%
\pgfpathlineto{\pgfqpoint{11.027243in}{3.893984in}}%
\pgfusepath{stroke,fill}%
\end{pgfscope}%
\begin{pgfscope}%
\pgfpathrectangle{\pgfqpoint{7.105882in}{3.134211in}}{\pgfqpoint{4.376471in}{0.953684in}} %
\pgfusepath{clip}%
\pgfsetbuttcap%
\pgfsetroundjoin%
\definecolor{currentfill}{rgb}{1.000000,0.000000,0.000000}%
\pgfsetfillcolor{currentfill}%
\pgfsetlinewidth{2.007500pt}%
\definecolor{currentstroke}{rgb}{1.000000,0.000000,0.000000}%
\pgfsetstrokecolor{currentstroke}%
\pgfsetdash{}{0pt}%
\pgfpathmoveto{\pgfqpoint{11.365825in}{3.745955in}}%
\pgfpathlineto{\pgfqpoint{11.449159in}{3.745955in}}%
\pgfpathmoveto{\pgfqpoint{11.407492in}{3.704288in}}%
\pgfpathlineto{\pgfqpoint{11.407492in}{3.787621in}}%
\pgfusepath{stroke,fill}%
\end{pgfscope}%
\begin{pgfscope}%
\pgfpathrectangle{\pgfqpoint{7.105882in}{3.134211in}}{\pgfqpoint{4.376471in}{0.953684in}} %
\pgfusepath{clip}%
\pgfsetbuttcap%
\pgfsetroundjoin%
\definecolor{currentfill}{rgb}{1.000000,0.000000,0.000000}%
\pgfsetfillcolor{currentfill}%
\pgfsetlinewidth{2.007500pt}%
\definecolor{currentstroke}{rgb}{1.000000,0.000000,0.000000}%
\pgfsetstrokecolor{currentstroke}%
\pgfsetdash{}{0pt}%
\pgfpathmoveto{\pgfqpoint{10.737505in}{3.403883in}}%
\pgfpathlineto{\pgfqpoint{10.820838in}{3.403883in}}%
\pgfpathmoveto{\pgfqpoint{10.779172in}{3.362216in}}%
\pgfpathlineto{\pgfqpoint{10.779172in}{3.445550in}}%
\pgfusepath{stroke,fill}%
\end{pgfscope}%
\begin{pgfscope}%
\pgfpathrectangle{\pgfqpoint{7.105882in}{3.134211in}}{\pgfqpoint{4.376471in}{0.953684in}} %
\pgfusepath{clip}%
\pgfsetbuttcap%
\pgfsetroundjoin%
\definecolor{currentfill}{rgb}{1.000000,0.000000,0.000000}%
\pgfsetfillcolor{currentfill}%
\pgfsetlinewidth{2.007500pt}%
\definecolor{currentstroke}{rgb}{1.000000,0.000000,0.000000}%
\pgfsetstrokecolor{currentstroke}%
\pgfsetdash{}{0pt}%
\pgfpathmoveto{\pgfqpoint{9.555230in}{3.373457in}}%
\pgfpathlineto{\pgfqpoint{9.638564in}{3.373457in}}%
\pgfpathmoveto{\pgfqpoint{9.596897in}{3.331790in}}%
\pgfpathlineto{\pgfqpoint{9.596897in}{3.415123in}}%
\pgfusepath{stroke,fill}%
\end{pgfscope}%
\begin{pgfscope}%
\pgfpathrectangle{\pgfqpoint{7.105882in}{3.134211in}}{\pgfqpoint{4.376471in}{0.953684in}} %
\pgfusepath{clip}%
\pgfsetbuttcap%
\pgfsetroundjoin%
\definecolor{currentfill}{rgb}{1.000000,0.000000,0.000000}%
\pgfsetfillcolor{currentfill}%
\pgfsetlinewidth{2.007500pt}%
\definecolor{currentstroke}{rgb}{1.000000,0.000000,0.000000}%
\pgfsetstrokecolor{currentstroke}%
\pgfsetdash{}{0pt}%
\pgfpathmoveto{\pgfqpoint{10.672280in}{3.453230in}}%
\pgfpathlineto{\pgfqpoint{10.755614in}{3.453230in}}%
\pgfpathmoveto{\pgfqpoint{10.713947in}{3.411563in}}%
\pgfpathlineto{\pgfqpoint{10.713947in}{3.494896in}}%
\pgfusepath{stroke,fill}%
\end{pgfscope}%
\begin{pgfscope}%
\pgfpathrectangle{\pgfqpoint{7.105882in}{3.134211in}}{\pgfqpoint{4.376471in}{0.953684in}} %
\pgfusepath{clip}%
\pgfsetbuttcap%
\pgfsetroundjoin%
\definecolor{currentfill}{rgb}{1.000000,0.000000,0.000000}%
\pgfsetfillcolor{currentfill}%
\pgfsetlinewidth{2.007500pt}%
\definecolor{currentstroke}{rgb}{1.000000,0.000000,0.000000}%
\pgfsetstrokecolor{currentstroke}%
\pgfsetdash{}{0pt}%
\pgfpathmoveto{\pgfqpoint{8.353609in}{3.367586in}}%
\pgfpathlineto{\pgfqpoint{8.436943in}{3.367586in}}%
\pgfpathmoveto{\pgfqpoint{8.395276in}{3.325919in}}%
\pgfpathlineto{\pgfqpoint{8.395276in}{3.409252in}}%
\pgfusepath{stroke,fill}%
\end{pgfscope}%
\begin{pgfscope}%
\pgfpathrectangle{\pgfqpoint{7.105882in}{3.134211in}}{\pgfqpoint{4.376471in}{0.953684in}} %
\pgfusepath{clip}%
\pgfsetbuttcap%
\pgfsetroundjoin%
\definecolor{currentfill}{rgb}{1.000000,0.000000,0.000000}%
\pgfsetfillcolor{currentfill}%
\pgfsetlinewidth{2.007500pt}%
\definecolor{currentstroke}{rgb}{1.000000,0.000000,0.000000}%
\pgfsetstrokecolor{currentstroke}%
\pgfsetdash{}{0pt}%
\pgfpathmoveto{\pgfqpoint{10.179986in}{3.635787in}}%
\pgfpathlineto{\pgfqpoint{10.263320in}{3.635787in}}%
\pgfpathmoveto{\pgfqpoint{10.221653in}{3.594120in}}%
\pgfpathlineto{\pgfqpoint{10.221653in}{3.677454in}}%
\pgfusepath{stroke,fill}%
\end{pgfscope}%
\begin{pgfscope}%
\pgfpathrectangle{\pgfqpoint{7.105882in}{3.134211in}}{\pgfqpoint{4.376471in}{0.953684in}} %
\pgfusepath{clip}%
\pgfsetbuttcap%
\pgfsetroundjoin%
\definecolor{currentfill}{rgb}{1.000000,0.000000,0.000000}%
\pgfsetfillcolor{currentfill}%
\pgfsetlinewidth{2.007500pt}%
\definecolor{currentstroke}{rgb}{1.000000,0.000000,0.000000}%
\pgfsetstrokecolor{currentstroke}%
\pgfsetdash{}{0pt}%
\pgfpathmoveto{\pgfqpoint{8.441415in}{3.454651in}}%
\pgfpathlineto{\pgfqpoint{8.524748in}{3.454651in}}%
\pgfpathmoveto{\pgfqpoint{8.483082in}{3.412984in}}%
\pgfpathlineto{\pgfqpoint{8.483082in}{3.496317in}}%
\pgfusepath{stroke,fill}%
\end{pgfscope}%
\begin{pgfscope}%
\pgfpathrectangle{\pgfqpoint{7.105882in}{3.134211in}}{\pgfqpoint{4.376471in}{0.953684in}} %
\pgfusepath{clip}%
\pgfsetbuttcap%
\pgfsetroundjoin%
\definecolor{currentfill}{rgb}{1.000000,0.000000,0.000000}%
\pgfsetfillcolor{currentfill}%
\pgfsetlinewidth{2.007500pt}%
\definecolor{currentstroke}{rgb}{1.000000,0.000000,0.000000}%
\pgfsetstrokecolor{currentstroke}%
\pgfsetdash{}{0pt}%
\pgfpathmoveto{\pgfqpoint{11.246962in}{3.911971in}}%
\pgfpathlineto{\pgfqpoint{11.330296in}{3.911971in}}%
\pgfpathmoveto{\pgfqpoint{11.288629in}{3.870304in}}%
\pgfpathlineto{\pgfqpoint{11.288629in}{3.953638in}}%
\pgfusepath{stroke,fill}%
\end{pgfscope}%
\begin{pgfscope}%
\pgfpathrectangle{\pgfqpoint{7.105882in}{3.134211in}}{\pgfqpoint{4.376471in}{0.953684in}} %
\pgfusepath{clip}%
\pgfsetbuttcap%
\pgfsetroundjoin%
\definecolor{currentfill}{rgb}{1.000000,0.000000,0.000000}%
\pgfsetfillcolor{currentfill}%
\pgfsetlinewidth{2.007500pt}%
\definecolor{currentstroke}{rgb}{1.000000,0.000000,0.000000}%
\pgfsetstrokecolor{currentstroke}%
\pgfsetdash{}{0pt}%
\pgfpathmoveto{\pgfqpoint{9.766593in}{3.649112in}}%
\pgfpathlineto{\pgfqpoint{9.849926in}{3.649112in}}%
\pgfpathmoveto{\pgfqpoint{9.808260in}{3.607445in}}%
\pgfpathlineto{\pgfqpoint{9.808260in}{3.690779in}}%
\pgfusepath{stroke,fill}%
\end{pgfscope}%
\begin{pgfscope}%
\pgfpathrectangle{\pgfqpoint{7.105882in}{3.134211in}}{\pgfqpoint{4.376471in}{0.953684in}} %
\pgfusepath{clip}%
\pgfsetbuttcap%
\pgfsetroundjoin%
\definecolor{currentfill}{rgb}{1.000000,0.000000,0.000000}%
\pgfsetfillcolor{currentfill}%
\pgfsetlinewidth{2.007500pt}%
\definecolor{currentstroke}{rgb}{1.000000,0.000000,0.000000}%
\pgfsetstrokecolor{currentstroke}%
\pgfsetdash{}{0pt}%
\pgfpathmoveto{\pgfqpoint{9.391314in}{3.319380in}}%
\pgfpathlineto{\pgfqpoint{9.474648in}{3.319380in}}%
\pgfpathmoveto{\pgfqpoint{9.432981in}{3.277714in}}%
\pgfpathlineto{\pgfqpoint{9.432981in}{3.361047in}}%
\pgfusepath{stroke,fill}%
\end{pgfscope}%
\begin{pgfscope}%
\pgfpathrectangle{\pgfqpoint{7.105882in}{3.134211in}}{\pgfqpoint{4.376471in}{0.953684in}} %
\pgfusepath{clip}%
\pgfsetbuttcap%
\pgfsetroundjoin%
\definecolor{currentfill}{rgb}{1.000000,0.000000,0.000000}%
\pgfsetfillcolor{currentfill}%
\pgfsetlinewidth{2.007500pt}%
\definecolor{currentstroke}{rgb}{1.000000,0.000000,0.000000}%
\pgfsetstrokecolor{currentstroke}%
\pgfsetdash{}{0pt}%
\pgfpathmoveto{\pgfqpoint{8.865766in}{3.891827in}}%
\pgfpathlineto{\pgfqpoint{8.949099in}{3.891827in}}%
\pgfpathmoveto{\pgfqpoint{8.907432in}{3.850160in}}%
\pgfpathlineto{\pgfqpoint{8.907432in}{3.933493in}}%
\pgfusepath{stroke,fill}%
\end{pgfscope}%
\begin{pgfscope}%
\pgfpathrectangle{\pgfqpoint{7.105882in}{3.134211in}}{\pgfqpoint{4.376471in}{0.953684in}} %
\pgfusepath{clip}%
\pgfsetbuttcap%
\pgfsetroundjoin%
\definecolor{currentfill}{rgb}{1.000000,0.000000,0.000000}%
\pgfsetfillcolor{currentfill}%
\pgfsetlinewidth{2.007500pt}%
\definecolor{currentstroke}{rgb}{1.000000,0.000000,0.000000}%
\pgfsetstrokecolor{currentstroke}%
\pgfsetdash{}{0pt}%
\pgfpathmoveto{\pgfqpoint{10.650239in}{3.346386in}}%
\pgfpathlineto{\pgfqpoint{10.733572in}{3.346386in}}%
\pgfpathmoveto{\pgfqpoint{10.691905in}{3.304719in}}%
\pgfpathlineto{\pgfqpoint{10.691905in}{3.388052in}}%
\pgfusepath{stroke,fill}%
\end{pgfscope}%
\begin{pgfscope}%
\pgfpathrectangle{\pgfqpoint{7.105882in}{3.134211in}}{\pgfqpoint{4.376471in}{0.953684in}} %
\pgfusepath{clip}%
\pgfsetbuttcap%
\pgfsetroundjoin%
\definecolor{currentfill}{rgb}{1.000000,0.000000,0.000000}%
\pgfsetfillcolor{currentfill}%
\pgfsetlinewidth{2.007500pt}%
\definecolor{currentstroke}{rgb}{1.000000,0.000000,0.000000}%
\pgfsetstrokecolor{currentstroke}%
\pgfsetdash{}{0pt}%
\pgfpathmoveto{\pgfqpoint{9.536573in}{3.382115in}}%
\pgfpathlineto{\pgfqpoint{9.619906in}{3.382115in}}%
\pgfpathmoveto{\pgfqpoint{9.578239in}{3.340449in}}%
\pgfpathlineto{\pgfqpoint{9.578239in}{3.423782in}}%
\pgfusepath{stroke,fill}%
\end{pgfscope}%
\begin{pgfscope}%
\pgfpathrectangle{\pgfqpoint{7.105882in}{3.134211in}}{\pgfqpoint{4.376471in}{0.953684in}} %
\pgfusepath{clip}%
\pgfsetbuttcap%
\pgfsetroundjoin%
\definecolor{currentfill}{rgb}{1.000000,0.000000,0.000000}%
\pgfsetfillcolor{currentfill}%
\pgfsetlinewidth{2.007500pt}%
\definecolor{currentstroke}{rgb}{1.000000,0.000000,0.000000}%
\pgfsetstrokecolor{currentstroke}%
\pgfsetdash{}{0pt}%
\pgfpathmoveto{\pgfqpoint{9.929697in}{3.625895in}}%
\pgfpathlineto{\pgfqpoint{10.013031in}{3.625895in}}%
\pgfpathmoveto{\pgfqpoint{9.971364in}{3.584229in}}%
\pgfpathlineto{\pgfqpoint{9.971364in}{3.667562in}}%
\pgfusepath{stroke,fill}%
\end{pgfscope}%
\begin{pgfscope}%
\pgfpathrectangle{\pgfqpoint{7.105882in}{3.134211in}}{\pgfqpoint{4.376471in}{0.953684in}} %
\pgfusepath{clip}%
\pgfsetbuttcap%
\pgfsetroundjoin%
\definecolor{currentfill}{rgb}{1.000000,0.000000,0.000000}%
\pgfsetfillcolor{currentfill}%
\pgfsetlinewidth{2.007500pt}%
\definecolor{currentstroke}{rgb}{1.000000,0.000000,0.000000}%
\pgfsetstrokecolor{currentstroke}%
\pgfsetdash{}{0pt}%
\pgfpathmoveto{\pgfqpoint{8.005296in}{3.812760in}}%
\pgfpathlineto{\pgfqpoint{8.088630in}{3.812760in}}%
\pgfpathmoveto{\pgfqpoint{8.046963in}{3.771093in}}%
\pgfpathlineto{\pgfqpoint{8.046963in}{3.854426in}}%
\pgfusepath{stroke,fill}%
\end{pgfscope}%
\begin{pgfscope}%
\pgfpathrectangle{\pgfqpoint{7.105882in}{3.134211in}}{\pgfqpoint{4.376471in}{0.953684in}} %
\pgfusepath{clip}%
\pgfsetbuttcap%
\pgfsetroundjoin%
\definecolor{currentfill}{rgb}{1.000000,0.000000,0.000000}%
\pgfsetfillcolor{currentfill}%
\pgfsetlinewidth{2.007500pt}%
\definecolor{currentstroke}{rgb}{1.000000,0.000000,0.000000}%
\pgfsetstrokecolor{currentstroke}%
\pgfsetdash{}{0pt}%
\pgfpathmoveto{\pgfqpoint{10.101961in}{3.597313in}}%
\pgfpathlineto{\pgfqpoint{10.185294in}{3.597313in}}%
\pgfpathmoveto{\pgfqpoint{10.143627in}{3.555646in}}%
\pgfpathlineto{\pgfqpoint{10.143627in}{3.638980in}}%
\pgfusepath{stroke,fill}%
\end{pgfscope}%
\begin{pgfscope}%
\pgfpathrectangle{\pgfqpoint{7.105882in}{3.134211in}}{\pgfqpoint{4.376471in}{0.953684in}} %
\pgfusepath{clip}%
\pgfsetbuttcap%
\pgfsetroundjoin%
\definecolor{currentfill}{rgb}{1.000000,0.000000,0.000000}%
\pgfsetfillcolor{currentfill}%
\pgfsetlinewidth{2.007500pt}%
\definecolor{currentstroke}{rgb}{1.000000,0.000000,0.000000}%
\pgfsetstrokecolor{currentstroke}%
\pgfsetdash{}{0pt}%
\pgfpathmoveto{\pgfqpoint{10.082565in}{3.636459in}}%
\pgfpathlineto{\pgfqpoint{10.165898in}{3.636459in}}%
\pgfpathmoveto{\pgfqpoint{10.124232in}{3.594793in}}%
\pgfpathlineto{\pgfqpoint{10.124232in}{3.678126in}}%
\pgfusepath{stroke,fill}%
\end{pgfscope}%
\begin{pgfscope}%
\pgfpathrectangle{\pgfqpoint{7.105882in}{3.134211in}}{\pgfqpoint{4.376471in}{0.953684in}} %
\pgfusepath{clip}%
\pgfsetbuttcap%
\pgfsetroundjoin%
\definecolor{currentfill}{rgb}{1.000000,0.000000,0.000000}%
\pgfsetfillcolor{currentfill}%
\pgfsetlinewidth{2.007500pt}%
\definecolor{currentstroke}{rgb}{1.000000,0.000000,0.000000}%
\pgfsetstrokecolor{currentstroke}%
\pgfsetdash{}{0pt}%
\pgfpathmoveto{\pgfqpoint{10.099505in}{3.662085in}}%
\pgfpathlineto{\pgfqpoint{10.182838in}{3.662085in}}%
\pgfpathmoveto{\pgfqpoint{10.141171in}{3.620419in}}%
\pgfpathlineto{\pgfqpoint{10.141171in}{3.703752in}}%
\pgfusepath{stroke,fill}%
\end{pgfscope}%
\begin{pgfscope}%
\pgfpathrectangle{\pgfqpoint{7.105882in}{3.134211in}}{\pgfqpoint{4.376471in}{0.953684in}} %
\pgfusepath{clip}%
\pgfsetbuttcap%
\pgfsetroundjoin%
\definecolor{currentfill}{rgb}{1.000000,0.000000,0.000000}%
\pgfsetfillcolor{currentfill}%
\pgfsetlinewidth{2.007500pt}%
\definecolor{currentstroke}{rgb}{1.000000,0.000000,0.000000}%
\pgfsetstrokecolor{currentstroke}%
\pgfsetdash{}{0pt}%
\pgfpathmoveto{\pgfqpoint{11.243738in}{3.792899in}}%
\pgfpathlineto{\pgfqpoint{11.327072in}{3.792899in}}%
\pgfpathmoveto{\pgfqpoint{11.285405in}{3.751233in}}%
\pgfpathlineto{\pgfqpoint{11.285405in}{3.834566in}}%
\pgfusepath{stroke,fill}%
\end{pgfscope}%
\begin{pgfscope}%
\pgfpathrectangle{\pgfqpoint{7.105882in}{3.134211in}}{\pgfqpoint{4.376471in}{0.953684in}} %
\pgfusepath{clip}%
\pgfsetbuttcap%
\pgfsetroundjoin%
\definecolor{currentfill}{rgb}{1.000000,0.000000,0.000000}%
\pgfsetfillcolor{currentfill}%
\pgfsetlinewidth{2.007500pt}%
\definecolor{currentstroke}{rgb}{1.000000,0.000000,0.000000}%
\pgfsetstrokecolor{currentstroke}%
\pgfsetdash{}{0pt}%
\pgfpathmoveto{\pgfqpoint{10.326683in}{3.539529in}}%
\pgfpathlineto{\pgfqpoint{10.410016in}{3.539529in}}%
\pgfpathmoveto{\pgfqpoint{10.368350in}{3.497863in}}%
\pgfpathlineto{\pgfqpoint{10.368350in}{3.581196in}}%
\pgfusepath{stroke,fill}%
\end{pgfscope}%
\begin{pgfscope}%
\pgfpathrectangle{\pgfqpoint{7.105882in}{3.134211in}}{\pgfqpoint{4.376471in}{0.953684in}} %
\pgfusepath{clip}%
\pgfsetbuttcap%
\pgfsetroundjoin%
\definecolor{currentfill}{rgb}{1.000000,0.000000,0.000000}%
\pgfsetfillcolor{currentfill}%
\pgfsetlinewidth{2.007500pt}%
\definecolor{currentstroke}{rgb}{1.000000,0.000000,0.000000}%
\pgfsetstrokecolor{currentstroke}%
\pgfsetdash{}{0pt}%
\pgfpathmoveto{\pgfqpoint{9.198210in}{3.504764in}}%
\pgfpathlineto{\pgfqpoint{9.281544in}{3.504764in}}%
\pgfpathmoveto{\pgfqpoint{9.239877in}{3.463097in}}%
\pgfpathlineto{\pgfqpoint{9.239877in}{3.546431in}}%
\pgfusepath{stroke,fill}%
\end{pgfscope}%
\begin{pgfscope}%
\pgfpathrectangle{\pgfqpoint{7.105882in}{3.134211in}}{\pgfqpoint{4.376471in}{0.953684in}} %
\pgfusepath{clip}%
\pgfsetbuttcap%
\pgfsetroundjoin%
\definecolor{currentfill}{rgb}{1.000000,0.000000,0.000000}%
\pgfsetfillcolor{currentfill}%
\pgfsetlinewidth{2.007500pt}%
\definecolor{currentstroke}{rgb}{1.000000,0.000000,0.000000}%
\pgfsetstrokecolor{currentstroke}%
\pgfsetdash{}{0pt}%
\pgfpathmoveto{\pgfqpoint{9.469636in}{3.156404in}}%
\pgfpathlineto{\pgfqpoint{9.552969in}{3.156404in}}%
\pgfpathmoveto{\pgfqpoint{9.511302in}{3.114738in}}%
\pgfpathlineto{\pgfqpoint{9.511302in}{3.198071in}}%
\pgfusepath{stroke,fill}%
\end{pgfscope}%
\begin{pgfscope}%
\pgfpathrectangle{\pgfqpoint{7.105882in}{3.134211in}}{\pgfqpoint{4.376471in}{0.953684in}} %
\pgfusepath{clip}%
\pgfsetbuttcap%
\pgfsetroundjoin%
\definecolor{currentfill}{rgb}{1.000000,0.000000,0.000000}%
\pgfsetfillcolor{currentfill}%
\pgfsetlinewidth{2.007500pt}%
\definecolor{currentstroke}{rgb}{1.000000,0.000000,0.000000}%
\pgfsetstrokecolor{currentstroke}%
\pgfsetdash{}{0pt}%
\pgfpathmoveto{\pgfqpoint{10.382040in}{3.557724in}}%
\pgfpathlineto{\pgfqpoint{10.465373in}{3.557724in}}%
\pgfpathmoveto{\pgfqpoint{10.423706in}{3.516058in}}%
\pgfpathlineto{\pgfqpoint{10.423706in}{3.599391in}}%
\pgfusepath{stroke,fill}%
\end{pgfscope}%
\begin{pgfscope}%
\pgfpathrectangle{\pgfqpoint{7.105882in}{3.134211in}}{\pgfqpoint{4.376471in}{0.953684in}} %
\pgfusepath{clip}%
\pgfsetbuttcap%
\pgfsetroundjoin%
\definecolor{currentfill}{rgb}{1.000000,0.000000,0.000000}%
\pgfsetfillcolor{currentfill}%
\pgfsetlinewidth{2.007500pt}%
\definecolor{currentstroke}{rgb}{1.000000,0.000000,0.000000}%
\pgfsetstrokecolor{currentstroke}%
\pgfsetdash{}{0pt}%
\pgfpathmoveto{\pgfqpoint{8.150370in}{3.870651in}}%
\pgfpathlineto{\pgfqpoint{8.233703in}{3.870651in}}%
\pgfpathmoveto{\pgfqpoint{8.192036in}{3.828985in}}%
\pgfpathlineto{\pgfqpoint{8.192036in}{3.912318in}}%
\pgfusepath{stroke,fill}%
\end{pgfscope}%
\begin{pgfscope}%
\pgfpathrectangle{\pgfqpoint{7.105882in}{3.134211in}}{\pgfqpoint{4.376471in}{0.953684in}} %
\pgfusepath{clip}%
\pgfsetbuttcap%
\pgfsetroundjoin%
\definecolor{currentfill}{rgb}{1.000000,0.000000,0.000000}%
\pgfsetfillcolor{currentfill}%
\pgfsetlinewidth{2.007500pt}%
\definecolor{currentstroke}{rgb}{1.000000,0.000000,0.000000}%
\pgfsetstrokecolor{currentstroke}%
\pgfsetdash{}{0pt}%
\pgfpathmoveto{\pgfqpoint{10.273978in}{3.614897in}}%
\pgfpathlineto{\pgfqpoint{10.357311in}{3.614897in}}%
\pgfpathmoveto{\pgfqpoint{10.315644in}{3.573230in}}%
\pgfpathlineto{\pgfqpoint{10.315644in}{3.656563in}}%
\pgfusepath{stroke,fill}%
\end{pgfscope}%
\begin{pgfscope}%
\pgfpathrectangle{\pgfqpoint{7.105882in}{3.134211in}}{\pgfqpoint{4.376471in}{0.953684in}} %
\pgfusepath{clip}%
\pgfsetbuttcap%
\pgfsetroundjoin%
\definecolor{currentfill}{rgb}{1.000000,0.000000,0.000000}%
\pgfsetfillcolor{currentfill}%
\pgfsetlinewidth{2.007500pt}%
\definecolor{currentstroke}{rgb}{1.000000,0.000000,0.000000}%
\pgfsetstrokecolor{currentstroke}%
\pgfsetdash{}{0pt}%
\pgfpathmoveto{\pgfqpoint{10.287531in}{3.467031in}}%
\pgfpathlineto{\pgfqpoint{10.370865in}{3.467031in}}%
\pgfpathmoveto{\pgfqpoint{10.329198in}{3.425364in}}%
\pgfpathlineto{\pgfqpoint{10.329198in}{3.508697in}}%
\pgfusepath{stroke,fill}%
\end{pgfscope}%
\begin{pgfscope}%
\pgfpathrectangle{\pgfqpoint{7.105882in}{3.134211in}}{\pgfqpoint{4.376471in}{0.953684in}} %
\pgfusepath{clip}%
\pgfsetbuttcap%
\pgfsetroundjoin%
\definecolor{currentfill}{rgb}{1.000000,0.000000,0.000000}%
\pgfsetfillcolor{currentfill}%
\pgfsetlinewidth{2.007500pt}%
\definecolor{currentstroke}{rgb}{1.000000,0.000000,0.000000}%
\pgfsetstrokecolor{currentstroke}%
\pgfsetdash{}{0pt}%
\pgfpathmoveto{\pgfqpoint{8.676096in}{3.612446in}}%
\pgfpathlineto{\pgfqpoint{8.759430in}{3.612446in}}%
\pgfpathmoveto{\pgfqpoint{8.717763in}{3.570779in}}%
\pgfpathlineto{\pgfqpoint{8.717763in}{3.654113in}}%
\pgfusepath{stroke,fill}%
\end{pgfscope}%
\begin{pgfscope}%
\pgfpathrectangle{\pgfqpoint{7.105882in}{3.134211in}}{\pgfqpoint{4.376471in}{0.953684in}} %
\pgfusepath{clip}%
\pgfsetbuttcap%
\pgfsetroundjoin%
\definecolor{currentfill}{rgb}{1.000000,0.000000,0.000000}%
\pgfsetfillcolor{currentfill}%
\pgfsetlinewidth{2.007500pt}%
\definecolor{currentstroke}{rgb}{1.000000,0.000000,0.000000}%
\pgfsetstrokecolor{currentstroke}%
\pgfsetdash{}{0pt}%
\pgfpathmoveto{\pgfqpoint{8.390904in}{3.635068in}}%
\pgfpathlineto{\pgfqpoint{8.474237in}{3.635068in}}%
\pgfpathmoveto{\pgfqpoint{8.432570in}{3.593402in}}%
\pgfpathlineto{\pgfqpoint{8.432570in}{3.676735in}}%
\pgfusepath{stroke,fill}%
\end{pgfscope}%
\begin{pgfscope}%
\pgfpathrectangle{\pgfqpoint{7.105882in}{3.134211in}}{\pgfqpoint{4.376471in}{0.953684in}} %
\pgfusepath{clip}%
\pgfsetbuttcap%
\pgfsetroundjoin%
\definecolor{currentfill}{rgb}{1.000000,0.000000,0.000000}%
\pgfsetfillcolor{currentfill}%
\pgfsetlinewidth{2.007500pt}%
\definecolor{currentstroke}{rgb}{1.000000,0.000000,0.000000}%
\pgfsetstrokecolor{currentstroke}%
\pgfsetdash{}{0pt}%
\pgfpathmoveto{\pgfqpoint{9.043880in}{3.609760in}}%
\pgfpathlineto{\pgfqpoint{9.127213in}{3.609760in}}%
\pgfpathmoveto{\pgfqpoint{9.085547in}{3.568094in}}%
\pgfpathlineto{\pgfqpoint{9.085547in}{3.651427in}}%
\pgfusepath{stroke,fill}%
\end{pgfscope}%
\begin{pgfscope}%
\pgfpathrectangle{\pgfqpoint{7.105882in}{3.134211in}}{\pgfqpoint{4.376471in}{0.953684in}} %
\pgfusepath{clip}%
\pgfsetbuttcap%
\pgfsetroundjoin%
\definecolor{currentfill}{rgb}{1.000000,0.000000,0.000000}%
\pgfsetfillcolor{currentfill}%
\pgfsetlinewidth{2.007500pt}%
\definecolor{currentstroke}{rgb}{1.000000,0.000000,0.000000}%
\pgfsetstrokecolor{currentstroke}%
\pgfsetdash{}{0pt}%
\pgfpathmoveto{\pgfqpoint{9.212925in}{3.562552in}}%
\pgfpathlineto{\pgfqpoint{9.296259in}{3.562552in}}%
\pgfpathmoveto{\pgfqpoint{9.254592in}{3.520885in}}%
\pgfpathlineto{\pgfqpoint{9.254592in}{3.604219in}}%
\pgfusepath{stroke,fill}%
\end{pgfscope}%
\begin{pgfscope}%
\pgfpathrectangle{\pgfqpoint{7.105882in}{3.134211in}}{\pgfqpoint{4.376471in}{0.953684in}} %
\pgfusepath{clip}%
\pgfsetbuttcap%
\pgfsetroundjoin%
\definecolor{currentfill}{rgb}{0.000000,0.000000,0.000000}%
\pgfsetfillcolor{currentfill}%
\pgfsetlinewidth{1.003750pt}%
\definecolor{currentstroke}{rgb}{0.000000,0.000000,0.000000}%
\pgfsetstrokecolor{currentstroke}%
\pgfsetdash{}{0pt}%
\pgfsys@defobject{currentmarker}{\pgfqpoint{-0.020833in}{-0.020833in}}{\pgfqpoint{0.020833in}{0.020833in}}{%
\pgfpathmoveto{\pgfqpoint{0.000000in}{-0.020833in}}%
\pgfpathcurveto{\pgfqpoint{0.005525in}{-0.020833in}}{\pgfqpoint{0.010825in}{-0.018638in}}{\pgfqpoint{0.014731in}{-0.014731in}}%
\pgfpathcurveto{\pgfqpoint{0.018638in}{-0.010825in}}{\pgfqpoint{0.020833in}{-0.005525in}}{\pgfqpoint{0.020833in}{0.000000in}}%
\pgfpathcurveto{\pgfqpoint{0.020833in}{0.005525in}}{\pgfqpoint{0.018638in}{0.010825in}}{\pgfqpoint{0.014731in}{0.014731in}}%
\pgfpathcurveto{\pgfqpoint{0.010825in}{0.018638in}}{\pgfqpoint{0.005525in}{0.020833in}}{\pgfqpoint{0.000000in}{0.020833in}}%
\pgfpathcurveto{\pgfqpoint{-0.005525in}{0.020833in}}{\pgfqpoint{-0.010825in}{0.018638in}}{\pgfqpoint{-0.014731in}{0.014731in}}%
\pgfpathcurveto{\pgfqpoint{-0.018638in}{0.010825in}}{\pgfqpoint{-0.020833in}{0.005525in}}{\pgfqpoint{-0.020833in}{0.000000in}}%
\pgfpathcurveto{\pgfqpoint{-0.020833in}{-0.005525in}}{\pgfqpoint{-0.018638in}{-0.010825in}}{\pgfqpoint{-0.014731in}{-0.014731in}}%
\pgfpathcurveto{\pgfqpoint{-0.010825in}{-0.018638in}}{\pgfqpoint{-0.005525in}{-0.020833in}}{\pgfqpoint{0.000000in}{-0.020833in}}%
\pgfpathclose%
\pgfusepath{stroke,fill}%
}%
\begin{pgfscope}%
\pgfsys@transformshift{7.981176in}{4.017989in}%
\pgfsys@useobject{currentmarker}{}%
\end{pgfscope}%
\begin{pgfscope}%
\pgfsys@transformshift{7.998770in}{4.069101in}%
\pgfsys@useobject{currentmarker}{}%
\end{pgfscope}%
\begin{pgfscope}%
\pgfsys@transformshift{8.016364in}{3.981426in}%
\pgfsys@useobject{currentmarker}{}%
\end{pgfscope}%
\begin{pgfscope}%
\pgfsys@transformshift{8.033958in}{3.989509in}%
\pgfsys@useobject{currentmarker}{}%
\end{pgfscope}%
\begin{pgfscope}%
\pgfsys@transformshift{8.051552in}{3.893189in}%
\pgfsys@useobject{currentmarker}{}%
\end{pgfscope}%
\begin{pgfscope}%
\pgfsys@transformshift{8.069146in}{4.042123in}%
\pgfsys@useobject{currentmarker}{}%
\end{pgfscope}%
\begin{pgfscope}%
\pgfsys@transformshift{8.086740in}{3.849818in}%
\pgfsys@useobject{currentmarker}{}%
\end{pgfscope}%
\begin{pgfscope}%
\pgfsys@transformshift{8.104333in}{3.849780in}%
\pgfsys@useobject{currentmarker}{}%
\end{pgfscope}%
\begin{pgfscope}%
\pgfsys@transformshift{8.121927in}{3.969894in}%
\pgfsys@useobject{currentmarker}{}%
\end{pgfscope}%
\begin{pgfscope}%
\pgfsys@transformshift{8.139521in}{3.622436in}%
\pgfsys@useobject{currentmarker}{}%
\end{pgfscope}%
\begin{pgfscope}%
\pgfsys@transformshift{8.157115in}{3.604348in}%
\pgfsys@useobject{currentmarker}{}%
\end{pgfscope}%
\begin{pgfscope}%
\pgfsys@transformshift{8.174709in}{3.802015in}%
\pgfsys@useobject{currentmarker}{}%
\end{pgfscope}%
\begin{pgfscope}%
\pgfsys@transformshift{8.192303in}{3.565419in}%
\pgfsys@useobject{currentmarker}{}%
\end{pgfscope}%
\begin{pgfscope}%
\pgfsys@transformshift{8.209897in}{3.852552in}%
\pgfsys@useobject{currentmarker}{}%
\end{pgfscope}%
\begin{pgfscope}%
\pgfsys@transformshift{8.227490in}{3.597307in}%
\pgfsys@useobject{currentmarker}{}%
\end{pgfscope}%
\begin{pgfscope}%
\pgfsys@transformshift{8.245084in}{3.544634in}%
\pgfsys@useobject{currentmarker}{}%
\end{pgfscope}%
\begin{pgfscope}%
\pgfsys@transformshift{8.262678in}{3.792099in}%
\pgfsys@useobject{currentmarker}{}%
\end{pgfscope}%
\begin{pgfscope}%
\pgfsys@transformshift{8.280272in}{3.732131in}%
\pgfsys@useobject{currentmarker}{}%
\end{pgfscope}%
\begin{pgfscope}%
\pgfsys@transformshift{8.297866in}{3.756461in}%
\pgfsys@useobject{currentmarker}{}%
\end{pgfscope}%
\begin{pgfscope}%
\pgfsys@transformshift{8.315460in}{3.648795in}%
\pgfsys@useobject{currentmarker}{}%
\end{pgfscope}%
\begin{pgfscope}%
\pgfsys@transformshift{8.333054in}{3.463113in}%
\pgfsys@useobject{currentmarker}{}%
\end{pgfscope}%
\begin{pgfscope}%
\pgfsys@transformshift{8.350647in}{3.730368in}%
\pgfsys@useobject{currentmarker}{}%
\end{pgfscope}%
\begin{pgfscope}%
\pgfsys@transformshift{8.368241in}{3.508028in}%
\pgfsys@useobject{currentmarker}{}%
\end{pgfscope}%
\begin{pgfscope}%
\pgfsys@transformshift{8.385835in}{3.610485in}%
\pgfsys@useobject{currentmarker}{}%
\end{pgfscope}%
\begin{pgfscope}%
\pgfsys@transformshift{8.403429in}{3.623042in}%
\pgfsys@useobject{currentmarker}{}%
\end{pgfscope}%
\begin{pgfscope}%
\pgfsys@transformshift{8.421023in}{3.513729in}%
\pgfsys@useobject{currentmarker}{}%
\end{pgfscope}%
\begin{pgfscope}%
\pgfsys@transformshift{8.438617in}{3.592272in}%
\pgfsys@useobject{currentmarker}{}%
\end{pgfscope}%
\begin{pgfscope}%
\pgfsys@transformshift{8.456210in}{3.626893in}%
\pgfsys@useobject{currentmarker}{}%
\end{pgfscope}%
\begin{pgfscope}%
\pgfsys@transformshift{8.473804in}{3.578455in}%
\pgfsys@useobject{currentmarker}{}%
\end{pgfscope}%
\begin{pgfscope}%
\pgfsys@transformshift{8.491398in}{3.439280in}%
\pgfsys@useobject{currentmarker}{}%
\end{pgfscope}%
\begin{pgfscope}%
\pgfsys@transformshift{8.508992in}{3.587082in}%
\pgfsys@useobject{currentmarker}{}%
\end{pgfscope}%
\begin{pgfscope}%
\pgfsys@transformshift{8.526586in}{3.699554in}%
\pgfsys@useobject{currentmarker}{}%
\end{pgfscope}%
\begin{pgfscope}%
\pgfsys@transformshift{8.544180in}{3.510335in}%
\pgfsys@useobject{currentmarker}{}%
\end{pgfscope}%
\begin{pgfscope}%
\pgfsys@transformshift{8.561774in}{3.577060in}%
\pgfsys@useobject{currentmarker}{}%
\end{pgfscope}%
\begin{pgfscope}%
\pgfsys@transformshift{8.579367in}{3.562148in}%
\pgfsys@useobject{currentmarker}{}%
\end{pgfscope}%
\begin{pgfscope}%
\pgfsys@transformshift{8.596961in}{3.803211in}%
\pgfsys@useobject{currentmarker}{}%
\end{pgfscope}%
\begin{pgfscope}%
\pgfsys@transformshift{8.614555in}{3.700889in}%
\pgfsys@useobject{currentmarker}{}%
\end{pgfscope}%
\begin{pgfscope}%
\pgfsys@transformshift{8.632149in}{3.689460in}%
\pgfsys@useobject{currentmarker}{}%
\end{pgfscope}%
\begin{pgfscope}%
\pgfsys@transformshift{8.649743in}{3.587452in}%
\pgfsys@useobject{currentmarker}{}%
\end{pgfscope}%
\begin{pgfscope}%
\pgfsys@transformshift{8.667337in}{3.732339in}%
\pgfsys@useobject{currentmarker}{}%
\end{pgfscope}%
\begin{pgfscope}%
\pgfsys@transformshift{8.684931in}{3.626287in}%
\pgfsys@useobject{currentmarker}{}%
\end{pgfscope}%
\begin{pgfscope}%
\pgfsys@transformshift{8.702524in}{3.710428in}%
\pgfsys@useobject{currentmarker}{}%
\end{pgfscope}%
\begin{pgfscope}%
\pgfsys@transformshift{8.720118in}{3.657381in}%
\pgfsys@useobject{currentmarker}{}%
\end{pgfscope}%
\begin{pgfscope}%
\pgfsys@transformshift{8.737712in}{3.800139in}%
\pgfsys@useobject{currentmarker}{}%
\end{pgfscope}%
\begin{pgfscope}%
\pgfsys@transformshift{8.755306in}{3.801553in}%
\pgfsys@useobject{currentmarker}{}%
\end{pgfscope}%
\begin{pgfscope}%
\pgfsys@transformshift{8.772900in}{3.733694in}%
\pgfsys@useobject{currentmarker}{}%
\end{pgfscope}%
\begin{pgfscope}%
\pgfsys@transformshift{8.790494in}{3.802475in}%
\pgfsys@useobject{currentmarker}{}%
\end{pgfscope}%
\begin{pgfscope}%
\pgfsys@transformshift{8.808087in}{3.661797in}%
\pgfsys@useobject{currentmarker}{}%
\end{pgfscope}%
\begin{pgfscope}%
\pgfsys@transformshift{8.825681in}{3.627874in}%
\pgfsys@useobject{currentmarker}{}%
\end{pgfscope}%
\begin{pgfscope}%
\pgfsys@transformshift{8.843275in}{3.823590in}%
\pgfsys@useobject{currentmarker}{}%
\end{pgfscope}%
\begin{pgfscope}%
\pgfsys@transformshift{8.860869in}{3.798625in}%
\pgfsys@useobject{currentmarker}{}%
\end{pgfscope}%
\begin{pgfscope}%
\pgfsys@transformshift{8.878463in}{3.845424in}%
\pgfsys@useobject{currentmarker}{}%
\end{pgfscope}%
\begin{pgfscope}%
\pgfsys@transformshift{8.896057in}{4.017457in}%
\pgfsys@useobject{currentmarker}{}%
\end{pgfscope}%
\begin{pgfscope}%
\pgfsys@transformshift{8.913651in}{3.870988in}%
\pgfsys@useobject{currentmarker}{}%
\end{pgfscope}%
\begin{pgfscope}%
\pgfsys@transformshift{8.931244in}{3.681008in}%
\pgfsys@useobject{currentmarker}{}%
\end{pgfscope}%
\begin{pgfscope}%
\pgfsys@transformshift{8.948838in}{3.875279in}%
\pgfsys@useobject{currentmarker}{}%
\end{pgfscope}%
\begin{pgfscope}%
\pgfsys@transformshift{8.966432in}{3.624342in}%
\pgfsys@useobject{currentmarker}{}%
\end{pgfscope}%
\begin{pgfscope}%
\pgfsys@transformshift{8.984026in}{3.698158in}%
\pgfsys@useobject{currentmarker}{}%
\end{pgfscope}%
\begin{pgfscope}%
\pgfsys@transformshift{9.001620in}{3.724445in}%
\pgfsys@useobject{currentmarker}{}%
\end{pgfscope}%
\begin{pgfscope}%
\pgfsys@transformshift{9.019214in}{3.887006in}%
\pgfsys@useobject{currentmarker}{}%
\end{pgfscope}%
\begin{pgfscope}%
\pgfsys@transformshift{9.036808in}{3.626862in}%
\pgfsys@useobject{currentmarker}{}%
\end{pgfscope}%
\begin{pgfscope}%
\pgfsys@transformshift{9.054401in}{3.601399in}%
\pgfsys@useobject{currentmarker}{}%
\end{pgfscope}%
\begin{pgfscope}%
\pgfsys@transformshift{9.071995in}{3.655117in}%
\pgfsys@useobject{currentmarker}{}%
\end{pgfscope}%
\begin{pgfscope}%
\pgfsys@transformshift{9.089589in}{3.579287in}%
\pgfsys@useobject{currentmarker}{}%
\end{pgfscope}%
\begin{pgfscope}%
\pgfsys@transformshift{9.107183in}{3.736495in}%
\pgfsys@useobject{currentmarker}{}%
\end{pgfscope}%
\begin{pgfscope}%
\pgfsys@transformshift{9.124777in}{3.495914in}%
\pgfsys@useobject{currentmarker}{}%
\end{pgfscope}%
\begin{pgfscope}%
\pgfsys@transformshift{9.142371in}{3.467269in}%
\pgfsys@useobject{currentmarker}{}%
\end{pgfscope}%
\begin{pgfscope}%
\pgfsys@transformshift{9.159965in}{3.515463in}%
\pgfsys@useobject{currentmarker}{}%
\end{pgfscope}%
\begin{pgfscope}%
\pgfsys@transformshift{9.177558in}{3.487008in}%
\pgfsys@useobject{currentmarker}{}%
\end{pgfscope}%
\begin{pgfscope}%
\pgfsys@transformshift{9.195152in}{3.705725in}%
\pgfsys@useobject{currentmarker}{}%
\end{pgfscope}%
\begin{pgfscope}%
\pgfsys@transformshift{9.212746in}{3.586150in}%
\pgfsys@useobject{currentmarker}{}%
\end{pgfscope}%
\begin{pgfscope}%
\pgfsys@transformshift{9.230340in}{3.478755in}%
\pgfsys@useobject{currentmarker}{}%
\end{pgfscope}%
\begin{pgfscope}%
\pgfsys@transformshift{9.247934in}{3.327173in}%
\pgfsys@useobject{currentmarker}{}%
\end{pgfscope}%
\begin{pgfscope}%
\pgfsys@transformshift{9.265528in}{3.512465in}%
\pgfsys@useobject{currentmarker}{}%
\end{pgfscope}%
\begin{pgfscope}%
\pgfsys@transformshift{9.283121in}{3.309897in}%
\pgfsys@useobject{currentmarker}{}%
\end{pgfscope}%
\begin{pgfscope}%
\pgfsys@transformshift{9.300715in}{3.237654in}%
\pgfsys@useobject{currentmarker}{}%
\end{pgfscope}%
\begin{pgfscope}%
\pgfsys@transformshift{9.318309in}{3.492329in}%
\pgfsys@useobject{currentmarker}{}%
\end{pgfscope}%
\begin{pgfscope}%
\pgfsys@transformshift{9.335903in}{3.390477in}%
\pgfsys@useobject{currentmarker}{}%
\end{pgfscope}%
\begin{pgfscope}%
\pgfsys@transformshift{9.353497in}{3.436782in}%
\pgfsys@useobject{currentmarker}{}%
\end{pgfscope}%
\begin{pgfscope}%
\pgfsys@transformshift{9.371091in}{3.365021in}%
\pgfsys@useobject{currentmarker}{}%
\end{pgfscope}%
\begin{pgfscope}%
\pgfsys@transformshift{9.388685in}{3.408384in}%
\pgfsys@useobject{currentmarker}{}%
\end{pgfscope}%
\begin{pgfscope}%
\pgfsys@transformshift{9.406278in}{3.250432in}%
\pgfsys@useobject{currentmarker}{}%
\end{pgfscope}%
\begin{pgfscope}%
\pgfsys@transformshift{9.423872in}{3.206207in}%
\pgfsys@useobject{currentmarker}{}%
\end{pgfscope}%
\begin{pgfscope}%
\pgfsys@transformshift{9.441466in}{3.372556in}%
\pgfsys@useobject{currentmarker}{}%
\end{pgfscope}%
\begin{pgfscope}%
\pgfsys@transformshift{9.459060in}{3.223005in}%
\pgfsys@useobject{currentmarker}{}%
\end{pgfscope}%
\begin{pgfscope}%
\pgfsys@transformshift{9.476654in}{3.234374in}%
\pgfsys@useobject{currentmarker}{}%
\end{pgfscope}%
\begin{pgfscope}%
\pgfsys@transformshift{9.494248in}{3.259752in}%
\pgfsys@useobject{currentmarker}{}%
\end{pgfscope}%
\begin{pgfscope}%
\pgfsys@transformshift{9.511842in}{3.310912in}%
\pgfsys@useobject{currentmarker}{}%
\end{pgfscope}%
\begin{pgfscope}%
\pgfsys@transformshift{9.529435in}{3.280166in}%
\pgfsys@useobject{currentmarker}{}%
\end{pgfscope}%
\begin{pgfscope}%
\pgfsys@transformshift{9.547029in}{3.186814in}%
\pgfsys@useobject{currentmarker}{}%
\end{pgfscope}%
\begin{pgfscope}%
\pgfsys@transformshift{9.564623in}{3.269363in}%
\pgfsys@useobject{currentmarker}{}%
\end{pgfscope}%
\begin{pgfscope}%
\pgfsys@transformshift{9.582217in}{3.124038in}%
\pgfsys@useobject{currentmarker}{}%
\end{pgfscope}%
\begin{pgfscope}%
\pgfsys@transformshift{9.599811in}{3.420234in}%
\pgfsys@useobject{currentmarker}{}%
\end{pgfscope}%
\begin{pgfscope}%
\pgfsys@transformshift{9.617405in}{3.213629in}%
\pgfsys@useobject{currentmarker}{}%
\end{pgfscope}%
\begin{pgfscope}%
\pgfsys@transformshift{9.634999in}{3.279027in}%
\pgfsys@useobject{currentmarker}{}%
\end{pgfscope}%
\begin{pgfscope}%
\pgfsys@transformshift{9.652592in}{3.411012in}%
\pgfsys@useobject{currentmarker}{}%
\end{pgfscope}%
\begin{pgfscope}%
\pgfsys@transformshift{9.670186in}{3.350408in}%
\pgfsys@useobject{currentmarker}{}%
\end{pgfscope}%
\begin{pgfscope}%
\pgfsys@transformshift{9.687780in}{3.595982in}%
\pgfsys@useobject{currentmarker}{}%
\end{pgfscope}%
\begin{pgfscope}%
\pgfsys@transformshift{9.705374in}{3.333658in}%
\pgfsys@useobject{currentmarker}{}%
\end{pgfscope}%
\begin{pgfscope}%
\pgfsys@transformshift{9.722968in}{3.508402in}%
\pgfsys@useobject{currentmarker}{}%
\end{pgfscope}%
\begin{pgfscope}%
\pgfsys@transformshift{9.740562in}{3.497930in}%
\pgfsys@useobject{currentmarker}{}%
\end{pgfscope}%
\begin{pgfscope}%
\pgfsys@transformshift{9.758155in}{3.405731in}%
\pgfsys@useobject{currentmarker}{}%
\end{pgfscope}%
\begin{pgfscope}%
\pgfsys@transformshift{9.775749in}{3.593473in}%
\pgfsys@useobject{currentmarker}{}%
\end{pgfscope}%
\begin{pgfscope}%
\pgfsys@transformshift{9.793343in}{3.543749in}%
\pgfsys@useobject{currentmarker}{}%
\end{pgfscope}%
\begin{pgfscope}%
\pgfsys@transformshift{9.810937in}{3.656151in}%
\pgfsys@useobject{currentmarker}{}%
\end{pgfscope}%
\begin{pgfscope}%
\pgfsys@transformshift{9.828531in}{3.679182in}%
\pgfsys@useobject{currentmarker}{}%
\end{pgfscope}%
\begin{pgfscope}%
\pgfsys@transformshift{9.846125in}{3.829195in}%
\pgfsys@useobject{currentmarker}{}%
\end{pgfscope}%
\begin{pgfscope}%
\pgfsys@transformshift{9.863719in}{3.762851in}%
\pgfsys@useobject{currentmarker}{}%
\end{pgfscope}%
\begin{pgfscope}%
\pgfsys@transformshift{9.881312in}{3.607893in}%
\pgfsys@useobject{currentmarker}{}%
\end{pgfscope}%
\begin{pgfscope}%
\pgfsys@transformshift{9.898906in}{3.633847in}%
\pgfsys@useobject{currentmarker}{}%
\end{pgfscope}%
\begin{pgfscope}%
\pgfsys@transformshift{9.916500in}{3.778372in}%
\pgfsys@useobject{currentmarker}{}%
\end{pgfscope}%
\begin{pgfscope}%
\pgfsys@transformshift{9.934094in}{3.744063in}%
\pgfsys@useobject{currentmarker}{}%
\end{pgfscope}%
\begin{pgfscope}%
\pgfsys@transformshift{9.951688in}{3.750644in}%
\pgfsys@useobject{currentmarker}{}%
\end{pgfscope}%
\begin{pgfscope}%
\pgfsys@transformshift{9.969282in}{3.532677in}%
\pgfsys@useobject{currentmarker}{}%
\end{pgfscope}%
\begin{pgfscope}%
\pgfsys@transformshift{9.986876in}{3.695270in}%
\pgfsys@useobject{currentmarker}{}%
\end{pgfscope}%
\begin{pgfscope}%
\pgfsys@transformshift{10.004469in}{3.626835in}%
\pgfsys@useobject{currentmarker}{}%
\end{pgfscope}%
\begin{pgfscope}%
\pgfsys@transformshift{10.022063in}{3.728502in}%
\pgfsys@useobject{currentmarker}{}%
\end{pgfscope}%
\begin{pgfscope}%
\pgfsys@transformshift{10.039657in}{3.689567in}%
\pgfsys@useobject{currentmarker}{}%
\end{pgfscope}%
\begin{pgfscope}%
\pgfsys@transformshift{10.057251in}{3.786471in}%
\pgfsys@useobject{currentmarker}{}%
\end{pgfscope}%
\begin{pgfscope}%
\pgfsys@transformshift{10.074845in}{3.722457in}%
\pgfsys@useobject{currentmarker}{}%
\end{pgfscope}%
\begin{pgfscope}%
\pgfsys@transformshift{10.092439in}{3.762177in}%
\pgfsys@useobject{currentmarker}{}%
\end{pgfscope}%
\begin{pgfscope}%
\pgfsys@transformshift{10.110033in}{3.629154in}%
\pgfsys@useobject{currentmarker}{}%
\end{pgfscope}%
\begin{pgfscope}%
\pgfsys@transformshift{10.127626in}{3.571508in}%
\pgfsys@useobject{currentmarker}{}%
\end{pgfscope}%
\begin{pgfscope}%
\pgfsys@transformshift{10.145220in}{3.612973in}%
\pgfsys@useobject{currentmarker}{}%
\end{pgfscope}%
\begin{pgfscope}%
\pgfsys@transformshift{10.162814in}{3.639064in}%
\pgfsys@useobject{currentmarker}{}%
\end{pgfscope}%
\begin{pgfscope}%
\pgfsys@transformshift{10.180408in}{3.664288in}%
\pgfsys@useobject{currentmarker}{}%
\end{pgfscope}%
\begin{pgfscope}%
\pgfsys@transformshift{10.198002in}{3.835917in}%
\pgfsys@useobject{currentmarker}{}%
\end{pgfscope}%
\begin{pgfscope}%
\pgfsys@transformshift{10.215596in}{3.591283in}%
\pgfsys@useobject{currentmarker}{}%
\end{pgfscope}%
\begin{pgfscope}%
\pgfsys@transformshift{10.233189in}{3.483864in}%
\pgfsys@useobject{currentmarker}{}%
\end{pgfscope}%
\begin{pgfscope}%
\pgfsys@transformshift{10.250783in}{3.527339in}%
\pgfsys@useobject{currentmarker}{}%
\end{pgfscope}%
\begin{pgfscope}%
\pgfsys@transformshift{10.268377in}{3.498263in}%
\pgfsys@useobject{currentmarker}{}%
\end{pgfscope}%
\begin{pgfscope}%
\pgfsys@transformshift{10.285971in}{3.574685in}%
\pgfsys@useobject{currentmarker}{}%
\end{pgfscope}%
\begin{pgfscope}%
\pgfsys@transformshift{10.303565in}{3.356452in}%
\pgfsys@useobject{currentmarker}{}%
\end{pgfscope}%
\begin{pgfscope}%
\pgfsys@transformshift{10.321159in}{3.498796in}%
\pgfsys@useobject{currentmarker}{}%
\end{pgfscope}%
\begin{pgfscope}%
\pgfsys@transformshift{10.338753in}{3.491581in}%
\pgfsys@useobject{currentmarker}{}%
\end{pgfscope}%
\begin{pgfscope}%
\pgfsys@transformshift{10.356346in}{3.483343in}%
\pgfsys@useobject{currentmarker}{}%
\end{pgfscope}%
\begin{pgfscope}%
\pgfsys@transformshift{10.373940in}{3.386093in}%
\pgfsys@useobject{currentmarker}{}%
\end{pgfscope}%
\begin{pgfscope}%
\pgfsys@transformshift{10.391534in}{3.408037in}%
\pgfsys@useobject{currentmarker}{}%
\end{pgfscope}%
\begin{pgfscope}%
\pgfsys@transformshift{10.409128in}{3.277703in}%
\pgfsys@useobject{currentmarker}{}%
\end{pgfscope}%
\begin{pgfscope}%
\pgfsys@transformshift{10.426722in}{3.359077in}%
\pgfsys@useobject{currentmarker}{}%
\end{pgfscope}%
\begin{pgfscope}%
\pgfsys@transformshift{10.444316in}{3.344626in}%
\pgfsys@useobject{currentmarker}{}%
\end{pgfscope}%
\begin{pgfscope}%
\pgfsys@transformshift{10.461910in}{3.432044in}%
\pgfsys@useobject{currentmarker}{}%
\end{pgfscope}%
\begin{pgfscope}%
\pgfsys@transformshift{10.479503in}{3.269810in}%
\pgfsys@useobject{currentmarker}{}%
\end{pgfscope}%
\begin{pgfscope}%
\pgfsys@transformshift{10.497097in}{3.458122in}%
\pgfsys@useobject{currentmarker}{}%
\end{pgfscope}%
\begin{pgfscope}%
\pgfsys@transformshift{10.514691in}{3.526801in}%
\pgfsys@useobject{currentmarker}{}%
\end{pgfscope}%
\begin{pgfscope}%
\pgfsys@transformshift{10.532285in}{3.172546in}%
\pgfsys@useobject{currentmarker}{}%
\end{pgfscope}%
\begin{pgfscope}%
\pgfsys@transformshift{10.549879in}{3.422441in}%
\pgfsys@useobject{currentmarker}{}%
\end{pgfscope}%
\begin{pgfscope}%
\pgfsys@transformshift{10.567473in}{3.451338in}%
\pgfsys@useobject{currentmarker}{}%
\end{pgfscope}%
\begin{pgfscope}%
\pgfsys@transformshift{10.585067in}{3.326847in}%
\pgfsys@useobject{currentmarker}{}%
\end{pgfscope}%
\begin{pgfscope}%
\pgfsys@transformshift{10.602660in}{3.358844in}%
\pgfsys@useobject{currentmarker}{}%
\end{pgfscope}%
\begin{pgfscope}%
\pgfsys@transformshift{10.620254in}{3.395263in}%
\pgfsys@useobject{currentmarker}{}%
\end{pgfscope}%
\begin{pgfscope}%
\pgfsys@transformshift{10.637848in}{3.390990in}%
\pgfsys@useobject{currentmarker}{}%
\end{pgfscope}%
\begin{pgfscope}%
\pgfsys@transformshift{10.655442in}{3.403859in}%
\pgfsys@useobject{currentmarker}{}%
\end{pgfscope}%
\begin{pgfscope}%
\pgfsys@transformshift{10.673036in}{3.283817in}%
\pgfsys@useobject{currentmarker}{}%
\end{pgfscope}%
\begin{pgfscope}%
\pgfsys@transformshift{10.690630in}{3.582207in}%
\pgfsys@useobject{currentmarker}{}%
\end{pgfscope}%
\begin{pgfscope}%
\pgfsys@transformshift{10.708223in}{3.593927in}%
\pgfsys@useobject{currentmarker}{}%
\end{pgfscope}%
\begin{pgfscope}%
\pgfsys@transformshift{10.725817in}{3.426199in}%
\pgfsys@useobject{currentmarker}{}%
\end{pgfscope}%
\begin{pgfscope}%
\pgfsys@transformshift{10.743411in}{3.382867in}%
\pgfsys@useobject{currentmarker}{}%
\end{pgfscope}%
\begin{pgfscope}%
\pgfsys@transformshift{10.761005in}{3.602901in}%
\pgfsys@useobject{currentmarker}{}%
\end{pgfscope}%
\begin{pgfscope}%
\pgfsys@transformshift{10.778599in}{3.517399in}%
\pgfsys@useobject{currentmarker}{}%
\end{pgfscope}%
\begin{pgfscope}%
\pgfsys@transformshift{10.796193in}{3.612842in}%
\pgfsys@useobject{currentmarker}{}%
\end{pgfscope}%
\begin{pgfscope}%
\pgfsys@transformshift{10.813787in}{3.591642in}%
\pgfsys@useobject{currentmarker}{}%
\end{pgfscope}%
\begin{pgfscope}%
\pgfsys@transformshift{10.831380in}{3.716974in}%
\pgfsys@useobject{currentmarker}{}%
\end{pgfscope}%
\begin{pgfscope}%
\pgfsys@transformshift{10.848974in}{3.742286in}%
\pgfsys@useobject{currentmarker}{}%
\end{pgfscope}%
\begin{pgfscope}%
\pgfsys@transformshift{10.866568in}{3.626062in}%
\pgfsys@useobject{currentmarker}{}%
\end{pgfscope}%
\begin{pgfscope}%
\pgfsys@transformshift{10.884162in}{3.585278in}%
\pgfsys@useobject{currentmarker}{}%
\end{pgfscope}%
\begin{pgfscope}%
\pgfsys@transformshift{10.901756in}{3.589500in}%
\pgfsys@useobject{currentmarker}{}%
\end{pgfscope}%
\begin{pgfscope}%
\pgfsys@transformshift{10.919350in}{3.830736in}%
\pgfsys@useobject{currentmarker}{}%
\end{pgfscope}%
\begin{pgfscope}%
\pgfsys@transformshift{10.936944in}{3.674460in}%
\pgfsys@useobject{currentmarker}{}%
\end{pgfscope}%
\begin{pgfscope}%
\pgfsys@transformshift{10.954537in}{3.763627in}%
\pgfsys@useobject{currentmarker}{}%
\end{pgfscope}%
\begin{pgfscope}%
\pgfsys@transformshift{10.972131in}{3.774932in}%
\pgfsys@useobject{currentmarker}{}%
\end{pgfscope}%
\begin{pgfscope}%
\pgfsys@transformshift{10.989725in}{3.847784in}%
\pgfsys@useobject{currentmarker}{}%
\end{pgfscope}%
\begin{pgfscope}%
\pgfsys@transformshift{11.007319in}{3.677994in}%
\pgfsys@useobject{currentmarker}{}%
\end{pgfscope}%
\begin{pgfscope}%
\pgfsys@transformshift{11.024913in}{3.904833in}%
\pgfsys@useobject{currentmarker}{}%
\end{pgfscope}%
\begin{pgfscope}%
\pgfsys@transformshift{11.042507in}{3.952144in}%
\pgfsys@useobject{currentmarker}{}%
\end{pgfscope}%
\begin{pgfscope}%
\pgfsys@transformshift{11.060101in}{3.921113in}%
\pgfsys@useobject{currentmarker}{}%
\end{pgfscope}%
\begin{pgfscope}%
\pgfsys@transformshift{11.077694in}{3.891819in}%
\pgfsys@useobject{currentmarker}{}%
\end{pgfscope}%
\begin{pgfscope}%
\pgfsys@transformshift{11.095288in}{3.940856in}%
\pgfsys@useobject{currentmarker}{}%
\end{pgfscope}%
\begin{pgfscope}%
\pgfsys@transformshift{11.112882in}{3.977379in}%
\pgfsys@useobject{currentmarker}{}%
\end{pgfscope}%
\begin{pgfscope}%
\pgfsys@transformshift{11.130476in}{3.666935in}%
\pgfsys@useobject{currentmarker}{}%
\end{pgfscope}%
\begin{pgfscope}%
\pgfsys@transformshift{11.148070in}{4.139401in}%
\pgfsys@useobject{currentmarker}{}%
\end{pgfscope}%
\begin{pgfscope}%
\pgfsys@transformshift{11.165664in}{3.984706in}%
\pgfsys@useobject{currentmarker}{}%
\end{pgfscope}%
\begin{pgfscope}%
\pgfsys@transformshift{11.183257in}{3.880187in}%
\pgfsys@useobject{currentmarker}{}%
\end{pgfscope}%
\begin{pgfscope}%
\pgfsys@transformshift{11.200851in}{3.903470in}%
\pgfsys@useobject{currentmarker}{}%
\end{pgfscope}%
\begin{pgfscope}%
\pgfsys@transformshift{11.218445in}{3.987025in}%
\pgfsys@useobject{currentmarker}{}%
\end{pgfscope}%
\begin{pgfscope}%
\pgfsys@transformshift{11.236039in}{3.920579in}%
\pgfsys@useobject{currentmarker}{}%
\end{pgfscope}%
\begin{pgfscope}%
\pgfsys@transformshift{11.253633in}{3.723117in}%
\pgfsys@useobject{currentmarker}{}%
\end{pgfscope}%
\begin{pgfscope}%
\pgfsys@transformshift{11.271227in}{4.121417in}%
\pgfsys@useobject{currentmarker}{}%
\end{pgfscope}%
\begin{pgfscope}%
\pgfsys@transformshift{11.288821in}{3.895719in}%
\pgfsys@useobject{currentmarker}{}%
\end{pgfscope}%
\begin{pgfscope}%
\pgfsys@transformshift{11.306414in}{3.997487in}%
\pgfsys@useobject{currentmarker}{}%
\end{pgfscope}%
\begin{pgfscope}%
\pgfsys@transformshift{11.324008in}{3.816077in}%
\pgfsys@useobject{currentmarker}{}%
\end{pgfscope}%
\begin{pgfscope}%
\pgfsys@transformshift{11.341602in}{4.025571in}%
\pgfsys@useobject{currentmarker}{}%
\end{pgfscope}%
\begin{pgfscope}%
\pgfsys@transformshift{11.359196in}{3.889133in}%
\pgfsys@useobject{currentmarker}{}%
\end{pgfscope}%
\begin{pgfscope}%
\pgfsys@transformshift{11.376790in}{3.908737in}%
\pgfsys@useobject{currentmarker}{}%
\end{pgfscope}%
\begin{pgfscope}%
\pgfsys@transformshift{11.394384in}{3.732006in}%
\pgfsys@useobject{currentmarker}{}%
\end{pgfscope}%
\begin{pgfscope}%
\pgfsys@transformshift{11.411978in}{3.943979in}%
\pgfsys@useobject{currentmarker}{}%
\end{pgfscope}%
\begin{pgfscope}%
\pgfsys@transformshift{11.429571in}{3.880432in}%
\pgfsys@useobject{currentmarker}{}%
\end{pgfscope}%
\begin{pgfscope}%
\pgfsys@transformshift{11.447165in}{3.930097in}%
\pgfsys@useobject{currentmarker}{}%
\end{pgfscope}%
\begin{pgfscope}%
\pgfsys@transformshift{11.464759in}{3.728051in}%
\pgfsys@useobject{currentmarker}{}%
\end{pgfscope}%
\begin{pgfscope}%
\pgfsys@transformshift{11.482353in}{3.733336in}%
\pgfsys@useobject{currentmarker}{}%
\end{pgfscope}%
\end{pgfscope}%
\begin{pgfscope}%
\pgfsetbuttcap%
\pgfsetroundjoin%
\definecolor{currentfill}{rgb}{0.000000,0.000000,0.000000}%
\pgfsetfillcolor{currentfill}%
\pgfsetlinewidth{0.803000pt}%
\definecolor{currentstroke}{rgb}{0.000000,0.000000,0.000000}%
\pgfsetstrokecolor{currentstroke}%
\pgfsetdash{}{0pt}%
\pgfsys@defobject{currentmarker}{\pgfqpoint{0.000000in}{-0.048611in}}{\pgfqpoint{0.000000in}{0.000000in}}{%
\pgfpathmoveto{\pgfqpoint{0.000000in}{0.000000in}}%
\pgfpathlineto{\pgfqpoint{0.000000in}{-0.048611in}}%
\pgfusepath{stroke,fill}%
}%
\begin{pgfscope}%
\pgfsys@transformshift{7.105882in}{3.134211in}%
\pgfsys@useobject{currentmarker}{}%
\end{pgfscope}%
\end{pgfscope}%
\begin{pgfscope}%
\pgfsetbuttcap%
\pgfsetroundjoin%
\definecolor{currentfill}{rgb}{0.000000,0.000000,0.000000}%
\pgfsetfillcolor{currentfill}%
\pgfsetlinewidth{0.803000pt}%
\definecolor{currentstroke}{rgb}{0.000000,0.000000,0.000000}%
\pgfsetstrokecolor{currentstroke}%
\pgfsetdash{}{0pt}%
\pgfsys@defobject{currentmarker}{\pgfqpoint{0.000000in}{-0.048611in}}{\pgfqpoint{0.000000in}{0.000000in}}{%
\pgfpathmoveto{\pgfqpoint{0.000000in}{0.000000in}}%
\pgfpathlineto{\pgfqpoint{0.000000in}{-0.048611in}}%
\pgfusepath{stroke,fill}%
}%
\begin{pgfscope}%
\pgfsys@transformshift{7.981176in}{3.134211in}%
\pgfsys@useobject{currentmarker}{}%
\end{pgfscope}%
\end{pgfscope}%
\begin{pgfscope}%
\pgfsetbuttcap%
\pgfsetroundjoin%
\definecolor{currentfill}{rgb}{0.000000,0.000000,0.000000}%
\pgfsetfillcolor{currentfill}%
\pgfsetlinewidth{0.803000pt}%
\definecolor{currentstroke}{rgb}{0.000000,0.000000,0.000000}%
\pgfsetstrokecolor{currentstroke}%
\pgfsetdash{}{0pt}%
\pgfsys@defobject{currentmarker}{\pgfqpoint{0.000000in}{-0.048611in}}{\pgfqpoint{0.000000in}{0.000000in}}{%
\pgfpathmoveto{\pgfqpoint{0.000000in}{0.000000in}}%
\pgfpathlineto{\pgfqpoint{0.000000in}{-0.048611in}}%
\pgfusepath{stroke,fill}%
}%
\begin{pgfscope}%
\pgfsys@transformshift{8.856471in}{3.134211in}%
\pgfsys@useobject{currentmarker}{}%
\end{pgfscope}%
\end{pgfscope}%
\begin{pgfscope}%
\pgfsetbuttcap%
\pgfsetroundjoin%
\definecolor{currentfill}{rgb}{0.000000,0.000000,0.000000}%
\pgfsetfillcolor{currentfill}%
\pgfsetlinewidth{0.803000pt}%
\definecolor{currentstroke}{rgb}{0.000000,0.000000,0.000000}%
\pgfsetstrokecolor{currentstroke}%
\pgfsetdash{}{0pt}%
\pgfsys@defobject{currentmarker}{\pgfqpoint{0.000000in}{-0.048611in}}{\pgfqpoint{0.000000in}{0.000000in}}{%
\pgfpathmoveto{\pgfqpoint{0.000000in}{0.000000in}}%
\pgfpathlineto{\pgfqpoint{0.000000in}{-0.048611in}}%
\pgfusepath{stroke,fill}%
}%
\begin{pgfscope}%
\pgfsys@transformshift{9.731765in}{3.134211in}%
\pgfsys@useobject{currentmarker}{}%
\end{pgfscope}%
\end{pgfscope}%
\begin{pgfscope}%
\pgfsetbuttcap%
\pgfsetroundjoin%
\definecolor{currentfill}{rgb}{0.000000,0.000000,0.000000}%
\pgfsetfillcolor{currentfill}%
\pgfsetlinewidth{0.803000pt}%
\definecolor{currentstroke}{rgb}{0.000000,0.000000,0.000000}%
\pgfsetstrokecolor{currentstroke}%
\pgfsetdash{}{0pt}%
\pgfsys@defobject{currentmarker}{\pgfqpoint{0.000000in}{-0.048611in}}{\pgfqpoint{0.000000in}{0.000000in}}{%
\pgfpathmoveto{\pgfqpoint{0.000000in}{0.000000in}}%
\pgfpathlineto{\pgfqpoint{0.000000in}{-0.048611in}}%
\pgfusepath{stroke,fill}%
}%
\begin{pgfscope}%
\pgfsys@transformshift{10.607059in}{3.134211in}%
\pgfsys@useobject{currentmarker}{}%
\end{pgfscope}%
\end{pgfscope}%
\begin{pgfscope}%
\pgfsetbuttcap%
\pgfsetroundjoin%
\definecolor{currentfill}{rgb}{0.000000,0.000000,0.000000}%
\pgfsetfillcolor{currentfill}%
\pgfsetlinewidth{0.803000pt}%
\definecolor{currentstroke}{rgb}{0.000000,0.000000,0.000000}%
\pgfsetstrokecolor{currentstroke}%
\pgfsetdash{}{0pt}%
\pgfsys@defobject{currentmarker}{\pgfqpoint{0.000000in}{-0.048611in}}{\pgfqpoint{0.000000in}{0.000000in}}{%
\pgfpathmoveto{\pgfqpoint{0.000000in}{0.000000in}}%
\pgfpathlineto{\pgfqpoint{0.000000in}{-0.048611in}}%
\pgfusepath{stroke,fill}%
}%
\begin{pgfscope}%
\pgfsys@transformshift{11.482353in}{3.134211in}%
\pgfsys@useobject{currentmarker}{}%
\end{pgfscope}%
\end{pgfscope}%
\begin{pgfscope}%
\pgfsetbuttcap%
\pgfsetroundjoin%
\definecolor{currentfill}{rgb}{0.000000,0.000000,0.000000}%
\pgfsetfillcolor{currentfill}%
\pgfsetlinewidth{0.803000pt}%
\definecolor{currentstroke}{rgb}{0.000000,0.000000,0.000000}%
\pgfsetstrokecolor{currentstroke}%
\pgfsetdash{}{0pt}%
\pgfsys@defobject{currentmarker}{\pgfqpoint{-0.048611in}{0.000000in}}{\pgfqpoint{0.000000in}{0.000000in}}{%
\pgfpathmoveto{\pgfqpoint{0.000000in}{0.000000in}}%
\pgfpathlineto{\pgfqpoint{-0.048611in}{0.000000in}}%
\pgfusepath{stroke,fill}%
}%
\begin{pgfscope}%
\pgfsys@transformshift{7.105882in}{3.491842in}%
\pgfsys@useobject{currentmarker}{}%
\end{pgfscope}%
\end{pgfscope}%
\begin{pgfscope}%
\pgftext[x=6.939215in,y=3.443624in,left,base]{\rmfamily\fontsize{10.000000}{12.000000}\selectfont \(\displaystyle 0\)}%
\end{pgfscope}%
\begin{pgfscope}%
\pgfsetbuttcap%
\pgfsetroundjoin%
\definecolor{currentfill}{rgb}{0.000000,0.000000,0.000000}%
\pgfsetfillcolor{currentfill}%
\pgfsetlinewidth{0.803000pt}%
\definecolor{currentstroke}{rgb}{0.000000,0.000000,0.000000}%
\pgfsetstrokecolor{currentstroke}%
\pgfsetdash{}{0pt}%
\pgfsys@defobject{currentmarker}{\pgfqpoint{-0.048611in}{0.000000in}}{\pgfqpoint{0.000000in}{0.000000in}}{%
\pgfpathmoveto{\pgfqpoint{0.000000in}{0.000000in}}%
\pgfpathlineto{\pgfqpoint{-0.048611in}{0.000000in}}%
\pgfusepath{stroke,fill}%
}%
\begin{pgfscope}%
\pgfsys@transformshift{7.105882in}{3.889211in}%
\pgfsys@useobject{currentmarker}{}%
\end{pgfscope}%
\end{pgfscope}%
\begin{pgfscope}%
\pgftext[x=6.939215in,y=3.840993in,left,base]{\rmfamily\fontsize{10.000000}{12.000000}\selectfont \(\displaystyle 2\)}%
\end{pgfscope}%
\begin{pgfscope}%
\pgfpathrectangle{\pgfqpoint{7.105882in}{3.134211in}}{\pgfqpoint{4.376471in}{0.953684in}} %
\pgfusepath{clip}%
\pgfsetrectcap%
\pgfsetroundjoin%
\pgfsetlinewidth{1.505625pt}%
\definecolor{currentstroke}{rgb}{0.121569,0.466667,0.705882}%
\pgfsetstrokecolor{currentstroke}%
\pgfsetdash{}{0pt}%
\pgfpathmoveto{\pgfqpoint{7.981176in}{3.491842in}}%
\pgfpathlineto{\pgfqpoint{11.482353in}{3.491842in}}%
\pgfpathlineto{\pgfqpoint{11.482353in}{3.491842in}}%
\pgfusepath{stroke}%
\end{pgfscope}%
\begin{pgfscope}%
\pgfsetrectcap%
\pgfsetmiterjoin%
\pgfsetlinewidth{0.803000pt}%
\definecolor{currentstroke}{rgb}{0.000000,0.000000,0.000000}%
\pgfsetstrokecolor{currentstroke}%
\pgfsetdash{}{0pt}%
\pgfpathmoveto{\pgfqpoint{7.105882in}{3.134211in}}%
\pgfpathlineto{\pgfqpoint{7.105882in}{4.087895in}}%
\pgfusepath{stroke}%
\end{pgfscope}%
\begin{pgfscope}%
\pgfsetrectcap%
\pgfsetmiterjoin%
\pgfsetlinewidth{0.803000pt}%
\definecolor{currentstroke}{rgb}{0.000000,0.000000,0.000000}%
\pgfsetstrokecolor{currentstroke}%
\pgfsetdash{}{0pt}%
\pgfpathmoveto{\pgfqpoint{11.482353in}{3.134211in}}%
\pgfpathlineto{\pgfqpoint{11.482353in}{4.087895in}}%
\pgfusepath{stroke}%
\end{pgfscope}%
\begin{pgfscope}%
\pgfsetrectcap%
\pgfsetmiterjoin%
\pgfsetlinewidth{0.803000pt}%
\definecolor{currentstroke}{rgb}{0.000000,0.000000,0.000000}%
\pgfsetstrokecolor{currentstroke}%
\pgfsetdash{}{0pt}%
\pgfpathmoveto{\pgfqpoint{7.105882in}{3.134211in}}%
\pgfpathlineto{\pgfqpoint{11.482353in}{3.134211in}}%
\pgfusepath{stroke}%
\end{pgfscope}%
\begin{pgfscope}%
\pgfsetrectcap%
\pgfsetmiterjoin%
\pgfsetlinewidth{0.803000pt}%
\definecolor{currentstroke}{rgb}{0.000000,0.000000,0.000000}%
\pgfsetstrokecolor{currentstroke}%
\pgfsetdash{}{0pt}%
\pgfpathmoveto{\pgfqpoint{7.105882in}{4.087895in}}%
\pgfpathlineto{\pgfqpoint{11.482353in}{4.087895in}}%
\pgfusepath{stroke}%
\end{pgfscope}%
\begin{pgfscope}%
\pgfsetbuttcap%
\pgfsetmiterjoin%
\definecolor{currentfill}{rgb}{1.000000,1.000000,1.000000}%
\pgfsetfillcolor{currentfill}%
\pgfsetfillopacity{0.800000}%
\pgfsetlinewidth{1.003750pt}%
\definecolor{currentstroke}{rgb}{0.800000,0.800000,0.800000}%
\pgfsetstrokecolor{currentstroke}%
\pgfsetstrokeopacity{0.800000}%
\pgfsetdash{}{0pt}%
\pgfpathmoveto{\pgfqpoint{7.203105in}{3.203655in}}%
\pgfpathlineto{\pgfqpoint{7.944617in}{3.203655in}}%
\pgfpathquadraticcurveto{\pgfqpoint{7.972395in}{3.203655in}}{\pgfqpoint{7.972395in}{3.231433in}}%
\pgfpathlineto{\pgfqpoint{7.972395in}{3.815146in}}%
\pgfpathquadraticcurveto{\pgfqpoint{7.972395in}{3.842923in}}{\pgfqpoint{7.944617in}{3.842923in}}%
\pgfpathlineto{\pgfqpoint{7.203105in}{3.842923in}}%
\pgfpathquadraticcurveto{\pgfqpoint{7.175327in}{3.842923in}}{\pgfqpoint{7.175327in}{3.815146in}}%
\pgfpathlineto{\pgfqpoint{7.175327in}{3.231433in}}%
\pgfpathquadraticcurveto{\pgfqpoint{7.175327in}{3.203655in}}{\pgfqpoint{7.203105in}{3.203655in}}%
\pgfpathclose%
\pgfusepath{stroke,fill}%
\end{pgfscope}%
\begin{pgfscope}%
\pgfsetrectcap%
\pgfsetroundjoin%
\pgfsetlinewidth{1.505625pt}%
\definecolor{currentstroke}{rgb}{0.121569,0.466667,0.705882}%
\pgfsetstrokecolor{currentstroke}%
\pgfsetdash{}{0pt}%
\pgfpathmoveto{\pgfqpoint{7.230882in}{3.730266in}}%
\pgfpathlineto{\pgfqpoint{7.508660in}{3.730266in}}%
\pgfusepath{stroke}%
\end{pgfscope}%
\begin{pgfscope}%
\pgftext[x=7.619771in,y=3.681655in,left,base]{\rmfamily\fontsize{10.000000}{12.000000}\selectfont \(\displaystyle \widetilde{\Phi}^* \theta^{\parallel}\)}%
\end{pgfscope}%
\begin{pgfscope}%
\pgfsetbuttcap%
\pgfsetroundjoin%
\definecolor{currentfill}{rgb}{1.000000,0.000000,0.000000}%
\pgfsetfillcolor{currentfill}%
\pgfsetlinewidth{2.007500pt}%
\definecolor{currentstroke}{rgb}{1.000000,0.000000,0.000000}%
\pgfsetstrokecolor{currentstroke}%
\pgfsetdash{}{0pt}%
\pgfpathmoveto{\pgfqpoint{7.328105in}{3.521743in}}%
\pgfpathlineto{\pgfqpoint{7.411438in}{3.521743in}}%
\pgfpathmoveto{\pgfqpoint{7.369771in}{3.480076in}}%
\pgfpathlineto{\pgfqpoint{7.369771in}{3.563409in}}%
\pgfusepath{stroke,fill}%
\end{pgfscope}%
\begin{pgfscope}%
\pgftext[x=7.619771in,y=3.485284in,left,base]{\rmfamily\fontsize{10.000000}{12.000000}\selectfont train}%
\end{pgfscope}%
\begin{pgfscope}%
\pgfsetbuttcap%
\pgfsetroundjoin%
\definecolor{currentfill}{rgb}{0.000000,0.000000,0.000000}%
\pgfsetfillcolor{currentfill}%
\pgfsetlinewidth{1.003750pt}%
\definecolor{currentstroke}{rgb}{0.000000,0.000000,0.000000}%
\pgfsetstrokecolor{currentstroke}%
\pgfsetdash{}{0pt}%
\pgfsys@defobject{currentmarker}{\pgfqpoint{-0.020833in}{-0.020833in}}{\pgfqpoint{0.020833in}{0.020833in}}{%
\pgfpathmoveto{\pgfqpoint{0.000000in}{-0.020833in}}%
\pgfpathcurveto{\pgfqpoint{0.005525in}{-0.020833in}}{\pgfqpoint{0.010825in}{-0.018638in}}{\pgfqpoint{0.014731in}{-0.014731in}}%
\pgfpathcurveto{\pgfqpoint{0.018638in}{-0.010825in}}{\pgfqpoint{0.020833in}{-0.005525in}}{\pgfqpoint{0.020833in}{0.000000in}}%
\pgfpathcurveto{\pgfqpoint{0.020833in}{0.005525in}}{\pgfqpoint{0.018638in}{0.010825in}}{\pgfqpoint{0.014731in}{0.014731in}}%
\pgfpathcurveto{\pgfqpoint{0.010825in}{0.018638in}}{\pgfqpoint{0.005525in}{0.020833in}}{\pgfqpoint{0.000000in}{0.020833in}}%
\pgfpathcurveto{\pgfqpoint{-0.005525in}{0.020833in}}{\pgfqpoint{-0.010825in}{0.018638in}}{\pgfqpoint{-0.014731in}{0.014731in}}%
\pgfpathcurveto{\pgfqpoint{-0.018638in}{0.010825in}}{\pgfqpoint{-0.020833in}{0.005525in}}{\pgfqpoint{-0.020833in}{0.000000in}}%
\pgfpathcurveto{\pgfqpoint{-0.020833in}{-0.005525in}}{\pgfqpoint{-0.018638in}{-0.010825in}}{\pgfqpoint{-0.014731in}{-0.014731in}}%
\pgfpathcurveto{\pgfqpoint{-0.010825in}{-0.018638in}}{\pgfqpoint{-0.005525in}{-0.020833in}}{\pgfqpoint{0.000000in}{-0.020833in}}%
\pgfpathclose%
\pgfusepath{stroke,fill}%
}%
\begin{pgfscope}%
\pgfsys@transformshift{7.369771in}{3.325372in}%
\pgfsys@useobject{currentmarker}{}%
\end{pgfscope}%
\end{pgfscope}%
\begin{pgfscope}%
\pgftext[x=7.619771in,y=3.288914in,left,base]{\rmfamily\fontsize{10.000000}{12.000000}\selectfont test}%
\end{pgfscope}%
\begin{pgfscope}%
\pgfsetbuttcap%
\pgfsetmiterjoin%
\definecolor{currentfill}{rgb}{1.000000,1.000000,1.000000}%
\pgfsetfillcolor{currentfill}%
\pgfsetlinewidth{0.000000pt}%
\definecolor{currentstroke}{rgb}{0.000000,0.000000,0.000000}%
\pgfsetstrokecolor{currentstroke}%
\pgfsetstrokeopacity{0.000000}%
\pgfsetdash{}{0pt}%
\pgfpathmoveto{\pgfqpoint{12.211765in}{3.134211in}}%
\pgfpathlineto{\pgfqpoint{14.400000in}{3.134211in}}%
\pgfpathlineto{\pgfqpoint{14.400000in}{4.087895in}}%
\pgfpathlineto{\pgfqpoint{12.211765in}{4.087895in}}%
\pgfpathclose%
\pgfusepath{fill}%
\end{pgfscope}%
\begin{pgfscope}%
\pgfpathrectangle{\pgfqpoint{12.211765in}{3.134211in}}{\pgfqpoint{2.188235in}{0.953684in}} %
\pgfusepath{clip}%
\pgfsetbuttcap%
\pgfsetmiterjoin%
\definecolor{currentfill}{rgb}{0.121569,0.466667,0.705882}%
\pgfsetfillcolor{currentfill}%
\pgfsetlinewidth{0.000000pt}%
\definecolor{currentstroke}{rgb}{0.000000,0.000000,0.000000}%
\pgfsetstrokecolor{currentstroke}%
\pgfsetstrokeopacity{0.000000}%
\pgfsetdash{}{0pt}%
\pgfpathmoveto{\pgfqpoint{-23.809001in}{3.177560in}}%
\pgfpathlineto{\pgfqpoint{13.088857in}{3.177560in}}%
\pgfpathlineto{\pgfqpoint{13.088857in}{3.184510in}}%
\pgfpathlineto{\pgfqpoint{-23.809001in}{3.184510in}}%
\pgfpathclose%
\pgfusepath{fill}%
\end{pgfscope}%
\begin{pgfscope}%
\pgfpathrectangle{\pgfqpoint{12.211765in}{3.134211in}}{\pgfqpoint{2.188235in}{0.953684in}} %
\pgfusepath{clip}%
\pgfsetbuttcap%
\pgfsetmiterjoin%
\definecolor{currentfill}{rgb}{0.121569,0.466667,0.705882}%
\pgfsetfillcolor{currentfill}%
\pgfsetlinewidth{0.000000pt}%
\definecolor{currentstroke}{rgb}{0.000000,0.000000,0.000000}%
\pgfsetstrokecolor{currentstroke}%
\pgfsetstrokeopacity{0.000000}%
\pgfsetdash{}{0pt}%
\pgfpathmoveto{\pgfqpoint{-23.809001in}{3.186247in}}%
\pgfpathlineto{\pgfqpoint{13.173032in}{3.186247in}}%
\pgfpathlineto{\pgfqpoint{13.173032in}{3.193197in}}%
\pgfpathlineto{\pgfqpoint{-23.809001in}{3.193197in}}%
\pgfpathclose%
\pgfusepath{fill}%
\end{pgfscope}%
\begin{pgfscope}%
\pgfpathrectangle{\pgfqpoint{12.211765in}{3.134211in}}{\pgfqpoint{2.188235in}{0.953684in}} %
\pgfusepath{clip}%
\pgfsetbuttcap%
\pgfsetmiterjoin%
\definecolor{currentfill}{rgb}{0.121569,0.466667,0.705882}%
\pgfsetfillcolor{currentfill}%
\pgfsetlinewidth{0.000000pt}%
\definecolor{currentstroke}{rgb}{0.000000,0.000000,0.000000}%
\pgfsetstrokecolor{currentstroke}%
\pgfsetstrokeopacity{0.000000}%
\pgfsetdash{}{0pt}%
\pgfpathmoveto{\pgfqpoint{-23.809001in}{3.194934in}}%
\pgfpathlineto{\pgfqpoint{13.171347in}{3.194934in}}%
\pgfpathlineto{\pgfqpoint{13.171347in}{3.201884in}}%
\pgfpathlineto{\pgfqpoint{-23.809001in}{3.201884in}}%
\pgfpathclose%
\pgfusepath{fill}%
\end{pgfscope}%
\begin{pgfscope}%
\pgfpathrectangle{\pgfqpoint{12.211765in}{3.134211in}}{\pgfqpoint{2.188235in}{0.953684in}} %
\pgfusepath{clip}%
\pgfsetbuttcap%
\pgfsetmiterjoin%
\definecolor{currentfill}{rgb}{0.121569,0.466667,0.705882}%
\pgfsetfillcolor{currentfill}%
\pgfsetlinewidth{0.000000pt}%
\definecolor{currentstroke}{rgb}{0.000000,0.000000,0.000000}%
\pgfsetstrokecolor{currentstroke}%
\pgfsetstrokeopacity{0.000000}%
\pgfsetdash{}{0pt}%
\pgfpathmoveto{\pgfqpoint{-23.809001in}{3.203622in}}%
\pgfpathlineto{\pgfqpoint{13.047961in}{3.203622in}}%
\pgfpathlineto{\pgfqpoint{13.047961in}{3.210571in}}%
\pgfpathlineto{\pgfqpoint{-23.809001in}{3.210571in}}%
\pgfpathclose%
\pgfusepath{fill}%
\end{pgfscope}%
\begin{pgfscope}%
\pgfpathrectangle{\pgfqpoint{12.211765in}{3.134211in}}{\pgfqpoint{2.188235in}{0.953684in}} %
\pgfusepath{clip}%
\pgfsetbuttcap%
\pgfsetmiterjoin%
\definecolor{currentfill}{rgb}{0.121569,0.466667,0.705882}%
\pgfsetfillcolor{currentfill}%
\pgfsetlinewidth{0.000000pt}%
\definecolor{currentstroke}{rgb}{0.000000,0.000000,0.000000}%
\pgfsetstrokecolor{currentstroke}%
\pgfsetstrokeopacity{0.000000}%
\pgfsetdash{}{0pt}%
\pgfpathmoveto{\pgfqpoint{-23.809001in}{3.212309in}}%
\pgfpathlineto{\pgfqpoint{13.081245in}{3.212309in}}%
\pgfpathlineto{\pgfqpoint{13.081245in}{3.219259in}}%
\pgfpathlineto{\pgfqpoint{-23.809001in}{3.219259in}}%
\pgfpathclose%
\pgfusepath{fill}%
\end{pgfscope}%
\begin{pgfscope}%
\pgfpathrectangle{\pgfqpoint{12.211765in}{3.134211in}}{\pgfqpoint{2.188235in}{0.953684in}} %
\pgfusepath{clip}%
\pgfsetbuttcap%
\pgfsetmiterjoin%
\definecolor{currentfill}{rgb}{0.121569,0.466667,0.705882}%
\pgfsetfillcolor{currentfill}%
\pgfsetlinewidth{0.000000pt}%
\definecolor{currentstroke}{rgb}{0.000000,0.000000,0.000000}%
\pgfsetstrokecolor{currentstroke}%
\pgfsetstrokeopacity{0.000000}%
\pgfsetdash{}{0pt}%
\pgfpathmoveto{\pgfqpoint{-23.809001in}{3.220996in}}%
\pgfpathlineto{\pgfqpoint{13.106233in}{3.220996in}}%
\pgfpathlineto{\pgfqpoint{13.106233in}{3.227946in}}%
\pgfpathlineto{\pgfqpoint{-23.809001in}{3.227946in}}%
\pgfpathclose%
\pgfusepath{fill}%
\end{pgfscope}%
\begin{pgfscope}%
\pgfpathrectangle{\pgfqpoint{12.211765in}{3.134211in}}{\pgfqpoint{2.188235in}{0.953684in}} %
\pgfusepath{clip}%
\pgfsetbuttcap%
\pgfsetmiterjoin%
\definecolor{currentfill}{rgb}{0.121569,0.466667,0.705882}%
\pgfsetfillcolor{currentfill}%
\pgfsetlinewidth{0.000000pt}%
\definecolor{currentstroke}{rgb}{0.000000,0.000000,0.000000}%
\pgfsetstrokecolor{currentstroke}%
\pgfsetstrokeopacity{0.000000}%
\pgfsetdash{}{0pt}%
\pgfpathmoveto{\pgfqpoint{-23.809001in}{3.229683in}}%
\pgfpathlineto{\pgfqpoint{13.065992in}{3.229683in}}%
\pgfpathlineto{\pgfqpoint{13.065992in}{3.236633in}}%
\pgfpathlineto{\pgfqpoint{-23.809001in}{3.236633in}}%
\pgfpathclose%
\pgfusepath{fill}%
\end{pgfscope}%
\begin{pgfscope}%
\pgfpathrectangle{\pgfqpoint{12.211765in}{3.134211in}}{\pgfqpoint{2.188235in}{0.953684in}} %
\pgfusepath{clip}%
\pgfsetbuttcap%
\pgfsetmiterjoin%
\definecolor{currentfill}{rgb}{0.121569,0.466667,0.705882}%
\pgfsetfillcolor{currentfill}%
\pgfsetlinewidth{0.000000pt}%
\definecolor{currentstroke}{rgb}{0.000000,0.000000,0.000000}%
\pgfsetstrokecolor{currentstroke}%
\pgfsetstrokeopacity{0.000000}%
\pgfsetdash{}{0pt}%
\pgfpathmoveto{\pgfqpoint{-23.809001in}{3.238370in}}%
\pgfpathlineto{\pgfqpoint{13.165057in}{3.238370in}}%
\pgfpathlineto{\pgfqpoint{13.165057in}{3.245320in}}%
\pgfpathlineto{\pgfqpoint{-23.809001in}{3.245320in}}%
\pgfpathclose%
\pgfusepath{fill}%
\end{pgfscope}%
\begin{pgfscope}%
\pgfpathrectangle{\pgfqpoint{12.211765in}{3.134211in}}{\pgfqpoint{2.188235in}{0.953684in}} %
\pgfusepath{clip}%
\pgfsetbuttcap%
\pgfsetmiterjoin%
\definecolor{currentfill}{rgb}{0.121569,0.466667,0.705882}%
\pgfsetfillcolor{currentfill}%
\pgfsetlinewidth{0.000000pt}%
\definecolor{currentstroke}{rgb}{0.000000,0.000000,0.000000}%
\pgfsetstrokecolor{currentstroke}%
\pgfsetstrokeopacity{0.000000}%
\pgfsetdash{}{0pt}%
\pgfpathmoveto{\pgfqpoint{-23.809001in}{3.247058in}}%
\pgfpathlineto{\pgfqpoint{13.060758in}{3.247058in}}%
\pgfpathlineto{\pgfqpoint{13.060758in}{3.254007in}}%
\pgfpathlineto{\pgfqpoint{-23.809001in}{3.254007in}}%
\pgfpathclose%
\pgfusepath{fill}%
\end{pgfscope}%
\begin{pgfscope}%
\pgfpathrectangle{\pgfqpoint{12.211765in}{3.134211in}}{\pgfqpoint{2.188235in}{0.953684in}} %
\pgfusepath{clip}%
\pgfsetbuttcap%
\pgfsetmiterjoin%
\definecolor{currentfill}{rgb}{0.121569,0.466667,0.705882}%
\pgfsetfillcolor{currentfill}%
\pgfsetlinewidth{0.000000pt}%
\definecolor{currentstroke}{rgb}{0.000000,0.000000,0.000000}%
\pgfsetstrokecolor{currentstroke}%
\pgfsetstrokeopacity{0.000000}%
\pgfsetdash{}{0pt}%
\pgfpathmoveto{\pgfqpoint{-23.809001in}{3.255745in}}%
\pgfpathlineto{\pgfqpoint{13.078597in}{3.255745in}}%
\pgfpathlineto{\pgfqpoint{13.078597in}{3.262695in}}%
\pgfpathlineto{\pgfqpoint{-23.809001in}{3.262695in}}%
\pgfpathclose%
\pgfusepath{fill}%
\end{pgfscope}%
\begin{pgfscope}%
\pgfpathrectangle{\pgfqpoint{12.211765in}{3.134211in}}{\pgfqpoint{2.188235in}{0.953684in}} %
\pgfusepath{clip}%
\pgfsetbuttcap%
\pgfsetmiterjoin%
\definecolor{currentfill}{rgb}{0.121569,0.466667,0.705882}%
\pgfsetfillcolor{currentfill}%
\pgfsetlinewidth{0.000000pt}%
\definecolor{currentstroke}{rgb}{0.000000,0.000000,0.000000}%
\pgfsetstrokecolor{currentstroke}%
\pgfsetstrokeopacity{0.000000}%
\pgfsetdash{}{0pt}%
\pgfpathmoveto{\pgfqpoint{-23.809001in}{3.264432in}}%
\pgfpathlineto{\pgfqpoint{13.121670in}{3.264432in}}%
\pgfpathlineto{\pgfqpoint{13.121670in}{3.271382in}}%
\pgfpathlineto{\pgfqpoint{-23.809001in}{3.271382in}}%
\pgfpathclose%
\pgfusepath{fill}%
\end{pgfscope}%
\begin{pgfscope}%
\pgfpathrectangle{\pgfqpoint{12.211765in}{3.134211in}}{\pgfqpoint{2.188235in}{0.953684in}} %
\pgfusepath{clip}%
\pgfsetbuttcap%
\pgfsetmiterjoin%
\definecolor{currentfill}{rgb}{0.121569,0.466667,0.705882}%
\pgfsetfillcolor{currentfill}%
\pgfsetlinewidth{0.000000pt}%
\definecolor{currentstroke}{rgb}{0.000000,0.000000,0.000000}%
\pgfsetstrokecolor{currentstroke}%
\pgfsetstrokeopacity{0.000000}%
\pgfsetdash{}{0pt}%
\pgfpathmoveto{\pgfqpoint{-23.809001in}{3.273119in}}%
\pgfpathlineto{\pgfqpoint{13.125972in}{3.273119in}}%
\pgfpathlineto{\pgfqpoint{13.125972in}{3.280069in}}%
\pgfpathlineto{\pgfqpoint{-23.809001in}{3.280069in}}%
\pgfpathclose%
\pgfusepath{fill}%
\end{pgfscope}%
\begin{pgfscope}%
\pgfpathrectangle{\pgfqpoint{12.211765in}{3.134211in}}{\pgfqpoint{2.188235in}{0.953684in}} %
\pgfusepath{clip}%
\pgfsetbuttcap%
\pgfsetmiterjoin%
\definecolor{currentfill}{rgb}{0.121569,0.466667,0.705882}%
\pgfsetfillcolor{currentfill}%
\pgfsetlinewidth{0.000000pt}%
\definecolor{currentstroke}{rgb}{0.000000,0.000000,0.000000}%
\pgfsetstrokecolor{currentstroke}%
\pgfsetstrokeopacity{0.000000}%
\pgfsetdash{}{0pt}%
\pgfpathmoveto{\pgfqpoint{-23.809001in}{3.281807in}}%
\pgfpathlineto{\pgfqpoint{12.932805in}{3.281807in}}%
\pgfpathlineto{\pgfqpoint{12.932805in}{3.288756in}}%
\pgfpathlineto{\pgfqpoint{-23.809001in}{3.288756in}}%
\pgfpathclose%
\pgfusepath{fill}%
\end{pgfscope}%
\begin{pgfscope}%
\pgfpathrectangle{\pgfqpoint{12.211765in}{3.134211in}}{\pgfqpoint{2.188235in}{0.953684in}} %
\pgfusepath{clip}%
\pgfsetbuttcap%
\pgfsetmiterjoin%
\definecolor{currentfill}{rgb}{0.121569,0.466667,0.705882}%
\pgfsetfillcolor{currentfill}%
\pgfsetlinewidth{0.000000pt}%
\definecolor{currentstroke}{rgb}{0.000000,0.000000,0.000000}%
\pgfsetstrokecolor{currentstroke}%
\pgfsetstrokeopacity{0.000000}%
\pgfsetdash{}{0pt}%
\pgfpathmoveto{\pgfqpoint{-23.809001in}{3.290494in}}%
\pgfpathlineto{\pgfqpoint{13.026647in}{3.290494in}}%
\pgfpathlineto{\pgfqpoint{13.026647in}{3.297444in}}%
\pgfpathlineto{\pgfqpoint{-23.809001in}{3.297444in}}%
\pgfpathclose%
\pgfusepath{fill}%
\end{pgfscope}%
\begin{pgfscope}%
\pgfpathrectangle{\pgfqpoint{12.211765in}{3.134211in}}{\pgfqpoint{2.188235in}{0.953684in}} %
\pgfusepath{clip}%
\pgfsetbuttcap%
\pgfsetmiterjoin%
\definecolor{currentfill}{rgb}{0.121569,0.466667,0.705882}%
\pgfsetfillcolor{currentfill}%
\pgfsetlinewidth{0.000000pt}%
\definecolor{currentstroke}{rgb}{0.000000,0.000000,0.000000}%
\pgfsetstrokecolor{currentstroke}%
\pgfsetstrokeopacity{0.000000}%
\pgfsetdash{}{0pt}%
\pgfpathmoveto{\pgfqpoint{-23.809001in}{3.299181in}}%
\pgfpathlineto{\pgfqpoint{13.125172in}{3.299181in}}%
\pgfpathlineto{\pgfqpoint{13.125172in}{3.306131in}}%
\pgfpathlineto{\pgfqpoint{-23.809001in}{3.306131in}}%
\pgfpathclose%
\pgfusepath{fill}%
\end{pgfscope}%
\begin{pgfscope}%
\pgfpathrectangle{\pgfqpoint{12.211765in}{3.134211in}}{\pgfqpoint{2.188235in}{0.953684in}} %
\pgfusepath{clip}%
\pgfsetbuttcap%
\pgfsetmiterjoin%
\definecolor{currentfill}{rgb}{0.121569,0.466667,0.705882}%
\pgfsetfillcolor{currentfill}%
\pgfsetlinewidth{0.000000pt}%
\definecolor{currentstroke}{rgb}{0.000000,0.000000,0.000000}%
\pgfsetstrokecolor{currentstroke}%
\pgfsetstrokeopacity{0.000000}%
\pgfsetdash{}{0pt}%
\pgfpathmoveto{\pgfqpoint{-23.809001in}{3.307868in}}%
\pgfpathlineto{\pgfqpoint{13.088804in}{3.307868in}}%
\pgfpathlineto{\pgfqpoint{13.088804in}{3.314818in}}%
\pgfpathlineto{\pgfqpoint{-23.809001in}{3.314818in}}%
\pgfpathclose%
\pgfusepath{fill}%
\end{pgfscope}%
\begin{pgfscope}%
\pgfpathrectangle{\pgfqpoint{12.211765in}{3.134211in}}{\pgfqpoint{2.188235in}{0.953684in}} %
\pgfusepath{clip}%
\pgfsetbuttcap%
\pgfsetmiterjoin%
\definecolor{currentfill}{rgb}{0.121569,0.466667,0.705882}%
\pgfsetfillcolor{currentfill}%
\pgfsetlinewidth{0.000000pt}%
\definecolor{currentstroke}{rgb}{0.000000,0.000000,0.000000}%
\pgfsetstrokecolor{currentstroke}%
\pgfsetstrokeopacity{0.000000}%
\pgfsetdash{}{0pt}%
\pgfpathmoveto{\pgfqpoint{-23.809001in}{3.316556in}}%
\pgfpathlineto{\pgfqpoint{13.103635in}{3.316556in}}%
\pgfpathlineto{\pgfqpoint{13.103635in}{3.323505in}}%
\pgfpathlineto{\pgfqpoint{-23.809001in}{3.323505in}}%
\pgfpathclose%
\pgfusepath{fill}%
\end{pgfscope}%
\begin{pgfscope}%
\pgfpathrectangle{\pgfqpoint{12.211765in}{3.134211in}}{\pgfqpoint{2.188235in}{0.953684in}} %
\pgfusepath{clip}%
\pgfsetbuttcap%
\pgfsetmiterjoin%
\definecolor{currentfill}{rgb}{0.121569,0.466667,0.705882}%
\pgfsetfillcolor{currentfill}%
\pgfsetlinewidth{0.000000pt}%
\definecolor{currentstroke}{rgb}{0.000000,0.000000,0.000000}%
\pgfsetstrokecolor{currentstroke}%
\pgfsetstrokeopacity{0.000000}%
\pgfsetdash{}{0pt}%
\pgfpathmoveto{\pgfqpoint{-23.809001in}{3.325243in}}%
\pgfpathlineto{\pgfqpoint{13.066974in}{3.325243in}}%
\pgfpathlineto{\pgfqpoint{13.066974in}{3.332193in}}%
\pgfpathlineto{\pgfqpoint{-23.809001in}{3.332193in}}%
\pgfpathclose%
\pgfusepath{fill}%
\end{pgfscope}%
\begin{pgfscope}%
\pgfpathrectangle{\pgfqpoint{12.211765in}{3.134211in}}{\pgfqpoint{2.188235in}{0.953684in}} %
\pgfusepath{clip}%
\pgfsetbuttcap%
\pgfsetmiterjoin%
\definecolor{currentfill}{rgb}{0.121569,0.466667,0.705882}%
\pgfsetfillcolor{currentfill}%
\pgfsetlinewidth{0.000000pt}%
\definecolor{currentstroke}{rgb}{0.000000,0.000000,0.000000}%
\pgfsetstrokecolor{currentstroke}%
\pgfsetstrokeopacity{0.000000}%
\pgfsetdash{}{0pt}%
\pgfpathmoveto{\pgfqpoint{-23.809001in}{3.333930in}}%
\pgfpathlineto{\pgfqpoint{13.120305in}{3.333930in}}%
\pgfpathlineto{\pgfqpoint{13.120305in}{3.340880in}}%
\pgfpathlineto{\pgfqpoint{-23.809001in}{3.340880in}}%
\pgfpathclose%
\pgfusepath{fill}%
\end{pgfscope}%
\begin{pgfscope}%
\pgfpathrectangle{\pgfqpoint{12.211765in}{3.134211in}}{\pgfqpoint{2.188235in}{0.953684in}} %
\pgfusepath{clip}%
\pgfsetbuttcap%
\pgfsetmiterjoin%
\definecolor{currentfill}{rgb}{0.121569,0.466667,0.705882}%
\pgfsetfillcolor{currentfill}%
\pgfsetlinewidth{0.000000pt}%
\definecolor{currentstroke}{rgb}{0.000000,0.000000,0.000000}%
\pgfsetstrokecolor{currentstroke}%
\pgfsetstrokeopacity{0.000000}%
\pgfsetdash{}{0pt}%
\pgfpathmoveto{\pgfqpoint{-23.809001in}{3.342617in}}%
\pgfpathlineto{\pgfqpoint{13.091588in}{3.342617in}}%
\pgfpathlineto{\pgfqpoint{13.091588in}{3.349567in}}%
\pgfpathlineto{\pgfqpoint{-23.809001in}{3.349567in}}%
\pgfpathclose%
\pgfusepath{fill}%
\end{pgfscope}%
\begin{pgfscope}%
\pgfpathrectangle{\pgfqpoint{12.211765in}{3.134211in}}{\pgfqpoint{2.188235in}{0.953684in}} %
\pgfusepath{clip}%
\pgfsetbuttcap%
\pgfsetmiterjoin%
\definecolor{currentfill}{rgb}{0.121569,0.466667,0.705882}%
\pgfsetfillcolor{currentfill}%
\pgfsetlinewidth{0.000000pt}%
\definecolor{currentstroke}{rgb}{0.000000,0.000000,0.000000}%
\pgfsetstrokecolor{currentstroke}%
\pgfsetstrokeopacity{0.000000}%
\pgfsetdash{}{0pt}%
\pgfpathmoveto{\pgfqpoint{-23.809001in}{3.351304in}}%
\pgfpathlineto{\pgfqpoint{13.135393in}{3.351304in}}%
\pgfpathlineto{\pgfqpoint{13.135393in}{3.358254in}}%
\pgfpathlineto{\pgfqpoint{-23.809001in}{3.358254in}}%
\pgfpathclose%
\pgfusepath{fill}%
\end{pgfscope}%
\begin{pgfscope}%
\pgfpathrectangle{\pgfqpoint{12.211765in}{3.134211in}}{\pgfqpoint{2.188235in}{0.953684in}} %
\pgfusepath{clip}%
\pgfsetbuttcap%
\pgfsetmiterjoin%
\definecolor{currentfill}{rgb}{0.121569,0.466667,0.705882}%
\pgfsetfillcolor{currentfill}%
\pgfsetlinewidth{0.000000pt}%
\definecolor{currentstroke}{rgb}{0.000000,0.000000,0.000000}%
\pgfsetstrokecolor{currentstroke}%
\pgfsetstrokeopacity{0.000000}%
\pgfsetdash{}{0pt}%
\pgfpathmoveto{\pgfqpoint{-23.809001in}{3.359992in}}%
\pgfpathlineto{\pgfqpoint{13.085578in}{3.359992in}}%
\pgfpathlineto{\pgfqpoint{13.085578in}{3.366941in}}%
\pgfpathlineto{\pgfqpoint{-23.809001in}{3.366941in}}%
\pgfpathclose%
\pgfusepath{fill}%
\end{pgfscope}%
\begin{pgfscope}%
\pgfpathrectangle{\pgfqpoint{12.211765in}{3.134211in}}{\pgfqpoint{2.188235in}{0.953684in}} %
\pgfusepath{clip}%
\pgfsetbuttcap%
\pgfsetmiterjoin%
\definecolor{currentfill}{rgb}{0.121569,0.466667,0.705882}%
\pgfsetfillcolor{currentfill}%
\pgfsetlinewidth{0.000000pt}%
\definecolor{currentstroke}{rgb}{0.000000,0.000000,0.000000}%
\pgfsetstrokecolor{currentstroke}%
\pgfsetstrokeopacity{0.000000}%
\pgfsetdash{}{0pt}%
\pgfpathmoveto{\pgfqpoint{-23.809001in}{3.368679in}}%
\pgfpathlineto{\pgfqpoint{12.968034in}{3.368679in}}%
\pgfpathlineto{\pgfqpoint{12.968034in}{3.375629in}}%
\pgfpathlineto{\pgfqpoint{-23.809001in}{3.375629in}}%
\pgfpathclose%
\pgfusepath{fill}%
\end{pgfscope}%
\begin{pgfscope}%
\pgfpathrectangle{\pgfqpoint{12.211765in}{3.134211in}}{\pgfqpoint{2.188235in}{0.953684in}} %
\pgfusepath{clip}%
\pgfsetbuttcap%
\pgfsetmiterjoin%
\definecolor{currentfill}{rgb}{0.121569,0.466667,0.705882}%
\pgfsetfillcolor{currentfill}%
\pgfsetlinewidth{0.000000pt}%
\definecolor{currentstroke}{rgb}{0.000000,0.000000,0.000000}%
\pgfsetstrokecolor{currentstroke}%
\pgfsetstrokeopacity{0.000000}%
\pgfsetdash{}{0pt}%
\pgfpathmoveto{\pgfqpoint{-23.809001in}{3.377366in}}%
\pgfpathlineto{\pgfqpoint{13.071618in}{3.377366in}}%
\pgfpathlineto{\pgfqpoint{13.071618in}{3.384316in}}%
\pgfpathlineto{\pgfqpoint{-23.809001in}{3.384316in}}%
\pgfpathclose%
\pgfusepath{fill}%
\end{pgfscope}%
\begin{pgfscope}%
\pgfpathrectangle{\pgfqpoint{12.211765in}{3.134211in}}{\pgfqpoint{2.188235in}{0.953684in}} %
\pgfusepath{clip}%
\pgfsetbuttcap%
\pgfsetmiterjoin%
\definecolor{currentfill}{rgb}{0.121569,0.466667,0.705882}%
\pgfsetfillcolor{currentfill}%
\pgfsetlinewidth{0.000000pt}%
\definecolor{currentstroke}{rgb}{0.000000,0.000000,0.000000}%
\pgfsetstrokecolor{currentstroke}%
\pgfsetstrokeopacity{0.000000}%
\pgfsetdash{}{0pt}%
\pgfpathmoveto{\pgfqpoint{-23.809001in}{3.386053in}}%
\pgfpathlineto{\pgfqpoint{13.104478in}{3.386053in}}%
\pgfpathlineto{\pgfqpoint{13.104478in}{3.393003in}}%
\pgfpathlineto{\pgfqpoint{-23.809001in}{3.393003in}}%
\pgfpathclose%
\pgfusepath{fill}%
\end{pgfscope}%
\begin{pgfscope}%
\pgfpathrectangle{\pgfqpoint{12.211765in}{3.134211in}}{\pgfqpoint{2.188235in}{0.953684in}} %
\pgfusepath{clip}%
\pgfsetbuttcap%
\pgfsetmiterjoin%
\definecolor{currentfill}{rgb}{0.121569,0.466667,0.705882}%
\pgfsetfillcolor{currentfill}%
\pgfsetlinewidth{0.000000pt}%
\definecolor{currentstroke}{rgb}{0.000000,0.000000,0.000000}%
\pgfsetstrokecolor{currentstroke}%
\pgfsetstrokeopacity{0.000000}%
\pgfsetdash{}{0pt}%
\pgfpathmoveto{\pgfqpoint{-23.809001in}{3.394741in}}%
\pgfpathlineto{\pgfqpoint{13.104730in}{3.394741in}}%
\pgfpathlineto{\pgfqpoint{13.104730in}{3.401690in}}%
\pgfpathlineto{\pgfqpoint{-23.809001in}{3.401690in}}%
\pgfpathclose%
\pgfusepath{fill}%
\end{pgfscope}%
\begin{pgfscope}%
\pgfpathrectangle{\pgfqpoint{12.211765in}{3.134211in}}{\pgfqpoint{2.188235in}{0.953684in}} %
\pgfusepath{clip}%
\pgfsetbuttcap%
\pgfsetmiterjoin%
\definecolor{currentfill}{rgb}{0.121569,0.466667,0.705882}%
\pgfsetfillcolor{currentfill}%
\pgfsetlinewidth{0.000000pt}%
\definecolor{currentstroke}{rgb}{0.000000,0.000000,0.000000}%
\pgfsetstrokecolor{currentstroke}%
\pgfsetstrokeopacity{0.000000}%
\pgfsetdash{}{0pt}%
\pgfpathmoveto{\pgfqpoint{-23.809001in}{3.403428in}}%
\pgfpathlineto{\pgfqpoint{13.096076in}{3.403428in}}%
\pgfpathlineto{\pgfqpoint{13.096076in}{3.410378in}}%
\pgfpathlineto{\pgfqpoint{-23.809001in}{3.410378in}}%
\pgfpathclose%
\pgfusepath{fill}%
\end{pgfscope}%
\begin{pgfscope}%
\pgfpathrectangle{\pgfqpoint{12.211765in}{3.134211in}}{\pgfqpoint{2.188235in}{0.953684in}} %
\pgfusepath{clip}%
\pgfsetbuttcap%
\pgfsetmiterjoin%
\definecolor{currentfill}{rgb}{0.121569,0.466667,0.705882}%
\pgfsetfillcolor{currentfill}%
\pgfsetlinewidth{0.000000pt}%
\definecolor{currentstroke}{rgb}{0.000000,0.000000,0.000000}%
\pgfsetstrokecolor{currentstroke}%
\pgfsetstrokeopacity{0.000000}%
\pgfsetdash{}{0pt}%
\pgfpathmoveto{\pgfqpoint{-23.809001in}{3.412115in}}%
\pgfpathlineto{\pgfqpoint{13.092830in}{3.412115in}}%
\pgfpathlineto{\pgfqpoint{13.092830in}{3.419065in}}%
\pgfpathlineto{\pgfqpoint{-23.809001in}{3.419065in}}%
\pgfpathclose%
\pgfusepath{fill}%
\end{pgfscope}%
\begin{pgfscope}%
\pgfpathrectangle{\pgfqpoint{12.211765in}{3.134211in}}{\pgfqpoint{2.188235in}{0.953684in}} %
\pgfusepath{clip}%
\pgfsetbuttcap%
\pgfsetmiterjoin%
\definecolor{currentfill}{rgb}{0.121569,0.466667,0.705882}%
\pgfsetfillcolor{currentfill}%
\pgfsetlinewidth{0.000000pt}%
\definecolor{currentstroke}{rgb}{0.000000,0.000000,0.000000}%
\pgfsetstrokecolor{currentstroke}%
\pgfsetstrokeopacity{0.000000}%
\pgfsetdash{}{0pt}%
\pgfpathmoveto{\pgfqpoint{-23.809001in}{3.420802in}}%
\pgfpathlineto{\pgfqpoint{13.035616in}{3.420802in}}%
\pgfpathlineto{\pgfqpoint{13.035616in}{3.427752in}}%
\pgfpathlineto{\pgfqpoint{-23.809001in}{3.427752in}}%
\pgfpathclose%
\pgfusepath{fill}%
\end{pgfscope}%
\begin{pgfscope}%
\pgfpathrectangle{\pgfqpoint{12.211765in}{3.134211in}}{\pgfqpoint{2.188235in}{0.953684in}} %
\pgfusepath{clip}%
\pgfsetbuttcap%
\pgfsetmiterjoin%
\definecolor{currentfill}{rgb}{0.121569,0.466667,0.705882}%
\pgfsetfillcolor{currentfill}%
\pgfsetlinewidth{0.000000pt}%
\definecolor{currentstroke}{rgb}{0.000000,0.000000,0.000000}%
\pgfsetstrokecolor{currentstroke}%
\pgfsetstrokeopacity{0.000000}%
\pgfsetdash{}{0pt}%
\pgfpathmoveto{\pgfqpoint{-23.809001in}{3.429490in}}%
\pgfpathlineto{\pgfqpoint{13.016003in}{3.429490in}}%
\pgfpathlineto{\pgfqpoint{13.016003in}{3.436439in}}%
\pgfpathlineto{\pgfqpoint{-23.809001in}{3.436439in}}%
\pgfpathclose%
\pgfusepath{fill}%
\end{pgfscope}%
\begin{pgfscope}%
\pgfpathrectangle{\pgfqpoint{12.211765in}{3.134211in}}{\pgfqpoint{2.188235in}{0.953684in}} %
\pgfusepath{clip}%
\pgfsetbuttcap%
\pgfsetmiterjoin%
\definecolor{currentfill}{rgb}{0.121569,0.466667,0.705882}%
\pgfsetfillcolor{currentfill}%
\pgfsetlinewidth{0.000000pt}%
\definecolor{currentstroke}{rgb}{0.000000,0.000000,0.000000}%
\pgfsetstrokecolor{currentstroke}%
\pgfsetstrokeopacity{0.000000}%
\pgfsetdash{}{0pt}%
\pgfpathmoveto{\pgfqpoint{-23.809001in}{3.438177in}}%
\pgfpathlineto{\pgfqpoint{13.130176in}{3.438177in}}%
\pgfpathlineto{\pgfqpoint{13.130176in}{3.445127in}}%
\pgfpathlineto{\pgfqpoint{-23.809001in}{3.445127in}}%
\pgfpathclose%
\pgfusepath{fill}%
\end{pgfscope}%
\begin{pgfscope}%
\pgfpathrectangle{\pgfqpoint{12.211765in}{3.134211in}}{\pgfqpoint{2.188235in}{0.953684in}} %
\pgfusepath{clip}%
\pgfsetbuttcap%
\pgfsetmiterjoin%
\definecolor{currentfill}{rgb}{0.121569,0.466667,0.705882}%
\pgfsetfillcolor{currentfill}%
\pgfsetlinewidth{0.000000pt}%
\definecolor{currentstroke}{rgb}{0.000000,0.000000,0.000000}%
\pgfsetstrokecolor{currentstroke}%
\pgfsetstrokeopacity{0.000000}%
\pgfsetdash{}{0pt}%
\pgfpathmoveto{\pgfqpoint{-23.809001in}{3.446864in}}%
\pgfpathlineto{\pgfqpoint{13.024046in}{3.446864in}}%
\pgfpathlineto{\pgfqpoint{13.024046in}{3.453814in}}%
\pgfpathlineto{\pgfqpoint{-23.809001in}{3.453814in}}%
\pgfpathclose%
\pgfusepath{fill}%
\end{pgfscope}%
\begin{pgfscope}%
\pgfpathrectangle{\pgfqpoint{12.211765in}{3.134211in}}{\pgfqpoint{2.188235in}{0.953684in}} %
\pgfusepath{clip}%
\pgfsetbuttcap%
\pgfsetmiterjoin%
\definecolor{currentfill}{rgb}{0.121569,0.466667,0.705882}%
\pgfsetfillcolor{currentfill}%
\pgfsetlinewidth{0.000000pt}%
\definecolor{currentstroke}{rgb}{0.000000,0.000000,0.000000}%
\pgfsetstrokecolor{currentstroke}%
\pgfsetstrokeopacity{0.000000}%
\pgfsetdash{}{0pt}%
\pgfpathmoveto{\pgfqpoint{-23.809001in}{3.455551in}}%
\pgfpathlineto{\pgfqpoint{13.093310in}{3.455551in}}%
\pgfpathlineto{\pgfqpoint{13.093310in}{3.462501in}}%
\pgfpathlineto{\pgfqpoint{-23.809001in}{3.462501in}}%
\pgfpathclose%
\pgfusepath{fill}%
\end{pgfscope}%
\begin{pgfscope}%
\pgfpathrectangle{\pgfqpoint{12.211765in}{3.134211in}}{\pgfqpoint{2.188235in}{0.953684in}} %
\pgfusepath{clip}%
\pgfsetbuttcap%
\pgfsetmiterjoin%
\definecolor{currentfill}{rgb}{0.121569,0.466667,0.705882}%
\pgfsetfillcolor{currentfill}%
\pgfsetlinewidth{0.000000pt}%
\definecolor{currentstroke}{rgb}{0.000000,0.000000,0.000000}%
\pgfsetstrokecolor{currentstroke}%
\pgfsetstrokeopacity{0.000000}%
\pgfsetdash{}{0pt}%
\pgfpathmoveto{\pgfqpoint{-23.809001in}{3.464238in}}%
\pgfpathlineto{\pgfqpoint{12.994373in}{3.464238in}}%
\pgfpathlineto{\pgfqpoint{12.994373in}{3.471188in}}%
\pgfpathlineto{\pgfqpoint{-23.809001in}{3.471188in}}%
\pgfpathclose%
\pgfusepath{fill}%
\end{pgfscope}%
\begin{pgfscope}%
\pgfpathrectangle{\pgfqpoint{12.211765in}{3.134211in}}{\pgfqpoint{2.188235in}{0.953684in}} %
\pgfusepath{clip}%
\pgfsetbuttcap%
\pgfsetmiterjoin%
\definecolor{currentfill}{rgb}{0.121569,0.466667,0.705882}%
\pgfsetfillcolor{currentfill}%
\pgfsetlinewidth{0.000000pt}%
\definecolor{currentstroke}{rgb}{0.000000,0.000000,0.000000}%
\pgfsetstrokecolor{currentstroke}%
\pgfsetstrokeopacity{0.000000}%
\pgfsetdash{}{0pt}%
\pgfpathmoveto{\pgfqpoint{-23.809001in}{3.472926in}}%
\pgfpathlineto{\pgfqpoint{13.074638in}{3.472926in}}%
\pgfpathlineto{\pgfqpoint{13.074638in}{3.479875in}}%
\pgfpathlineto{\pgfqpoint{-23.809001in}{3.479875in}}%
\pgfpathclose%
\pgfusepath{fill}%
\end{pgfscope}%
\begin{pgfscope}%
\pgfpathrectangle{\pgfqpoint{12.211765in}{3.134211in}}{\pgfqpoint{2.188235in}{0.953684in}} %
\pgfusepath{clip}%
\pgfsetbuttcap%
\pgfsetmiterjoin%
\definecolor{currentfill}{rgb}{0.121569,0.466667,0.705882}%
\pgfsetfillcolor{currentfill}%
\pgfsetlinewidth{0.000000pt}%
\definecolor{currentstroke}{rgb}{0.000000,0.000000,0.000000}%
\pgfsetstrokecolor{currentstroke}%
\pgfsetstrokeopacity{0.000000}%
\pgfsetdash{}{0pt}%
\pgfpathmoveto{\pgfqpoint{-23.809001in}{3.481613in}}%
\pgfpathlineto{\pgfqpoint{13.084945in}{3.481613in}}%
\pgfpathlineto{\pgfqpoint{13.084945in}{3.488563in}}%
\pgfpathlineto{\pgfqpoint{-23.809001in}{3.488563in}}%
\pgfpathclose%
\pgfusepath{fill}%
\end{pgfscope}%
\begin{pgfscope}%
\pgfpathrectangle{\pgfqpoint{12.211765in}{3.134211in}}{\pgfqpoint{2.188235in}{0.953684in}} %
\pgfusepath{clip}%
\pgfsetbuttcap%
\pgfsetmiterjoin%
\definecolor{currentfill}{rgb}{0.121569,0.466667,0.705882}%
\pgfsetfillcolor{currentfill}%
\pgfsetlinewidth{0.000000pt}%
\definecolor{currentstroke}{rgb}{0.000000,0.000000,0.000000}%
\pgfsetstrokecolor{currentstroke}%
\pgfsetstrokeopacity{0.000000}%
\pgfsetdash{}{0pt}%
\pgfpathmoveto{\pgfqpoint{-23.809001in}{3.490300in}}%
\pgfpathlineto{\pgfqpoint{13.076249in}{3.490300in}}%
\pgfpathlineto{\pgfqpoint{13.076249in}{3.497250in}}%
\pgfpathlineto{\pgfqpoint{-23.809001in}{3.497250in}}%
\pgfpathclose%
\pgfusepath{fill}%
\end{pgfscope}%
\begin{pgfscope}%
\pgfpathrectangle{\pgfqpoint{12.211765in}{3.134211in}}{\pgfqpoint{2.188235in}{0.953684in}} %
\pgfusepath{clip}%
\pgfsetbuttcap%
\pgfsetmiterjoin%
\definecolor{currentfill}{rgb}{0.121569,0.466667,0.705882}%
\pgfsetfillcolor{currentfill}%
\pgfsetlinewidth{0.000000pt}%
\definecolor{currentstroke}{rgb}{0.000000,0.000000,0.000000}%
\pgfsetstrokecolor{currentstroke}%
\pgfsetstrokeopacity{0.000000}%
\pgfsetdash{}{0pt}%
\pgfpathmoveto{\pgfqpoint{-23.809001in}{3.498987in}}%
\pgfpathlineto{\pgfqpoint{13.037738in}{3.498987in}}%
\pgfpathlineto{\pgfqpoint{13.037738in}{3.505937in}}%
\pgfpathlineto{\pgfqpoint{-23.809001in}{3.505937in}}%
\pgfpathclose%
\pgfusepath{fill}%
\end{pgfscope}%
\begin{pgfscope}%
\pgfpathrectangle{\pgfqpoint{12.211765in}{3.134211in}}{\pgfqpoint{2.188235in}{0.953684in}} %
\pgfusepath{clip}%
\pgfsetbuttcap%
\pgfsetmiterjoin%
\definecolor{currentfill}{rgb}{0.121569,0.466667,0.705882}%
\pgfsetfillcolor{currentfill}%
\pgfsetlinewidth{0.000000pt}%
\definecolor{currentstroke}{rgb}{0.000000,0.000000,0.000000}%
\pgfsetstrokecolor{currentstroke}%
\pgfsetstrokeopacity{0.000000}%
\pgfsetdash{}{0pt}%
\pgfpathmoveto{\pgfqpoint{-23.809001in}{3.507675in}}%
\pgfpathlineto{\pgfqpoint{13.043971in}{3.507675in}}%
\pgfpathlineto{\pgfqpoint{13.043971in}{3.514624in}}%
\pgfpathlineto{\pgfqpoint{-23.809001in}{3.514624in}}%
\pgfpathclose%
\pgfusepath{fill}%
\end{pgfscope}%
\begin{pgfscope}%
\pgfpathrectangle{\pgfqpoint{12.211765in}{3.134211in}}{\pgfqpoint{2.188235in}{0.953684in}} %
\pgfusepath{clip}%
\pgfsetbuttcap%
\pgfsetmiterjoin%
\definecolor{currentfill}{rgb}{0.121569,0.466667,0.705882}%
\pgfsetfillcolor{currentfill}%
\pgfsetlinewidth{0.000000pt}%
\definecolor{currentstroke}{rgb}{0.000000,0.000000,0.000000}%
\pgfsetstrokecolor{currentstroke}%
\pgfsetstrokeopacity{0.000000}%
\pgfsetdash{}{0pt}%
\pgfpathmoveto{\pgfqpoint{-23.809001in}{3.516362in}}%
\pgfpathlineto{\pgfqpoint{13.077208in}{3.516362in}}%
\pgfpathlineto{\pgfqpoint{13.077208in}{3.523312in}}%
\pgfpathlineto{\pgfqpoint{-23.809001in}{3.523312in}}%
\pgfpathclose%
\pgfusepath{fill}%
\end{pgfscope}%
\begin{pgfscope}%
\pgfpathrectangle{\pgfqpoint{12.211765in}{3.134211in}}{\pgfqpoint{2.188235in}{0.953684in}} %
\pgfusepath{clip}%
\pgfsetbuttcap%
\pgfsetmiterjoin%
\definecolor{currentfill}{rgb}{0.121569,0.466667,0.705882}%
\pgfsetfillcolor{currentfill}%
\pgfsetlinewidth{0.000000pt}%
\definecolor{currentstroke}{rgb}{0.000000,0.000000,0.000000}%
\pgfsetstrokecolor{currentstroke}%
\pgfsetstrokeopacity{0.000000}%
\pgfsetdash{}{0pt}%
\pgfpathmoveto{\pgfqpoint{-23.809001in}{3.525049in}}%
\pgfpathlineto{\pgfqpoint{13.048010in}{3.525049in}}%
\pgfpathlineto{\pgfqpoint{13.048010in}{3.531999in}}%
\pgfpathlineto{\pgfqpoint{-23.809001in}{3.531999in}}%
\pgfpathclose%
\pgfusepath{fill}%
\end{pgfscope}%
\begin{pgfscope}%
\pgfpathrectangle{\pgfqpoint{12.211765in}{3.134211in}}{\pgfqpoint{2.188235in}{0.953684in}} %
\pgfusepath{clip}%
\pgfsetbuttcap%
\pgfsetmiterjoin%
\definecolor{currentfill}{rgb}{0.121569,0.466667,0.705882}%
\pgfsetfillcolor{currentfill}%
\pgfsetlinewidth{0.000000pt}%
\definecolor{currentstroke}{rgb}{0.000000,0.000000,0.000000}%
\pgfsetstrokecolor{currentstroke}%
\pgfsetstrokeopacity{0.000000}%
\pgfsetdash{}{0pt}%
\pgfpathmoveto{\pgfqpoint{-23.809001in}{3.533736in}}%
\pgfpathlineto{\pgfqpoint{13.054778in}{3.533736in}}%
\pgfpathlineto{\pgfqpoint{13.054778in}{3.540686in}}%
\pgfpathlineto{\pgfqpoint{-23.809001in}{3.540686in}}%
\pgfpathclose%
\pgfusepath{fill}%
\end{pgfscope}%
\begin{pgfscope}%
\pgfpathrectangle{\pgfqpoint{12.211765in}{3.134211in}}{\pgfqpoint{2.188235in}{0.953684in}} %
\pgfusepath{clip}%
\pgfsetbuttcap%
\pgfsetmiterjoin%
\definecolor{currentfill}{rgb}{0.121569,0.466667,0.705882}%
\pgfsetfillcolor{currentfill}%
\pgfsetlinewidth{0.000000pt}%
\definecolor{currentstroke}{rgb}{0.000000,0.000000,0.000000}%
\pgfsetstrokecolor{currentstroke}%
\pgfsetstrokeopacity{0.000000}%
\pgfsetdash{}{0pt}%
\pgfpathmoveto{\pgfqpoint{-23.809001in}{3.542424in}}%
\pgfpathlineto{\pgfqpoint{12.914659in}{3.542424in}}%
\pgfpathlineto{\pgfqpoint{12.914659in}{3.549373in}}%
\pgfpathlineto{\pgfqpoint{-23.809001in}{3.549373in}}%
\pgfpathclose%
\pgfusepath{fill}%
\end{pgfscope}%
\begin{pgfscope}%
\pgfpathrectangle{\pgfqpoint{12.211765in}{3.134211in}}{\pgfqpoint{2.188235in}{0.953684in}} %
\pgfusepath{clip}%
\pgfsetbuttcap%
\pgfsetmiterjoin%
\definecolor{currentfill}{rgb}{0.121569,0.466667,0.705882}%
\pgfsetfillcolor{currentfill}%
\pgfsetlinewidth{0.000000pt}%
\definecolor{currentstroke}{rgb}{0.000000,0.000000,0.000000}%
\pgfsetstrokecolor{currentstroke}%
\pgfsetstrokeopacity{0.000000}%
\pgfsetdash{}{0pt}%
\pgfpathmoveto{\pgfqpoint{-23.809001in}{3.551111in}}%
\pgfpathlineto{\pgfqpoint{13.085852in}{3.551111in}}%
\pgfpathlineto{\pgfqpoint{13.085852in}{3.558061in}}%
\pgfpathlineto{\pgfqpoint{-23.809001in}{3.558061in}}%
\pgfpathclose%
\pgfusepath{fill}%
\end{pgfscope}%
\begin{pgfscope}%
\pgfpathrectangle{\pgfqpoint{12.211765in}{3.134211in}}{\pgfqpoint{2.188235in}{0.953684in}} %
\pgfusepath{clip}%
\pgfsetbuttcap%
\pgfsetmiterjoin%
\definecolor{currentfill}{rgb}{0.121569,0.466667,0.705882}%
\pgfsetfillcolor{currentfill}%
\pgfsetlinewidth{0.000000pt}%
\definecolor{currentstroke}{rgb}{0.000000,0.000000,0.000000}%
\pgfsetstrokecolor{currentstroke}%
\pgfsetstrokeopacity{0.000000}%
\pgfsetdash{}{0pt}%
\pgfpathmoveto{\pgfqpoint{-23.809001in}{3.559798in}}%
\pgfpathlineto{\pgfqpoint{13.107822in}{3.559798in}}%
\pgfpathlineto{\pgfqpoint{13.107822in}{3.566748in}}%
\pgfpathlineto{\pgfqpoint{-23.809001in}{3.566748in}}%
\pgfpathclose%
\pgfusepath{fill}%
\end{pgfscope}%
\begin{pgfscope}%
\pgfpathrectangle{\pgfqpoint{12.211765in}{3.134211in}}{\pgfqpoint{2.188235in}{0.953684in}} %
\pgfusepath{clip}%
\pgfsetbuttcap%
\pgfsetmiterjoin%
\definecolor{currentfill}{rgb}{0.121569,0.466667,0.705882}%
\pgfsetfillcolor{currentfill}%
\pgfsetlinewidth{0.000000pt}%
\definecolor{currentstroke}{rgb}{0.000000,0.000000,0.000000}%
\pgfsetstrokecolor{currentstroke}%
\pgfsetstrokeopacity{0.000000}%
\pgfsetdash{}{0pt}%
\pgfpathmoveto{\pgfqpoint{-23.809001in}{3.568485in}}%
\pgfpathlineto{\pgfqpoint{13.050483in}{3.568485in}}%
\pgfpathlineto{\pgfqpoint{13.050483in}{3.575435in}}%
\pgfpathlineto{\pgfqpoint{-23.809001in}{3.575435in}}%
\pgfpathclose%
\pgfusepath{fill}%
\end{pgfscope}%
\begin{pgfscope}%
\pgfpathrectangle{\pgfqpoint{12.211765in}{3.134211in}}{\pgfqpoint{2.188235in}{0.953684in}} %
\pgfusepath{clip}%
\pgfsetbuttcap%
\pgfsetmiterjoin%
\definecolor{currentfill}{rgb}{0.121569,0.466667,0.705882}%
\pgfsetfillcolor{currentfill}%
\pgfsetlinewidth{0.000000pt}%
\definecolor{currentstroke}{rgb}{0.000000,0.000000,0.000000}%
\pgfsetstrokecolor{currentstroke}%
\pgfsetstrokeopacity{0.000000}%
\pgfsetdash{}{0pt}%
\pgfpathmoveto{\pgfqpoint{-23.809001in}{3.577172in}}%
\pgfpathlineto{\pgfqpoint{12.946618in}{3.577172in}}%
\pgfpathlineto{\pgfqpoint{12.946618in}{3.584122in}}%
\pgfpathlineto{\pgfqpoint{-23.809001in}{3.584122in}}%
\pgfpathclose%
\pgfusepath{fill}%
\end{pgfscope}%
\begin{pgfscope}%
\pgfpathrectangle{\pgfqpoint{12.211765in}{3.134211in}}{\pgfqpoint{2.188235in}{0.953684in}} %
\pgfusepath{clip}%
\pgfsetbuttcap%
\pgfsetmiterjoin%
\definecolor{currentfill}{rgb}{0.121569,0.466667,0.705882}%
\pgfsetfillcolor{currentfill}%
\pgfsetlinewidth{0.000000pt}%
\definecolor{currentstroke}{rgb}{0.000000,0.000000,0.000000}%
\pgfsetstrokecolor{currentstroke}%
\pgfsetstrokeopacity{0.000000}%
\pgfsetdash{}{0pt}%
\pgfpathmoveto{\pgfqpoint{-23.809001in}{3.585860in}}%
\pgfpathlineto{\pgfqpoint{13.131538in}{3.585860in}}%
\pgfpathlineto{\pgfqpoint{13.131538in}{3.592809in}}%
\pgfpathlineto{\pgfqpoint{-23.809001in}{3.592809in}}%
\pgfpathclose%
\pgfusepath{fill}%
\end{pgfscope}%
\begin{pgfscope}%
\pgfpathrectangle{\pgfqpoint{12.211765in}{3.134211in}}{\pgfqpoint{2.188235in}{0.953684in}} %
\pgfusepath{clip}%
\pgfsetbuttcap%
\pgfsetmiterjoin%
\definecolor{currentfill}{rgb}{0.121569,0.466667,0.705882}%
\pgfsetfillcolor{currentfill}%
\pgfsetlinewidth{0.000000pt}%
\definecolor{currentstroke}{rgb}{0.000000,0.000000,0.000000}%
\pgfsetstrokecolor{currentstroke}%
\pgfsetstrokeopacity{0.000000}%
\pgfsetdash{}{0pt}%
\pgfpathmoveto{\pgfqpoint{-23.809001in}{3.594547in}}%
\pgfpathlineto{\pgfqpoint{13.037042in}{3.594547in}}%
\pgfpathlineto{\pgfqpoint{13.037042in}{3.601497in}}%
\pgfpathlineto{\pgfqpoint{-23.809001in}{3.601497in}}%
\pgfpathclose%
\pgfusepath{fill}%
\end{pgfscope}%
\begin{pgfscope}%
\pgfpathrectangle{\pgfqpoint{12.211765in}{3.134211in}}{\pgfqpoint{2.188235in}{0.953684in}} %
\pgfusepath{clip}%
\pgfsetbuttcap%
\pgfsetmiterjoin%
\definecolor{currentfill}{rgb}{0.121569,0.466667,0.705882}%
\pgfsetfillcolor{currentfill}%
\pgfsetlinewidth{0.000000pt}%
\definecolor{currentstroke}{rgb}{0.000000,0.000000,0.000000}%
\pgfsetstrokecolor{currentstroke}%
\pgfsetstrokeopacity{0.000000}%
\pgfsetdash{}{0pt}%
\pgfpathmoveto{\pgfqpoint{-23.809001in}{3.603234in}}%
\pgfpathlineto{\pgfqpoint{13.112211in}{3.603234in}}%
\pgfpathlineto{\pgfqpoint{13.112211in}{3.610184in}}%
\pgfpathlineto{\pgfqpoint{-23.809001in}{3.610184in}}%
\pgfpathclose%
\pgfusepath{fill}%
\end{pgfscope}%
\begin{pgfscope}%
\pgfpathrectangle{\pgfqpoint{12.211765in}{3.134211in}}{\pgfqpoint{2.188235in}{0.953684in}} %
\pgfusepath{clip}%
\pgfsetbuttcap%
\pgfsetmiterjoin%
\definecolor{currentfill}{rgb}{0.121569,0.466667,0.705882}%
\pgfsetfillcolor{currentfill}%
\pgfsetlinewidth{0.000000pt}%
\definecolor{currentstroke}{rgb}{0.000000,0.000000,0.000000}%
\pgfsetstrokecolor{currentstroke}%
\pgfsetstrokeopacity{0.000000}%
\pgfsetdash{}{0pt}%
\pgfpathmoveto{\pgfqpoint{-23.809001in}{3.611921in}}%
\pgfpathlineto{\pgfqpoint{13.068983in}{3.611921in}}%
\pgfpathlineto{\pgfqpoint{13.068983in}{3.618871in}}%
\pgfpathlineto{\pgfqpoint{-23.809001in}{3.618871in}}%
\pgfpathclose%
\pgfusepath{fill}%
\end{pgfscope}%
\begin{pgfscope}%
\pgfpathrectangle{\pgfqpoint{12.211765in}{3.134211in}}{\pgfqpoint{2.188235in}{0.953684in}} %
\pgfusepath{clip}%
\pgfsetbuttcap%
\pgfsetmiterjoin%
\definecolor{currentfill}{rgb}{0.121569,0.466667,0.705882}%
\pgfsetfillcolor{currentfill}%
\pgfsetlinewidth{0.000000pt}%
\definecolor{currentstroke}{rgb}{0.000000,0.000000,0.000000}%
\pgfsetstrokecolor{currentstroke}%
\pgfsetstrokeopacity{0.000000}%
\pgfsetdash{}{0pt}%
\pgfpathmoveto{\pgfqpoint{-23.809001in}{3.620609in}}%
\pgfpathlineto{\pgfqpoint{13.085466in}{3.620609in}}%
\pgfpathlineto{\pgfqpoint{13.085466in}{3.627558in}}%
\pgfpathlineto{\pgfqpoint{-23.809001in}{3.627558in}}%
\pgfpathclose%
\pgfusepath{fill}%
\end{pgfscope}%
\begin{pgfscope}%
\pgfpathrectangle{\pgfqpoint{12.211765in}{3.134211in}}{\pgfqpoint{2.188235in}{0.953684in}} %
\pgfusepath{clip}%
\pgfsetbuttcap%
\pgfsetmiterjoin%
\definecolor{currentfill}{rgb}{0.121569,0.466667,0.705882}%
\pgfsetfillcolor{currentfill}%
\pgfsetlinewidth{0.000000pt}%
\definecolor{currentstroke}{rgb}{0.000000,0.000000,0.000000}%
\pgfsetstrokecolor{currentstroke}%
\pgfsetstrokeopacity{0.000000}%
\pgfsetdash{}{0pt}%
\pgfpathmoveto{\pgfqpoint{-23.809001in}{3.629296in}}%
\pgfpathlineto{\pgfqpoint{12.804110in}{3.629296in}}%
\pgfpathlineto{\pgfqpoint{12.804110in}{3.636246in}}%
\pgfpathlineto{\pgfqpoint{-23.809001in}{3.636246in}}%
\pgfpathclose%
\pgfusepath{fill}%
\end{pgfscope}%
\begin{pgfscope}%
\pgfpathrectangle{\pgfqpoint{12.211765in}{3.134211in}}{\pgfqpoint{2.188235in}{0.953684in}} %
\pgfusepath{clip}%
\pgfsetbuttcap%
\pgfsetmiterjoin%
\definecolor{currentfill}{rgb}{0.121569,0.466667,0.705882}%
\pgfsetfillcolor{currentfill}%
\pgfsetlinewidth{0.000000pt}%
\definecolor{currentstroke}{rgb}{0.000000,0.000000,0.000000}%
\pgfsetstrokecolor{currentstroke}%
\pgfsetstrokeopacity{0.000000}%
\pgfsetdash{}{0pt}%
\pgfpathmoveto{\pgfqpoint{-23.809001in}{3.637983in}}%
\pgfpathlineto{\pgfqpoint{12.971983in}{3.637983in}}%
\pgfpathlineto{\pgfqpoint{12.971983in}{3.644933in}}%
\pgfpathlineto{\pgfqpoint{-23.809001in}{3.644933in}}%
\pgfpathclose%
\pgfusepath{fill}%
\end{pgfscope}%
\begin{pgfscope}%
\pgfpathrectangle{\pgfqpoint{12.211765in}{3.134211in}}{\pgfqpoint{2.188235in}{0.953684in}} %
\pgfusepath{clip}%
\pgfsetbuttcap%
\pgfsetmiterjoin%
\definecolor{currentfill}{rgb}{0.121569,0.466667,0.705882}%
\pgfsetfillcolor{currentfill}%
\pgfsetlinewidth{0.000000pt}%
\definecolor{currentstroke}{rgb}{0.000000,0.000000,0.000000}%
\pgfsetstrokecolor{currentstroke}%
\pgfsetstrokeopacity{0.000000}%
\pgfsetdash{}{0pt}%
\pgfpathmoveto{\pgfqpoint{-23.809001in}{3.646670in}}%
\pgfpathlineto{\pgfqpoint{12.992147in}{3.646670in}}%
\pgfpathlineto{\pgfqpoint{12.992147in}{3.653620in}}%
\pgfpathlineto{\pgfqpoint{-23.809001in}{3.653620in}}%
\pgfpathclose%
\pgfusepath{fill}%
\end{pgfscope}%
\begin{pgfscope}%
\pgfpathrectangle{\pgfqpoint{12.211765in}{3.134211in}}{\pgfqpoint{2.188235in}{0.953684in}} %
\pgfusepath{clip}%
\pgfsetbuttcap%
\pgfsetmiterjoin%
\definecolor{currentfill}{rgb}{0.121569,0.466667,0.705882}%
\pgfsetfillcolor{currentfill}%
\pgfsetlinewidth{0.000000pt}%
\definecolor{currentstroke}{rgb}{0.000000,0.000000,0.000000}%
\pgfsetstrokecolor{currentstroke}%
\pgfsetstrokeopacity{0.000000}%
\pgfsetdash{}{0pt}%
\pgfpathmoveto{\pgfqpoint{-23.809001in}{3.655358in}}%
\pgfpathlineto{\pgfqpoint{13.037624in}{3.655358in}}%
\pgfpathlineto{\pgfqpoint{13.037624in}{3.662307in}}%
\pgfpathlineto{\pgfqpoint{-23.809001in}{3.662307in}}%
\pgfpathclose%
\pgfusepath{fill}%
\end{pgfscope}%
\begin{pgfscope}%
\pgfpathrectangle{\pgfqpoint{12.211765in}{3.134211in}}{\pgfqpoint{2.188235in}{0.953684in}} %
\pgfusepath{clip}%
\pgfsetbuttcap%
\pgfsetmiterjoin%
\definecolor{currentfill}{rgb}{0.121569,0.466667,0.705882}%
\pgfsetfillcolor{currentfill}%
\pgfsetlinewidth{0.000000pt}%
\definecolor{currentstroke}{rgb}{0.000000,0.000000,0.000000}%
\pgfsetstrokecolor{currentstroke}%
\pgfsetstrokeopacity{0.000000}%
\pgfsetdash{}{0pt}%
\pgfpathmoveto{\pgfqpoint{-23.809001in}{3.664045in}}%
\pgfpathlineto{\pgfqpoint{13.031639in}{3.664045in}}%
\pgfpathlineto{\pgfqpoint{13.031639in}{3.670995in}}%
\pgfpathlineto{\pgfqpoint{-23.809001in}{3.670995in}}%
\pgfpathclose%
\pgfusepath{fill}%
\end{pgfscope}%
\begin{pgfscope}%
\pgfpathrectangle{\pgfqpoint{12.211765in}{3.134211in}}{\pgfqpoint{2.188235in}{0.953684in}} %
\pgfusepath{clip}%
\pgfsetbuttcap%
\pgfsetmiterjoin%
\definecolor{currentfill}{rgb}{0.121569,0.466667,0.705882}%
\pgfsetfillcolor{currentfill}%
\pgfsetlinewidth{0.000000pt}%
\definecolor{currentstroke}{rgb}{0.000000,0.000000,0.000000}%
\pgfsetstrokecolor{currentstroke}%
\pgfsetstrokeopacity{0.000000}%
\pgfsetdash{}{0pt}%
\pgfpathmoveto{\pgfqpoint{-23.809001in}{3.672732in}}%
\pgfpathlineto{\pgfqpoint{13.039293in}{3.672732in}}%
\pgfpathlineto{\pgfqpoint{13.039293in}{3.679682in}}%
\pgfpathlineto{\pgfqpoint{-23.809001in}{3.679682in}}%
\pgfpathclose%
\pgfusepath{fill}%
\end{pgfscope}%
\begin{pgfscope}%
\pgfpathrectangle{\pgfqpoint{12.211765in}{3.134211in}}{\pgfqpoint{2.188235in}{0.953684in}} %
\pgfusepath{clip}%
\pgfsetbuttcap%
\pgfsetmiterjoin%
\definecolor{currentfill}{rgb}{0.121569,0.466667,0.705882}%
\pgfsetfillcolor{currentfill}%
\pgfsetlinewidth{0.000000pt}%
\definecolor{currentstroke}{rgb}{0.000000,0.000000,0.000000}%
\pgfsetstrokecolor{currentstroke}%
\pgfsetstrokeopacity{0.000000}%
\pgfsetdash{}{0pt}%
\pgfpathmoveto{\pgfqpoint{-23.809001in}{3.681419in}}%
\pgfpathlineto{\pgfqpoint{13.062392in}{3.681419in}}%
\pgfpathlineto{\pgfqpoint{13.062392in}{3.688369in}}%
\pgfpathlineto{\pgfqpoint{-23.809001in}{3.688369in}}%
\pgfpathclose%
\pgfusepath{fill}%
\end{pgfscope}%
\begin{pgfscope}%
\pgfpathrectangle{\pgfqpoint{12.211765in}{3.134211in}}{\pgfqpoint{2.188235in}{0.953684in}} %
\pgfusepath{clip}%
\pgfsetbuttcap%
\pgfsetmiterjoin%
\definecolor{currentfill}{rgb}{0.121569,0.466667,0.705882}%
\pgfsetfillcolor{currentfill}%
\pgfsetlinewidth{0.000000pt}%
\definecolor{currentstroke}{rgb}{0.000000,0.000000,0.000000}%
\pgfsetstrokecolor{currentstroke}%
\pgfsetstrokeopacity{0.000000}%
\pgfsetdash{}{0pt}%
\pgfpathmoveto{\pgfqpoint{-23.809001in}{3.690106in}}%
\pgfpathlineto{\pgfqpoint{12.975843in}{3.690106in}}%
\pgfpathlineto{\pgfqpoint{12.975843in}{3.697056in}}%
\pgfpathlineto{\pgfqpoint{-23.809001in}{3.697056in}}%
\pgfpathclose%
\pgfusepath{fill}%
\end{pgfscope}%
\begin{pgfscope}%
\pgfpathrectangle{\pgfqpoint{12.211765in}{3.134211in}}{\pgfqpoint{2.188235in}{0.953684in}} %
\pgfusepath{clip}%
\pgfsetbuttcap%
\pgfsetmiterjoin%
\definecolor{currentfill}{rgb}{0.121569,0.466667,0.705882}%
\pgfsetfillcolor{currentfill}%
\pgfsetlinewidth{0.000000pt}%
\definecolor{currentstroke}{rgb}{0.000000,0.000000,0.000000}%
\pgfsetstrokecolor{currentstroke}%
\pgfsetstrokeopacity{0.000000}%
\pgfsetdash{}{0pt}%
\pgfpathmoveto{\pgfqpoint{-23.809001in}{3.698794in}}%
\pgfpathlineto{\pgfqpoint{13.018618in}{3.698794in}}%
\pgfpathlineto{\pgfqpoint{13.018618in}{3.705743in}}%
\pgfpathlineto{\pgfqpoint{-23.809001in}{3.705743in}}%
\pgfpathclose%
\pgfusepath{fill}%
\end{pgfscope}%
\begin{pgfscope}%
\pgfpathrectangle{\pgfqpoint{12.211765in}{3.134211in}}{\pgfqpoint{2.188235in}{0.953684in}} %
\pgfusepath{clip}%
\pgfsetbuttcap%
\pgfsetmiterjoin%
\definecolor{currentfill}{rgb}{0.121569,0.466667,0.705882}%
\pgfsetfillcolor{currentfill}%
\pgfsetlinewidth{0.000000pt}%
\definecolor{currentstroke}{rgb}{0.000000,0.000000,0.000000}%
\pgfsetstrokecolor{currentstroke}%
\pgfsetstrokeopacity{0.000000}%
\pgfsetdash{}{0pt}%
\pgfpathmoveto{\pgfqpoint{-23.809001in}{3.707481in}}%
\pgfpathlineto{\pgfqpoint{13.006535in}{3.707481in}}%
\pgfpathlineto{\pgfqpoint{13.006535in}{3.714431in}}%
\pgfpathlineto{\pgfqpoint{-23.809001in}{3.714431in}}%
\pgfpathclose%
\pgfusepath{fill}%
\end{pgfscope}%
\begin{pgfscope}%
\pgfpathrectangle{\pgfqpoint{12.211765in}{3.134211in}}{\pgfqpoint{2.188235in}{0.953684in}} %
\pgfusepath{clip}%
\pgfsetbuttcap%
\pgfsetmiterjoin%
\definecolor{currentfill}{rgb}{0.121569,0.466667,0.705882}%
\pgfsetfillcolor{currentfill}%
\pgfsetlinewidth{0.000000pt}%
\definecolor{currentstroke}{rgb}{0.000000,0.000000,0.000000}%
\pgfsetstrokecolor{currentstroke}%
\pgfsetstrokeopacity{0.000000}%
\pgfsetdash{}{0pt}%
\pgfpathmoveto{\pgfqpoint{-23.809001in}{3.716168in}}%
\pgfpathlineto{\pgfqpoint{13.007937in}{3.716168in}}%
\pgfpathlineto{\pgfqpoint{13.007937in}{3.723118in}}%
\pgfpathlineto{\pgfqpoint{-23.809001in}{3.723118in}}%
\pgfpathclose%
\pgfusepath{fill}%
\end{pgfscope}%
\begin{pgfscope}%
\pgfpathrectangle{\pgfqpoint{12.211765in}{3.134211in}}{\pgfqpoint{2.188235in}{0.953684in}} %
\pgfusepath{clip}%
\pgfsetbuttcap%
\pgfsetmiterjoin%
\definecolor{currentfill}{rgb}{0.121569,0.466667,0.705882}%
\pgfsetfillcolor{currentfill}%
\pgfsetlinewidth{0.000000pt}%
\definecolor{currentstroke}{rgb}{0.000000,0.000000,0.000000}%
\pgfsetstrokecolor{currentstroke}%
\pgfsetstrokeopacity{0.000000}%
\pgfsetdash{}{0pt}%
\pgfpathmoveto{\pgfqpoint{-23.809001in}{3.724855in}}%
\pgfpathlineto{\pgfqpoint{12.976364in}{3.724855in}}%
\pgfpathlineto{\pgfqpoint{12.976364in}{3.731805in}}%
\pgfpathlineto{\pgfqpoint{-23.809001in}{3.731805in}}%
\pgfpathclose%
\pgfusepath{fill}%
\end{pgfscope}%
\begin{pgfscope}%
\pgfpathrectangle{\pgfqpoint{12.211765in}{3.134211in}}{\pgfqpoint{2.188235in}{0.953684in}} %
\pgfusepath{clip}%
\pgfsetbuttcap%
\pgfsetmiterjoin%
\definecolor{currentfill}{rgb}{0.121569,0.466667,0.705882}%
\pgfsetfillcolor{currentfill}%
\pgfsetlinewidth{0.000000pt}%
\definecolor{currentstroke}{rgb}{0.000000,0.000000,0.000000}%
\pgfsetstrokecolor{currentstroke}%
\pgfsetstrokeopacity{0.000000}%
\pgfsetdash{}{0pt}%
\pgfpathmoveto{\pgfqpoint{-23.809001in}{3.733543in}}%
\pgfpathlineto{\pgfqpoint{12.962264in}{3.733543in}}%
\pgfpathlineto{\pgfqpoint{12.962264in}{3.740492in}}%
\pgfpathlineto{\pgfqpoint{-23.809001in}{3.740492in}}%
\pgfpathclose%
\pgfusepath{fill}%
\end{pgfscope}%
\begin{pgfscope}%
\pgfpathrectangle{\pgfqpoint{12.211765in}{3.134211in}}{\pgfqpoint{2.188235in}{0.953684in}} %
\pgfusepath{clip}%
\pgfsetbuttcap%
\pgfsetmiterjoin%
\definecolor{currentfill}{rgb}{0.121569,0.466667,0.705882}%
\pgfsetfillcolor{currentfill}%
\pgfsetlinewidth{0.000000pt}%
\definecolor{currentstroke}{rgb}{0.000000,0.000000,0.000000}%
\pgfsetstrokecolor{currentstroke}%
\pgfsetstrokeopacity{0.000000}%
\pgfsetdash{}{0pt}%
\pgfpathmoveto{\pgfqpoint{-23.809001in}{3.742230in}}%
\pgfpathlineto{\pgfqpoint{13.039515in}{3.742230in}}%
\pgfpathlineto{\pgfqpoint{13.039515in}{3.749180in}}%
\pgfpathlineto{\pgfqpoint{-23.809001in}{3.749180in}}%
\pgfpathclose%
\pgfusepath{fill}%
\end{pgfscope}%
\begin{pgfscope}%
\pgfpathrectangle{\pgfqpoint{12.211765in}{3.134211in}}{\pgfqpoint{2.188235in}{0.953684in}} %
\pgfusepath{clip}%
\pgfsetbuttcap%
\pgfsetmiterjoin%
\definecolor{currentfill}{rgb}{0.121569,0.466667,0.705882}%
\pgfsetfillcolor{currentfill}%
\pgfsetlinewidth{0.000000pt}%
\definecolor{currentstroke}{rgb}{0.000000,0.000000,0.000000}%
\pgfsetstrokecolor{currentstroke}%
\pgfsetstrokeopacity{0.000000}%
\pgfsetdash{}{0pt}%
\pgfpathmoveto{\pgfqpoint{-23.809001in}{3.750917in}}%
\pgfpathlineto{\pgfqpoint{13.070408in}{3.750917in}}%
\pgfpathlineto{\pgfqpoint{13.070408in}{3.757867in}}%
\pgfpathlineto{\pgfqpoint{-23.809001in}{3.757867in}}%
\pgfpathclose%
\pgfusepath{fill}%
\end{pgfscope}%
\begin{pgfscope}%
\pgfpathrectangle{\pgfqpoint{12.211765in}{3.134211in}}{\pgfqpoint{2.188235in}{0.953684in}} %
\pgfusepath{clip}%
\pgfsetbuttcap%
\pgfsetmiterjoin%
\definecolor{currentfill}{rgb}{0.121569,0.466667,0.705882}%
\pgfsetfillcolor{currentfill}%
\pgfsetlinewidth{0.000000pt}%
\definecolor{currentstroke}{rgb}{0.000000,0.000000,0.000000}%
\pgfsetstrokecolor{currentstroke}%
\pgfsetstrokeopacity{0.000000}%
\pgfsetdash{}{0pt}%
\pgfpathmoveto{\pgfqpoint{-23.809001in}{3.759604in}}%
\pgfpathlineto{\pgfqpoint{13.024882in}{3.759604in}}%
\pgfpathlineto{\pgfqpoint{13.024882in}{3.766554in}}%
\pgfpathlineto{\pgfqpoint{-23.809001in}{3.766554in}}%
\pgfpathclose%
\pgfusepath{fill}%
\end{pgfscope}%
\begin{pgfscope}%
\pgfpathrectangle{\pgfqpoint{12.211765in}{3.134211in}}{\pgfqpoint{2.188235in}{0.953684in}} %
\pgfusepath{clip}%
\pgfsetbuttcap%
\pgfsetmiterjoin%
\definecolor{currentfill}{rgb}{0.121569,0.466667,0.705882}%
\pgfsetfillcolor{currentfill}%
\pgfsetlinewidth{0.000000pt}%
\definecolor{currentstroke}{rgb}{0.000000,0.000000,0.000000}%
\pgfsetstrokecolor{currentstroke}%
\pgfsetstrokeopacity{0.000000}%
\pgfsetdash{}{0pt}%
\pgfpathmoveto{\pgfqpoint{-23.809001in}{3.768292in}}%
\pgfpathlineto{\pgfqpoint{12.992556in}{3.768292in}}%
\pgfpathlineto{\pgfqpoint{12.992556in}{3.775241in}}%
\pgfpathlineto{\pgfqpoint{-23.809001in}{3.775241in}}%
\pgfpathclose%
\pgfusepath{fill}%
\end{pgfscope}%
\begin{pgfscope}%
\pgfpathrectangle{\pgfqpoint{12.211765in}{3.134211in}}{\pgfqpoint{2.188235in}{0.953684in}} %
\pgfusepath{clip}%
\pgfsetbuttcap%
\pgfsetmiterjoin%
\definecolor{currentfill}{rgb}{0.121569,0.466667,0.705882}%
\pgfsetfillcolor{currentfill}%
\pgfsetlinewidth{0.000000pt}%
\definecolor{currentstroke}{rgb}{0.000000,0.000000,0.000000}%
\pgfsetstrokecolor{currentstroke}%
\pgfsetstrokeopacity{0.000000}%
\pgfsetdash{}{0pt}%
\pgfpathmoveto{\pgfqpoint{-23.809001in}{3.776979in}}%
\pgfpathlineto{\pgfqpoint{13.027883in}{3.776979in}}%
\pgfpathlineto{\pgfqpoint{13.027883in}{3.783929in}}%
\pgfpathlineto{\pgfqpoint{-23.809001in}{3.783929in}}%
\pgfpathclose%
\pgfusepath{fill}%
\end{pgfscope}%
\begin{pgfscope}%
\pgfpathrectangle{\pgfqpoint{12.211765in}{3.134211in}}{\pgfqpoint{2.188235in}{0.953684in}} %
\pgfusepath{clip}%
\pgfsetbuttcap%
\pgfsetmiterjoin%
\definecolor{currentfill}{rgb}{0.121569,0.466667,0.705882}%
\pgfsetfillcolor{currentfill}%
\pgfsetlinewidth{0.000000pt}%
\definecolor{currentstroke}{rgb}{0.000000,0.000000,0.000000}%
\pgfsetstrokecolor{currentstroke}%
\pgfsetstrokeopacity{0.000000}%
\pgfsetdash{}{0pt}%
\pgfpathmoveto{\pgfqpoint{-23.809001in}{3.785666in}}%
\pgfpathlineto{\pgfqpoint{12.937514in}{3.785666in}}%
\pgfpathlineto{\pgfqpoint{12.937514in}{3.792616in}}%
\pgfpathlineto{\pgfqpoint{-23.809001in}{3.792616in}}%
\pgfpathclose%
\pgfusepath{fill}%
\end{pgfscope}%
\begin{pgfscope}%
\pgfpathrectangle{\pgfqpoint{12.211765in}{3.134211in}}{\pgfqpoint{2.188235in}{0.953684in}} %
\pgfusepath{clip}%
\pgfsetbuttcap%
\pgfsetmiterjoin%
\definecolor{currentfill}{rgb}{0.121569,0.466667,0.705882}%
\pgfsetfillcolor{currentfill}%
\pgfsetlinewidth{0.000000pt}%
\definecolor{currentstroke}{rgb}{0.000000,0.000000,0.000000}%
\pgfsetstrokecolor{currentstroke}%
\pgfsetstrokeopacity{0.000000}%
\pgfsetdash{}{0pt}%
\pgfpathmoveto{\pgfqpoint{-23.809001in}{3.794353in}}%
\pgfpathlineto{\pgfqpoint{13.091406in}{3.794353in}}%
\pgfpathlineto{\pgfqpoint{13.091406in}{3.801303in}}%
\pgfpathlineto{\pgfqpoint{-23.809001in}{3.801303in}}%
\pgfpathclose%
\pgfusepath{fill}%
\end{pgfscope}%
\begin{pgfscope}%
\pgfpathrectangle{\pgfqpoint{12.211765in}{3.134211in}}{\pgfqpoint{2.188235in}{0.953684in}} %
\pgfusepath{clip}%
\pgfsetbuttcap%
\pgfsetmiterjoin%
\definecolor{currentfill}{rgb}{0.121569,0.466667,0.705882}%
\pgfsetfillcolor{currentfill}%
\pgfsetlinewidth{0.000000pt}%
\definecolor{currentstroke}{rgb}{0.000000,0.000000,0.000000}%
\pgfsetstrokecolor{currentstroke}%
\pgfsetstrokeopacity{0.000000}%
\pgfsetdash{}{0pt}%
\pgfpathmoveto{\pgfqpoint{-23.809001in}{3.803040in}}%
\pgfpathlineto{\pgfqpoint{13.050637in}{3.803040in}}%
\pgfpathlineto{\pgfqpoint{13.050637in}{3.809990in}}%
\pgfpathlineto{\pgfqpoint{-23.809001in}{3.809990in}}%
\pgfpathclose%
\pgfusepath{fill}%
\end{pgfscope}%
\begin{pgfscope}%
\pgfpathrectangle{\pgfqpoint{12.211765in}{3.134211in}}{\pgfqpoint{2.188235in}{0.953684in}} %
\pgfusepath{clip}%
\pgfsetbuttcap%
\pgfsetmiterjoin%
\definecolor{currentfill}{rgb}{0.121569,0.466667,0.705882}%
\pgfsetfillcolor{currentfill}%
\pgfsetlinewidth{0.000000pt}%
\definecolor{currentstroke}{rgb}{0.000000,0.000000,0.000000}%
\pgfsetstrokecolor{currentstroke}%
\pgfsetstrokeopacity{0.000000}%
\pgfsetdash{}{0pt}%
\pgfpathmoveto{\pgfqpoint{-23.809001in}{3.811728in}}%
\pgfpathlineto{\pgfqpoint{13.021490in}{3.811728in}}%
\pgfpathlineto{\pgfqpoint{13.021490in}{3.818677in}}%
\pgfpathlineto{\pgfqpoint{-23.809001in}{3.818677in}}%
\pgfpathclose%
\pgfusepath{fill}%
\end{pgfscope}%
\begin{pgfscope}%
\pgfpathrectangle{\pgfqpoint{12.211765in}{3.134211in}}{\pgfqpoint{2.188235in}{0.953684in}} %
\pgfusepath{clip}%
\pgfsetbuttcap%
\pgfsetmiterjoin%
\definecolor{currentfill}{rgb}{0.121569,0.466667,0.705882}%
\pgfsetfillcolor{currentfill}%
\pgfsetlinewidth{0.000000pt}%
\definecolor{currentstroke}{rgb}{0.000000,0.000000,0.000000}%
\pgfsetstrokecolor{currentstroke}%
\pgfsetstrokeopacity{0.000000}%
\pgfsetdash{}{0pt}%
\pgfpathmoveto{\pgfqpoint{-23.809001in}{3.820415in}}%
\pgfpathlineto{\pgfqpoint{12.931564in}{3.820415in}}%
\pgfpathlineto{\pgfqpoint{12.931564in}{3.827365in}}%
\pgfpathlineto{\pgfqpoint{-23.809001in}{3.827365in}}%
\pgfpathclose%
\pgfusepath{fill}%
\end{pgfscope}%
\begin{pgfscope}%
\pgfpathrectangle{\pgfqpoint{12.211765in}{3.134211in}}{\pgfqpoint{2.188235in}{0.953684in}} %
\pgfusepath{clip}%
\pgfsetbuttcap%
\pgfsetmiterjoin%
\definecolor{currentfill}{rgb}{0.121569,0.466667,0.705882}%
\pgfsetfillcolor{currentfill}%
\pgfsetlinewidth{0.000000pt}%
\definecolor{currentstroke}{rgb}{0.000000,0.000000,0.000000}%
\pgfsetstrokecolor{currentstroke}%
\pgfsetstrokeopacity{0.000000}%
\pgfsetdash{}{0pt}%
\pgfpathmoveto{\pgfqpoint{-23.809001in}{3.829102in}}%
\pgfpathlineto{\pgfqpoint{13.030397in}{3.829102in}}%
\pgfpathlineto{\pgfqpoint{13.030397in}{3.836052in}}%
\pgfpathlineto{\pgfqpoint{-23.809001in}{3.836052in}}%
\pgfpathclose%
\pgfusepath{fill}%
\end{pgfscope}%
\begin{pgfscope}%
\pgfpathrectangle{\pgfqpoint{12.211765in}{3.134211in}}{\pgfqpoint{2.188235in}{0.953684in}} %
\pgfusepath{clip}%
\pgfsetbuttcap%
\pgfsetmiterjoin%
\definecolor{currentfill}{rgb}{0.121569,0.466667,0.705882}%
\pgfsetfillcolor{currentfill}%
\pgfsetlinewidth{0.000000pt}%
\definecolor{currentstroke}{rgb}{0.000000,0.000000,0.000000}%
\pgfsetstrokecolor{currentstroke}%
\pgfsetstrokeopacity{0.000000}%
\pgfsetdash{}{0pt}%
\pgfpathmoveto{\pgfqpoint{-23.809001in}{3.837789in}}%
\pgfpathlineto{\pgfqpoint{12.959212in}{3.837789in}}%
\pgfpathlineto{\pgfqpoint{12.959212in}{3.844739in}}%
\pgfpathlineto{\pgfqpoint{-23.809001in}{3.844739in}}%
\pgfpathclose%
\pgfusepath{fill}%
\end{pgfscope}%
\begin{pgfscope}%
\pgfpathrectangle{\pgfqpoint{12.211765in}{3.134211in}}{\pgfqpoint{2.188235in}{0.953684in}} %
\pgfusepath{clip}%
\pgfsetbuttcap%
\pgfsetmiterjoin%
\definecolor{currentfill}{rgb}{0.121569,0.466667,0.705882}%
\pgfsetfillcolor{currentfill}%
\pgfsetlinewidth{0.000000pt}%
\definecolor{currentstroke}{rgb}{0.000000,0.000000,0.000000}%
\pgfsetstrokecolor{currentstroke}%
\pgfsetstrokeopacity{0.000000}%
\pgfsetdash{}{0pt}%
\pgfpathmoveto{\pgfqpoint{-23.809001in}{3.846477in}}%
\pgfpathlineto{\pgfqpoint{13.025772in}{3.846477in}}%
\pgfpathlineto{\pgfqpoint{13.025772in}{3.853426in}}%
\pgfpathlineto{\pgfqpoint{-23.809001in}{3.853426in}}%
\pgfpathclose%
\pgfusepath{fill}%
\end{pgfscope}%
\begin{pgfscope}%
\pgfpathrectangle{\pgfqpoint{12.211765in}{3.134211in}}{\pgfqpoint{2.188235in}{0.953684in}} %
\pgfusepath{clip}%
\pgfsetbuttcap%
\pgfsetmiterjoin%
\definecolor{currentfill}{rgb}{0.121569,0.466667,0.705882}%
\pgfsetfillcolor{currentfill}%
\pgfsetlinewidth{0.000000pt}%
\definecolor{currentstroke}{rgb}{0.000000,0.000000,0.000000}%
\pgfsetstrokecolor{currentstroke}%
\pgfsetstrokeopacity{0.000000}%
\pgfsetdash{}{0pt}%
\pgfpathmoveto{\pgfqpoint{-23.809001in}{3.855164in}}%
\pgfpathlineto{\pgfqpoint{13.037485in}{3.855164in}}%
\pgfpathlineto{\pgfqpoint{13.037485in}{3.862114in}}%
\pgfpathlineto{\pgfqpoint{-23.809001in}{3.862114in}}%
\pgfpathclose%
\pgfusepath{fill}%
\end{pgfscope}%
\begin{pgfscope}%
\pgfpathrectangle{\pgfqpoint{12.211765in}{3.134211in}}{\pgfqpoint{2.188235in}{0.953684in}} %
\pgfusepath{clip}%
\pgfsetbuttcap%
\pgfsetmiterjoin%
\definecolor{currentfill}{rgb}{0.121569,0.466667,0.705882}%
\pgfsetfillcolor{currentfill}%
\pgfsetlinewidth{0.000000pt}%
\definecolor{currentstroke}{rgb}{0.000000,0.000000,0.000000}%
\pgfsetstrokecolor{currentstroke}%
\pgfsetstrokeopacity{0.000000}%
\pgfsetdash{}{0pt}%
\pgfpathmoveto{\pgfqpoint{-23.809001in}{3.863851in}}%
\pgfpathlineto{\pgfqpoint{13.002774in}{3.863851in}}%
\pgfpathlineto{\pgfqpoint{13.002774in}{3.870801in}}%
\pgfpathlineto{\pgfqpoint{-23.809001in}{3.870801in}}%
\pgfpathclose%
\pgfusepath{fill}%
\end{pgfscope}%
\begin{pgfscope}%
\pgfpathrectangle{\pgfqpoint{12.211765in}{3.134211in}}{\pgfqpoint{2.188235in}{0.953684in}} %
\pgfusepath{clip}%
\pgfsetbuttcap%
\pgfsetmiterjoin%
\definecolor{currentfill}{rgb}{0.121569,0.466667,0.705882}%
\pgfsetfillcolor{currentfill}%
\pgfsetlinewidth{0.000000pt}%
\definecolor{currentstroke}{rgb}{0.000000,0.000000,0.000000}%
\pgfsetstrokecolor{currentstroke}%
\pgfsetstrokeopacity{0.000000}%
\pgfsetdash{}{0pt}%
\pgfpathmoveto{\pgfqpoint{-23.809001in}{3.872538in}}%
\pgfpathlineto{\pgfqpoint{12.982265in}{3.872538in}}%
\pgfpathlineto{\pgfqpoint{12.982265in}{3.879488in}}%
\pgfpathlineto{\pgfqpoint{-23.809001in}{3.879488in}}%
\pgfpathclose%
\pgfusepath{fill}%
\end{pgfscope}%
\begin{pgfscope}%
\pgfpathrectangle{\pgfqpoint{12.211765in}{3.134211in}}{\pgfqpoint{2.188235in}{0.953684in}} %
\pgfusepath{clip}%
\pgfsetbuttcap%
\pgfsetmiterjoin%
\definecolor{currentfill}{rgb}{0.121569,0.466667,0.705882}%
\pgfsetfillcolor{currentfill}%
\pgfsetlinewidth{0.000000pt}%
\definecolor{currentstroke}{rgb}{0.000000,0.000000,0.000000}%
\pgfsetstrokecolor{currentstroke}%
\pgfsetstrokeopacity{0.000000}%
\pgfsetdash{}{0pt}%
\pgfpathmoveto{\pgfqpoint{-23.809001in}{3.881226in}}%
\pgfpathlineto{\pgfqpoint{13.054450in}{3.881226in}}%
\pgfpathlineto{\pgfqpoint{13.054450in}{3.888175in}}%
\pgfpathlineto{\pgfqpoint{-23.809001in}{3.888175in}}%
\pgfpathclose%
\pgfusepath{fill}%
\end{pgfscope}%
\begin{pgfscope}%
\pgfpathrectangle{\pgfqpoint{12.211765in}{3.134211in}}{\pgfqpoint{2.188235in}{0.953684in}} %
\pgfusepath{clip}%
\pgfsetbuttcap%
\pgfsetmiterjoin%
\definecolor{currentfill}{rgb}{0.121569,0.466667,0.705882}%
\pgfsetfillcolor{currentfill}%
\pgfsetlinewidth{0.000000pt}%
\definecolor{currentstroke}{rgb}{0.000000,0.000000,0.000000}%
\pgfsetstrokecolor{currentstroke}%
\pgfsetstrokeopacity{0.000000}%
\pgfsetdash{}{0pt}%
\pgfpathmoveto{\pgfqpoint{-23.809001in}{3.889913in}}%
\pgfpathlineto{\pgfqpoint{13.028938in}{3.889913in}}%
\pgfpathlineto{\pgfqpoint{13.028938in}{3.896863in}}%
\pgfpathlineto{\pgfqpoint{-23.809001in}{3.896863in}}%
\pgfpathclose%
\pgfusepath{fill}%
\end{pgfscope}%
\begin{pgfscope}%
\pgfpathrectangle{\pgfqpoint{12.211765in}{3.134211in}}{\pgfqpoint{2.188235in}{0.953684in}} %
\pgfusepath{clip}%
\pgfsetbuttcap%
\pgfsetmiterjoin%
\definecolor{currentfill}{rgb}{0.121569,0.466667,0.705882}%
\pgfsetfillcolor{currentfill}%
\pgfsetlinewidth{0.000000pt}%
\definecolor{currentstroke}{rgb}{0.000000,0.000000,0.000000}%
\pgfsetstrokecolor{currentstroke}%
\pgfsetstrokeopacity{0.000000}%
\pgfsetdash{}{0pt}%
\pgfpathmoveto{\pgfqpoint{-23.809001in}{3.898600in}}%
\pgfpathlineto{\pgfqpoint{13.110205in}{3.898600in}}%
\pgfpathlineto{\pgfqpoint{13.110205in}{3.905550in}}%
\pgfpathlineto{\pgfqpoint{-23.809001in}{3.905550in}}%
\pgfpathclose%
\pgfusepath{fill}%
\end{pgfscope}%
\begin{pgfscope}%
\pgfpathrectangle{\pgfqpoint{12.211765in}{3.134211in}}{\pgfqpoint{2.188235in}{0.953684in}} %
\pgfusepath{clip}%
\pgfsetbuttcap%
\pgfsetmiterjoin%
\definecolor{currentfill}{rgb}{0.121569,0.466667,0.705882}%
\pgfsetfillcolor{currentfill}%
\pgfsetlinewidth{0.000000pt}%
\definecolor{currentstroke}{rgb}{0.000000,0.000000,0.000000}%
\pgfsetstrokecolor{currentstroke}%
\pgfsetstrokeopacity{0.000000}%
\pgfsetdash{}{0pt}%
\pgfpathmoveto{\pgfqpoint{-23.809001in}{3.907287in}}%
\pgfpathlineto{\pgfqpoint{13.012233in}{3.907287in}}%
\pgfpathlineto{\pgfqpoint{13.012233in}{3.914237in}}%
\pgfpathlineto{\pgfqpoint{-23.809001in}{3.914237in}}%
\pgfpathclose%
\pgfusepath{fill}%
\end{pgfscope}%
\begin{pgfscope}%
\pgfpathrectangle{\pgfqpoint{12.211765in}{3.134211in}}{\pgfqpoint{2.188235in}{0.953684in}} %
\pgfusepath{clip}%
\pgfsetbuttcap%
\pgfsetmiterjoin%
\definecolor{currentfill}{rgb}{0.121569,0.466667,0.705882}%
\pgfsetfillcolor{currentfill}%
\pgfsetlinewidth{0.000000pt}%
\definecolor{currentstroke}{rgb}{0.000000,0.000000,0.000000}%
\pgfsetstrokecolor{currentstroke}%
\pgfsetstrokeopacity{0.000000}%
\pgfsetdash{}{0pt}%
\pgfpathmoveto{\pgfqpoint{-23.809001in}{3.915974in}}%
\pgfpathlineto{\pgfqpoint{13.080345in}{3.915974in}}%
\pgfpathlineto{\pgfqpoint{13.080345in}{3.922924in}}%
\pgfpathlineto{\pgfqpoint{-23.809001in}{3.922924in}}%
\pgfpathclose%
\pgfusepath{fill}%
\end{pgfscope}%
\begin{pgfscope}%
\pgfpathrectangle{\pgfqpoint{12.211765in}{3.134211in}}{\pgfqpoint{2.188235in}{0.953684in}} %
\pgfusepath{clip}%
\pgfsetbuttcap%
\pgfsetmiterjoin%
\definecolor{currentfill}{rgb}{0.121569,0.466667,0.705882}%
\pgfsetfillcolor{currentfill}%
\pgfsetlinewidth{0.000000pt}%
\definecolor{currentstroke}{rgb}{0.000000,0.000000,0.000000}%
\pgfsetstrokecolor{currentstroke}%
\pgfsetstrokeopacity{0.000000}%
\pgfsetdash{}{0pt}%
\pgfpathmoveto{\pgfqpoint{-23.809001in}{3.924662in}}%
\pgfpathlineto{\pgfqpoint{13.060157in}{3.924662in}}%
\pgfpathlineto{\pgfqpoint{13.060157in}{3.931611in}}%
\pgfpathlineto{\pgfqpoint{-23.809001in}{3.931611in}}%
\pgfpathclose%
\pgfusepath{fill}%
\end{pgfscope}%
\begin{pgfscope}%
\pgfpathrectangle{\pgfqpoint{12.211765in}{3.134211in}}{\pgfqpoint{2.188235in}{0.953684in}} %
\pgfusepath{clip}%
\pgfsetbuttcap%
\pgfsetmiterjoin%
\definecolor{currentfill}{rgb}{0.121569,0.466667,0.705882}%
\pgfsetfillcolor{currentfill}%
\pgfsetlinewidth{0.000000pt}%
\definecolor{currentstroke}{rgb}{0.000000,0.000000,0.000000}%
\pgfsetstrokecolor{currentstroke}%
\pgfsetstrokeopacity{0.000000}%
\pgfsetdash{}{0pt}%
\pgfpathmoveto{\pgfqpoint{-23.809001in}{3.933349in}}%
\pgfpathlineto{\pgfqpoint{12.923399in}{3.933349in}}%
\pgfpathlineto{\pgfqpoint{12.923399in}{3.940299in}}%
\pgfpathlineto{\pgfqpoint{-23.809001in}{3.940299in}}%
\pgfpathclose%
\pgfusepath{fill}%
\end{pgfscope}%
\begin{pgfscope}%
\pgfpathrectangle{\pgfqpoint{12.211765in}{3.134211in}}{\pgfqpoint{2.188235in}{0.953684in}} %
\pgfusepath{clip}%
\pgfsetbuttcap%
\pgfsetmiterjoin%
\definecolor{currentfill}{rgb}{0.121569,0.466667,0.705882}%
\pgfsetfillcolor{currentfill}%
\pgfsetlinewidth{0.000000pt}%
\definecolor{currentstroke}{rgb}{0.000000,0.000000,0.000000}%
\pgfsetstrokecolor{currentstroke}%
\pgfsetstrokeopacity{0.000000}%
\pgfsetdash{}{0pt}%
\pgfpathmoveto{\pgfqpoint{-23.809001in}{3.942036in}}%
\pgfpathlineto{\pgfqpoint{12.982301in}{3.942036in}}%
\pgfpathlineto{\pgfqpoint{12.982301in}{3.948986in}}%
\pgfpathlineto{\pgfqpoint{-23.809001in}{3.948986in}}%
\pgfpathclose%
\pgfusepath{fill}%
\end{pgfscope}%
\begin{pgfscope}%
\pgfpathrectangle{\pgfqpoint{12.211765in}{3.134211in}}{\pgfqpoint{2.188235in}{0.953684in}} %
\pgfusepath{clip}%
\pgfsetbuttcap%
\pgfsetmiterjoin%
\definecolor{currentfill}{rgb}{0.121569,0.466667,0.705882}%
\pgfsetfillcolor{currentfill}%
\pgfsetlinewidth{0.000000pt}%
\definecolor{currentstroke}{rgb}{0.000000,0.000000,0.000000}%
\pgfsetstrokecolor{currentstroke}%
\pgfsetstrokeopacity{0.000000}%
\pgfsetdash{}{0pt}%
\pgfpathmoveto{\pgfqpoint{-23.809001in}{3.950723in}}%
\pgfpathlineto{\pgfqpoint{13.072662in}{3.950723in}}%
\pgfpathlineto{\pgfqpoint{13.072662in}{3.957673in}}%
\pgfpathlineto{\pgfqpoint{-23.809001in}{3.957673in}}%
\pgfpathclose%
\pgfusepath{fill}%
\end{pgfscope}%
\begin{pgfscope}%
\pgfpathrectangle{\pgfqpoint{12.211765in}{3.134211in}}{\pgfqpoint{2.188235in}{0.953684in}} %
\pgfusepath{clip}%
\pgfsetbuttcap%
\pgfsetmiterjoin%
\definecolor{currentfill}{rgb}{0.121569,0.466667,0.705882}%
\pgfsetfillcolor{currentfill}%
\pgfsetlinewidth{0.000000pt}%
\definecolor{currentstroke}{rgb}{0.000000,0.000000,0.000000}%
\pgfsetstrokecolor{currentstroke}%
\pgfsetstrokeopacity{0.000000}%
\pgfsetdash{}{0pt}%
\pgfpathmoveto{\pgfqpoint{-23.809001in}{3.959411in}}%
\pgfpathlineto{\pgfqpoint{12.987162in}{3.959411in}}%
\pgfpathlineto{\pgfqpoint{12.987162in}{3.966360in}}%
\pgfpathlineto{\pgfqpoint{-23.809001in}{3.966360in}}%
\pgfpathclose%
\pgfusepath{fill}%
\end{pgfscope}%
\begin{pgfscope}%
\pgfpathrectangle{\pgfqpoint{12.211765in}{3.134211in}}{\pgfqpoint{2.188235in}{0.953684in}} %
\pgfusepath{clip}%
\pgfsetbuttcap%
\pgfsetmiterjoin%
\definecolor{currentfill}{rgb}{0.121569,0.466667,0.705882}%
\pgfsetfillcolor{currentfill}%
\pgfsetlinewidth{0.000000pt}%
\definecolor{currentstroke}{rgb}{0.000000,0.000000,0.000000}%
\pgfsetstrokecolor{currentstroke}%
\pgfsetstrokeopacity{0.000000}%
\pgfsetdash{}{0pt}%
\pgfpathmoveto{\pgfqpoint{-23.809001in}{3.968098in}}%
\pgfpathlineto{\pgfqpoint{13.045477in}{3.968098in}}%
\pgfpathlineto{\pgfqpoint{13.045477in}{3.975048in}}%
\pgfpathlineto{\pgfqpoint{-23.809001in}{3.975048in}}%
\pgfpathclose%
\pgfusepath{fill}%
\end{pgfscope}%
\begin{pgfscope}%
\pgfpathrectangle{\pgfqpoint{12.211765in}{3.134211in}}{\pgfqpoint{2.188235in}{0.953684in}} %
\pgfusepath{clip}%
\pgfsetbuttcap%
\pgfsetmiterjoin%
\definecolor{currentfill}{rgb}{0.121569,0.466667,0.705882}%
\pgfsetfillcolor{currentfill}%
\pgfsetlinewidth{0.000000pt}%
\definecolor{currentstroke}{rgb}{0.000000,0.000000,0.000000}%
\pgfsetstrokecolor{currentstroke}%
\pgfsetstrokeopacity{0.000000}%
\pgfsetdash{}{0pt}%
\pgfpathmoveto{\pgfqpoint{-23.809001in}{3.976785in}}%
\pgfpathlineto{\pgfqpoint{13.077467in}{3.976785in}}%
\pgfpathlineto{\pgfqpoint{13.077467in}{3.983735in}}%
\pgfpathlineto{\pgfqpoint{-23.809001in}{3.983735in}}%
\pgfpathclose%
\pgfusepath{fill}%
\end{pgfscope}%
\begin{pgfscope}%
\pgfpathrectangle{\pgfqpoint{12.211765in}{3.134211in}}{\pgfqpoint{2.188235in}{0.953684in}} %
\pgfusepath{clip}%
\pgfsetbuttcap%
\pgfsetmiterjoin%
\definecolor{currentfill}{rgb}{0.121569,0.466667,0.705882}%
\pgfsetfillcolor{currentfill}%
\pgfsetlinewidth{0.000000pt}%
\definecolor{currentstroke}{rgb}{0.000000,0.000000,0.000000}%
\pgfsetstrokecolor{currentstroke}%
\pgfsetstrokeopacity{0.000000}%
\pgfsetdash{}{0pt}%
\pgfpathmoveto{\pgfqpoint{-23.809001in}{3.985472in}}%
\pgfpathlineto{\pgfqpoint{13.087893in}{3.985472in}}%
\pgfpathlineto{\pgfqpoint{13.087893in}{3.992422in}}%
\pgfpathlineto{\pgfqpoint{-23.809001in}{3.992422in}}%
\pgfpathclose%
\pgfusepath{fill}%
\end{pgfscope}%
\begin{pgfscope}%
\pgfpathrectangle{\pgfqpoint{12.211765in}{3.134211in}}{\pgfqpoint{2.188235in}{0.953684in}} %
\pgfusepath{clip}%
\pgfsetbuttcap%
\pgfsetmiterjoin%
\definecolor{currentfill}{rgb}{0.121569,0.466667,0.705882}%
\pgfsetfillcolor{currentfill}%
\pgfsetlinewidth{0.000000pt}%
\definecolor{currentstroke}{rgb}{0.000000,0.000000,0.000000}%
\pgfsetstrokecolor{currentstroke}%
\pgfsetstrokeopacity{0.000000}%
\pgfsetdash{}{0pt}%
\pgfpathmoveto{\pgfqpoint{-23.809001in}{3.994160in}}%
\pgfpathlineto{\pgfqpoint{13.089771in}{3.994160in}}%
\pgfpathlineto{\pgfqpoint{13.089771in}{4.001109in}}%
\pgfpathlineto{\pgfqpoint{-23.809001in}{4.001109in}}%
\pgfpathclose%
\pgfusepath{fill}%
\end{pgfscope}%
\begin{pgfscope}%
\pgfpathrectangle{\pgfqpoint{12.211765in}{3.134211in}}{\pgfqpoint{2.188235in}{0.953684in}} %
\pgfusepath{clip}%
\pgfsetbuttcap%
\pgfsetmiterjoin%
\definecolor{currentfill}{rgb}{0.121569,0.466667,0.705882}%
\pgfsetfillcolor{currentfill}%
\pgfsetlinewidth{0.000000pt}%
\definecolor{currentstroke}{rgb}{0.000000,0.000000,0.000000}%
\pgfsetstrokecolor{currentstroke}%
\pgfsetstrokeopacity{0.000000}%
\pgfsetdash{}{0pt}%
\pgfpathmoveto{\pgfqpoint{-23.809001in}{4.002847in}}%
\pgfpathlineto{\pgfqpoint{12.963908in}{4.002847in}}%
\pgfpathlineto{\pgfqpoint{12.963908in}{4.009797in}}%
\pgfpathlineto{\pgfqpoint{-23.809001in}{4.009797in}}%
\pgfpathclose%
\pgfusepath{fill}%
\end{pgfscope}%
\begin{pgfscope}%
\pgfpathrectangle{\pgfqpoint{12.211765in}{3.134211in}}{\pgfqpoint{2.188235in}{0.953684in}} %
\pgfusepath{clip}%
\pgfsetbuttcap%
\pgfsetmiterjoin%
\definecolor{currentfill}{rgb}{0.121569,0.466667,0.705882}%
\pgfsetfillcolor{currentfill}%
\pgfsetlinewidth{0.000000pt}%
\definecolor{currentstroke}{rgb}{0.000000,0.000000,0.000000}%
\pgfsetstrokecolor{currentstroke}%
\pgfsetstrokeopacity{0.000000}%
\pgfsetdash{}{0pt}%
\pgfpathmoveto{\pgfqpoint{-23.809001in}{4.011534in}}%
\pgfpathlineto{\pgfqpoint{13.049911in}{4.011534in}}%
\pgfpathlineto{\pgfqpoint{13.049911in}{4.018484in}}%
\pgfpathlineto{\pgfqpoint{-23.809001in}{4.018484in}}%
\pgfpathclose%
\pgfusepath{fill}%
\end{pgfscope}%
\begin{pgfscope}%
\pgfpathrectangle{\pgfqpoint{12.211765in}{3.134211in}}{\pgfqpoint{2.188235in}{0.953684in}} %
\pgfusepath{clip}%
\pgfsetbuttcap%
\pgfsetmiterjoin%
\definecolor{currentfill}{rgb}{0.121569,0.466667,0.705882}%
\pgfsetfillcolor{currentfill}%
\pgfsetlinewidth{0.000000pt}%
\definecolor{currentstroke}{rgb}{0.000000,0.000000,0.000000}%
\pgfsetstrokecolor{currentstroke}%
\pgfsetstrokeopacity{0.000000}%
\pgfsetdash{}{0pt}%
\pgfpathmoveto{\pgfqpoint{-23.809001in}{4.020221in}}%
\pgfpathlineto{\pgfqpoint{13.057207in}{4.020221in}}%
\pgfpathlineto{\pgfqpoint{13.057207in}{4.027171in}}%
\pgfpathlineto{\pgfqpoint{-23.809001in}{4.027171in}}%
\pgfpathclose%
\pgfusepath{fill}%
\end{pgfscope}%
\begin{pgfscope}%
\pgfpathrectangle{\pgfqpoint{12.211765in}{3.134211in}}{\pgfqpoint{2.188235in}{0.953684in}} %
\pgfusepath{clip}%
\pgfsetbuttcap%
\pgfsetmiterjoin%
\definecolor{currentfill}{rgb}{0.121569,0.466667,0.705882}%
\pgfsetfillcolor{currentfill}%
\pgfsetlinewidth{0.000000pt}%
\definecolor{currentstroke}{rgb}{0.000000,0.000000,0.000000}%
\pgfsetstrokecolor{currentstroke}%
\pgfsetstrokeopacity{0.000000}%
\pgfsetdash{}{0pt}%
\pgfpathmoveto{\pgfqpoint{-23.809001in}{4.028908in}}%
\pgfpathlineto{\pgfqpoint{13.033114in}{4.028908in}}%
\pgfpathlineto{\pgfqpoint{13.033114in}{4.035858in}}%
\pgfpathlineto{\pgfqpoint{-23.809001in}{4.035858in}}%
\pgfpathclose%
\pgfusepath{fill}%
\end{pgfscope}%
\begin{pgfscope}%
\pgfpathrectangle{\pgfqpoint{12.211765in}{3.134211in}}{\pgfqpoint{2.188235in}{0.953684in}} %
\pgfusepath{clip}%
\pgfsetbuttcap%
\pgfsetmiterjoin%
\definecolor{currentfill}{rgb}{0.121569,0.466667,0.705882}%
\pgfsetfillcolor{currentfill}%
\pgfsetlinewidth{0.000000pt}%
\definecolor{currentstroke}{rgb}{0.000000,0.000000,0.000000}%
\pgfsetstrokecolor{currentstroke}%
\pgfsetstrokeopacity{0.000000}%
\pgfsetdash{}{0pt}%
\pgfpathmoveto{\pgfqpoint{-23.809001in}{4.037596in}}%
\pgfpathlineto{\pgfqpoint{13.068007in}{4.037596in}}%
\pgfpathlineto{\pgfqpoint{13.068007in}{4.044545in}}%
\pgfpathlineto{\pgfqpoint{-23.809001in}{4.044545in}}%
\pgfpathclose%
\pgfusepath{fill}%
\end{pgfscope}%
\begin{pgfscope}%
\pgfsetbuttcap%
\pgfsetroundjoin%
\definecolor{currentfill}{rgb}{0.000000,0.000000,0.000000}%
\pgfsetfillcolor{currentfill}%
\pgfsetlinewidth{0.803000pt}%
\definecolor{currentstroke}{rgb}{0.000000,0.000000,0.000000}%
\pgfsetstrokecolor{currentstroke}%
\pgfsetdash{}{0pt}%
\pgfsys@defobject{currentmarker}{\pgfqpoint{0.000000in}{-0.048611in}}{\pgfqpoint{0.000000in}{0.000000in}}{%
\pgfpathmoveto{\pgfqpoint{0.000000in}{0.000000in}}%
\pgfpathlineto{\pgfqpoint{0.000000in}{-0.048611in}}%
\pgfusepath{stroke,fill}%
}%
\begin{pgfscope}%
\pgfsys@transformshift{12.624272in}{3.134211in}%
\pgfsys@useobject{currentmarker}{}%
\end{pgfscope}%
\end{pgfscope}%
\begin{pgfscope}%
\pgfsetbuttcap%
\pgfsetroundjoin%
\definecolor{currentfill}{rgb}{0.000000,0.000000,0.000000}%
\pgfsetfillcolor{currentfill}%
\pgfsetlinewidth{0.803000pt}%
\definecolor{currentstroke}{rgb}{0.000000,0.000000,0.000000}%
\pgfsetstrokecolor{currentstroke}%
\pgfsetdash{}{0pt}%
\pgfsys@defobject{currentmarker}{\pgfqpoint{0.000000in}{-0.048611in}}{\pgfqpoint{0.000000in}{0.000000in}}{%
\pgfpathmoveto{\pgfqpoint{0.000000in}{0.000000in}}%
\pgfpathlineto{\pgfqpoint{0.000000in}{-0.048611in}}%
\pgfusepath{stroke,fill}%
}%
\begin{pgfscope}%
\pgfsys@transformshift{13.130289in}{3.134211in}%
\pgfsys@useobject{currentmarker}{}%
\end{pgfscope}%
\end{pgfscope}%
\begin{pgfscope}%
\pgfsetbuttcap%
\pgfsetroundjoin%
\definecolor{currentfill}{rgb}{0.000000,0.000000,0.000000}%
\pgfsetfillcolor{currentfill}%
\pgfsetlinewidth{0.803000pt}%
\definecolor{currentstroke}{rgb}{0.000000,0.000000,0.000000}%
\pgfsetstrokecolor{currentstroke}%
\pgfsetdash{}{0pt}%
\pgfsys@defobject{currentmarker}{\pgfqpoint{0.000000in}{-0.048611in}}{\pgfqpoint{0.000000in}{0.000000in}}{%
\pgfpathmoveto{\pgfqpoint{0.000000in}{0.000000in}}%
\pgfpathlineto{\pgfqpoint{0.000000in}{-0.048611in}}%
\pgfusepath{stroke,fill}%
}%
\begin{pgfscope}%
\pgfsys@transformshift{13.636307in}{3.134211in}%
\pgfsys@useobject{currentmarker}{}%
\end{pgfscope}%
\end{pgfscope}%
\begin{pgfscope}%
\pgfsetbuttcap%
\pgfsetroundjoin%
\definecolor{currentfill}{rgb}{0.000000,0.000000,0.000000}%
\pgfsetfillcolor{currentfill}%
\pgfsetlinewidth{0.803000pt}%
\definecolor{currentstroke}{rgb}{0.000000,0.000000,0.000000}%
\pgfsetstrokecolor{currentstroke}%
\pgfsetdash{}{0pt}%
\pgfsys@defobject{currentmarker}{\pgfqpoint{0.000000in}{-0.048611in}}{\pgfqpoint{0.000000in}{0.000000in}}{%
\pgfpathmoveto{\pgfqpoint{0.000000in}{0.000000in}}%
\pgfpathlineto{\pgfqpoint{0.000000in}{-0.048611in}}%
\pgfusepath{stroke,fill}%
}%
\begin{pgfscope}%
\pgfsys@transformshift{14.142325in}{3.134211in}%
\pgfsys@useobject{currentmarker}{}%
\end{pgfscope}%
\end{pgfscope}%
\begin{pgfscope}%
\pgfsetbuttcap%
\pgfsetroundjoin%
\definecolor{currentfill}{rgb}{0.000000,0.000000,0.000000}%
\pgfsetfillcolor{currentfill}%
\pgfsetlinewidth{0.803000pt}%
\definecolor{currentstroke}{rgb}{0.000000,0.000000,0.000000}%
\pgfsetstrokecolor{currentstroke}%
\pgfsetdash{}{0pt}%
\pgfsys@defobject{currentmarker}{\pgfqpoint{-0.048611in}{0.000000in}}{\pgfqpoint{0.000000in}{0.000000in}}{%
\pgfpathmoveto{\pgfqpoint{0.000000in}{0.000000in}}%
\pgfpathlineto{\pgfqpoint{-0.048611in}{0.000000in}}%
\pgfusepath{stroke,fill}%
}%
\begin{pgfscope}%
\pgfsys@transformshift{12.211765in}{3.181035in}%
\pgfsys@useobject{currentmarker}{}%
\end{pgfscope}%
\end{pgfscope}%
\begin{pgfscope}%
\pgftext[x=12.045098in,y=3.132817in,left,base]{\rmfamily\fontsize{10.000000}{12.000000}\selectfont \(\displaystyle 0\)}%
\end{pgfscope}%
\begin{pgfscope}%
\pgfsetbuttcap%
\pgfsetroundjoin%
\definecolor{currentfill}{rgb}{0.000000,0.000000,0.000000}%
\pgfsetfillcolor{currentfill}%
\pgfsetlinewidth{0.803000pt}%
\definecolor{currentstroke}{rgb}{0.000000,0.000000,0.000000}%
\pgfsetstrokecolor{currentstroke}%
\pgfsetdash{}{0pt}%
\pgfsys@defobject{currentmarker}{\pgfqpoint{-0.048611in}{0.000000in}}{\pgfqpoint{0.000000in}{0.000000in}}{%
\pgfpathmoveto{\pgfqpoint{0.000000in}{0.000000in}}%
\pgfpathlineto{\pgfqpoint{-0.048611in}{0.000000in}}%
\pgfusepath{stroke,fill}%
}%
\begin{pgfscope}%
\pgfsys@transformshift{12.211765in}{3.615396in}%
\pgfsys@useobject{currentmarker}{}%
\end{pgfscope}%
\end{pgfscope}%
\begin{pgfscope}%
\pgftext[x=11.975653in,y=3.567178in,left,base]{\rmfamily\fontsize{10.000000}{12.000000}\selectfont \(\displaystyle 50\)}%
\end{pgfscope}%
\begin{pgfscope}%
\pgfsetbuttcap%
\pgfsetroundjoin%
\definecolor{currentfill}{rgb}{0.000000,0.000000,0.000000}%
\pgfsetfillcolor{currentfill}%
\pgfsetlinewidth{0.803000pt}%
\definecolor{currentstroke}{rgb}{0.000000,0.000000,0.000000}%
\pgfsetstrokecolor{currentstroke}%
\pgfsetdash{}{0pt}%
\pgfsys@defobject{currentmarker}{\pgfqpoint{-0.048611in}{0.000000in}}{\pgfqpoint{0.000000in}{0.000000in}}{%
\pgfpathmoveto{\pgfqpoint{0.000000in}{0.000000in}}%
\pgfpathlineto{\pgfqpoint{-0.048611in}{0.000000in}}%
\pgfusepath{stroke,fill}%
}%
\begin{pgfscope}%
\pgfsys@transformshift{12.211765in}{4.049758in}%
\pgfsys@useobject{currentmarker}{}%
\end{pgfscope}%
\end{pgfscope}%
\begin{pgfscope}%
\pgftext[x=11.906208in,y=4.001540in,left,base]{\rmfamily\fontsize{10.000000}{12.000000}\selectfont \(\displaystyle 100\)}%
\end{pgfscope}%
\begin{pgfscope}%
\pgftext[x=11.850653in,y=3.611053in,,bottom,rotate=90.000000]{\rmfamily\fontsize{10.000000}{12.000000}\selectfont \(\displaystyle j\)}%
\end{pgfscope}%
\begin{pgfscope}%
\pgfsetrectcap%
\pgfsetmiterjoin%
\pgfsetlinewidth{0.803000pt}%
\definecolor{currentstroke}{rgb}{0.000000,0.000000,0.000000}%
\pgfsetstrokecolor{currentstroke}%
\pgfsetdash{}{0pt}%
\pgfpathmoveto{\pgfqpoint{12.211765in}{3.134211in}}%
\pgfpathlineto{\pgfqpoint{12.211765in}{4.087895in}}%
\pgfusepath{stroke}%
\end{pgfscope}%
\begin{pgfscope}%
\pgfsetrectcap%
\pgfsetmiterjoin%
\pgfsetlinewidth{0.803000pt}%
\definecolor{currentstroke}{rgb}{0.000000,0.000000,0.000000}%
\pgfsetstrokecolor{currentstroke}%
\pgfsetdash{}{0pt}%
\pgfpathmoveto{\pgfqpoint{14.400000in}{3.134211in}}%
\pgfpathlineto{\pgfqpoint{14.400000in}{4.087895in}}%
\pgfusepath{stroke}%
\end{pgfscope}%
\begin{pgfscope}%
\pgfsetrectcap%
\pgfsetmiterjoin%
\pgfsetlinewidth{0.803000pt}%
\definecolor{currentstroke}{rgb}{0.000000,0.000000,0.000000}%
\pgfsetstrokecolor{currentstroke}%
\pgfsetdash{}{0pt}%
\pgfpathmoveto{\pgfqpoint{12.211765in}{3.134211in}}%
\pgfpathlineto{\pgfqpoint{14.400000in}{3.134211in}}%
\pgfusepath{stroke}%
\end{pgfscope}%
\begin{pgfscope}%
\pgfsetrectcap%
\pgfsetmiterjoin%
\pgfsetlinewidth{0.803000pt}%
\definecolor{currentstroke}{rgb}{0.000000,0.000000,0.000000}%
\pgfsetstrokecolor{currentstroke}%
\pgfsetdash{}{0pt}%
\pgfpathmoveto{\pgfqpoint{12.211765in}{4.087895in}}%
\pgfpathlineto{\pgfqpoint{14.400000in}{4.087895in}}%
\pgfusepath{stroke}%
\end{pgfscope}%
\begin{pgfscope}%
\pgfsetbuttcap%
\pgfsetmiterjoin%
\definecolor{currentfill}{rgb}{1.000000,1.000000,1.000000}%
\pgfsetfillcolor{currentfill}%
\pgfsetlinewidth{0.000000pt}%
\definecolor{currentstroke}{rgb}{0.000000,0.000000,0.000000}%
\pgfsetstrokecolor{currentstroke}%
\pgfsetstrokeopacity{0.000000}%
\pgfsetdash{}{0pt}%
\pgfpathmoveto{\pgfqpoint{2.000000in}{1.942105in}}%
\pgfpathlineto{\pgfqpoint{6.376471in}{1.942105in}}%
\pgfpathlineto{\pgfqpoint{6.376471in}{2.895789in}}%
\pgfpathlineto{\pgfqpoint{2.000000in}{2.895789in}}%
\pgfpathclose%
\pgfusepath{fill}%
\end{pgfscope}%
\begin{pgfscope}%
\pgfpathrectangle{\pgfqpoint{2.000000in}{1.942105in}}{\pgfqpoint{4.376471in}{0.953684in}} %
\pgfusepath{clip}%
\pgfsetbuttcap%
\pgfsetroundjoin%
\definecolor{currentfill}{rgb}{1.000000,0.000000,0.000000}%
\pgfsetfillcolor{currentfill}%
\pgfsetlinewidth{2.007500pt}%
\definecolor{currentstroke}{rgb}{1.000000,0.000000,0.000000}%
\pgfsetstrokecolor{currentstroke}%
\pgfsetdash{}{0pt}%
\pgfpathmoveto{\pgfqpoint{4.755120in}{2.298612in}}%
\pgfpathlineto{\pgfqpoint{4.838454in}{2.298612in}}%
\pgfpathmoveto{\pgfqpoint{4.796787in}{2.256946in}}%
\pgfpathlineto{\pgfqpoint{4.796787in}{2.340279in}}%
\pgfusepath{stroke,fill}%
\end{pgfscope}%
\begin{pgfscope}%
\pgfpathrectangle{\pgfqpoint{2.000000in}{1.942105in}}{\pgfqpoint{4.376471in}{0.953684in}} %
\pgfusepath{clip}%
\pgfsetbuttcap%
\pgfsetroundjoin%
\definecolor{currentfill}{rgb}{1.000000,0.000000,0.000000}%
\pgfsetfillcolor{currentfill}%
\pgfsetlinewidth{2.007500pt}%
\definecolor{currentstroke}{rgb}{1.000000,0.000000,0.000000}%
\pgfsetstrokecolor{currentstroke}%
\pgfsetdash{}{0pt}%
\pgfpathmoveto{\pgfqpoint{5.337632in}{2.384806in}}%
\pgfpathlineto{\pgfqpoint{5.420965in}{2.384806in}}%
\pgfpathmoveto{\pgfqpoint{5.379298in}{2.343139in}}%
\pgfpathlineto{\pgfqpoint{5.379298in}{2.426473in}}%
\pgfusepath{stroke,fill}%
\end{pgfscope}%
\begin{pgfscope}%
\pgfpathrectangle{\pgfqpoint{2.000000in}{1.942105in}}{\pgfqpoint{4.376471in}{0.953684in}} %
\pgfusepath{clip}%
\pgfsetbuttcap%
\pgfsetroundjoin%
\definecolor{currentfill}{rgb}{1.000000,0.000000,0.000000}%
\pgfsetfillcolor{currentfill}%
\pgfsetlinewidth{2.007500pt}%
\definecolor{currentstroke}{rgb}{1.000000,0.000000,0.000000}%
\pgfsetstrokecolor{currentstroke}%
\pgfsetdash{}{0pt}%
\pgfpathmoveto{\pgfqpoint{4.944008in}{2.441712in}}%
\pgfpathlineto{\pgfqpoint{5.027342in}{2.441712in}}%
\pgfpathmoveto{\pgfqpoint{4.985675in}{2.400045in}}%
\pgfpathlineto{\pgfqpoint{4.985675in}{2.483379in}}%
\pgfusepath{stroke,fill}%
\end{pgfscope}%
\begin{pgfscope}%
\pgfpathrectangle{\pgfqpoint{2.000000in}{1.942105in}}{\pgfqpoint{4.376471in}{0.953684in}} %
\pgfusepath{clip}%
\pgfsetbuttcap%
\pgfsetroundjoin%
\definecolor{currentfill}{rgb}{1.000000,0.000000,0.000000}%
\pgfsetfillcolor{currentfill}%
\pgfsetlinewidth{2.007500pt}%
\definecolor{currentstroke}{rgb}{1.000000,0.000000,0.000000}%
\pgfsetstrokecolor{currentstroke}%
\pgfsetdash{}{0pt}%
\pgfpathmoveto{\pgfqpoint{4.741360in}{2.414765in}}%
\pgfpathlineto{\pgfqpoint{4.824693in}{2.414765in}}%
\pgfpathmoveto{\pgfqpoint{4.783026in}{2.373099in}}%
\pgfpathlineto{\pgfqpoint{4.783026in}{2.456432in}}%
\pgfusepath{stroke,fill}%
\end{pgfscope}%
\begin{pgfscope}%
\pgfpathrectangle{\pgfqpoint{2.000000in}{1.942105in}}{\pgfqpoint{4.376471in}{0.953684in}} %
\pgfusepath{clip}%
\pgfsetbuttcap%
\pgfsetroundjoin%
\definecolor{currentfill}{rgb}{1.000000,0.000000,0.000000}%
\pgfsetfillcolor{currentfill}%
\pgfsetlinewidth{2.007500pt}%
\definecolor{currentstroke}{rgb}{1.000000,0.000000,0.000000}%
\pgfsetstrokecolor{currentstroke}%
\pgfsetdash{}{0pt}%
\pgfpathmoveto{\pgfqpoint{4.316918in}{1.986053in}}%
\pgfpathlineto{\pgfqpoint{4.400251in}{1.986053in}}%
\pgfpathmoveto{\pgfqpoint{4.358584in}{1.944386in}}%
\pgfpathlineto{\pgfqpoint{4.358584in}{2.027720in}}%
\pgfusepath{stroke,fill}%
\end{pgfscope}%
\begin{pgfscope}%
\pgfpathrectangle{\pgfqpoint{2.000000in}{1.942105in}}{\pgfqpoint{4.376471in}{0.953684in}} %
\pgfusepath{clip}%
\pgfsetbuttcap%
\pgfsetroundjoin%
\definecolor{currentfill}{rgb}{1.000000,0.000000,0.000000}%
\pgfsetfillcolor{currentfill}%
\pgfsetlinewidth{2.007500pt}%
\definecolor{currentstroke}{rgb}{1.000000,0.000000,0.000000}%
\pgfsetstrokecolor{currentstroke}%
\pgfsetdash{}{0pt}%
\pgfpathmoveto{\pgfqpoint{5.095017in}{2.454884in}}%
\pgfpathlineto{\pgfqpoint{5.178350in}{2.454884in}}%
\pgfpathmoveto{\pgfqpoint{5.136683in}{2.413218in}}%
\pgfpathlineto{\pgfqpoint{5.136683in}{2.496551in}}%
\pgfusepath{stroke,fill}%
\end{pgfscope}%
\begin{pgfscope}%
\pgfpathrectangle{\pgfqpoint{2.000000in}{1.942105in}}{\pgfqpoint{4.376471in}{0.953684in}} %
\pgfusepath{clip}%
\pgfsetbuttcap%
\pgfsetroundjoin%
\definecolor{currentfill}{rgb}{1.000000,0.000000,0.000000}%
\pgfsetfillcolor{currentfill}%
\pgfsetlinewidth{2.007500pt}%
\definecolor{currentstroke}{rgb}{1.000000,0.000000,0.000000}%
\pgfsetstrokecolor{currentstroke}%
\pgfsetdash{}{0pt}%
\pgfpathmoveto{\pgfqpoint{4.365697in}{1.957162in}}%
\pgfpathlineto{\pgfqpoint{4.449031in}{1.957162in}}%
\pgfpathmoveto{\pgfqpoint{4.407364in}{1.915495in}}%
\pgfpathlineto{\pgfqpoint{4.407364in}{1.998828in}}%
\pgfusepath{stroke,fill}%
\end{pgfscope}%
\begin{pgfscope}%
\pgfpathrectangle{\pgfqpoint{2.000000in}{1.942105in}}{\pgfqpoint{4.376471in}{0.953684in}} %
\pgfusepath{clip}%
\pgfsetbuttcap%
\pgfsetroundjoin%
\definecolor{currentfill}{rgb}{1.000000,0.000000,0.000000}%
\pgfsetfillcolor{currentfill}%
\pgfsetlinewidth{2.007500pt}%
\definecolor{currentstroke}{rgb}{1.000000,0.000000,0.000000}%
\pgfsetstrokecolor{currentstroke}%
\pgfsetdash{}{0pt}%
\pgfpathmoveto{\pgfqpoint{5.955882in}{2.721215in}}%
\pgfpathlineto{\pgfqpoint{6.039215in}{2.721215in}}%
\pgfpathmoveto{\pgfqpoint{5.997549in}{2.679548in}}%
\pgfpathlineto{\pgfqpoint{5.997549in}{2.762882in}}%
\pgfusepath{stroke,fill}%
\end{pgfscope}%
\begin{pgfscope}%
\pgfpathrectangle{\pgfqpoint{2.000000in}{1.942105in}}{\pgfqpoint{4.376471in}{0.953684in}} %
\pgfusepath{clip}%
\pgfsetbuttcap%
\pgfsetroundjoin%
\definecolor{currentfill}{rgb}{1.000000,0.000000,0.000000}%
\pgfsetfillcolor{currentfill}%
\pgfsetlinewidth{2.007500pt}%
\definecolor{currentstroke}{rgb}{1.000000,0.000000,0.000000}%
\pgfsetstrokecolor{currentstroke}%
\pgfsetdash{}{0pt}%
\pgfpathmoveto{\pgfqpoint{6.207581in}{2.582479in}}%
\pgfpathlineto{\pgfqpoint{6.290914in}{2.582479in}}%
\pgfpathmoveto{\pgfqpoint{6.249248in}{2.540813in}}%
\pgfpathlineto{\pgfqpoint{6.249248in}{2.624146in}}%
\pgfusepath{stroke,fill}%
\end{pgfscope}%
\begin{pgfscope}%
\pgfpathrectangle{\pgfqpoint{2.000000in}{1.942105in}}{\pgfqpoint{4.376471in}{0.953684in}} %
\pgfusepath{clip}%
\pgfsetbuttcap%
\pgfsetroundjoin%
\definecolor{currentfill}{rgb}{1.000000,0.000000,0.000000}%
\pgfsetfillcolor{currentfill}%
\pgfsetlinewidth{2.007500pt}%
\definecolor{currentstroke}{rgb}{1.000000,0.000000,0.000000}%
\pgfsetstrokecolor{currentstroke}%
\pgfsetdash{}{0pt}%
\pgfpathmoveto{\pgfqpoint{4.176124in}{2.215805in}}%
\pgfpathlineto{\pgfqpoint{4.259457in}{2.215805in}}%
\pgfpathmoveto{\pgfqpoint{4.217791in}{2.174138in}}%
\pgfpathlineto{\pgfqpoint{4.217791in}{2.257471in}}%
\pgfusepath{stroke,fill}%
\end{pgfscope}%
\begin{pgfscope}%
\pgfpathrectangle{\pgfqpoint{2.000000in}{1.942105in}}{\pgfqpoint{4.376471in}{0.953684in}} %
\pgfusepath{clip}%
\pgfsetbuttcap%
\pgfsetroundjoin%
\definecolor{currentfill}{rgb}{1.000000,0.000000,0.000000}%
\pgfsetfillcolor{currentfill}%
\pgfsetlinewidth{2.007500pt}%
\definecolor{currentstroke}{rgb}{1.000000,0.000000,0.000000}%
\pgfsetstrokecolor{currentstroke}%
\pgfsetdash{}{0pt}%
\pgfpathmoveto{\pgfqpoint{5.605597in}{2.298077in}}%
\pgfpathlineto{\pgfqpoint{5.688930in}{2.298077in}}%
\pgfpathmoveto{\pgfqpoint{5.647263in}{2.256410in}}%
\pgfpathlineto{\pgfqpoint{5.647263in}{2.339743in}}%
\pgfusepath{stroke,fill}%
\end{pgfscope}%
\begin{pgfscope}%
\pgfpathrectangle{\pgfqpoint{2.000000in}{1.942105in}}{\pgfqpoint{4.376471in}{0.953684in}} %
\pgfusepath{clip}%
\pgfsetbuttcap%
\pgfsetroundjoin%
\definecolor{currentfill}{rgb}{1.000000,0.000000,0.000000}%
\pgfsetfillcolor{currentfill}%
\pgfsetlinewidth{2.007500pt}%
\definecolor{currentstroke}{rgb}{1.000000,0.000000,0.000000}%
\pgfsetstrokecolor{currentstroke}%
\pgfsetdash{}{0pt}%
\pgfpathmoveto{\pgfqpoint{4.685382in}{2.292312in}}%
\pgfpathlineto{\pgfqpoint{4.768715in}{2.292312in}}%
\pgfpathmoveto{\pgfqpoint{4.727049in}{2.250645in}}%
\pgfpathlineto{\pgfqpoint{4.727049in}{2.333978in}}%
\pgfusepath{stroke,fill}%
\end{pgfscope}%
\begin{pgfscope}%
\pgfpathrectangle{\pgfqpoint{2.000000in}{1.942105in}}{\pgfqpoint{4.376471in}{0.953684in}} %
\pgfusepath{clip}%
\pgfsetbuttcap%
\pgfsetroundjoin%
\definecolor{currentfill}{rgb}{1.000000,0.000000,0.000000}%
\pgfsetfillcolor{currentfill}%
\pgfsetlinewidth{2.007500pt}%
\definecolor{currentstroke}{rgb}{1.000000,0.000000,0.000000}%
\pgfsetstrokecolor{currentstroke}%
\pgfsetdash{}{0pt}%
\pgfpathmoveto{\pgfqpoint{4.822452in}{2.498620in}}%
\pgfpathlineto{\pgfqpoint{4.905785in}{2.498620in}}%
\pgfpathmoveto{\pgfqpoint{4.864118in}{2.456953in}}%
\pgfpathlineto{\pgfqpoint{4.864118in}{2.540286in}}%
\pgfusepath{stroke,fill}%
\end{pgfscope}%
\begin{pgfscope}%
\pgfpathrectangle{\pgfqpoint{2.000000in}{1.942105in}}{\pgfqpoint{4.376471in}{0.953684in}} %
\pgfusepath{clip}%
\pgfsetbuttcap%
\pgfsetroundjoin%
\definecolor{currentfill}{rgb}{1.000000,0.000000,0.000000}%
\pgfsetfillcolor{currentfill}%
\pgfsetlinewidth{2.007500pt}%
\definecolor{currentstroke}{rgb}{1.000000,0.000000,0.000000}%
\pgfsetstrokecolor{currentstroke}%
\pgfsetdash{}{0pt}%
\pgfpathmoveto{\pgfqpoint{6.074305in}{2.787402in}}%
\pgfpathlineto{\pgfqpoint{6.157638in}{2.787402in}}%
\pgfpathmoveto{\pgfqpoint{6.115971in}{2.745735in}}%
\pgfpathlineto{\pgfqpoint{6.115971in}{2.829068in}}%
\pgfusepath{stroke,fill}%
\end{pgfscope}%
\begin{pgfscope}%
\pgfpathrectangle{\pgfqpoint{2.000000in}{1.942105in}}{\pgfqpoint{4.376471in}{0.953684in}} %
\pgfusepath{clip}%
\pgfsetbuttcap%
\pgfsetroundjoin%
\definecolor{currentfill}{rgb}{1.000000,0.000000,0.000000}%
\pgfsetfillcolor{currentfill}%
\pgfsetlinewidth{2.007500pt}%
\definecolor{currentstroke}{rgb}{1.000000,0.000000,0.000000}%
\pgfsetstrokecolor{currentstroke}%
\pgfsetdash{}{0pt}%
\pgfpathmoveto{\pgfqpoint{3.082337in}{2.450148in}}%
\pgfpathlineto{\pgfqpoint{3.165671in}{2.450148in}}%
\pgfpathmoveto{\pgfqpoint{3.124004in}{2.408481in}}%
\pgfpathlineto{\pgfqpoint{3.124004in}{2.491815in}}%
\pgfusepath{stroke,fill}%
\end{pgfscope}%
\begin{pgfscope}%
\pgfpathrectangle{\pgfqpoint{2.000000in}{1.942105in}}{\pgfqpoint{4.376471in}{0.953684in}} %
\pgfusepath{clip}%
\pgfsetbuttcap%
\pgfsetroundjoin%
\definecolor{currentfill}{rgb}{1.000000,0.000000,0.000000}%
\pgfsetfillcolor{currentfill}%
\pgfsetlinewidth{2.007500pt}%
\definecolor{currentstroke}{rgb}{1.000000,0.000000,0.000000}%
\pgfsetstrokecolor{currentstroke}%
\pgfsetdash{}{0pt}%
\pgfpathmoveto{\pgfqpoint{3.138683in}{2.417997in}}%
\pgfpathlineto{\pgfqpoint{3.222016in}{2.417997in}}%
\pgfpathmoveto{\pgfqpoint{3.180349in}{2.376330in}}%
\pgfpathlineto{\pgfqpoint{3.180349in}{2.459663in}}%
\pgfusepath{stroke,fill}%
\end{pgfscope}%
\begin{pgfscope}%
\pgfpathrectangle{\pgfqpoint{2.000000in}{1.942105in}}{\pgfqpoint{4.376471in}{0.953684in}} %
\pgfusepath{clip}%
\pgfsetbuttcap%
\pgfsetroundjoin%
\definecolor{currentfill}{rgb}{1.000000,0.000000,0.000000}%
\pgfsetfillcolor{currentfill}%
\pgfsetlinewidth{2.007500pt}%
\definecolor{currentstroke}{rgb}{1.000000,0.000000,0.000000}%
\pgfsetstrokecolor{currentstroke}%
\pgfsetdash{}{0pt}%
\pgfpathmoveto{\pgfqpoint{2.904416in}{2.636385in}}%
\pgfpathlineto{\pgfqpoint{2.987749in}{2.636385in}}%
\pgfpathmoveto{\pgfqpoint{2.946082in}{2.594718in}}%
\pgfpathlineto{\pgfqpoint{2.946082in}{2.678051in}}%
\pgfusepath{stroke,fill}%
\end{pgfscope}%
\begin{pgfscope}%
\pgfpathrectangle{\pgfqpoint{2.000000in}{1.942105in}}{\pgfqpoint{4.376471in}{0.953684in}} %
\pgfusepath{clip}%
\pgfsetbuttcap%
\pgfsetroundjoin%
\definecolor{currentfill}{rgb}{1.000000,0.000000,0.000000}%
\pgfsetfillcolor{currentfill}%
\pgfsetlinewidth{2.007500pt}%
\definecolor{currentstroke}{rgb}{1.000000,0.000000,0.000000}%
\pgfsetstrokecolor{currentstroke}%
\pgfsetdash{}{0pt}%
\pgfpathmoveto{\pgfqpoint{5.748776in}{2.511256in}}%
\pgfpathlineto{\pgfqpoint{5.832110in}{2.511256in}}%
\pgfpathmoveto{\pgfqpoint{5.790443in}{2.469589in}}%
\pgfpathlineto{\pgfqpoint{5.790443in}{2.552922in}}%
\pgfusepath{stroke,fill}%
\end{pgfscope}%
\begin{pgfscope}%
\pgfpathrectangle{\pgfqpoint{2.000000in}{1.942105in}}{\pgfqpoint{4.376471in}{0.953684in}} %
\pgfusepath{clip}%
\pgfsetbuttcap%
\pgfsetroundjoin%
\definecolor{currentfill}{rgb}{1.000000,0.000000,0.000000}%
\pgfsetfillcolor{currentfill}%
\pgfsetlinewidth{2.007500pt}%
\definecolor{currentstroke}{rgb}{1.000000,0.000000,0.000000}%
\pgfsetstrokecolor{currentstroke}%
\pgfsetdash{}{0pt}%
\pgfpathmoveto{\pgfqpoint{5.558092in}{2.224896in}}%
\pgfpathlineto{\pgfqpoint{5.641425in}{2.224896in}}%
\pgfpathmoveto{\pgfqpoint{5.599758in}{2.183229in}}%
\pgfpathlineto{\pgfqpoint{5.599758in}{2.266562in}}%
\pgfusepath{stroke,fill}%
\end{pgfscope}%
\begin{pgfscope}%
\pgfpathrectangle{\pgfqpoint{2.000000in}{1.942105in}}{\pgfqpoint{4.376471in}{0.953684in}} %
\pgfusepath{clip}%
\pgfsetbuttcap%
\pgfsetroundjoin%
\definecolor{currentfill}{rgb}{1.000000,0.000000,0.000000}%
\pgfsetfillcolor{currentfill}%
\pgfsetlinewidth{2.007500pt}%
\definecolor{currentstroke}{rgb}{1.000000,0.000000,0.000000}%
\pgfsetstrokecolor{currentstroke}%
\pgfsetdash{}{0pt}%
\pgfpathmoveto{\pgfqpoint{5.879694in}{2.660212in}}%
\pgfpathlineto{\pgfqpoint{5.963027in}{2.660212in}}%
\pgfpathmoveto{\pgfqpoint{5.921360in}{2.618545in}}%
\pgfpathlineto{\pgfqpoint{5.921360in}{2.701878in}}%
\pgfusepath{stroke,fill}%
\end{pgfscope}%
\begin{pgfscope}%
\pgfpathrectangle{\pgfqpoint{2.000000in}{1.942105in}}{\pgfqpoint{4.376471in}{0.953684in}} %
\pgfusepath{clip}%
\pgfsetbuttcap%
\pgfsetroundjoin%
\definecolor{currentfill}{rgb}{1.000000,0.000000,0.000000}%
\pgfsetfillcolor{currentfill}%
\pgfsetlinewidth{2.007500pt}%
\definecolor{currentstroke}{rgb}{1.000000,0.000000,0.000000}%
\pgfsetstrokecolor{currentstroke}%
\pgfsetdash{}{0pt}%
\pgfpathmoveto{\pgfqpoint{6.259943in}{2.553849in}}%
\pgfpathlineto{\pgfqpoint{6.343276in}{2.553849in}}%
\pgfpathmoveto{\pgfqpoint{6.301610in}{2.512183in}}%
\pgfpathlineto{\pgfqpoint{6.301610in}{2.595516in}}%
\pgfusepath{stroke,fill}%
\end{pgfscope}%
\begin{pgfscope}%
\pgfpathrectangle{\pgfqpoint{2.000000in}{1.942105in}}{\pgfqpoint{4.376471in}{0.953684in}} %
\pgfusepath{clip}%
\pgfsetbuttcap%
\pgfsetroundjoin%
\definecolor{currentfill}{rgb}{1.000000,0.000000,0.000000}%
\pgfsetfillcolor{currentfill}%
\pgfsetlinewidth{2.007500pt}%
\definecolor{currentstroke}{rgb}{1.000000,0.000000,0.000000}%
\pgfsetstrokecolor{currentstroke}%
\pgfsetdash{}{0pt}%
\pgfpathmoveto{\pgfqpoint{5.631623in}{2.211778in}}%
\pgfpathlineto{\pgfqpoint{5.714956in}{2.211778in}}%
\pgfpathmoveto{\pgfqpoint{5.673289in}{2.170111in}}%
\pgfpathlineto{\pgfqpoint{5.673289in}{2.253444in}}%
\pgfusepath{stroke,fill}%
\end{pgfscope}%
\begin{pgfscope}%
\pgfpathrectangle{\pgfqpoint{2.000000in}{1.942105in}}{\pgfqpoint{4.376471in}{0.953684in}} %
\pgfusepath{clip}%
\pgfsetbuttcap%
\pgfsetroundjoin%
\definecolor{currentfill}{rgb}{1.000000,0.000000,0.000000}%
\pgfsetfillcolor{currentfill}%
\pgfsetlinewidth{2.007500pt}%
\definecolor{currentstroke}{rgb}{1.000000,0.000000,0.000000}%
\pgfsetstrokecolor{currentstroke}%
\pgfsetdash{}{0pt}%
\pgfpathmoveto{\pgfqpoint{4.449348in}{2.181351in}}%
\pgfpathlineto{\pgfqpoint{4.532681in}{2.181351in}}%
\pgfpathmoveto{\pgfqpoint{4.491015in}{2.139685in}}%
\pgfpathlineto{\pgfqpoint{4.491015in}{2.223018in}}%
\pgfusepath{stroke,fill}%
\end{pgfscope}%
\begin{pgfscope}%
\pgfpathrectangle{\pgfqpoint{2.000000in}{1.942105in}}{\pgfqpoint{4.376471in}{0.953684in}} %
\pgfusepath{clip}%
\pgfsetbuttcap%
\pgfsetroundjoin%
\definecolor{currentfill}{rgb}{1.000000,0.000000,0.000000}%
\pgfsetfillcolor{currentfill}%
\pgfsetlinewidth{2.007500pt}%
\definecolor{currentstroke}{rgb}{1.000000,0.000000,0.000000}%
\pgfsetstrokecolor{currentstroke}%
\pgfsetdash{}{0pt}%
\pgfpathmoveto{\pgfqpoint{5.566398in}{2.261124in}}%
\pgfpathlineto{\pgfqpoint{5.649731in}{2.261124in}}%
\pgfpathmoveto{\pgfqpoint{5.608065in}{2.219458in}}%
\pgfpathlineto{\pgfqpoint{5.608065in}{2.302791in}}%
\pgfusepath{stroke,fill}%
\end{pgfscope}%
\begin{pgfscope}%
\pgfpathrectangle{\pgfqpoint{2.000000in}{1.942105in}}{\pgfqpoint{4.376471in}{0.953684in}} %
\pgfusepath{clip}%
\pgfsetbuttcap%
\pgfsetroundjoin%
\definecolor{currentfill}{rgb}{1.000000,0.000000,0.000000}%
\pgfsetfillcolor{currentfill}%
\pgfsetlinewidth{2.007500pt}%
\definecolor{currentstroke}{rgb}{1.000000,0.000000,0.000000}%
\pgfsetstrokecolor{currentstroke}%
\pgfsetdash{}{0pt}%
\pgfpathmoveto{\pgfqpoint{3.247727in}{2.175480in}}%
\pgfpathlineto{\pgfqpoint{3.331060in}{2.175480in}}%
\pgfpathmoveto{\pgfqpoint{3.289394in}{2.133814in}}%
\pgfpathlineto{\pgfqpoint{3.289394in}{2.217147in}}%
\pgfusepath{stroke,fill}%
\end{pgfscope}%
\begin{pgfscope}%
\pgfpathrectangle{\pgfqpoint{2.000000in}{1.942105in}}{\pgfqpoint{4.376471in}{0.953684in}} %
\pgfusepath{clip}%
\pgfsetbuttcap%
\pgfsetroundjoin%
\definecolor{currentfill}{rgb}{1.000000,0.000000,0.000000}%
\pgfsetfillcolor{currentfill}%
\pgfsetlinewidth{2.007500pt}%
\definecolor{currentstroke}{rgb}{1.000000,0.000000,0.000000}%
\pgfsetstrokecolor{currentstroke}%
\pgfsetdash{}{0pt}%
\pgfpathmoveto{\pgfqpoint{5.074104in}{2.443682in}}%
\pgfpathlineto{\pgfqpoint{5.157437in}{2.443682in}}%
\pgfpathmoveto{\pgfqpoint{5.115771in}{2.402015in}}%
\pgfpathlineto{\pgfqpoint{5.115771in}{2.485348in}}%
\pgfusepath{stroke,fill}%
\end{pgfscope}%
\begin{pgfscope}%
\pgfpathrectangle{\pgfqpoint{2.000000in}{1.942105in}}{\pgfqpoint{4.376471in}{0.953684in}} %
\pgfusepath{clip}%
\pgfsetbuttcap%
\pgfsetroundjoin%
\definecolor{currentfill}{rgb}{1.000000,0.000000,0.000000}%
\pgfsetfillcolor{currentfill}%
\pgfsetlinewidth{2.007500pt}%
\definecolor{currentstroke}{rgb}{1.000000,0.000000,0.000000}%
\pgfsetstrokecolor{currentstroke}%
\pgfsetdash{}{0pt}%
\pgfpathmoveto{\pgfqpoint{3.335533in}{2.262545in}}%
\pgfpathlineto{\pgfqpoint{3.418866in}{2.262545in}}%
\pgfpathmoveto{\pgfqpoint{3.377199in}{2.220879in}}%
\pgfpathlineto{\pgfqpoint{3.377199in}{2.304212in}}%
\pgfusepath{stroke,fill}%
\end{pgfscope}%
\begin{pgfscope}%
\pgfpathrectangle{\pgfqpoint{2.000000in}{1.942105in}}{\pgfqpoint{4.376471in}{0.953684in}} %
\pgfusepath{clip}%
\pgfsetbuttcap%
\pgfsetroundjoin%
\definecolor{currentfill}{rgb}{1.000000,0.000000,0.000000}%
\pgfsetfillcolor{currentfill}%
\pgfsetlinewidth{2.007500pt}%
\definecolor{currentstroke}{rgb}{1.000000,0.000000,0.000000}%
\pgfsetstrokecolor{currentstroke}%
\pgfsetdash{}{0pt}%
\pgfpathmoveto{\pgfqpoint{6.141080in}{2.719866in}}%
\pgfpathlineto{\pgfqpoint{6.224413in}{2.719866in}}%
\pgfpathmoveto{\pgfqpoint{6.182747in}{2.678199in}}%
\pgfpathlineto{\pgfqpoint{6.182747in}{2.761533in}}%
\pgfusepath{stroke,fill}%
\end{pgfscope}%
\begin{pgfscope}%
\pgfpathrectangle{\pgfqpoint{2.000000in}{1.942105in}}{\pgfqpoint{4.376471in}{0.953684in}} %
\pgfusepath{clip}%
\pgfsetbuttcap%
\pgfsetroundjoin%
\definecolor{currentfill}{rgb}{1.000000,0.000000,0.000000}%
\pgfsetfillcolor{currentfill}%
\pgfsetlinewidth{2.007500pt}%
\definecolor{currentstroke}{rgb}{1.000000,0.000000,0.000000}%
\pgfsetstrokecolor{currentstroke}%
\pgfsetdash{}{0pt}%
\pgfpathmoveto{\pgfqpoint{4.660711in}{2.457007in}}%
\pgfpathlineto{\pgfqpoint{4.744044in}{2.457007in}}%
\pgfpathmoveto{\pgfqpoint{4.702377in}{2.415340in}}%
\pgfpathlineto{\pgfqpoint{4.702377in}{2.498673in}}%
\pgfusepath{stroke,fill}%
\end{pgfscope}%
\begin{pgfscope}%
\pgfpathrectangle{\pgfqpoint{2.000000in}{1.942105in}}{\pgfqpoint{4.376471in}{0.953684in}} %
\pgfusepath{clip}%
\pgfsetbuttcap%
\pgfsetroundjoin%
\definecolor{currentfill}{rgb}{1.000000,0.000000,0.000000}%
\pgfsetfillcolor{currentfill}%
\pgfsetlinewidth{2.007500pt}%
\definecolor{currentstroke}{rgb}{1.000000,0.000000,0.000000}%
\pgfsetstrokecolor{currentstroke}%
\pgfsetdash{}{0pt}%
\pgfpathmoveto{\pgfqpoint{4.285432in}{2.127275in}}%
\pgfpathlineto{\pgfqpoint{4.368765in}{2.127275in}}%
\pgfpathmoveto{\pgfqpoint{4.327099in}{2.085609in}}%
\pgfpathlineto{\pgfqpoint{4.327099in}{2.168942in}}%
\pgfusepath{stroke,fill}%
\end{pgfscope}%
\begin{pgfscope}%
\pgfpathrectangle{\pgfqpoint{2.000000in}{1.942105in}}{\pgfqpoint{4.376471in}{0.953684in}} %
\pgfusepath{clip}%
\pgfsetbuttcap%
\pgfsetroundjoin%
\definecolor{currentfill}{rgb}{1.000000,0.000000,0.000000}%
\pgfsetfillcolor{currentfill}%
\pgfsetlinewidth{2.007500pt}%
\definecolor{currentstroke}{rgb}{1.000000,0.000000,0.000000}%
\pgfsetstrokecolor{currentstroke}%
\pgfsetdash{}{0pt}%
\pgfpathmoveto{\pgfqpoint{3.759883in}{2.699721in}}%
\pgfpathlineto{\pgfqpoint{3.843217in}{2.699721in}}%
\pgfpathmoveto{\pgfqpoint{3.801550in}{2.658055in}}%
\pgfpathlineto{\pgfqpoint{3.801550in}{2.741388in}}%
\pgfusepath{stroke,fill}%
\end{pgfscope}%
\begin{pgfscope}%
\pgfpathrectangle{\pgfqpoint{2.000000in}{1.942105in}}{\pgfqpoint{4.376471in}{0.953684in}} %
\pgfusepath{clip}%
\pgfsetbuttcap%
\pgfsetroundjoin%
\definecolor{currentfill}{rgb}{1.000000,0.000000,0.000000}%
\pgfsetfillcolor{currentfill}%
\pgfsetlinewidth{2.007500pt}%
\definecolor{currentstroke}{rgb}{1.000000,0.000000,0.000000}%
\pgfsetstrokecolor{currentstroke}%
\pgfsetdash{}{0pt}%
\pgfpathmoveto{\pgfqpoint{5.544356in}{2.154280in}}%
\pgfpathlineto{\pgfqpoint{5.627690in}{2.154280in}}%
\pgfpathmoveto{\pgfqpoint{5.586023in}{2.112614in}}%
\pgfpathlineto{\pgfqpoint{5.586023in}{2.195947in}}%
\pgfusepath{stroke,fill}%
\end{pgfscope}%
\begin{pgfscope}%
\pgfpathrectangle{\pgfqpoint{2.000000in}{1.942105in}}{\pgfqpoint{4.376471in}{0.953684in}} %
\pgfusepath{clip}%
\pgfsetbuttcap%
\pgfsetroundjoin%
\definecolor{currentfill}{rgb}{1.000000,0.000000,0.000000}%
\pgfsetfillcolor{currentfill}%
\pgfsetlinewidth{2.007500pt}%
\definecolor{currentstroke}{rgb}{1.000000,0.000000,0.000000}%
\pgfsetstrokecolor{currentstroke}%
\pgfsetdash{}{0pt}%
\pgfpathmoveto{\pgfqpoint{4.430690in}{2.190010in}}%
\pgfpathlineto{\pgfqpoint{4.514024in}{2.190010in}}%
\pgfpathmoveto{\pgfqpoint{4.472357in}{2.148344in}}%
\pgfpathlineto{\pgfqpoint{4.472357in}{2.231677in}}%
\pgfusepath{stroke,fill}%
\end{pgfscope}%
\begin{pgfscope}%
\pgfpathrectangle{\pgfqpoint{2.000000in}{1.942105in}}{\pgfqpoint{4.376471in}{0.953684in}} %
\pgfusepath{clip}%
\pgfsetbuttcap%
\pgfsetroundjoin%
\definecolor{currentfill}{rgb}{1.000000,0.000000,0.000000}%
\pgfsetfillcolor{currentfill}%
\pgfsetlinewidth{2.007500pt}%
\definecolor{currentstroke}{rgb}{1.000000,0.000000,0.000000}%
\pgfsetstrokecolor{currentstroke}%
\pgfsetdash{}{0pt}%
\pgfpathmoveto{\pgfqpoint{4.823815in}{2.433790in}}%
\pgfpathlineto{\pgfqpoint{4.907148in}{2.433790in}}%
\pgfpathmoveto{\pgfqpoint{4.865482in}{2.392123in}}%
\pgfpathlineto{\pgfqpoint{4.865482in}{2.475457in}}%
\pgfusepath{stroke,fill}%
\end{pgfscope}%
\begin{pgfscope}%
\pgfpathrectangle{\pgfqpoint{2.000000in}{1.942105in}}{\pgfqpoint{4.376471in}{0.953684in}} %
\pgfusepath{clip}%
\pgfsetbuttcap%
\pgfsetroundjoin%
\definecolor{currentfill}{rgb}{1.000000,0.000000,0.000000}%
\pgfsetfillcolor{currentfill}%
\pgfsetlinewidth{2.007500pt}%
\definecolor{currentstroke}{rgb}{1.000000,0.000000,0.000000}%
\pgfsetstrokecolor{currentstroke}%
\pgfsetdash{}{0pt}%
\pgfpathmoveto{\pgfqpoint{2.899414in}{2.620654in}}%
\pgfpathlineto{\pgfqpoint{2.982747in}{2.620654in}}%
\pgfpathmoveto{\pgfqpoint{2.941081in}{2.578988in}}%
\pgfpathlineto{\pgfqpoint{2.941081in}{2.662321in}}%
\pgfusepath{stroke,fill}%
\end{pgfscope}%
\begin{pgfscope}%
\pgfpathrectangle{\pgfqpoint{2.000000in}{1.942105in}}{\pgfqpoint{4.376471in}{0.953684in}} %
\pgfusepath{clip}%
\pgfsetbuttcap%
\pgfsetroundjoin%
\definecolor{currentfill}{rgb}{1.000000,0.000000,0.000000}%
\pgfsetfillcolor{currentfill}%
\pgfsetlinewidth{2.007500pt}%
\definecolor{currentstroke}{rgb}{1.000000,0.000000,0.000000}%
\pgfsetstrokecolor{currentstroke}%
\pgfsetdash{}{0pt}%
\pgfpathmoveto{\pgfqpoint{4.996078in}{2.405208in}}%
\pgfpathlineto{\pgfqpoint{5.079412in}{2.405208in}}%
\pgfpathmoveto{\pgfqpoint{5.037745in}{2.363541in}}%
\pgfpathlineto{\pgfqpoint{5.037745in}{2.446874in}}%
\pgfusepath{stroke,fill}%
\end{pgfscope}%
\begin{pgfscope}%
\pgfpathrectangle{\pgfqpoint{2.000000in}{1.942105in}}{\pgfqpoint{4.376471in}{0.953684in}} %
\pgfusepath{clip}%
\pgfsetbuttcap%
\pgfsetroundjoin%
\definecolor{currentfill}{rgb}{1.000000,0.000000,0.000000}%
\pgfsetfillcolor{currentfill}%
\pgfsetlinewidth{2.007500pt}%
\definecolor{currentstroke}{rgb}{1.000000,0.000000,0.000000}%
\pgfsetstrokecolor{currentstroke}%
\pgfsetdash{}{0pt}%
\pgfpathmoveto{\pgfqpoint{4.976683in}{2.444354in}}%
\pgfpathlineto{\pgfqpoint{5.060016in}{2.444354in}}%
\pgfpathmoveto{\pgfqpoint{5.018349in}{2.402687in}}%
\pgfpathlineto{\pgfqpoint{5.018349in}{2.486021in}}%
\pgfusepath{stroke,fill}%
\end{pgfscope}%
\begin{pgfscope}%
\pgfpathrectangle{\pgfqpoint{2.000000in}{1.942105in}}{\pgfqpoint{4.376471in}{0.953684in}} %
\pgfusepath{clip}%
\pgfsetbuttcap%
\pgfsetroundjoin%
\definecolor{currentfill}{rgb}{1.000000,0.000000,0.000000}%
\pgfsetfillcolor{currentfill}%
\pgfsetlinewidth{2.007500pt}%
\definecolor{currentstroke}{rgb}{1.000000,0.000000,0.000000}%
\pgfsetstrokecolor{currentstroke}%
\pgfsetdash{}{0pt}%
\pgfpathmoveto{\pgfqpoint{4.993622in}{2.469980in}}%
\pgfpathlineto{\pgfqpoint{5.076956in}{2.469980in}}%
\pgfpathmoveto{\pgfqpoint{5.035289in}{2.428313in}}%
\pgfpathlineto{\pgfqpoint{5.035289in}{2.511647in}}%
\pgfusepath{stroke,fill}%
\end{pgfscope}%
\begin{pgfscope}%
\pgfpathrectangle{\pgfqpoint{2.000000in}{1.942105in}}{\pgfqpoint{4.376471in}{0.953684in}} %
\pgfusepath{clip}%
\pgfsetbuttcap%
\pgfsetroundjoin%
\definecolor{currentfill}{rgb}{1.000000,0.000000,0.000000}%
\pgfsetfillcolor{currentfill}%
\pgfsetlinewidth{2.007500pt}%
\definecolor{currentstroke}{rgb}{1.000000,0.000000,0.000000}%
\pgfsetstrokecolor{currentstroke}%
\pgfsetdash{}{0pt}%
\pgfpathmoveto{\pgfqpoint{6.137856in}{2.600794in}}%
\pgfpathlineto{\pgfqpoint{6.221189in}{2.600794in}}%
\pgfpathmoveto{\pgfqpoint{6.179523in}{2.559127in}}%
\pgfpathlineto{\pgfqpoint{6.179523in}{2.642461in}}%
\pgfusepath{stroke,fill}%
\end{pgfscope}%
\begin{pgfscope}%
\pgfpathrectangle{\pgfqpoint{2.000000in}{1.942105in}}{\pgfqpoint{4.376471in}{0.953684in}} %
\pgfusepath{clip}%
\pgfsetbuttcap%
\pgfsetroundjoin%
\definecolor{currentfill}{rgb}{1.000000,0.000000,0.000000}%
\pgfsetfillcolor{currentfill}%
\pgfsetlinewidth{2.007500pt}%
\definecolor{currentstroke}{rgb}{1.000000,0.000000,0.000000}%
\pgfsetstrokecolor{currentstroke}%
\pgfsetdash{}{0pt}%
\pgfpathmoveto{\pgfqpoint{5.220801in}{2.347424in}}%
\pgfpathlineto{\pgfqpoint{5.304134in}{2.347424in}}%
\pgfpathmoveto{\pgfqpoint{5.262467in}{2.305757in}}%
\pgfpathlineto{\pgfqpoint{5.262467in}{2.389091in}}%
\pgfusepath{stroke,fill}%
\end{pgfscope}%
\begin{pgfscope}%
\pgfpathrectangle{\pgfqpoint{2.000000in}{1.942105in}}{\pgfqpoint{4.376471in}{0.953684in}} %
\pgfusepath{clip}%
\pgfsetbuttcap%
\pgfsetroundjoin%
\definecolor{currentfill}{rgb}{1.000000,0.000000,0.000000}%
\pgfsetfillcolor{currentfill}%
\pgfsetlinewidth{2.007500pt}%
\definecolor{currentstroke}{rgb}{1.000000,0.000000,0.000000}%
\pgfsetstrokecolor{currentstroke}%
\pgfsetdash{}{0pt}%
\pgfpathmoveto{\pgfqpoint{4.092328in}{2.312659in}}%
\pgfpathlineto{\pgfqpoint{4.175661in}{2.312659in}}%
\pgfpathmoveto{\pgfqpoint{4.133995in}{2.270992in}}%
\pgfpathlineto{\pgfqpoint{4.133995in}{2.354325in}}%
\pgfusepath{stroke,fill}%
\end{pgfscope}%
\begin{pgfscope}%
\pgfpathrectangle{\pgfqpoint{2.000000in}{1.942105in}}{\pgfqpoint{4.376471in}{0.953684in}} %
\pgfusepath{clip}%
\pgfsetbuttcap%
\pgfsetroundjoin%
\definecolor{currentfill}{rgb}{1.000000,0.000000,0.000000}%
\pgfsetfillcolor{currentfill}%
\pgfsetlinewidth{2.007500pt}%
\definecolor{currentstroke}{rgb}{1.000000,0.000000,0.000000}%
\pgfsetstrokecolor{currentstroke}%
\pgfsetdash{}{0pt}%
\pgfpathmoveto{\pgfqpoint{4.363753in}{1.964299in}}%
\pgfpathlineto{\pgfqpoint{4.447087in}{1.964299in}}%
\pgfpathmoveto{\pgfqpoint{4.405420in}{1.922632in}}%
\pgfpathlineto{\pgfqpoint{4.405420in}{2.005966in}}%
\pgfusepath{stroke,fill}%
\end{pgfscope}%
\begin{pgfscope}%
\pgfpathrectangle{\pgfqpoint{2.000000in}{1.942105in}}{\pgfqpoint{4.376471in}{0.953684in}} %
\pgfusepath{clip}%
\pgfsetbuttcap%
\pgfsetroundjoin%
\definecolor{currentfill}{rgb}{1.000000,0.000000,0.000000}%
\pgfsetfillcolor{currentfill}%
\pgfsetlinewidth{2.007500pt}%
\definecolor{currentstroke}{rgb}{1.000000,0.000000,0.000000}%
\pgfsetstrokecolor{currentstroke}%
\pgfsetdash{}{0pt}%
\pgfpathmoveto{\pgfqpoint{5.276157in}{2.365619in}}%
\pgfpathlineto{\pgfqpoint{5.359491in}{2.365619in}}%
\pgfpathmoveto{\pgfqpoint{5.317824in}{2.323952in}}%
\pgfpathlineto{\pgfqpoint{5.317824in}{2.407286in}}%
\pgfusepath{stroke,fill}%
\end{pgfscope}%
\begin{pgfscope}%
\pgfpathrectangle{\pgfqpoint{2.000000in}{1.942105in}}{\pgfqpoint{4.376471in}{0.953684in}} %
\pgfusepath{clip}%
\pgfsetbuttcap%
\pgfsetroundjoin%
\definecolor{currentfill}{rgb}{1.000000,0.000000,0.000000}%
\pgfsetfillcolor{currentfill}%
\pgfsetlinewidth{2.007500pt}%
\definecolor{currentstroke}{rgb}{1.000000,0.000000,0.000000}%
\pgfsetstrokecolor{currentstroke}%
\pgfsetdash{}{0pt}%
\pgfpathmoveto{\pgfqpoint{3.044487in}{2.678546in}}%
\pgfpathlineto{\pgfqpoint{3.127821in}{2.678546in}}%
\pgfpathmoveto{\pgfqpoint{3.086154in}{2.636879in}}%
\pgfpathlineto{\pgfqpoint{3.086154in}{2.720213in}}%
\pgfusepath{stroke,fill}%
\end{pgfscope}%
\begin{pgfscope}%
\pgfpathrectangle{\pgfqpoint{2.000000in}{1.942105in}}{\pgfqpoint{4.376471in}{0.953684in}} %
\pgfusepath{clip}%
\pgfsetbuttcap%
\pgfsetroundjoin%
\definecolor{currentfill}{rgb}{1.000000,0.000000,0.000000}%
\pgfsetfillcolor{currentfill}%
\pgfsetlinewidth{2.007500pt}%
\definecolor{currentstroke}{rgb}{1.000000,0.000000,0.000000}%
\pgfsetstrokecolor{currentstroke}%
\pgfsetdash{}{0pt}%
\pgfpathmoveto{\pgfqpoint{5.168095in}{2.422791in}}%
\pgfpathlineto{\pgfqpoint{5.251429in}{2.422791in}}%
\pgfpathmoveto{\pgfqpoint{5.209762in}{2.381125in}}%
\pgfpathlineto{\pgfqpoint{5.209762in}{2.464458in}}%
\pgfusepath{stroke,fill}%
\end{pgfscope}%
\begin{pgfscope}%
\pgfpathrectangle{\pgfqpoint{2.000000in}{1.942105in}}{\pgfqpoint{4.376471in}{0.953684in}} %
\pgfusepath{clip}%
\pgfsetbuttcap%
\pgfsetroundjoin%
\definecolor{currentfill}{rgb}{1.000000,0.000000,0.000000}%
\pgfsetfillcolor{currentfill}%
\pgfsetlinewidth{2.007500pt}%
\definecolor{currentstroke}{rgb}{1.000000,0.000000,0.000000}%
\pgfsetstrokecolor{currentstroke}%
\pgfsetdash{}{0pt}%
\pgfpathmoveto{\pgfqpoint{5.181649in}{2.274926in}}%
\pgfpathlineto{\pgfqpoint{5.264982in}{2.274926in}}%
\pgfpathmoveto{\pgfqpoint{5.223316in}{2.233259in}}%
\pgfpathlineto{\pgfqpoint{5.223316in}{2.316592in}}%
\pgfusepath{stroke,fill}%
\end{pgfscope}%
\begin{pgfscope}%
\pgfpathrectangle{\pgfqpoint{2.000000in}{1.942105in}}{\pgfqpoint{4.376471in}{0.953684in}} %
\pgfusepath{clip}%
\pgfsetbuttcap%
\pgfsetroundjoin%
\definecolor{currentfill}{rgb}{1.000000,0.000000,0.000000}%
\pgfsetfillcolor{currentfill}%
\pgfsetlinewidth{2.007500pt}%
\definecolor{currentstroke}{rgb}{1.000000,0.000000,0.000000}%
\pgfsetstrokecolor{currentstroke}%
\pgfsetdash{}{0pt}%
\pgfpathmoveto{\pgfqpoint{3.570214in}{2.420341in}}%
\pgfpathlineto{\pgfqpoint{3.653547in}{2.420341in}}%
\pgfpathmoveto{\pgfqpoint{3.611881in}{2.378674in}}%
\pgfpathlineto{\pgfqpoint{3.611881in}{2.462008in}}%
\pgfusepath{stroke,fill}%
\end{pgfscope}%
\begin{pgfscope}%
\pgfpathrectangle{\pgfqpoint{2.000000in}{1.942105in}}{\pgfqpoint{4.376471in}{0.953684in}} %
\pgfusepath{clip}%
\pgfsetbuttcap%
\pgfsetroundjoin%
\definecolor{currentfill}{rgb}{1.000000,0.000000,0.000000}%
\pgfsetfillcolor{currentfill}%
\pgfsetlinewidth{2.007500pt}%
\definecolor{currentstroke}{rgb}{1.000000,0.000000,0.000000}%
\pgfsetstrokecolor{currentstroke}%
\pgfsetdash{}{0pt}%
\pgfpathmoveto{\pgfqpoint{3.285021in}{2.442963in}}%
\pgfpathlineto{\pgfqpoint{3.368355in}{2.442963in}}%
\pgfpathmoveto{\pgfqpoint{3.326688in}{2.401296in}}%
\pgfpathlineto{\pgfqpoint{3.326688in}{2.484630in}}%
\pgfusepath{stroke,fill}%
\end{pgfscope}%
\begin{pgfscope}%
\pgfpathrectangle{\pgfqpoint{2.000000in}{1.942105in}}{\pgfqpoint{4.376471in}{0.953684in}} %
\pgfusepath{clip}%
\pgfsetbuttcap%
\pgfsetroundjoin%
\definecolor{currentfill}{rgb}{1.000000,0.000000,0.000000}%
\pgfsetfillcolor{currentfill}%
\pgfsetlinewidth{2.007500pt}%
\definecolor{currentstroke}{rgb}{1.000000,0.000000,0.000000}%
\pgfsetstrokecolor{currentstroke}%
\pgfsetdash{}{0pt}%
\pgfpathmoveto{\pgfqpoint{3.937998in}{2.417655in}}%
\pgfpathlineto{\pgfqpoint{4.021331in}{2.417655in}}%
\pgfpathmoveto{\pgfqpoint{3.979664in}{2.375988in}}%
\pgfpathlineto{\pgfqpoint{3.979664in}{2.459322in}}%
\pgfusepath{stroke,fill}%
\end{pgfscope}%
\begin{pgfscope}%
\pgfpathrectangle{\pgfqpoint{2.000000in}{1.942105in}}{\pgfqpoint{4.376471in}{0.953684in}} %
\pgfusepath{clip}%
\pgfsetbuttcap%
\pgfsetroundjoin%
\definecolor{currentfill}{rgb}{1.000000,0.000000,0.000000}%
\pgfsetfillcolor{currentfill}%
\pgfsetlinewidth{2.007500pt}%
\definecolor{currentstroke}{rgb}{1.000000,0.000000,0.000000}%
\pgfsetstrokecolor{currentstroke}%
\pgfsetdash{}{0pt}%
\pgfpathmoveto{\pgfqpoint{4.107043in}{2.370447in}}%
\pgfpathlineto{\pgfqpoint{4.190376in}{2.370447in}}%
\pgfpathmoveto{\pgfqpoint{4.148710in}{2.328780in}}%
\pgfpathlineto{\pgfqpoint{4.148710in}{2.412113in}}%
\pgfusepath{stroke,fill}%
\end{pgfscope}%
\begin{pgfscope}%
\pgfpathrectangle{\pgfqpoint{2.000000in}{1.942105in}}{\pgfqpoint{4.376471in}{0.953684in}} %
\pgfusepath{clip}%
\pgfsetbuttcap%
\pgfsetroundjoin%
\definecolor{currentfill}{rgb}{0.000000,0.000000,0.000000}%
\pgfsetfillcolor{currentfill}%
\pgfsetlinewidth{1.003750pt}%
\definecolor{currentstroke}{rgb}{0.000000,0.000000,0.000000}%
\pgfsetstrokecolor{currentstroke}%
\pgfsetdash{}{0pt}%
\pgfsys@defobject{currentmarker}{\pgfqpoint{-0.020833in}{-0.020833in}}{\pgfqpoint{0.020833in}{0.020833in}}{%
\pgfpathmoveto{\pgfqpoint{0.000000in}{-0.020833in}}%
\pgfpathcurveto{\pgfqpoint{0.005525in}{-0.020833in}}{\pgfqpoint{0.010825in}{-0.018638in}}{\pgfqpoint{0.014731in}{-0.014731in}}%
\pgfpathcurveto{\pgfqpoint{0.018638in}{-0.010825in}}{\pgfqpoint{0.020833in}{-0.005525in}}{\pgfqpoint{0.020833in}{0.000000in}}%
\pgfpathcurveto{\pgfqpoint{0.020833in}{0.005525in}}{\pgfqpoint{0.018638in}{0.010825in}}{\pgfqpoint{0.014731in}{0.014731in}}%
\pgfpathcurveto{\pgfqpoint{0.010825in}{0.018638in}}{\pgfqpoint{0.005525in}{0.020833in}}{\pgfqpoint{0.000000in}{0.020833in}}%
\pgfpathcurveto{\pgfqpoint{-0.005525in}{0.020833in}}{\pgfqpoint{-0.010825in}{0.018638in}}{\pgfqpoint{-0.014731in}{0.014731in}}%
\pgfpathcurveto{\pgfqpoint{-0.018638in}{0.010825in}}{\pgfqpoint{-0.020833in}{0.005525in}}{\pgfqpoint{-0.020833in}{0.000000in}}%
\pgfpathcurveto{\pgfqpoint{-0.020833in}{-0.005525in}}{\pgfqpoint{-0.018638in}{-0.010825in}}{\pgfqpoint{-0.014731in}{-0.014731in}}%
\pgfpathcurveto{\pgfqpoint{-0.010825in}{-0.018638in}}{\pgfqpoint{-0.005525in}{-0.020833in}}{\pgfqpoint{0.000000in}{-0.020833in}}%
\pgfpathclose%
\pgfusepath{stroke,fill}%
}%
\begin{pgfscope}%
\pgfsys@transformshift{2.875294in}{2.825884in}%
\pgfsys@useobject{currentmarker}{}%
\end{pgfscope}%
\begin{pgfscope}%
\pgfsys@transformshift{2.892888in}{2.876996in}%
\pgfsys@useobject{currentmarker}{}%
\end{pgfscope}%
\begin{pgfscope}%
\pgfsys@transformshift{2.910482in}{2.789321in}%
\pgfsys@useobject{currentmarker}{}%
\end{pgfscope}%
\begin{pgfscope}%
\pgfsys@transformshift{2.928076in}{2.797403in}%
\pgfsys@useobject{currentmarker}{}%
\end{pgfscope}%
\begin{pgfscope}%
\pgfsys@transformshift{2.945670in}{2.701084in}%
\pgfsys@useobject{currentmarker}{}%
\end{pgfscope}%
\begin{pgfscope}%
\pgfsys@transformshift{2.963263in}{2.850018in}%
\pgfsys@useobject{currentmarker}{}%
\end{pgfscope}%
\begin{pgfscope}%
\pgfsys@transformshift{2.980857in}{2.657713in}%
\pgfsys@useobject{currentmarker}{}%
\end{pgfscope}%
\begin{pgfscope}%
\pgfsys@transformshift{2.998451in}{2.657675in}%
\pgfsys@useobject{currentmarker}{}%
\end{pgfscope}%
\begin{pgfscope}%
\pgfsys@transformshift{3.016045in}{2.777789in}%
\pgfsys@useobject{currentmarker}{}%
\end{pgfscope}%
\begin{pgfscope}%
\pgfsys@transformshift{3.033639in}{2.430331in}%
\pgfsys@useobject{currentmarker}{}%
\end{pgfscope}%
\begin{pgfscope}%
\pgfsys@transformshift{3.051233in}{2.412243in}%
\pgfsys@useobject{currentmarker}{}%
\end{pgfscope}%
\begin{pgfscope}%
\pgfsys@transformshift{3.068826in}{2.609910in}%
\pgfsys@useobject{currentmarker}{}%
\end{pgfscope}%
\begin{pgfscope}%
\pgfsys@transformshift{3.086420in}{2.373313in}%
\pgfsys@useobject{currentmarker}{}%
\end{pgfscope}%
\begin{pgfscope}%
\pgfsys@transformshift{3.104014in}{2.660447in}%
\pgfsys@useobject{currentmarker}{}%
\end{pgfscope}%
\begin{pgfscope}%
\pgfsys@transformshift{3.121608in}{2.405202in}%
\pgfsys@useobject{currentmarker}{}%
\end{pgfscope}%
\begin{pgfscope}%
\pgfsys@transformshift{3.139202in}{2.352529in}%
\pgfsys@useobject{currentmarker}{}%
\end{pgfscope}%
\begin{pgfscope}%
\pgfsys@transformshift{3.156796in}{2.599993in}%
\pgfsys@useobject{currentmarker}{}%
\end{pgfscope}%
\begin{pgfscope}%
\pgfsys@transformshift{3.174390in}{2.540025in}%
\pgfsys@useobject{currentmarker}{}%
\end{pgfscope}%
\begin{pgfscope}%
\pgfsys@transformshift{3.191983in}{2.564355in}%
\pgfsys@useobject{currentmarker}{}%
\end{pgfscope}%
\begin{pgfscope}%
\pgfsys@transformshift{3.209577in}{2.456689in}%
\pgfsys@useobject{currentmarker}{}%
\end{pgfscope}%
\begin{pgfscope}%
\pgfsys@transformshift{3.227171in}{2.271008in}%
\pgfsys@useobject{currentmarker}{}%
\end{pgfscope}%
\begin{pgfscope}%
\pgfsys@transformshift{3.244765in}{2.538263in}%
\pgfsys@useobject{currentmarker}{}%
\end{pgfscope}%
\begin{pgfscope}%
\pgfsys@transformshift{3.262359in}{2.315923in}%
\pgfsys@useobject{currentmarker}{}%
\end{pgfscope}%
\begin{pgfscope}%
\pgfsys@transformshift{3.279953in}{2.418379in}%
\pgfsys@useobject{currentmarker}{}%
\end{pgfscope}%
\begin{pgfscope}%
\pgfsys@transformshift{3.297547in}{2.430937in}%
\pgfsys@useobject{currentmarker}{}%
\end{pgfscope}%
\begin{pgfscope}%
\pgfsys@transformshift{3.315140in}{2.321623in}%
\pgfsys@useobject{currentmarker}{}%
\end{pgfscope}%
\begin{pgfscope}%
\pgfsys@transformshift{3.332734in}{2.400167in}%
\pgfsys@useobject{currentmarker}{}%
\end{pgfscope}%
\begin{pgfscope}%
\pgfsys@transformshift{3.350328in}{2.434787in}%
\pgfsys@useobject{currentmarker}{}%
\end{pgfscope}%
\begin{pgfscope}%
\pgfsys@transformshift{3.367922in}{2.386350in}%
\pgfsys@useobject{currentmarker}{}%
\end{pgfscope}%
\begin{pgfscope}%
\pgfsys@transformshift{3.385516in}{2.247175in}%
\pgfsys@useobject{currentmarker}{}%
\end{pgfscope}%
\begin{pgfscope}%
\pgfsys@transformshift{3.403110in}{2.394977in}%
\pgfsys@useobject{currentmarker}{}%
\end{pgfscope}%
\begin{pgfscope}%
\pgfsys@transformshift{3.420704in}{2.507448in}%
\pgfsys@useobject{currentmarker}{}%
\end{pgfscope}%
\begin{pgfscope}%
\pgfsys@transformshift{3.438297in}{2.318230in}%
\pgfsys@useobject{currentmarker}{}%
\end{pgfscope}%
\begin{pgfscope}%
\pgfsys@transformshift{3.455891in}{2.384955in}%
\pgfsys@useobject{currentmarker}{}%
\end{pgfscope}%
\begin{pgfscope}%
\pgfsys@transformshift{3.473485in}{2.370043in}%
\pgfsys@useobject{currentmarker}{}%
\end{pgfscope}%
\begin{pgfscope}%
\pgfsys@transformshift{3.491079in}{2.611106in}%
\pgfsys@useobject{currentmarker}{}%
\end{pgfscope}%
\begin{pgfscope}%
\pgfsys@transformshift{3.508673in}{2.508784in}%
\pgfsys@useobject{currentmarker}{}%
\end{pgfscope}%
\begin{pgfscope}%
\pgfsys@transformshift{3.526267in}{2.497354in}%
\pgfsys@useobject{currentmarker}{}%
\end{pgfscope}%
\begin{pgfscope}%
\pgfsys@transformshift{3.543860in}{2.395347in}%
\pgfsys@useobject{currentmarker}{}%
\end{pgfscope}%
\begin{pgfscope}%
\pgfsys@transformshift{3.561454in}{2.540233in}%
\pgfsys@useobject{currentmarker}{}%
\end{pgfscope}%
\begin{pgfscope}%
\pgfsys@transformshift{3.579048in}{2.434182in}%
\pgfsys@useobject{currentmarker}{}%
\end{pgfscope}%
\begin{pgfscope}%
\pgfsys@transformshift{3.596642in}{2.518323in}%
\pgfsys@useobject{currentmarker}{}%
\end{pgfscope}%
\begin{pgfscope}%
\pgfsys@transformshift{3.614236in}{2.465276in}%
\pgfsys@useobject{currentmarker}{}%
\end{pgfscope}%
\begin{pgfscope}%
\pgfsys@transformshift{3.631830in}{2.608033in}%
\pgfsys@useobject{currentmarker}{}%
\end{pgfscope}%
\begin{pgfscope}%
\pgfsys@transformshift{3.649424in}{2.609448in}%
\pgfsys@useobject{currentmarker}{}%
\end{pgfscope}%
\begin{pgfscope}%
\pgfsys@transformshift{3.667017in}{2.541589in}%
\pgfsys@useobject{currentmarker}{}%
\end{pgfscope}%
\begin{pgfscope}%
\pgfsys@transformshift{3.684611in}{2.610369in}%
\pgfsys@useobject{currentmarker}{}%
\end{pgfscope}%
\begin{pgfscope}%
\pgfsys@transformshift{3.702205in}{2.469691in}%
\pgfsys@useobject{currentmarker}{}%
\end{pgfscope}%
\begin{pgfscope}%
\pgfsys@transformshift{3.719799in}{2.435769in}%
\pgfsys@useobject{currentmarker}{}%
\end{pgfscope}%
\begin{pgfscope}%
\pgfsys@transformshift{3.737393in}{2.631485in}%
\pgfsys@useobject{currentmarker}{}%
\end{pgfscope}%
\begin{pgfscope}%
\pgfsys@transformshift{3.754987in}{2.606520in}%
\pgfsys@useobject{currentmarker}{}%
\end{pgfscope}%
\begin{pgfscope}%
\pgfsys@transformshift{3.772581in}{2.653318in}%
\pgfsys@useobject{currentmarker}{}%
\end{pgfscope}%
\begin{pgfscope}%
\pgfsys@transformshift{3.790174in}{2.825352in}%
\pgfsys@useobject{currentmarker}{}%
\end{pgfscope}%
\begin{pgfscope}%
\pgfsys@transformshift{3.807768in}{2.678883in}%
\pgfsys@useobject{currentmarker}{}%
\end{pgfscope}%
\begin{pgfscope}%
\pgfsys@transformshift{3.825362in}{2.488903in}%
\pgfsys@useobject{currentmarker}{}%
\end{pgfscope}%
\begin{pgfscope}%
\pgfsys@transformshift{3.842956in}{2.683174in}%
\pgfsys@useobject{currentmarker}{}%
\end{pgfscope}%
\begin{pgfscope}%
\pgfsys@transformshift{3.860550in}{2.432237in}%
\pgfsys@useobject{currentmarker}{}%
\end{pgfscope}%
\begin{pgfscope}%
\pgfsys@transformshift{3.878144in}{2.506052in}%
\pgfsys@useobject{currentmarker}{}%
\end{pgfscope}%
\begin{pgfscope}%
\pgfsys@transformshift{3.895738in}{2.532340in}%
\pgfsys@useobject{currentmarker}{}%
\end{pgfscope}%
\begin{pgfscope}%
\pgfsys@transformshift{3.913331in}{2.694901in}%
\pgfsys@useobject{currentmarker}{}%
\end{pgfscope}%
\begin{pgfscope}%
\pgfsys@transformshift{3.930925in}{2.434757in}%
\pgfsys@useobject{currentmarker}{}%
\end{pgfscope}%
\begin{pgfscope}%
\pgfsys@transformshift{3.948519in}{2.409294in}%
\pgfsys@useobject{currentmarker}{}%
\end{pgfscope}%
\begin{pgfscope}%
\pgfsys@transformshift{3.966113in}{2.463011in}%
\pgfsys@useobject{currentmarker}{}%
\end{pgfscope}%
\begin{pgfscope}%
\pgfsys@transformshift{3.983707in}{2.387182in}%
\pgfsys@useobject{currentmarker}{}%
\end{pgfscope}%
\begin{pgfscope}%
\pgfsys@transformshift{4.001301in}{2.544389in}%
\pgfsys@useobject{currentmarker}{}%
\end{pgfscope}%
\begin{pgfscope}%
\pgfsys@transformshift{4.018894in}{2.303809in}%
\pgfsys@useobject{currentmarker}{}%
\end{pgfscope}%
\begin{pgfscope}%
\pgfsys@transformshift{4.036488in}{2.275163in}%
\pgfsys@useobject{currentmarker}{}%
\end{pgfscope}%
\begin{pgfscope}%
\pgfsys@transformshift{4.054082in}{2.323357in}%
\pgfsys@useobject{currentmarker}{}%
\end{pgfscope}%
\begin{pgfscope}%
\pgfsys@transformshift{4.071676in}{2.294902in}%
\pgfsys@useobject{currentmarker}{}%
\end{pgfscope}%
\begin{pgfscope}%
\pgfsys@transformshift{4.089270in}{2.513620in}%
\pgfsys@useobject{currentmarker}{}%
\end{pgfscope}%
\begin{pgfscope}%
\pgfsys@transformshift{4.106864in}{2.394045in}%
\pgfsys@useobject{currentmarker}{}%
\end{pgfscope}%
\begin{pgfscope}%
\pgfsys@transformshift{4.124458in}{2.286650in}%
\pgfsys@useobject{currentmarker}{}%
\end{pgfscope}%
\begin{pgfscope}%
\pgfsys@transformshift{4.142051in}{2.135068in}%
\pgfsys@useobject{currentmarker}{}%
\end{pgfscope}%
\begin{pgfscope}%
\pgfsys@transformshift{4.159645in}{2.320360in}%
\pgfsys@useobject{currentmarker}{}%
\end{pgfscope}%
\begin{pgfscope}%
\pgfsys@transformshift{4.177239in}{2.117792in}%
\pgfsys@useobject{currentmarker}{}%
\end{pgfscope}%
\begin{pgfscope}%
\pgfsys@transformshift{4.194833in}{2.045548in}%
\pgfsys@useobject{currentmarker}{}%
\end{pgfscope}%
\begin{pgfscope}%
\pgfsys@transformshift{4.212427in}{2.300224in}%
\pgfsys@useobject{currentmarker}{}%
\end{pgfscope}%
\begin{pgfscope}%
\pgfsys@transformshift{4.230021in}{2.198372in}%
\pgfsys@useobject{currentmarker}{}%
\end{pgfscope}%
\begin{pgfscope}%
\pgfsys@transformshift{4.247615in}{2.244676in}%
\pgfsys@useobject{currentmarker}{}%
\end{pgfscope}%
\begin{pgfscope}%
\pgfsys@transformshift{4.265208in}{2.172916in}%
\pgfsys@useobject{currentmarker}{}%
\end{pgfscope}%
\begin{pgfscope}%
\pgfsys@transformshift{4.282802in}{2.216278in}%
\pgfsys@useobject{currentmarker}{}%
\end{pgfscope}%
\begin{pgfscope}%
\pgfsys@transformshift{4.300396in}{2.058327in}%
\pgfsys@useobject{currentmarker}{}%
\end{pgfscope}%
\begin{pgfscope}%
\pgfsys@transformshift{4.317990in}{2.014102in}%
\pgfsys@useobject{currentmarker}{}%
\end{pgfscope}%
\begin{pgfscope}%
\pgfsys@transformshift{4.335584in}{2.180451in}%
\pgfsys@useobject{currentmarker}{}%
\end{pgfscope}%
\begin{pgfscope}%
\pgfsys@transformshift{4.353178in}{2.030900in}%
\pgfsys@useobject{currentmarker}{}%
\end{pgfscope}%
\begin{pgfscope}%
\pgfsys@transformshift{4.370772in}{2.042269in}%
\pgfsys@useobject{currentmarker}{}%
\end{pgfscope}%
\begin{pgfscope}%
\pgfsys@transformshift{4.388365in}{2.067647in}%
\pgfsys@useobject{currentmarker}{}%
\end{pgfscope}%
\begin{pgfscope}%
\pgfsys@transformshift{4.405959in}{2.118807in}%
\pgfsys@useobject{currentmarker}{}%
\end{pgfscope}%
\begin{pgfscope}%
\pgfsys@transformshift{4.423553in}{2.088061in}%
\pgfsys@useobject{currentmarker}{}%
\end{pgfscope}%
\begin{pgfscope}%
\pgfsys@transformshift{4.441147in}{1.994709in}%
\pgfsys@useobject{currentmarker}{}%
\end{pgfscope}%
\begin{pgfscope}%
\pgfsys@transformshift{4.458741in}{2.077258in}%
\pgfsys@useobject{currentmarker}{}%
\end{pgfscope}%
\begin{pgfscope}%
\pgfsys@transformshift{4.476335in}{1.931933in}%
\pgfsys@useobject{currentmarker}{}%
\end{pgfscope}%
\begin{pgfscope}%
\pgfsys@transformshift{4.493928in}{2.228129in}%
\pgfsys@useobject{currentmarker}{}%
\end{pgfscope}%
\begin{pgfscope}%
\pgfsys@transformshift{4.511522in}{2.021524in}%
\pgfsys@useobject{currentmarker}{}%
\end{pgfscope}%
\begin{pgfscope}%
\pgfsys@transformshift{4.529116in}{2.086922in}%
\pgfsys@useobject{currentmarker}{}%
\end{pgfscope}%
\begin{pgfscope}%
\pgfsys@transformshift{4.546710in}{2.218907in}%
\pgfsys@useobject{currentmarker}{}%
\end{pgfscope}%
\begin{pgfscope}%
\pgfsys@transformshift{4.564304in}{2.158302in}%
\pgfsys@useobject{currentmarker}{}%
\end{pgfscope}%
\begin{pgfscope}%
\pgfsys@transformshift{4.581898in}{2.403877in}%
\pgfsys@useobject{currentmarker}{}%
\end{pgfscope}%
\begin{pgfscope}%
\pgfsys@transformshift{4.599492in}{2.141553in}%
\pgfsys@useobject{currentmarker}{}%
\end{pgfscope}%
\begin{pgfscope}%
\pgfsys@transformshift{4.617085in}{2.316296in}%
\pgfsys@useobject{currentmarker}{}%
\end{pgfscope}%
\begin{pgfscope}%
\pgfsys@transformshift{4.634679in}{2.305824in}%
\pgfsys@useobject{currentmarker}{}%
\end{pgfscope}%
\begin{pgfscope}%
\pgfsys@transformshift{4.652273in}{2.213625in}%
\pgfsys@useobject{currentmarker}{}%
\end{pgfscope}%
\begin{pgfscope}%
\pgfsys@transformshift{4.669867in}{2.401368in}%
\pgfsys@useobject{currentmarker}{}%
\end{pgfscope}%
\begin{pgfscope}%
\pgfsys@transformshift{4.687461in}{2.351643in}%
\pgfsys@useobject{currentmarker}{}%
\end{pgfscope}%
\begin{pgfscope}%
\pgfsys@transformshift{4.705055in}{2.464046in}%
\pgfsys@useobject{currentmarker}{}%
\end{pgfscope}%
\begin{pgfscope}%
\pgfsys@transformshift{4.722649in}{2.487077in}%
\pgfsys@useobject{currentmarker}{}%
\end{pgfscope}%
\begin{pgfscope}%
\pgfsys@transformshift{4.740242in}{2.637090in}%
\pgfsys@useobject{currentmarker}{}%
\end{pgfscope}%
\begin{pgfscope}%
\pgfsys@transformshift{4.757836in}{2.570746in}%
\pgfsys@useobject{currentmarker}{}%
\end{pgfscope}%
\begin{pgfscope}%
\pgfsys@transformshift{4.775430in}{2.415788in}%
\pgfsys@useobject{currentmarker}{}%
\end{pgfscope}%
\begin{pgfscope}%
\pgfsys@transformshift{4.793024in}{2.441741in}%
\pgfsys@useobject{currentmarker}{}%
\end{pgfscope}%
\begin{pgfscope}%
\pgfsys@transformshift{4.810618in}{2.586267in}%
\pgfsys@useobject{currentmarker}{}%
\end{pgfscope}%
\begin{pgfscope}%
\pgfsys@transformshift{4.828212in}{2.551958in}%
\pgfsys@useobject{currentmarker}{}%
\end{pgfscope}%
\begin{pgfscope}%
\pgfsys@transformshift{4.845805in}{2.558538in}%
\pgfsys@useobject{currentmarker}{}%
\end{pgfscope}%
\begin{pgfscope}%
\pgfsys@transformshift{4.863399in}{2.340572in}%
\pgfsys@useobject{currentmarker}{}%
\end{pgfscope}%
\begin{pgfscope}%
\pgfsys@transformshift{4.880993in}{2.503165in}%
\pgfsys@useobject{currentmarker}{}%
\end{pgfscope}%
\begin{pgfscope}%
\pgfsys@transformshift{4.898587in}{2.434730in}%
\pgfsys@useobject{currentmarker}{}%
\end{pgfscope}%
\begin{pgfscope}%
\pgfsys@transformshift{4.916181in}{2.536397in}%
\pgfsys@useobject{currentmarker}{}%
\end{pgfscope}%
\begin{pgfscope}%
\pgfsys@transformshift{4.933775in}{2.497462in}%
\pgfsys@useobject{currentmarker}{}%
\end{pgfscope}%
\begin{pgfscope}%
\pgfsys@transformshift{4.951369in}{2.594366in}%
\pgfsys@useobject{currentmarker}{}%
\end{pgfscope}%
\begin{pgfscope}%
\pgfsys@transformshift{4.968962in}{2.530352in}%
\pgfsys@useobject{currentmarker}{}%
\end{pgfscope}%
\begin{pgfscope}%
\pgfsys@transformshift{4.986556in}{2.570072in}%
\pgfsys@useobject{currentmarker}{}%
\end{pgfscope}%
\begin{pgfscope}%
\pgfsys@transformshift{5.004150in}{2.437049in}%
\pgfsys@useobject{currentmarker}{}%
\end{pgfscope}%
\begin{pgfscope}%
\pgfsys@transformshift{5.021744in}{2.379403in}%
\pgfsys@useobject{currentmarker}{}%
\end{pgfscope}%
\begin{pgfscope}%
\pgfsys@transformshift{5.039338in}{2.420867in}%
\pgfsys@useobject{currentmarker}{}%
\end{pgfscope}%
\begin{pgfscope}%
\pgfsys@transformshift{5.056932in}{2.446959in}%
\pgfsys@useobject{currentmarker}{}%
\end{pgfscope}%
\begin{pgfscope}%
\pgfsys@transformshift{5.074526in}{2.472182in}%
\pgfsys@useobject{currentmarker}{}%
\end{pgfscope}%
\begin{pgfscope}%
\pgfsys@transformshift{5.092119in}{2.643812in}%
\pgfsys@useobject{currentmarker}{}%
\end{pgfscope}%
\begin{pgfscope}%
\pgfsys@transformshift{5.109713in}{2.399178in}%
\pgfsys@useobject{currentmarker}{}%
\end{pgfscope}%
\begin{pgfscope}%
\pgfsys@transformshift{5.127307in}{2.291759in}%
\pgfsys@useobject{currentmarker}{}%
\end{pgfscope}%
\begin{pgfscope}%
\pgfsys@transformshift{5.144901in}{2.335234in}%
\pgfsys@useobject{currentmarker}{}%
\end{pgfscope}%
\begin{pgfscope}%
\pgfsys@transformshift{5.162495in}{2.306158in}%
\pgfsys@useobject{currentmarker}{}%
\end{pgfscope}%
\begin{pgfscope}%
\pgfsys@transformshift{5.180089in}{2.382579in}%
\pgfsys@useobject{currentmarker}{}%
\end{pgfscope}%
\begin{pgfscope}%
\pgfsys@transformshift{5.197683in}{2.164346in}%
\pgfsys@useobject{currentmarker}{}%
\end{pgfscope}%
\begin{pgfscope}%
\pgfsys@transformshift{5.215276in}{2.306691in}%
\pgfsys@useobject{currentmarker}{}%
\end{pgfscope}%
\begin{pgfscope}%
\pgfsys@transformshift{5.232870in}{2.299476in}%
\pgfsys@useobject{currentmarker}{}%
\end{pgfscope}%
\begin{pgfscope}%
\pgfsys@transformshift{5.250464in}{2.291238in}%
\pgfsys@useobject{currentmarker}{}%
\end{pgfscope}%
\begin{pgfscope}%
\pgfsys@transformshift{5.268058in}{2.193987in}%
\pgfsys@useobject{currentmarker}{}%
\end{pgfscope}%
\begin{pgfscope}%
\pgfsys@transformshift{5.285652in}{2.215932in}%
\pgfsys@useobject{currentmarker}{}%
\end{pgfscope}%
\begin{pgfscope}%
\pgfsys@transformshift{5.303246in}{2.085597in}%
\pgfsys@useobject{currentmarker}{}%
\end{pgfscope}%
\begin{pgfscope}%
\pgfsys@transformshift{5.320839in}{2.166972in}%
\pgfsys@useobject{currentmarker}{}%
\end{pgfscope}%
\begin{pgfscope}%
\pgfsys@transformshift{5.338433in}{2.152521in}%
\pgfsys@useobject{currentmarker}{}%
\end{pgfscope}%
\begin{pgfscope}%
\pgfsys@transformshift{5.356027in}{2.239938in}%
\pgfsys@useobject{currentmarker}{}%
\end{pgfscope}%
\begin{pgfscope}%
\pgfsys@transformshift{5.373621in}{2.077705in}%
\pgfsys@useobject{currentmarker}{}%
\end{pgfscope}%
\begin{pgfscope}%
\pgfsys@transformshift{5.391215in}{2.266017in}%
\pgfsys@useobject{currentmarker}{}%
\end{pgfscope}%
\begin{pgfscope}%
\pgfsys@transformshift{5.408809in}{2.334696in}%
\pgfsys@useobject{currentmarker}{}%
\end{pgfscope}%
\begin{pgfscope}%
\pgfsys@transformshift{5.426403in}{1.980441in}%
\pgfsys@useobject{currentmarker}{}%
\end{pgfscope}%
\begin{pgfscope}%
\pgfsys@transformshift{5.443996in}{2.230336in}%
\pgfsys@useobject{currentmarker}{}%
\end{pgfscope}%
\begin{pgfscope}%
\pgfsys@transformshift{5.461590in}{2.259233in}%
\pgfsys@useobject{currentmarker}{}%
\end{pgfscope}%
\begin{pgfscope}%
\pgfsys@transformshift{5.479184in}{2.134742in}%
\pgfsys@useobject{currentmarker}{}%
\end{pgfscope}%
\begin{pgfscope}%
\pgfsys@transformshift{5.496778in}{2.166738in}%
\pgfsys@useobject{currentmarker}{}%
\end{pgfscope}%
\begin{pgfscope}%
\pgfsys@transformshift{5.514372in}{2.203158in}%
\pgfsys@useobject{currentmarker}{}%
\end{pgfscope}%
\begin{pgfscope}%
\pgfsys@transformshift{5.531966in}{2.198885in}%
\pgfsys@useobject{currentmarker}{}%
\end{pgfscope}%
\begin{pgfscope}%
\pgfsys@transformshift{5.549560in}{2.211754in}%
\pgfsys@useobject{currentmarker}{}%
\end{pgfscope}%
\begin{pgfscope}%
\pgfsys@transformshift{5.567153in}{2.091712in}%
\pgfsys@useobject{currentmarker}{}%
\end{pgfscope}%
\begin{pgfscope}%
\pgfsys@transformshift{5.584747in}{2.390102in}%
\pgfsys@useobject{currentmarker}{}%
\end{pgfscope}%
\begin{pgfscope}%
\pgfsys@transformshift{5.602341in}{2.401822in}%
\pgfsys@useobject{currentmarker}{}%
\end{pgfscope}%
\begin{pgfscope}%
\pgfsys@transformshift{5.619935in}{2.234093in}%
\pgfsys@useobject{currentmarker}{}%
\end{pgfscope}%
\begin{pgfscope}%
\pgfsys@transformshift{5.637529in}{2.190761in}%
\pgfsys@useobject{currentmarker}{}%
\end{pgfscope}%
\begin{pgfscope}%
\pgfsys@transformshift{5.655123in}{2.410796in}%
\pgfsys@useobject{currentmarker}{}%
\end{pgfscope}%
\begin{pgfscope}%
\pgfsys@transformshift{5.672717in}{2.325294in}%
\pgfsys@useobject{currentmarker}{}%
\end{pgfscope}%
\begin{pgfscope}%
\pgfsys@transformshift{5.690310in}{2.420737in}%
\pgfsys@useobject{currentmarker}{}%
\end{pgfscope}%
\begin{pgfscope}%
\pgfsys@transformshift{5.707904in}{2.399537in}%
\pgfsys@useobject{currentmarker}{}%
\end{pgfscope}%
\begin{pgfscope}%
\pgfsys@transformshift{5.725498in}{2.524868in}%
\pgfsys@useobject{currentmarker}{}%
\end{pgfscope}%
\begin{pgfscope}%
\pgfsys@transformshift{5.743092in}{2.550181in}%
\pgfsys@useobject{currentmarker}{}%
\end{pgfscope}%
\begin{pgfscope}%
\pgfsys@transformshift{5.760686in}{2.433957in}%
\pgfsys@useobject{currentmarker}{}%
\end{pgfscope}%
\begin{pgfscope}%
\pgfsys@transformshift{5.778280in}{2.393173in}%
\pgfsys@useobject{currentmarker}{}%
\end{pgfscope}%
\begin{pgfscope}%
\pgfsys@transformshift{5.795873in}{2.397395in}%
\pgfsys@useobject{currentmarker}{}%
\end{pgfscope}%
\begin{pgfscope}%
\pgfsys@transformshift{5.813467in}{2.638631in}%
\pgfsys@useobject{currentmarker}{}%
\end{pgfscope}%
\begin{pgfscope}%
\pgfsys@transformshift{5.831061in}{2.482355in}%
\pgfsys@useobject{currentmarker}{}%
\end{pgfscope}%
\begin{pgfscope}%
\pgfsys@transformshift{5.848655in}{2.571521in}%
\pgfsys@useobject{currentmarker}{}%
\end{pgfscope}%
\begin{pgfscope}%
\pgfsys@transformshift{5.866249in}{2.582827in}%
\pgfsys@useobject{currentmarker}{}%
\end{pgfscope}%
\begin{pgfscope}%
\pgfsys@transformshift{5.883843in}{2.655679in}%
\pgfsys@useobject{currentmarker}{}%
\end{pgfscope}%
\begin{pgfscope}%
\pgfsys@transformshift{5.901437in}{2.485889in}%
\pgfsys@useobject{currentmarker}{}%
\end{pgfscope}%
\begin{pgfscope}%
\pgfsys@transformshift{5.919030in}{2.712727in}%
\pgfsys@useobject{currentmarker}{}%
\end{pgfscope}%
\begin{pgfscope}%
\pgfsys@transformshift{5.936624in}{2.760039in}%
\pgfsys@useobject{currentmarker}{}%
\end{pgfscope}%
\begin{pgfscope}%
\pgfsys@transformshift{5.954218in}{2.729007in}%
\pgfsys@useobject{currentmarker}{}%
\end{pgfscope}%
\begin{pgfscope}%
\pgfsys@transformshift{5.971812in}{2.699714in}%
\pgfsys@useobject{currentmarker}{}%
\end{pgfscope}%
\begin{pgfscope}%
\pgfsys@transformshift{5.989406in}{2.748751in}%
\pgfsys@useobject{currentmarker}{}%
\end{pgfscope}%
\begin{pgfscope}%
\pgfsys@transformshift{6.007000in}{2.785274in}%
\pgfsys@useobject{currentmarker}{}%
\end{pgfscope}%
\begin{pgfscope}%
\pgfsys@transformshift{6.024594in}{2.474829in}%
\pgfsys@useobject{currentmarker}{}%
\end{pgfscope}%
\begin{pgfscope}%
\pgfsys@transformshift{6.042187in}{2.947295in}%
\pgfsys@useobject{currentmarker}{}%
\end{pgfscope}%
\begin{pgfscope}%
\pgfsys@transformshift{6.059781in}{2.792601in}%
\pgfsys@useobject{currentmarker}{}%
\end{pgfscope}%
\begin{pgfscope}%
\pgfsys@transformshift{6.077375in}{2.688082in}%
\pgfsys@useobject{currentmarker}{}%
\end{pgfscope}%
\begin{pgfscope}%
\pgfsys@transformshift{6.094969in}{2.711365in}%
\pgfsys@useobject{currentmarker}{}%
\end{pgfscope}%
\begin{pgfscope}%
\pgfsys@transformshift{6.112563in}{2.794920in}%
\pgfsys@useobject{currentmarker}{}%
\end{pgfscope}%
\begin{pgfscope}%
\pgfsys@transformshift{6.130157in}{2.728474in}%
\pgfsys@useobject{currentmarker}{}%
\end{pgfscope}%
\begin{pgfscope}%
\pgfsys@transformshift{6.147751in}{2.531011in}%
\pgfsys@useobject{currentmarker}{}%
\end{pgfscope}%
\begin{pgfscope}%
\pgfsys@transformshift{6.165344in}{2.929312in}%
\pgfsys@useobject{currentmarker}{}%
\end{pgfscope}%
\begin{pgfscope}%
\pgfsys@transformshift{6.182938in}{2.703614in}%
\pgfsys@useobject{currentmarker}{}%
\end{pgfscope}%
\begin{pgfscope}%
\pgfsys@transformshift{6.200532in}{2.805382in}%
\pgfsys@useobject{currentmarker}{}%
\end{pgfscope}%
\begin{pgfscope}%
\pgfsys@transformshift{6.218126in}{2.623972in}%
\pgfsys@useobject{currentmarker}{}%
\end{pgfscope}%
\begin{pgfscope}%
\pgfsys@transformshift{6.235720in}{2.833466in}%
\pgfsys@useobject{currentmarker}{}%
\end{pgfscope}%
\begin{pgfscope}%
\pgfsys@transformshift{6.253314in}{2.697027in}%
\pgfsys@useobject{currentmarker}{}%
\end{pgfscope}%
\begin{pgfscope}%
\pgfsys@transformshift{6.270907in}{2.716632in}%
\pgfsys@useobject{currentmarker}{}%
\end{pgfscope}%
\begin{pgfscope}%
\pgfsys@transformshift{6.288501in}{2.539901in}%
\pgfsys@useobject{currentmarker}{}%
\end{pgfscope}%
\begin{pgfscope}%
\pgfsys@transformshift{6.306095in}{2.751873in}%
\pgfsys@useobject{currentmarker}{}%
\end{pgfscope}%
\begin{pgfscope}%
\pgfsys@transformshift{6.323689in}{2.688326in}%
\pgfsys@useobject{currentmarker}{}%
\end{pgfscope}%
\begin{pgfscope}%
\pgfsys@transformshift{6.341283in}{2.737992in}%
\pgfsys@useobject{currentmarker}{}%
\end{pgfscope}%
\begin{pgfscope}%
\pgfsys@transformshift{6.358877in}{2.535946in}%
\pgfsys@useobject{currentmarker}{}%
\end{pgfscope}%
\begin{pgfscope}%
\pgfsys@transformshift{6.376471in}{2.541231in}%
\pgfsys@useobject{currentmarker}{}%
\end{pgfscope}%
\end{pgfscope}%
\begin{pgfscope}%
\pgfsetbuttcap%
\pgfsetroundjoin%
\definecolor{currentfill}{rgb}{0.000000,0.000000,0.000000}%
\pgfsetfillcolor{currentfill}%
\pgfsetlinewidth{0.803000pt}%
\definecolor{currentstroke}{rgb}{0.000000,0.000000,0.000000}%
\pgfsetstrokecolor{currentstroke}%
\pgfsetdash{}{0pt}%
\pgfsys@defobject{currentmarker}{\pgfqpoint{0.000000in}{-0.048611in}}{\pgfqpoint{0.000000in}{0.000000in}}{%
\pgfpathmoveto{\pgfqpoint{0.000000in}{0.000000in}}%
\pgfpathlineto{\pgfqpoint{0.000000in}{-0.048611in}}%
\pgfusepath{stroke,fill}%
}%
\begin{pgfscope}%
\pgfsys@transformshift{2.000000in}{1.942105in}%
\pgfsys@useobject{currentmarker}{}%
\end{pgfscope}%
\end{pgfscope}%
\begin{pgfscope}%
\pgfsetbuttcap%
\pgfsetroundjoin%
\definecolor{currentfill}{rgb}{0.000000,0.000000,0.000000}%
\pgfsetfillcolor{currentfill}%
\pgfsetlinewidth{0.803000pt}%
\definecolor{currentstroke}{rgb}{0.000000,0.000000,0.000000}%
\pgfsetstrokecolor{currentstroke}%
\pgfsetdash{}{0pt}%
\pgfsys@defobject{currentmarker}{\pgfqpoint{0.000000in}{-0.048611in}}{\pgfqpoint{0.000000in}{0.000000in}}{%
\pgfpathmoveto{\pgfqpoint{0.000000in}{0.000000in}}%
\pgfpathlineto{\pgfqpoint{0.000000in}{-0.048611in}}%
\pgfusepath{stroke,fill}%
}%
\begin{pgfscope}%
\pgfsys@transformshift{2.875294in}{1.942105in}%
\pgfsys@useobject{currentmarker}{}%
\end{pgfscope}%
\end{pgfscope}%
\begin{pgfscope}%
\pgfsetbuttcap%
\pgfsetroundjoin%
\definecolor{currentfill}{rgb}{0.000000,0.000000,0.000000}%
\pgfsetfillcolor{currentfill}%
\pgfsetlinewidth{0.803000pt}%
\definecolor{currentstroke}{rgb}{0.000000,0.000000,0.000000}%
\pgfsetstrokecolor{currentstroke}%
\pgfsetdash{}{0pt}%
\pgfsys@defobject{currentmarker}{\pgfqpoint{0.000000in}{-0.048611in}}{\pgfqpoint{0.000000in}{0.000000in}}{%
\pgfpathmoveto{\pgfqpoint{0.000000in}{0.000000in}}%
\pgfpathlineto{\pgfqpoint{0.000000in}{-0.048611in}}%
\pgfusepath{stroke,fill}%
}%
\begin{pgfscope}%
\pgfsys@transformshift{3.750588in}{1.942105in}%
\pgfsys@useobject{currentmarker}{}%
\end{pgfscope}%
\end{pgfscope}%
\begin{pgfscope}%
\pgfsetbuttcap%
\pgfsetroundjoin%
\definecolor{currentfill}{rgb}{0.000000,0.000000,0.000000}%
\pgfsetfillcolor{currentfill}%
\pgfsetlinewidth{0.803000pt}%
\definecolor{currentstroke}{rgb}{0.000000,0.000000,0.000000}%
\pgfsetstrokecolor{currentstroke}%
\pgfsetdash{}{0pt}%
\pgfsys@defobject{currentmarker}{\pgfqpoint{0.000000in}{-0.048611in}}{\pgfqpoint{0.000000in}{0.000000in}}{%
\pgfpathmoveto{\pgfqpoint{0.000000in}{0.000000in}}%
\pgfpathlineto{\pgfqpoint{0.000000in}{-0.048611in}}%
\pgfusepath{stroke,fill}%
}%
\begin{pgfscope}%
\pgfsys@transformshift{4.625882in}{1.942105in}%
\pgfsys@useobject{currentmarker}{}%
\end{pgfscope}%
\end{pgfscope}%
\begin{pgfscope}%
\pgfsetbuttcap%
\pgfsetroundjoin%
\definecolor{currentfill}{rgb}{0.000000,0.000000,0.000000}%
\pgfsetfillcolor{currentfill}%
\pgfsetlinewidth{0.803000pt}%
\definecolor{currentstroke}{rgb}{0.000000,0.000000,0.000000}%
\pgfsetstrokecolor{currentstroke}%
\pgfsetdash{}{0pt}%
\pgfsys@defobject{currentmarker}{\pgfqpoint{0.000000in}{-0.048611in}}{\pgfqpoint{0.000000in}{0.000000in}}{%
\pgfpathmoveto{\pgfqpoint{0.000000in}{0.000000in}}%
\pgfpathlineto{\pgfqpoint{0.000000in}{-0.048611in}}%
\pgfusepath{stroke,fill}%
}%
\begin{pgfscope}%
\pgfsys@transformshift{5.501176in}{1.942105in}%
\pgfsys@useobject{currentmarker}{}%
\end{pgfscope}%
\end{pgfscope}%
\begin{pgfscope}%
\pgfsetbuttcap%
\pgfsetroundjoin%
\definecolor{currentfill}{rgb}{0.000000,0.000000,0.000000}%
\pgfsetfillcolor{currentfill}%
\pgfsetlinewidth{0.803000pt}%
\definecolor{currentstroke}{rgb}{0.000000,0.000000,0.000000}%
\pgfsetstrokecolor{currentstroke}%
\pgfsetdash{}{0pt}%
\pgfsys@defobject{currentmarker}{\pgfqpoint{0.000000in}{-0.048611in}}{\pgfqpoint{0.000000in}{0.000000in}}{%
\pgfpathmoveto{\pgfqpoint{0.000000in}{0.000000in}}%
\pgfpathlineto{\pgfqpoint{0.000000in}{-0.048611in}}%
\pgfusepath{stroke,fill}%
}%
\begin{pgfscope}%
\pgfsys@transformshift{6.376471in}{1.942105in}%
\pgfsys@useobject{currentmarker}{}%
\end{pgfscope}%
\end{pgfscope}%
\begin{pgfscope}%
\pgfsetbuttcap%
\pgfsetroundjoin%
\definecolor{currentfill}{rgb}{0.000000,0.000000,0.000000}%
\pgfsetfillcolor{currentfill}%
\pgfsetlinewidth{0.803000pt}%
\definecolor{currentstroke}{rgb}{0.000000,0.000000,0.000000}%
\pgfsetstrokecolor{currentstroke}%
\pgfsetdash{}{0pt}%
\pgfsys@defobject{currentmarker}{\pgfqpoint{-0.048611in}{0.000000in}}{\pgfqpoint{0.000000in}{0.000000in}}{%
\pgfpathmoveto{\pgfqpoint{0.000000in}{0.000000in}}%
\pgfpathlineto{\pgfqpoint{-0.048611in}{0.000000in}}%
\pgfusepath{stroke,fill}%
}%
\begin{pgfscope}%
\pgfsys@transformshift{2.000000in}{2.299737in}%
\pgfsys@useobject{currentmarker}{}%
\end{pgfscope}%
\end{pgfscope}%
\begin{pgfscope}%
\pgftext[x=1.833333in,y=2.251519in,left,base]{\rmfamily\fontsize{10.000000}{12.000000}\selectfont \(\displaystyle 0\)}%
\end{pgfscope}%
\begin{pgfscope}%
\pgfsetbuttcap%
\pgfsetroundjoin%
\definecolor{currentfill}{rgb}{0.000000,0.000000,0.000000}%
\pgfsetfillcolor{currentfill}%
\pgfsetlinewidth{0.803000pt}%
\definecolor{currentstroke}{rgb}{0.000000,0.000000,0.000000}%
\pgfsetstrokecolor{currentstroke}%
\pgfsetdash{}{0pt}%
\pgfsys@defobject{currentmarker}{\pgfqpoint{-0.048611in}{0.000000in}}{\pgfqpoint{0.000000in}{0.000000in}}{%
\pgfpathmoveto{\pgfqpoint{0.000000in}{0.000000in}}%
\pgfpathlineto{\pgfqpoint{-0.048611in}{0.000000in}}%
\pgfusepath{stroke,fill}%
}%
\begin{pgfscope}%
\pgfsys@transformshift{2.000000in}{2.697105in}%
\pgfsys@useobject{currentmarker}{}%
\end{pgfscope}%
\end{pgfscope}%
\begin{pgfscope}%
\pgftext[x=1.833333in,y=2.648887in,left,base]{\rmfamily\fontsize{10.000000}{12.000000}\selectfont \(\displaystyle 2\)}%
\end{pgfscope}%
\begin{pgfscope}%
\pgftext[x=1.777777in,y=2.418947in,,bottom,rotate=90.000000]{\rmfamily\fontsize{10.000000}{12.000000}\selectfont y}%
\end{pgfscope}%
\begin{pgfscope}%
\pgfpathrectangle{\pgfqpoint{2.000000in}{1.942105in}}{\pgfqpoint{4.376471in}{0.953684in}} %
\pgfusepath{clip}%
\pgfsetrectcap%
\pgfsetroundjoin%
\pgfsetlinewidth{1.505625pt}%
\definecolor{currentstroke}{rgb}{0.121569,0.466667,0.705882}%
\pgfsetstrokecolor{currentstroke}%
\pgfsetdash{}{0pt}%
\pgfpathmoveto{\pgfqpoint{2.875294in}{2.491415in}}%
\pgfpathlineto{\pgfqpoint{2.892888in}{2.541628in}}%
\pgfpathlineto{\pgfqpoint{2.910482in}{2.581795in}}%
\pgfpathlineto{\pgfqpoint{2.928076in}{2.612617in}}%
\pgfpathlineto{\pgfqpoint{2.945670in}{2.634791in}}%
\pgfpathlineto{\pgfqpoint{2.963263in}{2.649011in}}%
\pgfpathlineto{\pgfqpoint{2.980857in}{2.655973in}}%
\pgfpathlineto{\pgfqpoint{2.998451in}{2.656374in}}%
\pgfpathlineto{\pgfqpoint{3.016045in}{2.650910in}}%
\pgfpathlineto{\pgfqpoint{3.033639in}{2.640279in}}%
\pgfpathlineto{\pgfqpoint{3.051233in}{2.625174in}}%
\pgfpathlineto{\pgfqpoint{3.068826in}{2.606283in}}%
\pgfpathlineto{\pgfqpoint{3.104014in}{2.559840in}}%
\pgfpathlineto{\pgfqpoint{3.139202in}{2.506187in}}%
\pgfpathlineto{\pgfqpoint{3.209577in}{2.396282in}}%
\pgfpathlineto{\pgfqpoint{3.244765in}{2.348185in}}%
\pgfpathlineto{\pgfqpoint{3.262359in}{2.327300in}}%
\pgfpathlineto{\pgfqpoint{3.279953in}{2.308948in}}%
\pgfpathlineto{\pgfqpoint{3.297547in}{2.293381in}}%
\pgfpathlineto{\pgfqpoint{3.315140in}{2.280798in}}%
\pgfpathlineto{\pgfqpoint{3.332734in}{2.271345in}}%
\pgfpathlineto{\pgfqpoint{3.350328in}{2.265113in}}%
\pgfpathlineto{\pgfqpoint{3.367922in}{2.262140in}}%
\pgfpathlineto{\pgfqpoint{3.385516in}{2.262414in}}%
\pgfpathlineto{\pgfqpoint{3.403110in}{2.265872in}}%
\pgfpathlineto{\pgfqpoint{3.420704in}{2.272404in}}%
\pgfpathlineto{\pgfqpoint{3.438297in}{2.281857in}}%
\pgfpathlineto{\pgfqpoint{3.455891in}{2.294035in}}%
\pgfpathlineto{\pgfqpoint{3.473485in}{2.308706in}}%
\pgfpathlineto{\pgfqpoint{3.508673in}{2.344447in}}%
\pgfpathlineto{\pgfqpoint{3.543860in}{2.386656in}}%
\pgfpathlineto{\pgfqpoint{3.667017in}{2.545268in}}%
\pgfpathlineto{\pgfqpoint{3.702205in}{2.582375in}}%
\pgfpathlineto{\pgfqpoint{3.719799in}{2.598053in}}%
\pgfpathlineto{\pgfqpoint{3.737393in}{2.611498in}}%
\pgfpathlineto{\pgfqpoint{3.754987in}{2.622516in}}%
\pgfpathlineto{\pgfqpoint{3.772581in}{2.630941in}}%
\pgfpathlineto{\pgfqpoint{3.790174in}{2.636646in}}%
\pgfpathlineto{\pgfqpoint{3.807768in}{2.639538in}}%
\pgfpathlineto{\pgfqpoint{3.825362in}{2.639562in}}%
\pgfpathlineto{\pgfqpoint{3.842956in}{2.636699in}}%
\pgfpathlineto{\pgfqpoint{3.860550in}{2.630967in}}%
\pgfpathlineto{\pgfqpoint{3.878144in}{2.622419in}}%
\pgfpathlineto{\pgfqpoint{3.895738in}{2.611141in}}%
\pgfpathlineto{\pgfqpoint{3.913331in}{2.597254in}}%
\pgfpathlineto{\pgfqpoint{3.930925in}{2.580907in}}%
\pgfpathlineto{\pgfqpoint{3.966113in}{2.541572in}}%
\pgfpathlineto{\pgfqpoint{4.001301in}{2.494846in}}%
\pgfpathlineto{\pgfqpoint{4.036488in}{2.442742in}}%
\pgfpathlineto{\pgfqpoint{4.177239in}{2.225337in}}%
\pgfpathlineto{\pgfqpoint{4.212427in}{2.179523in}}%
\pgfpathlineto{\pgfqpoint{4.247615in}{2.140723in}}%
\pgfpathlineto{\pgfqpoint{4.265208in}{2.124351in}}%
\pgfpathlineto{\pgfqpoint{4.282802in}{2.110162in}}%
\pgfpathlineto{\pgfqpoint{4.300396in}{2.098244in}}%
\pgfpathlineto{\pgfqpoint{4.317990in}{2.088662in}}%
\pgfpathlineto{\pgfqpoint{4.335584in}{2.081451in}}%
\pgfpathlineto{\pgfqpoint{4.353178in}{2.076621in}}%
\pgfpathlineto{\pgfqpoint{4.370772in}{2.074158in}}%
\pgfpathlineto{\pgfqpoint{4.388365in}{2.074021in}}%
\pgfpathlineto{\pgfqpoint{4.405959in}{2.076147in}}%
\pgfpathlineto{\pgfqpoint{4.423553in}{2.080451in}}%
\pgfpathlineto{\pgfqpoint{4.441147in}{2.086825in}}%
\pgfpathlineto{\pgfqpoint{4.458741in}{2.095146in}}%
\pgfpathlineto{\pgfqpoint{4.493928in}{2.117050in}}%
\pgfpathlineto{\pgfqpoint{4.529116in}{2.144881in}}%
\pgfpathlineto{\pgfqpoint{4.564304in}{2.177200in}}%
\pgfpathlineto{\pgfqpoint{4.617085in}{2.230779in}}%
\pgfpathlineto{\pgfqpoint{4.705055in}{2.321423in}}%
\pgfpathlineto{\pgfqpoint{4.740242in}{2.354364in}}%
\pgfpathlineto{\pgfqpoint{4.775430in}{2.383863in}}%
\pgfpathlineto{\pgfqpoint{4.810618in}{2.409190in}}%
\pgfpathlineto{\pgfqpoint{4.845805in}{2.429841in}}%
\pgfpathlineto{\pgfqpoint{4.880993in}{2.445537in}}%
\pgfpathlineto{\pgfqpoint{4.916181in}{2.456201in}}%
\pgfpathlineto{\pgfqpoint{4.951369in}{2.461934in}}%
\pgfpathlineto{\pgfqpoint{4.986556in}{2.462978in}}%
\pgfpathlineto{\pgfqpoint{5.021744in}{2.459685in}}%
\pgfpathlineto{\pgfqpoint{5.056932in}{2.452482in}}%
\pgfpathlineto{\pgfqpoint{5.092119in}{2.441839in}}%
\pgfpathlineto{\pgfqpoint{5.127307in}{2.428243in}}%
\pgfpathlineto{\pgfqpoint{5.180089in}{2.403394in}}%
\pgfpathlineto{\pgfqpoint{5.232870in}{2.374667in}}%
\pgfpathlineto{\pgfqpoint{5.408809in}{2.274097in}}%
\pgfpathlineto{\pgfqpoint{5.443996in}{2.258088in}}%
\pgfpathlineto{\pgfqpoint{5.479184in}{2.245453in}}%
\pgfpathlineto{\pgfqpoint{5.514372in}{2.237236in}}%
\pgfpathlineto{\pgfqpoint{5.549560in}{2.234531in}}%
\pgfpathlineto{\pgfqpoint{5.567153in}{2.235586in}}%
\pgfpathlineto{\pgfqpoint{5.584747in}{2.238420in}}%
\pgfpathlineto{\pgfqpoint{5.602341in}{2.243154in}}%
\pgfpathlineto{\pgfqpoint{5.619935in}{2.249897in}}%
\pgfpathlineto{\pgfqpoint{5.637529in}{2.258743in}}%
\pgfpathlineto{\pgfqpoint{5.655123in}{2.269769in}}%
\pgfpathlineto{\pgfqpoint{5.672717in}{2.283027in}}%
\pgfpathlineto{\pgfqpoint{5.690310in}{2.298545in}}%
\pgfpathlineto{\pgfqpoint{5.707904in}{2.316321in}}%
\pgfpathlineto{\pgfqpoint{5.743092in}{2.358463in}}%
\pgfpathlineto{\pgfqpoint{5.778280in}{2.408700in}}%
\pgfpathlineto{\pgfqpoint{5.813467in}{2.465595in}}%
\pgfpathlineto{\pgfqpoint{5.866249in}{2.558447in}}%
\pgfpathlineto{\pgfqpoint{5.919030in}{2.650758in}}%
\pgfpathlineto{\pgfqpoint{5.954218in}{2.705424in}}%
\pgfpathlineto{\pgfqpoint{5.971812in}{2.729055in}}%
\pgfpathlineto{\pgfqpoint{5.989406in}{2.749482in}}%
\pgfpathlineto{\pgfqpoint{6.007000in}{2.766176in}}%
\pgfpathlineto{\pgfqpoint{6.024594in}{2.778646in}}%
\pgfpathlineto{\pgfqpoint{6.042187in}{2.786462in}}%
\pgfpathlineto{\pgfqpoint{6.059781in}{2.789275in}}%
\pgfpathlineto{\pgfqpoint{6.077375in}{2.786845in}}%
\pgfpathlineto{\pgfqpoint{6.094969in}{2.779063in}}%
\pgfpathlineto{\pgfqpoint{6.112563in}{2.765990in}}%
\pgfpathlineto{\pgfqpoint{6.130157in}{2.747880in}}%
\pgfpathlineto{\pgfqpoint{6.147751in}{2.725226in}}%
\pgfpathlineto{\pgfqpoint{6.182938in}{2.669641in}}%
\pgfpathlineto{\pgfqpoint{6.218126in}{2.609336in}}%
\pgfpathlineto{\pgfqpoint{6.235720in}{2.582285in}}%
\pgfpathlineto{\pgfqpoint{6.253314in}{2.560834in}}%
\pgfpathlineto{\pgfqpoint{6.270907in}{2.548303in}}%
\pgfpathlineto{\pgfqpoint{6.288501in}{2.548609in}}%
\pgfpathlineto{\pgfqpoint{6.306095in}{2.566324in}}%
\pgfpathlineto{\pgfqpoint{6.323689in}{2.606721in}}%
\pgfpathlineto{\pgfqpoint{6.341283in}{2.675835in}}%
\pgfpathlineto{\pgfqpoint{6.358877in}{2.780514in}}%
\pgfpathlineto{\pgfqpoint{6.374236in}{2.909678in}}%
\pgfpathlineto{\pgfqpoint{6.374236in}{2.909678in}}%
\pgfusepath{stroke}%
\end{pgfscope}%
\begin{pgfscope}%
\pgfpathrectangle{\pgfqpoint{2.000000in}{1.942105in}}{\pgfqpoint{4.376471in}{0.953684in}} %
\pgfusepath{clip}%
\pgfsetbuttcap%
\pgfsetroundjoin%
\pgfsetlinewidth{1.505625pt}%
\definecolor{currentstroke}{rgb}{1.000000,0.498039,0.054902}%
\pgfsetstrokecolor{currentstroke}%
\pgfsetdash{{9.600000pt}{2.400000pt}{1.600000pt}{2.400000pt}}{0.000000pt}%
\pgfpathmoveto{\pgfqpoint{2.875294in}{2.491414in}}%
\pgfpathlineto{\pgfqpoint{2.892888in}{2.541626in}}%
\pgfpathlineto{\pgfqpoint{2.910482in}{2.581793in}}%
\pgfpathlineto{\pgfqpoint{2.928076in}{2.612616in}}%
\pgfpathlineto{\pgfqpoint{2.945670in}{2.634791in}}%
\pgfpathlineto{\pgfqpoint{2.963263in}{2.649010in}}%
\pgfpathlineto{\pgfqpoint{2.980857in}{2.655973in}}%
\pgfpathlineto{\pgfqpoint{2.998451in}{2.656373in}}%
\pgfpathlineto{\pgfqpoint{3.016045in}{2.650910in}}%
\pgfpathlineto{\pgfqpoint{3.033639in}{2.640278in}}%
\pgfpathlineto{\pgfqpoint{3.051233in}{2.625173in}}%
\pgfpathlineto{\pgfqpoint{3.068826in}{2.606282in}}%
\pgfpathlineto{\pgfqpoint{3.104014in}{2.559838in}}%
\pgfpathlineto{\pgfqpoint{3.139202in}{2.506185in}}%
\pgfpathlineto{\pgfqpoint{3.209577in}{2.396280in}}%
\pgfpathlineto{\pgfqpoint{3.244765in}{2.348183in}}%
\pgfpathlineto{\pgfqpoint{3.262359in}{2.327297in}}%
\pgfpathlineto{\pgfqpoint{3.279953in}{2.308946in}}%
\pgfpathlineto{\pgfqpoint{3.297547in}{2.293379in}}%
\pgfpathlineto{\pgfqpoint{3.315140in}{2.280797in}}%
\pgfpathlineto{\pgfqpoint{3.332734in}{2.271343in}}%
\pgfpathlineto{\pgfqpoint{3.350328in}{2.265113in}}%
\pgfpathlineto{\pgfqpoint{3.367922in}{2.262139in}}%
\pgfpathlineto{\pgfqpoint{3.385516in}{2.262414in}}%
\pgfpathlineto{\pgfqpoint{3.403110in}{2.265873in}}%
\pgfpathlineto{\pgfqpoint{3.420704in}{2.272403in}}%
\pgfpathlineto{\pgfqpoint{3.438297in}{2.281857in}}%
\pgfpathlineto{\pgfqpoint{3.455891in}{2.294033in}}%
\pgfpathlineto{\pgfqpoint{3.473485in}{2.308707in}}%
\pgfpathlineto{\pgfqpoint{3.508673in}{2.344447in}}%
\pgfpathlineto{\pgfqpoint{3.543860in}{2.386657in}}%
\pgfpathlineto{\pgfqpoint{3.667017in}{2.545268in}}%
\pgfpathlineto{\pgfqpoint{3.702205in}{2.582377in}}%
\pgfpathlineto{\pgfqpoint{3.719799in}{2.598055in}}%
\pgfpathlineto{\pgfqpoint{3.737393in}{2.611500in}}%
\pgfpathlineto{\pgfqpoint{3.754987in}{2.622518in}}%
\pgfpathlineto{\pgfqpoint{3.772581in}{2.630941in}}%
\pgfpathlineto{\pgfqpoint{3.790174in}{2.636646in}}%
\pgfpathlineto{\pgfqpoint{3.807768in}{2.639539in}}%
\pgfpathlineto{\pgfqpoint{3.825362in}{2.639563in}}%
\pgfpathlineto{\pgfqpoint{3.842956in}{2.636701in}}%
\pgfpathlineto{\pgfqpoint{3.860550in}{2.630969in}}%
\pgfpathlineto{\pgfqpoint{3.878144in}{2.622419in}}%
\pgfpathlineto{\pgfqpoint{3.895738in}{2.611143in}}%
\pgfpathlineto{\pgfqpoint{3.913331in}{2.597256in}}%
\pgfpathlineto{\pgfqpoint{3.930925in}{2.580910in}}%
\pgfpathlineto{\pgfqpoint{3.966113in}{2.541573in}}%
\pgfpathlineto{\pgfqpoint{4.001301in}{2.494845in}}%
\pgfpathlineto{\pgfqpoint{4.036488in}{2.442742in}}%
\pgfpathlineto{\pgfqpoint{4.177239in}{2.225335in}}%
\pgfpathlineto{\pgfqpoint{4.212427in}{2.179521in}}%
\pgfpathlineto{\pgfqpoint{4.247615in}{2.140724in}}%
\pgfpathlineto{\pgfqpoint{4.265208in}{2.124351in}}%
\pgfpathlineto{\pgfqpoint{4.282802in}{2.110160in}}%
\pgfpathlineto{\pgfqpoint{4.300396in}{2.098243in}}%
\pgfpathlineto{\pgfqpoint{4.317990in}{2.088661in}}%
\pgfpathlineto{\pgfqpoint{4.335584in}{2.081449in}}%
\pgfpathlineto{\pgfqpoint{4.353178in}{2.076621in}}%
\pgfpathlineto{\pgfqpoint{4.370772in}{2.074159in}}%
\pgfpathlineto{\pgfqpoint{4.388365in}{2.074020in}}%
\pgfpathlineto{\pgfqpoint{4.405959in}{2.076147in}}%
\pgfpathlineto{\pgfqpoint{4.423553in}{2.080448in}}%
\pgfpathlineto{\pgfqpoint{4.441147in}{2.086825in}}%
\pgfpathlineto{\pgfqpoint{4.458741in}{2.095146in}}%
\pgfpathlineto{\pgfqpoint{4.493928in}{2.117051in}}%
\pgfpathlineto{\pgfqpoint{4.529116in}{2.144881in}}%
\pgfpathlineto{\pgfqpoint{4.564304in}{2.177199in}}%
\pgfpathlineto{\pgfqpoint{4.617085in}{2.230780in}}%
\pgfpathlineto{\pgfqpoint{4.705055in}{2.321423in}}%
\pgfpathlineto{\pgfqpoint{4.740242in}{2.354362in}}%
\pgfpathlineto{\pgfqpoint{4.775430in}{2.383864in}}%
\pgfpathlineto{\pgfqpoint{4.810618in}{2.409189in}}%
\pgfpathlineto{\pgfqpoint{4.845805in}{2.429840in}}%
\pgfpathlineto{\pgfqpoint{4.880993in}{2.445538in}}%
\pgfpathlineto{\pgfqpoint{4.916181in}{2.456202in}}%
\pgfpathlineto{\pgfqpoint{4.951369in}{2.461935in}}%
\pgfpathlineto{\pgfqpoint{4.986556in}{2.462978in}}%
\pgfpathlineto{\pgfqpoint{5.021744in}{2.459686in}}%
\pgfpathlineto{\pgfqpoint{5.056932in}{2.452484in}}%
\pgfpathlineto{\pgfqpoint{5.092119in}{2.441839in}}%
\pgfpathlineto{\pgfqpoint{5.127307in}{2.428244in}}%
\pgfpathlineto{\pgfqpoint{5.180089in}{2.403396in}}%
\pgfpathlineto{\pgfqpoint{5.232870in}{2.374669in}}%
\pgfpathlineto{\pgfqpoint{5.408809in}{2.274100in}}%
\pgfpathlineto{\pgfqpoint{5.443996in}{2.258092in}}%
\pgfpathlineto{\pgfqpoint{5.479184in}{2.245458in}}%
\pgfpathlineto{\pgfqpoint{5.514372in}{2.237240in}}%
\pgfpathlineto{\pgfqpoint{5.549560in}{2.234534in}}%
\pgfpathlineto{\pgfqpoint{5.567153in}{2.235588in}}%
\pgfpathlineto{\pgfqpoint{5.584747in}{2.238423in}}%
\pgfpathlineto{\pgfqpoint{5.602341in}{2.243157in}}%
\pgfpathlineto{\pgfqpoint{5.619935in}{2.249900in}}%
\pgfpathlineto{\pgfqpoint{5.637529in}{2.258744in}}%
\pgfpathlineto{\pgfqpoint{5.655123in}{2.269772in}}%
\pgfpathlineto{\pgfqpoint{5.672717in}{2.283028in}}%
\pgfpathlineto{\pgfqpoint{5.690310in}{2.298548in}}%
\pgfpathlineto{\pgfqpoint{5.707904in}{2.316322in}}%
\pgfpathlineto{\pgfqpoint{5.743092in}{2.358464in}}%
\pgfpathlineto{\pgfqpoint{5.778280in}{2.408701in}}%
\pgfpathlineto{\pgfqpoint{5.813467in}{2.465596in}}%
\pgfpathlineto{\pgfqpoint{5.866249in}{2.558449in}}%
\pgfpathlineto{\pgfqpoint{5.919030in}{2.650759in}}%
\pgfpathlineto{\pgfqpoint{5.954218in}{2.705425in}}%
\pgfpathlineto{\pgfqpoint{5.971812in}{2.729058in}}%
\pgfpathlineto{\pgfqpoint{5.989406in}{2.749483in}}%
\pgfpathlineto{\pgfqpoint{6.007000in}{2.766179in}}%
\pgfpathlineto{\pgfqpoint{6.024594in}{2.778648in}}%
\pgfpathlineto{\pgfqpoint{6.042187in}{2.786464in}}%
\pgfpathlineto{\pgfqpoint{6.059781in}{2.789278in}}%
\pgfpathlineto{\pgfqpoint{6.077375in}{2.786847in}}%
\pgfpathlineto{\pgfqpoint{6.094969in}{2.779065in}}%
\pgfpathlineto{\pgfqpoint{6.112563in}{2.765993in}}%
\pgfpathlineto{\pgfqpoint{6.130157in}{2.747882in}}%
\pgfpathlineto{\pgfqpoint{6.147751in}{2.725228in}}%
\pgfpathlineto{\pgfqpoint{6.182938in}{2.669644in}}%
\pgfpathlineto{\pgfqpoint{6.218126in}{2.609337in}}%
\pgfpathlineto{\pgfqpoint{6.235720in}{2.582288in}}%
\pgfpathlineto{\pgfqpoint{6.253314in}{2.560836in}}%
\pgfpathlineto{\pgfqpoint{6.270907in}{2.548305in}}%
\pgfpathlineto{\pgfqpoint{6.288501in}{2.548612in}}%
\pgfpathlineto{\pgfqpoint{6.306095in}{2.566326in}}%
\pgfpathlineto{\pgfqpoint{6.323689in}{2.606723in}}%
\pgfpathlineto{\pgfqpoint{6.341283in}{2.675837in}}%
\pgfpathlineto{\pgfqpoint{6.358877in}{2.780517in}}%
\pgfpathlineto{\pgfqpoint{6.374236in}{2.909678in}}%
\pgfpathlineto{\pgfqpoint{6.374236in}{2.909678in}}%
\pgfusepath{stroke}%
\end{pgfscope}%
\begin{pgfscope}%
\pgfsetrectcap%
\pgfsetmiterjoin%
\pgfsetlinewidth{0.803000pt}%
\definecolor{currentstroke}{rgb}{0.000000,0.000000,0.000000}%
\pgfsetstrokecolor{currentstroke}%
\pgfsetdash{}{0pt}%
\pgfpathmoveto{\pgfqpoint{2.000000in}{1.942105in}}%
\pgfpathlineto{\pgfqpoint{2.000000in}{2.895789in}}%
\pgfusepath{stroke}%
\end{pgfscope}%
\begin{pgfscope}%
\pgfsetrectcap%
\pgfsetmiterjoin%
\pgfsetlinewidth{0.803000pt}%
\definecolor{currentstroke}{rgb}{0.000000,0.000000,0.000000}%
\pgfsetstrokecolor{currentstroke}%
\pgfsetdash{}{0pt}%
\pgfpathmoveto{\pgfqpoint{6.376471in}{1.942105in}}%
\pgfpathlineto{\pgfqpoint{6.376471in}{2.895789in}}%
\pgfusepath{stroke}%
\end{pgfscope}%
\begin{pgfscope}%
\pgfsetrectcap%
\pgfsetmiterjoin%
\pgfsetlinewidth{0.803000pt}%
\definecolor{currentstroke}{rgb}{0.000000,0.000000,0.000000}%
\pgfsetstrokecolor{currentstroke}%
\pgfsetdash{}{0pt}%
\pgfpathmoveto{\pgfqpoint{2.000000in}{1.942105in}}%
\pgfpathlineto{\pgfqpoint{6.376471in}{1.942105in}}%
\pgfusepath{stroke}%
\end{pgfscope}%
\begin{pgfscope}%
\pgfsetrectcap%
\pgfsetmiterjoin%
\pgfsetlinewidth{0.803000pt}%
\definecolor{currentstroke}{rgb}{0.000000,0.000000,0.000000}%
\pgfsetstrokecolor{currentstroke}%
\pgfsetdash{}{0pt}%
\pgfpathmoveto{\pgfqpoint{2.000000in}{2.895789in}}%
\pgfpathlineto{\pgfqpoint{6.376471in}{2.895789in}}%
\pgfusepath{stroke}%
\end{pgfscope}%
\begin{pgfscope}%
\pgfsetbuttcap%
\pgfsetmiterjoin%
\definecolor{currentfill}{rgb}{1.000000,1.000000,1.000000}%
\pgfsetfillcolor{currentfill}%
\pgfsetfillopacity{0.800000}%
\pgfsetlinewidth{1.003750pt}%
\definecolor{currentstroke}{rgb}{0.800000,0.800000,0.800000}%
\pgfsetstrokecolor{currentstroke}%
\pgfsetstrokeopacity{0.800000}%
\pgfsetdash{}{0pt}%
\pgfpathmoveto{\pgfqpoint{2.097222in}{2.011550in}}%
\pgfpathlineto{\pgfqpoint{2.796183in}{2.011550in}}%
\pgfpathquadraticcurveto{\pgfqpoint{2.823961in}{2.011550in}}{\pgfqpoint{2.823961in}{2.039327in}}%
\pgfpathlineto{\pgfqpoint{2.823961in}{2.827901in}}%
\pgfpathquadraticcurveto{\pgfqpoint{2.823961in}{2.855679in}}{\pgfqpoint{2.796183in}{2.855679in}}%
\pgfpathlineto{\pgfqpoint{2.097222in}{2.855679in}}%
\pgfpathquadraticcurveto{\pgfqpoint{2.069444in}{2.855679in}}{\pgfqpoint{2.069444in}{2.827901in}}%
\pgfpathlineto{\pgfqpoint{2.069444in}{2.039327in}}%
\pgfpathquadraticcurveto{\pgfqpoint{2.069444in}{2.011550in}}{\pgfqpoint{2.097222in}{2.011550in}}%
\pgfpathclose%
\pgfusepath{stroke,fill}%
\end{pgfscope}%
\begin{pgfscope}%
\pgfsetrectcap%
\pgfsetroundjoin%
\pgfsetlinewidth{1.505625pt}%
\definecolor{currentstroke}{rgb}{0.121569,0.466667,0.705882}%
\pgfsetstrokecolor{currentstroke}%
\pgfsetdash{}{0pt}%
\pgfpathmoveto{\pgfqpoint{2.125000in}{2.743022in}}%
\pgfpathlineto{\pgfqpoint{2.402778in}{2.743022in}}%
\pgfusepath{stroke}%
\end{pgfscope}%
\begin{pgfscope}%
\pgftext[x=2.513889in,y=2.694411in,left,base]{\rmfamily\fontsize{10.000000}{12.000000}\selectfont \(\displaystyle \widetilde{\Phi}^* \theta\)}%
\end{pgfscope}%
\begin{pgfscope}%
\pgfsetbuttcap%
\pgfsetroundjoin%
\pgfsetlinewidth{1.505625pt}%
\definecolor{currentstroke}{rgb}{1.000000,0.498039,0.054902}%
\pgfsetstrokecolor{currentstroke}%
\pgfsetdash{{9.600000pt}{2.400000pt}{1.600000pt}{2.400000pt}}{0.000000pt}%
\pgfpathmoveto{\pgfqpoint{2.125000in}{2.538161in}}%
\pgfpathlineto{\pgfqpoint{2.402778in}{2.538161in}}%
\pgfusepath{stroke}%
\end{pgfscope}%
\begin{pgfscope}%
\pgftext[x=2.513889in,y=2.489549in,left,base]{\rmfamily\fontsize{10.000000}{12.000000}\selectfont \(\displaystyle \widetilde{K}u\)}%
\end{pgfscope}%
\begin{pgfscope}%
\pgfsetbuttcap%
\pgfsetroundjoin%
\definecolor{currentfill}{rgb}{1.000000,0.000000,0.000000}%
\pgfsetfillcolor{currentfill}%
\pgfsetlinewidth{2.007500pt}%
\definecolor{currentstroke}{rgb}{1.000000,0.000000,0.000000}%
\pgfsetstrokecolor{currentstroke}%
\pgfsetdash{}{0pt}%
\pgfpathmoveto{\pgfqpoint{2.222222in}{2.329637in}}%
\pgfpathlineto{\pgfqpoint{2.305556in}{2.329637in}}%
\pgfpathmoveto{\pgfqpoint{2.263889in}{2.287971in}}%
\pgfpathlineto{\pgfqpoint{2.263889in}{2.371304in}}%
\pgfusepath{stroke,fill}%
\end{pgfscope}%
\begin{pgfscope}%
\pgftext[x=2.513889in,y=2.293179in,left,base]{\rmfamily\fontsize{10.000000}{12.000000}\selectfont train}%
\end{pgfscope}%
\begin{pgfscope}%
\pgfsetbuttcap%
\pgfsetroundjoin%
\definecolor{currentfill}{rgb}{0.000000,0.000000,0.000000}%
\pgfsetfillcolor{currentfill}%
\pgfsetlinewidth{1.003750pt}%
\definecolor{currentstroke}{rgb}{0.000000,0.000000,0.000000}%
\pgfsetstrokecolor{currentstroke}%
\pgfsetdash{}{0pt}%
\pgfsys@defobject{currentmarker}{\pgfqpoint{-0.020833in}{-0.020833in}}{\pgfqpoint{0.020833in}{0.020833in}}{%
\pgfpathmoveto{\pgfqpoint{0.000000in}{-0.020833in}}%
\pgfpathcurveto{\pgfqpoint{0.005525in}{-0.020833in}}{\pgfqpoint{0.010825in}{-0.018638in}}{\pgfqpoint{0.014731in}{-0.014731in}}%
\pgfpathcurveto{\pgfqpoint{0.018638in}{-0.010825in}}{\pgfqpoint{0.020833in}{-0.005525in}}{\pgfqpoint{0.020833in}{0.000000in}}%
\pgfpathcurveto{\pgfqpoint{0.020833in}{0.005525in}}{\pgfqpoint{0.018638in}{0.010825in}}{\pgfqpoint{0.014731in}{0.014731in}}%
\pgfpathcurveto{\pgfqpoint{0.010825in}{0.018638in}}{\pgfqpoint{0.005525in}{0.020833in}}{\pgfqpoint{0.000000in}{0.020833in}}%
\pgfpathcurveto{\pgfqpoint{-0.005525in}{0.020833in}}{\pgfqpoint{-0.010825in}{0.018638in}}{\pgfqpoint{-0.014731in}{0.014731in}}%
\pgfpathcurveto{\pgfqpoint{-0.018638in}{0.010825in}}{\pgfqpoint{-0.020833in}{0.005525in}}{\pgfqpoint{-0.020833in}{0.000000in}}%
\pgfpathcurveto{\pgfqpoint{-0.020833in}{-0.005525in}}{\pgfqpoint{-0.018638in}{-0.010825in}}{\pgfqpoint{-0.014731in}{-0.014731in}}%
\pgfpathcurveto{\pgfqpoint{-0.010825in}{-0.018638in}}{\pgfqpoint{-0.005525in}{-0.020833in}}{\pgfqpoint{0.000000in}{-0.020833in}}%
\pgfpathclose%
\pgfusepath{stroke,fill}%
}%
\begin{pgfscope}%
\pgfsys@transformshift{2.263889in}{2.133267in}%
\pgfsys@useobject{currentmarker}{}%
\end{pgfscope}%
\end{pgfscope}%
\begin{pgfscope}%
\pgftext[x=2.513889in,y=2.096809in,left,base]{\rmfamily\fontsize{10.000000}{12.000000}\selectfont test}%
\end{pgfscope}%
\begin{pgfscope}%
\pgfsetbuttcap%
\pgfsetmiterjoin%
\definecolor{currentfill}{rgb}{1.000000,1.000000,1.000000}%
\pgfsetfillcolor{currentfill}%
\pgfsetlinewidth{0.000000pt}%
\definecolor{currentstroke}{rgb}{0.000000,0.000000,0.000000}%
\pgfsetstrokecolor{currentstroke}%
\pgfsetstrokeopacity{0.000000}%
\pgfsetdash{}{0pt}%
\pgfpathmoveto{\pgfqpoint{7.105882in}{1.942105in}}%
\pgfpathlineto{\pgfqpoint{11.482353in}{1.942105in}}%
\pgfpathlineto{\pgfqpoint{11.482353in}{2.895789in}}%
\pgfpathlineto{\pgfqpoint{7.105882in}{2.895789in}}%
\pgfpathclose%
\pgfusepath{fill}%
\end{pgfscope}%
\begin{pgfscope}%
\pgfpathrectangle{\pgfqpoint{7.105882in}{1.942105in}}{\pgfqpoint{4.376471in}{0.953684in}} %
\pgfusepath{clip}%
\pgfsetbuttcap%
\pgfsetroundjoin%
\definecolor{currentfill}{rgb}{1.000000,0.000000,0.000000}%
\pgfsetfillcolor{currentfill}%
\pgfsetlinewidth{2.007500pt}%
\definecolor{currentstroke}{rgb}{1.000000,0.000000,0.000000}%
\pgfsetstrokecolor{currentstroke}%
\pgfsetdash{}{0pt}%
\pgfpathmoveto{\pgfqpoint{9.861003in}{2.298612in}}%
\pgfpathlineto{\pgfqpoint{9.944336in}{2.298612in}}%
\pgfpathmoveto{\pgfqpoint{9.902669in}{2.256946in}}%
\pgfpathlineto{\pgfqpoint{9.902669in}{2.340279in}}%
\pgfusepath{stroke,fill}%
\end{pgfscope}%
\begin{pgfscope}%
\pgfpathrectangle{\pgfqpoint{7.105882in}{1.942105in}}{\pgfqpoint{4.376471in}{0.953684in}} %
\pgfusepath{clip}%
\pgfsetbuttcap%
\pgfsetroundjoin%
\definecolor{currentfill}{rgb}{1.000000,0.000000,0.000000}%
\pgfsetfillcolor{currentfill}%
\pgfsetlinewidth{2.007500pt}%
\definecolor{currentstroke}{rgb}{1.000000,0.000000,0.000000}%
\pgfsetstrokecolor{currentstroke}%
\pgfsetdash{}{0pt}%
\pgfpathmoveto{\pgfqpoint{10.443514in}{2.384806in}}%
\pgfpathlineto{\pgfqpoint{10.526847in}{2.384806in}}%
\pgfpathmoveto{\pgfqpoint{10.485181in}{2.343139in}}%
\pgfpathlineto{\pgfqpoint{10.485181in}{2.426473in}}%
\pgfusepath{stroke,fill}%
\end{pgfscope}%
\begin{pgfscope}%
\pgfpathrectangle{\pgfqpoint{7.105882in}{1.942105in}}{\pgfqpoint{4.376471in}{0.953684in}} %
\pgfusepath{clip}%
\pgfsetbuttcap%
\pgfsetroundjoin%
\definecolor{currentfill}{rgb}{1.000000,0.000000,0.000000}%
\pgfsetfillcolor{currentfill}%
\pgfsetlinewidth{2.007500pt}%
\definecolor{currentstroke}{rgb}{1.000000,0.000000,0.000000}%
\pgfsetstrokecolor{currentstroke}%
\pgfsetdash{}{0pt}%
\pgfpathmoveto{\pgfqpoint{10.049891in}{2.441712in}}%
\pgfpathlineto{\pgfqpoint{10.133224in}{2.441712in}}%
\pgfpathmoveto{\pgfqpoint{10.091557in}{2.400045in}}%
\pgfpathlineto{\pgfqpoint{10.091557in}{2.483379in}}%
\pgfusepath{stroke,fill}%
\end{pgfscope}%
\begin{pgfscope}%
\pgfpathrectangle{\pgfqpoint{7.105882in}{1.942105in}}{\pgfqpoint{4.376471in}{0.953684in}} %
\pgfusepath{clip}%
\pgfsetbuttcap%
\pgfsetroundjoin%
\definecolor{currentfill}{rgb}{1.000000,0.000000,0.000000}%
\pgfsetfillcolor{currentfill}%
\pgfsetlinewidth{2.007500pt}%
\definecolor{currentstroke}{rgb}{1.000000,0.000000,0.000000}%
\pgfsetstrokecolor{currentstroke}%
\pgfsetdash{}{0pt}%
\pgfpathmoveto{\pgfqpoint{9.847242in}{2.414765in}}%
\pgfpathlineto{\pgfqpoint{9.930575in}{2.414765in}}%
\pgfpathmoveto{\pgfqpoint{9.888909in}{2.373099in}}%
\pgfpathlineto{\pgfqpoint{9.888909in}{2.456432in}}%
\pgfusepath{stroke,fill}%
\end{pgfscope}%
\begin{pgfscope}%
\pgfpathrectangle{\pgfqpoint{7.105882in}{1.942105in}}{\pgfqpoint{4.376471in}{0.953684in}} %
\pgfusepath{clip}%
\pgfsetbuttcap%
\pgfsetroundjoin%
\definecolor{currentfill}{rgb}{1.000000,0.000000,0.000000}%
\pgfsetfillcolor{currentfill}%
\pgfsetlinewidth{2.007500pt}%
\definecolor{currentstroke}{rgb}{1.000000,0.000000,0.000000}%
\pgfsetstrokecolor{currentstroke}%
\pgfsetdash{}{0pt}%
\pgfpathmoveto{\pgfqpoint{9.422800in}{1.986053in}}%
\pgfpathlineto{\pgfqpoint{9.506133in}{1.986053in}}%
\pgfpathmoveto{\pgfqpoint{9.464467in}{1.944386in}}%
\pgfpathlineto{\pgfqpoint{9.464467in}{2.027720in}}%
\pgfusepath{stroke,fill}%
\end{pgfscope}%
\begin{pgfscope}%
\pgfpathrectangle{\pgfqpoint{7.105882in}{1.942105in}}{\pgfqpoint{4.376471in}{0.953684in}} %
\pgfusepath{clip}%
\pgfsetbuttcap%
\pgfsetroundjoin%
\definecolor{currentfill}{rgb}{1.000000,0.000000,0.000000}%
\pgfsetfillcolor{currentfill}%
\pgfsetlinewidth{2.007500pt}%
\definecolor{currentstroke}{rgb}{1.000000,0.000000,0.000000}%
\pgfsetstrokecolor{currentstroke}%
\pgfsetdash{}{0pt}%
\pgfpathmoveto{\pgfqpoint{10.200899in}{2.454884in}}%
\pgfpathlineto{\pgfqpoint{10.284232in}{2.454884in}}%
\pgfpathmoveto{\pgfqpoint{10.242566in}{2.413218in}}%
\pgfpathlineto{\pgfqpoint{10.242566in}{2.496551in}}%
\pgfusepath{stroke,fill}%
\end{pgfscope}%
\begin{pgfscope}%
\pgfpathrectangle{\pgfqpoint{7.105882in}{1.942105in}}{\pgfqpoint{4.376471in}{0.953684in}} %
\pgfusepath{clip}%
\pgfsetbuttcap%
\pgfsetroundjoin%
\definecolor{currentfill}{rgb}{1.000000,0.000000,0.000000}%
\pgfsetfillcolor{currentfill}%
\pgfsetlinewidth{2.007500pt}%
\definecolor{currentstroke}{rgb}{1.000000,0.000000,0.000000}%
\pgfsetstrokecolor{currentstroke}%
\pgfsetdash{}{0pt}%
\pgfpathmoveto{\pgfqpoint{9.471580in}{1.957162in}}%
\pgfpathlineto{\pgfqpoint{9.554913in}{1.957162in}}%
\pgfpathmoveto{\pgfqpoint{9.513247in}{1.915495in}}%
\pgfpathlineto{\pgfqpoint{9.513247in}{1.998828in}}%
\pgfusepath{stroke,fill}%
\end{pgfscope}%
\begin{pgfscope}%
\pgfpathrectangle{\pgfqpoint{7.105882in}{1.942105in}}{\pgfqpoint{4.376471in}{0.953684in}} %
\pgfusepath{clip}%
\pgfsetbuttcap%
\pgfsetroundjoin%
\definecolor{currentfill}{rgb}{1.000000,0.000000,0.000000}%
\pgfsetfillcolor{currentfill}%
\pgfsetlinewidth{2.007500pt}%
\definecolor{currentstroke}{rgb}{1.000000,0.000000,0.000000}%
\pgfsetstrokecolor{currentstroke}%
\pgfsetdash{}{0pt}%
\pgfpathmoveto{\pgfqpoint{11.061764in}{2.721215in}}%
\pgfpathlineto{\pgfqpoint{11.145098in}{2.721215in}}%
\pgfpathmoveto{\pgfqpoint{11.103431in}{2.679548in}}%
\pgfpathlineto{\pgfqpoint{11.103431in}{2.762882in}}%
\pgfusepath{stroke,fill}%
\end{pgfscope}%
\begin{pgfscope}%
\pgfpathrectangle{\pgfqpoint{7.105882in}{1.942105in}}{\pgfqpoint{4.376471in}{0.953684in}} %
\pgfusepath{clip}%
\pgfsetbuttcap%
\pgfsetroundjoin%
\definecolor{currentfill}{rgb}{1.000000,0.000000,0.000000}%
\pgfsetfillcolor{currentfill}%
\pgfsetlinewidth{2.007500pt}%
\definecolor{currentstroke}{rgb}{1.000000,0.000000,0.000000}%
\pgfsetstrokecolor{currentstroke}%
\pgfsetdash{}{0pt}%
\pgfpathmoveto{\pgfqpoint{11.313463in}{2.582479in}}%
\pgfpathlineto{\pgfqpoint{11.396797in}{2.582479in}}%
\pgfpathmoveto{\pgfqpoint{11.355130in}{2.540813in}}%
\pgfpathlineto{\pgfqpoint{11.355130in}{2.624146in}}%
\pgfusepath{stroke,fill}%
\end{pgfscope}%
\begin{pgfscope}%
\pgfpathrectangle{\pgfqpoint{7.105882in}{1.942105in}}{\pgfqpoint{4.376471in}{0.953684in}} %
\pgfusepath{clip}%
\pgfsetbuttcap%
\pgfsetroundjoin%
\definecolor{currentfill}{rgb}{1.000000,0.000000,0.000000}%
\pgfsetfillcolor{currentfill}%
\pgfsetlinewidth{2.007500pt}%
\definecolor{currentstroke}{rgb}{1.000000,0.000000,0.000000}%
\pgfsetstrokecolor{currentstroke}%
\pgfsetdash{}{0pt}%
\pgfpathmoveto{\pgfqpoint{9.282006in}{2.215805in}}%
\pgfpathlineto{\pgfqpoint{9.365340in}{2.215805in}}%
\pgfpathmoveto{\pgfqpoint{9.323673in}{2.174138in}}%
\pgfpathlineto{\pgfqpoint{9.323673in}{2.257471in}}%
\pgfusepath{stroke,fill}%
\end{pgfscope}%
\begin{pgfscope}%
\pgfpathrectangle{\pgfqpoint{7.105882in}{1.942105in}}{\pgfqpoint{4.376471in}{0.953684in}} %
\pgfusepath{clip}%
\pgfsetbuttcap%
\pgfsetroundjoin%
\definecolor{currentfill}{rgb}{1.000000,0.000000,0.000000}%
\pgfsetfillcolor{currentfill}%
\pgfsetlinewidth{2.007500pt}%
\definecolor{currentstroke}{rgb}{1.000000,0.000000,0.000000}%
\pgfsetstrokecolor{currentstroke}%
\pgfsetdash{}{0pt}%
\pgfpathmoveto{\pgfqpoint{10.711479in}{2.298077in}}%
\pgfpathlineto{\pgfqpoint{10.794812in}{2.298077in}}%
\pgfpathmoveto{\pgfqpoint{10.753146in}{2.256410in}}%
\pgfpathlineto{\pgfqpoint{10.753146in}{2.339743in}}%
\pgfusepath{stroke,fill}%
\end{pgfscope}%
\begin{pgfscope}%
\pgfpathrectangle{\pgfqpoint{7.105882in}{1.942105in}}{\pgfqpoint{4.376471in}{0.953684in}} %
\pgfusepath{clip}%
\pgfsetbuttcap%
\pgfsetroundjoin%
\definecolor{currentfill}{rgb}{1.000000,0.000000,0.000000}%
\pgfsetfillcolor{currentfill}%
\pgfsetlinewidth{2.007500pt}%
\definecolor{currentstroke}{rgb}{1.000000,0.000000,0.000000}%
\pgfsetstrokecolor{currentstroke}%
\pgfsetdash{}{0pt}%
\pgfpathmoveto{\pgfqpoint{9.791264in}{2.292312in}}%
\pgfpathlineto{\pgfqpoint{9.874598in}{2.292312in}}%
\pgfpathmoveto{\pgfqpoint{9.832931in}{2.250645in}}%
\pgfpathlineto{\pgfqpoint{9.832931in}{2.333978in}}%
\pgfusepath{stroke,fill}%
\end{pgfscope}%
\begin{pgfscope}%
\pgfpathrectangle{\pgfqpoint{7.105882in}{1.942105in}}{\pgfqpoint{4.376471in}{0.953684in}} %
\pgfusepath{clip}%
\pgfsetbuttcap%
\pgfsetroundjoin%
\definecolor{currentfill}{rgb}{1.000000,0.000000,0.000000}%
\pgfsetfillcolor{currentfill}%
\pgfsetlinewidth{2.007500pt}%
\definecolor{currentstroke}{rgb}{1.000000,0.000000,0.000000}%
\pgfsetstrokecolor{currentstroke}%
\pgfsetdash{}{0pt}%
\pgfpathmoveto{\pgfqpoint{9.928334in}{2.498620in}}%
\pgfpathlineto{\pgfqpoint{10.011667in}{2.498620in}}%
\pgfpathmoveto{\pgfqpoint{9.970001in}{2.456953in}}%
\pgfpathlineto{\pgfqpoint{9.970001in}{2.540286in}}%
\pgfusepath{stroke,fill}%
\end{pgfscope}%
\begin{pgfscope}%
\pgfpathrectangle{\pgfqpoint{7.105882in}{1.942105in}}{\pgfqpoint{4.376471in}{0.953684in}} %
\pgfusepath{clip}%
\pgfsetbuttcap%
\pgfsetroundjoin%
\definecolor{currentfill}{rgb}{1.000000,0.000000,0.000000}%
\pgfsetfillcolor{currentfill}%
\pgfsetlinewidth{2.007500pt}%
\definecolor{currentstroke}{rgb}{1.000000,0.000000,0.000000}%
\pgfsetstrokecolor{currentstroke}%
\pgfsetdash{}{0pt}%
\pgfpathmoveto{\pgfqpoint{11.180187in}{2.787402in}}%
\pgfpathlineto{\pgfqpoint{11.263520in}{2.787402in}}%
\pgfpathmoveto{\pgfqpoint{11.221854in}{2.745735in}}%
\pgfpathlineto{\pgfqpoint{11.221854in}{2.829068in}}%
\pgfusepath{stroke,fill}%
\end{pgfscope}%
\begin{pgfscope}%
\pgfpathrectangle{\pgfqpoint{7.105882in}{1.942105in}}{\pgfqpoint{4.376471in}{0.953684in}} %
\pgfusepath{clip}%
\pgfsetbuttcap%
\pgfsetroundjoin%
\definecolor{currentfill}{rgb}{1.000000,0.000000,0.000000}%
\pgfsetfillcolor{currentfill}%
\pgfsetlinewidth{2.007500pt}%
\definecolor{currentstroke}{rgb}{1.000000,0.000000,0.000000}%
\pgfsetstrokecolor{currentstroke}%
\pgfsetdash{}{0pt}%
\pgfpathmoveto{\pgfqpoint{8.188220in}{2.450148in}}%
\pgfpathlineto{\pgfqpoint{8.271553in}{2.450148in}}%
\pgfpathmoveto{\pgfqpoint{8.229886in}{2.408481in}}%
\pgfpathlineto{\pgfqpoint{8.229886in}{2.491815in}}%
\pgfusepath{stroke,fill}%
\end{pgfscope}%
\begin{pgfscope}%
\pgfpathrectangle{\pgfqpoint{7.105882in}{1.942105in}}{\pgfqpoint{4.376471in}{0.953684in}} %
\pgfusepath{clip}%
\pgfsetbuttcap%
\pgfsetroundjoin%
\definecolor{currentfill}{rgb}{1.000000,0.000000,0.000000}%
\pgfsetfillcolor{currentfill}%
\pgfsetlinewidth{2.007500pt}%
\definecolor{currentstroke}{rgb}{1.000000,0.000000,0.000000}%
\pgfsetstrokecolor{currentstroke}%
\pgfsetdash{}{0pt}%
\pgfpathmoveto{\pgfqpoint{8.244565in}{2.417997in}}%
\pgfpathlineto{\pgfqpoint{8.327898in}{2.417997in}}%
\pgfpathmoveto{\pgfqpoint{8.286232in}{2.376330in}}%
\pgfpathlineto{\pgfqpoint{8.286232in}{2.459663in}}%
\pgfusepath{stroke,fill}%
\end{pgfscope}%
\begin{pgfscope}%
\pgfpathrectangle{\pgfqpoint{7.105882in}{1.942105in}}{\pgfqpoint{4.376471in}{0.953684in}} %
\pgfusepath{clip}%
\pgfsetbuttcap%
\pgfsetroundjoin%
\definecolor{currentfill}{rgb}{1.000000,0.000000,0.000000}%
\pgfsetfillcolor{currentfill}%
\pgfsetlinewidth{2.007500pt}%
\definecolor{currentstroke}{rgb}{1.000000,0.000000,0.000000}%
\pgfsetstrokecolor{currentstroke}%
\pgfsetdash{}{0pt}%
\pgfpathmoveto{\pgfqpoint{8.010298in}{2.636385in}}%
\pgfpathlineto{\pgfqpoint{8.093631in}{2.636385in}}%
\pgfpathmoveto{\pgfqpoint{8.051965in}{2.594718in}}%
\pgfpathlineto{\pgfqpoint{8.051965in}{2.678051in}}%
\pgfusepath{stroke,fill}%
\end{pgfscope}%
\begin{pgfscope}%
\pgfpathrectangle{\pgfqpoint{7.105882in}{1.942105in}}{\pgfqpoint{4.376471in}{0.953684in}} %
\pgfusepath{clip}%
\pgfsetbuttcap%
\pgfsetroundjoin%
\definecolor{currentfill}{rgb}{1.000000,0.000000,0.000000}%
\pgfsetfillcolor{currentfill}%
\pgfsetlinewidth{2.007500pt}%
\definecolor{currentstroke}{rgb}{1.000000,0.000000,0.000000}%
\pgfsetstrokecolor{currentstroke}%
\pgfsetdash{}{0pt}%
\pgfpathmoveto{\pgfqpoint{10.854659in}{2.511256in}}%
\pgfpathlineto{\pgfqpoint{10.937992in}{2.511256in}}%
\pgfpathmoveto{\pgfqpoint{10.896325in}{2.469589in}}%
\pgfpathlineto{\pgfqpoint{10.896325in}{2.552922in}}%
\pgfusepath{stroke,fill}%
\end{pgfscope}%
\begin{pgfscope}%
\pgfpathrectangle{\pgfqpoint{7.105882in}{1.942105in}}{\pgfqpoint{4.376471in}{0.953684in}} %
\pgfusepath{clip}%
\pgfsetbuttcap%
\pgfsetroundjoin%
\definecolor{currentfill}{rgb}{1.000000,0.000000,0.000000}%
\pgfsetfillcolor{currentfill}%
\pgfsetlinewidth{2.007500pt}%
\definecolor{currentstroke}{rgb}{1.000000,0.000000,0.000000}%
\pgfsetstrokecolor{currentstroke}%
\pgfsetdash{}{0pt}%
\pgfpathmoveto{\pgfqpoint{10.663974in}{2.224896in}}%
\pgfpathlineto{\pgfqpoint{10.747307in}{2.224896in}}%
\pgfpathmoveto{\pgfqpoint{10.705641in}{2.183229in}}%
\pgfpathlineto{\pgfqpoint{10.705641in}{2.266562in}}%
\pgfusepath{stroke,fill}%
\end{pgfscope}%
\begin{pgfscope}%
\pgfpathrectangle{\pgfqpoint{7.105882in}{1.942105in}}{\pgfqpoint{4.376471in}{0.953684in}} %
\pgfusepath{clip}%
\pgfsetbuttcap%
\pgfsetroundjoin%
\definecolor{currentfill}{rgb}{1.000000,0.000000,0.000000}%
\pgfsetfillcolor{currentfill}%
\pgfsetlinewidth{2.007500pt}%
\definecolor{currentstroke}{rgb}{1.000000,0.000000,0.000000}%
\pgfsetstrokecolor{currentstroke}%
\pgfsetdash{}{0pt}%
\pgfpathmoveto{\pgfqpoint{10.985576in}{2.660212in}}%
\pgfpathlineto{\pgfqpoint{11.068909in}{2.660212in}}%
\pgfpathmoveto{\pgfqpoint{11.027243in}{2.618545in}}%
\pgfpathlineto{\pgfqpoint{11.027243in}{2.701878in}}%
\pgfusepath{stroke,fill}%
\end{pgfscope}%
\begin{pgfscope}%
\pgfpathrectangle{\pgfqpoint{7.105882in}{1.942105in}}{\pgfqpoint{4.376471in}{0.953684in}} %
\pgfusepath{clip}%
\pgfsetbuttcap%
\pgfsetroundjoin%
\definecolor{currentfill}{rgb}{1.000000,0.000000,0.000000}%
\pgfsetfillcolor{currentfill}%
\pgfsetlinewidth{2.007500pt}%
\definecolor{currentstroke}{rgb}{1.000000,0.000000,0.000000}%
\pgfsetstrokecolor{currentstroke}%
\pgfsetdash{}{0pt}%
\pgfpathmoveto{\pgfqpoint{11.365825in}{2.553849in}}%
\pgfpathlineto{\pgfqpoint{11.449159in}{2.553849in}}%
\pgfpathmoveto{\pgfqpoint{11.407492in}{2.512183in}}%
\pgfpathlineto{\pgfqpoint{11.407492in}{2.595516in}}%
\pgfusepath{stroke,fill}%
\end{pgfscope}%
\begin{pgfscope}%
\pgfpathrectangle{\pgfqpoint{7.105882in}{1.942105in}}{\pgfqpoint{4.376471in}{0.953684in}} %
\pgfusepath{clip}%
\pgfsetbuttcap%
\pgfsetroundjoin%
\definecolor{currentfill}{rgb}{1.000000,0.000000,0.000000}%
\pgfsetfillcolor{currentfill}%
\pgfsetlinewidth{2.007500pt}%
\definecolor{currentstroke}{rgb}{1.000000,0.000000,0.000000}%
\pgfsetstrokecolor{currentstroke}%
\pgfsetdash{}{0pt}%
\pgfpathmoveto{\pgfqpoint{10.737505in}{2.211778in}}%
\pgfpathlineto{\pgfqpoint{10.820838in}{2.211778in}}%
\pgfpathmoveto{\pgfqpoint{10.779172in}{2.170111in}}%
\pgfpathlineto{\pgfqpoint{10.779172in}{2.253444in}}%
\pgfusepath{stroke,fill}%
\end{pgfscope}%
\begin{pgfscope}%
\pgfpathrectangle{\pgfqpoint{7.105882in}{1.942105in}}{\pgfqpoint{4.376471in}{0.953684in}} %
\pgfusepath{clip}%
\pgfsetbuttcap%
\pgfsetroundjoin%
\definecolor{currentfill}{rgb}{1.000000,0.000000,0.000000}%
\pgfsetfillcolor{currentfill}%
\pgfsetlinewidth{2.007500pt}%
\definecolor{currentstroke}{rgb}{1.000000,0.000000,0.000000}%
\pgfsetstrokecolor{currentstroke}%
\pgfsetdash{}{0pt}%
\pgfpathmoveto{\pgfqpoint{9.555230in}{2.181351in}}%
\pgfpathlineto{\pgfqpoint{9.638564in}{2.181351in}}%
\pgfpathmoveto{\pgfqpoint{9.596897in}{2.139685in}}%
\pgfpathlineto{\pgfqpoint{9.596897in}{2.223018in}}%
\pgfusepath{stroke,fill}%
\end{pgfscope}%
\begin{pgfscope}%
\pgfpathrectangle{\pgfqpoint{7.105882in}{1.942105in}}{\pgfqpoint{4.376471in}{0.953684in}} %
\pgfusepath{clip}%
\pgfsetbuttcap%
\pgfsetroundjoin%
\definecolor{currentfill}{rgb}{1.000000,0.000000,0.000000}%
\pgfsetfillcolor{currentfill}%
\pgfsetlinewidth{2.007500pt}%
\definecolor{currentstroke}{rgb}{1.000000,0.000000,0.000000}%
\pgfsetstrokecolor{currentstroke}%
\pgfsetdash{}{0pt}%
\pgfpathmoveto{\pgfqpoint{10.672280in}{2.261124in}}%
\pgfpathlineto{\pgfqpoint{10.755614in}{2.261124in}}%
\pgfpathmoveto{\pgfqpoint{10.713947in}{2.219458in}}%
\pgfpathlineto{\pgfqpoint{10.713947in}{2.302791in}}%
\pgfusepath{stroke,fill}%
\end{pgfscope}%
\begin{pgfscope}%
\pgfpathrectangle{\pgfqpoint{7.105882in}{1.942105in}}{\pgfqpoint{4.376471in}{0.953684in}} %
\pgfusepath{clip}%
\pgfsetbuttcap%
\pgfsetroundjoin%
\definecolor{currentfill}{rgb}{1.000000,0.000000,0.000000}%
\pgfsetfillcolor{currentfill}%
\pgfsetlinewidth{2.007500pt}%
\definecolor{currentstroke}{rgb}{1.000000,0.000000,0.000000}%
\pgfsetstrokecolor{currentstroke}%
\pgfsetdash{}{0pt}%
\pgfpathmoveto{\pgfqpoint{8.353609in}{2.175480in}}%
\pgfpathlineto{\pgfqpoint{8.436943in}{2.175480in}}%
\pgfpathmoveto{\pgfqpoint{8.395276in}{2.133814in}}%
\pgfpathlineto{\pgfqpoint{8.395276in}{2.217147in}}%
\pgfusepath{stroke,fill}%
\end{pgfscope}%
\begin{pgfscope}%
\pgfpathrectangle{\pgfqpoint{7.105882in}{1.942105in}}{\pgfqpoint{4.376471in}{0.953684in}} %
\pgfusepath{clip}%
\pgfsetbuttcap%
\pgfsetroundjoin%
\definecolor{currentfill}{rgb}{1.000000,0.000000,0.000000}%
\pgfsetfillcolor{currentfill}%
\pgfsetlinewidth{2.007500pt}%
\definecolor{currentstroke}{rgb}{1.000000,0.000000,0.000000}%
\pgfsetstrokecolor{currentstroke}%
\pgfsetdash{}{0pt}%
\pgfpathmoveto{\pgfqpoint{10.179986in}{2.443682in}}%
\pgfpathlineto{\pgfqpoint{10.263320in}{2.443682in}}%
\pgfpathmoveto{\pgfqpoint{10.221653in}{2.402015in}}%
\pgfpathlineto{\pgfqpoint{10.221653in}{2.485348in}}%
\pgfusepath{stroke,fill}%
\end{pgfscope}%
\begin{pgfscope}%
\pgfpathrectangle{\pgfqpoint{7.105882in}{1.942105in}}{\pgfqpoint{4.376471in}{0.953684in}} %
\pgfusepath{clip}%
\pgfsetbuttcap%
\pgfsetroundjoin%
\definecolor{currentfill}{rgb}{1.000000,0.000000,0.000000}%
\pgfsetfillcolor{currentfill}%
\pgfsetlinewidth{2.007500pt}%
\definecolor{currentstroke}{rgb}{1.000000,0.000000,0.000000}%
\pgfsetstrokecolor{currentstroke}%
\pgfsetdash{}{0pt}%
\pgfpathmoveto{\pgfqpoint{8.441415in}{2.262545in}}%
\pgfpathlineto{\pgfqpoint{8.524748in}{2.262545in}}%
\pgfpathmoveto{\pgfqpoint{8.483082in}{2.220879in}}%
\pgfpathlineto{\pgfqpoint{8.483082in}{2.304212in}}%
\pgfusepath{stroke,fill}%
\end{pgfscope}%
\begin{pgfscope}%
\pgfpathrectangle{\pgfqpoint{7.105882in}{1.942105in}}{\pgfqpoint{4.376471in}{0.953684in}} %
\pgfusepath{clip}%
\pgfsetbuttcap%
\pgfsetroundjoin%
\definecolor{currentfill}{rgb}{1.000000,0.000000,0.000000}%
\pgfsetfillcolor{currentfill}%
\pgfsetlinewidth{2.007500pt}%
\definecolor{currentstroke}{rgb}{1.000000,0.000000,0.000000}%
\pgfsetstrokecolor{currentstroke}%
\pgfsetdash{}{0pt}%
\pgfpathmoveto{\pgfqpoint{11.246962in}{2.719866in}}%
\pgfpathlineto{\pgfqpoint{11.330296in}{2.719866in}}%
\pgfpathmoveto{\pgfqpoint{11.288629in}{2.678199in}}%
\pgfpathlineto{\pgfqpoint{11.288629in}{2.761533in}}%
\pgfusepath{stroke,fill}%
\end{pgfscope}%
\begin{pgfscope}%
\pgfpathrectangle{\pgfqpoint{7.105882in}{1.942105in}}{\pgfqpoint{4.376471in}{0.953684in}} %
\pgfusepath{clip}%
\pgfsetbuttcap%
\pgfsetroundjoin%
\definecolor{currentfill}{rgb}{1.000000,0.000000,0.000000}%
\pgfsetfillcolor{currentfill}%
\pgfsetlinewidth{2.007500pt}%
\definecolor{currentstroke}{rgb}{1.000000,0.000000,0.000000}%
\pgfsetstrokecolor{currentstroke}%
\pgfsetdash{}{0pt}%
\pgfpathmoveto{\pgfqpoint{9.766593in}{2.457007in}}%
\pgfpathlineto{\pgfqpoint{9.849926in}{2.457007in}}%
\pgfpathmoveto{\pgfqpoint{9.808260in}{2.415340in}}%
\pgfpathlineto{\pgfqpoint{9.808260in}{2.498673in}}%
\pgfusepath{stroke,fill}%
\end{pgfscope}%
\begin{pgfscope}%
\pgfpathrectangle{\pgfqpoint{7.105882in}{1.942105in}}{\pgfqpoint{4.376471in}{0.953684in}} %
\pgfusepath{clip}%
\pgfsetbuttcap%
\pgfsetroundjoin%
\definecolor{currentfill}{rgb}{1.000000,0.000000,0.000000}%
\pgfsetfillcolor{currentfill}%
\pgfsetlinewidth{2.007500pt}%
\definecolor{currentstroke}{rgb}{1.000000,0.000000,0.000000}%
\pgfsetstrokecolor{currentstroke}%
\pgfsetdash{}{0pt}%
\pgfpathmoveto{\pgfqpoint{9.391314in}{2.127275in}}%
\pgfpathlineto{\pgfqpoint{9.474648in}{2.127275in}}%
\pgfpathmoveto{\pgfqpoint{9.432981in}{2.085609in}}%
\pgfpathlineto{\pgfqpoint{9.432981in}{2.168942in}}%
\pgfusepath{stroke,fill}%
\end{pgfscope}%
\begin{pgfscope}%
\pgfpathrectangle{\pgfqpoint{7.105882in}{1.942105in}}{\pgfqpoint{4.376471in}{0.953684in}} %
\pgfusepath{clip}%
\pgfsetbuttcap%
\pgfsetroundjoin%
\definecolor{currentfill}{rgb}{1.000000,0.000000,0.000000}%
\pgfsetfillcolor{currentfill}%
\pgfsetlinewidth{2.007500pt}%
\definecolor{currentstroke}{rgb}{1.000000,0.000000,0.000000}%
\pgfsetstrokecolor{currentstroke}%
\pgfsetdash{}{0pt}%
\pgfpathmoveto{\pgfqpoint{8.865766in}{2.699721in}}%
\pgfpathlineto{\pgfqpoint{8.949099in}{2.699721in}}%
\pgfpathmoveto{\pgfqpoint{8.907432in}{2.658055in}}%
\pgfpathlineto{\pgfqpoint{8.907432in}{2.741388in}}%
\pgfusepath{stroke,fill}%
\end{pgfscope}%
\begin{pgfscope}%
\pgfpathrectangle{\pgfqpoint{7.105882in}{1.942105in}}{\pgfqpoint{4.376471in}{0.953684in}} %
\pgfusepath{clip}%
\pgfsetbuttcap%
\pgfsetroundjoin%
\definecolor{currentfill}{rgb}{1.000000,0.000000,0.000000}%
\pgfsetfillcolor{currentfill}%
\pgfsetlinewidth{2.007500pt}%
\definecolor{currentstroke}{rgb}{1.000000,0.000000,0.000000}%
\pgfsetstrokecolor{currentstroke}%
\pgfsetdash{}{0pt}%
\pgfpathmoveto{\pgfqpoint{10.650239in}{2.154280in}}%
\pgfpathlineto{\pgfqpoint{10.733572in}{2.154280in}}%
\pgfpathmoveto{\pgfqpoint{10.691905in}{2.112614in}}%
\pgfpathlineto{\pgfqpoint{10.691905in}{2.195947in}}%
\pgfusepath{stroke,fill}%
\end{pgfscope}%
\begin{pgfscope}%
\pgfpathrectangle{\pgfqpoint{7.105882in}{1.942105in}}{\pgfqpoint{4.376471in}{0.953684in}} %
\pgfusepath{clip}%
\pgfsetbuttcap%
\pgfsetroundjoin%
\definecolor{currentfill}{rgb}{1.000000,0.000000,0.000000}%
\pgfsetfillcolor{currentfill}%
\pgfsetlinewidth{2.007500pt}%
\definecolor{currentstroke}{rgb}{1.000000,0.000000,0.000000}%
\pgfsetstrokecolor{currentstroke}%
\pgfsetdash{}{0pt}%
\pgfpathmoveto{\pgfqpoint{9.536573in}{2.190010in}}%
\pgfpathlineto{\pgfqpoint{9.619906in}{2.190010in}}%
\pgfpathmoveto{\pgfqpoint{9.578239in}{2.148344in}}%
\pgfpathlineto{\pgfqpoint{9.578239in}{2.231677in}}%
\pgfusepath{stroke,fill}%
\end{pgfscope}%
\begin{pgfscope}%
\pgfpathrectangle{\pgfqpoint{7.105882in}{1.942105in}}{\pgfqpoint{4.376471in}{0.953684in}} %
\pgfusepath{clip}%
\pgfsetbuttcap%
\pgfsetroundjoin%
\definecolor{currentfill}{rgb}{1.000000,0.000000,0.000000}%
\pgfsetfillcolor{currentfill}%
\pgfsetlinewidth{2.007500pt}%
\definecolor{currentstroke}{rgb}{1.000000,0.000000,0.000000}%
\pgfsetstrokecolor{currentstroke}%
\pgfsetdash{}{0pt}%
\pgfpathmoveto{\pgfqpoint{9.929697in}{2.433790in}}%
\pgfpathlineto{\pgfqpoint{10.013031in}{2.433790in}}%
\pgfpathmoveto{\pgfqpoint{9.971364in}{2.392123in}}%
\pgfpathlineto{\pgfqpoint{9.971364in}{2.475457in}}%
\pgfusepath{stroke,fill}%
\end{pgfscope}%
\begin{pgfscope}%
\pgfpathrectangle{\pgfqpoint{7.105882in}{1.942105in}}{\pgfqpoint{4.376471in}{0.953684in}} %
\pgfusepath{clip}%
\pgfsetbuttcap%
\pgfsetroundjoin%
\definecolor{currentfill}{rgb}{1.000000,0.000000,0.000000}%
\pgfsetfillcolor{currentfill}%
\pgfsetlinewidth{2.007500pt}%
\definecolor{currentstroke}{rgb}{1.000000,0.000000,0.000000}%
\pgfsetstrokecolor{currentstroke}%
\pgfsetdash{}{0pt}%
\pgfpathmoveto{\pgfqpoint{8.005296in}{2.620654in}}%
\pgfpathlineto{\pgfqpoint{8.088630in}{2.620654in}}%
\pgfpathmoveto{\pgfqpoint{8.046963in}{2.578988in}}%
\pgfpathlineto{\pgfqpoint{8.046963in}{2.662321in}}%
\pgfusepath{stroke,fill}%
\end{pgfscope}%
\begin{pgfscope}%
\pgfpathrectangle{\pgfqpoint{7.105882in}{1.942105in}}{\pgfqpoint{4.376471in}{0.953684in}} %
\pgfusepath{clip}%
\pgfsetbuttcap%
\pgfsetroundjoin%
\definecolor{currentfill}{rgb}{1.000000,0.000000,0.000000}%
\pgfsetfillcolor{currentfill}%
\pgfsetlinewidth{2.007500pt}%
\definecolor{currentstroke}{rgb}{1.000000,0.000000,0.000000}%
\pgfsetstrokecolor{currentstroke}%
\pgfsetdash{}{0pt}%
\pgfpathmoveto{\pgfqpoint{10.101961in}{2.405208in}}%
\pgfpathlineto{\pgfqpoint{10.185294in}{2.405208in}}%
\pgfpathmoveto{\pgfqpoint{10.143627in}{2.363541in}}%
\pgfpathlineto{\pgfqpoint{10.143627in}{2.446874in}}%
\pgfusepath{stroke,fill}%
\end{pgfscope}%
\begin{pgfscope}%
\pgfpathrectangle{\pgfqpoint{7.105882in}{1.942105in}}{\pgfqpoint{4.376471in}{0.953684in}} %
\pgfusepath{clip}%
\pgfsetbuttcap%
\pgfsetroundjoin%
\definecolor{currentfill}{rgb}{1.000000,0.000000,0.000000}%
\pgfsetfillcolor{currentfill}%
\pgfsetlinewidth{2.007500pt}%
\definecolor{currentstroke}{rgb}{1.000000,0.000000,0.000000}%
\pgfsetstrokecolor{currentstroke}%
\pgfsetdash{}{0pt}%
\pgfpathmoveto{\pgfqpoint{10.082565in}{2.444354in}}%
\pgfpathlineto{\pgfqpoint{10.165898in}{2.444354in}}%
\pgfpathmoveto{\pgfqpoint{10.124232in}{2.402687in}}%
\pgfpathlineto{\pgfqpoint{10.124232in}{2.486021in}}%
\pgfusepath{stroke,fill}%
\end{pgfscope}%
\begin{pgfscope}%
\pgfpathrectangle{\pgfqpoint{7.105882in}{1.942105in}}{\pgfqpoint{4.376471in}{0.953684in}} %
\pgfusepath{clip}%
\pgfsetbuttcap%
\pgfsetroundjoin%
\definecolor{currentfill}{rgb}{1.000000,0.000000,0.000000}%
\pgfsetfillcolor{currentfill}%
\pgfsetlinewidth{2.007500pt}%
\definecolor{currentstroke}{rgb}{1.000000,0.000000,0.000000}%
\pgfsetstrokecolor{currentstroke}%
\pgfsetdash{}{0pt}%
\pgfpathmoveto{\pgfqpoint{10.099505in}{2.469980in}}%
\pgfpathlineto{\pgfqpoint{10.182838in}{2.469980in}}%
\pgfpathmoveto{\pgfqpoint{10.141171in}{2.428313in}}%
\pgfpathlineto{\pgfqpoint{10.141171in}{2.511647in}}%
\pgfusepath{stroke,fill}%
\end{pgfscope}%
\begin{pgfscope}%
\pgfpathrectangle{\pgfqpoint{7.105882in}{1.942105in}}{\pgfqpoint{4.376471in}{0.953684in}} %
\pgfusepath{clip}%
\pgfsetbuttcap%
\pgfsetroundjoin%
\definecolor{currentfill}{rgb}{1.000000,0.000000,0.000000}%
\pgfsetfillcolor{currentfill}%
\pgfsetlinewidth{2.007500pt}%
\definecolor{currentstroke}{rgb}{1.000000,0.000000,0.000000}%
\pgfsetstrokecolor{currentstroke}%
\pgfsetdash{}{0pt}%
\pgfpathmoveto{\pgfqpoint{11.243738in}{2.600794in}}%
\pgfpathlineto{\pgfqpoint{11.327072in}{2.600794in}}%
\pgfpathmoveto{\pgfqpoint{11.285405in}{2.559127in}}%
\pgfpathlineto{\pgfqpoint{11.285405in}{2.642461in}}%
\pgfusepath{stroke,fill}%
\end{pgfscope}%
\begin{pgfscope}%
\pgfpathrectangle{\pgfqpoint{7.105882in}{1.942105in}}{\pgfqpoint{4.376471in}{0.953684in}} %
\pgfusepath{clip}%
\pgfsetbuttcap%
\pgfsetroundjoin%
\definecolor{currentfill}{rgb}{1.000000,0.000000,0.000000}%
\pgfsetfillcolor{currentfill}%
\pgfsetlinewidth{2.007500pt}%
\definecolor{currentstroke}{rgb}{1.000000,0.000000,0.000000}%
\pgfsetstrokecolor{currentstroke}%
\pgfsetdash{}{0pt}%
\pgfpathmoveto{\pgfqpoint{10.326683in}{2.347424in}}%
\pgfpathlineto{\pgfqpoint{10.410016in}{2.347424in}}%
\pgfpathmoveto{\pgfqpoint{10.368350in}{2.305757in}}%
\pgfpathlineto{\pgfqpoint{10.368350in}{2.389091in}}%
\pgfusepath{stroke,fill}%
\end{pgfscope}%
\begin{pgfscope}%
\pgfpathrectangle{\pgfqpoint{7.105882in}{1.942105in}}{\pgfqpoint{4.376471in}{0.953684in}} %
\pgfusepath{clip}%
\pgfsetbuttcap%
\pgfsetroundjoin%
\definecolor{currentfill}{rgb}{1.000000,0.000000,0.000000}%
\pgfsetfillcolor{currentfill}%
\pgfsetlinewidth{2.007500pt}%
\definecolor{currentstroke}{rgb}{1.000000,0.000000,0.000000}%
\pgfsetstrokecolor{currentstroke}%
\pgfsetdash{}{0pt}%
\pgfpathmoveto{\pgfqpoint{9.198210in}{2.312659in}}%
\pgfpathlineto{\pgfqpoint{9.281544in}{2.312659in}}%
\pgfpathmoveto{\pgfqpoint{9.239877in}{2.270992in}}%
\pgfpathlineto{\pgfqpoint{9.239877in}{2.354325in}}%
\pgfusepath{stroke,fill}%
\end{pgfscope}%
\begin{pgfscope}%
\pgfpathrectangle{\pgfqpoint{7.105882in}{1.942105in}}{\pgfqpoint{4.376471in}{0.953684in}} %
\pgfusepath{clip}%
\pgfsetbuttcap%
\pgfsetroundjoin%
\definecolor{currentfill}{rgb}{1.000000,0.000000,0.000000}%
\pgfsetfillcolor{currentfill}%
\pgfsetlinewidth{2.007500pt}%
\definecolor{currentstroke}{rgb}{1.000000,0.000000,0.000000}%
\pgfsetstrokecolor{currentstroke}%
\pgfsetdash{}{0pt}%
\pgfpathmoveto{\pgfqpoint{9.469636in}{1.964299in}}%
\pgfpathlineto{\pgfqpoint{9.552969in}{1.964299in}}%
\pgfpathmoveto{\pgfqpoint{9.511302in}{1.922632in}}%
\pgfpathlineto{\pgfqpoint{9.511302in}{2.005966in}}%
\pgfusepath{stroke,fill}%
\end{pgfscope}%
\begin{pgfscope}%
\pgfpathrectangle{\pgfqpoint{7.105882in}{1.942105in}}{\pgfqpoint{4.376471in}{0.953684in}} %
\pgfusepath{clip}%
\pgfsetbuttcap%
\pgfsetroundjoin%
\definecolor{currentfill}{rgb}{1.000000,0.000000,0.000000}%
\pgfsetfillcolor{currentfill}%
\pgfsetlinewidth{2.007500pt}%
\definecolor{currentstroke}{rgb}{1.000000,0.000000,0.000000}%
\pgfsetstrokecolor{currentstroke}%
\pgfsetdash{}{0pt}%
\pgfpathmoveto{\pgfqpoint{10.382040in}{2.365619in}}%
\pgfpathlineto{\pgfqpoint{10.465373in}{2.365619in}}%
\pgfpathmoveto{\pgfqpoint{10.423706in}{2.323952in}}%
\pgfpathlineto{\pgfqpoint{10.423706in}{2.407286in}}%
\pgfusepath{stroke,fill}%
\end{pgfscope}%
\begin{pgfscope}%
\pgfpathrectangle{\pgfqpoint{7.105882in}{1.942105in}}{\pgfqpoint{4.376471in}{0.953684in}} %
\pgfusepath{clip}%
\pgfsetbuttcap%
\pgfsetroundjoin%
\definecolor{currentfill}{rgb}{1.000000,0.000000,0.000000}%
\pgfsetfillcolor{currentfill}%
\pgfsetlinewidth{2.007500pt}%
\definecolor{currentstroke}{rgb}{1.000000,0.000000,0.000000}%
\pgfsetstrokecolor{currentstroke}%
\pgfsetdash{}{0pt}%
\pgfpathmoveto{\pgfqpoint{8.150370in}{2.678546in}}%
\pgfpathlineto{\pgfqpoint{8.233703in}{2.678546in}}%
\pgfpathmoveto{\pgfqpoint{8.192036in}{2.636879in}}%
\pgfpathlineto{\pgfqpoint{8.192036in}{2.720213in}}%
\pgfusepath{stroke,fill}%
\end{pgfscope}%
\begin{pgfscope}%
\pgfpathrectangle{\pgfqpoint{7.105882in}{1.942105in}}{\pgfqpoint{4.376471in}{0.953684in}} %
\pgfusepath{clip}%
\pgfsetbuttcap%
\pgfsetroundjoin%
\definecolor{currentfill}{rgb}{1.000000,0.000000,0.000000}%
\pgfsetfillcolor{currentfill}%
\pgfsetlinewidth{2.007500pt}%
\definecolor{currentstroke}{rgb}{1.000000,0.000000,0.000000}%
\pgfsetstrokecolor{currentstroke}%
\pgfsetdash{}{0pt}%
\pgfpathmoveto{\pgfqpoint{10.273978in}{2.422791in}}%
\pgfpathlineto{\pgfqpoint{10.357311in}{2.422791in}}%
\pgfpathmoveto{\pgfqpoint{10.315644in}{2.381125in}}%
\pgfpathlineto{\pgfqpoint{10.315644in}{2.464458in}}%
\pgfusepath{stroke,fill}%
\end{pgfscope}%
\begin{pgfscope}%
\pgfpathrectangle{\pgfqpoint{7.105882in}{1.942105in}}{\pgfqpoint{4.376471in}{0.953684in}} %
\pgfusepath{clip}%
\pgfsetbuttcap%
\pgfsetroundjoin%
\definecolor{currentfill}{rgb}{1.000000,0.000000,0.000000}%
\pgfsetfillcolor{currentfill}%
\pgfsetlinewidth{2.007500pt}%
\definecolor{currentstroke}{rgb}{1.000000,0.000000,0.000000}%
\pgfsetstrokecolor{currentstroke}%
\pgfsetdash{}{0pt}%
\pgfpathmoveto{\pgfqpoint{10.287531in}{2.274926in}}%
\pgfpathlineto{\pgfqpoint{10.370865in}{2.274926in}}%
\pgfpathmoveto{\pgfqpoint{10.329198in}{2.233259in}}%
\pgfpathlineto{\pgfqpoint{10.329198in}{2.316592in}}%
\pgfusepath{stroke,fill}%
\end{pgfscope}%
\begin{pgfscope}%
\pgfpathrectangle{\pgfqpoint{7.105882in}{1.942105in}}{\pgfqpoint{4.376471in}{0.953684in}} %
\pgfusepath{clip}%
\pgfsetbuttcap%
\pgfsetroundjoin%
\definecolor{currentfill}{rgb}{1.000000,0.000000,0.000000}%
\pgfsetfillcolor{currentfill}%
\pgfsetlinewidth{2.007500pt}%
\definecolor{currentstroke}{rgb}{1.000000,0.000000,0.000000}%
\pgfsetstrokecolor{currentstroke}%
\pgfsetdash{}{0pt}%
\pgfpathmoveto{\pgfqpoint{8.676096in}{2.420341in}}%
\pgfpathlineto{\pgfqpoint{8.759430in}{2.420341in}}%
\pgfpathmoveto{\pgfqpoint{8.717763in}{2.378674in}}%
\pgfpathlineto{\pgfqpoint{8.717763in}{2.462008in}}%
\pgfusepath{stroke,fill}%
\end{pgfscope}%
\begin{pgfscope}%
\pgfpathrectangle{\pgfqpoint{7.105882in}{1.942105in}}{\pgfqpoint{4.376471in}{0.953684in}} %
\pgfusepath{clip}%
\pgfsetbuttcap%
\pgfsetroundjoin%
\definecolor{currentfill}{rgb}{1.000000,0.000000,0.000000}%
\pgfsetfillcolor{currentfill}%
\pgfsetlinewidth{2.007500pt}%
\definecolor{currentstroke}{rgb}{1.000000,0.000000,0.000000}%
\pgfsetstrokecolor{currentstroke}%
\pgfsetdash{}{0pt}%
\pgfpathmoveto{\pgfqpoint{8.390904in}{2.442963in}}%
\pgfpathlineto{\pgfqpoint{8.474237in}{2.442963in}}%
\pgfpathmoveto{\pgfqpoint{8.432570in}{2.401296in}}%
\pgfpathlineto{\pgfqpoint{8.432570in}{2.484630in}}%
\pgfusepath{stroke,fill}%
\end{pgfscope}%
\begin{pgfscope}%
\pgfpathrectangle{\pgfqpoint{7.105882in}{1.942105in}}{\pgfqpoint{4.376471in}{0.953684in}} %
\pgfusepath{clip}%
\pgfsetbuttcap%
\pgfsetroundjoin%
\definecolor{currentfill}{rgb}{1.000000,0.000000,0.000000}%
\pgfsetfillcolor{currentfill}%
\pgfsetlinewidth{2.007500pt}%
\definecolor{currentstroke}{rgb}{1.000000,0.000000,0.000000}%
\pgfsetstrokecolor{currentstroke}%
\pgfsetdash{}{0pt}%
\pgfpathmoveto{\pgfqpoint{9.043880in}{2.417655in}}%
\pgfpathlineto{\pgfqpoint{9.127213in}{2.417655in}}%
\pgfpathmoveto{\pgfqpoint{9.085547in}{2.375988in}}%
\pgfpathlineto{\pgfqpoint{9.085547in}{2.459322in}}%
\pgfusepath{stroke,fill}%
\end{pgfscope}%
\begin{pgfscope}%
\pgfpathrectangle{\pgfqpoint{7.105882in}{1.942105in}}{\pgfqpoint{4.376471in}{0.953684in}} %
\pgfusepath{clip}%
\pgfsetbuttcap%
\pgfsetroundjoin%
\definecolor{currentfill}{rgb}{1.000000,0.000000,0.000000}%
\pgfsetfillcolor{currentfill}%
\pgfsetlinewidth{2.007500pt}%
\definecolor{currentstroke}{rgb}{1.000000,0.000000,0.000000}%
\pgfsetstrokecolor{currentstroke}%
\pgfsetdash{}{0pt}%
\pgfpathmoveto{\pgfqpoint{9.212925in}{2.370447in}}%
\pgfpathlineto{\pgfqpoint{9.296259in}{2.370447in}}%
\pgfpathmoveto{\pgfqpoint{9.254592in}{2.328780in}}%
\pgfpathlineto{\pgfqpoint{9.254592in}{2.412113in}}%
\pgfusepath{stroke,fill}%
\end{pgfscope}%
\begin{pgfscope}%
\pgfpathrectangle{\pgfqpoint{7.105882in}{1.942105in}}{\pgfqpoint{4.376471in}{0.953684in}} %
\pgfusepath{clip}%
\pgfsetbuttcap%
\pgfsetroundjoin%
\definecolor{currentfill}{rgb}{0.000000,0.000000,0.000000}%
\pgfsetfillcolor{currentfill}%
\pgfsetlinewidth{1.003750pt}%
\definecolor{currentstroke}{rgb}{0.000000,0.000000,0.000000}%
\pgfsetstrokecolor{currentstroke}%
\pgfsetdash{}{0pt}%
\pgfsys@defobject{currentmarker}{\pgfqpoint{-0.020833in}{-0.020833in}}{\pgfqpoint{0.020833in}{0.020833in}}{%
\pgfpathmoveto{\pgfqpoint{0.000000in}{-0.020833in}}%
\pgfpathcurveto{\pgfqpoint{0.005525in}{-0.020833in}}{\pgfqpoint{0.010825in}{-0.018638in}}{\pgfqpoint{0.014731in}{-0.014731in}}%
\pgfpathcurveto{\pgfqpoint{0.018638in}{-0.010825in}}{\pgfqpoint{0.020833in}{-0.005525in}}{\pgfqpoint{0.020833in}{0.000000in}}%
\pgfpathcurveto{\pgfqpoint{0.020833in}{0.005525in}}{\pgfqpoint{0.018638in}{0.010825in}}{\pgfqpoint{0.014731in}{0.014731in}}%
\pgfpathcurveto{\pgfqpoint{0.010825in}{0.018638in}}{\pgfqpoint{0.005525in}{0.020833in}}{\pgfqpoint{0.000000in}{0.020833in}}%
\pgfpathcurveto{\pgfqpoint{-0.005525in}{0.020833in}}{\pgfqpoint{-0.010825in}{0.018638in}}{\pgfqpoint{-0.014731in}{0.014731in}}%
\pgfpathcurveto{\pgfqpoint{-0.018638in}{0.010825in}}{\pgfqpoint{-0.020833in}{0.005525in}}{\pgfqpoint{-0.020833in}{0.000000in}}%
\pgfpathcurveto{\pgfqpoint{-0.020833in}{-0.005525in}}{\pgfqpoint{-0.018638in}{-0.010825in}}{\pgfqpoint{-0.014731in}{-0.014731in}}%
\pgfpathcurveto{\pgfqpoint{-0.010825in}{-0.018638in}}{\pgfqpoint{-0.005525in}{-0.020833in}}{\pgfqpoint{0.000000in}{-0.020833in}}%
\pgfpathclose%
\pgfusepath{stroke,fill}%
}%
\begin{pgfscope}%
\pgfsys@transformshift{7.981176in}{2.825884in}%
\pgfsys@useobject{currentmarker}{}%
\end{pgfscope}%
\begin{pgfscope}%
\pgfsys@transformshift{7.998770in}{2.876996in}%
\pgfsys@useobject{currentmarker}{}%
\end{pgfscope}%
\begin{pgfscope}%
\pgfsys@transformshift{8.016364in}{2.789321in}%
\pgfsys@useobject{currentmarker}{}%
\end{pgfscope}%
\begin{pgfscope}%
\pgfsys@transformshift{8.033958in}{2.797403in}%
\pgfsys@useobject{currentmarker}{}%
\end{pgfscope}%
\begin{pgfscope}%
\pgfsys@transformshift{8.051552in}{2.701084in}%
\pgfsys@useobject{currentmarker}{}%
\end{pgfscope}%
\begin{pgfscope}%
\pgfsys@transformshift{8.069146in}{2.850018in}%
\pgfsys@useobject{currentmarker}{}%
\end{pgfscope}%
\begin{pgfscope}%
\pgfsys@transformshift{8.086740in}{2.657713in}%
\pgfsys@useobject{currentmarker}{}%
\end{pgfscope}%
\begin{pgfscope}%
\pgfsys@transformshift{8.104333in}{2.657675in}%
\pgfsys@useobject{currentmarker}{}%
\end{pgfscope}%
\begin{pgfscope}%
\pgfsys@transformshift{8.121927in}{2.777789in}%
\pgfsys@useobject{currentmarker}{}%
\end{pgfscope}%
\begin{pgfscope}%
\pgfsys@transformshift{8.139521in}{2.430331in}%
\pgfsys@useobject{currentmarker}{}%
\end{pgfscope}%
\begin{pgfscope}%
\pgfsys@transformshift{8.157115in}{2.412243in}%
\pgfsys@useobject{currentmarker}{}%
\end{pgfscope}%
\begin{pgfscope}%
\pgfsys@transformshift{8.174709in}{2.609910in}%
\pgfsys@useobject{currentmarker}{}%
\end{pgfscope}%
\begin{pgfscope}%
\pgfsys@transformshift{8.192303in}{2.373313in}%
\pgfsys@useobject{currentmarker}{}%
\end{pgfscope}%
\begin{pgfscope}%
\pgfsys@transformshift{8.209897in}{2.660447in}%
\pgfsys@useobject{currentmarker}{}%
\end{pgfscope}%
\begin{pgfscope}%
\pgfsys@transformshift{8.227490in}{2.405202in}%
\pgfsys@useobject{currentmarker}{}%
\end{pgfscope}%
\begin{pgfscope}%
\pgfsys@transformshift{8.245084in}{2.352529in}%
\pgfsys@useobject{currentmarker}{}%
\end{pgfscope}%
\begin{pgfscope}%
\pgfsys@transformshift{8.262678in}{2.599993in}%
\pgfsys@useobject{currentmarker}{}%
\end{pgfscope}%
\begin{pgfscope}%
\pgfsys@transformshift{8.280272in}{2.540025in}%
\pgfsys@useobject{currentmarker}{}%
\end{pgfscope}%
\begin{pgfscope}%
\pgfsys@transformshift{8.297866in}{2.564355in}%
\pgfsys@useobject{currentmarker}{}%
\end{pgfscope}%
\begin{pgfscope}%
\pgfsys@transformshift{8.315460in}{2.456689in}%
\pgfsys@useobject{currentmarker}{}%
\end{pgfscope}%
\begin{pgfscope}%
\pgfsys@transformshift{8.333054in}{2.271008in}%
\pgfsys@useobject{currentmarker}{}%
\end{pgfscope}%
\begin{pgfscope}%
\pgfsys@transformshift{8.350647in}{2.538263in}%
\pgfsys@useobject{currentmarker}{}%
\end{pgfscope}%
\begin{pgfscope}%
\pgfsys@transformshift{8.368241in}{2.315923in}%
\pgfsys@useobject{currentmarker}{}%
\end{pgfscope}%
\begin{pgfscope}%
\pgfsys@transformshift{8.385835in}{2.418379in}%
\pgfsys@useobject{currentmarker}{}%
\end{pgfscope}%
\begin{pgfscope}%
\pgfsys@transformshift{8.403429in}{2.430937in}%
\pgfsys@useobject{currentmarker}{}%
\end{pgfscope}%
\begin{pgfscope}%
\pgfsys@transformshift{8.421023in}{2.321623in}%
\pgfsys@useobject{currentmarker}{}%
\end{pgfscope}%
\begin{pgfscope}%
\pgfsys@transformshift{8.438617in}{2.400167in}%
\pgfsys@useobject{currentmarker}{}%
\end{pgfscope}%
\begin{pgfscope}%
\pgfsys@transformshift{8.456210in}{2.434787in}%
\pgfsys@useobject{currentmarker}{}%
\end{pgfscope}%
\begin{pgfscope}%
\pgfsys@transformshift{8.473804in}{2.386350in}%
\pgfsys@useobject{currentmarker}{}%
\end{pgfscope}%
\begin{pgfscope}%
\pgfsys@transformshift{8.491398in}{2.247175in}%
\pgfsys@useobject{currentmarker}{}%
\end{pgfscope}%
\begin{pgfscope}%
\pgfsys@transformshift{8.508992in}{2.394977in}%
\pgfsys@useobject{currentmarker}{}%
\end{pgfscope}%
\begin{pgfscope}%
\pgfsys@transformshift{8.526586in}{2.507448in}%
\pgfsys@useobject{currentmarker}{}%
\end{pgfscope}%
\begin{pgfscope}%
\pgfsys@transformshift{8.544180in}{2.318230in}%
\pgfsys@useobject{currentmarker}{}%
\end{pgfscope}%
\begin{pgfscope}%
\pgfsys@transformshift{8.561774in}{2.384955in}%
\pgfsys@useobject{currentmarker}{}%
\end{pgfscope}%
\begin{pgfscope}%
\pgfsys@transformshift{8.579367in}{2.370043in}%
\pgfsys@useobject{currentmarker}{}%
\end{pgfscope}%
\begin{pgfscope}%
\pgfsys@transformshift{8.596961in}{2.611106in}%
\pgfsys@useobject{currentmarker}{}%
\end{pgfscope}%
\begin{pgfscope}%
\pgfsys@transformshift{8.614555in}{2.508784in}%
\pgfsys@useobject{currentmarker}{}%
\end{pgfscope}%
\begin{pgfscope}%
\pgfsys@transformshift{8.632149in}{2.497354in}%
\pgfsys@useobject{currentmarker}{}%
\end{pgfscope}%
\begin{pgfscope}%
\pgfsys@transformshift{8.649743in}{2.395347in}%
\pgfsys@useobject{currentmarker}{}%
\end{pgfscope}%
\begin{pgfscope}%
\pgfsys@transformshift{8.667337in}{2.540233in}%
\pgfsys@useobject{currentmarker}{}%
\end{pgfscope}%
\begin{pgfscope}%
\pgfsys@transformshift{8.684931in}{2.434182in}%
\pgfsys@useobject{currentmarker}{}%
\end{pgfscope}%
\begin{pgfscope}%
\pgfsys@transformshift{8.702524in}{2.518323in}%
\pgfsys@useobject{currentmarker}{}%
\end{pgfscope}%
\begin{pgfscope}%
\pgfsys@transformshift{8.720118in}{2.465276in}%
\pgfsys@useobject{currentmarker}{}%
\end{pgfscope}%
\begin{pgfscope}%
\pgfsys@transformshift{8.737712in}{2.608033in}%
\pgfsys@useobject{currentmarker}{}%
\end{pgfscope}%
\begin{pgfscope}%
\pgfsys@transformshift{8.755306in}{2.609448in}%
\pgfsys@useobject{currentmarker}{}%
\end{pgfscope}%
\begin{pgfscope}%
\pgfsys@transformshift{8.772900in}{2.541589in}%
\pgfsys@useobject{currentmarker}{}%
\end{pgfscope}%
\begin{pgfscope}%
\pgfsys@transformshift{8.790494in}{2.610369in}%
\pgfsys@useobject{currentmarker}{}%
\end{pgfscope}%
\begin{pgfscope}%
\pgfsys@transformshift{8.808087in}{2.469691in}%
\pgfsys@useobject{currentmarker}{}%
\end{pgfscope}%
\begin{pgfscope}%
\pgfsys@transformshift{8.825681in}{2.435769in}%
\pgfsys@useobject{currentmarker}{}%
\end{pgfscope}%
\begin{pgfscope}%
\pgfsys@transformshift{8.843275in}{2.631485in}%
\pgfsys@useobject{currentmarker}{}%
\end{pgfscope}%
\begin{pgfscope}%
\pgfsys@transformshift{8.860869in}{2.606520in}%
\pgfsys@useobject{currentmarker}{}%
\end{pgfscope}%
\begin{pgfscope}%
\pgfsys@transformshift{8.878463in}{2.653318in}%
\pgfsys@useobject{currentmarker}{}%
\end{pgfscope}%
\begin{pgfscope}%
\pgfsys@transformshift{8.896057in}{2.825352in}%
\pgfsys@useobject{currentmarker}{}%
\end{pgfscope}%
\begin{pgfscope}%
\pgfsys@transformshift{8.913651in}{2.678883in}%
\pgfsys@useobject{currentmarker}{}%
\end{pgfscope}%
\begin{pgfscope}%
\pgfsys@transformshift{8.931244in}{2.488903in}%
\pgfsys@useobject{currentmarker}{}%
\end{pgfscope}%
\begin{pgfscope}%
\pgfsys@transformshift{8.948838in}{2.683174in}%
\pgfsys@useobject{currentmarker}{}%
\end{pgfscope}%
\begin{pgfscope}%
\pgfsys@transformshift{8.966432in}{2.432237in}%
\pgfsys@useobject{currentmarker}{}%
\end{pgfscope}%
\begin{pgfscope}%
\pgfsys@transformshift{8.984026in}{2.506052in}%
\pgfsys@useobject{currentmarker}{}%
\end{pgfscope}%
\begin{pgfscope}%
\pgfsys@transformshift{9.001620in}{2.532340in}%
\pgfsys@useobject{currentmarker}{}%
\end{pgfscope}%
\begin{pgfscope}%
\pgfsys@transformshift{9.019214in}{2.694901in}%
\pgfsys@useobject{currentmarker}{}%
\end{pgfscope}%
\begin{pgfscope}%
\pgfsys@transformshift{9.036808in}{2.434757in}%
\pgfsys@useobject{currentmarker}{}%
\end{pgfscope}%
\begin{pgfscope}%
\pgfsys@transformshift{9.054401in}{2.409294in}%
\pgfsys@useobject{currentmarker}{}%
\end{pgfscope}%
\begin{pgfscope}%
\pgfsys@transformshift{9.071995in}{2.463011in}%
\pgfsys@useobject{currentmarker}{}%
\end{pgfscope}%
\begin{pgfscope}%
\pgfsys@transformshift{9.089589in}{2.387182in}%
\pgfsys@useobject{currentmarker}{}%
\end{pgfscope}%
\begin{pgfscope}%
\pgfsys@transformshift{9.107183in}{2.544389in}%
\pgfsys@useobject{currentmarker}{}%
\end{pgfscope}%
\begin{pgfscope}%
\pgfsys@transformshift{9.124777in}{2.303809in}%
\pgfsys@useobject{currentmarker}{}%
\end{pgfscope}%
\begin{pgfscope}%
\pgfsys@transformshift{9.142371in}{2.275163in}%
\pgfsys@useobject{currentmarker}{}%
\end{pgfscope}%
\begin{pgfscope}%
\pgfsys@transformshift{9.159965in}{2.323357in}%
\pgfsys@useobject{currentmarker}{}%
\end{pgfscope}%
\begin{pgfscope}%
\pgfsys@transformshift{9.177558in}{2.294902in}%
\pgfsys@useobject{currentmarker}{}%
\end{pgfscope}%
\begin{pgfscope}%
\pgfsys@transformshift{9.195152in}{2.513620in}%
\pgfsys@useobject{currentmarker}{}%
\end{pgfscope}%
\begin{pgfscope}%
\pgfsys@transformshift{9.212746in}{2.394045in}%
\pgfsys@useobject{currentmarker}{}%
\end{pgfscope}%
\begin{pgfscope}%
\pgfsys@transformshift{9.230340in}{2.286650in}%
\pgfsys@useobject{currentmarker}{}%
\end{pgfscope}%
\begin{pgfscope}%
\pgfsys@transformshift{9.247934in}{2.135068in}%
\pgfsys@useobject{currentmarker}{}%
\end{pgfscope}%
\begin{pgfscope}%
\pgfsys@transformshift{9.265528in}{2.320360in}%
\pgfsys@useobject{currentmarker}{}%
\end{pgfscope}%
\begin{pgfscope}%
\pgfsys@transformshift{9.283121in}{2.117792in}%
\pgfsys@useobject{currentmarker}{}%
\end{pgfscope}%
\begin{pgfscope}%
\pgfsys@transformshift{9.300715in}{2.045548in}%
\pgfsys@useobject{currentmarker}{}%
\end{pgfscope}%
\begin{pgfscope}%
\pgfsys@transformshift{9.318309in}{2.300224in}%
\pgfsys@useobject{currentmarker}{}%
\end{pgfscope}%
\begin{pgfscope}%
\pgfsys@transformshift{9.335903in}{2.198372in}%
\pgfsys@useobject{currentmarker}{}%
\end{pgfscope}%
\begin{pgfscope}%
\pgfsys@transformshift{9.353497in}{2.244676in}%
\pgfsys@useobject{currentmarker}{}%
\end{pgfscope}%
\begin{pgfscope}%
\pgfsys@transformshift{9.371091in}{2.172916in}%
\pgfsys@useobject{currentmarker}{}%
\end{pgfscope}%
\begin{pgfscope}%
\pgfsys@transformshift{9.388685in}{2.216278in}%
\pgfsys@useobject{currentmarker}{}%
\end{pgfscope}%
\begin{pgfscope}%
\pgfsys@transformshift{9.406278in}{2.058327in}%
\pgfsys@useobject{currentmarker}{}%
\end{pgfscope}%
\begin{pgfscope}%
\pgfsys@transformshift{9.423872in}{2.014102in}%
\pgfsys@useobject{currentmarker}{}%
\end{pgfscope}%
\begin{pgfscope}%
\pgfsys@transformshift{9.441466in}{2.180451in}%
\pgfsys@useobject{currentmarker}{}%
\end{pgfscope}%
\begin{pgfscope}%
\pgfsys@transformshift{9.459060in}{2.030900in}%
\pgfsys@useobject{currentmarker}{}%
\end{pgfscope}%
\begin{pgfscope}%
\pgfsys@transformshift{9.476654in}{2.042269in}%
\pgfsys@useobject{currentmarker}{}%
\end{pgfscope}%
\begin{pgfscope}%
\pgfsys@transformshift{9.494248in}{2.067647in}%
\pgfsys@useobject{currentmarker}{}%
\end{pgfscope}%
\begin{pgfscope}%
\pgfsys@transformshift{9.511842in}{2.118807in}%
\pgfsys@useobject{currentmarker}{}%
\end{pgfscope}%
\begin{pgfscope}%
\pgfsys@transformshift{9.529435in}{2.088061in}%
\pgfsys@useobject{currentmarker}{}%
\end{pgfscope}%
\begin{pgfscope}%
\pgfsys@transformshift{9.547029in}{1.994709in}%
\pgfsys@useobject{currentmarker}{}%
\end{pgfscope}%
\begin{pgfscope}%
\pgfsys@transformshift{9.564623in}{2.077258in}%
\pgfsys@useobject{currentmarker}{}%
\end{pgfscope}%
\begin{pgfscope}%
\pgfsys@transformshift{9.582217in}{1.931933in}%
\pgfsys@useobject{currentmarker}{}%
\end{pgfscope}%
\begin{pgfscope}%
\pgfsys@transformshift{9.599811in}{2.228129in}%
\pgfsys@useobject{currentmarker}{}%
\end{pgfscope}%
\begin{pgfscope}%
\pgfsys@transformshift{9.617405in}{2.021524in}%
\pgfsys@useobject{currentmarker}{}%
\end{pgfscope}%
\begin{pgfscope}%
\pgfsys@transformshift{9.634999in}{2.086922in}%
\pgfsys@useobject{currentmarker}{}%
\end{pgfscope}%
\begin{pgfscope}%
\pgfsys@transformshift{9.652592in}{2.218907in}%
\pgfsys@useobject{currentmarker}{}%
\end{pgfscope}%
\begin{pgfscope}%
\pgfsys@transformshift{9.670186in}{2.158302in}%
\pgfsys@useobject{currentmarker}{}%
\end{pgfscope}%
\begin{pgfscope}%
\pgfsys@transformshift{9.687780in}{2.403877in}%
\pgfsys@useobject{currentmarker}{}%
\end{pgfscope}%
\begin{pgfscope}%
\pgfsys@transformshift{9.705374in}{2.141553in}%
\pgfsys@useobject{currentmarker}{}%
\end{pgfscope}%
\begin{pgfscope}%
\pgfsys@transformshift{9.722968in}{2.316296in}%
\pgfsys@useobject{currentmarker}{}%
\end{pgfscope}%
\begin{pgfscope}%
\pgfsys@transformshift{9.740562in}{2.305824in}%
\pgfsys@useobject{currentmarker}{}%
\end{pgfscope}%
\begin{pgfscope}%
\pgfsys@transformshift{9.758155in}{2.213625in}%
\pgfsys@useobject{currentmarker}{}%
\end{pgfscope}%
\begin{pgfscope}%
\pgfsys@transformshift{9.775749in}{2.401368in}%
\pgfsys@useobject{currentmarker}{}%
\end{pgfscope}%
\begin{pgfscope}%
\pgfsys@transformshift{9.793343in}{2.351643in}%
\pgfsys@useobject{currentmarker}{}%
\end{pgfscope}%
\begin{pgfscope}%
\pgfsys@transformshift{9.810937in}{2.464046in}%
\pgfsys@useobject{currentmarker}{}%
\end{pgfscope}%
\begin{pgfscope}%
\pgfsys@transformshift{9.828531in}{2.487077in}%
\pgfsys@useobject{currentmarker}{}%
\end{pgfscope}%
\begin{pgfscope}%
\pgfsys@transformshift{9.846125in}{2.637090in}%
\pgfsys@useobject{currentmarker}{}%
\end{pgfscope}%
\begin{pgfscope}%
\pgfsys@transformshift{9.863719in}{2.570746in}%
\pgfsys@useobject{currentmarker}{}%
\end{pgfscope}%
\begin{pgfscope}%
\pgfsys@transformshift{9.881312in}{2.415788in}%
\pgfsys@useobject{currentmarker}{}%
\end{pgfscope}%
\begin{pgfscope}%
\pgfsys@transformshift{9.898906in}{2.441741in}%
\pgfsys@useobject{currentmarker}{}%
\end{pgfscope}%
\begin{pgfscope}%
\pgfsys@transformshift{9.916500in}{2.586267in}%
\pgfsys@useobject{currentmarker}{}%
\end{pgfscope}%
\begin{pgfscope}%
\pgfsys@transformshift{9.934094in}{2.551958in}%
\pgfsys@useobject{currentmarker}{}%
\end{pgfscope}%
\begin{pgfscope}%
\pgfsys@transformshift{9.951688in}{2.558538in}%
\pgfsys@useobject{currentmarker}{}%
\end{pgfscope}%
\begin{pgfscope}%
\pgfsys@transformshift{9.969282in}{2.340572in}%
\pgfsys@useobject{currentmarker}{}%
\end{pgfscope}%
\begin{pgfscope}%
\pgfsys@transformshift{9.986876in}{2.503165in}%
\pgfsys@useobject{currentmarker}{}%
\end{pgfscope}%
\begin{pgfscope}%
\pgfsys@transformshift{10.004469in}{2.434730in}%
\pgfsys@useobject{currentmarker}{}%
\end{pgfscope}%
\begin{pgfscope}%
\pgfsys@transformshift{10.022063in}{2.536397in}%
\pgfsys@useobject{currentmarker}{}%
\end{pgfscope}%
\begin{pgfscope}%
\pgfsys@transformshift{10.039657in}{2.497462in}%
\pgfsys@useobject{currentmarker}{}%
\end{pgfscope}%
\begin{pgfscope}%
\pgfsys@transformshift{10.057251in}{2.594366in}%
\pgfsys@useobject{currentmarker}{}%
\end{pgfscope}%
\begin{pgfscope}%
\pgfsys@transformshift{10.074845in}{2.530352in}%
\pgfsys@useobject{currentmarker}{}%
\end{pgfscope}%
\begin{pgfscope}%
\pgfsys@transformshift{10.092439in}{2.570072in}%
\pgfsys@useobject{currentmarker}{}%
\end{pgfscope}%
\begin{pgfscope}%
\pgfsys@transformshift{10.110033in}{2.437049in}%
\pgfsys@useobject{currentmarker}{}%
\end{pgfscope}%
\begin{pgfscope}%
\pgfsys@transformshift{10.127626in}{2.379403in}%
\pgfsys@useobject{currentmarker}{}%
\end{pgfscope}%
\begin{pgfscope}%
\pgfsys@transformshift{10.145220in}{2.420867in}%
\pgfsys@useobject{currentmarker}{}%
\end{pgfscope}%
\begin{pgfscope}%
\pgfsys@transformshift{10.162814in}{2.446959in}%
\pgfsys@useobject{currentmarker}{}%
\end{pgfscope}%
\begin{pgfscope}%
\pgfsys@transformshift{10.180408in}{2.472182in}%
\pgfsys@useobject{currentmarker}{}%
\end{pgfscope}%
\begin{pgfscope}%
\pgfsys@transformshift{10.198002in}{2.643812in}%
\pgfsys@useobject{currentmarker}{}%
\end{pgfscope}%
\begin{pgfscope}%
\pgfsys@transformshift{10.215596in}{2.399178in}%
\pgfsys@useobject{currentmarker}{}%
\end{pgfscope}%
\begin{pgfscope}%
\pgfsys@transformshift{10.233189in}{2.291759in}%
\pgfsys@useobject{currentmarker}{}%
\end{pgfscope}%
\begin{pgfscope}%
\pgfsys@transformshift{10.250783in}{2.335234in}%
\pgfsys@useobject{currentmarker}{}%
\end{pgfscope}%
\begin{pgfscope}%
\pgfsys@transformshift{10.268377in}{2.306158in}%
\pgfsys@useobject{currentmarker}{}%
\end{pgfscope}%
\begin{pgfscope}%
\pgfsys@transformshift{10.285971in}{2.382579in}%
\pgfsys@useobject{currentmarker}{}%
\end{pgfscope}%
\begin{pgfscope}%
\pgfsys@transformshift{10.303565in}{2.164346in}%
\pgfsys@useobject{currentmarker}{}%
\end{pgfscope}%
\begin{pgfscope}%
\pgfsys@transformshift{10.321159in}{2.306691in}%
\pgfsys@useobject{currentmarker}{}%
\end{pgfscope}%
\begin{pgfscope}%
\pgfsys@transformshift{10.338753in}{2.299476in}%
\pgfsys@useobject{currentmarker}{}%
\end{pgfscope}%
\begin{pgfscope}%
\pgfsys@transformshift{10.356346in}{2.291238in}%
\pgfsys@useobject{currentmarker}{}%
\end{pgfscope}%
\begin{pgfscope}%
\pgfsys@transformshift{10.373940in}{2.193987in}%
\pgfsys@useobject{currentmarker}{}%
\end{pgfscope}%
\begin{pgfscope}%
\pgfsys@transformshift{10.391534in}{2.215932in}%
\pgfsys@useobject{currentmarker}{}%
\end{pgfscope}%
\begin{pgfscope}%
\pgfsys@transformshift{10.409128in}{2.085597in}%
\pgfsys@useobject{currentmarker}{}%
\end{pgfscope}%
\begin{pgfscope}%
\pgfsys@transformshift{10.426722in}{2.166972in}%
\pgfsys@useobject{currentmarker}{}%
\end{pgfscope}%
\begin{pgfscope}%
\pgfsys@transformshift{10.444316in}{2.152521in}%
\pgfsys@useobject{currentmarker}{}%
\end{pgfscope}%
\begin{pgfscope}%
\pgfsys@transformshift{10.461910in}{2.239938in}%
\pgfsys@useobject{currentmarker}{}%
\end{pgfscope}%
\begin{pgfscope}%
\pgfsys@transformshift{10.479503in}{2.077705in}%
\pgfsys@useobject{currentmarker}{}%
\end{pgfscope}%
\begin{pgfscope}%
\pgfsys@transformshift{10.497097in}{2.266017in}%
\pgfsys@useobject{currentmarker}{}%
\end{pgfscope}%
\begin{pgfscope}%
\pgfsys@transformshift{10.514691in}{2.334696in}%
\pgfsys@useobject{currentmarker}{}%
\end{pgfscope}%
\begin{pgfscope}%
\pgfsys@transformshift{10.532285in}{1.980441in}%
\pgfsys@useobject{currentmarker}{}%
\end{pgfscope}%
\begin{pgfscope}%
\pgfsys@transformshift{10.549879in}{2.230336in}%
\pgfsys@useobject{currentmarker}{}%
\end{pgfscope}%
\begin{pgfscope}%
\pgfsys@transformshift{10.567473in}{2.259233in}%
\pgfsys@useobject{currentmarker}{}%
\end{pgfscope}%
\begin{pgfscope}%
\pgfsys@transformshift{10.585067in}{2.134742in}%
\pgfsys@useobject{currentmarker}{}%
\end{pgfscope}%
\begin{pgfscope}%
\pgfsys@transformshift{10.602660in}{2.166738in}%
\pgfsys@useobject{currentmarker}{}%
\end{pgfscope}%
\begin{pgfscope}%
\pgfsys@transformshift{10.620254in}{2.203158in}%
\pgfsys@useobject{currentmarker}{}%
\end{pgfscope}%
\begin{pgfscope}%
\pgfsys@transformshift{10.637848in}{2.198885in}%
\pgfsys@useobject{currentmarker}{}%
\end{pgfscope}%
\begin{pgfscope}%
\pgfsys@transformshift{10.655442in}{2.211754in}%
\pgfsys@useobject{currentmarker}{}%
\end{pgfscope}%
\begin{pgfscope}%
\pgfsys@transformshift{10.673036in}{2.091712in}%
\pgfsys@useobject{currentmarker}{}%
\end{pgfscope}%
\begin{pgfscope}%
\pgfsys@transformshift{10.690630in}{2.390102in}%
\pgfsys@useobject{currentmarker}{}%
\end{pgfscope}%
\begin{pgfscope}%
\pgfsys@transformshift{10.708223in}{2.401822in}%
\pgfsys@useobject{currentmarker}{}%
\end{pgfscope}%
\begin{pgfscope}%
\pgfsys@transformshift{10.725817in}{2.234093in}%
\pgfsys@useobject{currentmarker}{}%
\end{pgfscope}%
\begin{pgfscope}%
\pgfsys@transformshift{10.743411in}{2.190761in}%
\pgfsys@useobject{currentmarker}{}%
\end{pgfscope}%
\begin{pgfscope}%
\pgfsys@transformshift{10.761005in}{2.410796in}%
\pgfsys@useobject{currentmarker}{}%
\end{pgfscope}%
\begin{pgfscope}%
\pgfsys@transformshift{10.778599in}{2.325294in}%
\pgfsys@useobject{currentmarker}{}%
\end{pgfscope}%
\begin{pgfscope}%
\pgfsys@transformshift{10.796193in}{2.420737in}%
\pgfsys@useobject{currentmarker}{}%
\end{pgfscope}%
\begin{pgfscope}%
\pgfsys@transformshift{10.813787in}{2.399537in}%
\pgfsys@useobject{currentmarker}{}%
\end{pgfscope}%
\begin{pgfscope}%
\pgfsys@transformshift{10.831380in}{2.524868in}%
\pgfsys@useobject{currentmarker}{}%
\end{pgfscope}%
\begin{pgfscope}%
\pgfsys@transformshift{10.848974in}{2.550181in}%
\pgfsys@useobject{currentmarker}{}%
\end{pgfscope}%
\begin{pgfscope}%
\pgfsys@transformshift{10.866568in}{2.433957in}%
\pgfsys@useobject{currentmarker}{}%
\end{pgfscope}%
\begin{pgfscope}%
\pgfsys@transformshift{10.884162in}{2.393173in}%
\pgfsys@useobject{currentmarker}{}%
\end{pgfscope}%
\begin{pgfscope}%
\pgfsys@transformshift{10.901756in}{2.397395in}%
\pgfsys@useobject{currentmarker}{}%
\end{pgfscope}%
\begin{pgfscope}%
\pgfsys@transformshift{10.919350in}{2.638631in}%
\pgfsys@useobject{currentmarker}{}%
\end{pgfscope}%
\begin{pgfscope}%
\pgfsys@transformshift{10.936944in}{2.482355in}%
\pgfsys@useobject{currentmarker}{}%
\end{pgfscope}%
\begin{pgfscope}%
\pgfsys@transformshift{10.954537in}{2.571521in}%
\pgfsys@useobject{currentmarker}{}%
\end{pgfscope}%
\begin{pgfscope}%
\pgfsys@transformshift{10.972131in}{2.582827in}%
\pgfsys@useobject{currentmarker}{}%
\end{pgfscope}%
\begin{pgfscope}%
\pgfsys@transformshift{10.989725in}{2.655679in}%
\pgfsys@useobject{currentmarker}{}%
\end{pgfscope}%
\begin{pgfscope}%
\pgfsys@transformshift{11.007319in}{2.485889in}%
\pgfsys@useobject{currentmarker}{}%
\end{pgfscope}%
\begin{pgfscope}%
\pgfsys@transformshift{11.024913in}{2.712727in}%
\pgfsys@useobject{currentmarker}{}%
\end{pgfscope}%
\begin{pgfscope}%
\pgfsys@transformshift{11.042507in}{2.760039in}%
\pgfsys@useobject{currentmarker}{}%
\end{pgfscope}%
\begin{pgfscope}%
\pgfsys@transformshift{11.060101in}{2.729007in}%
\pgfsys@useobject{currentmarker}{}%
\end{pgfscope}%
\begin{pgfscope}%
\pgfsys@transformshift{11.077694in}{2.699714in}%
\pgfsys@useobject{currentmarker}{}%
\end{pgfscope}%
\begin{pgfscope}%
\pgfsys@transformshift{11.095288in}{2.748751in}%
\pgfsys@useobject{currentmarker}{}%
\end{pgfscope}%
\begin{pgfscope}%
\pgfsys@transformshift{11.112882in}{2.785274in}%
\pgfsys@useobject{currentmarker}{}%
\end{pgfscope}%
\begin{pgfscope}%
\pgfsys@transformshift{11.130476in}{2.474829in}%
\pgfsys@useobject{currentmarker}{}%
\end{pgfscope}%
\begin{pgfscope}%
\pgfsys@transformshift{11.148070in}{2.947295in}%
\pgfsys@useobject{currentmarker}{}%
\end{pgfscope}%
\begin{pgfscope}%
\pgfsys@transformshift{11.165664in}{2.792601in}%
\pgfsys@useobject{currentmarker}{}%
\end{pgfscope}%
\begin{pgfscope}%
\pgfsys@transformshift{11.183257in}{2.688082in}%
\pgfsys@useobject{currentmarker}{}%
\end{pgfscope}%
\begin{pgfscope}%
\pgfsys@transformshift{11.200851in}{2.711365in}%
\pgfsys@useobject{currentmarker}{}%
\end{pgfscope}%
\begin{pgfscope}%
\pgfsys@transformshift{11.218445in}{2.794920in}%
\pgfsys@useobject{currentmarker}{}%
\end{pgfscope}%
\begin{pgfscope}%
\pgfsys@transformshift{11.236039in}{2.728474in}%
\pgfsys@useobject{currentmarker}{}%
\end{pgfscope}%
\begin{pgfscope}%
\pgfsys@transformshift{11.253633in}{2.531011in}%
\pgfsys@useobject{currentmarker}{}%
\end{pgfscope}%
\begin{pgfscope}%
\pgfsys@transformshift{11.271227in}{2.929312in}%
\pgfsys@useobject{currentmarker}{}%
\end{pgfscope}%
\begin{pgfscope}%
\pgfsys@transformshift{11.288821in}{2.703614in}%
\pgfsys@useobject{currentmarker}{}%
\end{pgfscope}%
\begin{pgfscope}%
\pgfsys@transformshift{11.306414in}{2.805382in}%
\pgfsys@useobject{currentmarker}{}%
\end{pgfscope}%
\begin{pgfscope}%
\pgfsys@transformshift{11.324008in}{2.623972in}%
\pgfsys@useobject{currentmarker}{}%
\end{pgfscope}%
\begin{pgfscope}%
\pgfsys@transformshift{11.341602in}{2.833466in}%
\pgfsys@useobject{currentmarker}{}%
\end{pgfscope}%
\begin{pgfscope}%
\pgfsys@transformshift{11.359196in}{2.697027in}%
\pgfsys@useobject{currentmarker}{}%
\end{pgfscope}%
\begin{pgfscope}%
\pgfsys@transformshift{11.376790in}{2.716632in}%
\pgfsys@useobject{currentmarker}{}%
\end{pgfscope}%
\begin{pgfscope}%
\pgfsys@transformshift{11.394384in}{2.539901in}%
\pgfsys@useobject{currentmarker}{}%
\end{pgfscope}%
\begin{pgfscope}%
\pgfsys@transformshift{11.411978in}{2.751873in}%
\pgfsys@useobject{currentmarker}{}%
\end{pgfscope}%
\begin{pgfscope}%
\pgfsys@transformshift{11.429571in}{2.688326in}%
\pgfsys@useobject{currentmarker}{}%
\end{pgfscope}%
\begin{pgfscope}%
\pgfsys@transformshift{11.447165in}{2.737992in}%
\pgfsys@useobject{currentmarker}{}%
\end{pgfscope}%
\begin{pgfscope}%
\pgfsys@transformshift{11.464759in}{2.535946in}%
\pgfsys@useobject{currentmarker}{}%
\end{pgfscope}%
\begin{pgfscope}%
\pgfsys@transformshift{11.482353in}{2.541231in}%
\pgfsys@useobject{currentmarker}{}%
\end{pgfscope}%
\end{pgfscope}%
\begin{pgfscope}%
\pgfsetbuttcap%
\pgfsetroundjoin%
\definecolor{currentfill}{rgb}{0.000000,0.000000,0.000000}%
\pgfsetfillcolor{currentfill}%
\pgfsetlinewidth{0.803000pt}%
\definecolor{currentstroke}{rgb}{0.000000,0.000000,0.000000}%
\pgfsetstrokecolor{currentstroke}%
\pgfsetdash{}{0pt}%
\pgfsys@defobject{currentmarker}{\pgfqpoint{0.000000in}{-0.048611in}}{\pgfqpoint{0.000000in}{0.000000in}}{%
\pgfpathmoveto{\pgfqpoint{0.000000in}{0.000000in}}%
\pgfpathlineto{\pgfqpoint{0.000000in}{-0.048611in}}%
\pgfusepath{stroke,fill}%
}%
\begin{pgfscope}%
\pgfsys@transformshift{7.105882in}{1.942105in}%
\pgfsys@useobject{currentmarker}{}%
\end{pgfscope}%
\end{pgfscope}%
\begin{pgfscope}%
\pgfsetbuttcap%
\pgfsetroundjoin%
\definecolor{currentfill}{rgb}{0.000000,0.000000,0.000000}%
\pgfsetfillcolor{currentfill}%
\pgfsetlinewidth{0.803000pt}%
\definecolor{currentstroke}{rgb}{0.000000,0.000000,0.000000}%
\pgfsetstrokecolor{currentstroke}%
\pgfsetdash{}{0pt}%
\pgfsys@defobject{currentmarker}{\pgfqpoint{0.000000in}{-0.048611in}}{\pgfqpoint{0.000000in}{0.000000in}}{%
\pgfpathmoveto{\pgfqpoint{0.000000in}{0.000000in}}%
\pgfpathlineto{\pgfqpoint{0.000000in}{-0.048611in}}%
\pgfusepath{stroke,fill}%
}%
\begin{pgfscope}%
\pgfsys@transformshift{7.981176in}{1.942105in}%
\pgfsys@useobject{currentmarker}{}%
\end{pgfscope}%
\end{pgfscope}%
\begin{pgfscope}%
\pgfsetbuttcap%
\pgfsetroundjoin%
\definecolor{currentfill}{rgb}{0.000000,0.000000,0.000000}%
\pgfsetfillcolor{currentfill}%
\pgfsetlinewidth{0.803000pt}%
\definecolor{currentstroke}{rgb}{0.000000,0.000000,0.000000}%
\pgfsetstrokecolor{currentstroke}%
\pgfsetdash{}{0pt}%
\pgfsys@defobject{currentmarker}{\pgfqpoint{0.000000in}{-0.048611in}}{\pgfqpoint{0.000000in}{0.000000in}}{%
\pgfpathmoveto{\pgfqpoint{0.000000in}{0.000000in}}%
\pgfpathlineto{\pgfqpoint{0.000000in}{-0.048611in}}%
\pgfusepath{stroke,fill}%
}%
\begin{pgfscope}%
\pgfsys@transformshift{8.856471in}{1.942105in}%
\pgfsys@useobject{currentmarker}{}%
\end{pgfscope}%
\end{pgfscope}%
\begin{pgfscope}%
\pgfsetbuttcap%
\pgfsetroundjoin%
\definecolor{currentfill}{rgb}{0.000000,0.000000,0.000000}%
\pgfsetfillcolor{currentfill}%
\pgfsetlinewidth{0.803000pt}%
\definecolor{currentstroke}{rgb}{0.000000,0.000000,0.000000}%
\pgfsetstrokecolor{currentstroke}%
\pgfsetdash{}{0pt}%
\pgfsys@defobject{currentmarker}{\pgfqpoint{0.000000in}{-0.048611in}}{\pgfqpoint{0.000000in}{0.000000in}}{%
\pgfpathmoveto{\pgfqpoint{0.000000in}{0.000000in}}%
\pgfpathlineto{\pgfqpoint{0.000000in}{-0.048611in}}%
\pgfusepath{stroke,fill}%
}%
\begin{pgfscope}%
\pgfsys@transformshift{9.731765in}{1.942105in}%
\pgfsys@useobject{currentmarker}{}%
\end{pgfscope}%
\end{pgfscope}%
\begin{pgfscope}%
\pgfsetbuttcap%
\pgfsetroundjoin%
\definecolor{currentfill}{rgb}{0.000000,0.000000,0.000000}%
\pgfsetfillcolor{currentfill}%
\pgfsetlinewidth{0.803000pt}%
\definecolor{currentstroke}{rgb}{0.000000,0.000000,0.000000}%
\pgfsetstrokecolor{currentstroke}%
\pgfsetdash{}{0pt}%
\pgfsys@defobject{currentmarker}{\pgfqpoint{0.000000in}{-0.048611in}}{\pgfqpoint{0.000000in}{0.000000in}}{%
\pgfpathmoveto{\pgfqpoint{0.000000in}{0.000000in}}%
\pgfpathlineto{\pgfqpoint{0.000000in}{-0.048611in}}%
\pgfusepath{stroke,fill}%
}%
\begin{pgfscope}%
\pgfsys@transformshift{10.607059in}{1.942105in}%
\pgfsys@useobject{currentmarker}{}%
\end{pgfscope}%
\end{pgfscope}%
\begin{pgfscope}%
\pgfsetbuttcap%
\pgfsetroundjoin%
\definecolor{currentfill}{rgb}{0.000000,0.000000,0.000000}%
\pgfsetfillcolor{currentfill}%
\pgfsetlinewidth{0.803000pt}%
\definecolor{currentstroke}{rgb}{0.000000,0.000000,0.000000}%
\pgfsetstrokecolor{currentstroke}%
\pgfsetdash{}{0pt}%
\pgfsys@defobject{currentmarker}{\pgfqpoint{0.000000in}{-0.048611in}}{\pgfqpoint{0.000000in}{0.000000in}}{%
\pgfpathmoveto{\pgfqpoint{0.000000in}{0.000000in}}%
\pgfpathlineto{\pgfqpoint{0.000000in}{-0.048611in}}%
\pgfusepath{stroke,fill}%
}%
\begin{pgfscope}%
\pgfsys@transformshift{11.482353in}{1.942105in}%
\pgfsys@useobject{currentmarker}{}%
\end{pgfscope}%
\end{pgfscope}%
\begin{pgfscope}%
\pgfsetbuttcap%
\pgfsetroundjoin%
\definecolor{currentfill}{rgb}{0.000000,0.000000,0.000000}%
\pgfsetfillcolor{currentfill}%
\pgfsetlinewidth{0.803000pt}%
\definecolor{currentstroke}{rgb}{0.000000,0.000000,0.000000}%
\pgfsetstrokecolor{currentstroke}%
\pgfsetdash{}{0pt}%
\pgfsys@defobject{currentmarker}{\pgfqpoint{-0.048611in}{0.000000in}}{\pgfqpoint{0.000000in}{0.000000in}}{%
\pgfpathmoveto{\pgfqpoint{0.000000in}{0.000000in}}%
\pgfpathlineto{\pgfqpoint{-0.048611in}{0.000000in}}%
\pgfusepath{stroke,fill}%
}%
\begin{pgfscope}%
\pgfsys@transformshift{7.105882in}{2.299737in}%
\pgfsys@useobject{currentmarker}{}%
\end{pgfscope}%
\end{pgfscope}%
\begin{pgfscope}%
\pgftext[x=6.939215in,y=2.251519in,left,base]{\rmfamily\fontsize{10.000000}{12.000000}\selectfont \(\displaystyle 0\)}%
\end{pgfscope}%
\begin{pgfscope}%
\pgfsetbuttcap%
\pgfsetroundjoin%
\definecolor{currentfill}{rgb}{0.000000,0.000000,0.000000}%
\pgfsetfillcolor{currentfill}%
\pgfsetlinewidth{0.803000pt}%
\definecolor{currentstroke}{rgb}{0.000000,0.000000,0.000000}%
\pgfsetstrokecolor{currentstroke}%
\pgfsetdash{}{0pt}%
\pgfsys@defobject{currentmarker}{\pgfqpoint{-0.048611in}{0.000000in}}{\pgfqpoint{0.000000in}{0.000000in}}{%
\pgfpathmoveto{\pgfqpoint{0.000000in}{0.000000in}}%
\pgfpathlineto{\pgfqpoint{-0.048611in}{0.000000in}}%
\pgfusepath{stroke,fill}%
}%
\begin{pgfscope}%
\pgfsys@transformshift{7.105882in}{2.697105in}%
\pgfsys@useobject{currentmarker}{}%
\end{pgfscope}%
\end{pgfscope}%
\begin{pgfscope}%
\pgftext[x=6.939215in,y=2.648887in,left,base]{\rmfamily\fontsize{10.000000}{12.000000}\selectfont \(\displaystyle 2\)}%
\end{pgfscope}%
\begin{pgfscope}%
\pgfpathrectangle{\pgfqpoint{7.105882in}{1.942105in}}{\pgfqpoint{4.376471in}{0.953684in}} %
\pgfusepath{clip}%
\pgfsetrectcap%
\pgfsetroundjoin%
\pgfsetlinewidth{1.505625pt}%
\definecolor{currentstroke}{rgb}{0.121569,0.466667,0.705882}%
\pgfsetstrokecolor{currentstroke}%
\pgfsetdash{}{0pt}%
\pgfpathmoveto{\pgfqpoint{7.981176in}{2.299737in}}%
\pgfpathlineto{\pgfqpoint{11.482353in}{2.299737in}}%
\pgfpathlineto{\pgfqpoint{11.482353in}{2.299737in}}%
\pgfusepath{stroke}%
\end{pgfscope}%
\begin{pgfscope}%
\pgfsetrectcap%
\pgfsetmiterjoin%
\pgfsetlinewidth{0.803000pt}%
\definecolor{currentstroke}{rgb}{0.000000,0.000000,0.000000}%
\pgfsetstrokecolor{currentstroke}%
\pgfsetdash{}{0pt}%
\pgfpathmoveto{\pgfqpoint{7.105882in}{1.942105in}}%
\pgfpathlineto{\pgfqpoint{7.105882in}{2.895789in}}%
\pgfusepath{stroke}%
\end{pgfscope}%
\begin{pgfscope}%
\pgfsetrectcap%
\pgfsetmiterjoin%
\pgfsetlinewidth{0.803000pt}%
\definecolor{currentstroke}{rgb}{0.000000,0.000000,0.000000}%
\pgfsetstrokecolor{currentstroke}%
\pgfsetdash{}{0pt}%
\pgfpathmoveto{\pgfqpoint{11.482353in}{1.942105in}}%
\pgfpathlineto{\pgfqpoint{11.482353in}{2.895789in}}%
\pgfusepath{stroke}%
\end{pgfscope}%
\begin{pgfscope}%
\pgfsetrectcap%
\pgfsetmiterjoin%
\pgfsetlinewidth{0.803000pt}%
\definecolor{currentstroke}{rgb}{0.000000,0.000000,0.000000}%
\pgfsetstrokecolor{currentstroke}%
\pgfsetdash{}{0pt}%
\pgfpathmoveto{\pgfqpoint{7.105882in}{1.942105in}}%
\pgfpathlineto{\pgfqpoint{11.482353in}{1.942105in}}%
\pgfusepath{stroke}%
\end{pgfscope}%
\begin{pgfscope}%
\pgfsetrectcap%
\pgfsetmiterjoin%
\pgfsetlinewidth{0.803000pt}%
\definecolor{currentstroke}{rgb}{0.000000,0.000000,0.000000}%
\pgfsetstrokecolor{currentstroke}%
\pgfsetdash{}{0pt}%
\pgfpathmoveto{\pgfqpoint{7.105882in}{2.895789in}}%
\pgfpathlineto{\pgfqpoint{11.482353in}{2.895789in}}%
\pgfusepath{stroke}%
\end{pgfscope}%
\begin{pgfscope}%
\pgfsetbuttcap%
\pgfsetmiterjoin%
\definecolor{currentfill}{rgb}{1.000000,1.000000,1.000000}%
\pgfsetfillcolor{currentfill}%
\pgfsetfillopacity{0.800000}%
\pgfsetlinewidth{1.003750pt}%
\definecolor{currentstroke}{rgb}{0.800000,0.800000,0.800000}%
\pgfsetstrokecolor{currentstroke}%
\pgfsetstrokeopacity{0.800000}%
\pgfsetdash{}{0pt}%
\pgfpathmoveto{\pgfqpoint{7.203105in}{2.011550in}}%
\pgfpathlineto{\pgfqpoint{7.944617in}{2.011550in}}%
\pgfpathquadraticcurveto{\pgfqpoint{7.972395in}{2.011550in}}{\pgfqpoint{7.972395in}{2.039327in}}%
\pgfpathlineto{\pgfqpoint{7.972395in}{2.623040in}}%
\pgfpathquadraticcurveto{\pgfqpoint{7.972395in}{2.650818in}}{\pgfqpoint{7.944617in}{2.650818in}}%
\pgfpathlineto{\pgfqpoint{7.203105in}{2.650818in}}%
\pgfpathquadraticcurveto{\pgfqpoint{7.175327in}{2.650818in}}{\pgfqpoint{7.175327in}{2.623040in}}%
\pgfpathlineto{\pgfqpoint{7.175327in}{2.039327in}}%
\pgfpathquadraticcurveto{\pgfqpoint{7.175327in}{2.011550in}}{\pgfqpoint{7.203105in}{2.011550in}}%
\pgfpathclose%
\pgfusepath{stroke,fill}%
\end{pgfscope}%
\begin{pgfscope}%
\pgfsetrectcap%
\pgfsetroundjoin%
\pgfsetlinewidth{1.505625pt}%
\definecolor{currentstroke}{rgb}{0.121569,0.466667,0.705882}%
\pgfsetstrokecolor{currentstroke}%
\pgfsetdash{}{0pt}%
\pgfpathmoveto{\pgfqpoint{7.230882in}{2.538161in}}%
\pgfpathlineto{\pgfqpoint{7.508660in}{2.538161in}}%
\pgfusepath{stroke}%
\end{pgfscope}%
\begin{pgfscope}%
\pgftext[x=7.619771in,y=2.489549in,left,base]{\rmfamily\fontsize{10.000000}{12.000000}\selectfont \(\displaystyle \widetilde{\Phi}^* \theta^{\parallel}\)}%
\end{pgfscope}%
\begin{pgfscope}%
\pgfsetbuttcap%
\pgfsetroundjoin%
\definecolor{currentfill}{rgb}{1.000000,0.000000,0.000000}%
\pgfsetfillcolor{currentfill}%
\pgfsetlinewidth{2.007500pt}%
\definecolor{currentstroke}{rgb}{1.000000,0.000000,0.000000}%
\pgfsetstrokecolor{currentstroke}%
\pgfsetdash{}{0pt}%
\pgfpathmoveto{\pgfqpoint{7.328105in}{2.329637in}}%
\pgfpathlineto{\pgfqpoint{7.411438in}{2.329637in}}%
\pgfpathmoveto{\pgfqpoint{7.369771in}{2.287971in}}%
\pgfpathlineto{\pgfqpoint{7.369771in}{2.371304in}}%
\pgfusepath{stroke,fill}%
\end{pgfscope}%
\begin{pgfscope}%
\pgftext[x=7.619771in,y=2.293179in,left,base]{\rmfamily\fontsize{10.000000}{12.000000}\selectfont train}%
\end{pgfscope}%
\begin{pgfscope}%
\pgfsetbuttcap%
\pgfsetroundjoin%
\definecolor{currentfill}{rgb}{0.000000,0.000000,0.000000}%
\pgfsetfillcolor{currentfill}%
\pgfsetlinewidth{1.003750pt}%
\definecolor{currentstroke}{rgb}{0.000000,0.000000,0.000000}%
\pgfsetstrokecolor{currentstroke}%
\pgfsetdash{}{0pt}%
\pgfsys@defobject{currentmarker}{\pgfqpoint{-0.020833in}{-0.020833in}}{\pgfqpoint{0.020833in}{0.020833in}}{%
\pgfpathmoveto{\pgfqpoint{0.000000in}{-0.020833in}}%
\pgfpathcurveto{\pgfqpoint{0.005525in}{-0.020833in}}{\pgfqpoint{0.010825in}{-0.018638in}}{\pgfqpoint{0.014731in}{-0.014731in}}%
\pgfpathcurveto{\pgfqpoint{0.018638in}{-0.010825in}}{\pgfqpoint{0.020833in}{-0.005525in}}{\pgfqpoint{0.020833in}{0.000000in}}%
\pgfpathcurveto{\pgfqpoint{0.020833in}{0.005525in}}{\pgfqpoint{0.018638in}{0.010825in}}{\pgfqpoint{0.014731in}{0.014731in}}%
\pgfpathcurveto{\pgfqpoint{0.010825in}{0.018638in}}{\pgfqpoint{0.005525in}{0.020833in}}{\pgfqpoint{0.000000in}{0.020833in}}%
\pgfpathcurveto{\pgfqpoint{-0.005525in}{0.020833in}}{\pgfqpoint{-0.010825in}{0.018638in}}{\pgfqpoint{-0.014731in}{0.014731in}}%
\pgfpathcurveto{\pgfqpoint{-0.018638in}{0.010825in}}{\pgfqpoint{-0.020833in}{0.005525in}}{\pgfqpoint{-0.020833in}{0.000000in}}%
\pgfpathcurveto{\pgfqpoint{-0.020833in}{-0.005525in}}{\pgfqpoint{-0.018638in}{-0.010825in}}{\pgfqpoint{-0.014731in}{-0.014731in}}%
\pgfpathcurveto{\pgfqpoint{-0.010825in}{-0.018638in}}{\pgfqpoint{-0.005525in}{-0.020833in}}{\pgfqpoint{0.000000in}{-0.020833in}}%
\pgfpathclose%
\pgfusepath{stroke,fill}%
}%
\begin{pgfscope}%
\pgfsys@transformshift{7.369771in}{2.133267in}%
\pgfsys@useobject{currentmarker}{}%
\end{pgfscope}%
\end{pgfscope}%
\begin{pgfscope}%
\pgftext[x=7.619771in,y=2.096809in,left,base]{\rmfamily\fontsize{10.000000}{12.000000}\selectfont test}%
\end{pgfscope}%
\begin{pgfscope}%
\pgfsetbuttcap%
\pgfsetmiterjoin%
\definecolor{currentfill}{rgb}{1.000000,1.000000,1.000000}%
\pgfsetfillcolor{currentfill}%
\pgfsetlinewidth{0.000000pt}%
\definecolor{currentstroke}{rgb}{0.000000,0.000000,0.000000}%
\pgfsetstrokecolor{currentstroke}%
\pgfsetstrokeopacity{0.000000}%
\pgfsetdash{}{0pt}%
\pgfpathmoveto{\pgfqpoint{12.211765in}{1.942105in}}%
\pgfpathlineto{\pgfqpoint{14.400000in}{1.942105in}}%
\pgfpathlineto{\pgfqpoint{14.400000in}{2.895789in}}%
\pgfpathlineto{\pgfqpoint{12.211765in}{2.895789in}}%
\pgfpathclose%
\pgfusepath{fill}%
\end{pgfscope}%
\begin{pgfscope}%
\pgfpathrectangle{\pgfqpoint{12.211765in}{1.942105in}}{\pgfqpoint{2.188235in}{0.953684in}} %
\pgfusepath{clip}%
\pgfsetbuttcap%
\pgfsetmiterjoin%
\definecolor{currentfill}{rgb}{0.121569,0.466667,0.705882}%
\pgfsetfillcolor{currentfill}%
\pgfsetlinewidth{0.000000pt}%
\definecolor{currentstroke}{rgb}{0.000000,0.000000,0.000000}%
\pgfsetstrokecolor{currentstroke}%
\pgfsetstrokeopacity{0.000000}%
\pgfsetdash{}{0pt}%
\pgfpathmoveto{\pgfqpoint{-23.809001in}{1.985455in}}%
\pgfpathlineto{\pgfqpoint{13.637665in}{1.985455in}}%
\pgfpathlineto{\pgfqpoint{13.637665in}{1.992404in}}%
\pgfpathlineto{\pgfqpoint{-23.809001in}{1.992404in}}%
\pgfpathclose%
\pgfusepath{fill}%
\end{pgfscope}%
\begin{pgfscope}%
\pgfpathrectangle{\pgfqpoint{12.211765in}{1.942105in}}{\pgfqpoint{2.188235in}{0.953684in}} %
\pgfusepath{clip}%
\pgfsetbuttcap%
\pgfsetmiterjoin%
\definecolor{currentfill}{rgb}{0.121569,0.466667,0.705882}%
\pgfsetfillcolor{currentfill}%
\pgfsetlinewidth{0.000000pt}%
\definecolor{currentstroke}{rgb}{0.000000,0.000000,0.000000}%
\pgfsetstrokecolor{currentstroke}%
\pgfsetstrokeopacity{0.000000}%
\pgfsetdash{}{0pt}%
\pgfpathmoveto{\pgfqpoint{-23.809001in}{1.994142in}}%
\pgfpathlineto{\pgfqpoint{13.674550in}{1.994142in}}%
\pgfpathlineto{\pgfqpoint{13.674550in}{2.001092in}}%
\pgfpathlineto{\pgfqpoint{-23.809001in}{2.001092in}}%
\pgfpathclose%
\pgfusepath{fill}%
\end{pgfscope}%
\begin{pgfscope}%
\pgfpathrectangle{\pgfqpoint{12.211765in}{1.942105in}}{\pgfqpoint{2.188235in}{0.953684in}} %
\pgfusepath{clip}%
\pgfsetbuttcap%
\pgfsetmiterjoin%
\definecolor{currentfill}{rgb}{0.121569,0.466667,0.705882}%
\pgfsetfillcolor{currentfill}%
\pgfsetlinewidth{0.000000pt}%
\definecolor{currentstroke}{rgb}{0.000000,0.000000,0.000000}%
\pgfsetstrokecolor{currentstroke}%
\pgfsetstrokeopacity{0.000000}%
\pgfsetdash{}{0pt}%
\pgfpathmoveto{\pgfqpoint{-23.809001in}{2.002829in}}%
\pgfpathlineto{\pgfqpoint{13.746779in}{2.002829in}}%
\pgfpathlineto{\pgfqpoint{13.746779in}{2.009779in}}%
\pgfpathlineto{\pgfqpoint{-23.809001in}{2.009779in}}%
\pgfpathclose%
\pgfusepath{fill}%
\end{pgfscope}%
\begin{pgfscope}%
\pgfpathrectangle{\pgfqpoint{12.211765in}{1.942105in}}{\pgfqpoint{2.188235in}{0.953684in}} %
\pgfusepath{clip}%
\pgfsetbuttcap%
\pgfsetmiterjoin%
\definecolor{currentfill}{rgb}{0.121569,0.466667,0.705882}%
\pgfsetfillcolor{currentfill}%
\pgfsetlinewidth{0.000000pt}%
\definecolor{currentstroke}{rgb}{0.000000,0.000000,0.000000}%
\pgfsetstrokecolor{currentstroke}%
\pgfsetstrokeopacity{0.000000}%
\pgfsetdash{}{0pt}%
\pgfpathmoveto{\pgfqpoint{-23.809001in}{2.011516in}}%
\pgfpathlineto{\pgfqpoint{13.704865in}{2.011516in}}%
\pgfpathlineto{\pgfqpoint{13.704865in}{2.018466in}}%
\pgfpathlineto{\pgfqpoint{-23.809001in}{2.018466in}}%
\pgfpathclose%
\pgfusepath{fill}%
\end{pgfscope}%
\begin{pgfscope}%
\pgfpathrectangle{\pgfqpoint{12.211765in}{1.942105in}}{\pgfqpoint{2.188235in}{0.953684in}} %
\pgfusepath{clip}%
\pgfsetbuttcap%
\pgfsetmiterjoin%
\definecolor{currentfill}{rgb}{0.121569,0.466667,0.705882}%
\pgfsetfillcolor{currentfill}%
\pgfsetlinewidth{0.000000pt}%
\definecolor{currentstroke}{rgb}{0.000000,0.000000,0.000000}%
\pgfsetstrokecolor{currentstroke}%
\pgfsetstrokeopacity{0.000000}%
\pgfsetdash{}{0pt}%
\pgfpathmoveto{\pgfqpoint{-23.809001in}{2.020203in}}%
\pgfpathlineto{\pgfqpoint{13.689241in}{2.020203in}}%
\pgfpathlineto{\pgfqpoint{13.689241in}{2.027153in}}%
\pgfpathlineto{\pgfqpoint{-23.809001in}{2.027153in}}%
\pgfpathclose%
\pgfusepath{fill}%
\end{pgfscope}%
\begin{pgfscope}%
\pgfpathrectangle{\pgfqpoint{12.211765in}{1.942105in}}{\pgfqpoint{2.188235in}{0.953684in}} %
\pgfusepath{clip}%
\pgfsetbuttcap%
\pgfsetmiterjoin%
\definecolor{currentfill}{rgb}{0.121569,0.466667,0.705882}%
\pgfsetfillcolor{currentfill}%
\pgfsetlinewidth{0.000000pt}%
\definecolor{currentstroke}{rgb}{0.000000,0.000000,0.000000}%
\pgfsetstrokecolor{currentstroke}%
\pgfsetstrokeopacity{0.000000}%
\pgfsetdash{}{0pt}%
\pgfpathmoveto{\pgfqpoint{-23.809001in}{2.028891in}}%
\pgfpathlineto{\pgfqpoint{13.707471in}{2.028891in}}%
\pgfpathlineto{\pgfqpoint{13.707471in}{2.035840in}}%
\pgfpathlineto{\pgfqpoint{-23.809001in}{2.035840in}}%
\pgfpathclose%
\pgfusepath{fill}%
\end{pgfscope}%
\begin{pgfscope}%
\pgfpathrectangle{\pgfqpoint{12.211765in}{1.942105in}}{\pgfqpoint{2.188235in}{0.953684in}} %
\pgfusepath{clip}%
\pgfsetbuttcap%
\pgfsetmiterjoin%
\definecolor{currentfill}{rgb}{0.121569,0.466667,0.705882}%
\pgfsetfillcolor{currentfill}%
\pgfsetlinewidth{0.000000pt}%
\definecolor{currentstroke}{rgb}{0.000000,0.000000,0.000000}%
\pgfsetstrokecolor{currentstroke}%
\pgfsetstrokeopacity{0.000000}%
\pgfsetdash{}{0pt}%
\pgfpathmoveto{\pgfqpoint{-23.809001in}{2.037578in}}%
\pgfpathlineto{\pgfqpoint{13.649565in}{2.037578in}}%
\pgfpathlineto{\pgfqpoint{13.649565in}{2.044528in}}%
\pgfpathlineto{\pgfqpoint{-23.809001in}{2.044528in}}%
\pgfpathclose%
\pgfusepath{fill}%
\end{pgfscope}%
\begin{pgfscope}%
\pgfpathrectangle{\pgfqpoint{12.211765in}{1.942105in}}{\pgfqpoint{2.188235in}{0.953684in}} %
\pgfusepath{clip}%
\pgfsetbuttcap%
\pgfsetmiterjoin%
\definecolor{currentfill}{rgb}{0.121569,0.466667,0.705882}%
\pgfsetfillcolor{currentfill}%
\pgfsetlinewidth{0.000000pt}%
\definecolor{currentstroke}{rgb}{0.000000,0.000000,0.000000}%
\pgfsetstrokecolor{currentstroke}%
\pgfsetstrokeopacity{0.000000}%
\pgfsetdash{}{0pt}%
\pgfpathmoveto{\pgfqpoint{-23.809001in}{2.046265in}}%
\pgfpathlineto{\pgfqpoint{13.670979in}{2.046265in}}%
\pgfpathlineto{\pgfqpoint{13.670979in}{2.053215in}}%
\pgfpathlineto{\pgfqpoint{-23.809001in}{2.053215in}}%
\pgfpathclose%
\pgfusepath{fill}%
\end{pgfscope}%
\begin{pgfscope}%
\pgfpathrectangle{\pgfqpoint{12.211765in}{1.942105in}}{\pgfqpoint{2.188235in}{0.953684in}} %
\pgfusepath{clip}%
\pgfsetbuttcap%
\pgfsetmiterjoin%
\definecolor{currentfill}{rgb}{0.121569,0.466667,0.705882}%
\pgfsetfillcolor{currentfill}%
\pgfsetlinewidth{0.000000pt}%
\definecolor{currentstroke}{rgb}{0.000000,0.000000,0.000000}%
\pgfsetstrokecolor{currentstroke}%
\pgfsetstrokeopacity{0.000000}%
\pgfsetdash{}{0pt}%
\pgfpathmoveto{\pgfqpoint{-23.809001in}{2.054952in}}%
\pgfpathlineto{\pgfqpoint{13.541316in}{2.054952in}}%
\pgfpathlineto{\pgfqpoint{13.541316in}{2.061902in}}%
\pgfpathlineto{\pgfqpoint{-23.809001in}{2.061902in}}%
\pgfpathclose%
\pgfusepath{fill}%
\end{pgfscope}%
\begin{pgfscope}%
\pgfpathrectangle{\pgfqpoint{12.211765in}{1.942105in}}{\pgfqpoint{2.188235in}{0.953684in}} %
\pgfusepath{clip}%
\pgfsetbuttcap%
\pgfsetmiterjoin%
\definecolor{currentfill}{rgb}{0.121569,0.466667,0.705882}%
\pgfsetfillcolor{currentfill}%
\pgfsetlinewidth{0.000000pt}%
\definecolor{currentstroke}{rgb}{0.000000,0.000000,0.000000}%
\pgfsetstrokecolor{currentstroke}%
\pgfsetstrokeopacity{0.000000}%
\pgfsetdash{}{0pt}%
\pgfpathmoveto{\pgfqpoint{-23.809001in}{2.063640in}}%
\pgfpathlineto{\pgfqpoint{13.693736in}{2.063640in}}%
\pgfpathlineto{\pgfqpoint{13.693736in}{2.070589in}}%
\pgfpathlineto{\pgfqpoint{-23.809001in}{2.070589in}}%
\pgfpathclose%
\pgfusepath{fill}%
\end{pgfscope}%
\begin{pgfscope}%
\pgfpathrectangle{\pgfqpoint{12.211765in}{1.942105in}}{\pgfqpoint{2.188235in}{0.953684in}} %
\pgfusepath{clip}%
\pgfsetbuttcap%
\pgfsetmiterjoin%
\definecolor{currentfill}{rgb}{0.121569,0.466667,0.705882}%
\pgfsetfillcolor{currentfill}%
\pgfsetlinewidth{0.000000pt}%
\definecolor{currentstroke}{rgb}{0.000000,0.000000,0.000000}%
\pgfsetstrokecolor{currentstroke}%
\pgfsetstrokeopacity{0.000000}%
\pgfsetdash{}{0pt}%
\pgfpathmoveto{\pgfqpoint{-23.809001in}{2.072327in}}%
\pgfpathlineto{\pgfqpoint{13.665031in}{2.072327in}}%
\pgfpathlineto{\pgfqpoint{13.665031in}{2.079277in}}%
\pgfpathlineto{\pgfqpoint{-23.809001in}{2.079277in}}%
\pgfpathclose%
\pgfusepath{fill}%
\end{pgfscope}%
\begin{pgfscope}%
\pgfpathrectangle{\pgfqpoint{12.211765in}{1.942105in}}{\pgfqpoint{2.188235in}{0.953684in}} %
\pgfusepath{clip}%
\pgfsetbuttcap%
\pgfsetmiterjoin%
\definecolor{currentfill}{rgb}{0.121569,0.466667,0.705882}%
\pgfsetfillcolor{currentfill}%
\pgfsetlinewidth{0.000000pt}%
\definecolor{currentstroke}{rgb}{0.000000,0.000000,0.000000}%
\pgfsetstrokecolor{currentstroke}%
\pgfsetstrokeopacity{0.000000}%
\pgfsetdash{}{0pt}%
\pgfpathmoveto{\pgfqpoint{-23.809001in}{2.081014in}}%
\pgfpathlineto{\pgfqpoint{13.649474in}{2.081014in}}%
\pgfpathlineto{\pgfqpoint{13.649474in}{2.087964in}}%
\pgfpathlineto{\pgfqpoint{-23.809001in}{2.087964in}}%
\pgfpathclose%
\pgfusepath{fill}%
\end{pgfscope}%
\begin{pgfscope}%
\pgfpathrectangle{\pgfqpoint{12.211765in}{1.942105in}}{\pgfqpoint{2.188235in}{0.953684in}} %
\pgfusepath{clip}%
\pgfsetbuttcap%
\pgfsetmiterjoin%
\definecolor{currentfill}{rgb}{0.121569,0.466667,0.705882}%
\pgfsetfillcolor{currentfill}%
\pgfsetlinewidth{0.000000pt}%
\definecolor{currentstroke}{rgb}{0.000000,0.000000,0.000000}%
\pgfsetstrokecolor{currentstroke}%
\pgfsetstrokeopacity{0.000000}%
\pgfsetdash{}{0pt}%
\pgfpathmoveto{\pgfqpoint{-23.809001in}{2.089701in}}%
\pgfpathlineto{\pgfqpoint{13.665994in}{2.089701in}}%
\pgfpathlineto{\pgfqpoint{13.665994in}{2.096651in}}%
\pgfpathlineto{\pgfqpoint{-23.809001in}{2.096651in}}%
\pgfpathclose%
\pgfusepath{fill}%
\end{pgfscope}%
\begin{pgfscope}%
\pgfpathrectangle{\pgfqpoint{12.211765in}{1.942105in}}{\pgfqpoint{2.188235in}{0.953684in}} %
\pgfusepath{clip}%
\pgfsetbuttcap%
\pgfsetmiterjoin%
\definecolor{currentfill}{rgb}{0.121569,0.466667,0.705882}%
\pgfsetfillcolor{currentfill}%
\pgfsetlinewidth{0.000000pt}%
\definecolor{currentstroke}{rgb}{0.000000,0.000000,0.000000}%
\pgfsetstrokecolor{currentstroke}%
\pgfsetstrokeopacity{0.000000}%
\pgfsetdash{}{0pt}%
\pgfpathmoveto{\pgfqpoint{-23.809001in}{2.098389in}}%
\pgfpathlineto{\pgfqpoint{13.628340in}{2.098389in}}%
\pgfpathlineto{\pgfqpoint{13.628340in}{2.105338in}}%
\pgfpathlineto{\pgfqpoint{-23.809001in}{2.105338in}}%
\pgfpathclose%
\pgfusepath{fill}%
\end{pgfscope}%
\begin{pgfscope}%
\pgfpathrectangle{\pgfqpoint{12.211765in}{1.942105in}}{\pgfqpoint{2.188235in}{0.953684in}} %
\pgfusepath{clip}%
\pgfsetbuttcap%
\pgfsetmiterjoin%
\definecolor{currentfill}{rgb}{0.121569,0.466667,0.705882}%
\pgfsetfillcolor{currentfill}%
\pgfsetlinewidth{0.000000pt}%
\definecolor{currentstroke}{rgb}{0.000000,0.000000,0.000000}%
\pgfsetstrokecolor{currentstroke}%
\pgfsetstrokeopacity{0.000000}%
\pgfsetdash{}{0pt}%
\pgfpathmoveto{\pgfqpoint{-23.809001in}{2.107076in}}%
\pgfpathlineto{\pgfqpoint{13.601139in}{2.107076in}}%
\pgfpathlineto{\pgfqpoint{13.601139in}{2.114026in}}%
\pgfpathlineto{\pgfqpoint{-23.809001in}{2.114026in}}%
\pgfpathclose%
\pgfusepath{fill}%
\end{pgfscope}%
\begin{pgfscope}%
\pgfpathrectangle{\pgfqpoint{12.211765in}{1.942105in}}{\pgfqpoint{2.188235in}{0.953684in}} %
\pgfusepath{clip}%
\pgfsetbuttcap%
\pgfsetmiterjoin%
\definecolor{currentfill}{rgb}{0.121569,0.466667,0.705882}%
\pgfsetfillcolor{currentfill}%
\pgfsetlinewidth{0.000000pt}%
\definecolor{currentstroke}{rgb}{0.000000,0.000000,0.000000}%
\pgfsetstrokecolor{currentstroke}%
\pgfsetstrokeopacity{0.000000}%
\pgfsetdash{}{0pt}%
\pgfpathmoveto{\pgfqpoint{-23.809001in}{2.115763in}}%
\pgfpathlineto{\pgfqpoint{13.705468in}{2.115763in}}%
\pgfpathlineto{\pgfqpoint{13.705468in}{2.122713in}}%
\pgfpathlineto{\pgfqpoint{-23.809001in}{2.122713in}}%
\pgfpathclose%
\pgfusepath{fill}%
\end{pgfscope}%
\begin{pgfscope}%
\pgfpathrectangle{\pgfqpoint{12.211765in}{1.942105in}}{\pgfqpoint{2.188235in}{0.953684in}} %
\pgfusepath{clip}%
\pgfsetbuttcap%
\pgfsetmiterjoin%
\definecolor{currentfill}{rgb}{0.121569,0.466667,0.705882}%
\pgfsetfillcolor{currentfill}%
\pgfsetlinewidth{0.000000pt}%
\definecolor{currentstroke}{rgb}{0.000000,0.000000,0.000000}%
\pgfsetstrokecolor{currentstroke}%
\pgfsetstrokeopacity{0.000000}%
\pgfsetdash{}{0pt}%
\pgfpathmoveto{\pgfqpoint{-23.809001in}{2.124450in}}%
\pgfpathlineto{\pgfqpoint{13.650006in}{2.124450in}}%
\pgfpathlineto{\pgfqpoint{13.650006in}{2.131400in}}%
\pgfpathlineto{\pgfqpoint{-23.809001in}{2.131400in}}%
\pgfpathclose%
\pgfusepath{fill}%
\end{pgfscope}%
\begin{pgfscope}%
\pgfpathrectangle{\pgfqpoint{12.211765in}{1.942105in}}{\pgfqpoint{2.188235in}{0.953684in}} %
\pgfusepath{clip}%
\pgfsetbuttcap%
\pgfsetmiterjoin%
\definecolor{currentfill}{rgb}{0.121569,0.466667,0.705882}%
\pgfsetfillcolor{currentfill}%
\pgfsetlinewidth{0.000000pt}%
\definecolor{currentstroke}{rgb}{0.000000,0.000000,0.000000}%
\pgfsetstrokecolor{currentstroke}%
\pgfsetstrokeopacity{0.000000}%
\pgfsetdash{}{0pt}%
\pgfpathmoveto{\pgfqpoint{-23.809001in}{2.133137in}}%
\pgfpathlineto{\pgfqpoint{13.603672in}{2.133137in}}%
\pgfpathlineto{\pgfqpoint{13.603672in}{2.140087in}}%
\pgfpathlineto{\pgfqpoint{-23.809001in}{2.140087in}}%
\pgfpathclose%
\pgfusepath{fill}%
\end{pgfscope}%
\begin{pgfscope}%
\pgfpathrectangle{\pgfqpoint{12.211765in}{1.942105in}}{\pgfqpoint{2.188235in}{0.953684in}} %
\pgfusepath{clip}%
\pgfsetbuttcap%
\pgfsetmiterjoin%
\definecolor{currentfill}{rgb}{0.121569,0.466667,0.705882}%
\pgfsetfillcolor{currentfill}%
\pgfsetlinewidth{0.000000pt}%
\definecolor{currentstroke}{rgb}{0.000000,0.000000,0.000000}%
\pgfsetstrokecolor{currentstroke}%
\pgfsetstrokeopacity{0.000000}%
\pgfsetdash{}{0pt}%
\pgfpathmoveto{\pgfqpoint{-23.809001in}{2.141825in}}%
\pgfpathlineto{\pgfqpoint{13.607948in}{2.141825in}}%
\pgfpathlineto{\pgfqpoint{13.607948in}{2.148774in}}%
\pgfpathlineto{\pgfqpoint{-23.809001in}{2.148774in}}%
\pgfpathclose%
\pgfusepath{fill}%
\end{pgfscope}%
\begin{pgfscope}%
\pgfpathrectangle{\pgfqpoint{12.211765in}{1.942105in}}{\pgfqpoint{2.188235in}{0.953684in}} %
\pgfusepath{clip}%
\pgfsetbuttcap%
\pgfsetmiterjoin%
\definecolor{currentfill}{rgb}{0.121569,0.466667,0.705882}%
\pgfsetfillcolor{currentfill}%
\pgfsetlinewidth{0.000000pt}%
\definecolor{currentstroke}{rgb}{0.000000,0.000000,0.000000}%
\pgfsetstrokecolor{currentstroke}%
\pgfsetstrokeopacity{0.000000}%
\pgfsetdash{}{0pt}%
\pgfpathmoveto{\pgfqpoint{-23.809001in}{2.150512in}}%
\pgfpathlineto{\pgfqpoint{13.658820in}{2.150512in}}%
\pgfpathlineto{\pgfqpoint{13.658820in}{2.157462in}}%
\pgfpathlineto{\pgfqpoint{-23.809001in}{2.157462in}}%
\pgfpathclose%
\pgfusepath{fill}%
\end{pgfscope}%
\begin{pgfscope}%
\pgfpathrectangle{\pgfqpoint{12.211765in}{1.942105in}}{\pgfqpoint{2.188235in}{0.953684in}} %
\pgfusepath{clip}%
\pgfsetbuttcap%
\pgfsetmiterjoin%
\definecolor{currentfill}{rgb}{0.121569,0.466667,0.705882}%
\pgfsetfillcolor{currentfill}%
\pgfsetlinewidth{0.000000pt}%
\definecolor{currentstroke}{rgb}{0.000000,0.000000,0.000000}%
\pgfsetstrokecolor{currentstroke}%
\pgfsetstrokeopacity{0.000000}%
\pgfsetdash{}{0pt}%
\pgfpathmoveto{\pgfqpoint{-23.809001in}{2.159199in}}%
\pgfpathlineto{\pgfqpoint{13.690296in}{2.159199in}}%
\pgfpathlineto{\pgfqpoint{13.690296in}{2.166149in}}%
\pgfpathlineto{\pgfqpoint{-23.809001in}{2.166149in}}%
\pgfpathclose%
\pgfusepath{fill}%
\end{pgfscope}%
\begin{pgfscope}%
\pgfpathrectangle{\pgfqpoint{12.211765in}{1.942105in}}{\pgfqpoint{2.188235in}{0.953684in}} %
\pgfusepath{clip}%
\pgfsetbuttcap%
\pgfsetmiterjoin%
\definecolor{currentfill}{rgb}{0.121569,0.466667,0.705882}%
\pgfsetfillcolor{currentfill}%
\pgfsetlinewidth{0.000000pt}%
\definecolor{currentstroke}{rgb}{0.000000,0.000000,0.000000}%
\pgfsetstrokecolor{currentstroke}%
\pgfsetstrokeopacity{0.000000}%
\pgfsetdash{}{0pt}%
\pgfpathmoveto{\pgfqpoint{-23.809001in}{2.167886in}}%
\pgfpathlineto{\pgfqpoint{13.680439in}{2.167886in}}%
\pgfpathlineto{\pgfqpoint{13.680439in}{2.174836in}}%
\pgfpathlineto{\pgfqpoint{-23.809001in}{2.174836in}}%
\pgfpathclose%
\pgfusepath{fill}%
\end{pgfscope}%
\begin{pgfscope}%
\pgfpathrectangle{\pgfqpoint{12.211765in}{1.942105in}}{\pgfqpoint{2.188235in}{0.953684in}} %
\pgfusepath{clip}%
\pgfsetbuttcap%
\pgfsetmiterjoin%
\definecolor{currentfill}{rgb}{0.121569,0.466667,0.705882}%
\pgfsetfillcolor{currentfill}%
\pgfsetlinewidth{0.000000pt}%
\definecolor{currentstroke}{rgb}{0.000000,0.000000,0.000000}%
\pgfsetstrokecolor{currentstroke}%
\pgfsetstrokeopacity{0.000000}%
\pgfsetdash{}{0pt}%
\pgfpathmoveto{\pgfqpoint{-23.809001in}{2.176574in}}%
\pgfpathlineto{\pgfqpoint{13.470908in}{2.176574in}}%
\pgfpathlineto{\pgfqpoint{13.470908in}{2.183523in}}%
\pgfpathlineto{\pgfqpoint{-23.809001in}{2.183523in}}%
\pgfpathclose%
\pgfusepath{fill}%
\end{pgfscope}%
\begin{pgfscope}%
\pgfpathrectangle{\pgfqpoint{12.211765in}{1.942105in}}{\pgfqpoint{2.188235in}{0.953684in}} %
\pgfusepath{clip}%
\pgfsetbuttcap%
\pgfsetmiterjoin%
\definecolor{currentfill}{rgb}{0.121569,0.466667,0.705882}%
\pgfsetfillcolor{currentfill}%
\pgfsetlinewidth{0.000000pt}%
\definecolor{currentstroke}{rgb}{0.000000,0.000000,0.000000}%
\pgfsetstrokecolor{currentstroke}%
\pgfsetstrokeopacity{0.000000}%
\pgfsetdash{}{0pt}%
\pgfpathmoveto{\pgfqpoint{-23.809001in}{2.185261in}}%
\pgfpathlineto{\pgfqpoint{13.699652in}{2.185261in}}%
\pgfpathlineto{\pgfqpoint{13.699652in}{2.192211in}}%
\pgfpathlineto{\pgfqpoint{-23.809001in}{2.192211in}}%
\pgfpathclose%
\pgfusepath{fill}%
\end{pgfscope}%
\begin{pgfscope}%
\pgfpathrectangle{\pgfqpoint{12.211765in}{1.942105in}}{\pgfqpoint{2.188235in}{0.953684in}} %
\pgfusepath{clip}%
\pgfsetbuttcap%
\pgfsetmiterjoin%
\definecolor{currentfill}{rgb}{0.121569,0.466667,0.705882}%
\pgfsetfillcolor{currentfill}%
\pgfsetlinewidth{0.000000pt}%
\definecolor{currentstroke}{rgb}{0.000000,0.000000,0.000000}%
\pgfsetstrokecolor{currentstroke}%
\pgfsetstrokeopacity{0.000000}%
\pgfsetdash{}{0pt}%
\pgfpathmoveto{\pgfqpoint{-23.809001in}{2.193948in}}%
\pgfpathlineto{\pgfqpoint{13.662880in}{2.193948in}}%
\pgfpathlineto{\pgfqpoint{13.662880in}{2.200898in}}%
\pgfpathlineto{\pgfqpoint{-23.809001in}{2.200898in}}%
\pgfpathclose%
\pgfusepath{fill}%
\end{pgfscope}%
\begin{pgfscope}%
\pgfpathrectangle{\pgfqpoint{12.211765in}{1.942105in}}{\pgfqpoint{2.188235in}{0.953684in}} %
\pgfusepath{clip}%
\pgfsetbuttcap%
\pgfsetmiterjoin%
\definecolor{currentfill}{rgb}{0.121569,0.466667,0.705882}%
\pgfsetfillcolor{currentfill}%
\pgfsetlinewidth{0.000000pt}%
\definecolor{currentstroke}{rgb}{0.000000,0.000000,0.000000}%
\pgfsetstrokecolor{currentstroke}%
\pgfsetstrokeopacity{0.000000}%
\pgfsetdash{}{0pt}%
\pgfpathmoveto{\pgfqpoint{-23.809001in}{2.202635in}}%
\pgfpathlineto{\pgfqpoint{13.600147in}{2.202635in}}%
\pgfpathlineto{\pgfqpoint{13.600147in}{2.209585in}}%
\pgfpathlineto{\pgfqpoint{-23.809001in}{2.209585in}}%
\pgfpathclose%
\pgfusepath{fill}%
\end{pgfscope}%
\begin{pgfscope}%
\pgfpathrectangle{\pgfqpoint{12.211765in}{1.942105in}}{\pgfqpoint{2.188235in}{0.953684in}} %
\pgfusepath{clip}%
\pgfsetbuttcap%
\pgfsetmiterjoin%
\definecolor{currentfill}{rgb}{0.121569,0.466667,0.705882}%
\pgfsetfillcolor{currentfill}%
\pgfsetlinewidth{0.000000pt}%
\definecolor{currentstroke}{rgb}{0.000000,0.000000,0.000000}%
\pgfsetstrokecolor{currentstroke}%
\pgfsetstrokeopacity{0.000000}%
\pgfsetdash{}{0pt}%
\pgfpathmoveto{\pgfqpoint{-23.809001in}{2.211323in}}%
\pgfpathlineto{\pgfqpoint{13.638640in}{2.211323in}}%
\pgfpathlineto{\pgfqpoint{13.638640in}{2.218272in}}%
\pgfpathlineto{\pgfqpoint{-23.809001in}{2.218272in}}%
\pgfpathclose%
\pgfusepath{fill}%
\end{pgfscope}%
\begin{pgfscope}%
\pgfpathrectangle{\pgfqpoint{12.211765in}{1.942105in}}{\pgfqpoint{2.188235in}{0.953684in}} %
\pgfusepath{clip}%
\pgfsetbuttcap%
\pgfsetmiterjoin%
\definecolor{currentfill}{rgb}{0.121569,0.466667,0.705882}%
\pgfsetfillcolor{currentfill}%
\pgfsetlinewidth{0.000000pt}%
\definecolor{currentstroke}{rgb}{0.000000,0.000000,0.000000}%
\pgfsetstrokecolor{currentstroke}%
\pgfsetstrokeopacity{0.000000}%
\pgfsetdash{}{0pt}%
\pgfpathmoveto{\pgfqpoint{-23.809001in}{2.220010in}}%
\pgfpathlineto{\pgfqpoint{13.651466in}{2.220010in}}%
\pgfpathlineto{\pgfqpoint{13.651466in}{2.226960in}}%
\pgfpathlineto{\pgfqpoint{-23.809001in}{2.226960in}}%
\pgfpathclose%
\pgfusepath{fill}%
\end{pgfscope}%
\begin{pgfscope}%
\pgfpathrectangle{\pgfqpoint{12.211765in}{1.942105in}}{\pgfqpoint{2.188235in}{0.953684in}} %
\pgfusepath{clip}%
\pgfsetbuttcap%
\pgfsetmiterjoin%
\definecolor{currentfill}{rgb}{0.121569,0.466667,0.705882}%
\pgfsetfillcolor{currentfill}%
\pgfsetlinewidth{0.000000pt}%
\definecolor{currentstroke}{rgb}{0.000000,0.000000,0.000000}%
\pgfsetstrokecolor{currentstroke}%
\pgfsetstrokeopacity{0.000000}%
\pgfsetdash{}{0pt}%
\pgfpathmoveto{\pgfqpoint{-23.809001in}{2.228697in}}%
\pgfpathlineto{\pgfqpoint{13.576541in}{2.228697in}}%
\pgfpathlineto{\pgfqpoint{13.576541in}{2.235647in}}%
\pgfpathlineto{\pgfqpoint{-23.809001in}{2.235647in}}%
\pgfpathclose%
\pgfusepath{fill}%
\end{pgfscope}%
\begin{pgfscope}%
\pgfpathrectangle{\pgfqpoint{12.211765in}{1.942105in}}{\pgfqpoint{2.188235in}{0.953684in}} %
\pgfusepath{clip}%
\pgfsetbuttcap%
\pgfsetmiterjoin%
\definecolor{currentfill}{rgb}{0.121569,0.466667,0.705882}%
\pgfsetfillcolor{currentfill}%
\pgfsetlinewidth{0.000000pt}%
\definecolor{currentstroke}{rgb}{0.000000,0.000000,0.000000}%
\pgfsetstrokecolor{currentstroke}%
\pgfsetstrokeopacity{0.000000}%
\pgfsetdash{}{0pt}%
\pgfpathmoveto{\pgfqpoint{-23.809001in}{2.237384in}}%
\pgfpathlineto{\pgfqpoint{13.666591in}{2.237384in}}%
\pgfpathlineto{\pgfqpoint{13.666591in}{2.244334in}}%
\pgfpathlineto{\pgfqpoint{-23.809001in}{2.244334in}}%
\pgfpathclose%
\pgfusepath{fill}%
\end{pgfscope}%
\begin{pgfscope}%
\pgfpathrectangle{\pgfqpoint{12.211765in}{1.942105in}}{\pgfqpoint{2.188235in}{0.953684in}} %
\pgfusepath{clip}%
\pgfsetbuttcap%
\pgfsetmiterjoin%
\definecolor{currentfill}{rgb}{0.121569,0.466667,0.705882}%
\pgfsetfillcolor{currentfill}%
\pgfsetlinewidth{0.000000pt}%
\definecolor{currentstroke}{rgb}{0.000000,0.000000,0.000000}%
\pgfsetstrokecolor{currentstroke}%
\pgfsetstrokeopacity{0.000000}%
\pgfsetdash{}{0pt}%
\pgfpathmoveto{\pgfqpoint{-23.809001in}{2.246071in}}%
\pgfpathlineto{\pgfqpoint{13.548846in}{2.246071in}}%
\pgfpathlineto{\pgfqpoint{13.548846in}{2.253021in}}%
\pgfpathlineto{\pgfqpoint{-23.809001in}{2.253021in}}%
\pgfpathclose%
\pgfusepath{fill}%
\end{pgfscope}%
\begin{pgfscope}%
\pgfpathrectangle{\pgfqpoint{12.211765in}{1.942105in}}{\pgfqpoint{2.188235in}{0.953684in}} %
\pgfusepath{clip}%
\pgfsetbuttcap%
\pgfsetmiterjoin%
\definecolor{currentfill}{rgb}{0.121569,0.466667,0.705882}%
\pgfsetfillcolor{currentfill}%
\pgfsetlinewidth{0.000000pt}%
\definecolor{currentstroke}{rgb}{0.000000,0.000000,0.000000}%
\pgfsetstrokecolor{currentstroke}%
\pgfsetstrokeopacity{0.000000}%
\pgfsetdash{}{0pt}%
\pgfpathmoveto{\pgfqpoint{-23.809001in}{2.254759in}}%
\pgfpathlineto{\pgfqpoint{13.571207in}{2.254759in}}%
\pgfpathlineto{\pgfqpoint{13.571207in}{2.261708in}}%
\pgfpathlineto{\pgfqpoint{-23.809001in}{2.261708in}}%
\pgfpathclose%
\pgfusepath{fill}%
\end{pgfscope}%
\begin{pgfscope}%
\pgfpathrectangle{\pgfqpoint{12.211765in}{1.942105in}}{\pgfqpoint{2.188235in}{0.953684in}} %
\pgfusepath{clip}%
\pgfsetbuttcap%
\pgfsetmiterjoin%
\definecolor{currentfill}{rgb}{0.121569,0.466667,0.705882}%
\pgfsetfillcolor{currentfill}%
\pgfsetlinewidth{0.000000pt}%
\definecolor{currentstroke}{rgb}{0.000000,0.000000,0.000000}%
\pgfsetstrokecolor{currentstroke}%
\pgfsetstrokeopacity{0.000000}%
\pgfsetdash{}{0pt}%
\pgfpathmoveto{\pgfqpoint{-23.809001in}{2.263446in}}%
\pgfpathlineto{\pgfqpoint{13.675066in}{2.263446in}}%
\pgfpathlineto{\pgfqpoint{13.675066in}{2.270396in}}%
\pgfpathlineto{\pgfqpoint{-23.809001in}{2.270396in}}%
\pgfpathclose%
\pgfusepath{fill}%
\end{pgfscope}%
\begin{pgfscope}%
\pgfpathrectangle{\pgfqpoint{12.211765in}{1.942105in}}{\pgfqpoint{2.188235in}{0.953684in}} %
\pgfusepath{clip}%
\pgfsetbuttcap%
\pgfsetmiterjoin%
\definecolor{currentfill}{rgb}{0.121569,0.466667,0.705882}%
\pgfsetfillcolor{currentfill}%
\pgfsetlinewidth{0.000000pt}%
\definecolor{currentstroke}{rgb}{0.000000,0.000000,0.000000}%
\pgfsetstrokecolor{currentstroke}%
\pgfsetstrokeopacity{0.000000}%
\pgfsetdash{}{0pt}%
\pgfpathmoveto{\pgfqpoint{-23.809001in}{2.272133in}}%
\pgfpathlineto{\pgfqpoint{13.558011in}{2.272133in}}%
\pgfpathlineto{\pgfqpoint{13.558011in}{2.279083in}}%
\pgfpathlineto{\pgfqpoint{-23.809001in}{2.279083in}}%
\pgfpathclose%
\pgfusepath{fill}%
\end{pgfscope}%
\begin{pgfscope}%
\pgfpathrectangle{\pgfqpoint{12.211765in}{1.942105in}}{\pgfqpoint{2.188235in}{0.953684in}} %
\pgfusepath{clip}%
\pgfsetbuttcap%
\pgfsetmiterjoin%
\definecolor{currentfill}{rgb}{0.121569,0.466667,0.705882}%
\pgfsetfillcolor{currentfill}%
\pgfsetlinewidth{0.000000pt}%
\definecolor{currentstroke}{rgb}{0.000000,0.000000,0.000000}%
\pgfsetstrokecolor{currentstroke}%
\pgfsetstrokeopacity{0.000000}%
\pgfsetdash{}{0pt}%
\pgfpathmoveto{\pgfqpoint{-23.809001in}{2.280820in}}%
\pgfpathlineto{\pgfqpoint{13.623099in}{2.280820in}}%
\pgfpathlineto{\pgfqpoint{13.623099in}{2.287770in}}%
\pgfpathlineto{\pgfqpoint{-23.809001in}{2.287770in}}%
\pgfpathclose%
\pgfusepath{fill}%
\end{pgfscope}%
\begin{pgfscope}%
\pgfpathrectangle{\pgfqpoint{12.211765in}{1.942105in}}{\pgfqpoint{2.188235in}{0.953684in}} %
\pgfusepath{clip}%
\pgfsetbuttcap%
\pgfsetmiterjoin%
\definecolor{currentfill}{rgb}{0.121569,0.466667,0.705882}%
\pgfsetfillcolor{currentfill}%
\pgfsetlinewidth{0.000000pt}%
\definecolor{currentstroke}{rgb}{0.000000,0.000000,0.000000}%
\pgfsetstrokecolor{currentstroke}%
\pgfsetstrokeopacity{0.000000}%
\pgfsetdash{}{0pt}%
\pgfpathmoveto{\pgfqpoint{-23.809001in}{2.289508in}}%
\pgfpathlineto{\pgfqpoint{13.657714in}{2.289508in}}%
\pgfpathlineto{\pgfqpoint{13.657714in}{2.296457in}}%
\pgfpathlineto{\pgfqpoint{-23.809001in}{2.296457in}}%
\pgfpathclose%
\pgfusepath{fill}%
\end{pgfscope}%
\begin{pgfscope}%
\pgfpathrectangle{\pgfqpoint{12.211765in}{1.942105in}}{\pgfqpoint{2.188235in}{0.953684in}} %
\pgfusepath{clip}%
\pgfsetbuttcap%
\pgfsetmiterjoin%
\definecolor{currentfill}{rgb}{0.121569,0.466667,0.705882}%
\pgfsetfillcolor{currentfill}%
\pgfsetlinewidth{0.000000pt}%
\definecolor{currentstroke}{rgb}{0.000000,0.000000,0.000000}%
\pgfsetstrokecolor{currentstroke}%
\pgfsetstrokeopacity{0.000000}%
\pgfsetdash{}{0pt}%
\pgfpathmoveto{\pgfqpoint{-23.809001in}{2.298195in}}%
\pgfpathlineto{\pgfqpoint{13.613639in}{2.298195in}}%
\pgfpathlineto{\pgfqpoint{13.613639in}{2.305145in}}%
\pgfpathlineto{\pgfqpoint{-23.809001in}{2.305145in}}%
\pgfpathclose%
\pgfusepath{fill}%
\end{pgfscope}%
\begin{pgfscope}%
\pgfpathrectangle{\pgfqpoint{12.211765in}{1.942105in}}{\pgfqpoint{2.188235in}{0.953684in}} %
\pgfusepath{clip}%
\pgfsetbuttcap%
\pgfsetmiterjoin%
\definecolor{currentfill}{rgb}{0.121569,0.466667,0.705882}%
\pgfsetfillcolor{currentfill}%
\pgfsetlinewidth{0.000000pt}%
\definecolor{currentstroke}{rgb}{0.000000,0.000000,0.000000}%
\pgfsetstrokecolor{currentstroke}%
\pgfsetstrokeopacity{0.000000}%
\pgfsetdash{}{0pt}%
\pgfpathmoveto{\pgfqpoint{-23.809001in}{2.306882in}}%
\pgfpathlineto{\pgfqpoint{13.420700in}{2.306882in}}%
\pgfpathlineto{\pgfqpoint{13.420700in}{2.313832in}}%
\pgfpathlineto{\pgfqpoint{-23.809001in}{2.313832in}}%
\pgfpathclose%
\pgfusepath{fill}%
\end{pgfscope}%
\begin{pgfscope}%
\pgfpathrectangle{\pgfqpoint{12.211765in}{1.942105in}}{\pgfqpoint{2.188235in}{0.953684in}} %
\pgfusepath{clip}%
\pgfsetbuttcap%
\pgfsetmiterjoin%
\definecolor{currentfill}{rgb}{0.121569,0.466667,0.705882}%
\pgfsetfillcolor{currentfill}%
\pgfsetlinewidth{0.000000pt}%
\definecolor{currentstroke}{rgb}{0.000000,0.000000,0.000000}%
\pgfsetstrokecolor{currentstroke}%
\pgfsetstrokeopacity{0.000000}%
\pgfsetdash{}{0pt}%
\pgfpathmoveto{\pgfqpoint{-23.809001in}{2.315569in}}%
\pgfpathlineto{\pgfqpoint{13.639356in}{2.315569in}}%
\pgfpathlineto{\pgfqpoint{13.639356in}{2.322519in}}%
\pgfpathlineto{\pgfqpoint{-23.809001in}{2.322519in}}%
\pgfpathclose%
\pgfusepath{fill}%
\end{pgfscope}%
\begin{pgfscope}%
\pgfpathrectangle{\pgfqpoint{12.211765in}{1.942105in}}{\pgfqpoint{2.188235in}{0.953684in}} %
\pgfusepath{clip}%
\pgfsetbuttcap%
\pgfsetmiterjoin%
\definecolor{currentfill}{rgb}{0.121569,0.466667,0.705882}%
\pgfsetfillcolor{currentfill}%
\pgfsetlinewidth{0.000000pt}%
\definecolor{currentstroke}{rgb}{0.000000,0.000000,0.000000}%
\pgfsetstrokecolor{currentstroke}%
\pgfsetstrokeopacity{0.000000}%
\pgfsetdash{}{0pt}%
\pgfpathmoveto{\pgfqpoint{-23.809001in}{2.324257in}}%
\pgfpathlineto{\pgfqpoint{13.672921in}{2.324257in}}%
\pgfpathlineto{\pgfqpoint{13.672921in}{2.331206in}}%
\pgfpathlineto{\pgfqpoint{-23.809001in}{2.331206in}}%
\pgfpathclose%
\pgfusepath{fill}%
\end{pgfscope}%
\begin{pgfscope}%
\pgfpathrectangle{\pgfqpoint{12.211765in}{1.942105in}}{\pgfqpoint{2.188235in}{0.953684in}} %
\pgfusepath{clip}%
\pgfsetbuttcap%
\pgfsetmiterjoin%
\definecolor{currentfill}{rgb}{0.121569,0.466667,0.705882}%
\pgfsetfillcolor{currentfill}%
\pgfsetlinewidth{0.000000pt}%
\definecolor{currentstroke}{rgb}{0.000000,0.000000,0.000000}%
\pgfsetstrokecolor{currentstroke}%
\pgfsetstrokeopacity{0.000000}%
\pgfsetdash{}{0pt}%
\pgfpathmoveto{\pgfqpoint{-23.809001in}{2.332944in}}%
\pgfpathlineto{\pgfqpoint{13.651863in}{2.332944in}}%
\pgfpathlineto{\pgfqpoint{13.651863in}{2.339894in}}%
\pgfpathlineto{\pgfqpoint{-23.809001in}{2.339894in}}%
\pgfpathclose%
\pgfusepath{fill}%
\end{pgfscope}%
\begin{pgfscope}%
\pgfpathrectangle{\pgfqpoint{12.211765in}{1.942105in}}{\pgfqpoint{2.188235in}{0.953684in}} %
\pgfusepath{clip}%
\pgfsetbuttcap%
\pgfsetmiterjoin%
\definecolor{currentfill}{rgb}{0.121569,0.466667,0.705882}%
\pgfsetfillcolor{currentfill}%
\pgfsetlinewidth{0.000000pt}%
\definecolor{currentstroke}{rgb}{0.000000,0.000000,0.000000}%
\pgfsetstrokecolor{currentstroke}%
\pgfsetstrokeopacity{0.000000}%
\pgfsetdash{}{0pt}%
\pgfpathmoveto{\pgfqpoint{-23.809001in}{2.341631in}}%
\pgfpathlineto{\pgfqpoint{13.663131in}{2.341631in}}%
\pgfpathlineto{\pgfqpoint{13.663131in}{2.348581in}}%
\pgfpathlineto{\pgfqpoint{-23.809001in}{2.348581in}}%
\pgfpathclose%
\pgfusepath{fill}%
\end{pgfscope}%
\begin{pgfscope}%
\pgfpathrectangle{\pgfqpoint{12.211765in}{1.942105in}}{\pgfqpoint{2.188235in}{0.953684in}} %
\pgfusepath{clip}%
\pgfsetbuttcap%
\pgfsetmiterjoin%
\definecolor{currentfill}{rgb}{0.121569,0.466667,0.705882}%
\pgfsetfillcolor{currentfill}%
\pgfsetlinewidth{0.000000pt}%
\definecolor{currentstroke}{rgb}{0.000000,0.000000,0.000000}%
\pgfsetstrokecolor{currentstroke}%
\pgfsetstrokeopacity{0.000000}%
\pgfsetdash{}{0pt}%
\pgfpathmoveto{\pgfqpoint{-23.809001in}{2.350318in}}%
\pgfpathlineto{\pgfqpoint{13.444749in}{2.350318in}}%
\pgfpathlineto{\pgfqpoint{13.444749in}{2.357268in}}%
\pgfpathlineto{\pgfqpoint{-23.809001in}{2.357268in}}%
\pgfpathclose%
\pgfusepath{fill}%
\end{pgfscope}%
\begin{pgfscope}%
\pgfpathrectangle{\pgfqpoint{12.211765in}{1.942105in}}{\pgfqpoint{2.188235in}{0.953684in}} %
\pgfusepath{clip}%
\pgfsetbuttcap%
\pgfsetmiterjoin%
\definecolor{currentfill}{rgb}{0.121569,0.466667,0.705882}%
\pgfsetfillcolor{currentfill}%
\pgfsetlinewidth{0.000000pt}%
\definecolor{currentstroke}{rgb}{0.000000,0.000000,0.000000}%
\pgfsetstrokecolor{currentstroke}%
\pgfsetstrokeopacity{0.000000}%
\pgfsetdash{}{0pt}%
\pgfpathmoveto{\pgfqpoint{-23.809001in}{2.359005in}}%
\pgfpathlineto{\pgfqpoint{13.675787in}{2.359005in}}%
\pgfpathlineto{\pgfqpoint{13.675787in}{2.365955in}}%
\pgfpathlineto{\pgfqpoint{-23.809001in}{2.365955in}}%
\pgfpathclose%
\pgfusepath{fill}%
\end{pgfscope}%
\begin{pgfscope}%
\pgfpathrectangle{\pgfqpoint{12.211765in}{1.942105in}}{\pgfqpoint{2.188235in}{0.953684in}} %
\pgfusepath{clip}%
\pgfsetbuttcap%
\pgfsetmiterjoin%
\definecolor{currentfill}{rgb}{0.121569,0.466667,0.705882}%
\pgfsetfillcolor{currentfill}%
\pgfsetlinewidth{0.000000pt}%
\definecolor{currentstroke}{rgb}{0.000000,0.000000,0.000000}%
\pgfsetstrokecolor{currentstroke}%
\pgfsetstrokeopacity{0.000000}%
\pgfsetdash{}{0pt}%
\pgfpathmoveto{\pgfqpoint{-23.809001in}{2.367693in}}%
\pgfpathlineto{\pgfqpoint{13.698408in}{2.367693in}}%
\pgfpathlineto{\pgfqpoint{13.698408in}{2.374642in}}%
\pgfpathlineto{\pgfqpoint{-23.809001in}{2.374642in}}%
\pgfpathclose%
\pgfusepath{fill}%
\end{pgfscope}%
\begin{pgfscope}%
\pgfpathrectangle{\pgfqpoint{12.211765in}{1.942105in}}{\pgfqpoint{2.188235in}{0.953684in}} %
\pgfusepath{clip}%
\pgfsetbuttcap%
\pgfsetmiterjoin%
\definecolor{currentfill}{rgb}{0.121569,0.466667,0.705882}%
\pgfsetfillcolor{currentfill}%
\pgfsetlinewidth{0.000000pt}%
\definecolor{currentstroke}{rgb}{0.000000,0.000000,0.000000}%
\pgfsetstrokecolor{currentstroke}%
\pgfsetstrokeopacity{0.000000}%
\pgfsetdash{}{0pt}%
\pgfpathmoveto{\pgfqpoint{-23.809001in}{2.376380in}}%
\pgfpathlineto{\pgfqpoint{13.646674in}{2.376380in}}%
\pgfpathlineto{\pgfqpoint{13.646674in}{2.383330in}}%
\pgfpathlineto{\pgfqpoint{-23.809001in}{2.383330in}}%
\pgfpathclose%
\pgfusepath{fill}%
\end{pgfscope}%
\begin{pgfscope}%
\pgfpathrectangle{\pgfqpoint{12.211765in}{1.942105in}}{\pgfqpoint{2.188235in}{0.953684in}} %
\pgfusepath{clip}%
\pgfsetbuttcap%
\pgfsetmiterjoin%
\definecolor{currentfill}{rgb}{0.121569,0.466667,0.705882}%
\pgfsetfillcolor{currentfill}%
\pgfsetlinewidth{0.000000pt}%
\definecolor{currentstroke}{rgb}{0.000000,0.000000,0.000000}%
\pgfsetstrokecolor{currentstroke}%
\pgfsetstrokeopacity{0.000000}%
\pgfsetdash{}{0pt}%
\pgfpathmoveto{\pgfqpoint{-23.809001in}{2.385067in}}%
\pgfpathlineto{\pgfqpoint{13.553450in}{2.385067in}}%
\pgfpathlineto{\pgfqpoint{13.553450in}{2.392017in}}%
\pgfpathlineto{\pgfqpoint{-23.809001in}{2.392017in}}%
\pgfpathclose%
\pgfusepath{fill}%
\end{pgfscope}%
\begin{pgfscope}%
\pgfpathrectangle{\pgfqpoint{12.211765in}{1.942105in}}{\pgfqpoint{2.188235in}{0.953684in}} %
\pgfusepath{clip}%
\pgfsetbuttcap%
\pgfsetmiterjoin%
\definecolor{currentfill}{rgb}{0.121569,0.466667,0.705882}%
\pgfsetfillcolor{currentfill}%
\pgfsetlinewidth{0.000000pt}%
\definecolor{currentstroke}{rgb}{0.000000,0.000000,0.000000}%
\pgfsetstrokecolor{currentstroke}%
\pgfsetstrokeopacity{0.000000}%
\pgfsetdash{}{0pt}%
\pgfpathmoveto{\pgfqpoint{-23.809001in}{2.393754in}}%
\pgfpathlineto{\pgfqpoint{13.604849in}{2.393754in}}%
\pgfpathlineto{\pgfqpoint{13.604849in}{2.400704in}}%
\pgfpathlineto{\pgfqpoint{-23.809001in}{2.400704in}}%
\pgfpathclose%
\pgfusepath{fill}%
\end{pgfscope}%
\begin{pgfscope}%
\pgfpathrectangle{\pgfqpoint{12.211765in}{1.942105in}}{\pgfqpoint{2.188235in}{0.953684in}} %
\pgfusepath{clip}%
\pgfsetbuttcap%
\pgfsetmiterjoin%
\definecolor{currentfill}{rgb}{0.121569,0.466667,0.705882}%
\pgfsetfillcolor{currentfill}%
\pgfsetlinewidth{0.000000pt}%
\definecolor{currentstroke}{rgb}{0.000000,0.000000,0.000000}%
\pgfsetstrokecolor{currentstroke}%
\pgfsetstrokeopacity{0.000000}%
\pgfsetdash{}{0pt}%
\pgfpathmoveto{\pgfqpoint{-23.809001in}{2.402442in}}%
\pgfpathlineto{\pgfqpoint{13.602445in}{2.402442in}}%
\pgfpathlineto{\pgfqpoint{13.602445in}{2.409391in}}%
\pgfpathlineto{\pgfqpoint{-23.809001in}{2.409391in}}%
\pgfpathclose%
\pgfusepath{fill}%
\end{pgfscope}%
\begin{pgfscope}%
\pgfpathrectangle{\pgfqpoint{12.211765in}{1.942105in}}{\pgfqpoint{2.188235in}{0.953684in}} %
\pgfusepath{clip}%
\pgfsetbuttcap%
\pgfsetmiterjoin%
\definecolor{currentfill}{rgb}{0.121569,0.466667,0.705882}%
\pgfsetfillcolor{currentfill}%
\pgfsetlinewidth{0.000000pt}%
\definecolor{currentstroke}{rgb}{0.000000,0.000000,0.000000}%
\pgfsetstrokecolor{currentstroke}%
\pgfsetstrokeopacity{0.000000}%
\pgfsetdash{}{0pt}%
\pgfpathmoveto{\pgfqpoint{-23.809001in}{2.411129in}}%
\pgfpathlineto{\pgfqpoint{13.709131in}{2.411129in}}%
\pgfpathlineto{\pgfqpoint{13.709131in}{2.418079in}}%
\pgfpathlineto{\pgfqpoint{-23.809001in}{2.418079in}}%
\pgfpathclose%
\pgfusepath{fill}%
\end{pgfscope}%
\begin{pgfscope}%
\pgfpathrectangle{\pgfqpoint{12.211765in}{1.942105in}}{\pgfqpoint{2.188235in}{0.953684in}} %
\pgfusepath{clip}%
\pgfsetbuttcap%
\pgfsetmiterjoin%
\definecolor{currentfill}{rgb}{0.121569,0.466667,0.705882}%
\pgfsetfillcolor{currentfill}%
\pgfsetlinewidth{0.000000pt}%
\definecolor{currentstroke}{rgb}{0.000000,0.000000,0.000000}%
\pgfsetstrokecolor{currentstroke}%
\pgfsetstrokeopacity{0.000000}%
\pgfsetdash{}{0pt}%
\pgfpathmoveto{\pgfqpoint{-23.809001in}{2.419816in}}%
\pgfpathlineto{\pgfqpoint{13.715492in}{2.419816in}}%
\pgfpathlineto{\pgfqpoint{13.715492in}{2.426766in}}%
\pgfpathlineto{\pgfqpoint{-23.809001in}{2.426766in}}%
\pgfpathclose%
\pgfusepath{fill}%
\end{pgfscope}%
\begin{pgfscope}%
\pgfpathrectangle{\pgfqpoint{12.211765in}{1.942105in}}{\pgfqpoint{2.188235in}{0.953684in}} %
\pgfusepath{clip}%
\pgfsetbuttcap%
\pgfsetmiterjoin%
\definecolor{currentfill}{rgb}{0.121569,0.466667,0.705882}%
\pgfsetfillcolor{currentfill}%
\pgfsetlinewidth{0.000000pt}%
\definecolor{currentstroke}{rgb}{0.000000,0.000000,0.000000}%
\pgfsetstrokecolor{currentstroke}%
\pgfsetstrokeopacity{0.000000}%
\pgfsetdash{}{0pt}%
\pgfpathmoveto{\pgfqpoint{-23.809001in}{2.428503in}}%
\pgfpathlineto{\pgfqpoint{13.622772in}{2.428503in}}%
\pgfpathlineto{\pgfqpoint{13.622772in}{2.435453in}}%
\pgfpathlineto{\pgfqpoint{-23.809001in}{2.435453in}}%
\pgfpathclose%
\pgfusepath{fill}%
\end{pgfscope}%
\begin{pgfscope}%
\pgfpathrectangle{\pgfqpoint{12.211765in}{1.942105in}}{\pgfqpoint{2.188235in}{0.953684in}} %
\pgfusepath{clip}%
\pgfsetbuttcap%
\pgfsetmiterjoin%
\definecolor{currentfill}{rgb}{0.121569,0.466667,0.705882}%
\pgfsetfillcolor{currentfill}%
\pgfsetlinewidth{0.000000pt}%
\definecolor{currentstroke}{rgb}{0.000000,0.000000,0.000000}%
\pgfsetstrokecolor{currentstroke}%
\pgfsetstrokeopacity{0.000000}%
\pgfsetdash{}{0pt}%
\pgfpathmoveto{\pgfqpoint{-23.809001in}{2.437191in}}%
\pgfpathlineto{\pgfqpoint{13.592087in}{2.437191in}}%
\pgfpathlineto{\pgfqpoint{13.592087in}{2.444140in}}%
\pgfpathlineto{\pgfqpoint{-23.809001in}{2.444140in}}%
\pgfpathclose%
\pgfusepath{fill}%
\end{pgfscope}%
\begin{pgfscope}%
\pgfpathrectangle{\pgfqpoint{12.211765in}{1.942105in}}{\pgfqpoint{2.188235in}{0.953684in}} %
\pgfusepath{clip}%
\pgfsetbuttcap%
\pgfsetmiterjoin%
\definecolor{currentfill}{rgb}{0.121569,0.466667,0.705882}%
\pgfsetfillcolor{currentfill}%
\pgfsetlinewidth{0.000000pt}%
\definecolor{currentstroke}{rgb}{0.000000,0.000000,0.000000}%
\pgfsetstrokecolor{currentstroke}%
\pgfsetstrokeopacity{0.000000}%
\pgfsetdash{}{0pt}%
\pgfpathmoveto{\pgfqpoint{-23.809001in}{2.445878in}}%
\pgfpathlineto{\pgfqpoint{13.650810in}{2.445878in}}%
\pgfpathlineto{\pgfqpoint{13.650810in}{2.452828in}}%
\pgfpathlineto{\pgfqpoint{-23.809001in}{2.452828in}}%
\pgfpathclose%
\pgfusepath{fill}%
\end{pgfscope}%
\begin{pgfscope}%
\pgfpathrectangle{\pgfqpoint{12.211765in}{1.942105in}}{\pgfqpoint{2.188235in}{0.953684in}} %
\pgfusepath{clip}%
\pgfsetbuttcap%
\pgfsetmiterjoin%
\definecolor{currentfill}{rgb}{0.121569,0.466667,0.705882}%
\pgfsetfillcolor{currentfill}%
\pgfsetlinewidth{0.000000pt}%
\definecolor{currentstroke}{rgb}{0.000000,0.000000,0.000000}%
\pgfsetstrokecolor{currentstroke}%
\pgfsetstrokeopacity{0.000000}%
\pgfsetdash{}{0pt}%
\pgfpathmoveto{\pgfqpoint{-23.809001in}{2.454565in}}%
\pgfpathlineto{\pgfqpoint{13.477142in}{2.454565in}}%
\pgfpathlineto{\pgfqpoint{13.477142in}{2.461515in}}%
\pgfpathlineto{\pgfqpoint{-23.809001in}{2.461515in}}%
\pgfpathclose%
\pgfusepath{fill}%
\end{pgfscope}%
\begin{pgfscope}%
\pgfpathrectangle{\pgfqpoint{12.211765in}{1.942105in}}{\pgfqpoint{2.188235in}{0.953684in}} %
\pgfusepath{clip}%
\pgfsetbuttcap%
\pgfsetmiterjoin%
\definecolor{currentfill}{rgb}{0.121569,0.466667,0.705882}%
\pgfsetfillcolor{currentfill}%
\pgfsetlinewidth{0.000000pt}%
\definecolor{currentstroke}{rgb}{0.000000,0.000000,0.000000}%
\pgfsetstrokecolor{currentstroke}%
\pgfsetstrokeopacity{0.000000}%
\pgfsetdash{}{0pt}%
\pgfpathmoveto{\pgfqpoint{-23.809001in}{2.463252in}}%
\pgfpathlineto{\pgfqpoint{13.608082in}{2.463252in}}%
\pgfpathlineto{\pgfqpoint{13.608082in}{2.470202in}}%
\pgfpathlineto{\pgfqpoint{-23.809001in}{2.470202in}}%
\pgfpathclose%
\pgfusepath{fill}%
\end{pgfscope}%
\begin{pgfscope}%
\pgfpathrectangle{\pgfqpoint{12.211765in}{1.942105in}}{\pgfqpoint{2.188235in}{0.953684in}} %
\pgfusepath{clip}%
\pgfsetbuttcap%
\pgfsetmiterjoin%
\definecolor{currentfill}{rgb}{0.121569,0.466667,0.705882}%
\pgfsetfillcolor{currentfill}%
\pgfsetlinewidth{0.000000pt}%
\definecolor{currentstroke}{rgb}{0.000000,0.000000,0.000000}%
\pgfsetstrokecolor{currentstroke}%
\pgfsetstrokeopacity{0.000000}%
\pgfsetdash{}{0pt}%
\pgfpathmoveto{\pgfqpoint{-23.809001in}{2.471939in}}%
\pgfpathlineto{\pgfqpoint{13.676922in}{2.471939in}}%
\pgfpathlineto{\pgfqpoint{13.676922in}{2.478889in}}%
\pgfpathlineto{\pgfqpoint{-23.809001in}{2.478889in}}%
\pgfpathclose%
\pgfusepath{fill}%
\end{pgfscope}%
\begin{pgfscope}%
\pgfpathrectangle{\pgfqpoint{12.211765in}{1.942105in}}{\pgfqpoint{2.188235in}{0.953684in}} %
\pgfusepath{clip}%
\pgfsetbuttcap%
\pgfsetmiterjoin%
\definecolor{currentfill}{rgb}{0.121569,0.466667,0.705882}%
\pgfsetfillcolor{currentfill}%
\pgfsetlinewidth{0.000000pt}%
\definecolor{currentstroke}{rgb}{0.000000,0.000000,0.000000}%
\pgfsetstrokecolor{currentstroke}%
\pgfsetstrokeopacity{0.000000}%
\pgfsetdash{}{0pt}%
\pgfpathmoveto{\pgfqpoint{-23.809001in}{2.480627in}}%
\pgfpathlineto{\pgfqpoint{13.602860in}{2.480627in}}%
\pgfpathlineto{\pgfqpoint{13.602860in}{2.487576in}}%
\pgfpathlineto{\pgfqpoint{-23.809001in}{2.487576in}}%
\pgfpathclose%
\pgfusepath{fill}%
\end{pgfscope}%
\begin{pgfscope}%
\pgfpathrectangle{\pgfqpoint{12.211765in}{1.942105in}}{\pgfqpoint{2.188235in}{0.953684in}} %
\pgfusepath{clip}%
\pgfsetbuttcap%
\pgfsetmiterjoin%
\definecolor{currentfill}{rgb}{0.121569,0.466667,0.705882}%
\pgfsetfillcolor{currentfill}%
\pgfsetlinewidth{0.000000pt}%
\definecolor{currentstroke}{rgb}{0.000000,0.000000,0.000000}%
\pgfsetstrokecolor{currentstroke}%
\pgfsetstrokeopacity{0.000000}%
\pgfsetdash{}{0pt}%
\pgfpathmoveto{\pgfqpoint{-23.809001in}{2.489314in}}%
\pgfpathlineto{\pgfqpoint{13.587324in}{2.489314in}}%
\pgfpathlineto{\pgfqpoint{13.587324in}{2.496264in}}%
\pgfpathlineto{\pgfqpoint{-23.809001in}{2.496264in}}%
\pgfpathclose%
\pgfusepath{fill}%
\end{pgfscope}%
\begin{pgfscope}%
\pgfpathrectangle{\pgfqpoint{12.211765in}{1.942105in}}{\pgfqpoint{2.188235in}{0.953684in}} %
\pgfusepath{clip}%
\pgfsetbuttcap%
\pgfsetmiterjoin%
\definecolor{currentfill}{rgb}{0.121569,0.466667,0.705882}%
\pgfsetfillcolor{currentfill}%
\pgfsetlinewidth{0.000000pt}%
\definecolor{currentstroke}{rgb}{0.000000,0.000000,0.000000}%
\pgfsetstrokecolor{currentstroke}%
\pgfsetstrokeopacity{0.000000}%
\pgfsetdash{}{0pt}%
\pgfpathmoveto{\pgfqpoint{-23.809001in}{2.498001in}}%
\pgfpathlineto{\pgfqpoint{13.668241in}{2.498001in}}%
\pgfpathlineto{\pgfqpoint{13.668241in}{2.504951in}}%
\pgfpathlineto{\pgfqpoint{-23.809001in}{2.504951in}}%
\pgfpathclose%
\pgfusepath{fill}%
\end{pgfscope}%
\begin{pgfscope}%
\pgfpathrectangle{\pgfqpoint{12.211765in}{1.942105in}}{\pgfqpoint{2.188235in}{0.953684in}} %
\pgfusepath{clip}%
\pgfsetbuttcap%
\pgfsetmiterjoin%
\definecolor{currentfill}{rgb}{0.121569,0.466667,0.705882}%
\pgfsetfillcolor{currentfill}%
\pgfsetlinewidth{0.000000pt}%
\definecolor{currentstroke}{rgb}{0.000000,0.000000,0.000000}%
\pgfsetstrokecolor{currentstroke}%
\pgfsetstrokeopacity{0.000000}%
\pgfsetdash{}{0pt}%
\pgfpathmoveto{\pgfqpoint{-23.809001in}{2.506688in}}%
\pgfpathlineto{\pgfqpoint{13.574259in}{2.506688in}}%
\pgfpathlineto{\pgfqpoint{13.574259in}{2.513638in}}%
\pgfpathlineto{\pgfqpoint{-23.809001in}{2.513638in}}%
\pgfpathclose%
\pgfusepath{fill}%
\end{pgfscope}%
\begin{pgfscope}%
\pgfpathrectangle{\pgfqpoint{12.211765in}{1.942105in}}{\pgfqpoint{2.188235in}{0.953684in}} %
\pgfusepath{clip}%
\pgfsetbuttcap%
\pgfsetmiterjoin%
\definecolor{currentfill}{rgb}{0.121569,0.466667,0.705882}%
\pgfsetfillcolor{currentfill}%
\pgfsetlinewidth{0.000000pt}%
\definecolor{currentstroke}{rgb}{0.000000,0.000000,0.000000}%
\pgfsetstrokecolor{currentstroke}%
\pgfsetstrokeopacity{0.000000}%
\pgfsetdash{}{0pt}%
\pgfpathmoveto{\pgfqpoint{-23.809001in}{2.515376in}}%
\pgfpathlineto{\pgfqpoint{13.572303in}{2.515376in}}%
\pgfpathlineto{\pgfqpoint{13.572303in}{2.522325in}}%
\pgfpathlineto{\pgfqpoint{-23.809001in}{2.522325in}}%
\pgfpathclose%
\pgfusepath{fill}%
\end{pgfscope}%
\begin{pgfscope}%
\pgfpathrectangle{\pgfqpoint{12.211765in}{1.942105in}}{\pgfqpoint{2.188235in}{0.953684in}} %
\pgfusepath{clip}%
\pgfsetbuttcap%
\pgfsetmiterjoin%
\definecolor{currentfill}{rgb}{0.121569,0.466667,0.705882}%
\pgfsetfillcolor{currentfill}%
\pgfsetlinewidth{0.000000pt}%
\definecolor{currentstroke}{rgb}{0.000000,0.000000,0.000000}%
\pgfsetstrokecolor{currentstroke}%
\pgfsetstrokeopacity{0.000000}%
\pgfsetdash{}{0pt}%
\pgfpathmoveto{\pgfqpoint{-23.809001in}{2.524063in}}%
\pgfpathlineto{\pgfqpoint{13.670631in}{2.524063in}}%
\pgfpathlineto{\pgfqpoint{13.670631in}{2.531013in}}%
\pgfpathlineto{\pgfqpoint{-23.809001in}{2.531013in}}%
\pgfpathclose%
\pgfusepath{fill}%
\end{pgfscope}%
\begin{pgfscope}%
\pgfpathrectangle{\pgfqpoint{12.211765in}{1.942105in}}{\pgfqpoint{2.188235in}{0.953684in}} %
\pgfusepath{clip}%
\pgfsetbuttcap%
\pgfsetmiterjoin%
\definecolor{currentfill}{rgb}{0.121569,0.466667,0.705882}%
\pgfsetfillcolor{currentfill}%
\pgfsetlinewidth{0.000000pt}%
\definecolor{currentstroke}{rgb}{0.000000,0.000000,0.000000}%
\pgfsetstrokecolor{currentstroke}%
\pgfsetstrokeopacity{0.000000}%
\pgfsetdash{}{0pt}%
\pgfpathmoveto{\pgfqpoint{-23.809001in}{2.532750in}}%
\pgfpathlineto{\pgfqpoint{13.455481in}{2.532750in}}%
\pgfpathlineto{\pgfqpoint{13.455481in}{2.539700in}}%
\pgfpathlineto{\pgfqpoint{-23.809001in}{2.539700in}}%
\pgfpathclose%
\pgfusepath{fill}%
\end{pgfscope}%
\begin{pgfscope}%
\pgfpathrectangle{\pgfqpoint{12.211765in}{1.942105in}}{\pgfqpoint{2.188235in}{0.953684in}} %
\pgfusepath{clip}%
\pgfsetbuttcap%
\pgfsetmiterjoin%
\definecolor{currentfill}{rgb}{0.121569,0.466667,0.705882}%
\pgfsetfillcolor{currentfill}%
\pgfsetlinewidth{0.000000pt}%
\definecolor{currentstroke}{rgb}{0.000000,0.000000,0.000000}%
\pgfsetstrokecolor{currentstroke}%
\pgfsetstrokeopacity{0.000000}%
\pgfsetdash{}{0pt}%
\pgfpathmoveto{\pgfqpoint{-23.809001in}{2.541437in}}%
\pgfpathlineto{\pgfqpoint{13.605252in}{2.541437in}}%
\pgfpathlineto{\pgfqpoint{13.605252in}{2.548387in}}%
\pgfpathlineto{\pgfqpoint{-23.809001in}{2.548387in}}%
\pgfpathclose%
\pgfusepath{fill}%
\end{pgfscope}%
\begin{pgfscope}%
\pgfpathrectangle{\pgfqpoint{12.211765in}{1.942105in}}{\pgfqpoint{2.188235in}{0.953684in}} %
\pgfusepath{clip}%
\pgfsetbuttcap%
\pgfsetmiterjoin%
\definecolor{currentfill}{rgb}{0.121569,0.466667,0.705882}%
\pgfsetfillcolor{currentfill}%
\pgfsetlinewidth{0.000000pt}%
\definecolor{currentstroke}{rgb}{0.000000,0.000000,0.000000}%
\pgfsetstrokecolor{currentstroke}%
\pgfsetstrokeopacity{0.000000}%
\pgfsetdash{}{0pt}%
\pgfpathmoveto{\pgfqpoint{-23.809001in}{2.550125in}}%
\pgfpathlineto{\pgfqpoint{13.570504in}{2.550125in}}%
\pgfpathlineto{\pgfqpoint{13.570504in}{2.557074in}}%
\pgfpathlineto{\pgfqpoint{-23.809001in}{2.557074in}}%
\pgfpathclose%
\pgfusepath{fill}%
\end{pgfscope}%
\begin{pgfscope}%
\pgfpathrectangle{\pgfqpoint{12.211765in}{1.942105in}}{\pgfqpoint{2.188235in}{0.953684in}} %
\pgfusepath{clip}%
\pgfsetbuttcap%
\pgfsetmiterjoin%
\definecolor{currentfill}{rgb}{0.121569,0.466667,0.705882}%
\pgfsetfillcolor{currentfill}%
\pgfsetlinewidth{0.000000pt}%
\definecolor{currentstroke}{rgb}{0.000000,0.000000,0.000000}%
\pgfsetstrokecolor{currentstroke}%
\pgfsetstrokeopacity{0.000000}%
\pgfsetdash{}{0pt}%
\pgfpathmoveto{\pgfqpoint{-23.809001in}{2.558812in}}%
\pgfpathlineto{\pgfqpoint{13.319959in}{2.558812in}}%
\pgfpathlineto{\pgfqpoint{13.319959in}{2.565762in}}%
\pgfpathlineto{\pgfqpoint{-23.809001in}{2.565762in}}%
\pgfpathclose%
\pgfusepath{fill}%
\end{pgfscope}%
\begin{pgfscope}%
\pgfpathrectangle{\pgfqpoint{12.211765in}{1.942105in}}{\pgfqpoint{2.188235in}{0.953684in}} %
\pgfusepath{clip}%
\pgfsetbuttcap%
\pgfsetmiterjoin%
\definecolor{currentfill}{rgb}{0.121569,0.466667,0.705882}%
\pgfsetfillcolor{currentfill}%
\pgfsetlinewidth{0.000000pt}%
\definecolor{currentstroke}{rgb}{0.000000,0.000000,0.000000}%
\pgfsetstrokecolor{currentstroke}%
\pgfsetstrokeopacity{0.000000}%
\pgfsetdash{}{0pt}%
\pgfpathmoveto{\pgfqpoint{-23.809001in}{2.567499in}}%
\pgfpathlineto{\pgfqpoint{13.531984in}{2.567499in}}%
\pgfpathlineto{\pgfqpoint{13.531984in}{2.574449in}}%
\pgfpathlineto{\pgfqpoint{-23.809001in}{2.574449in}}%
\pgfpathclose%
\pgfusepath{fill}%
\end{pgfscope}%
\begin{pgfscope}%
\pgfpathrectangle{\pgfqpoint{12.211765in}{1.942105in}}{\pgfqpoint{2.188235in}{0.953684in}} %
\pgfusepath{clip}%
\pgfsetbuttcap%
\pgfsetmiterjoin%
\definecolor{currentfill}{rgb}{0.121569,0.466667,0.705882}%
\pgfsetfillcolor{currentfill}%
\pgfsetlinewidth{0.000000pt}%
\definecolor{currentstroke}{rgb}{0.000000,0.000000,0.000000}%
\pgfsetstrokecolor{currentstroke}%
\pgfsetstrokeopacity{0.000000}%
\pgfsetdash{}{0pt}%
\pgfpathmoveto{\pgfqpoint{-23.809001in}{2.576186in}}%
\pgfpathlineto{\pgfqpoint{13.605367in}{2.576186in}}%
\pgfpathlineto{\pgfqpoint{13.605367in}{2.583136in}}%
\pgfpathlineto{\pgfqpoint{-23.809001in}{2.583136in}}%
\pgfpathclose%
\pgfusepath{fill}%
\end{pgfscope}%
\begin{pgfscope}%
\pgfpathrectangle{\pgfqpoint{12.211765in}{1.942105in}}{\pgfqpoint{2.188235in}{0.953684in}} %
\pgfusepath{clip}%
\pgfsetbuttcap%
\pgfsetmiterjoin%
\definecolor{currentfill}{rgb}{0.121569,0.466667,0.705882}%
\pgfsetfillcolor{currentfill}%
\pgfsetlinewidth{0.000000pt}%
\definecolor{currentstroke}{rgb}{0.000000,0.000000,0.000000}%
\pgfsetstrokecolor{currentstroke}%
\pgfsetstrokeopacity{0.000000}%
\pgfsetdash{}{0pt}%
\pgfpathmoveto{\pgfqpoint{-23.809001in}{2.584873in}}%
\pgfpathlineto{\pgfqpoint{13.606665in}{2.584873in}}%
\pgfpathlineto{\pgfqpoint{13.606665in}{2.591823in}}%
\pgfpathlineto{\pgfqpoint{-23.809001in}{2.591823in}}%
\pgfpathclose%
\pgfusepath{fill}%
\end{pgfscope}%
\begin{pgfscope}%
\pgfpathrectangle{\pgfqpoint{12.211765in}{1.942105in}}{\pgfqpoint{2.188235in}{0.953684in}} %
\pgfusepath{clip}%
\pgfsetbuttcap%
\pgfsetmiterjoin%
\definecolor{currentfill}{rgb}{0.121569,0.466667,0.705882}%
\pgfsetfillcolor{currentfill}%
\pgfsetlinewidth{0.000000pt}%
\definecolor{currentstroke}{rgb}{0.000000,0.000000,0.000000}%
\pgfsetstrokecolor{currentstroke}%
\pgfsetstrokeopacity{0.000000}%
\pgfsetdash{}{0pt}%
\pgfpathmoveto{\pgfqpoint{-23.809001in}{2.593561in}}%
\pgfpathlineto{\pgfqpoint{13.606921in}{2.593561in}}%
\pgfpathlineto{\pgfqpoint{13.606921in}{2.600510in}}%
\pgfpathlineto{\pgfqpoint{-23.809001in}{2.600510in}}%
\pgfpathclose%
\pgfusepath{fill}%
\end{pgfscope}%
\begin{pgfscope}%
\pgfpathrectangle{\pgfqpoint{12.211765in}{1.942105in}}{\pgfqpoint{2.188235in}{0.953684in}} %
\pgfusepath{clip}%
\pgfsetbuttcap%
\pgfsetmiterjoin%
\definecolor{currentfill}{rgb}{0.121569,0.466667,0.705882}%
\pgfsetfillcolor{currentfill}%
\pgfsetlinewidth{0.000000pt}%
\definecolor{currentstroke}{rgb}{0.000000,0.000000,0.000000}%
\pgfsetstrokecolor{currentstroke}%
\pgfsetstrokeopacity{0.000000}%
\pgfsetdash{}{0pt}%
\pgfpathmoveto{\pgfqpoint{-23.809001in}{2.602248in}}%
\pgfpathlineto{\pgfqpoint{13.622318in}{2.602248in}}%
\pgfpathlineto{\pgfqpoint{13.622318in}{2.609198in}}%
\pgfpathlineto{\pgfqpoint{-23.809001in}{2.609198in}}%
\pgfpathclose%
\pgfusepath{fill}%
\end{pgfscope}%
\begin{pgfscope}%
\pgfpathrectangle{\pgfqpoint{12.211765in}{1.942105in}}{\pgfqpoint{2.188235in}{0.953684in}} %
\pgfusepath{clip}%
\pgfsetbuttcap%
\pgfsetmiterjoin%
\definecolor{currentfill}{rgb}{0.121569,0.466667,0.705882}%
\pgfsetfillcolor{currentfill}%
\pgfsetlinewidth{0.000000pt}%
\definecolor{currentstroke}{rgb}{0.000000,0.000000,0.000000}%
\pgfsetstrokecolor{currentstroke}%
\pgfsetstrokeopacity{0.000000}%
\pgfsetdash{}{0pt}%
\pgfpathmoveto{\pgfqpoint{-23.809001in}{2.610935in}}%
\pgfpathlineto{\pgfqpoint{13.599046in}{2.610935in}}%
\pgfpathlineto{\pgfqpoint{13.599046in}{2.617885in}}%
\pgfpathlineto{\pgfqpoint{-23.809001in}{2.617885in}}%
\pgfpathclose%
\pgfusepath{fill}%
\end{pgfscope}%
\begin{pgfscope}%
\pgfpathrectangle{\pgfqpoint{12.211765in}{1.942105in}}{\pgfqpoint{2.188235in}{0.953684in}} %
\pgfusepath{clip}%
\pgfsetbuttcap%
\pgfsetmiterjoin%
\definecolor{currentfill}{rgb}{0.121569,0.466667,0.705882}%
\pgfsetfillcolor{currentfill}%
\pgfsetlinewidth{0.000000pt}%
\definecolor{currentstroke}{rgb}{0.000000,0.000000,0.000000}%
\pgfsetstrokecolor{currentstroke}%
\pgfsetstrokeopacity{0.000000}%
\pgfsetdash{}{0pt}%
\pgfpathmoveto{\pgfqpoint{-23.809001in}{2.619622in}}%
\pgfpathlineto{\pgfqpoint{13.562609in}{2.619622in}}%
\pgfpathlineto{\pgfqpoint{13.562609in}{2.626572in}}%
\pgfpathlineto{\pgfqpoint{-23.809001in}{2.626572in}}%
\pgfpathclose%
\pgfusepath{fill}%
\end{pgfscope}%
\begin{pgfscope}%
\pgfpathrectangle{\pgfqpoint{12.211765in}{1.942105in}}{\pgfqpoint{2.188235in}{0.953684in}} %
\pgfusepath{clip}%
\pgfsetbuttcap%
\pgfsetmiterjoin%
\definecolor{currentfill}{rgb}{0.121569,0.466667,0.705882}%
\pgfsetfillcolor{currentfill}%
\pgfsetlinewidth{0.000000pt}%
\definecolor{currentstroke}{rgb}{0.000000,0.000000,0.000000}%
\pgfsetstrokecolor{currentstroke}%
\pgfsetstrokeopacity{0.000000}%
\pgfsetdash{}{0pt}%
\pgfpathmoveto{\pgfqpoint{-23.809001in}{2.628310in}}%
\pgfpathlineto{\pgfqpoint{13.540048in}{2.628310in}}%
\pgfpathlineto{\pgfqpoint{13.540048in}{2.635259in}}%
\pgfpathlineto{\pgfqpoint{-23.809001in}{2.635259in}}%
\pgfpathclose%
\pgfusepath{fill}%
\end{pgfscope}%
\begin{pgfscope}%
\pgfpathrectangle{\pgfqpoint{12.211765in}{1.942105in}}{\pgfqpoint{2.188235in}{0.953684in}} %
\pgfusepath{clip}%
\pgfsetbuttcap%
\pgfsetmiterjoin%
\definecolor{currentfill}{rgb}{0.121569,0.466667,0.705882}%
\pgfsetfillcolor{currentfill}%
\pgfsetlinewidth{0.000000pt}%
\definecolor{currentstroke}{rgb}{0.000000,0.000000,0.000000}%
\pgfsetstrokecolor{currentstroke}%
\pgfsetstrokeopacity{0.000000}%
\pgfsetdash{}{0pt}%
\pgfpathmoveto{\pgfqpoint{-23.809001in}{2.636997in}}%
\pgfpathlineto{\pgfqpoint{13.619622in}{2.636997in}}%
\pgfpathlineto{\pgfqpoint{13.619622in}{2.643947in}}%
\pgfpathlineto{\pgfqpoint{-23.809001in}{2.643947in}}%
\pgfpathclose%
\pgfusepath{fill}%
\end{pgfscope}%
\begin{pgfscope}%
\pgfpathrectangle{\pgfqpoint{12.211765in}{1.942105in}}{\pgfqpoint{2.188235in}{0.953684in}} %
\pgfusepath{clip}%
\pgfsetbuttcap%
\pgfsetmiterjoin%
\definecolor{currentfill}{rgb}{0.121569,0.466667,0.705882}%
\pgfsetfillcolor{currentfill}%
\pgfsetlinewidth{0.000000pt}%
\definecolor{currentstroke}{rgb}{0.000000,0.000000,0.000000}%
\pgfsetstrokecolor{currentstroke}%
\pgfsetstrokeopacity{0.000000}%
\pgfsetdash{}{0pt}%
\pgfpathmoveto{\pgfqpoint{-23.809001in}{2.645684in}}%
\pgfpathlineto{\pgfqpoint{13.503155in}{2.645684in}}%
\pgfpathlineto{\pgfqpoint{13.503155in}{2.652634in}}%
\pgfpathlineto{\pgfqpoint{-23.809001in}{2.652634in}}%
\pgfpathclose%
\pgfusepath{fill}%
\end{pgfscope}%
\begin{pgfscope}%
\pgfpathrectangle{\pgfqpoint{12.211765in}{1.942105in}}{\pgfqpoint{2.188235in}{0.953684in}} %
\pgfusepath{clip}%
\pgfsetbuttcap%
\pgfsetmiterjoin%
\definecolor{currentfill}{rgb}{0.121569,0.466667,0.705882}%
\pgfsetfillcolor{currentfill}%
\pgfsetlinewidth{0.000000pt}%
\definecolor{currentstroke}{rgb}{0.000000,0.000000,0.000000}%
\pgfsetstrokecolor{currentstroke}%
\pgfsetstrokeopacity{0.000000}%
\pgfsetdash{}{0pt}%
\pgfpathmoveto{\pgfqpoint{-23.809001in}{2.654371in}}%
\pgfpathlineto{\pgfqpoint{13.494306in}{2.654371in}}%
\pgfpathlineto{\pgfqpoint{13.494306in}{2.661321in}}%
\pgfpathlineto{\pgfqpoint{-23.809001in}{2.661321in}}%
\pgfpathclose%
\pgfusepath{fill}%
\end{pgfscope}%
\begin{pgfscope}%
\pgfpathrectangle{\pgfqpoint{12.211765in}{1.942105in}}{\pgfqpoint{2.188235in}{0.953684in}} %
\pgfusepath{clip}%
\pgfsetbuttcap%
\pgfsetmiterjoin%
\definecolor{currentfill}{rgb}{0.121569,0.466667,0.705882}%
\pgfsetfillcolor{currentfill}%
\pgfsetlinewidth{0.000000pt}%
\definecolor{currentstroke}{rgb}{0.000000,0.000000,0.000000}%
\pgfsetstrokecolor{currentstroke}%
\pgfsetstrokeopacity{0.000000}%
\pgfsetdash{}{0pt}%
\pgfpathmoveto{\pgfqpoint{-23.809001in}{2.663059in}}%
\pgfpathlineto{\pgfqpoint{13.636009in}{2.663059in}}%
\pgfpathlineto{\pgfqpoint{13.636009in}{2.670008in}}%
\pgfpathlineto{\pgfqpoint{-23.809001in}{2.670008in}}%
\pgfpathclose%
\pgfusepath{fill}%
\end{pgfscope}%
\begin{pgfscope}%
\pgfpathrectangle{\pgfqpoint{12.211765in}{1.942105in}}{\pgfqpoint{2.188235in}{0.953684in}} %
\pgfusepath{clip}%
\pgfsetbuttcap%
\pgfsetmiterjoin%
\definecolor{currentfill}{rgb}{0.121569,0.466667,0.705882}%
\pgfsetfillcolor{currentfill}%
\pgfsetlinewidth{0.000000pt}%
\definecolor{currentstroke}{rgb}{0.000000,0.000000,0.000000}%
\pgfsetstrokecolor{currentstroke}%
\pgfsetstrokeopacity{0.000000}%
\pgfsetdash{}{0pt}%
\pgfpathmoveto{\pgfqpoint{-23.809001in}{2.671746in}}%
\pgfpathlineto{\pgfqpoint{13.649841in}{2.671746in}}%
\pgfpathlineto{\pgfqpoint{13.649841in}{2.678696in}}%
\pgfpathlineto{\pgfqpoint{-23.809001in}{2.678696in}}%
\pgfpathclose%
\pgfusepath{fill}%
\end{pgfscope}%
\begin{pgfscope}%
\pgfpathrectangle{\pgfqpoint{12.211765in}{1.942105in}}{\pgfqpoint{2.188235in}{0.953684in}} %
\pgfusepath{clip}%
\pgfsetbuttcap%
\pgfsetmiterjoin%
\definecolor{currentfill}{rgb}{0.121569,0.466667,0.705882}%
\pgfsetfillcolor{currentfill}%
\pgfsetlinewidth{0.000000pt}%
\definecolor{currentstroke}{rgb}{0.000000,0.000000,0.000000}%
\pgfsetstrokecolor{currentstroke}%
\pgfsetstrokeopacity{0.000000}%
\pgfsetdash{}{0pt}%
\pgfpathmoveto{\pgfqpoint{-23.809001in}{2.680433in}}%
\pgfpathlineto{\pgfqpoint{13.611812in}{2.680433in}}%
\pgfpathlineto{\pgfqpoint{13.611812in}{2.687383in}}%
\pgfpathlineto{\pgfqpoint{-23.809001in}{2.687383in}}%
\pgfpathclose%
\pgfusepath{fill}%
\end{pgfscope}%
\begin{pgfscope}%
\pgfpathrectangle{\pgfqpoint{12.211765in}{1.942105in}}{\pgfqpoint{2.188235in}{0.953684in}} %
\pgfusepath{clip}%
\pgfsetbuttcap%
\pgfsetmiterjoin%
\definecolor{currentfill}{rgb}{0.121569,0.466667,0.705882}%
\pgfsetfillcolor{currentfill}%
\pgfsetlinewidth{0.000000pt}%
\definecolor{currentstroke}{rgb}{0.000000,0.000000,0.000000}%
\pgfsetstrokecolor{currentstroke}%
\pgfsetstrokeopacity{0.000000}%
\pgfsetdash{}{0pt}%
\pgfpathmoveto{\pgfqpoint{-23.809001in}{2.689120in}}%
\pgfpathlineto{\pgfqpoint{13.566496in}{2.689120in}}%
\pgfpathlineto{\pgfqpoint{13.566496in}{2.696070in}}%
\pgfpathlineto{\pgfqpoint{-23.809001in}{2.696070in}}%
\pgfpathclose%
\pgfusepath{fill}%
\end{pgfscope}%
\begin{pgfscope}%
\pgfpathrectangle{\pgfqpoint{12.211765in}{1.942105in}}{\pgfqpoint{2.188235in}{0.953684in}} %
\pgfusepath{clip}%
\pgfsetbuttcap%
\pgfsetmiterjoin%
\definecolor{currentfill}{rgb}{0.121569,0.466667,0.705882}%
\pgfsetfillcolor{currentfill}%
\pgfsetlinewidth{0.000000pt}%
\definecolor{currentstroke}{rgb}{0.000000,0.000000,0.000000}%
\pgfsetstrokecolor{currentstroke}%
\pgfsetstrokeopacity{0.000000}%
\pgfsetdash{}{0pt}%
\pgfpathmoveto{\pgfqpoint{-23.809001in}{2.697807in}}%
\pgfpathlineto{\pgfqpoint{13.612029in}{2.697807in}}%
\pgfpathlineto{\pgfqpoint{13.612029in}{2.704757in}}%
\pgfpathlineto{\pgfqpoint{-23.809001in}{2.704757in}}%
\pgfpathclose%
\pgfusepath{fill}%
\end{pgfscope}%
\begin{pgfscope}%
\pgfpathrectangle{\pgfqpoint{12.211765in}{1.942105in}}{\pgfqpoint{2.188235in}{0.953684in}} %
\pgfusepath{clip}%
\pgfsetbuttcap%
\pgfsetmiterjoin%
\definecolor{currentfill}{rgb}{0.121569,0.466667,0.705882}%
\pgfsetfillcolor{currentfill}%
\pgfsetlinewidth{0.000000pt}%
\definecolor{currentstroke}{rgb}{0.000000,0.000000,0.000000}%
\pgfsetstrokecolor{currentstroke}%
\pgfsetstrokeopacity{0.000000}%
\pgfsetdash{}{0pt}%
\pgfpathmoveto{\pgfqpoint{-23.809001in}{2.706495in}}%
\pgfpathlineto{\pgfqpoint{13.500663in}{2.706495in}}%
\pgfpathlineto{\pgfqpoint{13.500663in}{2.713444in}}%
\pgfpathlineto{\pgfqpoint{-23.809001in}{2.713444in}}%
\pgfpathclose%
\pgfusepath{fill}%
\end{pgfscope}%
\begin{pgfscope}%
\pgfpathrectangle{\pgfqpoint{12.211765in}{1.942105in}}{\pgfqpoint{2.188235in}{0.953684in}} %
\pgfusepath{clip}%
\pgfsetbuttcap%
\pgfsetmiterjoin%
\definecolor{currentfill}{rgb}{0.121569,0.466667,0.705882}%
\pgfsetfillcolor{currentfill}%
\pgfsetlinewidth{0.000000pt}%
\definecolor{currentstroke}{rgb}{0.000000,0.000000,0.000000}%
\pgfsetstrokecolor{currentstroke}%
\pgfsetstrokeopacity{0.000000}%
\pgfsetdash{}{0pt}%
\pgfpathmoveto{\pgfqpoint{-23.809001in}{2.715182in}}%
\pgfpathlineto{\pgfqpoint{13.641148in}{2.715182in}}%
\pgfpathlineto{\pgfqpoint{13.641148in}{2.722132in}}%
\pgfpathlineto{\pgfqpoint{-23.809001in}{2.722132in}}%
\pgfpathclose%
\pgfusepath{fill}%
\end{pgfscope}%
\begin{pgfscope}%
\pgfpathrectangle{\pgfqpoint{12.211765in}{1.942105in}}{\pgfqpoint{2.188235in}{0.953684in}} %
\pgfusepath{clip}%
\pgfsetbuttcap%
\pgfsetmiterjoin%
\definecolor{currentfill}{rgb}{0.121569,0.466667,0.705882}%
\pgfsetfillcolor{currentfill}%
\pgfsetlinewidth{0.000000pt}%
\definecolor{currentstroke}{rgb}{0.000000,0.000000,0.000000}%
\pgfsetstrokecolor{currentstroke}%
\pgfsetstrokeopacity{0.000000}%
\pgfsetdash{}{0pt}%
\pgfpathmoveto{\pgfqpoint{-23.809001in}{2.723869in}}%
\pgfpathlineto{\pgfqpoint{13.720360in}{2.723869in}}%
\pgfpathlineto{\pgfqpoint{13.720360in}{2.730819in}}%
\pgfpathlineto{\pgfqpoint{-23.809001in}{2.730819in}}%
\pgfpathclose%
\pgfusepath{fill}%
\end{pgfscope}%
\begin{pgfscope}%
\pgfpathrectangle{\pgfqpoint{12.211765in}{1.942105in}}{\pgfqpoint{2.188235in}{0.953684in}} %
\pgfusepath{clip}%
\pgfsetbuttcap%
\pgfsetmiterjoin%
\definecolor{currentfill}{rgb}{0.121569,0.466667,0.705882}%
\pgfsetfillcolor{currentfill}%
\pgfsetlinewidth{0.000000pt}%
\definecolor{currentstroke}{rgb}{0.000000,0.000000,0.000000}%
\pgfsetstrokecolor{currentstroke}%
\pgfsetstrokeopacity{0.000000}%
\pgfsetdash{}{0pt}%
\pgfpathmoveto{\pgfqpoint{-23.809001in}{2.732556in}}%
\pgfpathlineto{\pgfqpoint{13.685159in}{2.732556in}}%
\pgfpathlineto{\pgfqpoint{13.685159in}{2.739506in}}%
\pgfpathlineto{\pgfqpoint{-23.809001in}{2.739506in}}%
\pgfpathclose%
\pgfusepath{fill}%
\end{pgfscope}%
\begin{pgfscope}%
\pgfpathrectangle{\pgfqpoint{12.211765in}{1.942105in}}{\pgfqpoint{2.188235in}{0.953684in}} %
\pgfusepath{clip}%
\pgfsetbuttcap%
\pgfsetmiterjoin%
\definecolor{currentfill}{rgb}{0.121569,0.466667,0.705882}%
\pgfsetfillcolor{currentfill}%
\pgfsetlinewidth{0.000000pt}%
\definecolor{currentstroke}{rgb}{0.000000,0.000000,0.000000}%
\pgfsetstrokecolor{currentstroke}%
\pgfsetstrokeopacity{0.000000}%
\pgfsetdash{}{0pt}%
\pgfpathmoveto{\pgfqpoint{-23.809001in}{2.741244in}}%
\pgfpathlineto{\pgfqpoint{13.623014in}{2.741244in}}%
\pgfpathlineto{\pgfqpoint{13.623014in}{2.748193in}}%
\pgfpathlineto{\pgfqpoint{-23.809001in}{2.748193in}}%
\pgfpathclose%
\pgfusepath{fill}%
\end{pgfscope}%
\begin{pgfscope}%
\pgfpathrectangle{\pgfqpoint{12.211765in}{1.942105in}}{\pgfqpoint{2.188235in}{0.953684in}} %
\pgfusepath{clip}%
\pgfsetbuttcap%
\pgfsetmiterjoin%
\definecolor{currentfill}{rgb}{0.121569,0.466667,0.705882}%
\pgfsetfillcolor{currentfill}%
\pgfsetlinewidth{0.000000pt}%
\definecolor{currentstroke}{rgb}{0.000000,0.000000,0.000000}%
\pgfsetstrokecolor{currentstroke}%
\pgfsetstrokeopacity{0.000000}%
\pgfsetdash{}{0pt}%
\pgfpathmoveto{\pgfqpoint{-23.809001in}{2.749931in}}%
\pgfpathlineto{\pgfqpoint{13.629310in}{2.749931in}}%
\pgfpathlineto{\pgfqpoint{13.629310in}{2.756881in}}%
\pgfpathlineto{\pgfqpoint{-23.809001in}{2.756881in}}%
\pgfpathclose%
\pgfusepath{fill}%
\end{pgfscope}%
\begin{pgfscope}%
\pgfpathrectangle{\pgfqpoint{12.211765in}{1.942105in}}{\pgfqpoint{2.188235in}{0.953684in}} %
\pgfusepath{clip}%
\pgfsetbuttcap%
\pgfsetmiterjoin%
\definecolor{currentfill}{rgb}{0.121569,0.466667,0.705882}%
\pgfsetfillcolor{currentfill}%
\pgfsetlinewidth{0.000000pt}%
\definecolor{currentstroke}{rgb}{0.000000,0.000000,0.000000}%
\pgfsetstrokecolor{currentstroke}%
\pgfsetstrokeopacity{0.000000}%
\pgfsetdash{}{0pt}%
\pgfpathmoveto{\pgfqpoint{-23.809001in}{2.758618in}}%
\pgfpathlineto{\pgfqpoint{13.610244in}{2.758618in}}%
\pgfpathlineto{\pgfqpoint{13.610244in}{2.765568in}}%
\pgfpathlineto{\pgfqpoint{-23.809001in}{2.765568in}}%
\pgfpathclose%
\pgfusepath{fill}%
\end{pgfscope}%
\begin{pgfscope}%
\pgfpathrectangle{\pgfqpoint{12.211765in}{1.942105in}}{\pgfqpoint{2.188235in}{0.953684in}} %
\pgfusepath{clip}%
\pgfsetbuttcap%
\pgfsetmiterjoin%
\definecolor{currentfill}{rgb}{0.121569,0.466667,0.705882}%
\pgfsetfillcolor{currentfill}%
\pgfsetlinewidth{0.000000pt}%
\definecolor{currentstroke}{rgb}{0.000000,0.000000,0.000000}%
\pgfsetstrokecolor{currentstroke}%
\pgfsetstrokeopacity{0.000000}%
\pgfsetdash{}{0pt}%
\pgfpathmoveto{\pgfqpoint{-23.809001in}{2.767305in}}%
\pgfpathlineto{\pgfqpoint{13.589159in}{2.767305in}}%
\pgfpathlineto{\pgfqpoint{13.589159in}{2.774255in}}%
\pgfpathlineto{\pgfqpoint{-23.809001in}{2.774255in}}%
\pgfpathclose%
\pgfusepath{fill}%
\end{pgfscope}%
\begin{pgfscope}%
\pgfpathrectangle{\pgfqpoint{12.211765in}{1.942105in}}{\pgfqpoint{2.188235in}{0.953684in}} %
\pgfusepath{clip}%
\pgfsetbuttcap%
\pgfsetmiterjoin%
\definecolor{currentfill}{rgb}{0.121569,0.466667,0.705882}%
\pgfsetfillcolor{currentfill}%
\pgfsetlinewidth{0.000000pt}%
\definecolor{currentstroke}{rgb}{0.000000,0.000000,0.000000}%
\pgfsetstrokecolor{currentstroke}%
\pgfsetstrokeopacity{0.000000}%
\pgfsetdash{}{0pt}%
\pgfpathmoveto{\pgfqpoint{-23.809001in}{2.775993in}}%
\pgfpathlineto{\pgfqpoint{13.595097in}{2.775993in}}%
\pgfpathlineto{\pgfqpoint{13.595097in}{2.782942in}}%
\pgfpathlineto{\pgfqpoint{-23.809001in}{2.782942in}}%
\pgfpathclose%
\pgfusepath{fill}%
\end{pgfscope}%
\begin{pgfscope}%
\pgfpathrectangle{\pgfqpoint{12.211765in}{1.942105in}}{\pgfqpoint{2.188235in}{0.953684in}} %
\pgfusepath{clip}%
\pgfsetbuttcap%
\pgfsetmiterjoin%
\definecolor{currentfill}{rgb}{0.121569,0.466667,0.705882}%
\pgfsetfillcolor{currentfill}%
\pgfsetlinewidth{0.000000pt}%
\definecolor{currentstroke}{rgb}{0.000000,0.000000,0.000000}%
\pgfsetstrokecolor{currentstroke}%
\pgfsetstrokeopacity{0.000000}%
\pgfsetdash{}{0pt}%
\pgfpathmoveto{\pgfqpoint{-23.809001in}{2.784680in}}%
\pgfpathlineto{\pgfqpoint{13.523107in}{2.784680in}}%
\pgfpathlineto{\pgfqpoint{13.523107in}{2.791630in}}%
\pgfpathlineto{\pgfqpoint{-23.809001in}{2.791630in}}%
\pgfpathclose%
\pgfusepath{fill}%
\end{pgfscope}%
\begin{pgfscope}%
\pgfpathrectangle{\pgfqpoint{12.211765in}{1.942105in}}{\pgfqpoint{2.188235in}{0.953684in}} %
\pgfusepath{clip}%
\pgfsetbuttcap%
\pgfsetmiterjoin%
\definecolor{currentfill}{rgb}{0.121569,0.466667,0.705882}%
\pgfsetfillcolor{currentfill}%
\pgfsetlinewidth{0.000000pt}%
\definecolor{currentstroke}{rgb}{0.000000,0.000000,0.000000}%
\pgfsetstrokecolor{currentstroke}%
\pgfsetstrokeopacity{0.000000}%
\pgfsetdash{}{0pt}%
\pgfpathmoveto{\pgfqpoint{-23.809001in}{2.793367in}}%
\pgfpathlineto{\pgfqpoint{13.654218in}{2.793367in}}%
\pgfpathlineto{\pgfqpoint{13.654218in}{2.800317in}}%
\pgfpathlineto{\pgfqpoint{-23.809001in}{2.800317in}}%
\pgfpathclose%
\pgfusepath{fill}%
\end{pgfscope}%
\begin{pgfscope}%
\pgfpathrectangle{\pgfqpoint{12.211765in}{1.942105in}}{\pgfqpoint{2.188235in}{0.953684in}} %
\pgfusepath{clip}%
\pgfsetbuttcap%
\pgfsetmiterjoin%
\definecolor{currentfill}{rgb}{0.121569,0.466667,0.705882}%
\pgfsetfillcolor{currentfill}%
\pgfsetlinewidth{0.000000pt}%
\definecolor{currentstroke}{rgb}{0.000000,0.000000,0.000000}%
\pgfsetstrokecolor{currentstroke}%
\pgfsetstrokeopacity{0.000000}%
\pgfsetdash{}{0pt}%
\pgfpathmoveto{\pgfqpoint{-23.809001in}{2.802054in}}%
\pgfpathlineto{\pgfqpoint{13.601765in}{2.802054in}}%
\pgfpathlineto{\pgfqpoint{13.601765in}{2.809004in}}%
\pgfpathlineto{\pgfqpoint{-23.809001in}{2.809004in}}%
\pgfpathclose%
\pgfusepath{fill}%
\end{pgfscope}%
\begin{pgfscope}%
\pgfpathrectangle{\pgfqpoint{12.211765in}{1.942105in}}{\pgfqpoint{2.188235in}{0.953684in}} %
\pgfusepath{clip}%
\pgfsetbuttcap%
\pgfsetmiterjoin%
\definecolor{currentfill}{rgb}{0.121569,0.466667,0.705882}%
\pgfsetfillcolor{currentfill}%
\pgfsetlinewidth{0.000000pt}%
\definecolor{currentstroke}{rgb}{0.000000,0.000000,0.000000}%
\pgfsetstrokecolor{currentstroke}%
\pgfsetstrokeopacity{0.000000}%
\pgfsetdash{}{0pt}%
\pgfpathmoveto{\pgfqpoint{-23.809001in}{2.810741in}}%
\pgfpathlineto{\pgfqpoint{13.605318in}{2.810741in}}%
\pgfpathlineto{\pgfqpoint{13.605318in}{2.817691in}}%
\pgfpathlineto{\pgfqpoint{-23.809001in}{2.817691in}}%
\pgfpathclose%
\pgfusepath{fill}%
\end{pgfscope}%
\begin{pgfscope}%
\pgfpathrectangle{\pgfqpoint{12.211765in}{1.942105in}}{\pgfqpoint{2.188235in}{0.953684in}} %
\pgfusepath{clip}%
\pgfsetbuttcap%
\pgfsetmiterjoin%
\definecolor{currentfill}{rgb}{0.121569,0.466667,0.705882}%
\pgfsetfillcolor{currentfill}%
\pgfsetlinewidth{0.000000pt}%
\definecolor{currentstroke}{rgb}{0.000000,0.000000,0.000000}%
\pgfsetstrokecolor{currentstroke}%
\pgfsetstrokeopacity{0.000000}%
\pgfsetdash{}{0pt}%
\pgfpathmoveto{\pgfqpoint{-23.809001in}{2.819429in}}%
\pgfpathlineto{\pgfqpoint{13.660011in}{2.819429in}}%
\pgfpathlineto{\pgfqpoint{13.660011in}{2.826378in}}%
\pgfpathlineto{\pgfqpoint{-23.809001in}{2.826378in}}%
\pgfpathclose%
\pgfusepath{fill}%
\end{pgfscope}%
\begin{pgfscope}%
\pgfpathrectangle{\pgfqpoint{12.211765in}{1.942105in}}{\pgfqpoint{2.188235in}{0.953684in}} %
\pgfusepath{clip}%
\pgfsetbuttcap%
\pgfsetmiterjoin%
\definecolor{currentfill}{rgb}{0.121569,0.466667,0.705882}%
\pgfsetfillcolor{currentfill}%
\pgfsetlinewidth{0.000000pt}%
\definecolor{currentstroke}{rgb}{0.000000,0.000000,0.000000}%
\pgfsetstrokecolor{currentstroke}%
\pgfsetstrokeopacity{0.000000}%
\pgfsetdash{}{0pt}%
\pgfpathmoveto{\pgfqpoint{-23.809001in}{2.828116in}}%
\pgfpathlineto{\pgfqpoint{13.574215in}{2.828116in}}%
\pgfpathlineto{\pgfqpoint{13.574215in}{2.835066in}}%
\pgfpathlineto{\pgfqpoint{-23.809001in}{2.835066in}}%
\pgfpathclose%
\pgfusepath{fill}%
\end{pgfscope}%
\begin{pgfscope}%
\pgfpathrectangle{\pgfqpoint{12.211765in}{1.942105in}}{\pgfqpoint{2.188235in}{0.953684in}} %
\pgfusepath{clip}%
\pgfsetbuttcap%
\pgfsetmiterjoin%
\definecolor{currentfill}{rgb}{0.121569,0.466667,0.705882}%
\pgfsetfillcolor{currentfill}%
\pgfsetlinewidth{0.000000pt}%
\definecolor{currentstroke}{rgb}{0.000000,0.000000,0.000000}%
\pgfsetstrokecolor{currentstroke}%
\pgfsetstrokeopacity{0.000000}%
\pgfsetdash{}{0pt}%
\pgfpathmoveto{\pgfqpoint{-23.809001in}{2.836803in}}%
\pgfpathlineto{\pgfqpoint{13.538113in}{2.836803in}}%
\pgfpathlineto{\pgfqpoint{13.538113in}{2.843753in}}%
\pgfpathlineto{\pgfqpoint{-23.809001in}{2.843753in}}%
\pgfpathclose%
\pgfusepath{fill}%
\end{pgfscope}%
\begin{pgfscope}%
\pgfpathrectangle{\pgfqpoint{12.211765in}{1.942105in}}{\pgfqpoint{2.188235in}{0.953684in}} %
\pgfusepath{clip}%
\pgfsetbuttcap%
\pgfsetmiterjoin%
\definecolor{currentfill}{rgb}{0.121569,0.466667,0.705882}%
\pgfsetfillcolor{currentfill}%
\pgfsetlinewidth{0.000000pt}%
\definecolor{currentstroke}{rgb}{0.000000,0.000000,0.000000}%
\pgfsetstrokecolor{currentstroke}%
\pgfsetstrokeopacity{0.000000}%
\pgfsetdash{}{0pt}%
\pgfpathmoveto{\pgfqpoint{-23.809001in}{2.845490in}}%
\pgfpathlineto{\pgfqpoint{13.469093in}{2.845490in}}%
\pgfpathlineto{\pgfqpoint{13.469093in}{2.852440in}}%
\pgfpathlineto{\pgfqpoint{-23.809001in}{2.852440in}}%
\pgfpathclose%
\pgfusepath{fill}%
\end{pgfscope}%
\begin{pgfscope}%
\pgfsetbuttcap%
\pgfsetroundjoin%
\definecolor{currentfill}{rgb}{0.000000,0.000000,0.000000}%
\pgfsetfillcolor{currentfill}%
\pgfsetlinewidth{0.803000pt}%
\definecolor{currentstroke}{rgb}{0.000000,0.000000,0.000000}%
\pgfsetstrokecolor{currentstroke}%
\pgfsetdash{}{0pt}%
\pgfsys@defobject{currentmarker}{\pgfqpoint{0.000000in}{-0.048611in}}{\pgfqpoint{0.000000in}{0.000000in}}{%
\pgfpathmoveto{\pgfqpoint{0.000000in}{0.000000in}}%
\pgfpathlineto{\pgfqpoint{0.000000in}{-0.048611in}}%
\pgfusepath{stroke,fill}%
}%
\begin{pgfscope}%
\pgfsys@transformshift{12.624272in}{1.942105in}%
\pgfsys@useobject{currentmarker}{}%
\end{pgfscope}%
\end{pgfscope}%
\begin{pgfscope}%
\pgfsetbuttcap%
\pgfsetroundjoin%
\definecolor{currentfill}{rgb}{0.000000,0.000000,0.000000}%
\pgfsetfillcolor{currentfill}%
\pgfsetlinewidth{0.803000pt}%
\definecolor{currentstroke}{rgb}{0.000000,0.000000,0.000000}%
\pgfsetstrokecolor{currentstroke}%
\pgfsetdash{}{0pt}%
\pgfsys@defobject{currentmarker}{\pgfqpoint{0.000000in}{-0.048611in}}{\pgfqpoint{0.000000in}{0.000000in}}{%
\pgfpathmoveto{\pgfqpoint{0.000000in}{0.000000in}}%
\pgfpathlineto{\pgfqpoint{0.000000in}{-0.048611in}}%
\pgfusepath{stroke,fill}%
}%
\begin{pgfscope}%
\pgfsys@transformshift{13.130289in}{1.942105in}%
\pgfsys@useobject{currentmarker}{}%
\end{pgfscope}%
\end{pgfscope}%
\begin{pgfscope}%
\pgfsetbuttcap%
\pgfsetroundjoin%
\definecolor{currentfill}{rgb}{0.000000,0.000000,0.000000}%
\pgfsetfillcolor{currentfill}%
\pgfsetlinewidth{0.803000pt}%
\definecolor{currentstroke}{rgb}{0.000000,0.000000,0.000000}%
\pgfsetstrokecolor{currentstroke}%
\pgfsetdash{}{0pt}%
\pgfsys@defobject{currentmarker}{\pgfqpoint{0.000000in}{-0.048611in}}{\pgfqpoint{0.000000in}{0.000000in}}{%
\pgfpathmoveto{\pgfqpoint{0.000000in}{0.000000in}}%
\pgfpathlineto{\pgfqpoint{0.000000in}{-0.048611in}}%
\pgfusepath{stroke,fill}%
}%
\begin{pgfscope}%
\pgfsys@transformshift{13.636307in}{1.942105in}%
\pgfsys@useobject{currentmarker}{}%
\end{pgfscope}%
\end{pgfscope}%
\begin{pgfscope}%
\pgfsetbuttcap%
\pgfsetroundjoin%
\definecolor{currentfill}{rgb}{0.000000,0.000000,0.000000}%
\pgfsetfillcolor{currentfill}%
\pgfsetlinewidth{0.803000pt}%
\definecolor{currentstroke}{rgb}{0.000000,0.000000,0.000000}%
\pgfsetstrokecolor{currentstroke}%
\pgfsetdash{}{0pt}%
\pgfsys@defobject{currentmarker}{\pgfqpoint{0.000000in}{-0.048611in}}{\pgfqpoint{0.000000in}{0.000000in}}{%
\pgfpathmoveto{\pgfqpoint{0.000000in}{0.000000in}}%
\pgfpathlineto{\pgfqpoint{0.000000in}{-0.048611in}}%
\pgfusepath{stroke,fill}%
}%
\begin{pgfscope}%
\pgfsys@transformshift{14.142325in}{1.942105in}%
\pgfsys@useobject{currentmarker}{}%
\end{pgfscope}%
\end{pgfscope}%
\begin{pgfscope}%
\pgfsetbuttcap%
\pgfsetroundjoin%
\definecolor{currentfill}{rgb}{0.000000,0.000000,0.000000}%
\pgfsetfillcolor{currentfill}%
\pgfsetlinewidth{0.803000pt}%
\definecolor{currentstroke}{rgb}{0.000000,0.000000,0.000000}%
\pgfsetstrokecolor{currentstroke}%
\pgfsetdash{}{0pt}%
\pgfsys@defobject{currentmarker}{\pgfqpoint{-0.048611in}{0.000000in}}{\pgfqpoint{0.000000in}{0.000000in}}{%
\pgfpathmoveto{\pgfqpoint{0.000000in}{0.000000in}}%
\pgfpathlineto{\pgfqpoint{-0.048611in}{0.000000in}}%
\pgfusepath{stroke,fill}%
}%
\begin{pgfscope}%
\pgfsys@transformshift{12.211765in}{1.988929in}%
\pgfsys@useobject{currentmarker}{}%
\end{pgfscope}%
\end{pgfscope}%
\begin{pgfscope}%
\pgftext[x=12.045098in,y=1.940712in,left,base]{\rmfamily\fontsize{10.000000}{12.000000}\selectfont \(\displaystyle 0\)}%
\end{pgfscope}%
\begin{pgfscope}%
\pgfsetbuttcap%
\pgfsetroundjoin%
\definecolor{currentfill}{rgb}{0.000000,0.000000,0.000000}%
\pgfsetfillcolor{currentfill}%
\pgfsetlinewidth{0.803000pt}%
\definecolor{currentstroke}{rgb}{0.000000,0.000000,0.000000}%
\pgfsetstrokecolor{currentstroke}%
\pgfsetdash{}{0pt}%
\pgfsys@defobject{currentmarker}{\pgfqpoint{-0.048611in}{0.000000in}}{\pgfqpoint{0.000000in}{0.000000in}}{%
\pgfpathmoveto{\pgfqpoint{0.000000in}{0.000000in}}%
\pgfpathlineto{\pgfqpoint{-0.048611in}{0.000000in}}%
\pgfusepath{stroke,fill}%
}%
\begin{pgfscope}%
\pgfsys@transformshift{12.211765in}{2.423291in}%
\pgfsys@useobject{currentmarker}{}%
\end{pgfscope}%
\end{pgfscope}%
\begin{pgfscope}%
\pgftext[x=11.975653in,y=2.375073in,left,base]{\rmfamily\fontsize{10.000000}{12.000000}\selectfont \(\displaystyle 50\)}%
\end{pgfscope}%
\begin{pgfscope}%
\pgfsetbuttcap%
\pgfsetroundjoin%
\definecolor{currentfill}{rgb}{0.000000,0.000000,0.000000}%
\pgfsetfillcolor{currentfill}%
\pgfsetlinewidth{0.803000pt}%
\definecolor{currentstroke}{rgb}{0.000000,0.000000,0.000000}%
\pgfsetstrokecolor{currentstroke}%
\pgfsetdash{}{0pt}%
\pgfsys@defobject{currentmarker}{\pgfqpoint{-0.048611in}{0.000000in}}{\pgfqpoint{0.000000in}{0.000000in}}{%
\pgfpathmoveto{\pgfqpoint{0.000000in}{0.000000in}}%
\pgfpathlineto{\pgfqpoint{-0.048611in}{0.000000in}}%
\pgfusepath{stroke,fill}%
}%
\begin{pgfscope}%
\pgfsys@transformshift{12.211765in}{2.857653in}%
\pgfsys@useobject{currentmarker}{}%
\end{pgfscope}%
\end{pgfscope}%
\begin{pgfscope}%
\pgftext[x=11.906208in,y=2.809435in,left,base]{\rmfamily\fontsize{10.000000}{12.000000}\selectfont \(\displaystyle 100\)}%
\end{pgfscope}%
\begin{pgfscope}%
\pgftext[x=11.850653in,y=2.418947in,,bottom,rotate=90.000000]{\rmfamily\fontsize{10.000000}{12.000000}\selectfont \(\displaystyle j\)}%
\end{pgfscope}%
\begin{pgfscope}%
\pgfsetrectcap%
\pgfsetmiterjoin%
\pgfsetlinewidth{0.803000pt}%
\definecolor{currentstroke}{rgb}{0.000000,0.000000,0.000000}%
\pgfsetstrokecolor{currentstroke}%
\pgfsetdash{}{0pt}%
\pgfpathmoveto{\pgfqpoint{12.211765in}{1.942105in}}%
\pgfpathlineto{\pgfqpoint{12.211765in}{2.895789in}}%
\pgfusepath{stroke}%
\end{pgfscope}%
\begin{pgfscope}%
\pgfsetrectcap%
\pgfsetmiterjoin%
\pgfsetlinewidth{0.803000pt}%
\definecolor{currentstroke}{rgb}{0.000000,0.000000,0.000000}%
\pgfsetstrokecolor{currentstroke}%
\pgfsetdash{}{0pt}%
\pgfpathmoveto{\pgfqpoint{14.400000in}{1.942105in}}%
\pgfpathlineto{\pgfqpoint{14.400000in}{2.895789in}}%
\pgfusepath{stroke}%
\end{pgfscope}%
\begin{pgfscope}%
\pgfsetrectcap%
\pgfsetmiterjoin%
\pgfsetlinewidth{0.803000pt}%
\definecolor{currentstroke}{rgb}{0.000000,0.000000,0.000000}%
\pgfsetstrokecolor{currentstroke}%
\pgfsetdash{}{0pt}%
\pgfpathmoveto{\pgfqpoint{12.211765in}{1.942105in}}%
\pgfpathlineto{\pgfqpoint{14.400000in}{1.942105in}}%
\pgfusepath{stroke}%
\end{pgfscope}%
\begin{pgfscope}%
\pgfsetrectcap%
\pgfsetmiterjoin%
\pgfsetlinewidth{0.803000pt}%
\definecolor{currentstroke}{rgb}{0.000000,0.000000,0.000000}%
\pgfsetstrokecolor{currentstroke}%
\pgfsetdash{}{0pt}%
\pgfpathmoveto{\pgfqpoint{12.211765in}{2.895789in}}%
\pgfpathlineto{\pgfqpoint{14.400000in}{2.895789in}}%
\pgfusepath{stroke}%
\end{pgfscope}%
\begin{pgfscope}%
\pgfsetbuttcap%
\pgfsetmiterjoin%
\definecolor{currentfill}{rgb}{1.000000,1.000000,1.000000}%
\pgfsetfillcolor{currentfill}%
\pgfsetlinewidth{0.000000pt}%
\definecolor{currentstroke}{rgb}{0.000000,0.000000,0.000000}%
\pgfsetstrokecolor{currentstroke}%
\pgfsetstrokeopacity{0.000000}%
\pgfsetdash{}{0pt}%
\pgfpathmoveto{\pgfqpoint{2.000000in}{0.750000in}}%
\pgfpathlineto{\pgfqpoint{6.376471in}{0.750000in}}%
\pgfpathlineto{\pgfqpoint{6.376471in}{1.703684in}}%
\pgfpathlineto{\pgfqpoint{2.000000in}{1.703684in}}%
\pgfpathclose%
\pgfusepath{fill}%
\end{pgfscope}%
\begin{pgfscope}%
\pgfpathrectangle{\pgfqpoint{2.000000in}{0.750000in}}{\pgfqpoint{4.376471in}{0.953684in}} %
\pgfusepath{clip}%
\pgfsetbuttcap%
\pgfsetroundjoin%
\definecolor{currentfill}{rgb}{1.000000,0.000000,0.000000}%
\pgfsetfillcolor{currentfill}%
\pgfsetlinewidth{2.007500pt}%
\definecolor{currentstroke}{rgb}{1.000000,0.000000,0.000000}%
\pgfsetstrokecolor{currentstroke}%
\pgfsetdash{}{0pt}%
\pgfpathmoveto{\pgfqpoint{4.755120in}{1.106507in}}%
\pgfpathlineto{\pgfqpoint{4.838454in}{1.106507in}}%
\pgfpathmoveto{\pgfqpoint{4.796787in}{1.064841in}}%
\pgfpathlineto{\pgfqpoint{4.796787in}{1.148174in}}%
\pgfusepath{stroke,fill}%
\end{pgfscope}%
\begin{pgfscope}%
\pgfpathrectangle{\pgfqpoint{2.000000in}{0.750000in}}{\pgfqpoint{4.376471in}{0.953684in}} %
\pgfusepath{clip}%
\pgfsetbuttcap%
\pgfsetroundjoin%
\definecolor{currentfill}{rgb}{1.000000,0.000000,0.000000}%
\pgfsetfillcolor{currentfill}%
\pgfsetlinewidth{2.007500pt}%
\definecolor{currentstroke}{rgb}{1.000000,0.000000,0.000000}%
\pgfsetstrokecolor{currentstroke}%
\pgfsetdash{}{0pt}%
\pgfpathmoveto{\pgfqpoint{5.337632in}{1.192701in}}%
\pgfpathlineto{\pgfqpoint{5.420965in}{1.192701in}}%
\pgfpathmoveto{\pgfqpoint{5.379298in}{1.151034in}}%
\pgfpathlineto{\pgfqpoint{5.379298in}{1.234367in}}%
\pgfusepath{stroke,fill}%
\end{pgfscope}%
\begin{pgfscope}%
\pgfpathrectangle{\pgfqpoint{2.000000in}{0.750000in}}{\pgfqpoint{4.376471in}{0.953684in}} %
\pgfusepath{clip}%
\pgfsetbuttcap%
\pgfsetroundjoin%
\definecolor{currentfill}{rgb}{1.000000,0.000000,0.000000}%
\pgfsetfillcolor{currentfill}%
\pgfsetlinewidth{2.007500pt}%
\definecolor{currentstroke}{rgb}{1.000000,0.000000,0.000000}%
\pgfsetstrokecolor{currentstroke}%
\pgfsetdash{}{0pt}%
\pgfpathmoveto{\pgfqpoint{4.944008in}{1.249607in}}%
\pgfpathlineto{\pgfqpoint{5.027342in}{1.249607in}}%
\pgfpathmoveto{\pgfqpoint{4.985675in}{1.207940in}}%
\pgfpathlineto{\pgfqpoint{4.985675in}{1.291274in}}%
\pgfusepath{stroke,fill}%
\end{pgfscope}%
\begin{pgfscope}%
\pgfpathrectangle{\pgfqpoint{2.000000in}{0.750000in}}{\pgfqpoint{4.376471in}{0.953684in}} %
\pgfusepath{clip}%
\pgfsetbuttcap%
\pgfsetroundjoin%
\definecolor{currentfill}{rgb}{1.000000,0.000000,0.000000}%
\pgfsetfillcolor{currentfill}%
\pgfsetlinewidth{2.007500pt}%
\definecolor{currentstroke}{rgb}{1.000000,0.000000,0.000000}%
\pgfsetstrokecolor{currentstroke}%
\pgfsetdash{}{0pt}%
\pgfpathmoveto{\pgfqpoint{4.741360in}{1.222660in}}%
\pgfpathlineto{\pgfqpoint{4.824693in}{1.222660in}}%
\pgfpathmoveto{\pgfqpoint{4.783026in}{1.180993in}}%
\pgfpathlineto{\pgfqpoint{4.783026in}{1.264327in}}%
\pgfusepath{stroke,fill}%
\end{pgfscope}%
\begin{pgfscope}%
\pgfpathrectangle{\pgfqpoint{2.000000in}{0.750000in}}{\pgfqpoint{4.376471in}{0.953684in}} %
\pgfusepath{clip}%
\pgfsetbuttcap%
\pgfsetroundjoin%
\definecolor{currentfill}{rgb}{1.000000,0.000000,0.000000}%
\pgfsetfillcolor{currentfill}%
\pgfsetlinewidth{2.007500pt}%
\definecolor{currentstroke}{rgb}{1.000000,0.000000,0.000000}%
\pgfsetstrokecolor{currentstroke}%
\pgfsetdash{}{0pt}%
\pgfpathmoveto{\pgfqpoint{4.316918in}{0.793948in}}%
\pgfpathlineto{\pgfqpoint{4.400251in}{0.793948in}}%
\pgfpathmoveto{\pgfqpoint{4.358584in}{0.752281in}}%
\pgfpathlineto{\pgfqpoint{4.358584in}{0.835614in}}%
\pgfusepath{stroke,fill}%
\end{pgfscope}%
\begin{pgfscope}%
\pgfpathrectangle{\pgfqpoint{2.000000in}{0.750000in}}{\pgfqpoint{4.376471in}{0.953684in}} %
\pgfusepath{clip}%
\pgfsetbuttcap%
\pgfsetroundjoin%
\definecolor{currentfill}{rgb}{1.000000,0.000000,0.000000}%
\pgfsetfillcolor{currentfill}%
\pgfsetlinewidth{2.007500pt}%
\definecolor{currentstroke}{rgb}{1.000000,0.000000,0.000000}%
\pgfsetstrokecolor{currentstroke}%
\pgfsetdash{}{0pt}%
\pgfpathmoveto{\pgfqpoint{5.095017in}{1.262779in}}%
\pgfpathlineto{\pgfqpoint{5.178350in}{1.262779in}}%
\pgfpathmoveto{\pgfqpoint{5.136683in}{1.221112in}}%
\pgfpathlineto{\pgfqpoint{5.136683in}{1.304446in}}%
\pgfusepath{stroke,fill}%
\end{pgfscope}%
\begin{pgfscope}%
\pgfpathrectangle{\pgfqpoint{2.000000in}{0.750000in}}{\pgfqpoint{4.376471in}{0.953684in}} %
\pgfusepath{clip}%
\pgfsetbuttcap%
\pgfsetroundjoin%
\definecolor{currentfill}{rgb}{1.000000,0.000000,0.000000}%
\pgfsetfillcolor{currentfill}%
\pgfsetlinewidth{2.007500pt}%
\definecolor{currentstroke}{rgb}{1.000000,0.000000,0.000000}%
\pgfsetstrokecolor{currentstroke}%
\pgfsetdash{}{0pt}%
\pgfpathmoveto{\pgfqpoint{4.365697in}{0.765056in}}%
\pgfpathlineto{\pgfqpoint{4.449031in}{0.765056in}}%
\pgfpathmoveto{\pgfqpoint{4.407364in}{0.723390in}}%
\pgfpathlineto{\pgfqpoint{4.407364in}{0.806723in}}%
\pgfusepath{stroke,fill}%
\end{pgfscope}%
\begin{pgfscope}%
\pgfpathrectangle{\pgfqpoint{2.000000in}{0.750000in}}{\pgfqpoint{4.376471in}{0.953684in}} %
\pgfusepath{clip}%
\pgfsetbuttcap%
\pgfsetroundjoin%
\definecolor{currentfill}{rgb}{1.000000,0.000000,0.000000}%
\pgfsetfillcolor{currentfill}%
\pgfsetlinewidth{2.007500pt}%
\definecolor{currentstroke}{rgb}{1.000000,0.000000,0.000000}%
\pgfsetstrokecolor{currentstroke}%
\pgfsetdash{}{0pt}%
\pgfpathmoveto{\pgfqpoint{5.955882in}{1.529110in}}%
\pgfpathlineto{\pgfqpoint{6.039215in}{1.529110in}}%
\pgfpathmoveto{\pgfqpoint{5.997549in}{1.487443in}}%
\pgfpathlineto{\pgfqpoint{5.997549in}{1.570776in}}%
\pgfusepath{stroke,fill}%
\end{pgfscope}%
\begin{pgfscope}%
\pgfpathrectangle{\pgfqpoint{2.000000in}{0.750000in}}{\pgfqpoint{4.376471in}{0.953684in}} %
\pgfusepath{clip}%
\pgfsetbuttcap%
\pgfsetroundjoin%
\definecolor{currentfill}{rgb}{1.000000,0.000000,0.000000}%
\pgfsetfillcolor{currentfill}%
\pgfsetlinewidth{2.007500pt}%
\definecolor{currentstroke}{rgb}{1.000000,0.000000,0.000000}%
\pgfsetstrokecolor{currentstroke}%
\pgfsetdash{}{0pt}%
\pgfpathmoveto{\pgfqpoint{6.207581in}{1.390374in}}%
\pgfpathlineto{\pgfqpoint{6.290914in}{1.390374in}}%
\pgfpathmoveto{\pgfqpoint{6.249248in}{1.348708in}}%
\pgfpathlineto{\pgfqpoint{6.249248in}{1.432041in}}%
\pgfusepath{stroke,fill}%
\end{pgfscope}%
\begin{pgfscope}%
\pgfpathrectangle{\pgfqpoint{2.000000in}{0.750000in}}{\pgfqpoint{4.376471in}{0.953684in}} %
\pgfusepath{clip}%
\pgfsetbuttcap%
\pgfsetroundjoin%
\definecolor{currentfill}{rgb}{1.000000,0.000000,0.000000}%
\pgfsetfillcolor{currentfill}%
\pgfsetlinewidth{2.007500pt}%
\definecolor{currentstroke}{rgb}{1.000000,0.000000,0.000000}%
\pgfsetstrokecolor{currentstroke}%
\pgfsetdash{}{0pt}%
\pgfpathmoveto{\pgfqpoint{4.176124in}{1.023699in}}%
\pgfpathlineto{\pgfqpoint{4.259457in}{1.023699in}}%
\pgfpathmoveto{\pgfqpoint{4.217791in}{0.982033in}}%
\pgfpathlineto{\pgfqpoint{4.217791in}{1.065366in}}%
\pgfusepath{stroke,fill}%
\end{pgfscope}%
\begin{pgfscope}%
\pgfpathrectangle{\pgfqpoint{2.000000in}{0.750000in}}{\pgfqpoint{4.376471in}{0.953684in}} %
\pgfusepath{clip}%
\pgfsetbuttcap%
\pgfsetroundjoin%
\definecolor{currentfill}{rgb}{1.000000,0.000000,0.000000}%
\pgfsetfillcolor{currentfill}%
\pgfsetlinewidth{2.007500pt}%
\definecolor{currentstroke}{rgb}{1.000000,0.000000,0.000000}%
\pgfsetstrokecolor{currentstroke}%
\pgfsetdash{}{0pt}%
\pgfpathmoveto{\pgfqpoint{5.605597in}{1.105972in}}%
\pgfpathlineto{\pgfqpoint{5.688930in}{1.105972in}}%
\pgfpathmoveto{\pgfqpoint{5.647263in}{1.064305in}}%
\pgfpathlineto{\pgfqpoint{5.647263in}{1.147638in}}%
\pgfusepath{stroke,fill}%
\end{pgfscope}%
\begin{pgfscope}%
\pgfpathrectangle{\pgfqpoint{2.000000in}{0.750000in}}{\pgfqpoint{4.376471in}{0.953684in}} %
\pgfusepath{clip}%
\pgfsetbuttcap%
\pgfsetroundjoin%
\definecolor{currentfill}{rgb}{1.000000,0.000000,0.000000}%
\pgfsetfillcolor{currentfill}%
\pgfsetlinewidth{2.007500pt}%
\definecolor{currentstroke}{rgb}{1.000000,0.000000,0.000000}%
\pgfsetstrokecolor{currentstroke}%
\pgfsetdash{}{0pt}%
\pgfpathmoveto{\pgfqpoint{4.685382in}{1.100206in}}%
\pgfpathlineto{\pgfqpoint{4.768715in}{1.100206in}}%
\pgfpathmoveto{\pgfqpoint{4.727049in}{1.058540in}}%
\pgfpathlineto{\pgfqpoint{4.727049in}{1.141873in}}%
\pgfusepath{stroke,fill}%
\end{pgfscope}%
\begin{pgfscope}%
\pgfpathrectangle{\pgfqpoint{2.000000in}{0.750000in}}{\pgfqpoint{4.376471in}{0.953684in}} %
\pgfusepath{clip}%
\pgfsetbuttcap%
\pgfsetroundjoin%
\definecolor{currentfill}{rgb}{1.000000,0.000000,0.000000}%
\pgfsetfillcolor{currentfill}%
\pgfsetlinewidth{2.007500pt}%
\definecolor{currentstroke}{rgb}{1.000000,0.000000,0.000000}%
\pgfsetstrokecolor{currentstroke}%
\pgfsetdash{}{0pt}%
\pgfpathmoveto{\pgfqpoint{4.822452in}{1.306514in}}%
\pgfpathlineto{\pgfqpoint{4.905785in}{1.306514in}}%
\pgfpathmoveto{\pgfqpoint{4.864118in}{1.264848in}}%
\pgfpathlineto{\pgfqpoint{4.864118in}{1.348181in}}%
\pgfusepath{stroke,fill}%
\end{pgfscope}%
\begin{pgfscope}%
\pgfpathrectangle{\pgfqpoint{2.000000in}{0.750000in}}{\pgfqpoint{4.376471in}{0.953684in}} %
\pgfusepath{clip}%
\pgfsetbuttcap%
\pgfsetroundjoin%
\definecolor{currentfill}{rgb}{1.000000,0.000000,0.000000}%
\pgfsetfillcolor{currentfill}%
\pgfsetlinewidth{2.007500pt}%
\definecolor{currentstroke}{rgb}{1.000000,0.000000,0.000000}%
\pgfsetstrokecolor{currentstroke}%
\pgfsetdash{}{0pt}%
\pgfpathmoveto{\pgfqpoint{6.074305in}{1.595297in}}%
\pgfpathlineto{\pgfqpoint{6.157638in}{1.595297in}}%
\pgfpathmoveto{\pgfqpoint{6.115971in}{1.553630in}}%
\pgfpathlineto{\pgfqpoint{6.115971in}{1.636963in}}%
\pgfusepath{stroke,fill}%
\end{pgfscope}%
\begin{pgfscope}%
\pgfpathrectangle{\pgfqpoint{2.000000in}{0.750000in}}{\pgfqpoint{4.376471in}{0.953684in}} %
\pgfusepath{clip}%
\pgfsetbuttcap%
\pgfsetroundjoin%
\definecolor{currentfill}{rgb}{1.000000,0.000000,0.000000}%
\pgfsetfillcolor{currentfill}%
\pgfsetlinewidth{2.007500pt}%
\definecolor{currentstroke}{rgb}{1.000000,0.000000,0.000000}%
\pgfsetstrokecolor{currentstroke}%
\pgfsetdash{}{0pt}%
\pgfpathmoveto{\pgfqpoint{3.082337in}{1.258043in}}%
\pgfpathlineto{\pgfqpoint{3.165671in}{1.258043in}}%
\pgfpathmoveto{\pgfqpoint{3.124004in}{1.216376in}}%
\pgfpathlineto{\pgfqpoint{3.124004in}{1.299710in}}%
\pgfusepath{stroke,fill}%
\end{pgfscope}%
\begin{pgfscope}%
\pgfpathrectangle{\pgfqpoint{2.000000in}{0.750000in}}{\pgfqpoint{4.376471in}{0.953684in}} %
\pgfusepath{clip}%
\pgfsetbuttcap%
\pgfsetroundjoin%
\definecolor{currentfill}{rgb}{1.000000,0.000000,0.000000}%
\pgfsetfillcolor{currentfill}%
\pgfsetlinewidth{2.007500pt}%
\definecolor{currentstroke}{rgb}{1.000000,0.000000,0.000000}%
\pgfsetstrokecolor{currentstroke}%
\pgfsetdash{}{0pt}%
\pgfpathmoveto{\pgfqpoint{3.138683in}{1.225891in}}%
\pgfpathlineto{\pgfqpoint{3.222016in}{1.225891in}}%
\pgfpathmoveto{\pgfqpoint{3.180349in}{1.184225in}}%
\pgfpathlineto{\pgfqpoint{3.180349in}{1.267558in}}%
\pgfusepath{stroke,fill}%
\end{pgfscope}%
\begin{pgfscope}%
\pgfpathrectangle{\pgfqpoint{2.000000in}{0.750000in}}{\pgfqpoint{4.376471in}{0.953684in}} %
\pgfusepath{clip}%
\pgfsetbuttcap%
\pgfsetroundjoin%
\definecolor{currentfill}{rgb}{1.000000,0.000000,0.000000}%
\pgfsetfillcolor{currentfill}%
\pgfsetlinewidth{2.007500pt}%
\definecolor{currentstroke}{rgb}{1.000000,0.000000,0.000000}%
\pgfsetstrokecolor{currentstroke}%
\pgfsetdash{}{0pt}%
\pgfpathmoveto{\pgfqpoint{2.904416in}{1.444279in}}%
\pgfpathlineto{\pgfqpoint{2.987749in}{1.444279in}}%
\pgfpathmoveto{\pgfqpoint{2.946082in}{1.402613in}}%
\pgfpathlineto{\pgfqpoint{2.946082in}{1.485946in}}%
\pgfusepath{stroke,fill}%
\end{pgfscope}%
\begin{pgfscope}%
\pgfpathrectangle{\pgfqpoint{2.000000in}{0.750000in}}{\pgfqpoint{4.376471in}{0.953684in}} %
\pgfusepath{clip}%
\pgfsetbuttcap%
\pgfsetroundjoin%
\definecolor{currentfill}{rgb}{1.000000,0.000000,0.000000}%
\pgfsetfillcolor{currentfill}%
\pgfsetlinewidth{2.007500pt}%
\definecolor{currentstroke}{rgb}{1.000000,0.000000,0.000000}%
\pgfsetstrokecolor{currentstroke}%
\pgfsetdash{}{0pt}%
\pgfpathmoveto{\pgfqpoint{5.748776in}{1.319150in}}%
\pgfpathlineto{\pgfqpoint{5.832110in}{1.319150in}}%
\pgfpathmoveto{\pgfqpoint{5.790443in}{1.277484in}}%
\pgfpathlineto{\pgfqpoint{5.790443in}{1.360817in}}%
\pgfusepath{stroke,fill}%
\end{pgfscope}%
\begin{pgfscope}%
\pgfpathrectangle{\pgfqpoint{2.000000in}{0.750000in}}{\pgfqpoint{4.376471in}{0.953684in}} %
\pgfusepath{clip}%
\pgfsetbuttcap%
\pgfsetroundjoin%
\definecolor{currentfill}{rgb}{1.000000,0.000000,0.000000}%
\pgfsetfillcolor{currentfill}%
\pgfsetlinewidth{2.007500pt}%
\definecolor{currentstroke}{rgb}{1.000000,0.000000,0.000000}%
\pgfsetstrokecolor{currentstroke}%
\pgfsetdash{}{0pt}%
\pgfpathmoveto{\pgfqpoint{5.558092in}{1.032791in}}%
\pgfpathlineto{\pgfqpoint{5.641425in}{1.032791in}}%
\pgfpathmoveto{\pgfqpoint{5.599758in}{0.991124in}}%
\pgfpathlineto{\pgfqpoint{5.599758in}{1.074457in}}%
\pgfusepath{stroke,fill}%
\end{pgfscope}%
\begin{pgfscope}%
\pgfpathrectangle{\pgfqpoint{2.000000in}{0.750000in}}{\pgfqpoint{4.376471in}{0.953684in}} %
\pgfusepath{clip}%
\pgfsetbuttcap%
\pgfsetroundjoin%
\definecolor{currentfill}{rgb}{1.000000,0.000000,0.000000}%
\pgfsetfillcolor{currentfill}%
\pgfsetlinewidth{2.007500pt}%
\definecolor{currentstroke}{rgb}{1.000000,0.000000,0.000000}%
\pgfsetstrokecolor{currentstroke}%
\pgfsetdash{}{0pt}%
\pgfpathmoveto{\pgfqpoint{5.879694in}{1.468106in}}%
\pgfpathlineto{\pgfqpoint{5.963027in}{1.468106in}}%
\pgfpathmoveto{\pgfqpoint{5.921360in}{1.426440in}}%
\pgfpathlineto{\pgfqpoint{5.921360in}{1.509773in}}%
\pgfusepath{stroke,fill}%
\end{pgfscope}%
\begin{pgfscope}%
\pgfpathrectangle{\pgfqpoint{2.000000in}{0.750000in}}{\pgfqpoint{4.376471in}{0.953684in}} %
\pgfusepath{clip}%
\pgfsetbuttcap%
\pgfsetroundjoin%
\definecolor{currentfill}{rgb}{1.000000,0.000000,0.000000}%
\pgfsetfillcolor{currentfill}%
\pgfsetlinewidth{2.007500pt}%
\definecolor{currentstroke}{rgb}{1.000000,0.000000,0.000000}%
\pgfsetstrokecolor{currentstroke}%
\pgfsetdash{}{0pt}%
\pgfpathmoveto{\pgfqpoint{6.259943in}{1.361744in}}%
\pgfpathlineto{\pgfqpoint{6.343276in}{1.361744in}}%
\pgfpathmoveto{\pgfqpoint{6.301610in}{1.320078in}}%
\pgfpathlineto{\pgfqpoint{6.301610in}{1.403411in}}%
\pgfusepath{stroke,fill}%
\end{pgfscope}%
\begin{pgfscope}%
\pgfpathrectangle{\pgfqpoint{2.000000in}{0.750000in}}{\pgfqpoint{4.376471in}{0.953684in}} %
\pgfusepath{clip}%
\pgfsetbuttcap%
\pgfsetroundjoin%
\definecolor{currentfill}{rgb}{1.000000,0.000000,0.000000}%
\pgfsetfillcolor{currentfill}%
\pgfsetlinewidth{2.007500pt}%
\definecolor{currentstroke}{rgb}{1.000000,0.000000,0.000000}%
\pgfsetstrokecolor{currentstroke}%
\pgfsetdash{}{0pt}%
\pgfpathmoveto{\pgfqpoint{5.631623in}{1.019672in}}%
\pgfpathlineto{\pgfqpoint{5.714956in}{1.019672in}}%
\pgfpathmoveto{\pgfqpoint{5.673289in}{0.978006in}}%
\pgfpathlineto{\pgfqpoint{5.673289in}{1.061339in}}%
\pgfusepath{stroke,fill}%
\end{pgfscope}%
\begin{pgfscope}%
\pgfpathrectangle{\pgfqpoint{2.000000in}{0.750000in}}{\pgfqpoint{4.376471in}{0.953684in}} %
\pgfusepath{clip}%
\pgfsetbuttcap%
\pgfsetroundjoin%
\definecolor{currentfill}{rgb}{1.000000,0.000000,0.000000}%
\pgfsetfillcolor{currentfill}%
\pgfsetlinewidth{2.007500pt}%
\definecolor{currentstroke}{rgb}{1.000000,0.000000,0.000000}%
\pgfsetstrokecolor{currentstroke}%
\pgfsetdash{}{0pt}%
\pgfpathmoveto{\pgfqpoint{4.449348in}{0.989246in}}%
\pgfpathlineto{\pgfqpoint{4.532681in}{0.989246in}}%
\pgfpathmoveto{\pgfqpoint{4.491015in}{0.947579in}}%
\pgfpathlineto{\pgfqpoint{4.491015in}{1.030913in}}%
\pgfusepath{stroke,fill}%
\end{pgfscope}%
\begin{pgfscope}%
\pgfpathrectangle{\pgfqpoint{2.000000in}{0.750000in}}{\pgfqpoint{4.376471in}{0.953684in}} %
\pgfusepath{clip}%
\pgfsetbuttcap%
\pgfsetroundjoin%
\definecolor{currentfill}{rgb}{1.000000,0.000000,0.000000}%
\pgfsetfillcolor{currentfill}%
\pgfsetlinewidth{2.007500pt}%
\definecolor{currentstroke}{rgb}{1.000000,0.000000,0.000000}%
\pgfsetstrokecolor{currentstroke}%
\pgfsetdash{}{0pt}%
\pgfpathmoveto{\pgfqpoint{5.566398in}{1.069019in}}%
\pgfpathlineto{\pgfqpoint{5.649731in}{1.069019in}}%
\pgfpathmoveto{\pgfqpoint{5.608065in}{1.027352in}}%
\pgfpathlineto{\pgfqpoint{5.608065in}{1.110686in}}%
\pgfusepath{stroke,fill}%
\end{pgfscope}%
\begin{pgfscope}%
\pgfpathrectangle{\pgfqpoint{2.000000in}{0.750000in}}{\pgfqpoint{4.376471in}{0.953684in}} %
\pgfusepath{clip}%
\pgfsetbuttcap%
\pgfsetroundjoin%
\definecolor{currentfill}{rgb}{1.000000,0.000000,0.000000}%
\pgfsetfillcolor{currentfill}%
\pgfsetlinewidth{2.007500pt}%
\definecolor{currentstroke}{rgb}{1.000000,0.000000,0.000000}%
\pgfsetstrokecolor{currentstroke}%
\pgfsetdash{}{0pt}%
\pgfpathmoveto{\pgfqpoint{3.247727in}{0.983375in}}%
\pgfpathlineto{\pgfqpoint{3.331060in}{0.983375in}}%
\pgfpathmoveto{\pgfqpoint{3.289394in}{0.941708in}}%
\pgfpathlineto{\pgfqpoint{3.289394in}{1.025042in}}%
\pgfusepath{stroke,fill}%
\end{pgfscope}%
\begin{pgfscope}%
\pgfpathrectangle{\pgfqpoint{2.000000in}{0.750000in}}{\pgfqpoint{4.376471in}{0.953684in}} %
\pgfusepath{clip}%
\pgfsetbuttcap%
\pgfsetroundjoin%
\definecolor{currentfill}{rgb}{1.000000,0.000000,0.000000}%
\pgfsetfillcolor{currentfill}%
\pgfsetlinewidth{2.007500pt}%
\definecolor{currentstroke}{rgb}{1.000000,0.000000,0.000000}%
\pgfsetstrokecolor{currentstroke}%
\pgfsetdash{}{0pt}%
\pgfpathmoveto{\pgfqpoint{5.074104in}{1.251576in}}%
\pgfpathlineto{\pgfqpoint{5.157437in}{1.251576in}}%
\pgfpathmoveto{\pgfqpoint{5.115771in}{1.209910in}}%
\pgfpathlineto{\pgfqpoint{5.115771in}{1.293243in}}%
\pgfusepath{stroke,fill}%
\end{pgfscope}%
\begin{pgfscope}%
\pgfpathrectangle{\pgfqpoint{2.000000in}{0.750000in}}{\pgfqpoint{4.376471in}{0.953684in}} %
\pgfusepath{clip}%
\pgfsetbuttcap%
\pgfsetroundjoin%
\definecolor{currentfill}{rgb}{1.000000,0.000000,0.000000}%
\pgfsetfillcolor{currentfill}%
\pgfsetlinewidth{2.007500pt}%
\definecolor{currentstroke}{rgb}{1.000000,0.000000,0.000000}%
\pgfsetstrokecolor{currentstroke}%
\pgfsetdash{}{0pt}%
\pgfpathmoveto{\pgfqpoint{3.335533in}{1.070440in}}%
\pgfpathlineto{\pgfqpoint{3.418866in}{1.070440in}}%
\pgfpathmoveto{\pgfqpoint{3.377199in}{1.028773in}}%
\pgfpathlineto{\pgfqpoint{3.377199in}{1.112107in}}%
\pgfusepath{stroke,fill}%
\end{pgfscope}%
\begin{pgfscope}%
\pgfpathrectangle{\pgfqpoint{2.000000in}{0.750000in}}{\pgfqpoint{4.376471in}{0.953684in}} %
\pgfusepath{clip}%
\pgfsetbuttcap%
\pgfsetroundjoin%
\definecolor{currentfill}{rgb}{1.000000,0.000000,0.000000}%
\pgfsetfillcolor{currentfill}%
\pgfsetlinewidth{2.007500pt}%
\definecolor{currentstroke}{rgb}{1.000000,0.000000,0.000000}%
\pgfsetstrokecolor{currentstroke}%
\pgfsetdash{}{0pt}%
\pgfpathmoveto{\pgfqpoint{6.141080in}{1.527761in}}%
\pgfpathlineto{\pgfqpoint{6.224413in}{1.527761in}}%
\pgfpathmoveto{\pgfqpoint{6.182747in}{1.486094in}}%
\pgfpathlineto{\pgfqpoint{6.182747in}{1.569427in}}%
\pgfusepath{stroke,fill}%
\end{pgfscope}%
\begin{pgfscope}%
\pgfpathrectangle{\pgfqpoint{2.000000in}{0.750000in}}{\pgfqpoint{4.376471in}{0.953684in}} %
\pgfusepath{clip}%
\pgfsetbuttcap%
\pgfsetroundjoin%
\definecolor{currentfill}{rgb}{1.000000,0.000000,0.000000}%
\pgfsetfillcolor{currentfill}%
\pgfsetlinewidth{2.007500pt}%
\definecolor{currentstroke}{rgb}{1.000000,0.000000,0.000000}%
\pgfsetstrokecolor{currentstroke}%
\pgfsetdash{}{0pt}%
\pgfpathmoveto{\pgfqpoint{4.660711in}{1.264901in}}%
\pgfpathlineto{\pgfqpoint{4.744044in}{1.264901in}}%
\pgfpathmoveto{\pgfqpoint{4.702377in}{1.223235in}}%
\pgfpathlineto{\pgfqpoint{4.702377in}{1.306568in}}%
\pgfusepath{stroke,fill}%
\end{pgfscope}%
\begin{pgfscope}%
\pgfpathrectangle{\pgfqpoint{2.000000in}{0.750000in}}{\pgfqpoint{4.376471in}{0.953684in}} %
\pgfusepath{clip}%
\pgfsetbuttcap%
\pgfsetroundjoin%
\definecolor{currentfill}{rgb}{1.000000,0.000000,0.000000}%
\pgfsetfillcolor{currentfill}%
\pgfsetlinewidth{2.007500pt}%
\definecolor{currentstroke}{rgb}{1.000000,0.000000,0.000000}%
\pgfsetstrokecolor{currentstroke}%
\pgfsetdash{}{0pt}%
\pgfpathmoveto{\pgfqpoint{4.285432in}{0.935170in}}%
\pgfpathlineto{\pgfqpoint{4.368765in}{0.935170in}}%
\pgfpathmoveto{\pgfqpoint{4.327099in}{0.893503in}}%
\pgfpathlineto{\pgfqpoint{4.327099in}{0.976837in}}%
\pgfusepath{stroke,fill}%
\end{pgfscope}%
\begin{pgfscope}%
\pgfpathrectangle{\pgfqpoint{2.000000in}{0.750000in}}{\pgfqpoint{4.376471in}{0.953684in}} %
\pgfusepath{clip}%
\pgfsetbuttcap%
\pgfsetroundjoin%
\definecolor{currentfill}{rgb}{1.000000,0.000000,0.000000}%
\pgfsetfillcolor{currentfill}%
\pgfsetlinewidth{2.007500pt}%
\definecolor{currentstroke}{rgb}{1.000000,0.000000,0.000000}%
\pgfsetstrokecolor{currentstroke}%
\pgfsetdash{}{0pt}%
\pgfpathmoveto{\pgfqpoint{3.759883in}{1.507616in}}%
\pgfpathlineto{\pgfqpoint{3.843217in}{1.507616in}}%
\pgfpathmoveto{\pgfqpoint{3.801550in}{1.465949in}}%
\pgfpathlineto{\pgfqpoint{3.801550in}{1.549283in}}%
\pgfusepath{stroke,fill}%
\end{pgfscope}%
\begin{pgfscope}%
\pgfpathrectangle{\pgfqpoint{2.000000in}{0.750000in}}{\pgfqpoint{4.376471in}{0.953684in}} %
\pgfusepath{clip}%
\pgfsetbuttcap%
\pgfsetroundjoin%
\definecolor{currentfill}{rgb}{1.000000,0.000000,0.000000}%
\pgfsetfillcolor{currentfill}%
\pgfsetlinewidth{2.007500pt}%
\definecolor{currentstroke}{rgb}{1.000000,0.000000,0.000000}%
\pgfsetstrokecolor{currentstroke}%
\pgfsetdash{}{0pt}%
\pgfpathmoveto{\pgfqpoint{5.544356in}{0.962175in}}%
\pgfpathlineto{\pgfqpoint{5.627690in}{0.962175in}}%
\pgfpathmoveto{\pgfqpoint{5.586023in}{0.920509in}}%
\pgfpathlineto{\pgfqpoint{5.586023in}{1.003842in}}%
\pgfusepath{stroke,fill}%
\end{pgfscope}%
\begin{pgfscope}%
\pgfpathrectangle{\pgfqpoint{2.000000in}{0.750000in}}{\pgfqpoint{4.376471in}{0.953684in}} %
\pgfusepath{clip}%
\pgfsetbuttcap%
\pgfsetroundjoin%
\definecolor{currentfill}{rgb}{1.000000,0.000000,0.000000}%
\pgfsetfillcolor{currentfill}%
\pgfsetlinewidth{2.007500pt}%
\definecolor{currentstroke}{rgb}{1.000000,0.000000,0.000000}%
\pgfsetstrokecolor{currentstroke}%
\pgfsetdash{}{0pt}%
\pgfpathmoveto{\pgfqpoint{4.430690in}{0.997905in}}%
\pgfpathlineto{\pgfqpoint{4.514024in}{0.997905in}}%
\pgfpathmoveto{\pgfqpoint{4.472357in}{0.956238in}}%
\pgfpathlineto{\pgfqpoint{4.472357in}{1.039572in}}%
\pgfusepath{stroke,fill}%
\end{pgfscope}%
\begin{pgfscope}%
\pgfpathrectangle{\pgfqpoint{2.000000in}{0.750000in}}{\pgfqpoint{4.376471in}{0.953684in}} %
\pgfusepath{clip}%
\pgfsetbuttcap%
\pgfsetroundjoin%
\definecolor{currentfill}{rgb}{1.000000,0.000000,0.000000}%
\pgfsetfillcolor{currentfill}%
\pgfsetlinewidth{2.007500pt}%
\definecolor{currentstroke}{rgb}{1.000000,0.000000,0.000000}%
\pgfsetstrokecolor{currentstroke}%
\pgfsetdash{}{0pt}%
\pgfpathmoveto{\pgfqpoint{4.823815in}{1.241685in}}%
\pgfpathlineto{\pgfqpoint{4.907148in}{1.241685in}}%
\pgfpathmoveto{\pgfqpoint{4.865482in}{1.200018in}}%
\pgfpathlineto{\pgfqpoint{4.865482in}{1.283351in}}%
\pgfusepath{stroke,fill}%
\end{pgfscope}%
\begin{pgfscope}%
\pgfpathrectangle{\pgfqpoint{2.000000in}{0.750000in}}{\pgfqpoint{4.376471in}{0.953684in}} %
\pgfusepath{clip}%
\pgfsetbuttcap%
\pgfsetroundjoin%
\definecolor{currentfill}{rgb}{1.000000,0.000000,0.000000}%
\pgfsetfillcolor{currentfill}%
\pgfsetlinewidth{2.007500pt}%
\definecolor{currentstroke}{rgb}{1.000000,0.000000,0.000000}%
\pgfsetstrokecolor{currentstroke}%
\pgfsetdash{}{0pt}%
\pgfpathmoveto{\pgfqpoint{2.899414in}{1.428549in}}%
\pgfpathlineto{\pgfqpoint{2.982747in}{1.428549in}}%
\pgfpathmoveto{\pgfqpoint{2.941081in}{1.386882in}}%
\pgfpathlineto{\pgfqpoint{2.941081in}{1.470216in}}%
\pgfusepath{stroke,fill}%
\end{pgfscope}%
\begin{pgfscope}%
\pgfpathrectangle{\pgfqpoint{2.000000in}{0.750000in}}{\pgfqpoint{4.376471in}{0.953684in}} %
\pgfusepath{clip}%
\pgfsetbuttcap%
\pgfsetroundjoin%
\definecolor{currentfill}{rgb}{1.000000,0.000000,0.000000}%
\pgfsetfillcolor{currentfill}%
\pgfsetlinewidth{2.007500pt}%
\definecolor{currentstroke}{rgb}{1.000000,0.000000,0.000000}%
\pgfsetstrokecolor{currentstroke}%
\pgfsetdash{}{0pt}%
\pgfpathmoveto{\pgfqpoint{4.996078in}{1.213102in}}%
\pgfpathlineto{\pgfqpoint{5.079412in}{1.213102in}}%
\pgfpathmoveto{\pgfqpoint{5.037745in}{1.171436in}}%
\pgfpathlineto{\pgfqpoint{5.037745in}{1.254769in}}%
\pgfusepath{stroke,fill}%
\end{pgfscope}%
\begin{pgfscope}%
\pgfpathrectangle{\pgfqpoint{2.000000in}{0.750000in}}{\pgfqpoint{4.376471in}{0.953684in}} %
\pgfusepath{clip}%
\pgfsetbuttcap%
\pgfsetroundjoin%
\definecolor{currentfill}{rgb}{1.000000,0.000000,0.000000}%
\pgfsetfillcolor{currentfill}%
\pgfsetlinewidth{2.007500pt}%
\definecolor{currentstroke}{rgb}{1.000000,0.000000,0.000000}%
\pgfsetstrokecolor{currentstroke}%
\pgfsetdash{}{0pt}%
\pgfpathmoveto{\pgfqpoint{4.976683in}{1.252249in}}%
\pgfpathlineto{\pgfqpoint{5.060016in}{1.252249in}}%
\pgfpathmoveto{\pgfqpoint{5.018349in}{1.210582in}}%
\pgfpathlineto{\pgfqpoint{5.018349in}{1.293915in}}%
\pgfusepath{stroke,fill}%
\end{pgfscope}%
\begin{pgfscope}%
\pgfpathrectangle{\pgfqpoint{2.000000in}{0.750000in}}{\pgfqpoint{4.376471in}{0.953684in}} %
\pgfusepath{clip}%
\pgfsetbuttcap%
\pgfsetroundjoin%
\definecolor{currentfill}{rgb}{1.000000,0.000000,0.000000}%
\pgfsetfillcolor{currentfill}%
\pgfsetlinewidth{2.007500pt}%
\definecolor{currentstroke}{rgb}{1.000000,0.000000,0.000000}%
\pgfsetstrokecolor{currentstroke}%
\pgfsetdash{}{0pt}%
\pgfpathmoveto{\pgfqpoint{4.993622in}{1.277875in}}%
\pgfpathlineto{\pgfqpoint{5.076956in}{1.277875in}}%
\pgfpathmoveto{\pgfqpoint{5.035289in}{1.236208in}}%
\pgfpathlineto{\pgfqpoint{5.035289in}{1.319542in}}%
\pgfusepath{stroke,fill}%
\end{pgfscope}%
\begin{pgfscope}%
\pgfpathrectangle{\pgfqpoint{2.000000in}{0.750000in}}{\pgfqpoint{4.376471in}{0.953684in}} %
\pgfusepath{clip}%
\pgfsetbuttcap%
\pgfsetroundjoin%
\definecolor{currentfill}{rgb}{1.000000,0.000000,0.000000}%
\pgfsetfillcolor{currentfill}%
\pgfsetlinewidth{2.007500pt}%
\definecolor{currentstroke}{rgb}{1.000000,0.000000,0.000000}%
\pgfsetstrokecolor{currentstroke}%
\pgfsetdash{}{0pt}%
\pgfpathmoveto{\pgfqpoint{6.137856in}{1.408689in}}%
\pgfpathlineto{\pgfqpoint{6.221189in}{1.408689in}}%
\pgfpathmoveto{\pgfqpoint{6.179523in}{1.367022in}}%
\pgfpathlineto{\pgfqpoint{6.179523in}{1.450355in}}%
\pgfusepath{stroke,fill}%
\end{pgfscope}%
\begin{pgfscope}%
\pgfpathrectangle{\pgfqpoint{2.000000in}{0.750000in}}{\pgfqpoint{4.376471in}{0.953684in}} %
\pgfusepath{clip}%
\pgfsetbuttcap%
\pgfsetroundjoin%
\definecolor{currentfill}{rgb}{1.000000,0.000000,0.000000}%
\pgfsetfillcolor{currentfill}%
\pgfsetlinewidth{2.007500pt}%
\definecolor{currentstroke}{rgb}{1.000000,0.000000,0.000000}%
\pgfsetstrokecolor{currentstroke}%
\pgfsetdash{}{0pt}%
\pgfpathmoveto{\pgfqpoint{5.220801in}{1.155319in}}%
\pgfpathlineto{\pgfqpoint{5.304134in}{1.155319in}}%
\pgfpathmoveto{\pgfqpoint{5.262467in}{1.113652in}}%
\pgfpathlineto{\pgfqpoint{5.262467in}{1.196985in}}%
\pgfusepath{stroke,fill}%
\end{pgfscope}%
\begin{pgfscope}%
\pgfpathrectangle{\pgfqpoint{2.000000in}{0.750000in}}{\pgfqpoint{4.376471in}{0.953684in}} %
\pgfusepath{clip}%
\pgfsetbuttcap%
\pgfsetroundjoin%
\definecolor{currentfill}{rgb}{1.000000,0.000000,0.000000}%
\pgfsetfillcolor{currentfill}%
\pgfsetlinewidth{2.007500pt}%
\definecolor{currentstroke}{rgb}{1.000000,0.000000,0.000000}%
\pgfsetstrokecolor{currentstroke}%
\pgfsetdash{}{0pt}%
\pgfpathmoveto{\pgfqpoint{4.092328in}{1.120553in}}%
\pgfpathlineto{\pgfqpoint{4.175661in}{1.120553in}}%
\pgfpathmoveto{\pgfqpoint{4.133995in}{1.078887in}}%
\pgfpathlineto{\pgfqpoint{4.133995in}{1.162220in}}%
\pgfusepath{stroke,fill}%
\end{pgfscope}%
\begin{pgfscope}%
\pgfpathrectangle{\pgfqpoint{2.000000in}{0.750000in}}{\pgfqpoint{4.376471in}{0.953684in}} %
\pgfusepath{clip}%
\pgfsetbuttcap%
\pgfsetroundjoin%
\definecolor{currentfill}{rgb}{1.000000,0.000000,0.000000}%
\pgfsetfillcolor{currentfill}%
\pgfsetlinewidth{2.007500pt}%
\definecolor{currentstroke}{rgb}{1.000000,0.000000,0.000000}%
\pgfsetstrokecolor{currentstroke}%
\pgfsetdash{}{0pt}%
\pgfpathmoveto{\pgfqpoint{4.363753in}{0.772194in}}%
\pgfpathlineto{\pgfqpoint{4.447087in}{0.772194in}}%
\pgfpathmoveto{\pgfqpoint{4.405420in}{0.730527in}}%
\pgfpathlineto{\pgfqpoint{4.405420in}{0.813860in}}%
\pgfusepath{stroke,fill}%
\end{pgfscope}%
\begin{pgfscope}%
\pgfpathrectangle{\pgfqpoint{2.000000in}{0.750000in}}{\pgfqpoint{4.376471in}{0.953684in}} %
\pgfusepath{clip}%
\pgfsetbuttcap%
\pgfsetroundjoin%
\definecolor{currentfill}{rgb}{1.000000,0.000000,0.000000}%
\pgfsetfillcolor{currentfill}%
\pgfsetlinewidth{2.007500pt}%
\definecolor{currentstroke}{rgb}{1.000000,0.000000,0.000000}%
\pgfsetstrokecolor{currentstroke}%
\pgfsetdash{}{0pt}%
\pgfpathmoveto{\pgfqpoint{5.276157in}{1.173514in}}%
\pgfpathlineto{\pgfqpoint{5.359491in}{1.173514in}}%
\pgfpathmoveto{\pgfqpoint{5.317824in}{1.131847in}}%
\pgfpathlineto{\pgfqpoint{5.317824in}{1.215181in}}%
\pgfusepath{stroke,fill}%
\end{pgfscope}%
\begin{pgfscope}%
\pgfpathrectangle{\pgfqpoint{2.000000in}{0.750000in}}{\pgfqpoint{4.376471in}{0.953684in}} %
\pgfusepath{clip}%
\pgfsetbuttcap%
\pgfsetroundjoin%
\definecolor{currentfill}{rgb}{1.000000,0.000000,0.000000}%
\pgfsetfillcolor{currentfill}%
\pgfsetlinewidth{2.007500pt}%
\definecolor{currentstroke}{rgb}{1.000000,0.000000,0.000000}%
\pgfsetstrokecolor{currentstroke}%
\pgfsetdash{}{0pt}%
\pgfpathmoveto{\pgfqpoint{3.044487in}{1.486441in}}%
\pgfpathlineto{\pgfqpoint{3.127821in}{1.486441in}}%
\pgfpathmoveto{\pgfqpoint{3.086154in}{1.444774in}}%
\pgfpathlineto{\pgfqpoint{3.086154in}{1.528107in}}%
\pgfusepath{stroke,fill}%
\end{pgfscope}%
\begin{pgfscope}%
\pgfpathrectangle{\pgfqpoint{2.000000in}{0.750000in}}{\pgfqpoint{4.376471in}{0.953684in}} %
\pgfusepath{clip}%
\pgfsetbuttcap%
\pgfsetroundjoin%
\definecolor{currentfill}{rgb}{1.000000,0.000000,0.000000}%
\pgfsetfillcolor{currentfill}%
\pgfsetlinewidth{2.007500pt}%
\definecolor{currentstroke}{rgb}{1.000000,0.000000,0.000000}%
\pgfsetstrokecolor{currentstroke}%
\pgfsetdash{}{0pt}%
\pgfpathmoveto{\pgfqpoint{5.168095in}{1.230686in}}%
\pgfpathlineto{\pgfqpoint{5.251429in}{1.230686in}}%
\pgfpathmoveto{\pgfqpoint{5.209762in}{1.189019in}}%
\pgfpathlineto{\pgfqpoint{5.209762in}{1.272353in}}%
\pgfusepath{stroke,fill}%
\end{pgfscope}%
\begin{pgfscope}%
\pgfpathrectangle{\pgfqpoint{2.000000in}{0.750000in}}{\pgfqpoint{4.376471in}{0.953684in}} %
\pgfusepath{clip}%
\pgfsetbuttcap%
\pgfsetroundjoin%
\definecolor{currentfill}{rgb}{1.000000,0.000000,0.000000}%
\pgfsetfillcolor{currentfill}%
\pgfsetlinewidth{2.007500pt}%
\definecolor{currentstroke}{rgb}{1.000000,0.000000,0.000000}%
\pgfsetstrokecolor{currentstroke}%
\pgfsetdash{}{0pt}%
\pgfpathmoveto{\pgfqpoint{5.181649in}{1.082820in}}%
\pgfpathlineto{\pgfqpoint{5.264982in}{1.082820in}}%
\pgfpathmoveto{\pgfqpoint{5.223316in}{1.041154in}}%
\pgfpathlineto{\pgfqpoint{5.223316in}{1.124487in}}%
\pgfusepath{stroke,fill}%
\end{pgfscope}%
\begin{pgfscope}%
\pgfpathrectangle{\pgfqpoint{2.000000in}{0.750000in}}{\pgfqpoint{4.376471in}{0.953684in}} %
\pgfusepath{clip}%
\pgfsetbuttcap%
\pgfsetroundjoin%
\definecolor{currentfill}{rgb}{1.000000,0.000000,0.000000}%
\pgfsetfillcolor{currentfill}%
\pgfsetlinewidth{2.007500pt}%
\definecolor{currentstroke}{rgb}{1.000000,0.000000,0.000000}%
\pgfsetstrokecolor{currentstroke}%
\pgfsetdash{}{0pt}%
\pgfpathmoveto{\pgfqpoint{3.570214in}{1.228236in}}%
\pgfpathlineto{\pgfqpoint{3.653547in}{1.228236in}}%
\pgfpathmoveto{\pgfqpoint{3.611881in}{1.186569in}}%
\pgfpathlineto{\pgfqpoint{3.611881in}{1.269902in}}%
\pgfusepath{stroke,fill}%
\end{pgfscope}%
\begin{pgfscope}%
\pgfpathrectangle{\pgfqpoint{2.000000in}{0.750000in}}{\pgfqpoint{4.376471in}{0.953684in}} %
\pgfusepath{clip}%
\pgfsetbuttcap%
\pgfsetroundjoin%
\definecolor{currentfill}{rgb}{1.000000,0.000000,0.000000}%
\pgfsetfillcolor{currentfill}%
\pgfsetlinewidth{2.007500pt}%
\definecolor{currentstroke}{rgb}{1.000000,0.000000,0.000000}%
\pgfsetstrokecolor{currentstroke}%
\pgfsetdash{}{0pt}%
\pgfpathmoveto{\pgfqpoint{3.285021in}{1.250858in}}%
\pgfpathlineto{\pgfqpoint{3.368355in}{1.250858in}}%
\pgfpathmoveto{\pgfqpoint{3.326688in}{1.209191in}}%
\pgfpathlineto{\pgfqpoint{3.326688in}{1.292524in}}%
\pgfusepath{stroke,fill}%
\end{pgfscope}%
\begin{pgfscope}%
\pgfpathrectangle{\pgfqpoint{2.000000in}{0.750000in}}{\pgfqpoint{4.376471in}{0.953684in}} %
\pgfusepath{clip}%
\pgfsetbuttcap%
\pgfsetroundjoin%
\definecolor{currentfill}{rgb}{1.000000,0.000000,0.000000}%
\pgfsetfillcolor{currentfill}%
\pgfsetlinewidth{2.007500pt}%
\definecolor{currentstroke}{rgb}{1.000000,0.000000,0.000000}%
\pgfsetstrokecolor{currentstroke}%
\pgfsetdash{}{0pt}%
\pgfpathmoveto{\pgfqpoint{3.937998in}{1.225550in}}%
\pgfpathlineto{\pgfqpoint{4.021331in}{1.225550in}}%
\pgfpathmoveto{\pgfqpoint{3.979664in}{1.183883in}}%
\pgfpathlineto{\pgfqpoint{3.979664in}{1.267217in}}%
\pgfusepath{stroke,fill}%
\end{pgfscope}%
\begin{pgfscope}%
\pgfpathrectangle{\pgfqpoint{2.000000in}{0.750000in}}{\pgfqpoint{4.376471in}{0.953684in}} %
\pgfusepath{clip}%
\pgfsetbuttcap%
\pgfsetroundjoin%
\definecolor{currentfill}{rgb}{1.000000,0.000000,0.000000}%
\pgfsetfillcolor{currentfill}%
\pgfsetlinewidth{2.007500pt}%
\definecolor{currentstroke}{rgb}{1.000000,0.000000,0.000000}%
\pgfsetstrokecolor{currentstroke}%
\pgfsetdash{}{0pt}%
\pgfpathmoveto{\pgfqpoint{4.107043in}{1.178342in}}%
\pgfpathlineto{\pgfqpoint{4.190376in}{1.178342in}}%
\pgfpathmoveto{\pgfqpoint{4.148710in}{1.136675in}}%
\pgfpathlineto{\pgfqpoint{4.148710in}{1.220008in}}%
\pgfusepath{stroke,fill}%
\end{pgfscope}%
\begin{pgfscope}%
\pgfpathrectangle{\pgfqpoint{2.000000in}{0.750000in}}{\pgfqpoint{4.376471in}{0.953684in}} %
\pgfusepath{clip}%
\pgfsetbuttcap%
\pgfsetroundjoin%
\definecolor{currentfill}{rgb}{0.000000,0.000000,0.000000}%
\pgfsetfillcolor{currentfill}%
\pgfsetlinewidth{1.003750pt}%
\definecolor{currentstroke}{rgb}{0.000000,0.000000,0.000000}%
\pgfsetstrokecolor{currentstroke}%
\pgfsetdash{}{0pt}%
\pgfsys@defobject{currentmarker}{\pgfqpoint{-0.020833in}{-0.020833in}}{\pgfqpoint{0.020833in}{0.020833in}}{%
\pgfpathmoveto{\pgfqpoint{0.000000in}{-0.020833in}}%
\pgfpathcurveto{\pgfqpoint{0.005525in}{-0.020833in}}{\pgfqpoint{0.010825in}{-0.018638in}}{\pgfqpoint{0.014731in}{-0.014731in}}%
\pgfpathcurveto{\pgfqpoint{0.018638in}{-0.010825in}}{\pgfqpoint{0.020833in}{-0.005525in}}{\pgfqpoint{0.020833in}{0.000000in}}%
\pgfpathcurveto{\pgfqpoint{0.020833in}{0.005525in}}{\pgfqpoint{0.018638in}{0.010825in}}{\pgfqpoint{0.014731in}{0.014731in}}%
\pgfpathcurveto{\pgfqpoint{0.010825in}{0.018638in}}{\pgfqpoint{0.005525in}{0.020833in}}{\pgfqpoint{0.000000in}{0.020833in}}%
\pgfpathcurveto{\pgfqpoint{-0.005525in}{0.020833in}}{\pgfqpoint{-0.010825in}{0.018638in}}{\pgfqpoint{-0.014731in}{0.014731in}}%
\pgfpathcurveto{\pgfqpoint{-0.018638in}{0.010825in}}{\pgfqpoint{-0.020833in}{0.005525in}}{\pgfqpoint{-0.020833in}{0.000000in}}%
\pgfpathcurveto{\pgfqpoint{-0.020833in}{-0.005525in}}{\pgfqpoint{-0.018638in}{-0.010825in}}{\pgfqpoint{-0.014731in}{-0.014731in}}%
\pgfpathcurveto{\pgfqpoint{-0.010825in}{-0.018638in}}{\pgfqpoint{-0.005525in}{-0.020833in}}{\pgfqpoint{0.000000in}{-0.020833in}}%
\pgfpathclose%
\pgfusepath{stroke,fill}%
}%
\begin{pgfscope}%
\pgfsys@transformshift{2.875294in}{1.633779in}%
\pgfsys@useobject{currentmarker}{}%
\end{pgfscope}%
\begin{pgfscope}%
\pgfsys@transformshift{2.892888in}{1.684891in}%
\pgfsys@useobject{currentmarker}{}%
\end{pgfscope}%
\begin{pgfscope}%
\pgfsys@transformshift{2.910482in}{1.597216in}%
\pgfsys@useobject{currentmarker}{}%
\end{pgfscope}%
\begin{pgfscope}%
\pgfsys@transformshift{2.928076in}{1.605298in}%
\pgfsys@useobject{currentmarker}{}%
\end{pgfscope}%
\begin{pgfscope}%
\pgfsys@transformshift{2.945670in}{1.508979in}%
\pgfsys@useobject{currentmarker}{}%
\end{pgfscope}%
\begin{pgfscope}%
\pgfsys@transformshift{2.963263in}{1.657913in}%
\pgfsys@useobject{currentmarker}{}%
\end{pgfscope}%
\begin{pgfscope}%
\pgfsys@transformshift{2.980857in}{1.465607in}%
\pgfsys@useobject{currentmarker}{}%
\end{pgfscope}%
\begin{pgfscope}%
\pgfsys@transformshift{2.998451in}{1.465570in}%
\pgfsys@useobject{currentmarker}{}%
\end{pgfscope}%
\begin{pgfscope}%
\pgfsys@transformshift{3.016045in}{1.585684in}%
\pgfsys@useobject{currentmarker}{}%
\end{pgfscope}%
\begin{pgfscope}%
\pgfsys@transformshift{3.033639in}{1.238225in}%
\pgfsys@useobject{currentmarker}{}%
\end{pgfscope}%
\begin{pgfscope}%
\pgfsys@transformshift{3.051233in}{1.220138in}%
\pgfsys@useobject{currentmarker}{}%
\end{pgfscope}%
\begin{pgfscope}%
\pgfsys@transformshift{3.068826in}{1.417804in}%
\pgfsys@useobject{currentmarker}{}%
\end{pgfscope}%
\begin{pgfscope}%
\pgfsys@transformshift{3.086420in}{1.181208in}%
\pgfsys@useobject{currentmarker}{}%
\end{pgfscope}%
\begin{pgfscope}%
\pgfsys@transformshift{3.104014in}{1.468341in}%
\pgfsys@useobject{currentmarker}{}%
\end{pgfscope}%
\begin{pgfscope}%
\pgfsys@transformshift{3.121608in}{1.213097in}%
\pgfsys@useobject{currentmarker}{}%
\end{pgfscope}%
\begin{pgfscope}%
\pgfsys@transformshift{3.139202in}{1.160424in}%
\pgfsys@useobject{currentmarker}{}%
\end{pgfscope}%
\begin{pgfscope}%
\pgfsys@transformshift{3.156796in}{1.407888in}%
\pgfsys@useobject{currentmarker}{}%
\end{pgfscope}%
\begin{pgfscope}%
\pgfsys@transformshift{3.174390in}{1.347920in}%
\pgfsys@useobject{currentmarker}{}%
\end{pgfscope}%
\begin{pgfscope}%
\pgfsys@transformshift{3.191983in}{1.372250in}%
\pgfsys@useobject{currentmarker}{}%
\end{pgfscope}%
\begin{pgfscope}%
\pgfsys@transformshift{3.209577in}{1.264584in}%
\pgfsys@useobject{currentmarker}{}%
\end{pgfscope}%
\begin{pgfscope}%
\pgfsys@transformshift{3.227171in}{1.078902in}%
\pgfsys@useobject{currentmarker}{}%
\end{pgfscope}%
\begin{pgfscope}%
\pgfsys@transformshift{3.244765in}{1.346158in}%
\pgfsys@useobject{currentmarker}{}%
\end{pgfscope}%
\begin{pgfscope}%
\pgfsys@transformshift{3.262359in}{1.123818in}%
\pgfsys@useobject{currentmarker}{}%
\end{pgfscope}%
\begin{pgfscope}%
\pgfsys@transformshift{3.279953in}{1.226274in}%
\pgfsys@useobject{currentmarker}{}%
\end{pgfscope}%
\begin{pgfscope}%
\pgfsys@transformshift{3.297547in}{1.238831in}%
\pgfsys@useobject{currentmarker}{}%
\end{pgfscope}%
\begin{pgfscope}%
\pgfsys@transformshift{3.315140in}{1.129518in}%
\pgfsys@useobject{currentmarker}{}%
\end{pgfscope}%
\begin{pgfscope}%
\pgfsys@transformshift{3.332734in}{1.208061in}%
\pgfsys@useobject{currentmarker}{}%
\end{pgfscope}%
\begin{pgfscope}%
\pgfsys@transformshift{3.350328in}{1.242682in}%
\pgfsys@useobject{currentmarker}{}%
\end{pgfscope}%
\begin{pgfscope}%
\pgfsys@transformshift{3.367922in}{1.194245in}%
\pgfsys@useobject{currentmarker}{}%
\end{pgfscope}%
\begin{pgfscope}%
\pgfsys@transformshift{3.385516in}{1.055069in}%
\pgfsys@useobject{currentmarker}{}%
\end{pgfscope}%
\begin{pgfscope}%
\pgfsys@transformshift{3.403110in}{1.202872in}%
\pgfsys@useobject{currentmarker}{}%
\end{pgfscope}%
\begin{pgfscope}%
\pgfsys@transformshift{3.420704in}{1.315343in}%
\pgfsys@useobject{currentmarker}{}%
\end{pgfscope}%
\begin{pgfscope}%
\pgfsys@transformshift{3.438297in}{1.126125in}%
\pgfsys@useobject{currentmarker}{}%
\end{pgfscope}%
\begin{pgfscope}%
\pgfsys@transformshift{3.455891in}{1.192850in}%
\pgfsys@useobject{currentmarker}{}%
\end{pgfscope}%
\begin{pgfscope}%
\pgfsys@transformshift{3.473485in}{1.177937in}%
\pgfsys@useobject{currentmarker}{}%
\end{pgfscope}%
\begin{pgfscope}%
\pgfsys@transformshift{3.491079in}{1.419000in}%
\pgfsys@useobject{currentmarker}{}%
\end{pgfscope}%
\begin{pgfscope}%
\pgfsys@transformshift{3.508673in}{1.316679in}%
\pgfsys@useobject{currentmarker}{}%
\end{pgfscope}%
\begin{pgfscope}%
\pgfsys@transformshift{3.526267in}{1.305249in}%
\pgfsys@useobject{currentmarker}{}%
\end{pgfscope}%
\begin{pgfscope}%
\pgfsys@transformshift{3.543860in}{1.203242in}%
\pgfsys@useobject{currentmarker}{}%
\end{pgfscope}%
\begin{pgfscope}%
\pgfsys@transformshift{3.561454in}{1.348128in}%
\pgfsys@useobject{currentmarker}{}%
\end{pgfscope}%
\begin{pgfscope}%
\pgfsys@transformshift{3.579048in}{1.242076in}%
\pgfsys@useobject{currentmarker}{}%
\end{pgfscope}%
\begin{pgfscope}%
\pgfsys@transformshift{3.596642in}{1.326217in}%
\pgfsys@useobject{currentmarker}{}%
\end{pgfscope}%
\begin{pgfscope}%
\pgfsys@transformshift{3.614236in}{1.273171in}%
\pgfsys@useobject{currentmarker}{}%
\end{pgfscope}%
\begin{pgfscope}%
\pgfsys@transformshift{3.631830in}{1.415928in}%
\pgfsys@useobject{currentmarker}{}%
\end{pgfscope}%
\begin{pgfscope}%
\pgfsys@transformshift{3.649424in}{1.417343in}%
\pgfsys@useobject{currentmarker}{}%
\end{pgfscope}%
\begin{pgfscope}%
\pgfsys@transformshift{3.667017in}{1.349484in}%
\pgfsys@useobject{currentmarker}{}%
\end{pgfscope}%
\begin{pgfscope}%
\pgfsys@transformshift{3.684611in}{1.418264in}%
\pgfsys@useobject{currentmarker}{}%
\end{pgfscope}%
\begin{pgfscope}%
\pgfsys@transformshift{3.702205in}{1.277586in}%
\pgfsys@useobject{currentmarker}{}%
\end{pgfscope}%
\begin{pgfscope}%
\pgfsys@transformshift{3.719799in}{1.243663in}%
\pgfsys@useobject{currentmarker}{}%
\end{pgfscope}%
\begin{pgfscope}%
\pgfsys@transformshift{3.737393in}{1.439380in}%
\pgfsys@useobject{currentmarker}{}%
\end{pgfscope}%
\begin{pgfscope}%
\pgfsys@transformshift{3.754987in}{1.414414in}%
\pgfsys@useobject{currentmarker}{}%
\end{pgfscope}%
\begin{pgfscope}%
\pgfsys@transformshift{3.772581in}{1.461213in}%
\pgfsys@useobject{currentmarker}{}%
\end{pgfscope}%
\begin{pgfscope}%
\pgfsys@transformshift{3.790174in}{1.633247in}%
\pgfsys@useobject{currentmarker}{}%
\end{pgfscope}%
\begin{pgfscope}%
\pgfsys@transformshift{3.807768in}{1.486778in}%
\pgfsys@useobject{currentmarker}{}%
\end{pgfscope}%
\begin{pgfscope}%
\pgfsys@transformshift{3.825362in}{1.296797in}%
\pgfsys@useobject{currentmarker}{}%
\end{pgfscope}%
\begin{pgfscope}%
\pgfsys@transformshift{3.842956in}{1.491068in}%
\pgfsys@useobject{currentmarker}{}%
\end{pgfscope}%
\begin{pgfscope}%
\pgfsys@transformshift{3.860550in}{1.240132in}%
\pgfsys@useobject{currentmarker}{}%
\end{pgfscope}%
\begin{pgfscope}%
\pgfsys@transformshift{3.878144in}{1.313947in}%
\pgfsys@useobject{currentmarker}{}%
\end{pgfscope}%
\begin{pgfscope}%
\pgfsys@transformshift{3.895738in}{1.340235in}%
\pgfsys@useobject{currentmarker}{}%
\end{pgfscope}%
\begin{pgfscope}%
\pgfsys@transformshift{3.913331in}{1.502795in}%
\pgfsys@useobject{currentmarker}{}%
\end{pgfscope}%
\begin{pgfscope}%
\pgfsys@transformshift{3.930925in}{1.242651in}%
\pgfsys@useobject{currentmarker}{}%
\end{pgfscope}%
\begin{pgfscope}%
\pgfsys@transformshift{3.948519in}{1.217188in}%
\pgfsys@useobject{currentmarker}{}%
\end{pgfscope}%
\begin{pgfscope}%
\pgfsys@transformshift{3.966113in}{1.270906in}%
\pgfsys@useobject{currentmarker}{}%
\end{pgfscope}%
\begin{pgfscope}%
\pgfsys@transformshift{3.983707in}{1.195077in}%
\pgfsys@useobject{currentmarker}{}%
\end{pgfscope}%
\begin{pgfscope}%
\pgfsys@transformshift{4.001301in}{1.352284in}%
\pgfsys@useobject{currentmarker}{}%
\end{pgfscope}%
\begin{pgfscope}%
\pgfsys@transformshift{4.018894in}{1.111704in}%
\pgfsys@useobject{currentmarker}{}%
\end{pgfscope}%
\begin{pgfscope}%
\pgfsys@transformshift{4.036488in}{1.083058in}%
\pgfsys@useobject{currentmarker}{}%
\end{pgfscope}%
\begin{pgfscope}%
\pgfsys@transformshift{4.054082in}{1.131252in}%
\pgfsys@useobject{currentmarker}{}%
\end{pgfscope}%
\begin{pgfscope}%
\pgfsys@transformshift{4.071676in}{1.102797in}%
\pgfsys@useobject{currentmarker}{}%
\end{pgfscope}%
\begin{pgfscope}%
\pgfsys@transformshift{4.089270in}{1.321515in}%
\pgfsys@useobject{currentmarker}{}%
\end{pgfscope}%
\begin{pgfscope}%
\pgfsys@transformshift{4.106864in}{1.201940in}%
\pgfsys@useobject{currentmarker}{}%
\end{pgfscope}%
\begin{pgfscope}%
\pgfsys@transformshift{4.124458in}{1.094544in}%
\pgfsys@useobject{currentmarker}{}%
\end{pgfscope}%
\begin{pgfscope}%
\pgfsys@transformshift{4.142051in}{0.942963in}%
\pgfsys@useobject{currentmarker}{}%
\end{pgfscope}%
\begin{pgfscope}%
\pgfsys@transformshift{4.159645in}{1.128254in}%
\pgfsys@useobject{currentmarker}{}%
\end{pgfscope}%
\begin{pgfscope}%
\pgfsys@transformshift{4.177239in}{0.925687in}%
\pgfsys@useobject{currentmarker}{}%
\end{pgfscope}%
\begin{pgfscope}%
\pgfsys@transformshift{4.194833in}{0.853443in}%
\pgfsys@useobject{currentmarker}{}%
\end{pgfscope}%
\begin{pgfscope}%
\pgfsys@transformshift{4.212427in}{1.108119in}%
\pgfsys@useobject{currentmarker}{}%
\end{pgfscope}%
\begin{pgfscope}%
\pgfsys@transformshift{4.230021in}{1.006267in}%
\pgfsys@useobject{currentmarker}{}%
\end{pgfscope}%
\begin{pgfscope}%
\pgfsys@transformshift{4.247615in}{1.052571in}%
\pgfsys@useobject{currentmarker}{}%
\end{pgfscope}%
\begin{pgfscope}%
\pgfsys@transformshift{4.265208in}{0.980810in}%
\pgfsys@useobject{currentmarker}{}%
\end{pgfscope}%
\begin{pgfscope}%
\pgfsys@transformshift{4.282802in}{1.024173in}%
\pgfsys@useobject{currentmarker}{}%
\end{pgfscope}%
\begin{pgfscope}%
\pgfsys@transformshift{4.300396in}{0.866221in}%
\pgfsys@useobject{currentmarker}{}%
\end{pgfscope}%
\begin{pgfscope}%
\pgfsys@transformshift{4.317990in}{0.821996in}%
\pgfsys@useobject{currentmarker}{}%
\end{pgfscope}%
\begin{pgfscope}%
\pgfsys@transformshift{4.335584in}{0.988346in}%
\pgfsys@useobject{currentmarker}{}%
\end{pgfscope}%
\begin{pgfscope}%
\pgfsys@transformshift{4.353178in}{0.838795in}%
\pgfsys@useobject{currentmarker}{}%
\end{pgfscope}%
\begin{pgfscope}%
\pgfsys@transformshift{4.370772in}{0.850163in}%
\pgfsys@useobject{currentmarker}{}%
\end{pgfscope}%
\begin{pgfscope}%
\pgfsys@transformshift{4.388365in}{0.875541in}%
\pgfsys@useobject{currentmarker}{}%
\end{pgfscope}%
\begin{pgfscope}%
\pgfsys@transformshift{4.405959in}{0.926701in}%
\pgfsys@useobject{currentmarker}{}%
\end{pgfscope}%
\begin{pgfscope}%
\pgfsys@transformshift{4.423553in}{0.895955in}%
\pgfsys@useobject{currentmarker}{}%
\end{pgfscope}%
\begin{pgfscope}%
\pgfsys@transformshift{4.441147in}{0.802604in}%
\pgfsys@useobject{currentmarker}{}%
\end{pgfscope}%
\begin{pgfscope}%
\pgfsys@transformshift{4.458741in}{0.885153in}%
\pgfsys@useobject{currentmarker}{}%
\end{pgfscope}%
\begin{pgfscope}%
\pgfsys@transformshift{4.476335in}{0.739828in}%
\pgfsys@useobject{currentmarker}{}%
\end{pgfscope}%
\begin{pgfscope}%
\pgfsys@transformshift{4.493928in}{1.036023in}%
\pgfsys@useobject{currentmarker}{}%
\end{pgfscope}%
\begin{pgfscope}%
\pgfsys@transformshift{4.511522in}{0.829419in}%
\pgfsys@useobject{currentmarker}{}%
\end{pgfscope}%
\begin{pgfscope}%
\pgfsys@transformshift{4.529116in}{0.894817in}%
\pgfsys@useobject{currentmarker}{}%
\end{pgfscope}%
\begin{pgfscope}%
\pgfsys@transformshift{4.546710in}{1.026801in}%
\pgfsys@useobject{currentmarker}{}%
\end{pgfscope}%
\begin{pgfscope}%
\pgfsys@transformshift{4.564304in}{0.966197in}%
\pgfsys@useobject{currentmarker}{}%
\end{pgfscope}%
\begin{pgfscope}%
\pgfsys@transformshift{4.581898in}{1.211772in}%
\pgfsys@useobject{currentmarker}{}%
\end{pgfscope}%
\begin{pgfscope}%
\pgfsys@transformshift{4.599492in}{0.949448in}%
\pgfsys@useobject{currentmarker}{}%
\end{pgfscope}%
\begin{pgfscope}%
\pgfsys@transformshift{4.617085in}{1.124191in}%
\pgfsys@useobject{currentmarker}{}%
\end{pgfscope}%
\begin{pgfscope}%
\pgfsys@transformshift{4.634679in}{1.113719in}%
\pgfsys@useobject{currentmarker}{}%
\end{pgfscope}%
\begin{pgfscope}%
\pgfsys@transformshift{4.652273in}{1.021520in}%
\pgfsys@useobject{currentmarker}{}%
\end{pgfscope}%
\begin{pgfscope}%
\pgfsys@transformshift{4.669867in}{1.209263in}%
\pgfsys@useobject{currentmarker}{}%
\end{pgfscope}%
\begin{pgfscope}%
\pgfsys@transformshift{4.687461in}{1.159538in}%
\pgfsys@useobject{currentmarker}{}%
\end{pgfscope}%
\begin{pgfscope}%
\pgfsys@transformshift{4.705055in}{1.271941in}%
\pgfsys@useobject{currentmarker}{}%
\end{pgfscope}%
\begin{pgfscope}%
\pgfsys@transformshift{4.722649in}{1.294972in}%
\pgfsys@useobject{currentmarker}{}%
\end{pgfscope}%
\begin{pgfscope}%
\pgfsys@transformshift{4.740242in}{1.444985in}%
\pgfsys@useobject{currentmarker}{}%
\end{pgfscope}%
\begin{pgfscope}%
\pgfsys@transformshift{4.757836in}{1.378641in}%
\pgfsys@useobject{currentmarker}{}%
\end{pgfscope}%
\begin{pgfscope}%
\pgfsys@transformshift{4.775430in}{1.223682in}%
\pgfsys@useobject{currentmarker}{}%
\end{pgfscope}%
\begin{pgfscope}%
\pgfsys@transformshift{4.793024in}{1.249636in}%
\pgfsys@useobject{currentmarker}{}%
\end{pgfscope}%
\begin{pgfscope}%
\pgfsys@transformshift{4.810618in}{1.394162in}%
\pgfsys@useobject{currentmarker}{}%
\end{pgfscope}%
\begin{pgfscope}%
\pgfsys@transformshift{4.828212in}{1.359853in}%
\pgfsys@useobject{currentmarker}{}%
\end{pgfscope}%
\begin{pgfscope}%
\pgfsys@transformshift{4.845805in}{1.366433in}%
\pgfsys@useobject{currentmarker}{}%
\end{pgfscope}%
\begin{pgfscope}%
\pgfsys@transformshift{4.863399in}{1.148466in}%
\pgfsys@useobject{currentmarker}{}%
\end{pgfscope}%
\begin{pgfscope}%
\pgfsys@transformshift{4.880993in}{1.311059in}%
\pgfsys@useobject{currentmarker}{}%
\end{pgfscope}%
\begin{pgfscope}%
\pgfsys@transformshift{4.898587in}{1.242624in}%
\pgfsys@useobject{currentmarker}{}%
\end{pgfscope}%
\begin{pgfscope}%
\pgfsys@transformshift{4.916181in}{1.344291in}%
\pgfsys@useobject{currentmarker}{}%
\end{pgfscope}%
\begin{pgfscope}%
\pgfsys@transformshift{4.933775in}{1.305357in}%
\pgfsys@useobject{currentmarker}{}%
\end{pgfscope}%
\begin{pgfscope}%
\pgfsys@transformshift{4.951369in}{1.402261in}%
\pgfsys@useobject{currentmarker}{}%
\end{pgfscope}%
\begin{pgfscope}%
\pgfsys@transformshift{4.968962in}{1.338247in}%
\pgfsys@useobject{currentmarker}{}%
\end{pgfscope}%
\begin{pgfscope}%
\pgfsys@transformshift{4.986556in}{1.377967in}%
\pgfsys@useobject{currentmarker}{}%
\end{pgfscope}%
\begin{pgfscope}%
\pgfsys@transformshift{5.004150in}{1.244944in}%
\pgfsys@useobject{currentmarker}{}%
\end{pgfscope}%
\begin{pgfscope}%
\pgfsys@transformshift{5.021744in}{1.187298in}%
\pgfsys@useobject{currentmarker}{}%
\end{pgfscope}%
\begin{pgfscope}%
\pgfsys@transformshift{5.039338in}{1.228762in}%
\pgfsys@useobject{currentmarker}{}%
\end{pgfscope}%
\begin{pgfscope}%
\pgfsys@transformshift{5.056932in}{1.254853in}%
\pgfsys@useobject{currentmarker}{}%
\end{pgfscope}%
\begin{pgfscope}%
\pgfsys@transformshift{5.074526in}{1.280077in}%
\pgfsys@useobject{currentmarker}{}%
\end{pgfscope}%
\begin{pgfscope}%
\pgfsys@transformshift{5.092119in}{1.451707in}%
\pgfsys@useobject{currentmarker}{}%
\end{pgfscope}%
\begin{pgfscope}%
\pgfsys@transformshift{5.109713in}{1.207072in}%
\pgfsys@useobject{currentmarker}{}%
\end{pgfscope}%
\begin{pgfscope}%
\pgfsys@transformshift{5.127307in}{1.099654in}%
\pgfsys@useobject{currentmarker}{}%
\end{pgfscope}%
\begin{pgfscope}%
\pgfsys@transformshift{5.144901in}{1.143129in}%
\pgfsys@useobject{currentmarker}{}%
\end{pgfscope}%
\begin{pgfscope}%
\pgfsys@transformshift{5.162495in}{1.114053in}%
\pgfsys@useobject{currentmarker}{}%
\end{pgfscope}%
\begin{pgfscope}%
\pgfsys@transformshift{5.180089in}{1.190474in}%
\pgfsys@useobject{currentmarker}{}%
\end{pgfscope}%
\begin{pgfscope}%
\pgfsys@transformshift{5.197683in}{0.972241in}%
\pgfsys@useobject{currentmarker}{}%
\end{pgfscope}%
\begin{pgfscope}%
\pgfsys@transformshift{5.215276in}{1.114586in}%
\pgfsys@useobject{currentmarker}{}%
\end{pgfscope}%
\begin{pgfscope}%
\pgfsys@transformshift{5.232870in}{1.107371in}%
\pgfsys@useobject{currentmarker}{}%
\end{pgfscope}%
\begin{pgfscope}%
\pgfsys@transformshift{5.250464in}{1.099133in}%
\pgfsys@useobject{currentmarker}{}%
\end{pgfscope}%
\begin{pgfscope}%
\pgfsys@transformshift{5.268058in}{1.001882in}%
\pgfsys@useobject{currentmarker}{}%
\end{pgfscope}%
\begin{pgfscope}%
\pgfsys@transformshift{5.285652in}{1.023827in}%
\pgfsys@useobject{currentmarker}{}%
\end{pgfscope}%
\begin{pgfscope}%
\pgfsys@transformshift{5.303246in}{0.893492in}%
\pgfsys@useobject{currentmarker}{}%
\end{pgfscope}%
\begin{pgfscope}%
\pgfsys@transformshift{5.320839in}{0.974867in}%
\pgfsys@useobject{currentmarker}{}%
\end{pgfscope}%
\begin{pgfscope}%
\pgfsys@transformshift{5.338433in}{0.960416in}%
\pgfsys@useobject{currentmarker}{}%
\end{pgfscope}%
\begin{pgfscope}%
\pgfsys@transformshift{5.356027in}{1.047833in}%
\pgfsys@useobject{currentmarker}{}%
\end{pgfscope}%
\begin{pgfscope}%
\pgfsys@transformshift{5.373621in}{0.885599in}%
\pgfsys@useobject{currentmarker}{}%
\end{pgfscope}%
\begin{pgfscope}%
\pgfsys@transformshift{5.391215in}{1.073912in}%
\pgfsys@useobject{currentmarker}{}%
\end{pgfscope}%
\begin{pgfscope}%
\pgfsys@transformshift{5.408809in}{1.142590in}%
\pgfsys@useobject{currentmarker}{}%
\end{pgfscope}%
\begin{pgfscope}%
\pgfsys@transformshift{5.426403in}{0.788336in}%
\pgfsys@useobject{currentmarker}{}%
\end{pgfscope}%
\begin{pgfscope}%
\pgfsys@transformshift{5.443996in}{1.038231in}%
\pgfsys@useobject{currentmarker}{}%
\end{pgfscope}%
\begin{pgfscope}%
\pgfsys@transformshift{5.461590in}{1.067128in}%
\pgfsys@useobject{currentmarker}{}%
\end{pgfscope}%
\begin{pgfscope}%
\pgfsys@transformshift{5.479184in}{0.942637in}%
\pgfsys@useobject{currentmarker}{}%
\end{pgfscope}%
\begin{pgfscope}%
\pgfsys@transformshift{5.496778in}{0.974633in}%
\pgfsys@useobject{currentmarker}{}%
\end{pgfscope}%
\begin{pgfscope}%
\pgfsys@transformshift{5.514372in}{1.011053in}%
\pgfsys@useobject{currentmarker}{}%
\end{pgfscope}%
\begin{pgfscope}%
\pgfsys@transformshift{5.531966in}{1.006780in}%
\pgfsys@useobject{currentmarker}{}%
\end{pgfscope}%
\begin{pgfscope}%
\pgfsys@transformshift{5.549560in}{1.019648in}%
\pgfsys@useobject{currentmarker}{}%
\end{pgfscope}%
\begin{pgfscope}%
\pgfsys@transformshift{5.567153in}{0.899607in}%
\pgfsys@useobject{currentmarker}{}%
\end{pgfscope}%
\begin{pgfscope}%
\pgfsys@transformshift{5.584747in}{1.197997in}%
\pgfsys@useobject{currentmarker}{}%
\end{pgfscope}%
\begin{pgfscope}%
\pgfsys@transformshift{5.602341in}{1.209717in}%
\pgfsys@useobject{currentmarker}{}%
\end{pgfscope}%
\begin{pgfscope}%
\pgfsys@transformshift{5.619935in}{1.041988in}%
\pgfsys@useobject{currentmarker}{}%
\end{pgfscope}%
\begin{pgfscope}%
\pgfsys@transformshift{5.637529in}{0.998656in}%
\pgfsys@useobject{currentmarker}{}%
\end{pgfscope}%
\begin{pgfscope}%
\pgfsys@transformshift{5.655123in}{1.218691in}%
\pgfsys@useobject{currentmarker}{}%
\end{pgfscope}%
\begin{pgfscope}%
\pgfsys@transformshift{5.672717in}{1.133189in}%
\pgfsys@useobject{currentmarker}{}%
\end{pgfscope}%
\begin{pgfscope}%
\pgfsys@transformshift{5.690310in}{1.228631in}%
\pgfsys@useobject{currentmarker}{}%
\end{pgfscope}%
\begin{pgfscope}%
\pgfsys@transformshift{5.707904in}{1.207432in}%
\pgfsys@useobject{currentmarker}{}%
\end{pgfscope}%
\begin{pgfscope}%
\pgfsys@transformshift{5.725498in}{1.332763in}%
\pgfsys@useobject{currentmarker}{}%
\end{pgfscope}%
\begin{pgfscope}%
\pgfsys@transformshift{5.743092in}{1.358076in}%
\pgfsys@useobject{currentmarker}{}%
\end{pgfscope}%
\begin{pgfscope}%
\pgfsys@transformshift{5.760686in}{1.241852in}%
\pgfsys@useobject{currentmarker}{}%
\end{pgfscope}%
\begin{pgfscope}%
\pgfsys@transformshift{5.778280in}{1.201068in}%
\pgfsys@useobject{currentmarker}{}%
\end{pgfscope}%
\begin{pgfscope}%
\pgfsys@transformshift{5.795873in}{1.205290in}%
\pgfsys@useobject{currentmarker}{}%
\end{pgfscope}%
\begin{pgfscope}%
\pgfsys@transformshift{5.813467in}{1.446525in}%
\pgfsys@useobject{currentmarker}{}%
\end{pgfscope}%
\begin{pgfscope}%
\pgfsys@transformshift{5.831061in}{1.290250in}%
\pgfsys@useobject{currentmarker}{}%
\end{pgfscope}%
\begin{pgfscope}%
\pgfsys@transformshift{5.848655in}{1.379416in}%
\pgfsys@useobject{currentmarker}{}%
\end{pgfscope}%
\begin{pgfscope}%
\pgfsys@transformshift{5.866249in}{1.390722in}%
\pgfsys@useobject{currentmarker}{}%
\end{pgfscope}%
\begin{pgfscope}%
\pgfsys@transformshift{5.883843in}{1.463574in}%
\pgfsys@useobject{currentmarker}{}%
\end{pgfscope}%
\begin{pgfscope}%
\pgfsys@transformshift{5.901437in}{1.293783in}%
\pgfsys@useobject{currentmarker}{}%
\end{pgfscope}%
\begin{pgfscope}%
\pgfsys@transformshift{5.919030in}{1.520622in}%
\pgfsys@useobject{currentmarker}{}%
\end{pgfscope}%
\begin{pgfscope}%
\pgfsys@transformshift{5.936624in}{1.567933in}%
\pgfsys@useobject{currentmarker}{}%
\end{pgfscope}%
\begin{pgfscope}%
\pgfsys@transformshift{5.954218in}{1.536902in}%
\pgfsys@useobject{currentmarker}{}%
\end{pgfscope}%
\begin{pgfscope}%
\pgfsys@transformshift{5.971812in}{1.507609in}%
\pgfsys@useobject{currentmarker}{}%
\end{pgfscope}%
\begin{pgfscope}%
\pgfsys@transformshift{5.989406in}{1.556646in}%
\pgfsys@useobject{currentmarker}{}%
\end{pgfscope}%
\begin{pgfscope}%
\pgfsys@transformshift{6.007000in}{1.593168in}%
\pgfsys@useobject{currentmarker}{}%
\end{pgfscope}%
\begin{pgfscope}%
\pgfsys@transformshift{6.024594in}{1.282724in}%
\pgfsys@useobject{currentmarker}{}%
\end{pgfscope}%
\begin{pgfscope}%
\pgfsys@transformshift{6.042187in}{1.755190in}%
\pgfsys@useobject{currentmarker}{}%
\end{pgfscope}%
\begin{pgfscope}%
\pgfsys@transformshift{6.059781in}{1.600496in}%
\pgfsys@useobject{currentmarker}{}%
\end{pgfscope}%
\begin{pgfscope}%
\pgfsys@transformshift{6.077375in}{1.495977in}%
\pgfsys@useobject{currentmarker}{}%
\end{pgfscope}%
\begin{pgfscope}%
\pgfsys@transformshift{6.094969in}{1.519259in}%
\pgfsys@useobject{currentmarker}{}%
\end{pgfscope}%
\begin{pgfscope}%
\pgfsys@transformshift{6.112563in}{1.602814in}%
\pgfsys@useobject{currentmarker}{}%
\end{pgfscope}%
\begin{pgfscope}%
\pgfsys@transformshift{6.130157in}{1.536369in}%
\pgfsys@useobject{currentmarker}{}%
\end{pgfscope}%
\begin{pgfscope}%
\pgfsys@transformshift{6.147751in}{1.338906in}%
\pgfsys@useobject{currentmarker}{}%
\end{pgfscope}%
\begin{pgfscope}%
\pgfsys@transformshift{6.165344in}{1.737207in}%
\pgfsys@useobject{currentmarker}{}%
\end{pgfscope}%
\begin{pgfscope}%
\pgfsys@transformshift{6.182938in}{1.511509in}%
\pgfsys@useobject{currentmarker}{}%
\end{pgfscope}%
\begin{pgfscope}%
\pgfsys@transformshift{6.200532in}{1.613277in}%
\pgfsys@useobject{currentmarker}{}%
\end{pgfscope}%
\begin{pgfscope}%
\pgfsys@transformshift{6.218126in}{1.431866in}%
\pgfsys@useobject{currentmarker}{}%
\end{pgfscope}%
\begin{pgfscope}%
\pgfsys@transformshift{6.235720in}{1.641360in}%
\pgfsys@useobject{currentmarker}{}%
\end{pgfscope}%
\begin{pgfscope}%
\pgfsys@transformshift{6.253314in}{1.504922in}%
\pgfsys@useobject{currentmarker}{}%
\end{pgfscope}%
\begin{pgfscope}%
\pgfsys@transformshift{6.270907in}{1.524526in}%
\pgfsys@useobject{currentmarker}{}%
\end{pgfscope}%
\begin{pgfscope}%
\pgfsys@transformshift{6.288501in}{1.347796in}%
\pgfsys@useobject{currentmarker}{}%
\end{pgfscope}%
\begin{pgfscope}%
\pgfsys@transformshift{6.306095in}{1.559768in}%
\pgfsys@useobject{currentmarker}{}%
\end{pgfscope}%
\begin{pgfscope}%
\pgfsys@transformshift{6.323689in}{1.496221in}%
\pgfsys@useobject{currentmarker}{}%
\end{pgfscope}%
\begin{pgfscope}%
\pgfsys@transformshift{6.341283in}{1.545886in}%
\pgfsys@useobject{currentmarker}{}%
\end{pgfscope}%
\begin{pgfscope}%
\pgfsys@transformshift{6.358877in}{1.343841in}%
\pgfsys@useobject{currentmarker}{}%
\end{pgfscope}%
\begin{pgfscope}%
\pgfsys@transformshift{6.376471in}{1.349125in}%
\pgfsys@useobject{currentmarker}{}%
\end{pgfscope}%
\end{pgfscope}%
\begin{pgfscope}%
\pgfsetbuttcap%
\pgfsetroundjoin%
\definecolor{currentfill}{rgb}{0.000000,0.000000,0.000000}%
\pgfsetfillcolor{currentfill}%
\pgfsetlinewidth{0.803000pt}%
\definecolor{currentstroke}{rgb}{0.000000,0.000000,0.000000}%
\pgfsetstrokecolor{currentstroke}%
\pgfsetdash{}{0pt}%
\pgfsys@defobject{currentmarker}{\pgfqpoint{0.000000in}{-0.048611in}}{\pgfqpoint{0.000000in}{0.000000in}}{%
\pgfpathmoveto{\pgfqpoint{0.000000in}{0.000000in}}%
\pgfpathlineto{\pgfqpoint{0.000000in}{-0.048611in}}%
\pgfusepath{stroke,fill}%
}%
\begin{pgfscope}%
\pgfsys@transformshift{2.000000in}{0.750000in}%
\pgfsys@useobject{currentmarker}{}%
\end{pgfscope}%
\end{pgfscope}%
\begin{pgfscope}%
\pgftext[x=2.000000in,y=0.652778in,,top]{\rmfamily\fontsize{10.000000}{12.000000}\selectfont \(\displaystyle -1.5\)}%
\end{pgfscope}%
\begin{pgfscope}%
\pgfsetbuttcap%
\pgfsetroundjoin%
\definecolor{currentfill}{rgb}{0.000000,0.000000,0.000000}%
\pgfsetfillcolor{currentfill}%
\pgfsetlinewidth{0.803000pt}%
\definecolor{currentstroke}{rgb}{0.000000,0.000000,0.000000}%
\pgfsetstrokecolor{currentstroke}%
\pgfsetdash{}{0pt}%
\pgfsys@defobject{currentmarker}{\pgfqpoint{0.000000in}{-0.048611in}}{\pgfqpoint{0.000000in}{0.000000in}}{%
\pgfpathmoveto{\pgfqpoint{0.000000in}{0.000000in}}%
\pgfpathlineto{\pgfqpoint{0.000000in}{-0.048611in}}%
\pgfusepath{stroke,fill}%
}%
\begin{pgfscope}%
\pgfsys@transformshift{2.875294in}{0.750000in}%
\pgfsys@useobject{currentmarker}{}%
\end{pgfscope}%
\end{pgfscope}%
\begin{pgfscope}%
\pgftext[x=2.875294in,y=0.652778in,,top]{\rmfamily\fontsize{10.000000}{12.000000}\selectfont \(\displaystyle -1.0\)}%
\end{pgfscope}%
\begin{pgfscope}%
\pgfsetbuttcap%
\pgfsetroundjoin%
\definecolor{currentfill}{rgb}{0.000000,0.000000,0.000000}%
\pgfsetfillcolor{currentfill}%
\pgfsetlinewidth{0.803000pt}%
\definecolor{currentstroke}{rgb}{0.000000,0.000000,0.000000}%
\pgfsetstrokecolor{currentstroke}%
\pgfsetdash{}{0pt}%
\pgfsys@defobject{currentmarker}{\pgfqpoint{0.000000in}{-0.048611in}}{\pgfqpoint{0.000000in}{0.000000in}}{%
\pgfpathmoveto{\pgfqpoint{0.000000in}{0.000000in}}%
\pgfpathlineto{\pgfqpoint{0.000000in}{-0.048611in}}%
\pgfusepath{stroke,fill}%
}%
\begin{pgfscope}%
\pgfsys@transformshift{3.750588in}{0.750000in}%
\pgfsys@useobject{currentmarker}{}%
\end{pgfscope}%
\end{pgfscope}%
\begin{pgfscope}%
\pgftext[x=3.750588in,y=0.652778in,,top]{\rmfamily\fontsize{10.000000}{12.000000}\selectfont \(\displaystyle -0.5\)}%
\end{pgfscope}%
\begin{pgfscope}%
\pgfsetbuttcap%
\pgfsetroundjoin%
\definecolor{currentfill}{rgb}{0.000000,0.000000,0.000000}%
\pgfsetfillcolor{currentfill}%
\pgfsetlinewidth{0.803000pt}%
\definecolor{currentstroke}{rgb}{0.000000,0.000000,0.000000}%
\pgfsetstrokecolor{currentstroke}%
\pgfsetdash{}{0pt}%
\pgfsys@defobject{currentmarker}{\pgfqpoint{0.000000in}{-0.048611in}}{\pgfqpoint{0.000000in}{0.000000in}}{%
\pgfpathmoveto{\pgfqpoint{0.000000in}{0.000000in}}%
\pgfpathlineto{\pgfqpoint{0.000000in}{-0.048611in}}%
\pgfusepath{stroke,fill}%
}%
\begin{pgfscope}%
\pgfsys@transformshift{4.625882in}{0.750000in}%
\pgfsys@useobject{currentmarker}{}%
\end{pgfscope}%
\end{pgfscope}%
\begin{pgfscope}%
\pgftext[x=4.625882in,y=0.652778in,,top]{\rmfamily\fontsize{10.000000}{12.000000}\selectfont \(\displaystyle 0.0\)}%
\end{pgfscope}%
\begin{pgfscope}%
\pgfsetbuttcap%
\pgfsetroundjoin%
\definecolor{currentfill}{rgb}{0.000000,0.000000,0.000000}%
\pgfsetfillcolor{currentfill}%
\pgfsetlinewidth{0.803000pt}%
\definecolor{currentstroke}{rgb}{0.000000,0.000000,0.000000}%
\pgfsetstrokecolor{currentstroke}%
\pgfsetdash{}{0pt}%
\pgfsys@defobject{currentmarker}{\pgfqpoint{0.000000in}{-0.048611in}}{\pgfqpoint{0.000000in}{0.000000in}}{%
\pgfpathmoveto{\pgfqpoint{0.000000in}{0.000000in}}%
\pgfpathlineto{\pgfqpoint{0.000000in}{-0.048611in}}%
\pgfusepath{stroke,fill}%
}%
\begin{pgfscope}%
\pgfsys@transformshift{5.501176in}{0.750000in}%
\pgfsys@useobject{currentmarker}{}%
\end{pgfscope}%
\end{pgfscope}%
\begin{pgfscope}%
\pgftext[x=5.501176in,y=0.652778in,,top]{\rmfamily\fontsize{10.000000}{12.000000}\selectfont \(\displaystyle 0.5\)}%
\end{pgfscope}%
\begin{pgfscope}%
\pgfsetbuttcap%
\pgfsetroundjoin%
\definecolor{currentfill}{rgb}{0.000000,0.000000,0.000000}%
\pgfsetfillcolor{currentfill}%
\pgfsetlinewidth{0.803000pt}%
\definecolor{currentstroke}{rgb}{0.000000,0.000000,0.000000}%
\pgfsetstrokecolor{currentstroke}%
\pgfsetdash{}{0pt}%
\pgfsys@defobject{currentmarker}{\pgfqpoint{0.000000in}{-0.048611in}}{\pgfqpoint{0.000000in}{0.000000in}}{%
\pgfpathmoveto{\pgfqpoint{0.000000in}{0.000000in}}%
\pgfpathlineto{\pgfqpoint{0.000000in}{-0.048611in}}%
\pgfusepath{stroke,fill}%
}%
\begin{pgfscope}%
\pgfsys@transformshift{6.376471in}{0.750000in}%
\pgfsys@useobject{currentmarker}{}%
\end{pgfscope}%
\end{pgfscope}%
\begin{pgfscope}%
\pgftext[x=6.376471in,y=0.652778in,,top]{\rmfamily\fontsize{10.000000}{12.000000}\selectfont \(\displaystyle 1.0\)}%
\end{pgfscope}%
\begin{pgfscope}%
\pgftext[x=4.188235in,y=0.471083in,,top]{\rmfamily\fontsize{10.000000}{12.000000}\selectfont x}%
\end{pgfscope}%
\begin{pgfscope}%
\pgfsetbuttcap%
\pgfsetroundjoin%
\definecolor{currentfill}{rgb}{0.000000,0.000000,0.000000}%
\pgfsetfillcolor{currentfill}%
\pgfsetlinewidth{0.803000pt}%
\definecolor{currentstroke}{rgb}{0.000000,0.000000,0.000000}%
\pgfsetstrokecolor{currentstroke}%
\pgfsetdash{}{0pt}%
\pgfsys@defobject{currentmarker}{\pgfqpoint{-0.048611in}{0.000000in}}{\pgfqpoint{0.000000in}{0.000000in}}{%
\pgfpathmoveto{\pgfqpoint{0.000000in}{0.000000in}}%
\pgfpathlineto{\pgfqpoint{-0.048611in}{0.000000in}}%
\pgfusepath{stroke,fill}%
}%
\begin{pgfscope}%
\pgfsys@transformshift{2.000000in}{1.107632in}%
\pgfsys@useobject{currentmarker}{}%
\end{pgfscope}%
\end{pgfscope}%
\begin{pgfscope}%
\pgftext[x=1.833333in,y=1.059414in,left,base]{\rmfamily\fontsize{10.000000}{12.000000}\selectfont \(\displaystyle 0\)}%
\end{pgfscope}%
\begin{pgfscope}%
\pgfsetbuttcap%
\pgfsetroundjoin%
\definecolor{currentfill}{rgb}{0.000000,0.000000,0.000000}%
\pgfsetfillcolor{currentfill}%
\pgfsetlinewidth{0.803000pt}%
\definecolor{currentstroke}{rgb}{0.000000,0.000000,0.000000}%
\pgfsetstrokecolor{currentstroke}%
\pgfsetdash{}{0pt}%
\pgfsys@defobject{currentmarker}{\pgfqpoint{-0.048611in}{0.000000in}}{\pgfqpoint{0.000000in}{0.000000in}}{%
\pgfpathmoveto{\pgfqpoint{0.000000in}{0.000000in}}%
\pgfpathlineto{\pgfqpoint{-0.048611in}{0.000000in}}%
\pgfusepath{stroke,fill}%
}%
\begin{pgfscope}%
\pgfsys@transformshift{2.000000in}{1.505000in}%
\pgfsys@useobject{currentmarker}{}%
\end{pgfscope}%
\end{pgfscope}%
\begin{pgfscope}%
\pgftext[x=1.833333in,y=1.456782in,left,base]{\rmfamily\fontsize{10.000000}{12.000000}\selectfont \(\displaystyle 2\)}%
\end{pgfscope}%
\begin{pgfscope}%
\pgftext[x=1.777777in,y=1.226842in,,bottom,rotate=90.000000]{\rmfamily\fontsize{10.000000}{12.000000}\selectfont y}%
\end{pgfscope}%
\begin{pgfscope}%
\pgfpathrectangle{\pgfqpoint{2.000000in}{0.750000in}}{\pgfqpoint{4.376471in}{0.953684in}} %
\pgfusepath{clip}%
\pgfsetrectcap%
\pgfsetroundjoin%
\pgfsetlinewidth{1.505625pt}%
\definecolor{currentstroke}{rgb}{0.121569,0.466667,0.705882}%
\pgfsetstrokecolor{currentstroke}%
\pgfsetdash{}{0pt}%
\pgfpathmoveto{\pgfqpoint{2.906712in}{0.736111in}}%
\pgfpathlineto{\pgfqpoint{2.910482in}{0.849980in}}%
\pgfpathlineto{\pgfqpoint{2.928076in}{1.219326in}}%
\pgfpathlineto{\pgfqpoint{2.945670in}{1.461936in}}%
\pgfpathlineto{\pgfqpoint{2.963263in}{1.607243in}}%
\pgfpathlineto{\pgfqpoint{2.980857in}{1.679541in}}%
\pgfpathlineto{\pgfqpoint{2.998451in}{1.698654in}}%
\pgfpathlineto{\pgfqpoint{3.016045in}{1.680551in}}%
\pgfpathlineto{\pgfqpoint{3.033639in}{1.637896in}}%
\pgfpathlineto{\pgfqpoint{3.051233in}{1.580551in}}%
\pgfpathlineto{\pgfqpoint{3.121608in}{1.327043in}}%
\pgfpathlineto{\pgfqpoint{3.139202in}{1.274663in}}%
\pgfpathlineto{\pgfqpoint{3.156796in}{1.229647in}}%
\pgfpathlineto{\pgfqpoint{3.174390in}{1.192207in}}%
\pgfpathlineto{\pgfqpoint{3.191983in}{1.162116in}}%
\pgfpathlineto{\pgfqpoint{3.209577in}{1.138852in}}%
\pgfpathlineto{\pgfqpoint{3.227171in}{1.121705in}}%
\pgfpathlineto{\pgfqpoint{3.244765in}{1.109868in}}%
\pgfpathlineto{\pgfqpoint{3.262359in}{1.102509in}}%
\pgfpathlineto{\pgfqpoint{3.279953in}{1.098822in}}%
\pgfpathlineto{\pgfqpoint{3.297547in}{1.098067in}}%
\pgfpathlineto{\pgfqpoint{3.315140in}{1.099590in}}%
\pgfpathlineto{\pgfqpoint{3.350328in}{1.107384in}}%
\pgfpathlineto{\pgfqpoint{3.385516in}{1.119110in}}%
\pgfpathlineto{\pgfqpoint{3.438297in}{1.141115in}}%
\pgfpathlineto{\pgfqpoint{3.473485in}{1.158592in}}%
\pgfpathlineto{\pgfqpoint{3.508673in}{1.178964in}}%
\pgfpathlineto{\pgfqpoint{3.543860in}{1.202863in}}%
\pgfpathlineto{\pgfqpoint{3.579048in}{1.230570in}}%
\pgfpathlineto{\pgfqpoint{3.614236in}{1.261791in}}%
\pgfpathlineto{\pgfqpoint{3.684611in}{1.330199in}}%
\pgfpathlineto{\pgfqpoint{3.719799in}{1.363601in}}%
\pgfpathlineto{\pgfqpoint{3.754987in}{1.393278in}}%
\pgfpathlineto{\pgfqpoint{3.790174in}{1.416697in}}%
\pgfpathlineto{\pgfqpoint{3.807768in}{1.425311in}}%
\pgfpathlineto{\pgfqpoint{3.825362in}{1.431511in}}%
\pgfpathlineto{\pgfqpoint{3.842956in}{1.435072in}}%
\pgfpathlineto{\pgfqpoint{3.860550in}{1.435809in}}%
\pgfpathlineto{\pgfqpoint{3.878144in}{1.433585in}}%
\pgfpathlineto{\pgfqpoint{3.895738in}{1.428307in}}%
\pgfpathlineto{\pgfqpoint{3.913331in}{1.419936in}}%
\pgfpathlineto{\pgfqpoint{3.930925in}{1.408483in}}%
\pgfpathlineto{\pgfqpoint{3.948519in}{1.394014in}}%
\pgfpathlineto{\pgfqpoint{3.966113in}{1.376645in}}%
\pgfpathlineto{\pgfqpoint{3.983707in}{1.356542in}}%
\pgfpathlineto{\pgfqpoint{4.018894in}{1.309035in}}%
\pgfpathlineto{\pgfqpoint{4.054082in}{1.253692in}}%
\pgfpathlineto{\pgfqpoint{4.106864in}{1.162039in}}%
\pgfpathlineto{\pgfqpoint{4.159645in}{1.069311in}}%
\pgfpathlineto{\pgfqpoint{4.194833in}{1.012068in}}%
\pgfpathlineto{\pgfqpoint{4.230021in}{0.961746in}}%
\pgfpathlineto{\pgfqpoint{4.247615in}{0.939914in}}%
\pgfpathlineto{\pgfqpoint{4.265208in}{0.920622in}}%
\pgfpathlineto{\pgfqpoint{4.282802in}{0.904058in}}%
\pgfpathlineto{\pgfqpoint{4.300396in}{0.890367in}}%
\pgfpathlineto{\pgfqpoint{4.317990in}{0.879650in}}%
\pgfpathlineto{\pgfqpoint{4.335584in}{0.871962in}}%
\pgfpathlineto{\pgfqpoint{4.353178in}{0.867312in}}%
\pgfpathlineto{\pgfqpoint{4.370772in}{0.865669in}}%
\pgfpathlineto{\pgfqpoint{4.388365in}{0.866956in}}%
\pgfpathlineto{\pgfqpoint{4.405959in}{0.871058in}}%
\pgfpathlineto{\pgfqpoint{4.423553in}{0.877826in}}%
\pgfpathlineto{\pgfqpoint{4.441147in}{0.887075in}}%
\pgfpathlineto{\pgfqpoint{4.458741in}{0.898595in}}%
\pgfpathlineto{\pgfqpoint{4.493928in}{0.927487in}}%
\pgfpathlineto{\pgfqpoint{4.529116in}{0.962423in}}%
\pgfpathlineto{\pgfqpoint{4.581898in}{1.021306in}}%
\pgfpathlineto{\pgfqpoint{4.652273in}{1.100752in}}%
\pgfpathlineto{\pgfqpoint{4.687461in}{1.136827in}}%
\pgfpathlineto{\pgfqpoint{4.722649in}{1.168719in}}%
\pgfpathlineto{\pgfqpoint{4.757836in}{1.195639in}}%
\pgfpathlineto{\pgfqpoint{4.793024in}{1.217214in}}%
\pgfpathlineto{\pgfqpoint{4.828212in}{1.233452in}}%
\pgfpathlineto{\pgfqpoint{4.863399in}{1.244676in}}%
\pgfpathlineto{\pgfqpoint{4.898587in}{1.251431in}}%
\pgfpathlineto{\pgfqpoint{4.933775in}{1.254394in}}%
\pgfpathlineto{\pgfqpoint{4.968962in}{1.254269in}}%
\pgfpathlineto{\pgfqpoint{5.004150in}{1.251700in}}%
\pgfpathlineto{\pgfqpoint{5.056932in}{1.244340in}}%
\pgfpathlineto{\pgfqpoint{5.109713in}{1.233561in}}%
\pgfpathlineto{\pgfqpoint{5.162495in}{1.219394in}}%
\pgfpathlineto{\pgfqpoint{5.215276in}{1.201213in}}%
\pgfpathlineto{\pgfqpoint{5.268058in}{1.178290in}}%
\pgfpathlineto{\pgfqpoint{5.320839in}{1.150495in}}%
\pgfpathlineto{\pgfqpoint{5.391215in}{1.108083in}}%
\pgfpathlineto{\pgfqpoint{5.443996in}{1.076379in}}%
\pgfpathlineto{\pgfqpoint{5.479184in}{1.058035in}}%
\pgfpathlineto{\pgfqpoint{5.514372in}{1.044049in}}%
\pgfpathlineto{\pgfqpoint{5.531966in}{1.039299in}}%
\pgfpathlineto{\pgfqpoint{5.549560in}{1.036366in}}%
\pgfpathlineto{\pgfqpoint{5.567153in}{1.035480in}}%
\pgfpathlineto{\pgfqpoint{5.584747in}{1.036853in}}%
\pgfpathlineto{\pgfqpoint{5.602341in}{1.040675in}}%
\pgfpathlineto{\pgfqpoint{5.619935in}{1.047105in}}%
\pgfpathlineto{\pgfqpoint{5.637529in}{1.056269in}}%
\pgfpathlineto{\pgfqpoint{5.655123in}{1.068248in}}%
\pgfpathlineto{\pgfqpoint{5.672717in}{1.083078in}}%
\pgfpathlineto{\pgfqpoint{5.690310in}{1.100745in}}%
\pgfpathlineto{\pgfqpoint{5.707904in}{1.121175in}}%
\pgfpathlineto{\pgfqpoint{5.743092in}{1.169754in}}%
\pgfpathlineto{\pgfqpoint{5.778280in}{1.227066in}}%
\pgfpathlineto{\pgfqpoint{5.831061in}{1.323330in}}%
\pgfpathlineto{\pgfqpoint{5.883843in}{1.420805in}}%
\pgfpathlineto{\pgfqpoint{5.919030in}{1.479358in}}%
\pgfpathlineto{\pgfqpoint{5.936624in}{1.505089in}}%
\pgfpathlineto{\pgfqpoint{5.954218in}{1.527795in}}%
\pgfpathlineto{\pgfqpoint{5.971812in}{1.547040in}}%
\pgfpathlineto{\pgfqpoint{5.989406in}{1.562456in}}%
\pgfpathlineto{\pgfqpoint{6.007000in}{1.573755in}}%
\pgfpathlineto{\pgfqpoint{6.024594in}{1.580742in}}%
\pgfpathlineto{\pgfqpoint{6.042187in}{1.583329in}}%
\pgfpathlineto{\pgfqpoint{6.059781in}{1.581545in}}%
\pgfpathlineto{\pgfqpoint{6.077375in}{1.575539in}}%
\pgfpathlineto{\pgfqpoint{6.094969in}{1.565587in}}%
\pgfpathlineto{\pgfqpoint{6.112563in}{1.552088in}}%
\pgfpathlineto{\pgfqpoint{6.130157in}{1.535560in}}%
\pgfpathlineto{\pgfqpoint{6.165344in}{1.495992in}}%
\pgfpathlineto{\pgfqpoint{6.235720in}{1.412067in}}%
\pgfpathlineto{\pgfqpoint{6.253314in}{1.394368in}}%
\pgfpathlineto{\pgfqpoint{6.270907in}{1.379009in}}%
\pgfpathlineto{\pgfqpoint{6.288501in}{1.366086in}}%
\pgfpathlineto{\pgfqpoint{6.323689in}{1.345917in}}%
\pgfpathlineto{\pgfqpoint{6.341283in}{1.336512in}}%
\pgfpathlineto{\pgfqpoint{6.358877in}{1.324979in}}%
\pgfpathlineto{\pgfqpoint{6.376471in}{1.308314in}}%
\pgfpathlineto{\pgfqpoint{6.376471in}{1.308314in}}%
\pgfusepath{stroke}%
\end{pgfscope}%
\begin{pgfscope}%
\pgfpathrectangle{\pgfqpoint{2.000000in}{0.750000in}}{\pgfqpoint{4.376471in}{0.953684in}} %
\pgfusepath{clip}%
\pgfsetbuttcap%
\pgfsetroundjoin%
\pgfsetlinewidth{1.505625pt}%
\definecolor{currentstroke}{rgb}{1.000000,0.498039,0.054902}%
\pgfsetstrokecolor{currentstroke}%
\pgfsetdash{{9.600000pt}{2.400000pt}{1.600000pt}{2.400000pt}}{0.000000pt}%
\pgfpathmoveto{\pgfqpoint{2.905523in}{0.736111in}}%
\pgfpathlineto{\pgfqpoint{2.910482in}{0.879067in}}%
\pgfpathlineto{\pgfqpoint{2.928076in}{1.233361in}}%
\pgfpathlineto{\pgfqpoint{2.945670in}{1.465612in}}%
\pgfpathlineto{\pgfqpoint{2.963263in}{1.606476in}}%
\pgfpathlineto{\pgfqpoint{2.980857in}{1.679043in}}%
\pgfpathlineto{\pgfqpoint{2.998451in}{1.697475in}}%
\pgfpathlineto{\pgfqpoint{3.016045in}{1.679043in}}%
\pgfpathlineto{\pgfqpoint{3.033639in}{1.640431in}}%
\pgfpathlineto{\pgfqpoint{3.051233in}{1.583387in}}%
\pgfpathlineto{\pgfqpoint{3.121608in}{1.330763in}}%
\pgfpathlineto{\pgfqpoint{3.139202in}{1.278182in}}%
\pgfpathlineto{\pgfqpoint{3.156796in}{1.232973in}}%
\pgfpathlineto{\pgfqpoint{3.174390in}{1.195720in}}%
\pgfpathlineto{\pgfqpoint{3.191983in}{1.166034in}}%
\pgfpathlineto{\pgfqpoint{3.209577in}{1.141974in}}%
\pgfpathlineto{\pgfqpoint{3.227171in}{1.122572in}}%
\pgfpathlineto{\pgfqpoint{3.262359in}{1.104527in}}%
\pgfpathlineto{\pgfqpoint{3.279953in}{1.100647in}}%
\pgfpathlineto{\pgfqpoint{3.297547in}{1.100550in}}%
\pgfpathlineto{\pgfqpoint{3.332734in}{1.105109in}}%
\pgfpathlineto{\pgfqpoint{3.350328in}{1.113161in}}%
\pgfpathlineto{\pgfqpoint{3.385516in}{1.122378in}}%
\pgfpathlineto{\pgfqpoint{3.403110in}{1.130915in}}%
\pgfpathlineto{\pgfqpoint{3.420704in}{1.134601in}}%
\pgfpathlineto{\pgfqpoint{3.438297in}{1.144594in}}%
\pgfpathlineto{\pgfqpoint{3.455891in}{1.151918in}}%
\pgfpathlineto{\pgfqpoint{3.491079in}{1.172097in}}%
\pgfpathlineto{\pgfqpoint{3.508673in}{1.179907in}}%
\pgfpathlineto{\pgfqpoint{3.543860in}{1.202244in}}%
\pgfpathlineto{\pgfqpoint{3.561454in}{1.210830in}}%
\pgfpathlineto{\pgfqpoint{3.579048in}{1.228098in}}%
\pgfpathlineto{\pgfqpoint{3.614236in}{1.256693in}}%
\pgfpathlineto{\pgfqpoint{3.631830in}{1.271973in}}%
\pgfpathlineto{\pgfqpoint{3.649424in}{1.282838in}}%
\pgfpathlineto{\pgfqpoint{3.667017in}{1.303017in}}%
\pgfpathlineto{\pgfqpoint{3.684611in}{1.320286in}}%
\pgfpathlineto{\pgfqpoint{3.737393in}{1.365300in}}%
\pgfpathlineto{\pgfqpoint{3.754987in}{1.371897in}}%
\pgfpathlineto{\pgfqpoint{3.772581in}{1.390718in}}%
\pgfpathlineto{\pgfqpoint{3.807768in}{1.410315in}}%
\pgfpathlineto{\pgfqpoint{3.825362in}{1.416911in}}%
\pgfpathlineto{\pgfqpoint{3.860550in}{1.420792in}}%
\pgfpathlineto{\pgfqpoint{3.878144in}{1.425837in}}%
\pgfpathlineto{\pgfqpoint{3.895738in}{1.412255in}}%
\pgfpathlineto{\pgfqpoint{3.913331in}{1.409927in}}%
\pgfpathlineto{\pgfqpoint{3.930925in}{1.399061in}}%
\pgfpathlineto{\pgfqpoint{3.948519in}{1.382375in}}%
\pgfpathlineto{\pgfqpoint{3.966113in}{1.369181in}}%
\pgfpathlineto{\pgfqpoint{3.983707in}{1.344733in}}%
\pgfpathlineto{\pgfqpoint{4.001301in}{1.326107in}}%
\pgfpathlineto{\pgfqpoint{4.036488in}{1.279152in}}%
\pgfpathlineto{\pgfqpoint{4.071676in}{1.220944in}}%
\pgfpathlineto{\pgfqpoint{4.089270in}{1.185243in}}%
\pgfpathlineto{\pgfqpoint{4.106864in}{1.154198in}}%
\pgfpathlineto{\pgfqpoint{4.124458in}{1.129363in}}%
\pgfpathlineto{\pgfqpoint{4.159645in}{1.057961in}}%
\pgfpathlineto{\pgfqpoint{4.177239in}{1.033901in}}%
\pgfpathlineto{\pgfqpoint{4.194833in}{1.006737in}}%
\pgfpathlineto{\pgfqpoint{4.212427in}{0.975693in}}%
\pgfpathlineto{\pgfqpoint{4.247615in}{0.919813in}}%
\pgfpathlineto{\pgfqpoint{4.282802in}{0.894201in}}%
\pgfpathlineto{\pgfqpoint{4.300396in}{0.874799in}}%
\pgfpathlineto{\pgfqpoint{4.317990in}{0.865485in}}%
\pgfpathlineto{\pgfqpoint{4.335584in}{0.846859in}}%
\pgfpathlineto{\pgfqpoint{4.353178in}{0.848411in}}%
\pgfpathlineto{\pgfqpoint{4.370772in}{0.846859in}}%
\pgfpathlineto{\pgfqpoint{4.405959in}{0.853067in}}%
\pgfpathlineto{\pgfqpoint{4.423553in}{0.857724in}}%
\pgfpathlineto{\pgfqpoint{4.441147in}{0.865485in}}%
\pgfpathlineto{\pgfqpoint{4.458741in}{0.878679in}}%
\pgfpathlineto{\pgfqpoint{4.476335in}{0.897306in}}%
\pgfpathlineto{\pgfqpoint{4.493928in}{0.905067in}}%
\pgfpathlineto{\pgfqpoint{4.529116in}{0.937663in}}%
\pgfpathlineto{\pgfqpoint{4.546710in}{0.957842in}}%
\pgfpathlineto{\pgfqpoint{4.564304in}{0.993543in}}%
\pgfpathlineto{\pgfqpoint{4.581898in}{1.012170in}}%
\pgfpathlineto{\pgfqpoint{4.599492in}{1.026916in}}%
\pgfpathlineto{\pgfqpoint{4.652273in}{1.090557in}}%
\pgfpathlineto{\pgfqpoint{4.669867in}{1.106855in}}%
\pgfpathlineto{\pgfqpoint{4.687461in}{1.131691in}}%
\pgfpathlineto{\pgfqpoint{4.705055in}{1.150318in}}%
\pgfpathlineto{\pgfqpoint{4.722649in}{1.162735in}}%
\pgfpathlineto{\pgfqpoint{4.740242in}{1.181362in}}%
\pgfpathlineto{\pgfqpoint{4.757836in}{1.195332in}}%
\pgfpathlineto{\pgfqpoint{4.775430in}{1.202317in}}%
\pgfpathlineto{\pgfqpoint{4.793024in}{1.216287in}}%
\pgfpathlineto{\pgfqpoint{4.810618in}{1.225600in}}%
\pgfpathlineto{\pgfqpoint{4.845805in}{1.248884in}}%
\pgfpathlineto{\pgfqpoint{4.863399in}{1.244227in}}%
\pgfpathlineto{\pgfqpoint{4.880993in}{1.245779in}}%
\pgfpathlineto{\pgfqpoint{4.898587in}{1.258973in}}%
\pgfpathlineto{\pgfqpoint{4.916181in}{1.256645in}}%
\pgfpathlineto{\pgfqpoint{4.933775in}{1.256645in}}%
\pgfpathlineto{\pgfqpoint{4.951369in}{1.249660in}}%
\pgfpathlineto{\pgfqpoint{4.968962in}{1.250436in}}%
\pgfpathlineto{\pgfqpoint{4.986556in}{1.245779in}}%
\pgfpathlineto{\pgfqpoint{5.004150in}{1.245779in}}%
\pgfpathlineto{\pgfqpoint{5.039338in}{1.242675in}}%
\pgfpathlineto{\pgfqpoint{5.056932in}{1.246555in}}%
\pgfpathlineto{\pgfqpoint{5.074526in}{1.234138in}}%
\pgfpathlineto{\pgfqpoint{5.109713in}{1.221720in}}%
\pgfpathlineto{\pgfqpoint{5.127307in}{1.224824in}}%
\pgfpathlineto{\pgfqpoint{5.144901in}{1.210854in}}%
\pgfpathlineto{\pgfqpoint{5.162495in}{1.211630in}}%
\pgfpathlineto{\pgfqpoint{5.180089in}{1.203869in}}%
\pgfpathlineto{\pgfqpoint{5.197683in}{1.200765in}}%
\pgfpathlineto{\pgfqpoint{5.215276in}{1.182914in}}%
\pgfpathlineto{\pgfqpoint{5.232870in}{1.179034in}}%
\pgfpathlineto{\pgfqpoint{5.250464in}{1.170497in}}%
\pgfpathlineto{\pgfqpoint{5.268058in}{1.170497in}}%
\pgfpathlineto{\pgfqpoint{5.285652in}{1.164288in}}%
\pgfpathlineto{\pgfqpoint{5.303246in}{1.161959in}}%
\pgfpathlineto{\pgfqpoint{5.320839in}{1.141780in}}%
\pgfpathlineto{\pgfqpoint{5.356027in}{1.129363in}}%
\pgfpathlineto{\pgfqpoint{5.373621in}{1.109960in}}%
\pgfpathlineto{\pgfqpoint{5.391215in}{1.109960in}}%
\pgfpathlineto{\pgfqpoint{5.408809in}{1.101423in}}%
\pgfpathlineto{\pgfqpoint{5.426403in}{1.099094in}}%
\pgfpathlineto{\pgfqpoint{5.443996in}{1.087453in}}%
\pgfpathlineto{\pgfqpoint{5.461590in}{1.079692in}}%
\pgfpathlineto{\pgfqpoint{5.479184in}{1.062617in}}%
\pgfpathlineto{\pgfqpoint{5.496778in}{1.053304in}}%
\pgfpathlineto{\pgfqpoint{5.514372in}{1.054080in}}%
\pgfpathlineto{\pgfqpoint{5.531966in}{1.045543in}}%
\pgfpathlineto{\pgfqpoint{5.549560in}{1.049423in}}%
\pgfpathlineto{\pgfqpoint{5.584747in}{1.037006in}}%
\pgfpathlineto{\pgfqpoint{5.602341in}{1.043991in}}%
\pgfpathlineto{\pgfqpoint{5.619935in}{1.057961in}}%
\pgfpathlineto{\pgfqpoint{5.637529in}{1.062617in}}%
\pgfpathlineto{\pgfqpoint{5.655123in}{1.078916in}}%
\pgfpathlineto{\pgfqpoint{5.672717in}{1.089781in}}%
\pgfpathlineto{\pgfqpoint{5.707904in}{1.120049in}}%
\pgfpathlineto{\pgfqpoint{5.725498in}{1.141004in}}%
\pgfpathlineto{\pgfqpoint{5.760686in}{1.193004in}}%
\pgfpathlineto{\pgfqpoint{5.778280in}{1.231809in}}%
\pgfpathlineto{\pgfqpoint{5.795873in}{1.263630in}}%
\pgfpathlineto{\pgfqpoint{5.831061in}{1.319122in}}%
\pgfpathlineto{\pgfqpoint{5.848655in}{1.362972in}}%
\pgfpathlineto{\pgfqpoint{5.866249in}{1.393240in}}%
\pgfpathlineto{\pgfqpoint{5.883843in}{1.429329in}}%
\pgfpathlineto{\pgfqpoint{5.901437in}{1.455717in}}%
\pgfpathlineto{\pgfqpoint{5.936624in}{1.520910in}}%
\pgfpathlineto{\pgfqpoint{5.954218in}{1.536821in}}%
\pgfpathlineto{\pgfqpoint{5.971812in}{1.563790in}}%
\pgfpathlineto{\pgfqpoint{5.989406in}{1.575238in}}%
\pgfpathlineto{\pgfqpoint{6.007000in}{1.584551in}}%
\pgfpathlineto{\pgfqpoint{6.024594in}{1.595805in}}%
\pgfpathlineto{\pgfqpoint{6.042187in}{1.602014in}}%
\pgfpathlineto{\pgfqpoint{6.059781in}{1.596387in}}%
\pgfpathlineto{\pgfqpoint{6.077375in}{1.592894in}}%
\pgfpathlineto{\pgfqpoint{6.094969in}{1.581932in}}%
\pgfpathlineto{\pgfqpoint{6.112563in}{1.564178in}}%
\pgfpathlineto{\pgfqpoint{6.130157in}{1.549820in}}%
\pgfpathlineto{\pgfqpoint{6.147751in}{1.529059in}}%
\pgfpathlineto{\pgfqpoint{6.200532in}{1.457754in}}%
\pgfpathlineto{\pgfqpoint{6.253314in}{1.398570in}}%
\pgfpathlineto{\pgfqpoint{6.270907in}{1.387668in}}%
\pgfpathlineto{\pgfqpoint{6.288501in}{1.371679in}}%
\pgfpathlineto{\pgfqpoint{6.341283in}{1.336924in}}%
\pgfpathlineto{\pgfqpoint{6.358877in}{1.318006in}}%
\pgfpathlineto{\pgfqpoint{6.376471in}{1.285894in}}%
\pgfpathlineto{\pgfqpoint{6.376471in}{1.285894in}}%
\pgfusepath{stroke}%
\end{pgfscope}%
\begin{pgfscope}%
\pgfsetrectcap%
\pgfsetmiterjoin%
\pgfsetlinewidth{0.803000pt}%
\definecolor{currentstroke}{rgb}{0.000000,0.000000,0.000000}%
\pgfsetstrokecolor{currentstroke}%
\pgfsetdash{}{0pt}%
\pgfpathmoveto{\pgfqpoint{2.000000in}{0.750000in}}%
\pgfpathlineto{\pgfqpoint{2.000000in}{1.703684in}}%
\pgfusepath{stroke}%
\end{pgfscope}%
\begin{pgfscope}%
\pgfsetrectcap%
\pgfsetmiterjoin%
\pgfsetlinewidth{0.803000pt}%
\definecolor{currentstroke}{rgb}{0.000000,0.000000,0.000000}%
\pgfsetstrokecolor{currentstroke}%
\pgfsetdash{}{0pt}%
\pgfpathmoveto{\pgfqpoint{6.376471in}{0.750000in}}%
\pgfpathlineto{\pgfqpoint{6.376471in}{1.703684in}}%
\pgfusepath{stroke}%
\end{pgfscope}%
\begin{pgfscope}%
\pgfsetrectcap%
\pgfsetmiterjoin%
\pgfsetlinewidth{0.803000pt}%
\definecolor{currentstroke}{rgb}{0.000000,0.000000,0.000000}%
\pgfsetstrokecolor{currentstroke}%
\pgfsetdash{}{0pt}%
\pgfpathmoveto{\pgfqpoint{2.000000in}{0.750000in}}%
\pgfpathlineto{\pgfqpoint{6.376471in}{0.750000in}}%
\pgfusepath{stroke}%
\end{pgfscope}%
\begin{pgfscope}%
\pgfsetrectcap%
\pgfsetmiterjoin%
\pgfsetlinewidth{0.803000pt}%
\definecolor{currentstroke}{rgb}{0.000000,0.000000,0.000000}%
\pgfsetstrokecolor{currentstroke}%
\pgfsetdash{}{0pt}%
\pgfpathmoveto{\pgfqpoint{2.000000in}{1.703684in}}%
\pgfpathlineto{\pgfqpoint{6.376471in}{1.703684in}}%
\pgfusepath{stroke}%
\end{pgfscope}%
\begin{pgfscope}%
\pgfsetbuttcap%
\pgfsetmiterjoin%
\definecolor{currentfill}{rgb}{1.000000,1.000000,1.000000}%
\pgfsetfillcolor{currentfill}%
\pgfsetfillopacity{0.800000}%
\pgfsetlinewidth{1.003750pt}%
\definecolor{currentstroke}{rgb}{0.800000,0.800000,0.800000}%
\pgfsetstrokecolor{currentstroke}%
\pgfsetstrokeopacity{0.800000}%
\pgfsetdash{}{0pt}%
\pgfpathmoveto{\pgfqpoint{2.097222in}{0.819444in}}%
\pgfpathlineto{\pgfqpoint{2.796183in}{0.819444in}}%
\pgfpathquadraticcurveto{\pgfqpoint{2.823961in}{0.819444in}}{\pgfqpoint{2.823961in}{0.847222in}}%
\pgfpathlineto{\pgfqpoint{2.823961in}{1.635796in}}%
\pgfpathquadraticcurveto{\pgfqpoint{2.823961in}{1.663574in}}{\pgfqpoint{2.796183in}{1.663574in}}%
\pgfpathlineto{\pgfqpoint{2.097222in}{1.663574in}}%
\pgfpathquadraticcurveto{\pgfqpoint{2.069444in}{1.663574in}}{\pgfqpoint{2.069444in}{1.635796in}}%
\pgfpathlineto{\pgfqpoint{2.069444in}{0.847222in}}%
\pgfpathquadraticcurveto{\pgfqpoint{2.069444in}{0.819444in}}{\pgfqpoint{2.097222in}{0.819444in}}%
\pgfpathclose%
\pgfusepath{stroke,fill}%
\end{pgfscope}%
\begin{pgfscope}%
\pgfsetrectcap%
\pgfsetroundjoin%
\pgfsetlinewidth{1.505625pt}%
\definecolor{currentstroke}{rgb}{0.121569,0.466667,0.705882}%
\pgfsetstrokecolor{currentstroke}%
\pgfsetdash{}{0pt}%
\pgfpathmoveto{\pgfqpoint{2.125000in}{1.550916in}}%
\pgfpathlineto{\pgfqpoint{2.402778in}{1.550916in}}%
\pgfusepath{stroke}%
\end{pgfscope}%
\begin{pgfscope}%
\pgftext[x=2.513889in,y=1.502305in,left,base]{\rmfamily\fontsize{10.000000}{12.000000}\selectfont \(\displaystyle \widetilde{\Phi}^* \theta\)}%
\end{pgfscope}%
\begin{pgfscope}%
\pgfsetbuttcap%
\pgfsetroundjoin%
\pgfsetlinewidth{1.505625pt}%
\definecolor{currentstroke}{rgb}{1.000000,0.498039,0.054902}%
\pgfsetstrokecolor{currentstroke}%
\pgfsetdash{{9.600000pt}{2.400000pt}{1.600000pt}{2.400000pt}}{0.000000pt}%
\pgfpathmoveto{\pgfqpoint{2.125000in}{1.346055in}}%
\pgfpathlineto{\pgfqpoint{2.402778in}{1.346055in}}%
\pgfusepath{stroke}%
\end{pgfscope}%
\begin{pgfscope}%
\pgftext[x=2.513889in,y=1.297444in,left,base]{\rmfamily\fontsize{10.000000}{12.000000}\selectfont \(\displaystyle \widetilde{K}u\)}%
\end{pgfscope}%
\begin{pgfscope}%
\pgfsetbuttcap%
\pgfsetroundjoin%
\definecolor{currentfill}{rgb}{1.000000,0.000000,0.000000}%
\pgfsetfillcolor{currentfill}%
\pgfsetlinewidth{2.007500pt}%
\definecolor{currentstroke}{rgb}{1.000000,0.000000,0.000000}%
\pgfsetstrokecolor{currentstroke}%
\pgfsetdash{}{0pt}%
\pgfpathmoveto{\pgfqpoint{2.222222in}{1.137532in}}%
\pgfpathlineto{\pgfqpoint{2.305556in}{1.137532in}}%
\pgfpathmoveto{\pgfqpoint{2.263889in}{1.095866in}}%
\pgfpathlineto{\pgfqpoint{2.263889in}{1.179199in}}%
\pgfusepath{stroke,fill}%
\end{pgfscope}%
\begin{pgfscope}%
\pgftext[x=2.513889in,y=1.101074in,left,base]{\rmfamily\fontsize{10.000000}{12.000000}\selectfont train}%
\end{pgfscope}%
\begin{pgfscope}%
\pgfsetbuttcap%
\pgfsetroundjoin%
\definecolor{currentfill}{rgb}{0.000000,0.000000,0.000000}%
\pgfsetfillcolor{currentfill}%
\pgfsetlinewidth{1.003750pt}%
\definecolor{currentstroke}{rgb}{0.000000,0.000000,0.000000}%
\pgfsetstrokecolor{currentstroke}%
\pgfsetdash{}{0pt}%
\pgfsys@defobject{currentmarker}{\pgfqpoint{-0.020833in}{-0.020833in}}{\pgfqpoint{0.020833in}{0.020833in}}{%
\pgfpathmoveto{\pgfqpoint{0.000000in}{-0.020833in}}%
\pgfpathcurveto{\pgfqpoint{0.005525in}{-0.020833in}}{\pgfqpoint{0.010825in}{-0.018638in}}{\pgfqpoint{0.014731in}{-0.014731in}}%
\pgfpathcurveto{\pgfqpoint{0.018638in}{-0.010825in}}{\pgfqpoint{0.020833in}{-0.005525in}}{\pgfqpoint{0.020833in}{0.000000in}}%
\pgfpathcurveto{\pgfqpoint{0.020833in}{0.005525in}}{\pgfqpoint{0.018638in}{0.010825in}}{\pgfqpoint{0.014731in}{0.014731in}}%
\pgfpathcurveto{\pgfqpoint{0.010825in}{0.018638in}}{\pgfqpoint{0.005525in}{0.020833in}}{\pgfqpoint{0.000000in}{0.020833in}}%
\pgfpathcurveto{\pgfqpoint{-0.005525in}{0.020833in}}{\pgfqpoint{-0.010825in}{0.018638in}}{\pgfqpoint{-0.014731in}{0.014731in}}%
\pgfpathcurveto{\pgfqpoint{-0.018638in}{0.010825in}}{\pgfqpoint{-0.020833in}{0.005525in}}{\pgfqpoint{-0.020833in}{0.000000in}}%
\pgfpathcurveto{\pgfqpoint{-0.020833in}{-0.005525in}}{\pgfqpoint{-0.018638in}{-0.010825in}}{\pgfqpoint{-0.014731in}{-0.014731in}}%
\pgfpathcurveto{\pgfqpoint{-0.010825in}{-0.018638in}}{\pgfqpoint{-0.005525in}{-0.020833in}}{\pgfqpoint{0.000000in}{-0.020833in}}%
\pgfpathclose%
\pgfusepath{stroke,fill}%
}%
\begin{pgfscope}%
\pgfsys@transformshift{2.263889in}{0.941162in}%
\pgfsys@useobject{currentmarker}{}%
\end{pgfscope}%
\end{pgfscope}%
\begin{pgfscope}%
\pgftext[x=2.513889in,y=0.904704in,left,base]{\rmfamily\fontsize{10.000000}{12.000000}\selectfont test}%
\end{pgfscope}%
\begin{pgfscope}%
\pgfsetbuttcap%
\pgfsetmiterjoin%
\definecolor{currentfill}{rgb}{1.000000,1.000000,1.000000}%
\pgfsetfillcolor{currentfill}%
\pgfsetlinewidth{0.000000pt}%
\definecolor{currentstroke}{rgb}{0.000000,0.000000,0.000000}%
\pgfsetstrokecolor{currentstroke}%
\pgfsetstrokeopacity{0.000000}%
\pgfsetdash{}{0pt}%
\pgfpathmoveto{\pgfqpoint{7.105882in}{0.750000in}}%
\pgfpathlineto{\pgfqpoint{11.482353in}{0.750000in}}%
\pgfpathlineto{\pgfqpoint{11.482353in}{1.703684in}}%
\pgfpathlineto{\pgfqpoint{7.105882in}{1.703684in}}%
\pgfpathclose%
\pgfusepath{fill}%
\end{pgfscope}%
\begin{pgfscope}%
\pgfpathrectangle{\pgfqpoint{7.105882in}{0.750000in}}{\pgfqpoint{4.376471in}{0.953684in}} %
\pgfusepath{clip}%
\pgfsetbuttcap%
\pgfsetroundjoin%
\definecolor{currentfill}{rgb}{1.000000,0.000000,0.000000}%
\pgfsetfillcolor{currentfill}%
\pgfsetlinewidth{2.007500pt}%
\definecolor{currentstroke}{rgb}{1.000000,0.000000,0.000000}%
\pgfsetstrokecolor{currentstroke}%
\pgfsetdash{}{0pt}%
\pgfpathmoveto{\pgfqpoint{9.861003in}{1.106507in}}%
\pgfpathlineto{\pgfqpoint{9.944336in}{1.106507in}}%
\pgfpathmoveto{\pgfqpoint{9.902669in}{1.064841in}}%
\pgfpathlineto{\pgfqpoint{9.902669in}{1.148174in}}%
\pgfusepath{stroke,fill}%
\end{pgfscope}%
\begin{pgfscope}%
\pgfpathrectangle{\pgfqpoint{7.105882in}{0.750000in}}{\pgfqpoint{4.376471in}{0.953684in}} %
\pgfusepath{clip}%
\pgfsetbuttcap%
\pgfsetroundjoin%
\definecolor{currentfill}{rgb}{1.000000,0.000000,0.000000}%
\pgfsetfillcolor{currentfill}%
\pgfsetlinewidth{2.007500pt}%
\definecolor{currentstroke}{rgb}{1.000000,0.000000,0.000000}%
\pgfsetstrokecolor{currentstroke}%
\pgfsetdash{}{0pt}%
\pgfpathmoveto{\pgfqpoint{10.443514in}{1.192701in}}%
\pgfpathlineto{\pgfqpoint{10.526847in}{1.192701in}}%
\pgfpathmoveto{\pgfqpoint{10.485181in}{1.151034in}}%
\pgfpathlineto{\pgfqpoint{10.485181in}{1.234367in}}%
\pgfusepath{stroke,fill}%
\end{pgfscope}%
\begin{pgfscope}%
\pgfpathrectangle{\pgfqpoint{7.105882in}{0.750000in}}{\pgfqpoint{4.376471in}{0.953684in}} %
\pgfusepath{clip}%
\pgfsetbuttcap%
\pgfsetroundjoin%
\definecolor{currentfill}{rgb}{1.000000,0.000000,0.000000}%
\pgfsetfillcolor{currentfill}%
\pgfsetlinewidth{2.007500pt}%
\definecolor{currentstroke}{rgb}{1.000000,0.000000,0.000000}%
\pgfsetstrokecolor{currentstroke}%
\pgfsetdash{}{0pt}%
\pgfpathmoveto{\pgfqpoint{10.049891in}{1.249607in}}%
\pgfpathlineto{\pgfqpoint{10.133224in}{1.249607in}}%
\pgfpathmoveto{\pgfqpoint{10.091557in}{1.207940in}}%
\pgfpathlineto{\pgfqpoint{10.091557in}{1.291274in}}%
\pgfusepath{stroke,fill}%
\end{pgfscope}%
\begin{pgfscope}%
\pgfpathrectangle{\pgfqpoint{7.105882in}{0.750000in}}{\pgfqpoint{4.376471in}{0.953684in}} %
\pgfusepath{clip}%
\pgfsetbuttcap%
\pgfsetroundjoin%
\definecolor{currentfill}{rgb}{1.000000,0.000000,0.000000}%
\pgfsetfillcolor{currentfill}%
\pgfsetlinewidth{2.007500pt}%
\definecolor{currentstroke}{rgb}{1.000000,0.000000,0.000000}%
\pgfsetstrokecolor{currentstroke}%
\pgfsetdash{}{0pt}%
\pgfpathmoveto{\pgfqpoint{9.847242in}{1.222660in}}%
\pgfpathlineto{\pgfqpoint{9.930575in}{1.222660in}}%
\pgfpathmoveto{\pgfqpoint{9.888909in}{1.180993in}}%
\pgfpathlineto{\pgfqpoint{9.888909in}{1.264327in}}%
\pgfusepath{stroke,fill}%
\end{pgfscope}%
\begin{pgfscope}%
\pgfpathrectangle{\pgfqpoint{7.105882in}{0.750000in}}{\pgfqpoint{4.376471in}{0.953684in}} %
\pgfusepath{clip}%
\pgfsetbuttcap%
\pgfsetroundjoin%
\definecolor{currentfill}{rgb}{1.000000,0.000000,0.000000}%
\pgfsetfillcolor{currentfill}%
\pgfsetlinewidth{2.007500pt}%
\definecolor{currentstroke}{rgb}{1.000000,0.000000,0.000000}%
\pgfsetstrokecolor{currentstroke}%
\pgfsetdash{}{0pt}%
\pgfpathmoveto{\pgfqpoint{9.422800in}{0.793948in}}%
\pgfpathlineto{\pgfqpoint{9.506133in}{0.793948in}}%
\pgfpathmoveto{\pgfqpoint{9.464467in}{0.752281in}}%
\pgfpathlineto{\pgfqpoint{9.464467in}{0.835614in}}%
\pgfusepath{stroke,fill}%
\end{pgfscope}%
\begin{pgfscope}%
\pgfpathrectangle{\pgfqpoint{7.105882in}{0.750000in}}{\pgfqpoint{4.376471in}{0.953684in}} %
\pgfusepath{clip}%
\pgfsetbuttcap%
\pgfsetroundjoin%
\definecolor{currentfill}{rgb}{1.000000,0.000000,0.000000}%
\pgfsetfillcolor{currentfill}%
\pgfsetlinewidth{2.007500pt}%
\definecolor{currentstroke}{rgb}{1.000000,0.000000,0.000000}%
\pgfsetstrokecolor{currentstroke}%
\pgfsetdash{}{0pt}%
\pgfpathmoveto{\pgfqpoint{10.200899in}{1.262779in}}%
\pgfpathlineto{\pgfqpoint{10.284232in}{1.262779in}}%
\pgfpathmoveto{\pgfqpoint{10.242566in}{1.221112in}}%
\pgfpathlineto{\pgfqpoint{10.242566in}{1.304446in}}%
\pgfusepath{stroke,fill}%
\end{pgfscope}%
\begin{pgfscope}%
\pgfpathrectangle{\pgfqpoint{7.105882in}{0.750000in}}{\pgfqpoint{4.376471in}{0.953684in}} %
\pgfusepath{clip}%
\pgfsetbuttcap%
\pgfsetroundjoin%
\definecolor{currentfill}{rgb}{1.000000,0.000000,0.000000}%
\pgfsetfillcolor{currentfill}%
\pgfsetlinewidth{2.007500pt}%
\definecolor{currentstroke}{rgb}{1.000000,0.000000,0.000000}%
\pgfsetstrokecolor{currentstroke}%
\pgfsetdash{}{0pt}%
\pgfpathmoveto{\pgfqpoint{9.471580in}{0.765056in}}%
\pgfpathlineto{\pgfqpoint{9.554913in}{0.765056in}}%
\pgfpathmoveto{\pgfqpoint{9.513247in}{0.723390in}}%
\pgfpathlineto{\pgfqpoint{9.513247in}{0.806723in}}%
\pgfusepath{stroke,fill}%
\end{pgfscope}%
\begin{pgfscope}%
\pgfpathrectangle{\pgfqpoint{7.105882in}{0.750000in}}{\pgfqpoint{4.376471in}{0.953684in}} %
\pgfusepath{clip}%
\pgfsetbuttcap%
\pgfsetroundjoin%
\definecolor{currentfill}{rgb}{1.000000,0.000000,0.000000}%
\pgfsetfillcolor{currentfill}%
\pgfsetlinewidth{2.007500pt}%
\definecolor{currentstroke}{rgb}{1.000000,0.000000,0.000000}%
\pgfsetstrokecolor{currentstroke}%
\pgfsetdash{}{0pt}%
\pgfpathmoveto{\pgfqpoint{11.061764in}{1.529110in}}%
\pgfpathlineto{\pgfqpoint{11.145098in}{1.529110in}}%
\pgfpathmoveto{\pgfqpoint{11.103431in}{1.487443in}}%
\pgfpathlineto{\pgfqpoint{11.103431in}{1.570776in}}%
\pgfusepath{stroke,fill}%
\end{pgfscope}%
\begin{pgfscope}%
\pgfpathrectangle{\pgfqpoint{7.105882in}{0.750000in}}{\pgfqpoint{4.376471in}{0.953684in}} %
\pgfusepath{clip}%
\pgfsetbuttcap%
\pgfsetroundjoin%
\definecolor{currentfill}{rgb}{1.000000,0.000000,0.000000}%
\pgfsetfillcolor{currentfill}%
\pgfsetlinewidth{2.007500pt}%
\definecolor{currentstroke}{rgb}{1.000000,0.000000,0.000000}%
\pgfsetstrokecolor{currentstroke}%
\pgfsetdash{}{0pt}%
\pgfpathmoveto{\pgfqpoint{11.313463in}{1.390374in}}%
\pgfpathlineto{\pgfqpoint{11.396797in}{1.390374in}}%
\pgfpathmoveto{\pgfqpoint{11.355130in}{1.348708in}}%
\pgfpathlineto{\pgfqpoint{11.355130in}{1.432041in}}%
\pgfusepath{stroke,fill}%
\end{pgfscope}%
\begin{pgfscope}%
\pgfpathrectangle{\pgfqpoint{7.105882in}{0.750000in}}{\pgfqpoint{4.376471in}{0.953684in}} %
\pgfusepath{clip}%
\pgfsetbuttcap%
\pgfsetroundjoin%
\definecolor{currentfill}{rgb}{1.000000,0.000000,0.000000}%
\pgfsetfillcolor{currentfill}%
\pgfsetlinewidth{2.007500pt}%
\definecolor{currentstroke}{rgb}{1.000000,0.000000,0.000000}%
\pgfsetstrokecolor{currentstroke}%
\pgfsetdash{}{0pt}%
\pgfpathmoveto{\pgfqpoint{9.282006in}{1.023699in}}%
\pgfpathlineto{\pgfqpoint{9.365340in}{1.023699in}}%
\pgfpathmoveto{\pgfqpoint{9.323673in}{0.982033in}}%
\pgfpathlineto{\pgfqpoint{9.323673in}{1.065366in}}%
\pgfusepath{stroke,fill}%
\end{pgfscope}%
\begin{pgfscope}%
\pgfpathrectangle{\pgfqpoint{7.105882in}{0.750000in}}{\pgfqpoint{4.376471in}{0.953684in}} %
\pgfusepath{clip}%
\pgfsetbuttcap%
\pgfsetroundjoin%
\definecolor{currentfill}{rgb}{1.000000,0.000000,0.000000}%
\pgfsetfillcolor{currentfill}%
\pgfsetlinewidth{2.007500pt}%
\definecolor{currentstroke}{rgb}{1.000000,0.000000,0.000000}%
\pgfsetstrokecolor{currentstroke}%
\pgfsetdash{}{0pt}%
\pgfpathmoveto{\pgfqpoint{10.711479in}{1.105972in}}%
\pgfpathlineto{\pgfqpoint{10.794812in}{1.105972in}}%
\pgfpathmoveto{\pgfqpoint{10.753146in}{1.064305in}}%
\pgfpathlineto{\pgfqpoint{10.753146in}{1.147638in}}%
\pgfusepath{stroke,fill}%
\end{pgfscope}%
\begin{pgfscope}%
\pgfpathrectangle{\pgfqpoint{7.105882in}{0.750000in}}{\pgfqpoint{4.376471in}{0.953684in}} %
\pgfusepath{clip}%
\pgfsetbuttcap%
\pgfsetroundjoin%
\definecolor{currentfill}{rgb}{1.000000,0.000000,0.000000}%
\pgfsetfillcolor{currentfill}%
\pgfsetlinewidth{2.007500pt}%
\definecolor{currentstroke}{rgb}{1.000000,0.000000,0.000000}%
\pgfsetstrokecolor{currentstroke}%
\pgfsetdash{}{0pt}%
\pgfpathmoveto{\pgfqpoint{9.791264in}{1.100206in}}%
\pgfpathlineto{\pgfqpoint{9.874598in}{1.100206in}}%
\pgfpathmoveto{\pgfqpoint{9.832931in}{1.058540in}}%
\pgfpathlineto{\pgfqpoint{9.832931in}{1.141873in}}%
\pgfusepath{stroke,fill}%
\end{pgfscope}%
\begin{pgfscope}%
\pgfpathrectangle{\pgfqpoint{7.105882in}{0.750000in}}{\pgfqpoint{4.376471in}{0.953684in}} %
\pgfusepath{clip}%
\pgfsetbuttcap%
\pgfsetroundjoin%
\definecolor{currentfill}{rgb}{1.000000,0.000000,0.000000}%
\pgfsetfillcolor{currentfill}%
\pgfsetlinewidth{2.007500pt}%
\definecolor{currentstroke}{rgb}{1.000000,0.000000,0.000000}%
\pgfsetstrokecolor{currentstroke}%
\pgfsetdash{}{0pt}%
\pgfpathmoveto{\pgfqpoint{9.928334in}{1.306514in}}%
\pgfpathlineto{\pgfqpoint{10.011667in}{1.306514in}}%
\pgfpathmoveto{\pgfqpoint{9.970001in}{1.264848in}}%
\pgfpathlineto{\pgfqpoint{9.970001in}{1.348181in}}%
\pgfusepath{stroke,fill}%
\end{pgfscope}%
\begin{pgfscope}%
\pgfpathrectangle{\pgfqpoint{7.105882in}{0.750000in}}{\pgfqpoint{4.376471in}{0.953684in}} %
\pgfusepath{clip}%
\pgfsetbuttcap%
\pgfsetroundjoin%
\definecolor{currentfill}{rgb}{1.000000,0.000000,0.000000}%
\pgfsetfillcolor{currentfill}%
\pgfsetlinewidth{2.007500pt}%
\definecolor{currentstroke}{rgb}{1.000000,0.000000,0.000000}%
\pgfsetstrokecolor{currentstroke}%
\pgfsetdash{}{0pt}%
\pgfpathmoveto{\pgfqpoint{11.180187in}{1.595297in}}%
\pgfpathlineto{\pgfqpoint{11.263520in}{1.595297in}}%
\pgfpathmoveto{\pgfqpoint{11.221854in}{1.553630in}}%
\pgfpathlineto{\pgfqpoint{11.221854in}{1.636963in}}%
\pgfusepath{stroke,fill}%
\end{pgfscope}%
\begin{pgfscope}%
\pgfpathrectangle{\pgfqpoint{7.105882in}{0.750000in}}{\pgfqpoint{4.376471in}{0.953684in}} %
\pgfusepath{clip}%
\pgfsetbuttcap%
\pgfsetroundjoin%
\definecolor{currentfill}{rgb}{1.000000,0.000000,0.000000}%
\pgfsetfillcolor{currentfill}%
\pgfsetlinewidth{2.007500pt}%
\definecolor{currentstroke}{rgb}{1.000000,0.000000,0.000000}%
\pgfsetstrokecolor{currentstroke}%
\pgfsetdash{}{0pt}%
\pgfpathmoveto{\pgfqpoint{8.188220in}{1.258043in}}%
\pgfpathlineto{\pgfqpoint{8.271553in}{1.258043in}}%
\pgfpathmoveto{\pgfqpoint{8.229886in}{1.216376in}}%
\pgfpathlineto{\pgfqpoint{8.229886in}{1.299710in}}%
\pgfusepath{stroke,fill}%
\end{pgfscope}%
\begin{pgfscope}%
\pgfpathrectangle{\pgfqpoint{7.105882in}{0.750000in}}{\pgfqpoint{4.376471in}{0.953684in}} %
\pgfusepath{clip}%
\pgfsetbuttcap%
\pgfsetroundjoin%
\definecolor{currentfill}{rgb}{1.000000,0.000000,0.000000}%
\pgfsetfillcolor{currentfill}%
\pgfsetlinewidth{2.007500pt}%
\definecolor{currentstroke}{rgb}{1.000000,0.000000,0.000000}%
\pgfsetstrokecolor{currentstroke}%
\pgfsetdash{}{0pt}%
\pgfpathmoveto{\pgfqpoint{8.244565in}{1.225891in}}%
\pgfpathlineto{\pgfqpoint{8.327898in}{1.225891in}}%
\pgfpathmoveto{\pgfqpoint{8.286232in}{1.184225in}}%
\pgfpathlineto{\pgfqpoint{8.286232in}{1.267558in}}%
\pgfusepath{stroke,fill}%
\end{pgfscope}%
\begin{pgfscope}%
\pgfpathrectangle{\pgfqpoint{7.105882in}{0.750000in}}{\pgfqpoint{4.376471in}{0.953684in}} %
\pgfusepath{clip}%
\pgfsetbuttcap%
\pgfsetroundjoin%
\definecolor{currentfill}{rgb}{1.000000,0.000000,0.000000}%
\pgfsetfillcolor{currentfill}%
\pgfsetlinewidth{2.007500pt}%
\definecolor{currentstroke}{rgb}{1.000000,0.000000,0.000000}%
\pgfsetstrokecolor{currentstroke}%
\pgfsetdash{}{0pt}%
\pgfpathmoveto{\pgfqpoint{8.010298in}{1.444279in}}%
\pgfpathlineto{\pgfqpoint{8.093631in}{1.444279in}}%
\pgfpathmoveto{\pgfqpoint{8.051965in}{1.402613in}}%
\pgfpathlineto{\pgfqpoint{8.051965in}{1.485946in}}%
\pgfusepath{stroke,fill}%
\end{pgfscope}%
\begin{pgfscope}%
\pgfpathrectangle{\pgfqpoint{7.105882in}{0.750000in}}{\pgfqpoint{4.376471in}{0.953684in}} %
\pgfusepath{clip}%
\pgfsetbuttcap%
\pgfsetroundjoin%
\definecolor{currentfill}{rgb}{1.000000,0.000000,0.000000}%
\pgfsetfillcolor{currentfill}%
\pgfsetlinewidth{2.007500pt}%
\definecolor{currentstroke}{rgb}{1.000000,0.000000,0.000000}%
\pgfsetstrokecolor{currentstroke}%
\pgfsetdash{}{0pt}%
\pgfpathmoveto{\pgfqpoint{10.854659in}{1.319150in}}%
\pgfpathlineto{\pgfqpoint{10.937992in}{1.319150in}}%
\pgfpathmoveto{\pgfqpoint{10.896325in}{1.277484in}}%
\pgfpathlineto{\pgfqpoint{10.896325in}{1.360817in}}%
\pgfusepath{stroke,fill}%
\end{pgfscope}%
\begin{pgfscope}%
\pgfpathrectangle{\pgfqpoint{7.105882in}{0.750000in}}{\pgfqpoint{4.376471in}{0.953684in}} %
\pgfusepath{clip}%
\pgfsetbuttcap%
\pgfsetroundjoin%
\definecolor{currentfill}{rgb}{1.000000,0.000000,0.000000}%
\pgfsetfillcolor{currentfill}%
\pgfsetlinewidth{2.007500pt}%
\definecolor{currentstroke}{rgb}{1.000000,0.000000,0.000000}%
\pgfsetstrokecolor{currentstroke}%
\pgfsetdash{}{0pt}%
\pgfpathmoveto{\pgfqpoint{10.663974in}{1.032791in}}%
\pgfpathlineto{\pgfqpoint{10.747307in}{1.032791in}}%
\pgfpathmoveto{\pgfqpoint{10.705641in}{0.991124in}}%
\pgfpathlineto{\pgfqpoint{10.705641in}{1.074457in}}%
\pgfusepath{stroke,fill}%
\end{pgfscope}%
\begin{pgfscope}%
\pgfpathrectangle{\pgfqpoint{7.105882in}{0.750000in}}{\pgfqpoint{4.376471in}{0.953684in}} %
\pgfusepath{clip}%
\pgfsetbuttcap%
\pgfsetroundjoin%
\definecolor{currentfill}{rgb}{1.000000,0.000000,0.000000}%
\pgfsetfillcolor{currentfill}%
\pgfsetlinewidth{2.007500pt}%
\definecolor{currentstroke}{rgb}{1.000000,0.000000,0.000000}%
\pgfsetstrokecolor{currentstroke}%
\pgfsetdash{}{0pt}%
\pgfpathmoveto{\pgfqpoint{10.985576in}{1.468106in}}%
\pgfpathlineto{\pgfqpoint{11.068909in}{1.468106in}}%
\pgfpathmoveto{\pgfqpoint{11.027243in}{1.426440in}}%
\pgfpathlineto{\pgfqpoint{11.027243in}{1.509773in}}%
\pgfusepath{stroke,fill}%
\end{pgfscope}%
\begin{pgfscope}%
\pgfpathrectangle{\pgfqpoint{7.105882in}{0.750000in}}{\pgfqpoint{4.376471in}{0.953684in}} %
\pgfusepath{clip}%
\pgfsetbuttcap%
\pgfsetroundjoin%
\definecolor{currentfill}{rgb}{1.000000,0.000000,0.000000}%
\pgfsetfillcolor{currentfill}%
\pgfsetlinewidth{2.007500pt}%
\definecolor{currentstroke}{rgb}{1.000000,0.000000,0.000000}%
\pgfsetstrokecolor{currentstroke}%
\pgfsetdash{}{0pt}%
\pgfpathmoveto{\pgfqpoint{11.365825in}{1.361744in}}%
\pgfpathlineto{\pgfqpoint{11.449159in}{1.361744in}}%
\pgfpathmoveto{\pgfqpoint{11.407492in}{1.320078in}}%
\pgfpathlineto{\pgfqpoint{11.407492in}{1.403411in}}%
\pgfusepath{stroke,fill}%
\end{pgfscope}%
\begin{pgfscope}%
\pgfpathrectangle{\pgfqpoint{7.105882in}{0.750000in}}{\pgfqpoint{4.376471in}{0.953684in}} %
\pgfusepath{clip}%
\pgfsetbuttcap%
\pgfsetroundjoin%
\definecolor{currentfill}{rgb}{1.000000,0.000000,0.000000}%
\pgfsetfillcolor{currentfill}%
\pgfsetlinewidth{2.007500pt}%
\definecolor{currentstroke}{rgb}{1.000000,0.000000,0.000000}%
\pgfsetstrokecolor{currentstroke}%
\pgfsetdash{}{0pt}%
\pgfpathmoveto{\pgfqpoint{10.737505in}{1.019672in}}%
\pgfpathlineto{\pgfqpoint{10.820838in}{1.019672in}}%
\pgfpathmoveto{\pgfqpoint{10.779172in}{0.978006in}}%
\pgfpathlineto{\pgfqpoint{10.779172in}{1.061339in}}%
\pgfusepath{stroke,fill}%
\end{pgfscope}%
\begin{pgfscope}%
\pgfpathrectangle{\pgfqpoint{7.105882in}{0.750000in}}{\pgfqpoint{4.376471in}{0.953684in}} %
\pgfusepath{clip}%
\pgfsetbuttcap%
\pgfsetroundjoin%
\definecolor{currentfill}{rgb}{1.000000,0.000000,0.000000}%
\pgfsetfillcolor{currentfill}%
\pgfsetlinewidth{2.007500pt}%
\definecolor{currentstroke}{rgb}{1.000000,0.000000,0.000000}%
\pgfsetstrokecolor{currentstroke}%
\pgfsetdash{}{0pt}%
\pgfpathmoveto{\pgfqpoint{9.555230in}{0.989246in}}%
\pgfpathlineto{\pgfqpoint{9.638564in}{0.989246in}}%
\pgfpathmoveto{\pgfqpoint{9.596897in}{0.947579in}}%
\pgfpathlineto{\pgfqpoint{9.596897in}{1.030913in}}%
\pgfusepath{stroke,fill}%
\end{pgfscope}%
\begin{pgfscope}%
\pgfpathrectangle{\pgfqpoint{7.105882in}{0.750000in}}{\pgfqpoint{4.376471in}{0.953684in}} %
\pgfusepath{clip}%
\pgfsetbuttcap%
\pgfsetroundjoin%
\definecolor{currentfill}{rgb}{1.000000,0.000000,0.000000}%
\pgfsetfillcolor{currentfill}%
\pgfsetlinewidth{2.007500pt}%
\definecolor{currentstroke}{rgb}{1.000000,0.000000,0.000000}%
\pgfsetstrokecolor{currentstroke}%
\pgfsetdash{}{0pt}%
\pgfpathmoveto{\pgfqpoint{10.672280in}{1.069019in}}%
\pgfpathlineto{\pgfqpoint{10.755614in}{1.069019in}}%
\pgfpathmoveto{\pgfqpoint{10.713947in}{1.027352in}}%
\pgfpathlineto{\pgfqpoint{10.713947in}{1.110686in}}%
\pgfusepath{stroke,fill}%
\end{pgfscope}%
\begin{pgfscope}%
\pgfpathrectangle{\pgfqpoint{7.105882in}{0.750000in}}{\pgfqpoint{4.376471in}{0.953684in}} %
\pgfusepath{clip}%
\pgfsetbuttcap%
\pgfsetroundjoin%
\definecolor{currentfill}{rgb}{1.000000,0.000000,0.000000}%
\pgfsetfillcolor{currentfill}%
\pgfsetlinewidth{2.007500pt}%
\definecolor{currentstroke}{rgb}{1.000000,0.000000,0.000000}%
\pgfsetstrokecolor{currentstroke}%
\pgfsetdash{}{0pt}%
\pgfpathmoveto{\pgfqpoint{8.353609in}{0.983375in}}%
\pgfpathlineto{\pgfqpoint{8.436943in}{0.983375in}}%
\pgfpathmoveto{\pgfqpoint{8.395276in}{0.941708in}}%
\pgfpathlineto{\pgfqpoint{8.395276in}{1.025042in}}%
\pgfusepath{stroke,fill}%
\end{pgfscope}%
\begin{pgfscope}%
\pgfpathrectangle{\pgfqpoint{7.105882in}{0.750000in}}{\pgfqpoint{4.376471in}{0.953684in}} %
\pgfusepath{clip}%
\pgfsetbuttcap%
\pgfsetroundjoin%
\definecolor{currentfill}{rgb}{1.000000,0.000000,0.000000}%
\pgfsetfillcolor{currentfill}%
\pgfsetlinewidth{2.007500pt}%
\definecolor{currentstroke}{rgb}{1.000000,0.000000,0.000000}%
\pgfsetstrokecolor{currentstroke}%
\pgfsetdash{}{0pt}%
\pgfpathmoveto{\pgfqpoint{10.179986in}{1.251576in}}%
\pgfpathlineto{\pgfqpoint{10.263320in}{1.251576in}}%
\pgfpathmoveto{\pgfqpoint{10.221653in}{1.209910in}}%
\pgfpathlineto{\pgfqpoint{10.221653in}{1.293243in}}%
\pgfusepath{stroke,fill}%
\end{pgfscope}%
\begin{pgfscope}%
\pgfpathrectangle{\pgfqpoint{7.105882in}{0.750000in}}{\pgfqpoint{4.376471in}{0.953684in}} %
\pgfusepath{clip}%
\pgfsetbuttcap%
\pgfsetroundjoin%
\definecolor{currentfill}{rgb}{1.000000,0.000000,0.000000}%
\pgfsetfillcolor{currentfill}%
\pgfsetlinewidth{2.007500pt}%
\definecolor{currentstroke}{rgb}{1.000000,0.000000,0.000000}%
\pgfsetstrokecolor{currentstroke}%
\pgfsetdash{}{0pt}%
\pgfpathmoveto{\pgfqpoint{8.441415in}{1.070440in}}%
\pgfpathlineto{\pgfqpoint{8.524748in}{1.070440in}}%
\pgfpathmoveto{\pgfqpoint{8.483082in}{1.028773in}}%
\pgfpathlineto{\pgfqpoint{8.483082in}{1.112107in}}%
\pgfusepath{stroke,fill}%
\end{pgfscope}%
\begin{pgfscope}%
\pgfpathrectangle{\pgfqpoint{7.105882in}{0.750000in}}{\pgfqpoint{4.376471in}{0.953684in}} %
\pgfusepath{clip}%
\pgfsetbuttcap%
\pgfsetroundjoin%
\definecolor{currentfill}{rgb}{1.000000,0.000000,0.000000}%
\pgfsetfillcolor{currentfill}%
\pgfsetlinewidth{2.007500pt}%
\definecolor{currentstroke}{rgb}{1.000000,0.000000,0.000000}%
\pgfsetstrokecolor{currentstroke}%
\pgfsetdash{}{0pt}%
\pgfpathmoveto{\pgfqpoint{11.246962in}{1.527761in}}%
\pgfpathlineto{\pgfqpoint{11.330296in}{1.527761in}}%
\pgfpathmoveto{\pgfqpoint{11.288629in}{1.486094in}}%
\pgfpathlineto{\pgfqpoint{11.288629in}{1.569427in}}%
\pgfusepath{stroke,fill}%
\end{pgfscope}%
\begin{pgfscope}%
\pgfpathrectangle{\pgfqpoint{7.105882in}{0.750000in}}{\pgfqpoint{4.376471in}{0.953684in}} %
\pgfusepath{clip}%
\pgfsetbuttcap%
\pgfsetroundjoin%
\definecolor{currentfill}{rgb}{1.000000,0.000000,0.000000}%
\pgfsetfillcolor{currentfill}%
\pgfsetlinewidth{2.007500pt}%
\definecolor{currentstroke}{rgb}{1.000000,0.000000,0.000000}%
\pgfsetstrokecolor{currentstroke}%
\pgfsetdash{}{0pt}%
\pgfpathmoveto{\pgfqpoint{9.766593in}{1.264901in}}%
\pgfpathlineto{\pgfqpoint{9.849926in}{1.264901in}}%
\pgfpathmoveto{\pgfqpoint{9.808260in}{1.223235in}}%
\pgfpathlineto{\pgfqpoint{9.808260in}{1.306568in}}%
\pgfusepath{stroke,fill}%
\end{pgfscope}%
\begin{pgfscope}%
\pgfpathrectangle{\pgfqpoint{7.105882in}{0.750000in}}{\pgfqpoint{4.376471in}{0.953684in}} %
\pgfusepath{clip}%
\pgfsetbuttcap%
\pgfsetroundjoin%
\definecolor{currentfill}{rgb}{1.000000,0.000000,0.000000}%
\pgfsetfillcolor{currentfill}%
\pgfsetlinewidth{2.007500pt}%
\definecolor{currentstroke}{rgb}{1.000000,0.000000,0.000000}%
\pgfsetstrokecolor{currentstroke}%
\pgfsetdash{}{0pt}%
\pgfpathmoveto{\pgfqpoint{9.391314in}{0.935170in}}%
\pgfpathlineto{\pgfqpoint{9.474648in}{0.935170in}}%
\pgfpathmoveto{\pgfqpoint{9.432981in}{0.893503in}}%
\pgfpathlineto{\pgfqpoint{9.432981in}{0.976837in}}%
\pgfusepath{stroke,fill}%
\end{pgfscope}%
\begin{pgfscope}%
\pgfpathrectangle{\pgfqpoint{7.105882in}{0.750000in}}{\pgfqpoint{4.376471in}{0.953684in}} %
\pgfusepath{clip}%
\pgfsetbuttcap%
\pgfsetroundjoin%
\definecolor{currentfill}{rgb}{1.000000,0.000000,0.000000}%
\pgfsetfillcolor{currentfill}%
\pgfsetlinewidth{2.007500pt}%
\definecolor{currentstroke}{rgb}{1.000000,0.000000,0.000000}%
\pgfsetstrokecolor{currentstroke}%
\pgfsetdash{}{0pt}%
\pgfpathmoveto{\pgfqpoint{8.865766in}{1.507616in}}%
\pgfpathlineto{\pgfqpoint{8.949099in}{1.507616in}}%
\pgfpathmoveto{\pgfqpoint{8.907432in}{1.465949in}}%
\pgfpathlineto{\pgfqpoint{8.907432in}{1.549283in}}%
\pgfusepath{stroke,fill}%
\end{pgfscope}%
\begin{pgfscope}%
\pgfpathrectangle{\pgfqpoint{7.105882in}{0.750000in}}{\pgfqpoint{4.376471in}{0.953684in}} %
\pgfusepath{clip}%
\pgfsetbuttcap%
\pgfsetroundjoin%
\definecolor{currentfill}{rgb}{1.000000,0.000000,0.000000}%
\pgfsetfillcolor{currentfill}%
\pgfsetlinewidth{2.007500pt}%
\definecolor{currentstroke}{rgb}{1.000000,0.000000,0.000000}%
\pgfsetstrokecolor{currentstroke}%
\pgfsetdash{}{0pt}%
\pgfpathmoveto{\pgfqpoint{10.650239in}{0.962175in}}%
\pgfpathlineto{\pgfqpoint{10.733572in}{0.962175in}}%
\pgfpathmoveto{\pgfqpoint{10.691905in}{0.920509in}}%
\pgfpathlineto{\pgfqpoint{10.691905in}{1.003842in}}%
\pgfusepath{stroke,fill}%
\end{pgfscope}%
\begin{pgfscope}%
\pgfpathrectangle{\pgfqpoint{7.105882in}{0.750000in}}{\pgfqpoint{4.376471in}{0.953684in}} %
\pgfusepath{clip}%
\pgfsetbuttcap%
\pgfsetroundjoin%
\definecolor{currentfill}{rgb}{1.000000,0.000000,0.000000}%
\pgfsetfillcolor{currentfill}%
\pgfsetlinewidth{2.007500pt}%
\definecolor{currentstroke}{rgb}{1.000000,0.000000,0.000000}%
\pgfsetstrokecolor{currentstroke}%
\pgfsetdash{}{0pt}%
\pgfpathmoveto{\pgfqpoint{9.536573in}{0.997905in}}%
\pgfpathlineto{\pgfqpoint{9.619906in}{0.997905in}}%
\pgfpathmoveto{\pgfqpoint{9.578239in}{0.956238in}}%
\pgfpathlineto{\pgfqpoint{9.578239in}{1.039572in}}%
\pgfusepath{stroke,fill}%
\end{pgfscope}%
\begin{pgfscope}%
\pgfpathrectangle{\pgfqpoint{7.105882in}{0.750000in}}{\pgfqpoint{4.376471in}{0.953684in}} %
\pgfusepath{clip}%
\pgfsetbuttcap%
\pgfsetroundjoin%
\definecolor{currentfill}{rgb}{1.000000,0.000000,0.000000}%
\pgfsetfillcolor{currentfill}%
\pgfsetlinewidth{2.007500pt}%
\definecolor{currentstroke}{rgb}{1.000000,0.000000,0.000000}%
\pgfsetstrokecolor{currentstroke}%
\pgfsetdash{}{0pt}%
\pgfpathmoveto{\pgfqpoint{9.929697in}{1.241685in}}%
\pgfpathlineto{\pgfqpoint{10.013031in}{1.241685in}}%
\pgfpathmoveto{\pgfqpoint{9.971364in}{1.200018in}}%
\pgfpathlineto{\pgfqpoint{9.971364in}{1.283351in}}%
\pgfusepath{stroke,fill}%
\end{pgfscope}%
\begin{pgfscope}%
\pgfpathrectangle{\pgfqpoint{7.105882in}{0.750000in}}{\pgfqpoint{4.376471in}{0.953684in}} %
\pgfusepath{clip}%
\pgfsetbuttcap%
\pgfsetroundjoin%
\definecolor{currentfill}{rgb}{1.000000,0.000000,0.000000}%
\pgfsetfillcolor{currentfill}%
\pgfsetlinewidth{2.007500pt}%
\definecolor{currentstroke}{rgb}{1.000000,0.000000,0.000000}%
\pgfsetstrokecolor{currentstroke}%
\pgfsetdash{}{0pt}%
\pgfpathmoveto{\pgfqpoint{8.005296in}{1.428549in}}%
\pgfpathlineto{\pgfqpoint{8.088630in}{1.428549in}}%
\pgfpathmoveto{\pgfqpoint{8.046963in}{1.386882in}}%
\pgfpathlineto{\pgfqpoint{8.046963in}{1.470216in}}%
\pgfusepath{stroke,fill}%
\end{pgfscope}%
\begin{pgfscope}%
\pgfpathrectangle{\pgfqpoint{7.105882in}{0.750000in}}{\pgfqpoint{4.376471in}{0.953684in}} %
\pgfusepath{clip}%
\pgfsetbuttcap%
\pgfsetroundjoin%
\definecolor{currentfill}{rgb}{1.000000,0.000000,0.000000}%
\pgfsetfillcolor{currentfill}%
\pgfsetlinewidth{2.007500pt}%
\definecolor{currentstroke}{rgb}{1.000000,0.000000,0.000000}%
\pgfsetstrokecolor{currentstroke}%
\pgfsetdash{}{0pt}%
\pgfpathmoveto{\pgfqpoint{10.101961in}{1.213102in}}%
\pgfpathlineto{\pgfqpoint{10.185294in}{1.213102in}}%
\pgfpathmoveto{\pgfqpoint{10.143627in}{1.171436in}}%
\pgfpathlineto{\pgfqpoint{10.143627in}{1.254769in}}%
\pgfusepath{stroke,fill}%
\end{pgfscope}%
\begin{pgfscope}%
\pgfpathrectangle{\pgfqpoint{7.105882in}{0.750000in}}{\pgfqpoint{4.376471in}{0.953684in}} %
\pgfusepath{clip}%
\pgfsetbuttcap%
\pgfsetroundjoin%
\definecolor{currentfill}{rgb}{1.000000,0.000000,0.000000}%
\pgfsetfillcolor{currentfill}%
\pgfsetlinewidth{2.007500pt}%
\definecolor{currentstroke}{rgb}{1.000000,0.000000,0.000000}%
\pgfsetstrokecolor{currentstroke}%
\pgfsetdash{}{0pt}%
\pgfpathmoveto{\pgfqpoint{10.082565in}{1.252249in}}%
\pgfpathlineto{\pgfqpoint{10.165898in}{1.252249in}}%
\pgfpathmoveto{\pgfqpoint{10.124232in}{1.210582in}}%
\pgfpathlineto{\pgfqpoint{10.124232in}{1.293915in}}%
\pgfusepath{stroke,fill}%
\end{pgfscope}%
\begin{pgfscope}%
\pgfpathrectangle{\pgfqpoint{7.105882in}{0.750000in}}{\pgfqpoint{4.376471in}{0.953684in}} %
\pgfusepath{clip}%
\pgfsetbuttcap%
\pgfsetroundjoin%
\definecolor{currentfill}{rgb}{1.000000,0.000000,0.000000}%
\pgfsetfillcolor{currentfill}%
\pgfsetlinewidth{2.007500pt}%
\definecolor{currentstroke}{rgb}{1.000000,0.000000,0.000000}%
\pgfsetstrokecolor{currentstroke}%
\pgfsetdash{}{0pt}%
\pgfpathmoveto{\pgfqpoint{10.099505in}{1.277875in}}%
\pgfpathlineto{\pgfqpoint{10.182838in}{1.277875in}}%
\pgfpathmoveto{\pgfqpoint{10.141171in}{1.236208in}}%
\pgfpathlineto{\pgfqpoint{10.141171in}{1.319542in}}%
\pgfusepath{stroke,fill}%
\end{pgfscope}%
\begin{pgfscope}%
\pgfpathrectangle{\pgfqpoint{7.105882in}{0.750000in}}{\pgfqpoint{4.376471in}{0.953684in}} %
\pgfusepath{clip}%
\pgfsetbuttcap%
\pgfsetroundjoin%
\definecolor{currentfill}{rgb}{1.000000,0.000000,0.000000}%
\pgfsetfillcolor{currentfill}%
\pgfsetlinewidth{2.007500pt}%
\definecolor{currentstroke}{rgb}{1.000000,0.000000,0.000000}%
\pgfsetstrokecolor{currentstroke}%
\pgfsetdash{}{0pt}%
\pgfpathmoveto{\pgfqpoint{11.243738in}{1.408689in}}%
\pgfpathlineto{\pgfqpoint{11.327072in}{1.408689in}}%
\pgfpathmoveto{\pgfqpoint{11.285405in}{1.367022in}}%
\pgfpathlineto{\pgfqpoint{11.285405in}{1.450355in}}%
\pgfusepath{stroke,fill}%
\end{pgfscope}%
\begin{pgfscope}%
\pgfpathrectangle{\pgfqpoint{7.105882in}{0.750000in}}{\pgfqpoint{4.376471in}{0.953684in}} %
\pgfusepath{clip}%
\pgfsetbuttcap%
\pgfsetroundjoin%
\definecolor{currentfill}{rgb}{1.000000,0.000000,0.000000}%
\pgfsetfillcolor{currentfill}%
\pgfsetlinewidth{2.007500pt}%
\definecolor{currentstroke}{rgb}{1.000000,0.000000,0.000000}%
\pgfsetstrokecolor{currentstroke}%
\pgfsetdash{}{0pt}%
\pgfpathmoveto{\pgfqpoint{10.326683in}{1.155319in}}%
\pgfpathlineto{\pgfqpoint{10.410016in}{1.155319in}}%
\pgfpathmoveto{\pgfqpoint{10.368350in}{1.113652in}}%
\pgfpathlineto{\pgfqpoint{10.368350in}{1.196985in}}%
\pgfusepath{stroke,fill}%
\end{pgfscope}%
\begin{pgfscope}%
\pgfpathrectangle{\pgfqpoint{7.105882in}{0.750000in}}{\pgfqpoint{4.376471in}{0.953684in}} %
\pgfusepath{clip}%
\pgfsetbuttcap%
\pgfsetroundjoin%
\definecolor{currentfill}{rgb}{1.000000,0.000000,0.000000}%
\pgfsetfillcolor{currentfill}%
\pgfsetlinewidth{2.007500pt}%
\definecolor{currentstroke}{rgb}{1.000000,0.000000,0.000000}%
\pgfsetstrokecolor{currentstroke}%
\pgfsetdash{}{0pt}%
\pgfpathmoveto{\pgfqpoint{9.198210in}{1.120553in}}%
\pgfpathlineto{\pgfqpoint{9.281544in}{1.120553in}}%
\pgfpathmoveto{\pgfqpoint{9.239877in}{1.078887in}}%
\pgfpathlineto{\pgfqpoint{9.239877in}{1.162220in}}%
\pgfusepath{stroke,fill}%
\end{pgfscope}%
\begin{pgfscope}%
\pgfpathrectangle{\pgfqpoint{7.105882in}{0.750000in}}{\pgfqpoint{4.376471in}{0.953684in}} %
\pgfusepath{clip}%
\pgfsetbuttcap%
\pgfsetroundjoin%
\definecolor{currentfill}{rgb}{1.000000,0.000000,0.000000}%
\pgfsetfillcolor{currentfill}%
\pgfsetlinewidth{2.007500pt}%
\definecolor{currentstroke}{rgb}{1.000000,0.000000,0.000000}%
\pgfsetstrokecolor{currentstroke}%
\pgfsetdash{}{0pt}%
\pgfpathmoveto{\pgfqpoint{9.469636in}{0.772194in}}%
\pgfpathlineto{\pgfqpoint{9.552969in}{0.772194in}}%
\pgfpathmoveto{\pgfqpoint{9.511302in}{0.730527in}}%
\pgfpathlineto{\pgfqpoint{9.511302in}{0.813860in}}%
\pgfusepath{stroke,fill}%
\end{pgfscope}%
\begin{pgfscope}%
\pgfpathrectangle{\pgfqpoint{7.105882in}{0.750000in}}{\pgfqpoint{4.376471in}{0.953684in}} %
\pgfusepath{clip}%
\pgfsetbuttcap%
\pgfsetroundjoin%
\definecolor{currentfill}{rgb}{1.000000,0.000000,0.000000}%
\pgfsetfillcolor{currentfill}%
\pgfsetlinewidth{2.007500pt}%
\definecolor{currentstroke}{rgb}{1.000000,0.000000,0.000000}%
\pgfsetstrokecolor{currentstroke}%
\pgfsetdash{}{0pt}%
\pgfpathmoveto{\pgfqpoint{10.382040in}{1.173514in}}%
\pgfpathlineto{\pgfqpoint{10.465373in}{1.173514in}}%
\pgfpathmoveto{\pgfqpoint{10.423706in}{1.131847in}}%
\pgfpathlineto{\pgfqpoint{10.423706in}{1.215181in}}%
\pgfusepath{stroke,fill}%
\end{pgfscope}%
\begin{pgfscope}%
\pgfpathrectangle{\pgfqpoint{7.105882in}{0.750000in}}{\pgfqpoint{4.376471in}{0.953684in}} %
\pgfusepath{clip}%
\pgfsetbuttcap%
\pgfsetroundjoin%
\definecolor{currentfill}{rgb}{1.000000,0.000000,0.000000}%
\pgfsetfillcolor{currentfill}%
\pgfsetlinewidth{2.007500pt}%
\definecolor{currentstroke}{rgb}{1.000000,0.000000,0.000000}%
\pgfsetstrokecolor{currentstroke}%
\pgfsetdash{}{0pt}%
\pgfpathmoveto{\pgfqpoint{8.150370in}{1.486441in}}%
\pgfpathlineto{\pgfqpoint{8.233703in}{1.486441in}}%
\pgfpathmoveto{\pgfqpoint{8.192036in}{1.444774in}}%
\pgfpathlineto{\pgfqpoint{8.192036in}{1.528107in}}%
\pgfusepath{stroke,fill}%
\end{pgfscope}%
\begin{pgfscope}%
\pgfpathrectangle{\pgfqpoint{7.105882in}{0.750000in}}{\pgfqpoint{4.376471in}{0.953684in}} %
\pgfusepath{clip}%
\pgfsetbuttcap%
\pgfsetroundjoin%
\definecolor{currentfill}{rgb}{1.000000,0.000000,0.000000}%
\pgfsetfillcolor{currentfill}%
\pgfsetlinewidth{2.007500pt}%
\definecolor{currentstroke}{rgb}{1.000000,0.000000,0.000000}%
\pgfsetstrokecolor{currentstroke}%
\pgfsetdash{}{0pt}%
\pgfpathmoveto{\pgfqpoint{10.273978in}{1.230686in}}%
\pgfpathlineto{\pgfqpoint{10.357311in}{1.230686in}}%
\pgfpathmoveto{\pgfqpoint{10.315644in}{1.189019in}}%
\pgfpathlineto{\pgfqpoint{10.315644in}{1.272353in}}%
\pgfusepath{stroke,fill}%
\end{pgfscope}%
\begin{pgfscope}%
\pgfpathrectangle{\pgfqpoint{7.105882in}{0.750000in}}{\pgfqpoint{4.376471in}{0.953684in}} %
\pgfusepath{clip}%
\pgfsetbuttcap%
\pgfsetroundjoin%
\definecolor{currentfill}{rgb}{1.000000,0.000000,0.000000}%
\pgfsetfillcolor{currentfill}%
\pgfsetlinewidth{2.007500pt}%
\definecolor{currentstroke}{rgb}{1.000000,0.000000,0.000000}%
\pgfsetstrokecolor{currentstroke}%
\pgfsetdash{}{0pt}%
\pgfpathmoveto{\pgfqpoint{10.287531in}{1.082820in}}%
\pgfpathlineto{\pgfqpoint{10.370865in}{1.082820in}}%
\pgfpathmoveto{\pgfqpoint{10.329198in}{1.041154in}}%
\pgfpathlineto{\pgfqpoint{10.329198in}{1.124487in}}%
\pgfusepath{stroke,fill}%
\end{pgfscope}%
\begin{pgfscope}%
\pgfpathrectangle{\pgfqpoint{7.105882in}{0.750000in}}{\pgfqpoint{4.376471in}{0.953684in}} %
\pgfusepath{clip}%
\pgfsetbuttcap%
\pgfsetroundjoin%
\definecolor{currentfill}{rgb}{1.000000,0.000000,0.000000}%
\pgfsetfillcolor{currentfill}%
\pgfsetlinewidth{2.007500pt}%
\definecolor{currentstroke}{rgb}{1.000000,0.000000,0.000000}%
\pgfsetstrokecolor{currentstroke}%
\pgfsetdash{}{0pt}%
\pgfpathmoveto{\pgfqpoint{8.676096in}{1.228236in}}%
\pgfpathlineto{\pgfqpoint{8.759430in}{1.228236in}}%
\pgfpathmoveto{\pgfqpoint{8.717763in}{1.186569in}}%
\pgfpathlineto{\pgfqpoint{8.717763in}{1.269902in}}%
\pgfusepath{stroke,fill}%
\end{pgfscope}%
\begin{pgfscope}%
\pgfpathrectangle{\pgfqpoint{7.105882in}{0.750000in}}{\pgfqpoint{4.376471in}{0.953684in}} %
\pgfusepath{clip}%
\pgfsetbuttcap%
\pgfsetroundjoin%
\definecolor{currentfill}{rgb}{1.000000,0.000000,0.000000}%
\pgfsetfillcolor{currentfill}%
\pgfsetlinewidth{2.007500pt}%
\definecolor{currentstroke}{rgb}{1.000000,0.000000,0.000000}%
\pgfsetstrokecolor{currentstroke}%
\pgfsetdash{}{0pt}%
\pgfpathmoveto{\pgfqpoint{8.390904in}{1.250858in}}%
\pgfpathlineto{\pgfqpoint{8.474237in}{1.250858in}}%
\pgfpathmoveto{\pgfqpoint{8.432570in}{1.209191in}}%
\pgfpathlineto{\pgfqpoint{8.432570in}{1.292524in}}%
\pgfusepath{stroke,fill}%
\end{pgfscope}%
\begin{pgfscope}%
\pgfpathrectangle{\pgfqpoint{7.105882in}{0.750000in}}{\pgfqpoint{4.376471in}{0.953684in}} %
\pgfusepath{clip}%
\pgfsetbuttcap%
\pgfsetroundjoin%
\definecolor{currentfill}{rgb}{1.000000,0.000000,0.000000}%
\pgfsetfillcolor{currentfill}%
\pgfsetlinewidth{2.007500pt}%
\definecolor{currentstroke}{rgb}{1.000000,0.000000,0.000000}%
\pgfsetstrokecolor{currentstroke}%
\pgfsetdash{}{0pt}%
\pgfpathmoveto{\pgfqpoint{9.043880in}{1.225550in}}%
\pgfpathlineto{\pgfqpoint{9.127213in}{1.225550in}}%
\pgfpathmoveto{\pgfqpoint{9.085547in}{1.183883in}}%
\pgfpathlineto{\pgfqpoint{9.085547in}{1.267217in}}%
\pgfusepath{stroke,fill}%
\end{pgfscope}%
\begin{pgfscope}%
\pgfpathrectangle{\pgfqpoint{7.105882in}{0.750000in}}{\pgfqpoint{4.376471in}{0.953684in}} %
\pgfusepath{clip}%
\pgfsetbuttcap%
\pgfsetroundjoin%
\definecolor{currentfill}{rgb}{1.000000,0.000000,0.000000}%
\pgfsetfillcolor{currentfill}%
\pgfsetlinewidth{2.007500pt}%
\definecolor{currentstroke}{rgb}{1.000000,0.000000,0.000000}%
\pgfsetstrokecolor{currentstroke}%
\pgfsetdash{}{0pt}%
\pgfpathmoveto{\pgfqpoint{9.212925in}{1.178342in}}%
\pgfpathlineto{\pgfqpoint{9.296259in}{1.178342in}}%
\pgfpathmoveto{\pgfqpoint{9.254592in}{1.136675in}}%
\pgfpathlineto{\pgfqpoint{9.254592in}{1.220008in}}%
\pgfusepath{stroke,fill}%
\end{pgfscope}%
\begin{pgfscope}%
\pgfpathrectangle{\pgfqpoint{7.105882in}{0.750000in}}{\pgfqpoint{4.376471in}{0.953684in}} %
\pgfusepath{clip}%
\pgfsetbuttcap%
\pgfsetroundjoin%
\definecolor{currentfill}{rgb}{0.000000,0.000000,0.000000}%
\pgfsetfillcolor{currentfill}%
\pgfsetlinewidth{1.003750pt}%
\definecolor{currentstroke}{rgb}{0.000000,0.000000,0.000000}%
\pgfsetstrokecolor{currentstroke}%
\pgfsetdash{}{0pt}%
\pgfsys@defobject{currentmarker}{\pgfqpoint{-0.020833in}{-0.020833in}}{\pgfqpoint{0.020833in}{0.020833in}}{%
\pgfpathmoveto{\pgfqpoint{0.000000in}{-0.020833in}}%
\pgfpathcurveto{\pgfqpoint{0.005525in}{-0.020833in}}{\pgfqpoint{0.010825in}{-0.018638in}}{\pgfqpoint{0.014731in}{-0.014731in}}%
\pgfpathcurveto{\pgfqpoint{0.018638in}{-0.010825in}}{\pgfqpoint{0.020833in}{-0.005525in}}{\pgfqpoint{0.020833in}{0.000000in}}%
\pgfpathcurveto{\pgfqpoint{0.020833in}{0.005525in}}{\pgfqpoint{0.018638in}{0.010825in}}{\pgfqpoint{0.014731in}{0.014731in}}%
\pgfpathcurveto{\pgfqpoint{0.010825in}{0.018638in}}{\pgfqpoint{0.005525in}{0.020833in}}{\pgfqpoint{0.000000in}{0.020833in}}%
\pgfpathcurveto{\pgfqpoint{-0.005525in}{0.020833in}}{\pgfqpoint{-0.010825in}{0.018638in}}{\pgfqpoint{-0.014731in}{0.014731in}}%
\pgfpathcurveto{\pgfqpoint{-0.018638in}{0.010825in}}{\pgfqpoint{-0.020833in}{0.005525in}}{\pgfqpoint{-0.020833in}{0.000000in}}%
\pgfpathcurveto{\pgfqpoint{-0.020833in}{-0.005525in}}{\pgfqpoint{-0.018638in}{-0.010825in}}{\pgfqpoint{-0.014731in}{-0.014731in}}%
\pgfpathcurveto{\pgfqpoint{-0.010825in}{-0.018638in}}{\pgfqpoint{-0.005525in}{-0.020833in}}{\pgfqpoint{0.000000in}{-0.020833in}}%
\pgfpathclose%
\pgfusepath{stroke,fill}%
}%
\begin{pgfscope}%
\pgfsys@transformshift{7.981176in}{1.633779in}%
\pgfsys@useobject{currentmarker}{}%
\end{pgfscope}%
\begin{pgfscope}%
\pgfsys@transformshift{7.998770in}{1.684891in}%
\pgfsys@useobject{currentmarker}{}%
\end{pgfscope}%
\begin{pgfscope}%
\pgfsys@transformshift{8.016364in}{1.597216in}%
\pgfsys@useobject{currentmarker}{}%
\end{pgfscope}%
\begin{pgfscope}%
\pgfsys@transformshift{8.033958in}{1.605298in}%
\pgfsys@useobject{currentmarker}{}%
\end{pgfscope}%
\begin{pgfscope}%
\pgfsys@transformshift{8.051552in}{1.508979in}%
\pgfsys@useobject{currentmarker}{}%
\end{pgfscope}%
\begin{pgfscope}%
\pgfsys@transformshift{8.069146in}{1.657913in}%
\pgfsys@useobject{currentmarker}{}%
\end{pgfscope}%
\begin{pgfscope}%
\pgfsys@transformshift{8.086740in}{1.465607in}%
\pgfsys@useobject{currentmarker}{}%
\end{pgfscope}%
\begin{pgfscope}%
\pgfsys@transformshift{8.104333in}{1.465570in}%
\pgfsys@useobject{currentmarker}{}%
\end{pgfscope}%
\begin{pgfscope}%
\pgfsys@transformshift{8.121927in}{1.585684in}%
\pgfsys@useobject{currentmarker}{}%
\end{pgfscope}%
\begin{pgfscope}%
\pgfsys@transformshift{8.139521in}{1.238225in}%
\pgfsys@useobject{currentmarker}{}%
\end{pgfscope}%
\begin{pgfscope}%
\pgfsys@transformshift{8.157115in}{1.220138in}%
\pgfsys@useobject{currentmarker}{}%
\end{pgfscope}%
\begin{pgfscope}%
\pgfsys@transformshift{8.174709in}{1.417804in}%
\pgfsys@useobject{currentmarker}{}%
\end{pgfscope}%
\begin{pgfscope}%
\pgfsys@transformshift{8.192303in}{1.181208in}%
\pgfsys@useobject{currentmarker}{}%
\end{pgfscope}%
\begin{pgfscope}%
\pgfsys@transformshift{8.209897in}{1.468341in}%
\pgfsys@useobject{currentmarker}{}%
\end{pgfscope}%
\begin{pgfscope}%
\pgfsys@transformshift{8.227490in}{1.213097in}%
\pgfsys@useobject{currentmarker}{}%
\end{pgfscope}%
\begin{pgfscope}%
\pgfsys@transformshift{8.245084in}{1.160424in}%
\pgfsys@useobject{currentmarker}{}%
\end{pgfscope}%
\begin{pgfscope}%
\pgfsys@transformshift{8.262678in}{1.407888in}%
\pgfsys@useobject{currentmarker}{}%
\end{pgfscope}%
\begin{pgfscope}%
\pgfsys@transformshift{8.280272in}{1.347920in}%
\pgfsys@useobject{currentmarker}{}%
\end{pgfscope}%
\begin{pgfscope}%
\pgfsys@transformshift{8.297866in}{1.372250in}%
\pgfsys@useobject{currentmarker}{}%
\end{pgfscope}%
\begin{pgfscope}%
\pgfsys@transformshift{8.315460in}{1.264584in}%
\pgfsys@useobject{currentmarker}{}%
\end{pgfscope}%
\begin{pgfscope}%
\pgfsys@transformshift{8.333054in}{1.078902in}%
\pgfsys@useobject{currentmarker}{}%
\end{pgfscope}%
\begin{pgfscope}%
\pgfsys@transformshift{8.350647in}{1.346158in}%
\pgfsys@useobject{currentmarker}{}%
\end{pgfscope}%
\begin{pgfscope}%
\pgfsys@transformshift{8.368241in}{1.123818in}%
\pgfsys@useobject{currentmarker}{}%
\end{pgfscope}%
\begin{pgfscope}%
\pgfsys@transformshift{8.385835in}{1.226274in}%
\pgfsys@useobject{currentmarker}{}%
\end{pgfscope}%
\begin{pgfscope}%
\pgfsys@transformshift{8.403429in}{1.238831in}%
\pgfsys@useobject{currentmarker}{}%
\end{pgfscope}%
\begin{pgfscope}%
\pgfsys@transformshift{8.421023in}{1.129518in}%
\pgfsys@useobject{currentmarker}{}%
\end{pgfscope}%
\begin{pgfscope}%
\pgfsys@transformshift{8.438617in}{1.208061in}%
\pgfsys@useobject{currentmarker}{}%
\end{pgfscope}%
\begin{pgfscope}%
\pgfsys@transformshift{8.456210in}{1.242682in}%
\pgfsys@useobject{currentmarker}{}%
\end{pgfscope}%
\begin{pgfscope}%
\pgfsys@transformshift{8.473804in}{1.194245in}%
\pgfsys@useobject{currentmarker}{}%
\end{pgfscope}%
\begin{pgfscope}%
\pgfsys@transformshift{8.491398in}{1.055069in}%
\pgfsys@useobject{currentmarker}{}%
\end{pgfscope}%
\begin{pgfscope}%
\pgfsys@transformshift{8.508992in}{1.202872in}%
\pgfsys@useobject{currentmarker}{}%
\end{pgfscope}%
\begin{pgfscope}%
\pgfsys@transformshift{8.526586in}{1.315343in}%
\pgfsys@useobject{currentmarker}{}%
\end{pgfscope}%
\begin{pgfscope}%
\pgfsys@transformshift{8.544180in}{1.126125in}%
\pgfsys@useobject{currentmarker}{}%
\end{pgfscope}%
\begin{pgfscope}%
\pgfsys@transformshift{8.561774in}{1.192850in}%
\pgfsys@useobject{currentmarker}{}%
\end{pgfscope}%
\begin{pgfscope}%
\pgfsys@transformshift{8.579367in}{1.177937in}%
\pgfsys@useobject{currentmarker}{}%
\end{pgfscope}%
\begin{pgfscope}%
\pgfsys@transformshift{8.596961in}{1.419000in}%
\pgfsys@useobject{currentmarker}{}%
\end{pgfscope}%
\begin{pgfscope}%
\pgfsys@transformshift{8.614555in}{1.316679in}%
\pgfsys@useobject{currentmarker}{}%
\end{pgfscope}%
\begin{pgfscope}%
\pgfsys@transformshift{8.632149in}{1.305249in}%
\pgfsys@useobject{currentmarker}{}%
\end{pgfscope}%
\begin{pgfscope}%
\pgfsys@transformshift{8.649743in}{1.203242in}%
\pgfsys@useobject{currentmarker}{}%
\end{pgfscope}%
\begin{pgfscope}%
\pgfsys@transformshift{8.667337in}{1.348128in}%
\pgfsys@useobject{currentmarker}{}%
\end{pgfscope}%
\begin{pgfscope}%
\pgfsys@transformshift{8.684931in}{1.242076in}%
\pgfsys@useobject{currentmarker}{}%
\end{pgfscope}%
\begin{pgfscope}%
\pgfsys@transformshift{8.702524in}{1.326217in}%
\pgfsys@useobject{currentmarker}{}%
\end{pgfscope}%
\begin{pgfscope}%
\pgfsys@transformshift{8.720118in}{1.273171in}%
\pgfsys@useobject{currentmarker}{}%
\end{pgfscope}%
\begin{pgfscope}%
\pgfsys@transformshift{8.737712in}{1.415928in}%
\pgfsys@useobject{currentmarker}{}%
\end{pgfscope}%
\begin{pgfscope}%
\pgfsys@transformshift{8.755306in}{1.417343in}%
\pgfsys@useobject{currentmarker}{}%
\end{pgfscope}%
\begin{pgfscope}%
\pgfsys@transformshift{8.772900in}{1.349484in}%
\pgfsys@useobject{currentmarker}{}%
\end{pgfscope}%
\begin{pgfscope}%
\pgfsys@transformshift{8.790494in}{1.418264in}%
\pgfsys@useobject{currentmarker}{}%
\end{pgfscope}%
\begin{pgfscope}%
\pgfsys@transformshift{8.808087in}{1.277586in}%
\pgfsys@useobject{currentmarker}{}%
\end{pgfscope}%
\begin{pgfscope}%
\pgfsys@transformshift{8.825681in}{1.243663in}%
\pgfsys@useobject{currentmarker}{}%
\end{pgfscope}%
\begin{pgfscope}%
\pgfsys@transformshift{8.843275in}{1.439380in}%
\pgfsys@useobject{currentmarker}{}%
\end{pgfscope}%
\begin{pgfscope}%
\pgfsys@transformshift{8.860869in}{1.414414in}%
\pgfsys@useobject{currentmarker}{}%
\end{pgfscope}%
\begin{pgfscope}%
\pgfsys@transformshift{8.878463in}{1.461213in}%
\pgfsys@useobject{currentmarker}{}%
\end{pgfscope}%
\begin{pgfscope}%
\pgfsys@transformshift{8.896057in}{1.633247in}%
\pgfsys@useobject{currentmarker}{}%
\end{pgfscope}%
\begin{pgfscope}%
\pgfsys@transformshift{8.913651in}{1.486778in}%
\pgfsys@useobject{currentmarker}{}%
\end{pgfscope}%
\begin{pgfscope}%
\pgfsys@transformshift{8.931244in}{1.296797in}%
\pgfsys@useobject{currentmarker}{}%
\end{pgfscope}%
\begin{pgfscope}%
\pgfsys@transformshift{8.948838in}{1.491068in}%
\pgfsys@useobject{currentmarker}{}%
\end{pgfscope}%
\begin{pgfscope}%
\pgfsys@transformshift{8.966432in}{1.240132in}%
\pgfsys@useobject{currentmarker}{}%
\end{pgfscope}%
\begin{pgfscope}%
\pgfsys@transformshift{8.984026in}{1.313947in}%
\pgfsys@useobject{currentmarker}{}%
\end{pgfscope}%
\begin{pgfscope}%
\pgfsys@transformshift{9.001620in}{1.340235in}%
\pgfsys@useobject{currentmarker}{}%
\end{pgfscope}%
\begin{pgfscope}%
\pgfsys@transformshift{9.019214in}{1.502795in}%
\pgfsys@useobject{currentmarker}{}%
\end{pgfscope}%
\begin{pgfscope}%
\pgfsys@transformshift{9.036808in}{1.242651in}%
\pgfsys@useobject{currentmarker}{}%
\end{pgfscope}%
\begin{pgfscope}%
\pgfsys@transformshift{9.054401in}{1.217188in}%
\pgfsys@useobject{currentmarker}{}%
\end{pgfscope}%
\begin{pgfscope}%
\pgfsys@transformshift{9.071995in}{1.270906in}%
\pgfsys@useobject{currentmarker}{}%
\end{pgfscope}%
\begin{pgfscope}%
\pgfsys@transformshift{9.089589in}{1.195077in}%
\pgfsys@useobject{currentmarker}{}%
\end{pgfscope}%
\begin{pgfscope}%
\pgfsys@transformshift{9.107183in}{1.352284in}%
\pgfsys@useobject{currentmarker}{}%
\end{pgfscope}%
\begin{pgfscope}%
\pgfsys@transformshift{9.124777in}{1.111704in}%
\pgfsys@useobject{currentmarker}{}%
\end{pgfscope}%
\begin{pgfscope}%
\pgfsys@transformshift{9.142371in}{1.083058in}%
\pgfsys@useobject{currentmarker}{}%
\end{pgfscope}%
\begin{pgfscope}%
\pgfsys@transformshift{9.159965in}{1.131252in}%
\pgfsys@useobject{currentmarker}{}%
\end{pgfscope}%
\begin{pgfscope}%
\pgfsys@transformshift{9.177558in}{1.102797in}%
\pgfsys@useobject{currentmarker}{}%
\end{pgfscope}%
\begin{pgfscope}%
\pgfsys@transformshift{9.195152in}{1.321515in}%
\pgfsys@useobject{currentmarker}{}%
\end{pgfscope}%
\begin{pgfscope}%
\pgfsys@transformshift{9.212746in}{1.201940in}%
\pgfsys@useobject{currentmarker}{}%
\end{pgfscope}%
\begin{pgfscope}%
\pgfsys@transformshift{9.230340in}{1.094544in}%
\pgfsys@useobject{currentmarker}{}%
\end{pgfscope}%
\begin{pgfscope}%
\pgfsys@transformshift{9.247934in}{0.942963in}%
\pgfsys@useobject{currentmarker}{}%
\end{pgfscope}%
\begin{pgfscope}%
\pgfsys@transformshift{9.265528in}{1.128254in}%
\pgfsys@useobject{currentmarker}{}%
\end{pgfscope}%
\begin{pgfscope}%
\pgfsys@transformshift{9.283121in}{0.925687in}%
\pgfsys@useobject{currentmarker}{}%
\end{pgfscope}%
\begin{pgfscope}%
\pgfsys@transformshift{9.300715in}{0.853443in}%
\pgfsys@useobject{currentmarker}{}%
\end{pgfscope}%
\begin{pgfscope}%
\pgfsys@transformshift{9.318309in}{1.108119in}%
\pgfsys@useobject{currentmarker}{}%
\end{pgfscope}%
\begin{pgfscope}%
\pgfsys@transformshift{9.335903in}{1.006267in}%
\pgfsys@useobject{currentmarker}{}%
\end{pgfscope}%
\begin{pgfscope}%
\pgfsys@transformshift{9.353497in}{1.052571in}%
\pgfsys@useobject{currentmarker}{}%
\end{pgfscope}%
\begin{pgfscope}%
\pgfsys@transformshift{9.371091in}{0.980810in}%
\pgfsys@useobject{currentmarker}{}%
\end{pgfscope}%
\begin{pgfscope}%
\pgfsys@transformshift{9.388685in}{1.024173in}%
\pgfsys@useobject{currentmarker}{}%
\end{pgfscope}%
\begin{pgfscope}%
\pgfsys@transformshift{9.406278in}{0.866221in}%
\pgfsys@useobject{currentmarker}{}%
\end{pgfscope}%
\begin{pgfscope}%
\pgfsys@transformshift{9.423872in}{0.821996in}%
\pgfsys@useobject{currentmarker}{}%
\end{pgfscope}%
\begin{pgfscope}%
\pgfsys@transformshift{9.441466in}{0.988346in}%
\pgfsys@useobject{currentmarker}{}%
\end{pgfscope}%
\begin{pgfscope}%
\pgfsys@transformshift{9.459060in}{0.838795in}%
\pgfsys@useobject{currentmarker}{}%
\end{pgfscope}%
\begin{pgfscope}%
\pgfsys@transformshift{9.476654in}{0.850163in}%
\pgfsys@useobject{currentmarker}{}%
\end{pgfscope}%
\begin{pgfscope}%
\pgfsys@transformshift{9.494248in}{0.875541in}%
\pgfsys@useobject{currentmarker}{}%
\end{pgfscope}%
\begin{pgfscope}%
\pgfsys@transformshift{9.511842in}{0.926701in}%
\pgfsys@useobject{currentmarker}{}%
\end{pgfscope}%
\begin{pgfscope}%
\pgfsys@transformshift{9.529435in}{0.895955in}%
\pgfsys@useobject{currentmarker}{}%
\end{pgfscope}%
\begin{pgfscope}%
\pgfsys@transformshift{9.547029in}{0.802604in}%
\pgfsys@useobject{currentmarker}{}%
\end{pgfscope}%
\begin{pgfscope}%
\pgfsys@transformshift{9.564623in}{0.885153in}%
\pgfsys@useobject{currentmarker}{}%
\end{pgfscope}%
\begin{pgfscope}%
\pgfsys@transformshift{9.582217in}{0.739828in}%
\pgfsys@useobject{currentmarker}{}%
\end{pgfscope}%
\begin{pgfscope}%
\pgfsys@transformshift{9.599811in}{1.036023in}%
\pgfsys@useobject{currentmarker}{}%
\end{pgfscope}%
\begin{pgfscope}%
\pgfsys@transformshift{9.617405in}{0.829419in}%
\pgfsys@useobject{currentmarker}{}%
\end{pgfscope}%
\begin{pgfscope}%
\pgfsys@transformshift{9.634999in}{0.894817in}%
\pgfsys@useobject{currentmarker}{}%
\end{pgfscope}%
\begin{pgfscope}%
\pgfsys@transformshift{9.652592in}{1.026801in}%
\pgfsys@useobject{currentmarker}{}%
\end{pgfscope}%
\begin{pgfscope}%
\pgfsys@transformshift{9.670186in}{0.966197in}%
\pgfsys@useobject{currentmarker}{}%
\end{pgfscope}%
\begin{pgfscope}%
\pgfsys@transformshift{9.687780in}{1.211772in}%
\pgfsys@useobject{currentmarker}{}%
\end{pgfscope}%
\begin{pgfscope}%
\pgfsys@transformshift{9.705374in}{0.949448in}%
\pgfsys@useobject{currentmarker}{}%
\end{pgfscope}%
\begin{pgfscope}%
\pgfsys@transformshift{9.722968in}{1.124191in}%
\pgfsys@useobject{currentmarker}{}%
\end{pgfscope}%
\begin{pgfscope}%
\pgfsys@transformshift{9.740562in}{1.113719in}%
\pgfsys@useobject{currentmarker}{}%
\end{pgfscope}%
\begin{pgfscope}%
\pgfsys@transformshift{9.758155in}{1.021520in}%
\pgfsys@useobject{currentmarker}{}%
\end{pgfscope}%
\begin{pgfscope}%
\pgfsys@transformshift{9.775749in}{1.209263in}%
\pgfsys@useobject{currentmarker}{}%
\end{pgfscope}%
\begin{pgfscope}%
\pgfsys@transformshift{9.793343in}{1.159538in}%
\pgfsys@useobject{currentmarker}{}%
\end{pgfscope}%
\begin{pgfscope}%
\pgfsys@transformshift{9.810937in}{1.271941in}%
\pgfsys@useobject{currentmarker}{}%
\end{pgfscope}%
\begin{pgfscope}%
\pgfsys@transformshift{9.828531in}{1.294972in}%
\pgfsys@useobject{currentmarker}{}%
\end{pgfscope}%
\begin{pgfscope}%
\pgfsys@transformshift{9.846125in}{1.444985in}%
\pgfsys@useobject{currentmarker}{}%
\end{pgfscope}%
\begin{pgfscope}%
\pgfsys@transformshift{9.863719in}{1.378641in}%
\pgfsys@useobject{currentmarker}{}%
\end{pgfscope}%
\begin{pgfscope}%
\pgfsys@transformshift{9.881312in}{1.223682in}%
\pgfsys@useobject{currentmarker}{}%
\end{pgfscope}%
\begin{pgfscope}%
\pgfsys@transformshift{9.898906in}{1.249636in}%
\pgfsys@useobject{currentmarker}{}%
\end{pgfscope}%
\begin{pgfscope}%
\pgfsys@transformshift{9.916500in}{1.394162in}%
\pgfsys@useobject{currentmarker}{}%
\end{pgfscope}%
\begin{pgfscope}%
\pgfsys@transformshift{9.934094in}{1.359853in}%
\pgfsys@useobject{currentmarker}{}%
\end{pgfscope}%
\begin{pgfscope}%
\pgfsys@transformshift{9.951688in}{1.366433in}%
\pgfsys@useobject{currentmarker}{}%
\end{pgfscope}%
\begin{pgfscope}%
\pgfsys@transformshift{9.969282in}{1.148466in}%
\pgfsys@useobject{currentmarker}{}%
\end{pgfscope}%
\begin{pgfscope}%
\pgfsys@transformshift{9.986876in}{1.311059in}%
\pgfsys@useobject{currentmarker}{}%
\end{pgfscope}%
\begin{pgfscope}%
\pgfsys@transformshift{10.004469in}{1.242624in}%
\pgfsys@useobject{currentmarker}{}%
\end{pgfscope}%
\begin{pgfscope}%
\pgfsys@transformshift{10.022063in}{1.344291in}%
\pgfsys@useobject{currentmarker}{}%
\end{pgfscope}%
\begin{pgfscope}%
\pgfsys@transformshift{10.039657in}{1.305357in}%
\pgfsys@useobject{currentmarker}{}%
\end{pgfscope}%
\begin{pgfscope}%
\pgfsys@transformshift{10.057251in}{1.402261in}%
\pgfsys@useobject{currentmarker}{}%
\end{pgfscope}%
\begin{pgfscope}%
\pgfsys@transformshift{10.074845in}{1.338247in}%
\pgfsys@useobject{currentmarker}{}%
\end{pgfscope}%
\begin{pgfscope}%
\pgfsys@transformshift{10.092439in}{1.377967in}%
\pgfsys@useobject{currentmarker}{}%
\end{pgfscope}%
\begin{pgfscope}%
\pgfsys@transformshift{10.110033in}{1.244944in}%
\pgfsys@useobject{currentmarker}{}%
\end{pgfscope}%
\begin{pgfscope}%
\pgfsys@transformshift{10.127626in}{1.187298in}%
\pgfsys@useobject{currentmarker}{}%
\end{pgfscope}%
\begin{pgfscope}%
\pgfsys@transformshift{10.145220in}{1.228762in}%
\pgfsys@useobject{currentmarker}{}%
\end{pgfscope}%
\begin{pgfscope}%
\pgfsys@transformshift{10.162814in}{1.254853in}%
\pgfsys@useobject{currentmarker}{}%
\end{pgfscope}%
\begin{pgfscope}%
\pgfsys@transformshift{10.180408in}{1.280077in}%
\pgfsys@useobject{currentmarker}{}%
\end{pgfscope}%
\begin{pgfscope}%
\pgfsys@transformshift{10.198002in}{1.451707in}%
\pgfsys@useobject{currentmarker}{}%
\end{pgfscope}%
\begin{pgfscope}%
\pgfsys@transformshift{10.215596in}{1.207072in}%
\pgfsys@useobject{currentmarker}{}%
\end{pgfscope}%
\begin{pgfscope}%
\pgfsys@transformshift{10.233189in}{1.099654in}%
\pgfsys@useobject{currentmarker}{}%
\end{pgfscope}%
\begin{pgfscope}%
\pgfsys@transformshift{10.250783in}{1.143129in}%
\pgfsys@useobject{currentmarker}{}%
\end{pgfscope}%
\begin{pgfscope}%
\pgfsys@transformshift{10.268377in}{1.114053in}%
\pgfsys@useobject{currentmarker}{}%
\end{pgfscope}%
\begin{pgfscope}%
\pgfsys@transformshift{10.285971in}{1.190474in}%
\pgfsys@useobject{currentmarker}{}%
\end{pgfscope}%
\begin{pgfscope}%
\pgfsys@transformshift{10.303565in}{0.972241in}%
\pgfsys@useobject{currentmarker}{}%
\end{pgfscope}%
\begin{pgfscope}%
\pgfsys@transformshift{10.321159in}{1.114586in}%
\pgfsys@useobject{currentmarker}{}%
\end{pgfscope}%
\begin{pgfscope}%
\pgfsys@transformshift{10.338753in}{1.107371in}%
\pgfsys@useobject{currentmarker}{}%
\end{pgfscope}%
\begin{pgfscope}%
\pgfsys@transformshift{10.356346in}{1.099133in}%
\pgfsys@useobject{currentmarker}{}%
\end{pgfscope}%
\begin{pgfscope}%
\pgfsys@transformshift{10.373940in}{1.001882in}%
\pgfsys@useobject{currentmarker}{}%
\end{pgfscope}%
\begin{pgfscope}%
\pgfsys@transformshift{10.391534in}{1.023827in}%
\pgfsys@useobject{currentmarker}{}%
\end{pgfscope}%
\begin{pgfscope}%
\pgfsys@transformshift{10.409128in}{0.893492in}%
\pgfsys@useobject{currentmarker}{}%
\end{pgfscope}%
\begin{pgfscope}%
\pgfsys@transformshift{10.426722in}{0.974867in}%
\pgfsys@useobject{currentmarker}{}%
\end{pgfscope}%
\begin{pgfscope}%
\pgfsys@transformshift{10.444316in}{0.960416in}%
\pgfsys@useobject{currentmarker}{}%
\end{pgfscope}%
\begin{pgfscope}%
\pgfsys@transformshift{10.461910in}{1.047833in}%
\pgfsys@useobject{currentmarker}{}%
\end{pgfscope}%
\begin{pgfscope}%
\pgfsys@transformshift{10.479503in}{0.885599in}%
\pgfsys@useobject{currentmarker}{}%
\end{pgfscope}%
\begin{pgfscope}%
\pgfsys@transformshift{10.497097in}{1.073912in}%
\pgfsys@useobject{currentmarker}{}%
\end{pgfscope}%
\begin{pgfscope}%
\pgfsys@transformshift{10.514691in}{1.142590in}%
\pgfsys@useobject{currentmarker}{}%
\end{pgfscope}%
\begin{pgfscope}%
\pgfsys@transformshift{10.532285in}{0.788336in}%
\pgfsys@useobject{currentmarker}{}%
\end{pgfscope}%
\begin{pgfscope}%
\pgfsys@transformshift{10.549879in}{1.038231in}%
\pgfsys@useobject{currentmarker}{}%
\end{pgfscope}%
\begin{pgfscope}%
\pgfsys@transformshift{10.567473in}{1.067128in}%
\pgfsys@useobject{currentmarker}{}%
\end{pgfscope}%
\begin{pgfscope}%
\pgfsys@transformshift{10.585067in}{0.942637in}%
\pgfsys@useobject{currentmarker}{}%
\end{pgfscope}%
\begin{pgfscope}%
\pgfsys@transformshift{10.602660in}{0.974633in}%
\pgfsys@useobject{currentmarker}{}%
\end{pgfscope}%
\begin{pgfscope}%
\pgfsys@transformshift{10.620254in}{1.011053in}%
\pgfsys@useobject{currentmarker}{}%
\end{pgfscope}%
\begin{pgfscope}%
\pgfsys@transformshift{10.637848in}{1.006780in}%
\pgfsys@useobject{currentmarker}{}%
\end{pgfscope}%
\begin{pgfscope}%
\pgfsys@transformshift{10.655442in}{1.019648in}%
\pgfsys@useobject{currentmarker}{}%
\end{pgfscope}%
\begin{pgfscope}%
\pgfsys@transformshift{10.673036in}{0.899607in}%
\pgfsys@useobject{currentmarker}{}%
\end{pgfscope}%
\begin{pgfscope}%
\pgfsys@transformshift{10.690630in}{1.197997in}%
\pgfsys@useobject{currentmarker}{}%
\end{pgfscope}%
\begin{pgfscope}%
\pgfsys@transformshift{10.708223in}{1.209717in}%
\pgfsys@useobject{currentmarker}{}%
\end{pgfscope}%
\begin{pgfscope}%
\pgfsys@transformshift{10.725817in}{1.041988in}%
\pgfsys@useobject{currentmarker}{}%
\end{pgfscope}%
\begin{pgfscope}%
\pgfsys@transformshift{10.743411in}{0.998656in}%
\pgfsys@useobject{currentmarker}{}%
\end{pgfscope}%
\begin{pgfscope}%
\pgfsys@transformshift{10.761005in}{1.218691in}%
\pgfsys@useobject{currentmarker}{}%
\end{pgfscope}%
\begin{pgfscope}%
\pgfsys@transformshift{10.778599in}{1.133189in}%
\pgfsys@useobject{currentmarker}{}%
\end{pgfscope}%
\begin{pgfscope}%
\pgfsys@transformshift{10.796193in}{1.228631in}%
\pgfsys@useobject{currentmarker}{}%
\end{pgfscope}%
\begin{pgfscope}%
\pgfsys@transformshift{10.813787in}{1.207432in}%
\pgfsys@useobject{currentmarker}{}%
\end{pgfscope}%
\begin{pgfscope}%
\pgfsys@transformshift{10.831380in}{1.332763in}%
\pgfsys@useobject{currentmarker}{}%
\end{pgfscope}%
\begin{pgfscope}%
\pgfsys@transformshift{10.848974in}{1.358076in}%
\pgfsys@useobject{currentmarker}{}%
\end{pgfscope}%
\begin{pgfscope}%
\pgfsys@transformshift{10.866568in}{1.241852in}%
\pgfsys@useobject{currentmarker}{}%
\end{pgfscope}%
\begin{pgfscope}%
\pgfsys@transformshift{10.884162in}{1.201068in}%
\pgfsys@useobject{currentmarker}{}%
\end{pgfscope}%
\begin{pgfscope}%
\pgfsys@transformshift{10.901756in}{1.205290in}%
\pgfsys@useobject{currentmarker}{}%
\end{pgfscope}%
\begin{pgfscope}%
\pgfsys@transformshift{10.919350in}{1.446525in}%
\pgfsys@useobject{currentmarker}{}%
\end{pgfscope}%
\begin{pgfscope}%
\pgfsys@transformshift{10.936944in}{1.290250in}%
\pgfsys@useobject{currentmarker}{}%
\end{pgfscope}%
\begin{pgfscope}%
\pgfsys@transformshift{10.954537in}{1.379416in}%
\pgfsys@useobject{currentmarker}{}%
\end{pgfscope}%
\begin{pgfscope}%
\pgfsys@transformshift{10.972131in}{1.390722in}%
\pgfsys@useobject{currentmarker}{}%
\end{pgfscope}%
\begin{pgfscope}%
\pgfsys@transformshift{10.989725in}{1.463574in}%
\pgfsys@useobject{currentmarker}{}%
\end{pgfscope}%
\begin{pgfscope}%
\pgfsys@transformshift{11.007319in}{1.293783in}%
\pgfsys@useobject{currentmarker}{}%
\end{pgfscope}%
\begin{pgfscope}%
\pgfsys@transformshift{11.024913in}{1.520622in}%
\pgfsys@useobject{currentmarker}{}%
\end{pgfscope}%
\begin{pgfscope}%
\pgfsys@transformshift{11.042507in}{1.567933in}%
\pgfsys@useobject{currentmarker}{}%
\end{pgfscope}%
\begin{pgfscope}%
\pgfsys@transformshift{11.060101in}{1.536902in}%
\pgfsys@useobject{currentmarker}{}%
\end{pgfscope}%
\begin{pgfscope}%
\pgfsys@transformshift{11.077694in}{1.507609in}%
\pgfsys@useobject{currentmarker}{}%
\end{pgfscope}%
\begin{pgfscope}%
\pgfsys@transformshift{11.095288in}{1.556646in}%
\pgfsys@useobject{currentmarker}{}%
\end{pgfscope}%
\begin{pgfscope}%
\pgfsys@transformshift{11.112882in}{1.593168in}%
\pgfsys@useobject{currentmarker}{}%
\end{pgfscope}%
\begin{pgfscope}%
\pgfsys@transformshift{11.130476in}{1.282724in}%
\pgfsys@useobject{currentmarker}{}%
\end{pgfscope}%
\begin{pgfscope}%
\pgfsys@transformshift{11.148070in}{1.755190in}%
\pgfsys@useobject{currentmarker}{}%
\end{pgfscope}%
\begin{pgfscope}%
\pgfsys@transformshift{11.165664in}{1.600496in}%
\pgfsys@useobject{currentmarker}{}%
\end{pgfscope}%
\begin{pgfscope}%
\pgfsys@transformshift{11.183257in}{1.495977in}%
\pgfsys@useobject{currentmarker}{}%
\end{pgfscope}%
\begin{pgfscope}%
\pgfsys@transformshift{11.200851in}{1.519259in}%
\pgfsys@useobject{currentmarker}{}%
\end{pgfscope}%
\begin{pgfscope}%
\pgfsys@transformshift{11.218445in}{1.602814in}%
\pgfsys@useobject{currentmarker}{}%
\end{pgfscope}%
\begin{pgfscope}%
\pgfsys@transformshift{11.236039in}{1.536369in}%
\pgfsys@useobject{currentmarker}{}%
\end{pgfscope}%
\begin{pgfscope}%
\pgfsys@transformshift{11.253633in}{1.338906in}%
\pgfsys@useobject{currentmarker}{}%
\end{pgfscope}%
\begin{pgfscope}%
\pgfsys@transformshift{11.271227in}{1.737207in}%
\pgfsys@useobject{currentmarker}{}%
\end{pgfscope}%
\begin{pgfscope}%
\pgfsys@transformshift{11.288821in}{1.511509in}%
\pgfsys@useobject{currentmarker}{}%
\end{pgfscope}%
\begin{pgfscope}%
\pgfsys@transformshift{11.306414in}{1.613277in}%
\pgfsys@useobject{currentmarker}{}%
\end{pgfscope}%
\begin{pgfscope}%
\pgfsys@transformshift{11.324008in}{1.431866in}%
\pgfsys@useobject{currentmarker}{}%
\end{pgfscope}%
\begin{pgfscope}%
\pgfsys@transformshift{11.341602in}{1.641360in}%
\pgfsys@useobject{currentmarker}{}%
\end{pgfscope}%
\begin{pgfscope}%
\pgfsys@transformshift{11.359196in}{1.504922in}%
\pgfsys@useobject{currentmarker}{}%
\end{pgfscope}%
\begin{pgfscope}%
\pgfsys@transformshift{11.376790in}{1.524526in}%
\pgfsys@useobject{currentmarker}{}%
\end{pgfscope}%
\begin{pgfscope}%
\pgfsys@transformshift{11.394384in}{1.347796in}%
\pgfsys@useobject{currentmarker}{}%
\end{pgfscope}%
\begin{pgfscope}%
\pgfsys@transformshift{11.411978in}{1.559768in}%
\pgfsys@useobject{currentmarker}{}%
\end{pgfscope}%
\begin{pgfscope}%
\pgfsys@transformshift{11.429571in}{1.496221in}%
\pgfsys@useobject{currentmarker}{}%
\end{pgfscope}%
\begin{pgfscope}%
\pgfsys@transformshift{11.447165in}{1.545886in}%
\pgfsys@useobject{currentmarker}{}%
\end{pgfscope}%
\begin{pgfscope}%
\pgfsys@transformshift{11.464759in}{1.343841in}%
\pgfsys@useobject{currentmarker}{}%
\end{pgfscope}%
\begin{pgfscope}%
\pgfsys@transformshift{11.482353in}{1.349125in}%
\pgfsys@useobject{currentmarker}{}%
\end{pgfscope}%
\end{pgfscope}%
\begin{pgfscope}%
\pgfsetbuttcap%
\pgfsetroundjoin%
\definecolor{currentfill}{rgb}{0.000000,0.000000,0.000000}%
\pgfsetfillcolor{currentfill}%
\pgfsetlinewidth{0.803000pt}%
\definecolor{currentstroke}{rgb}{0.000000,0.000000,0.000000}%
\pgfsetstrokecolor{currentstroke}%
\pgfsetdash{}{0pt}%
\pgfsys@defobject{currentmarker}{\pgfqpoint{0.000000in}{-0.048611in}}{\pgfqpoint{0.000000in}{0.000000in}}{%
\pgfpathmoveto{\pgfqpoint{0.000000in}{0.000000in}}%
\pgfpathlineto{\pgfqpoint{0.000000in}{-0.048611in}}%
\pgfusepath{stroke,fill}%
}%
\begin{pgfscope}%
\pgfsys@transformshift{7.105882in}{0.750000in}%
\pgfsys@useobject{currentmarker}{}%
\end{pgfscope}%
\end{pgfscope}%
\begin{pgfscope}%
\pgftext[x=7.105882in,y=0.652778in,,top]{\rmfamily\fontsize{10.000000}{12.000000}\selectfont \(\displaystyle -1.5\)}%
\end{pgfscope}%
\begin{pgfscope}%
\pgfsetbuttcap%
\pgfsetroundjoin%
\definecolor{currentfill}{rgb}{0.000000,0.000000,0.000000}%
\pgfsetfillcolor{currentfill}%
\pgfsetlinewidth{0.803000pt}%
\definecolor{currentstroke}{rgb}{0.000000,0.000000,0.000000}%
\pgfsetstrokecolor{currentstroke}%
\pgfsetdash{}{0pt}%
\pgfsys@defobject{currentmarker}{\pgfqpoint{0.000000in}{-0.048611in}}{\pgfqpoint{0.000000in}{0.000000in}}{%
\pgfpathmoveto{\pgfqpoint{0.000000in}{0.000000in}}%
\pgfpathlineto{\pgfqpoint{0.000000in}{-0.048611in}}%
\pgfusepath{stroke,fill}%
}%
\begin{pgfscope}%
\pgfsys@transformshift{7.981176in}{0.750000in}%
\pgfsys@useobject{currentmarker}{}%
\end{pgfscope}%
\end{pgfscope}%
\begin{pgfscope}%
\pgftext[x=7.981176in,y=0.652778in,,top]{\rmfamily\fontsize{10.000000}{12.000000}\selectfont \(\displaystyle -1.0\)}%
\end{pgfscope}%
\begin{pgfscope}%
\pgfsetbuttcap%
\pgfsetroundjoin%
\definecolor{currentfill}{rgb}{0.000000,0.000000,0.000000}%
\pgfsetfillcolor{currentfill}%
\pgfsetlinewidth{0.803000pt}%
\definecolor{currentstroke}{rgb}{0.000000,0.000000,0.000000}%
\pgfsetstrokecolor{currentstroke}%
\pgfsetdash{}{0pt}%
\pgfsys@defobject{currentmarker}{\pgfqpoint{0.000000in}{-0.048611in}}{\pgfqpoint{0.000000in}{0.000000in}}{%
\pgfpathmoveto{\pgfqpoint{0.000000in}{0.000000in}}%
\pgfpathlineto{\pgfqpoint{0.000000in}{-0.048611in}}%
\pgfusepath{stroke,fill}%
}%
\begin{pgfscope}%
\pgfsys@transformshift{8.856471in}{0.750000in}%
\pgfsys@useobject{currentmarker}{}%
\end{pgfscope}%
\end{pgfscope}%
\begin{pgfscope}%
\pgftext[x=8.856471in,y=0.652778in,,top]{\rmfamily\fontsize{10.000000}{12.000000}\selectfont \(\displaystyle -0.5\)}%
\end{pgfscope}%
\begin{pgfscope}%
\pgfsetbuttcap%
\pgfsetroundjoin%
\definecolor{currentfill}{rgb}{0.000000,0.000000,0.000000}%
\pgfsetfillcolor{currentfill}%
\pgfsetlinewidth{0.803000pt}%
\definecolor{currentstroke}{rgb}{0.000000,0.000000,0.000000}%
\pgfsetstrokecolor{currentstroke}%
\pgfsetdash{}{0pt}%
\pgfsys@defobject{currentmarker}{\pgfqpoint{0.000000in}{-0.048611in}}{\pgfqpoint{0.000000in}{0.000000in}}{%
\pgfpathmoveto{\pgfqpoint{0.000000in}{0.000000in}}%
\pgfpathlineto{\pgfqpoint{0.000000in}{-0.048611in}}%
\pgfusepath{stroke,fill}%
}%
\begin{pgfscope}%
\pgfsys@transformshift{9.731765in}{0.750000in}%
\pgfsys@useobject{currentmarker}{}%
\end{pgfscope}%
\end{pgfscope}%
\begin{pgfscope}%
\pgftext[x=9.731765in,y=0.652778in,,top]{\rmfamily\fontsize{10.000000}{12.000000}\selectfont \(\displaystyle 0.0\)}%
\end{pgfscope}%
\begin{pgfscope}%
\pgfsetbuttcap%
\pgfsetroundjoin%
\definecolor{currentfill}{rgb}{0.000000,0.000000,0.000000}%
\pgfsetfillcolor{currentfill}%
\pgfsetlinewidth{0.803000pt}%
\definecolor{currentstroke}{rgb}{0.000000,0.000000,0.000000}%
\pgfsetstrokecolor{currentstroke}%
\pgfsetdash{}{0pt}%
\pgfsys@defobject{currentmarker}{\pgfqpoint{0.000000in}{-0.048611in}}{\pgfqpoint{0.000000in}{0.000000in}}{%
\pgfpathmoveto{\pgfqpoint{0.000000in}{0.000000in}}%
\pgfpathlineto{\pgfqpoint{0.000000in}{-0.048611in}}%
\pgfusepath{stroke,fill}%
}%
\begin{pgfscope}%
\pgfsys@transformshift{10.607059in}{0.750000in}%
\pgfsys@useobject{currentmarker}{}%
\end{pgfscope}%
\end{pgfscope}%
\begin{pgfscope}%
\pgftext[x=10.607059in,y=0.652778in,,top]{\rmfamily\fontsize{10.000000}{12.000000}\selectfont \(\displaystyle 0.5\)}%
\end{pgfscope}%
\begin{pgfscope}%
\pgfsetbuttcap%
\pgfsetroundjoin%
\definecolor{currentfill}{rgb}{0.000000,0.000000,0.000000}%
\pgfsetfillcolor{currentfill}%
\pgfsetlinewidth{0.803000pt}%
\definecolor{currentstroke}{rgb}{0.000000,0.000000,0.000000}%
\pgfsetstrokecolor{currentstroke}%
\pgfsetdash{}{0pt}%
\pgfsys@defobject{currentmarker}{\pgfqpoint{0.000000in}{-0.048611in}}{\pgfqpoint{0.000000in}{0.000000in}}{%
\pgfpathmoveto{\pgfqpoint{0.000000in}{0.000000in}}%
\pgfpathlineto{\pgfqpoint{0.000000in}{-0.048611in}}%
\pgfusepath{stroke,fill}%
}%
\begin{pgfscope}%
\pgfsys@transformshift{11.482353in}{0.750000in}%
\pgfsys@useobject{currentmarker}{}%
\end{pgfscope}%
\end{pgfscope}%
\begin{pgfscope}%
\pgftext[x=11.482353in,y=0.652778in,,top]{\rmfamily\fontsize{10.000000}{12.000000}\selectfont \(\displaystyle 1.0\)}%
\end{pgfscope}%
\begin{pgfscope}%
\pgftext[x=9.294118in,y=0.471083in,,top]{\rmfamily\fontsize{10.000000}{12.000000}\selectfont x}%
\end{pgfscope}%
\begin{pgfscope}%
\pgfsetbuttcap%
\pgfsetroundjoin%
\definecolor{currentfill}{rgb}{0.000000,0.000000,0.000000}%
\pgfsetfillcolor{currentfill}%
\pgfsetlinewidth{0.803000pt}%
\definecolor{currentstroke}{rgb}{0.000000,0.000000,0.000000}%
\pgfsetstrokecolor{currentstroke}%
\pgfsetdash{}{0pt}%
\pgfsys@defobject{currentmarker}{\pgfqpoint{-0.048611in}{0.000000in}}{\pgfqpoint{0.000000in}{0.000000in}}{%
\pgfpathmoveto{\pgfqpoint{0.000000in}{0.000000in}}%
\pgfpathlineto{\pgfqpoint{-0.048611in}{0.000000in}}%
\pgfusepath{stroke,fill}%
}%
\begin{pgfscope}%
\pgfsys@transformshift{7.105882in}{1.107632in}%
\pgfsys@useobject{currentmarker}{}%
\end{pgfscope}%
\end{pgfscope}%
\begin{pgfscope}%
\pgftext[x=6.939215in,y=1.059414in,left,base]{\rmfamily\fontsize{10.000000}{12.000000}\selectfont \(\displaystyle 0\)}%
\end{pgfscope}%
\begin{pgfscope}%
\pgfsetbuttcap%
\pgfsetroundjoin%
\definecolor{currentfill}{rgb}{0.000000,0.000000,0.000000}%
\pgfsetfillcolor{currentfill}%
\pgfsetlinewidth{0.803000pt}%
\definecolor{currentstroke}{rgb}{0.000000,0.000000,0.000000}%
\pgfsetstrokecolor{currentstroke}%
\pgfsetdash{}{0pt}%
\pgfsys@defobject{currentmarker}{\pgfqpoint{-0.048611in}{0.000000in}}{\pgfqpoint{0.000000in}{0.000000in}}{%
\pgfpathmoveto{\pgfqpoint{0.000000in}{0.000000in}}%
\pgfpathlineto{\pgfqpoint{-0.048611in}{0.000000in}}%
\pgfusepath{stroke,fill}%
}%
\begin{pgfscope}%
\pgfsys@transformshift{7.105882in}{1.505000in}%
\pgfsys@useobject{currentmarker}{}%
\end{pgfscope}%
\end{pgfscope}%
\begin{pgfscope}%
\pgftext[x=6.939215in,y=1.456782in,left,base]{\rmfamily\fontsize{10.000000}{12.000000}\selectfont \(\displaystyle 2\)}%
\end{pgfscope}%
\begin{pgfscope}%
\pgfpathrectangle{\pgfqpoint{7.105882in}{0.750000in}}{\pgfqpoint{4.376471in}{0.953684in}} %
\pgfusepath{clip}%
\pgfsetrectcap%
\pgfsetroundjoin%
\pgfsetlinewidth{1.505625pt}%
\definecolor{currentstroke}{rgb}{0.121569,0.466667,0.705882}%
\pgfsetstrokecolor{currentstroke}%
\pgfsetdash{}{0pt}%
\pgfpathmoveto{\pgfqpoint{7.981176in}{1.107640in}}%
\pgfpathlineto{\pgfqpoint{11.482353in}{1.107629in}}%
\pgfpathlineto{\pgfqpoint{11.482353in}{1.107629in}}%
\pgfusepath{stroke}%
\end{pgfscope}%
\begin{pgfscope}%
\pgfsetrectcap%
\pgfsetmiterjoin%
\pgfsetlinewidth{0.803000pt}%
\definecolor{currentstroke}{rgb}{0.000000,0.000000,0.000000}%
\pgfsetstrokecolor{currentstroke}%
\pgfsetdash{}{0pt}%
\pgfpathmoveto{\pgfqpoint{7.105882in}{0.750000in}}%
\pgfpathlineto{\pgfqpoint{7.105882in}{1.703684in}}%
\pgfusepath{stroke}%
\end{pgfscope}%
\begin{pgfscope}%
\pgfsetrectcap%
\pgfsetmiterjoin%
\pgfsetlinewidth{0.803000pt}%
\definecolor{currentstroke}{rgb}{0.000000,0.000000,0.000000}%
\pgfsetstrokecolor{currentstroke}%
\pgfsetdash{}{0pt}%
\pgfpathmoveto{\pgfqpoint{11.482353in}{0.750000in}}%
\pgfpathlineto{\pgfqpoint{11.482353in}{1.703684in}}%
\pgfusepath{stroke}%
\end{pgfscope}%
\begin{pgfscope}%
\pgfsetrectcap%
\pgfsetmiterjoin%
\pgfsetlinewidth{0.803000pt}%
\definecolor{currentstroke}{rgb}{0.000000,0.000000,0.000000}%
\pgfsetstrokecolor{currentstroke}%
\pgfsetdash{}{0pt}%
\pgfpathmoveto{\pgfqpoint{7.105882in}{0.750000in}}%
\pgfpathlineto{\pgfqpoint{11.482353in}{0.750000in}}%
\pgfusepath{stroke}%
\end{pgfscope}%
\begin{pgfscope}%
\pgfsetrectcap%
\pgfsetmiterjoin%
\pgfsetlinewidth{0.803000pt}%
\definecolor{currentstroke}{rgb}{0.000000,0.000000,0.000000}%
\pgfsetstrokecolor{currentstroke}%
\pgfsetdash{}{0pt}%
\pgfpathmoveto{\pgfqpoint{7.105882in}{1.703684in}}%
\pgfpathlineto{\pgfqpoint{11.482353in}{1.703684in}}%
\pgfusepath{stroke}%
\end{pgfscope}%
\begin{pgfscope}%
\pgfsetbuttcap%
\pgfsetmiterjoin%
\definecolor{currentfill}{rgb}{1.000000,1.000000,1.000000}%
\pgfsetfillcolor{currentfill}%
\pgfsetfillopacity{0.800000}%
\pgfsetlinewidth{1.003750pt}%
\definecolor{currentstroke}{rgb}{0.800000,0.800000,0.800000}%
\pgfsetstrokecolor{currentstroke}%
\pgfsetstrokeopacity{0.800000}%
\pgfsetdash{}{0pt}%
\pgfpathmoveto{\pgfqpoint{7.203105in}{0.819444in}}%
\pgfpathlineto{\pgfqpoint{7.944617in}{0.819444in}}%
\pgfpathquadraticcurveto{\pgfqpoint{7.972395in}{0.819444in}}{\pgfqpoint{7.972395in}{0.847222in}}%
\pgfpathlineto{\pgfqpoint{7.972395in}{1.430935in}}%
\pgfpathquadraticcurveto{\pgfqpoint{7.972395in}{1.458713in}}{\pgfqpoint{7.944617in}{1.458713in}}%
\pgfpathlineto{\pgfqpoint{7.203105in}{1.458713in}}%
\pgfpathquadraticcurveto{\pgfqpoint{7.175327in}{1.458713in}}{\pgfqpoint{7.175327in}{1.430935in}}%
\pgfpathlineto{\pgfqpoint{7.175327in}{0.847222in}}%
\pgfpathquadraticcurveto{\pgfqpoint{7.175327in}{0.819444in}}{\pgfqpoint{7.203105in}{0.819444in}}%
\pgfpathclose%
\pgfusepath{stroke,fill}%
\end{pgfscope}%
\begin{pgfscope}%
\pgfsetrectcap%
\pgfsetroundjoin%
\pgfsetlinewidth{1.505625pt}%
\definecolor{currentstroke}{rgb}{0.121569,0.466667,0.705882}%
\pgfsetstrokecolor{currentstroke}%
\pgfsetdash{}{0pt}%
\pgfpathmoveto{\pgfqpoint{7.230882in}{1.346055in}}%
\pgfpathlineto{\pgfqpoint{7.508660in}{1.346055in}}%
\pgfusepath{stroke}%
\end{pgfscope}%
\begin{pgfscope}%
\pgftext[x=7.619771in,y=1.297444in,left,base]{\rmfamily\fontsize{10.000000}{12.000000}\selectfont \(\displaystyle \widetilde{\Phi}^* \theta^{\parallel}\)}%
\end{pgfscope}%
\begin{pgfscope}%
\pgfsetbuttcap%
\pgfsetroundjoin%
\definecolor{currentfill}{rgb}{1.000000,0.000000,0.000000}%
\pgfsetfillcolor{currentfill}%
\pgfsetlinewidth{2.007500pt}%
\definecolor{currentstroke}{rgb}{1.000000,0.000000,0.000000}%
\pgfsetstrokecolor{currentstroke}%
\pgfsetdash{}{0pt}%
\pgfpathmoveto{\pgfqpoint{7.328105in}{1.137532in}}%
\pgfpathlineto{\pgfqpoint{7.411438in}{1.137532in}}%
\pgfpathmoveto{\pgfqpoint{7.369771in}{1.095866in}}%
\pgfpathlineto{\pgfqpoint{7.369771in}{1.179199in}}%
\pgfusepath{stroke,fill}%
\end{pgfscope}%
\begin{pgfscope}%
\pgftext[x=7.619771in,y=1.101074in,left,base]{\rmfamily\fontsize{10.000000}{12.000000}\selectfont train}%
\end{pgfscope}%
\begin{pgfscope}%
\pgfsetbuttcap%
\pgfsetroundjoin%
\definecolor{currentfill}{rgb}{0.000000,0.000000,0.000000}%
\pgfsetfillcolor{currentfill}%
\pgfsetlinewidth{1.003750pt}%
\definecolor{currentstroke}{rgb}{0.000000,0.000000,0.000000}%
\pgfsetstrokecolor{currentstroke}%
\pgfsetdash{}{0pt}%
\pgfsys@defobject{currentmarker}{\pgfqpoint{-0.020833in}{-0.020833in}}{\pgfqpoint{0.020833in}{0.020833in}}{%
\pgfpathmoveto{\pgfqpoint{0.000000in}{-0.020833in}}%
\pgfpathcurveto{\pgfqpoint{0.005525in}{-0.020833in}}{\pgfqpoint{0.010825in}{-0.018638in}}{\pgfqpoint{0.014731in}{-0.014731in}}%
\pgfpathcurveto{\pgfqpoint{0.018638in}{-0.010825in}}{\pgfqpoint{0.020833in}{-0.005525in}}{\pgfqpoint{0.020833in}{0.000000in}}%
\pgfpathcurveto{\pgfqpoint{0.020833in}{0.005525in}}{\pgfqpoint{0.018638in}{0.010825in}}{\pgfqpoint{0.014731in}{0.014731in}}%
\pgfpathcurveto{\pgfqpoint{0.010825in}{0.018638in}}{\pgfqpoint{0.005525in}{0.020833in}}{\pgfqpoint{0.000000in}{0.020833in}}%
\pgfpathcurveto{\pgfqpoint{-0.005525in}{0.020833in}}{\pgfqpoint{-0.010825in}{0.018638in}}{\pgfqpoint{-0.014731in}{0.014731in}}%
\pgfpathcurveto{\pgfqpoint{-0.018638in}{0.010825in}}{\pgfqpoint{-0.020833in}{0.005525in}}{\pgfqpoint{-0.020833in}{0.000000in}}%
\pgfpathcurveto{\pgfqpoint{-0.020833in}{-0.005525in}}{\pgfqpoint{-0.018638in}{-0.010825in}}{\pgfqpoint{-0.014731in}{-0.014731in}}%
\pgfpathcurveto{\pgfqpoint{-0.010825in}{-0.018638in}}{\pgfqpoint{-0.005525in}{-0.020833in}}{\pgfqpoint{0.000000in}{-0.020833in}}%
\pgfpathclose%
\pgfusepath{stroke,fill}%
}%
\begin{pgfscope}%
\pgfsys@transformshift{7.369771in}{0.941162in}%
\pgfsys@useobject{currentmarker}{}%
\end{pgfscope}%
\end{pgfscope}%
\begin{pgfscope}%
\pgftext[x=7.619771in,y=0.904704in,left,base]{\rmfamily\fontsize{10.000000}{12.000000}\selectfont test}%
\end{pgfscope}%
\begin{pgfscope}%
\pgfsetbuttcap%
\pgfsetmiterjoin%
\definecolor{currentfill}{rgb}{1.000000,1.000000,1.000000}%
\pgfsetfillcolor{currentfill}%
\pgfsetlinewidth{0.000000pt}%
\definecolor{currentstroke}{rgb}{0.000000,0.000000,0.000000}%
\pgfsetstrokecolor{currentstroke}%
\pgfsetstrokeopacity{0.000000}%
\pgfsetdash{}{0pt}%
\pgfpathmoveto{\pgfqpoint{12.211765in}{0.750000in}}%
\pgfpathlineto{\pgfqpoint{14.400000in}{0.750000in}}%
\pgfpathlineto{\pgfqpoint{14.400000in}{1.703684in}}%
\pgfpathlineto{\pgfqpoint{12.211765in}{1.703684in}}%
\pgfpathclose%
\pgfusepath{fill}%
\end{pgfscope}%
\begin{pgfscope}%
\pgfpathrectangle{\pgfqpoint{12.211765in}{0.750000in}}{\pgfqpoint{2.188235in}{0.953684in}} %
\pgfusepath{clip}%
\pgfsetbuttcap%
\pgfsetmiterjoin%
\definecolor{currentfill}{rgb}{0.121569,0.466667,0.705882}%
\pgfsetfillcolor{currentfill}%
\pgfsetlinewidth{0.000000pt}%
\definecolor{currentstroke}{rgb}{0.000000,0.000000,0.000000}%
\pgfsetstrokecolor{currentstroke}%
\pgfsetstrokeopacity{0.000000}%
\pgfsetdash{}{0pt}%
\pgfpathmoveto{\pgfqpoint{-23.809001in}{0.793349in}}%
\pgfpathlineto{\pgfqpoint{14.206019in}{0.793349in}}%
\pgfpathlineto{\pgfqpoint{14.206019in}{0.800299in}}%
\pgfpathlineto{\pgfqpoint{-23.809001in}{0.800299in}}%
\pgfpathclose%
\pgfusepath{fill}%
\end{pgfscope}%
\begin{pgfscope}%
\pgfpathrectangle{\pgfqpoint{12.211765in}{0.750000in}}{\pgfqpoint{2.188235in}{0.953684in}} %
\pgfusepath{clip}%
\pgfsetbuttcap%
\pgfsetmiterjoin%
\definecolor{currentfill}{rgb}{0.121569,0.466667,0.705882}%
\pgfsetfillcolor{currentfill}%
\pgfsetlinewidth{0.000000pt}%
\definecolor{currentstroke}{rgb}{0.000000,0.000000,0.000000}%
\pgfsetstrokecolor{currentstroke}%
\pgfsetstrokeopacity{0.000000}%
\pgfsetdash{}{0pt}%
\pgfpathmoveto{\pgfqpoint{-23.809001in}{0.802037in}}%
\pgfpathlineto{\pgfqpoint{14.259799in}{0.802037in}}%
\pgfpathlineto{\pgfqpoint{14.259799in}{0.808986in}}%
\pgfpathlineto{\pgfqpoint{-23.809001in}{0.808986in}}%
\pgfpathclose%
\pgfusepath{fill}%
\end{pgfscope}%
\begin{pgfscope}%
\pgfpathrectangle{\pgfqpoint{12.211765in}{0.750000in}}{\pgfqpoint{2.188235in}{0.953684in}} %
\pgfusepath{clip}%
\pgfsetbuttcap%
\pgfsetmiterjoin%
\definecolor{currentfill}{rgb}{0.121569,0.466667,0.705882}%
\pgfsetfillcolor{currentfill}%
\pgfsetlinewidth{0.000000pt}%
\definecolor{currentstroke}{rgb}{0.000000,0.000000,0.000000}%
\pgfsetstrokecolor{currentstroke}%
\pgfsetstrokeopacity{0.000000}%
\pgfsetdash{}{0pt}%
\pgfpathmoveto{\pgfqpoint{-23.809001in}{0.810724in}}%
\pgfpathlineto{\pgfqpoint{14.123328in}{0.810724in}}%
\pgfpathlineto{\pgfqpoint{14.123328in}{0.817674in}}%
\pgfpathlineto{\pgfqpoint{-23.809001in}{0.817674in}}%
\pgfpathclose%
\pgfusepath{fill}%
\end{pgfscope}%
\begin{pgfscope}%
\pgfpathrectangle{\pgfqpoint{12.211765in}{0.750000in}}{\pgfqpoint{2.188235in}{0.953684in}} %
\pgfusepath{clip}%
\pgfsetbuttcap%
\pgfsetmiterjoin%
\definecolor{currentfill}{rgb}{0.121569,0.466667,0.705882}%
\pgfsetfillcolor{currentfill}%
\pgfsetlinewidth{0.000000pt}%
\definecolor{currentstroke}{rgb}{0.000000,0.000000,0.000000}%
\pgfsetstrokecolor{currentstroke}%
\pgfsetstrokeopacity{0.000000}%
\pgfsetdash{}{0pt}%
\pgfpathmoveto{\pgfqpoint{-23.809001in}{0.819411in}}%
\pgfpathlineto{\pgfqpoint{14.232679in}{0.819411in}}%
\pgfpathlineto{\pgfqpoint{14.232679in}{0.826361in}}%
\pgfpathlineto{\pgfqpoint{-23.809001in}{0.826361in}}%
\pgfpathclose%
\pgfusepath{fill}%
\end{pgfscope}%
\begin{pgfscope}%
\pgfpathrectangle{\pgfqpoint{12.211765in}{0.750000in}}{\pgfqpoint{2.188235in}{0.953684in}} %
\pgfusepath{clip}%
\pgfsetbuttcap%
\pgfsetmiterjoin%
\definecolor{currentfill}{rgb}{0.121569,0.466667,0.705882}%
\pgfsetfillcolor{currentfill}%
\pgfsetlinewidth{0.000000pt}%
\definecolor{currentstroke}{rgb}{0.000000,0.000000,0.000000}%
\pgfsetstrokecolor{currentstroke}%
\pgfsetstrokeopacity{0.000000}%
\pgfsetdash{}{0pt}%
\pgfpathmoveto{\pgfqpoint{-23.809001in}{0.828098in}}%
\pgfpathlineto{\pgfqpoint{14.258827in}{0.828098in}}%
\pgfpathlineto{\pgfqpoint{14.258827in}{0.835048in}}%
\pgfpathlineto{\pgfqpoint{-23.809001in}{0.835048in}}%
\pgfpathclose%
\pgfusepath{fill}%
\end{pgfscope}%
\begin{pgfscope}%
\pgfpathrectangle{\pgfqpoint{12.211765in}{0.750000in}}{\pgfqpoint{2.188235in}{0.953684in}} %
\pgfusepath{clip}%
\pgfsetbuttcap%
\pgfsetmiterjoin%
\definecolor{currentfill}{rgb}{0.121569,0.466667,0.705882}%
\pgfsetfillcolor{currentfill}%
\pgfsetlinewidth{0.000000pt}%
\definecolor{currentstroke}{rgb}{0.000000,0.000000,0.000000}%
\pgfsetstrokecolor{currentstroke}%
\pgfsetstrokeopacity{0.000000}%
\pgfsetdash{}{0pt}%
\pgfpathmoveto{\pgfqpoint{-23.809001in}{0.836785in}}%
\pgfpathlineto{\pgfqpoint{14.288587in}{0.836785in}}%
\pgfpathlineto{\pgfqpoint{14.288587in}{0.843735in}}%
\pgfpathlineto{\pgfqpoint{-23.809001in}{0.843735in}}%
\pgfpathclose%
\pgfusepath{fill}%
\end{pgfscope}%
\begin{pgfscope}%
\pgfpathrectangle{\pgfqpoint{12.211765in}{0.750000in}}{\pgfqpoint{2.188235in}{0.953684in}} %
\pgfusepath{clip}%
\pgfsetbuttcap%
\pgfsetmiterjoin%
\definecolor{currentfill}{rgb}{0.121569,0.466667,0.705882}%
\pgfsetfillcolor{currentfill}%
\pgfsetlinewidth{0.000000pt}%
\definecolor{currentstroke}{rgb}{0.000000,0.000000,0.000000}%
\pgfsetstrokecolor{currentstroke}%
\pgfsetstrokeopacity{0.000000}%
\pgfsetdash{}{0pt}%
\pgfpathmoveto{\pgfqpoint{-23.809001in}{0.845473in}}%
\pgfpathlineto{\pgfqpoint{14.183092in}{0.845473in}}%
\pgfpathlineto{\pgfqpoint{14.183092in}{0.852422in}}%
\pgfpathlineto{\pgfqpoint{-23.809001in}{0.852422in}}%
\pgfpathclose%
\pgfusepath{fill}%
\end{pgfscope}%
\begin{pgfscope}%
\pgfpathrectangle{\pgfqpoint{12.211765in}{0.750000in}}{\pgfqpoint{2.188235in}{0.953684in}} %
\pgfusepath{clip}%
\pgfsetbuttcap%
\pgfsetmiterjoin%
\definecolor{currentfill}{rgb}{0.121569,0.466667,0.705882}%
\pgfsetfillcolor{currentfill}%
\pgfsetlinewidth{0.000000pt}%
\definecolor{currentstroke}{rgb}{0.000000,0.000000,0.000000}%
\pgfsetstrokecolor{currentstroke}%
\pgfsetstrokeopacity{0.000000}%
\pgfsetdash{}{0pt}%
\pgfpathmoveto{\pgfqpoint{-23.809001in}{0.854160in}}%
\pgfpathlineto{\pgfqpoint{14.256713in}{0.854160in}}%
\pgfpathlineto{\pgfqpoint{14.256713in}{0.861110in}}%
\pgfpathlineto{\pgfqpoint{-23.809001in}{0.861110in}}%
\pgfpathclose%
\pgfusepath{fill}%
\end{pgfscope}%
\begin{pgfscope}%
\pgfpathrectangle{\pgfqpoint{12.211765in}{0.750000in}}{\pgfqpoint{2.188235in}{0.953684in}} %
\pgfusepath{clip}%
\pgfsetbuttcap%
\pgfsetmiterjoin%
\definecolor{currentfill}{rgb}{0.121569,0.466667,0.705882}%
\pgfsetfillcolor{currentfill}%
\pgfsetlinewidth{0.000000pt}%
\definecolor{currentstroke}{rgb}{0.000000,0.000000,0.000000}%
\pgfsetstrokecolor{currentstroke}%
\pgfsetstrokeopacity{0.000000}%
\pgfsetdash{}{0pt}%
\pgfpathmoveto{\pgfqpoint{-23.809001in}{0.862847in}}%
\pgfpathlineto{\pgfqpoint{14.205748in}{0.862847in}}%
\pgfpathlineto{\pgfqpoint{14.205748in}{0.869797in}}%
\pgfpathlineto{\pgfqpoint{-23.809001in}{0.869797in}}%
\pgfpathclose%
\pgfusepath{fill}%
\end{pgfscope}%
\begin{pgfscope}%
\pgfpathrectangle{\pgfqpoint{12.211765in}{0.750000in}}{\pgfqpoint{2.188235in}{0.953684in}} %
\pgfusepath{clip}%
\pgfsetbuttcap%
\pgfsetmiterjoin%
\definecolor{currentfill}{rgb}{0.121569,0.466667,0.705882}%
\pgfsetfillcolor{currentfill}%
\pgfsetlinewidth{0.000000pt}%
\definecolor{currentstroke}{rgb}{0.000000,0.000000,0.000000}%
\pgfsetstrokecolor{currentstroke}%
\pgfsetstrokeopacity{0.000000}%
\pgfsetdash{}{0pt}%
\pgfpathmoveto{\pgfqpoint{-23.809001in}{0.871534in}}%
\pgfpathlineto{\pgfqpoint{14.247244in}{0.871534in}}%
\pgfpathlineto{\pgfqpoint{14.247244in}{0.878484in}}%
\pgfpathlineto{\pgfqpoint{-23.809001in}{0.878484in}}%
\pgfpathclose%
\pgfusepath{fill}%
\end{pgfscope}%
\begin{pgfscope}%
\pgfpathrectangle{\pgfqpoint{12.211765in}{0.750000in}}{\pgfqpoint{2.188235in}{0.953684in}} %
\pgfusepath{clip}%
\pgfsetbuttcap%
\pgfsetmiterjoin%
\definecolor{currentfill}{rgb}{0.121569,0.466667,0.705882}%
\pgfsetfillcolor{currentfill}%
\pgfsetlinewidth{0.000000pt}%
\definecolor{currentstroke}{rgb}{0.000000,0.000000,0.000000}%
\pgfsetstrokecolor{currentstroke}%
\pgfsetstrokeopacity{0.000000}%
\pgfsetdash{}{0pt}%
\pgfpathmoveto{\pgfqpoint{-23.809001in}{0.880222in}}%
\pgfpathlineto{\pgfqpoint{14.190177in}{0.880222in}}%
\pgfpathlineto{\pgfqpoint{14.190177in}{0.887171in}}%
\pgfpathlineto{\pgfqpoint{-23.809001in}{0.887171in}}%
\pgfpathclose%
\pgfusepath{fill}%
\end{pgfscope}%
\begin{pgfscope}%
\pgfpathrectangle{\pgfqpoint{12.211765in}{0.750000in}}{\pgfqpoint{2.188235in}{0.953684in}} %
\pgfusepath{clip}%
\pgfsetbuttcap%
\pgfsetmiterjoin%
\definecolor{currentfill}{rgb}{0.121569,0.466667,0.705882}%
\pgfsetfillcolor{currentfill}%
\pgfsetlinewidth{0.000000pt}%
\definecolor{currentstroke}{rgb}{0.000000,0.000000,0.000000}%
\pgfsetstrokecolor{currentstroke}%
\pgfsetstrokeopacity{0.000000}%
\pgfsetdash{}{0pt}%
\pgfpathmoveto{\pgfqpoint{-23.809001in}{0.888909in}}%
\pgfpathlineto{\pgfqpoint{14.277603in}{0.888909in}}%
\pgfpathlineto{\pgfqpoint{14.277603in}{0.895859in}}%
\pgfpathlineto{\pgfqpoint{-23.809001in}{0.895859in}}%
\pgfpathclose%
\pgfusepath{fill}%
\end{pgfscope}%
\begin{pgfscope}%
\pgfpathrectangle{\pgfqpoint{12.211765in}{0.750000in}}{\pgfqpoint{2.188235in}{0.953684in}} %
\pgfusepath{clip}%
\pgfsetbuttcap%
\pgfsetmiterjoin%
\definecolor{currentfill}{rgb}{0.121569,0.466667,0.705882}%
\pgfsetfillcolor{currentfill}%
\pgfsetlinewidth{0.000000pt}%
\definecolor{currentstroke}{rgb}{0.000000,0.000000,0.000000}%
\pgfsetstrokecolor{currentstroke}%
\pgfsetstrokeopacity{0.000000}%
\pgfsetdash{}{0pt}%
\pgfpathmoveto{\pgfqpoint{-23.809001in}{0.897596in}}%
\pgfpathlineto{\pgfqpoint{14.258364in}{0.897596in}}%
\pgfpathlineto{\pgfqpoint{14.258364in}{0.904546in}}%
\pgfpathlineto{\pgfqpoint{-23.809001in}{0.904546in}}%
\pgfpathclose%
\pgfusepath{fill}%
\end{pgfscope}%
\begin{pgfscope}%
\pgfpathrectangle{\pgfqpoint{12.211765in}{0.750000in}}{\pgfqpoint{2.188235in}{0.953684in}} %
\pgfusepath{clip}%
\pgfsetbuttcap%
\pgfsetmiterjoin%
\definecolor{currentfill}{rgb}{0.121569,0.466667,0.705882}%
\pgfsetfillcolor{currentfill}%
\pgfsetlinewidth{0.000000pt}%
\definecolor{currentstroke}{rgb}{0.000000,0.000000,0.000000}%
\pgfsetstrokecolor{currentstroke}%
\pgfsetstrokeopacity{0.000000}%
\pgfsetdash{}{0pt}%
\pgfpathmoveto{\pgfqpoint{-23.809001in}{0.906283in}}%
\pgfpathlineto{\pgfqpoint{14.170208in}{0.906283in}}%
\pgfpathlineto{\pgfqpoint{14.170208in}{0.913233in}}%
\pgfpathlineto{\pgfqpoint{-23.809001in}{0.913233in}}%
\pgfpathclose%
\pgfusepath{fill}%
\end{pgfscope}%
\begin{pgfscope}%
\pgfpathrectangle{\pgfqpoint{12.211765in}{0.750000in}}{\pgfqpoint{2.188235in}{0.953684in}} %
\pgfusepath{clip}%
\pgfsetbuttcap%
\pgfsetmiterjoin%
\definecolor{currentfill}{rgb}{0.121569,0.466667,0.705882}%
\pgfsetfillcolor{currentfill}%
\pgfsetlinewidth{0.000000pt}%
\definecolor{currentstroke}{rgb}{0.000000,0.000000,0.000000}%
\pgfsetstrokecolor{currentstroke}%
\pgfsetstrokeopacity{0.000000}%
\pgfsetdash{}{0pt}%
\pgfpathmoveto{\pgfqpoint{-23.809001in}{0.914971in}}%
\pgfpathlineto{\pgfqpoint{14.067398in}{0.914971in}}%
\pgfpathlineto{\pgfqpoint{14.067398in}{0.921920in}}%
\pgfpathlineto{\pgfqpoint{-23.809001in}{0.921920in}}%
\pgfpathclose%
\pgfusepath{fill}%
\end{pgfscope}%
\begin{pgfscope}%
\pgfpathrectangle{\pgfqpoint{12.211765in}{0.750000in}}{\pgfqpoint{2.188235in}{0.953684in}} %
\pgfusepath{clip}%
\pgfsetbuttcap%
\pgfsetmiterjoin%
\definecolor{currentfill}{rgb}{0.121569,0.466667,0.705882}%
\pgfsetfillcolor{currentfill}%
\pgfsetlinewidth{0.000000pt}%
\definecolor{currentstroke}{rgb}{0.000000,0.000000,0.000000}%
\pgfsetstrokecolor{currentstroke}%
\pgfsetstrokeopacity{0.000000}%
\pgfsetdash{}{0pt}%
\pgfpathmoveto{\pgfqpoint{-23.809001in}{0.923658in}}%
\pgfpathlineto{\pgfqpoint{14.150474in}{0.923658in}}%
\pgfpathlineto{\pgfqpoint{14.150474in}{0.930608in}}%
\pgfpathlineto{\pgfqpoint{-23.809001in}{0.930608in}}%
\pgfpathclose%
\pgfusepath{fill}%
\end{pgfscope}%
\begin{pgfscope}%
\pgfpathrectangle{\pgfqpoint{12.211765in}{0.750000in}}{\pgfqpoint{2.188235in}{0.953684in}} %
\pgfusepath{clip}%
\pgfsetbuttcap%
\pgfsetmiterjoin%
\definecolor{currentfill}{rgb}{0.121569,0.466667,0.705882}%
\pgfsetfillcolor{currentfill}%
\pgfsetlinewidth{0.000000pt}%
\definecolor{currentstroke}{rgb}{0.000000,0.000000,0.000000}%
\pgfsetstrokecolor{currentstroke}%
\pgfsetstrokeopacity{0.000000}%
\pgfsetdash{}{0pt}%
\pgfpathmoveto{\pgfqpoint{-23.809001in}{0.932345in}}%
\pgfpathlineto{\pgfqpoint{14.262202in}{0.932345in}}%
\pgfpathlineto{\pgfqpoint{14.262202in}{0.939295in}}%
\pgfpathlineto{\pgfqpoint{-23.809001in}{0.939295in}}%
\pgfpathclose%
\pgfusepath{fill}%
\end{pgfscope}%
\begin{pgfscope}%
\pgfpathrectangle{\pgfqpoint{12.211765in}{0.750000in}}{\pgfqpoint{2.188235in}{0.953684in}} %
\pgfusepath{clip}%
\pgfsetbuttcap%
\pgfsetmiterjoin%
\definecolor{currentfill}{rgb}{0.121569,0.466667,0.705882}%
\pgfsetfillcolor{currentfill}%
\pgfsetlinewidth{0.000000pt}%
\definecolor{currentstroke}{rgb}{0.000000,0.000000,0.000000}%
\pgfsetstrokecolor{currentstroke}%
\pgfsetstrokeopacity{0.000000}%
\pgfsetdash{}{0pt}%
\pgfpathmoveto{\pgfqpoint{-23.809001in}{0.941032in}}%
\pgfpathlineto{\pgfqpoint{14.220569in}{0.941032in}}%
\pgfpathlineto{\pgfqpoint{14.220569in}{0.947982in}}%
\pgfpathlineto{\pgfqpoint{-23.809001in}{0.947982in}}%
\pgfpathclose%
\pgfusepath{fill}%
\end{pgfscope}%
\begin{pgfscope}%
\pgfpathrectangle{\pgfqpoint{12.211765in}{0.750000in}}{\pgfqpoint{2.188235in}{0.953684in}} %
\pgfusepath{clip}%
\pgfsetbuttcap%
\pgfsetmiterjoin%
\definecolor{currentfill}{rgb}{0.121569,0.466667,0.705882}%
\pgfsetfillcolor{currentfill}%
\pgfsetlinewidth{0.000000pt}%
\definecolor{currentstroke}{rgb}{0.000000,0.000000,0.000000}%
\pgfsetstrokecolor{currentstroke}%
\pgfsetstrokeopacity{0.000000}%
\pgfsetdash{}{0pt}%
\pgfpathmoveto{\pgfqpoint{-23.809001in}{0.949719in}}%
\pgfpathlineto{\pgfqpoint{14.201077in}{0.949719in}}%
\pgfpathlineto{\pgfqpoint{14.201077in}{0.956669in}}%
\pgfpathlineto{\pgfqpoint{-23.809001in}{0.956669in}}%
\pgfpathclose%
\pgfusepath{fill}%
\end{pgfscope}%
\begin{pgfscope}%
\pgfpathrectangle{\pgfqpoint{12.211765in}{0.750000in}}{\pgfqpoint{2.188235in}{0.953684in}} %
\pgfusepath{clip}%
\pgfsetbuttcap%
\pgfsetmiterjoin%
\definecolor{currentfill}{rgb}{0.121569,0.466667,0.705882}%
\pgfsetfillcolor{currentfill}%
\pgfsetlinewidth{0.000000pt}%
\definecolor{currentstroke}{rgb}{0.000000,0.000000,0.000000}%
\pgfsetstrokecolor{currentstroke}%
\pgfsetstrokeopacity{0.000000}%
\pgfsetdash{}{0pt}%
\pgfpathmoveto{\pgfqpoint{-23.809001in}{0.958407in}}%
\pgfpathlineto{\pgfqpoint{14.137866in}{0.958407in}}%
\pgfpathlineto{\pgfqpoint{14.137866in}{0.965356in}}%
\pgfpathlineto{\pgfqpoint{-23.809001in}{0.965356in}}%
\pgfpathclose%
\pgfusepath{fill}%
\end{pgfscope}%
\begin{pgfscope}%
\pgfpathrectangle{\pgfqpoint{12.211765in}{0.750000in}}{\pgfqpoint{2.188235in}{0.953684in}} %
\pgfusepath{clip}%
\pgfsetbuttcap%
\pgfsetmiterjoin%
\definecolor{currentfill}{rgb}{0.121569,0.466667,0.705882}%
\pgfsetfillcolor{currentfill}%
\pgfsetlinewidth{0.000000pt}%
\definecolor{currentstroke}{rgb}{0.000000,0.000000,0.000000}%
\pgfsetstrokecolor{currentstroke}%
\pgfsetstrokeopacity{0.000000}%
\pgfsetdash{}{0pt}%
\pgfpathmoveto{\pgfqpoint{-23.809001in}{0.967094in}}%
\pgfpathlineto{\pgfqpoint{14.173794in}{0.967094in}}%
\pgfpathlineto{\pgfqpoint{14.173794in}{0.974044in}}%
\pgfpathlineto{\pgfqpoint{-23.809001in}{0.974044in}}%
\pgfpathclose%
\pgfusepath{fill}%
\end{pgfscope}%
\begin{pgfscope}%
\pgfpathrectangle{\pgfqpoint{12.211765in}{0.750000in}}{\pgfqpoint{2.188235in}{0.953684in}} %
\pgfusepath{clip}%
\pgfsetbuttcap%
\pgfsetmiterjoin%
\definecolor{currentfill}{rgb}{0.121569,0.466667,0.705882}%
\pgfsetfillcolor{currentfill}%
\pgfsetlinewidth{0.000000pt}%
\definecolor{currentstroke}{rgb}{0.000000,0.000000,0.000000}%
\pgfsetstrokecolor{currentstroke}%
\pgfsetstrokeopacity{0.000000}%
\pgfsetdash{}{0pt}%
\pgfpathmoveto{\pgfqpoint{-23.809001in}{0.975781in}}%
\pgfpathlineto{\pgfqpoint{14.206451in}{0.975781in}}%
\pgfpathlineto{\pgfqpoint{14.206451in}{0.982731in}}%
\pgfpathlineto{\pgfqpoint{-23.809001in}{0.982731in}}%
\pgfpathclose%
\pgfusepath{fill}%
\end{pgfscope}%
\begin{pgfscope}%
\pgfpathrectangle{\pgfqpoint{12.211765in}{0.750000in}}{\pgfqpoint{2.188235in}{0.953684in}} %
\pgfusepath{clip}%
\pgfsetbuttcap%
\pgfsetmiterjoin%
\definecolor{currentfill}{rgb}{0.121569,0.466667,0.705882}%
\pgfsetfillcolor{currentfill}%
\pgfsetlinewidth{0.000000pt}%
\definecolor{currentstroke}{rgb}{0.000000,0.000000,0.000000}%
\pgfsetstrokecolor{currentstroke}%
\pgfsetstrokeopacity{0.000000}%
\pgfsetdash{}{0pt}%
\pgfpathmoveto{\pgfqpoint{-23.809001in}{0.984468in}}%
\pgfpathlineto{\pgfqpoint{13.967044in}{0.984468in}}%
\pgfpathlineto{\pgfqpoint{13.967044in}{0.991418in}}%
\pgfpathlineto{\pgfqpoint{-23.809001in}{0.991418in}}%
\pgfpathclose%
\pgfusepath{fill}%
\end{pgfscope}%
\begin{pgfscope}%
\pgfpathrectangle{\pgfqpoint{12.211765in}{0.750000in}}{\pgfqpoint{2.188235in}{0.953684in}} %
\pgfusepath{clip}%
\pgfsetbuttcap%
\pgfsetmiterjoin%
\definecolor{currentfill}{rgb}{0.121569,0.466667,0.705882}%
\pgfsetfillcolor{currentfill}%
\pgfsetlinewidth{0.000000pt}%
\definecolor{currentstroke}{rgb}{0.000000,0.000000,0.000000}%
\pgfsetstrokecolor{currentstroke}%
\pgfsetstrokeopacity{0.000000}%
\pgfsetdash{}{0pt}%
\pgfpathmoveto{\pgfqpoint{-23.809001in}{0.993156in}}%
\pgfpathlineto{\pgfqpoint{14.147382in}{0.993156in}}%
\pgfpathlineto{\pgfqpoint{14.147382in}{1.000105in}}%
\pgfpathlineto{\pgfqpoint{-23.809001in}{1.000105in}}%
\pgfpathclose%
\pgfusepath{fill}%
\end{pgfscope}%
\begin{pgfscope}%
\pgfpathrectangle{\pgfqpoint{12.211765in}{0.750000in}}{\pgfqpoint{2.188235in}{0.953684in}} %
\pgfusepath{clip}%
\pgfsetbuttcap%
\pgfsetmiterjoin%
\definecolor{currentfill}{rgb}{0.121569,0.466667,0.705882}%
\pgfsetfillcolor{currentfill}%
\pgfsetlinewidth{0.000000pt}%
\definecolor{currentstroke}{rgb}{0.000000,0.000000,0.000000}%
\pgfsetstrokecolor{currentstroke}%
\pgfsetstrokeopacity{0.000000}%
\pgfsetdash{}{0pt}%
\pgfpathmoveto{\pgfqpoint{-23.809001in}{1.001843in}}%
\pgfpathlineto{\pgfqpoint{13.948724in}{1.001843in}}%
\pgfpathlineto{\pgfqpoint{13.948724in}{1.008793in}}%
\pgfpathlineto{\pgfqpoint{-23.809001in}{1.008793in}}%
\pgfpathclose%
\pgfusepath{fill}%
\end{pgfscope}%
\begin{pgfscope}%
\pgfpathrectangle{\pgfqpoint{12.211765in}{0.750000in}}{\pgfqpoint{2.188235in}{0.953684in}} %
\pgfusepath{clip}%
\pgfsetbuttcap%
\pgfsetmiterjoin%
\definecolor{currentfill}{rgb}{0.121569,0.466667,0.705882}%
\pgfsetfillcolor{currentfill}%
\pgfsetlinewidth{0.000000pt}%
\definecolor{currentstroke}{rgb}{0.000000,0.000000,0.000000}%
\pgfsetstrokecolor{currentstroke}%
\pgfsetstrokeopacity{0.000000}%
\pgfsetdash{}{0pt}%
\pgfpathmoveto{\pgfqpoint{-23.809001in}{1.010530in}}%
\pgfpathlineto{\pgfqpoint{14.196489in}{1.010530in}}%
\pgfpathlineto{\pgfqpoint{14.196489in}{1.017480in}}%
\pgfpathlineto{\pgfqpoint{-23.809001in}{1.017480in}}%
\pgfpathclose%
\pgfusepath{fill}%
\end{pgfscope}%
\begin{pgfscope}%
\pgfpathrectangle{\pgfqpoint{12.211765in}{0.750000in}}{\pgfqpoint{2.188235in}{0.953684in}} %
\pgfusepath{clip}%
\pgfsetbuttcap%
\pgfsetmiterjoin%
\definecolor{currentfill}{rgb}{0.121569,0.466667,0.705882}%
\pgfsetfillcolor{currentfill}%
\pgfsetlinewidth{0.000000pt}%
\definecolor{currentstroke}{rgb}{0.000000,0.000000,0.000000}%
\pgfsetstrokecolor{currentstroke}%
\pgfsetstrokeopacity{0.000000}%
\pgfsetdash{}{0pt}%
\pgfpathmoveto{\pgfqpoint{-23.809001in}{1.019217in}}%
\pgfpathlineto{\pgfqpoint{14.164308in}{1.019217in}}%
\pgfpathlineto{\pgfqpoint{14.164308in}{1.026167in}}%
\pgfpathlineto{\pgfqpoint{-23.809001in}{1.026167in}}%
\pgfpathclose%
\pgfusepath{fill}%
\end{pgfscope}%
\begin{pgfscope}%
\pgfpathrectangle{\pgfqpoint{12.211765in}{0.750000in}}{\pgfqpoint{2.188235in}{0.953684in}} %
\pgfusepath{clip}%
\pgfsetbuttcap%
\pgfsetmiterjoin%
\definecolor{currentfill}{rgb}{0.121569,0.466667,0.705882}%
\pgfsetfillcolor{currentfill}%
\pgfsetlinewidth{0.000000pt}%
\definecolor{currentstroke}{rgb}{0.000000,0.000000,0.000000}%
\pgfsetstrokecolor{currentstroke}%
\pgfsetstrokeopacity{0.000000}%
\pgfsetdash{}{0pt}%
\pgfpathmoveto{\pgfqpoint{-23.809001in}{1.027905in}}%
\pgfpathlineto{\pgfqpoint{14.185267in}{1.027905in}}%
\pgfpathlineto{\pgfqpoint{14.185267in}{1.034854in}}%
\pgfpathlineto{\pgfqpoint{-23.809001in}{1.034854in}}%
\pgfpathclose%
\pgfusepath{fill}%
\end{pgfscope}%
\begin{pgfscope}%
\pgfpathrectangle{\pgfqpoint{12.211765in}{0.750000in}}{\pgfqpoint{2.188235in}{0.953684in}} %
\pgfusepath{clip}%
\pgfsetbuttcap%
\pgfsetmiterjoin%
\definecolor{currentfill}{rgb}{0.121569,0.466667,0.705882}%
\pgfsetfillcolor{currentfill}%
\pgfsetlinewidth{0.000000pt}%
\definecolor{currentstroke}{rgb}{0.000000,0.000000,0.000000}%
\pgfsetstrokecolor{currentstroke}%
\pgfsetstrokeopacity{0.000000}%
\pgfsetdash{}{0pt}%
\pgfpathmoveto{\pgfqpoint{-23.809001in}{1.036592in}}%
\pgfpathlineto{\pgfqpoint{14.172649in}{1.036592in}}%
\pgfpathlineto{\pgfqpoint{14.172649in}{1.043542in}}%
\pgfpathlineto{\pgfqpoint{-23.809001in}{1.043542in}}%
\pgfpathclose%
\pgfusepath{fill}%
\end{pgfscope}%
\begin{pgfscope}%
\pgfpathrectangle{\pgfqpoint{12.211765in}{0.750000in}}{\pgfqpoint{2.188235in}{0.953684in}} %
\pgfusepath{clip}%
\pgfsetbuttcap%
\pgfsetmiterjoin%
\definecolor{currentfill}{rgb}{0.121569,0.466667,0.705882}%
\pgfsetfillcolor{currentfill}%
\pgfsetlinewidth{0.000000pt}%
\definecolor{currentstroke}{rgb}{0.000000,0.000000,0.000000}%
\pgfsetstrokecolor{currentstroke}%
\pgfsetstrokeopacity{0.000000}%
\pgfsetdash{}{0pt}%
\pgfpathmoveto{\pgfqpoint{-23.809001in}{1.045279in}}%
\pgfpathlineto{\pgfqpoint{14.231459in}{1.045279in}}%
\pgfpathlineto{\pgfqpoint{14.231459in}{1.052229in}}%
\pgfpathlineto{\pgfqpoint{-23.809001in}{1.052229in}}%
\pgfpathclose%
\pgfusepath{fill}%
\end{pgfscope}%
\begin{pgfscope}%
\pgfpathrectangle{\pgfqpoint{12.211765in}{0.750000in}}{\pgfqpoint{2.188235in}{0.953684in}} %
\pgfusepath{clip}%
\pgfsetbuttcap%
\pgfsetmiterjoin%
\definecolor{currentfill}{rgb}{0.121569,0.466667,0.705882}%
\pgfsetfillcolor{currentfill}%
\pgfsetlinewidth{0.000000pt}%
\definecolor{currentstroke}{rgb}{0.000000,0.000000,0.000000}%
\pgfsetstrokecolor{currentstroke}%
\pgfsetstrokeopacity{0.000000}%
\pgfsetdash{}{0pt}%
\pgfpathmoveto{\pgfqpoint{-23.809001in}{1.053966in}}%
\pgfpathlineto{\pgfqpoint{14.160075in}{1.053966in}}%
\pgfpathlineto{\pgfqpoint{14.160075in}{1.060916in}}%
\pgfpathlineto{\pgfqpoint{-23.809001in}{1.060916in}}%
\pgfpathclose%
\pgfusepath{fill}%
\end{pgfscope}%
\begin{pgfscope}%
\pgfpathrectangle{\pgfqpoint{12.211765in}{0.750000in}}{\pgfqpoint{2.188235in}{0.953684in}} %
\pgfusepath{clip}%
\pgfsetbuttcap%
\pgfsetmiterjoin%
\definecolor{currentfill}{rgb}{0.121569,0.466667,0.705882}%
\pgfsetfillcolor{currentfill}%
\pgfsetlinewidth{0.000000pt}%
\definecolor{currentstroke}{rgb}{0.000000,0.000000,0.000000}%
\pgfsetstrokecolor{currentstroke}%
\pgfsetstrokeopacity{0.000000}%
\pgfsetdash{}{0pt}%
\pgfpathmoveto{\pgfqpoint{-23.809001in}{1.062653in}}%
\pgfpathlineto{\pgfqpoint{14.127776in}{1.062653in}}%
\pgfpathlineto{\pgfqpoint{14.127776in}{1.069603in}}%
\pgfpathlineto{\pgfqpoint{-23.809001in}{1.069603in}}%
\pgfpathclose%
\pgfusepath{fill}%
\end{pgfscope}%
\begin{pgfscope}%
\pgfpathrectangle{\pgfqpoint{12.211765in}{0.750000in}}{\pgfqpoint{2.188235in}{0.953684in}} %
\pgfusepath{clip}%
\pgfsetbuttcap%
\pgfsetmiterjoin%
\definecolor{currentfill}{rgb}{0.121569,0.466667,0.705882}%
\pgfsetfillcolor{currentfill}%
\pgfsetlinewidth{0.000000pt}%
\definecolor{currentstroke}{rgb}{0.000000,0.000000,0.000000}%
\pgfsetstrokecolor{currentstroke}%
\pgfsetstrokeopacity{0.000000}%
\pgfsetdash{}{0pt}%
\pgfpathmoveto{\pgfqpoint{-23.809001in}{1.071341in}}%
\pgfpathlineto{\pgfqpoint{14.184633in}{1.071341in}}%
\pgfpathlineto{\pgfqpoint{14.184633in}{1.078290in}}%
\pgfpathlineto{\pgfqpoint{-23.809001in}{1.078290in}}%
\pgfpathclose%
\pgfusepath{fill}%
\end{pgfscope}%
\begin{pgfscope}%
\pgfpathrectangle{\pgfqpoint{12.211765in}{0.750000in}}{\pgfqpoint{2.188235in}{0.953684in}} %
\pgfusepath{clip}%
\pgfsetbuttcap%
\pgfsetmiterjoin%
\definecolor{currentfill}{rgb}{0.121569,0.466667,0.705882}%
\pgfsetfillcolor{currentfill}%
\pgfsetlinewidth{0.000000pt}%
\definecolor{currentstroke}{rgb}{0.000000,0.000000,0.000000}%
\pgfsetstrokecolor{currentstroke}%
\pgfsetstrokeopacity{0.000000}%
\pgfsetdash{}{0pt}%
\pgfpathmoveto{\pgfqpoint{-23.809001in}{1.080028in}}%
\pgfpathlineto{\pgfqpoint{14.164354in}{1.080028in}}%
\pgfpathlineto{\pgfqpoint{14.164354in}{1.086978in}}%
\pgfpathlineto{\pgfqpoint{-23.809001in}{1.086978in}}%
\pgfpathclose%
\pgfusepath{fill}%
\end{pgfscope}%
\begin{pgfscope}%
\pgfpathrectangle{\pgfqpoint{12.211765in}{0.750000in}}{\pgfqpoint{2.188235in}{0.953684in}} %
\pgfusepath{clip}%
\pgfsetbuttcap%
\pgfsetmiterjoin%
\definecolor{currentfill}{rgb}{0.121569,0.466667,0.705882}%
\pgfsetfillcolor{currentfill}%
\pgfsetlinewidth{0.000000pt}%
\definecolor{currentstroke}{rgb}{0.000000,0.000000,0.000000}%
\pgfsetstrokecolor{currentstroke}%
\pgfsetstrokeopacity{0.000000}%
\pgfsetdash{}{0pt}%
\pgfpathmoveto{\pgfqpoint{-23.809001in}{1.088715in}}%
\pgfpathlineto{\pgfqpoint{14.249809in}{1.088715in}}%
\pgfpathlineto{\pgfqpoint{14.249809in}{1.095665in}}%
\pgfpathlineto{\pgfqpoint{-23.809001in}{1.095665in}}%
\pgfpathclose%
\pgfusepath{fill}%
\end{pgfscope}%
\begin{pgfscope}%
\pgfpathrectangle{\pgfqpoint{12.211765in}{0.750000in}}{\pgfqpoint{2.188235in}{0.953684in}} %
\pgfusepath{clip}%
\pgfsetbuttcap%
\pgfsetmiterjoin%
\definecolor{currentfill}{rgb}{0.121569,0.466667,0.705882}%
\pgfsetfillcolor{currentfill}%
\pgfsetlinewidth{0.000000pt}%
\definecolor{currentstroke}{rgb}{0.000000,0.000000,0.000000}%
\pgfsetstrokecolor{currentstroke}%
\pgfsetstrokeopacity{0.000000}%
\pgfsetdash{}{0pt}%
\pgfpathmoveto{\pgfqpoint{-23.809001in}{1.097402in}}%
\pgfpathlineto{\pgfqpoint{14.219142in}{1.097402in}}%
\pgfpathlineto{\pgfqpoint{14.219142in}{1.104352in}}%
\pgfpathlineto{\pgfqpoint{-23.809001in}{1.104352in}}%
\pgfpathclose%
\pgfusepath{fill}%
\end{pgfscope}%
\begin{pgfscope}%
\pgfpathrectangle{\pgfqpoint{12.211765in}{0.750000in}}{\pgfqpoint{2.188235in}{0.953684in}} %
\pgfusepath{clip}%
\pgfsetbuttcap%
\pgfsetmiterjoin%
\definecolor{currentfill}{rgb}{0.121569,0.466667,0.705882}%
\pgfsetfillcolor{currentfill}%
\pgfsetlinewidth{0.000000pt}%
\definecolor{currentstroke}{rgb}{0.000000,0.000000,0.000000}%
\pgfsetstrokecolor{currentstroke}%
\pgfsetstrokeopacity{0.000000}%
\pgfsetdash{}{0pt}%
\pgfpathmoveto{\pgfqpoint{-23.809001in}{1.106090in}}%
\pgfpathlineto{\pgfqpoint{14.221619in}{1.106090in}}%
\pgfpathlineto{\pgfqpoint{14.221619in}{1.113039in}}%
\pgfpathlineto{\pgfqpoint{-23.809001in}{1.113039in}}%
\pgfpathclose%
\pgfusepath{fill}%
\end{pgfscope}%
\begin{pgfscope}%
\pgfpathrectangle{\pgfqpoint{12.211765in}{0.750000in}}{\pgfqpoint{2.188235in}{0.953684in}} %
\pgfusepath{clip}%
\pgfsetbuttcap%
\pgfsetmiterjoin%
\definecolor{currentfill}{rgb}{0.121569,0.466667,0.705882}%
\pgfsetfillcolor{currentfill}%
\pgfsetlinewidth{0.000000pt}%
\definecolor{currentstroke}{rgb}{0.000000,0.000000,0.000000}%
\pgfsetstrokecolor{currentstroke}%
\pgfsetstrokeopacity{0.000000}%
\pgfsetdash{}{0pt}%
\pgfpathmoveto{\pgfqpoint{-23.809001in}{1.114777in}}%
\pgfpathlineto{\pgfqpoint{14.243563in}{1.114777in}}%
\pgfpathlineto{\pgfqpoint{14.243563in}{1.121727in}}%
\pgfpathlineto{\pgfqpoint{-23.809001in}{1.121727in}}%
\pgfpathclose%
\pgfusepath{fill}%
\end{pgfscope}%
\begin{pgfscope}%
\pgfpathrectangle{\pgfqpoint{12.211765in}{0.750000in}}{\pgfqpoint{2.188235in}{0.953684in}} %
\pgfusepath{clip}%
\pgfsetbuttcap%
\pgfsetmiterjoin%
\definecolor{currentfill}{rgb}{0.121569,0.466667,0.705882}%
\pgfsetfillcolor{currentfill}%
\pgfsetlinewidth{0.000000pt}%
\definecolor{currentstroke}{rgb}{0.000000,0.000000,0.000000}%
\pgfsetstrokecolor{currentstroke}%
\pgfsetstrokeopacity{0.000000}%
\pgfsetdash{}{0pt}%
\pgfpathmoveto{\pgfqpoint{-23.809001in}{1.123464in}}%
\pgfpathlineto{\pgfqpoint{14.219412in}{1.123464in}}%
\pgfpathlineto{\pgfqpoint{14.219412in}{1.130414in}}%
\pgfpathlineto{\pgfqpoint{-23.809001in}{1.130414in}}%
\pgfpathclose%
\pgfusepath{fill}%
\end{pgfscope}%
\begin{pgfscope}%
\pgfpathrectangle{\pgfqpoint{12.211765in}{0.750000in}}{\pgfqpoint{2.188235in}{0.953684in}} %
\pgfusepath{clip}%
\pgfsetbuttcap%
\pgfsetmiterjoin%
\definecolor{currentfill}{rgb}{0.121569,0.466667,0.705882}%
\pgfsetfillcolor{currentfill}%
\pgfsetlinewidth{0.000000pt}%
\definecolor{currentstroke}{rgb}{0.000000,0.000000,0.000000}%
\pgfsetstrokecolor{currentstroke}%
\pgfsetstrokeopacity{0.000000}%
\pgfsetdash{}{0pt}%
\pgfpathmoveto{\pgfqpoint{-23.809001in}{1.132151in}}%
\pgfpathlineto{\pgfqpoint{14.192659in}{1.132151in}}%
\pgfpathlineto{\pgfqpoint{14.192659in}{1.139101in}}%
\pgfpathlineto{\pgfqpoint{-23.809001in}{1.139101in}}%
\pgfpathclose%
\pgfusepath{fill}%
\end{pgfscope}%
\begin{pgfscope}%
\pgfpathrectangle{\pgfqpoint{12.211765in}{0.750000in}}{\pgfqpoint{2.188235in}{0.953684in}} %
\pgfusepath{clip}%
\pgfsetbuttcap%
\pgfsetmiterjoin%
\definecolor{currentfill}{rgb}{0.121569,0.466667,0.705882}%
\pgfsetfillcolor{currentfill}%
\pgfsetlinewidth{0.000000pt}%
\definecolor{currentstroke}{rgb}{0.000000,0.000000,0.000000}%
\pgfsetstrokecolor{currentstroke}%
\pgfsetstrokeopacity{0.000000}%
\pgfsetdash{}{0pt}%
\pgfpathmoveto{\pgfqpoint{-23.809001in}{1.140839in}}%
\pgfpathlineto{\pgfqpoint{14.246906in}{1.140839in}}%
\pgfpathlineto{\pgfqpoint{14.246906in}{1.147788in}}%
\pgfpathlineto{\pgfqpoint{-23.809001in}{1.147788in}}%
\pgfpathclose%
\pgfusepath{fill}%
\end{pgfscope}%
\begin{pgfscope}%
\pgfpathrectangle{\pgfqpoint{12.211765in}{0.750000in}}{\pgfqpoint{2.188235in}{0.953684in}} %
\pgfusepath{clip}%
\pgfsetbuttcap%
\pgfsetmiterjoin%
\definecolor{currentfill}{rgb}{0.121569,0.466667,0.705882}%
\pgfsetfillcolor{currentfill}%
\pgfsetlinewidth{0.000000pt}%
\definecolor{currentstroke}{rgb}{0.000000,0.000000,0.000000}%
\pgfsetstrokecolor{currentstroke}%
\pgfsetstrokeopacity{0.000000}%
\pgfsetdash{}{0pt}%
\pgfpathmoveto{\pgfqpoint{-23.809001in}{1.149526in}}%
\pgfpathlineto{\pgfqpoint{14.181849in}{1.149526in}}%
\pgfpathlineto{\pgfqpoint{14.181849in}{1.156476in}}%
\pgfpathlineto{\pgfqpoint{-23.809001in}{1.156476in}}%
\pgfpathclose%
\pgfusepath{fill}%
\end{pgfscope}%
\begin{pgfscope}%
\pgfpathrectangle{\pgfqpoint{12.211765in}{0.750000in}}{\pgfqpoint{2.188235in}{0.953684in}} %
\pgfusepath{clip}%
\pgfsetbuttcap%
\pgfsetmiterjoin%
\definecolor{currentfill}{rgb}{0.121569,0.466667,0.705882}%
\pgfsetfillcolor{currentfill}%
\pgfsetlinewidth{0.000000pt}%
\definecolor{currentstroke}{rgb}{0.000000,0.000000,0.000000}%
\pgfsetstrokecolor{currentstroke}%
\pgfsetstrokeopacity{0.000000}%
\pgfsetdash{}{0pt}%
\pgfpathmoveto{\pgfqpoint{-23.809001in}{1.158213in}}%
\pgfpathlineto{\pgfqpoint{13.963876in}{1.158213in}}%
\pgfpathlineto{\pgfqpoint{13.963876in}{1.165163in}}%
\pgfpathlineto{\pgfqpoint{-23.809001in}{1.165163in}}%
\pgfpathclose%
\pgfusepath{fill}%
\end{pgfscope}%
\begin{pgfscope}%
\pgfpathrectangle{\pgfqpoint{12.211765in}{0.750000in}}{\pgfqpoint{2.188235in}{0.953684in}} %
\pgfusepath{clip}%
\pgfsetbuttcap%
\pgfsetmiterjoin%
\definecolor{currentfill}{rgb}{0.121569,0.466667,0.705882}%
\pgfsetfillcolor{currentfill}%
\pgfsetlinewidth{0.000000pt}%
\definecolor{currentstroke}{rgb}{0.000000,0.000000,0.000000}%
\pgfsetstrokecolor{currentstroke}%
\pgfsetstrokeopacity{0.000000}%
\pgfsetdash{}{0pt}%
\pgfpathmoveto{\pgfqpoint{-23.809001in}{1.166900in}}%
\pgfpathlineto{\pgfqpoint{14.194588in}{1.166900in}}%
\pgfpathlineto{\pgfqpoint{14.194588in}{1.173850in}}%
\pgfpathlineto{\pgfqpoint{-23.809001in}{1.173850in}}%
\pgfpathclose%
\pgfusepath{fill}%
\end{pgfscope}%
\begin{pgfscope}%
\pgfpathrectangle{\pgfqpoint{12.211765in}{0.750000in}}{\pgfqpoint{2.188235in}{0.953684in}} %
\pgfusepath{clip}%
\pgfsetbuttcap%
\pgfsetmiterjoin%
\definecolor{currentfill}{rgb}{0.121569,0.466667,0.705882}%
\pgfsetfillcolor{currentfill}%
\pgfsetlinewidth{0.000000pt}%
\definecolor{currentstroke}{rgb}{0.000000,0.000000,0.000000}%
\pgfsetstrokecolor{currentstroke}%
\pgfsetstrokeopacity{0.000000}%
\pgfsetdash{}{0pt}%
\pgfpathmoveto{\pgfqpoint{-23.809001in}{1.175587in}}%
\pgfpathlineto{\pgfqpoint{14.280396in}{1.175587in}}%
\pgfpathlineto{\pgfqpoint{14.280396in}{1.182537in}}%
\pgfpathlineto{\pgfqpoint{-23.809001in}{1.182537in}}%
\pgfpathclose%
\pgfusepath{fill}%
\end{pgfscope}%
\begin{pgfscope}%
\pgfpathrectangle{\pgfqpoint{12.211765in}{0.750000in}}{\pgfqpoint{2.188235in}{0.953684in}} %
\pgfusepath{clip}%
\pgfsetbuttcap%
\pgfsetmiterjoin%
\definecolor{currentfill}{rgb}{0.121569,0.466667,0.705882}%
\pgfsetfillcolor{currentfill}%
\pgfsetlinewidth{0.000000pt}%
\definecolor{currentstroke}{rgb}{0.000000,0.000000,0.000000}%
\pgfsetstrokecolor{currentstroke}%
\pgfsetstrokeopacity{0.000000}%
\pgfsetdash{}{0pt}%
\pgfpathmoveto{\pgfqpoint{-23.809001in}{1.184275in}}%
\pgfpathlineto{\pgfqpoint{14.235332in}{1.184275in}}%
\pgfpathlineto{\pgfqpoint{14.235332in}{1.191224in}}%
\pgfpathlineto{\pgfqpoint{-23.809001in}{1.191224in}}%
\pgfpathclose%
\pgfusepath{fill}%
\end{pgfscope}%
\begin{pgfscope}%
\pgfpathrectangle{\pgfqpoint{12.211765in}{0.750000in}}{\pgfqpoint{2.188235in}{0.953684in}} %
\pgfusepath{clip}%
\pgfsetbuttcap%
\pgfsetmiterjoin%
\definecolor{currentfill}{rgb}{0.121569,0.466667,0.705882}%
\pgfsetfillcolor{currentfill}%
\pgfsetlinewidth{0.000000pt}%
\definecolor{currentstroke}{rgb}{0.000000,0.000000,0.000000}%
\pgfsetstrokecolor{currentstroke}%
\pgfsetstrokeopacity{0.000000}%
\pgfsetdash{}{0pt}%
\pgfpathmoveto{\pgfqpoint{-23.809001in}{1.192962in}}%
\pgfpathlineto{\pgfqpoint{14.224974in}{1.192962in}}%
\pgfpathlineto{\pgfqpoint{14.224974in}{1.199912in}}%
\pgfpathlineto{\pgfqpoint{-23.809001in}{1.199912in}}%
\pgfpathclose%
\pgfusepath{fill}%
\end{pgfscope}%
\begin{pgfscope}%
\pgfpathrectangle{\pgfqpoint{12.211765in}{0.750000in}}{\pgfqpoint{2.188235in}{0.953684in}} %
\pgfusepath{clip}%
\pgfsetbuttcap%
\pgfsetmiterjoin%
\definecolor{currentfill}{rgb}{0.121569,0.466667,0.705882}%
\pgfsetfillcolor{currentfill}%
\pgfsetlinewidth{0.000000pt}%
\definecolor{currentstroke}{rgb}{0.000000,0.000000,0.000000}%
\pgfsetstrokecolor{currentstroke}%
\pgfsetstrokeopacity{0.000000}%
\pgfsetdash{}{0pt}%
\pgfpathmoveto{\pgfqpoint{-23.809001in}{1.201649in}}%
\pgfpathlineto{\pgfqpoint{14.258269in}{1.201649in}}%
\pgfpathlineto{\pgfqpoint{14.258269in}{1.208599in}}%
\pgfpathlineto{\pgfqpoint{-23.809001in}{1.208599in}}%
\pgfpathclose%
\pgfusepath{fill}%
\end{pgfscope}%
\begin{pgfscope}%
\pgfpathrectangle{\pgfqpoint{12.211765in}{0.750000in}}{\pgfqpoint{2.188235in}{0.953684in}} %
\pgfusepath{clip}%
\pgfsetbuttcap%
\pgfsetmiterjoin%
\definecolor{currentfill}{rgb}{0.121569,0.466667,0.705882}%
\pgfsetfillcolor{currentfill}%
\pgfsetlinewidth{0.000000pt}%
\definecolor{currentstroke}{rgb}{0.000000,0.000000,0.000000}%
\pgfsetstrokecolor{currentstroke}%
\pgfsetstrokeopacity{0.000000}%
\pgfsetdash{}{0pt}%
\pgfpathmoveto{\pgfqpoint{-23.809001in}{1.210336in}}%
\pgfpathlineto{\pgfqpoint{14.149229in}{1.210336in}}%
\pgfpathlineto{\pgfqpoint{14.149229in}{1.217286in}}%
\pgfpathlineto{\pgfqpoint{-23.809001in}{1.217286in}}%
\pgfpathclose%
\pgfusepath{fill}%
\end{pgfscope}%
\begin{pgfscope}%
\pgfpathrectangle{\pgfqpoint{12.211765in}{0.750000in}}{\pgfqpoint{2.188235in}{0.953684in}} %
\pgfusepath{clip}%
\pgfsetbuttcap%
\pgfsetmiterjoin%
\definecolor{currentfill}{rgb}{0.121569,0.466667,0.705882}%
\pgfsetfillcolor{currentfill}%
\pgfsetlinewidth{0.000000pt}%
\definecolor{currentstroke}{rgb}{0.000000,0.000000,0.000000}%
\pgfsetstrokecolor{currentstroke}%
\pgfsetstrokeopacity{0.000000}%
\pgfsetdash{}{0pt}%
\pgfpathmoveto{\pgfqpoint{-23.809001in}{1.219024in}}%
\pgfpathlineto{\pgfqpoint{14.231830in}{1.219024in}}%
\pgfpathlineto{\pgfqpoint{14.231830in}{1.225973in}}%
\pgfpathlineto{\pgfqpoint{-23.809001in}{1.225973in}}%
\pgfpathclose%
\pgfusepath{fill}%
\end{pgfscope}%
\begin{pgfscope}%
\pgfpathrectangle{\pgfqpoint{12.211765in}{0.750000in}}{\pgfqpoint{2.188235in}{0.953684in}} %
\pgfusepath{clip}%
\pgfsetbuttcap%
\pgfsetmiterjoin%
\definecolor{currentfill}{rgb}{0.121569,0.466667,0.705882}%
\pgfsetfillcolor{currentfill}%
\pgfsetlinewidth{0.000000pt}%
\definecolor{currentstroke}{rgb}{0.000000,0.000000,0.000000}%
\pgfsetstrokecolor{currentstroke}%
\pgfsetstrokeopacity{0.000000}%
\pgfsetdash{}{0pt}%
\pgfpathmoveto{\pgfqpoint{-23.809001in}{1.227711in}}%
\pgfpathlineto{\pgfqpoint{14.291169in}{1.227711in}}%
\pgfpathlineto{\pgfqpoint{14.291169in}{1.234661in}}%
\pgfpathlineto{\pgfqpoint{-23.809001in}{1.234661in}}%
\pgfpathclose%
\pgfusepath{fill}%
\end{pgfscope}%
\begin{pgfscope}%
\pgfpathrectangle{\pgfqpoint{12.211765in}{0.750000in}}{\pgfqpoint{2.188235in}{0.953684in}} %
\pgfusepath{clip}%
\pgfsetbuttcap%
\pgfsetmiterjoin%
\definecolor{currentfill}{rgb}{0.121569,0.466667,0.705882}%
\pgfsetfillcolor{currentfill}%
\pgfsetlinewidth{0.000000pt}%
\definecolor{currentstroke}{rgb}{0.000000,0.000000,0.000000}%
\pgfsetstrokecolor{currentstroke}%
\pgfsetstrokeopacity{0.000000}%
\pgfsetdash{}{0pt}%
\pgfpathmoveto{\pgfqpoint{-23.809001in}{1.236398in}}%
\pgfpathlineto{\pgfqpoint{14.233857in}{1.236398in}}%
\pgfpathlineto{\pgfqpoint{14.233857in}{1.243348in}}%
\pgfpathlineto{\pgfqpoint{-23.809001in}{1.243348in}}%
\pgfpathclose%
\pgfusepath{fill}%
\end{pgfscope}%
\begin{pgfscope}%
\pgfpathrectangle{\pgfqpoint{12.211765in}{0.750000in}}{\pgfqpoint{2.188235in}{0.953684in}} %
\pgfusepath{clip}%
\pgfsetbuttcap%
\pgfsetmiterjoin%
\definecolor{currentfill}{rgb}{0.121569,0.466667,0.705882}%
\pgfsetfillcolor{currentfill}%
\pgfsetlinewidth{0.000000pt}%
\definecolor{currentstroke}{rgb}{0.000000,0.000000,0.000000}%
\pgfsetstrokecolor{currentstroke}%
\pgfsetstrokeopacity{0.000000}%
\pgfsetdash{}{0pt}%
\pgfpathmoveto{\pgfqpoint{-23.809001in}{1.245085in}}%
\pgfpathlineto{\pgfqpoint{14.215089in}{1.245085in}}%
\pgfpathlineto{\pgfqpoint{14.215089in}{1.252035in}}%
\pgfpathlineto{\pgfqpoint{-23.809001in}{1.252035in}}%
\pgfpathclose%
\pgfusepath{fill}%
\end{pgfscope}%
\begin{pgfscope}%
\pgfpathrectangle{\pgfqpoint{12.211765in}{0.750000in}}{\pgfqpoint{2.188235in}{0.953684in}} %
\pgfusepath{clip}%
\pgfsetbuttcap%
\pgfsetmiterjoin%
\definecolor{currentfill}{rgb}{0.121569,0.466667,0.705882}%
\pgfsetfillcolor{currentfill}%
\pgfsetlinewidth{0.000000pt}%
\definecolor{currentstroke}{rgb}{0.000000,0.000000,0.000000}%
\pgfsetstrokecolor{currentstroke}%
\pgfsetstrokeopacity{0.000000}%
\pgfsetdash{}{0pt}%
\pgfpathmoveto{\pgfqpoint{-23.809001in}{1.253773in}}%
\pgfpathlineto{\pgfqpoint{14.214914in}{1.253773in}}%
\pgfpathlineto{\pgfqpoint{14.214914in}{1.260722in}}%
\pgfpathlineto{\pgfqpoint{-23.809001in}{1.260722in}}%
\pgfpathclose%
\pgfusepath{fill}%
\end{pgfscope}%
\begin{pgfscope}%
\pgfpathrectangle{\pgfqpoint{12.211765in}{0.750000in}}{\pgfqpoint{2.188235in}{0.953684in}} %
\pgfusepath{clip}%
\pgfsetbuttcap%
\pgfsetmiterjoin%
\definecolor{currentfill}{rgb}{0.121569,0.466667,0.705882}%
\pgfsetfillcolor{currentfill}%
\pgfsetlinewidth{0.000000pt}%
\definecolor{currentstroke}{rgb}{0.000000,0.000000,0.000000}%
\pgfsetstrokecolor{currentstroke}%
\pgfsetstrokeopacity{0.000000}%
\pgfsetdash{}{0pt}%
\pgfpathmoveto{\pgfqpoint{-23.809001in}{1.262460in}}%
\pgfpathlineto{\pgfqpoint{14.209437in}{1.262460in}}%
\pgfpathlineto{\pgfqpoint{14.209437in}{1.269410in}}%
\pgfpathlineto{\pgfqpoint{-23.809001in}{1.269410in}}%
\pgfpathclose%
\pgfusepath{fill}%
\end{pgfscope}%
\begin{pgfscope}%
\pgfpathrectangle{\pgfqpoint{12.211765in}{0.750000in}}{\pgfqpoint{2.188235in}{0.953684in}} %
\pgfusepath{clip}%
\pgfsetbuttcap%
\pgfsetmiterjoin%
\definecolor{currentfill}{rgb}{0.121569,0.466667,0.705882}%
\pgfsetfillcolor{currentfill}%
\pgfsetlinewidth{0.000000pt}%
\definecolor{currentstroke}{rgb}{0.000000,0.000000,0.000000}%
\pgfsetstrokecolor{currentstroke}%
\pgfsetstrokeopacity{0.000000}%
\pgfsetdash{}{0pt}%
\pgfpathmoveto{\pgfqpoint{-23.809001in}{1.271147in}}%
\pgfpathlineto{\pgfqpoint{14.220333in}{1.271147in}}%
\pgfpathlineto{\pgfqpoint{14.220333in}{1.278097in}}%
\pgfpathlineto{\pgfqpoint{-23.809001in}{1.278097in}}%
\pgfpathclose%
\pgfusepath{fill}%
\end{pgfscope}%
\begin{pgfscope}%
\pgfpathrectangle{\pgfqpoint{12.211765in}{0.750000in}}{\pgfqpoint{2.188235in}{0.953684in}} %
\pgfusepath{clip}%
\pgfsetbuttcap%
\pgfsetmiterjoin%
\definecolor{currentfill}{rgb}{0.121569,0.466667,0.705882}%
\pgfsetfillcolor{currentfill}%
\pgfsetlinewidth{0.000000pt}%
\definecolor{currentstroke}{rgb}{0.000000,0.000000,0.000000}%
\pgfsetstrokecolor{currentstroke}%
\pgfsetstrokeopacity{0.000000}%
\pgfsetdash{}{0pt}%
\pgfpathmoveto{\pgfqpoint{-23.809001in}{1.279834in}}%
\pgfpathlineto{\pgfqpoint{14.161788in}{1.279834in}}%
\pgfpathlineto{\pgfqpoint{14.161788in}{1.286784in}}%
\pgfpathlineto{\pgfqpoint{-23.809001in}{1.286784in}}%
\pgfpathclose%
\pgfusepath{fill}%
\end{pgfscope}%
\begin{pgfscope}%
\pgfpathrectangle{\pgfqpoint{12.211765in}{0.750000in}}{\pgfqpoint{2.188235in}{0.953684in}} %
\pgfusepath{clip}%
\pgfsetbuttcap%
\pgfsetmiterjoin%
\definecolor{currentfill}{rgb}{0.121569,0.466667,0.705882}%
\pgfsetfillcolor{currentfill}%
\pgfsetlinewidth{0.000000pt}%
\definecolor{currentstroke}{rgb}{0.000000,0.000000,0.000000}%
\pgfsetstrokecolor{currentstroke}%
\pgfsetstrokeopacity{0.000000}%
\pgfsetdash{}{0pt}%
\pgfpathmoveto{\pgfqpoint{-23.809001in}{1.288521in}}%
\pgfpathlineto{\pgfqpoint{14.189110in}{1.288521in}}%
\pgfpathlineto{\pgfqpoint{14.189110in}{1.295471in}}%
\pgfpathlineto{\pgfqpoint{-23.809001in}{1.295471in}}%
\pgfpathclose%
\pgfusepath{fill}%
\end{pgfscope}%
\begin{pgfscope}%
\pgfpathrectangle{\pgfqpoint{12.211765in}{0.750000in}}{\pgfqpoint{2.188235in}{0.953684in}} %
\pgfusepath{clip}%
\pgfsetbuttcap%
\pgfsetmiterjoin%
\definecolor{currentfill}{rgb}{0.121569,0.466667,0.705882}%
\pgfsetfillcolor{currentfill}%
\pgfsetlinewidth{0.000000pt}%
\definecolor{currentstroke}{rgb}{0.000000,0.000000,0.000000}%
\pgfsetstrokecolor{currentstroke}%
\pgfsetstrokeopacity{0.000000}%
\pgfsetdash{}{0pt}%
\pgfpathmoveto{\pgfqpoint{-23.809001in}{1.297209in}}%
\pgfpathlineto{\pgfqpoint{14.261859in}{1.297209in}}%
\pgfpathlineto{\pgfqpoint{14.261859in}{1.304158in}}%
\pgfpathlineto{\pgfqpoint{-23.809001in}{1.304158in}}%
\pgfpathclose%
\pgfusepath{fill}%
\end{pgfscope}%
\begin{pgfscope}%
\pgfpathrectangle{\pgfqpoint{12.211765in}{0.750000in}}{\pgfqpoint{2.188235in}{0.953684in}} %
\pgfusepath{clip}%
\pgfsetbuttcap%
\pgfsetmiterjoin%
\definecolor{currentfill}{rgb}{0.121569,0.466667,0.705882}%
\pgfsetfillcolor{currentfill}%
\pgfsetlinewidth{0.000000pt}%
\definecolor{currentstroke}{rgb}{0.000000,0.000000,0.000000}%
\pgfsetstrokecolor{currentstroke}%
\pgfsetstrokeopacity{0.000000}%
\pgfsetdash{}{0pt}%
\pgfpathmoveto{\pgfqpoint{-23.809001in}{1.305896in}}%
\pgfpathlineto{\pgfqpoint{14.181698in}{1.305896in}}%
\pgfpathlineto{\pgfqpoint{14.181698in}{1.312846in}}%
\pgfpathlineto{\pgfqpoint{-23.809001in}{1.312846in}}%
\pgfpathclose%
\pgfusepath{fill}%
\end{pgfscope}%
\begin{pgfscope}%
\pgfpathrectangle{\pgfqpoint{12.211765in}{0.750000in}}{\pgfqpoint{2.188235in}{0.953684in}} %
\pgfusepath{clip}%
\pgfsetbuttcap%
\pgfsetmiterjoin%
\definecolor{currentfill}{rgb}{0.121569,0.466667,0.705882}%
\pgfsetfillcolor{currentfill}%
\pgfsetlinewidth{0.000000pt}%
\definecolor{currentstroke}{rgb}{0.000000,0.000000,0.000000}%
\pgfsetstrokecolor{currentstroke}%
\pgfsetstrokeopacity{0.000000}%
\pgfsetdash{}{0pt}%
\pgfpathmoveto{\pgfqpoint{-23.809001in}{1.314583in}}%
\pgfpathlineto{\pgfqpoint{14.214578in}{1.314583in}}%
\pgfpathlineto{\pgfqpoint{14.214578in}{1.321533in}}%
\pgfpathlineto{\pgfqpoint{-23.809001in}{1.321533in}}%
\pgfpathclose%
\pgfusepath{fill}%
\end{pgfscope}%
\begin{pgfscope}%
\pgfpathrectangle{\pgfqpoint{12.211765in}{0.750000in}}{\pgfqpoint{2.188235in}{0.953684in}} %
\pgfusepath{clip}%
\pgfsetbuttcap%
\pgfsetmiterjoin%
\definecolor{currentfill}{rgb}{0.121569,0.466667,0.705882}%
\pgfsetfillcolor{currentfill}%
\pgfsetlinewidth{0.000000pt}%
\definecolor{currentstroke}{rgb}{0.000000,0.000000,0.000000}%
\pgfsetstrokecolor{currentstroke}%
\pgfsetstrokeopacity{0.000000}%
\pgfsetdash{}{0pt}%
\pgfpathmoveto{\pgfqpoint{-23.809001in}{1.323270in}}%
\pgfpathlineto{\pgfqpoint{14.222651in}{1.323270in}}%
\pgfpathlineto{\pgfqpoint{14.222651in}{1.330220in}}%
\pgfpathlineto{\pgfqpoint{-23.809001in}{1.330220in}}%
\pgfpathclose%
\pgfusepath{fill}%
\end{pgfscope}%
\begin{pgfscope}%
\pgfpathrectangle{\pgfqpoint{12.211765in}{0.750000in}}{\pgfqpoint{2.188235in}{0.953684in}} %
\pgfusepath{clip}%
\pgfsetbuttcap%
\pgfsetmiterjoin%
\definecolor{currentfill}{rgb}{0.121569,0.466667,0.705882}%
\pgfsetfillcolor{currentfill}%
\pgfsetlinewidth{0.000000pt}%
\definecolor{currentstroke}{rgb}{0.000000,0.000000,0.000000}%
\pgfsetstrokecolor{currentstroke}%
\pgfsetstrokeopacity{0.000000}%
\pgfsetdash{}{0pt}%
\pgfpathmoveto{\pgfqpoint{-23.809001in}{1.331958in}}%
\pgfpathlineto{\pgfqpoint{14.232710in}{1.331958in}}%
\pgfpathlineto{\pgfqpoint{14.232710in}{1.338907in}}%
\pgfpathlineto{\pgfqpoint{-23.809001in}{1.338907in}}%
\pgfpathclose%
\pgfusepath{fill}%
\end{pgfscope}%
\begin{pgfscope}%
\pgfpathrectangle{\pgfqpoint{12.211765in}{0.750000in}}{\pgfqpoint{2.188235in}{0.953684in}} %
\pgfusepath{clip}%
\pgfsetbuttcap%
\pgfsetmiterjoin%
\definecolor{currentfill}{rgb}{0.121569,0.466667,0.705882}%
\pgfsetfillcolor{currentfill}%
\pgfsetlinewidth{0.000000pt}%
\definecolor{currentstroke}{rgb}{0.000000,0.000000,0.000000}%
\pgfsetstrokecolor{currentstroke}%
\pgfsetstrokeopacity{0.000000}%
\pgfsetdash{}{0pt}%
\pgfpathmoveto{\pgfqpoint{-23.809001in}{1.340645in}}%
\pgfpathlineto{\pgfqpoint{14.151295in}{1.340645in}}%
\pgfpathlineto{\pgfqpoint{14.151295in}{1.347595in}}%
\pgfpathlineto{\pgfqpoint{-23.809001in}{1.347595in}}%
\pgfpathclose%
\pgfusepath{fill}%
\end{pgfscope}%
\begin{pgfscope}%
\pgfpathrectangle{\pgfqpoint{12.211765in}{0.750000in}}{\pgfqpoint{2.188235in}{0.953684in}} %
\pgfusepath{clip}%
\pgfsetbuttcap%
\pgfsetmiterjoin%
\definecolor{currentfill}{rgb}{0.121569,0.466667,0.705882}%
\pgfsetfillcolor{currentfill}%
\pgfsetlinewidth{0.000000pt}%
\definecolor{currentstroke}{rgb}{0.000000,0.000000,0.000000}%
\pgfsetstrokecolor{currentstroke}%
\pgfsetstrokeopacity{0.000000}%
\pgfsetdash{}{0pt}%
\pgfpathmoveto{\pgfqpoint{-23.809001in}{1.349332in}}%
\pgfpathlineto{\pgfqpoint{14.282301in}{1.349332in}}%
\pgfpathlineto{\pgfqpoint{14.282301in}{1.356282in}}%
\pgfpathlineto{\pgfqpoint{-23.809001in}{1.356282in}}%
\pgfpathclose%
\pgfusepath{fill}%
\end{pgfscope}%
\begin{pgfscope}%
\pgfpathrectangle{\pgfqpoint{12.211765in}{0.750000in}}{\pgfqpoint{2.188235in}{0.953684in}} %
\pgfusepath{clip}%
\pgfsetbuttcap%
\pgfsetmiterjoin%
\definecolor{currentfill}{rgb}{0.121569,0.466667,0.705882}%
\pgfsetfillcolor{currentfill}%
\pgfsetlinewidth{0.000000pt}%
\definecolor{currentstroke}{rgb}{0.000000,0.000000,0.000000}%
\pgfsetstrokecolor{currentstroke}%
\pgfsetstrokeopacity{0.000000}%
\pgfsetdash{}{0pt}%
\pgfpathmoveto{\pgfqpoint{-23.809001in}{1.358019in}}%
\pgfpathlineto{\pgfqpoint{14.277857in}{1.358019in}}%
\pgfpathlineto{\pgfqpoint{14.277857in}{1.364969in}}%
\pgfpathlineto{\pgfqpoint{-23.809001in}{1.364969in}}%
\pgfpathclose%
\pgfusepath{fill}%
\end{pgfscope}%
\begin{pgfscope}%
\pgfpathrectangle{\pgfqpoint{12.211765in}{0.750000in}}{\pgfqpoint{2.188235in}{0.953684in}} %
\pgfusepath{clip}%
\pgfsetbuttcap%
\pgfsetmiterjoin%
\definecolor{currentfill}{rgb}{0.121569,0.466667,0.705882}%
\pgfsetfillcolor{currentfill}%
\pgfsetlinewidth{0.000000pt}%
\definecolor{currentstroke}{rgb}{0.000000,0.000000,0.000000}%
\pgfsetstrokecolor{currentstroke}%
\pgfsetstrokeopacity{0.000000}%
\pgfsetdash{}{0pt}%
\pgfpathmoveto{\pgfqpoint{-23.809001in}{1.366707in}}%
\pgfpathlineto{\pgfqpoint{14.201697in}{1.366707in}}%
\pgfpathlineto{\pgfqpoint{14.201697in}{1.373656in}}%
\pgfpathlineto{\pgfqpoint{-23.809001in}{1.373656in}}%
\pgfpathclose%
\pgfusepath{fill}%
\end{pgfscope}%
\begin{pgfscope}%
\pgfpathrectangle{\pgfqpoint{12.211765in}{0.750000in}}{\pgfqpoint{2.188235in}{0.953684in}} %
\pgfusepath{clip}%
\pgfsetbuttcap%
\pgfsetmiterjoin%
\definecolor{currentfill}{rgb}{0.121569,0.466667,0.705882}%
\pgfsetfillcolor{currentfill}%
\pgfsetlinewidth{0.000000pt}%
\definecolor{currentstroke}{rgb}{0.000000,0.000000,0.000000}%
\pgfsetstrokecolor{currentstroke}%
\pgfsetstrokeopacity{0.000000}%
\pgfsetdash{}{0pt}%
\pgfpathmoveto{\pgfqpoint{-23.809001in}{1.375394in}}%
\pgfpathlineto{\pgfqpoint{14.221826in}{1.375394in}}%
\pgfpathlineto{\pgfqpoint{14.221826in}{1.382344in}}%
\pgfpathlineto{\pgfqpoint{-23.809001in}{1.382344in}}%
\pgfpathclose%
\pgfusepath{fill}%
\end{pgfscope}%
\begin{pgfscope}%
\pgfpathrectangle{\pgfqpoint{12.211765in}{0.750000in}}{\pgfqpoint{2.188235in}{0.953684in}} %
\pgfusepath{clip}%
\pgfsetbuttcap%
\pgfsetmiterjoin%
\definecolor{currentfill}{rgb}{0.121569,0.466667,0.705882}%
\pgfsetfillcolor{currentfill}%
\pgfsetlinewidth{0.000000pt}%
\definecolor{currentstroke}{rgb}{0.000000,0.000000,0.000000}%
\pgfsetstrokecolor{currentstroke}%
\pgfsetstrokeopacity{0.000000}%
\pgfsetdash{}{0pt}%
\pgfpathmoveto{\pgfqpoint{-23.809001in}{1.384081in}}%
\pgfpathlineto{\pgfqpoint{14.219688in}{1.384081in}}%
\pgfpathlineto{\pgfqpoint{14.219688in}{1.391031in}}%
\pgfpathlineto{\pgfqpoint{-23.809001in}{1.391031in}}%
\pgfpathclose%
\pgfusepath{fill}%
\end{pgfscope}%
\begin{pgfscope}%
\pgfpathrectangle{\pgfqpoint{12.211765in}{0.750000in}}{\pgfqpoint{2.188235in}{0.953684in}} %
\pgfusepath{clip}%
\pgfsetbuttcap%
\pgfsetmiterjoin%
\definecolor{currentfill}{rgb}{0.121569,0.466667,0.705882}%
\pgfsetfillcolor{currentfill}%
\pgfsetlinewidth{0.000000pt}%
\definecolor{currentstroke}{rgb}{0.000000,0.000000,0.000000}%
\pgfsetstrokecolor{currentstroke}%
\pgfsetstrokeopacity{0.000000}%
\pgfsetdash{}{0pt}%
\pgfpathmoveto{\pgfqpoint{-23.809001in}{1.392768in}}%
\pgfpathlineto{\pgfqpoint{14.243277in}{1.392768in}}%
\pgfpathlineto{\pgfqpoint{14.243277in}{1.399718in}}%
\pgfpathlineto{\pgfqpoint{-23.809001in}{1.399718in}}%
\pgfpathclose%
\pgfusepath{fill}%
\end{pgfscope}%
\begin{pgfscope}%
\pgfpathrectangle{\pgfqpoint{12.211765in}{0.750000in}}{\pgfqpoint{2.188235in}{0.953684in}} %
\pgfusepath{clip}%
\pgfsetbuttcap%
\pgfsetmiterjoin%
\definecolor{currentfill}{rgb}{0.121569,0.466667,0.705882}%
\pgfsetfillcolor{currentfill}%
\pgfsetlinewidth{0.000000pt}%
\definecolor{currentstroke}{rgb}{0.000000,0.000000,0.000000}%
\pgfsetstrokecolor{currentstroke}%
\pgfsetstrokeopacity{0.000000}%
\pgfsetdash{}{0pt}%
\pgfpathmoveto{\pgfqpoint{-23.809001in}{1.401455in}}%
\pgfpathlineto{\pgfqpoint{14.235580in}{1.401455in}}%
\pgfpathlineto{\pgfqpoint{14.235580in}{1.408405in}}%
\pgfpathlineto{\pgfqpoint{-23.809001in}{1.408405in}}%
\pgfpathclose%
\pgfusepath{fill}%
\end{pgfscope}%
\begin{pgfscope}%
\pgfpathrectangle{\pgfqpoint{12.211765in}{0.750000in}}{\pgfqpoint{2.188235in}{0.953684in}} %
\pgfusepath{clip}%
\pgfsetbuttcap%
\pgfsetmiterjoin%
\definecolor{currentfill}{rgb}{0.121569,0.466667,0.705882}%
\pgfsetfillcolor{currentfill}%
\pgfsetlinewidth{0.000000pt}%
\definecolor{currentstroke}{rgb}{0.000000,0.000000,0.000000}%
\pgfsetstrokecolor{currentstroke}%
\pgfsetstrokeopacity{0.000000}%
\pgfsetdash{}{0pt}%
\pgfpathmoveto{\pgfqpoint{-23.809001in}{1.410143in}}%
\pgfpathlineto{\pgfqpoint{14.207776in}{1.410143in}}%
\pgfpathlineto{\pgfqpoint{14.207776in}{1.417092in}}%
\pgfpathlineto{\pgfqpoint{-23.809001in}{1.417092in}}%
\pgfpathclose%
\pgfusepath{fill}%
\end{pgfscope}%
\begin{pgfscope}%
\pgfpathrectangle{\pgfqpoint{12.211765in}{0.750000in}}{\pgfqpoint{2.188235in}{0.953684in}} %
\pgfusepath{clip}%
\pgfsetbuttcap%
\pgfsetmiterjoin%
\definecolor{currentfill}{rgb}{0.121569,0.466667,0.705882}%
\pgfsetfillcolor{currentfill}%
\pgfsetlinewidth{0.000000pt}%
\definecolor{currentstroke}{rgb}{0.000000,0.000000,0.000000}%
\pgfsetstrokecolor{currentstroke}%
\pgfsetstrokeopacity{0.000000}%
\pgfsetdash{}{0pt}%
\pgfpathmoveto{\pgfqpoint{-23.809001in}{1.418830in}}%
\pgfpathlineto{\pgfqpoint{14.147342in}{1.418830in}}%
\pgfpathlineto{\pgfqpoint{14.147342in}{1.425780in}}%
\pgfpathlineto{\pgfqpoint{-23.809001in}{1.425780in}}%
\pgfpathclose%
\pgfusepath{fill}%
\end{pgfscope}%
\begin{pgfscope}%
\pgfpathrectangle{\pgfqpoint{12.211765in}{0.750000in}}{\pgfqpoint{2.188235in}{0.953684in}} %
\pgfusepath{clip}%
\pgfsetbuttcap%
\pgfsetmiterjoin%
\definecolor{currentfill}{rgb}{0.121569,0.466667,0.705882}%
\pgfsetfillcolor{currentfill}%
\pgfsetlinewidth{0.000000pt}%
\definecolor{currentstroke}{rgb}{0.000000,0.000000,0.000000}%
\pgfsetstrokecolor{currentstroke}%
\pgfsetstrokeopacity{0.000000}%
\pgfsetdash{}{0pt}%
\pgfpathmoveto{\pgfqpoint{-23.809001in}{1.427517in}}%
\pgfpathlineto{\pgfqpoint{14.203179in}{1.427517in}}%
\pgfpathlineto{\pgfqpoint{14.203179in}{1.434467in}}%
\pgfpathlineto{\pgfqpoint{-23.809001in}{1.434467in}}%
\pgfpathclose%
\pgfusepath{fill}%
\end{pgfscope}%
\begin{pgfscope}%
\pgfpathrectangle{\pgfqpoint{12.211765in}{0.750000in}}{\pgfqpoint{2.188235in}{0.953684in}} %
\pgfusepath{clip}%
\pgfsetbuttcap%
\pgfsetmiterjoin%
\definecolor{currentfill}{rgb}{0.121569,0.466667,0.705882}%
\pgfsetfillcolor{currentfill}%
\pgfsetlinewidth{0.000000pt}%
\definecolor{currentstroke}{rgb}{0.000000,0.000000,0.000000}%
\pgfsetstrokecolor{currentstroke}%
\pgfsetstrokeopacity{0.000000}%
\pgfsetdash{}{0pt}%
\pgfpathmoveto{\pgfqpoint{-23.809001in}{1.436204in}}%
\pgfpathlineto{\pgfqpoint{14.161596in}{1.436204in}}%
\pgfpathlineto{\pgfqpoint{14.161596in}{1.443154in}}%
\pgfpathlineto{\pgfqpoint{-23.809001in}{1.443154in}}%
\pgfpathclose%
\pgfusepath{fill}%
\end{pgfscope}%
\begin{pgfscope}%
\pgfpathrectangle{\pgfqpoint{12.211765in}{0.750000in}}{\pgfqpoint{2.188235in}{0.953684in}} %
\pgfusepath{clip}%
\pgfsetbuttcap%
\pgfsetmiterjoin%
\definecolor{currentfill}{rgb}{0.121569,0.466667,0.705882}%
\pgfsetfillcolor{currentfill}%
\pgfsetlinewidth{0.000000pt}%
\definecolor{currentstroke}{rgb}{0.000000,0.000000,0.000000}%
\pgfsetstrokecolor{currentstroke}%
\pgfsetstrokeopacity{0.000000}%
\pgfsetdash{}{0pt}%
\pgfpathmoveto{\pgfqpoint{-23.809001in}{1.444892in}}%
\pgfpathlineto{\pgfqpoint{14.227720in}{1.444892in}}%
\pgfpathlineto{\pgfqpoint{14.227720in}{1.451841in}}%
\pgfpathlineto{\pgfqpoint{-23.809001in}{1.451841in}}%
\pgfpathclose%
\pgfusepath{fill}%
\end{pgfscope}%
\begin{pgfscope}%
\pgfpathrectangle{\pgfqpoint{12.211765in}{0.750000in}}{\pgfqpoint{2.188235in}{0.953684in}} %
\pgfusepath{clip}%
\pgfsetbuttcap%
\pgfsetmiterjoin%
\definecolor{currentfill}{rgb}{0.121569,0.466667,0.705882}%
\pgfsetfillcolor{currentfill}%
\pgfsetlinewidth{0.000000pt}%
\definecolor{currentstroke}{rgb}{0.000000,0.000000,0.000000}%
\pgfsetstrokecolor{currentstroke}%
\pgfsetstrokeopacity{0.000000}%
\pgfsetdash{}{0pt}%
\pgfpathmoveto{\pgfqpoint{-23.809001in}{1.453579in}}%
\pgfpathlineto{\pgfqpoint{14.180409in}{1.453579in}}%
\pgfpathlineto{\pgfqpoint{14.180409in}{1.460529in}}%
\pgfpathlineto{\pgfqpoint{-23.809001in}{1.460529in}}%
\pgfpathclose%
\pgfusepath{fill}%
\end{pgfscope}%
\begin{pgfscope}%
\pgfpathrectangle{\pgfqpoint{12.211765in}{0.750000in}}{\pgfqpoint{2.188235in}{0.953684in}} %
\pgfusepath{clip}%
\pgfsetbuttcap%
\pgfsetmiterjoin%
\definecolor{currentfill}{rgb}{0.121569,0.466667,0.705882}%
\pgfsetfillcolor{currentfill}%
\pgfsetlinewidth{0.000000pt}%
\definecolor{currentstroke}{rgb}{0.000000,0.000000,0.000000}%
\pgfsetstrokecolor{currentstroke}%
\pgfsetstrokeopacity{0.000000}%
\pgfsetdash{}{0pt}%
\pgfpathmoveto{\pgfqpoint{-23.809001in}{1.462266in}}%
\pgfpathlineto{\pgfqpoint{14.212875in}{1.462266in}}%
\pgfpathlineto{\pgfqpoint{14.212875in}{1.469216in}}%
\pgfpathlineto{\pgfqpoint{-23.809001in}{1.469216in}}%
\pgfpathclose%
\pgfusepath{fill}%
\end{pgfscope}%
\begin{pgfscope}%
\pgfpathrectangle{\pgfqpoint{12.211765in}{0.750000in}}{\pgfqpoint{2.188235in}{0.953684in}} %
\pgfusepath{clip}%
\pgfsetbuttcap%
\pgfsetmiterjoin%
\definecolor{currentfill}{rgb}{0.121569,0.466667,0.705882}%
\pgfsetfillcolor{currentfill}%
\pgfsetlinewidth{0.000000pt}%
\definecolor{currentstroke}{rgb}{0.000000,0.000000,0.000000}%
\pgfsetstrokecolor{currentstroke}%
\pgfsetstrokeopacity{0.000000}%
\pgfsetdash{}{0pt}%
\pgfpathmoveto{\pgfqpoint{-23.809001in}{1.470953in}}%
\pgfpathlineto{\pgfqpoint{14.197795in}{1.470953in}}%
\pgfpathlineto{\pgfqpoint{14.197795in}{1.477903in}}%
\pgfpathlineto{\pgfqpoint{-23.809001in}{1.477903in}}%
\pgfpathclose%
\pgfusepath{fill}%
\end{pgfscope}%
\begin{pgfscope}%
\pgfpathrectangle{\pgfqpoint{12.211765in}{0.750000in}}{\pgfqpoint{2.188235in}{0.953684in}} %
\pgfusepath{clip}%
\pgfsetbuttcap%
\pgfsetmiterjoin%
\definecolor{currentfill}{rgb}{0.121569,0.466667,0.705882}%
\pgfsetfillcolor{currentfill}%
\pgfsetlinewidth{0.000000pt}%
\definecolor{currentstroke}{rgb}{0.000000,0.000000,0.000000}%
\pgfsetstrokecolor{currentstroke}%
\pgfsetstrokeopacity{0.000000}%
\pgfsetdash{}{0pt}%
\pgfpathmoveto{\pgfqpoint{-23.809001in}{1.479641in}}%
\pgfpathlineto{\pgfqpoint{14.102271in}{1.479641in}}%
\pgfpathlineto{\pgfqpoint{14.102271in}{1.486590in}}%
\pgfpathlineto{\pgfqpoint{-23.809001in}{1.486590in}}%
\pgfpathclose%
\pgfusepath{fill}%
\end{pgfscope}%
\begin{pgfscope}%
\pgfpathrectangle{\pgfqpoint{12.211765in}{0.750000in}}{\pgfqpoint{2.188235in}{0.953684in}} %
\pgfusepath{clip}%
\pgfsetbuttcap%
\pgfsetmiterjoin%
\definecolor{currentfill}{rgb}{0.121569,0.466667,0.705882}%
\pgfsetfillcolor{currentfill}%
\pgfsetlinewidth{0.000000pt}%
\definecolor{currentstroke}{rgb}{0.000000,0.000000,0.000000}%
\pgfsetstrokecolor{currentstroke}%
\pgfsetstrokeopacity{0.000000}%
\pgfsetdash{}{0pt}%
\pgfpathmoveto{\pgfqpoint{-23.809001in}{1.488328in}}%
\pgfpathlineto{\pgfqpoint{14.224371in}{1.488328in}}%
\pgfpathlineto{\pgfqpoint{14.224371in}{1.495278in}}%
\pgfpathlineto{\pgfqpoint{-23.809001in}{1.495278in}}%
\pgfpathclose%
\pgfusepath{fill}%
\end{pgfscope}%
\begin{pgfscope}%
\pgfpathrectangle{\pgfqpoint{12.211765in}{0.750000in}}{\pgfqpoint{2.188235in}{0.953684in}} %
\pgfusepath{clip}%
\pgfsetbuttcap%
\pgfsetmiterjoin%
\definecolor{currentfill}{rgb}{0.121569,0.466667,0.705882}%
\pgfsetfillcolor{currentfill}%
\pgfsetlinewidth{0.000000pt}%
\definecolor{currentstroke}{rgb}{0.000000,0.000000,0.000000}%
\pgfsetstrokecolor{currentstroke}%
\pgfsetstrokeopacity{0.000000}%
\pgfsetdash{}{0pt}%
\pgfpathmoveto{\pgfqpoint{-23.809001in}{1.497015in}}%
\pgfpathlineto{\pgfqpoint{14.230031in}{1.497015in}}%
\pgfpathlineto{\pgfqpoint{14.230031in}{1.503965in}}%
\pgfpathlineto{\pgfqpoint{-23.809001in}{1.503965in}}%
\pgfpathclose%
\pgfusepath{fill}%
\end{pgfscope}%
\begin{pgfscope}%
\pgfpathrectangle{\pgfqpoint{12.211765in}{0.750000in}}{\pgfqpoint{2.188235in}{0.953684in}} %
\pgfusepath{clip}%
\pgfsetbuttcap%
\pgfsetmiterjoin%
\definecolor{currentfill}{rgb}{0.121569,0.466667,0.705882}%
\pgfsetfillcolor{currentfill}%
\pgfsetlinewidth{0.000000pt}%
\definecolor{currentstroke}{rgb}{0.000000,0.000000,0.000000}%
\pgfsetstrokecolor{currentstroke}%
\pgfsetstrokeopacity{0.000000}%
\pgfsetdash{}{0pt}%
\pgfpathmoveto{\pgfqpoint{-23.809001in}{1.505702in}}%
\pgfpathlineto{\pgfqpoint{14.231506in}{1.505702in}}%
\pgfpathlineto{\pgfqpoint{14.231506in}{1.512652in}}%
\pgfpathlineto{\pgfqpoint{-23.809001in}{1.512652in}}%
\pgfpathclose%
\pgfusepath{fill}%
\end{pgfscope}%
\begin{pgfscope}%
\pgfpathrectangle{\pgfqpoint{12.211765in}{0.750000in}}{\pgfqpoint{2.188235in}{0.953684in}} %
\pgfusepath{clip}%
\pgfsetbuttcap%
\pgfsetmiterjoin%
\definecolor{currentfill}{rgb}{0.121569,0.466667,0.705882}%
\pgfsetfillcolor{currentfill}%
\pgfsetlinewidth{0.000000pt}%
\definecolor{currentstroke}{rgb}{0.000000,0.000000,0.000000}%
\pgfsetstrokecolor{currentstroke}%
\pgfsetstrokeopacity{0.000000}%
\pgfsetdash{}{0pt}%
\pgfpathmoveto{\pgfqpoint{-23.809001in}{1.514389in}}%
\pgfpathlineto{\pgfqpoint{14.210793in}{1.514389in}}%
\pgfpathlineto{\pgfqpoint{14.210793in}{1.521339in}}%
\pgfpathlineto{\pgfqpoint{-23.809001in}{1.521339in}}%
\pgfpathclose%
\pgfusepath{fill}%
\end{pgfscope}%
\begin{pgfscope}%
\pgfpathrectangle{\pgfqpoint{12.211765in}{0.750000in}}{\pgfqpoint{2.188235in}{0.953684in}} %
\pgfusepath{clip}%
\pgfsetbuttcap%
\pgfsetmiterjoin%
\definecolor{currentfill}{rgb}{0.121569,0.466667,0.705882}%
\pgfsetfillcolor{currentfill}%
\pgfsetlinewidth{0.000000pt}%
\definecolor{currentstroke}{rgb}{0.000000,0.000000,0.000000}%
\pgfsetstrokecolor{currentstroke}%
\pgfsetstrokeopacity{0.000000}%
\pgfsetdash{}{0pt}%
\pgfpathmoveto{\pgfqpoint{-23.809001in}{1.523077in}}%
\pgfpathlineto{\pgfqpoint{14.226910in}{1.523077in}}%
\pgfpathlineto{\pgfqpoint{14.226910in}{1.530026in}}%
\pgfpathlineto{\pgfqpoint{-23.809001in}{1.530026in}}%
\pgfpathclose%
\pgfusepath{fill}%
\end{pgfscope}%
\begin{pgfscope}%
\pgfpathrectangle{\pgfqpoint{12.211765in}{0.750000in}}{\pgfqpoint{2.188235in}{0.953684in}} %
\pgfusepath{clip}%
\pgfsetbuttcap%
\pgfsetmiterjoin%
\definecolor{currentfill}{rgb}{0.121569,0.466667,0.705882}%
\pgfsetfillcolor{currentfill}%
\pgfsetlinewidth{0.000000pt}%
\definecolor{currentstroke}{rgb}{0.000000,0.000000,0.000000}%
\pgfsetstrokecolor{currentstroke}%
\pgfsetstrokeopacity{0.000000}%
\pgfsetdash{}{0pt}%
\pgfpathmoveto{\pgfqpoint{-23.809001in}{1.531764in}}%
\pgfpathlineto{\pgfqpoint{14.300535in}{1.531764in}}%
\pgfpathlineto{\pgfqpoint{14.300535in}{1.538714in}}%
\pgfpathlineto{\pgfqpoint{-23.809001in}{1.538714in}}%
\pgfpathclose%
\pgfusepath{fill}%
\end{pgfscope}%
\begin{pgfscope}%
\pgfpathrectangle{\pgfqpoint{12.211765in}{0.750000in}}{\pgfqpoint{2.188235in}{0.953684in}} %
\pgfusepath{clip}%
\pgfsetbuttcap%
\pgfsetmiterjoin%
\definecolor{currentfill}{rgb}{0.121569,0.466667,0.705882}%
\pgfsetfillcolor{currentfill}%
\pgfsetlinewidth{0.000000pt}%
\definecolor{currentstroke}{rgb}{0.000000,0.000000,0.000000}%
\pgfsetstrokecolor{currentstroke}%
\pgfsetstrokeopacity{0.000000}%
\pgfsetdash{}{0pt}%
\pgfpathmoveto{\pgfqpoint{-23.809001in}{1.540451in}}%
\pgfpathlineto{\pgfqpoint{14.231664in}{1.540451in}}%
\pgfpathlineto{\pgfqpoint{14.231664in}{1.547401in}}%
\pgfpathlineto{\pgfqpoint{-23.809001in}{1.547401in}}%
\pgfpathclose%
\pgfusepath{fill}%
\end{pgfscope}%
\begin{pgfscope}%
\pgfpathrectangle{\pgfqpoint{12.211765in}{0.750000in}}{\pgfqpoint{2.188235in}{0.953684in}} %
\pgfusepath{clip}%
\pgfsetbuttcap%
\pgfsetmiterjoin%
\definecolor{currentfill}{rgb}{0.121569,0.466667,0.705882}%
\pgfsetfillcolor{currentfill}%
\pgfsetlinewidth{0.000000pt}%
\definecolor{currentstroke}{rgb}{0.000000,0.000000,0.000000}%
\pgfsetstrokecolor{currentstroke}%
\pgfsetstrokeopacity{0.000000}%
\pgfsetdash{}{0pt}%
\pgfpathmoveto{\pgfqpoint{-23.809001in}{1.549138in}}%
\pgfpathlineto{\pgfqpoint{14.240874in}{1.549138in}}%
\pgfpathlineto{\pgfqpoint{14.240874in}{1.556088in}}%
\pgfpathlineto{\pgfqpoint{-23.809001in}{1.556088in}}%
\pgfpathclose%
\pgfusepath{fill}%
\end{pgfscope}%
\begin{pgfscope}%
\pgfpathrectangle{\pgfqpoint{12.211765in}{0.750000in}}{\pgfqpoint{2.188235in}{0.953684in}} %
\pgfusepath{clip}%
\pgfsetbuttcap%
\pgfsetmiterjoin%
\definecolor{currentfill}{rgb}{0.121569,0.466667,0.705882}%
\pgfsetfillcolor{currentfill}%
\pgfsetlinewidth{0.000000pt}%
\definecolor{currentstroke}{rgb}{0.000000,0.000000,0.000000}%
\pgfsetstrokecolor{currentstroke}%
\pgfsetstrokeopacity{0.000000}%
\pgfsetdash{}{0pt}%
\pgfpathmoveto{\pgfqpoint{-23.809001in}{1.557826in}}%
\pgfpathlineto{\pgfqpoint{14.220819in}{1.557826in}}%
\pgfpathlineto{\pgfqpoint{14.220819in}{1.564775in}}%
\pgfpathlineto{\pgfqpoint{-23.809001in}{1.564775in}}%
\pgfpathclose%
\pgfusepath{fill}%
\end{pgfscope}%
\begin{pgfscope}%
\pgfpathrectangle{\pgfqpoint{12.211765in}{0.750000in}}{\pgfqpoint{2.188235in}{0.953684in}} %
\pgfusepath{clip}%
\pgfsetbuttcap%
\pgfsetmiterjoin%
\definecolor{currentfill}{rgb}{0.121569,0.466667,0.705882}%
\pgfsetfillcolor{currentfill}%
\pgfsetlinewidth{0.000000pt}%
\definecolor{currentstroke}{rgb}{0.000000,0.000000,0.000000}%
\pgfsetstrokecolor{currentstroke}%
\pgfsetstrokeopacity{0.000000}%
\pgfsetdash{}{0pt}%
\pgfpathmoveto{\pgfqpoint{-23.809001in}{1.566513in}}%
\pgfpathlineto{\pgfqpoint{14.175232in}{1.566513in}}%
\pgfpathlineto{\pgfqpoint{14.175232in}{1.573463in}}%
\pgfpathlineto{\pgfqpoint{-23.809001in}{1.573463in}}%
\pgfpathclose%
\pgfusepath{fill}%
\end{pgfscope}%
\begin{pgfscope}%
\pgfpathrectangle{\pgfqpoint{12.211765in}{0.750000in}}{\pgfqpoint{2.188235in}{0.953684in}} %
\pgfusepath{clip}%
\pgfsetbuttcap%
\pgfsetmiterjoin%
\definecolor{currentfill}{rgb}{0.121569,0.466667,0.705882}%
\pgfsetfillcolor{currentfill}%
\pgfsetlinewidth{0.000000pt}%
\definecolor{currentstroke}{rgb}{0.000000,0.000000,0.000000}%
\pgfsetstrokecolor{currentstroke}%
\pgfsetstrokeopacity{0.000000}%
\pgfsetdash{}{0pt}%
\pgfpathmoveto{\pgfqpoint{-23.809001in}{1.575200in}}%
\pgfpathlineto{\pgfqpoint{14.155370in}{1.575200in}}%
\pgfpathlineto{\pgfqpoint{14.155370in}{1.582150in}}%
\pgfpathlineto{\pgfqpoint{-23.809001in}{1.582150in}}%
\pgfpathclose%
\pgfusepath{fill}%
\end{pgfscope}%
\begin{pgfscope}%
\pgfpathrectangle{\pgfqpoint{12.211765in}{0.750000in}}{\pgfqpoint{2.188235in}{0.953684in}} %
\pgfusepath{clip}%
\pgfsetbuttcap%
\pgfsetmiterjoin%
\definecolor{currentfill}{rgb}{0.121569,0.466667,0.705882}%
\pgfsetfillcolor{currentfill}%
\pgfsetlinewidth{0.000000pt}%
\definecolor{currentstroke}{rgb}{0.000000,0.000000,0.000000}%
\pgfsetstrokecolor{currentstroke}%
\pgfsetstrokeopacity{0.000000}%
\pgfsetdash{}{0pt}%
\pgfpathmoveto{\pgfqpoint{-23.809001in}{1.583887in}}%
\pgfpathlineto{\pgfqpoint{14.207392in}{1.583887in}}%
\pgfpathlineto{\pgfqpoint{14.207392in}{1.590837in}}%
\pgfpathlineto{\pgfqpoint{-23.809001in}{1.590837in}}%
\pgfpathclose%
\pgfusepath{fill}%
\end{pgfscope}%
\begin{pgfscope}%
\pgfpathrectangle{\pgfqpoint{12.211765in}{0.750000in}}{\pgfqpoint{2.188235in}{0.953684in}} %
\pgfusepath{clip}%
\pgfsetbuttcap%
\pgfsetmiterjoin%
\definecolor{currentfill}{rgb}{0.121569,0.466667,0.705882}%
\pgfsetfillcolor{currentfill}%
\pgfsetlinewidth{0.000000pt}%
\definecolor{currentstroke}{rgb}{0.000000,0.000000,0.000000}%
\pgfsetstrokecolor{currentstroke}%
\pgfsetstrokeopacity{0.000000}%
\pgfsetdash{}{0pt}%
\pgfpathmoveto{\pgfqpoint{-23.809001in}{1.592575in}}%
\pgfpathlineto{\pgfqpoint{14.073969in}{1.592575in}}%
\pgfpathlineto{\pgfqpoint{14.073969in}{1.599524in}}%
\pgfpathlineto{\pgfqpoint{-23.809001in}{1.599524in}}%
\pgfpathclose%
\pgfusepath{fill}%
\end{pgfscope}%
\begin{pgfscope}%
\pgfpathrectangle{\pgfqpoint{12.211765in}{0.750000in}}{\pgfqpoint{2.188235in}{0.953684in}} %
\pgfusepath{clip}%
\pgfsetbuttcap%
\pgfsetmiterjoin%
\definecolor{currentfill}{rgb}{0.121569,0.466667,0.705882}%
\pgfsetfillcolor{currentfill}%
\pgfsetlinewidth{0.000000pt}%
\definecolor{currentstroke}{rgb}{0.000000,0.000000,0.000000}%
\pgfsetstrokecolor{currentstroke}%
\pgfsetstrokeopacity{0.000000}%
\pgfsetdash{}{0pt}%
\pgfpathmoveto{\pgfqpoint{-23.809001in}{1.601262in}}%
\pgfpathlineto{\pgfqpoint{14.235739in}{1.601262in}}%
\pgfpathlineto{\pgfqpoint{14.235739in}{1.608212in}}%
\pgfpathlineto{\pgfqpoint{-23.809001in}{1.608212in}}%
\pgfpathclose%
\pgfusepath{fill}%
\end{pgfscope}%
\begin{pgfscope}%
\pgfpathrectangle{\pgfqpoint{12.211765in}{0.750000in}}{\pgfqpoint{2.188235in}{0.953684in}} %
\pgfusepath{clip}%
\pgfsetbuttcap%
\pgfsetmiterjoin%
\definecolor{currentfill}{rgb}{0.121569,0.466667,0.705882}%
\pgfsetfillcolor{currentfill}%
\pgfsetlinewidth{0.000000pt}%
\definecolor{currentstroke}{rgb}{0.000000,0.000000,0.000000}%
\pgfsetstrokecolor{currentstroke}%
\pgfsetstrokeopacity{0.000000}%
\pgfsetdash{}{0pt}%
\pgfpathmoveto{\pgfqpoint{-23.809001in}{1.609949in}}%
\pgfpathlineto{\pgfqpoint{14.287911in}{1.609949in}}%
\pgfpathlineto{\pgfqpoint{14.287911in}{1.616899in}}%
\pgfpathlineto{\pgfqpoint{-23.809001in}{1.616899in}}%
\pgfpathclose%
\pgfusepath{fill}%
\end{pgfscope}%
\begin{pgfscope}%
\pgfpathrectangle{\pgfqpoint{12.211765in}{0.750000in}}{\pgfqpoint{2.188235in}{0.953684in}} %
\pgfusepath{clip}%
\pgfsetbuttcap%
\pgfsetmiterjoin%
\definecolor{currentfill}{rgb}{0.121569,0.466667,0.705882}%
\pgfsetfillcolor{currentfill}%
\pgfsetlinewidth{0.000000pt}%
\definecolor{currentstroke}{rgb}{0.000000,0.000000,0.000000}%
\pgfsetstrokecolor{currentstroke}%
\pgfsetstrokeopacity{0.000000}%
\pgfsetdash{}{0pt}%
\pgfpathmoveto{\pgfqpoint{-23.809001in}{1.618636in}}%
\pgfpathlineto{\pgfqpoint{13.901027in}{1.618636in}}%
\pgfpathlineto{\pgfqpoint{13.901027in}{1.625586in}}%
\pgfpathlineto{\pgfqpoint{-23.809001in}{1.625586in}}%
\pgfpathclose%
\pgfusepath{fill}%
\end{pgfscope}%
\begin{pgfscope}%
\pgfpathrectangle{\pgfqpoint{12.211765in}{0.750000in}}{\pgfqpoint{2.188235in}{0.953684in}} %
\pgfusepath{clip}%
\pgfsetbuttcap%
\pgfsetmiterjoin%
\definecolor{currentfill}{rgb}{0.121569,0.466667,0.705882}%
\pgfsetfillcolor{currentfill}%
\pgfsetlinewidth{0.000000pt}%
\definecolor{currentstroke}{rgb}{0.000000,0.000000,0.000000}%
\pgfsetstrokecolor{currentstroke}%
\pgfsetstrokeopacity{0.000000}%
\pgfsetdash{}{0pt}%
\pgfpathmoveto{\pgfqpoint{-23.809001in}{1.627323in}}%
\pgfpathlineto{\pgfqpoint{14.233596in}{1.627323in}}%
\pgfpathlineto{\pgfqpoint{14.233596in}{1.634273in}}%
\pgfpathlineto{\pgfqpoint{-23.809001in}{1.634273in}}%
\pgfpathclose%
\pgfusepath{fill}%
\end{pgfscope}%
\begin{pgfscope}%
\pgfpathrectangle{\pgfqpoint{12.211765in}{0.750000in}}{\pgfqpoint{2.188235in}{0.953684in}} %
\pgfusepath{clip}%
\pgfsetbuttcap%
\pgfsetmiterjoin%
\definecolor{currentfill}{rgb}{0.121569,0.466667,0.705882}%
\pgfsetfillcolor{currentfill}%
\pgfsetlinewidth{0.000000pt}%
\definecolor{currentstroke}{rgb}{0.000000,0.000000,0.000000}%
\pgfsetstrokecolor{currentstroke}%
\pgfsetstrokeopacity{0.000000}%
\pgfsetdash{}{0pt}%
\pgfpathmoveto{\pgfqpoint{-23.809001in}{1.636011in}}%
\pgfpathlineto{\pgfqpoint{14.197645in}{1.636011in}}%
\pgfpathlineto{\pgfqpoint{14.197645in}{1.642960in}}%
\pgfpathlineto{\pgfqpoint{-23.809001in}{1.642960in}}%
\pgfpathclose%
\pgfusepath{fill}%
\end{pgfscope}%
\begin{pgfscope}%
\pgfpathrectangle{\pgfqpoint{12.211765in}{0.750000in}}{\pgfqpoint{2.188235in}{0.953684in}} %
\pgfusepath{clip}%
\pgfsetbuttcap%
\pgfsetmiterjoin%
\definecolor{currentfill}{rgb}{0.121569,0.466667,0.705882}%
\pgfsetfillcolor{currentfill}%
\pgfsetlinewidth{0.000000pt}%
\definecolor{currentstroke}{rgb}{0.000000,0.000000,0.000000}%
\pgfsetstrokecolor{currentstroke}%
\pgfsetstrokeopacity{0.000000}%
\pgfsetdash{}{0pt}%
\pgfpathmoveto{\pgfqpoint{-23.809001in}{1.644698in}}%
\pgfpathlineto{\pgfqpoint{14.265177in}{1.644698in}}%
\pgfpathlineto{\pgfqpoint{14.265177in}{1.651648in}}%
\pgfpathlineto{\pgfqpoint{-23.809001in}{1.651648in}}%
\pgfpathclose%
\pgfusepath{fill}%
\end{pgfscope}%
\begin{pgfscope}%
\pgfpathrectangle{\pgfqpoint{12.211765in}{0.750000in}}{\pgfqpoint{2.188235in}{0.953684in}} %
\pgfusepath{clip}%
\pgfsetbuttcap%
\pgfsetmiterjoin%
\definecolor{currentfill}{rgb}{0.121569,0.466667,0.705882}%
\pgfsetfillcolor{currentfill}%
\pgfsetlinewidth{0.000000pt}%
\definecolor{currentstroke}{rgb}{0.000000,0.000000,0.000000}%
\pgfsetstrokecolor{currentstroke}%
\pgfsetstrokeopacity{0.000000}%
\pgfsetdash{}{0pt}%
\pgfpathmoveto{\pgfqpoint{-23.809001in}{1.653385in}}%
\pgfpathlineto{\pgfqpoint{14.150187in}{1.653385in}}%
\pgfpathlineto{\pgfqpoint{14.150187in}{1.660335in}}%
\pgfpathlineto{\pgfqpoint{-23.809001in}{1.660335in}}%
\pgfpathclose%
\pgfusepath{fill}%
\end{pgfscope}%
\begin{pgfscope}%
\pgfsetbuttcap%
\pgfsetroundjoin%
\definecolor{currentfill}{rgb}{0.000000,0.000000,0.000000}%
\pgfsetfillcolor{currentfill}%
\pgfsetlinewidth{0.803000pt}%
\definecolor{currentstroke}{rgb}{0.000000,0.000000,0.000000}%
\pgfsetstrokecolor{currentstroke}%
\pgfsetdash{}{0pt}%
\pgfsys@defobject{currentmarker}{\pgfqpoint{0.000000in}{-0.048611in}}{\pgfqpoint{0.000000in}{0.000000in}}{%
\pgfpathmoveto{\pgfqpoint{0.000000in}{0.000000in}}%
\pgfpathlineto{\pgfqpoint{0.000000in}{-0.048611in}}%
\pgfusepath{stroke,fill}%
}%
\begin{pgfscope}%
\pgfsys@transformshift{12.624272in}{0.750000in}%
\pgfsys@useobject{currentmarker}{}%
\end{pgfscope}%
\end{pgfscope}%
\begin{pgfscope}%
\pgftext[x=12.624272in,y=0.652778in,,top]{\rmfamily\fontsize{10.000000}{12.000000}\selectfont \(\displaystyle 10^{-12}\)}%
\end{pgfscope}%
\begin{pgfscope}%
\pgfsetbuttcap%
\pgfsetroundjoin%
\definecolor{currentfill}{rgb}{0.000000,0.000000,0.000000}%
\pgfsetfillcolor{currentfill}%
\pgfsetlinewidth{0.803000pt}%
\definecolor{currentstroke}{rgb}{0.000000,0.000000,0.000000}%
\pgfsetstrokecolor{currentstroke}%
\pgfsetdash{}{0pt}%
\pgfsys@defobject{currentmarker}{\pgfqpoint{0.000000in}{-0.048611in}}{\pgfqpoint{0.000000in}{0.000000in}}{%
\pgfpathmoveto{\pgfqpoint{0.000000in}{0.000000in}}%
\pgfpathlineto{\pgfqpoint{0.000000in}{-0.048611in}}%
\pgfusepath{stroke,fill}%
}%
\begin{pgfscope}%
\pgfsys@transformshift{13.130289in}{0.750000in}%
\pgfsys@useobject{currentmarker}{}%
\end{pgfscope}%
\end{pgfscope}%
\begin{pgfscope}%
\pgftext[x=13.130289in,y=0.652778in,,top]{\rmfamily\fontsize{10.000000}{12.000000}\selectfont \(\displaystyle 10^{-8}\)}%
\end{pgfscope}%
\begin{pgfscope}%
\pgfsetbuttcap%
\pgfsetroundjoin%
\definecolor{currentfill}{rgb}{0.000000,0.000000,0.000000}%
\pgfsetfillcolor{currentfill}%
\pgfsetlinewidth{0.803000pt}%
\definecolor{currentstroke}{rgb}{0.000000,0.000000,0.000000}%
\pgfsetstrokecolor{currentstroke}%
\pgfsetdash{}{0pt}%
\pgfsys@defobject{currentmarker}{\pgfqpoint{0.000000in}{-0.048611in}}{\pgfqpoint{0.000000in}{0.000000in}}{%
\pgfpathmoveto{\pgfqpoint{0.000000in}{0.000000in}}%
\pgfpathlineto{\pgfqpoint{0.000000in}{-0.048611in}}%
\pgfusepath{stroke,fill}%
}%
\begin{pgfscope}%
\pgfsys@transformshift{13.636307in}{0.750000in}%
\pgfsys@useobject{currentmarker}{}%
\end{pgfscope}%
\end{pgfscope}%
\begin{pgfscope}%
\pgftext[x=13.636307in,y=0.652778in,,top]{\rmfamily\fontsize{10.000000}{12.000000}\selectfont \(\displaystyle 10^{-4}\)}%
\end{pgfscope}%
\begin{pgfscope}%
\pgfsetbuttcap%
\pgfsetroundjoin%
\definecolor{currentfill}{rgb}{0.000000,0.000000,0.000000}%
\pgfsetfillcolor{currentfill}%
\pgfsetlinewidth{0.803000pt}%
\definecolor{currentstroke}{rgb}{0.000000,0.000000,0.000000}%
\pgfsetstrokecolor{currentstroke}%
\pgfsetdash{}{0pt}%
\pgfsys@defobject{currentmarker}{\pgfqpoint{0.000000in}{-0.048611in}}{\pgfqpoint{0.000000in}{0.000000in}}{%
\pgfpathmoveto{\pgfqpoint{0.000000in}{0.000000in}}%
\pgfpathlineto{\pgfqpoint{0.000000in}{-0.048611in}}%
\pgfusepath{stroke,fill}%
}%
\begin{pgfscope}%
\pgfsys@transformshift{14.142325in}{0.750000in}%
\pgfsys@useobject{currentmarker}{}%
\end{pgfscope}%
\end{pgfscope}%
\begin{pgfscope}%
\pgftext[x=14.142325in,y=0.652778in,,top]{\rmfamily\fontsize{10.000000}{12.000000}\selectfont \(\displaystyle 10^{0}\)}%
\end{pgfscope}%
\begin{pgfscope}%
\pgftext[x=13.305882in,y=0.471083in,,top]{\rmfamily\fontsize{10.000000}{12.000000}\selectfont \(\displaystyle |\theta^{\parallel}_j|\), \% of relative error}%
\end{pgfscope}%
\begin{pgfscope}%
\pgfsetbuttcap%
\pgfsetroundjoin%
\definecolor{currentfill}{rgb}{0.000000,0.000000,0.000000}%
\pgfsetfillcolor{currentfill}%
\pgfsetlinewidth{0.803000pt}%
\definecolor{currentstroke}{rgb}{0.000000,0.000000,0.000000}%
\pgfsetstrokecolor{currentstroke}%
\pgfsetdash{}{0pt}%
\pgfsys@defobject{currentmarker}{\pgfqpoint{-0.048611in}{0.000000in}}{\pgfqpoint{0.000000in}{0.000000in}}{%
\pgfpathmoveto{\pgfqpoint{0.000000in}{0.000000in}}%
\pgfpathlineto{\pgfqpoint{-0.048611in}{0.000000in}}%
\pgfusepath{stroke,fill}%
}%
\begin{pgfscope}%
\pgfsys@transformshift{12.211765in}{0.796824in}%
\pgfsys@useobject{currentmarker}{}%
\end{pgfscope}%
\end{pgfscope}%
\begin{pgfscope}%
\pgftext[x=12.045098in,y=0.748606in,left,base]{\rmfamily\fontsize{10.000000}{12.000000}\selectfont \(\displaystyle 0\)}%
\end{pgfscope}%
\begin{pgfscope}%
\pgfsetbuttcap%
\pgfsetroundjoin%
\definecolor{currentfill}{rgb}{0.000000,0.000000,0.000000}%
\pgfsetfillcolor{currentfill}%
\pgfsetlinewidth{0.803000pt}%
\definecolor{currentstroke}{rgb}{0.000000,0.000000,0.000000}%
\pgfsetstrokecolor{currentstroke}%
\pgfsetdash{}{0pt}%
\pgfsys@defobject{currentmarker}{\pgfqpoint{-0.048611in}{0.000000in}}{\pgfqpoint{0.000000in}{0.000000in}}{%
\pgfpathmoveto{\pgfqpoint{0.000000in}{0.000000in}}%
\pgfpathlineto{\pgfqpoint{-0.048611in}{0.000000in}}%
\pgfusepath{stroke,fill}%
}%
\begin{pgfscope}%
\pgfsys@transformshift{12.211765in}{1.231186in}%
\pgfsys@useobject{currentmarker}{}%
\end{pgfscope}%
\end{pgfscope}%
\begin{pgfscope}%
\pgftext[x=11.975653in,y=1.182968in,left,base]{\rmfamily\fontsize{10.000000}{12.000000}\selectfont \(\displaystyle 50\)}%
\end{pgfscope}%
\begin{pgfscope}%
\pgfsetbuttcap%
\pgfsetroundjoin%
\definecolor{currentfill}{rgb}{0.000000,0.000000,0.000000}%
\pgfsetfillcolor{currentfill}%
\pgfsetlinewidth{0.803000pt}%
\definecolor{currentstroke}{rgb}{0.000000,0.000000,0.000000}%
\pgfsetstrokecolor{currentstroke}%
\pgfsetdash{}{0pt}%
\pgfsys@defobject{currentmarker}{\pgfqpoint{-0.048611in}{0.000000in}}{\pgfqpoint{0.000000in}{0.000000in}}{%
\pgfpathmoveto{\pgfqpoint{0.000000in}{0.000000in}}%
\pgfpathlineto{\pgfqpoint{-0.048611in}{0.000000in}}%
\pgfusepath{stroke,fill}%
}%
\begin{pgfscope}%
\pgfsys@transformshift{12.211765in}{1.665547in}%
\pgfsys@useobject{currentmarker}{}%
\end{pgfscope}%
\end{pgfscope}%
\begin{pgfscope}%
\pgftext[x=11.906208in,y=1.617329in,left,base]{\rmfamily\fontsize{10.000000}{12.000000}\selectfont \(\displaystyle 100\)}%
\end{pgfscope}%
\begin{pgfscope}%
\pgftext[x=11.850653in,y=1.226842in,,bottom,rotate=90.000000]{\rmfamily\fontsize{10.000000}{12.000000}\selectfont \(\displaystyle j\)}%
\end{pgfscope}%
\begin{pgfscope}%
\pgfsetrectcap%
\pgfsetmiterjoin%
\pgfsetlinewidth{0.803000pt}%
\definecolor{currentstroke}{rgb}{0.000000,0.000000,0.000000}%
\pgfsetstrokecolor{currentstroke}%
\pgfsetdash{}{0pt}%
\pgfpathmoveto{\pgfqpoint{12.211765in}{0.750000in}}%
\pgfpathlineto{\pgfqpoint{12.211765in}{1.703684in}}%
\pgfusepath{stroke}%
\end{pgfscope}%
\begin{pgfscope}%
\pgfsetrectcap%
\pgfsetmiterjoin%
\pgfsetlinewidth{0.803000pt}%
\definecolor{currentstroke}{rgb}{0.000000,0.000000,0.000000}%
\pgfsetstrokecolor{currentstroke}%
\pgfsetdash{}{0pt}%
\pgfpathmoveto{\pgfqpoint{14.400000in}{0.750000in}}%
\pgfpathlineto{\pgfqpoint{14.400000in}{1.703684in}}%
\pgfusepath{stroke}%
\end{pgfscope}%
\begin{pgfscope}%
\pgfsetrectcap%
\pgfsetmiterjoin%
\pgfsetlinewidth{0.803000pt}%
\definecolor{currentstroke}{rgb}{0.000000,0.000000,0.000000}%
\pgfsetstrokecolor{currentstroke}%
\pgfsetdash{}{0pt}%
\pgfpathmoveto{\pgfqpoint{12.211765in}{0.750000in}}%
\pgfpathlineto{\pgfqpoint{14.400000in}{0.750000in}}%
\pgfusepath{stroke}%
\end{pgfscope}%
\begin{pgfscope}%
\pgfsetrectcap%
\pgfsetmiterjoin%
\pgfsetlinewidth{0.803000pt}%
\definecolor{currentstroke}{rgb}{0.000000,0.000000,0.000000}%
\pgfsetstrokecolor{currentstroke}%
\pgfsetdash{}{0pt}%
\pgfpathmoveto{\pgfqpoint{12.211765in}{1.703684in}}%
\pgfpathlineto{\pgfqpoint{14.400000in}{1.703684in}}%
\pgfusepath{stroke}%
\end{pgfscope}%
\end{pgfpicture}%
\makeatother%
\endgroup%

}
\caption[ORFF Representer theorem]{ORFF Representer theorem. We trained a first model named $\tildeK{\omega}$ following}
\label{fig:representer}
\end{figure}
\end{landscape}}

\subsection{Efficient linear operators}
When developping \cref{alg:close_form} we considered that the feature map $\tildePhi{\omega}(x)$ was a matrix from $\mathbb{R}^u$ to $\mathbb{R}^{r}$ for all $x\in\mathcal{X}$, and therefore that computing $\tildePhi{\omega}(x)^T \theta$ has a time complexity of $O(r^2u)$. While this holds true in the most generic senario, in many cases the feature maps presents some structure or sparsity allowing to reduce the computational cost of evaluating the feature map. We focus on the \acl{ORFF} given by \cref{alg:ORFF_construction}, developped in \cref{subsec:building_ORFF} and \cref{subsec:examples_ORFF} and treat the decomposable kernel, the curl-free kernel and the divergence-free kernel as an example. We recall that if $\mathcal{U}'=\mathbb{R}^{u'}$ and $\mathcal{U}=\mathbb{R}^u$, then $\tildeH{\omega}=\mathbb{R}^{2Du'}$ thus the \acl{ORFF}s given in \cref{ch:operator-valued_random_fourier_features} have the form
\begin{dmath*}
\begin{cases}\tildePhi{\omega}(x) \in\mathcal{L}\left(\mathbb{R}^u, \mathbb{R}^{2Du'}\right) &:  y \mapsto \frac{1}{\sqrt{D}}\Vect_{j=1}^D\pairing{x, \omega_j}B(\omega_j)^T y \\ \tildePhi{\omega}(x)^T \in\mathcal{L}\left(\mathbb{R}^{2Du'}, \mathbb{R}^u\right) &: \theta \mapsto \frac{1}{\sqrt{D}} \sum_{j=1}^D \pairing{x, \omega_j}B(\omega_j)\theta_j \end{cases},
\end{dmath*}
where $\omega_j\sim\probability_{\dual{\Haar}, \rho}$ \iid~and $B(\omega_j)\in\mathcal{L}\left(\mathbb{R}^u,\mathbb{R}^{u'}\right)$ for all $\omega_j\in\dual{\mathcal{X}}$. Hence the \acl{ORFF} can be seen as the block matrix
\begin{dmath*}
\tildePhi{\omega}(x) = \begin{pmatrix} \cos\inner{x,\omega_1}B(\omega_1)^T \\
\sin\inner{x,\omega_1}B(\omega_1)^T \\
\vdots \\
\cos\inner{x,\omega_D}B(\omega_D)^T \\
\sin\inner{x,\omega_D}B(\omega_D)^T
\end{pmatrix}\hiderel{\in}\mathcal{L}\left(\mathbb{R}^u, \mathbb{R}^{2Du'}\right)\condition{$\omega_j\sim\probability_{\dual{\Haar}, \rho}$ \iid.}
\end{dmath*}
\subsection{Decomposable kernel}
\subsection{Curl-free kernel}
\subsection{Divergence-free kernel}

\clearpage
%----------------------------------------------------------------------------------------
\section{Conclusions}
\label{sec:conclusions}

\chapterend
