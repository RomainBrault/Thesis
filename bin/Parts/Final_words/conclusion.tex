In machine learning, many algorithm focus on learning functions that model
dependencies in a dataset, where the the ouputs are real numbers. However
in numerous applications such as biology, economics, physics, etc.~the
output data are not reals: they can be a collection of reals, present
complex structures or can even be functions. To overcome this difficulty
and take into account the structure of the data, a common approach is to
see them as \emph{vectors} of some Hilbert space. From this observation, in
this thesis, we took interrest in vector-valued functions. Looking at the
litterature we focused on mathematical objects called \aclp{OVK} to learn
such functions.

\section{Contributions}
\acsp{OVK} naturally extend the celebrated kernel method used to learn
scalar-valued functions, to the case of learning vector-valued functions.
Yet, although \acsp{OVK} are appealing from a theoretical aspect, these
methods scale poorly in terms of computation time when the number of data
is high. Indeed, to evaluate the value of function with an \acl{OVK}, it
requires to evaluate an \acl{OVK} on all the point in the given dataset.
Hence naive learning with kernels usually scales cubicly in time with the
number of data. In the context of large-scale learning such scaling is not
acceptable. Through this work we propose a methodology to tackle this
difficulty.
\paragraph{}
Enlighted by the litterature on large-scale learning with
\emph{scalar}-valued kernel, in particular the work of Rahimi and Recht
\citep{Rahimi2007}, we propose to replace an \acs{OVK} by a random feature
map that we called \acl{ORFF}. Our contribution start with the formal
mathematical construction of this feature from an \acs{OVK}. Then we show
that it is also possible to obtain a kernel from an \acs{ORFF}. Eventually
we analyse the regularization properties in terms of \acl{FT} of
$\mathcal{Y}$-Mercer kernels. Then we moved on giving a bound on the error
due to the random approximation of the \acs{OVK} with high probability.
We showed that it is possible to bound the error eventhough the \acs{ORFF}
estimator of an \acs{OVK} is not a bounded random variable. Moreover we
also give a bound when the dimension of the ouput data infinite.
\paragraph{}
After ensuring that an \acs{ORFF} is a good approximation of a kernel, we
moved on giving a framework to learn with \aclp{OVK}. We showed that
learning with a feature map is equivalent to learn with the reconstructed
\acs{OVK} under some mild conditions. Then we focused on a efficient
implementation of \acs{ORFF} by viewing them as linear operator rather than
matrices and using matrix-free (iterative) solvers and conclued with some
numerical experiments. Eventually we gave a generalization bound for
\acs{ORFF} learning that suggest that the number of features sampled in an
\acs{ORFF} should be proportional to the number of data. We concluded our
contribution by applying the \acs{ORFF} framework to learning vector-valued
time series.

\section{Perspectives}
To start with the theoretical Perspectives, following Rahimi and Recht we
give a generalization bound for \acs{ORFF} kernel ridge that suggest that
the number of feature to draw is proportional to the number of data.
However new results of \citet{rudi2016generalization} suggest that the
number of feature should be proportional to the \emph{square root} of the
number of data. We shall investigate this results and extend it to
\acs{ORFF}.
\paragraph{}
On the methodological perspectives we gave an intuition on how operator-valued
kernels can be used to learn outputs that are functions. We used the \acs{ORFF}
framework to speed up quantile regression and at the same time retrieved the
full quantile function. We applied the same methodology to the anomaly
detection setting and showed that it is possible to learn jointly all the level
sets of a distribution with an extension of a \acl{OCSVM}. We are convinced
that this will open the door to many new applications. Given a problem with
some hyperparameters, the combination of \acs{ORFF} and \aclp{OVK} allow to
learn functions of the hyperparameters.
\paragraph{}
Another nice extension would be to be able to learn the structure of an
\acs{ORFF} \acs{ie} the spectral distribution and the operator from the data,
as in \citet{Yang2015} so that we avoid to inject directly ourselve a prior on
the data by the mean of an \acl{OVK}.
\paragraph{}
On the implementation level, we are really enthousiast about Operalib, a
library for learning with \aclp{OVK} started during this thesis as a project of
Paris-Saclay Center for Datascience, and will extend the library with other
\acs{OVK}-based algorithms. Moreover may work is remaning concerning the
implementation of efficient algorithms based on (O)\acsp{RFF}. We could extend
the Multiple Kernel learning setting to \acsp{OVK} to see if we can match the
performances of Deep Neural Networks as in \citet{lu2014scale}.
We could also improve the Doubly Stochastic Gradient descent in the light of
the recent results of \citet{rudi2016generalization} on generalization.
\paragraph{}
Eventually during this three years, we witnessed the rise of deep-learning
methods with neural network. As pointed out by many authors, random features
share deep connections with neural networks, as a neural network can be seen as
a compositional feature map. As in the work of \citet{yang2015deep} we could
replace the last layer of a convolutional neural network
\citep{lecun1995convolutional} with an \acs{ORFF} map to see if it can improve
its performances.

\chapterend
