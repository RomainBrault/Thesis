\section{Motivation}
This thesis is dedicated to the definition of a general and flexible approach
to learn vector-valued functions together with an efficient implementation of
the learning algorithms. To achieve this goal, we study shallow architectures,
namely the product of a (nonlinear) operator-valued feature
$\widetilde{\Phi}(x)$ and a parameter vector $\theta$ such that
$\widetilde{f}(x) = \widetilde{\Phi}(x)^* \theta$, and combine two appealing
methodologies: Operator-Valued Kernel Regression and Random Fourier Features.
\paragraph{}
Operator-Valued Kernels \citep{Micchelli2005,Carmeli2010,Kadri_aistat10,
Brouard2011,Alvarez2012} extend the classic scalar-valued kernels to functions
with values in some \emph{output} Hilbert space. As in the scalar case,
\acfp{OVK} are used to build Reproducing Kernel Hilbert Spaces (\acs{RKHS}) in
which representer theorems apply as for ridge regression or other appropriate
loss functional. In these cases, learning a model in the \acs{RKHS} boils down
to learning a function of the form $f(x)=\sum_{i=1}^N K(x,x_i)\alpha_i$ where
$x_1, \ldots, x_N$ are the training input data and each $\alpha_i, i=1, \ldots,
N$ is a vector of the output space $\mathcal{Y}$, and each $K(x,x_i)$ is an
operator on vectors of $\mathcal{Y}$.
\paragraph{}
However, \acsp{OVK} suffer from the same drawbacks as classic
(sca\-lar-va\-lued) kernel machines: they scale poorly to large datasets
because they are exceedingly demanding in terms of memory and computations. We
propose to approximate OVKs by extending a methodology called \acfp{RFF}
\citep{Rahimi2007, Le2013, Yang2015, sriper2015, Bach2015, sutherland2015,
rudi2016generalization} so far developed to speed up scalar-valued kernel
machines. The \acs{RFF} approach linearizes a shift-invariant kernel model by
generating explicitly an approximated feature map $\tilde{\phi}$. \acsp{RFF}
has been shown to be efficient on large datasets and has been further improved
by efficient matrix computations such as \citep[``FastFood'']{Le2013} and
\citep[``SORF'']{felix2016orthogonal}, which are considered as the best large
scale implementations of kernel methods, along with Nystr\"om approaches
proposed in \citet{drineas2005nystrom}. Moreover thanks to \acsp{RFF}, kernel
methods have been proved to be competitive with deep architectures
\citep{lu2014scale, dai2014scalable, yang2015deep}.

\section{Outline}
\textbf{Chapter 2.}
In this introductory chapter we recall some elements of the statistical
learning theory started by \citet{Vapnik1998}. Then we recall kernel methods
\citep{Aronszajn1950} which are used to construct spaces of
\emph{scalar-valued} functions (called \acsp{RKHS}) that are used model and
learn non linear dependencies from the data. We finish by a literature review
on large-scale implementations of kernel methods based on random Fourier
features \citep{Rahimi2007} and the Nystr\"om method
\citep{Williams2000-nystrom}.
\paragraph{}
\textbf{Chapter 3.}
In this chapter, to conclude the introduction, we develop briefly the
mathematical tools used throughout this manuscript. We give a full table of
notations, and present elements of functional analysis
\citep{kurdila2006convex} and abstract harmonic analysis
\citep{folland1994course}. Then we turn our attention to the case where the
functions we want to learn are not real-valued, but vector-valued.  To learn
vector-valued functions we define \aclp{OVK} \citep{Micchelli2005, Carmeli2010}
that generalize the scalar-valued kernel presented in \cref{ch:motivations}. We
conclude by giving a non-exhaustive list of \aclp{OVK} along with the context
in which they have been used.

\paragraph{}
\textbf{Chapter 4.}
In this first contribution chapter we present a generalization of the \acs{RFF}
framework introduced in \cref{ch:motivations} \citep{brault2016random}. This is
based on an operator-valued Bochner theorem proposed by \citet{Carmeli2010}. We
use this theorem to show how to construct an \acf{ORFF} from an \acs{OVK}.
Conversely we also show that it is possible to construct an \acs{ORFF} from the
regularization properties it induces rather than from an \acs{OVK}. We give
various examples of \acs{ORFF} maps such as an \acs{ORFF} map for the
decomposable kernel, the curl-free kernel and the divergence-free kernel.

\paragraph{}
\textbf{Chapter 5.}
In this contribution chapter we refine the bound on the \ac{OVK} approximation
with \ac{ORFF} we first proposed in \cite{brault2016random} and presented in
\cite{braultborne}. It generalizes the proof technique of \citet{Rahimi2007} to
\ac{OVK} on \ac{LCA} groups thanks to the recent results of
\citet{sutherland2015, tropp2015introduction, minsker2011some,
koltchinskii2013remark}. As a Bernstein bound it depends on the variance of the
estimator for which we derive an \say{upper bound}.

\paragraph{}
\textbf{Chapter 6.}
This contribution chapter focus on explaining how to define an efficient
implementation and algorithm to train an \acs{ORFF} model. First we recall
the supervised ridge regression with \acs{OVK} and the celebrated representer
theorem \citep{Wahba90}. Then we show under which conditions learning with an
\acs{ORFF} is equivalent to learn with a kernel approximation. Eventually
we give the gradient for the ridge regression problem, useful to find an
optimal solution with gradient descent algorithms, as well as a closed form
algorithm.  We conclude by showing how viewing \acsp{ORFF} as linear operators
rather than matrices yields a more efficient implementation and finish with
some numerical applications on toy and real-world datasets.

\paragraph{}
\textbf{Chapter 7.}
This contribution chapter deals with a generalization bound for the a
regression problem with ORFF based on the results of \citet{rahimi2009weighted,
maurer2016vector}.  We also discuss the case of Ridge regression presented in
\cref{ch:learning operator-valued_random_fourier_features}.

\paragraph{}
\textbf{Chapter 8.}
This contribution chapter shows how to use the \acs{ORFF} methodology for
non-linear vector autoregression. It is an instantiation of the \acs{ORFF}
framework to $\mathcal{X}=\mathcal{Y}=\left(\mathbb{R}^d, +\right)$. We also
give a generalization of a stochastic gradient descent \citep{dai2014scalable}
to \acs{ORFF}. This is a joint work with N\'eh\'emy Lim and Florence
d'Alch\'e-Buc and has been published at a workshop of \acs{ECML}. It is based
on the previous work \citet{Lim2015} for time series vector autoregression with
operator-valued kernels \cite{brault2016scaling}.

\paragraph{}
\textbf{Chapter 9.}
To conclude our work we present some work in progress. We show practical
applications of operator-valued kernels acting on an infinite dimensional space
$\mathcal{Y}$. We give two examples. First we show how to generalize many
quantile regression to learn a continuous function of the quantiles on the
data. Second we apply the same methodology to the \acf{OCSVM} algorithm in
order to learn a continuous function of all the level sets. We conclude by
presenting Operalib, a python library developed during this thesis which aims
at implementing \acs{OVK}-based algorithms in the spirit of Scikit-learn
\citep{pedregosa2011scikit}.


