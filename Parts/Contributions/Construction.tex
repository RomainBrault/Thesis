%!TEX root = ../../ThesisRomainbrault.tex

%----------------------------------------------------------------------------------------
\section{Motivations}
\label{sec:motivations}
Random Fourier Features have been proved useful to implement efficiently kernel methods in the scalar case, allowing to learn a linear model based on an approximated feature map. In this work, we are interested to construct approximated operator-valued feature maps to learn vector-valued functions. With an explicit (approximated) feature map, one converts the problem of learning a function $f$ in the vector-valued Reproducing Kernel Hilbert Space $\mathcal{H}_K$ into the learning of a linear model $\tilde{f}$ defined by:
 \begin{dmath*}
 \tilde{f}(x) = \tildePhi{\omega}(x)^\adjoint  \theta,
 \end{dmath*}
 where $\Phi: \mathcal{X} \to \mathcal{L}(\mathcal{H},\mathcal{Y})$ and $\theta \in \mathcal{H}$. The methodology we propose works for operator-valued kernels defined on any \acf{LCA} group, noted ($\mathcal{X}, \groupop)$, for some operation noted $\groupop$. This allows us to use the general context of Pontryagin duality for \acl{FT} of functions on \acs{LCA} groups. Building upon a generalization of Bochner's theorem for operator-valued measures, an operator-valued kernel is seen as the \emph{\acl{FT}} of an operator-valued positive measure. From that result, we extend the principle of Random Fourier Feature for scalar-valued kernels and derive a general methodology to build Operator Random Fourier Feature when operator-valued kernels are shift-invariant according to the chosen group operation. Elements of this chapter has been developped in \citet{brault2016random}.

\clearpage
%----------------------------------------------------------------------------------------
\section{Construction}
\label{sec:construction}
We present a construction of \acf{ORFF} such that $f: x\mapsto \tildePhi{\omega}(x)^\adjoint \theta$ is a continuous function that maps an arbitrary \acs{LCA} group $\mathcal{X}$ as input space to an arbitrary output Hilbert space $\mathcal{Y}$. First we define a functional \emph{Fourier feature map}, and then propose a Monte-Carlo sampling from this feature map to construct an approximation of a shift-invariant $\mathcal{Y}$-Mercer kernel.
Then, we prove the convergence of the kernel approximation $\tilde{K}(x,z)=\tildePhi{\omega}(x)^\adjoint \tildePhi{\omega}(z)$ with high probability on \emph{compact} subsets of the \acs{LCA} $\mathcal{X}$, when $\mathcal{Y}$ is \emph{finite dimensional}. Eventually we conclude with some numerical experiments.
\subsection{Theoretical study}
The following proposition of \citet{Zhang2012,Carmeli2010} extends Bochner's theorem to any shift-invariant $\mathcal{Y}$-Mercer kernel.
\begin{proposition}[Operator-valued Bochner's theorem \citep{Zhang2012,neeb1998operator}]\label{eq:bochner-gen}
If a continuous function $K$ from $\mathcal{X} \times \mathcal{X}$ to $\mathcal{Y}$ is a shift-invariant $\mathcal{Y}$-Mercer kernel on $\mathcal{X}$, then there exists a unique positive projection-valued measure $\dual{Q}: \mathcal{B}(\mathcal{X}) \to \mathcal{L}_+(\mathcal{Y})$ such that for all $x$, $z \in \mathcal{X}$,
\begin{dmath}
K(x, z) = \int_{\dual{\mathcal{X}}} \conj{\pairing{x \groupop \inv{z}, \omega}} d\dual{Q}(\omega),
\end{dmath}
where $\dual{Q}$ belongs to the set of all the projection-valued measures of bounded variation on the $\sigma$-algebra of Borel subsets of $\dual{\mathcal{X}}$. Conversely, from any positive operator-valued measure $M$, a shift-invariant kernel $K$ can be defined by \cref{eq:bochner-gen}.
\end{proposition}
Although this theorem is central to the spectral decomposition of shift-invariant $\mathcal{Y}$-Mercer \acs{OVK}, the following results proved by \citet{Carmeli2010} provides insights about this decomposition that are more relevant in practice. It first gives the necessary conditions to build shift-invariant $\mathcal{Y}$-Mercer kernel with  a pair $(A, \dual{\mu})$ where $A$ is an operator-valued function on $\dual{\mathcal{X}}$ and $\dual{\mu}$ is a real-valued positive measure on $\dual{\mathcal{X}}$. Note that obviously such a pair is not unique ad the choice of this paper may have an impact on theoretical properties as well as practical computations.
Secondly it also states that any \acs{OVK} have such a spectral decomposition when $\mathcal{Y}$ is finite dimensional or $\mathcal{X}$.

\begin{proposition}[\citet{Carmeli2010}]\label{pr:mercer_kernel_bochner}
Let $\dual{\mu}$ be a positive measure on $\mathcal{B}(\mathcal{\dual{\mathcal{X}}})$ and $A: \dual{\mathcal{X}}\to \mathcal{L}(\mathcal{Y})$ such that $\inner{A(\cdot)y,y'}\in L^1(\mathcal{X},\dual{\mu})$ for all $y,y'\in\mathcal{Y}$ and $A(\omega)\succcurlyeq 0$ for $\dual{\mu}$-almost all $\omega\in\dual{\mathcal{X}}$. Then, for all $\delta \in \mathcal{X}$,
\begin{dmath}
\label{eq:AK0}
K_e(\delta)=\int_{\dual{\mathcal{X}}}\conj{\pairing{\delta,\omega}}A(\omega)d\dual{\mu}(\omega)
\end{dmath}
is the kernel signature of a shift-invariant $\mathcal{Y}$-Mercer kernel $K$ such that $K(x,z)=K_e(x \groupop \inv{z})$. The \acs{vv-RKHS} $\mathcal{H}_K$ is embed in $L^2(\dual{\mathcal{X}},\dual{\mu};\mathcal{Y}')$ by mean of the feature operator
\begin{dmath}
\label{eq:feature_operator}
(Wg)(x)=\int_{\mathcal{\dual{X}}}\conj{\pairing{x,\omega}}B(\omega)g(\omega)d\dual{\mu}(\omega),
\end{dmath}
Where $B(\omega)B(\omega)^\adjoint=A(\omega)$ and both integral converges in the weak sense. If $\mathcal{Y}$ is finite dimensional or $\mathcal{X}$ is compact, any shift-invariant kernel is of the above form for some pair $(A, \dual{\mu})$.
\end{proposition}
\paragraph{}
When $p=1$ one can always assume $A$ is reduced to the scalar $1$, $\dual{\mu}$ is still a bounded positive measure and we retrieve the Bochner theorem applied to the scalar case (\cref{th:bochner-scalar}).
\paragraph{}
\Cref{pr:mercer_kernel_bochner} shows that a given pair $(A,\dual{\mu})$ characterize an \acs{OVK}. Namely given a measure $\dual{\mu}$ and a function $A$ such that $\inner{A(.)y,y'}\in L^1(\mathcal{X},\dual{\mu})$ for all $y,y'\in\mathcal{Y}$ and $A(\omega)\succcurlyeq 0$ for $\dual{\mu}$-almost all $\omega$, it gives rise to an \acs{OVK}. Since $(A,\dual{\mu})$ determine a unique kernel we can write $\mathcal{H}_{(A,\dual{\mu})}{\scriptstyle\implies}\mathcal{H}_K$ where $K$ is defined as in \cref{eq:AK0}. However the converse is to true: Given a $\mathcal{Y}$-Mercer shift invariant \acl{OVK}, there exist infinitely many pairs $(A,\dual{\mu})$ that characterize an \acs{OVK}.
\paragraph{}
The main difference between \cref{eq:bochner-gen} and \cref{pr:mercer_kernel_bochner} is that the first one characterize an \acs{OVK} by a unique \acf{POVM}, while the second one shows that the \acs{POVM} that uniquely characterize a $\mathcal{Y}$-Mercer \acs{OVK} has an operator-valued density with respect to a \emph{scalar} measure $\dual{\mu}$; and that this operator-valued density is not unique.
\paragraph{}
Finally \cref{pr:mercer_kernel_bochner} does not provide any \emph{constructive} way to obtain the pair $(A,\dual{\mu})$ that characterize an \acs{OVK}.
The following \cref{subsec:sufficient_conditions} is based on an other proposition of \citeauthor{carmeli2006vector} and show that if the kernel signature $K_e(\delta)$ of an $\acs{OVK}$ is in $L^1$ then it is possible to construct \emph{explicitly} a pair $(C,\dual{\Haar})$ from it. Additionally, we show that we can always extract a scalar-valued \emph{probability} density function from $C$ such that we obtain a pair $(A,\probability_{\dual{\mu},\rho})$ where $\probability_{\dual{\mu},\rho}$ is a \emph{probability} distribution absolutely continuous with respect to $\dual{\mu}$ and with associated probability density function (\pdf) $\rho$. Thus for all $\mathcal{Z}\subset\mathcal{B}(\dual{\mathcal{X}})$,
\begin{dmath*}
\probability_{\dual{\mu},\rho}(\mathcal{Z})=\int_{\mathcal{Z}} \rho(\omega)d\dual{\mu}(\omega).
\end{dmath*}
When the reference measure $\dual{\mu}$ is the Lebesgue measure, we note $\probability_{\dual{\mu},\rho}=\probability_\rho$.

\subsection{Sufficient conditions of existence}
\label{subsec:sufficient_conditions}
While \cref{pr:mercer_kernel_bochner} gives some insights on how to build an approximation of a $\mathcal{Y}$-Mercer kernel, we need a theorem that provides an explicit construction of the pair $(A, \probability_{\dual{\mu},\rho})$ from the kernel signature $K_e$. Proposition 14 in \citet{Carmeli2010} gives the solution, and also provide a sufficient condition for \cref{pr:mercer_kernel_bochner} to apply.
\begin{proposition}[\citet{Carmeli2010}]
\label{pr:inverse_ovk_Fourier_decomposition}
Let $K$ be a shift-invariant $\mathcal{Y}$-Mercer kernel. %
Suppose that for all $z \in \mathcal{X}$ and for all $y$, $y' \in\mathcal{Y}$, $\inner{K_e(.)y,y'}\in L^1(\mathcal{X},\Haar)$ where $\mathcal{X}$ is endowed with the group law $\groupop$. For all $\omega \in \dual{\mathcal{X}}$ and for all $y$, $y'$ in $\mathcal{Y}$, let
\begin{dmath}\label{eq:CK0}
\inner{y',C(\omega)y} = \int_{\mathcal{X}} \pairing{\delta, \omega}\inner{y', K_e(\delta)y}d\delta = \IFT{\inner{y', K_e(\cdot)y}}(\omega).
\end{dmath}
Then
\begin{propenum}
\item $C(\omega)$ is a bounded non-negative operator for all $\omega \in \dual{\mathcal{X}}$,
\item $\inner{y, C(\cdot)y'}\in L^1(\dual{\mathcal{X}},\dual{\Haar})$ for all $y,y'\in\mathcal{X}$,
\item for all $\delta\in\mathcal{X}$ and for all $y$, $y'$ in $\mathcal{Y}$,
\begin{dmath*}
\inner{y', K_e(\delta)y}= \int_{\dual{\mathcal{X}}}\conj{\pairing{\delta,\omega}}\inner{y', C(\omega)y}d\dual{\Haar}(\omega)
=\FT{\inner{y', C(\cdot)y}}(\delta).
\end{dmath*}
\end{propenum}
\end{proposition}
There have been a lot of confusion in the literature whether a kernel is the \acl{FT} or \acl{IFT} of a measure. However \cref{lm:C_characterization} clarify the relation between the \acl{FT} and \acl{IFT} for a translation invariant \acl{OVK}. Notice that in the real scalar case the \acl{FT} and \acl{IFT} of a shift-invariant kernel are the same, while the difference is significant for \acs{OVK}.
\paragraph{}
The following lemma is a direct consequence of the definition of $C(\omega)$ as the \acl{FT} of the adjoint of $K_e$ and also helps simplifying the definition of \acs{ORFF}.
\begin{lemma}
\label{lm:C_characterization}
Let $K_e$ be the signature of a shift-invariant $\mathcal{Y}$-Mercer kernel and let $\inner{y', C(\cdot)y}=\IFT{\inner{y', K_e(\cdot)y}}$ for all $y$. $y'\in\mathcal{Y}$. Then
\begin{propenum}
\item \label{lm:C_characterization_1} $C(\omega)$ is self-adjoint and $C$ is even.
\item \label{lm:C_characterization_2} $\IFT{\inner{y', K_e(\cdot)y}} = \FT{\inner{y', K_e(\cdot)y}}$.
\item \label{lm:C_characterization_3} $K_e(\delta)$ is self-adjoint and $K_e$ is even.
\end{propenum}
\end{lemma}
\begin{proof}
For any function $f$ on $(\mathcal{X},\groupop)$ define the flip operator $\mathcal{R}$ by
\begin{dmath*}
\mathcal{R}f(x) \colonequals f\left(\inv{x}\right).
\end{dmath*}
For any shift invariant $\mathcal{Y}$-Mercer kernel and for all $\delta\in\mathcal{X}$,  $K_e(\delta)=K_e\left(\inv{\delta}\right)^\adjoint$. Indeed,
\begin{dmath*}
\mathcal{R}\inner{y,K_e\left(x\groupop \inv{z}\right)y'}
=\inner{y,K_e\left(\inv{\left(x\groupop \inv{z}\right)}\right)y'}
=\inner{y,K_e\left(\inv{x}\groupop z\right)y'}
=\inner{y,K_e\left(x\groupop \inv{z}\right)^\adjoint y'}.
\end{dmath*}
\Cref{lm:C_characterization_1}: taking the \acl{FT} yields,
\begin{dmath*}
\IFT{\inner{y', K_e(\cdot)y}}=\mathcal{F}^{-1}\mathcal{R}\left[\inner{y', K_e(\cdot)y}\right]
\hiderel{=}\mathcal{R}\inner{y', C(\cdot)y}.
\end{dmath*}
Hence $C(\omega)=C\left(\inv{\omega}\right)^\adjoint$. Suppose that $\mathcal{Y}$ is a complex Hilbert space. Since for all $\omega\in\mathcal{\dual{X}}$, $C(\omega)$ is bounded and non-negative so $C(\omega)$ is self-adjoint. Besides we have $C(\omega)=C\left(\inv{\omega}\right)^\adjoint $ to $C$ must be pair. Suppose that $\mathcal{Y}$ is a real Hilbert space. Then we have the additional hypothesis that $K_e(\delta)=K_e(\delta)^\adjoint$. Taking the \acl{FT} yields that $C(\omega)=C(\omega)^\adjoint$. Since for any shift invariant $\mathcal{Y}$-Mercer kernel $C(\omega)=C\left(\inv{\omega}\right)^\adjoint$ we also conclude that $C\left(\inv{\omega}\right)=C(\omega)$.
\paragraph{}
\Cref{lm:C_characterization_2}: simply, for all $y$, $y'\in\mathcal{Y}$, $\inner{y, C(\inv{\omega})y'}$ $=$ $\inner{y', C(\omega)y}$ thus $\IFT{\inner{y', C(\cdot)y}}=\mathcal{F}\mathcal{R}\left[\inner{y', C(\cdot)y}\right]=\FT{\inner{y', C(\cdot)y}}$.
\paragraph{}
\Cref{lm:C_characterization_3}: from \cref{lm:C_characterization_2} we have $\IFT{\inner{y', K_e(\cdot)y}}$ $=$ $\mathcal{F}^{-1}\mathcal{R}{\inner{y', K_e(\cdot)y}}$. By injectivity of the \acl{FT}, $K_e$ is even. Since $K_e(\delta)=K_e(\inv{\delta})^\adjoint $, we must have $K_e(\delta)=K_e(\delta)^\adjoint $.
\end{proof}
While \cref{pr:inverse_ovk_Fourier_decomposition} gives an explicit form of the operator $C(\omega)$ defined as the \acl{FT} of the kernel $K$, it is not really convenient to work with the Haar measure $\dual{\Haar}$ on $\mathcal{B}(\dual{\mathcal{X}})$. However it is easily possible to turn $\dual{\Haar}$ into a probability measure to allow efficient integration over an infinite domain.
\paragraph{}
The following proposition allows to build a spectral decomposition of a shift-invariant $\mathcal{Y}$-Mercer kernel on a \acs{LCA} group $\mathcal{X}$ endowed with the group law $\groupop$ with respect to a scalar probability measure, by extracting a scalar probability density function from $C$.
\begin{proposition}[Shift-invariant $\mathcal{Y}$-Mercer kernel spectral decomposition]
\label{pr:spectral}
Let $K_e$ be the signature of a shift-invariant $\mathcal{Y}$-Mercer kernel. If for all $y$, $y' \in\mathcal{Y}$, $\inner{K_e(.)y,y'}\in L^1(\mathcal{X},\Haar)$ then there exists a positive probability measure $\probability_{\dual{\Haar},\rho}$ and an operator-valued function $A$ an such that for all $y,$ $y'\in\mathcal{Y}$,
\begin{dmath}
\label{eq:expectation_spec}
\inner{y', K_e(\delta)y}
=\expectation_{\dual{\Haar},\rho}\left[\conj{\pairing{\delta, \omega}}\inner{y', A(\omega)y}\right],
\end{dmath}
with
\begin{dmath}
\label{eq:comega}
\inner{y', A(\omega)y}\rho(\omega) = \FT{\inner{y', K_e(\cdot)y}}(\omega).
\end{dmath}
Moreover
\begin{propenum}
\item for all $y,$ $y'\in\mathcal{Y}$, $\inner{A(.)y,y'}\in L^1(\dual{\mathcal{X}}, \probability_{\dual{\Haar},\rho})$,
\item $A(\omega)$ is non-negative for $\probability_{\dual{\Haar},\rho}$-almost all $\omega\in\dual{\mathcal{X}}$,
\item $A(\cdot)$ and $\rho(\cdot)$ are even functions.
\end{propenum}
\end{proposition}
\begin{proof}
This is a simple consequence of \cref{pr:inverse_ovk_Fourier_decomposition} and \cref{lm:C_characterization}. By taking $\inner{y',C(\omega)y} = \IFT{\inner{y', K_e(\cdot)y}}(\omega)=\FT{\inner{y', K_e(\cdot)y}}(\omega)$ we can write the following equality concerning the \acs{OVK} signature $K_e$.
% Suppose that $\mu$ is absolutely continuous \wrt~$d\omega$. Then for all $\delta \in \mathcal{X}$ and for all $y,$ $y'$ in $\mathcal{Y}$
\begin{dmath*}
\inner{y', K_e(\delta)y}(\omega)=
\int_{\dual{\mathcal{X}}}\conj{\pairing{\delta, \omega}}\inner{y', C(\omega)y}d\dual{\Haar}(\omega)
=\int_{\dual{\mathcal{X}}}\conj{\pairing{\delta, \omega}}\inner*{y', \frac{1}{\rho(\omega)}C(\omega)y}\rho(\omega)d\dual{\Haar}(\omega).
\end{dmath*}
It is always possible to choose $\rho(\omega)$ such that $\int_{\dual{\mathcal{X}}}\rho(\omega)d\dual{\Haar}(\omega)=1$. For instance choose
\begin{dmath*}
\rho(\omega)=\frac{\norm{C(\omega)}_{\mathcal{Y},\mathcal{Y}}}{\int_{\dual{\mathcal{X}}}\norm{C(\omega)}_{\mathcal{Y},\mathcal{Y}}d\dual{\Haar}(\omega)}
\end{dmath*}
Since for all $y$, $y'\in\mathcal{Y}$, $\inner{y',C(\cdot)y}\in L^1(\dual{\mathcal{X}},\dual{\Haar})$ and $\mathcal{Y}$ is a separable Hilbert space, by pettis measurability theorem, $\int_{\dual{\mathcal{X}}}\norm{C(\omega)}_{\mathcal{Y},\mathcal{Y}}d\dual{\Haar}(\omega)$ is finite and so is $\norm{C(\omega)}_{\mathcal{Y},\mathcal{Y}}$ for all $\omega\in\dual{\mathcal{X}}$.
Therefore $\rho(\omega)$ is the density of a probability measure $\probability_{\dual{\Haar},\rho}$, \ie~conclude by taking
\begin{dmath*}
\probability_{\dual{\Haar},\rho}(\mathcal{Z}) = \int_{\mathcal{Z}}\rho(\omega)d\dual{\Haar}(\omega),
\end{dmath*}
for all $\mathcal{Z}\in\mathcal{B}(\dual{\mathcal{X}})$.
\end{proof}
\paragraph{}
In the case where $\mathcal{Y}=\mathbb{R}^p$, we rewrite \cref{eq:comega} coefficient-wise by choosing an orthonormal basis $\Set{e_j}_{j\in\mathbb{N}_p}$ of $\mathbb{R}^p$.
\begin{dmath}
\label{eq:operator_identification_real}
A(\omega)_{ij}\rho(\omega)\hiderel{=}\FT{K_e(\cdot)_{ij}}(\omega).
\end{dmath}
It follows that for all $i$ and $j$ in $\mathbb{N}_{p}$,
\begin{dmath}\label{eq:matrix-exp}
K_e(x\groupop \inv{z})_{ij} \hiderel{=} \FT{A(\cdot)_{ij}\rho(\cdot)}(x\groupop \inv{z})
\end{dmath}

\begin{remark}
Note that although the \acl{FT} of $K_e$ yields a unique operator-valued function $C(\cdot)$, the decomposition of $C(\cdot)$ into $A(\cdot)\rho(\cdot)$ is again not unique. The choice of the decomposition may be justified by the computational cost or by the nature of the constants involved in the uniform convergence of the estimator.
\end{remark}
Another difficulty arise from the fact that the quantity $\sup_{\omega\in\dual{\mathcal{X}}} \norm{A(\omega)}_{\mathcal{Y},\mathcal{Y}}$ obtained in \cref{pr:spectral} might not bounded. The unboundedness of $\norm{A(\cdot)}_{\mathcal{Y},\mathcal{Y}}$ forbid the use of the most simple concentrations inequalities -- which require the boundedness of the random variable to be controlled. Therefore in the context of \acl{OVK} concentration inequalities for unbounded random operators should be used. However, as pointed out by \citet{minh2016operator}, under some condition on the trace of $K_e(\delta)$, it is possible to turn $A(\cdot)$ into a bounded random operator for all $\omega$ in $\dual{\mathcal{X}}$. The idea is to define a sum measure $\rho=\sum_{j\in\mathbb{N}}\rho_{e_j}$, which gives rise to bounded operator $A(\omega)$ and is idependant of the $\Set{e_j}_{j\in\mathbb{N}}$ base, instead of constructing a measure from the operator norm as in \cref{pr:spectral}. Additionally with such construction the measure associated to $A(\cdot)$ is \emph{independant} from the basis of $\mathcal{Y}$. In this proof we relax the assumptions of \citet{minh2016operator} which requires $\int_{\mathcal{X}}\abs{\Tr{K_e(\delta)}}d\Haar(\delta)$ to be well defined. We only require $\Tr{K_e(e)}$ to be well defined.
\begin{proposition}[Bounded shift-invariant $\mathcal{Y}$-Mercer kernel spectral decomposition]
\label{pr:trace_measure}
Let $K_e$ be the signature of a shift-invariant $\mathcal{Y}$-Mercer kernel. If for all $y$ and $y'$ in $\mathcal{Y}$, $\inner{K_e(.)y,y'}\in L^1(\mathcal{X},\Haar)$ and $\Tr K_e(e)\in\mathbb{R}$, then
\begin{dmath}
\label{eq:expectation_tr}
\inner{y', K_e(\delta)y}
=\expectation_{\dual{\Haar},\rho_{\Tr}}\left[\conj{\pairing{\delta, \omega}}\inner{y, A_{\Tr}(\omega)y'}\right].
\end{dmath}
with
\begin{dgroup}
\begin{dmath}
\inner{y', C(\cdot)y}=\FT{\inner{y', K_e(\cdot)y}}
\end{dmath}
\begin{dmath}
c_{\Tr}=\Tr\left[K_e(e)\right]
\end{dmath}
\begin{dmath}
\label{eq:bounded_C}
A_{\Tr}(\omega)=c_{\Tr}\Tr\left[C(\omega)\right]^{-1}C(\omega)
\end{dmath}
\begin{dmath}
\rho_{\Tr}(\omega)=c_{\Tr}^{-1}\Tr\left[C(\omega)\right].
\end{dmath}

\end{dgroup}
\label{eq:bounded_mu}
Moreover
\begin{propenum}
\item For all $y$, $y'\in\mathcal{Y}$, $\inner{y, A_{\Tr}(\cdot)y'}\in L^1(\dual{\mathcal{X}}, \probability_{\dual{\Haar},\rho_{\Tr}})$.
\item $A_{\Tr}(\omega)$ is non-negative for all $\omega\in\dual{\mathcal{X}}$,
\item $\sup_{\omega\in\dual{\mathcal{X}}}\norm{A_{\Tr}(\omega)}_{\mathcal{Y},\mathcal{Y}}\le c_p$,
\item $A_{\Tr}(\cdot)$ and $\rho_{\Tr}$ are even functions.
\end{propenum}
\end{proposition}
\begin{proof}
Let $\Set{e_j}_{j\in\mathbb{N}}$ be an orthonormal basis of $\mathcal{Y}$. Notice that
\begin{dmath*}
\int_{\dual{\mathcal{X}}}\inner{e_j,C(\omega)e_j}d\dual{\Haar}(\omega)
=\int_{\dual{\mathcal{X}}}\underbrace{\conj{\pairing{e,\omega}}}_{=1}\inner{e_j,C(\omega)e_j}d\dual{\Haar}(\omega)
=\inner{e_j,K_e(e)e_j}.
\end{dmath*}
Since $C(\omega)$ is non-negative, all the $\inner{e_j,C(\omega)e_j}$. Thus using the monotone convergence theorem,
\begin{dmath*}
\int_{\dual{\mathcal{X}}}\Tr\left[C(\omega)\right] d\dual{\Haar}(\omega)
       =\int_{\dual{\mathcal{X}}}\sum_{j\in\mathbb{N}}\inner{e_j,C(\omega)e_j}d\dual{\Haar}(\omega)
       =\sum_{k\in\mathbb{N}}\inner{e_j,K_e(e)e_j}
       =\Tr\left[K_e(e)\right]\hiderel{=}c_{\Tr}\hiderel{<}\infty.
\end{dmath*}
Let $A_{\Tr}(\omega)$ and $\rho_{\Tr}(\omega)$ be defined respectively as in \cref{eq:bounded_C} and \cref{eq:bounded_mu}. By definition, $\int_{\dual{\mathcal{X}}}\rho_{\Tr}(\omega)d\dual{\Haar}(\omega)=1$ and $A_{\Tr}(\omega)\rho_{\Tr}(\omega)=C(\omega)$. Now it remains to check the finiteness of $\Tr\left[C(\omega)\right]$ for all $\omega\in\dual{\mathcal{X}}$. Since for all $\omega\in\dual{\mathcal{X}}$, $\Tr\left[C(\omega)\right]\ge 0$,
\begin{dmath*}
\Tr\left[C(\omega)\right] \le \int_{\dual{\mathcal{X}}} \Tr\left[C(\omega)\right] d\dual{\Haar}(\omega) \hiderel{=} \Tr\left[K_e(e)\right] \hiderel{<} \infty.
\end{dmath*}
Since $\Tr\left[C(\omega)\right]$ is positive and its integral is finite, $\rho_{\Tr}$ is a probability density function. The Schatten norms $\norm{\cdot}_p$ verifies $\Tr\left[\abs{\cdot}\right]=\norm{\cdot}_1\ge\norm{\cdot}_p\ge \norm{\cdot}_q \ge \norm{\cdot}_{\mathcal{Y},\mathcal{Y}}=\norm{\cdot}_{\infty}$ for all $p$, $q\in\mathbb{N}$ such that $1\le p \le q$. Therefore since for all $\omega\in\dual{\mathcal{X}}$, $C(\omega)$ is non-negative,
\begin{dmath*}
\norm{A_{\Tr}(\omega)}_{\mathcal{Y},\mathcal{Y}}=c_{\Tr}\Tr\left[C(\omega)\right]^{-1}\norm{C(\omega)}_{\infty}
\le c_{\Tr}\Tr\left[C(\omega)\right]^{-1}\norm{C(\omega)}_{1}
=c_{\Tr}\Tr\left[C(\omega)\right]^{-1}\Tr\left[\abs{C(\omega)}\right]
=c_{\Tr}\Tr\left[C(\omega)\right]^{-1}\Tr\left[C(\omega)\right]
\le c_{\Tr} \hiderel{<} \infty.
\end{dmath*}
Thus $\sup_{\omega\in\dual{\mathcal{X}}}\norm{A(\omega)}_{\mathcal{Y},\mathcal{Y}} \le c_{\Tr} < \infty$. As $C$ is an even function, so are $A_{\Tr}$ and $\rho_{\Tr}$. Eventually $\inner{y', C(\cdot)y}$ is in $L^1(\dual{\mathcal{X}}, \dual{\Haar})$, thus $\inner{y, A_{\Tr}(\cdot)\rho_{\Tr}(\cdot)y'}$ is in $L^1(\dual{\mathcal{X}}, \dual{\Haar})$, hence $\inner{y, A_{\Tr}(\cdot)y'}\in L^1(\dual{\mathcal{X}}, \probability_{\dual{\Haar},\rho_{\Tr}})$. Since the trace is idenpendent of the basis of $\mathcal{Y}$, so is $\rho_{\Tr}$.
\end{proof}
If $\mathcal{Y}$ is finite dimensional then $\Tr\left[K_e(e)\right]$ is well defined hence \cref{pr:trace_measure} is valid as long as $K_e(\cdot)_{ij}\in L^1(\mathcal{X}, \Haar)$ for all $i$, $j\in\mathbb{N}_{p}$, where $p$ is the dimension of $\mathcal{Y}$.

\subsection{Examples of spectral decomposition}
\label{subsec:dec_examples}
In this section we give exemple of spectral decomposition of various $\mathcal{Y}$-Mercer kernel, based on \cref{pr:spectral} and \cref{pr:trace_measure}.
\subsubsection{Gaussian decomposable kernel}
\label{par:gaussian_dec}
Recall that a decomposable $\mathbb{R}^p$-Mercer has the form $K(x,z)=k(x,z)\Gamma$, where $k(x,z)$ is a scalar Mercer kernel and $\Gamma\in\mathcal{L}(\mathbb{R}^p)$ is a non-negative operator. Let $K^{dec,gauss}_e(\cdot)=k_e^{gauss}(\cdot)\Gamma$ be the Gaussian decomposable kernel where $K_e$ and $k_e$ are respectively the signature of $K$ and $k$ on the additive group $\mathcal{X}=(\mathbb{R}^d,+)$ -- $\ie~\delta=x-z$ and $e=0$. The scalar Gaussian kernel reads for all $\delta\in\mathbb{R}^d$
\begin{dmath*}
k^{\text{gauss}}_0(\delta)\hiderel{=}\exp\left( -\frac{1}{2\sigma^2}\norm{\delta}^2_2\right)
\end{dmath*}
where $\sigma \in \mathbb{R}_+$ is an hyperparameter corresponding to the bandwith of the kernel. The --Pontryagin-- dual group of $\mathcal{X}=(\mathbb{R}^d,+)$ is $\dual{\mathcal{X}}\cong(\mathbb{R}^d,+)$ with the pairing
\begin{dmath*}
\pairing{\delta,\omega}=\exp\left(\iu\inner{\delta,\omega}\right)
\end{dmath*}
where $\delta$ and $\omega\in\mathbb{R}^d$. In this case the Haar measures on $\mathcal{X}$ and $\dual{\mathcal{X}}$ are in both case the Lebesgue measure. However in order to have the property that $\IFT{\FT{f}}=f$ and $\IFT{f}=\mathcal{R}\FT{f}$ one must normalize both measures by $\sqrt{2\pi}^{-d}$, \ie~for all $\mathcal{Z}\in\mathcal{B}\left(\mathbb{R}^d\right)$,
\begin{dgroup*}
\begin{dmath*}
\sqrt{2\pi}^{d}\Haar(\mathcal{Z}) = \Leb(\mathcal{Z}) \text{ and}
\end{dmath*}
\begin{dmath*}
\sqrt{2\pi}^{d}\dual{\Haar}(\mathcal{Z}) = \Leb(\mathcal{Z}).
\end{dmath*}
\end{dgroup*}
Then the \acl{FT} on $(\mathbb{R}^d,+)$ is
\begin{dmath*}
\FT{f}(\omega)
=\int_{\mathbb{R}^d}\exp\left(-\iu\inner{\delta,\omega}\right)f(x)d\Haar(\delta)
=\int_{\mathbb{R}^d}\exp\left(-\iu\inner{\delta,\omega}\right)f(x)\frac{d\Leb(\delta)}{\sqrt{2\pi}^d}.
\end{dmath*}
Since $k^{\text{gauss}}_0\in L^1$ and $\Gamma$ is bounded, it is possible to apply \cref{pr:spectral}, and obtain for all $i$, $j\in\mathbb{N}_p$,
\begin{dmath*}
C^{dec,gauss}(\omega)_{ij}=\FT{K^{dec,gauss}_0(\cdot)_{ij}}(\omega)
=\FT{k_0^{gauss}}(\omega)\Gamma_{ij}
=\int_{\mathbb{R}^d}\exp\left(-\iu\inner{\omega,x}\right)\exp\left( -\frac{\norm{\delta}^2_2}{2\sigma^2}\right)\frac{d\Leb(\delta)}{\sqrt{2\pi}^d} \Gamma_{ij}
=\frac{1}{\sqrt{2\pi\frac{1}{\sigma^2}}^d}\exp\left( -\frac{\sigma^2}{2}\norm{\omega}^2_2\right)\sqrt{2\pi}^d\Gamma_{ij}.
\end{dmath*}
Hence
\begin{dmath*}
C^{dec,gauss}(\omega)
=\underbrace{\frac{1}{\sqrt{2\pi\frac{1}{\sigma^2}}^d}\exp\left( -\frac{\sigma^2}{2}\norm{\omega}^2_2\right)\sqrt{2\pi}^d}_{\rho(\cdot)
=\mathcal{N}(0,\sigma^{-2}I_d)\sqrt{2\pi}^d}\underbrace{\Gamma}_{A(\cdot)=\Gamma}
\end{dmath*}
Therefore the canonical decomposition of $C^{dec,gauss}$ is $A^{dec,gauss}(\omega)=\Gamma$ and $\rho^{dec,gauss}=\mathcal{N}(0,\sigma^{-2}I_d)\sqrt{2\pi}^d$, where $\mathcal{N}$ is the Gaussian probability distribution. Note that this decomposition is done with respect to the \emph{normalized} Lebesgue measure $\dual{\Haar}$, meaning that for all $\mathcal{Z}\in\mathcal{B}(\dual{\mathcal{X}})$,
\begin{dmath*}
\probability_{\dual{\Haar},\mathcal{N}(0,\sigma^{-2}I_d)\sqrt{2\pi}^d}(\mathcal{Z})=\int_{\mathcal{Z}}\mathcal{N}(0,\sigma^{-2}I_d)\sqrt{2\pi}^dd\dual{\Haar}(\omega)
=\int_{\dual{\mathcal{X}}}\mathcal{N}(0,\sigma^{-2}I_d)d\Leb(\omega)
=\probability_{\mathcal{N}(0,\sigma^{-2}I_d)}(\mathcal{Z})
\end{dmath*}
Thus, the same decomposition with respect to the usual --non-normalized-- Lebesgue measure $\Leb$ yields
\begin{dgroup}
\begin{dmath}
A^{dec,gauss}(\cdot)=\Gamma
\end{dmath}
\begin{dmath}
\rho^{dec,gauss}=\mathcal{N}(0,\sigma^{-2}I_d)
\end{dmath}
\end{dgroup}
If $\Gamma$ is a trace class operator, applying \cref{pr:trace_measure} yields the same decomposition since $\Tr\left[K^{dec,gauss}_0(0)\right]=\Tr\left[\Gamma\right]$ and $\Tr\left[C^{dec,gauss}(\cdot)\right]=\mathcal{N}(0,\sigma^{-2}I_d)\sqrt{2\pi}^d\Tr\left[\Gamma\right]$.

\subsubsection{Skewed-$\chi^2$ decomposable kernel}
\label{subsubsec:skewedchi2}
The skewed-$\chi^2$ scalar kernel is defined on the \acs{LCA} product group $\mathcal{X}=((-c_k;+\infty)_{k=1}^d,\odot)$, with $c_k\in\mathbb{R}_{+}$. Let $(e_k)_{k=1}^d$ be the standard basis of $\mathcal{X}$ and ${}_k:x\mapsto \inner{x,e_k}$. The operator $\odot: \mathcal{X}\times\mathcal{X}\to\mathcal{X}$ is defined by
\begin{dmath*}
x\odot z = \left((x_k + c_k)(z_k + c_k) - c_k\right)_{k=1}^d.
\end{dmath*}
The identity element $e$ is $\left(1-c_k\right)_{k=1}^d$ since $(1-c) \odot x = x$. Thus the inverse element $x^{-1}$ is $((x_k+c_k)^{-1} - c_k)_{k=1}^d$. The skewed-$\chi^2$ scalar kernel reads
\begin{dmath}
k^{skewed}_{1-c}(\delta)=\prod_{k=1}^d\frac{2}{\sqrt{\delta_k+c_k}+\sqrt{\frac{1}{\delta_k+c_k}}}.
\end{dmath}
The dual of $\mathcal{X}$ is $\dual{\mathcal{X}}\cong\mathbb{R}$ with the pairing
\begin{dmath*}
\pairing{\delta,\omega}=\prod_{k=1}^d\exp\left(\iu\inner{\log(\delta_k+c_k),\omega_k}\right).
\end{dmath*}
The Haar measure are defined for all $\mathcal{Z}\in\mathcal{B}((-c;+\infty)^d)$ and all $\dual{\mathcal{Z}}\in\mathcal{B}(\mathbb{R}^d)$ by
\begin{dgroup*}
\begin{dmath*}
\sqrt{2\pi}^d\Haar(\mathcal{Z})=\int_{\mathcal{Z}}\prod_{k=1}^d\frac{1}{z_k+c_k}d\Leb(z)
\end{dmath*}
\begin{dmath*}
\sqrt{2\pi}^d\dual{\Haar}(\dual{\mathcal{Z}})=\Leb(\dual{\mathcal{Z}}).
\end{dmath*}
\end{dgroup*}
Thus the \acl{FT} is
\begin{dmath*}
\FT{f}(\omega)=\int_{(-c;+\infty)^d}\prod_{k=1}^d\frac{\exp\left(-\iu\inner{\log(\delta_k+c_k),\omega_k}\right)}{t_k + c_k}f(\delta)\frac{d\Leb(\delta)}{\sqrt{2\pi}^d}.
\end{dmath*}
Then, applying Fubini's theorem over product space,
\begin{dmath*}
\FT{k_0^{skewed}}(\omega)=\prod_{k=1}^d\int_{-c_k}^{\infty}\frac{2\exp\left(-\iu\inner{\log(\delta_k+c_k),\omega_k}\right)}{(t_k + c_k)\left(\sqrt{\delta_k+c_k}+\sqrt{\frac{1}{\delta_k+c_k}}\right)}\frac{d\Leb(\delta_k)}{\sqrt{2\pi}^d}
=\prod_{k=1}^d\int_{-c_k}^{\infty} \frac{2\exp\left(-\iu\inner{\delta_k,\omega_k}\right)}{\exp\left(\frac{1}{2}\delta_k\right)+\exp\left(-\frac{1}{2}\delta_k\right)} \frac{d\Leb(\delta_k)}{\sqrt{2\pi}^d}
=\sqrt{2\pi}^d\prod_{k=1}^d\sech(\pi\omega_k)
\end{dmath*}
Since $k^{\text{skewed}}_{1-c}\in L^1$ and $\Gamma$ is bounded, it is possible to apply \cref{pr:spectral}, and obtain for all $i$, $j\in\mathbb{N}_p$,
\begin{dmath*}
C^{dec,skewed}(\omega)
=\FT{k_{1-c}^{skewed}}(\omega)\Gamma
=\underbrace{\sqrt{2\pi}^d\prod_{k=1}^d\sech(\pi\omega_k)}_{\rho(\cdot)=\mathcal{S}(0,2^{-1})^d\sqrt{2\pi}^d}\underbrace{\Gamma}_{A(\cdot)}
\end{dmath*}
Hence the decomposition with respect to the usual --non-normalized-- Lebesgue measure $\Leb$ yields
\begin{dgroup}
\begin{dmath}
A^{dec,skewed}(\cdot)=\Gamma
\end{dmath}
\begin{dmath}
\rho^{dec,skewed}=\mathcal{S}(0,2^{-1})^d
\end{dmath}
\end{dgroup}
\subsubsection{Curl-free Gaussian kernel} The curl-free Gaussian kernel is defined as $K^{curl,gauss}_0=-\nabla\nabla^T k_0^{gauss}$. Here $\mathcal{X}=(\mathbb{R}^d, +)$ so the setting is the same than \cref{par:gaussian_dec}.
\begin{dmath*}
C^{curl,gauss}(\omega)_{ij}=
\FT{K^{curl,gauss}_{1-c}(\cdot)_{ij}}(\omega)
=\FT{-\frac{d^2}{d\delta_id\delta_j}k^{gauss}_0}(\omega)
=-(\iu\omega_i)(\iu\omega_j)\FT{k_0^{gauss}}(\omega)
=\omega_i\omega_j\FT{k_0^{gauss}}(\omega)
=\sqrt{2\pi\frac{1}{\sigma^2}}^d\exp\left( -\frac{\sigma^2}{2}\norm{\omega}^2_2\right)\sqrt{2\pi}^d\omega_i\omega_j.
\end{dmath*}
Hence
\begin{dmath*}
C^{curl,gauss}(\omega)=\underbrace{\frac{1}{\sqrt{2\pi\frac{1}{\sigma^2}}^d}\exp\left( -\frac{\sigma^2}{2}\norm{\omega}^2_2\right)\sqrt{2\pi}^d}_{\mu(\cdot)=\mathcal{N}(0,\sigma^{-2}I_d)\sqrt{2\pi}^d}\underbrace{\omega\omega^T}_{A(\omega)=\omega\omega^T}.
\end{dmath*}
Here a canonical decomposition is $A^{curl,gauss}(\omega)=\omega\omega^T$ for all $\omega\in\mathbb{R}^d$ and $\mu^{curl,gauss}=\mathcal{N}(0,\sigma^{-2}I_d)\sqrt{2\pi}^d$ with respect to the normalized Lebesgue measure $d\omega$. Again the decomposition with respect to the usual --non-normalized-- Lebesgue measure is for all $\omega\in\mathbb{R}^d$
\begin{dgroup}
\begin{dmath}
A^{curl,gauss}(\omega)=\omega\omega^T
\end{dmath}
\begin{dmath}
\mu^{curl,gauss}=\mathcal{N}(0,\sigma^{-2}I_d)
\end{dmath}
\end{dgroup}
Notice that in this case $\norm{A^{curl,gauss}(\cdot)}_{\mathbb{R}^d,\mathbb{R}^d}$ is not bounded. However applying \cref{pr:trace_measure} yields a different decomposition where the quantity $\norm{A^{curl,gauss}_{\Tr}(\cdot)}_{\mathbb{R}^d,\mathbb{R}^d}$ is bounded. First we have for all $\delta\in\mathbb{R}^d$ and for all $i$, $j\in\mathbb{N}_d$
\begin{dmath*}
\frac{d^2}{d\delta_id\delta_j}k^{gauss}_0(\delta)=\frac{\exp\left(-\frac{1}{2\sigma^2}\norm{\delta}^2_2\right)}{\sigma^2}\begin{cases}
\frac{\delta_i\delta_j}{\sigma^2} & \text{if } i\neq j \\
\left(1-\frac{\delta_i\delta_j}{\sigma^2}\right) & \text{otherwise.}
\end{cases}
\end{dmath*}
Hence
\begin{dmath*}
-\nabla\nabla^Tk^{gauss}_0(\delta)=\left(I_d -\frac{\delta\delta^T}{\sigma^2} \right)\frac{\exp\left(-\frac{1}{2\sigma^2}\norm{\delta}^2_2\right)}{\sigma^2}
\end{dmath*}
Thus $\Tr\left[K^{curl,gauss}_0(0)\right]=\Tr\left[\nabla\nabla^Tk^{gauss}_0(0)\right]=d\sigma^{-2}$ and $\Tr\left[C(\omega)\right]=\norm{\omega}_2^2\mathcal{N}(0,\sigma^{-2}I_d)\sqrt{2\pi}^d$. Apply \cref{pr:trace_measure} to obtain the decomposition $A^{curl,gauss}_{\Tr}(\omega)=\omega\omega^T\norm{\omega}_2^{-2}$ and the measure $\mu^{curl,gauss}_{\Tr}(\omega)=\sigma^2d^{-1}\norm{\omega}_2^2\mathcal{N}\left(0,\sigma^{-2}\right)\sqrt{2\pi}^d$ for all $\omega\in\mathbb{R}^d$, with respect to the normalized Lebesgue measure. Therefore the decomposition with respect to the usual non-normalized Lebesgue measure is
\begin{dgroup}
\begin{dmath}
A^{curl,gauss}_{\Tr}(\omega)=\frac{\omega\omega^T}{\norm{\omega}_2^{2}}
\end{dmath}
\begin{dmath}
\mu^{curl,gauss}_{\Tr}(\omega)=\frac{\sigma^2}{d}\norm{\omega}_2^2\mathcal{N}\left(0,\sigma^{-2}\right)(\omega)
\end{dmath}
\end{dgroup}
This example also illustrate that there exist many decomposition of $C(\omega)$ into $(A(\omega),\mu(\omega))$.
\subsubsection{Divergence-free kernel}
The divegence-free Gaussian kernel is defined as $K^{div,gauss}_0=(\nabla\nabla^T-\Delta)k_0^{gauss}$ on the group $\mathcal{X}=(\mathbb{R}^d, +)$. The setting is the same than \cref{par:gaussian_dec}. Hence
\begin{dmath*}
C^{div,gauss}(\omega)_{ij}=
\FT{K^{div,gauss}_0(\cdot)_{ij}}(\omega)
=\FT{\frac{d^2}{d\delta_id\delta_j}k^{gauss}_0-\delta_{i=j}\sum_{k=1}^d\frac{d^2}{d\delta_kd\delta_k}k^{gauss}_0}(\omega)
=\left(-(\iu\omega_i)(\iu\omega_j)-\delta_{i=j}\sum_{k=1}^d(\iu\omega_k)^2\right)\FT{k_0^{gauss}}=\left(\delta_{i=j}\sum_{k=1}^d\omega_k^2-\omega_i\omega_j\right)\FT{k_0^{gauss}}(\omega).
\end{dmath*}
Hence
\begin{dmath*}
C^{div,gauss}(\omega)=\underbrace{\frac{1}{\sqrt{2\pi\frac{1}{\sigma^2}}^d}\exp\left( -\frac{\sigma^2}{2}\norm{\omega}^2_2\right)\sqrt{2\pi}^d}_{\rho(\cdot)=\mathcal{N}(0,\sigma^{-2}I_d)\sqrt{2\pi}^d}\underbrace{\left(I_d\norm{\omega}_2^2-\omega\omega^T\right)}_{A(\omega)=I_d\norm{\omega}_2^2-\omega\omega^T}.
\end{dmath*}
Thus the canonical decomposition with respect to the normalized Lebesgue measure is $A^{div,gauss}(\omega)=I_d\norm{\omega}_2^2-\omega\omega^T$ and the measure $\rho^{div,gauss}=\mathcal{N}(0,\sigma^{-2}I_d)\sqrt{2\pi}^d$. The canonical decomposition with respect to the usual Lebesgue measure is
\begin{dgroup}
\begin{dmath}
A^{div,gauss}(\omega)=I_d\norm{\omega}_2^2-\omega\omega^T
\end{dmath}
\begin{dmath}
\rho^{div,gauss}=\mathcal{N}(0,\sigma^{-2}I_d).
\end{dmath}
\end{dgroup}
To obtain the bounded decomposition, again, apply \cref{pr:trace_measure}. For all $\delta\in\mathbb{R}^d$,
\begin{dmath*}
\sum_{k=1}^d\frac{d^2}{d\delta_k d\delta_k}k^{gauss}_0(\delta)=
\left(d-\frac{\norm{\delta}_2^2}{\sigma^2}\right)\frac{\exp\left(-\frac{1}{2\sigma^2}\norm{\delta}^2_2\right)}{\sigma^2}.
\end{dmath*}
Thus overall,
\begin{dmath*}
K^{div,gauss}_0(\delta)=\left(\frac{\delta\delta^T}{\sigma^2} + \left((d-1) - \frac{\norm{\delta}_2^2}{\sigma^2}\right)I_d \right)\frac{\exp\left(-\frac{1}{2\sigma^2}\norm{\delta}^2_2\right)}{\sigma^2},
\end{dmath*}
Eventually $\Tr\left[K^{div,gauss}_0(0)\right]=\Tr\left[(\nabla\nabla^T-\Delta)k^{gauss}_0(0)\right]=d(d-1)\sigma^{-2}$ and $\Tr\left[C(\omega)\right]=(d-1)\norm{\omega}_2^2\mathcal{N}(0,\sigma^2I_d)\sqrt{2\pi}^d$. As a result the decomposition with respect to the normalized Lebesgue measure is $A^{div,gauss}_{\Tr}(\omega)=(I_d-\omega\omega^T\norm{\omega}_2^{-2})$ and $\rho^{div,gauss}_{\Tr}(\omega)=d^{-1}\sigma^2\norm{\omega}_2^2\mathcal{N}(0,\sigma^2I_d)\sqrt{2\pi}^d$. The decomposition with respect to the normalized Lebesgue measure being
\begin{dgroup}
\begin{dmath}
A^{div,gauss}_{\Tr}(\omega)=I_d-\frac{\omega\omega^T}{\norm{\omega}_2^2}
\end{dmath}
\begin{dmath}
\rho^{div,gauss}_{\Tr}=\frac{\sigma^2}{d}\norm{\omega}_2^2\mathcal{N}(0,\sigma^{-2}I_d).
\end{dmath}
\end{dgroup}

\subsection{Functional Fourier feature map}
Let us introduce a functional feature map, we call here \emph{Fourier Feature map}, defined by the following proposition as a direct consequence of \cref{pr:mercer_kernel_bochner}.

\begin{proposition}[Functional Fourier feature map]\label{pr:fourier_feature_map}
Let $\mathcal{Y}$ and $\mathcal{Y}'$ be two Hilbert spaces. If there exist an operator-valued function $B:\dual{\mathcal{X}}\to\mathcal{L}(\mathcal{Y},\mathcal{Y}')$ such that for all $y$, $y'\in\mathcal{Y}$,
\begin{dmath*}
\inner{y, B(\omega)B(\omega)^\adjoint y'}_{\mathcal{Y}}=\inner{y', A(\omega)y}_{\mathcal{Y}}
\end{dmath*}
$\dual{\mu}$-almost everywhere and $\inner{y', A(\cdot)y}\in L^1(\dual{\mathcal{X}},\dual{\mu})$ then the operator $\Phi_x$ defined for all $y$ in $\mathcal{Y}$ by
\begin{dmath}
\label{eq:feature_shiftinv_map}
(\Phi_x y)(\omega)=\pairing{x,\omega}B(\omega)^\adjoint y,
\end{dmath}
is \emph{a feature map}\mpar{\Ie~it satisfies for all $x$, $z \in \mathcal{X}$, $\Phi_x^\adjoint \Phi_z=K(x,z)$ where $K$ is a $\mathcal{Y}$-Mercer \acs{OVK}.} of some shift-invariant $\mathcal{Y}$-Mercer kernel $K$.
\end{proposition}
\begin{proof}
For all $y$, $y'\in \mathcal{Y}$ and $x$, $z\in\mathcal{X}$,
\begin{dmath*}
\inner{y, \Phi_x^\adjoint \Phi_z y'}_{\mathcal{Y}} = \inner{\Phi_x y, \Phi_z y'}_{L^2(\dual{\mathcal{X}},\dual{\mu};\mathcal{Y}')}  \\
= \int_{\dual{\mathcal{X}}}\conj{\pairing{x,\omega}}\inner{y, B(\omega)\pairing{z,\omega}B(\omega)^\adjoint y'}d\dual{\mu}(\omega) \\
= \int_{\dual{\mathcal{X}}}\conj{\pairing{x \groupop \myinv{z},\omega}}\inner{y B(\omega)B(\omega)^\adjoint y'}d\dual{\mu}(\omega) \\
= \int_{\dual{\mathcal{X}}}\conj{\pairing{x \groupop \inv{z},\omega}}\inner{y,A(\omega)y'}d\dual{\mu}(\omega),
\end{dmath*}
which defines a $\mathcal{Y}$-Mercer according to \cref{pr:mercer_kernel_bochner} of \citet{Carmeli2010}.
\end{proof}
With this notation notice that $\Phi: \mathcal{X} \to \mathcal{L}(\mathcal{Y}; L^2(\dual{\mathcal{X}}, \dual{\mu}; \mathcal{Y}'))$ such that $\Phi_x\in \mathcal{L}(\mathcal{Y}; L^2(\dual{\mathcal{X}}, \dual{\mu}; \mathcal{Y}'))$ where $\Phi_x\colonequals\Phi(x)$.

\subsection{Building Operator-valued Random Fourier Features}
\label{subsec:building_ORFF}
As shown in \cref{pr:spectral,pr:trace_measure} it is always possible to find a pair $(A, \probability_{\dual{\Haar},\rho})$ from a shift invariant $\mathcal{Y}$-Mercer \acl{OVK} $K_e$ such that $\probability_{\dual{\Haar},\rho}$ is a probability measure --\ie~$\int_{\dual{\mathcal{X}}} \rho d\dual{\Haar}=1$ where $\rho$ is the density of $\probability_{\dual{\Haar},\rho}$-- and $K_e(\delta)=\expectation_\rho{\conj{\pairing{\delta,\omega}}A(\omega)}$. In order to obtain an approximation of $K$ from a decomposition $(A, \probability_{\dual{\Haar},\rho})$ we turn our attention to a Monte-Carlo estimation of the expectations \cref{eq:expectation_tr} and \cref{eq:expectation_spec} characterizing a $\mathcal{Y}$-Mercer shift-invariant \acl{OVK}.
\paragraph{}
However, for efficient computations as motivated in the introduction, we are interested in finding an approximated \emph{feature map} instead of a kernel approximation. Indeed, an approximated feature map will allow to build linear models in regression tasks. The idea is to start from the Monte-Carlo approximation of the expectation and provide a systematic decomposition of the Monte-Carlo sample mean into an approximate feature map. The following proposition provides the general form of an \acl{ORFF}.
\begin{proposition}[ORFF]
\label{pr:ORFF-map}
Let $\mathcal{Y}$ and $\mathcal{Y}'$ be two Hilbert spaces. If one can find $B: \dual{\mathcal{X}} \to \mathcal{L}(\mathcal{Y},\mathcal{Y}')$ and a probability measure $\probability_{\dual{\Haar},\rho}$ on $\mathcal{B}(\dual{\mathcal{X}})$, such that for all $y\in\mathcal{Y}$ and all $y'\in\mathcal{Y}'$, $\inner{y, B(\cdot)y'} \in L^2(\dual{\mathcal{X}}, \probability_{\dual{\Haar},\rho})$, then the operator-valued function given for all $y\in\mathcal{Y}$ by
\begin{equation}
\label{eq:phitilde}
\tildePhi{\omega}(x)y= \frac{1}{\sqrt{D}}\Vect_{j=1}^D\pairing{x, \omega_j}B(\omega_j)^\adjoint y \condition{$\omega_j \sim \probability_{\dual{\Haar},\rho}$ \iid,}
\end{equation}
is an approximated feature map\mpar{\Ie~it satisfies $\tildePhi{\omega}(x)^\adjoint \tildePhi{\omega}(z)\converges{\asurely}{D\to\infty}K(x,z)$ where $K$ is a $\mathcal{Y}$-Mercer \acs{OVK}.} of a shift-invariant $\mathcal{Y}$-Mercer \acl{OVK}.
\end{proposition}
\begin{proof}
Let $(\omega_j)_{j=1}^D$ be a sequence of $D\in\mathbb{N}$ \iid~random vectors, each of them following the law $\probability_{\dual{\Haar},\rho}$. For all $x$, $z \in \mathcal{X}$ and all $y$, $y' \in \mathcal{Y}$,
\begin{dmath*}
\inner*{\tildePhi{\omega}(x)y',\tildePhi{\omega}(z)y}=\frac{1}{D}\inner*{\Vect_{j=1}^D\pairing{x, \omega_j}B(\omega_j)^\adjoint y',  \Vect_{j=1}^D\pairing{z, \omega_j}B(\omega_j)^\adjoint y}
= \frac{1}{D} \sum_{j=1}^D \inner*{y', \conj{\pairing{x, \omega_j}}B(\omega_j)\pairing{z, \omega_j}B(\omega_j)^\adjoint y}_{\mathcal{Y}}
= \inner*{y', \left(\frac{1}{D} \sum_{j=1}^D \conj{\pairing{x\groupop \inv{z}, \omega_j}}A(\omega_j)\right)y}_{\mathcal{Y}},
\end{dmath*}
where $A(\omega)=B(\omega)B(\omega)^\adjoint $. By assumption $\inner{y, A(\cdot)y'}\in L^1(\dual{\mathcal{X}},\probability_{\dual{\Haar},\rho})$ and $\omega_j$ are \iid~. Hence from the strong law of large numbers and \cref{pr:mercer_kernel_bochner} with $\dual{\mu}=\probability_{\dual{\Haar},\rho}$,
\begin{dmath*}
\frac{1}{D} \sum_{j=1}^D \conj{\pairing{x\groupop\inv{z},\omega_j}}A(\omega_j)\converges{\asurely}{D\to\infty}\expectation_{\rho}[\conj{\pairing{x\groupop z^{-1},\omega_j}}A(\omega)]=K_e(x\groupop\inv{z})
\end{dmath*}
where the integral converges in the weak operator topology.
\end{proof}
\begin{remark}
The approximate feature map proposed in \cref{pr:ORFF-map} has direct link with the functional Fourier feature map defined in \cref{pr:fourier_feature_map} since we have for all $y\in\mathcal{Y}$
\begin{dmath}
\tildePhi{\omega}(x)y = \frac{1}{\sqrt{D}}\Vect_{j=1}^D\pairing{x, \omega_j}B(\omega_j)^\adjoint y \condition{$\omega_j \hiderel{\sim} \probability_{\dual{\Haar},\rho}$ \iid}
= \frac{1}{\sqrt{D}} \Vect_{j=1}^D (\Phi_x y)(\omega_j).
\end{dmath}
\end{remark}
Therefore $\tildePhi{\omega}(x)$ can be seen as an \say{operator-valued vector} corresponding the \say{stacking} of $D$ \iid~operator-valued random variable $\pairing{x, \omega_j}B(\omega_j)^\adjoint$. Note that we consider $\omega$ to be a \emph{variable} in $\dual{\mathcal{X}}$ while $\omega_j$ are $\dual{\mathcal{X}}$-valued \emph{random variables}. \Ie~$\omega_j$ are $(\dual{\mathcal{X}}, \mathcal{B}(\dual{\mathcal{X}}))$-valued measurable function and $\probability_{\dual{\Haar},\rho}$ is the distribution of each $\omega_j$. Let
\begin{dmath*}
\tildeH{\omega} = \Vect_{j=1}^D\mathcal{Y}',
\end{dmath*}
be a Hilbert space. Let $\seq{\omega}\in\dual{\mathcal{X}}^D$ be a sequence where $\seq{\omega}=(\omega_j)_{j=1}^D$ (\ie~ an outcome of the random sequence $(\omega_j)_{j=1}^D$, $\omega_j \sim \probability_{\dual{\Haar},\rho}$). Notice that for any $x\in\mathcal{X}$ and $y\in\mathcal{Y}$, $\tildePhi{\omega}(x)y\in\tildeH{\omega}$ as we can take $g(\omega_j)=\pairing{x,\omega_j}B(\omega_j)^\adjoint y$. Thus we have for all $y$, $y'\in\mathcal{Y}$ and all $x$, $z\in\mathcal{X}$
\begin{dmath*}
\inner{y', \tildeK{\omega}(x, z)y}_{\mathcal{Y}} \hiderel{=} \inner{y', \tildePhi{\omega}(x)^\adjoint \tildePhi{\omega}(z)y}_{\mathcal{Y}} \hiderel{=} \inner{\tildePhi{\omega}(x)y', \tildePhi{\omega}(z)y}_{\tildeH{\omega}}.
\end{dmath*}
Thus $\tildeK{\omega}(x,z)=\tildePhi{\omega}(x)^\adjoint \tildePhi{\omega}(z)$ is a proper random shift-invariant \acl{OVK} as shown in the following proposition.
\begin{proposition} Let $\seq{\omega}\in\dual{\mathcal{X}}^D$. If for all $y$, $y'\in\mathcal{Y}$
\begin{dmath*}
\inner{y', \tildeK{\omega}_e\left(x\groupop z^{-1}\right)y}_{\mathcal{Y}}=\inner{\tildePhi{\omega}(x)y', \tildePhi{\omega}(z)y}_{\tildeH{\omega}}
=\inner*{y', \frac{1}{D}\sum_{j=1}^D \conj{\pairing{x\groupop z^{-1},\omega_j}}B(\omega_j)B(\omega_j)^*y}_{\mathcal{Y}},
\end{dmath*}
for all $x$,  $z\in\mathcal{X}$, then $\tildeK{\omega}$ is a shift-invariant \acl{OVK}.
\end{proposition}
\begin{proof} Apply \cref{pr:feature_operator} to $\tildePhi{\omega}$ considering the Hilbert space $\tildeH{\omega}$ to show that $\tildeK{\omega}$ is an \acs{OVK}. Then \cref{pr:kernel_signature} shows that $\tildeK{\omega}$ is shift-invariant since $\tildeK{\omega}(x,z)=\tildeK{\omega}_e\left(x,\groupop z^{-1}\right)$.
\end{proof}
We stress out that if $\seq{\omega}=(\omega_j)_{j=1}^D\sim \probability_{\dual{\Haar},\rho}$ \iid~is a \emph{random sequence} then
\begin{dmath*}
\tildeK{\omega}_e\left(x\groupop z^{-1}\right)=\tildePhi{\omega}(x)^{\adjoint}\tildePhi{\omega}(z)
\end{dmath*}
is not \emph{sensus} stricto an \acl{OVK} since it is a random variable (and $\tildeH{\omega}$ is no longer a Hilbert space, since its inner product is a then random variable and not a scalar). However this is not a problem since any outcome of the random sequence $\seq{\omega}$ gives birth to a (different) \acl{OVK}, and $\expectation_{\dual{\Haar},\rho}\tildeK{\omega}$ is an \acs{OVK}. This illustrated by \cref{fig:not_Mercer} where we represented the same function for different realization of $\tildeK{\omega}\approx K$. We generated $250$ points equally separated on the segment $(-1;1)$.

\begin{pycode}[not_mercer]
sys.path.append('../src/')
import not_mercer

not_mercer.main()
\end{pycode}

\begin{figure}[htb]
\pyc{print(r'\centering\resizebox{\textwidth}{!}{%% Creator: Matplotlib, PGF backend
%%
%% To include the figure in your LaTeX document, write
%%   \input{<filename>.pgf}
%%
%% Make sure the required packages are loaded in your preamble
%%   \usepackage{pgf}
%%
%% Figures using additional raster images can only be included by \input if
%% they are in the same directory as the main LaTeX file. For loading figures
%% from other directories you can use the `import` package
%%   \usepackage{import}
%% and then include the figures with
%%   \import{<path to file>}{<filename>.pgf}
%%
%% Matplotlib used the following preamble
%%   \usepackage{fontspec}
%%   \setmainfont{DejaVu Serif}
%%   \setsansfont{DejaVu Sans}
%%   \setmonofont{DejaVu Sans Mono}
%%
\begingroup%
\makeatletter%
\begin{pgfpicture}%
\pgfpathrectangle{\pgfpointorigin}{\pgfqpoint{10.882492in}{7.151564in}}%
\pgfusepath{use as bounding box, clip}%
\begin{pgfscope}%
\pgfsetbuttcap%
\pgfsetmiterjoin%
\definecolor{currentfill}{rgb}{1.000000,1.000000,1.000000}%
\pgfsetfillcolor{currentfill}%
\pgfsetlinewidth{0.000000pt}%
\definecolor{currentstroke}{rgb}{1.000000,1.000000,1.000000}%
\pgfsetstrokecolor{currentstroke}%
\pgfsetdash{}{0pt}%
\pgfpathmoveto{\pgfqpoint{0.000000in}{0.000000in}}%
\pgfpathlineto{\pgfqpoint{10.882492in}{0.000000in}}%
\pgfpathlineto{\pgfqpoint{10.882492in}{7.151564in}}%
\pgfpathlineto{\pgfqpoint{0.000000in}{7.151564in}}%
\pgfpathclose%
\pgfusepath{fill}%
\end{pgfscope}%
\begin{pgfscope}%
\pgfsetbuttcap%
\pgfsetmiterjoin%
\definecolor{currentfill}{rgb}{1.000000,1.000000,1.000000}%
\pgfsetfillcolor{currentfill}%
\pgfsetlinewidth{0.000000pt}%
\definecolor{currentstroke}{rgb}{0.000000,0.000000,0.000000}%
\pgfsetstrokecolor{currentstroke}%
\pgfsetstrokeopacity{0.000000}%
\pgfsetdash{}{0pt}%
\pgfpathmoveto{\pgfqpoint{0.634105in}{3.881603in}}%
\pgfpathlineto{\pgfqpoint{4.861378in}{3.881603in}}%
\pgfpathlineto{\pgfqpoint{4.861378in}{6.681603in}}%
\pgfpathlineto{\pgfqpoint{0.634105in}{6.681603in}}%
\pgfpathclose%
\pgfusepath{fill}%
\end{pgfscope}%
\begin{pgfscope}%
\pgfsetbuttcap%
\pgfsetroundjoin%
\definecolor{currentfill}{rgb}{0.000000,0.000000,0.000000}%
\pgfsetfillcolor{currentfill}%
\pgfsetlinewidth{0.803000pt}%
\definecolor{currentstroke}{rgb}{0.000000,0.000000,0.000000}%
\pgfsetstrokecolor{currentstroke}%
\pgfsetdash{}{0pt}%
\pgfsys@defobject{currentmarker}{\pgfqpoint{0.000000in}{-0.048611in}}{\pgfqpoint{0.000000in}{0.000000in}}{%
\pgfpathmoveto{\pgfqpoint{0.000000in}{0.000000in}}%
\pgfpathlineto{\pgfqpoint{0.000000in}{-0.048611in}}%
\pgfusepath{stroke,fill}%
}%
\begin{pgfscope}%
\pgfsys@transformshift{0.826254in}{3.881603in}%
\pgfsys@useobject{currentmarker}{}%
\end{pgfscope}%
\end{pgfscope}%
\begin{pgfscope}%
\pgfsetbuttcap%
\pgfsetroundjoin%
\definecolor{currentfill}{rgb}{0.000000,0.000000,0.000000}%
\pgfsetfillcolor{currentfill}%
\pgfsetlinewidth{0.803000pt}%
\definecolor{currentstroke}{rgb}{0.000000,0.000000,0.000000}%
\pgfsetstrokecolor{currentstroke}%
\pgfsetdash{}{0pt}%
\pgfsys@defobject{currentmarker}{\pgfqpoint{0.000000in}{-0.048611in}}{\pgfqpoint{0.000000in}{0.000000in}}{%
\pgfpathmoveto{\pgfqpoint{0.000000in}{0.000000in}}%
\pgfpathlineto{\pgfqpoint{0.000000in}{-0.048611in}}%
\pgfusepath{stroke,fill}%
}%
\begin{pgfscope}%
\pgfsys@transformshift{1.306626in}{3.881603in}%
\pgfsys@useobject{currentmarker}{}%
\end{pgfscope}%
\end{pgfscope}%
\begin{pgfscope}%
\pgfsetbuttcap%
\pgfsetroundjoin%
\definecolor{currentfill}{rgb}{0.000000,0.000000,0.000000}%
\pgfsetfillcolor{currentfill}%
\pgfsetlinewidth{0.803000pt}%
\definecolor{currentstroke}{rgb}{0.000000,0.000000,0.000000}%
\pgfsetstrokecolor{currentstroke}%
\pgfsetdash{}{0pt}%
\pgfsys@defobject{currentmarker}{\pgfqpoint{0.000000in}{-0.048611in}}{\pgfqpoint{0.000000in}{0.000000in}}{%
\pgfpathmoveto{\pgfqpoint{0.000000in}{0.000000in}}%
\pgfpathlineto{\pgfqpoint{0.000000in}{-0.048611in}}%
\pgfusepath{stroke,fill}%
}%
\begin{pgfscope}%
\pgfsys@transformshift{1.786997in}{3.881603in}%
\pgfsys@useobject{currentmarker}{}%
\end{pgfscope}%
\end{pgfscope}%
\begin{pgfscope}%
\pgfsetbuttcap%
\pgfsetroundjoin%
\definecolor{currentfill}{rgb}{0.000000,0.000000,0.000000}%
\pgfsetfillcolor{currentfill}%
\pgfsetlinewidth{0.803000pt}%
\definecolor{currentstroke}{rgb}{0.000000,0.000000,0.000000}%
\pgfsetstrokecolor{currentstroke}%
\pgfsetdash{}{0pt}%
\pgfsys@defobject{currentmarker}{\pgfqpoint{0.000000in}{-0.048611in}}{\pgfqpoint{0.000000in}{0.000000in}}{%
\pgfpathmoveto{\pgfqpoint{0.000000in}{0.000000in}}%
\pgfpathlineto{\pgfqpoint{0.000000in}{-0.048611in}}%
\pgfusepath{stroke,fill}%
}%
\begin{pgfscope}%
\pgfsys@transformshift{2.267369in}{3.881603in}%
\pgfsys@useobject{currentmarker}{}%
\end{pgfscope}%
\end{pgfscope}%
\begin{pgfscope}%
\pgfsetbuttcap%
\pgfsetroundjoin%
\definecolor{currentfill}{rgb}{0.000000,0.000000,0.000000}%
\pgfsetfillcolor{currentfill}%
\pgfsetlinewidth{0.803000pt}%
\definecolor{currentstroke}{rgb}{0.000000,0.000000,0.000000}%
\pgfsetstrokecolor{currentstroke}%
\pgfsetdash{}{0pt}%
\pgfsys@defobject{currentmarker}{\pgfqpoint{0.000000in}{-0.048611in}}{\pgfqpoint{0.000000in}{0.000000in}}{%
\pgfpathmoveto{\pgfqpoint{0.000000in}{0.000000in}}%
\pgfpathlineto{\pgfqpoint{0.000000in}{-0.048611in}}%
\pgfusepath{stroke,fill}%
}%
\begin{pgfscope}%
\pgfsys@transformshift{2.747741in}{3.881603in}%
\pgfsys@useobject{currentmarker}{}%
\end{pgfscope}%
\end{pgfscope}%
\begin{pgfscope}%
\pgfsetbuttcap%
\pgfsetroundjoin%
\definecolor{currentfill}{rgb}{0.000000,0.000000,0.000000}%
\pgfsetfillcolor{currentfill}%
\pgfsetlinewidth{0.803000pt}%
\definecolor{currentstroke}{rgb}{0.000000,0.000000,0.000000}%
\pgfsetstrokecolor{currentstroke}%
\pgfsetdash{}{0pt}%
\pgfsys@defobject{currentmarker}{\pgfqpoint{0.000000in}{-0.048611in}}{\pgfqpoint{0.000000in}{0.000000in}}{%
\pgfpathmoveto{\pgfqpoint{0.000000in}{0.000000in}}%
\pgfpathlineto{\pgfqpoint{0.000000in}{-0.048611in}}%
\pgfusepath{stroke,fill}%
}%
\begin{pgfscope}%
\pgfsys@transformshift{3.228113in}{3.881603in}%
\pgfsys@useobject{currentmarker}{}%
\end{pgfscope}%
\end{pgfscope}%
\begin{pgfscope}%
\pgfsetbuttcap%
\pgfsetroundjoin%
\definecolor{currentfill}{rgb}{0.000000,0.000000,0.000000}%
\pgfsetfillcolor{currentfill}%
\pgfsetlinewidth{0.803000pt}%
\definecolor{currentstroke}{rgb}{0.000000,0.000000,0.000000}%
\pgfsetstrokecolor{currentstroke}%
\pgfsetdash{}{0pt}%
\pgfsys@defobject{currentmarker}{\pgfqpoint{0.000000in}{-0.048611in}}{\pgfqpoint{0.000000in}{0.000000in}}{%
\pgfpathmoveto{\pgfqpoint{0.000000in}{0.000000in}}%
\pgfpathlineto{\pgfqpoint{0.000000in}{-0.048611in}}%
\pgfusepath{stroke,fill}%
}%
\begin{pgfscope}%
\pgfsys@transformshift{3.708485in}{3.881603in}%
\pgfsys@useobject{currentmarker}{}%
\end{pgfscope}%
\end{pgfscope}%
\begin{pgfscope}%
\pgfsetbuttcap%
\pgfsetroundjoin%
\definecolor{currentfill}{rgb}{0.000000,0.000000,0.000000}%
\pgfsetfillcolor{currentfill}%
\pgfsetlinewidth{0.803000pt}%
\definecolor{currentstroke}{rgb}{0.000000,0.000000,0.000000}%
\pgfsetstrokecolor{currentstroke}%
\pgfsetdash{}{0pt}%
\pgfsys@defobject{currentmarker}{\pgfqpoint{0.000000in}{-0.048611in}}{\pgfqpoint{0.000000in}{0.000000in}}{%
\pgfpathmoveto{\pgfqpoint{0.000000in}{0.000000in}}%
\pgfpathlineto{\pgfqpoint{0.000000in}{-0.048611in}}%
\pgfusepath{stroke,fill}%
}%
\begin{pgfscope}%
\pgfsys@transformshift{4.188857in}{3.881603in}%
\pgfsys@useobject{currentmarker}{}%
\end{pgfscope}%
\end{pgfscope}%
\begin{pgfscope}%
\pgfsetbuttcap%
\pgfsetroundjoin%
\definecolor{currentfill}{rgb}{0.000000,0.000000,0.000000}%
\pgfsetfillcolor{currentfill}%
\pgfsetlinewidth{0.803000pt}%
\definecolor{currentstroke}{rgb}{0.000000,0.000000,0.000000}%
\pgfsetstrokecolor{currentstroke}%
\pgfsetdash{}{0pt}%
\pgfsys@defobject{currentmarker}{\pgfqpoint{0.000000in}{-0.048611in}}{\pgfqpoint{0.000000in}{0.000000in}}{%
\pgfpathmoveto{\pgfqpoint{0.000000in}{0.000000in}}%
\pgfpathlineto{\pgfqpoint{0.000000in}{-0.048611in}}%
\pgfusepath{stroke,fill}%
}%
\begin{pgfscope}%
\pgfsys@transformshift{4.669229in}{3.881603in}%
\pgfsys@useobject{currentmarker}{}%
\end{pgfscope}%
\end{pgfscope}%
\begin{pgfscope}%
\pgfsetbuttcap%
\pgfsetroundjoin%
\definecolor{currentfill}{rgb}{0.000000,0.000000,0.000000}%
\pgfsetfillcolor{currentfill}%
\pgfsetlinewidth{0.803000pt}%
\definecolor{currentstroke}{rgb}{0.000000,0.000000,0.000000}%
\pgfsetstrokecolor{currentstroke}%
\pgfsetdash{}{0pt}%
\pgfsys@defobject{currentmarker}{\pgfqpoint{-0.048611in}{0.000000in}}{\pgfqpoint{0.000000in}{0.000000in}}{%
\pgfpathmoveto{\pgfqpoint{0.000000in}{0.000000in}}%
\pgfpathlineto{\pgfqpoint{-0.048611in}{0.000000in}}%
\pgfusepath{stroke,fill}%
}%
\begin{pgfscope}%
\pgfsys@transformshift{0.634105in}{4.335954in}%
\pgfsys@useobject{currentmarker}{}%
\end{pgfscope}%
\end{pgfscope}%
\begin{pgfscope}%
\pgftext[x=0.289968in,y=4.283193in,left,base]{\rmfamily\fontsize{10.000000}{12.000000}\selectfont \(\displaystyle -20\)}%
\end{pgfscope}%
\begin{pgfscope}%
\pgfsetbuttcap%
\pgfsetroundjoin%
\definecolor{currentfill}{rgb}{0.000000,0.000000,0.000000}%
\pgfsetfillcolor{currentfill}%
\pgfsetlinewidth{0.803000pt}%
\definecolor{currentstroke}{rgb}{0.000000,0.000000,0.000000}%
\pgfsetstrokecolor{currentstroke}%
\pgfsetdash{}{0pt}%
\pgfsys@defobject{currentmarker}{\pgfqpoint{-0.048611in}{0.000000in}}{\pgfqpoint{0.000000in}{0.000000in}}{%
\pgfpathmoveto{\pgfqpoint{0.000000in}{0.000000in}}%
\pgfpathlineto{\pgfqpoint{-0.048611in}{0.000000in}}%
\pgfusepath{stroke,fill}%
}%
\begin{pgfscope}%
\pgfsys@transformshift{0.634105in}{4.805404in}%
\pgfsys@useobject{currentmarker}{}%
\end{pgfscope}%
\end{pgfscope}%
\begin{pgfscope}%
\pgftext[x=0.289968in,y=4.752642in,left,base]{\rmfamily\fontsize{10.000000}{12.000000}\selectfont \(\displaystyle -10\)}%
\end{pgfscope}%
\begin{pgfscope}%
\pgfsetbuttcap%
\pgfsetroundjoin%
\definecolor{currentfill}{rgb}{0.000000,0.000000,0.000000}%
\pgfsetfillcolor{currentfill}%
\pgfsetlinewidth{0.803000pt}%
\definecolor{currentstroke}{rgb}{0.000000,0.000000,0.000000}%
\pgfsetstrokecolor{currentstroke}%
\pgfsetdash{}{0pt}%
\pgfsys@defobject{currentmarker}{\pgfqpoint{-0.048611in}{0.000000in}}{\pgfqpoint{0.000000in}{0.000000in}}{%
\pgfpathmoveto{\pgfqpoint{0.000000in}{0.000000in}}%
\pgfpathlineto{\pgfqpoint{-0.048611in}{0.000000in}}%
\pgfusepath{stroke,fill}%
}%
\begin{pgfscope}%
\pgfsys@transformshift{0.634105in}{5.274853in}%
\pgfsys@useobject{currentmarker}{}%
\end{pgfscope}%
\end{pgfscope}%
\begin{pgfscope}%
\pgftext[x=0.467438in,y=5.222091in,left,base]{\rmfamily\fontsize{10.000000}{12.000000}\selectfont \(\displaystyle 0\)}%
\end{pgfscope}%
\begin{pgfscope}%
\pgfsetbuttcap%
\pgfsetroundjoin%
\definecolor{currentfill}{rgb}{0.000000,0.000000,0.000000}%
\pgfsetfillcolor{currentfill}%
\pgfsetlinewidth{0.803000pt}%
\definecolor{currentstroke}{rgb}{0.000000,0.000000,0.000000}%
\pgfsetstrokecolor{currentstroke}%
\pgfsetdash{}{0pt}%
\pgfsys@defobject{currentmarker}{\pgfqpoint{-0.048611in}{0.000000in}}{\pgfqpoint{0.000000in}{0.000000in}}{%
\pgfpathmoveto{\pgfqpoint{0.000000in}{0.000000in}}%
\pgfpathlineto{\pgfqpoint{-0.048611in}{0.000000in}}%
\pgfusepath{stroke,fill}%
}%
\begin{pgfscope}%
\pgfsys@transformshift{0.634105in}{5.744302in}%
\pgfsys@useobject{currentmarker}{}%
\end{pgfscope}%
\end{pgfscope}%
\begin{pgfscope}%
\pgftext[x=0.397993in,y=5.691541in,left,base]{\rmfamily\fontsize{10.000000}{12.000000}\selectfont \(\displaystyle 10\)}%
\end{pgfscope}%
\begin{pgfscope}%
\pgfsetbuttcap%
\pgfsetroundjoin%
\definecolor{currentfill}{rgb}{0.000000,0.000000,0.000000}%
\pgfsetfillcolor{currentfill}%
\pgfsetlinewidth{0.803000pt}%
\definecolor{currentstroke}{rgb}{0.000000,0.000000,0.000000}%
\pgfsetstrokecolor{currentstroke}%
\pgfsetdash{}{0pt}%
\pgfsys@defobject{currentmarker}{\pgfqpoint{-0.048611in}{0.000000in}}{\pgfqpoint{0.000000in}{0.000000in}}{%
\pgfpathmoveto{\pgfqpoint{0.000000in}{0.000000in}}%
\pgfpathlineto{\pgfqpoint{-0.048611in}{0.000000in}}%
\pgfusepath{stroke,fill}%
}%
\begin{pgfscope}%
\pgfsys@transformshift{0.634105in}{6.213751in}%
\pgfsys@useobject{currentmarker}{}%
\end{pgfscope}%
\end{pgfscope}%
\begin{pgfscope}%
\pgftext[x=0.397993in,y=6.160990in,left,base]{\rmfamily\fontsize{10.000000}{12.000000}\selectfont \(\displaystyle 20\)}%
\end{pgfscope}%
\begin{pgfscope}%
\pgfsetbuttcap%
\pgfsetroundjoin%
\definecolor{currentfill}{rgb}{0.000000,0.000000,0.000000}%
\pgfsetfillcolor{currentfill}%
\pgfsetlinewidth{0.803000pt}%
\definecolor{currentstroke}{rgb}{0.000000,0.000000,0.000000}%
\pgfsetstrokecolor{currentstroke}%
\pgfsetdash{}{0pt}%
\pgfsys@defobject{currentmarker}{\pgfqpoint{-0.048611in}{0.000000in}}{\pgfqpoint{0.000000in}{0.000000in}}{%
\pgfpathmoveto{\pgfqpoint{0.000000in}{0.000000in}}%
\pgfpathlineto{\pgfqpoint{-0.048611in}{0.000000in}}%
\pgfusepath{stroke,fill}%
}%
\begin{pgfscope}%
\pgfsys@transformshift{0.634105in}{6.683200in}%
\pgfsys@useobject{currentmarker}{}%
\end{pgfscope}%
\end{pgfscope}%
\begin{pgfscope}%
\pgftext[x=0.397993in,y=6.630439in,left,base]{\rmfamily\fontsize{10.000000}{12.000000}\selectfont \(\displaystyle 30\)}%
\end{pgfscope}%
\begin{pgfscope}%
\pgftext[x=0.234413in,y=5.281603in,,bottom,rotate=90.000000]{\rmfamily\fontsize{10.000000}{12.000000}\selectfont \(\displaystyle y_1\)}%
\end{pgfscope}%
\begin{pgfscope}%
\pgfpathrectangle{\pgfqpoint{0.634105in}{3.881603in}}{\pgfqpoint{4.227273in}{2.800000in}} %
\pgfusepath{clip}%
\pgfsetrectcap%
\pgfsetroundjoin%
\pgfsetlinewidth{0.501875pt}%
\definecolor{currentstroke}{rgb}{0.500000,0.000000,1.000000}%
\pgfsetstrokecolor{currentstroke}%
\pgfsetdash{}{0pt}%
\pgfpathmoveto{\pgfqpoint{0.826254in}{5.610399in}}%
\pgfpathlineto{\pgfqpoint{0.841687in}{5.776456in}}%
\pgfpathlineto{\pgfqpoint{0.857121in}{5.932461in}}%
\pgfpathlineto{\pgfqpoint{0.872555in}{6.075285in}}%
\pgfpathlineto{\pgfqpoint{0.887988in}{6.202066in}}%
\pgfpathlineto{\pgfqpoint{0.903422in}{6.310263in}}%
\pgfpathlineto{\pgfqpoint{0.918855in}{6.397708in}}%
\pgfpathlineto{\pgfqpoint{0.934289in}{6.462648in}}%
\pgfpathlineto{\pgfqpoint{0.949723in}{6.503781in}}%
\pgfpathlineto{\pgfqpoint{0.965156in}{6.520283in}}%
\pgfpathlineto{\pgfqpoint{0.980590in}{6.511823in}}%
\pgfpathlineto{\pgfqpoint{0.996024in}{6.478571in}}%
\pgfpathlineto{\pgfqpoint{1.011457in}{6.421193in}}%
\pgfpathlineto{\pgfqpoint{1.026891in}{6.340839in}}%
\pgfpathlineto{\pgfqpoint{1.042325in}{6.239120in}}%
\pgfpathlineto{\pgfqpoint{1.057758in}{6.118074in}}%
\pgfpathlineto{\pgfqpoint{1.073192in}{5.980128in}}%
\pgfpathlineto{\pgfqpoint{1.088625in}{5.828046in}}%
\pgfpathlineto{\pgfqpoint{1.119493in}{5.493890in}}%
\pgfpathlineto{\pgfqpoint{1.181227in}{4.801210in}}%
\pgfpathlineto{\pgfqpoint{1.196661in}{4.643245in}}%
\pgfpathlineto{\pgfqpoint{1.212095in}{4.497940in}}%
\pgfpathlineto{\pgfqpoint{1.227528in}{4.368207in}}%
\pgfpathlineto{\pgfqpoint{1.242962in}{4.256645in}}%
\pgfpathlineto{\pgfqpoint{1.258395in}{4.165490in}}%
\pgfpathlineto{\pgfqpoint{1.273829in}{4.096571in}}%
\pgfpathlineto{\pgfqpoint{1.289263in}{4.051267in}}%
\pgfpathlineto{\pgfqpoint{1.304696in}{4.030488in}}%
\pgfpathlineto{\pgfqpoint{1.320130in}{4.034649in}}%
\pgfpathlineto{\pgfqpoint{1.335564in}{4.063667in}}%
\pgfpathlineto{\pgfqpoint{1.350997in}{4.116960in}}%
\pgfpathlineto{\pgfqpoint{1.366431in}{4.193461in}}%
\pgfpathlineto{\pgfqpoint{1.381864in}{4.291636in}}%
\pgfpathlineto{\pgfqpoint{1.397298in}{4.409518in}}%
\pgfpathlineto{\pgfqpoint{1.412732in}{4.544743in}}%
\pgfpathlineto{\pgfqpoint{1.428165in}{4.694601in}}%
\pgfpathlineto{\pgfqpoint{1.459033in}{5.025971in}}%
\pgfpathlineto{\pgfqpoint{1.520767in}{5.720254in}}%
\pgfpathlineto{\pgfqpoint{1.536201in}{5.880084in}}%
\pgfpathlineto{\pgfqpoint{1.551634in}{6.027784in}}%
\pgfpathlineto{\pgfqpoint{1.567068in}{6.160393in}}%
\pgfpathlineto{\pgfqpoint{1.582502in}{6.275253in}}%
\pgfpathlineto{\pgfqpoint{1.597935in}{6.370062in}}%
\pgfpathlineto{\pgfqpoint{1.613369in}{6.442921in}}%
\pgfpathlineto{\pgfqpoint{1.628803in}{6.492368in}}%
\pgfpathlineto{\pgfqpoint{1.644236in}{6.517413in}}%
\pgfpathlineto{\pgfqpoint{1.659670in}{6.517553in}}%
\pgfpathlineto{\pgfqpoint{1.675104in}{6.492787in}}%
\pgfpathlineto{\pgfqpoint{1.690537in}{6.443609in}}%
\pgfpathlineto{\pgfqpoint{1.705971in}{6.371007in}}%
\pgfpathlineto{\pgfqpoint{1.721404in}{6.276434in}}%
\pgfpathlineto{\pgfqpoint{1.736838in}{6.161787in}}%
\pgfpathlineto{\pgfqpoint{1.752272in}{6.029364in}}%
\pgfpathlineto{\pgfqpoint{1.767705in}{5.881818in}}%
\pgfpathlineto{\pgfqpoint{1.798573in}{5.553431in}}%
\pgfpathlineto{\pgfqpoint{1.860307in}{4.857959in}}%
\pgfpathlineto{\pgfqpoint{1.891174in}{4.546352in}}%
\pgfpathlineto{\pgfqpoint{1.906608in}{4.410946in}}%
\pgfpathlineto{\pgfqpoint{1.922042in}{4.292856in}}%
\pgfpathlineto{\pgfqpoint{1.937475in}{4.194448in}}%
\pgfpathlineto{\pgfqpoint{1.952909in}{4.117694in}}%
\pgfpathlineto{\pgfqpoint{1.968343in}{4.064132in}}%
\pgfpathlineto{\pgfqpoint{1.983776in}{4.034837in}}%
\pgfpathlineto{\pgfqpoint{1.999210in}{4.030395in}}%
\pgfpathlineto{\pgfqpoint{2.014644in}{4.050896in}}%
\pgfpathlineto{\pgfqpoint{2.030077in}{4.095928in}}%
\pgfpathlineto{\pgfqpoint{2.045511in}{4.164589in}}%
\pgfpathlineto{\pgfqpoint{2.060944in}{4.255503in}}%
\pgfpathlineto{\pgfqpoint{2.076378in}{4.366847in}}%
\pgfpathlineto{\pgfqpoint{2.091812in}{4.496390in}}%
\pgfpathlineto{\pgfqpoint{2.107245in}{4.641536in}}%
\pgfpathlineto{\pgfqpoint{2.138113in}{4.966744in}}%
\pgfpathlineto{\pgfqpoint{2.215281in}{5.826267in}}%
\pgfpathlineto{\pgfqpoint{2.230714in}{5.978491in}}%
\pgfpathlineto{\pgfqpoint{2.246148in}{6.116612in}}%
\pgfpathlineto{\pgfqpoint{2.261582in}{6.237862in}}%
\pgfpathlineto{\pgfqpoint{2.277015in}{6.339810in}}%
\pgfpathlineto{\pgfqpoint{2.292449in}{6.420414in}}%
\pgfpathlineto{\pgfqpoint{2.307883in}{6.478058in}}%
\pgfpathlineto{\pgfqpoint{2.323316in}{6.511586in}}%
\pgfpathlineto{\pgfqpoint{2.338750in}{6.520327in}}%
\pgfpathlineto{\pgfqpoint{2.354184in}{6.504105in}}%
\pgfpathlineto{\pgfqpoint{2.369617in}{6.463245in}}%
\pgfpathlineto{\pgfqpoint{2.385051in}{6.398566in}}%
\pgfpathlineto{\pgfqpoint{2.400484in}{6.311366in}}%
\pgfpathlineto{\pgfqpoint{2.415918in}{6.203390in}}%
\pgfpathlineto{\pgfqpoint{2.431352in}{6.076805in}}%
\pgfpathlineto{\pgfqpoint{2.446785in}{5.934145in}}%
\pgfpathlineto{\pgfqpoint{2.477653in}{5.612309in}}%
\pgfpathlineto{\pgfqpoint{2.554821in}{4.750847in}}%
\pgfpathlineto{\pgfqpoint{2.570254in}{4.596496in}}%
\pgfpathlineto{\pgfqpoint{2.585688in}{4.455742in}}%
\pgfpathlineto{\pgfqpoint{2.601122in}{4.331404in}}%
\pgfpathlineto{\pgfqpoint{2.616555in}{4.225976in}}%
\pgfpathlineto{\pgfqpoint{2.631989in}{4.141570in}}%
\pgfpathlineto{\pgfqpoint{2.647423in}{4.079879in}}%
\pgfpathlineto{\pgfqpoint{2.662856in}{4.042137in}}%
\pgfpathlineto{\pgfqpoint{2.678290in}{4.029103in}}%
\pgfpathlineto{\pgfqpoint{2.693724in}{4.041037in}}%
\pgfpathlineto{\pgfqpoint{2.709157in}{4.077701in}}%
\pgfpathlineto{\pgfqpoint{2.724591in}{4.138358in}}%
\pgfpathlineto{\pgfqpoint{2.740024in}{4.221794in}}%
\pgfpathlineto{\pgfqpoint{2.755458in}{4.326336in}}%
\pgfpathlineto{\pgfqpoint{2.770892in}{4.449889in}}%
\pgfpathlineto{\pgfqpoint{2.786325in}{4.589977in}}%
\pgfpathlineto{\pgfqpoint{2.817193in}{4.908250in}}%
\pgfpathlineto{\pgfqpoint{2.909794in}{5.927524in}}%
\pgfpathlineto{\pgfqpoint{2.925228in}{6.070829in}}%
\pgfpathlineto{\pgfqpoint{2.940662in}{6.198180in}}%
\pgfpathlineto{\pgfqpoint{2.956095in}{6.307025in}}%
\pgfpathlineto{\pgfqpoint{2.971529in}{6.395182in}}%
\pgfpathlineto{\pgfqpoint{2.986963in}{6.460885in}}%
\pgfpathlineto{\pgfqpoint{3.002396in}{6.502817in}}%
\pgfpathlineto{\pgfqpoint{3.017830in}{6.520137in}}%
\pgfpathlineto{\pgfqpoint{3.033263in}{6.512498in}}%
\pgfpathlineto{\pgfqpoint{3.048697in}{6.480053in}}%
\pgfpathlineto{\pgfqpoint{3.064131in}{6.423452in}}%
\pgfpathlineto{\pgfqpoint{3.079564in}{6.343831in}}%
\pgfpathlineto{\pgfqpoint{3.094998in}{6.242784in}}%
\pgfpathlineto{\pgfqpoint{3.110432in}{6.122337in}}%
\pgfpathlineto{\pgfqpoint{3.125865in}{5.984905in}}%
\pgfpathlineto{\pgfqpoint{3.141299in}{5.833240in}}%
\pgfpathlineto{\pgfqpoint{3.172166in}{5.499601in}}%
\pgfpathlineto{\pgfqpoint{3.233901in}{4.806583in}}%
\pgfpathlineto{\pgfqpoint{3.249334in}{4.648255in}}%
\pgfpathlineto{\pgfqpoint{3.264768in}{4.502486in}}%
\pgfpathlineto{\pgfqpoint{3.280202in}{4.372197in}}%
\pgfpathlineto{\pgfqpoint{3.295635in}{4.260000in}}%
\pgfpathlineto{\pgfqpoint{3.311069in}{4.168143in}}%
\pgfpathlineto{\pgfqpoint{3.326503in}{4.098468in}}%
\pgfpathlineto{\pgfqpoint{3.341936in}{4.052371in}}%
\pgfpathlineto{\pgfqpoint{3.357370in}{4.030775in}}%
\pgfpathlineto{\pgfqpoint{3.372803in}{4.034115in}}%
\pgfpathlineto{\pgfqpoint{3.388237in}{4.062322in}}%
\pgfpathlineto{\pgfqpoint{3.403671in}{4.114832in}}%
\pgfpathlineto{\pgfqpoint{3.419104in}{4.190591in}}%
\pgfpathlineto{\pgfqpoint{3.434538in}{4.288083in}}%
\pgfpathlineto{\pgfqpoint{3.449972in}{4.405352in}}%
\pgfpathlineto{\pgfqpoint{3.465405in}{4.540048in}}%
\pgfpathlineto{\pgfqpoint{3.480839in}{4.689471in}}%
\pgfpathlineto{\pgfqpoint{3.511706in}{5.020287in}}%
\pgfpathlineto{\pgfqpoint{3.573441in}{5.714828in}}%
\pgfpathlineto{\pgfqpoint{3.588874in}{5.875004in}}%
\pgfpathlineto{\pgfqpoint{3.604308in}{6.023151in}}%
\pgfpathlineto{\pgfqpoint{3.619742in}{6.156300in}}%
\pgfpathlineto{\pgfqpoint{3.635175in}{6.271783in}}%
\pgfpathlineto{\pgfqpoint{3.650609in}{6.367284in}}%
\pgfpathlineto{\pgfqpoint{3.666043in}{6.440890in}}%
\pgfpathlineto{\pgfqpoint{3.681476in}{6.491126in}}%
\pgfpathlineto{\pgfqpoint{3.696910in}{6.516984in}}%
\pgfpathlineto{\pgfqpoint{3.712343in}{6.517946in}}%
\pgfpathlineto{\pgfqpoint{3.727777in}{6.493993in}}%
\pgfpathlineto{\pgfqpoint{3.743211in}{6.445606in}}%
\pgfpathlineto{\pgfqpoint{3.758644in}{6.373753in}}%
\pgfpathlineto{\pgfqpoint{3.774078in}{6.279875in}}%
\pgfpathlineto{\pgfqpoint{3.789512in}{6.165854in}}%
\pgfpathlineto{\pgfqpoint{3.804945in}{6.033974in}}%
\pgfpathlineto{\pgfqpoint{3.820379in}{5.886880in}}%
\pgfpathlineto{\pgfqpoint{3.851246in}{5.559085in}}%
\pgfpathlineto{\pgfqpoint{3.912981in}{4.863434in}}%
\pgfpathlineto{\pgfqpoint{3.943848in}{4.551068in}}%
\pgfpathlineto{\pgfqpoint{3.959282in}{4.415138in}}%
\pgfpathlineto{\pgfqpoint{3.974715in}{4.296438in}}%
\pgfpathlineto{\pgfqpoint{3.990149in}{4.197349in}}%
\pgfpathlineto{\pgfqpoint{4.005583in}{4.119856in}}%
\pgfpathlineto{\pgfqpoint{4.021016in}{4.065513in}}%
\pgfpathlineto{\pgfqpoint{4.036450in}{4.035407in}}%
\pgfpathlineto{\pgfqpoint{4.051883in}{4.030144in}}%
\pgfpathlineto{\pgfqpoint{4.067317in}{4.049828in}}%
\pgfpathlineto{\pgfqpoint{4.082751in}{4.094065in}}%
\pgfpathlineto{\pgfqpoint{4.098184in}{4.161968in}}%
\pgfpathlineto{\pgfqpoint{4.113618in}{4.252177in}}%
\pgfpathlineto{\pgfqpoint{4.129052in}{4.362883in}}%
\pgfpathlineto{\pgfqpoint{4.144485in}{4.491867in}}%
\pgfpathlineto{\pgfqpoint{4.159919in}{4.636544in}}%
\pgfpathlineto{\pgfqpoint{4.190786in}{4.961123in}}%
\pgfpathlineto{\pgfqpoint{4.267954in}{5.821056in}}%
\pgfpathlineto{\pgfqpoint{4.283388in}{5.973693in}}%
\pgfpathlineto{\pgfqpoint{4.298822in}{6.112324in}}%
\pgfpathlineto{\pgfqpoint{4.314255in}{6.234169in}}%
\pgfpathlineto{\pgfqpoint{4.329689in}{6.336787in}}%
\pgfpathlineto{\pgfqpoint{4.345122in}{6.418121in}}%
\pgfpathlineto{\pgfqpoint{4.360556in}{6.476541in}}%
\pgfpathlineto{\pgfqpoint{4.375990in}{6.510875in}}%
\pgfpathlineto{\pgfqpoint{4.391423in}{6.520437in}}%
\pgfpathlineto{\pgfqpoint{4.406857in}{6.505033in}}%
\pgfpathlineto{\pgfqpoint{4.422291in}{6.464973in}}%
\pgfpathlineto{\pgfqpoint{4.437724in}{6.401060in}}%
\pgfpathlineto{\pgfqpoint{4.453158in}{6.314574in}}%
\pgfpathlineto{\pgfqpoint{4.468592in}{6.207250in}}%
\pgfpathlineto{\pgfqpoint{4.484025in}{6.081238in}}%
\pgfpathlineto{\pgfqpoint{4.499459in}{5.939063in}}%
\pgfpathlineto{\pgfqpoint{4.530326in}{5.617893in}}%
\pgfpathlineto{\pgfqpoint{4.607494in}{4.756119in}}%
\pgfpathlineto{\pgfqpoint{4.622928in}{4.601372in}}%
\pgfpathlineto{\pgfqpoint{4.638362in}{4.460124in}}%
\pgfpathlineto{\pgfqpoint{4.653795in}{4.335205in}}%
\pgfpathlineto{\pgfqpoint{4.669229in}{4.229119in}}%
\pgfpathlineto{\pgfqpoint{4.669229in}{4.229119in}}%
\pgfusepath{stroke}%
\end{pgfscope}%
\begin{pgfscope}%
\pgfpathrectangle{\pgfqpoint{0.634105in}{3.881603in}}{\pgfqpoint{4.227273in}{2.800000in}} %
\pgfusepath{clip}%
\pgfsetrectcap%
\pgfsetroundjoin%
\pgfsetlinewidth{0.501875pt}%
\definecolor{currentstroke}{rgb}{0.421569,0.122888,0.998103}%
\pgfsetstrokecolor{currentstroke}%
\pgfsetdash{}{0pt}%
\pgfpathmoveto{\pgfqpoint{0.826254in}{5.610399in}}%
\pgfpathlineto{\pgfqpoint{0.841687in}{5.776456in}}%
\pgfpathlineto{\pgfqpoint{0.857121in}{5.932461in}}%
\pgfpathlineto{\pgfqpoint{0.872555in}{6.075285in}}%
\pgfpathlineto{\pgfqpoint{0.887988in}{6.202066in}}%
\pgfpathlineto{\pgfqpoint{0.903422in}{6.310263in}}%
\pgfpathlineto{\pgfqpoint{0.918855in}{6.397708in}}%
\pgfpathlineto{\pgfqpoint{0.934289in}{6.462648in}}%
\pgfpathlineto{\pgfqpoint{0.949723in}{6.503781in}}%
\pgfpathlineto{\pgfqpoint{0.965156in}{6.520283in}}%
\pgfpathlineto{\pgfqpoint{0.980590in}{6.511823in}}%
\pgfpathlineto{\pgfqpoint{0.996024in}{6.478571in}}%
\pgfpathlineto{\pgfqpoint{1.011457in}{6.421193in}}%
\pgfpathlineto{\pgfqpoint{1.026891in}{6.340839in}}%
\pgfpathlineto{\pgfqpoint{1.042325in}{6.239120in}}%
\pgfpathlineto{\pgfqpoint{1.057758in}{6.118074in}}%
\pgfpathlineto{\pgfqpoint{1.073192in}{5.980128in}}%
\pgfpathlineto{\pgfqpoint{1.088625in}{5.828046in}}%
\pgfpathlineto{\pgfqpoint{1.119493in}{5.493890in}}%
\pgfpathlineto{\pgfqpoint{1.181227in}{4.801210in}}%
\pgfpathlineto{\pgfqpoint{1.196661in}{4.643245in}}%
\pgfpathlineto{\pgfqpoint{1.212095in}{4.497940in}}%
\pgfpathlineto{\pgfqpoint{1.227528in}{4.368207in}}%
\pgfpathlineto{\pgfqpoint{1.242962in}{4.256645in}}%
\pgfpathlineto{\pgfqpoint{1.258395in}{4.165490in}}%
\pgfpathlineto{\pgfqpoint{1.273829in}{4.096571in}}%
\pgfpathlineto{\pgfqpoint{1.289263in}{4.051267in}}%
\pgfpathlineto{\pgfqpoint{1.304696in}{4.030488in}}%
\pgfpathlineto{\pgfqpoint{1.320130in}{4.034649in}}%
\pgfpathlineto{\pgfqpoint{1.335564in}{4.063667in}}%
\pgfpathlineto{\pgfqpoint{1.350997in}{4.116960in}}%
\pgfpathlineto{\pgfqpoint{1.366431in}{4.193461in}}%
\pgfpathlineto{\pgfqpoint{1.381864in}{4.291636in}}%
\pgfpathlineto{\pgfqpoint{1.397298in}{4.409518in}}%
\pgfpathlineto{\pgfqpoint{1.412732in}{4.544743in}}%
\pgfpathlineto{\pgfqpoint{1.428165in}{4.694601in}}%
\pgfpathlineto{\pgfqpoint{1.459033in}{5.025971in}}%
\pgfpathlineto{\pgfqpoint{1.520767in}{5.720254in}}%
\pgfpathlineto{\pgfqpoint{1.536201in}{5.880084in}}%
\pgfpathlineto{\pgfqpoint{1.551634in}{6.027784in}}%
\pgfpathlineto{\pgfqpoint{1.567068in}{6.160393in}}%
\pgfpathlineto{\pgfqpoint{1.582502in}{6.275253in}}%
\pgfpathlineto{\pgfqpoint{1.597935in}{6.370062in}}%
\pgfpathlineto{\pgfqpoint{1.613369in}{6.442921in}}%
\pgfpathlineto{\pgfqpoint{1.628803in}{6.492368in}}%
\pgfpathlineto{\pgfqpoint{1.644236in}{6.517413in}}%
\pgfpathlineto{\pgfqpoint{1.659670in}{6.517553in}}%
\pgfpathlineto{\pgfqpoint{1.675104in}{6.492787in}}%
\pgfpathlineto{\pgfqpoint{1.690537in}{6.443609in}}%
\pgfpathlineto{\pgfqpoint{1.705971in}{6.371007in}}%
\pgfpathlineto{\pgfqpoint{1.721404in}{6.276434in}}%
\pgfpathlineto{\pgfqpoint{1.736838in}{6.161787in}}%
\pgfpathlineto{\pgfqpoint{1.752272in}{6.029364in}}%
\pgfpathlineto{\pgfqpoint{1.767705in}{5.881818in}}%
\pgfpathlineto{\pgfqpoint{1.798573in}{5.553431in}}%
\pgfpathlineto{\pgfqpoint{1.860307in}{4.857959in}}%
\pgfpathlineto{\pgfqpoint{1.891174in}{4.546352in}}%
\pgfpathlineto{\pgfqpoint{1.906608in}{4.410946in}}%
\pgfpathlineto{\pgfqpoint{1.922042in}{4.292856in}}%
\pgfpathlineto{\pgfqpoint{1.937475in}{4.194448in}}%
\pgfpathlineto{\pgfqpoint{1.952909in}{4.117694in}}%
\pgfpathlineto{\pgfqpoint{1.968343in}{4.064132in}}%
\pgfpathlineto{\pgfqpoint{1.983776in}{4.034837in}}%
\pgfpathlineto{\pgfqpoint{1.999210in}{4.030395in}}%
\pgfpathlineto{\pgfqpoint{2.014644in}{4.050896in}}%
\pgfpathlineto{\pgfqpoint{2.030077in}{4.095928in}}%
\pgfpathlineto{\pgfqpoint{2.045511in}{4.164589in}}%
\pgfpathlineto{\pgfqpoint{2.060944in}{4.255503in}}%
\pgfpathlineto{\pgfqpoint{2.076378in}{4.366847in}}%
\pgfpathlineto{\pgfqpoint{2.091812in}{4.496390in}}%
\pgfpathlineto{\pgfqpoint{2.107245in}{4.641536in}}%
\pgfpathlineto{\pgfqpoint{2.138113in}{4.966744in}}%
\pgfpathlineto{\pgfqpoint{2.215281in}{5.826267in}}%
\pgfpathlineto{\pgfqpoint{2.230714in}{5.978491in}}%
\pgfpathlineto{\pgfqpoint{2.246148in}{6.116612in}}%
\pgfpathlineto{\pgfqpoint{2.261582in}{6.237862in}}%
\pgfpathlineto{\pgfqpoint{2.277015in}{6.339810in}}%
\pgfpathlineto{\pgfqpoint{2.292449in}{6.420414in}}%
\pgfpathlineto{\pgfqpoint{2.307883in}{6.478058in}}%
\pgfpathlineto{\pgfqpoint{2.323316in}{6.511586in}}%
\pgfpathlineto{\pgfqpoint{2.338750in}{6.520327in}}%
\pgfpathlineto{\pgfqpoint{2.354184in}{6.504105in}}%
\pgfpathlineto{\pgfqpoint{2.369617in}{6.463245in}}%
\pgfpathlineto{\pgfqpoint{2.385051in}{6.398566in}}%
\pgfpathlineto{\pgfqpoint{2.400484in}{6.311366in}}%
\pgfpathlineto{\pgfqpoint{2.415918in}{6.203390in}}%
\pgfpathlineto{\pgfqpoint{2.431352in}{6.076805in}}%
\pgfpathlineto{\pgfqpoint{2.446785in}{5.934145in}}%
\pgfpathlineto{\pgfqpoint{2.477653in}{5.612309in}}%
\pgfpathlineto{\pgfqpoint{2.554821in}{4.750847in}}%
\pgfpathlineto{\pgfqpoint{2.570254in}{4.596496in}}%
\pgfpathlineto{\pgfqpoint{2.585688in}{4.455742in}}%
\pgfpathlineto{\pgfqpoint{2.601122in}{4.331404in}}%
\pgfpathlineto{\pgfqpoint{2.616555in}{4.225976in}}%
\pgfpathlineto{\pgfqpoint{2.631989in}{4.141570in}}%
\pgfpathlineto{\pgfqpoint{2.647423in}{4.079879in}}%
\pgfpathlineto{\pgfqpoint{2.662856in}{4.042137in}}%
\pgfpathlineto{\pgfqpoint{2.678290in}{4.029103in}}%
\pgfpathlineto{\pgfqpoint{2.693724in}{4.041037in}}%
\pgfpathlineto{\pgfqpoint{2.709157in}{4.077701in}}%
\pgfpathlineto{\pgfqpoint{2.724591in}{4.138358in}}%
\pgfpathlineto{\pgfqpoint{2.740024in}{4.221794in}}%
\pgfpathlineto{\pgfqpoint{2.755458in}{4.326336in}}%
\pgfpathlineto{\pgfqpoint{2.770892in}{4.449889in}}%
\pgfpathlineto{\pgfqpoint{2.786325in}{4.589977in}}%
\pgfpathlineto{\pgfqpoint{2.817193in}{4.908250in}}%
\pgfpathlineto{\pgfqpoint{2.909794in}{5.927524in}}%
\pgfpathlineto{\pgfqpoint{2.925228in}{6.070829in}}%
\pgfpathlineto{\pgfqpoint{2.940662in}{6.198180in}}%
\pgfpathlineto{\pgfqpoint{2.956095in}{6.307025in}}%
\pgfpathlineto{\pgfqpoint{2.971529in}{6.395182in}}%
\pgfpathlineto{\pgfqpoint{2.986963in}{6.460885in}}%
\pgfpathlineto{\pgfqpoint{3.002396in}{6.502817in}}%
\pgfpathlineto{\pgfqpoint{3.017830in}{6.520137in}}%
\pgfpathlineto{\pgfqpoint{3.033263in}{6.512498in}}%
\pgfpathlineto{\pgfqpoint{3.048697in}{6.480053in}}%
\pgfpathlineto{\pgfqpoint{3.064131in}{6.423452in}}%
\pgfpathlineto{\pgfqpoint{3.079564in}{6.343831in}}%
\pgfpathlineto{\pgfqpoint{3.094998in}{6.242784in}}%
\pgfpathlineto{\pgfqpoint{3.110432in}{6.122337in}}%
\pgfpathlineto{\pgfqpoint{3.125865in}{5.984905in}}%
\pgfpathlineto{\pgfqpoint{3.141299in}{5.833240in}}%
\pgfpathlineto{\pgfqpoint{3.172166in}{5.499601in}}%
\pgfpathlineto{\pgfqpoint{3.233901in}{4.806583in}}%
\pgfpathlineto{\pgfqpoint{3.249334in}{4.648255in}}%
\pgfpathlineto{\pgfqpoint{3.264768in}{4.502486in}}%
\pgfpathlineto{\pgfqpoint{3.280202in}{4.372197in}}%
\pgfpathlineto{\pgfqpoint{3.295635in}{4.260000in}}%
\pgfpathlineto{\pgfqpoint{3.311069in}{4.168143in}}%
\pgfpathlineto{\pgfqpoint{3.326503in}{4.098468in}}%
\pgfpathlineto{\pgfqpoint{3.341936in}{4.052371in}}%
\pgfpathlineto{\pgfqpoint{3.357370in}{4.030775in}}%
\pgfpathlineto{\pgfqpoint{3.372803in}{4.034115in}}%
\pgfpathlineto{\pgfqpoint{3.388237in}{4.062322in}}%
\pgfpathlineto{\pgfqpoint{3.403671in}{4.114832in}}%
\pgfpathlineto{\pgfqpoint{3.419104in}{4.190591in}}%
\pgfpathlineto{\pgfqpoint{3.434538in}{4.288083in}}%
\pgfpathlineto{\pgfqpoint{3.449972in}{4.405352in}}%
\pgfpathlineto{\pgfqpoint{3.465405in}{4.540048in}}%
\pgfpathlineto{\pgfqpoint{3.480839in}{4.689471in}}%
\pgfpathlineto{\pgfqpoint{3.511706in}{5.020287in}}%
\pgfpathlineto{\pgfqpoint{3.573441in}{5.714828in}}%
\pgfpathlineto{\pgfqpoint{3.588874in}{5.875004in}}%
\pgfpathlineto{\pgfqpoint{3.604308in}{6.023151in}}%
\pgfpathlineto{\pgfqpoint{3.619742in}{6.156300in}}%
\pgfpathlineto{\pgfqpoint{3.635175in}{6.271783in}}%
\pgfpathlineto{\pgfqpoint{3.650609in}{6.367284in}}%
\pgfpathlineto{\pgfqpoint{3.666043in}{6.440890in}}%
\pgfpathlineto{\pgfqpoint{3.681476in}{6.491126in}}%
\pgfpathlineto{\pgfqpoint{3.696910in}{6.516984in}}%
\pgfpathlineto{\pgfqpoint{3.712343in}{6.517946in}}%
\pgfpathlineto{\pgfqpoint{3.727777in}{6.493993in}}%
\pgfpathlineto{\pgfqpoint{3.743211in}{6.445606in}}%
\pgfpathlineto{\pgfqpoint{3.758644in}{6.373753in}}%
\pgfpathlineto{\pgfqpoint{3.774078in}{6.279875in}}%
\pgfpathlineto{\pgfqpoint{3.789512in}{6.165854in}}%
\pgfpathlineto{\pgfqpoint{3.804945in}{6.033974in}}%
\pgfpathlineto{\pgfqpoint{3.820379in}{5.886880in}}%
\pgfpathlineto{\pgfqpoint{3.851246in}{5.559085in}}%
\pgfpathlineto{\pgfqpoint{3.912981in}{4.863434in}}%
\pgfpathlineto{\pgfqpoint{3.943848in}{4.551068in}}%
\pgfpathlineto{\pgfqpoint{3.959282in}{4.415138in}}%
\pgfpathlineto{\pgfqpoint{3.974715in}{4.296438in}}%
\pgfpathlineto{\pgfqpoint{3.990149in}{4.197349in}}%
\pgfpathlineto{\pgfqpoint{4.005583in}{4.119856in}}%
\pgfpathlineto{\pgfqpoint{4.021016in}{4.065513in}}%
\pgfpathlineto{\pgfqpoint{4.036450in}{4.035407in}}%
\pgfpathlineto{\pgfqpoint{4.051883in}{4.030144in}}%
\pgfpathlineto{\pgfqpoint{4.067317in}{4.049828in}}%
\pgfpathlineto{\pgfqpoint{4.082751in}{4.094065in}}%
\pgfpathlineto{\pgfqpoint{4.098184in}{4.161968in}}%
\pgfpathlineto{\pgfqpoint{4.113618in}{4.252177in}}%
\pgfpathlineto{\pgfqpoint{4.129052in}{4.362883in}}%
\pgfpathlineto{\pgfqpoint{4.144485in}{4.491867in}}%
\pgfpathlineto{\pgfqpoint{4.159919in}{4.636544in}}%
\pgfpathlineto{\pgfqpoint{4.190786in}{4.961123in}}%
\pgfpathlineto{\pgfqpoint{4.267954in}{5.821056in}}%
\pgfpathlineto{\pgfqpoint{4.283388in}{5.973693in}}%
\pgfpathlineto{\pgfqpoint{4.298822in}{6.112324in}}%
\pgfpathlineto{\pgfqpoint{4.314255in}{6.234169in}}%
\pgfpathlineto{\pgfqpoint{4.329689in}{6.336787in}}%
\pgfpathlineto{\pgfqpoint{4.345122in}{6.418121in}}%
\pgfpathlineto{\pgfqpoint{4.360556in}{6.476541in}}%
\pgfpathlineto{\pgfqpoint{4.375990in}{6.510875in}}%
\pgfpathlineto{\pgfqpoint{4.391423in}{6.520437in}}%
\pgfpathlineto{\pgfqpoint{4.406857in}{6.505033in}}%
\pgfpathlineto{\pgfqpoint{4.422291in}{6.464973in}}%
\pgfpathlineto{\pgfqpoint{4.437724in}{6.401060in}}%
\pgfpathlineto{\pgfqpoint{4.453158in}{6.314574in}}%
\pgfpathlineto{\pgfqpoint{4.468592in}{6.207250in}}%
\pgfpathlineto{\pgfqpoint{4.484025in}{6.081238in}}%
\pgfpathlineto{\pgfqpoint{4.499459in}{5.939063in}}%
\pgfpathlineto{\pgfqpoint{4.530326in}{5.617893in}}%
\pgfpathlineto{\pgfqpoint{4.607494in}{4.756119in}}%
\pgfpathlineto{\pgfqpoint{4.622928in}{4.601372in}}%
\pgfpathlineto{\pgfqpoint{4.638362in}{4.460124in}}%
\pgfpathlineto{\pgfqpoint{4.653795in}{4.335205in}}%
\pgfpathlineto{\pgfqpoint{4.669229in}{4.229119in}}%
\pgfpathlineto{\pgfqpoint{4.669229in}{4.229119in}}%
\pgfusepath{stroke}%
\end{pgfscope}%
\begin{pgfscope}%
\pgfpathrectangle{\pgfqpoint{0.634105in}{3.881603in}}{\pgfqpoint{4.227273in}{2.800000in}} %
\pgfusepath{clip}%
\pgfsetrectcap%
\pgfsetroundjoin%
\pgfsetlinewidth{0.501875pt}%
\definecolor{currentstroke}{rgb}{0.343137,0.243914,0.992421}%
\pgfsetstrokecolor{currentstroke}%
\pgfsetdash{}{0pt}%
\pgfpathmoveto{\pgfqpoint{0.826254in}{5.428612in}}%
\pgfpathlineto{\pgfqpoint{0.841687in}{5.524342in}}%
\pgfpathlineto{\pgfqpoint{0.857121in}{5.615046in}}%
\pgfpathlineto{\pgfqpoint{0.872555in}{5.699149in}}%
\pgfpathlineto{\pgfqpoint{0.887988in}{5.775205in}}%
\pgfpathlineto{\pgfqpoint{0.903422in}{5.841931in}}%
\pgfpathlineto{\pgfqpoint{0.918855in}{5.898230in}}%
\pgfpathlineto{\pgfqpoint{0.934289in}{5.943213in}}%
\pgfpathlineto{\pgfqpoint{0.949723in}{5.976215in}}%
\pgfpathlineto{\pgfqpoint{0.965156in}{5.996811in}}%
\pgfpathlineto{\pgfqpoint{0.980590in}{6.004825in}}%
\pgfpathlineto{\pgfqpoint{0.996024in}{6.000327in}}%
\pgfpathlineto{\pgfqpoint{1.011457in}{5.983638in}}%
\pgfpathlineto{\pgfqpoint{1.026891in}{5.955322in}}%
\pgfpathlineto{\pgfqpoint{1.042325in}{5.916170in}}%
\pgfpathlineto{\pgfqpoint{1.057758in}{5.867191in}}%
\pgfpathlineto{\pgfqpoint{1.073192in}{5.809585in}}%
\pgfpathlineto{\pgfqpoint{1.088625in}{5.744724in}}%
\pgfpathlineto{\pgfqpoint{1.119493in}{5.599394in}}%
\pgfpathlineto{\pgfqpoint{1.181227in}{5.293799in}}%
\pgfpathlineto{\pgfqpoint{1.196661in}{5.224368in}}%
\pgfpathlineto{\pgfqpoint{1.212095in}{5.160993in}}%
\pgfpathlineto{\pgfqpoint{1.227528in}{5.105120in}}%
\pgfpathlineto{\pgfqpoint{1.242962in}{5.058040in}}%
\pgfpathlineto{\pgfqpoint{1.258395in}{5.020862in}}%
\pgfpathlineto{\pgfqpoint{1.273829in}{4.994490in}}%
\pgfpathlineto{\pgfqpoint{1.289263in}{4.979608in}}%
\pgfpathlineto{\pgfqpoint{1.304696in}{4.976661in}}%
\pgfpathlineto{\pgfqpoint{1.320130in}{4.985849in}}%
\pgfpathlineto{\pgfqpoint{1.335564in}{5.007124in}}%
\pgfpathlineto{\pgfqpoint{1.350997in}{5.040187in}}%
\pgfpathlineto{\pgfqpoint{1.366431in}{5.084499in}}%
\pgfpathlineto{\pgfqpoint{1.381864in}{5.139286in}}%
\pgfpathlineto{\pgfqpoint{1.397298in}{5.203558in}}%
\pgfpathlineto{\pgfqpoint{1.412732in}{5.276128in}}%
\pgfpathlineto{\pgfqpoint{1.443599in}{5.440578in}}%
\pgfpathlineto{\pgfqpoint{1.536201in}{5.969494in}}%
\pgfpathlineto{\pgfqpoint{1.551634in}{6.044760in}}%
\pgfpathlineto{\pgfqpoint{1.567068in}{6.112074in}}%
\pgfpathlineto{\pgfqpoint{1.582502in}{6.170106in}}%
\pgfpathlineto{\pgfqpoint{1.597935in}{6.217705in}}%
\pgfpathlineto{\pgfqpoint{1.613369in}{6.253920in}}%
\pgfpathlineto{\pgfqpoint{1.628803in}{6.278021in}}%
\pgfpathlineto{\pgfqpoint{1.644236in}{6.289514in}}%
\pgfpathlineto{\pgfqpoint{1.659670in}{6.288148in}}%
\pgfpathlineto{\pgfqpoint{1.675104in}{6.273924in}}%
\pgfpathlineto{\pgfqpoint{1.690537in}{6.247091in}}%
\pgfpathlineto{\pgfqpoint{1.705971in}{6.208145in}}%
\pgfpathlineto{\pgfqpoint{1.721404in}{6.157817in}}%
\pgfpathlineto{\pgfqpoint{1.736838in}{6.097056in}}%
\pgfpathlineto{\pgfqpoint{1.752272in}{6.027016in}}%
\pgfpathlineto{\pgfqpoint{1.767705in}{5.949027in}}%
\pgfpathlineto{\pgfqpoint{1.798573in}{5.775258in}}%
\pgfpathlineto{\pgfqpoint{1.875741in}{5.316358in}}%
\pgfpathlineto{\pgfqpoint{1.891174in}{5.234231in}}%
\pgfpathlineto{\pgfqpoint{1.906608in}{5.159069in}}%
\pgfpathlineto{\pgfqpoint{1.922042in}{5.092239in}}%
\pgfpathlineto{\pgfqpoint{1.937475in}{5.034930in}}%
\pgfpathlineto{\pgfqpoint{1.952909in}{4.988139in}}%
\pgfpathlineto{\pgfqpoint{1.968343in}{4.952643in}}%
\pgfpathlineto{\pgfqpoint{1.983776in}{4.928987in}}%
\pgfpathlineto{\pgfqpoint{1.999210in}{4.917475in}}%
\pgfpathlineto{\pgfqpoint{2.014644in}{4.918160in}}%
\pgfpathlineto{\pgfqpoint{2.030077in}{4.930848in}}%
\pgfpathlineto{\pgfqpoint{2.045511in}{4.955096in}}%
\pgfpathlineto{\pgfqpoint{2.060944in}{4.990228in}}%
\pgfpathlineto{\pgfqpoint{2.076378in}{5.035342in}}%
\pgfpathlineto{\pgfqpoint{2.091812in}{5.089334in}}%
\pgfpathlineto{\pgfqpoint{2.107245in}{5.150917in}}%
\pgfpathlineto{\pgfqpoint{2.138113in}{5.290956in}}%
\pgfpathlineto{\pgfqpoint{2.199847in}{5.591937in}}%
\pgfpathlineto{\pgfqpoint{2.215281in}{5.661441in}}%
\pgfpathlineto{\pgfqpoint{2.230714in}{5.725304in}}%
\pgfpathlineto{\pgfqpoint{2.246148in}{5.782014in}}%
\pgfpathlineto{\pgfqpoint{2.261582in}{5.830199in}}%
\pgfpathlineto{\pgfqpoint{2.277015in}{5.868657in}}%
\pgfpathlineto{\pgfqpoint{2.292449in}{5.896380in}}%
\pgfpathlineto{\pgfqpoint{2.307883in}{5.912572in}}%
\pgfpathlineto{\pgfqpoint{2.323316in}{5.916669in}}%
\pgfpathlineto{\pgfqpoint{2.338750in}{5.908348in}}%
\pgfpathlineto{\pgfqpoint{2.354184in}{5.887535in}}%
\pgfpathlineto{\pgfqpoint{2.369617in}{5.854404in}}%
\pgfpathlineto{\pgfqpoint{2.385051in}{5.809380in}}%
\pgfpathlineto{\pgfqpoint{2.400484in}{5.753123in}}%
\pgfpathlineto{\pgfqpoint{2.415918in}{5.686519in}}%
\pgfpathlineto{\pgfqpoint{2.431352in}{5.610665in}}%
\pgfpathlineto{\pgfqpoint{2.446785in}{5.526841in}}%
\pgfpathlineto{\pgfqpoint{2.477653in}{5.341189in}}%
\pgfpathlineto{\pgfqpoint{2.554821in}{4.850691in}}%
\pgfpathlineto{\pgfqpoint{2.570254in}{4.762016in}}%
\pgfpathlineto{\pgfqpoint{2.585688in}{4.680318in}}%
\pgfpathlineto{\pgfqpoint{2.601122in}{4.607020in}}%
\pgfpathlineto{\pgfqpoint{2.616555in}{4.543380in}}%
\pgfpathlineto{\pgfqpoint{2.631989in}{4.490464in}}%
\pgfpathlineto{\pgfqpoint{2.647423in}{4.449130in}}%
\pgfpathlineto{\pgfqpoint{2.662856in}{4.420008in}}%
\pgfpathlineto{\pgfqpoint{2.678290in}{4.403485in}}%
\pgfpathlineto{\pgfqpoint{2.693724in}{4.399703in}}%
\pgfpathlineto{\pgfqpoint{2.709157in}{4.408553in}}%
\pgfpathlineto{\pgfqpoint{2.724591in}{4.429676in}}%
\pgfpathlineto{\pgfqpoint{2.740024in}{4.462474in}}%
\pgfpathlineto{\pgfqpoint{2.755458in}{4.506121in}}%
\pgfpathlineto{\pgfqpoint{2.770892in}{4.559577in}}%
\pgfpathlineto{\pgfqpoint{2.786325in}{4.621614in}}%
\pgfpathlineto{\pgfqpoint{2.801759in}{4.690835in}}%
\pgfpathlineto{\pgfqpoint{2.832626in}{4.844589in}}%
\pgfpathlineto{\pgfqpoint{2.894361in}{5.165552in}}%
\pgfpathlineto{\pgfqpoint{2.909794in}{5.238509in}}%
\pgfpathlineto{\pgfqpoint{2.925228in}{5.305300in}}%
\pgfpathlineto{\pgfqpoint{2.940662in}{5.364493in}}%
\pgfpathlineto{\pgfqpoint{2.956095in}{5.414818in}}%
\pgfpathlineto{\pgfqpoint{2.971529in}{5.455188in}}%
\pgfpathlineto{\pgfqpoint{2.986963in}{5.484723in}}%
\pgfpathlineto{\pgfqpoint{3.002396in}{5.502767in}}%
\pgfpathlineto{\pgfqpoint{3.017830in}{5.508905in}}%
\pgfpathlineto{\pgfqpoint{3.033263in}{5.502965in}}%
\pgfpathlineto{\pgfqpoint{3.048697in}{5.485025in}}%
\pgfpathlineto{\pgfqpoint{3.064131in}{5.455413in}}%
\pgfpathlineto{\pgfqpoint{3.079564in}{5.414697in}}%
\pgfpathlineto{\pgfqpoint{3.094998in}{5.363676in}}%
\pgfpathlineto{\pgfqpoint{3.110432in}{5.303364in}}%
\pgfpathlineto{\pgfqpoint{3.125865in}{5.234966in}}%
\pgfpathlineto{\pgfqpoint{3.156733in}{5.079567in}}%
\pgfpathlineto{\pgfqpoint{3.233901in}{4.659389in}}%
\pgfpathlineto{\pgfqpoint{3.249334in}{4.583763in}}%
\pgfpathlineto{\pgfqpoint{3.264768in}{4.514806in}}%
\pgfpathlineto{\pgfqpoint{3.280202in}{4.453976in}}%
\pgfpathlineto{\pgfqpoint{3.295635in}{4.402573in}}%
\pgfpathlineto{\pgfqpoint{3.311069in}{4.361718in}}%
\pgfpathlineto{\pgfqpoint{3.326503in}{4.332325in}}%
\pgfpathlineto{\pgfqpoint{3.341936in}{4.315087in}}%
\pgfpathlineto{\pgfqpoint{3.357370in}{4.310460in}}%
\pgfpathlineto{\pgfqpoint{3.372803in}{4.318654in}}%
\pgfpathlineto{\pgfqpoint{3.388237in}{4.339629in}}%
\pgfpathlineto{\pgfqpoint{3.403671in}{4.373094in}}%
\pgfpathlineto{\pgfqpoint{3.419104in}{4.418518in}}%
\pgfpathlineto{\pgfqpoint{3.434538in}{4.475131in}}%
\pgfpathlineto{\pgfqpoint{3.449972in}{4.541950in}}%
\pgfpathlineto{\pgfqpoint{3.465405in}{4.617790in}}%
\pgfpathlineto{\pgfqpoint{3.480839in}{4.701293in}}%
\pgfpathlineto{\pgfqpoint{3.511706in}{4.885145in}}%
\pgfpathlineto{\pgfqpoint{3.588874in}{5.363179in}}%
\pgfpathlineto{\pgfqpoint{3.604308in}{5.448087in}}%
\pgfpathlineto{\pgfqpoint{3.619742in}{5.525703in}}%
\pgfpathlineto{\pgfqpoint{3.635175in}{5.594682in}}%
\pgfpathlineto{\pgfqpoint{3.650609in}{5.653856in}}%
\pgfpathlineto{\pgfqpoint{3.666043in}{5.702254in}}%
\pgfpathlineto{\pgfqpoint{3.681476in}{5.739128in}}%
\pgfpathlineto{\pgfqpoint{3.696910in}{5.763961in}}%
\pgfpathlineto{\pgfqpoint{3.712343in}{5.776483in}}%
\pgfpathlineto{\pgfqpoint{3.727777in}{5.776672in}}%
\pgfpathlineto{\pgfqpoint{3.743211in}{5.764754in}}%
\pgfpathlineto{\pgfqpoint{3.758644in}{5.741202in}}%
\pgfpathlineto{\pgfqpoint{3.774078in}{5.706724in}}%
\pgfpathlineto{\pgfqpoint{3.789512in}{5.662247in}}%
\pgfpathlineto{\pgfqpoint{3.804945in}{5.608900in}}%
\pgfpathlineto{\pgfqpoint{3.820379in}{5.547994in}}%
\pgfpathlineto{\pgfqpoint{3.851246in}{5.409466in}}%
\pgfpathlineto{\pgfqpoint{3.912981in}{5.112368in}}%
\pgfpathlineto{\pgfqpoint{3.928414in}{5.044007in}}%
\pgfpathlineto{\pgfqpoint{3.943848in}{4.981334in}}%
\pgfpathlineto{\pgfqpoint{3.959282in}{4.925843in}}%
\pgfpathlineto{\pgfqpoint{3.974715in}{4.878885in}}%
\pgfpathlineto{\pgfqpoint{3.990149in}{4.841635in}}%
\pgfpathlineto{\pgfqpoint{4.005583in}{4.815075in}}%
\pgfpathlineto{\pgfqpoint{4.021016in}{4.799967in}}%
\pgfpathlineto{\pgfqpoint{4.036450in}{4.796845in}}%
\pgfpathlineto{\pgfqpoint{4.051883in}{4.805997in}}%
\pgfpathlineto{\pgfqpoint{4.067317in}{4.827464in}}%
\pgfpathlineto{\pgfqpoint{4.082751in}{4.861037in}}%
\pgfpathlineto{\pgfqpoint{4.098184in}{4.906261in}}%
\pgfpathlineto{\pgfqpoint{4.113618in}{4.962444in}}%
\pgfpathlineto{\pgfqpoint{4.129052in}{5.028670in}}%
\pgfpathlineto{\pgfqpoint{4.144485in}{5.103818in}}%
\pgfpathlineto{\pgfqpoint{4.159919in}{5.186585in}}%
\pgfpathlineto{\pgfqpoint{4.190786in}{5.369005in}}%
\pgfpathlineto{\pgfqpoint{4.267954in}{5.844590in}}%
\pgfpathlineto{\pgfqpoint{4.283388in}{5.929150in}}%
\pgfpathlineto{\pgfqpoint{4.298822in}{6.006391in}}%
\pgfpathlineto{\pgfqpoint{4.314255in}{6.074916in}}%
\pgfpathlineto{\pgfqpoint{4.329689in}{6.133497in}}%
\pgfpathlineto{\pgfqpoint{4.345122in}{6.181097in}}%
\pgfpathlineto{\pgfqpoint{4.360556in}{6.216895in}}%
\pgfpathlineto{\pgfqpoint{4.375990in}{6.240297in}}%
\pgfpathlineto{\pgfqpoint{4.391423in}{6.250954in}}%
\pgfpathlineto{\pgfqpoint{4.406857in}{6.248763in}}%
\pgfpathlineto{\pgfqpoint{4.422291in}{6.233874in}}%
\pgfpathlineto{\pgfqpoint{4.437724in}{6.206682in}}%
\pgfpathlineto{\pgfqpoint{4.453158in}{6.167823in}}%
\pgfpathlineto{\pgfqpoint{4.468592in}{6.118160in}}%
\pgfpathlineto{\pgfqpoint{4.484025in}{6.058763in}}%
\pgfpathlineto{\pgfqpoint{4.499459in}{5.990891in}}%
\pgfpathlineto{\pgfqpoint{4.530326in}{5.835545in}}%
\pgfpathlineto{\pgfqpoint{4.622928in}{5.333301in}}%
\pgfpathlineto{\pgfqpoint{4.638362in}{5.262253in}}%
\pgfpathlineto{\pgfqpoint{4.653795in}{5.198963in}}%
\pgfpathlineto{\pgfqpoint{4.669229in}{5.144682in}}%
\pgfpathlineto{\pgfqpoint{4.669229in}{5.144682in}}%
\pgfusepath{stroke}%
\end{pgfscope}%
\begin{pgfscope}%
\pgfpathrectangle{\pgfqpoint{0.634105in}{3.881603in}}{\pgfqpoint{4.227273in}{2.800000in}} %
\pgfusepath{clip}%
\pgfsetrectcap%
\pgfsetroundjoin%
\pgfsetlinewidth{0.501875pt}%
\definecolor{currentstroke}{rgb}{0.264706,0.361242,0.982973}%
\pgfsetstrokecolor{currentstroke}%
\pgfsetdash{}{0pt}%
\pgfpathmoveto{\pgfqpoint{0.826254in}{5.486831in}}%
\pgfpathlineto{\pgfqpoint{0.841687in}{5.528472in}}%
\pgfpathlineto{\pgfqpoint{0.857121in}{5.566224in}}%
\pgfpathlineto{\pgfqpoint{0.872555in}{5.599176in}}%
\pgfpathlineto{\pgfqpoint{0.887988in}{5.626508in}}%
\pgfpathlineto{\pgfqpoint{0.903422in}{5.647508in}}%
\pgfpathlineto{\pgfqpoint{0.918855in}{5.661589in}}%
\pgfpathlineto{\pgfqpoint{0.934289in}{5.668305in}}%
\pgfpathlineto{\pgfqpoint{0.949723in}{5.667356in}}%
\pgfpathlineto{\pgfqpoint{0.965156in}{5.658600in}}%
\pgfpathlineto{\pgfqpoint{0.980590in}{5.642058in}}%
\pgfpathlineto{\pgfqpoint{0.996024in}{5.617912in}}%
\pgfpathlineto{\pgfqpoint{1.011457in}{5.586508in}}%
\pgfpathlineto{\pgfqpoint{1.026891in}{5.548346in}}%
\pgfpathlineto{\pgfqpoint{1.042325in}{5.504073in}}%
\pgfpathlineto{\pgfqpoint{1.057758in}{5.454474in}}%
\pgfpathlineto{\pgfqpoint{1.088625in}{5.343021in}}%
\pgfpathlineto{\pgfqpoint{1.165794in}{5.044849in}}%
\pgfpathlineto{\pgfqpoint{1.181227in}{4.991438in}}%
\pgfpathlineto{\pgfqpoint{1.196661in}{4.942843in}}%
\pgfpathlineto{\pgfqpoint{1.212095in}{4.900129in}}%
\pgfpathlineto{\pgfqpoint{1.227528in}{4.864261in}}%
\pgfpathlineto{\pgfqpoint{1.242962in}{4.836093in}}%
\pgfpathlineto{\pgfqpoint{1.258395in}{4.816344in}}%
\pgfpathlineto{\pgfqpoint{1.273829in}{4.805585in}}%
\pgfpathlineto{\pgfqpoint{1.289263in}{4.804230in}}%
\pgfpathlineto{\pgfqpoint{1.304696in}{4.812522in}}%
\pgfpathlineto{\pgfqpoint{1.320130in}{4.830533in}}%
\pgfpathlineto{\pgfqpoint{1.335564in}{4.858156in}}%
\pgfpathlineto{\pgfqpoint{1.350997in}{4.895109in}}%
\pgfpathlineto{\pgfqpoint{1.366431in}{4.940941in}}%
\pgfpathlineto{\pgfqpoint{1.381864in}{4.995034in}}%
\pgfpathlineto{\pgfqpoint{1.397298in}{5.056622in}}%
\pgfpathlineto{\pgfqpoint{1.412732in}{5.124796in}}%
\pgfpathlineto{\pgfqpoint{1.443599in}{5.276685in}}%
\pgfpathlineto{\pgfqpoint{1.489900in}{5.525284in}}%
\pgfpathlineto{\pgfqpoint{1.520767in}{5.689735in}}%
\pgfpathlineto{\pgfqpoint{1.551634in}{5.841112in}}%
\pgfpathlineto{\pgfqpoint{1.567068in}{5.908924in}}%
\pgfpathlineto{\pgfqpoint{1.582502in}{5.970079in}}%
\pgfpathlineto{\pgfqpoint{1.597935in}{6.023671in}}%
\pgfpathlineto{\pgfqpoint{1.613369in}{6.068931in}}%
\pgfpathlineto{\pgfqpoint{1.628803in}{6.105242in}}%
\pgfpathlineto{\pgfqpoint{1.644236in}{6.132148in}}%
\pgfpathlineto{\pgfqpoint{1.659670in}{6.149366in}}%
\pgfpathlineto{\pgfqpoint{1.675104in}{6.156781in}}%
\pgfpathlineto{\pgfqpoint{1.690537in}{6.154458in}}%
\pgfpathlineto{\pgfqpoint{1.705971in}{6.142630in}}%
\pgfpathlineto{\pgfqpoint{1.721404in}{6.121696in}}%
\pgfpathlineto{\pgfqpoint{1.736838in}{6.092213in}}%
\pgfpathlineto{\pgfqpoint{1.752272in}{6.054880in}}%
\pgfpathlineto{\pgfqpoint{1.767705in}{6.010528in}}%
\pgfpathlineto{\pgfqpoint{1.783139in}{5.960097in}}%
\pgfpathlineto{\pgfqpoint{1.814006in}{5.845207in}}%
\pgfpathlineto{\pgfqpoint{1.860307in}{5.654974in}}%
\pgfpathlineto{\pgfqpoint{1.891174in}{5.529841in}}%
\pgfpathlineto{\pgfqpoint{1.922042in}{5.416229in}}%
\pgfpathlineto{\pgfqpoint{1.937475in}{5.366079in}}%
\pgfpathlineto{\pgfqpoint{1.952909in}{5.321388in}}%
\pgfpathlineto{\pgfqpoint{1.968343in}{5.282760in}}%
\pgfpathlineto{\pgfqpoint{1.983776in}{5.250649in}}%
\pgfpathlineto{\pgfqpoint{1.999210in}{5.225359in}}%
\pgfpathlineto{\pgfqpoint{2.014644in}{5.207035in}}%
\pgfpathlineto{\pgfqpoint{2.030077in}{5.195663in}}%
\pgfpathlineto{\pgfqpoint{2.045511in}{5.191070in}}%
\pgfpathlineto{\pgfqpoint{2.060944in}{5.192933in}}%
\pgfpathlineto{\pgfqpoint{2.076378in}{5.200786in}}%
\pgfpathlineto{\pgfqpoint{2.091812in}{5.214028in}}%
\pgfpathlineto{\pgfqpoint{2.107245in}{5.231943in}}%
\pgfpathlineto{\pgfqpoint{2.122679in}{5.253711in}}%
\pgfpathlineto{\pgfqpoint{2.153546in}{5.305134in}}%
\pgfpathlineto{\pgfqpoint{2.199847in}{5.387020in}}%
\pgfpathlineto{\pgfqpoint{2.215281in}{5.411522in}}%
\pgfpathlineto{\pgfqpoint{2.230714in}{5.433012in}}%
\pgfpathlineto{\pgfqpoint{2.246148in}{5.450610in}}%
\pgfpathlineto{\pgfqpoint{2.261582in}{5.463529in}}%
\pgfpathlineto{\pgfqpoint{2.277015in}{5.471084in}}%
\pgfpathlineto{\pgfqpoint{2.292449in}{5.472715in}}%
\pgfpathlineto{\pgfqpoint{2.307883in}{5.467996in}}%
\pgfpathlineto{\pgfqpoint{2.323316in}{5.456647in}}%
\pgfpathlineto{\pgfqpoint{2.338750in}{5.438540in}}%
\pgfpathlineto{\pgfqpoint{2.354184in}{5.413701in}}%
\pgfpathlineto{\pgfqpoint{2.369617in}{5.382317in}}%
\pgfpathlineto{\pgfqpoint{2.385051in}{5.344726in}}%
\pgfpathlineto{\pgfqpoint{2.400484in}{5.301416in}}%
\pgfpathlineto{\pgfqpoint{2.415918in}{5.253015in}}%
\pgfpathlineto{\pgfqpoint{2.446785in}{5.144067in}}%
\pgfpathlineto{\pgfqpoint{2.493086in}{4.964626in}}%
\pgfpathlineto{\pgfqpoint{2.523954in}{4.846957in}}%
\pgfpathlineto{\pgfqpoint{2.539387in}{4.792107in}}%
\pgfpathlineto{\pgfqpoint{2.554821in}{4.741382in}}%
\pgfpathlineto{\pgfqpoint{2.570254in}{4.695818in}}%
\pgfpathlineto{\pgfqpoint{2.585688in}{4.656371in}}%
\pgfpathlineto{\pgfqpoint{2.601122in}{4.623896in}}%
\pgfpathlineto{\pgfqpoint{2.616555in}{4.599130in}}%
\pgfpathlineto{\pgfqpoint{2.631989in}{4.582675in}}%
\pgfpathlineto{\pgfqpoint{2.647423in}{4.574987in}}%
\pgfpathlineto{\pgfqpoint{2.662856in}{4.576363in}}%
\pgfpathlineto{\pgfqpoint{2.678290in}{4.586932in}}%
\pgfpathlineto{\pgfqpoint{2.693724in}{4.606657in}}%
\pgfpathlineto{\pgfqpoint{2.709157in}{4.635326in}}%
\pgfpathlineto{\pgfqpoint{2.724591in}{4.672561in}}%
\pgfpathlineto{\pgfqpoint{2.740024in}{4.717818in}}%
\pgfpathlineto{\pgfqpoint{2.755458in}{4.770404in}}%
\pgfpathlineto{\pgfqpoint{2.770892in}{4.829480in}}%
\pgfpathlineto{\pgfqpoint{2.801759in}{4.963134in}}%
\pgfpathlineto{\pgfqpoint{2.848060in}{5.185045in}}%
\pgfpathlineto{\pgfqpoint{2.878927in}{5.332533in}}%
\pgfpathlineto{\pgfqpoint{2.909794in}{5.467719in}}%
\pgfpathlineto{\pgfqpoint{2.925228in}{5.527708in}}%
\pgfpathlineto{\pgfqpoint{2.940662in}{5.581195in}}%
\pgfpathlineto{\pgfqpoint{2.956095in}{5.627247in}}%
\pgfpathlineto{\pgfqpoint{2.971529in}{5.665061in}}%
\pgfpathlineto{\pgfqpoint{2.986963in}{5.693985in}}%
\pgfpathlineto{\pgfqpoint{3.002396in}{5.713524in}}%
\pgfpathlineto{\pgfqpoint{3.017830in}{5.723353in}}%
\pgfpathlineto{\pgfqpoint{3.033263in}{5.723325in}}%
\pgfpathlineto{\pgfqpoint{3.048697in}{5.713465in}}%
\pgfpathlineto{\pgfqpoint{3.064131in}{5.693981in}}%
\pgfpathlineto{\pgfqpoint{3.079564in}{5.665248in}}%
\pgfpathlineto{\pgfqpoint{3.094998in}{5.627810in}}%
\pgfpathlineto{\pgfqpoint{3.110432in}{5.582363in}}%
\pgfpathlineto{\pgfqpoint{3.125865in}{5.529744in}}%
\pgfpathlineto{\pgfqpoint{3.141299in}{5.470915in}}%
\pgfpathlineto{\pgfqpoint{3.172166in}{5.338977in}}%
\pgfpathlineto{\pgfqpoint{3.264768in}{4.916776in}}%
\pgfpathlineto{\pgfqpoint{3.280202in}{4.855399in}}%
\pgfpathlineto{\pgfqpoint{3.295635in}{4.799519in}}%
\pgfpathlineto{\pgfqpoint{3.311069in}{4.750014in}}%
\pgfpathlineto{\pgfqpoint{3.326503in}{4.707632in}}%
\pgfpathlineto{\pgfqpoint{3.341936in}{4.672976in}}%
\pgfpathlineto{\pgfqpoint{3.357370in}{4.646493in}}%
\pgfpathlineto{\pgfqpoint{3.372803in}{4.628467in}}%
\pgfpathlineto{\pgfqpoint{3.388237in}{4.619017in}}%
\pgfpathlineto{\pgfqpoint{3.403671in}{4.618094in}}%
\pgfpathlineto{\pgfqpoint{3.419104in}{4.625486in}}%
\pgfpathlineto{\pgfqpoint{3.434538in}{4.640823in}}%
\pgfpathlineto{\pgfqpoint{3.449972in}{4.663586in}}%
\pgfpathlineto{\pgfqpoint{3.465405in}{4.693122in}}%
\pgfpathlineto{\pgfqpoint{3.480839in}{4.728654in}}%
\pgfpathlineto{\pgfqpoint{3.496273in}{4.769302in}}%
\pgfpathlineto{\pgfqpoint{3.527140in}{4.862023in}}%
\pgfpathlineto{\pgfqpoint{3.604308in}{5.110810in}}%
\pgfpathlineto{\pgfqpoint{3.619742in}{5.155111in}}%
\pgfpathlineto{\pgfqpoint{3.635175in}{5.195427in}}%
\pgfpathlineto{\pgfqpoint{3.650609in}{5.231012in}}%
\pgfpathlineto{\pgfqpoint{3.666043in}{5.261246in}}%
\pgfpathlineto{\pgfqpoint{3.681476in}{5.285641in}}%
\pgfpathlineto{\pgfqpoint{3.696910in}{5.303855in}}%
\pgfpathlineto{\pgfqpoint{3.712343in}{5.315695in}}%
\pgfpathlineto{\pgfqpoint{3.727777in}{5.321127in}}%
\pgfpathlineto{\pgfqpoint{3.743211in}{5.320268in}}%
\pgfpathlineto{\pgfqpoint{3.758644in}{5.313391in}}%
\pgfpathlineto{\pgfqpoint{3.774078in}{5.300911in}}%
\pgfpathlineto{\pgfqpoint{3.789512in}{5.283383in}}%
\pgfpathlineto{\pgfqpoint{3.804945in}{5.261485in}}%
\pgfpathlineto{\pgfqpoint{3.820379in}{5.236004in}}%
\pgfpathlineto{\pgfqpoint{3.851246in}{5.177893in}}%
\pgfpathlineto{\pgfqpoint{3.897547in}{5.087822in}}%
\pgfpathlineto{\pgfqpoint{3.912981in}{5.061168in}}%
\pgfpathlineto{\pgfqpoint{3.928414in}{5.037884in}}%
\pgfpathlineto{\pgfqpoint{3.943848in}{5.018905in}}%
\pgfpathlineto{\pgfqpoint{3.959282in}{5.005086in}}%
\pgfpathlineto{\pgfqpoint{3.974715in}{4.997185in}}%
\pgfpathlineto{\pgfqpoint{3.990149in}{4.995841in}}%
\pgfpathlineto{\pgfqpoint{4.005583in}{5.001562in}}%
\pgfpathlineto{\pgfqpoint{4.021016in}{5.014714in}}%
\pgfpathlineto{\pgfqpoint{4.036450in}{5.035507in}}%
\pgfpathlineto{\pgfqpoint{4.051883in}{5.063993in}}%
\pgfpathlineto{\pgfqpoint{4.067317in}{5.100061in}}%
\pgfpathlineto{\pgfqpoint{4.082751in}{5.143437in}}%
\pgfpathlineto{\pgfqpoint{4.098184in}{5.193687in}}%
\pgfpathlineto{\pgfqpoint{4.113618in}{5.250227in}}%
\pgfpathlineto{\pgfqpoint{4.129052in}{5.312328in}}%
\pgfpathlineto{\pgfqpoint{4.159919in}{5.449667in}}%
\pgfpathlineto{\pgfqpoint{4.252521in}{5.887275in}}%
\pgfpathlineto{\pgfqpoint{4.267954in}{5.951296in}}%
\pgfpathlineto{\pgfqpoint{4.283388in}{6.009632in}}%
\pgfpathlineto{\pgfqpoint{4.298822in}{6.061244in}}%
\pgfpathlineto{\pgfqpoint{4.314255in}{6.105201in}}%
\pgfpathlineto{\pgfqpoint{4.329689in}{6.140694in}}%
\pgfpathlineto{\pgfqpoint{4.345122in}{6.167054in}}%
\pgfpathlineto{\pgfqpoint{4.360556in}{6.183766in}}%
\pgfpathlineto{\pgfqpoint{4.375990in}{6.190480in}}%
\pgfpathlineto{\pgfqpoint{4.391423in}{6.187017in}}%
\pgfpathlineto{\pgfqpoint{4.406857in}{6.173374in}}%
\pgfpathlineto{\pgfqpoint{4.422291in}{6.149726in}}%
\pgfpathlineto{\pgfqpoint{4.437724in}{6.116421in}}%
\pgfpathlineto{\pgfqpoint{4.453158in}{6.073977in}}%
\pgfpathlineto{\pgfqpoint{4.468592in}{6.023071in}}%
\pgfpathlineto{\pgfqpoint{4.484025in}{5.964527in}}%
\pgfpathlineto{\pgfqpoint{4.499459in}{5.899301in}}%
\pgfpathlineto{\pgfqpoint{4.530326in}{5.753185in}}%
\pgfpathlineto{\pgfqpoint{4.576627in}{5.513293in}}%
\pgfpathlineto{\pgfqpoint{4.607494in}{5.354731in}}%
\pgfpathlineto{\pgfqpoint{4.638362in}{5.209137in}}%
\pgfpathlineto{\pgfqpoint{4.653795in}{5.144084in}}%
\pgfpathlineto{\pgfqpoint{4.669229in}{5.085533in}}%
\pgfpathlineto{\pgfqpoint{4.669229in}{5.085533in}}%
\pgfusepath{stroke}%
\end{pgfscope}%
\begin{pgfscope}%
\pgfpathrectangle{\pgfqpoint{0.634105in}{3.881603in}}{\pgfqpoint{4.227273in}{2.800000in}} %
\pgfusepath{clip}%
\pgfsetrectcap%
\pgfsetroundjoin%
\pgfsetlinewidth{0.501875pt}%
\definecolor{currentstroke}{rgb}{0.186275,0.473094,0.969797}%
\pgfsetstrokecolor{currentstroke}%
\pgfsetdash{}{0pt}%
\pgfpathmoveto{\pgfqpoint{0.826254in}{5.717316in}}%
\pgfpathlineto{\pgfqpoint{0.841687in}{5.782577in}}%
\pgfpathlineto{\pgfqpoint{0.857121in}{5.834661in}}%
\pgfpathlineto{\pgfqpoint{0.872555in}{5.872120in}}%
\pgfpathlineto{\pgfqpoint{0.887988in}{5.893924in}}%
\pgfpathlineto{\pgfqpoint{0.903422in}{5.899499in}}%
\pgfpathlineto{\pgfqpoint{0.918855in}{5.888734in}}%
\pgfpathlineto{\pgfqpoint{0.934289in}{5.861982in}}%
\pgfpathlineto{\pgfqpoint{0.949723in}{5.820045in}}%
\pgfpathlineto{\pgfqpoint{0.965156in}{5.764149in}}%
\pgfpathlineto{\pgfqpoint{0.980590in}{5.695904in}}%
\pgfpathlineto{\pgfqpoint{0.996024in}{5.617251in}}%
\pgfpathlineto{\pgfqpoint{1.026891in}{5.437783in}}%
\pgfpathlineto{\pgfqpoint{1.088625in}{5.061054in}}%
\pgfpathlineto{\pgfqpoint{1.104059in}{4.977900in}}%
\pgfpathlineto{\pgfqpoint{1.119493in}{4.903679in}}%
\pgfpathlineto{\pgfqpoint{1.134926in}{4.840215in}}%
\pgfpathlineto{\pgfqpoint{1.150360in}{4.789002in}}%
\pgfpathlineto{\pgfqpoint{1.165794in}{4.751164in}}%
\pgfpathlineto{\pgfqpoint{1.181227in}{4.727431in}}%
\pgfpathlineto{\pgfqpoint{1.196661in}{4.718131in}}%
\pgfpathlineto{\pgfqpoint{1.212095in}{4.723183in}}%
\pgfpathlineto{\pgfqpoint{1.227528in}{4.742114in}}%
\pgfpathlineto{\pgfqpoint{1.242962in}{4.774082in}}%
\pgfpathlineto{\pgfqpoint{1.258395in}{4.817912in}}%
\pgfpathlineto{\pgfqpoint{1.273829in}{4.872138in}}%
\pgfpathlineto{\pgfqpoint{1.289263in}{4.935056in}}%
\pgfpathlineto{\pgfqpoint{1.320130in}{5.079326in}}%
\pgfpathlineto{\pgfqpoint{1.381864in}{5.385121in}}%
\pgfpathlineto{\pgfqpoint{1.397298in}{5.454097in}}%
\pgfpathlineto{\pgfqpoint{1.412732in}{5.516934in}}%
\pgfpathlineto{\pgfqpoint{1.428165in}{5.572481in}}%
\pgfpathlineto{\pgfqpoint{1.443599in}{5.619882in}}%
\pgfpathlineto{\pgfqpoint{1.459033in}{5.658587in}}%
\pgfpathlineto{\pgfqpoint{1.474466in}{5.688362in}}%
\pgfpathlineto{\pgfqpoint{1.489900in}{5.709280in}}%
\pgfpathlineto{\pgfqpoint{1.505334in}{5.721707in}}%
\pgfpathlineto{\pgfqpoint{1.520767in}{5.726274in}}%
\pgfpathlineto{\pgfqpoint{1.536201in}{5.723847in}}%
\pgfpathlineto{\pgfqpoint{1.551634in}{5.715483in}}%
\pgfpathlineto{\pgfqpoint{1.567068in}{5.702382in}}%
\pgfpathlineto{\pgfqpoint{1.582502in}{5.685841in}}%
\pgfpathlineto{\pgfqpoint{1.644236in}{5.611512in}}%
\pgfpathlineto{\pgfqpoint{1.659670in}{5.596830in}}%
\pgfpathlineto{\pgfqpoint{1.675104in}{5.585586in}}%
\pgfpathlineto{\pgfqpoint{1.690537in}{5.578368in}}%
\pgfpathlineto{\pgfqpoint{1.705971in}{5.575530in}}%
\pgfpathlineto{\pgfqpoint{1.721404in}{5.577176in}}%
\pgfpathlineto{\pgfqpoint{1.736838in}{5.583163in}}%
\pgfpathlineto{\pgfqpoint{1.752272in}{5.593108in}}%
\pgfpathlineto{\pgfqpoint{1.767705in}{5.606407in}}%
\pgfpathlineto{\pgfqpoint{1.783139in}{5.622262in}}%
\pgfpathlineto{\pgfqpoint{1.844874in}{5.690530in}}%
\pgfpathlineto{\pgfqpoint{1.860307in}{5.703062in}}%
\pgfpathlineto{\pgfqpoint{1.875741in}{5.711523in}}%
\pgfpathlineto{\pgfqpoint{1.891174in}{5.714937in}}%
\pgfpathlineto{\pgfqpoint{1.906608in}{5.712487in}}%
\pgfpathlineto{\pgfqpoint{1.922042in}{5.703552in}}%
\pgfpathlineto{\pgfqpoint{1.937475in}{5.687734in}}%
\pgfpathlineto{\pgfqpoint{1.952909in}{5.664881in}}%
\pgfpathlineto{\pgfqpoint{1.968343in}{5.635101in}}%
\pgfpathlineto{\pgfqpoint{1.983776in}{5.598762in}}%
\pgfpathlineto{\pgfqpoint{1.999210in}{5.556486in}}%
\pgfpathlineto{\pgfqpoint{2.014644in}{5.509136in}}%
\pgfpathlineto{\pgfqpoint{2.045511in}{5.403692in}}%
\pgfpathlineto{\pgfqpoint{2.091812in}{5.239291in}}%
\pgfpathlineto{\pgfqpoint{2.107245in}{5.188825in}}%
\pgfpathlineto{\pgfqpoint{2.122679in}{5.142987in}}%
\pgfpathlineto{\pgfqpoint{2.138113in}{5.103111in}}%
\pgfpathlineto{\pgfqpoint{2.153546in}{5.070360in}}%
\pgfpathlineto{\pgfqpoint{2.168980in}{5.045683in}}%
\pgfpathlineto{\pgfqpoint{2.184414in}{5.029781in}}%
\pgfpathlineto{\pgfqpoint{2.199847in}{5.023069in}}%
\pgfpathlineto{\pgfqpoint{2.215281in}{5.025661in}}%
\pgfpathlineto{\pgfqpoint{2.230714in}{5.037357in}}%
\pgfpathlineto{\pgfqpoint{2.246148in}{5.057641in}}%
\pgfpathlineto{\pgfqpoint{2.261582in}{5.085692in}}%
\pgfpathlineto{\pgfqpoint{2.277015in}{5.120404in}}%
\pgfpathlineto{\pgfqpoint{2.292449in}{5.160418in}}%
\pgfpathlineto{\pgfqpoint{2.323316in}{5.249897in}}%
\pgfpathlineto{\pgfqpoint{2.354184in}{5.339895in}}%
\pgfpathlineto{\pgfqpoint{2.369617in}{5.380369in}}%
\pgfpathlineto{\pgfqpoint{2.385051in}{5.415384in}}%
\pgfpathlineto{\pgfqpoint{2.400484in}{5.443261in}}%
\pgfpathlineto{\pgfqpoint{2.415918in}{5.462522in}}%
\pgfpathlineto{\pgfqpoint{2.431352in}{5.471937in}}%
\pgfpathlineto{\pgfqpoint{2.446785in}{5.470576in}}%
\pgfpathlineto{\pgfqpoint{2.462219in}{5.457842in}}%
\pgfpathlineto{\pgfqpoint{2.477653in}{5.433505in}}%
\pgfpathlineto{\pgfqpoint{2.493086in}{5.397716in}}%
\pgfpathlineto{\pgfqpoint{2.508520in}{5.351010in}}%
\pgfpathlineto{\pgfqpoint{2.523954in}{5.294306in}}%
\pgfpathlineto{\pgfqpoint{2.539387in}{5.228878in}}%
\pgfpathlineto{\pgfqpoint{2.570254in}{5.078559in}}%
\pgfpathlineto{\pgfqpoint{2.631989in}{4.759895in}}%
\pgfpathlineto{\pgfqpoint{2.647423in}{4.690300in}}%
\pgfpathlineto{\pgfqpoint{2.662856in}{4.629430in}}%
\pgfpathlineto{\pgfqpoint{2.678290in}{4.579383in}}%
\pgfpathlineto{\pgfqpoint{2.693724in}{4.542003in}}%
\pgfpathlineto{\pgfqpoint{2.709157in}{4.518817in}}%
\pgfpathlineto{\pgfqpoint{2.724591in}{4.510986in}}%
\pgfpathlineto{\pgfqpoint{2.740024in}{4.519257in}}%
\pgfpathlineto{\pgfqpoint{2.755458in}{4.543939in}}%
\pgfpathlineto{\pgfqpoint{2.770892in}{4.584880in}}%
\pgfpathlineto{\pgfqpoint{2.786325in}{4.641465in}}%
\pgfpathlineto{\pgfqpoint{2.801759in}{4.712627in}}%
\pgfpathlineto{\pgfqpoint{2.817193in}{4.796867in}}%
\pgfpathlineto{\pgfqpoint{2.832626in}{4.892290in}}%
\pgfpathlineto{\pgfqpoint{2.863493in}{5.107451in}}%
\pgfpathlineto{\pgfqpoint{2.925228in}{5.558307in}}%
\pgfpathlineto{\pgfqpoint{2.940662in}{5.658421in}}%
\pgfpathlineto{\pgfqpoint{2.956095in}{5.747990in}}%
\pgfpathlineto{\pgfqpoint{2.971529in}{5.824632in}}%
\pgfpathlineto{\pgfqpoint{2.986963in}{5.886299in}}%
\pgfpathlineto{\pgfqpoint{3.002396in}{5.931329in}}%
\pgfpathlineto{\pgfqpoint{3.017830in}{5.958496in}}%
\pgfpathlineto{\pgfqpoint{3.033263in}{5.967049in}}%
\pgfpathlineto{\pgfqpoint{3.048697in}{5.956732in}}%
\pgfpathlineto{\pgfqpoint{3.064131in}{5.927794in}}%
\pgfpathlineto{\pgfqpoint{3.079564in}{5.880979in}}%
\pgfpathlineto{\pgfqpoint{3.094998in}{5.817512in}}%
\pgfpathlineto{\pgfqpoint{3.110432in}{5.739058in}}%
\pgfpathlineto{\pgfqpoint{3.125865in}{5.647679in}}%
\pgfpathlineto{\pgfqpoint{3.141299in}{5.545772in}}%
\pgfpathlineto{\pgfqpoint{3.172166in}{5.321239in}}%
\pgfpathlineto{\pgfqpoint{3.218467in}{4.976714in}}%
\pgfpathlineto{\pgfqpoint{3.233901in}{4.871524in}}%
\pgfpathlineto{\pgfqpoint{3.249334in}{4.775639in}}%
\pgfpathlineto{\pgfqpoint{3.264768in}{4.691356in}}%
\pgfpathlineto{\pgfqpoint{3.280202in}{4.620624in}}%
\pgfpathlineto{\pgfqpoint{3.295635in}{4.564987in}}%
\pgfpathlineto{\pgfqpoint{3.311069in}{4.525549in}}%
\pgfpathlineto{\pgfqpoint{3.326503in}{4.502945in}}%
\pgfpathlineto{\pgfqpoint{3.341936in}{4.497330in}}%
\pgfpathlineto{\pgfqpoint{3.357370in}{4.508382in}}%
\pgfpathlineto{\pgfqpoint{3.372803in}{4.535314in}}%
\pgfpathlineto{\pgfqpoint{3.388237in}{4.576909in}}%
\pgfpathlineto{\pgfqpoint{3.403671in}{4.631557in}}%
\pgfpathlineto{\pgfqpoint{3.419104in}{4.697312in}}%
\pgfpathlineto{\pgfqpoint{3.434538in}{4.771949in}}%
\pgfpathlineto{\pgfqpoint{3.465405in}{4.938029in}}%
\pgfpathlineto{\pgfqpoint{3.496273in}{5.109267in}}%
\pgfpathlineto{\pgfqpoint{3.511706in}{5.190448in}}%
\pgfpathlineto{\pgfqpoint{3.527140in}{5.265533in}}%
\pgfpathlineto{\pgfqpoint{3.542573in}{5.332454in}}%
\pgfpathlineto{\pgfqpoint{3.558007in}{5.389442in}}%
\pgfpathlineto{\pgfqpoint{3.573441in}{5.435077in}}%
\pgfpathlineto{\pgfqpoint{3.588874in}{5.468321in}}%
\pgfpathlineto{\pgfqpoint{3.604308in}{5.488544in}}%
\pgfpathlineto{\pgfqpoint{3.619742in}{5.495537in}}%
\pgfpathlineto{\pgfqpoint{3.635175in}{5.489508in}}%
\pgfpathlineto{\pgfqpoint{3.650609in}{5.471073in}}%
\pgfpathlineto{\pgfqpoint{3.666043in}{5.441221in}}%
\pgfpathlineto{\pgfqpoint{3.681476in}{5.401286in}}%
\pgfpathlineto{\pgfqpoint{3.696910in}{5.352890in}}%
\pgfpathlineto{\pgfqpoint{3.712343in}{5.297896in}}%
\pgfpathlineto{\pgfqpoint{3.743211in}{5.176369in}}%
\pgfpathlineto{\pgfqpoint{3.774078in}{5.053886in}}%
\pgfpathlineto{\pgfqpoint{3.789512in}{4.997591in}}%
\pgfpathlineto{\pgfqpoint{3.804945in}{4.947183in}}%
\pgfpathlineto{\pgfqpoint{3.820379in}{4.904352in}}%
\pgfpathlineto{\pgfqpoint{3.835813in}{4.870523in}}%
\pgfpathlineto{\pgfqpoint{3.851246in}{4.846824in}}%
\pgfpathlineto{\pgfqpoint{3.866680in}{4.834049in}}%
\pgfpathlineto{\pgfqpoint{3.882113in}{4.832646in}}%
\pgfpathlineto{\pgfqpoint{3.897547in}{4.842710in}}%
\pgfpathlineto{\pgfqpoint{3.912981in}{4.863985in}}%
\pgfpathlineto{\pgfqpoint{3.928414in}{4.895885in}}%
\pgfpathlineto{\pgfqpoint{3.943848in}{4.937515in}}%
\pgfpathlineto{\pgfqpoint{3.959282in}{4.987713in}}%
\pgfpathlineto{\pgfqpoint{3.974715in}{5.045089in}}%
\pgfpathlineto{\pgfqpoint{4.005583in}{5.174994in}}%
\pgfpathlineto{\pgfqpoint{4.067317in}{5.447183in}}%
\pgfpathlineto{\pgfqpoint{4.082751in}{5.508212in}}%
\pgfpathlineto{\pgfqpoint{4.098184in}{5.563673in}}%
\pgfpathlineto{\pgfqpoint{4.113618in}{5.612553in}}%
\pgfpathlineto{\pgfqpoint{4.129052in}{5.654096in}}%
\pgfpathlineto{\pgfqpoint{4.144485in}{5.687821in}}%
\pgfpathlineto{\pgfqpoint{4.159919in}{5.713528in}}%
\pgfpathlineto{\pgfqpoint{4.175352in}{5.731295in}}%
\pgfpathlineto{\pgfqpoint{4.190786in}{5.741462in}}%
\pgfpathlineto{\pgfqpoint{4.206220in}{5.744611in}}%
\pgfpathlineto{\pgfqpoint{4.221653in}{5.741536in}}%
\pgfpathlineto{\pgfqpoint{4.237087in}{5.733201in}}%
\pgfpathlineto{\pgfqpoint{4.252521in}{5.720701in}}%
\pgfpathlineto{\pgfqpoint{4.267954in}{5.705215in}}%
\pgfpathlineto{\pgfqpoint{4.329689in}{5.637055in}}%
\pgfpathlineto{\pgfqpoint{4.345122in}{5.623742in}}%
\pgfpathlineto{\pgfqpoint{4.360556in}{5.613540in}}%
\pgfpathlineto{\pgfqpoint{4.375990in}{5.606922in}}%
\pgfpathlineto{\pgfqpoint{4.391423in}{5.604129in}}%
\pgfpathlineto{\pgfqpoint{4.406857in}{5.605167in}}%
\pgfpathlineto{\pgfqpoint{4.422291in}{5.609801in}}%
\pgfpathlineto{\pgfqpoint{4.437724in}{5.617571in}}%
\pgfpathlineto{\pgfqpoint{4.453158in}{5.627806in}}%
\pgfpathlineto{\pgfqpoint{4.514892in}{5.674424in}}%
\pgfpathlineto{\pgfqpoint{4.530326in}{5.681920in}}%
\pgfpathlineto{\pgfqpoint{4.545760in}{5.685452in}}%
\pgfpathlineto{\pgfqpoint{4.561193in}{5.683968in}}%
\pgfpathlineto{\pgfqpoint{4.576627in}{5.676557in}}%
\pgfpathlineto{\pgfqpoint{4.592061in}{5.662483in}}%
\pgfpathlineto{\pgfqpoint{4.607494in}{5.641225in}}%
\pgfpathlineto{\pgfqpoint{4.622928in}{5.612510in}}%
\pgfpathlineto{\pgfqpoint{4.638362in}{5.576331in}}%
\pgfpathlineto{\pgfqpoint{4.653795in}{5.532964in}}%
\pgfpathlineto{\pgfqpoint{4.669229in}{5.482965in}}%
\pgfpathlineto{\pgfqpoint{4.669229in}{5.482965in}}%
\pgfusepath{stroke}%
\end{pgfscope}%
\begin{pgfscope}%
\pgfpathrectangle{\pgfqpoint{0.634105in}{3.881603in}}{\pgfqpoint{4.227273in}{2.800000in}} %
\pgfusepath{clip}%
\pgfsetrectcap%
\pgfsetroundjoin%
\pgfsetlinewidth{0.501875pt}%
\definecolor{currentstroke}{rgb}{0.100000,0.587785,0.951057}%
\pgfsetstrokecolor{currentstroke}%
\pgfsetdash{}{0pt}%
\pgfpathmoveto{\pgfqpoint{0.826254in}{5.671939in}}%
\pgfpathlineto{\pgfqpoint{0.841687in}{5.725988in}}%
\pgfpathlineto{\pgfqpoint{0.857121in}{5.769419in}}%
\pgfpathlineto{\pgfqpoint{0.872555in}{5.800885in}}%
\pgfpathlineto{\pgfqpoint{0.887988in}{5.819368in}}%
\pgfpathlineto{\pgfqpoint{0.903422in}{5.824206in}}%
\pgfpathlineto{\pgfqpoint{0.918855in}{5.815109in}}%
\pgfpathlineto{\pgfqpoint{0.934289in}{5.792167in}}%
\pgfpathlineto{\pgfqpoint{0.949723in}{5.755839in}}%
\pgfpathlineto{\pgfqpoint{0.965156in}{5.706947in}}%
\pgfpathlineto{\pgfqpoint{0.980590in}{5.646642in}}%
\pgfpathlineto{\pgfqpoint{0.996024in}{5.576374in}}%
\pgfpathlineto{\pgfqpoint{1.011457in}{5.497848in}}%
\pgfpathlineto{\pgfqpoint{1.042325in}{5.323829in}}%
\pgfpathlineto{\pgfqpoint{1.088625in}{5.052450in}}%
\pgfpathlineto{\pgfqpoint{1.104059in}{4.967870in}}%
\pgfpathlineto{\pgfqpoint{1.119493in}{4.889598in}}%
\pgfpathlineto{\pgfqpoint{1.134926in}{4.819409in}}%
\pgfpathlineto{\pgfqpoint{1.150360in}{4.758847in}}%
\pgfpathlineto{\pgfqpoint{1.165794in}{4.709194in}}%
\pgfpathlineto{\pgfqpoint{1.181227in}{4.671439in}}%
\pgfpathlineto{\pgfqpoint{1.196661in}{4.646257in}}%
\pgfpathlineto{\pgfqpoint{1.212095in}{4.634005in}}%
\pgfpathlineto{\pgfqpoint{1.227528in}{4.634714in}}%
\pgfpathlineto{\pgfqpoint{1.242962in}{4.648102in}}%
\pgfpathlineto{\pgfqpoint{1.258395in}{4.673591in}}%
\pgfpathlineto{\pgfqpoint{1.273829in}{4.710327in}}%
\pgfpathlineto{\pgfqpoint{1.289263in}{4.757218in}}%
\pgfpathlineto{\pgfqpoint{1.304696in}{4.812968in}}%
\pgfpathlineto{\pgfqpoint{1.320130in}{4.876119in}}%
\pgfpathlineto{\pgfqpoint{1.350997in}{5.018263in}}%
\pgfpathlineto{\pgfqpoint{1.412732in}{5.320126in}}%
\pgfpathlineto{\pgfqpoint{1.443599in}{5.456009in}}%
\pgfpathlineto{\pgfqpoint{1.459033in}{5.516059in}}%
\pgfpathlineto{\pgfqpoint{1.474466in}{5.569784in}}%
\pgfpathlineto{\pgfqpoint{1.489900in}{5.616656in}}%
\pgfpathlineto{\pgfqpoint{1.505334in}{5.656366in}}%
\pgfpathlineto{\pgfqpoint{1.520767in}{5.688821in}}%
\pgfpathlineto{\pgfqpoint{1.536201in}{5.714126in}}%
\pgfpathlineto{\pgfqpoint{1.551634in}{5.732571in}}%
\pgfpathlineto{\pgfqpoint{1.567068in}{5.744606in}}%
\pgfpathlineto{\pgfqpoint{1.582502in}{5.750813in}}%
\pgfpathlineto{\pgfqpoint{1.597935in}{5.751879in}}%
\pgfpathlineto{\pgfqpoint{1.613369in}{5.748561in}}%
\pgfpathlineto{\pgfqpoint{1.628803in}{5.741654in}}%
\pgfpathlineto{\pgfqpoint{1.644236in}{5.731961in}}%
\pgfpathlineto{\pgfqpoint{1.675104in}{5.707284in}}%
\pgfpathlineto{\pgfqpoint{1.736838in}{5.654154in}}%
\pgfpathlineto{\pgfqpoint{1.767705in}{5.631849in}}%
\pgfpathlineto{\pgfqpoint{1.798573in}{5.613324in}}%
\pgfpathlineto{\pgfqpoint{1.875741in}{5.571672in}}%
\pgfpathlineto{\pgfqpoint{1.891174in}{5.561366in}}%
\pgfpathlineto{\pgfqpoint{1.906608in}{5.549596in}}%
\pgfpathlineto{\pgfqpoint{1.922042in}{5.536106in}}%
\pgfpathlineto{\pgfqpoint{1.937475in}{5.520717in}}%
\pgfpathlineto{\pgfqpoint{1.952909in}{5.503339in}}%
\pgfpathlineto{\pgfqpoint{1.968343in}{5.483985in}}%
\pgfpathlineto{\pgfqpoint{1.999210in}{5.439900in}}%
\pgfpathlineto{\pgfqpoint{2.030077in}{5.390562in}}%
\pgfpathlineto{\pgfqpoint{2.076378in}{5.314510in}}%
\pgfpathlineto{\pgfqpoint{2.107245in}{5.268816in}}%
\pgfpathlineto{\pgfqpoint{2.122679in}{5.249109in}}%
\pgfpathlineto{\pgfqpoint{2.138113in}{5.232124in}}%
\pgfpathlineto{\pgfqpoint{2.153546in}{5.218230in}}%
\pgfpathlineto{\pgfqpoint{2.168980in}{5.207691in}}%
\pgfpathlineto{\pgfqpoint{2.184414in}{5.200653in}}%
\pgfpathlineto{\pgfqpoint{2.199847in}{5.197128in}}%
\pgfpathlineto{\pgfqpoint{2.215281in}{5.196988in}}%
\pgfpathlineto{\pgfqpoint{2.230714in}{5.199963in}}%
\pgfpathlineto{\pgfqpoint{2.246148in}{5.205648in}}%
\pgfpathlineto{\pgfqpoint{2.261582in}{5.213506in}}%
\pgfpathlineto{\pgfqpoint{2.292449in}{5.233049in}}%
\pgfpathlineto{\pgfqpoint{2.323316in}{5.252410in}}%
\pgfpathlineto{\pgfqpoint{2.338750in}{5.259881in}}%
\pgfpathlineto{\pgfqpoint{2.354184in}{5.264738in}}%
\pgfpathlineto{\pgfqpoint{2.369617in}{5.266185in}}%
\pgfpathlineto{\pgfqpoint{2.385051in}{5.263509in}}%
\pgfpathlineto{\pgfqpoint{2.400484in}{5.256114in}}%
\pgfpathlineto{\pgfqpoint{2.415918in}{5.243549in}}%
\pgfpathlineto{\pgfqpoint{2.431352in}{5.225532in}}%
\pgfpathlineto{\pgfqpoint{2.446785in}{5.201968in}}%
\pgfpathlineto{\pgfqpoint{2.462219in}{5.172966in}}%
\pgfpathlineto{\pgfqpoint{2.477653in}{5.138843in}}%
\pgfpathlineto{\pgfqpoint{2.493086in}{5.100128in}}%
\pgfpathlineto{\pgfqpoint{2.523954in}{5.012038in}}%
\pgfpathlineto{\pgfqpoint{2.585688in}{4.824425in}}%
\pgfpathlineto{\pgfqpoint{2.601122in}{4.782977in}}%
\pgfpathlineto{\pgfqpoint{2.616555in}{4.746588in}}%
\pgfpathlineto{\pgfqpoint{2.631989in}{4.716644in}}%
\pgfpathlineto{\pgfqpoint{2.647423in}{4.694431in}}%
\pgfpathlineto{\pgfqpoint{2.662856in}{4.681088in}}%
\pgfpathlineto{\pgfqpoint{2.678290in}{4.677570in}}%
\pgfpathlineto{\pgfqpoint{2.693724in}{4.684611in}}%
\pgfpathlineto{\pgfqpoint{2.709157in}{4.702690in}}%
\pgfpathlineto{\pgfqpoint{2.724591in}{4.732011in}}%
\pgfpathlineto{\pgfqpoint{2.740024in}{4.772486in}}%
\pgfpathlineto{\pgfqpoint{2.755458in}{4.823723in}}%
\pgfpathlineto{\pgfqpoint{2.770892in}{4.885031in}}%
\pgfpathlineto{\pgfqpoint{2.786325in}{4.955426in}}%
\pgfpathlineto{\pgfqpoint{2.801759in}{5.033650in}}%
\pgfpathlineto{\pgfqpoint{2.832626in}{5.207355in}}%
\pgfpathlineto{\pgfqpoint{2.894361in}{5.570738in}}%
\pgfpathlineto{\pgfqpoint{2.909794in}{5.652918in}}%
\pgfpathlineto{\pgfqpoint{2.925228in}{5.727588in}}%
\pgfpathlineto{\pgfqpoint{2.940662in}{5.792934in}}%
\pgfpathlineto{\pgfqpoint{2.956095in}{5.847343in}}%
\pgfpathlineto{\pgfqpoint{2.971529in}{5.889450in}}%
\pgfpathlineto{\pgfqpoint{2.986963in}{5.918180in}}%
\pgfpathlineto{\pgfqpoint{3.002396in}{5.932782in}}%
\pgfpathlineto{\pgfqpoint{3.017830in}{5.932846in}}%
\pgfpathlineto{\pgfqpoint{3.033263in}{5.918326in}}%
\pgfpathlineto{\pgfqpoint{3.048697in}{5.889540in}}%
\pgfpathlineto{\pgfqpoint{3.064131in}{5.847163in}}%
\pgfpathlineto{\pgfqpoint{3.079564in}{5.792217in}}%
\pgfpathlineto{\pgfqpoint{3.094998in}{5.726039in}}%
\pgfpathlineto{\pgfqpoint{3.110432in}{5.650249in}}%
\pgfpathlineto{\pgfqpoint{3.141299in}{5.477474in}}%
\pgfpathlineto{\pgfqpoint{3.203033in}{5.108079in}}%
\pgfpathlineto{\pgfqpoint{3.218467in}{5.023767in}}%
\pgfpathlineto{\pgfqpoint{3.233901in}{4.946794in}}%
\pgfpathlineto{\pgfqpoint{3.249334in}{4.878899in}}%
\pgfpathlineto{\pgfqpoint{3.264768in}{4.821550in}}%
\pgfpathlineto{\pgfqpoint{3.280202in}{4.775912in}}%
\pgfpathlineto{\pgfqpoint{3.295635in}{4.742813in}}%
\pgfpathlineto{\pgfqpoint{3.311069in}{4.722726in}}%
\pgfpathlineto{\pgfqpoint{3.326503in}{4.715754in}}%
\pgfpathlineto{\pgfqpoint{3.341936in}{4.721636in}}%
\pgfpathlineto{\pgfqpoint{3.357370in}{4.739753in}}%
\pgfpathlineto{\pgfqpoint{3.372803in}{4.769150in}}%
\pgfpathlineto{\pgfqpoint{3.388237in}{4.808569in}}%
\pgfpathlineto{\pgfqpoint{3.403671in}{4.856483in}}%
\pgfpathlineto{\pgfqpoint{3.419104in}{4.911148in}}%
\pgfpathlineto{\pgfqpoint{3.449972in}{5.032977in}}%
\pgfpathlineto{\pgfqpoint{3.480839in}{5.157821in}}%
\pgfpathlineto{\pgfqpoint{3.496273in}{5.216298in}}%
\pgfpathlineto{\pgfqpoint{3.511706in}{5.269631in}}%
\pgfpathlineto{\pgfqpoint{3.527140in}{5.316149in}}%
\pgfpathlineto{\pgfqpoint{3.542573in}{5.354418in}}%
\pgfpathlineto{\pgfqpoint{3.558007in}{5.383272in}}%
\pgfpathlineto{\pgfqpoint{3.573441in}{5.401854in}}%
\pgfpathlineto{\pgfqpoint{3.588874in}{5.409635in}}%
\pgfpathlineto{\pgfqpoint{3.604308in}{5.406428in}}%
\pgfpathlineto{\pgfqpoint{3.619742in}{5.392391in}}%
\pgfpathlineto{\pgfqpoint{3.635175in}{5.368021in}}%
\pgfpathlineto{\pgfqpoint{3.650609in}{5.334134in}}%
\pgfpathlineto{\pgfqpoint{3.666043in}{5.291842in}}%
\pgfpathlineto{\pgfqpoint{3.681476in}{5.242518in}}%
\pgfpathlineto{\pgfqpoint{3.712343in}{5.129289in}}%
\pgfpathlineto{\pgfqpoint{3.758644in}{4.950749in}}%
\pgfpathlineto{\pgfqpoint{3.774078in}{4.896571in}}%
\pgfpathlineto{\pgfqpoint{3.789512in}{4.848092in}}%
\pgfpathlineto{\pgfqpoint{3.804945in}{4.806923in}}%
\pgfpathlineto{\pgfqpoint{3.820379in}{4.774468in}}%
\pgfpathlineto{\pgfqpoint{3.835813in}{4.751884in}}%
\pgfpathlineto{\pgfqpoint{3.851246in}{4.740056in}}%
\pgfpathlineto{\pgfqpoint{3.866680in}{4.739565in}}%
\pgfpathlineto{\pgfqpoint{3.882113in}{4.750681in}}%
\pgfpathlineto{\pgfqpoint{3.897547in}{4.773357in}}%
\pgfpathlineto{\pgfqpoint{3.912981in}{4.807229in}}%
\pgfpathlineto{\pgfqpoint{3.928414in}{4.851640in}}%
\pgfpathlineto{\pgfqpoint{3.943848in}{4.905653in}}%
\pgfpathlineto{\pgfqpoint{3.959282in}{4.968086in}}%
\pgfpathlineto{\pgfqpoint{3.974715in}{5.037550in}}%
\pgfpathlineto{\pgfqpoint{4.005583in}{5.191231in}}%
\pgfpathlineto{\pgfqpoint{4.067317in}{5.509371in}}%
\pgfpathlineto{\pgfqpoint{4.082751in}{5.581342in}}%
\pgfpathlineto{\pgfqpoint{4.098184in}{5.647342in}}%
\pgfpathlineto{\pgfqpoint{4.113618in}{5.706208in}}%
\pgfpathlineto{\pgfqpoint{4.129052in}{5.757005in}}%
\pgfpathlineto{\pgfqpoint{4.144485in}{5.799039in}}%
\pgfpathlineto{\pgfqpoint{4.159919in}{5.831876in}}%
\pgfpathlineto{\pgfqpoint{4.175352in}{5.855343in}}%
\pgfpathlineto{\pgfqpoint{4.190786in}{5.869522in}}%
\pgfpathlineto{\pgfqpoint{4.206220in}{5.874737in}}%
\pgfpathlineto{\pgfqpoint{4.221653in}{5.871539in}}%
\pgfpathlineto{\pgfqpoint{4.237087in}{5.860674in}}%
\pgfpathlineto{\pgfqpoint{4.252521in}{5.843054in}}%
\pgfpathlineto{\pgfqpoint{4.267954in}{5.819722in}}%
\pgfpathlineto{\pgfqpoint{4.283388in}{5.791811in}}%
\pgfpathlineto{\pgfqpoint{4.314255in}{5.726998in}}%
\pgfpathlineto{\pgfqpoint{4.360556in}{5.624506in}}%
\pgfpathlineto{\pgfqpoint{4.375990in}{5.592960in}}%
\pgfpathlineto{\pgfqpoint{4.391423in}{5.564019in}}%
\pgfpathlineto{\pgfqpoint{4.406857in}{5.538214in}}%
\pgfpathlineto{\pgfqpoint{4.422291in}{5.515888in}}%
\pgfpathlineto{\pgfqpoint{4.437724in}{5.497197in}}%
\pgfpathlineto{\pgfqpoint{4.453158in}{5.482111in}}%
\pgfpathlineto{\pgfqpoint{4.468592in}{5.470424in}}%
\pgfpathlineto{\pgfqpoint{4.484025in}{5.461768in}}%
\pgfpathlineto{\pgfqpoint{4.499459in}{5.455637in}}%
\pgfpathlineto{\pgfqpoint{4.514892in}{5.451413in}}%
\pgfpathlineto{\pgfqpoint{4.576627in}{5.439001in}}%
\pgfpathlineto{\pgfqpoint{4.592061in}{5.433289in}}%
\pgfpathlineto{\pgfqpoint{4.607494in}{5.425195in}}%
\pgfpathlineto{\pgfqpoint{4.622928in}{5.414208in}}%
\pgfpathlineto{\pgfqpoint{4.638362in}{5.399950in}}%
\pgfpathlineto{\pgfqpoint{4.653795in}{5.382196in}}%
\pgfpathlineto{\pgfqpoint{4.669229in}{5.360879in}}%
\pgfpathlineto{\pgfqpoint{4.669229in}{5.360879in}}%
\pgfusepath{stroke}%
\end{pgfscope}%
\begin{pgfscope}%
\pgfpathrectangle{\pgfqpoint{0.634105in}{3.881603in}}{\pgfqpoint{4.227273in}{2.800000in}} %
\pgfusepath{clip}%
\pgfsetrectcap%
\pgfsetroundjoin%
\pgfsetlinewidth{0.501875pt}%
\definecolor{currentstroke}{rgb}{0.021569,0.682749,0.930229}%
\pgfsetstrokecolor{currentstroke}%
\pgfsetdash{}{0pt}%
\pgfpathmoveto{\pgfqpoint{0.826254in}{5.586433in}}%
\pgfpathlineto{\pgfqpoint{0.841687in}{5.613463in}}%
\pgfpathlineto{\pgfqpoint{0.857121in}{5.634034in}}%
\pgfpathlineto{\pgfqpoint{0.872555in}{5.647371in}}%
\pgfpathlineto{\pgfqpoint{0.887988in}{5.652898in}}%
\pgfpathlineto{\pgfqpoint{0.903422in}{5.650250in}}%
\pgfpathlineto{\pgfqpoint{0.918855in}{5.639288in}}%
\pgfpathlineto{\pgfqpoint{0.934289in}{5.620099in}}%
\pgfpathlineto{\pgfqpoint{0.949723in}{5.592991in}}%
\pgfpathlineto{\pgfqpoint{0.965156in}{5.558489in}}%
\pgfpathlineto{\pgfqpoint{0.980590in}{5.517314in}}%
\pgfpathlineto{\pgfqpoint{0.996024in}{5.470368in}}%
\pgfpathlineto{\pgfqpoint{1.026891in}{5.363491in}}%
\pgfpathlineto{\pgfqpoint{1.104059in}{5.079736in}}%
\pgfpathlineto{\pgfqpoint{1.119493in}{5.030523in}}%
\pgfpathlineto{\pgfqpoint{1.134926in}{4.986558in}}%
\pgfpathlineto{\pgfqpoint{1.150360in}{4.948778in}}%
\pgfpathlineto{\pgfqpoint{1.165794in}{4.917959in}}%
\pgfpathlineto{\pgfqpoint{1.181227in}{4.894700in}}%
\pgfpathlineto{\pgfqpoint{1.196661in}{4.879408in}}%
\pgfpathlineto{\pgfqpoint{1.212095in}{4.872297in}}%
\pgfpathlineto{\pgfqpoint{1.227528in}{4.873382in}}%
\pgfpathlineto{\pgfqpoint{1.242962in}{4.882489in}}%
\pgfpathlineto{\pgfqpoint{1.258395in}{4.899262in}}%
\pgfpathlineto{\pgfqpoint{1.273829in}{4.923178in}}%
\pgfpathlineto{\pgfqpoint{1.289263in}{4.953568in}}%
\pgfpathlineto{\pgfqpoint{1.304696in}{4.989638in}}%
\pgfpathlineto{\pgfqpoint{1.320130in}{5.030494in}}%
\pgfpathlineto{\pgfqpoint{1.350997in}{5.122665in}}%
\pgfpathlineto{\pgfqpoint{1.428165in}{5.367485in}}%
\pgfpathlineto{\pgfqpoint{1.459033in}{5.452613in}}%
\pgfpathlineto{\pgfqpoint{1.474466in}{5.489809in}}%
\pgfpathlineto{\pgfqpoint{1.489900in}{5.522913in}}%
\pgfpathlineto{\pgfqpoint{1.505334in}{5.551704in}}%
\pgfpathlineto{\pgfqpoint{1.520767in}{5.576088in}}%
\pgfpathlineto{\pgfqpoint{1.536201in}{5.596092in}}%
\pgfpathlineto{\pgfqpoint{1.551634in}{5.611854in}}%
\pgfpathlineto{\pgfqpoint{1.567068in}{5.623608in}}%
\pgfpathlineto{\pgfqpoint{1.582502in}{5.631667in}}%
\pgfpathlineto{\pgfqpoint{1.597935in}{5.636408in}}%
\pgfpathlineto{\pgfqpoint{1.613369in}{5.638253in}}%
\pgfpathlineto{\pgfqpoint{1.628803in}{5.637645in}}%
\pgfpathlineto{\pgfqpoint{1.644236in}{5.635035in}}%
\pgfpathlineto{\pgfqpoint{1.659670in}{5.630861in}}%
\pgfpathlineto{\pgfqpoint{1.690537in}{5.619415in}}%
\pgfpathlineto{\pgfqpoint{1.798573in}{5.572185in}}%
\pgfpathlineto{\pgfqpoint{1.829440in}{5.558076in}}%
\pgfpathlineto{\pgfqpoint{1.860307in}{5.541697in}}%
\pgfpathlineto{\pgfqpoint{1.891174in}{5.521295in}}%
\pgfpathlineto{\pgfqpoint{1.906608in}{5.509090in}}%
\pgfpathlineto{\pgfqpoint{1.922042in}{5.495348in}}%
\pgfpathlineto{\pgfqpoint{1.937475in}{5.479977in}}%
\pgfpathlineto{\pgfqpoint{1.968343in}{5.444274in}}%
\pgfpathlineto{\pgfqpoint{1.999210in}{5.402456in}}%
\pgfpathlineto{\pgfqpoint{2.030077in}{5.355994in}}%
\pgfpathlineto{\pgfqpoint{2.107245in}{5.235930in}}%
\pgfpathlineto{\pgfqpoint{2.138113in}{5.193903in}}%
\pgfpathlineto{\pgfqpoint{2.153546in}{5.175452in}}%
\pgfpathlineto{\pgfqpoint{2.168980in}{5.158995in}}%
\pgfpathlineto{\pgfqpoint{2.184414in}{5.144657in}}%
\pgfpathlineto{\pgfqpoint{2.199847in}{5.132484in}}%
\pgfpathlineto{\pgfqpoint{2.215281in}{5.122436in}}%
\pgfpathlineto{\pgfqpoint{2.230714in}{5.114391in}}%
\pgfpathlineto{\pgfqpoint{2.246148in}{5.108141in}}%
\pgfpathlineto{\pgfqpoint{2.277015in}{5.099811in}}%
\pgfpathlineto{\pgfqpoint{2.338750in}{5.088166in}}%
\pgfpathlineto{\pgfqpoint{2.354184in}{5.083519in}}%
\pgfpathlineto{\pgfqpoint{2.369617in}{5.077237in}}%
\pgfpathlineto{\pgfqpoint{2.385051in}{5.068911in}}%
\pgfpathlineto{\pgfqpoint{2.400484in}{5.058204in}}%
\pgfpathlineto{\pgfqpoint{2.415918in}{5.044861in}}%
\pgfpathlineto{\pgfqpoint{2.431352in}{5.028725in}}%
\pgfpathlineto{\pgfqpoint{2.446785in}{5.009752in}}%
\pgfpathlineto{\pgfqpoint{2.462219in}{4.988013in}}%
\pgfpathlineto{\pgfqpoint{2.477653in}{4.963705in}}%
\pgfpathlineto{\pgfqpoint{2.508520in}{4.908780in}}%
\pgfpathlineto{\pgfqpoint{2.585688in}{4.762180in}}%
\pgfpathlineto{\pgfqpoint{2.601122in}{4.737367in}}%
\pgfpathlineto{\pgfqpoint{2.616555in}{4.716019in}}%
\pgfpathlineto{\pgfqpoint{2.631989in}{4.698947in}}%
\pgfpathlineto{\pgfqpoint{2.647423in}{4.686899in}}%
\pgfpathlineto{\pgfqpoint{2.662856in}{4.680535in}}%
\pgfpathlineto{\pgfqpoint{2.678290in}{4.680401in}}%
\pgfpathlineto{\pgfqpoint{2.693724in}{4.686909in}}%
\pgfpathlineto{\pgfqpoint{2.709157in}{4.700317in}}%
\pgfpathlineto{\pgfqpoint{2.724591in}{4.720717in}}%
\pgfpathlineto{\pgfqpoint{2.740024in}{4.748024in}}%
\pgfpathlineto{\pgfqpoint{2.755458in}{4.781969in}}%
\pgfpathlineto{\pgfqpoint{2.770892in}{4.822105in}}%
\pgfpathlineto{\pgfqpoint{2.786325in}{4.867808in}}%
\pgfpathlineto{\pgfqpoint{2.801759in}{4.918287in}}%
\pgfpathlineto{\pgfqpoint{2.832626in}{5.029704in}}%
\pgfpathlineto{\pgfqpoint{2.894361in}{5.261587in}}%
\pgfpathlineto{\pgfqpoint{2.909794in}{5.314059in}}%
\pgfpathlineto{\pgfqpoint{2.925228in}{5.361846in}}%
\pgfpathlineto{\pgfqpoint{2.940662in}{5.403830in}}%
\pgfpathlineto{\pgfqpoint{2.956095in}{5.439019in}}%
\pgfpathlineto{\pgfqpoint{2.971529in}{5.466570in}}%
\pgfpathlineto{\pgfqpoint{2.986963in}{5.485815in}}%
\pgfpathlineto{\pgfqpoint{3.002396in}{5.496283in}}%
\pgfpathlineto{\pgfqpoint{3.017830in}{5.497711in}}%
\pgfpathlineto{\pgfqpoint{3.033263in}{5.490055in}}%
\pgfpathlineto{\pgfqpoint{3.048697in}{5.473494in}}%
\pgfpathlineto{\pgfqpoint{3.064131in}{5.448421in}}%
\pgfpathlineto{\pgfqpoint{3.079564in}{5.415442in}}%
\pgfpathlineto{\pgfqpoint{3.094998in}{5.375353in}}%
\pgfpathlineto{\pgfqpoint{3.110432in}{5.329124in}}%
\pgfpathlineto{\pgfqpoint{3.141299in}{5.222832in}}%
\pgfpathlineto{\pgfqpoint{3.218467in}{4.937831in}}%
\pgfpathlineto{\pgfqpoint{3.233901in}{4.888126in}}%
\pgfpathlineto{\pgfqpoint{3.249334in}{4.843586in}}%
\pgfpathlineto{\pgfqpoint{3.264768in}{4.805115in}}%
\pgfpathlineto{\pgfqpoint{3.280202in}{4.773435in}}%
\pgfpathlineto{\pgfqpoint{3.295635in}{4.749071in}}%
\pgfpathlineto{\pgfqpoint{3.311069in}{4.732332in}}%
\pgfpathlineto{\pgfqpoint{3.326503in}{4.723310in}}%
\pgfpathlineto{\pgfqpoint{3.341936in}{4.721879in}}%
\pgfpathlineto{\pgfqpoint{3.357370in}{4.727699in}}%
\pgfpathlineto{\pgfqpoint{3.372803in}{4.740231in}}%
\pgfpathlineto{\pgfqpoint{3.388237in}{4.758750in}}%
\pgfpathlineto{\pgfqpoint{3.403671in}{4.782378in}}%
\pgfpathlineto{\pgfqpoint{3.419104in}{4.810101in}}%
\pgfpathlineto{\pgfqpoint{3.449972in}{4.873318in}}%
\pgfpathlineto{\pgfqpoint{3.480839in}{4.938936in}}%
\pgfpathlineto{\pgfqpoint{3.496273in}{4.969684in}}%
\pgfpathlineto{\pgfqpoint{3.511706in}{4.997592in}}%
\pgfpathlineto{\pgfqpoint{3.527140in}{5.021691in}}%
\pgfpathlineto{\pgfqpoint{3.542573in}{5.041148in}}%
\pgfpathlineto{\pgfqpoint{3.558007in}{5.055294in}}%
\pgfpathlineto{\pgfqpoint{3.573441in}{5.063641in}}%
\pgfpathlineto{\pgfqpoint{3.588874in}{5.065896in}}%
\pgfpathlineto{\pgfqpoint{3.604308in}{5.061971in}}%
\pgfpathlineto{\pgfqpoint{3.619742in}{5.051981in}}%
\pgfpathlineto{\pgfqpoint{3.635175in}{5.036245in}}%
\pgfpathlineto{\pgfqpoint{3.650609in}{5.015267in}}%
\pgfpathlineto{\pgfqpoint{3.666043in}{4.989730in}}%
\pgfpathlineto{\pgfqpoint{3.681476in}{4.960469in}}%
\pgfpathlineto{\pgfqpoint{3.712343in}{4.894717in}}%
\pgfpathlineto{\pgfqpoint{3.743211in}{4.826695in}}%
\pgfpathlineto{\pgfqpoint{3.758644in}{4.794720in}}%
\pgfpathlineto{\pgfqpoint{3.774078in}{4.765610in}}%
\pgfpathlineto{\pgfqpoint{3.789512in}{4.740421in}}%
\pgfpathlineto{\pgfqpoint{3.804945in}{4.720108in}}%
\pgfpathlineto{\pgfqpoint{3.820379in}{4.705501in}}%
\pgfpathlineto{\pgfqpoint{3.835813in}{4.697279in}}%
\pgfpathlineto{\pgfqpoint{3.851246in}{4.695954in}}%
\pgfpathlineto{\pgfqpoint{3.866680in}{4.701855in}}%
\pgfpathlineto{\pgfqpoint{3.882113in}{4.715121in}}%
\pgfpathlineto{\pgfqpoint{3.897547in}{4.735697in}}%
\pgfpathlineto{\pgfqpoint{3.912981in}{4.763340in}}%
\pgfpathlineto{\pgfqpoint{3.928414in}{4.797625in}}%
\pgfpathlineto{\pgfqpoint{3.943848in}{4.837959in}}%
\pgfpathlineto{\pgfqpoint{3.959282in}{4.883601in}}%
\pgfpathlineto{\pgfqpoint{3.990149in}{4.987235in}}%
\pgfpathlineto{\pgfqpoint{4.082751in}{5.321902in}}%
\pgfpathlineto{\pgfqpoint{4.098184in}{5.370341in}}%
\pgfpathlineto{\pgfqpoint{4.113618in}{5.414382in}}%
\pgfpathlineto{\pgfqpoint{4.129052in}{5.453426in}}%
\pgfpathlineto{\pgfqpoint{4.144485in}{5.487020in}}%
\pgfpathlineto{\pgfqpoint{4.159919in}{5.514870in}}%
\pgfpathlineto{\pgfqpoint{4.175352in}{5.536835in}}%
\pgfpathlineto{\pgfqpoint{4.190786in}{5.552932in}}%
\pgfpathlineto{\pgfqpoint{4.206220in}{5.563324in}}%
\pgfpathlineto{\pgfqpoint{4.221653in}{5.568313in}}%
\pgfpathlineto{\pgfqpoint{4.237087in}{5.568316in}}%
\pgfpathlineto{\pgfqpoint{4.252521in}{5.563858in}}%
\pgfpathlineto{\pgfqpoint{4.267954in}{5.555539in}}%
\pgfpathlineto{\pgfqpoint{4.283388in}{5.544019in}}%
\pgfpathlineto{\pgfqpoint{4.298822in}{5.529988in}}%
\pgfpathlineto{\pgfqpoint{4.329689in}{5.497179in}}%
\pgfpathlineto{\pgfqpoint{4.375990in}{5.445655in}}%
\pgfpathlineto{\pgfqpoint{4.406857in}{5.415616in}}%
\pgfpathlineto{\pgfqpoint{4.422291in}{5.402834in}}%
\pgfpathlineto{\pgfqpoint{4.437724in}{5.391714in}}%
\pgfpathlineto{\pgfqpoint{4.453158in}{5.382245in}}%
\pgfpathlineto{\pgfqpoint{4.468592in}{5.374317in}}%
\pgfpathlineto{\pgfqpoint{4.499459in}{5.362173in}}%
\pgfpathlineto{\pgfqpoint{4.561193in}{5.342811in}}%
\pgfpathlineto{\pgfqpoint{4.576627in}{5.336563in}}%
\pgfpathlineto{\pgfqpoint{4.592061in}{5.328934in}}%
\pgfpathlineto{\pgfqpoint{4.607494in}{5.319585in}}%
\pgfpathlineto{\pgfqpoint{4.622928in}{5.308243in}}%
\pgfpathlineto{\pgfqpoint{4.638362in}{5.294715in}}%
\pgfpathlineto{\pgfqpoint{4.653795in}{5.278899in}}%
\pgfpathlineto{\pgfqpoint{4.669229in}{5.260791in}}%
\pgfpathlineto{\pgfqpoint{4.669229in}{5.260791in}}%
\pgfusepath{stroke}%
\end{pgfscope}%
\begin{pgfscope}%
\pgfpathrectangle{\pgfqpoint{0.634105in}{3.881603in}}{\pgfqpoint{4.227273in}{2.800000in}} %
\pgfusepath{clip}%
\pgfsetrectcap%
\pgfsetroundjoin%
\pgfsetlinewidth{0.501875pt}%
\definecolor{currentstroke}{rgb}{0.056863,0.767363,0.905873}%
\pgfsetstrokecolor{currentstroke}%
\pgfsetdash{}{0pt}%
\pgfpathmoveto{\pgfqpoint{0.826254in}{5.658121in}}%
\pgfpathlineto{\pgfqpoint{0.841687in}{5.669583in}}%
\pgfpathlineto{\pgfqpoint{0.857121in}{5.675081in}}%
\pgfpathlineto{\pgfqpoint{0.872555in}{5.674162in}}%
\pgfpathlineto{\pgfqpoint{0.887988in}{5.666528in}}%
\pgfpathlineto{\pgfqpoint{0.903422in}{5.652051in}}%
\pgfpathlineto{\pgfqpoint{0.918855in}{5.630778in}}%
\pgfpathlineto{\pgfqpoint{0.934289in}{5.602925in}}%
\pgfpathlineto{\pgfqpoint{0.949723in}{5.568882in}}%
\pgfpathlineto{\pgfqpoint{0.965156in}{5.529194in}}%
\pgfpathlineto{\pgfqpoint{0.980590in}{5.484558in}}%
\pgfpathlineto{\pgfqpoint{1.011457in}{5.383846in}}%
\pgfpathlineto{\pgfqpoint{1.088625in}{5.113381in}}%
\pgfpathlineto{\pgfqpoint{1.104059in}{5.064803in}}%
\pgfpathlineto{\pgfqpoint{1.119493in}{5.020461in}}%
\pgfpathlineto{\pgfqpoint{1.134926in}{4.981227in}}%
\pgfpathlineto{\pgfqpoint{1.150360in}{4.947852in}}%
\pgfpathlineto{\pgfqpoint{1.165794in}{4.920957in}}%
\pgfpathlineto{\pgfqpoint{1.181227in}{4.901010in}}%
\pgfpathlineto{\pgfqpoint{1.196661in}{4.888328in}}%
\pgfpathlineto{\pgfqpoint{1.212095in}{4.883064in}}%
\pgfpathlineto{\pgfqpoint{1.227528in}{4.885211in}}%
\pgfpathlineto{\pgfqpoint{1.242962in}{4.894604in}}%
\pgfpathlineto{\pgfqpoint{1.258395in}{4.910928in}}%
\pgfpathlineto{\pgfqpoint{1.273829in}{4.933731in}}%
\pgfpathlineto{\pgfqpoint{1.289263in}{4.962437in}}%
\pgfpathlineto{\pgfqpoint{1.304696in}{4.996364in}}%
\pgfpathlineto{\pgfqpoint{1.320130in}{5.034744in}}%
\pgfpathlineto{\pgfqpoint{1.350997in}{5.121488in}}%
\pgfpathlineto{\pgfqpoint{1.443599in}{5.398654in}}%
\pgfpathlineto{\pgfqpoint{1.459033in}{5.438990in}}%
\pgfpathlineto{\pgfqpoint{1.474466in}{5.475997in}}%
\pgfpathlineto{\pgfqpoint{1.489900in}{5.509267in}}%
\pgfpathlineto{\pgfqpoint{1.505334in}{5.538497in}}%
\pgfpathlineto{\pgfqpoint{1.520767in}{5.563486in}}%
\pgfpathlineto{\pgfqpoint{1.536201in}{5.584139in}}%
\pgfpathlineto{\pgfqpoint{1.551634in}{5.600452in}}%
\pgfpathlineto{\pgfqpoint{1.567068in}{5.612511in}}%
\pgfpathlineto{\pgfqpoint{1.582502in}{5.620478in}}%
\pgfpathlineto{\pgfqpoint{1.597935in}{5.624578in}}%
\pgfpathlineto{\pgfqpoint{1.613369in}{5.625089in}}%
\pgfpathlineto{\pgfqpoint{1.628803in}{5.622328in}}%
\pgfpathlineto{\pgfqpoint{1.644236in}{5.616635in}}%
\pgfpathlineto{\pgfqpoint{1.659670in}{5.608364in}}%
\pgfpathlineto{\pgfqpoint{1.675104in}{5.597866in}}%
\pgfpathlineto{\pgfqpoint{1.690537in}{5.585482in}}%
\pgfpathlineto{\pgfqpoint{1.721404in}{5.556302in}}%
\pgfpathlineto{\pgfqpoint{1.752272in}{5.523001in}}%
\pgfpathlineto{\pgfqpoint{1.798573in}{5.468635in}}%
\pgfpathlineto{\pgfqpoint{1.860307in}{5.391819in}}%
\pgfpathlineto{\pgfqpoint{1.937475in}{5.291450in}}%
\pgfpathlineto{\pgfqpoint{1.999210in}{5.211679in}}%
\pgfpathlineto{\pgfqpoint{2.030077in}{5.174920in}}%
\pgfpathlineto{\pgfqpoint{2.060944in}{5.142429in}}%
\pgfpathlineto{\pgfqpoint{2.076378in}{5.128341in}}%
\pgfpathlineto{\pgfqpoint{2.091812in}{5.115969in}}%
\pgfpathlineto{\pgfqpoint{2.107245in}{5.105495in}}%
\pgfpathlineto{\pgfqpoint{2.122679in}{5.097068in}}%
\pgfpathlineto{\pgfqpoint{2.138113in}{5.090791in}}%
\pgfpathlineto{\pgfqpoint{2.153546in}{5.086720in}}%
\pgfpathlineto{\pgfqpoint{2.168980in}{5.084850in}}%
\pgfpathlineto{\pgfqpoint{2.184414in}{5.085116in}}%
\pgfpathlineto{\pgfqpoint{2.199847in}{5.087386in}}%
\pgfpathlineto{\pgfqpoint{2.215281in}{5.091461in}}%
\pgfpathlineto{\pgfqpoint{2.230714in}{5.097078in}}%
\pgfpathlineto{\pgfqpoint{2.261582in}{5.111583in}}%
\pgfpathlineto{\pgfqpoint{2.307883in}{5.135172in}}%
\pgfpathlineto{\pgfqpoint{2.323316in}{5.141620in}}%
\pgfpathlineto{\pgfqpoint{2.338750in}{5.146540in}}%
\pgfpathlineto{\pgfqpoint{2.354184in}{5.149461in}}%
\pgfpathlineto{\pgfqpoint{2.369617in}{5.149950in}}%
\pgfpathlineto{\pgfqpoint{2.385051in}{5.147626in}}%
\pgfpathlineto{\pgfqpoint{2.400484in}{5.142174in}}%
\pgfpathlineto{\pgfqpoint{2.415918in}{5.133359in}}%
\pgfpathlineto{\pgfqpoint{2.431352in}{5.121037in}}%
\pgfpathlineto{\pgfqpoint{2.446785in}{5.105162in}}%
\pgfpathlineto{\pgfqpoint{2.462219in}{5.085796in}}%
\pgfpathlineto{\pgfqpoint{2.477653in}{5.063106in}}%
\pgfpathlineto{\pgfqpoint{2.493086in}{5.037373in}}%
\pgfpathlineto{\pgfqpoint{2.523954in}{4.978404in}}%
\pgfpathlineto{\pgfqpoint{2.570254in}{4.879676in}}%
\pgfpathlineto{\pgfqpoint{2.601122in}{4.815131in}}%
\pgfpathlineto{\pgfqpoint{2.616555in}{4.785533in}}%
\pgfpathlineto{\pgfqpoint{2.631989in}{4.758731in}}%
\pgfpathlineto{\pgfqpoint{2.647423in}{4.735440in}}%
\pgfpathlineto{\pgfqpoint{2.662856in}{4.716312in}}%
\pgfpathlineto{\pgfqpoint{2.678290in}{4.701917in}}%
\pgfpathlineto{\pgfqpoint{2.693724in}{4.692726in}}%
\pgfpathlineto{\pgfqpoint{2.709157in}{4.689096in}}%
\pgfpathlineto{\pgfqpoint{2.724591in}{4.691254in}}%
\pgfpathlineto{\pgfqpoint{2.740024in}{4.699293in}}%
\pgfpathlineto{\pgfqpoint{2.755458in}{4.713163in}}%
\pgfpathlineto{\pgfqpoint{2.770892in}{4.732669in}}%
\pgfpathlineto{\pgfqpoint{2.786325in}{4.757475in}}%
\pgfpathlineto{\pgfqpoint{2.801759in}{4.787113in}}%
\pgfpathlineto{\pgfqpoint{2.817193in}{4.820985in}}%
\pgfpathlineto{\pgfqpoint{2.848060in}{4.898521in}}%
\pgfpathlineto{\pgfqpoint{2.925228in}{5.108287in}}%
\pgfpathlineto{\pgfqpoint{2.940662in}{5.145515in}}%
\pgfpathlineto{\pgfqpoint{2.956095in}{5.179134in}}%
\pgfpathlineto{\pgfqpoint{2.971529in}{5.208421in}}%
\pgfpathlineto{\pgfqpoint{2.986963in}{5.232760in}}%
\pgfpathlineto{\pgfqpoint{3.002396in}{5.251667in}}%
\pgfpathlineto{\pgfqpoint{3.017830in}{5.264791in}}%
\pgfpathlineto{\pgfqpoint{3.033263in}{5.271933in}}%
\pgfpathlineto{\pgfqpoint{3.048697in}{5.273044in}}%
\pgfpathlineto{\pgfqpoint{3.064131in}{5.268227in}}%
\pgfpathlineto{\pgfqpoint{3.079564in}{5.257732in}}%
\pgfpathlineto{\pgfqpoint{3.094998in}{5.241948in}}%
\pgfpathlineto{\pgfqpoint{3.110432in}{5.221392in}}%
\pgfpathlineto{\pgfqpoint{3.125865in}{5.196689in}}%
\pgfpathlineto{\pgfqpoint{3.141299in}{5.168558in}}%
\pgfpathlineto{\pgfqpoint{3.172166in}{5.105214in}}%
\pgfpathlineto{\pgfqpoint{3.218467in}{5.005201in}}%
\pgfpathlineto{\pgfqpoint{3.249334in}{4.944684in}}%
\pgfpathlineto{\pgfqpoint{3.264768in}{4.918359in}}%
\pgfpathlineto{\pgfqpoint{3.280202in}{4.895363in}}%
\pgfpathlineto{\pgfqpoint{3.295635in}{4.876076in}}%
\pgfpathlineto{\pgfqpoint{3.311069in}{4.860748in}}%
\pgfpathlineto{\pgfqpoint{3.326503in}{4.849493in}}%
\pgfpathlineto{\pgfqpoint{3.341936in}{4.842289in}}%
\pgfpathlineto{\pgfqpoint{3.357370in}{4.838979in}}%
\pgfpathlineto{\pgfqpoint{3.372803in}{4.839282in}}%
\pgfpathlineto{\pgfqpoint{3.388237in}{4.842803in}}%
\pgfpathlineto{\pgfqpoint{3.403671in}{4.849046in}}%
\pgfpathlineto{\pgfqpoint{3.419104in}{4.857437in}}%
\pgfpathlineto{\pgfqpoint{3.449972in}{4.878075in}}%
\pgfpathlineto{\pgfqpoint{3.480839in}{4.899304in}}%
\pgfpathlineto{\pgfqpoint{3.496273in}{4.908465in}}%
\pgfpathlineto{\pgfqpoint{3.511706in}{4.915846in}}%
\pgfpathlineto{\pgfqpoint{3.527140in}{4.920926in}}%
\pgfpathlineto{\pgfqpoint{3.542573in}{4.923278in}}%
\pgfpathlineto{\pgfqpoint{3.558007in}{4.922581in}}%
\pgfpathlineto{\pgfqpoint{3.573441in}{4.918635in}}%
\pgfpathlineto{\pgfqpoint{3.588874in}{4.911368in}}%
\pgfpathlineto{\pgfqpoint{3.604308in}{4.900836in}}%
\pgfpathlineto{\pgfqpoint{3.619742in}{4.887230in}}%
\pgfpathlineto{\pgfqpoint{3.635175in}{4.870862in}}%
\pgfpathlineto{\pgfqpoint{3.650609in}{4.852164in}}%
\pgfpathlineto{\pgfqpoint{3.681476in}{4.810017in}}%
\pgfpathlineto{\pgfqpoint{3.727777in}{4.745232in}}%
\pgfpathlineto{\pgfqpoint{3.743211in}{4.726252in}}%
\pgfpathlineto{\pgfqpoint{3.758644in}{4.709852in}}%
\pgfpathlineto{\pgfqpoint{3.774078in}{4.696745in}}%
\pgfpathlineto{\pgfqpoint{3.789512in}{4.687577in}}%
\pgfpathlineto{\pgfqpoint{3.804945in}{4.682907in}}%
\pgfpathlineto{\pgfqpoint{3.820379in}{4.683192in}}%
\pgfpathlineto{\pgfqpoint{3.835813in}{4.688767in}}%
\pgfpathlineto{\pgfqpoint{3.851246in}{4.699838in}}%
\pgfpathlineto{\pgfqpoint{3.866680in}{4.716475in}}%
\pgfpathlineto{\pgfqpoint{3.882113in}{4.738603in}}%
\pgfpathlineto{\pgfqpoint{3.897547in}{4.766008in}}%
\pgfpathlineto{\pgfqpoint{3.912981in}{4.798340in}}%
\pgfpathlineto{\pgfqpoint{3.928414in}{4.835119in}}%
\pgfpathlineto{\pgfqpoint{3.943848in}{4.875751in}}%
\pgfpathlineto{\pgfqpoint{3.974715in}{4.965719in}}%
\pgfpathlineto{\pgfqpoint{4.051883in}{5.202195in}}%
\pgfpathlineto{\pgfqpoint{4.067317in}{5.244555in}}%
\pgfpathlineto{\pgfqpoint{4.082751in}{5.283432in}}%
\pgfpathlineto{\pgfqpoint{4.098184in}{5.318192in}}%
\pgfpathlineto{\pgfqpoint{4.113618in}{5.348305in}}%
\pgfpathlineto{\pgfqpoint{4.129052in}{5.373360in}}%
\pgfpathlineto{\pgfqpoint{4.144485in}{5.393071in}}%
\pgfpathlineto{\pgfqpoint{4.159919in}{5.407286in}}%
\pgfpathlineto{\pgfqpoint{4.175352in}{5.415982in}}%
\pgfpathlineto{\pgfqpoint{4.190786in}{5.419266in}}%
\pgfpathlineto{\pgfqpoint{4.206220in}{5.417368in}}%
\pgfpathlineto{\pgfqpoint{4.221653in}{5.410630in}}%
\pgfpathlineto{\pgfqpoint{4.237087in}{5.399495in}}%
\pgfpathlineto{\pgfqpoint{4.252521in}{5.384489in}}%
\pgfpathlineto{\pgfqpoint{4.267954in}{5.366206in}}%
\pgfpathlineto{\pgfqpoint{4.283388in}{5.345289in}}%
\pgfpathlineto{\pgfqpoint{4.314255in}{5.298241in}}%
\pgfpathlineto{\pgfqpoint{4.360556in}{5.224558in}}%
\pgfpathlineto{\pgfqpoint{4.391423in}{5.180173in}}%
\pgfpathlineto{\pgfqpoint{4.406857in}{5.160762in}}%
\pgfpathlineto{\pgfqpoint{4.422291in}{5.143614in}}%
\pgfpathlineto{\pgfqpoint{4.437724in}{5.128919in}}%
\pgfpathlineto{\pgfqpoint{4.453158in}{5.116773in}}%
\pgfpathlineto{\pgfqpoint{4.468592in}{5.107179in}}%
\pgfpathlineto{\pgfqpoint{4.484025in}{5.100056in}}%
\pgfpathlineto{\pgfqpoint{4.499459in}{5.095243in}}%
\pgfpathlineto{\pgfqpoint{4.514892in}{5.092514in}}%
\pgfpathlineto{\pgfqpoint{4.530326in}{5.091587in}}%
\pgfpathlineto{\pgfqpoint{4.545760in}{5.092138in}}%
\pgfpathlineto{\pgfqpoint{4.576627in}{5.096269in}}%
\pgfpathlineto{\pgfqpoint{4.638362in}{5.106954in}}%
\pgfpathlineto{\pgfqpoint{4.669229in}{5.109021in}}%
\pgfpathlineto{\pgfqpoint{4.669229in}{5.109021in}}%
\pgfusepath{stroke}%
\end{pgfscope}%
\begin{pgfscope}%
\pgfpathrectangle{\pgfqpoint{0.634105in}{3.881603in}}{\pgfqpoint{4.227273in}{2.800000in}} %
\pgfusepath{clip}%
\pgfsetrectcap%
\pgfsetroundjoin%
\pgfsetlinewidth{0.501875pt}%
\definecolor{currentstroke}{rgb}{0.135294,0.840344,0.878081}%
\pgfsetstrokecolor{currentstroke}%
\pgfsetdash{}{0pt}%
\pgfpathmoveto{\pgfqpoint{0.826254in}{5.724469in}}%
\pgfpathlineto{\pgfqpoint{0.841687in}{5.729952in}}%
\pgfpathlineto{\pgfqpoint{0.857121in}{5.730238in}}%
\pgfpathlineto{\pgfqpoint{0.872555in}{5.725089in}}%
\pgfpathlineto{\pgfqpoint{0.887988in}{5.714389in}}%
\pgfpathlineto{\pgfqpoint{0.903422in}{5.698150in}}%
\pgfpathlineto{\pgfqpoint{0.918855in}{5.676509in}}%
\pgfpathlineto{\pgfqpoint{0.934289in}{5.649730in}}%
\pgfpathlineto{\pgfqpoint{0.949723in}{5.618203in}}%
\pgfpathlineto{\pgfqpoint{0.965156in}{5.582427in}}%
\pgfpathlineto{\pgfqpoint{0.996024in}{5.500642in}}%
\pgfpathlineto{\pgfqpoint{1.042325in}{5.363835in}}%
\pgfpathlineto{\pgfqpoint{1.073192in}{5.273176in}}%
\pgfpathlineto{\pgfqpoint{1.104059in}{5.191053in}}%
\pgfpathlineto{\pgfqpoint{1.119493in}{5.155194in}}%
\pgfpathlineto{\pgfqpoint{1.134926in}{5.123738in}}%
\pgfpathlineto{\pgfqpoint{1.150360in}{5.097273in}}%
\pgfpathlineto{\pgfqpoint{1.165794in}{5.076286in}}%
\pgfpathlineto{\pgfqpoint{1.181227in}{5.061147in}}%
\pgfpathlineto{\pgfqpoint{1.196661in}{5.052108in}}%
\pgfpathlineto{\pgfqpoint{1.212095in}{5.049295in}}%
\pgfpathlineto{\pgfqpoint{1.227528in}{5.052709in}}%
\pgfpathlineto{\pgfqpoint{1.242962in}{5.062229in}}%
\pgfpathlineto{\pgfqpoint{1.258395in}{5.077613in}}%
\pgfpathlineto{\pgfqpoint{1.273829in}{5.098513in}}%
\pgfpathlineto{\pgfqpoint{1.289263in}{5.124478in}}%
\pgfpathlineto{\pgfqpoint{1.304696in}{5.154971in}}%
\pgfpathlineto{\pgfqpoint{1.320130in}{5.189380in}}%
\pgfpathlineto{\pgfqpoint{1.350997in}{5.267230in}}%
\pgfpathlineto{\pgfqpoint{1.397298in}{5.395633in}}%
\pgfpathlineto{\pgfqpoint{1.428165in}{5.480480in}}%
\pgfpathlineto{\pgfqpoint{1.459033in}{5.558551in}}%
\pgfpathlineto{\pgfqpoint{1.474466in}{5.593666in}}%
\pgfpathlineto{\pgfqpoint{1.489900in}{5.625572in}}%
\pgfpathlineto{\pgfqpoint{1.505334in}{5.653921in}}%
\pgfpathlineto{\pgfqpoint{1.520767in}{5.678445in}}%
\pgfpathlineto{\pgfqpoint{1.536201in}{5.698958in}}%
\pgfpathlineto{\pgfqpoint{1.551634in}{5.715352in}}%
\pgfpathlineto{\pgfqpoint{1.567068in}{5.727595in}}%
\pgfpathlineto{\pgfqpoint{1.582502in}{5.735722in}}%
\pgfpathlineto{\pgfqpoint{1.597935in}{5.739833in}}%
\pgfpathlineto{\pgfqpoint{1.613369in}{5.740082in}}%
\pgfpathlineto{\pgfqpoint{1.628803in}{5.736668in}}%
\pgfpathlineto{\pgfqpoint{1.644236in}{5.729829in}}%
\pgfpathlineto{\pgfqpoint{1.659670in}{5.719829in}}%
\pgfpathlineto{\pgfqpoint{1.675104in}{5.706955in}}%
\pgfpathlineto{\pgfqpoint{1.690537in}{5.691501in}}%
\pgfpathlineto{\pgfqpoint{1.705971in}{5.673768in}}%
\pgfpathlineto{\pgfqpoint{1.736838in}{5.632641in}}%
\pgfpathlineto{\pgfqpoint{1.767705in}{5.585823in}}%
\pgfpathlineto{\pgfqpoint{1.814006in}{5.509239in}}%
\pgfpathlineto{\pgfqpoint{1.906608in}{5.352303in}}%
\pgfpathlineto{\pgfqpoint{1.937475in}{5.303504in}}%
\pgfpathlineto{\pgfqpoint{1.968343in}{5.258423in}}%
\pgfpathlineto{\pgfqpoint{1.999210in}{5.218278in}}%
\pgfpathlineto{\pgfqpoint{2.030077in}{5.184289in}}%
\pgfpathlineto{\pgfqpoint{2.045511in}{5.169965in}}%
\pgfpathlineto{\pgfqpoint{2.060944in}{5.157593in}}%
\pgfpathlineto{\pgfqpoint{2.076378in}{5.147277in}}%
\pgfpathlineto{\pgfqpoint{2.091812in}{5.139098in}}%
\pgfpathlineto{\pgfqpoint{2.107245in}{5.133106in}}%
\pgfpathlineto{\pgfqpoint{2.122679in}{5.129317in}}%
\pgfpathlineto{\pgfqpoint{2.138113in}{5.127703in}}%
\pgfpathlineto{\pgfqpoint{2.153546in}{5.128196in}}%
\pgfpathlineto{\pgfqpoint{2.168980in}{5.130677in}}%
\pgfpathlineto{\pgfqpoint{2.184414in}{5.134979in}}%
\pgfpathlineto{\pgfqpoint{2.199847in}{5.140887in}}%
\pgfpathlineto{\pgfqpoint{2.230714in}{5.156411in}}%
\pgfpathlineto{\pgfqpoint{2.292449in}{5.192277in}}%
\pgfpathlineto{\pgfqpoint{2.307883in}{5.199825in}}%
\pgfpathlineto{\pgfqpoint{2.323316in}{5.205926in}}%
\pgfpathlineto{\pgfqpoint{2.338750in}{5.210152in}}%
\pgfpathlineto{\pgfqpoint{2.354184in}{5.212101in}}%
\pgfpathlineto{\pgfqpoint{2.369617in}{5.211407in}}%
\pgfpathlineto{\pgfqpoint{2.385051in}{5.207755in}}%
\pgfpathlineto{\pgfqpoint{2.400484in}{5.200890in}}%
\pgfpathlineto{\pgfqpoint{2.415918in}{5.190626in}}%
\pgfpathlineto{\pgfqpoint{2.431352in}{5.176855in}}%
\pgfpathlineto{\pgfqpoint{2.446785in}{5.159553in}}%
\pgfpathlineto{\pgfqpoint{2.462219in}{5.138783in}}%
\pgfpathlineto{\pgfqpoint{2.477653in}{5.114702in}}%
\pgfpathlineto{\pgfqpoint{2.493086in}{5.087553in}}%
\pgfpathlineto{\pgfqpoint{2.523954in}{5.025466in}}%
\pgfpathlineto{\pgfqpoint{2.554821in}{4.956116in}}%
\pgfpathlineto{\pgfqpoint{2.616555in}{4.814669in}}%
\pgfpathlineto{\pgfqpoint{2.631989in}{4.782584in}}%
\pgfpathlineto{\pgfqpoint{2.647423in}{4.753117in}}%
\pgfpathlineto{\pgfqpoint{2.662856in}{4.726866in}}%
\pgfpathlineto{\pgfqpoint{2.678290in}{4.704367in}}%
\pgfpathlineto{\pgfqpoint{2.693724in}{4.686088in}}%
\pgfpathlineto{\pgfqpoint{2.709157in}{4.672407in}}%
\pgfpathlineto{\pgfqpoint{2.724591in}{4.663608in}}%
\pgfpathlineto{\pgfqpoint{2.740024in}{4.659870in}}%
\pgfpathlineto{\pgfqpoint{2.755458in}{4.661262in}}%
\pgfpathlineto{\pgfqpoint{2.770892in}{4.667741in}}%
\pgfpathlineto{\pgfqpoint{2.786325in}{4.679151in}}%
\pgfpathlineto{\pgfqpoint{2.801759in}{4.695226in}}%
\pgfpathlineto{\pgfqpoint{2.817193in}{4.715597in}}%
\pgfpathlineto{\pgfqpoint{2.832626in}{4.739802in}}%
\pgfpathlineto{\pgfqpoint{2.848060in}{4.767295in}}%
\pgfpathlineto{\pgfqpoint{2.878927in}{4.829627in}}%
\pgfpathlineto{\pgfqpoint{2.956095in}{4.995475in}}%
\pgfpathlineto{\pgfqpoint{2.971529in}{5.024770in}}%
\pgfpathlineto{\pgfqpoint{2.986963in}{5.051335in}}%
\pgfpathlineto{\pgfqpoint{3.002396in}{5.074688in}}%
\pgfpathlineto{\pgfqpoint{3.017830in}{5.094437in}}%
\pgfpathlineto{\pgfqpoint{3.033263in}{5.110286in}}%
\pgfpathlineto{\pgfqpoint{3.048697in}{5.122042in}}%
\pgfpathlineto{\pgfqpoint{3.064131in}{5.129617in}}%
\pgfpathlineto{\pgfqpoint{3.079564in}{5.133025in}}%
\pgfpathlineto{\pgfqpoint{3.094998in}{5.132381in}}%
\pgfpathlineto{\pgfqpoint{3.110432in}{5.127895in}}%
\pgfpathlineto{\pgfqpoint{3.125865in}{5.119860in}}%
\pgfpathlineto{\pgfqpoint{3.141299in}{5.108644in}}%
\pgfpathlineto{\pgfqpoint{3.156733in}{5.094676in}}%
\pgfpathlineto{\pgfqpoint{3.172166in}{5.078430in}}%
\pgfpathlineto{\pgfqpoint{3.203033in}{5.041139in}}%
\pgfpathlineto{\pgfqpoint{3.264768in}{4.961574in}}%
\pgfpathlineto{\pgfqpoint{3.295635in}{4.926338in}}%
\pgfpathlineto{\pgfqpoint{3.311069in}{4.910971in}}%
\pgfpathlineto{\pgfqpoint{3.326503in}{4.897342in}}%
\pgfpathlineto{\pgfqpoint{3.341936in}{4.885533in}}%
\pgfpathlineto{\pgfqpoint{3.357370in}{4.875549in}}%
\pgfpathlineto{\pgfqpoint{3.372803in}{4.867321in}}%
\pgfpathlineto{\pgfqpoint{3.388237in}{4.860717in}}%
\pgfpathlineto{\pgfqpoint{3.403671in}{4.855542in}}%
\pgfpathlineto{\pgfqpoint{3.434538in}{4.848489in}}%
\pgfpathlineto{\pgfqpoint{3.511706in}{4.836340in}}%
\pgfpathlineto{\pgfqpoint{3.542573in}{4.827978in}}%
\pgfpathlineto{\pgfqpoint{3.573441in}{4.815629in}}%
\pgfpathlineto{\pgfqpoint{3.604308in}{4.799315in}}%
\pgfpathlineto{\pgfqpoint{3.650609in}{4.770203in}}%
\pgfpathlineto{\pgfqpoint{3.681476in}{4.751193in}}%
\pgfpathlineto{\pgfqpoint{3.696910in}{4.742980in}}%
\pgfpathlineto{\pgfqpoint{3.712343in}{4.736207in}}%
\pgfpathlineto{\pgfqpoint{3.727777in}{4.731311in}}%
\pgfpathlineto{\pgfqpoint{3.743211in}{4.728717in}}%
\pgfpathlineto{\pgfqpoint{3.758644in}{4.728817in}}%
\pgfpathlineto{\pgfqpoint{3.774078in}{4.731963in}}%
\pgfpathlineto{\pgfqpoint{3.789512in}{4.738450in}}%
\pgfpathlineto{\pgfqpoint{3.804945in}{4.748505in}}%
\pgfpathlineto{\pgfqpoint{3.820379in}{4.762277in}}%
\pgfpathlineto{\pgfqpoint{3.835813in}{4.779834in}}%
\pgfpathlineto{\pgfqpoint{3.851246in}{4.801149in}}%
\pgfpathlineto{\pgfqpoint{3.866680in}{4.826107in}}%
\pgfpathlineto{\pgfqpoint{3.882113in}{4.854500in}}%
\pgfpathlineto{\pgfqpoint{3.897547in}{4.886028in}}%
\pgfpathlineto{\pgfqpoint{3.928414in}{4.956895in}}%
\pgfpathlineto{\pgfqpoint{3.959282in}{5.034815in}}%
\pgfpathlineto{\pgfqpoint{4.021016in}{5.192536in}}%
\pgfpathlineto{\pgfqpoint{4.051883in}{5.262267in}}%
\pgfpathlineto{\pgfqpoint{4.067317in}{5.292837in}}%
\pgfpathlineto{\pgfqpoint{4.082751in}{5.319920in}}%
\pgfpathlineto{\pgfqpoint{4.098184in}{5.343144in}}%
\pgfpathlineto{\pgfqpoint{4.113618in}{5.362224in}}%
\pgfpathlineto{\pgfqpoint{4.129052in}{5.376970in}}%
\pgfpathlineto{\pgfqpoint{4.144485in}{5.387291in}}%
\pgfpathlineto{\pgfqpoint{4.159919in}{5.393195in}}%
\pgfpathlineto{\pgfqpoint{4.175352in}{5.394788in}}%
\pgfpathlineto{\pgfqpoint{4.190786in}{5.392271in}}%
\pgfpathlineto{\pgfqpoint{4.206220in}{5.385933in}}%
\pgfpathlineto{\pgfqpoint{4.221653in}{5.376144in}}%
\pgfpathlineto{\pgfqpoint{4.237087in}{5.363340in}}%
\pgfpathlineto{\pgfqpoint{4.252521in}{5.348015in}}%
\pgfpathlineto{\pgfqpoint{4.283388in}{5.311977in}}%
\pgfpathlineto{\pgfqpoint{4.345122in}{5.234303in}}%
\pgfpathlineto{\pgfqpoint{4.360556in}{5.216889in}}%
\pgfpathlineto{\pgfqpoint{4.375990in}{5.201213in}}%
\pgfpathlineto{\pgfqpoint{4.391423in}{5.187644in}}%
\pgfpathlineto{\pgfqpoint{4.406857in}{5.176478in}}%
\pgfpathlineto{\pgfqpoint{4.422291in}{5.167937in}}%
\pgfpathlineto{\pgfqpoint{4.437724in}{5.162163in}}%
\pgfpathlineto{\pgfqpoint{4.453158in}{5.159219in}}%
\pgfpathlineto{\pgfqpoint{4.468592in}{5.159090in}}%
\pgfpathlineto{\pgfqpoint{4.484025in}{5.161688in}}%
\pgfpathlineto{\pgfqpoint{4.499459in}{5.166858in}}%
\pgfpathlineto{\pgfqpoint{4.514892in}{5.174381in}}%
\pgfpathlineto{\pgfqpoint{4.530326in}{5.183993in}}%
\pgfpathlineto{\pgfqpoint{4.545760in}{5.195382in}}%
\pgfpathlineto{\pgfqpoint{4.576627in}{5.222120in}}%
\pgfpathlineto{\pgfqpoint{4.669229in}{5.308536in}}%
\pgfpathlineto{\pgfqpoint{4.669229in}{5.308536in}}%
\pgfusepath{stroke}%
\end{pgfscope}%
\begin{pgfscope}%
\pgfpathrectangle{\pgfqpoint{0.634105in}{3.881603in}}{\pgfqpoint{4.227273in}{2.800000in}} %
\pgfusepath{clip}%
\pgfsetrectcap%
\pgfsetroundjoin%
\pgfsetlinewidth{0.501875pt}%
\definecolor{currentstroke}{rgb}{0.221569,0.905873,0.843667}%
\pgfsetstrokecolor{currentstroke}%
\pgfsetdash{}{0pt}%
\pgfpathmoveto{\pgfqpoint{0.826254in}{5.819127in}}%
\pgfpathlineto{\pgfqpoint{0.841687in}{5.821722in}}%
\pgfpathlineto{\pgfqpoint{0.857121in}{5.817236in}}%
\pgfpathlineto{\pgfqpoint{0.872555in}{5.805441in}}%
\pgfpathlineto{\pgfqpoint{0.887988in}{5.786268in}}%
\pgfpathlineto{\pgfqpoint{0.903422in}{5.759814in}}%
\pgfpathlineto{\pgfqpoint{0.918855in}{5.726343in}}%
\pgfpathlineto{\pgfqpoint{0.934289in}{5.686285in}}%
\pgfpathlineto{\pgfqpoint{0.949723in}{5.640229in}}%
\pgfpathlineto{\pgfqpoint{0.965156in}{5.588911in}}%
\pgfpathlineto{\pgfqpoint{0.996024in}{5.474067in}}%
\pgfpathlineto{\pgfqpoint{1.073192in}{5.166440in}}%
\pgfpathlineto{\pgfqpoint{1.088625in}{5.110699in}}%
\pgfpathlineto{\pgfqpoint{1.104059in}{5.059495in}}%
\pgfpathlineto{\pgfqpoint{1.119493in}{5.013796in}}%
\pgfpathlineto{\pgfqpoint{1.134926in}{4.974460in}}%
\pgfpathlineto{\pgfqpoint{1.150360in}{4.942204in}}%
\pgfpathlineto{\pgfqpoint{1.165794in}{4.917599in}}%
\pgfpathlineto{\pgfqpoint{1.181227in}{4.901051in}}%
\pgfpathlineto{\pgfqpoint{1.196661in}{4.892796in}}%
\pgfpathlineto{\pgfqpoint{1.212095in}{4.892897in}}%
\pgfpathlineto{\pgfqpoint{1.227528in}{4.901242in}}%
\pgfpathlineto{\pgfqpoint{1.242962in}{4.917555in}}%
\pgfpathlineto{\pgfqpoint{1.258395in}{4.941401in}}%
\pgfpathlineto{\pgfqpoint{1.273829in}{4.972197in}}%
\pgfpathlineto{\pgfqpoint{1.289263in}{5.009237in}}%
\pgfpathlineto{\pgfqpoint{1.304696in}{5.051701in}}%
\pgfpathlineto{\pgfqpoint{1.335564in}{5.149213in}}%
\pgfpathlineto{\pgfqpoint{1.381864in}{5.311914in}}%
\pgfpathlineto{\pgfqpoint{1.412732in}{5.419586in}}%
\pgfpathlineto{\pgfqpoint{1.443599in}{5.518092in}}%
\pgfpathlineto{\pgfqpoint{1.459033in}{5.562017in}}%
\pgfpathlineto{\pgfqpoint{1.474466in}{5.601588in}}%
\pgfpathlineto{\pgfqpoint{1.489900in}{5.636349in}}%
\pgfpathlineto{\pgfqpoint{1.505334in}{5.665961in}}%
\pgfpathlineto{\pgfqpoint{1.520767in}{5.690207in}}%
\pgfpathlineto{\pgfqpoint{1.536201in}{5.708985in}}%
\pgfpathlineto{\pgfqpoint{1.551634in}{5.722303in}}%
\pgfpathlineto{\pgfqpoint{1.567068in}{5.730271in}}%
\pgfpathlineto{\pgfqpoint{1.582502in}{5.733089in}}%
\pgfpathlineto{\pgfqpoint{1.597935in}{5.731033in}}%
\pgfpathlineto{\pgfqpoint{1.613369in}{5.724443in}}%
\pgfpathlineto{\pgfqpoint{1.628803in}{5.713708in}}%
\pgfpathlineto{\pgfqpoint{1.644236in}{5.699251in}}%
\pgfpathlineto{\pgfqpoint{1.659670in}{5.681512in}}%
\pgfpathlineto{\pgfqpoint{1.675104in}{5.660939in}}%
\pgfpathlineto{\pgfqpoint{1.705971in}{5.613031in}}%
\pgfpathlineto{\pgfqpoint{1.736838in}{5.558781in}}%
\pgfpathlineto{\pgfqpoint{1.783139in}{5.471320in}}%
\pgfpathlineto{\pgfqpoint{1.860307in}{5.323827in}}%
\pgfpathlineto{\pgfqpoint{1.906608in}{5.239920in}}%
\pgfpathlineto{\pgfqpoint{1.937475in}{5.187589in}}%
\pgfpathlineto{\pgfqpoint{1.968343in}{5.139426in}}%
\pgfpathlineto{\pgfqpoint{1.999210in}{5.096940in}}%
\pgfpathlineto{\pgfqpoint{2.014644in}{5.078395in}}%
\pgfpathlineto{\pgfqpoint{2.030077in}{5.061984in}}%
\pgfpathlineto{\pgfqpoint{2.045511in}{5.047967in}}%
\pgfpathlineto{\pgfqpoint{2.060944in}{5.036601in}}%
\pgfpathlineto{\pgfqpoint{2.076378in}{5.028127in}}%
\pgfpathlineto{\pgfqpoint{2.091812in}{5.022763in}}%
\pgfpathlineto{\pgfqpoint{2.107245in}{5.020690in}}%
\pgfpathlineto{\pgfqpoint{2.122679in}{5.022043in}}%
\pgfpathlineto{\pgfqpoint{2.138113in}{5.026897in}}%
\pgfpathlineto{\pgfqpoint{2.153546in}{5.035262in}}%
\pgfpathlineto{\pgfqpoint{2.168980in}{5.047069in}}%
\pgfpathlineto{\pgfqpoint{2.184414in}{5.062171in}}%
\pgfpathlineto{\pgfqpoint{2.199847in}{5.080331in}}%
\pgfpathlineto{\pgfqpoint{2.215281in}{5.101229in}}%
\pgfpathlineto{\pgfqpoint{2.246148in}{5.149519in}}%
\pgfpathlineto{\pgfqpoint{2.323316in}{5.280612in}}%
\pgfpathlineto{\pgfqpoint{2.338750in}{5.303026in}}%
\pgfpathlineto{\pgfqpoint{2.354184in}{5.322457in}}%
\pgfpathlineto{\pgfqpoint{2.369617in}{5.338211in}}%
\pgfpathlineto{\pgfqpoint{2.385051in}{5.349650in}}%
\pgfpathlineto{\pgfqpoint{2.400484in}{5.356210in}}%
\pgfpathlineto{\pgfqpoint{2.415918in}{5.357420in}}%
\pgfpathlineto{\pgfqpoint{2.431352in}{5.352919in}}%
\pgfpathlineto{\pgfqpoint{2.446785in}{5.342467in}}%
\pgfpathlineto{\pgfqpoint{2.462219in}{5.325954in}}%
\pgfpathlineto{\pgfqpoint{2.477653in}{5.303415in}}%
\pgfpathlineto{\pgfqpoint{2.493086in}{5.275024in}}%
\pgfpathlineto{\pgfqpoint{2.508520in}{5.241102in}}%
\pgfpathlineto{\pgfqpoint{2.523954in}{5.202106in}}%
\pgfpathlineto{\pgfqpoint{2.539387in}{5.158630in}}%
\pgfpathlineto{\pgfqpoint{2.570254in}{5.061181in}}%
\pgfpathlineto{\pgfqpoint{2.647423in}{4.799341in}}%
\pgfpathlineto{\pgfqpoint{2.662856in}{4.751898in}}%
\pgfpathlineto{\pgfqpoint{2.678290in}{4.708387in}}%
\pgfpathlineto{\pgfqpoint{2.693724in}{4.669667in}}%
\pgfpathlineto{\pgfqpoint{2.709157in}{4.636499in}}%
\pgfpathlineto{\pgfqpoint{2.724591in}{4.609523in}}%
\pgfpathlineto{\pgfqpoint{2.740024in}{4.589242in}}%
\pgfpathlineto{\pgfqpoint{2.755458in}{4.576013in}}%
\pgfpathlineto{\pgfqpoint{2.770892in}{4.570034in}}%
\pgfpathlineto{\pgfqpoint{2.786325in}{4.571343in}}%
\pgfpathlineto{\pgfqpoint{2.801759in}{4.579817in}}%
\pgfpathlineto{\pgfqpoint{2.817193in}{4.595178in}}%
\pgfpathlineto{\pgfqpoint{2.832626in}{4.616995in}}%
\pgfpathlineto{\pgfqpoint{2.848060in}{4.644704in}}%
\pgfpathlineto{\pgfqpoint{2.863493in}{4.677620in}}%
\pgfpathlineto{\pgfqpoint{2.878927in}{4.714953in}}%
\pgfpathlineto{\pgfqpoint{2.909794in}{4.799338in}}%
\pgfpathlineto{\pgfqpoint{2.986963in}{5.022688in}}%
\pgfpathlineto{\pgfqpoint{3.002396in}{5.062223in}}%
\pgfpathlineto{\pgfqpoint{3.017830in}{5.098259in}}%
\pgfpathlineto{\pgfqpoint{3.033263in}{5.130224in}}%
\pgfpathlineto{\pgfqpoint{3.048697in}{5.157671in}}%
\pgfpathlineto{\pgfqpoint{3.064131in}{5.180276in}}%
\pgfpathlineto{\pgfqpoint{3.079564in}{5.197850in}}%
\pgfpathlineto{\pgfqpoint{3.094998in}{5.210328in}}%
\pgfpathlineto{\pgfqpoint{3.110432in}{5.217772in}}%
\pgfpathlineto{\pgfqpoint{3.125865in}{5.220359in}}%
\pgfpathlineto{\pgfqpoint{3.141299in}{5.218370in}}%
\pgfpathlineto{\pgfqpoint{3.156733in}{5.212175in}}%
\pgfpathlineto{\pgfqpoint{3.172166in}{5.202219in}}%
\pgfpathlineto{\pgfqpoint{3.187600in}{5.189002in}}%
\pgfpathlineto{\pgfqpoint{3.203033in}{5.173061in}}%
\pgfpathlineto{\pgfqpoint{3.233901in}{5.135229in}}%
\pgfpathlineto{\pgfqpoint{3.341936in}{4.991015in}}%
\pgfpathlineto{\pgfqpoint{3.372803in}{4.956639in}}%
\pgfpathlineto{\pgfqpoint{3.403671in}{4.926385in}}%
\pgfpathlineto{\pgfqpoint{3.434538in}{4.899182in}}%
\pgfpathlineto{\pgfqpoint{3.511706in}{4.833890in}}%
\pgfpathlineto{\pgfqpoint{3.542573in}{4.805320in}}%
\pgfpathlineto{\pgfqpoint{3.588874in}{4.759185in}}%
\pgfpathlineto{\pgfqpoint{3.635175in}{4.713795in}}%
\pgfpathlineto{\pgfqpoint{3.650609in}{4.700439in}}%
\pgfpathlineto{\pgfqpoint{3.666043in}{4.688715in}}%
\pgfpathlineto{\pgfqpoint{3.681476in}{4.679111in}}%
\pgfpathlineto{\pgfqpoint{3.696910in}{4.672120in}}%
\pgfpathlineto{\pgfqpoint{3.712343in}{4.668225in}}%
\pgfpathlineto{\pgfqpoint{3.727777in}{4.667878in}}%
\pgfpathlineto{\pgfqpoint{3.743211in}{4.671484in}}%
\pgfpathlineto{\pgfqpoint{3.758644in}{4.679384in}}%
\pgfpathlineto{\pgfqpoint{3.774078in}{4.691840in}}%
\pgfpathlineto{\pgfqpoint{3.789512in}{4.709017in}}%
\pgfpathlineto{\pgfqpoint{3.804945in}{4.730981in}}%
\pgfpathlineto{\pgfqpoint{3.820379in}{4.757680in}}%
\pgfpathlineto{\pgfqpoint{3.835813in}{4.788947in}}%
\pgfpathlineto{\pgfqpoint{3.851246in}{4.824496in}}%
\pgfpathlineto{\pgfqpoint{3.866680in}{4.863923in}}%
\pgfpathlineto{\pgfqpoint{3.897547in}{4.952247in}}%
\pgfpathlineto{\pgfqpoint{3.943848in}{5.097899in}}%
\pgfpathlineto{\pgfqpoint{3.974715in}{5.194148in}}%
\pgfpathlineto{\pgfqpoint{4.005583in}{5.281549in}}%
\pgfpathlineto{\pgfqpoint{4.021016in}{5.319854in}}%
\pgfpathlineto{\pgfqpoint{4.036450in}{5.353577in}}%
\pgfpathlineto{\pgfqpoint{4.051883in}{5.382093in}}%
\pgfpathlineto{\pgfqpoint{4.067317in}{5.404887in}}%
\pgfpathlineto{\pgfqpoint{4.082751in}{5.421572in}}%
\pgfpathlineto{\pgfqpoint{4.098184in}{5.431895in}}%
\pgfpathlineto{\pgfqpoint{4.113618in}{5.435749in}}%
\pgfpathlineto{\pgfqpoint{4.129052in}{5.433172in}}%
\pgfpathlineto{\pgfqpoint{4.144485in}{5.424352in}}%
\pgfpathlineto{\pgfqpoint{4.159919in}{5.409618in}}%
\pgfpathlineto{\pgfqpoint{4.175352in}{5.389433in}}%
\pgfpathlineto{\pgfqpoint{4.190786in}{5.364384in}}%
\pgfpathlineto{\pgfqpoint{4.206220in}{5.335164in}}%
\pgfpathlineto{\pgfqpoint{4.237087in}{5.267417in}}%
\pgfpathlineto{\pgfqpoint{4.314255in}{5.085555in}}%
\pgfpathlineto{\pgfqpoint{4.329689in}{5.054144in}}%
\pgfpathlineto{\pgfqpoint{4.345122in}{5.026208in}}%
\pgfpathlineto{\pgfqpoint{4.360556in}{5.002353in}}%
\pgfpathlineto{\pgfqpoint{4.375990in}{4.983073in}}%
\pgfpathlineto{\pgfqpoint{4.391423in}{4.968737in}}%
\pgfpathlineto{\pgfqpoint{4.406857in}{4.959587in}}%
\pgfpathlineto{\pgfqpoint{4.422291in}{4.955729in}}%
\pgfpathlineto{\pgfqpoint{4.437724in}{4.957142in}}%
\pgfpathlineto{\pgfqpoint{4.453158in}{4.963675in}}%
\pgfpathlineto{\pgfqpoint{4.468592in}{4.975061in}}%
\pgfpathlineto{\pgfqpoint{4.484025in}{4.990926in}}%
\pgfpathlineto{\pgfqpoint{4.499459in}{5.010801in}}%
\pgfpathlineto{\pgfqpoint{4.514892in}{5.034141in}}%
\pgfpathlineto{\pgfqpoint{4.530326in}{5.060340in}}%
\pgfpathlineto{\pgfqpoint{4.561193in}{5.118706in}}%
\pgfpathlineto{\pgfqpoint{4.622928in}{5.240815in}}%
\pgfpathlineto{\pgfqpoint{4.653795in}{5.295078in}}%
\pgfpathlineto{\pgfqpoint{4.669229in}{5.318895in}}%
\pgfpathlineto{\pgfqpoint{4.669229in}{5.318895in}}%
\pgfusepath{stroke}%
\end{pgfscope}%
\begin{pgfscope}%
\pgfpathrectangle{\pgfqpoint{0.634105in}{3.881603in}}{\pgfqpoint{4.227273in}{2.800000in}} %
\pgfusepath{clip}%
\pgfsetrectcap%
\pgfsetroundjoin%
\pgfsetlinewidth{0.501875pt}%
\definecolor{currentstroke}{rgb}{0.300000,0.951057,0.809017}%
\pgfsetstrokecolor{currentstroke}%
\pgfsetdash{}{0pt}%
\pgfpathmoveto{\pgfqpoint{0.826254in}{5.779353in}}%
\pgfpathlineto{\pgfqpoint{0.841687in}{5.787640in}}%
\pgfpathlineto{\pgfqpoint{0.857121in}{5.789867in}}%
\pgfpathlineto{\pgfqpoint{0.872555in}{5.785721in}}%
\pgfpathlineto{\pgfqpoint{0.887988in}{5.775022in}}%
\pgfpathlineto{\pgfqpoint{0.903422in}{5.757722in}}%
\pgfpathlineto{\pgfqpoint{0.918855in}{5.733921in}}%
\pgfpathlineto{\pgfqpoint{0.934289in}{5.703858in}}%
\pgfpathlineto{\pgfqpoint{0.949723in}{5.667917in}}%
\pgfpathlineto{\pgfqpoint{0.965156in}{5.626614in}}%
\pgfpathlineto{\pgfqpoint{0.980590in}{5.580591in}}%
\pgfpathlineto{\pgfqpoint{1.011457in}{5.477505in}}%
\pgfpathlineto{\pgfqpoint{1.104059in}{5.148613in}}%
\pgfpathlineto{\pgfqpoint{1.119493in}{5.101560in}}%
\pgfpathlineto{\pgfqpoint{1.134926in}{5.059251in}}%
\pgfpathlineto{\pgfqpoint{1.150360in}{5.022470in}}%
\pgfpathlineto{\pgfqpoint{1.165794in}{4.991890in}}%
\pgfpathlineto{\pgfqpoint{1.181227in}{4.968065in}}%
\pgfpathlineto{\pgfqpoint{1.196661in}{4.951410in}}%
\pgfpathlineto{\pgfqpoint{1.212095in}{4.942203in}}%
\pgfpathlineto{\pgfqpoint{1.227528in}{4.940574in}}%
\pgfpathlineto{\pgfqpoint{1.242962in}{4.946508in}}%
\pgfpathlineto{\pgfqpoint{1.258395in}{4.959843in}}%
\pgfpathlineto{\pgfqpoint{1.273829in}{4.980282in}}%
\pgfpathlineto{\pgfqpoint{1.289263in}{5.007397in}}%
\pgfpathlineto{\pgfqpoint{1.304696in}{5.040643in}}%
\pgfpathlineto{\pgfqpoint{1.320130in}{5.079371in}}%
\pgfpathlineto{\pgfqpoint{1.335564in}{5.122843in}}%
\pgfpathlineto{\pgfqpoint{1.366431in}{5.220740in}}%
\pgfpathlineto{\pgfqpoint{1.412732in}{5.381698in}}%
\pgfpathlineto{\pgfqpoint{1.443599in}{5.488021in}}%
\pgfpathlineto{\pgfqpoint{1.474466in}{5.585915in}}%
\pgfpathlineto{\pgfqpoint{1.489900in}{5.629958in}}%
\pgfpathlineto{\pgfqpoint{1.505334in}{5.669964in}}%
\pgfpathlineto{\pgfqpoint{1.520767in}{5.705473in}}%
\pgfpathlineto{\pgfqpoint{1.536201in}{5.736117in}}%
\pgfpathlineto{\pgfqpoint{1.551634in}{5.761630in}}%
\pgfpathlineto{\pgfqpoint{1.567068in}{5.781837in}}%
\pgfpathlineto{\pgfqpoint{1.582502in}{5.796660in}}%
\pgfpathlineto{\pgfqpoint{1.597935in}{5.806108in}}%
\pgfpathlineto{\pgfqpoint{1.613369in}{5.810270in}}%
\pgfpathlineto{\pgfqpoint{1.628803in}{5.809310in}}%
\pgfpathlineto{\pgfqpoint{1.644236in}{5.803457in}}%
\pgfpathlineto{\pgfqpoint{1.659670in}{5.792994in}}%
\pgfpathlineto{\pgfqpoint{1.675104in}{5.778252in}}%
\pgfpathlineto{\pgfqpoint{1.690537in}{5.759595in}}%
\pgfpathlineto{\pgfqpoint{1.705971in}{5.737416in}}%
\pgfpathlineto{\pgfqpoint{1.721404in}{5.712123in}}%
\pgfpathlineto{\pgfqpoint{1.752272in}{5.653874in}}%
\pgfpathlineto{\pgfqpoint{1.783139in}{5.588184in}}%
\pgfpathlineto{\pgfqpoint{1.844874in}{5.446990in}}%
\pgfpathlineto{\pgfqpoint{1.891174in}{5.343321in}}%
\pgfpathlineto{\pgfqpoint{1.922042in}{5.279633in}}%
\pgfpathlineto{\pgfqpoint{1.952909in}{5.222552in}}%
\pgfpathlineto{\pgfqpoint{1.968343in}{5.197016in}}%
\pgfpathlineto{\pgfqpoint{1.983776in}{5.173736in}}%
\pgfpathlineto{\pgfqpoint{1.999210in}{5.152876in}}%
\pgfpathlineto{\pgfqpoint{2.014644in}{5.134586in}}%
\pgfpathlineto{\pgfqpoint{2.030077in}{5.118994in}}%
\pgfpathlineto{\pgfqpoint{2.045511in}{5.106202in}}%
\pgfpathlineto{\pgfqpoint{2.060944in}{5.096290in}}%
\pgfpathlineto{\pgfqpoint{2.076378in}{5.089302in}}%
\pgfpathlineto{\pgfqpoint{2.091812in}{5.085252in}}%
\pgfpathlineto{\pgfqpoint{2.107245in}{5.084115in}}%
\pgfpathlineto{\pgfqpoint{2.122679in}{5.085824in}}%
\pgfpathlineto{\pgfqpoint{2.138113in}{5.090271in}}%
\pgfpathlineto{\pgfqpoint{2.153546in}{5.097298in}}%
\pgfpathlineto{\pgfqpoint{2.168980in}{5.106702in}}%
\pgfpathlineto{\pgfqpoint{2.184414in}{5.118232in}}%
\pgfpathlineto{\pgfqpoint{2.199847in}{5.131592in}}%
\pgfpathlineto{\pgfqpoint{2.230714in}{5.162388in}}%
\pgfpathlineto{\pgfqpoint{2.292449in}{5.228472in}}%
\pgfpathlineto{\pgfqpoint{2.307883in}{5.243206in}}%
\pgfpathlineto{\pgfqpoint{2.323316in}{5.256268in}}%
\pgfpathlineto{\pgfqpoint{2.338750in}{5.267199in}}%
\pgfpathlineto{\pgfqpoint{2.354184in}{5.275573in}}%
\pgfpathlineto{\pgfqpoint{2.369617in}{5.281004in}}%
\pgfpathlineto{\pgfqpoint{2.385051in}{5.283158in}}%
\pgfpathlineto{\pgfqpoint{2.400484in}{5.281766in}}%
\pgfpathlineto{\pgfqpoint{2.415918in}{5.276625in}}%
\pgfpathlineto{\pgfqpoint{2.431352in}{5.267614in}}%
\pgfpathlineto{\pgfqpoint{2.446785in}{5.254695in}}%
\pgfpathlineto{\pgfqpoint{2.462219in}{5.237916in}}%
\pgfpathlineto{\pgfqpoint{2.477653in}{5.217415in}}%
\pgfpathlineto{\pgfqpoint{2.493086in}{5.193420in}}%
\pgfpathlineto{\pgfqpoint{2.508520in}{5.166242in}}%
\pgfpathlineto{\pgfqpoint{2.539387in}{5.103984in}}%
\pgfpathlineto{\pgfqpoint{2.570254in}{5.034600in}}%
\pgfpathlineto{\pgfqpoint{2.616555in}{4.927744in}}%
\pgfpathlineto{\pgfqpoint{2.647423in}{4.862194in}}%
\pgfpathlineto{\pgfqpoint{2.662856in}{4.832992in}}%
\pgfpathlineto{\pgfqpoint{2.678290in}{4.806909in}}%
\pgfpathlineto{\pgfqpoint{2.693724in}{4.784422in}}%
\pgfpathlineto{\pgfqpoint{2.709157in}{4.765929in}}%
\pgfpathlineto{\pgfqpoint{2.724591in}{4.751741in}}%
\pgfpathlineto{\pgfqpoint{2.740024in}{4.742078in}}%
\pgfpathlineto{\pgfqpoint{2.755458in}{4.737057in}}%
\pgfpathlineto{\pgfqpoint{2.770892in}{4.736693in}}%
\pgfpathlineto{\pgfqpoint{2.786325in}{4.740901in}}%
\pgfpathlineto{\pgfqpoint{2.801759in}{4.749493in}}%
\pgfpathlineto{\pgfqpoint{2.817193in}{4.762188in}}%
\pgfpathlineto{\pgfqpoint{2.832626in}{4.778618in}}%
\pgfpathlineto{\pgfqpoint{2.848060in}{4.798337in}}%
\pgfpathlineto{\pgfqpoint{2.863493in}{4.820835in}}%
\pgfpathlineto{\pgfqpoint{2.894361in}{4.871882in}}%
\pgfpathlineto{\pgfqpoint{2.956095in}{4.980982in}}%
\pgfpathlineto{\pgfqpoint{2.986963in}{5.029563in}}%
\pgfpathlineto{\pgfqpoint{3.002396in}{5.050594in}}%
\pgfpathlineto{\pgfqpoint{3.017830in}{5.068944in}}%
\pgfpathlineto{\pgfqpoint{3.033263in}{5.084334in}}%
\pgfpathlineto{\pgfqpoint{3.048697in}{5.096564in}}%
\pgfpathlineto{\pgfqpoint{3.064131in}{5.105523in}}%
\pgfpathlineto{\pgfqpoint{3.079564in}{5.111185in}}%
\pgfpathlineto{\pgfqpoint{3.094998in}{5.113605in}}%
\pgfpathlineto{\pgfqpoint{3.110432in}{5.112914in}}%
\pgfpathlineto{\pgfqpoint{3.125865in}{5.109314in}}%
\pgfpathlineto{\pgfqpoint{3.141299in}{5.103066in}}%
\pgfpathlineto{\pgfqpoint{3.156733in}{5.094480in}}%
\pgfpathlineto{\pgfqpoint{3.172166in}{5.083901in}}%
\pgfpathlineto{\pgfqpoint{3.203033in}{5.058265in}}%
\pgfpathlineto{\pgfqpoint{3.249334in}{5.014258in}}%
\pgfpathlineto{\pgfqpoint{3.295635in}{4.971470in}}%
\pgfpathlineto{\pgfqpoint{3.326503in}{4.946643in}}%
\pgfpathlineto{\pgfqpoint{3.357370in}{4.925671in}}%
\pgfpathlineto{\pgfqpoint{3.388237in}{4.908389in}}%
\pgfpathlineto{\pgfqpoint{3.419104in}{4.893967in}}%
\pgfpathlineto{\pgfqpoint{3.542573in}{4.841205in}}%
\pgfpathlineto{\pgfqpoint{3.619742in}{4.803703in}}%
\pgfpathlineto{\pgfqpoint{3.650609in}{4.792389in}}%
\pgfpathlineto{\pgfqpoint{3.666043in}{4.788731in}}%
\pgfpathlineto{\pgfqpoint{3.681476in}{4.786869in}}%
\pgfpathlineto{\pgfqpoint{3.696910in}{4.787146in}}%
\pgfpathlineto{\pgfqpoint{3.712343in}{4.789888in}}%
\pgfpathlineto{\pgfqpoint{3.727777in}{4.795387in}}%
\pgfpathlineto{\pgfqpoint{3.743211in}{4.803896in}}%
\pgfpathlineto{\pgfqpoint{3.758644in}{4.815609in}}%
\pgfpathlineto{\pgfqpoint{3.774078in}{4.830658in}}%
\pgfpathlineto{\pgfqpoint{3.789512in}{4.849098in}}%
\pgfpathlineto{\pgfqpoint{3.804945in}{4.870907in}}%
\pgfpathlineto{\pgfqpoint{3.820379in}{4.895977in}}%
\pgfpathlineto{\pgfqpoint{3.835813in}{4.924114in}}%
\pgfpathlineto{\pgfqpoint{3.866680in}{4.988387in}}%
\pgfpathlineto{\pgfqpoint{3.897547in}{5.060544in}}%
\pgfpathlineto{\pgfqpoint{3.959282in}{5.210846in}}%
\pgfpathlineto{\pgfqpoint{3.990149in}{5.278914in}}%
\pgfpathlineto{\pgfqpoint{4.005583in}{5.308997in}}%
\pgfpathlineto{\pgfqpoint{4.021016in}{5.335712in}}%
\pgfpathlineto{\pgfqpoint{4.036450in}{5.358588in}}%
\pgfpathlineto{\pgfqpoint{4.051883in}{5.377236in}}%
\pgfpathlineto{\pgfqpoint{4.067317in}{5.391357in}}%
\pgfpathlineto{\pgfqpoint{4.082751in}{5.400754in}}%
\pgfpathlineto{\pgfqpoint{4.098184in}{5.405335in}}%
\pgfpathlineto{\pgfqpoint{4.113618in}{5.405121in}}%
\pgfpathlineto{\pgfqpoint{4.129052in}{5.400243in}}%
\pgfpathlineto{\pgfqpoint{4.144485in}{5.390940in}}%
\pgfpathlineto{\pgfqpoint{4.159919in}{5.377553in}}%
\pgfpathlineto{\pgfqpoint{4.175352in}{5.360522in}}%
\pgfpathlineto{\pgfqpoint{4.190786in}{5.340365in}}%
\pgfpathlineto{\pgfqpoint{4.221653in}{5.293102in}}%
\pgfpathlineto{\pgfqpoint{4.298822in}{5.166374in}}%
\pgfpathlineto{\pgfqpoint{4.314255in}{5.145041in}}%
\pgfpathlineto{\pgfqpoint{4.329689in}{5.126461in}}%
\pgfpathlineto{\pgfqpoint{4.345122in}{5.111109in}}%
\pgfpathlineto{\pgfqpoint{4.360556in}{5.099371in}}%
\pgfpathlineto{\pgfqpoint{4.375990in}{5.091536in}}%
\pgfpathlineto{\pgfqpoint{4.391423in}{5.087790in}}%
\pgfpathlineto{\pgfqpoint{4.406857in}{5.088215in}}%
\pgfpathlineto{\pgfqpoint{4.422291in}{5.092784in}}%
\pgfpathlineto{\pgfqpoint{4.437724in}{5.101368in}}%
\pgfpathlineto{\pgfqpoint{4.453158in}{5.113744in}}%
\pgfpathlineto{\pgfqpoint{4.468592in}{5.129595in}}%
\pgfpathlineto{\pgfqpoint{4.484025in}{5.148529in}}%
\pgfpathlineto{\pgfqpoint{4.499459in}{5.170088in}}%
\pgfpathlineto{\pgfqpoint{4.530326in}{5.219000in}}%
\pgfpathlineto{\pgfqpoint{4.607494in}{5.348950in}}%
\pgfpathlineto{\pgfqpoint{4.622928in}{5.372008in}}%
\pgfpathlineto{\pgfqpoint{4.638362in}{5.393062in}}%
\pgfpathlineto{\pgfqpoint{4.653795in}{5.411796in}}%
\pgfpathlineto{\pgfqpoint{4.669229in}{5.427970in}}%
\pgfpathlineto{\pgfqpoint{4.669229in}{5.427970in}}%
\pgfusepath{stroke}%
\end{pgfscope}%
\begin{pgfscope}%
\pgfpathrectangle{\pgfqpoint{0.634105in}{3.881603in}}{\pgfqpoint{4.227273in}{2.800000in}} %
\pgfusepath{clip}%
\pgfsetrectcap%
\pgfsetroundjoin%
\pgfsetlinewidth{0.501875pt}%
\definecolor{currentstroke}{rgb}{0.378431,0.981823,0.771298}%
\pgfsetstrokecolor{currentstroke}%
\pgfsetdash{}{0pt}%
\pgfpathmoveto{\pgfqpoint{0.826254in}{5.738709in}}%
\pgfpathlineto{\pgfqpoint{0.841687in}{5.750766in}}%
\pgfpathlineto{\pgfqpoint{0.857121in}{5.757260in}}%
\pgfpathlineto{\pgfqpoint{0.872555in}{5.757888in}}%
\pgfpathlineto{\pgfqpoint{0.887988in}{5.752459in}}%
\pgfpathlineto{\pgfqpoint{0.903422in}{5.740902in}}%
\pgfpathlineto{\pgfqpoint{0.918855in}{5.723269in}}%
\pgfpathlineto{\pgfqpoint{0.934289in}{5.699738in}}%
\pgfpathlineto{\pgfqpoint{0.949723in}{5.670608in}}%
\pgfpathlineto{\pgfqpoint{0.965156in}{5.636298in}}%
\pgfpathlineto{\pgfqpoint{0.980590in}{5.597337in}}%
\pgfpathlineto{\pgfqpoint{0.996024in}{5.554356in}}%
\pgfpathlineto{\pgfqpoint{1.026891in}{5.459288in}}%
\pgfpathlineto{\pgfqpoint{1.104059in}{5.208631in}}%
\pgfpathlineto{\pgfqpoint{1.119493in}{5.163480in}}%
\pgfpathlineto{\pgfqpoint{1.134926in}{5.121990in}}%
\pgfpathlineto{\pgfqpoint{1.150360in}{5.084910in}}%
\pgfpathlineto{\pgfqpoint{1.165794in}{5.052904in}}%
\pgfpathlineto{\pgfqpoint{1.181227in}{5.026540in}}%
\pgfpathlineto{\pgfqpoint{1.196661in}{5.006281in}}%
\pgfpathlineto{\pgfqpoint{1.212095in}{4.992473in}}%
\pgfpathlineto{\pgfqpoint{1.227528in}{4.985341in}}%
\pgfpathlineto{\pgfqpoint{1.242962in}{4.984990in}}%
\pgfpathlineto{\pgfqpoint{1.258395in}{4.991396in}}%
\pgfpathlineto{\pgfqpoint{1.273829in}{5.004417in}}%
\pgfpathlineto{\pgfqpoint{1.289263in}{5.023795in}}%
\pgfpathlineto{\pgfqpoint{1.304696in}{5.049161in}}%
\pgfpathlineto{\pgfqpoint{1.320130in}{5.080047in}}%
\pgfpathlineto{\pgfqpoint{1.335564in}{5.115896in}}%
\pgfpathlineto{\pgfqpoint{1.350997in}{5.156075in}}%
\pgfpathlineto{\pgfqpoint{1.381864in}{5.246597in}}%
\pgfpathlineto{\pgfqpoint{1.428165in}{5.396266in}}%
\pgfpathlineto{\pgfqpoint{1.474466in}{5.543866in}}%
\pgfpathlineto{\pgfqpoint{1.505334in}{5.631542in}}%
\pgfpathlineto{\pgfqpoint{1.520767in}{5.670349in}}%
\pgfpathlineto{\pgfqpoint{1.536201in}{5.705152in}}%
\pgfpathlineto{\pgfqpoint{1.551634in}{5.735561in}}%
\pgfpathlineto{\pgfqpoint{1.567068in}{5.761266in}}%
\pgfpathlineto{\pgfqpoint{1.582502in}{5.782037in}}%
\pgfpathlineto{\pgfqpoint{1.597935in}{5.797723in}}%
\pgfpathlineto{\pgfqpoint{1.613369in}{5.808256in}}%
\pgfpathlineto{\pgfqpoint{1.628803in}{5.813638in}}%
\pgfpathlineto{\pgfqpoint{1.644236in}{5.813945in}}%
\pgfpathlineto{\pgfqpoint{1.659670in}{5.809319in}}%
\pgfpathlineto{\pgfqpoint{1.675104in}{5.799959in}}%
\pgfpathlineto{\pgfqpoint{1.690537in}{5.786119in}}%
\pgfpathlineto{\pgfqpoint{1.705971in}{5.768099in}}%
\pgfpathlineto{\pgfqpoint{1.721404in}{5.746238in}}%
\pgfpathlineto{\pgfqpoint{1.736838in}{5.720909in}}%
\pgfpathlineto{\pgfqpoint{1.752272in}{5.692510in}}%
\pgfpathlineto{\pgfqpoint{1.783139in}{5.628187in}}%
\pgfpathlineto{\pgfqpoint{1.814006in}{5.556745in}}%
\pgfpathlineto{\pgfqpoint{1.906608in}{5.334422in}}%
\pgfpathlineto{\pgfqpoint{1.937475in}{5.268330in}}%
\pgfpathlineto{\pgfqpoint{1.968343in}{5.210796in}}%
\pgfpathlineto{\pgfqpoint{1.983776in}{5.185906in}}%
\pgfpathlineto{\pgfqpoint{1.999210in}{5.163892in}}%
\pgfpathlineto{\pgfqpoint{2.014644in}{5.144926in}}%
\pgfpathlineto{\pgfqpoint{2.030077in}{5.129141in}}%
\pgfpathlineto{\pgfqpoint{2.045511in}{5.116628in}}%
\pgfpathlineto{\pgfqpoint{2.060944in}{5.107433in}}%
\pgfpathlineto{\pgfqpoint{2.076378in}{5.101557in}}%
\pgfpathlineto{\pgfqpoint{2.091812in}{5.098954in}}%
\pgfpathlineto{\pgfqpoint{2.107245in}{5.099525in}}%
\pgfpathlineto{\pgfqpoint{2.122679in}{5.103127in}}%
\pgfpathlineto{\pgfqpoint{2.138113in}{5.109564in}}%
\pgfpathlineto{\pgfqpoint{2.153546in}{5.118592in}}%
\pgfpathlineto{\pgfqpoint{2.168980in}{5.129922in}}%
\pgfpathlineto{\pgfqpoint{2.184414in}{5.143219in}}%
\pgfpathlineto{\pgfqpoint{2.215281in}{5.174189in}}%
\pgfpathlineto{\pgfqpoint{2.277015in}{5.241305in}}%
\pgfpathlineto{\pgfqpoint{2.292449in}{5.256410in}}%
\pgfpathlineto{\pgfqpoint{2.307883in}{5.269898in}}%
\pgfpathlineto{\pgfqpoint{2.323316in}{5.281326in}}%
\pgfpathlineto{\pgfqpoint{2.338750in}{5.290278in}}%
\pgfpathlineto{\pgfqpoint{2.354184in}{5.296382in}}%
\pgfpathlineto{\pgfqpoint{2.369617in}{5.299317in}}%
\pgfpathlineto{\pgfqpoint{2.385051in}{5.298817in}}%
\pgfpathlineto{\pgfqpoint{2.400484in}{5.294685in}}%
\pgfpathlineto{\pgfqpoint{2.415918in}{5.286794in}}%
\pgfpathlineto{\pgfqpoint{2.431352in}{5.275093in}}%
\pgfpathlineto{\pgfqpoint{2.446785in}{5.259613in}}%
\pgfpathlineto{\pgfqpoint{2.462219in}{5.240464in}}%
\pgfpathlineto{\pgfqpoint{2.477653in}{5.217839in}}%
\pgfpathlineto{\pgfqpoint{2.493086in}{5.192005in}}%
\pgfpathlineto{\pgfqpoint{2.508520in}{5.163309in}}%
\pgfpathlineto{\pgfqpoint{2.539387in}{5.099034in}}%
\pgfpathlineto{\pgfqpoint{2.585688in}{4.993208in}}%
\pgfpathlineto{\pgfqpoint{2.616555in}{4.923200in}}%
\pgfpathlineto{\pgfqpoint{2.647423in}{4.859232in}}%
\pgfpathlineto{\pgfqpoint{2.662856in}{4.830918in}}%
\pgfpathlineto{\pgfqpoint{2.678290in}{4.805712in}}%
\pgfpathlineto{\pgfqpoint{2.693724in}{4.784038in}}%
\pgfpathlineto{\pgfqpoint{2.709157in}{4.766244in}}%
\pgfpathlineto{\pgfqpoint{2.724591in}{4.752604in}}%
\pgfpathlineto{\pgfqpoint{2.740024in}{4.743299in}}%
\pgfpathlineto{\pgfqpoint{2.755458in}{4.738425in}}%
\pgfpathlineto{\pgfqpoint{2.770892in}{4.737983in}}%
\pgfpathlineto{\pgfqpoint{2.786325in}{4.741880in}}%
\pgfpathlineto{\pgfqpoint{2.801759in}{4.749936in}}%
\pgfpathlineto{\pgfqpoint{2.817193in}{4.761883in}}%
\pgfpathlineto{\pgfqpoint{2.832626in}{4.777375in}}%
\pgfpathlineto{\pgfqpoint{2.848060in}{4.795995in}}%
\pgfpathlineto{\pgfqpoint{2.863493in}{4.817267in}}%
\pgfpathlineto{\pgfqpoint{2.894361in}{4.865626in}}%
\pgfpathlineto{\pgfqpoint{2.971529in}{4.993424in}}%
\pgfpathlineto{\pgfqpoint{2.986963in}{5.015664in}}%
\pgfpathlineto{\pgfqpoint{3.002396in}{5.035658in}}%
\pgfpathlineto{\pgfqpoint{3.017830in}{5.053026in}}%
\pgfpathlineto{\pgfqpoint{3.033263in}{5.067458in}}%
\pgfpathlineto{\pgfqpoint{3.048697in}{5.078719in}}%
\pgfpathlineto{\pgfqpoint{3.064131in}{5.086653in}}%
\pgfpathlineto{\pgfqpoint{3.079564in}{5.091182in}}%
\pgfpathlineto{\pgfqpoint{3.094998in}{5.092308in}}%
\pgfpathlineto{\pgfqpoint{3.110432in}{5.090105in}}%
\pgfpathlineto{\pgfqpoint{3.125865in}{5.084722in}}%
\pgfpathlineto{\pgfqpoint{3.141299in}{5.076369in}}%
\pgfpathlineto{\pgfqpoint{3.156733in}{5.065314in}}%
\pgfpathlineto{\pgfqpoint{3.172166in}{5.051871in}}%
\pgfpathlineto{\pgfqpoint{3.187600in}{5.036392in}}%
\pgfpathlineto{\pgfqpoint{3.218467in}{5.000864in}}%
\pgfpathlineto{\pgfqpoint{3.311069in}{4.885735in}}%
\pgfpathlineto{\pgfqpoint{3.341936in}{4.853479in}}%
\pgfpathlineto{\pgfqpoint{3.357370in}{4.839579in}}%
\pgfpathlineto{\pgfqpoint{3.372803in}{4.827339in}}%
\pgfpathlineto{\pgfqpoint{3.388237in}{4.816834in}}%
\pgfpathlineto{\pgfqpoint{3.403671in}{4.808100in}}%
\pgfpathlineto{\pgfqpoint{3.419104in}{4.801133in}}%
\pgfpathlineto{\pgfqpoint{3.434538in}{4.795899in}}%
\pgfpathlineto{\pgfqpoint{3.449972in}{4.792337in}}%
\pgfpathlineto{\pgfqpoint{3.465405in}{4.790364in}}%
\pgfpathlineto{\pgfqpoint{3.480839in}{4.789885in}}%
\pgfpathlineto{\pgfqpoint{3.496273in}{4.790797in}}%
\pgfpathlineto{\pgfqpoint{3.511706in}{4.792995in}}%
\pgfpathlineto{\pgfqpoint{3.542573in}{4.800859in}}%
\pgfpathlineto{\pgfqpoint{3.573441in}{4.812806in}}%
\pgfpathlineto{\pgfqpoint{3.604308in}{4.828418in}}%
\pgfpathlineto{\pgfqpoint{3.635175in}{4.847564in}}%
\pgfpathlineto{\pgfqpoint{3.666043in}{4.870367in}}%
\pgfpathlineto{\pgfqpoint{3.696910in}{4.897100in}}%
\pgfpathlineto{\pgfqpoint{3.727777in}{4.928041in}}%
\pgfpathlineto{\pgfqpoint{3.758644in}{4.963296in}}%
\pgfpathlineto{\pgfqpoint{3.789512in}{5.002637in}}%
\pgfpathlineto{\pgfqpoint{3.820379in}{5.045377in}}%
\pgfpathlineto{\pgfqpoint{3.928414in}{5.199974in}}%
\pgfpathlineto{\pgfqpoint{3.959282in}{5.236815in}}%
\pgfpathlineto{\pgfqpoint{3.974715in}{5.252571in}}%
\pgfpathlineto{\pgfqpoint{3.990149in}{5.266212in}}%
\pgfpathlineto{\pgfqpoint{4.005583in}{5.277523in}}%
\pgfpathlineto{\pgfqpoint{4.021016in}{5.286331in}}%
\pgfpathlineto{\pgfqpoint{4.036450in}{5.292515in}}%
\pgfpathlineto{\pgfqpoint{4.051883in}{5.296004in}}%
\pgfpathlineto{\pgfqpoint{4.067317in}{5.296789in}}%
\pgfpathlineto{\pgfqpoint{4.082751in}{5.294921in}}%
\pgfpathlineto{\pgfqpoint{4.098184in}{5.290515in}}%
\pgfpathlineto{\pgfqpoint{4.113618in}{5.283747in}}%
\pgfpathlineto{\pgfqpoint{4.129052in}{5.274855in}}%
\pgfpathlineto{\pgfqpoint{4.144485in}{5.264130in}}%
\pgfpathlineto{\pgfqpoint{4.175352in}{5.238597in}}%
\pgfpathlineto{\pgfqpoint{4.237087in}{5.182964in}}%
\pgfpathlineto{\pgfqpoint{4.252521in}{5.170730in}}%
\pgfpathlineto{\pgfqpoint{4.267954in}{5.160045in}}%
\pgfpathlineto{\pgfqpoint{4.283388in}{5.151304in}}%
\pgfpathlineto{\pgfqpoint{4.298822in}{5.144858in}}%
\pgfpathlineto{\pgfqpoint{4.314255in}{5.141004in}}%
\pgfpathlineto{\pgfqpoint{4.329689in}{5.139980in}}%
\pgfpathlineto{\pgfqpoint{4.345122in}{5.141954in}}%
\pgfpathlineto{\pgfqpoint{4.360556in}{5.147024in}}%
\pgfpathlineto{\pgfqpoint{4.375990in}{5.155210in}}%
\pgfpathlineto{\pgfqpoint{4.391423in}{5.166455in}}%
\pgfpathlineto{\pgfqpoint{4.406857in}{5.180629in}}%
\pgfpathlineto{\pgfqpoint{4.422291in}{5.197525in}}%
\pgfpathlineto{\pgfqpoint{4.437724in}{5.216873in}}%
\pgfpathlineto{\pgfqpoint{4.468592in}{5.261535in}}%
\pgfpathlineto{\pgfqpoint{4.514892in}{5.337091in}}%
\pgfpathlineto{\pgfqpoint{4.545760in}{5.387661in}}%
\pgfpathlineto{\pgfqpoint{4.576627in}{5.433980in}}%
\pgfpathlineto{\pgfqpoint{4.592061in}{5.454482in}}%
\pgfpathlineto{\pgfqpoint{4.607494in}{5.472733in}}%
\pgfpathlineto{\pgfqpoint{4.622928in}{5.488439in}}%
\pgfpathlineto{\pgfqpoint{4.638362in}{5.501373in}}%
\pgfpathlineto{\pgfqpoint{4.653795in}{5.511374in}}%
\pgfpathlineto{\pgfqpoint{4.669229in}{5.518353in}}%
\pgfpathlineto{\pgfqpoint{4.669229in}{5.518353in}}%
\pgfusepath{stroke}%
\end{pgfscope}%
\begin{pgfscope}%
\pgfpathrectangle{\pgfqpoint{0.634105in}{3.881603in}}{\pgfqpoint{4.227273in}{2.800000in}} %
\pgfusepath{clip}%
\pgfsetrectcap%
\pgfsetroundjoin%
\pgfsetlinewidth{0.501875pt}%
\definecolor{currentstroke}{rgb}{0.456863,0.997705,0.730653}%
\pgfsetstrokecolor{currentstroke}%
\pgfsetdash{}{0pt}%
\pgfpathmoveto{\pgfqpoint{0.826254in}{5.731110in}}%
\pgfpathlineto{\pgfqpoint{0.841687in}{5.736238in}}%
\pgfpathlineto{\pgfqpoint{0.857121in}{5.736230in}}%
\pgfpathlineto{\pgfqpoint{0.872555in}{5.730884in}}%
\pgfpathlineto{\pgfqpoint{0.887988in}{5.720096in}}%
\pgfpathlineto{\pgfqpoint{0.903422in}{5.703858in}}%
\pgfpathlineto{\pgfqpoint{0.918855in}{5.682268in}}%
\pgfpathlineto{\pgfqpoint{0.934289in}{5.655526in}}%
\pgfpathlineto{\pgfqpoint{0.949723in}{5.623932in}}%
\pgfpathlineto{\pgfqpoint{0.965156in}{5.587886in}}%
\pgfpathlineto{\pgfqpoint{0.980590in}{5.547878in}}%
\pgfpathlineto{\pgfqpoint{1.011457in}{5.458346in}}%
\pgfpathlineto{\pgfqpoint{1.057758in}{5.310802in}}%
\pgfpathlineto{\pgfqpoint{1.088625in}{5.212726in}}%
\pgfpathlineto{\pgfqpoint{1.119493in}{5.122367in}}%
\pgfpathlineto{\pgfqpoint{1.134926in}{5.081980in}}%
\pgfpathlineto{\pgfqpoint{1.150360in}{5.045703in}}%
\pgfpathlineto{\pgfqpoint{1.165794in}{5.014140in}}%
\pgfpathlineto{\pgfqpoint{1.181227in}{4.987814in}}%
\pgfpathlineto{\pgfqpoint{1.196661in}{4.967158in}}%
\pgfpathlineto{\pgfqpoint{1.212095in}{4.952508in}}%
\pgfpathlineto{\pgfqpoint{1.227528in}{4.944098in}}%
\pgfpathlineto{\pgfqpoint{1.242962in}{4.942053in}}%
\pgfpathlineto{\pgfqpoint{1.258395in}{4.946393in}}%
\pgfpathlineto{\pgfqpoint{1.273829in}{4.957032in}}%
\pgfpathlineto{\pgfqpoint{1.289263in}{4.973777in}}%
\pgfpathlineto{\pgfqpoint{1.304696in}{4.996341in}}%
\pgfpathlineto{\pgfqpoint{1.320130in}{5.024342in}}%
\pgfpathlineto{\pgfqpoint{1.335564in}{5.057317in}}%
\pgfpathlineto{\pgfqpoint{1.350997in}{5.094728in}}%
\pgfpathlineto{\pgfqpoint{1.366431in}{5.135976in}}%
\pgfpathlineto{\pgfqpoint{1.397298in}{5.227345in}}%
\pgfpathlineto{\pgfqpoint{1.505334in}{5.565890in}}%
\pgfpathlineto{\pgfqpoint{1.520767in}{5.607594in}}%
\pgfpathlineto{\pgfqpoint{1.536201in}{5.645899in}}%
\pgfpathlineto{\pgfqpoint{1.551634in}{5.680370in}}%
\pgfpathlineto{\pgfqpoint{1.567068in}{5.710636in}}%
\pgfpathlineto{\pgfqpoint{1.582502in}{5.736402in}}%
\pgfpathlineto{\pgfqpoint{1.597935in}{5.757443in}}%
\pgfpathlineto{\pgfqpoint{1.613369in}{5.773607in}}%
\pgfpathlineto{\pgfqpoint{1.628803in}{5.784814in}}%
\pgfpathlineto{\pgfqpoint{1.644236in}{5.791056in}}%
\pgfpathlineto{\pgfqpoint{1.659670in}{5.792388in}}%
\pgfpathlineto{\pgfqpoint{1.675104in}{5.788928in}}%
\pgfpathlineto{\pgfqpoint{1.690537in}{5.780855in}}%
\pgfpathlineto{\pgfqpoint{1.705971in}{5.768398in}}%
\pgfpathlineto{\pgfqpoint{1.721404in}{5.751832in}}%
\pgfpathlineto{\pgfqpoint{1.736838in}{5.731476in}}%
\pgfpathlineto{\pgfqpoint{1.752272in}{5.707682in}}%
\pgfpathlineto{\pgfqpoint{1.767705in}{5.680831in}}%
\pgfpathlineto{\pgfqpoint{1.798573in}{5.619592in}}%
\pgfpathlineto{\pgfqpoint{1.829440in}{5.551157in}}%
\pgfpathlineto{\pgfqpoint{1.922042in}{5.337016in}}%
\pgfpathlineto{\pgfqpoint{1.952909in}{5.273193in}}%
\pgfpathlineto{\pgfqpoint{1.983776in}{5.217479in}}%
\pgfpathlineto{\pgfqpoint{1.999210in}{5.193259in}}%
\pgfpathlineto{\pgfqpoint{2.014644in}{5.171705in}}%
\pgfpathlineto{\pgfqpoint{2.030077in}{5.152949in}}%
\pgfpathlineto{\pgfqpoint{2.045511in}{5.137077in}}%
\pgfpathlineto{\pgfqpoint{2.060944in}{5.124135in}}%
\pgfpathlineto{\pgfqpoint{2.076378in}{5.114121in}}%
\pgfpathlineto{\pgfqpoint{2.091812in}{5.106991in}}%
\pgfpathlineto{\pgfqpoint{2.107245in}{5.102656in}}%
\pgfpathlineto{\pgfqpoint{2.122679in}{5.100983in}}%
\pgfpathlineto{\pgfqpoint{2.138113in}{5.101797in}}%
\pgfpathlineto{\pgfqpoint{2.153546in}{5.104883in}}%
\pgfpathlineto{\pgfqpoint{2.168980in}{5.109991in}}%
\pgfpathlineto{\pgfqpoint{2.184414in}{5.116836in}}%
\pgfpathlineto{\pgfqpoint{2.215281in}{5.134456in}}%
\pgfpathlineto{\pgfqpoint{2.292449in}{5.184698in}}%
\pgfpathlineto{\pgfqpoint{2.307883in}{5.192858in}}%
\pgfpathlineto{\pgfqpoint{2.323316in}{5.199577in}}%
\pgfpathlineto{\pgfqpoint{2.338750in}{5.204556in}}%
\pgfpathlineto{\pgfqpoint{2.354184in}{5.207530in}}%
\pgfpathlineto{\pgfqpoint{2.369617in}{5.208276in}}%
\pgfpathlineto{\pgfqpoint{2.385051in}{5.206622in}}%
\pgfpathlineto{\pgfqpoint{2.400484in}{5.202443in}}%
\pgfpathlineto{\pgfqpoint{2.415918in}{5.195674in}}%
\pgfpathlineto{\pgfqpoint{2.431352in}{5.186305in}}%
\pgfpathlineto{\pgfqpoint{2.446785in}{5.174387in}}%
\pgfpathlineto{\pgfqpoint{2.462219in}{5.160028in}}%
\pgfpathlineto{\pgfqpoint{2.477653in}{5.143397in}}%
\pgfpathlineto{\pgfqpoint{2.493086in}{5.124715in}}%
\pgfpathlineto{\pgfqpoint{2.523954in}{5.082337in}}%
\pgfpathlineto{\pgfqpoint{2.570254in}{5.011556in}}%
\pgfpathlineto{\pgfqpoint{2.601122in}{4.964307in}}%
\pgfpathlineto{\pgfqpoint{2.631989in}{4.920980in}}%
\pgfpathlineto{\pgfqpoint{2.647423in}{4.901795in}}%
\pgfpathlineto{\pgfqpoint{2.662856in}{4.884744in}}%
\pgfpathlineto{\pgfqpoint{2.678290in}{4.870136in}}%
\pgfpathlineto{\pgfqpoint{2.693724in}{4.858231in}}%
\pgfpathlineto{\pgfqpoint{2.709157in}{4.849235in}}%
\pgfpathlineto{\pgfqpoint{2.724591in}{4.843289in}}%
\pgfpathlineto{\pgfqpoint{2.740024in}{4.840472in}}%
\pgfpathlineto{\pgfqpoint{2.755458in}{4.840794in}}%
\pgfpathlineto{\pgfqpoint{2.770892in}{4.844199in}}%
\pgfpathlineto{\pgfqpoint{2.786325in}{4.850563in}}%
\pgfpathlineto{\pgfqpoint{2.801759in}{4.859701in}}%
\pgfpathlineto{\pgfqpoint{2.817193in}{4.871365in}}%
\pgfpathlineto{\pgfqpoint{2.832626in}{4.885256in}}%
\pgfpathlineto{\pgfqpoint{2.863493in}{4.918288in}}%
\pgfpathlineto{\pgfqpoint{2.909794in}{4.974788in}}%
\pgfpathlineto{\pgfqpoint{2.940662in}{5.011955in}}%
\pgfpathlineto{\pgfqpoint{2.971529in}{5.044748in}}%
\pgfpathlineto{\pgfqpoint{2.986963in}{5.058565in}}%
\pgfpathlineto{\pgfqpoint{3.002396in}{5.070243in}}%
\pgfpathlineto{\pgfqpoint{3.017830in}{5.079531in}}%
\pgfpathlineto{\pgfqpoint{3.033263in}{5.086232in}}%
\pgfpathlineto{\pgfqpoint{3.048697in}{5.090210in}}%
\pgfpathlineto{\pgfqpoint{3.064131in}{5.091387in}}%
\pgfpathlineto{\pgfqpoint{3.079564in}{5.089751in}}%
\pgfpathlineto{\pgfqpoint{3.094998in}{5.085345in}}%
\pgfpathlineto{\pgfqpoint{3.110432in}{5.078274in}}%
\pgfpathlineto{\pgfqpoint{3.125865in}{5.068692in}}%
\pgfpathlineto{\pgfqpoint{3.141299in}{5.056805in}}%
\pgfpathlineto{\pgfqpoint{3.156733in}{5.042857in}}%
\pgfpathlineto{\pgfqpoint{3.187600in}{5.009933in}}%
\pgfpathlineto{\pgfqpoint{3.218467in}{4.972444in}}%
\pgfpathlineto{\pgfqpoint{3.280202in}{4.894495in}}%
\pgfpathlineto{\pgfqpoint{3.311069in}{4.858997in}}%
\pgfpathlineto{\pgfqpoint{3.341936in}{4.828473in}}%
\pgfpathlineto{\pgfqpoint{3.357370in}{4.815514in}}%
\pgfpathlineto{\pgfqpoint{3.372803in}{4.804252in}}%
\pgfpathlineto{\pgfqpoint{3.388237in}{4.794763in}}%
\pgfpathlineto{\pgfqpoint{3.403671in}{4.787088in}}%
\pgfpathlineto{\pgfqpoint{3.419104in}{4.781239in}}%
\pgfpathlineto{\pgfqpoint{3.434538in}{4.777201in}}%
\pgfpathlineto{\pgfqpoint{3.449972in}{4.774934in}}%
\pgfpathlineto{\pgfqpoint{3.465405in}{4.774380in}}%
\pgfpathlineto{\pgfqpoint{3.480839in}{4.775468in}}%
\pgfpathlineto{\pgfqpoint{3.496273in}{4.778115in}}%
\pgfpathlineto{\pgfqpoint{3.511706in}{4.782236in}}%
\pgfpathlineto{\pgfqpoint{3.527140in}{4.787741in}}%
\pgfpathlineto{\pgfqpoint{3.542573in}{4.794547in}}%
\pgfpathlineto{\pgfqpoint{3.573441in}{4.811751in}}%
\pgfpathlineto{\pgfqpoint{3.604308in}{4.833319in}}%
\pgfpathlineto{\pgfqpoint{3.635175in}{4.858879in}}%
\pgfpathlineto{\pgfqpoint{3.666043in}{4.888196in}}%
\pgfpathlineto{\pgfqpoint{3.696910in}{4.921113in}}%
\pgfpathlineto{\pgfqpoint{3.727777in}{4.957445in}}%
\pgfpathlineto{\pgfqpoint{3.758644in}{4.996876in}}%
\pgfpathlineto{\pgfqpoint{3.804945in}{5.060519in}}%
\pgfpathlineto{\pgfqpoint{3.882113in}{5.169404in}}%
\pgfpathlineto{\pgfqpoint{3.912981in}{5.209325in}}%
\pgfpathlineto{\pgfqpoint{3.943848in}{5.244343in}}%
\pgfpathlineto{\pgfqpoint{3.959282in}{5.259423in}}%
\pgfpathlineto{\pgfqpoint{3.974715in}{5.272595in}}%
\pgfpathlineto{\pgfqpoint{3.990149in}{5.283669in}}%
\pgfpathlineto{\pgfqpoint{4.005583in}{5.292486in}}%
\pgfpathlineto{\pgfqpoint{4.021016in}{5.298921in}}%
\pgfpathlineto{\pgfqpoint{4.036450in}{5.302890in}}%
\pgfpathlineto{\pgfqpoint{4.051883in}{5.304354in}}%
\pgfpathlineto{\pgfqpoint{4.067317in}{5.303322in}}%
\pgfpathlineto{\pgfqpoint{4.082751in}{5.299851in}}%
\pgfpathlineto{\pgfqpoint{4.098184in}{5.294051in}}%
\pgfpathlineto{\pgfqpoint{4.113618in}{5.286083in}}%
\pgfpathlineto{\pgfqpoint{4.129052in}{5.276155in}}%
\pgfpathlineto{\pgfqpoint{4.144485in}{5.264520in}}%
\pgfpathlineto{\pgfqpoint{4.175352in}{5.237350in}}%
\pgfpathlineto{\pgfqpoint{4.252521in}{5.163716in}}%
\pgfpathlineto{\pgfqpoint{4.267954in}{5.151130in}}%
\pgfpathlineto{\pgfqpoint{4.283388in}{5.140134in}}%
\pgfpathlineto{\pgfqpoint{4.298822in}{5.131053in}}%
\pgfpathlineto{\pgfqpoint{4.314255in}{5.124175in}}%
\pgfpathlineto{\pgfqpoint{4.329689in}{5.119741in}}%
\pgfpathlineto{\pgfqpoint{4.345122in}{5.117940in}}%
\pgfpathlineto{\pgfqpoint{4.360556in}{5.118905in}}%
\pgfpathlineto{\pgfqpoint{4.375990in}{5.122708in}}%
\pgfpathlineto{\pgfqpoint{4.391423in}{5.129362in}}%
\pgfpathlineto{\pgfqpoint{4.406857in}{5.138815in}}%
\pgfpathlineto{\pgfqpoint{4.422291in}{5.150955in}}%
\pgfpathlineto{\pgfqpoint{4.437724in}{5.165614in}}%
\pgfpathlineto{\pgfqpoint{4.453158in}{5.182570in}}%
\pgfpathlineto{\pgfqpoint{4.484025in}{5.222241in}}%
\pgfpathlineto{\pgfqpoint{4.514892in}{5.267345in}}%
\pgfpathlineto{\pgfqpoint{4.592061in}{5.383720in}}%
\pgfpathlineto{\pgfqpoint{4.622928in}{5.423809in}}%
\pgfpathlineto{\pgfqpoint{4.638362in}{5.441189in}}%
\pgfpathlineto{\pgfqpoint{4.653795in}{5.456480in}}%
\pgfpathlineto{\pgfqpoint{4.669229in}{5.469504in}}%
\pgfpathlineto{\pgfqpoint{4.669229in}{5.469504in}}%
\pgfusepath{stroke}%
\end{pgfscope}%
\begin{pgfscope}%
\pgfpathrectangle{\pgfqpoint{0.634105in}{3.881603in}}{\pgfqpoint{4.227273in}{2.800000in}} %
\pgfusepath{clip}%
\pgfsetrectcap%
\pgfsetroundjoin%
\pgfsetlinewidth{0.501875pt}%
\definecolor{currentstroke}{rgb}{0.543137,0.997705,0.682749}%
\pgfsetstrokecolor{currentstroke}%
\pgfsetdash{}{0pt}%
\pgfpathmoveto{\pgfqpoint{0.826254in}{5.691709in}}%
\pgfpathlineto{\pgfqpoint{0.841687in}{5.697124in}}%
\pgfpathlineto{\pgfqpoint{0.857121in}{5.698065in}}%
\pgfpathlineto{\pgfqpoint{0.872555in}{5.694261in}}%
\pgfpathlineto{\pgfqpoint{0.887988in}{5.685523in}}%
\pgfpathlineto{\pgfqpoint{0.903422in}{5.671755in}}%
\pgfpathlineto{\pgfqpoint{0.918855in}{5.652953in}}%
\pgfpathlineto{\pgfqpoint{0.934289in}{5.629215in}}%
\pgfpathlineto{\pgfqpoint{0.949723in}{5.600740in}}%
\pgfpathlineto{\pgfqpoint{0.965156in}{5.567826in}}%
\pgfpathlineto{\pgfqpoint{0.980590in}{5.530868in}}%
\pgfpathlineto{\pgfqpoint{1.011457in}{5.446857in}}%
\pgfpathlineto{\pgfqpoint{1.042325in}{5.353545in}}%
\pgfpathlineto{\pgfqpoint{1.104059in}{5.162812in}}%
\pgfpathlineto{\pgfqpoint{1.134926in}{5.078118in}}%
\pgfpathlineto{\pgfqpoint{1.150360in}{5.041108in}}%
\pgfpathlineto{\pgfqpoint{1.165794in}{5.008550in}}%
\pgfpathlineto{\pgfqpoint{1.181227in}{4.981024in}}%
\pgfpathlineto{\pgfqpoint{1.196661in}{4.959023in}}%
\pgfpathlineto{\pgfqpoint{1.212095in}{4.942944in}}%
\pgfpathlineto{\pgfqpoint{1.227528in}{4.933079in}}%
\pgfpathlineto{\pgfqpoint{1.242962in}{4.929609in}}%
\pgfpathlineto{\pgfqpoint{1.258395in}{4.932603in}}%
\pgfpathlineto{\pgfqpoint{1.273829in}{4.942015in}}%
\pgfpathlineto{\pgfqpoint{1.289263in}{4.957687in}}%
\pgfpathlineto{\pgfqpoint{1.304696in}{4.979352in}}%
\pgfpathlineto{\pgfqpoint{1.320130in}{5.006640in}}%
\pgfpathlineto{\pgfqpoint{1.335564in}{5.039089in}}%
\pgfpathlineto{\pgfqpoint{1.350997in}{5.076152in}}%
\pgfpathlineto{\pgfqpoint{1.366431in}{5.117211in}}%
\pgfpathlineto{\pgfqpoint{1.397298in}{5.208562in}}%
\pgfpathlineto{\pgfqpoint{1.505334in}{5.546617in}}%
\pgfpathlineto{\pgfqpoint{1.520767in}{5.587696in}}%
\pgfpathlineto{\pgfqpoint{1.536201in}{5.625170in}}%
\pgfpathlineto{\pgfqpoint{1.551634in}{5.658605in}}%
\pgfpathlineto{\pgfqpoint{1.567068in}{5.687642in}}%
\pgfpathlineto{\pgfqpoint{1.582502in}{5.712008in}}%
\pgfpathlineto{\pgfqpoint{1.597935in}{5.731514in}}%
\pgfpathlineto{\pgfqpoint{1.613369in}{5.746051in}}%
\pgfpathlineto{\pgfqpoint{1.628803in}{5.755595in}}%
\pgfpathlineto{\pgfqpoint{1.644236in}{5.760196in}}%
\pgfpathlineto{\pgfqpoint{1.659670in}{5.759977in}}%
\pgfpathlineto{\pgfqpoint{1.675104in}{5.755128in}}%
\pgfpathlineto{\pgfqpoint{1.690537in}{5.745899in}}%
\pgfpathlineto{\pgfqpoint{1.705971in}{5.732589in}}%
\pgfpathlineto{\pgfqpoint{1.721404in}{5.715545in}}%
\pgfpathlineto{\pgfqpoint{1.736838in}{5.695145in}}%
\pgfpathlineto{\pgfqpoint{1.752272in}{5.671797in}}%
\pgfpathlineto{\pgfqpoint{1.783139in}{5.617973in}}%
\pgfpathlineto{\pgfqpoint{1.814006in}{5.557573in}}%
\pgfpathlineto{\pgfqpoint{1.906608in}{5.370110in}}%
\pgfpathlineto{\pgfqpoint{1.937475in}{5.314896in}}%
\pgfpathlineto{\pgfqpoint{1.968343in}{5.266794in}}%
\pgfpathlineto{\pgfqpoint{1.983776in}{5.245826in}}%
\pgfpathlineto{\pgfqpoint{1.999210in}{5.227066in}}%
\pgfpathlineto{\pgfqpoint{2.014644in}{5.210581in}}%
\pgfpathlineto{\pgfqpoint{2.030077in}{5.196407in}}%
\pgfpathlineto{\pgfqpoint{2.045511in}{5.184547in}}%
\pgfpathlineto{\pgfqpoint{2.060944in}{5.174967in}}%
\pgfpathlineto{\pgfqpoint{2.076378in}{5.167607in}}%
\pgfpathlineto{\pgfqpoint{2.091812in}{5.162372in}}%
\pgfpathlineto{\pgfqpoint{2.107245in}{5.159141in}}%
\pgfpathlineto{\pgfqpoint{2.122679in}{5.157765in}}%
\pgfpathlineto{\pgfqpoint{2.138113in}{5.158071in}}%
\pgfpathlineto{\pgfqpoint{2.153546in}{5.159860in}}%
\pgfpathlineto{\pgfqpoint{2.184414in}{5.167005in}}%
\pgfpathlineto{\pgfqpoint{2.215281in}{5.177267in}}%
\pgfpathlineto{\pgfqpoint{2.277015in}{5.198663in}}%
\pgfpathlineto{\pgfqpoint{2.307883in}{5.205592in}}%
\pgfpathlineto{\pgfqpoint{2.323316in}{5.207280in}}%
\pgfpathlineto{\pgfqpoint{2.338750in}{5.207524in}}%
\pgfpathlineto{\pgfqpoint{2.354184in}{5.206163in}}%
\pgfpathlineto{\pgfqpoint{2.369617in}{5.203072in}}%
\pgfpathlineto{\pgfqpoint{2.385051in}{5.198164in}}%
\pgfpathlineto{\pgfqpoint{2.400484in}{5.191391in}}%
\pgfpathlineto{\pgfqpoint{2.415918in}{5.182752in}}%
\pgfpathlineto{\pgfqpoint{2.431352in}{5.172285in}}%
\pgfpathlineto{\pgfqpoint{2.446785in}{5.160078in}}%
\pgfpathlineto{\pgfqpoint{2.477653in}{5.130998in}}%
\pgfpathlineto{\pgfqpoint{2.508520in}{5.097031in}}%
\pgfpathlineto{\pgfqpoint{2.601122in}{4.988289in}}%
\pgfpathlineto{\pgfqpoint{2.631989in}{4.958280in}}%
\pgfpathlineto{\pgfqpoint{2.647423in}{4.945799in}}%
\pgfpathlineto{\pgfqpoint{2.662856in}{4.935315in}}%
\pgfpathlineto{\pgfqpoint{2.678290in}{4.927017in}}%
\pgfpathlineto{\pgfqpoint{2.693724in}{4.921041in}}%
\pgfpathlineto{\pgfqpoint{2.709157in}{4.917477in}}%
\pgfpathlineto{\pgfqpoint{2.724591in}{4.916356in}}%
\pgfpathlineto{\pgfqpoint{2.740024in}{4.917657in}}%
\pgfpathlineto{\pgfqpoint{2.755458in}{4.921300in}}%
\pgfpathlineto{\pgfqpoint{2.770892in}{4.927153in}}%
\pgfpathlineto{\pgfqpoint{2.786325in}{4.935030in}}%
\pgfpathlineto{\pgfqpoint{2.801759in}{4.944701in}}%
\pgfpathlineto{\pgfqpoint{2.832626in}{4.968291in}}%
\pgfpathlineto{\pgfqpoint{2.878927in}{5.009307in}}%
\pgfpathlineto{\pgfqpoint{2.909794in}{5.036209in}}%
\pgfpathlineto{\pgfqpoint{2.940662in}{5.059498in}}%
\pgfpathlineto{\pgfqpoint{2.956095in}{5.069026in}}%
\pgfpathlineto{\pgfqpoint{2.971529in}{5.076811in}}%
\pgfpathlineto{\pgfqpoint{2.986963in}{5.082657in}}%
\pgfpathlineto{\pgfqpoint{3.002396in}{5.086423in}}%
\pgfpathlineto{\pgfqpoint{3.017830in}{5.088018in}}%
\pgfpathlineto{\pgfqpoint{3.033263in}{5.087407in}}%
\pgfpathlineto{\pgfqpoint{3.048697in}{5.084611in}}%
\pgfpathlineto{\pgfqpoint{3.064131in}{5.079701in}}%
\pgfpathlineto{\pgfqpoint{3.079564in}{5.072801in}}%
\pgfpathlineto{\pgfqpoint{3.094998in}{5.064077in}}%
\pgfpathlineto{\pgfqpoint{3.110432in}{5.053738in}}%
\pgfpathlineto{\pgfqpoint{3.141299in}{5.029199in}}%
\pgfpathlineto{\pgfqpoint{3.187600in}{4.986935in}}%
\pgfpathlineto{\pgfqpoint{3.233901in}{4.944957in}}%
\pgfpathlineto{\pgfqpoint{3.264768in}{4.920438in}}%
\pgfpathlineto{\pgfqpoint{3.295635in}{4.900314in}}%
\pgfpathlineto{\pgfqpoint{3.311069in}{4.892146in}}%
\pgfpathlineto{\pgfqpoint{3.326503in}{4.885289in}}%
\pgfpathlineto{\pgfqpoint{3.341936in}{4.879734in}}%
\pgfpathlineto{\pgfqpoint{3.357370in}{4.875434in}}%
\pgfpathlineto{\pgfqpoint{3.388237in}{4.870266in}}%
\pgfpathlineto{\pgfqpoint{3.419104in}{4.868900in}}%
\pgfpathlineto{\pgfqpoint{3.449972in}{4.870243in}}%
\pgfpathlineto{\pgfqpoint{3.496273in}{4.875028in}}%
\pgfpathlineto{\pgfqpoint{3.573441in}{4.885322in}}%
\pgfpathlineto{\pgfqpoint{3.619742in}{4.893641in}}%
\pgfpathlineto{\pgfqpoint{3.650609in}{4.901907in}}%
\pgfpathlineto{\pgfqpoint{3.681476in}{4.913872in}}%
\pgfpathlineto{\pgfqpoint{3.696910in}{4.921659in}}%
\pgfpathlineto{\pgfqpoint{3.712343in}{4.930848in}}%
\pgfpathlineto{\pgfqpoint{3.727777in}{4.941563in}}%
\pgfpathlineto{\pgfqpoint{3.743211in}{4.953896in}}%
\pgfpathlineto{\pgfqpoint{3.758644in}{4.967904in}}%
\pgfpathlineto{\pgfqpoint{3.774078in}{4.983606in}}%
\pgfpathlineto{\pgfqpoint{3.804945in}{5.019931in}}%
\pgfpathlineto{\pgfqpoint{3.835813in}{5.062086in}}%
\pgfpathlineto{\pgfqpoint{3.866680in}{5.108548in}}%
\pgfpathlineto{\pgfqpoint{3.943848in}{5.228142in}}%
\pgfpathlineto{\pgfqpoint{3.974715in}{5.270035in}}%
\pgfpathlineto{\pgfqpoint{3.990149in}{5.288318in}}%
\pgfpathlineto{\pgfqpoint{4.005583in}{5.304424in}}%
\pgfpathlineto{\pgfqpoint{4.021016in}{5.318103in}}%
\pgfpathlineto{\pgfqpoint{4.036450in}{5.329148in}}%
\pgfpathlineto{\pgfqpoint{4.051883in}{5.337411in}}%
\pgfpathlineto{\pgfqpoint{4.067317in}{5.342798in}}%
\pgfpathlineto{\pgfqpoint{4.082751in}{5.345280in}}%
\pgfpathlineto{\pgfqpoint{4.098184in}{5.344892in}}%
\pgfpathlineto{\pgfqpoint{4.113618in}{5.341732in}}%
\pgfpathlineto{\pgfqpoint{4.129052in}{5.335960in}}%
\pgfpathlineto{\pgfqpoint{4.144485in}{5.327795in}}%
\pgfpathlineto{\pgfqpoint{4.159919in}{5.317510in}}%
\pgfpathlineto{\pgfqpoint{4.175352in}{5.305429in}}%
\pgfpathlineto{\pgfqpoint{4.206220in}{5.277351in}}%
\pgfpathlineto{\pgfqpoint{4.267954in}{5.217123in}}%
\pgfpathlineto{\pgfqpoint{4.298822in}{5.191668in}}%
\pgfpathlineto{\pgfqpoint{4.314255in}{5.181429in}}%
\pgfpathlineto{\pgfqpoint{4.329689in}{5.173258in}}%
\pgfpathlineto{\pgfqpoint{4.345122in}{5.167398in}}%
\pgfpathlineto{\pgfqpoint{4.360556in}{5.164039in}}%
\pgfpathlineto{\pgfqpoint{4.375990in}{5.163311in}}%
\pgfpathlineto{\pgfqpoint{4.391423in}{5.165280in}}%
\pgfpathlineto{\pgfqpoint{4.406857in}{5.169952in}}%
\pgfpathlineto{\pgfqpoint{4.422291in}{5.177272in}}%
\pgfpathlineto{\pgfqpoint{4.437724in}{5.187125in}}%
\pgfpathlineto{\pgfqpoint{4.453158in}{5.199342in}}%
\pgfpathlineto{\pgfqpoint{4.468592in}{5.213703in}}%
\pgfpathlineto{\pgfqpoint{4.499459in}{5.247770in}}%
\pgfpathlineto{\pgfqpoint{4.530326in}{5.286831in}}%
\pgfpathlineto{\pgfqpoint{4.607494in}{5.387714in}}%
\pgfpathlineto{\pgfqpoint{4.638362in}{5.422224in}}%
\pgfpathlineto{\pgfqpoint{4.653795in}{5.437091in}}%
\pgfpathlineto{\pgfqpoint{4.669229in}{5.450090in}}%
\pgfpathlineto{\pgfqpoint{4.669229in}{5.450090in}}%
\pgfusepath{stroke}%
\end{pgfscope}%
\begin{pgfscope}%
\pgfpathrectangle{\pgfqpoint{0.634105in}{3.881603in}}{\pgfqpoint{4.227273in}{2.800000in}} %
\pgfusepath{clip}%
\pgfsetrectcap%
\pgfsetroundjoin%
\pgfsetlinewidth{0.501875pt}%
\definecolor{currentstroke}{rgb}{0.621569,0.981823,0.636474}%
\pgfsetstrokecolor{currentstroke}%
\pgfsetdash{}{0pt}%
\pgfpathmoveto{\pgfqpoint{0.826254in}{5.691586in}}%
\pgfpathlineto{\pgfqpoint{0.841687in}{5.692614in}}%
\pgfpathlineto{\pgfqpoint{0.857121in}{5.689029in}}%
\pgfpathlineto{\pgfqpoint{0.872555in}{5.680653in}}%
\pgfpathlineto{\pgfqpoint{0.887988in}{5.667388in}}%
\pgfpathlineto{\pgfqpoint{0.903422in}{5.649228in}}%
\pgfpathlineto{\pgfqpoint{0.918855in}{5.626255in}}%
\pgfpathlineto{\pgfqpoint{0.934289in}{5.598646in}}%
\pgfpathlineto{\pgfqpoint{0.949723in}{5.566668in}}%
\pgfpathlineto{\pgfqpoint{0.965156in}{5.530679in}}%
\pgfpathlineto{\pgfqpoint{0.996024in}{5.448514in}}%
\pgfpathlineto{\pgfqpoint{1.026891in}{5.356566in}}%
\pgfpathlineto{\pgfqpoint{1.104059in}{5.120179in}}%
\pgfpathlineto{\pgfqpoint{1.119493in}{5.077712in}}%
\pgfpathlineto{\pgfqpoint{1.134926in}{5.038518in}}%
\pgfpathlineto{\pgfqpoint{1.150360in}{5.003234in}}%
\pgfpathlineto{\pgfqpoint{1.165794in}{4.972433in}}%
\pgfpathlineto{\pgfqpoint{1.181227in}{4.946616in}}%
\pgfpathlineto{\pgfqpoint{1.196661in}{4.926200in}}%
\pgfpathlineto{\pgfqpoint{1.212095in}{4.911513in}}%
\pgfpathlineto{\pgfqpoint{1.227528in}{4.902784in}}%
\pgfpathlineto{\pgfqpoint{1.242962in}{4.900143in}}%
\pgfpathlineto{\pgfqpoint{1.258395in}{4.903621in}}%
\pgfpathlineto{\pgfqpoint{1.273829in}{4.913144in}}%
\pgfpathlineto{\pgfqpoint{1.289263in}{4.928540in}}%
\pgfpathlineto{\pgfqpoint{1.304696in}{4.949545in}}%
\pgfpathlineto{\pgfqpoint{1.320130in}{4.975806in}}%
\pgfpathlineto{\pgfqpoint{1.335564in}{5.006887in}}%
\pgfpathlineto{\pgfqpoint{1.350997in}{5.042286in}}%
\pgfpathlineto{\pgfqpoint{1.366431in}{5.081438in}}%
\pgfpathlineto{\pgfqpoint{1.397298in}{5.168511in}}%
\pgfpathlineto{\pgfqpoint{1.443599in}{5.310999in}}%
\pgfpathlineto{\pgfqpoint{1.489900in}{5.451547in}}%
\pgfpathlineto{\pgfqpoint{1.520767in}{5.535918in}}%
\pgfpathlineto{\pgfqpoint{1.536201in}{5.573743in}}%
\pgfpathlineto{\pgfqpoint{1.551634in}{5.608100in}}%
\pgfpathlineto{\pgfqpoint{1.567068in}{5.638651in}}%
\pgfpathlineto{\pgfqpoint{1.582502in}{5.665126in}}%
\pgfpathlineto{\pgfqpoint{1.597935in}{5.687322in}}%
\pgfpathlineto{\pgfqpoint{1.613369in}{5.705106in}}%
\pgfpathlineto{\pgfqpoint{1.628803in}{5.718414in}}%
\pgfpathlineto{\pgfqpoint{1.644236in}{5.727242in}}%
\pgfpathlineto{\pgfqpoint{1.659670in}{5.731651in}}%
\pgfpathlineto{\pgfqpoint{1.675104in}{5.731753in}}%
\pgfpathlineto{\pgfqpoint{1.690537in}{5.727715in}}%
\pgfpathlineto{\pgfqpoint{1.705971in}{5.719748in}}%
\pgfpathlineto{\pgfqpoint{1.721404in}{5.708102in}}%
\pgfpathlineto{\pgfqpoint{1.736838in}{5.693060in}}%
\pgfpathlineto{\pgfqpoint{1.752272in}{5.674934in}}%
\pgfpathlineto{\pgfqpoint{1.767705in}{5.654055in}}%
\pgfpathlineto{\pgfqpoint{1.783139in}{5.630772in}}%
\pgfpathlineto{\pgfqpoint{1.814006in}{5.578428in}}%
\pgfpathlineto{\pgfqpoint{1.844874in}{5.520799in}}%
\pgfpathlineto{\pgfqpoint{1.937475in}{5.343600in}}%
\pgfpathlineto{\pgfqpoint{1.968343in}{5.291019in}}%
\pgfpathlineto{\pgfqpoint{1.999210in}{5.244790in}}%
\pgfpathlineto{\pgfqpoint{2.014644in}{5.224463in}}%
\pgfpathlineto{\pgfqpoint{2.030077in}{5.206157in}}%
\pgfpathlineto{\pgfqpoint{2.045511in}{5.189954in}}%
\pgfpathlineto{\pgfqpoint{2.060944in}{5.175906in}}%
\pgfpathlineto{\pgfqpoint{2.076378in}{5.164036in}}%
\pgfpathlineto{\pgfqpoint{2.091812in}{5.154332in}}%
\pgfpathlineto{\pgfqpoint{2.107245in}{5.146754in}}%
\pgfpathlineto{\pgfqpoint{2.122679in}{5.141230in}}%
\pgfpathlineto{\pgfqpoint{2.138113in}{5.137655in}}%
\pgfpathlineto{\pgfqpoint{2.153546in}{5.135898in}}%
\pgfpathlineto{\pgfqpoint{2.168980in}{5.135796in}}%
\pgfpathlineto{\pgfqpoint{2.184414in}{5.137165in}}%
\pgfpathlineto{\pgfqpoint{2.215281in}{5.143450in}}%
\pgfpathlineto{\pgfqpoint{2.246148in}{5.152852in}}%
\pgfpathlineto{\pgfqpoint{2.307883in}{5.172561in}}%
\pgfpathlineto{\pgfqpoint{2.338750in}{5.178706in}}%
\pgfpathlineto{\pgfqpoint{2.354184in}{5.180047in}}%
\pgfpathlineto{\pgfqpoint{2.369617in}{5.179990in}}%
\pgfpathlineto{\pgfqpoint{2.385051in}{5.178397in}}%
\pgfpathlineto{\pgfqpoint{2.400484in}{5.175164in}}%
\pgfpathlineto{\pgfqpoint{2.415918in}{5.170232in}}%
\pgfpathlineto{\pgfqpoint{2.431352in}{5.163580in}}%
\pgfpathlineto{\pgfqpoint{2.446785in}{5.155238in}}%
\pgfpathlineto{\pgfqpoint{2.462219in}{5.145277in}}%
\pgfpathlineto{\pgfqpoint{2.493086in}{5.121007in}}%
\pgfpathlineto{\pgfqpoint{2.523954in}{5.092190in}}%
\pgfpathlineto{\pgfqpoint{2.616555in}{4.999894in}}%
\pgfpathlineto{\pgfqpoint{2.631989in}{4.986866in}}%
\pgfpathlineto{\pgfqpoint{2.647423in}{4.975298in}}%
\pgfpathlineto{\pgfqpoint{2.662856in}{4.965434in}}%
\pgfpathlineto{\pgfqpoint{2.678290in}{4.957485in}}%
\pgfpathlineto{\pgfqpoint{2.693724in}{4.951619in}}%
\pgfpathlineto{\pgfqpoint{2.709157in}{4.947958in}}%
\pgfpathlineto{\pgfqpoint{2.724591in}{4.946572in}}%
\pgfpathlineto{\pgfqpoint{2.740024in}{4.947481in}}%
\pgfpathlineto{\pgfqpoint{2.755458in}{4.950646in}}%
\pgfpathlineto{\pgfqpoint{2.770892in}{4.955980in}}%
\pgfpathlineto{\pgfqpoint{2.786325in}{4.963338in}}%
\pgfpathlineto{\pgfqpoint{2.801759in}{4.972529in}}%
\pgfpathlineto{\pgfqpoint{2.832626in}{4.995426in}}%
\pgfpathlineto{\pgfqpoint{2.863493in}{5.022351in}}%
\pgfpathlineto{\pgfqpoint{2.909794in}{5.064305in}}%
\pgfpathlineto{\pgfqpoint{2.940662in}{5.089206in}}%
\pgfpathlineto{\pgfqpoint{2.956095in}{5.099755in}}%
\pgfpathlineto{\pgfqpoint{2.971529in}{5.108665in}}%
\pgfpathlineto{\pgfqpoint{2.986963in}{5.115704in}}%
\pgfpathlineto{\pgfqpoint{3.002396in}{5.120686in}}%
\pgfpathlineto{\pgfqpoint{3.017830in}{5.123475in}}%
\pgfpathlineto{\pgfqpoint{3.033263in}{5.123984in}}%
\pgfpathlineto{\pgfqpoint{3.048697in}{5.122183in}}%
\pgfpathlineto{\pgfqpoint{3.064131in}{5.118090in}}%
\pgfpathlineto{\pgfqpoint{3.079564in}{5.111780in}}%
\pgfpathlineto{\pgfqpoint{3.094998in}{5.103371in}}%
\pgfpathlineto{\pgfqpoint{3.110432in}{5.093030in}}%
\pgfpathlineto{\pgfqpoint{3.125865in}{5.080959in}}%
\pgfpathlineto{\pgfqpoint{3.156733in}{5.052606in}}%
\pgfpathlineto{\pgfqpoint{3.187600in}{5.020478in}}%
\pgfpathlineto{\pgfqpoint{3.249334in}{4.954209in}}%
\pgfpathlineto{\pgfqpoint{3.280202in}{4.924353in}}%
\pgfpathlineto{\pgfqpoint{3.311069in}{4.898910in}}%
\pgfpathlineto{\pgfqpoint{3.326503in}{4.888183in}}%
\pgfpathlineto{\pgfqpoint{3.341936in}{4.878895in}}%
\pgfpathlineto{\pgfqpoint{3.357370in}{4.871084in}}%
\pgfpathlineto{\pgfqpoint{3.372803in}{4.864756in}}%
\pgfpathlineto{\pgfqpoint{3.388237in}{4.859885in}}%
\pgfpathlineto{\pgfqpoint{3.403671in}{4.856421in}}%
\pgfpathlineto{\pgfqpoint{3.419104in}{4.854292in}}%
\pgfpathlineto{\pgfqpoint{3.434538in}{4.853410in}}%
\pgfpathlineto{\pgfqpoint{3.465405in}{4.854984in}}%
\pgfpathlineto{\pgfqpoint{3.496273in}{4.860322in}}%
\pgfpathlineto{\pgfqpoint{3.527140in}{4.868680in}}%
\pgfpathlineto{\pgfqpoint{3.558007in}{4.879528in}}%
\pgfpathlineto{\pgfqpoint{3.588874in}{4.892629in}}%
\pgfpathlineto{\pgfqpoint{3.619742in}{4.908061in}}%
\pgfpathlineto{\pgfqpoint{3.650609in}{4.926172in}}%
\pgfpathlineto{\pgfqpoint{3.681476in}{4.947475in}}%
\pgfpathlineto{\pgfqpoint{3.712343in}{4.972509in}}%
\pgfpathlineto{\pgfqpoint{3.743211in}{5.001672in}}%
\pgfpathlineto{\pgfqpoint{3.774078in}{5.035059in}}%
\pgfpathlineto{\pgfqpoint{3.804945in}{5.072340in}}%
\pgfpathlineto{\pgfqpoint{3.851246in}{5.133591in}}%
\pgfpathlineto{\pgfqpoint{3.912981in}{5.216965in}}%
\pgfpathlineto{\pgfqpoint{3.943848in}{5.254763in}}%
\pgfpathlineto{\pgfqpoint{3.959282in}{5.271716in}}%
\pgfpathlineto{\pgfqpoint{3.974715in}{5.287019in}}%
\pgfpathlineto{\pgfqpoint{3.990149in}{5.300429in}}%
\pgfpathlineto{\pgfqpoint{4.005583in}{5.311738in}}%
\pgfpathlineto{\pgfqpoint{4.021016in}{5.320767in}}%
\pgfpathlineto{\pgfqpoint{4.036450in}{5.327384in}}%
\pgfpathlineto{\pgfqpoint{4.051883in}{5.331499in}}%
\pgfpathlineto{\pgfqpoint{4.067317in}{5.333075in}}%
\pgfpathlineto{\pgfqpoint{4.082751in}{5.332126in}}%
\pgfpathlineto{\pgfqpoint{4.098184in}{5.328722in}}%
\pgfpathlineto{\pgfqpoint{4.113618in}{5.322986in}}%
\pgfpathlineto{\pgfqpoint{4.129052in}{5.315094in}}%
\pgfpathlineto{\pgfqpoint{4.144485in}{5.305270in}}%
\pgfpathlineto{\pgfqpoint{4.175352in}{5.280953in}}%
\pgfpathlineto{\pgfqpoint{4.206220in}{5.252633in}}%
\pgfpathlineto{\pgfqpoint{4.252521in}{5.209249in}}%
\pgfpathlineto{\pgfqpoint{4.283388in}{5.184234in}}%
\pgfpathlineto{\pgfqpoint{4.298822in}{5.173975in}}%
\pgfpathlineto{\pgfqpoint{4.314255in}{5.165623in}}%
\pgfpathlineto{\pgfqpoint{4.329689in}{5.159434in}}%
\pgfpathlineto{\pgfqpoint{4.345122in}{5.155611in}}%
\pgfpathlineto{\pgfqpoint{4.360556in}{5.154308in}}%
\pgfpathlineto{\pgfqpoint{4.375990in}{5.155618in}}%
\pgfpathlineto{\pgfqpoint{4.391423in}{5.159575in}}%
\pgfpathlineto{\pgfqpoint{4.406857in}{5.166153in}}%
\pgfpathlineto{\pgfqpoint{4.422291in}{5.175270in}}%
\pgfpathlineto{\pgfqpoint{4.437724in}{5.186782in}}%
\pgfpathlineto{\pgfqpoint{4.453158in}{5.200498in}}%
\pgfpathlineto{\pgfqpoint{4.468592in}{5.216174in}}%
\pgfpathlineto{\pgfqpoint{4.499459in}{5.252239in}}%
\pgfpathlineto{\pgfqpoint{4.545760in}{5.312988in}}%
\pgfpathlineto{\pgfqpoint{4.592061in}{5.372863in}}%
\pgfpathlineto{\pgfqpoint{4.622928in}{5.407596in}}%
\pgfpathlineto{\pgfqpoint{4.638362in}{5.422508in}}%
\pgfpathlineto{\pgfqpoint{4.653795in}{5.435468in}}%
\pgfpathlineto{\pgfqpoint{4.669229in}{5.446295in}}%
\pgfpathlineto{\pgfqpoint{4.669229in}{5.446295in}}%
\pgfusepath{stroke}%
\end{pgfscope}%
\begin{pgfscope}%
\pgfpathrectangle{\pgfqpoint{0.634105in}{3.881603in}}{\pgfqpoint{4.227273in}{2.800000in}} %
\pgfusepath{clip}%
\pgfsetrectcap%
\pgfsetroundjoin%
\pgfsetlinewidth{0.501875pt}%
\definecolor{currentstroke}{rgb}{0.700000,0.951057,0.587785}%
\pgfsetstrokecolor{currentstroke}%
\pgfsetdash{}{0pt}%
\pgfpathmoveto{\pgfqpoint{0.826254in}{5.674094in}}%
\pgfpathlineto{\pgfqpoint{0.841687in}{5.675239in}}%
\pgfpathlineto{\pgfqpoint{0.857121in}{5.672028in}}%
\pgfpathlineto{\pgfqpoint{0.872555in}{5.664288in}}%
\pgfpathlineto{\pgfqpoint{0.887988in}{5.651921in}}%
\pgfpathlineto{\pgfqpoint{0.903422in}{5.634912in}}%
\pgfpathlineto{\pgfqpoint{0.918855in}{5.613334in}}%
\pgfpathlineto{\pgfqpoint{0.934289in}{5.587345in}}%
\pgfpathlineto{\pgfqpoint{0.949723in}{5.557192in}}%
\pgfpathlineto{\pgfqpoint{0.965156in}{5.523210in}}%
\pgfpathlineto{\pgfqpoint{0.980590in}{5.485812in}}%
\pgfpathlineto{\pgfqpoint{1.011457in}{5.402799in}}%
\pgfpathlineto{\pgfqpoint{1.057758in}{5.266908in}}%
\pgfpathlineto{\pgfqpoint{1.088625in}{5.176769in}}%
\pgfpathlineto{\pgfqpoint{1.119493in}{5.093684in}}%
\pgfpathlineto{\pgfqpoint{1.134926in}{5.056499in}}%
\pgfpathlineto{\pgfqpoint{1.150360in}{5.023053in}}%
\pgfpathlineto{\pgfqpoint{1.165794in}{4.993895in}}%
\pgfpathlineto{\pgfqpoint{1.181227in}{4.969505in}}%
\pgfpathlineto{\pgfqpoint{1.196661in}{4.950281in}}%
\pgfpathlineto{\pgfqpoint{1.212095in}{4.936533in}}%
\pgfpathlineto{\pgfqpoint{1.227528in}{4.928476in}}%
\pgfpathlineto{\pgfqpoint{1.242962in}{4.926227in}}%
\pgfpathlineto{\pgfqpoint{1.258395in}{4.929803in}}%
\pgfpathlineto{\pgfqpoint{1.273829in}{4.939125in}}%
\pgfpathlineto{\pgfqpoint{1.289263in}{4.954014in}}%
\pgfpathlineto{\pgfqpoint{1.304696in}{4.974203in}}%
\pgfpathlineto{\pgfqpoint{1.320130in}{4.999339in}}%
\pgfpathlineto{\pgfqpoint{1.335564in}{5.028993in}}%
\pgfpathlineto{\pgfqpoint{1.350997in}{5.062670in}}%
\pgfpathlineto{\pgfqpoint{1.381864in}{5.139836in}}%
\pgfpathlineto{\pgfqpoint{1.412732in}{5.225963in}}%
\pgfpathlineto{\pgfqpoint{1.489900in}{5.446650in}}%
\pgfpathlineto{\pgfqpoint{1.520767in}{5.524527in}}%
\pgfpathlineto{\pgfqpoint{1.536201in}{5.559248in}}%
\pgfpathlineto{\pgfqpoint{1.551634in}{5.590659in}}%
\pgfpathlineto{\pgfqpoint{1.567068in}{5.618465in}}%
\pgfpathlineto{\pgfqpoint{1.582502in}{5.642438in}}%
\pgfpathlineto{\pgfqpoint{1.597935in}{5.662413in}}%
\pgfpathlineto{\pgfqpoint{1.613369in}{5.678288in}}%
\pgfpathlineto{\pgfqpoint{1.628803in}{5.690024in}}%
\pgfpathlineto{\pgfqpoint{1.644236in}{5.697638in}}%
\pgfpathlineto{\pgfqpoint{1.659670in}{5.701201in}}%
\pgfpathlineto{\pgfqpoint{1.675104in}{5.700830in}}%
\pgfpathlineto{\pgfqpoint{1.690537in}{5.696690in}}%
\pgfpathlineto{\pgfqpoint{1.705971in}{5.688978in}}%
\pgfpathlineto{\pgfqpoint{1.721404in}{5.677927in}}%
\pgfpathlineto{\pgfqpoint{1.736838in}{5.663796in}}%
\pgfpathlineto{\pgfqpoint{1.752272in}{5.646863in}}%
\pgfpathlineto{\pgfqpoint{1.767705in}{5.627424in}}%
\pgfpathlineto{\pgfqpoint{1.798573in}{5.582264in}}%
\pgfpathlineto{\pgfqpoint{1.829440in}{5.530834in}}%
\pgfpathlineto{\pgfqpoint{1.875741in}{5.447431in}}%
\pgfpathlineto{\pgfqpoint{1.922042in}{5.363652in}}%
\pgfpathlineto{\pgfqpoint{1.952909in}{5.311178in}}%
\pgfpathlineto{\pgfqpoint{1.983776in}{5.263564in}}%
\pgfpathlineto{\pgfqpoint{2.014644in}{5.222321in}}%
\pgfpathlineto{\pgfqpoint{2.030077in}{5.204465in}}%
\pgfpathlineto{\pgfqpoint{2.045511in}{5.188601in}}%
\pgfpathlineto{\pgfqpoint{2.060944in}{5.174811in}}%
\pgfpathlineto{\pgfqpoint{2.076378in}{5.163142in}}%
\pgfpathlineto{\pgfqpoint{2.091812in}{5.153616in}}%
\pgfpathlineto{\pgfqpoint{2.107245in}{5.146220in}}%
\pgfpathlineto{\pgfqpoint{2.122679in}{5.140910in}}%
\pgfpathlineto{\pgfqpoint{2.138113in}{5.137610in}}%
\pgfpathlineto{\pgfqpoint{2.153546in}{5.136210in}}%
\pgfpathlineto{\pgfqpoint{2.168980in}{5.136570in}}%
\pgfpathlineto{\pgfqpoint{2.184414in}{5.138521in}}%
\pgfpathlineto{\pgfqpoint{2.199847in}{5.141864in}}%
\pgfpathlineto{\pgfqpoint{2.230714in}{5.151817in}}%
\pgfpathlineto{\pgfqpoint{2.277015in}{5.171025in}}%
\pgfpathlineto{\pgfqpoint{2.307883in}{5.183450in}}%
\pgfpathlineto{\pgfqpoint{2.338750in}{5.193028in}}%
\pgfpathlineto{\pgfqpoint{2.354184in}{5.196140in}}%
\pgfpathlineto{\pgfqpoint{2.369617in}{5.197858in}}%
\pgfpathlineto{\pgfqpoint{2.385051in}{5.198015in}}%
\pgfpathlineto{\pgfqpoint{2.400484in}{5.196482in}}%
\pgfpathlineto{\pgfqpoint{2.415918in}{5.193170in}}%
\pgfpathlineto{\pgfqpoint{2.431352in}{5.188034in}}%
\pgfpathlineto{\pgfqpoint{2.446785in}{5.181075in}}%
\pgfpathlineto{\pgfqpoint{2.462219in}{5.172339in}}%
\pgfpathlineto{\pgfqpoint{2.477653in}{5.161919in}}%
\pgfpathlineto{\pgfqpoint{2.508520in}{5.136611in}}%
\pgfpathlineto{\pgfqpoint{2.539387in}{5.106704in}}%
\pgfpathlineto{\pgfqpoint{2.631989in}{5.011340in}}%
\pgfpathlineto{\pgfqpoint{2.662856in}{4.985617in}}%
\pgfpathlineto{\pgfqpoint{2.678290in}{4.975091in}}%
\pgfpathlineto{\pgfqpoint{2.693724in}{4.966383in}}%
\pgfpathlineto{\pgfqpoint{2.709157in}{4.959636in}}%
\pgfpathlineto{\pgfqpoint{2.724591in}{4.954947in}}%
\pgfpathlineto{\pgfqpoint{2.740024in}{4.952366in}}%
\pgfpathlineto{\pgfqpoint{2.755458in}{4.951890in}}%
\pgfpathlineto{\pgfqpoint{2.770892in}{4.953468in}}%
\pgfpathlineto{\pgfqpoint{2.786325in}{4.957000in}}%
\pgfpathlineto{\pgfqpoint{2.801759in}{4.962336in}}%
\pgfpathlineto{\pgfqpoint{2.817193in}{4.969286in}}%
\pgfpathlineto{\pgfqpoint{2.848060in}{4.987080in}}%
\pgfpathlineto{\pgfqpoint{2.894361in}{5.019223in}}%
\pgfpathlineto{\pgfqpoint{2.925228in}{5.040629in}}%
\pgfpathlineto{\pgfqpoint{2.956095in}{5.059116in}}%
\pgfpathlineto{\pgfqpoint{2.971529in}{5.066572in}}%
\pgfpathlineto{\pgfqpoint{2.986963in}{5.072524in}}%
\pgfpathlineto{\pgfqpoint{3.002396in}{5.076785in}}%
\pgfpathlineto{\pgfqpoint{3.017830in}{5.079208in}}%
\pgfpathlineto{\pgfqpoint{3.033263in}{5.079693in}}%
\pgfpathlineto{\pgfqpoint{3.048697in}{5.078185in}}%
\pgfpathlineto{\pgfqpoint{3.064131in}{5.074676in}}%
\pgfpathlineto{\pgfqpoint{3.079564in}{5.069207in}}%
\pgfpathlineto{\pgfqpoint{3.094998in}{5.061860in}}%
\pgfpathlineto{\pgfqpoint{3.110432in}{5.052762in}}%
\pgfpathlineto{\pgfqpoint{3.125865in}{5.042073in}}%
\pgfpathlineto{\pgfqpoint{3.156733in}{5.016734in}}%
\pgfpathlineto{\pgfqpoint{3.187600in}{4.987680in}}%
\pgfpathlineto{\pgfqpoint{3.264768in}{4.912257in}}%
\pgfpathlineto{\pgfqpoint{3.295635in}{4.885949in}}%
\pgfpathlineto{\pgfqpoint{3.326503in}{4.864009in}}%
\pgfpathlineto{\pgfqpoint{3.341936in}{4.854949in}}%
\pgfpathlineto{\pgfqpoint{3.357370in}{4.847245in}}%
\pgfpathlineto{\pgfqpoint{3.372803in}{4.840920in}}%
\pgfpathlineto{\pgfqpoint{3.388237in}{4.835972in}}%
\pgfpathlineto{\pgfqpoint{3.403671in}{4.832372in}}%
\pgfpathlineto{\pgfqpoint{3.419104in}{4.830073in}}%
\pgfpathlineto{\pgfqpoint{3.434538in}{4.829012in}}%
\pgfpathlineto{\pgfqpoint{3.465405in}{4.830297in}}%
\pgfpathlineto{\pgfqpoint{3.496273in}{4.835587in}}%
\pgfpathlineto{\pgfqpoint{3.527140in}{4.844291in}}%
\pgfpathlineto{\pgfqpoint{3.558007in}{4.855994in}}%
\pgfpathlineto{\pgfqpoint{3.588874in}{4.870531in}}%
\pgfpathlineto{\pgfqpoint{3.619742in}{4.888015in}}%
\pgfpathlineto{\pgfqpoint{3.650609in}{4.908783in}}%
\pgfpathlineto{\pgfqpoint{3.681476in}{4.933305in}}%
\pgfpathlineto{\pgfqpoint{3.712343in}{4.962040in}}%
\pgfpathlineto{\pgfqpoint{3.743211in}{4.995271in}}%
\pgfpathlineto{\pgfqpoint{3.774078in}{5.032949in}}%
\pgfpathlineto{\pgfqpoint{3.804945in}{5.074566in}}%
\pgfpathlineto{\pgfqpoint{3.851246in}{5.141964in}}%
\pgfpathlineto{\pgfqpoint{3.897547in}{5.210113in}}%
\pgfpathlineto{\pgfqpoint{3.928414in}{5.252286in}}%
\pgfpathlineto{\pgfqpoint{3.959282in}{5.289056in}}%
\pgfpathlineto{\pgfqpoint{3.974715in}{5.304702in}}%
\pgfpathlineto{\pgfqpoint{3.990149in}{5.318185in}}%
\pgfpathlineto{\pgfqpoint{4.005583in}{5.329293in}}%
\pgfpathlineto{\pgfqpoint{4.021016in}{5.337855in}}%
\pgfpathlineto{\pgfqpoint{4.036450in}{5.343748in}}%
\pgfpathlineto{\pgfqpoint{4.051883in}{5.346901in}}%
\pgfpathlineto{\pgfqpoint{4.067317in}{5.347298in}}%
\pgfpathlineto{\pgfqpoint{4.082751in}{5.344983in}}%
\pgfpathlineto{\pgfqpoint{4.098184in}{5.340057in}}%
\pgfpathlineto{\pgfqpoint{4.113618in}{5.332680in}}%
\pgfpathlineto{\pgfqpoint{4.129052in}{5.323069in}}%
\pgfpathlineto{\pgfqpoint{4.144485in}{5.311490in}}%
\pgfpathlineto{\pgfqpoint{4.175352in}{5.283722in}}%
\pgfpathlineto{\pgfqpoint{4.267954in}{5.191688in}}%
\pgfpathlineto{\pgfqpoint{4.283388in}{5.179348in}}%
\pgfpathlineto{\pgfqpoint{4.298822in}{5.168847in}}%
\pgfpathlineto{\pgfqpoint{4.314255in}{5.160472in}}%
\pgfpathlineto{\pgfqpoint{4.329689in}{5.154461in}}%
\pgfpathlineto{\pgfqpoint{4.345122in}{5.150994in}}%
\pgfpathlineto{\pgfqpoint{4.360556in}{5.150189in}}%
\pgfpathlineto{\pgfqpoint{4.375990in}{5.152103in}}%
\pgfpathlineto{\pgfqpoint{4.391423in}{5.156726in}}%
\pgfpathlineto{\pgfqpoint{4.406857in}{5.163988in}}%
\pgfpathlineto{\pgfqpoint{4.422291in}{5.173757in}}%
\pgfpathlineto{\pgfqpoint{4.437724in}{5.185843in}}%
\pgfpathlineto{\pgfqpoint{4.453158in}{5.200007in}}%
\pgfpathlineto{\pgfqpoint{4.484025in}{5.233390in}}%
\pgfpathlineto{\pgfqpoint{4.514892in}{5.271233in}}%
\pgfpathlineto{\pgfqpoint{4.576627in}{5.348336in}}%
\pgfpathlineto{\pgfqpoint{4.607494in}{5.381847in}}%
\pgfpathlineto{\pgfqpoint{4.622928in}{5.396274in}}%
\pgfpathlineto{\pgfqpoint{4.638362in}{5.408840in}}%
\pgfpathlineto{\pgfqpoint{4.653795in}{5.419371in}}%
\pgfpathlineto{\pgfqpoint{4.669229in}{5.427744in}}%
\pgfpathlineto{\pgfqpoint{4.669229in}{5.427744in}}%
\pgfusepath{stroke}%
\end{pgfscope}%
\begin{pgfscope}%
\pgfpathrectangle{\pgfqpoint{0.634105in}{3.881603in}}{\pgfqpoint{4.227273in}{2.800000in}} %
\pgfusepath{clip}%
\pgfsetrectcap%
\pgfsetroundjoin%
\pgfsetlinewidth{0.501875pt}%
\definecolor{currentstroke}{rgb}{0.778431,0.905873,0.536867}%
\pgfsetstrokecolor{currentstroke}%
\pgfsetdash{}{0pt}%
\pgfpathmoveto{\pgfqpoint{0.826254in}{5.670411in}}%
\pgfpathlineto{\pgfqpoint{0.841687in}{5.671928in}}%
\pgfpathlineto{\pgfqpoint{0.857121in}{5.669274in}}%
\pgfpathlineto{\pgfqpoint{0.872555in}{5.662270in}}%
\pgfpathlineto{\pgfqpoint{0.887988in}{5.650815in}}%
\pgfpathlineto{\pgfqpoint{0.903422in}{5.634887in}}%
\pgfpathlineto{\pgfqpoint{0.918855in}{5.614549in}}%
\pgfpathlineto{\pgfqpoint{0.934289in}{5.589954in}}%
\pgfpathlineto{\pgfqpoint{0.949723in}{5.561337in}}%
\pgfpathlineto{\pgfqpoint{0.965156in}{5.529021in}}%
\pgfpathlineto{\pgfqpoint{0.980590in}{5.493408in}}%
\pgfpathlineto{\pgfqpoint{1.011457in}{5.414259in}}%
\pgfpathlineto{\pgfqpoint{1.057758in}{5.284625in}}%
\pgfpathlineto{\pgfqpoint{1.088625in}{5.198708in}}%
\pgfpathlineto{\pgfqpoint{1.119493in}{5.119641in}}%
\pgfpathlineto{\pgfqpoint{1.134926in}{5.084317in}}%
\pgfpathlineto{\pgfqpoint{1.150360in}{5.052589in}}%
\pgfpathlineto{\pgfqpoint{1.165794in}{5.024979in}}%
\pgfpathlineto{\pgfqpoint{1.181227in}{5.001936in}}%
\pgfpathlineto{\pgfqpoint{1.196661in}{4.983829in}}%
\pgfpathlineto{\pgfqpoint{1.212095in}{4.970941in}}%
\pgfpathlineto{\pgfqpoint{1.227528in}{4.963461in}}%
\pgfpathlineto{\pgfqpoint{1.242962in}{4.961488in}}%
\pgfpathlineto{\pgfqpoint{1.258395in}{4.965021in}}%
\pgfpathlineto{\pgfqpoint{1.273829in}{4.973969in}}%
\pgfpathlineto{\pgfqpoint{1.289263in}{4.988149in}}%
\pgfpathlineto{\pgfqpoint{1.304696in}{5.007293in}}%
\pgfpathlineto{\pgfqpoint{1.320130in}{5.031055in}}%
\pgfpathlineto{\pgfqpoint{1.335564in}{5.059020in}}%
\pgfpathlineto{\pgfqpoint{1.350997in}{5.090712in}}%
\pgfpathlineto{\pgfqpoint{1.381864in}{5.163141in}}%
\pgfpathlineto{\pgfqpoint{1.412732in}{5.243751in}}%
\pgfpathlineto{\pgfqpoint{1.474466in}{5.410318in}}%
\pgfpathlineto{\pgfqpoint{1.505334in}{5.487271in}}%
\pgfpathlineto{\pgfqpoint{1.536201in}{5.554946in}}%
\pgfpathlineto{\pgfqpoint{1.551634in}{5.584394in}}%
\pgfpathlineto{\pgfqpoint{1.567068in}{5.610542in}}%
\pgfpathlineto{\pgfqpoint{1.582502in}{5.633180in}}%
\pgfpathlineto{\pgfqpoint{1.597935in}{5.652154in}}%
\pgfpathlineto{\pgfqpoint{1.613369in}{5.667367in}}%
\pgfpathlineto{\pgfqpoint{1.628803in}{5.678772in}}%
\pgfpathlineto{\pgfqpoint{1.644236in}{5.686373in}}%
\pgfpathlineto{\pgfqpoint{1.659670in}{5.690217in}}%
\pgfpathlineto{\pgfqpoint{1.675104in}{5.690395in}}%
\pgfpathlineto{\pgfqpoint{1.690537in}{5.687034in}}%
\pgfpathlineto{\pgfqpoint{1.705971in}{5.680291in}}%
\pgfpathlineto{\pgfqpoint{1.721404in}{5.670355in}}%
\pgfpathlineto{\pgfqpoint{1.736838in}{5.657439in}}%
\pgfpathlineto{\pgfqpoint{1.752272in}{5.641775in}}%
\pgfpathlineto{\pgfqpoint{1.767705in}{5.623612in}}%
\pgfpathlineto{\pgfqpoint{1.783139in}{5.603215in}}%
\pgfpathlineto{\pgfqpoint{1.814006in}{5.556823in}}%
\pgfpathlineto{\pgfqpoint{1.844874in}{5.504877in}}%
\pgfpathlineto{\pgfqpoint{1.968343in}{5.287925in}}%
\pgfpathlineto{\pgfqpoint{1.999210in}{5.242227in}}%
\pgfpathlineto{\pgfqpoint{2.014644in}{5.221924in}}%
\pgfpathlineto{\pgfqpoint{2.030077in}{5.203544in}}%
\pgfpathlineto{\pgfqpoint{2.045511in}{5.187219in}}%
\pgfpathlineto{\pgfqpoint{2.060944in}{5.173054in}}%
\pgfpathlineto{\pgfqpoint{2.076378in}{5.161116in}}%
\pgfpathlineto{\pgfqpoint{2.091812in}{5.151439in}}%
\pgfpathlineto{\pgfqpoint{2.107245in}{5.144017in}}%
\pgfpathlineto{\pgfqpoint{2.122679in}{5.138807in}}%
\pgfpathlineto{\pgfqpoint{2.138113in}{5.135726in}}%
\pgfpathlineto{\pgfqpoint{2.153546in}{5.134655in}}%
\pgfpathlineto{\pgfqpoint{2.168980in}{5.135434in}}%
\pgfpathlineto{\pgfqpoint{2.184414in}{5.137872in}}%
\pgfpathlineto{\pgfqpoint{2.199847in}{5.141743in}}%
\pgfpathlineto{\pgfqpoint{2.230714in}{5.152753in}}%
\pgfpathlineto{\pgfqpoint{2.307883in}{5.185507in}}%
\pgfpathlineto{\pgfqpoint{2.323316in}{5.190507in}}%
\pgfpathlineto{\pgfqpoint{2.338750in}{5.194323in}}%
\pgfpathlineto{\pgfqpoint{2.354184in}{5.196716in}}%
\pgfpathlineto{\pgfqpoint{2.369617in}{5.197478in}}%
\pgfpathlineto{\pgfqpoint{2.385051in}{5.196442in}}%
\pgfpathlineto{\pgfqpoint{2.400484in}{5.193481in}}%
\pgfpathlineto{\pgfqpoint{2.415918in}{5.188517in}}%
\pgfpathlineto{\pgfqpoint{2.431352in}{5.181520in}}%
\pgfpathlineto{\pgfqpoint{2.446785in}{5.172510in}}%
\pgfpathlineto{\pgfqpoint{2.462219in}{5.161557in}}%
\pgfpathlineto{\pgfqpoint{2.477653in}{5.148780in}}%
\pgfpathlineto{\pgfqpoint{2.508520in}{5.118454in}}%
\pgfpathlineto{\pgfqpoint{2.539387in}{5.083339in}}%
\pgfpathlineto{\pgfqpoint{2.616555in}{4.990550in}}%
\pgfpathlineto{\pgfqpoint{2.647423in}{4.958389in}}%
\pgfpathlineto{\pgfqpoint{2.662856in}{4.944608in}}%
\pgfpathlineto{\pgfqpoint{2.678290in}{4.932698in}}%
\pgfpathlineto{\pgfqpoint{2.693724in}{4.922864in}}%
\pgfpathlineto{\pgfqpoint{2.709157in}{4.915267in}}%
\pgfpathlineto{\pgfqpoint{2.724591in}{4.910019in}}%
\pgfpathlineto{\pgfqpoint{2.740024in}{4.907182in}}%
\pgfpathlineto{\pgfqpoint{2.755458in}{4.906765in}}%
\pgfpathlineto{\pgfqpoint{2.770892in}{4.908724in}}%
\pgfpathlineto{\pgfqpoint{2.786325in}{4.912963in}}%
\pgfpathlineto{\pgfqpoint{2.801759in}{4.919334in}}%
\pgfpathlineto{\pgfqpoint{2.817193in}{4.927646in}}%
\pgfpathlineto{\pgfqpoint{2.832626in}{4.937662in}}%
\pgfpathlineto{\pgfqpoint{2.863493in}{4.961685in}}%
\pgfpathlineto{\pgfqpoint{2.956095in}{5.041777in}}%
\pgfpathlineto{\pgfqpoint{2.971529in}{5.052708in}}%
\pgfpathlineto{\pgfqpoint{2.986963in}{5.062123in}}%
\pgfpathlineto{\pgfqpoint{3.002396in}{5.069779in}}%
\pgfpathlineto{\pgfqpoint{3.017830in}{5.075473in}}%
\pgfpathlineto{\pgfqpoint{3.033263in}{5.079045in}}%
\pgfpathlineto{\pgfqpoint{3.048697in}{5.080383in}}%
\pgfpathlineto{\pgfqpoint{3.064131in}{5.079428in}}%
\pgfpathlineto{\pgfqpoint{3.079564in}{5.076166in}}%
\pgfpathlineto{\pgfqpoint{3.094998in}{5.070637in}}%
\pgfpathlineto{\pgfqpoint{3.110432in}{5.062928in}}%
\pgfpathlineto{\pgfqpoint{3.125865in}{5.053172in}}%
\pgfpathlineto{\pgfqpoint{3.141299in}{5.041542in}}%
\pgfpathlineto{\pgfqpoint{3.172166in}{5.013534in}}%
\pgfpathlineto{\pgfqpoint{3.203033in}{4.980924in}}%
\pgfpathlineto{\pgfqpoint{3.295635in}{4.878682in}}%
\pgfpathlineto{\pgfqpoint{3.326503in}{4.850358in}}%
\pgfpathlineto{\pgfqpoint{3.341936in}{4.838233in}}%
\pgfpathlineto{\pgfqpoint{3.357370in}{4.827649in}}%
\pgfpathlineto{\pgfqpoint{3.372803in}{4.818697in}}%
\pgfpathlineto{\pgfqpoint{3.388237in}{4.811436in}}%
\pgfpathlineto{\pgfqpoint{3.403671in}{4.805891in}}%
\pgfpathlineto{\pgfqpoint{3.419104in}{4.802056in}}%
\pgfpathlineto{\pgfqpoint{3.434538in}{4.799897in}}%
\pgfpathlineto{\pgfqpoint{3.449972in}{4.799358in}}%
\pgfpathlineto{\pgfqpoint{3.465405in}{4.800363in}}%
\pgfpathlineto{\pgfqpoint{3.480839in}{4.802821in}}%
\pgfpathlineto{\pgfqpoint{3.496273in}{4.806631in}}%
\pgfpathlineto{\pgfqpoint{3.527140in}{4.817891in}}%
\pgfpathlineto{\pgfqpoint{3.558007in}{4.833329in}}%
\pgfpathlineto{\pgfqpoint{3.588874in}{4.852254in}}%
\pgfpathlineto{\pgfqpoint{3.619742in}{4.874175in}}%
\pgfpathlineto{\pgfqpoint{3.650609in}{4.898835in}}%
\pgfpathlineto{\pgfqpoint{3.681476in}{4.926172in}}%
\pgfpathlineto{\pgfqpoint{3.712343in}{4.956236in}}%
\pgfpathlineto{\pgfqpoint{3.743211in}{4.989073in}}%
\pgfpathlineto{\pgfqpoint{3.774078in}{5.024587in}}%
\pgfpathlineto{\pgfqpoint{3.820379in}{5.081984in}}%
\pgfpathlineto{\pgfqpoint{3.897547in}{5.180606in}}%
\pgfpathlineto{\pgfqpoint{3.928414in}{5.216732in}}%
\pgfpathlineto{\pgfqpoint{3.959282in}{5.248242in}}%
\pgfpathlineto{\pgfqpoint{3.974715in}{5.261706in}}%
\pgfpathlineto{\pgfqpoint{3.990149in}{5.273374in}}%
\pgfpathlineto{\pgfqpoint{4.005583in}{5.283078in}}%
\pgfpathlineto{\pgfqpoint{4.021016in}{5.290679in}}%
\pgfpathlineto{\pgfqpoint{4.036450in}{5.296081in}}%
\pgfpathlineto{\pgfqpoint{4.051883in}{5.299227in}}%
\pgfpathlineto{\pgfqpoint{4.067317in}{5.300109in}}%
\pgfpathlineto{\pgfqpoint{4.082751in}{5.298768in}}%
\pgfpathlineto{\pgfqpoint{4.098184in}{5.295295in}}%
\pgfpathlineto{\pgfqpoint{4.113618in}{5.289830in}}%
\pgfpathlineto{\pgfqpoint{4.129052in}{5.282562in}}%
\pgfpathlineto{\pgfqpoint{4.144485in}{5.273723in}}%
\pgfpathlineto{\pgfqpoint{4.175352in}{5.252456in}}%
\pgfpathlineto{\pgfqpoint{4.252521in}{5.194019in}}%
\pgfpathlineto{\pgfqpoint{4.267954in}{5.184249in}}%
\pgfpathlineto{\pgfqpoint{4.283388in}{5.175887in}}%
\pgfpathlineto{\pgfqpoint{4.298822in}{5.169205in}}%
\pgfpathlineto{\pgfqpoint{4.314255in}{5.164435in}}%
\pgfpathlineto{\pgfqpoint{4.329689in}{5.161760in}}%
\pgfpathlineto{\pgfqpoint{4.345122in}{5.161311in}}%
\pgfpathlineto{\pgfqpoint{4.360556in}{5.163163in}}%
\pgfpathlineto{\pgfqpoint{4.375990in}{5.167334in}}%
\pgfpathlineto{\pgfqpoint{4.391423in}{5.173783in}}%
\pgfpathlineto{\pgfqpoint{4.406857in}{5.182415in}}%
\pgfpathlineto{\pgfqpoint{4.422291in}{5.193081in}}%
\pgfpathlineto{\pgfqpoint{4.437724in}{5.205581in}}%
\pgfpathlineto{\pgfqpoint{4.468592in}{5.235085in}}%
\pgfpathlineto{\pgfqpoint{4.499459in}{5.268612in}}%
\pgfpathlineto{\pgfqpoint{4.561193in}{5.337142in}}%
\pgfpathlineto{\pgfqpoint{4.592061in}{5.367009in}}%
\pgfpathlineto{\pgfqpoint{4.607494in}{5.379890in}}%
\pgfpathlineto{\pgfqpoint{4.622928in}{5.391132in}}%
\pgfpathlineto{\pgfqpoint{4.638362in}{5.400583in}}%
\pgfpathlineto{\pgfqpoint{4.653795in}{5.408141in}}%
\pgfpathlineto{\pgfqpoint{4.669229in}{5.413746in}}%
\pgfpathlineto{\pgfqpoint{4.669229in}{5.413746in}}%
\pgfusepath{stroke}%
\end{pgfscope}%
\begin{pgfscope}%
\pgfpathrectangle{\pgfqpoint{0.634105in}{3.881603in}}{\pgfqpoint{4.227273in}{2.800000in}} %
\pgfusepath{clip}%
\pgfsetrectcap%
\pgfsetroundjoin%
\pgfsetlinewidth{0.501875pt}%
\definecolor{currentstroke}{rgb}{0.864706,0.840344,0.478512}%
\pgfsetstrokecolor{currentstroke}%
\pgfsetdash{}{0pt}%
\pgfpathmoveto{\pgfqpoint{0.826254in}{5.666965in}}%
\pgfpathlineto{\pgfqpoint{0.841687in}{5.669468in}}%
\pgfpathlineto{\pgfqpoint{0.857121in}{5.668019in}}%
\pgfpathlineto{\pgfqpoint{0.872555in}{5.662393in}}%
\pgfpathlineto{\pgfqpoint{0.887988in}{5.652436in}}%
\pgfpathlineto{\pgfqpoint{0.903422in}{5.638079in}}%
\pgfpathlineto{\pgfqpoint{0.918855in}{5.619333in}}%
\pgfpathlineto{\pgfqpoint{0.934289in}{5.596304in}}%
\pgfpathlineto{\pgfqpoint{0.949723in}{5.569182in}}%
\pgfpathlineto{\pgfqpoint{0.965156in}{5.538251in}}%
\pgfpathlineto{\pgfqpoint{0.980590in}{5.503879in}}%
\pgfpathlineto{\pgfqpoint{1.011457in}{5.426677in}}%
\pgfpathlineto{\pgfqpoint{1.057758in}{5.298436in}}%
\pgfpathlineto{\pgfqpoint{1.104059in}{5.171432in}}%
\pgfpathlineto{\pgfqpoint{1.134926in}{5.096791in}}%
\pgfpathlineto{\pgfqpoint{1.150360in}{5.064436in}}%
\pgfpathlineto{\pgfqpoint{1.165794in}{5.036147in}}%
\pgfpathlineto{\pgfqpoint{1.181227in}{5.012403in}}%
\pgfpathlineto{\pgfqpoint{1.196661in}{4.993602in}}%
\pgfpathlineto{\pgfqpoint{1.212095in}{4.980053in}}%
\pgfpathlineto{\pgfqpoint{1.227528in}{4.971971in}}%
\pgfpathlineto{\pgfqpoint{1.242962in}{4.969472in}}%
\pgfpathlineto{\pgfqpoint{1.258395in}{4.972573in}}%
\pgfpathlineto{\pgfqpoint{1.273829in}{4.981191in}}%
\pgfpathlineto{\pgfqpoint{1.289263in}{4.995149in}}%
\pgfpathlineto{\pgfqpoint{1.304696in}{5.014181in}}%
\pgfpathlineto{\pgfqpoint{1.320130in}{5.037934in}}%
\pgfpathlineto{\pgfqpoint{1.335564in}{5.065984in}}%
\pgfpathlineto{\pgfqpoint{1.350997in}{5.097842in}}%
\pgfpathlineto{\pgfqpoint{1.381864in}{5.170776in}}%
\pgfpathlineto{\pgfqpoint{1.412732in}{5.251986in}}%
\pgfpathlineto{\pgfqpoint{1.474466in}{5.419382in}}%
\pgfpathlineto{\pgfqpoint{1.505334in}{5.496311in}}%
\pgfpathlineto{\pgfqpoint{1.520767in}{5.531356in}}%
\pgfpathlineto{\pgfqpoint{1.536201in}{5.563603in}}%
\pgfpathlineto{\pgfqpoint{1.551634in}{5.592735in}}%
\pgfpathlineto{\pgfqpoint{1.567068in}{5.618499in}}%
\pgfpathlineto{\pgfqpoint{1.582502in}{5.640703in}}%
\pgfpathlineto{\pgfqpoint{1.597935in}{5.659212in}}%
\pgfpathlineto{\pgfqpoint{1.613369in}{5.673953in}}%
\pgfpathlineto{\pgfqpoint{1.628803in}{5.684905in}}%
\pgfpathlineto{\pgfqpoint{1.644236in}{5.692096in}}%
\pgfpathlineto{\pgfqpoint{1.659670in}{5.695600in}}%
\pgfpathlineto{\pgfqpoint{1.675104in}{5.695532in}}%
\pgfpathlineto{\pgfqpoint{1.690537in}{5.692039in}}%
\pgfpathlineto{\pgfqpoint{1.705971in}{5.685299in}}%
\pgfpathlineto{\pgfqpoint{1.721404in}{5.675512in}}%
\pgfpathlineto{\pgfqpoint{1.736838in}{5.662899in}}%
\pgfpathlineto{\pgfqpoint{1.752272in}{5.647694in}}%
\pgfpathlineto{\pgfqpoint{1.767705in}{5.630141in}}%
\pgfpathlineto{\pgfqpoint{1.783139in}{5.610493in}}%
\pgfpathlineto{\pgfqpoint{1.814006in}{5.565940in}}%
\pgfpathlineto{\pgfqpoint{1.844874in}{5.516100in}}%
\pgfpathlineto{\pgfqpoint{1.891174in}{5.435912in}}%
\pgfpathlineto{\pgfqpoint{1.952909in}{5.329253in}}%
\pgfpathlineto{\pgfqpoint{1.983776in}{5.280231in}}%
\pgfpathlineto{\pgfqpoint{2.014644in}{5.236318in}}%
\pgfpathlineto{\pgfqpoint{2.045511in}{5.198914in}}%
\pgfpathlineto{\pgfqpoint{2.060944in}{5.183005in}}%
\pgfpathlineto{\pgfqpoint{2.076378in}{5.169093in}}%
\pgfpathlineto{\pgfqpoint{2.091812in}{5.157248in}}%
\pgfpathlineto{\pgfqpoint{2.107245in}{5.147503in}}%
\pgfpathlineto{\pgfqpoint{2.122679in}{5.139859in}}%
\pgfpathlineto{\pgfqpoint{2.138113in}{5.134281in}}%
\pgfpathlineto{\pgfqpoint{2.153546in}{5.130696in}}%
\pgfpathlineto{\pgfqpoint{2.168980in}{5.128995in}}%
\pgfpathlineto{\pgfqpoint{2.184414in}{5.129032in}}%
\pgfpathlineto{\pgfqpoint{2.199847in}{5.130629in}}%
\pgfpathlineto{\pgfqpoint{2.230714in}{5.137621in}}%
\pgfpathlineto{\pgfqpoint{2.261582in}{5.147965in}}%
\pgfpathlineto{\pgfqpoint{2.307883in}{5.164705in}}%
\pgfpathlineto{\pgfqpoint{2.338750in}{5.173230in}}%
\pgfpathlineto{\pgfqpoint{2.354184in}{5.175880in}}%
\pgfpathlineto{\pgfqpoint{2.369617in}{5.177146in}}%
\pgfpathlineto{\pgfqpoint{2.385051in}{5.176836in}}%
\pgfpathlineto{\pgfqpoint{2.400484in}{5.174801in}}%
\pgfpathlineto{\pgfqpoint{2.415918in}{5.170933in}}%
\pgfpathlineto{\pgfqpoint{2.431352in}{5.165176in}}%
\pgfpathlineto{\pgfqpoint{2.446785in}{5.157520in}}%
\pgfpathlineto{\pgfqpoint{2.462219in}{5.148007in}}%
\pgfpathlineto{\pgfqpoint{2.477653in}{5.136728in}}%
\pgfpathlineto{\pgfqpoint{2.508520in}{5.109477in}}%
\pgfpathlineto{\pgfqpoint{2.539387in}{5.077395in}}%
\pgfpathlineto{\pgfqpoint{2.631989in}{4.975484in}}%
\pgfpathlineto{\pgfqpoint{2.647423in}{4.961038in}}%
\pgfpathlineto{\pgfqpoint{2.662856in}{4.948115in}}%
\pgfpathlineto{\pgfqpoint{2.678290in}{4.936955in}}%
\pgfpathlineto{\pgfqpoint{2.693724in}{4.927762in}}%
\pgfpathlineto{\pgfqpoint{2.709157in}{4.920693in}}%
\pgfpathlineto{\pgfqpoint{2.724591in}{4.915858in}}%
\pgfpathlineto{\pgfqpoint{2.740024in}{4.913317in}}%
\pgfpathlineto{\pgfqpoint{2.755458in}{4.913076in}}%
\pgfpathlineto{\pgfqpoint{2.770892in}{4.915088in}}%
\pgfpathlineto{\pgfqpoint{2.786325in}{4.919255in}}%
\pgfpathlineto{\pgfqpoint{2.801759in}{4.925428in}}%
\pgfpathlineto{\pgfqpoint{2.817193in}{4.933414in}}%
\pgfpathlineto{\pgfqpoint{2.848060in}{4.953842in}}%
\pgfpathlineto{\pgfqpoint{2.878927in}{4.978253in}}%
\pgfpathlineto{\pgfqpoint{2.940662in}{5.028354in}}%
\pgfpathlineto{\pgfqpoint{2.971529in}{5.048687in}}%
\pgfpathlineto{\pgfqpoint{2.986963in}{5.056636in}}%
\pgfpathlineto{\pgfqpoint{3.002396in}{5.062791in}}%
\pgfpathlineto{\pgfqpoint{3.017830in}{5.066971in}}%
\pgfpathlineto{\pgfqpoint{3.033263in}{5.069038in}}%
\pgfpathlineto{\pgfqpoint{3.048697in}{5.068905in}}%
\pgfpathlineto{\pgfqpoint{3.064131in}{5.066533in}}%
\pgfpathlineto{\pgfqpoint{3.079564in}{5.061938in}}%
\pgfpathlineto{\pgfqpoint{3.094998in}{5.055183in}}%
\pgfpathlineto{\pgfqpoint{3.110432in}{5.046381in}}%
\pgfpathlineto{\pgfqpoint{3.125865in}{5.035687in}}%
\pgfpathlineto{\pgfqpoint{3.141299in}{5.023299in}}%
\pgfpathlineto{\pgfqpoint{3.172166in}{4.994393in}}%
\pgfpathlineto{\pgfqpoint{3.218467in}{4.944882in}}%
\pgfpathlineto{\pgfqpoint{3.264768in}{4.895101in}}%
\pgfpathlineto{\pgfqpoint{3.295635in}{4.865452in}}%
\pgfpathlineto{\pgfqpoint{3.326503in}{4.840677in}}%
\pgfpathlineto{\pgfqpoint{3.341936in}{4.830499in}}%
\pgfpathlineto{\pgfqpoint{3.357370in}{4.821921in}}%
\pgfpathlineto{\pgfqpoint{3.372803in}{4.814993in}}%
\pgfpathlineto{\pgfqpoint{3.388237in}{4.809727in}}%
\pgfpathlineto{\pgfqpoint{3.403671in}{4.806101in}}%
\pgfpathlineto{\pgfqpoint{3.419104in}{4.804062in}}%
\pgfpathlineto{\pgfqpoint{3.434538in}{4.803533in}}%
\pgfpathlineto{\pgfqpoint{3.449972in}{4.804414in}}%
\pgfpathlineto{\pgfqpoint{3.465405in}{4.806592in}}%
\pgfpathlineto{\pgfqpoint{3.496273in}{4.814343in}}%
\pgfpathlineto{\pgfqpoint{3.527140in}{4.825799in}}%
\pgfpathlineto{\pgfqpoint{3.558007in}{4.840094in}}%
\pgfpathlineto{\pgfqpoint{3.588874in}{4.856623in}}%
\pgfpathlineto{\pgfqpoint{3.619742in}{4.875107in}}%
\pgfpathlineto{\pgfqpoint{3.650609in}{4.895605in}}%
\pgfpathlineto{\pgfqpoint{3.681476in}{4.918433in}}%
\pgfpathlineto{\pgfqpoint{3.712343in}{4.944043in}}%
\pgfpathlineto{\pgfqpoint{3.743211in}{4.972846in}}%
\pgfpathlineto{\pgfqpoint{3.774078in}{5.005038in}}%
\pgfpathlineto{\pgfqpoint{3.804945in}{5.040444in}}%
\pgfpathlineto{\pgfqpoint{3.851246in}{5.098008in}}%
\pgfpathlineto{\pgfqpoint{3.912981in}{5.175794in}}%
\pgfpathlineto{\pgfqpoint{3.943848in}{5.210907in}}%
\pgfpathlineto{\pgfqpoint{3.974715in}{5.240792in}}%
\pgfpathlineto{\pgfqpoint{3.990149in}{5.253203in}}%
\pgfpathlineto{\pgfqpoint{4.005583in}{5.263674in}}%
\pgfpathlineto{\pgfqpoint{4.021016in}{5.272065in}}%
\pgfpathlineto{\pgfqpoint{4.036450in}{5.278277in}}%
\pgfpathlineto{\pgfqpoint{4.051883in}{5.282261in}}%
\pgfpathlineto{\pgfqpoint{4.067317in}{5.284021in}}%
\pgfpathlineto{\pgfqpoint{4.082751in}{5.283612in}}%
\pgfpathlineto{\pgfqpoint{4.098184in}{5.281147in}}%
\pgfpathlineto{\pgfqpoint{4.113618in}{5.276787in}}%
\pgfpathlineto{\pgfqpoint{4.129052in}{5.270745in}}%
\pgfpathlineto{\pgfqpoint{4.159919in}{5.254677in}}%
\pgfpathlineto{\pgfqpoint{4.206220in}{5.225422in}}%
\pgfpathlineto{\pgfqpoint{4.237087in}{5.206597in}}%
\pgfpathlineto{\pgfqpoint{4.252521in}{5.198438in}}%
\pgfpathlineto{\pgfqpoint{4.267954in}{5.191536in}}%
\pgfpathlineto{\pgfqpoint{4.283388in}{5.186166in}}%
\pgfpathlineto{\pgfqpoint{4.298822in}{5.182560in}}%
\pgfpathlineto{\pgfqpoint{4.314255in}{5.180903in}}%
\pgfpathlineto{\pgfqpoint{4.329689in}{5.181322in}}%
\pgfpathlineto{\pgfqpoint{4.345122in}{5.183888in}}%
\pgfpathlineto{\pgfqpoint{4.360556in}{5.188616in}}%
\pgfpathlineto{\pgfqpoint{4.375990in}{5.195457in}}%
\pgfpathlineto{\pgfqpoint{4.391423in}{5.204309in}}%
\pgfpathlineto{\pgfqpoint{4.406857in}{5.215014in}}%
\pgfpathlineto{\pgfqpoint{4.437724in}{5.241115in}}%
\pgfpathlineto{\pgfqpoint{4.468592in}{5.271645in}}%
\pgfpathlineto{\pgfqpoint{4.530326in}{5.335696in}}%
\pgfpathlineto{\pgfqpoint{4.561193in}{5.364023in}}%
\pgfpathlineto{\pgfqpoint{4.576627in}{5.376269in}}%
\pgfpathlineto{\pgfqpoint{4.592061in}{5.386948in}}%
\pgfpathlineto{\pgfqpoint{4.607494in}{5.395894in}}%
\pgfpathlineto{\pgfqpoint{4.622928in}{5.402993in}}%
\pgfpathlineto{\pgfqpoint{4.638362in}{5.408177in}}%
\pgfpathlineto{\pgfqpoint{4.653795in}{5.411427in}}%
\pgfpathlineto{\pgfqpoint{4.669229in}{5.412773in}}%
\pgfpathlineto{\pgfqpoint{4.669229in}{5.412773in}}%
\pgfusepath{stroke}%
\end{pgfscope}%
\begin{pgfscope}%
\pgfpathrectangle{\pgfqpoint{0.634105in}{3.881603in}}{\pgfqpoint{4.227273in}{2.800000in}} %
\pgfusepath{clip}%
\pgfsetrectcap%
\pgfsetroundjoin%
\pgfsetlinewidth{0.501875pt}%
\definecolor{currentstroke}{rgb}{0.943137,0.767363,0.423549}%
\pgfsetstrokecolor{currentstroke}%
\pgfsetdash{}{0pt}%
\pgfpathmoveto{\pgfqpoint{0.826254in}{5.651045in}}%
\pgfpathlineto{\pgfqpoint{0.841687in}{5.652361in}}%
\pgfpathlineto{\pgfqpoint{0.857121in}{5.649915in}}%
\pgfpathlineto{\pgfqpoint{0.872555in}{5.643513in}}%
\pgfpathlineto{\pgfqpoint{0.887988in}{5.633026in}}%
\pgfpathlineto{\pgfqpoint{0.903422in}{5.618399in}}%
\pgfpathlineto{\pgfqpoint{0.918855in}{5.599658in}}%
\pgfpathlineto{\pgfqpoint{0.934289in}{5.576910in}}%
\pgfpathlineto{\pgfqpoint{0.949723in}{5.550345in}}%
\pgfpathlineto{\pgfqpoint{0.965156in}{5.520235in}}%
\pgfpathlineto{\pgfqpoint{0.980590in}{5.486933in}}%
\pgfpathlineto{\pgfqpoint{1.011457in}{5.412520in}}%
\pgfpathlineto{\pgfqpoint{1.057758in}{5.289605in}}%
\pgfpathlineto{\pgfqpoint{1.088625in}{5.207480in}}%
\pgfpathlineto{\pgfqpoint{1.119493in}{5.131420in}}%
\pgfpathlineto{\pgfqpoint{1.134926in}{5.097271in}}%
\pgfpathlineto{\pgfqpoint{1.150360in}{5.066495in}}%
\pgfpathlineto{\pgfqpoint{1.165794in}{5.039612in}}%
\pgfpathlineto{\pgfqpoint{1.181227in}{5.017080in}}%
\pgfpathlineto{\pgfqpoint{1.196661in}{4.999276in}}%
\pgfpathlineto{\pgfqpoint{1.212095in}{4.986498in}}%
\pgfpathlineto{\pgfqpoint{1.227528in}{4.978952in}}%
\pgfpathlineto{\pgfqpoint{1.242962in}{4.976753in}}%
\pgfpathlineto{\pgfqpoint{1.258395in}{4.979921in}}%
\pgfpathlineto{\pgfqpoint{1.273829in}{4.988383in}}%
\pgfpathlineto{\pgfqpoint{1.289263in}{5.001975in}}%
\pgfpathlineto{\pgfqpoint{1.304696in}{5.020448in}}%
\pgfpathlineto{\pgfqpoint{1.320130in}{5.043473in}}%
\pgfpathlineto{\pgfqpoint{1.335564in}{5.070650in}}%
\pgfpathlineto{\pgfqpoint{1.350997in}{5.101516in}}%
\pgfpathlineto{\pgfqpoint{1.381864in}{5.172221in}}%
\pgfpathlineto{\pgfqpoint{1.412732in}{5.251070in}}%
\pgfpathlineto{\pgfqpoint{1.474466in}{5.414169in}}%
\pgfpathlineto{\pgfqpoint{1.505334in}{5.489488in}}%
\pgfpathlineto{\pgfqpoint{1.536201in}{5.555665in}}%
\pgfpathlineto{\pgfqpoint{1.551634in}{5.584436in}}%
\pgfpathlineto{\pgfqpoint{1.567068in}{5.609971in}}%
\pgfpathlineto{\pgfqpoint{1.582502in}{5.632073in}}%
\pgfpathlineto{\pgfqpoint{1.597935in}{5.650602in}}%
\pgfpathlineto{\pgfqpoint{1.613369in}{5.665475in}}%
\pgfpathlineto{\pgfqpoint{1.628803in}{5.676661in}}%
\pgfpathlineto{\pgfqpoint{1.644236in}{5.684177in}}%
\pgfpathlineto{\pgfqpoint{1.659670in}{5.688081in}}%
\pgfpathlineto{\pgfqpoint{1.675104in}{5.688475in}}%
\pgfpathlineto{\pgfqpoint{1.690537in}{5.685492in}}%
\pgfpathlineto{\pgfqpoint{1.705971in}{5.679294in}}%
\pgfpathlineto{\pgfqpoint{1.721404in}{5.670068in}}%
\pgfpathlineto{\pgfqpoint{1.736838in}{5.658020in}}%
\pgfpathlineto{\pgfqpoint{1.752272in}{5.643371in}}%
\pgfpathlineto{\pgfqpoint{1.767705in}{5.626355in}}%
\pgfpathlineto{\pgfqpoint{1.783139in}{5.607213in}}%
\pgfpathlineto{\pgfqpoint{1.814006in}{5.563542in}}%
\pgfpathlineto{\pgfqpoint{1.844874in}{5.514369in}}%
\pgfpathlineto{\pgfqpoint{1.891174in}{5.434709in}}%
\pgfpathlineto{\pgfqpoint{1.952909in}{5.327874in}}%
\pgfpathlineto{\pgfqpoint{1.983776in}{5.278441in}}%
\pgfpathlineto{\pgfqpoint{2.014644in}{5.233959in}}%
\pgfpathlineto{\pgfqpoint{2.045511in}{5.195884in}}%
\pgfpathlineto{\pgfqpoint{2.060944in}{5.179618in}}%
\pgfpathlineto{\pgfqpoint{2.076378in}{5.165348in}}%
\pgfpathlineto{\pgfqpoint{2.091812in}{5.153149in}}%
\pgfpathlineto{\pgfqpoint{2.107245in}{5.143063in}}%
\pgfpathlineto{\pgfqpoint{2.122679in}{5.135098in}}%
\pgfpathlineto{\pgfqpoint{2.138113in}{5.129226in}}%
\pgfpathlineto{\pgfqpoint{2.153546in}{5.125382in}}%
\pgfpathlineto{\pgfqpoint{2.168980in}{5.123462in}}%
\pgfpathlineto{\pgfqpoint{2.184414in}{5.123327in}}%
\pgfpathlineto{\pgfqpoint{2.199847in}{5.124799in}}%
\pgfpathlineto{\pgfqpoint{2.215281in}{5.127672in}}%
\pgfpathlineto{\pgfqpoint{2.246148in}{5.136641in}}%
\pgfpathlineto{\pgfqpoint{2.323316in}{5.164535in}}%
\pgfpathlineto{\pgfqpoint{2.354184in}{5.171653in}}%
\pgfpathlineto{\pgfqpoint{2.369617in}{5.173271in}}%
\pgfpathlineto{\pgfqpoint{2.385051in}{5.173314in}}%
\pgfpathlineto{\pgfqpoint{2.400484in}{5.171620in}}%
\pgfpathlineto{\pgfqpoint{2.415918in}{5.168071in}}%
\pgfpathlineto{\pgfqpoint{2.431352in}{5.162597in}}%
\pgfpathlineto{\pgfqpoint{2.446785in}{5.155177in}}%
\pgfpathlineto{\pgfqpoint{2.462219in}{5.145841in}}%
\pgfpathlineto{\pgfqpoint{2.477653in}{5.134671in}}%
\pgfpathlineto{\pgfqpoint{2.493086in}{5.121797in}}%
\pgfpathlineto{\pgfqpoint{2.523954in}{5.091688in}}%
\pgfpathlineto{\pgfqpoint{2.554821in}{5.057407in}}%
\pgfpathlineto{\pgfqpoint{2.616555in}{4.986130in}}%
\pgfpathlineto{\pgfqpoint{2.647423in}{4.954426in}}%
\pgfpathlineto{\pgfqpoint{2.662856in}{4.940650in}}%
\pgfpathlineto{\pgfqpoint{2.678290in}{4.928612in}}%
\pgfpathlineto{\pgfqpoint{2.693724in}{4.918529in}}%
\pgfpathlineto{\pgfqpoint{2.709157in}{4.910577in}}%
\pgfpathlineto{\pgfqpoint{2.724591in}{4.904883in}}%
\pgfpathlineto{\pgfqpoint{2.740024in}{4.901523in}}%
\pgfpathlineto{\pgfqpoint{2.755458in}{4.900522in}}%
\pgfpathlineto{\pgfqpoint{2.770892in}{4.901848in}}%
\pgfpathlineto{\pgfqpoint{2.786325in}{4.905418in}}%
\pgfpathlineto{\pgfqpoint{2.801759in}{4.911099in}}%
\pgfpathlineto{\pgfqpoint{2.817193in}{4.918707in}}%
\pgfpathlineto{\pgfqpoint{2.832626in}{4.928017in}}%
\pgfpathlineto{\pgfqpoint{2.863493in}{4.950657in}}%
\pgfpathlineto{\pgfqpoint{2.909794in}{4.989905in}}%
\pgfpathlineto{\pgfqpoint{2.940662in}{5.015608in}}%
\pgfpathlineto{\pgfqpoint{2.971529in}{5.037831in}}%
\pgfpathlineto{\pgfqpoint{2.986963in}{5.046902in}}%
\pgfpathlineto{\pgfqpoint{3.002396in}{5.054283in}}%
\pgfpathlineto{\pgfqpoint{3.017830in}{5.059779in}}%
\pgfpathlineto{\pgfqpoint{3.033263in}{5.063238in}}%
\pgfpathlineto{\pgfqpoint{3.048697in}{5.064556in}}%
\pgfpathlineto{\pgfqpoint{3.064131in}{5.063680in}}%
\pgfpathlineto{\pgfqpoint{3.079564in}{5.060607in}}%
\pgfpathlineto{\pgfqpoint{3.094998in}{5.055383in}}%
\pgfpathlineto{\pgfqpoint{3.110432in}{5.048103in}}%
\pgfpathlineto{\pgfqpoint{3.125865in}{5.038907in}}%
\pgfpathlineto{\pgfqpoint{3.141299in}{5.027973in}}%
\pgfpathlineto{\pgfqpoint{3.172166in}{5.001784in}}%
\pgfpathlineto{\pgfqpoint{3.203033in}{4.971565in}}%
\pgfpathlineto{\pgfqpoint{3.280202in}{4.893320in}}%
\pgfpathlineto{\pgfqpoint{3.311069in}{4.866510in}}%
\pgfpathlineto{\pgfqpoint{3.326503in}{4.854911in}}%
\pgfpathlineto{\pgfqpoint{3.341936in}{4.844704in}}%
\pgfpathlineto{\pgfqpoint{3.357370in}{4.835983in}}%
\pgfpathlineto{\pgfqpoint{3.372803in}{4.828807in}}%
\pgfpathlineto{\pgfqpoint{3.388237in}{4.823202in}}%
\pgfpathlineto{\pgfqpoint{3.403671in}{4.819156in}}%
\pgfpathlineto{\pgfqpoint{3.419104in}{4.816632in}}%
\pgfpathlineto{\pgfqpoint{3.434538in}{4.815565in}}%
\pgfpathlineto{\pgfqpoint{3.449972in}{4.815867in}}%
\pgfpathlineto{\pgfqpoint{3.465405in}{4.817437in}}%
\pgfpathlineto{\pgfqpoint{3.496273in}{4.823926in}}%
\pgfpathlineto{\pgfqpoint{3.527140in}{4.834101in}}%
\pgfpathlineto{\pgfqpoint{3.558007in}{4.847119in}}%
\pgfpathlineto{\pgfqpoint{3.588874in}{4.862348in}}%
\pgfpathlineto{\pgfqpoint{3.619742in}{4.879449in}}%
\pgfpathlineto{\pgfqpoint{3.650609in}{4.898386in}}%
\pgfpathlineto{\pgfqpoint{3.681476in}{4.919371in}}%
\pgfpathlineto{\pgfqpoint{3.712343in}{4.942756in}}%
\pgfpathlineto{\pgfqpoint{3.743211in}{4.968888in}}%
\pgfpathlineto{\pgfqpoint{3.774078in}{4.997946in}}%
\pgfpathlineto{\pgfqpoint{3.804945in}{5.029806in}}%
\pgfpathlineto{\pgfqpoint{3.851246in}{5.081579in}}%
\pgfpathlineto{\pgfqpoint{3.912981in}{5.151970in}}%
\pgfpathlineto{\pgfqpoint{3.943848in}{5.184182in}}%
\pgfpathlineto{\pgfqpoint{3.974715in}{5.212104in}}%
\pgfpathlineto{\pgfqpoint{3.990149in}{5.223969in}}%
\pgfpathlineto{\pgfqpoint{4.005583in}{5.234222in}}%
\pgfpathlineto{\pgfqpoint{4.021016in}{5.242742in}}%
\pgfpathlineto{\pgfqpoint{4.036450in}{5.249447in}}%
\pgfpathlineto{\pgfqpoint{4.051883in}{5.254295in}}%
\pgfpathlineto{\pgfqpoint{4.067317in}{5.257291in}}%
\pgfpathlineto{\pgfqpoint{4.082751in}{5.258482in}}%
\pgfpathlineto{\pgfqpoint{4.098184in}{5.257966in}}%
\pgfpathlineto{\pgfqpoint{4.113618in}{5.255883in}}%
\pgfpathlineto{\pgfqpoint{4.129052in}{5.252413in}}%
\pgfpathlineto{\pgfqpoint{4.159919in}{5.242225in}}%
\pgfpathlineto{\pgfqpoint{4.252521in}{5.205839in}}%
\pgfpathlineto{\pgfqpoint{4.267954in}{5.201965in}}%
\pgfpathlineto{\pgfqpoint{4.283388in}{5.199393in}}%
\pgfpathlineto{\pgfqpoint{4.298822in}{5.198302in}}%
\pgfpathlineto{\pgfqpoint{4.314255in}{5.198831in}}%
\pgfpathlineto{\pgfqpoint{4.329689in}{5.201070in}}%
\pgfpathlineto{\pgfqpoint{4.345122in}{5.205058in}}%
\pgfpathlineto{\pgfqpoint{4.360556in}{5.210784in}}%
\pgfpathlineto{\pgfqpoint{4.375990in}{5.218186in}}%
\pgfpathlineto{\pgfqpoint{4.391423in}{5.227154in}}%
\pgfpathlineto{\pgfqpoint{4.422291in}{5.249122in}}%
\pgfpathlineto{\pgfqpoint{4.453158in}{5.275002in}}%
\pgfpathlineto{\pgfqpoint{4.530326in}{5.342568in}}%
\pgfpathlineto{\pgfqpoint{4.561193in}{5.364970in}}%
\pgfpathlineto{\pgfqpoint{4.576627in}{5.374254in}}%
\pgfpathlineto{\pgfqpoint{4.592061in}{5.382047in}}%
\pgfpathlineto{\pgfqpoint{4.607494in}{5.388239in}}%
\pgfpathlineto{\pgfqpoint{4.622928in}{5.392764in}}%
\pgfpathlineto{\pgfqpoint{4.638362in}{5.395600in}}%
\pgfpathlineto{\pgfqpoint{4.653795in}{5.396767in}}%
\pgfpathlineto{\pgfqpoint{4.669229in}{5.396327in}}%
\pgfpathlineto{\pgfqpoint{4.669229in}{5.396327in}}%
\pgfusepath{stroke}%
\end{pgfscope}%
\begin{pgfscope}%
\pgfpathrectangle{\pgfqpoint{0.634105in}{3.881603in}}{\pgfqpoint{4.227273in}{2.800000in}} %
\pgfusepath{clip}%
\pgfsetrectcap%
\pgfsetroundjoin%
\pgfsetlinewidth{0.501875pt}%
\definecolor{currentstroke}{rgb}{1.000000,0.682749,0.366979}%
\pgfsetstrokecolor{currentstroke}%
\pgfsetdash{}{0pt}%
\pgfpathmoveto{\pgfqpoint{0.826254in}{5.653248in}}%
\pgfpathlineto{\pgfqpoint{0.841687in}{5.653496in}}%
\pgfpathlineto{\pgfqpoint{0.857121in}{5.649990in}}%
\pgfpathlineto{\pgfqpoint{0.872555in}{5.642561in}}%
\pgfpathlineto{\pgfqpoint{0.887988in}{5.631105in}}%
\pgfpathlineto{\pgfqpoint{0.903422in}{5.615593in}}%
\pgfpathlineto{\pgfqpoint{0.918855in}{5.596071in}}%
\pgfpathlineto{\pgfqpoint{0.934289in}{5.572662in}}%
\pgfpathlineto{\pgfqpoint{0.949723in}{5.545572in}}%
\pgfpathlineto{\pgfqpoint{0.965156in}{5.515079in}}%
\pgfpathlineto{\pgfqpoint{0.996024in}{5.445373in}}%
\pgfpathlineto{\pgfqpoint{1.026891in}{5.367161in}}%
\pgfpathlineto{\pgfqpoint{1.104059in}{5.164872in}}%
\pgfpathlineto{\pgfqpoint{1.134926in}{5.094465in}}%
\pgfpathlineto{\pgfqpoint{1.150360in}{5.063916in}}%
\pgfpathlineto{\pgfqpoint{1.165794in}{5.037157in}}%
\pgfpathlineto{\pgfqpoint{1.181227in}{5.014629in}}%
\pgfpathlineto{\pgfqpoint{1.196661in}{4.996705in}}%
\pgfpathlineto{\pgfqpoint{1.212095in}{4.983678in}}%
\pgfpathlineto{\pgfqpoint{1.227528in}{4.975760in}}%
\pgfpathlineto{\pgfqpoint{1.242962in}{4.973081in}}%
\pgfpathlineto{\pgfqpoint{1.258395in}{4.975677in}}%
\pgfpathlineto{\pgfqpoint{1.273829in}{4.983503in}}%
\pgfpathlineto{\pgfqpoint{1.289263in}{4.996422in}}%
\pgfpathlineto{\pgfqpoint{1.304696in}{5.014221in}}%
\pgfpathlineto{\pgfqpoint{1.320130in}{5.036604in}}%
\pgfpathlineto{\pgfqpoint{1.335564in}{5.063207in}}%
\pgfpathlineto{\pgfqpoint{1.350997in}{5.093604in}}%
\pgfpathlineto{\pgfqpoint{1.366431in}{5.127312in}}%
\pgfpathlineto{\pgfqpoint{1.397298in}{5.202526in}}%
\pgfpathlineto{\pgfqpoint{1.443599in}{5.326161in}}%
\pgfpathlineto{\pgfqpoint{1.489900in}{5.448683in}}%
\pgfpathlineto{\pgfqpoint{1.520767in}{5.522526in}}%
\pgfpathlineto{\pgfqpoint{1.536201in}{5.555722in}}%
\pgfpathlineto{\pgfqpoint{1.551634in}{5.585939in}}%
\pgfpathlineto{\pgfqpoint{1.567068in}{5.612878in}}%
\pgfpathlineto{\pgfqpoint{1.582502in}{5.636298in}}%
\pgfpathlineto{\pgfqpoint{1.597935in}{5.656018in}}%
\pgfpathlineto{\pgfqpoint{1.613369in}{5.671915in}}%
\pgfpathlineto{\pgfqpoint{1.628803in}{5.683927in}}%
\pgfpathlineto{\pgfqpoint{1.644236in}{5.692042in}}%
\pgfpathlineto{\pgfqpoint{1.659670in}{5.696305in}}%
\pgfpathlineto{\pgfqpoint{1.675104in}{5.696803in}}%
\pgfpathlineto{\pgfqpoint{1.690537in}{5.693668in}}%
\pgfpathlineto{\pgfqpoint{1.705971in}{5.687069in}}%
\pgfpathlineto{\pgfqpoint{1.721404in}{5.677207in}}%
\pgfpathlineto{\pgfqpoint{1.736838in}{5.664309in}}%
\pgfpathlineto{\pgfqpoint{1.752272in}{5.648626in}}%
\pgfpathlineto{\pgfqpoint{1.767705in}{5.630424in}}%
\pgfpathlineto{\pgfqpoint{1.783139in}{5.609981in}}%
\pgfpathlineto{\pgfqpoint{1.814006in}{5.563530in}}%
\pgfpathlineto{\pgfqpoint{1.844874in}{5.511616in}}%
\pgfpathlineto{\pgfqpoint{1.968343in}{5.294699in}}%
\pgfpathlineto{\pgfqpoint{1.999210in}{5.248386in}}%
\pgfpathlineto{\pgfqpoint{2.030077in}{5.208618in}}%
\pgfpathlineto{\pgfqpoint{2.045511in}{5.191566in}}%
\pgfpathlineto{\pgfqpoint{2.060944in}{5.176554in}}%
\pgfpathlineto{\pgfqpoint{2.076378in}{5.163659in}}%
\pgfpathlineto{\pgfqpoint{2.091812in}{5.152928in}}%
\pgfpathlineto{\pgfqpoint{2.107245in}{5.144373in}}%
\pgfpathlineto{\pgfqpoint{2.122679in}{5.137970in}}%
\pgfpathlineto{\pgfqpoint{2.138113in}{5.133660in}}%
\pgfpathlineto{\pgfqpoint{2.153546in}{5.131346in}}%
\pgfpathlineto{\pgfqpoint{2.168980in}{5.130894in}}%
\pgfpathlineto{\pgfqpoint{2.184414in}{5.132135in}}%
\pgfpathlineto{\pgfqpoint{2.199847in}{5.134866in}}%
\pgfpathlineto{\pgfqpoint{2.230714in}{5.143841in}}%
\pgfpathlineto{\pgfqpoint{2.277015in}{5.161914in}}%
\pgfpathlineto{\pgfqpoint{2.307883in}{5.173519in}}%
\pgfpathlineto{\pgfqpoint{2.338750in}{5.181994in}}%
\pgfpathlineto{\pgfqpoint{2.354184in}{5.184395in}}%
\pgfpathlineto{\pgfqpoint{2.369617in}{5.185274in}}%
\pgfpathlineto{\pgfqpoint{2.385051in}{5.184455in}}%
\pgfpathlineto{\pgfqpoint{2.400484in}{5.181804in}}%
\pgfpathlineto{\pgfqpoint{2.415918in}{5.177232in}}%
\pgfpathlineto{\pgfqpoint{2.431352in}{5.170701in}}%
\pgfpathlineto{\pgfqpoint{2.446785in}{5.162220in}}%
\pgfpathlineto{\pgfqpoint{2.462219in}{5.151852in}}%
\pgfpathlineto{\pgfqpoint{2.477653in}{5.139704in}}%
\pgfpathlineto{\pgfqpoint{2.508520in}{5.110740in}}%
\pgfpathlineto{\pgfqpoint{2.539387in}{5.077052in}}%
\pgfpathlineto{\pgfqpoint{2.616555in}{4.987461in}}%
\pgfpathlineto{\pgfqpoint{2.647423in}{4.956118in}}%
\pgfpathlineto{\pgfqpoint{2.662856in}{4.942584in}}%
\pgfpathlineto{\pgfqpoint{2.678290in}{4.930795in}}%
\pgfpathlineto{\pgfqpoint{2.693724in}{4.920944in}}%
\pgfpathlineto{\pgfqpoint{2.709157in}{4.913184in}}%
\pgfpathlineto{\pgfqpoint{2.724591in}{4.907622in}}%
\pgfpathlineto{\pgfqpoint{2.740024in}{4.904319in}}%
\pgfpathlineto{\pgfqpoint{2.755458in}{4.903288in}}%
\pgfpathlineto{\pgfqpoint{2.770892in}{4.904492in}}%
\pgfpathlineto{\pgfqpoint{2.786325in}{4.907848in}}%
\pgfpathlineto{\pgfqpoint{2.801759in}{4.913229in}}%
\pgfpathlineto{\pgfqpoint{2.817193in}{4.920463in}}%
\pgfpathlineto{\pgfqpoint{2.832626in}{4.929341in}}%
\pgfpathlineto{\pgfqpoint{2.863493in}{4.951028in}}%
\pgfpathlineto{\pgfqpoint{2.909794in}{4.989007in}}%
\pgfpathlineto{\pgfqpoint{2.940662in}{5.014273in}}%
\pgfpathlineto{\pgfqpoint{2.971529in}{5.036581in}}%
\pgfpathlineto{\pgfqpoint{2.986963in}{5.045921in}}%
\pgfpathlineto{\pgfqpoint{3.002396in}{5.053728in}}%
\pgfpathlineto{\pgfqpoint{3.017830in}{5.059802in}}%
\pgfpathlineto{\pgfqpoint{3.033263in}{5.063981in}}%
\pgfpathlineto{\pgfqpoint{3.048697in}{5.066147in}}%
\pgfpathlineto{\pgfqpoint{3.064131in}{5.066225in}}%
\pgfpathlineto{\pgfqpoint{3.079564in}{5.064187in}}%
\pgfpathlineto{\pgfqpoint{3.094998in}{5.060050in}}%
\pgfpathlineto{\pgfqpoint{3.110432in}{5.053876in}}%
\pgfpathlineto{\pgfqpoint{3.125865in}{5.045772in}}%
\pgfpathlineto{\pgfqpoint{3.141299in}{5.035882in}}%
\pgfpathlineto{\pgfqpoint{3.156733in}{5.024387in}}%
\pgfpathlineto{\pgfqpoint{3.187600in}{4.997459in}}%
\pgfpathlineto{\pgfqpoint{3.218467in}{4.966948in}}%
\pgfpathlineto{\pgfqpoint{3.280202in}{4.903807in}}%
\pgfpathlineto{\pgfqpoint{3.311069in}{4.875259in}}%
\pgfpathlineto{\pgfqpoint{3.341936in}{4.850933in}}%
\pgfpathlineto{\pgfqpoint{3.357370in}{4.840713in}}%
\pgfpathlineto{\pgfqpoint{3.372803in}{4.831915in}}%
\pgfpathlineto{\pgfqpoint{3.388237in}{4.824596in}}%
\pgfpathlineto{\pgfqpoint{3.403671in}{4.818780in}}%
\pgfpathlineto{\pgfqpoint{3.419104in}{4.814463in}}%
\pgfpathlineto{\pgfqpoint{3.434538in}{4.811617in}}%
\pgfpathlineto{\pgfqpoint{3.449972in}{4.810193in}}%
\pgfpathlineto{\pgfqpoint{3.465405in}{4.810125in}}%
\pgfpathlineto{\pgfqpoint{3.480839in}{4.811336in}}%
\pgfpathlineto{\pgfqpoint{3.511706in}{4.817253in}}%
\pgfpathlineto{\pgfqpoint{3.542573in}{4.827266in}}%
\pgfpathlineto{\pgfqpoint{3.573441in}{4.840779in}}%
\pgfpathlineto{\pgfqpoint{3.604308in}{4.857375in}}%
\pgfpathlineto{\pgfqpoint{3.635175in}{4.876839in}}%
\pgfpathlineto{\pgfqpoint{3.666043in}{4.899140in}}%
\pgfpathlineto{\pgfqpoint{3.696910in}{4.924355in}}%
\pgfpathlineto{\pgfqpoint{3.727777in}{4.952557in}}%
\pgfpathlineto{\pgfqpoint{3.758644in}{4.983694in}}%
\pgfpathlineto{\pgfqpoint{3.789512in}{5.017473in}}%
\pgfpathlineto{\pgfqpoint{3.835813in}{5.071667in}}%
\pgfpathlineto{\pgfqpoint{3.897547in}{5.144654in}}%
\pgfpathlineto{\pgfqpoint{3.928414in}{5.178177in}}%
\pgfpathlineto{\pgfqpoint{3.959282in}{5.207626in}}%
\pgfpathlineto{\pgfqpoint{3.974715in}{5.220373in}}%
\pgfpathlineto{\pgfqpoint{3.990149in}{5.231600in}}%
\pgfpathlineto{\pgfqpoint{4.005583in}{5.241190in}}%
\pgfpathlineto{\pgfqpoint{4.021016in}{5.249055in}}%
\pgfpathlineto{\pgfqpoint{4.036450in}{5.255144in}}%
\pgfpathlineto{\pgfqpoint{4.051883in}{5.259443in}}%
\pgfpathlineto{\pgfqpoint{4.067317in}{5.261977in}}%
\pgfpathlineto{\pgfqpoint{4.082751in}{5.262812in}}%
\pgfpathlineto{\pgfqpoint{4.098184in}{5.262053in}}%
\pgfpathlineto{\pgfqpoint{4.113618in}{5.259843in}}%
\pgfpathlineto{\pgfqpoint{4.144485in}{5.251804in}}%
\pgfpathlineto{\pgfqpoint{4.175352in}{5.240446in}}%
\pgfpathlineto{\pgfqpoint{4.237087in}{5.216109in}}%
\pgfpathlineto{\pgfqpoint{4.267954in}{5.207325in}}%
\pgfpathlineto{\pgfqpoint{4.283388in}{5.204581in}}%
\pgfpathlineto{\pgfqpoint{4.298822in}{5.203167in}}%
\pgfpathlineto{\pgfqpoint{4.314255in}{5.203211in}}%
\pgfpathlineto{\pgfqpoint{4.329689in}{5.204799in}}%
\pgfpathlineto{\pgfqpoint{4.345122in}{5.207976in}}%
\pgfpathlineto{\pgfqpoint{4.360556in}{5.212739in}}%
\pgfpathlineto{\pgfqpoint{4.375990in}{5.219044in}}%
\pgfpathlineto{\pgfqpoint{4.391423in}{5.226803in}}%
\pgfpathlineto{\pgfqpoint{4.422291in}{5.246134in}}%
\pgfpathlineto{\pgfqpoint{4.453158in}{5.269305in}}%
\pgfpathlineto{\pgfqpoint{4.545760in}{5.342824in}}%
\pgfpathlineto{\pgfqpoint{4.576627in}{5.362253in}}%
\pgfpathlineto{\pgfqpoint{4.592061in}{5.370137in}}%
\pgfpathlineto{\pgfqpoint{4.607494in}{5.376639in}}%
\pgfpathlineto{\pgfqpoint{4.622928in}{5.381684in}}%
\pgfpathlineto{\pgfqpoint{4.638362in}{5.385233in}}%
\pgfpathlineto{\pgfqpoint{4.653795in}{5.387290in}}%
\pgfpathlineto{\pgfqpoint{4.669229in}{5.387892in}}%
\pgfpathlineto{\pgfqpoint{4.669229in}{5.387892in}}%
\pgfusepath{stroke}%
\end{pgfscope}%
\begin{pgfscope}%
\pgfpathrectangle{\pgfqpoint{0.634105in}{3.881603in}}{\pgfqpoint{4.227273in}{2.800000in}} %
\pgfusepath{clip}%
\pgfsetrectcap%
\pgfsetroundjoin%
\pgfsetlinewidth{0.501875pt}%
\definecolor{currentstroke}{rgb}{1.000000,0.587785,0.309017}%
\pgfsetstrokecolor{currentstroke}%
\pgfsetdash{}{0pt}%
\pgfpathmoveto{\pgfqpoint{0.826254in}{5.656812in}}%
\pgfpathlineto{\pgfqpoint{0.841687in}{5.656570in}}%
\pgfpathlineto{\pgfqpoint{0.857121in}{5.652395in}}%
\pgfpathlineto{\pgfqpoint{0.872555in}{5.644140in}}%
\pgfpathlineto{\pgfqpoint{0.887988in}{5.631731in}}%
\pgfpathlineto{\pgfqpoint{0.903422in}{5.615163in}}%
\pgfpathlineto{\pgfqpoint{0.918855in}{5.594511in}}%
\pgfpathlineto{\pgfqpoint{0.934289in}{5.569924in}}%
\pgfpathlineto{\pgfqpoint{0.949723in}{5.541631in}}%
\pgfpathlineto{\pgfqpoint{0.965156in}{5.509934in}}%
\pgfpathlineto{\pgfqpoint{0.996024in}{5.437886in}}%
\pgfpathlineto{\pgfqpoint{1.026891in}{5.357505in}}%
\pgfpathlineto{\pgfqpoint{1.104059in}{5.150752in}}%
\pgfpathlineto{\pgfqpoint{1.134926in}{5.078879in}}%
\pgfpathlineto{\pgfqpoint{1.150360in}{5.047642in}}%
\pgfpathlineto{\pgfqpoint{1.165794in}{5.020219in}}%
\pgfpathlineto{\pgfqpoint{1.181227in}{4.997048in}}%
\pgfpathlineto{\pgfqpoint{1.196661in}{4.978500in}}%
\pgfpathlineto{\pgfqpoint{1.212095in}{4.964872in}}%
\pgfpathlineto{\pgfqpoint{1.227528in}{4.956381in}}%
\pgfpathlineto{\pgfqpoint{1.242962in}{4.953162in}}%
\pgfpathlineto{\pgfqpoint{1.258395in}{4.955264in}}%
\pgfpathlineto{\pgfqpoint{1.273829in}{4.962651in}}%
\pgfpathlineto{\pgfqpoint{1.289263in}{4.975203in}}%
\pgfpathlineto{\pgfqpoint{1.304696in}{4.992719in}}%
\pgfpathlineto{\pgfqpoint{1.320130in}{5.014921in}}%
\pgfpathlineto{\pgfqpoint{1.335564in}{5.041459in}}%
\pgfpathlineto{\pgfqpoint{1.350997in}{5.071920in}}%
\pgfpathlineto{\pgfqpoint{1.366431in}{5.105833in}}%
\pgfpathlineto{\pgfqpoint{1.397298in}{5.181914in}}%
\pgfpathlineto{\pgfqpoint{1.443599in}{5.308011in}}%
\pgfpathlineto{\pgfqpoint{1.489900in}{5.434165in}}%
\pgfpathlineto{\pgfqpoint{1.520767in}{5.510796in}}%
\pgfpathlineto{\pgfqpoint{1.536201in}{5.545404in}}%
\pgfpathlineto{\pgfqpoint{1.551634in}{5.576998in}}%
\pgfpathlineto{\pgfqpoint{1.567068in}{5.605241in}}%
\pgfpathlineto{\pgfqpoint{1.582502in}{5.629856in}}%
\pgfpathlineto{\pgfqpoint{1.597935in}{5.650623in}}%
\pgfpathlineto{\pgfqpoint{1.613369in}{5.667389in}}%
\pgfpathlineto{\pgfqpoint{1.628803in}{5.680057in}}%
\pgfpathlineto{\pgfqpoint{1.644236in}{5.688591in}}%
\pgfpathlineto{\pgfqpoint{1.659670in}{5.693013in}}%
\pgfpathlineto{\pgfqpoint{1.675104in}{5.693394in}}%
\pgfpathlineto{\pgfqpoint{1.690537in}{5.689858in}}%
\pgfpathlineto{\pgfqpoint{1.705971in}{5.682571in}}%
\pgfpathlineto{\pgfqpoint{1.721404in}{5.671740in}}%
\pgfpathlineto{\pgfqpoint{1.736838in}{5.657605in}}%
\pgfpathlineto{\pgfqpoint{1.752272in}{5.640433in}}%
\pgfpathlineto{\pgfqpoint{1.767705in}{5.620520in}}%
\pgfpathlineto{\pgfqpoint{1.783139in}{5.598174in}}%
\pgfpathlineto{\pgfqpoint{1.814006in}{5.547502in}}%
\pgfpathlineto{\pgfqpoint{1.844874in}{5.491110in}}%
\pgfpathlineto{\pgfqpoint{1.937475in}{5.314420in}}%
\pgfpathlineto{\pgfqpoint{1.968343in}{5.261265in}}%
\pgfpathlineto{\pgfqpoint{1.999210in}{5.214485in}}%
\pgfpathlineto{\pgfqpoint{2.014644in}{5.193996in}}%
\pgfpathlineto{\pgfqpoint{2.030077in}{5.175658in}}%
\pgfpathlineto{\pgfqpoint{2.045511in}{5.159595in}}%
\pgfpathlineto{\pgfqpoint{2.060944in}{5.145900in}}%
\pgfpathlineto{\pgfqpoint{2.076378in}{5.134627in}}%
\pgfpathlineto{\pgfqpoint{2.091812in}{5.125795in}}%
\pgfpathlineto{\pgfqpoint{2.107245in}{5.119385in}}%
\pgfpathlineto{\pgfqpoint{2.122679in}{5.115335in}}%
\pgfpathlineto{\pgfqpoint{2.138113in}{5.113545in}}%
\pgfpathlineto{\pgfqpoint{2.153546in}{5.113878in}}%
\pgfpathlineto{\pgfqpoint{2.168980in}{5.116154in}}%
\pgfpathlineto{\pgfqpoint{2.184414in}{5.120161in}}%
\pgfpathlineto{\pgfqpoint{2.199847in}{5.125652in}}%
\pgfpathlineto{\pgfqpoint{2.230714in}{5.139960in}}%
\pgfpathlineto{\pgfqpoint{2.307883in}{5.180216in}}%
\pgfpathlineto{\pgfqpoint{2.323316in}{5.186409in}}%
\pgfpathlineto{\pgfqpoint{2.338750in}{5.191274in}}%
\pgfpathlineto{\pgfqpoint{2.354184in}{5.194554in}}%
\pgfpathlineto{\pgfqpoint{2.369617in}{5.196032in}}%
\pgfpathlineto{\pgfqpoint{2.385051in}{5.195533in}}%
\pgfpathlineto{\pgfqpoint{2.400484in}{5.192929in}}%
\pgfpathlineto{\pgfqpoint{2.415918in}{5.188141in}}%
\pgfpathlineto{\pgfqpoint{2.431352in}{5.181142in}}%
\pgfpathlineto{\pgfqpoint{2.446785in}{5.171960in}}%
\pgfpathlineto{\pgfqpoint{2.462219in}{5.160675in}}%
\pgfpathlineto{\pgfqpoint{2.477653in}{5.147416in}}%
\pgfpathlineto{\pgfqpoint{2.493086in}{5.132364in}}%
\pgfpathlineto{\pgfqpoint{2.523954in}{5.097810in}}%
\pgfpathlineto{\pgfqpoint{2.570254in}{5.039231in}}%
\pgfpathlineto{\pgfqpoint{2.616555in}{4.980612in}}%
\pgfpathlineto{\pgfqpoint{2.647423in}{4.946074in}}%
\pgfpathlineto{\pgfqpoint{2.662856in}{4.931095in}}%
\pgfpathlineto{\pgfqpoint{2.678290in}{4.917989in}}%
\pgfpathlineto{\pgfqpoint{2.693724in}{4.906969in}}%
\pgfpathlineto{\pgfqpoint{2.709157in}{4.898202in}}%
\pgfpathlineto{\pgfqpoint{2.724591in}{4.891811in}}%
\pgfpathlineto{\pgfqpoint{2.740024in}{4.887869in}}%
\pgfpathlineto{\pgfqpoint{2.755458in}{4.886397in}}%
\pgfpathlineto{\pgfqpoint{2.770892in}{4.887368in}}%
\pgfpathlineto{\pgfqpoint{2.786325in}{4.890703in}}%
\pgfpathlineto{\pgfqpoint{2.801759in}{4.896277in}}%
\pgfpathlineto{\pgfqpoint{2.817193in}{4.903919in}}%
\pgfpathlineto{\pgfqpoint{2.832626in}{4.913416in}}%
\pgfpathlineto{\pgfqpoint{2.863493in}{4.936954in}}%
\pgfpathlineto{\pgfqpoint{2.894361in}{4.964565in}}%
\pgfpathlineto{\pgfqpoint{2.956095in}{5.021437in}}%
\pgfpathlineto{\pgfqpoint{2.986963in}{5.045437in}}%
\pgfpathlineto{\pgfqpoint{3.002396in}{5.055299in}}%
\pgfpathlineto{\pgfqpoint{3.017830in}{5.063421in}}%
\pgfpathlineto{\pgfqpoint{3.033263in}{5.069610in}}%
\pgfpathlineto{\pgfqpoint{3.048697in}{5.073717in}}%
\pgfpathlineto{\pgfqpoint{3.064131in}{5.075638in}}%
\pgfpathlineto{\pgfqpoint{3.079564in}{5.075316in}}%
\pgfpathlineto{\pgfqpoint{3.094998in}{5.072738in}}%
\pgfpathlineto{\pgfqpoint{3.110432in}{5.067942in}}%
\pgfpathlineto{\pgfqpoint{3.125865in}{5.061007in}}%
\pgfpathlineto{\pgfqpoint{3.141299in}{5.052059in}}%
\pgfpathlineto{\pgfqpoint{3.156733in}{5.041258in}}%
\pgfpathlineto{\pgfqpoint{3.187600in}{5.014922in}}%
\pgfpathlineto{\pgfqpoint{3.218467in}{4.983895in}}%
\pgfpathlineto{\pgfqpoint{3.311069in}{4.884573in}}%
\pgfpathlineto{\pgfqpoint{3.341936in}{4.856256in}}%
\pgfpathlineto{\pgfqpoint{3.372803in}{4.832979in}}%
\pgfpathlineto{\pgfqpoint{3.388237in}{4.823531in}}%
\pgfpathlineto{\pgfqpoint{3.403671in}{4.815641in}}%
\pgfpathlineto{\pgfqpoint{3.419104in}{4.809346in}}%
\pgfpathlineto{\pgfqpoint{3.434538in}{4.804655in}}%
\pgfpathlineto{\pgfqpoint{3.449972in}{4.801553in}}%
\pgfpathlineto{\pgfqpoint{3.465405in}{4.800005in}}%
\pgfpathlineto{\pgfqpoint{3.480839in}{4.799960in}}%
\pgfpathlineto{\pgfqpoint{3.496273in}{4.801351in}}%
\pgfpathlineto{\pgfqpoint{3.511706in}{4.804107in}}%
\pgfpathlineto{\pgfqpoint{3.527140in}{4.808149in}}%
\pgfpathlineto{\pgfqpoint{3.558007in}{4.819775in}}%
\pgfpathlineto{\pgfqpoint{3.588874in}{4.835634in}}%
\pgfpathlineto{\pgfqpoint{3.619742in}{4.855228in}}%
\pgfpathlineto{\pgfqpoint{3.650609in}{4.878173in}}%
\pgfpathlineto{\pgfqpoint{3.681476in}{4.904187in}}%
\pgfpathlineto{\pgfqpoint{3.712343in}{4.933022in}}%
\pgfpathlineto{\pgfqpoint{3.743211in}{4.964390in}}%
\pgfpathlineto{\pgfqpoint{3.789512in}{5.015224in}}%
\pgfpathlineto{\pgfqpoint{3.897547in}{5.137267in}}%
\pgfpathlineto{\pgfqpoint{3.928414in}{5.167838in}}%
\pgfpathlineto{\pgfqpoint{3.959282in}{5.194157in}}%
\pgfpathlineto{\pgfqpoint{3.974715in}{5.205376in}}%
\pgfpathlineto{\pgfqpoint{3.990149in}{5.215152in}}%
\pgfpathlineto{\pgfqpoint{4.005583in}{5.223404in}}%
\pgfpathlineto{\pgfqpoint{4.021016in}{5.230079in}}%
\pgfpathlineto{\pgfqpoint{4.036450in}{5.235153in}}%
\pgfpathlineto{\pgfqpoint{4.051883in}{5.238635in}}%
\pgfpathlineto{\pgfqpoint{4.067317in}{5.240568in}}%
\pgfpathlineto{\pgfqpoint{4.082751in}{5.241028in}}%
\pgfpathlineto{\pgfqpoint{4.098184in}{5.240122in}}%
\pgfpathlineto{\pgfqpoint{4.129052in}{5.234794in}}%
\pgfpathlineto{\pgfqpoint{4.159919in}{5.226004in}}%
\pgfpathlineto{\pgfqpoint{4.237087in}{5.200559in}}%
\pgfpathlineto{\pgfqpoint{4.267954in}{5.193800in}}%
\pgfpathlineto{\pgfqpoint{4.283388in}{5.191952in}}%
\pgfpathlineto{\pgfqpoint{4.298822in}{5.191321in}}%
\pgfpathlineto{\pgfqpoint{4.314255in}{5.192013in}}%
\pgfpathlineto{\pgfqpoint{4.329689in}{5.194100in}}%
\pgfpathlineto{\pgfqpoint{4.345122in}{5.197614in}}%
\pgfpathlineto{\pgfqpoint{4.360556in}{5.202550in}}%
\pgfpathlineto{\pgfqpoint{4.375990in}{5.208864in}}%
\pgfpathlineto{\pgfqpoint{4.406857in}{5.225267in}}%
\pgfpathlineto{\pgfqpoint{4.437724in}{5.245775in}}%
\pgfpathlineto{\pgfqpoint{4.484025in}{5.280910in}}%
\pgfpathlineto{\pgfqpoint{4.530326in}{5.315906in}}%
\pgfpathlineto{\pgfqpoint{4.561193in}{5.336192in}}%
\pgfpathlineto{\pgfqpoint{4.592061in}{5.352307in}}%
\pgfpathlineto{\pgfqpoint{4.607494in}{5.358464in}}%
\pgfpathlineto{\pgfqpoint{4.622928in}{5.363235in}}%
\pgfpathlineto{\pgfqpoint{4.638362in}{5.366577in}}%
\pgfpathlineto{\pgfqpoint{4.653795in}{5.368478in}}%
\pgfpathlineto{\pgfqpoint{4.669229in}{5.368962in}}%
\pgfpathlineto{\pgfqpoint{4.669229in}{5.368962in}}%
\pgfusepath{stroke}%
\end{pgfscope}%
\begin{pgfscope}%
\pgfpathrectangle{\pgfqpoint{0.634105in}{3.881603in}}{\pgfqpoint{4.227273in}{2.800000in}} %
\pgfusepath{clip}%
\pgfsetrectcap%
\pgfsetroundjoin%
\pgfsetlinewidth{0.501875pt}%
\definecolor{currentstroke}{rgb}{1.000000,0.473094,0.243914}%
\pgfsetstrokecolor{currentstroke}%
\pgfsetdash{}{0pt}%
\pgfpathmoveto{\pgfqpoint{0.826254in}{5.657672in}}%
\pgfpathlineto{\pgfqpoint{0.841687in}{5.656413in}}%
\pgfpathlineto{\pgfqpoint{0.857121in}{5.651272in}}%
\pgfpathlineto{\pgfqpoint{0.872555in}{5.642120in}}%
\pgfpathlineto{\pgfqpoint{0.887988in}{5.628897in}}%
\pgfpathlineto{\pgfqpoint{0.903422in}{5.611616in}}%
\pgfpathlineto{\pgfqpoint{0.918855in}{5.590362in}}%
\pgfpathlineto{\pgfqpoint{0.934289in}{5.565298in}}%
\pgfpathlineto{\pgfqpoint{0.949723in}{5.536657in}}%
\pgfpathlineto{\pgfqpoint{0.965156in}{5.504744in}}%
\pgfpathlineto{\pgfqpoint{0.996024in}{5.432647in}}%
\pgfpathlineto{\pgfqpoint{1.026891in}{5.352671in}}%
\pgfpathlineto{\pgfqpoint{1.088625in}{5.187119in}}%
\pgfpathlineto{\pgfqpoint{1.119493in}{5.111341in}}%
\pgfpathlineto{\pgfqpoint{1.134926in}{5.077311in}}%
\pgfpathlineto{\pgfqpoint{1.150360in}{5.046586in}}%
\pgfpathlineto{\pgfqpoint{1.165794in}{5.019657in}}%
\pgfpathlineto{\pgfqpoint{1.181227in}{4.996959in}}%
\pgfpathlineto{\pgfqpoint{1.196661in}{4.978858in}}%
\pgfpathlineto{\pgfqpoint{1.212095in}{4.965648in}}%
\pgfpathlineto{\pgfqpoint{1.227528in}{4.957542in}}%
\pgfpathlineto{\pgfqpoint{1.242962in}{4.954671in}}%
\pgfpathlineto{\pgfqpoint{1.258395in}{4.957076in}}%
\pgfpathlineto{\pgfqpoint{1.273829in}{4.964714in}}%
\pgfpathlineto{\pgfqpoint{1.289263in}{4.977457in}}%
\pgfpathlineto{\pgfqpoint{1.304696in}{4.995092in}}%
\pgfpathlineto{\pgfqpoint{1.320130in}{5.017331in}}%
\pgfpathlineto{\pgfqpoint{1.335564in}{5.043814in}}%
\pgfpathlineto{\pgfqpoint{1.350997in}{5.074120in}}%
\pgfpathlineto{\pgfqpoint{1.366431in}{5.107772in}}%
\pgfpathlineto{\pgfqpoint{1.397298in}{5.183011in}}%
\pgfpathlineto{\pgfqpoint{1.443599in}{5.307162in}}%
\pgfpathlineto{\pgfqpoint{1.489900in}{5.431027in}}%
\pgfpathlineto{\pgfqpoint{1.520767in}{5.506329in}}%
\pgfpathlineto{\pgfqpoint{1.536201in}{5.540426in}}%
\pgfpathlineto{\pgfqpoint{1.551634in}{5.571647in}}%
\pgfpathlineto{\pgfqpoint{1.567068in}{5.599678in}}%
\pgfpathlineto{\pgfqpoint{1.582502in}{5.624256in}}%
\pgfpathlineto{\pgfqpoint{1.597935in}{5.645172in}}%
\pgfpathlineto{\pgfqpoint{1.613369in}{5.662273in}}%
\pgfpathlineto{\pgfqpoint{1.628803in}{5.675460in}}%
\pgfpathlineto{\pgfqpoint{1.644236in}{5.684685in}}%
\pgfpathlineto{\pgfqpoint{1.659670in}{5.689954in}}%
\pgfpathlineto{\pgfqpoint{1.675104in}{5.691321in}}%
\pgfpathlineto{\pgfqpoint{1.690537in}{5.688887in}}%
\pgfpathlineto{\pgfqpoint{1.705971in}{5.682797in}}%
\pgfpathlineto{\pgfqpoint{1.721404in}{5.673234in}}%
\pgfpathlineto{\pgfqpoint{1.736838in}{5.660418in}}%
\pgfpathlineto{\pgfqpoint{1.752272in}{5.644600in}}%
\pgfpathlineto{\pgfqpoint{1.767705in}{5.626057in}}%
\pgfpathlineto{\pgfqpoint{1.783139in}{5.605089in}}%
\pgfpathlineto{\pgfqpoint{1.814006in}{5.557149in}}%
\pgfpathlineto{\pgfqpoint{1.844874in}{5.503410in}}%
\pgfpathlineto{\pgfqpoint{1.952909in}{5.306861in}}%
\pgfpathlineto{\pgfqpoint{1.983776in}{5.257610in}}%
\pgfpathlineto{\pgfqpoint{2.014644in}{5.214695in}}%
\pgfpathlineto{\pgfqpoint{2.030077in}{5.196009in}}%
\pgfpathlineto{\pgfqpoint{2.045511in}{5.179333in}}%
\pgfpathlineto{\pgfqpoint{2.060944in}{5.164758in}}%
\pgfpathlineto{\pgfqpoint{2.076378in}{5.152346in}}%
\pgfpathlineto{\pgfqpoint{2.091812in}{5.142130in}}%
\pgfpathlineto{\pgfqpoint{2.107245in}{5.134115in}}%
\pgfpathlineto{\pgfqpoint{2.122679in}{5.128271in}}%
\pgfpathlineto{\pgfqpoint{2.138113in}{5.124534in}}%
\pgfpathlineto{\pgfqpoint{2.153546in}{5.122806in}}%
\pgfpathlineto{\pgfqpoint{2.168980in}{5.122954in}}%
\pgfpathlineto{\pgfqpoint{2.184414in}{5.124807in}}%
\pgfpathlineto{\pgfqpoint{2.199847in}{5.128163in}}%
\pgfpathlineto{\pgfqpoint{2.230714in}{5.138413in}}%
\pgfpathlineto{\pgfqpoint{2.323316in}{5.176312in}}%
\pgfpathlineto{\pgfqpoint{2.338750in}{5.180425in}}%
\pgfpathlineto{\pgfqpoint{2.354184in}{5.183157in}}%
\pgfpathlineto{\pgfqpoint{2.369617in}{5.184291in}}%
\pgfpathlineto{\pgfqpoint{2.385051in}{5.183650in}}%
\pgfpathlineto{\pgfqpoint{2.400484in}{5.181106in}}%
\pgfpathlineto{\pgfqpoint{2.415918in}{5.176581in}}%
\pgfpathlineto{\pgfqpoint{2.431352in}{5.170048in}}%
\pgfpathlineto{\pgfqpoint{2.446785in}{5.161535in}}%
\pgfpathlineto{\pgfqpoint{2.462219in}{5.151121in}}%
\pgfpathlineto{\pgfqpoint{2.477653in}{5.138934in}}%
\pgfpathlineto{\pgfqpoint{2.508520in}{5.109973in}}%
\pgfpathlineto{\pgfqpoint{2.539387in}{5.076481in}}%
\pgfpathlineto{\pgfqpoint{2.616555in}{4.988262in}}%
\pgfpathlineto{\pgfqpoint{2.647423in}{4.957594in}}%
\pgfpathlineto{\pgfqpoint{2.662856in}{4.944362in}}%
\pgfpathlineto{\pgfqpoint{2.678290in}{4.932833in}}%
\pgfpathlineto{\pgfqpoint{2.693724in}{4.923191in}}%
\pgfpathlineto{\pgfqpoint{2.709157in}{4.915585in}}%
\pgfpathlineto{\pgfqpoint{2.724591in}{4.910121in}}%
\pgfpathlineto{\pgfqpoint{2.740024in}{4.906867in}}%
\pgfpathlineto{\pgfqpoint{2.755458in}{4.905847in}}%
\pgfpathlineto{\pgfqpoint{2.770892in}{4.907037in}}%
\pgfpathlineto{\pgfqpoint{2.786325in}{4.910372in}}%
\pgfpathlineto{\pgfqpoint{2.801759in}{4.915742in}}%
\pgfpathlineto{\pgfqpoint{2.817193in}{4.922994in}}%
\pgfpathlineto{\pgfqpoint{2.832626in}{4.931936in}}%
\pgfpathlineto{\pgfqpoint{2.863493in}{4.953934in}}%
\pgfpathlineto{\pgfqpoint{2.894361in}{4.979544in}}%
\pgfpathlineto{\pgfqpoint{2.940662in}{5.019165in}}%
\pgfpathlineto{\pgfqpoint{2.971529in}{5.042579in}}%
\pgfpathlineto{\pgfqpoint{2.986963in}{5.052478in}}%
\pgfpathlineto{\pgfqpoint{3.002396in}{5.060829in}}%
\pgfpathlineto{\pgfqpoint{3.017830in}{5.067417in}}%
\pgfpathlineto{\pgfqpoint{3.033263in}{5.072071in}}%
\pgfpathlineto{\pgfqpoint{3.048697in}{5.074663in}}%
\pgfpathlineto{\pgfqpoint{3.064131in}{5.075116in}}%
\pgfpathlineto{\pgfqpoint{3.079564in}{5.073398in}}%
\pgfpathlineto{\pgfqpoint{3.094998in}{5.069529in}}%
\pgfpathlineto{\pgfqpoint{3.110432in}{5.063571in}}%
\pgfpathlineto{\pgfqpoint{3.125865in}{5.055633in}}%
\pgfpathlineto{\pgfqpoint{3.141299in}{5.045861in}}%
\pgfpathlineto{\pgfqpoint{3.156733in}{5.034436in}}%
\pgfpathlineto{\pgfqpoint{3.187600in}{5.007487in}}%
\pgfpathlineto{\pgfqpoint{3.218467in}{4.976698in}}%
\pgfpathlineto{\pgfqpoint{3.295635in}{4.896427in}}%
\pgfpathlineto{\pgfqpoint{3.326503in}{4.867973in}}%
\pgfpathlineto{\pgfqpoint{3.357370in}{4.843858in}}%
\pgfpathlineto{\pgfqpoint{3.372803in}{4.833746in}}%
\pgfpathlineto{\pgfqpoint{3.388237in}{4.825047in}}%
\pgfpathlineto{\pgfqpoint{3.403671in}{4.817817in}}%
\pgfpathlineto{\pgfqpoint{3.419104in}{4.812085in}}%
\pgfpathlineto{\pgfqpoint{3.434538in}{4.807859in}}%
\pgfpathlineto{\pgfqpoint{3.449972in}{4.805125in}}%
\pgfpathlineto{\pgfqpoint{3.465405in}{4.803851in}}%
\pgfpathlineto{\pgfqpoint{3.480839in}{4.803988in}}%
\pgfpathlineto{\pgfqpoint{3.496273in}{4.805479in}}%
\pgfpathlineto{\pgfqpoint{3.511706in}{4.808255in}}%
\pgfpathlineto{\pgfqpoint{3.542573in}{4.817371in}}%
\pgfpathlineto{\pgfqpoint{3.573441in}{4.830760in}}%
\pgfpathlineto{\pgfqpoint{3.604308in}{4.847913in}}%
\pgfpathlineto{\pgfqpoint{3.635175in}{4.868453in}}%
\pgfpathlineto{\pgfqpoint{3.666043in}{4.892144in}}%
\pgfpathlineto{\pgfqpoint{3.696910in}{4.918853in}}%
\pgfpathlineto{\pgfqpoint{3.727777in}{4.948470in}}%
\pgfpathlineto{\pgfqpoint{3.758644in}{4.980790in}}%
\pgfpathlineto{\pgfqpoint{3.804945in}{5.033371in}}%
\pgfpathlineto{\pgfqpoint{3.897547in}{5.141896in}}%
\pgfpathlineto{\pgfqpoint{3.928414in}{5.174196in}}%
\pgfpathlineto{\pgfqpoint{3.959282in}{5.202124in}}%
\pgfpathlineto{\pgfqpoint{3.974715in}{5.214054in}}%
\pgfpathlineto{\pgfqpoint{3.990149in}{5.224462in}}%
\pgfpathlineto{\pgfqpoint{4.005583in}{5.233256in}}%
\pgfpathlineto{\pgfqpoint{4.021016in}{5.240376in}}%
\pgfpathlineto{\pgfqpoint{4.036450in}{5.245798in}}%
\pgfpathlineto{\pgfqpoint{4.051883in}{5.249528in}}%
\pgfpathlineto{\pgfqpoint{4.067317in}{5.251611in}}%
\pgfpathlineto{\pgfqpoint{4.082751in}{5.252124in}}%
\pgfpathlineto{\pgfqpoint{4.098184in}{5.251175in}}%
\pgfpathlineto{\pgfqpoint{4.113618in}{5.248906in}}%
\pgfpathlineto{\pgfqpoint{4.144485in}{5.241092in}}%
\pgfpathlineto{\pgfqpoint{4.175352in}{5.230266in}}%
\pgfpathlineto{\pgfqpoint{4.237087in}{5.206933in}}%
\pgfpathlineto{\pgfqpoint{4.267954in}{5.198145in}}%
\pgfpathlineto{\pgfqpoint{4.283388in}{5.195206in}}%
\pgfpathlineto{\pgfqpoint{4.298822in}{5.193463in}}%
\pgfpathlineto{\pgfqpoint{4.314255in}{5.193053in}}%
\pgfpathlineto{\pgfqpoint{4.329689in}{5.194076in}}%
\pgfpathlineto{\pgfqpoint{4.345122in}{5.196601in}}%
\pgfpathlineto{\pgfqpoint{4.360556in}{5.200654in}}%
\pgfpathlineto{\pgfqpoint{4.375990in}{5.206225in}}%
\pgfpathlineto{\pgfqpoint{4.391423in}{5.213258in}}%
\pgfpathlineto{\pgfqpoint{4.422291in}{5.231308in}}%
\pgfpathlineto{\pgfqpoint{4.453158in}{5.253618in}}%
\pgfpathlineto{\pgfqpoint{4.499459in}{5.291323in}}%
\pgfpathlineto{\pgfqpoint{4.545760in}{5.328076in}}%
\pgfpathlineto{\pgfqpoint{4.576627in}{5.348787in}}%
\pgfpathlineto{\pgfqpoint{4.592061in}{5.357400in}}%
\pgfpathlineto{\pgfqpoint{4.607494in}{5.364654in}}%
\pgfpathlineto{\pgfqpoint{4.622928in}{5.370449in}}%
\pgfpathlineto{\pgfqpoint{4.638362in}{5.374723in}}%
\pgfpathlineto{\pgfqpoint{4.653795in}{5.377457in}}%
\pgfpathlineto{\pgfqpoint{4.669229in}{5.378669in}}%
\pgfpathlineto{\pgfqpoint{4.669229in}{5.378669in}}%
\pgfusepath{stroke}%
\end{pgfscope}%
\begin{pgfscope}%
\pgfpathrectangle{\pgfqpoint{0.634105in}{3.881603in}}{\pgfqpoint{4.227273in}{2.800000in}} %
\pgfusepath{clip}%
\pgfsetrectcap%
\pgfsetroundjoin%
\pgfsetlinewidth{0.501875pt}%
\definecolor{currentstroke}{rgb}{1.000000,0.361242,0.183750}%
\pgfsetstrokecolor{currentstroke}%
\pgfsetdash{}{0pt}%
\pgfpathmoveto{\pgfqpoint{0.826254in}{5.664598in}}%
\pgfpathlineto{\pgfqpoint{0.841687in}{5.663203in}}%
\pgfpathlineto{\pgfqpoint{0.857121in}{5.657856in}}%
\pgfpathlineto{\pgfqpoint{0.872555in}{5.648427in}}%
\pgfpathlineto{\pgfqpoint{0.887988in}{5.634857in}}%
\pgfpathlineto{\pgfqpoint{0.903422in}{5.617155in}}%
\pgfpathlineto{\pgfqpoint{0.918855in}{5.595408in}}%
\pgfpathlineto{\pgfqpoint{0.934289in}{5.569772in}}%
\pgfpathlineto{\pgfqpoint{0.949723in}{5.540480in}}%
\pgfpathlineto{\pgfqpoint{0.965156in}{5.507835in}}%
\pgfpathlineto{\pgfqpoint{0.996024in}{5.434040in}}%
\pgfpathlineto{\pgfqpoint{1.026891in}{5.352078in}}%
\pgfpathlineto{\pgfqpoint{1.104059in}{5.141938in}}%
\pgfpathlineto{\pgfqpoint{1.134926in}{5.068991in}}%
\pgfpathlineto{\pgfqpoint{1.150360in}{5.037311in}}%
\pgfpathlineto{\pgfqpoint{1.165794in}{5.009522in}}%
\pgfpathlineto{\pgfqpoint{1.181227in}{4.986073in}}%
\pgfpathlineto{\pgfqpoint{1.196661in}{4.967343in}}%
\pgfpathlineto{\pgfqpoint{1.212095in}{4.953632in}}%
\pgfpathlineto{\pgfqpoint{1.227528in}{4.945158in}}%
\pgfpathlineto{\pgfqpoint{1.242962in}{4.942051in}}%
\pgfpathlineto{\pgfqpoint{1.258395in}{4.944350in}}%
\pgfpathlineto{\pgfqpoint{1.273829in}{4.952006in}}%
\pgfpathlineto{\pgfqpoint{1.289263in}{4.964878in}}%
\pgfpathlineto{\pgfqpoint{1.304696in}{4.982747in}}%
\pgfpathlineto{\pgfqpoint{1.320130in}{5.005310in}}%
\pgfpathlineto{\pgfqpoint{1.335564in}{5.032196in}}%
\pgfpathlineto{\pgfqpoint{1.350997in}{5.062972in}}%
\pgfpathlineto{\pgfqpoint{1.366431in}{5.097152in}}%
\pgfpathlineto{\pgfqpoint{1.397298in}{5.173575in}}%
\pgfpathlineto{\pgfqpoint{1.443599in}{5.299732in}}%
\pgfpathlineto{\pgfqpoint{1.489900in}{5.425761in}}%
\pgfpathlineto{\pgfqpoint{1.520767in}{5.502528in}}%
\pgfpathlineto{\pgfqpoint{1.536201in}{5.537340in}}%
\pgfpathlineto{\pgfqpoint{1.551634in}{5.569253in}}%
\pgfpathlineto{\pgfqpoint{1.567068in}{5.597938in}}%
\pgfpathlineto{\pgfqpoint{1.582502in}{5.623120in}}%
\pgfpathlineto{\pgfqpoint{1.597935in}{5.644573in}}%
\pgfpathlineto{\pgfqpoint{1.613369in}{5.662131in}}%
\pgfpathlineto{\pgfqpoint{1.628803in}{5.675679in}}%
\pgfpathlineto{\pgfqpoint{1.644236in}{5.685157in}}%
\pgfpathlineto{\pgfqpoint{1.659670in}{5.690560in}}%
\pgfpathlineto{\pgfqpoint{1.675104in}{5.691932in}}%
\pgfpathlineto{\pgfqpoint{1.690537in}{5.689370in}}%
\pgfpathlineto{\pgfqpoint{1.705971in}{5.683016in}}%
\pgfpathlineto{\pgfqpoint{1.721404in}{5.673059in}}%
\pgfpathlineto{\pgfqpoint{1.736838in}{5.659724in}}%
\pgfpathlineto{\pgfqpoint{1.752272in}{5.643273in}}%
\pgfpathlineto{\pgfqpoint{1.767705in}{5.623998in}}%
\pgfpathlineto{\pgfqpoint{1.783139in}{5.602217in}}%
\pgfpathlineto{\pgfqpoint{1.814006in}{5.552486in}}%
\pgfpathlineto{\pgfqpoint{1.844874in}{5.496884in}}%
\pgfpathlineto{\pgfqpoint{1.952909in}{5.295425in}}%
\pgfpathlineto{\pgfqpoint{1.983776in}{5.245613in}}%
\pgfpathlineto{\pgfqpoint{2.014644in}{5.202566in}}%
\pgfpathlineto{\pgfqpoint{2.030077in}{5.183971in}}%
\pgfpathlineto{\pgfqpoint{2.045511in}{5.167486in}}%
\pgfpathlineto{\pgfqpoint{2.060944in}{5.153200in}}%
\pgfpathlineto{\pgfqpoint{2.076378in}{5.141171in}}%
\pgfpathlineto{\pgfqpoint{2.091812in}{5.131431in}}%
\pgfpathlineto{\pgfqpoint{2.107245in}{5.123978in}}%
\pgfpathlineto{\pgfqpoint{2.122679in}{5.118779in}}%
\pgfpathlineto{\pgfqpoint{2.138113in}{5.115767in}}%
\pgfpathlineto{\pgfqpoint{2.153546in}{5.114837in}}%
\pgfpathlineto{\pgfqpoint{2.168980in}{5.115848in}}%
\pgfpathlineto{\pgfqpoint{2.184414in}{5.118625in}}%
\pgfpathlineto{\pgfqpoint{2.199847in}{5.122953in}}%
\pgfpathlineto{\pgfqpoint{2.230714in}{5.135249in}}%
\pgfpathlineto{\pgfqpoint{2.323316in}{5.179036in}}%
\pgfpathlineto{\pgfqpoint{2.338750in}{5.183928in}}%
\pgfpathlineto{\pgfqpoint{2.354184in}{5.187342in}}%
\pgfpathlineto{\pgfqpoint{2.369617in}{5.189050in}}%
\pgfpathlineto{\pgfqpoint{2.385051in}{5.188870in}}%
\pgfpathlineto{\pgfqpoint{2.400484in}{5.186666in}}%
\pgfpathlineto{\pgfqpoint{2.415918in}{5.182360in}}%
\pgfpathlineto{\pgfqpoint{2.431352in}{5.175925in}}%
\pgfpathlineto{\pgfqpoint{2.446785in}{5.167394in}}%
\pgfpathlineto{\pgfqpoint{2.462219in}{5.156851in}}%
\pgfpathlineto{\pgfqpoint{2.477653in}{5.144434in}}%
\pgfpathlineto{\pgfqpoint{2.508520in}{5.114753in}}%
\pgfpathlineto{\pgfqpoint{2.539387in}{5.080282in}}%
\pgfpathlineto{\pgfqpoint{2.616555in}{4.989232in}}%
\pgfpathlineto{\pgfqpoint{2.647423in}{4.957558in}}%
\pgfpathlineto{\pgfqpoint{2.662856in}{4.943897in}}%
\pgfpathlineto{\pgfqpoint{2.678290in}{4.932000in}}%
\pgfpathlineto{\pgfqpoint{2.693724in}{4.922062in}}%
\pgfpathlineto{\pgfqpoint{2.709157in}{4.914236in}}%
\pgfpathlineto{\pgfqpoint{2.724591in}{4.908639in}}%
\pgfpathlineto{\pgfqpoint{2.740024in}{4.905342in}}%
\pgfpathlineto{\pgfqpoint{2.755458in}{4.904372in}}%
\pgfpathlineto{\pgfqpoint{2.770892in}{4.905711in}}%
\pgfpathlineto{\pgfqpoint{2.786325in}{4.909292in}}%
\pgfpathlineto{\pgfqpoint{2.801759in}{4.915007in}}%
\pgfpathlineto{\pgfqpoint{2.817193in}{4.922698in}}%
\pgfpathlineto{\pgfqpoint{2.832626in}{4.932169in}}%
\pgfpathlineto{\pgfqpoint{2.863493in}{4.955467in}}%
\pgfpathlineto{\pgfqpoint{2.894361in}{4.982627in}}%
\pgfpathlineto{\pgfqpoint{2.940662in}{5.024792in}}%
\pgfpathlineto{\pgfqpoint{2.971529in}{5.049853in}}%
\pgfpathlineto{\pgfqpoint{2.986963in}{5.060511in}}%
\pgfpathlineto{\pgfqpoint{3.002396in}{5.069555in}}%
\pgfpathlineto{\pgfqpoint{3.017830in}{5.076756in}}%
\pgfpathlineto{\pgfqpoint{3.033263in}{5.081931in}}%
\pgfpathlineto{\pgfqpoint{3.048697in}{5.084943in}}%
\pgfpathlineto{\pgfqpoint{3.064131in}{5.085705in}}%
\pgfpathlineto{\pgfqpoint{3.079564in}{5.084182in}}%
\pgfpathlineto{\pgfqpoint{3.094998in}{5.080388in}}%
\pgfpathlineto{\pgfqpoint{3.110432in}{5.074388in}}%
\pgfpathlineto{\pgfqpoint{3.125865in}{5.066289in}}%
\pgfpathlineto{\pgfqpoint{3.141299in}{5.056243in}}%
\pgfpathlineto{\pgfqpoint{3.156733in}{5.044436in}}%
\pgfpathlineto{\pgfqpoint{3.187600in}{5.016425in}}%
\pgfpathlineto{\pgfqpoint{3.218467in}{4.984232in}}%
\pgfpathlineto{\pgfqpoint{3.311069in}{4.883779in}}%
\pgfpathlineto{\pgfqpoint{3.341936in}{4.855381in}}%
\pgfpathlineto{\pgfqpoint{3.372803in}{4.831950in}}%
\pgfpathlineto{\pgfqpoint{3.388237in}{4.822382in}}%
\pgfpathlineto{\pgfqpoint{3.403671in}{4.814347in}}%
\pgfpathlineto{\pgfqpoint{3.419104in}{4.807887in}}%
\pgfpathlineto{\pgfqpoint{3.434538in}{4.803024in}}%
\pgfpathlineto{\pgfqpoint{3.449972in}{4.799755in}}%
\pgfpathlineto{\pgfqpoint{3.465405in}{4.798059in}}%
\pgfpathlineto{\pgfqpoint{3.480839in}{4.797898in}}%
\pgfpathlineto{\pgfqpoint{3.496273in}{4.799219in}}%
\pgfpathlineto{\pgfqpoint{3.511706in}{4.801959in}}%
\pgfpathlineto{\pgfqpoint{3.527140in}{4.806048in}}%
\pgfpathlineto{\pgfqpoint{3.542573in}{4.811410in}}%
\pgfpathlineto{\pgfqpoint{3.573441in}{4.825649in}}%
\pgfpathlineto{\pgfqpoint{3.604308in}{4.844095in}}%
\pgfpathlineto{\pgfqpoint{3.635175in}{4.866256in}}%
\pgfpathlineto{\pgfqpoint{3.666043in}{4.891755in}}%
\pgfpathlineto{\pgfqpoint{3.696910in}{4.920307in}}%
\pgfpathlineto{\pgfqpoint{3.727777in}{4.951659in}}%
\pgfpathlineto{\pgfqpoint{3.758644in}{4.985487in}}%
\pgfpathlineto{\pgfqpoint{3.804945in}{5.039728in}}%
\pgfpathlineto{\pgfqpoint{3.882113in}{5.131826in}}%
\pgfpathlineto{\pgfqpoint{3.912981in}{5.165514in}}%
\pgfpathlineto{\pgfqpoint{3.943848in}{5.195214in}}%
\pgfpathlineto{\pgfqpoint{3.959282in}{5.208144in}}%
\pgfpathlineto{\pgfqpoint{3.974715in}{5.219600in}}%
\pgfpathlineto{\pgfqpoint{3.990149in}{5.229467in}}%
\pgfpathlineto{\pgfqpoint{4.005583in}{5.237657in}}%
\pgfpathlineto{\pgfqpoint{4.021016in}{5.244115in}}%
\pgfpathlineto{\pgfqpoint{4.036450in}{5.248818in}}%
\pgfpathlineto{\pgfqpoint{4.051883in}{5.251780in}}%
\pgfpathlineto{\pgfqpoint{4.067317in}{5.253049in}}%
\pgfpathlineto{\pgfqpoint{4.082751in}{5.252709in}}%
\pgfpathlineto{\pgfqpoint{4.098184in}{5.250876in}}%
\pgfpathlineto{\pgfqpoint{4.113618in}{5.247696in}}%
\pgfpathlineto{\pgfqpoint{4.144485in}{5.238018in}}%
\pgfpathlineto{\pgfqpoint{4.175352in}{5.225328in}}%
\pgfpathlineto{\pgfqpoint{4.237087in}{5.198544in}}%
\pgfpathlineto{\pgfqpoint{4.267954in}{5.188287in}}%
\pgfpathlineto{\pgfqpoint{4.283388in}{5.184695in}}%
\pgfpathlineto{\pgfqpoint{4.298822in}{5.182359in}}%
\pgfpathlineto{\pgfqpoint{4.314255in}{5.181416in}}%
\pgfpathlineto{\pgfqpoint{4.329689in}{5.181967in}}%
\pgfpathlineto{\pgfqpoint{4.345122in}{5.184079in}}%
\pgfpathlineto{\pgfqpoint{4.360556in}{5.187777in}}%
\pgfpathlineto{\pgfqpoint{4.375990in}{5.193047in}}%
\pgfpathlineto{\pgfqpoint{4.391423in}{5.199832in}}%
\pgfpathlineto{\pgfqpoint{4.406857in}{5.208036in}}%
\pgfpathlineto{\pgfqpoint{4.437724in}{5.228124in}}%
\pgfpathlineto{\pgfqpoint{4.468592in}{5.251833in}}%
\pgfpathlineto{\pgfqpoint{4.545760in}{5.314231in}}%
\pgfpathlineto{\pgfqpoint{4.576627in}{5.335219in}}%
\pgfpathlineto{\pgfqpoint{4.592061in}{5.344022in}}%
\pgfpathlineto{\pgfqpoint{4.607494in}{5.351506in}}%
\pgfpathlineto{\pgfqpoint{4.622928in}{5.357574in}}%
\pgfpathlineto{\pgfqpoint{4.638362in}{5.362170in}}%
\pgfpathlineto{\pgfqpoint{4.653795in}{5.365277in}}%
\pgfpathlineto{\pgfqpoint{4.669229in}{5.366916in}}%
\pgfpathlineto{\pgfqpoint{4.669229in}{5.366916in}}%
\pgfusepath{stroke}%
\end{pgfscope}%
\begin{pgfscope}%
\pgfpathrectangle{\pgfqpoint{0.634105in}{3.881603in}}{\pgfqpoint{4.227273in}{2.800000in}} %
\pgfusepath{clip}%
\pgfsetrectcap%
\pgfsetroundjoin%
\pgfsetlinewidth{0.501875pt}%
\definecolor{currentstroke}{rgb}{1.000000,0.243914,0.122888}%
\pgfsetstrokecolor{currentstroke}%
\pgfsetdash{}{0pt}%
\pgfpathmoveto{\pgfqpoint{0.826254in}{5.666125in}}%
\pgfpathlineto{\pgfqpoint{0.841687in}{5.664592in}}%
\pgfpathlineto{\pgfqpoint{0.857121in}{5.659175in}}%
\pgfpathlineto{\pgfqpoint{0.872555in}{5.649741in}}%
\pgfpathlineto{\pgfqpoint{0.887988in}{5.636223in}}%
\pgfpathlineto{\pgfqpoint{0.903422in}{5.618626in}}%
\pgfpathlineto{\pgfqpoint{0.918855in}{5.597026in}}%
\pgfpathlineto{\pgfqpoint{0.934289in}{5.571575in}}%
\pgfpathlineto{\pgfqpoint{0.949723in}{5.542500in}}%
\pgfpathlineto{\pgfqpoint{0.965156in}{5.510098in}}%
\pgfpathlineto{\pgfqpoint{0.996024in}{5.436846in}}%
\pgfpathlineto{\pgfqpoint{1.026891in}{5.355485in}}%
\pgfpathlineto{\pgfqpoint{1.088625in}{5.186755in}}%
\pgfpathlineto{\pgfqpoint{1.119493in}{5.109416in}}%
\pgfpathlineto{\pgfqpoint{1.134926in}{5.074669in}}%
\pgfpathlineto{\pgfqpoint{1.150360in}{5.043288in}}%
\pgfpathlineto{\pgfqpoint{1.165794in}{5.015776in}}%
\pgfpathlineto{\pgfqpoint{1.181227in}{4.992575in}}%
\pgfpathlineto{\pgfqpoint{1.196661in}{4.974057in}}%
\pgfpathlineto{\pgfqpoint{1.212095in}{4.960515in}}%
\pgfpathlineto{\pgfqpoint{1.227528in}{4.952162in}}%
\pgfpathlineto{\pgfqpoint{1.242962in}{4.949121in}}%
\pgfpathlineto{\pgfqpoint{1.258395in}{4.951428in}}%
\pgfpathlineto{\pgfqpoint{1.273829in}{4.959028in}}%
\pgfpathlineto{\pgfqpoint{1.289263in}{4.971781in}}%
\pgfpathlineto{\pgfqpoint{1.304696in}{4.989463in}}%
\pgfpathlineto{\pgfqpoint{1.320130in}{5.011772in}}%
\pgfpathlineto{\pgfqpoint{1.335564in}{5.038338in}}%
\pgfpathlineto{\pgfqpoint{1.350997in}{5.068727in}}%
\pgfpathlineto{\pgfqpoint{1.366431in}{5.102454in}}%
\pgfpathlineto{\pgfqpoint{1.397298in}{5.177788in}}%
\pgfpathlineto{\pgfqpoint{1.443599in}{5.301885in}}%
\pgfpathlineto{\pgfqpoint{1.489900in}{5.425444in}}%
\pgfpathlineto{\pgfqpoint{1.520767in}{5.500434in}}%
\pgfpathlineto{\pgfqpoint{1.536201in}{5.534353in}}%
\pgfpathlineto{\pgfqpoint{1.551634in}{5.565388in}}%
\pgfpathlineto{\pgfqpoint{1.567068in}{5.593230in}}%
\pgfpathlineto{\pgfqpoint{1.582502in}{5.617618in}}%
\pgfpathlineto{\pgfqpoint{1.597935in}{5.638351in}}%
\pgfpathlineto{\pgfqpoint{1.613369in}{5.655280in}}%
\pgfpathlineto{\pgfqpoint{1.628803in}{5.668308in}}%
\pgfpathlineto{\pgfqpoint{1.644236in}{5.677395in}}%
\pgfpathlineto{\pgfqpoint{1.659670in}{5.682550in}}%
\pgfpathlineto{\pgfqpoint{1.675104in}{5.683830in}}%
\pgfpathlineto{\pgfqpoint{1.690537in}{5.681339in}}%
\pgfpathlineto{\pgfqpoint{1.705971in}{5.675225in}}%
\pgfpathlineto{\pgfqpoint{1.721404in}{5.665671in}}%
\pgfpathlineto{\pgfqpoint{1.736838in}{5.652898in}}%
\pgfpathlineto{\pgfqpoint{1.752272in}{5.637154in}}%
\pgfpathlineto{\pgfqpoint{1.767705in}{5.618714in}}%
\pgfpathlineto{\pgfqpoint{1.783139in}{5.597873in}}%
\pgfpathlineto{\pgfqpoint{1.814006in}{5.550234in}}%
\pgfpathlineto{\pgfqpoint{1.844874in}{5.496808in}}%
\pgfpathlineto{\pgfqpoint{1.952909in}{5.300512in}}%
\pgfpathlineto{\pgfqpoint{1.983776in}{5.250923in}}%
\pgfpathlineto{\pgfqpoint{2.014644in}{5.207496in}}%
\pgfpathlineto{\pgfqpoint{2.030077in}{5.188505in}}%
\pgfpathlineto{\pgfqpoint{2.045511in}{5.171503in}}%
\pgfpathlineto{\pgfqpoint{2.060944in}{5.156590in}}%
\pgfpathlineto{\pgfqpoint{2.076378in}{5.143841in}}%
\pgfpathlineto{\pgfqpoint{2.091812in}{5.133302in}}%
\pgfpathlineto{\pgfqpoint{2.107245in}{5.124987in}}%
\pgfpathlineto{\pgfqpoint{2.122679in}{5.118881in}}%
\pgfpathlineto{\pgfqpoint{2.138113in}{5.114931in}}%
\pgfpathlineto{\pgfqpoint{2.153546in}{5.113050in}}%
\pgfpathlineto{\pgfqpoint{2.168980in}{5.113117in}}%
\pgfpathlineto{\pgfqpoint{2.184414in}{5.114971in}}%
\pgfpathlineto{\pgfqpoint{2.199847in}{5.118417in}}%
\pgfpathlineto{\pgfqpoint{2.230714in}{5.129149in}}%
\pgfpathlineto{\pgfqpoint{2.261582in}{5.143159in}}%
\pgfpathlineto{\pgfqpoint{2.307883in}{5.164893in}}%
\pgfpathlineto{\pgfqpoint{2.338750in}{5.176171in}}%
\pgfpathlineto{\pgfqpoint{2.354184in}{5.179976in}}%
\pgfpathlineto{\pgfqpoint{2.369617in}{5.182235in}}%
\pgfpathlineto{\pgfqpoint{2.385051in}{5.182755in}}%
\pgfpathlineto{\pgfqpoint{2.400484in}{5.181386in}}%
\pgfpathlineto{\pgfqpoint{2.415918in}{5.178030in}}%
\pgfpathlineto{\pgfqpoint{2.431352in}{5.172638in}}%
\pgfpathlineto{\pgfqpoint{2.446785in}{5.165218in}}%
\pgfpathlineto{\pgfqpoint{2.462219in}{5.155827in}}%
\pgfpathlineto{\pgfqpoint{2.477653in}{5.144576in}}%
\pgfpathlineto{\pgfqpoint{2.493086in}{5.131622in}}%
\pgfpathlineto{\pgfqpoint{2.523954in}{5.101447in}}%
\pgfpathlineto{\pgfqpoint{2.554821in}{5.067327in}}%
\pgfpathlineto{\pgfqpoint{2.616555in}{4.997074in}}%
\pgfpathlineto{\pgfqpoint{2.647423in}{4.966022in}}%
\pgfpathlineto{\pgfqpoint{2.662856in}{4.952535in}}%
\pgfpathlineto{\pgfqpoint{2.678290in}{4.940732in}}%
\pgfpathlineto{\pgfqpoint{2.693724in}{4.930814in}}%
\pgfpathlineto{\pgfqpoint{2.709157in}{4.922944in}}%
\pgfpathlineto{\pgfqpoint{2.724591in}{4.917240in}}%
\pgfpathlineto{\pgfqpoint{2.740024in}{4.913778in}}%
\pgfpathlineto{\pgfqpoint{2.755458in}{4.912585in}}%
\pgfpathlineto{\pgfqpoint{2.770892in}{4.913640in}}%
\pgfpathlineto{\pgfqpoint{2.786325in}{4.916874in}}%
\pgfpathlineto{\pgfqpoint{2.801759in}{4.922173in}}%
\pgfpathlineto{\pgfqpoint{2.817193in}{4.929374in}}%
\pgfpathlineto{\pgfqpoint{2.832626in}{4.938276in}}%
\pgfpathlineto{\pgfqpoint{2.863493in}{4.960185in}}%
\pgfpathlineto{\pgfqpoint{2.909794in}{4.998830in}}%
\pgfpathlineto{\pgfqpoint{2.940662in}{5.024596in}}%
\pgfpathlineto{\pgfqpoint{2.971529in}{5.047281in}}%
\pgfpathlineto{\pgfqpoint{2.986963in}{5.056719in}}%
\pgfpathlineto{\pgfqpoint{3.002396in}{5.064548in}}%
\pgfpathlineto{\pgfqpoint{3.017830in}{5.070559in}}%
\pgfpathlineto{\pgfqpoint{3.033263in}{5.074589in}}%
\pgfpathlineto{\pgfqpoint{3.048697in}{5.076521in}}%
\pgfpathlineto{\pgfqpoint{3.064131in}{5.076287in}}%
\pgfpathlineto{\pgfqpoint{3.079564in}{5.073869in}}%
\pgfpathlineto{\pgfqpoint{3.094998in}{5.069300in}}%
\pgfpathlineto{\pgfqpoint{3.110432in}{5.062658in}}%
\pgfpathlineto{\pgfqpoint{3.125865in}{5.054064in}}%
\pgfpathlineto{\pgfqpoint{3.141299in}{5.043681in}}%
\pgfpathlineto{\pgfqpoint{3.172166in}{5.018356in}}%
\pgfpathlineto{\pgfqpoint{3.203033in}{4.988541in}}%
\pgfpathlineto{\pgfqpoint{3.295635in}{4.893139in}}%
\pgfpathlineto{\pgfqpoint{3.326503in}{4.865812in}}%
\pgfpathlineto{\pgfqpoint{3.357370in}{4.843184in}}%
\pgfpathlineto{\pgfqpoint{3.372803in}{4.833914in}}%
\pgfpathlineto{\pgfqpoint{3.388237in}{4.826101in}}%
\pgfpathlineto{\pgfqpoint{3.403671in}{4.819782in}}%
\pgfpathlineto{\pgfqpoint{3.419104in}{4.814967in}}%
\pgfpathlineto{\pgfqpoint{3.434538in}{4.811646in}}%
\pgfpathlineto{\pgfqpoint{3.449972in}{4.809786in}}%
\pgfpathlineto{\pgfqpoint{3.465405in}{4.809339in}}%
\pgfpathlineto{\pgfqpoint{3.480839in}{4.810242in}}%
\pgfpathlineto{\pgfqpoint{3.496273in}{4.812422in}}%
\pgfpathlineto{\pgfqpoint{3.527140in}{4.820296in}}%
\pgfpathlineto{\pgfqpoint{3.558007in}{4.832318in}}%
\pgfpathlineto{\pgfqpoint{3.588874in}{4.847905in}}%
\pgfpathlineto{\pgfqpoint{3.619742in}{4.866610in}}%
\pgfpathlineto{\pgfqpoint{3.650609in}{4.888155in}}%
\pgfpathlineto{\pgfqpoint{3.681476in}{4.912413in}}%
\pgfpathlineto{\pgfqpoint{3.712343in}{4.939345in}}%
\pgfpathlineto{\pgfqpoint{3.743211in}{4.968897in}}%
\pgfpathlineto{\pgfqpoint{3.774078in}{5.000881in}}%
\pgfpathlineto{\pgfqpoint{3.820379in}{5.052405in}}%
\pgfpathlineto{\pgfqpoint{3.897547in}{5.139978in}}%
\pgfpathlineto{\pgfqpoint{3.928414in}{5.171739in}}%
\pgfpathlineto{\pgfqpoint{3.959282in}{5.199409in}}%
\pgfpathlineto{\pgfqpoint{3.974715in}{5.211290in}}%
\pgfpathlineto{\pgfqpoint{3.990149in}{5.221681in}}%
\pgfpathlineto{\pgfqpoint{4.005583in}{5.230475in}}%
\pgfpathlineto{\pgfqpoint{4.021016in}{5.237594in}}%
\pgfpathlineto{\pgfqpoint{4.036450in}{5.242995in}}%
\pgfpathlineto{\pgfqpoint{4.051883in}{5.246670in}}%
\pgfpathlineto{\pgfqpoint{4.067317in}{5.248648in}}%
\pgfpathlineto{\pgfqpoint{4.082751in}{5.248995in}}%
\pgfpathlineto{\pgfqpoint{4.098184in}{5.247812in}}%
\pgfpathlineto{\pgfqpoint{4.113618in}{5.245234in}}%
\pgfpathlineto{\pgfqpoint{4.144485in}{5.236587in}}%
\pgfpathlineto{\pgfqpoint{4.175352in}{5.224680in}}%
\pgfpathlineto{\pgfqpoint{4.237087in}{5.198824in}}%
\pgfpathlineto{\pgfqpoint{4.267954in}{5.188813in}}%
\pgfpathlineto{\pgfqpoint{4.283388in}{5.185310in}}%
\pgfpathlineto{\pgfqpoint{4.298822in}{5.183044in}}%
\pgfpathlineto{\pgfqpoint{4.314255in}{5.182151in}}%
\pgfpathlineto{\pgfqpoint{4.329689in}{5.182732in}}%
\pgfpathlineto{\pgfqpoint{4.345122in}{5.184852in}}%
\pgfpathlineto{\pgfqpoint{4.360556in}{5.188531in}}%
\pgfpathlineto{\pgfqpoint{4.375990in}{5.193750in}}%
\pgfpathlineto{\pgfqpoint{4.391423in}{5.200448in}}%
\pgfpathlineto{\pgfqpoint{4.422291in}{5.217845in}}%
\pgfpathlineto{\pgfqpoint{4.453158in}{5.239491in}}%
\pgfpathlineto{\pgfqpoint{4.499459in}{5.276175in}}%
\pgfpathlineto{\pgfqpoint{4.545760in}{5.312046in}}%
\pgfpathlineto{\pgfqpoint{4.576627in}{5.332415in}}%
\pgfpathlineto{\pgfqpoint{4.607494in}{5.348273in}}%
\pgfpathlineto{\pgfqpoint{4.622928in}{5.354221in}}%
\pgfpathlineto{\pgfqpoint{4.638362in}{5.358768in}}%
\pgfpathlineto{\pgfqpoint{4.653795in}{5.361898in}}%
\pgfpathlineto{\pgfqpoint{4.669229in}{5.363634in}}%
\pgfpathlineto{\pgfqpoint{4.669229in}{5.363634in}}%
\pgfusepath{stroke}%
\end{pgfscope}%
\begin{pgfscope}%
\pgfpathrectangle{\pgfqpoint{0.634105in}{3.881603in}}{\pgfqpoint{4.227273in}{2.800000in}} %
\pgfusepath{clip}%
\pgfsetrectcap%
\pgfsetroundjoin%
\pgfsetlinewidth{0.501875pt}%
\definecolor{currentstroke}{rgb}{1.000000,0.122888,0.061561}%
\pgfsetstrokecolor{currentstroke}%
\pgfsetdash{}{0pt}%
\pgfpathmoveto{\pgfqpoint{0.826254in}{5.668349in}}%
\pgfpathlineto{\pgfqpoint{0.841687in}{5.666803in}}%
\pgfpathlineto{\pgfqpoint{0.857121in}{5.661399in}}%
\pgfpathlineto{\pgfqpoint{0.872555in}{5.652007in}}%
\pgfpathlineto{\pgfqpoint{0.887988in}{5.638563in}}%
\pgfpathlineto{\pgfqpoint{0.903422in}{5.621072in}}%
\pgfpathlineto{\pgfqpoint{0.918855in}{5.599614in}}%
\pgfpathlineto{\pgfqpoint{0.934289in}{5.574339in}}%
\pgfpathlineto{\pgfqpoint{0.949723in}{5.545474in}}%
\pgfpathlineto{\pgfqpoint{0.965156in}{5.513315in}}%
\pgfpathlineto{\pgfqpoint{0.996024in}{5.440636in}}%
\pgfpathlineto{\pgfqpoint{1.026891in}{5.359932in}}%
\pgfpathlineto{\pgfqpoint{1.104059in}{5.153001in}}%
\pgfpathlineto{\pgfqpoint{1.134926in}{5.081061in}}%
\pgfpathlineto{\pgfqpoint{1.150360in}{5.049758in}}%
\pgfpathlineto{\pgfqpoint{1.165794in}{5.022240in}}%
\pgfpathlineto{\pgfqpoint{1.181227in}{4.998941in}}%
\pgfpathlineto{\pgfqpoint{1.196661in}{4.980228in}}%
\pgfpathlineto{\pgfqpoint{1.212095in}{4.966394in}}%
\pgfpathlineto{\pgfqpoint{1.227528in}{4.957651in}}%
\pgfpathlineto{\pgfqpoint{1.242962in}{4.954131in}}%
\pgfpathlineto{\pgfqpoint{1.258395in}{4.955877in}}%
\pgfpathlineto{\pgfqpoint{1.273829in}{4.962850in}}%
\pgfpathlineto{\pgfqpoint{1.289263in}{4.974924in}}%
\pgfpathlineto{\pgfqpoint{1.304696in}{4.991895in}}%
\pgfpathlineto{\pgfqpoint{1.320130in}{5.013482in}}%
\pgfpathlineto{\pgfqpoint{1.335564in}{5.039335in}}%
\pgfpathlineto{\pgfqpoint{1.350997in}{5.069044in}}%
\pgfpathlineto{\pgfqpoint{1.366431in}{5.102142in}}%
\pgfpathlineto{\pgfqpoint{1.397298in}{5.176444in}}%
\pgfpathlineto{\pgfqpoint{1.443599in}{5.299729in}}%
\pgfpathlineto{\pgfqpoint{1.489900in}{5.423408in}}%
\pgfpathlineto{\pgfqpoint{1.520767in}{5.498885in}}%
\pgfpathlineto{\pgfqpoint{1.536201in}{5.533124in}}%
\pgfpathlineto{\pgfqpoint{1.551634in}{5.564507in}}%
\pgfpathlineto{\pgfqpoint{1.567068in}{5.592704in}}%
\pgfpathlineto{\pgfqpoint{1.582502in}{5.617441in}}%
\pgfpathlineto{\pgfqpoint{1.597935in}{5.638497in}}%
\pgfpathlineto{\pgfqpoint{1.613369in}{5.655710in}}%
\pgfpathlineto{\pgfqpoint{1.628803in}{5.668974in}}%
\pgfpathlineto{\pgfqpoint{1.644236in}{5.678238in}}%
\pgfpathlineto{\pgfqpoint{1.659670in}{5.683505in}}%
\pgfpathlineto{\pgfqpoint{1.675104in}{5.684831in}}%
\pgfpathlineto{\pgfqpoint{1.690537in}{5.682318in}}%
\pgfpathlineto{\pgfqpoint{1.705971in}{5.676115in}}%
\pgfpathlineto{\pgfqpoint{1.721404in}{5.666411in}}%
\pgfpathlineto{\pgfqpoint{1.736838in}{5.653431in}}%
\pgfpathlineto{\pgfqpoint{1.752272in}{5.637429in}}%
\pgfpathlineto{\pgfqpoint{1.767705in}{5.618688in}}%
\pgfpathlineto{\pgfqpoint{1.783139in}{5.597510in}}%
\pgfpathlineto{\pgfqpoint{1.814006in}{5.549123in}}%
\pgfpathlineto{\pgfqpoint{1.844874in}{5.494900in}}%
\pgfpathlineto{\pgfqpoint{1.952909in}{5.296305in}}%
\pgfpathlineto{\pgfqpoint{1.983776in}{5.246416in}}%
\pgfpathlineto{\pgfqpoint{2.014644in}{5.202916in}}%
\pgfpathlineto{\pgfqpoint{2.030077in}{5.183976in}}%
\pgfpathlineto{\pgfqpoint{2.045511in}{5.167081in}}%
\pgfpathlineto{\pgfqpoint{2.060944in}{5.152330in}}%
\pgfpathlineto{\pgfqpoint{2.076378in}{5.139791in}}%
\pgfpathlineto{\pgfqpoint{2.091812in}{5.129504in}}%
\pgfpathlineto{\pgfqpoint{2.107245in}{5.121474in}}%
\pgfpathlineto{\pgfqpoint{2.122679in}{5.115678in}}%
\pgfpathlineto{\pgfqpoint{2.138113in}{5.112052in}}%
\pgfpathlineto{\pgfqpoint{2.153546in}{5.110502in}}%
\pgfpathlineto{\pgfqpoint{2.168980in}{5.110894in}}%
\pgfpathlineto{\pgfqpoint{2.184414in}{5.113060in}}%
\pgfpathlineto{\pgfqpoint{2.199847in}{5.116798in}}%
\pgfpathlineto{\pgfqpoint{2.230714in}{5.128025in}}%
\pgfpathlineto{\pgfqpoint{2.277015in}{5.149993in}}%
\pgfpathlineto{\pgfqpoint{2.307883in}{5.164422in}}%
\pgfpathlineto{\pgfqpoint{2.338750in}{5.175774in}}%
\pgfpathlineto{\pgfqpoint{2.354184in}{5.179596in}}%
\pgfpathlineto{\pgfqpoint{2.369617in}{5.181867in}}%
\pgfpathlineto{\pgfqpoint{2.385051in}{5.182397in}}%
\pgfpathlineto{\pgfqpoint{2.400484in}{5.181042in}}%
\pgfpathlineto{\pgfqpoint{2.415918in}{5.177708in}}%
\pgfpathlineto{\pgfqpoint{2.431352in}{5.172349in}}%
\pgfpathlineto{\pgfqpoint{2.446785in}{5.164972in}}%
\pgfpathlineto{\pgfqpoint{2.462219in}{5.155638in}}%
\pgfpathlineto{\pgfqpoint{2.477653in}{5.144455in}}%
\pgfpathlineto{\pgfqpoint{2.493086in}{5.131580in}}%
\pgfpathlineto{\pgfqpoint{2.523954in}{5.101589in}}%
\pgfpathlineto{\pgfqpoint{2.554821in}{5.067668in}}%
\pgfpathlineto{\pgfqpoint{2.616555in}{4.997762in}}%
\pgfpathlineto{\pgfqpoint{2.647423in}{4.966792in}}%
\pgfpathlineto{\pgfqpoint{2.662856in}{4.953307in}}%
\pgfpathlineto{\pgfqpoint{2.678290in}{4.941473in}}%
\pgfpathlineto{\pgfqpoint{2.693724in}{4.931484in}}%
\pgfpathlineto{\pgfqpoint{2.709157in}{4.923499in}}%
\pgfpathlineto{\pgfqpoint{2.724591in}{4.917629in}}%
\pgfpathlineto{\pgfqpoint{2.740024in}{4.913945in}}%
\pgfpathlineto{\pgfqpoint{2.755458in}{4.912470in}}%
\pgfpathlineto{\pgfqpoint{2.770892in}{4.913180in}}%
\pgfpathlineto{\pgfqpoint{2.786325in}{4.916006in}}%
\pgfpathlineto{\pgfqpoint{2.801759in}{4.920833in}}%
\pgfpathlineto{\pgfqpoint{2.817193in}{4.927502in}}%
\pgfpathlineto{\pgfqpoint{2.832626in}{4.935817in}}%
\pgfpathlineto{\pgfqpoint{2.863493in}{4.956417in}}%
\pgfpathlineto{\pgfqpoint{2.909794in}{4.992948in}}%
\pgfpathlineto{\pgfqpoint{2.940662in}{5.017375in}}%
\pgfpathlineto{\pgfqpoint{2.971529in}{5.038915in}}%
\pgfpathlineto{\pgfqpoint{2.986963in}{5.047886in}}%
\pgfpathlineto{\pgfqpoint{3.002396in}{5.055331in}}%
\pgfpathlineto{\pgfqpoint{3.017830in}{5.061046in}}%
\pgfpathlineto{\pgfqpoint{3.033263in}{5.064870in}}%
\pgfpathlineto{\pgfqpoint{3.048697in}{5.066687in}}%
\pgfpathlineto{\pgfqpoint{3.064131in}{5.066424in}}%
\pgfpathlineto{\pgfqpoint{3.079564in}{5.064057in}}%
\pgfpathlineto{\pgfqpoint{3.094998in}{5.059610in}}%
\pgfpathlineto{\pgfqpoint{3.110432in}{5.053148in}}%
\pgfpathlineto{\pgfqpoint{3.125865in}{5.044784in}}%
\pgfpathlineto{\pgfqpoint{3.141299in}{5.034665in}}%
\pgfpathlineto{\pgfqpoint{3.172166in}{5.009924in}}%
\pgfpathlineto{\pgfqpoint{3.203033in}{4.980700in}}%
\pgfpathlineto{\pgfqpoint{3.311069in}{4.872630in}}%
\pgfpathlineto{\pgfqpoint{3.341936in}{4.847815in}}%
\pgfpathlineto{\pgfqpoint{3.357370in}{4.837299in}}%
\pgfpathlineto{\pgfqpoint{3.372803in}{4.828182in}}%
\pgfpathlineto{\pgfqpoint{3.388237in}{4.820532in}}%
\pgfpathlineto{\pgfqpoint{3.403671in}{4.814391in}}%
\pgfpathlineto{\pgfqpoint{3.419104in}{4.809773in}}%
\pgfpathlineto{\pgfqpoint{3.434538in}{4.806672in}}%
\pgfpathlineto{\pgfqpoint{3.449972in}{4.805058in}}%
\pgfpathlineto{\pgfqpoint{3.465405in}{4.804885in}}%
\pgfpathlineto{\pgfqpoint{3.480839in}{4.806093in}}%
\pgfpathlineto{\pgfqpoint{3.496273in}{4.808611in}}%
\pgfpathlineto{\pgfqpoint{3.511706in}{4.812361in}}%
\pgfpathlineto{\pgfqpoint{3.542573in}{4.823237in}}%
\pgfpathlineto{\pgfqpoint{3.573441in}{4.838091in}}%
\pgfpathlineto{\pgfqpoint{3.604308in}{4.856380in}}%
\pgfpathlineto{\pgfqpoint{3.635175in}{4.877688in}}%
\pgfpathlineto{\pgfqpoint{3.666043in}{4.901737in}}%
\pgfpathlineto{\pgfqpoint{3.696910in}{4.928346in}}%
\pgfpathlineto{\pgfqpoint{3.727777in}{4.957353in}}%
\pgfpathlineto{\pgfqpoint{3.774078in}{5.004800in}}%
\pgfpathlineto{\pgfqpoint{3.835813in}{5.072660in}}%
\pgfpathlineto{\pgfqpoint{3.882113in}{5.123065in}}%
\pgfpathlineto{\pgfqpoint{3.912981in}{5.154146in}}%
\pgfpathlineto{\pgfqpoint{3.943848in}{5.181653in}}%
\pgfpathlineto{\pgfqpoint{3.974715in}{5.204364in}}%
\pgfpathlineto{\pgfqpoint{3.990149in}{5.213610in}}%
\pgfpathlineto{\pgfqpoint{4.005583in}{5.221328in}}%
\pgfpathlineto{\pgfqpoint{4.021016in}{5.227465in}}%
\pgfpathlineto{\pgfqpoint{4.036450in}{5.231994in}}%
\pgfpathlineto{\pgfqpoint{4.051883in}{5.234926in}}%
\pgfpathlineto{\pgfqpoint{4.067317in}{5.236302in}}%
\pgfpathlineto{\pgfqpoint{4.082751in}{5.236197in}}%
\pgfpathlineto{\pgfqpoint{4.098184in}{5.234720in}}%
\pgfpathlineto{\pgfqpoint{4.129052in}{5.228229in}}%
\pgfpathlineto{\pgfqpoint{4.159919in}{5.218241in}}%
\pgfpathlineto{\pgfqpoint{4.237087in}{5.189621in}}%
\pgfpathlineto{\pgfqpoint{4.267954in}{5.181550in}}%
\pgfpathlineto{\pgfqpoint{4.283388in}{5.179058in}}%
\pgfpathlineto{\pgfqpoint{4.298822in}{5.177803in}}%
\pgfpathlineto{\pgfqpoint{4.314255in}{5.177901in}}%
\pgfpathlineto{\pgfqpoint{4.329689in}{5.179436in}}%
\pgfpathlineto{\pgfqpoint{4.345122in}{5.182454in}}%
\pgfpathlineto{\pgfqpoint{4.360556in}{5.186960in}}%
\pgfpathlineto{\pgfqpoint{4.375990in}{5.192923in}}%
\pgfpathlineto{\pgfqpoint{4.391423in}{5.200273in}}%
\pgfpathlineto{\pgfqpoint{4.422291in}{5.218666in}}%
\pgfpathlineto{\pgfqpoint{4.453158in}{5.240899in}}%
\pgfpathlineto{\pgfqpoint{4.545760in}{5.313329in}}%
\pgfpathlineto{\pgfqpoint{4.576627in}{5.333379in}}%
\pgfpathlineto{\pgfqpoint{4.607494in}{5.348961in}}%
\pgfpathlineto{\pgfqpoint{4.622928in}{5.354813in}}%
\pgfpathlineto{\pgfqpoint{4.638362in}{5.359299in}}%
\pgfpathlineto{\pgfqpoint{4.653795in}{5.362409in}}%
\pgfpathlineto{\pgfqpoint{4.669229in}{5.364166in}}%
\pgfpathlineto{\pgfqpoint{4.669229in}{5.364166in}}%
\pgfusepath{stroke}%
\end{pgfscope}%
\begin{pgfscope}%
\pgfpathrectangle{\pgfqpoint{0.634105in}{3.881603in}}{\pgfqpoint{4.227273in}{2.800000in}} %
\pgfusepath{clip}%
\pgfsetrectcap%
\pgfsetroundjoin%
\pgfsetlinewidth{0.501875pt}%
\definecolor{currentstroke}{rgb}{0.000000,0.000000,0.000000}%
\pgfsetstrokecolor{currentstroke}%
\pgfsetdash{}{0pt}%
\pgfpathmoveto{\pgfqpoint{0.826254in}{5.669386in}}%
\pgfpathlineto{\pgfqpoint{0.841687in}{5.668002in}}%
\pgfpathlineto{\pgfqpoint{0.857121in}{5.662788in}}%
\pgfpathlineto{\pgfqpoint{0.872555in}{5.653596in}}%
\pgfpathlineto{\pgfqpoint{0.887988in}{5.640346in}}%
\pgfpathlineto{\pgfqpoint{0.903422in}{5.623029in}}%
\pgfpathlineto{\pgfqpoint{0.918855in}{5.601710in}}%
\pgfpathlineto{\pgfqpoint{0.934289in}{5.576530in}}%
\pgfpathlineto{\pgfqpoint{0.949723in}{5.547708in}}%
\pgfpathlineto{\pgfqpoint{0.965156in}{5.515536in}}%
\pgfpathlineto{\pgfqpoint{0.996024in}{5.442665in}}%
\pgfpathlineto{\pgfqpoint{1.026891in}{5.361582in}}%
\pgfpathlineto{\pgfqpoint{1.104059in}{5.153340in}}%
\pgfpathlineto{\pgfqpoint{1.134926in}{5.080920in}}%
\pgfpathlineto{\pgfqpoint{1.150360in}{5.049410in}}%
\pgfpathlineto{\pgfqpoint{1.165794in}{5.021709in}}%
\pgfpathlineto{\pgfqpoint{1.181227in}{4.998254in}}%
\pgfpathlineto{\pgfqpoint{1.196661in}{4.979409in}}%
\pgfpathlineto{\pgfqpoint{1.212095in}{4.965466in}}%
\pgfpathlineto{\pgfqpoint{1.227528in}{4.956637in}}%
\pgfpathlineto{\pgfqpoint{1.242962in}{4.953050in}}%
\pgfpathlineto{\pgfqpoint{1.258395in}{4.954749in}}%
\pgfpathlineto{\pgfqpoint{1.273829in}{4.961693in}}%
\pgfpathlineto{\pgfqpoint{1.289263in}{4.973757in}}%
\pgfpathlineto{\pgfqpoint{1.304696in}{4.990737in}}%
\pgfpathlineto{\pgfqpoint{1.320130in}{5.012355in}}%
\pgfpathlineto{\pgfqpoint{1.335564in}{5.038262in}}%
\pgfpathlineto{\pgfqpoint{1.350997in}{5.068049in}}%
\pgfpathlineto{\pgfqpoint{1.366431in}{5.101252in}}%
\pgfpathlineto{\pgfqpoint{1.397298in}{5.175843in}}%
\pgfpathlineto{\pgfqpoint{1.443599in}{5.299732in}}%
\pgfpathlineto{\pgfqpoint{1.489900in}{5.424111in}}%
\pgfpathlineto{\pgfqpoint{1.520767in}{5.500017in}}%
\pgfpathlineto{\pgfqpoint{1.536201in}{5.534440in}}%
\pgfpathlineto{\pgfqpoint{1.551634in}{5.565981in}}%
\pgfpathlineto{\pgfqpoint{1.567068in}{5.594309in}}%
\pgfpathlineto{\pgfqpoint{1.582502in}{5.619149in}}%
\pgfpathlineto{\pgfqpoint{1.597935in}{5.640284in}}%
\pgfpathlineto{\pgfqpoint{1.613369in}{5.657558in}}%
\pgfpathlineto{\pgfqpoint{1.628803in}{5.670874in}}%
\pgfpathlineto{\pgfqpoint{1.644236in}{5.680191in}}%
\pgfpathlineto{\pgfqpoint{1.659670in}{5.685524in}}%
\pgfpathlineto{\pgfqpoint{1.675104in}{5.686943in}}%
\pgfpathlineto{\pgfqpoint{1.690537in}{5.684564in}}%
\pgfpathlineto{\pgfqpoint{1.705971in}{5.678549in}}%
\pgfpathlineto{\pgfqpoint{1.721404in}{5.669099in}}%
\pgfpathlineto{\pgfqpoint{1.736838in}{5.656447in}}%
\pgfpathlineto{\pgfqpoint{1.752272in}{5.640856in}}%
\pgfpathlineto{\pgfqpoint{1.767705in}{5.622612in}}%
\pgfpathlineto{\pgfqpoint{1.783139in}{5.602015in}}%
\pgfpathlineto{\pgfqpoint{1.814006in}{5.555023in}}%
\pgfpathlineto{\pgfqpoint{1.844874in}{5.502429in}}%
\pgfpathlineto{\pgfqpoint{1.968343in}{5.283707in}}%
\pgfpathlineto{\pgfqpoint{1.999210in}{5.237076in}}%
\pgfpathlineto{\pgfqpoint{2.030077in}{5.196836in}}%
\pgfpathlineto{\pgfqpoint{2.045511in}{5.179467in}}%
\pgfpathlineto{\pgfqpoint{2.060944in}{5.164078in}}%
\pgfpathlineto{\pgfqpoint{2.076378in}{5.150753in}}%
\pgfpathlineto{\pgfqpoint{2.091812in}{5.139547in}}%
\pgfpathlineto{\pgfqpoint{2.107245in}{5.130490in}}%
\pgfpathlineto{\pgfqpoint{2.122679in}{5.123578in}}%
\pgfpathlineto{\pgfqpoint{2.138113in}{5.118778in}}%
\pgfpathlineto{\pgfqpoint{2.153546in}{5.116020in}}%
\pgfpathlineto{\pgfqpoint{2.168980in}{5.115198in}}%
\pgfpathlineto{\pgfqpoint{2.184414in}{5.116171in}}%
\pgfpathlineto{\pgfqpoint{2.199847in}{5.118762in}}%
\pgfpathlineto{\pgfqpoint{2.215281in}{5.122759in}}%
\pgfpathlineto{\pgfqpoint{2.246148in}{5.133971in}}%
\pgfpathlineto{\pgfqpoint{2.323316in}{5.167018in}}%
\pgfpathlineto{\pgfqpoint{2.338750in}{5.172016in}}%
\pgfpathlineto{\pgfqpoint{2.354184in}{5.175808in}}%
\pgfpathlineto{\pgfqpoint{2.369617in}{5.178159in}}%
\pgfpathlineto{\pgfqpoint{2.385051in}{5.178876in}}%
\pgfpathlineto{\pgfqpoint{2.400484in}{5.177807in}}%
\pgfpathlineto{\pgfqpoint{2.415918in}{5.174852in}}%
\pgfpathlineto{\pgfqpoint{2.431352in}{5.169963in}}%
\pgfpathlineto{\pgfqpoint{2.446785in}{5.163144in}}%
\pgfpathlineto{\pgfqpoint{2.462219in}{5.154453in}}%
\pgfpathlineto{\pgfqpoint{2.477653in}{5.143998in}}%
\pgfpathlineto{\pgfqpoint{2.493086in}{5.131935in}}%
\pgfpathlineto{\pgfqpoint{2.523954in}{5.103809in}}%
\pgfpathlineto{\pgfqpoint{2.554821in}{5.072055in}}%
\pgfpathlineto{\pgfqpoint{2.616555in}{5.007103in}}%
\pgfpathlineto{\pgfqpoint{2.647423in}{4.978684in}}%
\pgfpathlineto{\pgfqpoint{2.662856in}{4.966425in}}%
\pgfpathlineto{\pgfqpoint{2.678290in}{4.955755in}}%
\pgfpathlineto{\pgfqpoint{2.693724in}{4.946850in}}%
\pgfpathlineto{\pgfqpoint{2.709157in}{4.939845in}}%
\pgfpathlineto{\pgfqpoint{2.724591in}{4.934836in}}%
\pgfpathlineto{\pgfqpoint{2.740024in}{4.931879in}}%
\pgfpathlineto{\pgfqpoint{2.755458in}{4.930985in}}%
\pgfpathlineto{\pgfqpoint{2.770892in}{4.932121in}}%
\pgfpathlineto{\pgfqpoint{2.786325in}{4.935212in}}%
\pgfpathlineto{\pgfqpoint{2.801759in}{4.940141in}}%
\pgfpathlineto{\pgfqpoint{2.817193in}{4.946751in}}%
\pgfpathlineto{\pgfqpoint{2.832626in}{4.954848in}}%
\pgfpathlineto{\pgfqpoint{2.863493in}{4.974562in}}%
\pgfpathlineto{\pgfqpoint{2.956095in}{5.041167in}}%
\pgfpathlineto{\pgfqpoint{2.971529in}{5.050158in}}%
\pgfpathlineto{\pgfqpoint{2.986963in}{5.057793in}}%
\pgfpathlineto{\pgfqpoint{3.002396in}{5.063844in}}%
\pgfpathlineto{\pgfqpoint{3.017830in}{5.068124in}}%
\pgfpathlineto{\pgfqpoint{3.033263in}{5.070482in}}%
\pgfpathlineto{\pgfqpoint{3.048697in}{5.070816in}}%
\pgfpathlineto{\pgfqpoint{3.064131in}{5.069071in}}%
\pgfpathlineto{\pgfqpoint{3.079564in}{5.065242in}}%
\pgfpathlineto{\pgfqpoint{3.094998in}{5.059371in}}%
\pgfpathlineto{\pgfqpoint{3.110432in}{5.051545in}}%
\pgfpathlineto{\pgfqpoint{3.125865in}{5.041898in}}%
\pgfpathlineto{\pgfqpoint{3.141299in}{5.030601in}}%
\pgfpathlineto{\pgfqpoint{3.172166in}{5.003903in}}%
\pgfpathlineto{\pgfqpoint{3.203033in}{4.973387in}}%
\pgfpathlineto{\pgfqpoint{3.280202in}{4.894517in}}%
\pgfpathlineto{\pgfqpoint{3.311069in}{4.867173in}}%
\pgfpathlineto{\pgfqpoint{3.341936in}{4.844533in}}%
\pgfpathlineto{\pgfqpoint{3.357370in}{4.835277in}}%
\pgfpathlineto{\pgfqpoint{3.372803in}{4.827495in}}%
\pgfpathlineto{\pgfqpoint{3.388237in}{4.821221in}}%
\pgfpathlineto{\pgfqpoint{3.403671in}{4.816463in}}%
\pgfpathlineto{\pgfqpoint{3.419104in}{4.813202in}}%
\pgfpathlineto{\pgfqpoint{3.434538in}{4.811398in}}%
\pgfpathlineto{\pgfqpoint{3.449972in}{4.810993in}}%
\pgfpathlineto{\pgfqpoint{3.465405in}{4.811915in}}%
\pgfpathlineto{\pgfqpoint{3.480839in}{4.814082in}}%
\pgfpathlineto{\pgfqpoint{3.511706in}{4.821801in}}%
\pgfpathlineto{\pgfqpoint{3.542573in}{4.833466in}}%
\pgfpathlineto{\pgfqpoint{3.573441in}{4.848482in}}%
\pgfpathlineto{\pgfqpoint{3.604308in}{4.866423in}}%
\pgfpathlineto{\pgfqpoint{3.635175in}{4.887058in}}%
\pgfpathlineto{\pgfqpoint{3.666043in}{4.910322in}}%
\pgfpathlineto{\pgfqpoint{3.696910in}{4.936245in}}%
\pgfpathlineto{\pgfqpoint{3.727777in}{4.964839in}}%
\pgfpathlineto{\pgfqpoint{3.758644in}{4.995983in}}%
\pgfpathlineto{\pgfqpoint{3.804945in}{5.046603in}}%
\pgfpathlineto{\pgfqpoint{3.897547in}{5.150737in}}%
\pgfpathlineto{\pgfqpoint{3.928414in}{5.181526in}}%
\pgfpathlineto{\pgfqpoint{3.959282in}{5.207968in}}%
\pgfpathlineto{\pgfqpoint{3.974715in}{5.219172in}}%
\pgfpathlineto{\pgfqpoint{3.990149in}{5.228864in}}%
\pgfpathlineto{\pgfqpoint{4.005583in}{5.236951in}}%
\pgfpathlineto{\pgfqpoint{4.021016in}{5.243374in}}%
\pgfpathlineto{\pgfqpoint{4.036450in}{5.248100in}}%
\pgfpathlineto{\pgfqpoint{4.051883in}{5.251136in}}%
\pgfpathlineto{\pgfqpoint{4.067317in}{5.252520in}}%
\pgfpathlineto{\pgfqpoint{4.082751in}{5.252325in}}%
\pgfpathlineto{\pgfqpoint{4.098184in}{5.250658in}}%
\pgfpathlineto{\pgfqpoint{4.113618in}{5.247657in}}%
\pgfpathlineto{\pgfqpoint{4.144485in}{5.238342in}}%
\pgfpathlineto{\pgfqpoint{4.175352in}{5.225992in}}%
\pgfpathlineto{\pgfqpoint{4.237087in}{5.199791in}}%
\pgfpathlineto{\pgfqpoint{4.267954in}{5.189814in}}%
\pgfpathlineto{\pgfqpoint{4.283388in}{5.186369in}}%
\pgfpathlineto{\pgfqpoint{4.298822in}{5.184186in}}%
\pgfpathlineto{\pgfqpoint{4.314255in}{5.183399in}}%
\pgfpathlineto{\pgfqpoint{4.329689in}{5.184110in}}%
\pgfpathlineto{\pgfqpoint{4.345122in}{5.186379in}}%
\pgfpathlineto{\pgfqpoint{4.360556in}{5.190227in}}%
\pgfpathlineto{\pgfqpoint{4.375990in}{5.195633in}}%
\pgfpathlineto{\pgfqpoint{4.391423in}{5.202535in}}%
\pgfpathlineto{\pgfqpoint{4.422291in}{5.220377in}}%
\pgfpathlineto{\pgfqpoint{4.453158in}{5.242514in}}%
\pgfpathlineto{\pgfqpoint{4.499459in}{5.280007in}}%
\pgfpathlineto{\pgfqpoint{4.545760in}{5.316805in}}%
\pgfpathlineto{\pgfqpoint{4.576627in}{5.337898in}}%
\pgfpathlineto{\pgfqpoint{4.607494in}{5.354594in}}%
\pgfpathlineto{\pgfqpoint{4.622928in}{5.361009in}}%
\pgfpathlineto{\pgfqpoint{4.638362in}{5.366053in}}%
\pgfpathlineto{\pgfqpoint{4.653795in}{5.369712in}}%
\pgfpathlineto{\pgfqpoint{4.669229in}{5.372003in}}%
\pgfpathlineto{\pgfqpoint{4.669229in}{5.372003in}}%
\pgfusepath{stroke}%
\end{pgfscope}%
\begin{pgfscope}%
\pgfsetrectcap%
\pgfsetmiterjoin%
\pgfsetlinewidth{0.803000pt}%
\definecolor{currentstroke}{rgb}{0.000000,0.000000,0.000000}%
\pgfsetstrokecolor{currentstroke}%
\pgfsetdash{}{0pt}%
\pgfpathmoveto{\pgfqpoint{0.634105in}{3.881603in}}%
\pgfpathlineto{\pgfqpoint{0.634105in}{6.681603in}}%
\pgfusepath{stroke}%
\end{pgfscope}%
\begin{pgfscope}%
\pgfsetrectcap%
\pgfsetmiterjoin%
\pgfsetlinewidth{0.803000pt}%
\definecolor{currentstroke}{rgb}{0.000000,0.000000,0.000000}%
\pgfsetstrokecolor{currentstroke}%
\pgfsetdash{}{0pt}%
\pgfpathmoveto{\pgfqpoint{4.861378in}{3.881603in}}%
\pgfpathlineto{\pgfqpoint{4.861378in}{6.681603in}}%
\pgfusepath{stroke}%
\end{pgfscope}%
\begin{pgfscope}%
\pgfsetrectcap%
\pgfsetmiterjoin%
\pgfsetlinewidth{0.803000pt}%
\definecolor{currentstroke}{rgb}{0.000000,0.000000,0.000000}%
\pgfsetstrokecolor{currentstroke}%
\pgfsetdash{}{0pt}%
\pgfpathmoveto{\pgfqpoint{0.634105in}{3.881603in}}%
\pgfpathlineto{\pgfqpoint{4.861378in}{3.881603in}}%
\pgfusepath{stroke}%
\end{pgfscope}%
\begin{pgfscope}%
\pgfsetrectcap%
\pgfsetmiterjoin%
\pgfsetlinewidth{0.803000pt}%
\definecolor{currentstroke}{rgb}{0.000000,0.000000,0.000000}%
\pgfsetstrokecolor{currentstroke}%
\pgfsetdash{}{0pt}%
\pgfpathmoveto{\pgfqpoint{0.634105in}{6.681603in}}%
\pgfpathlineto{\pgfqpoint{4.861378in}{6.681603in}}%
\pgfusepath{stroke}%
\end{pgfscope}%
\begin{pgfscope}%
\pgftext[x=5.284105in,y=6.764937in,,base]{\rmfamily\fontsize{12.000000}{14.400000}\selectfont \(\displaystyle \widetilde{K}u \approx Ku\), realization 1}%
\end{pgfscope}%
\begin{pgfscope}%
\pgfsetbuttcap%
\pgfsetmiterjoin%
\definecolor{currentfill}{rgb}{1.000000,1.000000,1.000000}%
\pgfsetfillcolor{currentfill}%
\pgfsetfillopacity{0.800000}%
\pgfsetlinewidth{1.003750pt}%
\definecolor{currentstroke}{rgb}{0.800000,0.800000,0.800000}%
\pgfsetstrokecolor{currentstroke}%
\pgfsetstrokeopacity{0.800000}%
\pgfsetdash{}{0pt}%
\pgfpathmoveto{\pgfqpoint{4.215951in}{3.951048in}}%
\pgfpathlineto{\pgfqpoint{4.764155in}{3.951048in}}%
\pgfpathquadraticcurveto{\pgfqpoint{4.791933in}{3.951048in}}{\pgfqpoint{4.791933in}{3.978826in}}%
\pgfpathlineto{\pgfqpoint{4.791933in}{4.168794in}}%
\pgfpathquadraticcurveto{\pgfqpoint{4.791933in}{4.196572in}}{\pgfqpoint{4.764155in}{4.196572in}}%
\pgfpathlineto{\pgfqpoint{4.215951in}{4.196572in}}%
\pgfpathquadraticcurveto{\pgfqpoint{4.188173in}{4.196572in}}{\pgfqpoint{4.188173in}{4.168794in}}%
\pgfpathlineto{\pgfqpoint{4.188173in}{3.978826in}}%
\pgfpathquadraticcurveto{\pgfqpoint{4.188173in}{3.951048in}}{\pgfqpoint{4.215951in}{3.951048in}}%
\pgfpathclose%
\pgfusepath{stroke,fill}%
\end{pgfscope}%
\begin{pgfscope}%
\pgfsetrectcap%
\pgfsetroundjoin%
\pgfsetlinewidth{0.501875pt}%
\definecolor{currentstroke}{rgb}{0.000000,0.000000,0.000000}%
\pgfsetstrokecolor{currentstroke}%
\pgfsetdash{}{0pt}%
\pgfpathmoveto{\pgfqpoint{4.243729in}{4.084104in}}%
\pgfpathlineto{\pgfqpoint{4.521507in}{4.084104in}}%
\pgfusepath{stroke}%
\end{pgfscope}%
\begin{pgfscope}%
\pgftext[x=4.632618in,y=4.035493in,left,base]{\rmfamily\fontsize{10.000000}{12.000000}\selectfont K}%
\end{pgfscope}%
\begin{pgfscope}%
\pgfsetbuttcap%
\pgfsetmiterjoin%
\definecolor{currentfill}{rgb}{1.000000,1.000000,1.000000}%
\pgfsetfillcolor{currentfill}%
\pgfsetlinewidth{0.000000pt}%
\definecolor{currentstroke}{rgb}{0.000000,0.000000,0.000000}%
\pgfsetstrokecolor{currentstroke}%
\pgfsetstrokeopacity{0.000000}%
\pgfsetdash{}{0pt}%
\pgfpathmoveto{\pgfqpoint{5.706832in}{3.881603in}}%
\pgfpathlineto{\pgfqpoint{9.934105in}{3.881603in}}%
\pgfpathlineto{\pgfqpoint{9.934105in}{6.681603in}}%
\pgfpathlineto{\pgfqpoint{5.706832in}{6.681603in}}%
\pgfpathclose%
\pgfusepath{fill}%
\end{pgfscope}%
\begin{pgfscope}%
\pgfsetbuttcap%
\pgfsetroundjoin%
\definecolor{currentfill}{rgb}{0.000000,0.000000,0.000000}%
\pgfsetfillcolor{currentfill}%
\pgfsetlinewidth{0.803000pt}%
\definecolor{currentstroke}{rgb}{0.000000,0.000000,0.000000}%
\pgfsetstrokecolor{currentstroke}%
\pgfsetdash{}{0pt}%
\pgfsys@defobject{currentmarker}{\pgfqpoint{0.000000in}{-0.048611in}}{\pgfqpoint{0.000000in}{0.000000in}}{%
\pgfpathmoveto{\pgfqpoint{0.000000in}{0.000000in}}%
\pgfpathlineto{\pgfqpoint{0.000000in}{-0.048611in}}%
\pgfusepath{stroke,fill}%
}%
\begin{pgfscope}%
\pgfsys@transformshift{5.898981in}{3.881603in}%
\pgfsys@useobject{currentmarker}{}%
\end{pgfscope}%
\end{pgfscope}%
\begin{pgfscope}%
\pgfsetbuttcap%
\pgfsetroundjoin%
\definecolor{currentfill}{rgb}{0.000000,0.000000,0.000000}%
\pgfsetfillcolor{currentfill}%
\pgfsetlinewidth{0.803000pt}%
\definecolor{currentstroke}{rgb}{0.000000,0.000000,0.000000}%
\pgfsetstrokecolor{currentstroke}%
\pgfsetdash{}{0pt}%
\pgfsys@defobject{currentmarker}{\pgfqpoint{0.000000in}{-0.048611in}}{\pgfqpoint{0.000000in}{0.000000in}}{%
\pgfpathmoveto{\pgfqpoint{0.000000in}{0.000000in}}%
\pgfpathlineto{\pgfqpoint{0.000000in}{-0.048611in}}%
\pgfusepath{stroke,fill}%
}%
\begin{pgfscope}%
\pgfsys@transformshift{6.379353in}{3.881603in}%
\pgfsys@useobject{currentmarker}{}%
\end{pgfscope}%
\end{pgfscope}%
\begin{pgfscope}%
\pgfsetbuttcap%
\pgfsetroundjoin%
\definecolor{currentfill}{rgb}{0.000000,0.000000,0.000000}%
\pgfsetfillcolor{currentfill}%
\pgfsetlinewidth{0.803000pt}%
\definecolor{currentstroke}{rgb}{0.000000,0.000000,0.000000}%
\pgfsetstrokecolor{currentstroke}%
\pgfsetdash{}{0pt}%
\pgfsys@defobject{currentmarker}{\pgfqpoint{0.000000in}{-0.048611in}}{\pgfqpoint{0.000000in}{0.000000in}}{%
\pgfpathmoveto{\pgfqpoint{0.000000in}{0.000000in}}%
\pgfpathlineto{\pgfqpoint{0.000000in}{-0.048611in}}%
\pgfusepath{stroke,fill}%
}%
\begin{pgfscope}%
\pgfsys@transformshift{6.859725in}{3.881603in}%
\pgfsys@useobject{currentmarker}{}%
\end{pgfscope}%
\end{pgfscope}%
\begin{pgfscope}%
\pgfsetbuttcap%
\pgfsetroundjoin%
\definecolor{currentfill}{rgb}{0.000000,0.000000,0.000000}%
\pgfsetfillcolor{currentfill}%
\pgfsetlinewidth{0.803000pt}%
\definecolor{currentstroke}{rgb}{0.000000,0.000000,0.000000}%
\pgfsetstrokecolor{currentstroke}%
\pgfsetdash{}{0pt}%
\pgfsys@defobject{currentmarker}{\pgfqpoint{0.000000in}{-0.048611in}}{\pgfqpoint{0.000000in}{0.000000in}}{%
\pgfpathmoveto{\pgfqpoint{0.000000in}{0.000000in}}%
\pgfpathlineto{\pgfqpoint{0.000000in}{-0.048611in}}%
\pgfusepath{stroke,fill}%
}%
\begin{pgfscope}%
\pgfsys@transformshift{7.340097in}{3.881603in}%
\pgfsys@useobject{currentmarker}{}%
\end{pgfscope}%
\end{pgfscope}%
\begin{pgfscope}%
\pgfsetbuttcap%
\pgfsetroundjoin%
\definecolor{currentfill}{rgb}{0.000000,0.000000,0.000000}%
\pgfsetfillcolor{currentfill}%
\pgfsetlinewidth{0.803000pt}%
\definecolor{currentstroke}{rgb}{0.000000,0.000000,0.000000}%
\pgfsetstrokecolor{currentstroke}%
\pgfsetdash{}{0pt}%
\pgfsys@defobject{currentmarker}{\pgfqpoint{0.000000in}{-0.048611in}}{\pgfqpoint{0.000000in}{0.000000in}}{%
\pgfpathmoveto{\pgfqpoint{0.000000in}{0.000000in}}%
\pgfpathlineto{\pgfqpoint{0.000000in}{-0.048611in}}%
\pgfusepath{stroke,fill}%
}%
\begin{pgfscope}%
\pgfsys@transformshift{7.820468in}{3.881603in}%
\pgfsys@useobject{currentmarker}{}%
\end{pgfscope}%
\end{pgfscope}%
\begin{pgfscope}%
\pgfsetbuttcap%
\pgfsetroundjoin%
\definecolor{currentfill}{rgb}{0.000000,0.000000,0.000000}%
\pgfsetfillcolor{currentfill}%
\pgfsetlinewidth{0.803000pt}%
\definecolor{currentstroke}{rgb}{0.000000,0.000000,0.000000}%
\pgfsetstrokecolor{currentstroke}%
\pgfsetdash{}{0pt}%
\pgfsys@defobject{currentmarker}{\pgfqpoint{0.000000in}{-0.048611in}}{\pgfqpoint{0.000000in}{0.000000in}}{%
\pgfpathmoveto{\pgfqpoint{0.000000in}{0.000000in}}%
\pgfpathlineto{\pgfqpoint{0.000000in}{-0.048611in}}%
\pgfusepath{stroke,fill}%
}%
\begin{pgfscope}%
\pgfsys@transformshift{8.300840in}{3.881603in}%
\pgfsys@useobject{currentmarker}{}%
\end{pgfscope}%
\end{pgfscope}%
\begin{pgfscope}%
\pgfsetbuttcap%
\pgfsetroundjoin%
\definecolor{currentfill}{rgb}{0.000000,0.000000,0.000000}%
\pgfsetfillcolor{currentfill}%
\pgfsetlinewidth{0.803000pt}%
\definecolor{currentstroke}{rgb}{0.000000,0.000000,0.000000}%
\pgfsetstrokecolor{currentstroke}%
\pgfsetdash{}{0pt}%
\pgfsys@defobject{currentmarker}{\pgfqpoint{0.000000in}{-0.048611in}}{\pgfqpoint{0.000000in}{0.000000in}}{%
\pgfpathmoveto{\pgfqpoint{0.000000in}{0.000000in}}%
\pgfpathlineto{\pgfqpoint{0.000000in}{-0.048611in}}%
\pgfusepath{stroke,fill}%
}%
\begin{pgfscope}%
\pgfsys@transformshift{8.781212in}{3.881603in}%
\pgfsys@useobject{currentmarker}{}%
\end{pgfscope}%
\end{pgfscope}%
\begin{pgfscope}%
\pgfsetbuttcap%
\pgfsetroundjoin%
\definecolor{currentfill}{rgb}{0.000000,0.000000,0.000000}%
\pgfsetfillcolor{currentfill}%
\pgfsetlinewidth{0.803000pt}%
\definecolor{currentstroke}{rgb}{0.000000,0.000000,0.000000}%
\pgfsetstrokecolor{currentstroke}%
\pgfsetdash{}{0pt}%
\pgfsys@defobject{currentmarker}{\pgfqpoint{0.000000in}{-0.048611in}}{\pgfqpoint{0.000000in}{0.000000in}}{%
\pgfpathmoveto{\pgfqpoint{0.000000in}{0.000000in}}%
\pgfpathlineto{\pgfqpoint{0.000000in}{-0.048611in}}%
\pgfusepath{stroke,fill}%
}%
\begin{pgfscope}%
\pgfsys@transformshift{9.261584in}{3.881603in}%
\pgfsys@useobject{currentmarker}{}%
\end{pgfscope}%
\end{pgfscope}%
\begin{pgfscope}%
\pgfsetbuttcap%
\pgfsetroundjoin%
\definecolor{currentfill}{rgb}{0.000000,0.000000,0.000000}%
\pgfsetfillcolor{currentfill}%
\pgfsetlinewidth{0.803000pt}%
\definecolor{currentstroke}{rgb}{0.000000,0.000000,0.000000}%
\pgfsetstrokecolor{currentstroke}%
\pgfsetdash{}{0pt}%
\pgfsys@defobject{currentmarker}{\pgfqpoint{0.000000in}{-0.048611in}}{\pgfqpoint{0.000000in}{0.000000in}}{%
\pgfpathmoveto{\pgfqpoint{0.000000in}{0.000000in}}%
\pgfpathlineto{\pgfqpoint{0.000000in}{-0.048611in}}%
\pgfusepath{stroke,fill}%
}%
\begin{pgfscope}%
\pgfsys@transformshift{9.741956in}{3.881603in}%
\pgfsys@useobject{currentmarker}{}%
\end{pgfscope}%
\end{pgfscope}%
\begin{pgfscope}%
\pgfsetbuttcap%
\pgfsetroundjoin%
\definecolor{currentfill}{rgb}{0.000000,0.000000,0.000000}%
\pgfsetfillcolor{currentfill}%
\pgfsetlinewidth{0.803000pt}%
\definecolor{currentstroke}{rgb}{0.000000,0.000000,0.000000}%
\pgfsetstrokecolor{currentstroke}%
\pgfsetdash{}{0pt}%
\pgfsys@defobject{currentmarker}{\pgfqpoint{-0.048611in}{0.000000in}}{\pgfqpoint{0.000000in}{0.000000in}}{%
\pgfpathmoveto{\pgfqpoint{0.000000in}{0.000000in}}%
\pgfpathlineto{\pgfqpoint{-0.048611in}{0.000000in}}%
\pgfusepath{stroke,fill}%
}%
\begin{pgfscope}%
\pgfsys@transformshift{5.706832in}{4.335954in}%
\pgfsys@useobject{currentmarker}{}%
\end{pgfscope}%
\end{pgfscope}%
\begin{pgfscope}%
\pgfsetbuttcap%
\pgfsetroundjoin%
\definecolor{currentfill}{rgb}{0.000000,0.000000,0.000000}%
\pgfsetfillcolor{currentfill}%
\pgfsetlinewidth{0.803000pt}%
\definecolor{currentstroke}{rgb}{0.000000,0.000000,0.000000}%
\pgfsetstrokecolor{currentstroke}%
\pgfsetdash{}{0pt}%
\pgfsys@defobject{currentmarker}{\pgfqpoint{-0.048611in}{0.000000in}}{\pgfqpoint{0.000000in}{0.000000in}}{%
\pgfpathmoveto{\pgfqpoint{0.000000in}{0.000000in}}%
\pgfpathlineto{\pgfqpoint{-0.048611in}{0.000000in}}%
\pgfusepath{stroke,fill}%
}%
\begin{pgfscope}%
\pgfsys@transformshift{5.706832in}{4.805404in}%
\pgfsys@useobject{currentmarker}{}%
\end{pgfscope}%
\end{pgfscope}%
\begin{pgfscope}%
\pgfsetbuttcap%
\pgfsetroundjoin%
\definecolor{currentfill}{rgb}{0.000000,0.000000,0.000000}%
\pgfsetfillcolor{currentfill}%
\pgfsetlinewidth{0.803000pt}%
\definecolor{currentstroke}{rgb}{0.000000,0.000000,0.000000}%
\pgfsetstrokecolor{currentstroke}%
\pgfsetdash{}{0pt}%
\pgfsys@defobject{currentmarker}{\pgfqpoint{-0.048611in}{0.000000in}}{\pgfqpoint{0.000000in}{0.000000in}}{%
\pgfpathmoveto{\pgfqpoint{0.000000in}{0.000000in}}%
\pgfpathlineto{\pgfqpoint{-0.048611in}{0.000000in}}%
\pgfusepath{stroke,fill}%
}%
\begin{pgfscope}%
\pgfsys@transformshift{5.706832in}{5.274853in}%
\pgfsys@useobject{currentmarker}{}%
\end{pgfscope}%
\end{pgfscope}%
\begin{pgfscope}%
\pgfsetbuttcap%
\pgfsetroundjoin%
\definecolor{currentfill}{rgb}{0.000000,0.000000,0.000000}%
\pgfsetfillcolor{currentfill}%
\pgfsetlinewidth{0.803000pt}%
\definecolor{currentstroke}{rgb}{0.000000,0.000000,0.000000}%
\pgfsetstrokecolor{currentstroke}%
\pgfsetdash{}{0pt}%
\pgfsys@defobject{currentmarker}{\pgfqpoint{-0.048611in}{0.000000in}}{\pgfqpoint{0.000000in}{0.000000in}}{%
\pgfpathmoveto{\pgfqpoint{0.000000in}{0.000000in}}%
\pgfpathlineto{\pgfqpoint{-0.048611in}{0.000000in}}%
\pgfusepath{stroke,fill}%
}%
\begin{pgfscope}%
\pgfsys@transformshift{5.706832in}{5.744302in}%
\pgfsys@useobject{currentmarker}{}%
\end{pgfscope}%
\end{pgfscope}%
\begin{pgfscope}%
\pgfsetbuttcap%
\pgfsetroundjoin%
\definecolor{currentfill}{rgb}{0.000000,0.000000,0.000000}%
\pgfsetfillcolor{currentfill}%
\pgfsetlinewidth{0.803000pt}%
\definecolor{currentstroke}{rgb}{0.000000,0.000000,0.000000}%
\pgfsetstrokecolor{currentstroke}%
\pgfsetdash{}{0pt}%
\pgfsys@defobject{currentmarker}{\pgfqpoint{-0.048611in}{0.000000in}}{\pgfqpoint{0.000000in}{0.000000in}}{%
\pgfpathmoveto{\pgfqpoint{0.000000in}{0.000000in}}%
\pgfpathlineto{\pgfqpoint{-0.048611in}{0.000000in}}%
\pgfusepath{stroke,fill}%
}%
\begin{pgfscope}%
\pgfsys@transformshift{5.706832in}{6.213751in}%
\pgfsys@useobject{currentmarker}{}%
\end{pgfscope}%
\end{pgfscope}%
\begin{pgfscope}%
\pgfsetbuttcap%
\pgfsetroundjoin%
\definecolor{currentfill}{rgb}{0.000000,0.000000,0.000000}%
\pgfsetfillcolor{currentfill}%
\pgfsetlinewidth{0.803000pt}%
\definecolor{currentstroke}{rgb}{0.000000,0.000000,0.000000}%
\pgfsetstrokecolor{currentstroke}%
\pgfsetdash{}{0pt}%
\pgfsys@defobject{currentmarker}{\pgfqpoint{-0.048611in}{0.000000in}}{\pgfqpoint{0.000000in}{0.000000in}}{%
\pgfpathmoveto{\pgfqpoint{0.000000in}{0.000000in}}%
\pgfpathlineto{\pgfqpoint{-0.048611in}{0.000000in}}%
\pgfusepath{stroke,fill}%
}%
\begin{pgfscope}%
\pgfsys@transformshift{5.706832in}{6.683200in}%
\pgfsys@useobject{currentmarker}{}%
\end{pgfscope}%
\end{pgfscope}%
\begin{pgfscope}%
\pgftext[x=5.651277in,y=5.281603in,,bottom,rotate=90.000000]{\rmfamily\fontsize{10.000000}{12.000000}\selectfont \(\displaystyle y_2\)}%
\end{pgfscope}%
\begin{pgfscope}%
\pgfpathrectangle{\pgfqpoint{5.706832in}{3.881603in}}{\pgfqpoint{4.227273in}{2.800000in}} %
\pgfusepath{clip}%
\pgfsetrectcap%
\pgfsetroundjoin%
\pgfsetlinewidth{0.501875pt}%
\definecolor{currentstroke}{rgb}{0.500000,0.000000,1.000000}%
\pgfsetstrokecolor{currentstroke}%
\pgfsetdash{}{0pt}%
\pgfpathmoveto{\pgfqpoint{5.898981in}{5.410248in}}%
\pgfpathlineto{\pgfqpoint{5.914415in}{5.566349in}}%
\pgfpathlineto{\pgfqpoint{5.929848in}{5.716607in}}%
\pgfpathlineto{\pgfqpoint{5.945282in}{5.858012in}}%
\pgfpathlineto{\pgfqpoint{5.960715in}{5.987728in}}%
\pgfpathlineto{\pgfqpoint{5.976149in}{6.103156in}}%
\pgfpathlineto{\pgfqpoint{5.991583in}{6.201983in}}%
\pgfpathlineto{\pgfqpoint{6.007016in}{6.282227in}}%
\pgfpathlineto{\pgfqpoint{6.022450in}{6.342281in}}%
\pgfpathlineto{\pgfqpoint{6.037884in}{6.380941in}}%
\pgfpathlineto{\pgfqpoint{6.053317in}{6.397432in}}%
\pgfpathlineto{\pgfqpoint{6.068751in}{6.391423in}}%
\pgfpathlineto{\pgfqpoint{6.084185in}{6.363035in}}%
\pgfpathlineto{\pgfqpoint{6.099618in}{6.312837in}}%
\pgfpathlineto{\pgfqpoint{6.115052in}{6.241834in}}%
\pgfpathlineto{\pgfqpoint{6.130485in}{6.151451in}}%
\pgfpathlineto{\pgfqpoint{6.145919in}{6.043498in}}%
\pgfpathlineto{\pgfqpoint{6.161353in}{5.920140in}}%
\pgfpathlineto{\pgfqpoint{6.176786in}{5.783848in}}%
\pgfpathlineto{\pgfqpoint{6.207654in}{5.483596in}}%
\pgfpathlineto{\pgfqpoint{6.269388in}{4.858416in}}%
\pgfpathlineto{\pgfqpoint{6.284822in}{4.715280in}}%
\pgfpathlineto{\pgfqpoint{6.300255in}{4.583359in}}%
\pgfpathlineto{\pgfqpoint{6.315689in}{4.465298in}}%
\pgfpathlineto{\pgfqpoint{6.331123in}{4.363462in}}%
\pgfpathlineto{\pgfqpoint{6.346556in}{4.279893in}}%
\pgfpathlineto{\pgfqpoint{6.361990in}{4.216266in}}%
\pgfpathlineto{\pgfqpoint{6.377424in}{4.173856in}}%
\pgfpathlineto{\pgfqpoint{6.392857in}{4.153513in}}%
\pgfpathlineto{\pgfqpoint{6.408291in}{4.155645in}}%
\pgfpathlineto{\pgfqpoint{6.423724in}{4.180208in}}%
\pgfpathlineto{\pgfqpoint{6.439158in}{4.226712in}}%
\pgfpathlineto{\pgfqpoint{6.454592in}{4.294223in}}%
\pgfpathlineto{\pgfqpoint{6.470025in}{4.381388in}}%
\pgfpathlineto{\pgfqpoint{6.485459in}{4.486461in}}%
\pgfpathlineto{\pgfqpoint{6.500893in}{4.607336in}}%
\pgfpathlineto{\pgfqpoint{6.516326in}{4.741589in}}%
\pgfpathlineto{\pgfqpoint{6.547194in}{5.039255in}}%
\pgfpathlineto{\pgfqpoint{6.608928in}{5.665724in}}%
\pgfpathlineto{\pgfqpoint{6.624362in}{5.810507in}}%
\pgfpathlineto{\pgfqpoint{6.639795in}{5.944554in}}%
\pgfpathlineto{\pgfqpoint{6.655229in}{6.065178in}}%
\pgfpathlineto{\pgfqpoint{6.670663in}{6.169962in}}%
\pgfpathlineto{\pgfqpoint{6.686096in}{6.256806in}}%
\pgfpathlineto{\pgfqpoint{6.701530in}{6.323968in}}%
\pgfpathlineto{\pgfqpoint{6.716964in}{6.370103in}}%
\pgfpathlineto{\pgfqpoint{6.732397in}{6.394287in}}%
\pgfpathlineto{\pgfqpoint{6.747831in}{6.396033in}}%
\pgfpathlineto{\pgfqpoint{6.763264in}{6.375309in}}%
\pgfpathlineto{\pgfqpoint{6.778698in}{6.332528in}}%
\pgfpathlineto{\pgfqpoint{6.794132in}{6.268548in}}%
\pgfpathlineto{\pgfqpoint{6.809565in}{6.184652in}}%
\pgfpathlineto{\pgfqpoint{6.824999in}{6.082521in}}%
\pgfpathlineto{\pgfqpoint{6.840433in}{5.964203in}}%
\pgfpathlineto{\pgfqpoint{6.855866in}{5.832067in}}%
\pgfpathlineto{\pgfqpoint{6.886734in}{5.537165in}}%
\pgfpathlineto{\pgfqpoint{6.963902in}{4.763436in}}%
\pgfpathlineto{\pgfqpoint{6.979335in}{4.627343in}}%
\pgfpathlineto{\pgfqpoint{6.994769in}{4.504227in}}%
\pgfpathlineto{\pgfqpoint{7.010203in}{4.396558in}}%
\pgfpathlineto{\pgfqpoint{7.025636in}{4.306491in}}%
\pgfpathlineto{\pgfqpoint{7.041070in}{4.235833in}}%
\pgfpathlineto{\pgfqpoint{7.056504in}{4.186001in}}%
\pgfpathlineto{\pgfqpoint{7.071937in}{4.157991in}}%
\pgfpathlineto{\pgfqpoint{7.087371in}{4.152367in}}%
\pgfpathlineto{\pgfqpoint{7.102804in}{4.169240in}}%
\pgfpathlineto{\pgfqpoint{7.118238in}{4.208273in}}%
\pgfpathlineto{\pgfqpoint{7.133672in}{4.268683in}}%
\pgfpathlineto{\pgfqpoint{7.149105in}{4.349260in}}%
\pgfpathlineto{\pgfqpoint{7.164539in}{4.448388in}}%
\pgfpathlineto{\pgfqpoint{7.179973in}{4.564081in}}%
\pgfpathlineto{\pgfqpoint{7.195406in}{4.694019in}}%
\pgfpathlineto{\pgfqpoint{7.226274in}{4.985983in}}%
\pgfpathlineto{\pgfqpoint{7.303442in}{5.761728in}}%
\pgfpathlineto{\pgfqpoint{7.318875in}{5.899787in}}%
\pgfpathlineto{\pgfqpoint{7.334309in}{6.025320in}}%
\pgfpathlineto{\pgfqpoint{7.349743in}{6.135812in}}%
\pgfpathlineto{\pgfqpoint{7.365176in}{6.229047in}}%
\pgfpathlineto{\pgfqpoint{7.380610in}{6.303158in}}%
\pgfpathlineto{\pgfqpoint{7.396044in}{6.356659in}}%
\pgfpathlineto{\pgfqpoint{7.411477in}{6.388478in}}%
\pgfpathlineto{\pgfqpoint{7.426911in}{6.397976in}}%
\pgfpathlineto{\pgfqpoint{7.442344in}{6.384964in}}%
\pgfpathlineto{\pgfqpoint{7.457778in}{6.349702in}}%
\pgfpathlineto{\pgfqpoint{7.473212in}{6.292898in}}%
\pgfpathlineto{\pgfqpoint{7.488645in}{6.215689in}}%
\pgfpathlineto{\pgfqpoint{7.504079in}{6.119622in}}%
\pgfpathlineto{\pgfqpoint{7.519513in}{6.006625in}}%
\pgfpathlineto{\pgfqpoint{7.534946in}{5.878961in}}%
\pgfpathlineto{\pgfqpoint{7.565814in}{5.590109in}}%
\pgfpathlineto{\pgfqpoint{7.642982in}{4.812809in}}%
\pgfpathlineto{\pgfqpoint{7.658415in}{4.672867in}}%
\pgfpathlineto{\pgfqpoint{7.673849in}{4.544991in}}%
\pgfpathlineto{\pgfqpoint{7.689283in}{4.431743in}}%
\pgfpathlineto{\pgfqpoint{7.704716in}{4.335393in}}%
\pgfpathlineto{\pgfqpoint{7.720150in}{4.257873in}}%
\pgfpathlineto{\pgfqpoint{7.735584in}{4.200736in}}%
\pgfpathlineto{\pgfqpoint{7.751017in}{4.165127in}}%
\pgfpathlineto{\pgfqpoint{7.766451in}{4.151760in}}%
\pgfpathlineto{\pgfqpoint{7.781884in}{4.160903in}}%
\pgfpathlineto{\pgfqpoint{7.797318in}{4.192373in}}%
\pgfpathlineto{\pgfqpoint{7.812752in}{4.245538in}}%
\pgfpathlineto{\pgfqpoint{7.828185in}{4.319334in}}%
\pgfpathlineto{\pgfqpoint{7.843619in}{4.412281in}}%
\pgfpathlineto{\pgfqpoint{7.859053in}{4.522516in}}%
\pgfpathlineto{\pgfqpoint{7.874486in}{4.647831in}}%
\pgfpathlineto{\pgfqpoint{7.905353in}{4.933397in}}%
\pgfpathlineto{\pgfqpoint{7.997955in}{5.853533in}}%
\pgfpathlineto{\pgfqpoint{8.013389in}{5.983676in}}%
\pgfpathlineto{\pgfqpoint{8.028823in}{6.099613in}}%
\pgfpathlineto{\pgfqpoint{8.044256in}{6.199019in}}%
\pgfpathlineto{\pgfqpoint{8.059690in}{6.279902in}}%
\pgfpathlineto{\pgfqpoint{8.075123in}{6.340641in}}%
\pgfpathlineto{\pgfqpoint{8.090557in}{6.380019in}}%
\pgfpathlineto{\pgfqpoint{8.105991in}{6.397247in}}%
\pgfpathlineto{\pgfqpoint{8.121424in}{6.391978in}}%
\pgfpathlineto{\pgfqpoint{8.136858in}{6.364320in}}%
\pgfpathlineto{\pgfqpoint{8.152292in}{6.314825in}}%
\pgfpathlineto{\pgfqpoint{8.167725in}{6.244486in}}%
\pgfpathlineto{\pgfqpoint{8.183159in}{6.154714in}}%
\pgfpathlineto{\pgfqpoint{8.198593in}{6.047306in}}%
\pgfpathlineto{\pgfqpoint{8.214026in}{5.924416in}}%
\pgfpathlineto{\pgfqpoint{8.229460in}{5.788508in}}%
\pgfpathlineto{\pgfqpoint{8.260327in}{5.488736in}}%
\pgfpathlineto{\pgfqpoint{8.322062in}{4.863281in}}%
\pgfpathlineto{\pgfqpoint{8.337495in}{4.719824in}}%
\pgfpathlineto{\pgfqpoint{8.352929in}{4.587490in}}%
\pgfpathlineto{\pgfqpoint{8.368363in}{4.468934in}}%
\pgfpathlineto{\pgfqpoint{8.383796in}{4.366531in}}%
\pgfpathlineto{\pgfqpoint{8.399230in}{4.282332in}}%
\pgfpathlineto{\pgfqpoint{8.414663in}{4.218027in}}%
\pgfpathlineto{\pgfqpoint{8.430097in}{4.174903in}}%
\pgfpathlineto{\pgfqpoint{8.445531in}{4.153825in}}%
\pgfpathlineto{\pgfqpoint{8.460964in}{4.155216in}}%
\pgfpathlineto{\pgfqpoint{8.476398in}{4.179048in}}%
\pgfpathlineto{\pgfqpoint{8.491832in}{4.224842in}}%
\pgfpathlineto{\pgfqpoint{8.507265in}{4.291682in}}%
\pgfpathlineto{\pgfqpoint{8.522699in}{4.378227in}}%
\pgfpathlineto{\pgfqpoint{8.538133in}{4.482742in}}%
\pgfpathlineto{\pgfqpoint{8.553566in}{4.603134in}}%
\pgfpathlineto{\pgfqpoint{8.569000in}{4.736989in}}%
\pgfpathlineto{\pgfqpoint{8.599867in}{5.034141in}}%
\pgfpathlineto{\pgfqpoint{8.661602in}{5.660814in}}%
\pgfpathlineto{\pgfqpoint{8.677035in}{5.805901in}}%
\pgfpathlineto{\pgfqpoint{8.692469in}{5.940345in}}%
\pgfpathlineto{\pgfqpoint{8.707903in}{6.061451in}}%
\pgfpathlineto{\pgfqpoint{8.723336in}{6.166791in}}%
\pgfpathlineto{\pgfqpoint{8.738770in}{6.254254in}}%
\pgfpathlineto{\pgfqpoint{8.754203in}{6.322087in}}%
\pgfpathlineto{\pgfqpoint{8.769637in}{6.368931in}}%
\pgfpathlineto{\pgfqpoint{8.785071in}{6.393847in}}%
\pgfpathlineto{\pgfqpoint{8.800504in}{6.396334in}}%
\pgfpathlineto{\pgfqpoint{8.815938in}{6.376344in}}%
\pgfpathlineto{\pgfqpoint{8.831372in}{6.334278in}}%
\pgfpathlineto{\pgfqpoint{8.846805in}{6.270977in}}%
\pgfpathlineto{\pgfqpoint{8.862239in}{6.187711in}}%
\pgfpathlineto{\pgfqpoint{8.877673in}{6.086149in}}%
\pgfpathlineto{\pgfqpoint{8.893106in}{5.968327in}}%
\pgfpathlineto{\pgfqpoint{8.908540in}{5.836605in}}%
\pgfpathlineto{\pgfqpoint{8.939407in}{5.542250in}}%
\pgfpathlineto{\pgfqpoint{9.016575in}{4.768101in}}%
\pgfpathlineto{\pgfqpoint{9.032009in}{4.631626in}}%
\pgfpathlineto{\pgfqpoint{9.047443in}{4.508043in}}%
\pgfpathlineto{\pgfqpoint{9.062876in}{4.399829in}}%
\pgfpathlineto{\pgfqpoint{9.078310in}{4.309153in}}%
\pgfpathlineto{\pgfqpoint{9.093743in}{4.237832in}}%
\pgfpathlineto{\pgfqpoint{9.109177in}{4.187297in}}%
\pgfpathlineto{\pgfqpoint{9.124611in}{4.158558in}}%
\pgfpathlineto{\pgfqpoint{9.140044in}{4.152194in}}%
\pgfpathlineto{\pgfqpoint{9.155478in}{4.168330in}}%
\pgfpathlineto{\pgfqpoint{9.170912in}{4.206644in}}%
\pgfpathlineto{\pgfqpoint{9.186345in}{4.266368in}}%
\pgfpathlineto{\pgfqpoint{9.201779in}{4.346305in}}%
\pgfpathlineto{\pgfqpoint{9.217212in}{4.444853in}}%
\pgfpathlineto{\pgfqpoint{9.232646in}{4.560036in}}%
\pgfpathlineto{\pgfqpoint{9.248080in}{4.689546in}}%
\pgfpathlineto{\pgfqpoint{9.278947in}{4.980929in}}%
\pgfpathlineto{\pgfqpoint{9.356115in}{5.757007in}}%
\pgfpathlineto{\pgfqpoint{9.371549in}{5.895431in}}%
\pgfpathlineto{\pgfqpoint{9.386982in}{6.021418in}}%
\pgfpathlineto{\pgfqpoint{9.402416in}{6.132441in}}%
\pgfpathlineto{\pgfqpoint{9.417850in}{6.226276in}}%
\pgfpathlineto{\pgfqpoint{9.433283in}{6.301042in}}%
\pgfpathlineto{\pgfqpoint{9.448717in}{6.355240in}}%
\pgfpathlineto{\pgfqpoint{9.464151in}{6.387784in}}%
\pgfpathlineto{\pgfqpoint{9.479584in}{6.398022in}}%
\pgfpathlineto{\pgfqpoint{9.495018in}{6.385748in}}%
\pgfpathlineto{\pgfqpoint{9.510452in}{6.351210in}}%
\pgfpathlineto{\pgfqpoint{9.525885in}{6.295097in}}%
\pgfpathlineto{\pgfqpoint{9.541319in}{6.218537in}}%
\pgfpathlineto{\pgfqpoint{9.556752in}{6.123062in}}%
\pgfpathlineto{\pgfqpoint{9.572186in}{6.010587in}}%
\pgfpathlineto{\pgfqpoint{9.587620in}{5.883366in}}%
\pgfpathlineto{\pgfqpoint{9.618487in}{5.595129in}}%
\pgfpathlineto{\pgfqpoint{9.695655in}{4.817584in}}%
\pgfpathlineto{\pgfqpoint{9.711089in}{4.677292in}}%
\pgfpathlineto{\pgfqpoint{9.726522in}{4.548976in}}%
\pgfpathlineto{\pgfqpoint{9.741956in}{4.435210in}}%
\pgfpathlineto{\pgfqpoint{9.741956in}{4.435210in}}%
\pgfusepath{stroke}%
\end{pgfscope}%
\begin{pgfscope}%
\pgfpathrectangle{\pgfqpoint{5.706832in}{3.881603in}}{\pgfqpoint{4.227273in}{2.800000in}} %
\pgfusepath{clip}%
\pgfsetrectcap%
\pgfsetroundjoin%
\pgfsetlinewidth{0.501875pt}%
\definecolor{currentstroke}{rgb}{0.421569,0.122888,0.998103}%
\pgfsetstrokecolor{currentstroke}%
\pgfsetdash{}{0pt}%
\pgfpathmoveto{\pgfqpoint{5.898981in}{5.410248in}}%
\pgfpathlineto{\pgfqpoint{5.914415in}{5.566349in}}%
\pgfpathlineto{\pgfqpoint{5.929848in}{5.716607in}}%
\pgfpathlineto{\pgfqpoint{5.945282in}{5.858012in}}%
\pgfpathlineto{\pgfqpoint{5.960715in}{5.987728in}}%
\pgfpathlineto{\pgfqpoint{5.976149in}{6.103156in}}%
\pgfpathlineto{\pgfqpoint{5.991583in}{6.201983in}}%
\pgfpathlineto{\pgfqpoint{6.007016in}{6.282227in}}%
\pgfpathlineto{\pgfqpoint{6.022450in}{6.342281in}}%
\pgfpathlineto{\pgfqpoint{6.037884in}{6.380941in}}%
\pgfpathlineto{\pgfqpoint{6.053317in}{6.397432in}}%
\pgfpathlineto{\pgfqpoint{6.068751in}{6.391423in}}%
\pgfpathlineto{\pgfqpoint{6.084185in}{6.363035in}}%
\pgfpathlineto{\pgfqpoint{6.099618in}{6.312837in}}%
\pgfpathlineto{\pgfqpoint{6.115052in}{6.241834in}}%
\pgfpathlineto{\pgfqpoint{6.130485in}{6.151451in}}%
\pgfpathlineto{\pgfqpoint{6.145919in}{6.043498in}}%
\pgfpathlineto{\pgfqpoint{6.161353in}{5.920140in}}%
\pgfpathlineto{\pgfqpoint{6.176786in}{5.783848in}}%
\pgfpathlineto{\pgfqpoint{6.207654in}{5.483596in}}%
\pgfpathlineto{\pgfqpoint{6.269388in}{4.858416in}}%
\pgfpathlineto{\pgfqpoint{6.284822in}{4.715280in}}%
\pgfpathlineto{\pgfqpoint{6.300255in}{4.583359in}}%
\pgfpathlineto{\pgfqpoint{6.315689in}{4.465298in}}%
\pgfpathlineto{\pgfqpoint{6.331123in}{4.363462in}}%
\pgfpathlineto{\pgfqpoint{6.346556in}{4.279893in}}%
\pgfpathlineto{\pgfqpoint{6.361990in}{4.216266in}}%
\pgfpathlineto{\pgfqpoint{6.377424in}{4.173856in}}%
\pgfpathlineto{\pgfqpoint{6.392857in}{4.153513in}}%
\pgfpathlineto{\pgfqpoint{6.408291in}{4.155645in}}%
\pgfpathlineto{\pgfqpoint{6.423724in}{4.180208in}}%
\pgfpathlineto{\pgfqpoint{6.439158in}{4.226712in}}%
\pgfpathlineto{\pgfqpoint{6.454592in}{4.294223in}}%
\pgfpathlineto{\pgfqpoint{6.470025in}{4.381388in}}%
\pgfpathlineto{\pgfqpoint{6.485459in}{4.486461in}}%
\pgfpathlineto{\pgfqpoint{6.500893in}{4.607336in}}%
\pgfpathlineto{\pgfqpoint{6.516326in}{4.741589in}}%
\pgfpathlineto{\pgfqpoint{6.547194in}{5.039255in}}%
\pgfpathlineto{\pgfqpoint{6.608928in}{5.665724in}}%
\pgfpathlineto{\pgfqpoint{6.624362in}{5.810507in}}%
\pgfpathlineto{\pgfqpoint{6.639795in}{5.944554in}}%
\pgfpathlineto{\pgfqpoint{6.655229in}{6.065178in}}%
\pgfpathlineto{\pgfqpoint{6.670663in}{6.169962in}}%
\pgfpathlineto{\pgfqpoint{6.686096in}{6.256806in}}%
\pgfpathlineto{\pgfqpoint{6.701530in}{6.323968in}}%
\pgfpathlineto{\pgfqpoint{6.716964in}{6.370103in}}%
\pgfpathlineto{\pgfqpoint{6.732397in}{6.394287in}}%
\pgfpathlineto{\pgfqpoint{6.747831in}{6.396033in}}%
\pgfpathlineto{\pgfqpoint{6.763264in}{6.375309in}}%
\pgfpathlineto{\pgfqpoint{6.778698in}{6.332528in}}%
\pgfpathlineto{\pgfqpoint{6.794132in}{6.268548in}}%
\pgfpathlineto{\pgfqpoint{6.809565in}{6.184652in}}%
\pgfpathlineto{\pgfqpoint{6.824999in}{6.082521in}}%
\pgfpathlineto{\pgfqpoint{6.840433in}{5.964203in}}%
\pgfpathlineto{\pgfqpoint{6.855866in}{5.832067in}}%
\pgfpathlineto{\pgfqpoint{6.886734in}{5.537165in}}%
\pgfpathlineto{\pgfqpoint{6.963902in}{4.763436in}}%
\pgfpathlineto{\pgfqpoint{6.979335in}{4.627343in}}%
\pgfpathlineto{\pgfqpoint{6.994769in}{4.504227in}}%
\pgfpathlineto{\pgfqpoint{7.010203in}{4.396558in}}%
\pgfpathlineto{\pgfqpoint{7.025636in}{4.306491in}}%
\pgfpathlineto{\pgfqpoint{7.041070in}{4.235833in}}%
\pgfpathlineto{\pgfqpoint{7.056504in}{4.186001in}}%
\pgfpathlineto{\pgfqpoint{7.071937in}{4.157991in}}%
\pgfpathlineto{\pgfqpoint{7.087371in}{4.152367in}}%
\pgfpathlineto{\pgfqpoint{7.102804in}{4.169240in}}%
\pgfpathlineto{\pgfqpoint{7.118238in}{4.208273in}}%
\pgfpathlineto{\pgfqpoint{7.133672in}{4.268683in}}%
\pgfpathlineto{\pgfqpoint{7.149105in}{4.349260in}}%
\pgfpathlineto{\pgfqpoint{7.164539in}{4.448388in}}%
\pgfpathlineto{\pgfqpoint{7.179973in}{4.564081in}}%
\pgfpathlineto{\pgfqpoint{7.195406in}{4.694019in}}%
\pgfpathlineto{\pgfqpoint{7.226274in}{4.985983in}}%
\pgfpathlineto{\pgfqpoint{7.303442in}{5.761728in}}%
\pgfpathlineto{\pgfqpoint{7.318875in}{5.899787in}}%
\pgfpathlineto{\pgfqpoint{7.334309in}{6.025320in}}%
\pgfpathlineto{\pgfqpoint{7.349743in}{6.135812in}}%
\pgfpathlineto{\pgfqpoint{7.365176in}{6.229047in}}%
\pgfpathlineto{\pgfqpoint{7.380610in}{6.303158in}}%
\pgfpathlineto{\pgfqpoint{7.396044in}{6.356659in}}%
\pgfpathlineto{\pgfqpoint{7.411477in}{6.388478in}}%
\pgfpathlineto{\pgfqpoint{7.426911in}{6.397976in}}%
\pgfpathlineto{\pgfqpoint{7.442344in}{6.384964in}}%
\pgfpathlineto{\pgfqpoint{7.457778in}{6.349702in}}%
\pgfpathlineto{\pgfqpoint{7.473212in}{6.292898in}}%
\pgfpathlineto{\pgfqpoint{7.488645in}{6.215689in}}%
\pgfpathlineto{\pgfqpoint{7.504079in}{6.119622in}}%
\pgfpathlineto{\pgfqpoint{7.519513in}{6.006625in}}%
\pgfpathlineto{\pgfqpoint{7.534946in}{5.878961in}}%
\pgfpathlineto{\pgfqpoint{7.565814in}{5.590109in}}%
\pgfpathlineto{\pgfqpoint{7.642982in}{4.812809in}}%
\pgfpathlineto{\pgfqpoint{7.658415in}{4.672867in}}%
\pgfpathlineto{\pgfqpoint{7.673849in}{4.544991in}}%
\pgfpathlineto{\pgfqpoint{7.689283in}{4.431743in}}%
\pgfpathlineto{\pgfqpoint{7.704716in}{4.335393in}}%
\pgfpathlineto{\pgfqpoint{7.720150in}{4.257873in}}%
\pgfpathlineto{\pgfqpoint{7.735584in}{4.200736in}}%
\pgfpathlineto{\pgfqpoint{7.751017in}{4.165127in}}%
\pgfpathlineto{\pgfqpoint{7.766451in}{4.151760in}}%
\pgfpathlineto{\pgfqpoint{7.781884in}{4.160903in}}%
\pgfpathlineto{\pgfqpoint{7.797318in}{4.192373in}}%
\pgfpathlineto{\pgfqpoint{7.812752in}{4.245538in}}%
\pgfpathlineto{\pgfqpoint{7.828185in}{4.319334in}}%
\pgfpathlineto{\pgfqpoint{7.843619in}{4.412281in}}%
\pgfpathlineto{\pgfqpoint{7.859053in}{4.522516in}}%
\pgfpathlineto{\pgfqpoint{7.874486in}{4.647831in}}%
\pgfpathlineto{\pgfqpoint{7.905353in}{4.933397in}}%
\pgfpathlineto{\pgfqpoint{7.997955in}{5.853533in}}%
\pgfpathlineto{\pgfqpoint{8.013389in}{5.983676in}}%
\pgfpathlineto{\pgfqpoint{8.028823in}{6.099613in}}%
\pgfpathlineto{\pgfqpoint{8.044256in}{6.199019in}}%
\pgfpathlineto{\pgfqpoint{8.059690in}{6.279902in}}%
\pgfpathlineto{\pgfqpoint{8.075123in}{6.340641in}}%
\pgfpathlineto{\pgfqpoint{8.090557in}{6.380019in}}%
\pgfpathlineto{\pgfqpoint{8.105991in}{6.397247in}}%
\pgfpathlineto{\pgfqpoint{8.121424in}{6.391978in}}%
\pgfpathlineto{\pgfqpoint{8.136858in}{6.364320in}}%
\pgfpathlineto{\pgfqpoint{8.152292in}{6.314825in}}%
\pgfpathlineto{\pgfqpoint{8.167725in}{6.244486in}}%
\pgfpathlineto{\pgfqpoint{8.183159in}{6.154714in}}%
\pgfpathlineto{\pgfqpoint{8.198593in}{6.047306in}}%
\pgfpathlineto{\pgfqpoint{8.214026in}{5.924416in}}%
\pgfpathlineto{\pgfqpoint{8.229460in}{5.788508in}}%
\pgfpathlineto{\pgfqpoint{8.260327in}{5.488736in}}%
\pgfpathlineto{\pgfqpoint{8.322062in}{4.863281in}}%
\pgfpathlineto{\pgfqpoint{8.337495in}{4.719824in}}%
\pgfpathlineto{\pgfqpoint{8.352929in}{4.587490in}}%
\pgfpathlineto{\pgfqpoint{8.368363in}{4.468934in}}%
\pgfpathlineto{\pgfqpoint{8.383796in}{4.366531in}}%
\pgfpathlineto{\pgfqpoint{8.399230in}{4.282332in}}%
\pgfpathlineto{\pgfqpoint{8.414663in}{4.218027in}}%
\pgfpathlineto{\pgfqpoint{8.430097in}{4.174903in}}%
\pgfpathlineto{\pgfqpoint{8.445531in}{4.153825in}}%
\pgfpathlineto{\pgfqpoint{8.460964in}{4.155216in}}%
\pgfpathlineto{\pgfqpoint{8.476398in}{4.179048in}}%
\pgfpathlineto{\pgfqpoint{8.491832in}{4.224842in}}%
\pgfpathlineto{\pgfqpoint{8.507265in}{4.291682in}}%
\pgfpathlineto{\pgfqpoint{8.522699in}{4.378227in}}%
\pgfpathlineto{\pgfqpoint{8.538133in}{4.482742in}}%
\pgfpathlineto{\pgfqpoint{8.553566in}{4.603134in}}%
\pgfpathlineto{\pgfqpoint{8.569000in}{4.736989in}}%
\pgfpathlineto{\pgfqpoint{8.599867in}{5.034141in}}%
\pgfpathlineto{\pgfqpoint{8.661602in}{5.660814in}}%
\pgfpathlineto{\pgfqpoint{8.677035in}{5.805901in}}%
\pgfpathlineto{\pgfqpoint{8.692469in}{5.940345in}}%
\pgfpathlineto{\pgfqpoint{8.707903in}{6.061451in}}%
\pgfpathlineto{\pgfqpoint{8.723336in}{6.166791in}}%
\pgfpathlineto{\pgfqpoint{8.738770in}{6.254254in}}%
\pgfpathlineto{\pgfqpoint{8.754203in}{6.322087in}}%
\pgfpathlineto{\pgfqpoint{8.769637in}{6.368931in}}%
\pgfpathlineto{\pgfqpoint{8.785071in}{6.393847in}}%
\pgfpathlineto{\pgfqpoint{8.800504in}{6.396334in}}%
\pgfpathlineto{\pgfqpoint{8.815938in}{6.376344in}}%
\pgfpathlineto{\pgfqpoint{8.831372in}{6.334278in}}%
\pgfpathlineto{\pgfqpoint{8.846805in}{6.270977in}}%
\pgfpathlineto{\pgfqpoint{8.862239in}{6.187711in}}%
\pgfpathlineto{\pgfqpoint{8.877673in}{6.086149in}}%
\pgfpathlineto{\pgfqpoint{8.893106in}{5.968327in}}%
\pgfpathlineto{\pgfqpoint{8.908540in}{5.836605in}}%
\pgfpathlineto{\pgfqpoint{8.939407in}{5.542250in}}%
\pgfpathlineto{\pgfqpoint{9.016575in}{4.768101in}}%
\pgfpathlineto{\pgfqpoint{9.032009in}{4.631626in}}%
\pgfpathlineto{\pgfqpoint{9.047443in}{4.508043in}}%
\pgfpathlineto{\pgfqpoint{9.062876in}{4.399829in}}%
\pgfpathlineto{\pgfqpoint{9.078310in}{4.309153in}}%
\pgfpathlineto{\pgfqpoint{9.093743in}{4.237832in}}%
\pgfpathlineto{\pgfqpoint{9.109177in}{4.187297in}}%
\pgfpathlineto{\pgfqpoint{9.124611in}{4.158558in}}%
\pgfpathlineto{\pgfqpoint{9.140044in}{4.152194in}}%
\pgfpathlineto{\pgfqpoint{9.155478in}{4.168330in}}%
\pgfpathlineto{\pgfqpoint{9.170912in}{4.206644in}}%
\pgfpathlineto{\pgfqpoint{9.186345in}{4.266368in}}%
\pgfpathlineto{\pgfqpoint{9.201779in}{4.346305in}}%
\pgfpathlineto{\pgfqpoint{9.217212in}{4.444853in}}%
\pgfpathlineto{\pgfqpoint{9.232646in}{4.560036in}}%
\pgfpathlineto{\pgfqpoint{9.248080in}{4.689546in}}%
\pgfpathlineto{\pgfqpoint{9.278947in}{4.980929in}}%
\pgfpathlineto{\pgfqpoint{9.356115in}{5.757007in}}%
\pgfpathlineto{\pgfqpoint{9.371549in}{5.895431in}}%
\pgfpathlineto{\pgfqpoint{9.386982in}{6.021418in}}%
\pgfpathlineto{\pgfqpoint{9.402416in}{6.132441in}}%
\pgfpathlineto{\pgfqpoint{9.417850in}{6.226276in}}%
\pgfpathlineto{\pgfqpoint{9.433283in}{6.301042in}}%
\pgfpathlineto{\pgfqpoint{9.448717in}{6.355240in}}%
\pgfpathlineto{\pgfqpoint{9.464151in}{6.387784in}}%
\pgfpathlineto{\pgfqpoint{9.479584in}{6.398022in}}%
\pgfpathlineto{\pgfqpoint{9.495018in}{6.385748in}}%
\pgfpathlineto{\pgfqpoint{9.510452in}{6.351210in}}%
\pgfpathlineto{\pgfqpoint{9.525885in}{6.295097in}}%
\pgfpathlineto{\pgfqpoint{9.541319in}{6.218537in}}%
\pgfpathlineto{\pgfqpoint{9.556752in}{6.123062in}}%
\pgfpathlineto{\pgfqpoint{9.572186in}{6.010587in}}%
\pgfpathlineto{\pgfqpoint{9.587620in}{5.883366in}}%
\pgfpathlineto{\pgfqpoint{9.618487in}{5.595129in}}%
\pgfpathlineto{\pgfqpoint{9.695655in}{4.817584in}}%
\pgfpathlineto{\pgfqpoint{9.711089in}{4.677292in}}%
\pgfpathlineto{\pgfqpoint{9.726522in}{4.548976in}}%
\pgfpathlineto{\pgfqpoint{9.741956in}{4.435210in}}%
\pgfpathlineto{\pgfqpoint{9.741956in}{4.435210in}}%
\pgfusepath{stroke}%
\end{pgfscope}%
\begin{pgfscope}%
\pgfpathrectangle{\pgfqpoint{5.706832in}{3.881603in}}{\pgfqpoint{4.227273in}{2.800000in}} %
\pgfusepath{clip}%
\pgfsetrectcap%
\pgfsetroundjoin%
\pgfsetlinewidth{0.501875pt}%
\definecolor{currentstroke}{rgb}{0.343137,0.243914,0.992421}%
\pgfsetstrokecolor{currentstroke}%
\pgfsetdash{}{0pt}%
\pgfpathmoveto{\pgfqpoint{5.898981in}{5.242066in}}%
\pgfpathlineto{\pgfqpoint{5.914415in}{5.333846in}}%
\pgfpathlineto{\pgfqpoint{5.929848in}{5.422795in}}%
\pgfpathlineto{\pgfqpoint{5.960715in}{5.586207in}}%
\pgfpathlineto{\pgfqpoint{5.976149in}{5.657924in}}%
\pgfpathlineto{\pgfqpoint{5.991583in}{5.721372in}}%
\pgfpathlineto{\pgfqpoint{6.007016in}{5.775547in}}%
\pgfpathlineto{\pgfqpoint{6.022450in}{5.819629in}}%
\pgfpathlineto{\pgfqpoint{6.037884in}{5.853003in}}%
\pgfpathlineto{\pgfqpoint{6.053317in}{5.875267in}}%
\pgfpathlineto{\pgfqpoint{6.068751in}{5.886240in}}%
\pgfpathlineto{\pgfqpoint{6.084185in}{5.885969in}}%
\pgfpathlineto{\pgfqpoint{6.099618in}{5.874724in}}%
\pgfpathlineto{\pgfqpoint{6.115052in}{5.852993in}}%
\pgfpathlineto{\pgfqpoint{6.130485in}{5.821474in}}%
\pgfpathlineto{\pgfqpoint{6.145919in}{5.781058in}}%
\pgfpathlineto{\pgfqpoint{6.161353in}{5.732813in}}%
\pgfpathlineto{\pgfqpoint{6.176786in}{5.677962in}}%
\pgfpathlineto{\pgfqpoint{6.207654in}{5.553951in}}%
\pgfpathlineto{\pgfqpoint{6.269388in}{5.291322in}}%
\pgfpathlineto{\pgfqpoint{6.284822in}{5.231696in}}%
\pgfpathlineto{\pgfqpoint{6.300255in}{5.177433in}}%
\pgfpathlineto{\pgfqpoint{6.315689in}{5.129843in}}%
\pgfpathlineto{\pgfqpoint{6.331123in}{5.090098in}}%
\pgfpathlineto{\pgfqpoint{6.346556in}{5.059206in}}%
\pgfpathlineto{\pgfqpoint{6.361990in}{5.037994in}}%
\pgfpathlineto{\pgfqpoint{6.377424in}{5.027088in}}%
\pgfpathlineto{\pgfqpoint{6.392857in}{5.026903in}}%
\pgfpathlineto{\pgfqpoint{6.408291in}{5.037633in}}%
\pgfpathlineto{\pgfqpoint{6.423724in}{5.059245in}}%
\pgfpathlineto{\pgfqpoint{6.439158in}{5.091485in}}%
\pgfpathlineto{\pgfqpoint{6.454592in}{5.133876in}}%
\pgfpathlineto{\pgfqpoint{6.470025in}{5.185734in}}%
\pgfpathlineto{\pgfqpoint{6.485459in}{5.246176in}}%
\pgfpathlineto{\pgfqpoint{6.500893in}{5.314142in}}%
\pgfpathlineto{\pgfqpoint{6.531760in}{5.467633in}}%
\pgfpathlineto{\pgfqpoint{6.624362in}{5.961513in}}%
\pgfpathlineto{\pgfqpoint{6.639795in}{6.032366in}}%
\pgfpathlineto{\pgfqpoint{6.655229in}{6.096071in}}%
\pgfpathlineto{\pgfqpoint{6.670663in}{6.151416in}}%
\pgfpathlineto{\pgfqpoint{6.686096in}{6.197347in}}%
\pgfpathlineto{\pgfqpoint{6.701530in}{6.232992in}}%
\pgfpathlineto{\pgfqpoint{6.716964in}{6.257675in}}%
\pgfpathlineto{\pgfqpoint{6.732397in}{6.270934in}}%
\pgfpathlineto{\pgfqpoint{6.747831in}{6.272523in}}%
\pgfpathlineto{\pgfqpoint{6.763264in}{6.262424in}}%
\pgfpathlineto{\pgfqpoint{6.778698in}{6.240846in}}%
\pgfpathlineto{\pgfqpoint{6.794132in}{6.208217in}}%
\pgfpathlineto{\pgfqpoint{6.809565in}{6.165179in}}%
\pgfpathlineto{\pgfqpoint{6.824999in}{6.112574in}}%
\pgfpathlineto{\pgfqpoint{6.840433in}{6.051427in}}%
\pgfpathlineto{\pgfqpoint{6.855866in}{5.982925in}}%
\pgfpathlineto{\pgfqpoint{6.886734in}{5.829277in}}%
\pgfpathlineto{\pgfqpoint{6.963902in}{5.419347in}}%
\pgfpathlineto{\pgfqpoint{6.979335in}{5.345425in}}%
\pgfpathlineto{\pgfqpoint{6.994769in}{5.277586in}}%
\pgfpathlineto{\pgfqpoint{7.010203in}{5.217068in}}%
\pgfpathlineto{\pgfqpoint{7.025636in}{5.164958in}}%
\pgfpathlineto{\pgfqpoint{7.041070in}{5.122166in}}%
\pgfpathlineto{\pgfqpoint{7.056504in}{5.089407in}}%
\pgfpathlineto{\pgfqpoint{7.071937in}{5.067189in}}%
\pgfpathlineto{\pgfqpoint{7.087371in}{5.055802in}}%
\pgfpathlineto{\pgfqpoint{7.102804in}{5.055309in}}%
\pgfpathlineto{\pgfqpoint{7.118238in}{5.065553in}}%
\pgfpathlineto{\pgfqpoint{7.133672in}{5.086150in}}%
\pgfpathlineto{\pgfqpoint{7.149105in}{5.116505in}}%
\pgfpathlineto{\pgfqpoint{7.164539in}{5.155822in}}%
\pgfpathlineto{\pgfqpoint{7.179973in}{5.203118in}}%
\pgfpathlineto{\pgfqpoint{7.195406in}{5.257243in}}%
\pgfpathlineto{\pgfqpoint{7.226274in}{5.380703in}}%
\pgfpathlineto{\pgfqpoint{7.303442in}{5.707962in}}%
\pgfpathlineto{\pgfqpoint{7.318875in}{5.764229in}}%
\pgfpathlineto{\pgfqpoint{7.334309in}{5.814051in}}%
\pgfpathlineto{\pgfqpoint{7.349743in}{5.856183in}}%
\pgfpathlineto{\pgfqpoint{7.365176in}{5.889531in}}%
\pgfpathlineto{\pgfqpoint{7.380610in}{5.913175in}}%
\pgfpathlineto{\pgfqpoint{7.396044in}{5.926385in}}%
\pgfpathlineto{\pgfqpoint{7.411477in}{5.928640in}}%
\pgfpathlineto{\pgfqpoint{7.426911in}{5.919635in}}%
\pgfpathlineto{\pgfqpoint{7.442344in}{5.899288in}}%
\pgfpathlineto{\pgfqpoint{7.457778in}{5.867745in}}%
\pgfpathlineto{\pgfqpoint{7.473212in}{5.825373in}}%
\pgfpathlineto{\pgfqpoint{7.488645in}{5.772756in}}%
\pgfpathlineto{\pgfqpoint{7.504079in}{5.710682in}}%
\pgfpathlineto{\pgfqpoint{7.519513in}{5.640129in}}%
\pgfpathlineto{\pgfqpoint{7.534946in}{5.562243in}}%
\pgfpathlineto{\pgfqpoint{7.565814in}{5.389769in}}%
\pgfpathlineto{\pgfqpoint{7.642982in}{4.932022in}}%
\pgfpathlineto{\pgfqpoint{7.658415in}{4.848521in}}%
\pgfpathlineto{\pgfqpoint{7.673849in}{4.771183in}}%
\pgfpathlineto{\pgfqpoint{7.689283in}{4.701302in}}%
\pgfpathlineto{\pgfqpoint{7.704716in}{4.640028in}}%
\pgfpathlineto{\pgfqpoint{7.720150in}{4.588338in}}%
\pgfpathlineto{\pgfqpoint{7.735584in}{4.547025in}}%
\pgfpathlineto{\pgfqpoint{7.751017in}{4.516672in}}%
\pgfpathlineto{\pgfqpoint{7.766451in}{4.497650in}}%
\pgfpathlineto{\pgfqpoint{7.781884in}{4.490106in}}%
\pgfpathlineto{\pgfqpoint{7.797318in}{4.493961in}}%
\pgfpathlineto{\pgfqpoint{7.812752in}{4.508911in}}%
\pgfpathlineto{\pgfqpoint{7.828185in}{4.534436in}}%
\pgfpathlineto{\pgfqpoint{7.843619in}{4.569807in}}%
\pgfpathlineto{\pgfqpoint{7.859053in}{4.614106in}}%
\pgfpathlineto{\pgfqpoint{7.874486in}{4.666238in}}%
\pgfpathlineto{\pgfqpoint{7.889920in}{4.724959in}}%
\pgfpathlineto{\pgfqpoint{7.920787in}{4.856582in}}%
\pgfpathlineto{\pgfqpoint{7.982522in}{5.133646in}}%
\pgfpathlineto{\pgfqpoint{7.997955in}{5.196707in}}%
\pgfpathlineto{\pgfqpoint{8.013389in}{5.254349in}}%
\pgfpathlineto{\pgfqpoint{8.028823in}{5.305274in}}%
\pgfpathlineto{\pgfqpoint{8.044256in}{5.348329in}}%
\pgfpathlineto{\pgfqpoint{8.059690in}{5.382524in}}%
\pgfpathlineto{\pgfqpoint{8.075123in}{5.407054in}}%
\pgfpathlineto{\pgfqpoint{8.090557in}{5.421318in}}%
\pgfpathlineto{\pgfqpoint{8.105991in}{5.424926in}}%
\pgfpathlineto{\pgfqpoint{8.121424in}{5.417711in}}%
\pgfpathlineto{\pgfqpoint{8.136858in}{5.399730in}}%
\pgfpathlineto{\pgfqpoint{8.152292in}{5.371265in}}%
\pgfpathlineto{\pgfqpoint{8.167725in}{5.332815in}}%
\pgfpathlineto{\pgfqpoint{8.183159in}{5.285089in}}%
\pgfpathlineto{\pgfqpoint{8.198593in}{5.228990in}}%
\pgfpathlineto{\pgfqpoint{8.214026in}{5.165596in}}%
\pgfpathlineto{\pgfqpoint{8.244893in}{5.021988in}}%
\pgfpathlineto{\pgfqpoint{8.322062in}{4.633992in}}%
\pgfpathlineto{\pgfqpoint{8.337495in}{4.563918in}}%
\pgfpathlineto{\pgfqpoint{8.352929in}{4.499854in}}%
\pgfpathlineto{\pgfqpoint{8.368363in}{4.443124in}}%
\pgfpathlineto{\pgfqpoint{8.383796in}{4.394913in}}%
\pgfpathlineto{\pgfqpoint{8.399230in}{4.356245in}}%
\pgfpathlineto{\pgfqpoint{8.414663in}{4.327960in}}%
\pgfpathlineto{\pgfqpoint{8.430097in}{4.310697in}}%
\pgfpathlineto{\pgfqpoint{8.445531in}{4.304886in}}%
\pgfpathlineto{\pgfqpoint{8.460964in}{4.310731in}}%
\pgfpathlineto{\pgfqpoint{8.476398in}{4.328215in}}%
\pgfpathlineto{\pgfqpoint{8.491832in}{4.357092in}}%
\pgfpathlineto{\pgfqpoint{8.507265in}{4.396897in}}%
\pgfpathlineto{\pgfqpoint{8.522699in}{4.446954in}}%
\pgfpathlineto{\pgfqpoint{8.538133in}{4.506388in}}%
\pgfpathlineto{\pgfqpoint{8.553566in}{4.574145in}}%
\pgfpathlineto{\pgfqpoint{8.569000in}{4.649011in}}%
\pgfpathlineto{\pgfqpoint{8.599867in}{4.814561in}}%
\pgfpathlineto{\pgfqpoint{8.677035in}{5.248851in}}%
\pgfpathlineto{\pgfqpoint{8.692469in}{5.326700in}}%
\pgfpathlineto{\pgfqpoint{8.707903in}{5.398171in}}%
\pgfpathlineto{\pgfqpoint{8.723336in}{5.462037in}}%
\pgfpathlineto{\pgfqpoint{8.738770in}{5.517232in}}%
\pgfpathlineto{\pgfqpoint{8.754203in}{5.562867in}}%
\pgfpathlineto{\pgfqpoint{8.769637in}{5.598252in}}%
\pgfpathlineto{\pgfqpoint{8.785071in}{5.622903in}}%
\pgfpathlineto{\pgfqpoint{8.800504in}{5.636558in}}%
\pgfpathlineto{\pgfqpoint{8.815938in}{5.639181in}}%
\pgfpathlineto{\pgfqpoint{8.831372in}{5.630957in}}%
\pgfpathlineto{\pgfqpoint{8.846805in}{5.612296in}}%
\pgfpathlineto{\pgfqpoint{8.862239in}{5.583819in}}%
\pgfpathlineto{\pgfqpoint{8.877673in}{5.546345in}}%
\pgfpathlineto{\pgfqpoint{8.893106in}{5.500880in}}%
\pgfpathlineto{\pgfqpoint{8.908540in}{5.448590in}}%
\pgfpathlineto{\pgfqpoint{8.939407in}{5.328872in}}%
\pgfpathlineto{\pgfqpoint{9.001142in}{5.070904in}}%
\pgfpathlineto{\pgfqpoint{9.016575in}{5.011634in}}%
\pgfpathlineto{\pgfqpoint{9.032009in}{4.957453in}}%
\pgfpathlineto{\pgfqpoint{9.047443in}{4.909713in}}%
\pgfpathlineto{\pgfqpoint{9.062876in}{4.869639in}}%
\pgfpathlineto{\pgfqpoint{9.078310in}{4.838300in}}%
\pgfpathlineto{\pgfqpoint{9.093743in}{4.816591in}}%
\pgfpathlineto{\pgfqpoint{9.109177in}{4.805212in}}%
\pgfpathlineto{\pgfqpoint{9.124611in}{4.804656in}}%
\pgfpathlineto{\pgfqpoint{9.140044in}{4.815195in}}%
\pgfpathlineto{\pgfqpoint{9.155478in}{4.836880in}}%
\pgfpathlineto{\pgfqpoint{9.170912in}{4.869535in}}%
\pgfpathlineto{\pgfqpoint{9.186345in}{4.912762in}}%
\pgfpathlineto{\pgfqpoint{9.201779in}{4.965948in}}%
\pgfpathlineto{\pgfqpoint{9.217212in}{5.028279in}}%
\pgfpathlineto{\pgfqpoint{9.232646in}{5.098754in}}%
\pgfpathlineto{\pgfqpoint{9.263513in}{5.259321in}}%
\pgfpathlineto{\pgfqpoint{9.294381in}{5.436747in}}%
\pgfpathlineto{\pgfqpoint{9.340682in}{5.707416in}}%
\pgfpathlineto{\pgfqpoint{9.371549in}{5.872447in}}%
\pgfpathlineto{\pgfqpoint{9.386982in}{5.945888in}}%
\pgfpathlineto{\pgfqpoint{9.402416in}{6.011540in}}%
\pgfpathlineto{\pgfqpoint{9.417850in}{6.068280in}}%
\pgfpathlineto{\pgfqpoint{9.433283in}{6.115157in}}%
\pgfpathlineto{\pgfqpoint{9.448717in}{6.151412in}}%
\pgfpathlineto{\pgfqpoint{9.464151in}{6.176493in}}%
\pgfpathlineto{\pgfqpoint{9.479584in}{6.190064in}}%
\pgfpathlineto{\pgfqpoint{9.495018in}{6.192014in}}%
\pgfpathlineto{\pgfqpoint{9.510452in}{6.182458in}}%
\pgfpathlineto{\pgfqpoint{9.525885in}{6.161733in}}%
\pgfpathlineto{\pgfqpoint{9.541319in}{6.130396in}}%
\pgfpathlineto{\pgfqpoint{9.556752in}{6.089204in}}%
\pgfpathlineto{\pgfqpoint{9.572186in}{6.039110in}}%
\pgfpathlineto{\pgfqpoint{9.587620in}{5.981233in}}%
\pgfpathlineto{\pgfqpoint{9.603053in}{5.916842in}}%
\pgfpathlineto{\pgfqpoint{9.633921in}{5.774181in}}%
\pgfpathlineto{\pgfqpoint{9.695655in}{5.476568in}}%
\pgfpathlineto{\pgfqpoint{9.711089in}{5.408704in}}%
\pgfpathlineto{\pgfqpoint{9.726522in}{5.346381in}}%
\pgfpathlineto{\pgfqpoint{9.741956in}{5.290883in}}%
\pgfpathlineto{\pgfqpoint{9.741956in}{5.290883in}}%
\pgfusepath{stroke}%
\end{pgfscope}%
\begin{pgfscope}%
\pgfpathrectangle{\pgfqpoint{5.706832in}{3.881603in}}{\pgfqpoint{4.227273in}{2.800000in}} %
\pgfusepath{clip}%
\pgfsetrectcap%
\pgfsetroundjoin%
\pgfsetlinewidth{0.501875pt}%
\definecolor{currentstroke}{rgb}{0.264706,0.361242,0.982973}%
\pgfsetstrokecolor{currentstroke}%
\pgfsetdash{}{0pt}%
\pgfpathmoveto{\pgfqpoint{5.898981in}{5.377888in}}%
\pgfpathlineto{\pgfqpoint{5.914415in}{5.401089in}}%
\pgfpathlineto{\pgfqpoint{5.929848in}{5.421866in}}%
\pgfpathlineto{\pgfqpoint{5.945282in}{5.439442in}}%
\pgfpathlineto{\pgfqpoint{5.960715in}{5.453104in}}%
\pgfpathlineto{\pgfqpoint{5.976149in}{5.462216in}}%
\pgfpathlineto{\pgfqpoint{5.991583in}{5.466235in}}%
\pgfpathlineto{\pgfqpoint{6.007016in}{5.464728in}}%
\pgfpathlineto{\pgfqpoint{6.022450in}{5.457383in}}%
\pgfpathlineto{\pgfqpoint{6.037884in}{5.444014in}}%
\pgfpathlineto{\pgfqpoint{6.053317in}{5.424574in}}%
\pgfpathlineto{\pgfqpoint{6.068751in}{5.399155in}}%
\pgfpathlineto{\pgfqpoint{6.084185in}{5.367989in}}%
\pgfpathlineto{\pgfqpoint{6.099618in}{5.331449in}}%
\pgfpathlineto{\pgfqpoint{6.115052in}{5.290039in}}%
\pgfpathlineto{\pgfqpoint{6.130485in}{5.244390in}}%
\pgfpathlineto{\pgfqpoint{6.161353in}{5.143465in}}%
\pgfpathlineto{\pgfqpoint{6.238521in}{4.879705in}}%
\pgfpathlineto{\pgfqpoint{6.253955in}{4.833481in}}%
\pgfpathlineto{\pgfqpoint{6.269388in}{4.791937in}}%
\pgfpathlineto{\pgfqpoint{6.284822in}{4.756071in}}%
\pgfpathlineto{\pgfqpoint{6.300255in}{4.726807in}}%
\pgfpathlineto{\pgfqpoint{6.315689in}{4.704976in}}%
\pgfpathlineto{\pgfqpoint{6.331123in}{4.691298in}}%
\pgfpathlineto{\pgfqpoint{6.346556in}{4.686366in}}%
\pgfpathlineto{\pgfqpoint{6.361990in}{4.690634in}}%
\pgfpathlineto{\pgfqpoint{6.377424in}{4.704406in}}%
\pgfpathlineto{\pgfqpoint{6.392857in}{4.727829in}}%
\pgfpathlineto{\pgfqpoint{6.408291in}{4.760886in}}%
\pgfpathlineto{\pgfqpoint{6.423724in}{4.803396in}}%
\pgfpathlineto{\pgfqpoint{6.439158in}{4.855015in}}%
\pgfpathlineto{\pgfqpoint{6.454592in}{4.915240in}}%
\pgfpathlineto{\pgfqpoint{6.470025in}{4.983417in}}%
\pgfpathlineto{\pgfqpoint{6.485459in}{5.058750in}}%
\pgfpathlineto{\pgfqpoint{6.516326in}{5.227076in}}%
\pgfpathlineto{\pgfqpoint{6.547194in}{5.411581in}}%
\pgfpathlineto{\pgfqpoint{6.608928in}{5.789245in}}%
\pgfpathlineto{\pgfqpoint{6.639795in}{5.961873in}}%
\pgfpathlineto{\pgfqpoint{6.655229in}{6.039915in}}%
\pgfpathlineto{\pgfqpoint{6.670663in}{6.111039in}}%
\pgfpathlineto{\pgfqpoint{6.686096in}{6.174333in}}%
\pgfpathlineto{\pgfqpoint{6.701530in}{6.229012in}}%
\pgfpathlineto{\pgfqpoint{6.716964in}{6.274435in}}%
\pgfpathlineto{\pgfqpoint{6.732397in}{6.310113in}}%
\pgfpathlineto{\pgfqpoint{6.747831in}{6.335720in}}%
\pgfpathlineto{\pgfqpoint{6.763264in}{6.351090in}}%
\pgfpathlineto{\pgfqpoint{6.778698in}{6.356227in}}%
\pgfpathlineto{\pgfqpoint{6.794132in}{6.351294in}}%
\pgfpathlineto{\pgfqpoint{6.809565in}{6.336616in}}%
\pgfpathlineto{\pgfqpoint{6.824999in}{6.312669in}}%
\pgfpathlineto{\pgfqpoint{6.840433in}{6.280065in}}%
\pgfpathlineto{\pgfqpoint{6.855866in}{6.239548in}}%
\pgfpathlineto{\pgfqpoint{6.871300in}{6.191968in}}%
\pgfpathlineto{\pgfqpoint{6.886734in}{6.138274in}}%
\pgfpathlineto{\pgfqpoint{6.917601in}{6.016679in}}%
\pgfpathlineto{\pgfqpoint{6.963902in}{5.815263in}}%
\pgfpathlineto{\pgfqpoint{6.994769in}{5.681107in}}%
\pgfpathlineto{\pgfqpoint{7.025636in}{5.556395in}}%
\pgfpathlineto{\pgfqpoint{7.041070in}{5.499686in}}%
\pgfpathlineto{\pgfqpoint{7.056504in}{5.447648in}}%
\pgfpathlineto{\pgfqpoint{7.071937in}{5.400807in}}%
\pgfpathlineto{\pgfqpoint{7.087371in}{5.359553in}}%
\pgfpathlineto{\pgfqpoint{7.102804in}{5.324137in}}%
\pgfpathlineto{\pgfqpoint{7.118238in}{5.294669in}}%
\pgfpathlineto{\pgfqpoint{7.133672in}{5.271120in}}%
\pgfpathlineto{\pgfqpoint{7.149105in}{5.253323in}}%
\pgfpathlineto{\pgfqpoint{7.164539in}{5.240981in}}%
\pgfpathlineto{\pgfqpoint{7.179973in}{5.233676in}}%
\pgfpathlineto{\pgfqpoint{7.195406in}{5.230884in}}%
\pgfpathlineto{\pgfqpoint{7.210840in}{5.231984in}}%
\pgfpathlineto{\pgfqpoint{7.226274in}{5.236276in}}%
\pgfpathlineto{\pgfqpoint{7.241707in}{5.242999in}}%
\pgfpathlineto{\pgfqpoint{7.272574in}{5.260505in}}%
\pgfpathlineto{\pgfqpoint{7.303442in}{5.277916in}}%
\pgfpathlineto{\pgfqpoint{7.318875in}{5.284582in}}%
\pgfpathlineto{\pgfqpoint{7.334309in}{5.288906in}}%
\pgfpathlineto{\pgfqpoint{7.349743in}{5.290237in}}%
\pgfpathlineto{\pgfqpoint{7.365176in}{5.288011in}}%
\pgfpathlineto{\pgfqpoint{7.380610in}{5.281765in}}%
\pgfpathlineto{\pgfqpoint{7.396044in}{5.271149in}}%
\pgfpathlineto{\pgfqpoint{7.411477in}{5.255936in}}%
\pgfpathlineto{\pgfqpoint{7.426911in}{5.236025in}}%
\pgfpathlineto{\pgfqpoint{7.442344in}{5.211448in}}%
\pgfpathlineto{\pgfqpoint{7.457778in}{5.182370in}}%
\pgfpathlineto{\pgfqpoint{7.473212in}{5.149086in}}%
\pgfpathlineto{\pgfqpoint{7.488645in}{5.112015in}}%
\pgfpathlineto{\pgfqpoint{7.519513in}{5.028773in}}%
\pgfpathlineto{\pgfqpoint{7.612114in}{4.762528in}}%
\pgfpathlineto{\pgfqpoint{7.627548in}{4.725268in}}%
\pgfpathlineto{\pgfqpoint{7.642982in}{4.692511in}}%
\pgfpathlineto{\pgfqpoint{7.658415in}{4.665104in}}%
\pgfpathlineto{\pgfqpoint{7.673849in}{4.643816in}}%
\pgfpathlineto{\pgfqpoint{7.689283in}{4.629323in}}%
\pgfpathlineto{\pgfqpoint{7.704716in}{4.622193in}}%
\pgfpathlineto{\pgfqpoint{7.720150in}{4.622870in}}%
\pgfpathlineto{\pgfqpoint{7.735584in}{4.631665in}}%
\pgfpathlineto{\pgfqpoint{7.751017in}{4.648743in}}%
\pgfpathlineto{\pgfqpoint{7.766451in}{4.674120in}}%
\pgfpathlineto{\pgfqpoint{7.781884in}{4.707659in}}%
\pgfpathlineto{\pgfqpoint{7.797318in}{4.749069in}}%
\pgfpathlineto{\pgfqpoint{7.812752in}{4.797909in}}%
\pgfpathlineto{\pgfqpoint{7.828185in}{4.853590in}}%
\pgfpathlineto{\pgfqpoint{7.843619in}{4.915388in}}%
\pgfpathlineto{\pgfqpoint{7.874486in}{5.053830in}}%
\pgfpathlineto{\pgfqpoint{7.920787in}{5.282871in}}%
\pgfpathlineto{\pgfqpoint{7.951654in}{5.436074in}}%
\pgfpathlineto{\pgfqpoint{7.982522in}{5.578213in}}%
\pgfpathlineto{\pgfqpoint{7.997955in}{5.642167in}}%
\pgfpathlineto{\pgfqpoint{8.013389in}{5.699918in}}%
\pgfpathlineto{\pgfqpoint{8.028823in}{5.750481in}}%
\pgfpathlineto{\pgfqpoint{8.044256in}{5.792985in}}%
\pgfpathlineto{\pgfqpoint{8.059690in}{5.826684in}}%
\pgfpathlineto{\pgfqpoint{8.075123in}{5.850975in}}%
\pgfpathlineto{\pgfqpoint{8.090557in}{5.865407in}}%
\pgfpathlineto{\pgfqpoint{8.105991in}{5.869691in}}%
\pgfpathlineto{\pgfqpoint{8.121424in}{5.863702in}}%
\pgfpathlineto{\pgfqpoint{8.136858in}{5.847488in}}%
\pgfpathlineto{\pgfqpoint{8.152292in}{5.821262in}}%
\pgfpathlineto{\pgfqpoint{8.167725in}{5.785401in}}%
\pgfpathlineto{\pgfqpoint{8.183159in}{5.740441in}}%
\pgfpathlineto{\pgfqpoint{8.198593in}{5.687065in}}%
\pgfpathlineto{\pgfqpoint{8.214026in}{5.626091in}}%
\pgfpathlineto{\pgfqpoint{8.229460in}{5.558456in}}%
\pgfpathlineto{\pgfqpoint{8.260327in}{5.407453in}}%
\pgfpathlineto{\pgfqpoint{8.306628in}{5.159304in}}%
\pgfpathlineto{\pgfqpoint{8.352929in}{4.914773in}}%
\pgfpathlineto{\pgfqpoint{8.383796in}{4.769386in}}%
\pgfpathlineto{\pgfqpoint{8.399230in}{4.705024in}}%
\pgfpathlineto{\pgfqpoint{8.414663in}{4.647321in}}%
\pgfpathlineto{\pgfqpoint{8.430097in}{4.596944in}}%
\pgfpathlineto{\pgfqpoint{8.445531in}{4.554418in}}%
\pgfpathlineto{\pgfqpoint{8.460964in}{4.520118in}}%
\pgfpathlineto{\pgfqpoint{8.476398in}{4.494270in}}%
\pgfpathlineto{\pgfqpoint{8.491832in}{4.476940in}}%
\pgfpathlineto{\pgfqpoint{8.507265in}{4.468045in}}%
\pgfpathlineto{\pgfqpoint{8.522699in}{4.467353in}}%
\pgfpathlineto{\pgfqpoint{8.538133in}{4.474491in}}%
\pgfpathlineto{\pgfqpoint{8.553566in}{4.488957in}}%
\pgfpathlineto{\pgfqpoint{8.569000in}{4.510130in}}%
\pgfpathlineto{\pgfqpoint{8.584433in}{4.537290in}}%
\pgfpathlineto{\pgfqpoint{8.599867in}{4.569627in}}%
\pgfpathlineto{\pgfqpoint{8.615301in}{4.606267in}}%
\pgfpathlineto{\pgfqpoint{8.646168in}{4.688739in}}%
\pgfpathlineto{\pgfqpoint{8.707903in}{4.864029in}}%
\pgfpathlineto{\pgfqpoint{8.738770in}{4.942931in}}%
\pgfpathlineto{\pgfqpoint{8.754203in}{4.977696in}}%
\pgfpathlineto{\pgfqpoint{8.769637in}{5.008648in}}%
\pgfpathlineto{\pgfqpoint{8.785071in}{5.035420in}}%
\pgfpathlineto{\pgfqpoint{8.800504in}{5.057775in}}%
\pgfpathlineto{\pgfqpoint{8.815938in}{5.075604in}}%
\pgfpathlineto{\pgfqpoint{8.831372in}{5.088933in}}%
\pgfpathlineto{\pgfqpoint{8.846805in}{5.097919in}}%
\pgfpathlineto{\pgfqpoint{8.862239in}{5.102844in}}%
\pgfpathlineto{\pgfqpoint{8.877673in}{5.104105in}}%
\pgfpathlineto{\pgfqpoint{8.893106in}{5.102212in}}%
\pgfpathlineto{\pgfqpoint{8.908540in}{5.097766in}}%
\pgfpathlineto{\pgfqpoint{8.939407in}{5.084010in}}%
\pgfpathlineto{\pgfqpoint{8.970274in}{5.068961in}}%
\pgfpathlineto{\pgfqpoint{8.985708in}{5.062998in}}%
\pgfpathlineto{\pgfqpoint{9.001142in}{5.059167in}}%
\pgfpathlineto{\pgfqpoint{9.016575in}{5.058250in}}%
\pgfpathlineto{\pgfqpoint{9.032009in}{5.060980in}}%
\pgfpathlineto{\pgfqpoint{9.047443in}{5.068019in}}%
\pgfpathlineto{\pgfqpoint{9.062876in}{5.079942in}}%
\pgfpathlineto{\pgfqpoint{9.078310in}{5.097224in}}%
\pgfpathlineto{\pgfqpoint{9.093743in}{5.120225in}}%
\pgfpathlineto{\pgfqpoint{9.109177in}{5.149182in}}%
\pgfpathlineto{\pgfqpoint{9.124611in}{5.184195in}}%
\pgfpathlineto{\pgfqpoint{9.140044in}{5.225229in}}%
\pgfpathlineto{\pgfqpoint{9.155478in}{5.272108in}}%
\pgfpathlineto{\pgfqpoint{9.170912in}{5.324513in}}%
\pgfpathlineto{\pgfqpoint{9.186345in}{5.381990in}}%
\pgfpathlineto{\pgfqpoint{9.217212in}{5.509698in}}%
\pgfpathlineto{\pgfqpoint{9.248080in}{5.649163in}}%
\pgfpathlineto{\pgfqpoint{9.309814in}{5.932055in}}%
\pgfpathlineto{\pgfqpoint{9.340682in}{6.058035in}}%
\pgfpathlineto{\pgfqpoint{9.356115in}{6.113344in}}%
\pgfpathlineto{\pgfqpoint{9.371549in}{6.162194in}}%
\pgfpathlineto{\pgfqpoint{9.386982in}{6.203680in}}%
\pgfpathlineto{\pgfqpoint{9.402416in}{6.236999in}}%
\pgfpathlineto{\pgfqpoint{9.417850in}{6.261467in}}%
\pgfpathlineto{\pgfqpoint{9.433283in}{6.276533in}}%
\pgfpathlineto{\pgfqpoint{9.448717in}{6.281791in}}%
\pgfpathlineto{\pgfqpoint{9.464151in}{6.276990in}}%
\pgfpathlineto{\pgfqpoint{9.479584in}{6.262037in}}%
\pgfpathlineto{\pgfqpoint{9.495018in}{6.237009in}}%
\pgfpathlineto{\pgfqpoint{9.510452in}{6.202143in}}%
\pgfpathlineto{\pgfqpoint{9.525885in}{6.157842in}}%
\pgfpathlineto{\pgfqpoint{9.541319in}{6.104663in}}%
\pgfpathlineto{\pgfqpoint{9.556752in}{6.043312in}}%
\pgfpathlineto{\pgfqpoint{9.572186in}{5.974633in}}%
\pgfpathlineto{\pgfqpoint{9.603053in}{5.819254in}}%
\pgfpathlineto{\pgfqpoint{9.633921in}{5.647409in}}%
\pgfpathlineto{\pgfqpoint{9.695655in}{5.294504in}}%
\pgfpathlineto{\pgfqpoint{9.726522in}{5.133767in}}%
\pgfpathlineto{\pgfqpoint{9.741956in}{5.061432in}}%
\pgfpathlineto{\pgfqpoint{9.741956in}{5.061432in}}%
\pgfusepath{stroke}%
\end{pgfscope}%
\begin{pgfscope}%
\pgfpathrectangle{\pgfqpoint{5.706832in}{3.881603in}}{\pgfqpoint{4.227273in}{2.800000in}} %
\pgfusepath{clip}%
\pgfsetrectcap%
\pgfsetroundjoin%
\pgfsetlinewidth{0.501875pt}%
\definecolor{currentstroke}{rgb}{0.186275,0.473094,0.969797}%
\pgfsetstrokecolor{currentstroke}%
\pgfsetdash{}{0pt}%
\pgfpathmoveto{\pgfqpoint{5.898981in}{5.589799in}}%
\pgfpathlineto{\pgfqpoint{5.914415in}{5.626399in}}%
\pgfpathlineto{\pgfqpoint{5.929848in}{5.652880in}}%
\pgfpathlineto{\pgfqpoint{5.945282in}{5.668311in}}%
\pgfpathlineto{\pgfqpoint{5.960715in}{5.672081in}}%
\pgfpathlineto{\pgfqpoint{5.976149in}{5.663924in}}%
\pgfpathlineto{\pgfqpoint{5.991583in}{5.643917in}}%
\pgfpathlineto{\pgfqpoint{6.007016in}{5.612478in}}%
\pgfpathlineto{\pgfqpoint{6.022450in}{5.570354in}}%
\pgfpathlineto{\pgfqpoint{6.037884in}{5.518587in}}%
\pgfpathlineto{\pgfqpoint{6.053317in}{5.458491in}}%
\pgfpathlineto{\pgfqpoint{6.068751in}{5.391606in}}%
\pgfpathlineto{\pgfqpoint{6.099618in}{5.244474in}}%
\pgfpathlineto{\pgfqpoint{6.145919in}{5.018873in}}%
\pgfpathlineto{\pgfqpoint{6.161353in}{4.949943in}}%
\pgfpathlineto{\pgfqpoint{6.176786in}{4.887047in}}%
\pgfpathlineto{\pgfqpoint{6.192220in}{4.831675in}}%
\pgfpathlineto{\pgfqpoint{6.207654in}{4.785095in}}%
\pgfpathlineto{\pgfqpoint{6.223087in}{4.748325in}}%
\pgfpathlineto{\pgfqpoint{6.238521in}{4.722105in}}%
\pgfpathlineto{\pgfqpoint{6.253955in}{4.706889in}}%
\pgfpathlineto{\pgfqpoint{6.269388in}{4.702837in}}%
\pgfpathlineto{\pgfqpoint{6.284822in}{4.709824in}}%
\pgfpathlineto{\pgfqpoint{6.300255in}{4.727448in}}%
\pgfpathlineto{\pgfqpoint{6.315689in}{4.755052in}}%
\pgfpathlineto{\pgfqpoint{6.331123in}{4.791755in}}%
\pgfpathlineto{\pgfqpoint{6.346556in}{4.836486in}}%
\pgfpathlineto{\pgfqpoint{6.361990in}{4.888021in}}%
\pgfpathlineto{\pgfqpoint{6.392857in}{5.006100in}}%
\pgfpathlineto{\pgfqpoint{6.470025in}{5.324614in}}%
\pgfpathlineto{\pgfqpoint{6.500893in}{5.436463in}}%
\pgfpathlineto{\pgfqpoint{6.516326in}{5.485782in}}%
\pgfpathlineto{\pgfqpoint{6.531760in}{5.530176in}}%
\pgfpathlineto{\pgfqpoint{6.547194in}{5.569505in}}%
\pgfpathlineto{\pgfqpoint{6.562627in}{5.603822in}}%
\pgfpathlineto{\pgfqpoint{6.578061in}{5.633355in}}%
\pgfpathlineto{\pgfqpoint{6.593494in}{5.658486in}}%
\pgfpathlineto{\pgfqpoint{6.608928in}{5.679724in}}%
\pgfpathlineto{\pgfqpoint{6.624362in}{5.697674in}}%
\pgfpathlineto{\pgfqpoint{6.639795in}{5.713003in}}%
\pgfpathlineto{\pgfqpoint{6.670663in}{5.738572in}}%
\pgfpathlineto{\pgfqpoint{6.716964in}{5.773696in}}%
\pgfpathlineto{\pgfqpoint{6.747831in}{5.800292in}}%
\pgfpathlineto{\pgfqpoint{6.778698in}{5.830943in}}%
\pgfpathlineto{\pgfqpoint{6.840433in}{5.896263in}}%
\pgfpathlineto{\pgfqpoint{6.855866in}{5.910187in}}%
\pgfpathlineto{\pgfqpoint{6.871300in}{5.921664in}}%
\pgfpathlineto{\pgfqpoint{6.886734in}{5.929880in}}%
\pgfpathlineto{\pgfqpoint{6.902167in}{5.934035in}}%
\pgfpathlineto{\pgfqpoint{6.917601in}{5.933375in}}%
\pgfpathlineto{\pgfqpoint{6.933034in}{5.927232in}}%
\pgfpathlineto{\pgfqpoint{6.948468in}{5.915052in}}%
\pgfpathlineto{\pgfqpoint{6.963902in}{5.896424in}}%
\pgfpathlineto{\pgfqpoint{6.979335in}{5.871107in}}%
\pgfpathlineto{\pgfqpoint{6.994769in}{5.839047in}}%
\pgfpathlineto{\pgfqpoint{7.010203in}{5.800391in}}%
\pgfpathlineto{\pgfqpoint{7.025636in}{5.755488in}}%
\pgfpathlineto{\pgfqpoint{7.041070in}{5.704894in}}%
\pgfpathlineto{\pgfqpoint{7.071937in}{5.589804in}}%
\pgfpathlineto{\pgfqpoint{7.118238in}{5.398467in}}%
\pgfpathlineto{\pgfqpoint{7.149105in}{5.273328in}}%
\pgfpathlineto{\pgfqpoint{7.164539in}{5.215521in}}%
\pgfpathlineto{\pgfqpoint{7.179973in}{5.162589in}}%
\pgfpathlineto{\pgfqpoint{7.195406in}{5.115670in}}%
\pgfpathlineto{\pgfqpoint{7.210840in}{5.075748in}}%
\pgfpathlineto{\pgfqpoint{7.226274in}{5.043614in}}%
\pgfpathlineto{\pgfqpoint{7.241707in}{5.019841in}}%
\pgfpathlineto{\pgfqpoint{7.257141in}{5.004756in}}%
\pgfpathlineto{\pgfqpoint{7.272574in}{4.998430in}}%
\pgfpathlineto{\pgfqpoint{7.288008in}{5.000666in}}%
\pgfpathlineto{\pgfqpoint{7.303442in}{5.011004in}}%
\pgfpathlineto{\pgfqpoint{7.318875in}{5.028727in}}%
\pgfpathlineto{\pgfqpoint{7.334309in}{5.052881in}}%
\pgfpathlineto{\pgfqpoint{7.349743in}{5.082303in}}%
\pgfpathlineto{\pgfqpoint{7.380610in}{5.151446in}}%
\pgfpathlineto{\pgfqpoint{7.411477in}{5.224057in}}%
\pgfpathlineto{\pgfqpoint{7.426911in}{5.257654in}}%
\pgfpathlineto{\pgfqpoint{7.442344in}{5.287366in}}%
\pgfpathlineto{\pgfqpoint{7.457778in}{5.311762in}}%
\pgfpathlineto{\pgfqpoint{7.473212in}{5.329569in}}%
\pgfpathlineto{\pgfqpoint{7.488645in}{5.339721in}}%
\pgfpathlineto{\pgfqpoint{7.504079in}{5.341395in}}%
\pgfpathlineto{\pgfqpoint{7.519513in}{5.334044in}}%
\pgfpathlineto{\pgfqpoint{7.534946in}{5.317419in}}%
\pgfpathlineto{\pgfqpoint{7.550380in}{5.291586in}}%
\pgfpathlineto{\pgfqpoint{7.565814in}{5.256929in}}%
\pgfpathlineto{\pgfqpoint{7.581247in}{5.214149in}}%
\pgfpathlineto{\pgfqpoint{7.596681in}{5.164242in}}%
\pgfpathlineto{\pgfqpoint{7.612114in}{5.108483in}}%
\pgfpathlineto{\pgfqpoint{7.642982in}{4.985668in}}%
\pgfpathlineto{\pgfqpoint{7.673849in}{4.859925in}}%
\pgfpathlineto{\pgfqpoint{7.689283in}{4.800869in}}%
\pgfpathlineto{\pgfqpoint{7.704716in}{4.747007in}}%
\pgfpathlineto{\pgfqpoint{7.720150in}{4.700239in}}%
\pgfpathlineto{\pgfqpoint{7.735584in}{4.662326in}}%
\pgfpathlineto{\pgfqpoint{7.751017in}{4.634830in}}%
\pgfpathlineto{\pgfqpoint{7.766451in}{4.619066in}}%
\pgfpathlineto{\pgfqpoint{7.781884in}{4.616054in}}%
\pgfpathlineto{\pgfqpoint{7.797318in}{4.626486in}}%
\pgfpathlineto{\pgfqpoint{7.812752in}{4.650698in}}%
\pgfpathlineto{\pgfqpoint{7.828185in}{4.688649in}}%
\pgfpathlineto{\pgfqpoint{7.843619in}{4.739921in}}%
\pgfpathlineto{\pgfqpoint{7.859053in}{4.803717in}}%
\pgfpathlineto{\pgfqpoint{7.874486in}{4.878881in}}%
\pgfpathlineto{\pgfqpoint{7.889920in}{4.963920in}}%
\pgfpathlineto{\pgfqpoint{7.920787in}{5.156219in}}%
\pgfpathlineto{\pgfqpoint{7.982522in}{5.566783in}}%
\pgfpathlineto{\pgfqpoint{7.997955in}{5.660699in}}%
\pgfpathlineto{\pgfqpoint{8.013389in}{5.746396in}}%
\pgfpathlineto{\pgfqpoint{8.028823in}{5.821768in}}%
\pgfpathlineto{\pgfqpoint{8.044256in}{5.884950in}}%
\pgfpathlineto{\pgfqpoint{8.059690in}{5.934372in}}%
\pgfpathlineto{\pgfqpoint{8.075123in}{5.968801in}}%
\pgfpathlineto{\pgfqpoint{8.090557in}{5.987376in}}%
\pgfpathlineto{\pgfqpoint{8.105991in}{5.989628in}}%
\pgfpathlineto{\pgfqpoint{8.121424in}{5.975497in}}%
\pgfpathlineto{\pgfqpoint{8.136858in}{5.945328in}}%
\pgfpathlineto{\pgfqpoint{8.152292in}{5.899863in}}%
\pgfpathlineto{\pgfqpoint{8.167725in}{5.840219in}}%
\pgfpathlineto{\pgfqpoint{8.183159in}{5.767855in}}%
\pgfpathlineto{\pgfqpoint{8.198593in}{5.684534in}}%
\pgfpathlineto{\pgfqpoint{8.214026in}{5.592265in}}%
\pgfpathlineto{\pgfqpoint{8.244893in}{5.389851in}}%
\pgfpathlineto{\pgfqpoint{8.291194in}{5.077281in}}%
\pgfpathlineto{\pgfqpoint{8.322062in}{4.889912in}}%
\pgfpathlineto{\pgfqpoint{8.337495in}{4.808590in}}%
\pgfpathlineto{\pgfqpoint{8.352929in}{4.737634in}}%
\pgfpathlineto{\pgfqpoint{8.368363in}{4.678252in}}%
\pgfpathlineto{\pgfqpoint{8.383796in}{4.631305in}}%
\pgfpathlineto{\pgfqpoint{8.399230in}{4.597299in}}%
\pgfpathlineto{\pgfqpoint{8.414663in}{4.576369in}}%
\pgfpathlineto{\pgfqpoint{8.430097in}{4.568295in}}%
\pgfpathlineto{\pgfqpoint{8.445531in}{4.572511in}}%
\pgfpathlineto{\pgfqpoint{8.460964in}{4.588132in}}%
\pgfpathlineto{\pgfqpoint{8.476398in}{4.613989in}}%
\pgfpathlineto{\pgfqpoint{8.491832in}{4.648673in}}%
\pgfpathlineto{\pgfqpoint{8.507265in}{4.690579in}}%
\pgfpathlineto{\pgfqpoint{8.522699in}{4.737964in}}%
\pgfpathlineto{\pgfqpoint{8.599867in}{4.993408in}}%
\pgfpathlineto{\pgfqpoint{8.615301in}{5.036340in}}%
\pgfpathlineto{\pgfqpoint{8.630734in}{5.073313in}}%
\pgfpathlineto{\pgfqpoint{8.646168in}{5.103389in}}%
\pgfpathlineto{\pgfqpoint{8.661602in}{5.125907in}}%
\pgfpathlineto{\pgfqpoint{8.677035in}{5.140496in}}%
\pgfpathlineto{\pgfqpoint{8.692469in}{5.147080in}}%
\pgfpathlineto{\pgfqpoint{8.707903in}{5.145872in}}%
\pgfpathlineto{\pgfqpoint{8.723336in}{5.137358in}}%
\pgfpathlineto{\pgfqpoint{8.738770in}{5.122277in}}%
\pgfpathlineto{\pgfqpoint{8.754203in}{5.101589in}}%
\pgfpathlineto{\pgfqpoint{8.769637in}{5.076438in}}%
\pgfpathlineto{\pgfqpoint{8.800504in}{5.017999in}}%
\pgfpathlineto{\pgfqpoint{8.831372in}{4.958139in}}%
\pgfpathlineto{\pgfqpoint{8.846805in}{4.931220in}}%
\pgfpathlineto{\pgfqpoint{8.862239in}{4.908069in}}%
\pgfpathlineto{\pgfqpoint{8.877673in}{4.889854in}}%
\pgfpathlineto{\pgfqpoint{8.893106in}{4.877574in}}%
\pgfpathlineto{\pgfqpoint{8.908540in}{4.872032in}}%
\pgfpathlineto{\pgfqpoint{8.923973in}{4.873811in}}%
\pgfpathlineto{\pgfqpoint{8.939407in}{4.883261in}}%
\pgfpathlineto{\pgfqpoint{8.954841in}{4.900491in}}%
\pgfpathlineto{\pgfqpoint{8.970274in}{4.925371in}}%
\pgfpathlineto{\pgfqpoint{8.985708in}{4.957541in}}%
\pgfpathlineto{\pgfqpoint{9.001142in}{4.996428in}}%
\pgfpathlineto{\pgfqpoint{9.016575in}{5.041271in}}%
\pgfpathlineto{\pgfqpoint{9.032009in}{5.091146in}}%
\pgfpathlineto{\pgfqpoint{9.062876in}{5.201713in}}%
\pgfpathlineto{\pgfqpoint{9.124611in}{5.433351in}}%
\pgfpathlineto{\pgfqpoint{9.155478in}{5.536718in}}%
\pgfpathlineto{\pgfqpoint{9.170912in}{5.582185in}}%
\pgfpathlineto{\pgfqpoint{9.186345in}{5.622729in}}%
\pgfpathlineto{\pgfqpoint{9.201779in}{5.657994in}}%
\pgfpathlineto{\pgfqpoint{9.217212in}{5.687814in}}%
\pgfpathlineto{\pgfqpoint{9.232646in}{5.712203in}}%
\pgfpathlineto{\pgfqpoint{9.248080in}{5.731356in}}%
\pgfpathlineto{\pgfqpoint{9.263513in}{5.745624in}}%
\pgfpathlineto{\pgfqpoint{9.278947in}{5.755496in}}%
\pgfpathlineto{\pgfqpoint{9.294381in}{5.761571in}}%
\pgfpathlineto{\pgfqpoint{9.309814in}{5.764530in}}%
\pgfpathlineto{\pgfqpoint{9.325248in}{5.765101in}}%
\pgfpathlineto{\pgfqpoint{9.356115in}{5.762014in}}%
\pgfpathlineto{\pgfqpoint{9.386982in}{5.757778in}}%
\pgfpathlineto{\pgfqpoint{9.417850in}{5.756526in}}%
\pgfpathlineto{\pgfqpoint{9.433283in}{5.757749in}}%
\pgfpathlineto{\pgfqpoint{9.448717in}{5.760287in}}%
\pgfpathlineto{\pgfqpoint{9.479584in}{5.768572in}}%
\pgfpathlineto{\pgfqpoint{9.510452in}{5.778374in}}%
\pgfpathlineto{\pgfqpoint{9.525885in}{5.782311in}}%
\pgfpathlineto{\pgfqpoint{9.541319in}{5.784592in}}%
\pgfpathlineto{\pgfqpoint{9.556752in}{5.784383in}}%
\pgfpathlineto{\pgfqpoint{9.572186in}{5.780836in}}%
\pgfpathlineto{\pgfqpoint{9.587620in}{5.773125in}}%
\pgfpathlineto{\pgfqpoint{9.603053in}{5.760484in}}%
\pgfpathlineto{\pgfqpoint{9.618487in}{5.742245in}}%
\pgfpathlineto{\pgfqpoint{9.633921in}{5.717869in}}%
\pgfpathlineto{\pgfqpoint{9.649354in}{5.686982in}}%
\pgfpathlineto{\pgfqpoint{9.664788in}{5.649396in}}%
\pgfpathlineto{\pgfqpoint{9.680222in}{5.605134in}}%
\pgfpathlineto{\pgfqpoint{9.695655in}{5.554442in}}%
\pgfpathlineto{\pgfqpoint{9.711089in}{5.497793in}}%
\pgfpathlineto{\pgfqpoint{9.741956in}{5.369652in}}%
\pgfpathlineto{\pgfqpoint{9.741956in}{5.369652in}}%
\pgfusepath{stroke}%
\end{pgfscope}%
\begin{pgfscope}%
\pgfpathrectangle{\pgfqpoint{5.706832in}{3.881603in}}{\pgfqpoint{4.227273in}{2.800000in}} %
\pgfusepath{clip}%
\pgfsetrectcap%
\pgfsetroundjoin%
\pgfsetlinewidth{0.501875pt}%
\definecolor{currentstroke}{rgb}{0.100000,0.587785,0.951057}%
\pgfsetstrokecolor{currentstroke}%
\pgfsetdash{}{0pt}%
\pgfpathmoveto{\pgfqpoint{5.898981in}{5.538697in}}%
\pgfpathlineto{\pgfqpoint{5.914415in}{5.563445in}}%
\pgfpathlineto{\pgfqpoint{5.929848in}{5.580899in}}%
\pgfpathlineto{\pgfqpoint{5.945282in}{5.590127in}}%
\pgfpathlineto{\pgfqpoint{5.960715in}{5.590431in}}%
\pgfpathlineto{\pgfqpoint{5.976149in}{5.581367in}}%
\pgfpathlineto{\pgfqpoint{5.991583in}{5.562754in}}%
\pgfpathlineto{\pgfqpoint{6.007016in}{5.534677in}}%
\pgfpathlineto{\pgfqpoint{6.022450in}{5.497489in}}%
\pgfpathlineto{\pgfqpoint{6.037884in}{5.451793in}}%
\pgfpathlineto{\pgfqpoint{6.053317in}{5.398425in}}%
\pgfpathlineto{\pgfqpoint{6.068751in}{5.338435in}}%
\pgfpathlineto{\pgfqpoint{6.099618in}{5.203642in}}%
\pgfpathlineto{\pgfqpoint{6.161353in}{4.916158in}}%
\pgfpathlineto{\pgfqpoint{6.176786in}{4.849575in}}%
\pgfpathlineto{\pgfqpoint{6.192220in}{4.788046in}}%
\pgfpathlineto{\pgfqpoint{6.207654in}{4.732857in}}%
\pgfpathlineto{\pgfqpoint{6.223087in}{4.685138in}}%
\pgfpathlineto{\pgfqpoint{6.238521in}{4.645842in}}%
\pgfpathlineto{\pgfqpoint{6.253955in}{4.615718in}}%
\pgfpathlineto{\pgfqpoint{6.269388in}{4.595308in}}%
\pgfpathlineto{\pgfqpoint{6.284822in}{4.584933in}}%
\pgfpathlineto{\pgfqpoint{6.300255in}{4.584698in}}%
\pgfpathlineto{\pgfqpoint{6.315689in}{4.594493in}}%
\pgfpathlineto{\pgfqpoint{6.331123in}{4.614005in}}%
\pgfpathlineto{\pgfqpoint{6.346556in}{4.642739in}}%
\pgfpathlineto{\pgfqpoint{6.361990in}{4.680031in}}%
\pgfpathlineto{\pgfqpoint{6.377424in}{4.725078in}}%
\pgfpathlineto{\pgfqpoint{6.392857in}{4.776963in}}%
\pgfpathlineto{\pgfqpoint{6.408291in}{4.834682in}}%
\pgfpathlineto{\pgfqpoint{6.439158in}{4.963358in}}%
\pgfpathlineto{\pgfqpoint{6.485459in}{5.173365in}}%
\pgfpathlineto{\pgfqpoint{6.531760in}{5.380420in}}%
\pgfpathlineto{\pgfqpoint{6.562627in}{5.506694in}}%
\pgfpathlineto{\pgfqpoint{6.593494in}{5.619010in}}%
\pgfpathlineto{\pgfqpoint{6.608928in}{5.669268in}}%
\pgfpathlineto{\pgfqpoint{6.624362in}{5.715432in}}%
\pgfpathlineto{\pgfqpoint{6.639795in}{5.757485in}}%
\pgfpathlineto{\pgfqpoint{6.655229in}{5.795473in}}%
\pgfpathlineto{\pgfqpoint{6.670663in}{5.829492in}}%
\pgfpathlineto{\pgfqpoint{6.686096in}{5.859665in}}%
\pgfpathlineto{\pgfqpoint{6.701530in}{5.886137in}}%
\pgfpathlineto{\pgfqpoint{6.716964in}{5.909049in}}%
\pgfpathlineto{\pgfqpoint{6.732397in}{5.928533in}}%
\pgfpathlineto{\pgfqpoint{6.747831in}{5.944697in}}%
\pgfpathlineto{\pgfqpoint{6.763264in}{5.957617in}}%
\pgfpathlineto{\pgfqpoint{6.778698in}{5.967334in}}%
\pgfpathlineto{\pgfqpoint{6.794132in}{5.973850in}}%
\pgfpathlineto{\pgfqpoint{6.809565in}{5.977129in}}%
\pgfpathlineto{\pgfqpoint{6.824999in}{5.977098in}}%
\pgfpathlineto{\pgfqpoint{6.840433in}{5.973662in}}%
\pgfpathlineto{\pgfqpoint{6.855866in}{5.966705in}}%
\pgfpathlineto{\pgfqpoint{6.871300in}{5.956103in}}%
\pgfpathlineto{\pgfqpoint{6.886734in}{5.941741in}}%
\pgfpathlineto{\pgfqpoint{6.902167in}{5.923522in}}%
\pgfpathlineto{\pgfqpoint{6.917601in}{5.901383in}}%
\pgfpathlineto{\pgfqpoint{6.933034in}{5.875306in}}%
\pgfpathlineto{\pgfqpoint{6.948468in}{5.845335in}}%
\pgfpathlineto{\pgfqpoint{6.963902in}{5.811580in}}%
\pgfpathlineto{\pgfqpoint{6.979335in}{5.774231in}}%
\pgfpathlineto{\pgfqpoint{7.010203in}{5.689929in}}%
\pgfpathlineto{\pgfqpoint{7.041070in}{5.595627in}}%
\pgfpathlineto{\pgfqpoint{7.133672in}{5.301316in}}%
\pgfpathlineto{\pgfqpoint{7.149105in}{5.257997in}}%
\pgfpathlineto{\pgfqpoint{7.164539in}{5.218128in}}%
\pgfpathlineto{\pgfqpoint{7.179973in}{5.182264in}}%
\pgfpathlineto{\pgfqpoint{7.195406in}{5.150861in}}%
\pgfpathlineto{\pgfqpoint{7.210840in}{5.124261in}}%
\pgfpathlineto{\pgfqpoint{7.226274in}{5.102677in}}%
\pgfpathlineto{\pgfqpoint{7.241707in}{5.086183in}}%
\pgfpathlineto{\pgfqpoint{7.257141in}{5.074707in}}%
\pgfpathlineto{\pgfqpoint{7.272574in}{5.068029in}}%
\pgfpathlineto{\pgfqpoint{7.288008in}{5.065785in}}%
\pgfpathlineto{\pgfqpoint{7.303442in}{5.067474in}}%
\pgfpathlineto{\pgfqpoint{7.318875in}{5.072468in}}%
\pgfpathlineto{\pgfqpoint{7.334309in}{5.080036in}}%
\pgfpathlineto{\pgfqpoint{7.365176in}{5.099551in}}%
\pgfpathlineto{\pgfqpoint{7.396044in}{5.118895in}}%
\pgfpathlineto{\pgfqpoint{7.411477in}{5.126236in}}%
\pgfpathlineto{\pgfqpoint{7.426911in}{5.130892in}}%
\pgfpathlineto{\pgfqpoint{7.442344in}{5.132120in}}%
\pgfpathlineto{\pgfqpoint{7.457778in}{5.129292in}}%
\pgfpathlineto{\pgfqpoint{7.473212in}{5.121922in}}%
\pgfpathlineto{\pgfqpoint{7.488645in}{5.109690in}}%
\pgfpathlineto{\pgfqpoint{7.504079in}{5.092458in}}%
\pgfpathlineto{\pgfqpoint{7.519513in}{5.070280in}}%
\pgfpathlineto{\pgfqpoint{7.534946in}{5.043412in}}%
\pgfpathlineto{\pgfqpoint{7.550380in}{5.012310in}}%
\pgfpathlineto{\pgfqpoint{7.565814in}{4.977622in}}%
\pgfpathlineto{\pgfqpoint{7.596681in}{4.900979in}}%
\pgfpathlineto{\pgfqpoint{7.627548in}{4.821939in}}%
\pgfpathlineto{\pgfqpoint{7.642982in}{4.784657in}}%
\pgfpathlineto{\pgfqpoint{7.658415in}{4.750659in}}%
\pgfpathlineto{\pgfqpoint{7.673849in}{4.721298in}}%
\pgfpathlineto{\pgfqpoint{7.689283in}{4.697882in}}%
\pgfpathlineto{\pgfqpoint{7.704716in}{4.681634in}}%
\pgfpathlineto{\pgfqpoint{7.720150in}{4.673653in}}%
\pgfpathlineto{\pgfqpoint{7.735584in}{4.674878in}}%
\pgfpathlineto{\pgfqpoint{7.751017in}{4.686048in}}%
\pgfpathlineto{\pgfqpoint{7.766451in}{4.707682in}}%
\pgfpathlineto{\pgfqpoint{7.781884in}{4.740047in}}%
\pgfpathlineto{\pgfqpoint{7.797318in}{4.783146in}}%
\pgfpathlineto{\pgfqpoint{7.812752in}{4.836707in}}%
\pgfpathlineto{\pgfqpoint{7.828185in}{4.900178in}}%
\pgfpathlineto{\pgfqpoint{7.843619in}{4.972737in}}%
\pgfpathlineto{\pgfqpoint{7.859053in}{5.053303in}}%
\pgfpathlineto{\pgfqpoint{7.889920in}{5.232965in}}%
\pgfpathlineto{\pgfqpoint{7.967088in}{5.708856in}}%
\pgfpathlineto{\pgfqpoint{7.982522in}{5.793237in}}%
\pgfpathlineto{\pgfqpoint{7.997955in}{5.869620in}}%
\pgfpathlineto{\pgfqpoint{8.013389in}{5.936345in}}%
\pgfpathlineto{\pgfqpoint{8.028823in}{5.991952in}}%
\pgfpathlineto{\pgfqpoint{8.044256in}{6.035214in}}%
\pgfpathlineto{\pgfqpoint{8.059690in}{6.065173in}}%
\pgfpathlineto{\pgfqpoint{8.075123in}{6.081164in}}%
\pgfpathlineto{\pgfqpoint{8.090557in}{6.082837in}}%
\pgfpathlineto{\pgfqpoint{8.105991in}{6.070162in}}%
\pgfpathlineto{\pgfqpoint{8.121424in}{6.043434in}}%
\pgfpathlineto{\pgfqpoint{8.136858in}{6.003264in}}%
\pgfpathlineto{\pgfqpoint{8.152292in}{5.950566in}}%
\pgfpathlineto{\pgfqpoint{8.167725in}{5.886530in}}%
\pgfpathlineto{\pgfqpoint{8.183159in}{5.812595in}}%
\pgfpathlineto{\pgfqpoint{8.198593in}{5.730405in}}%
\pgfpathlineto{\pgfqpoint{8.229460in}{5.548630in}}%
\pgfpathlineto{\pgfqpoint{8.291194in}{5.170841in}}%
\pgfpathlineto{\pgfqpoint{8.306628in}{5.084670in}}%
\pgfpathlineto{\pgfqpoint{8.322062in}{5.005210in}}%
\pgfpathlineto{\pgfqpoint{8.337495in}{4.933827in}}%
\pgfpathlineto{\pgfqpoint{8.352929in}{4.871635in}}%
\pgfpathlineto{\pgfqpoint{8.368363in}{4.819481in}}%
\pgfpathlineto{\pgfqpoint{8.383796in}{4.777917in}}%
\pgfpathlineto{\pgfqpoint{8.399230in}{4.747201in}}%
\pgfpathlineto{\pgfqpoint{8.414663in}{4.727293in}}%
\pgfpathlineto{\pgfqpoint{8.430097in}{4.717861in}}%
\pgfpathlineto{\pgfqpoint{8.445531in}{4.718305in}}%
\pgfpathlineto{\pgfqpoint{8.460964in}{4.727772in}}%
\pgfpathlineto{\pgfqpoint{8.476398in}{4.745195in}}%
\pgfpathlineto{\pgfqpoint{8.491832in}{4.769321in}}%
\pgfpathlineto{\pgfqpoint{8.507265in}{4.798763in}}%
\pgfpathlineto{\pgfqpoint{8.538133in}{4.867598in}}%
\pgfpathlineto{\pgfqpoint{8.569000in}{4.939514in}}%
\pgfpathlineto{\pgfqpoint{8.584433in}{4.972973in}}%
\pgfpathlineto{\pgfqpoint{8.599867in}{5.003028in}}%
\pgfpathlineto{\pgfqpoint{8.615301in}{5.028577in}}%
\pgfpathlineto{\pgfqpoint{8.630734in}{5.048714in}}%
\pgfpathlineto{\pgfqpoint{8.646168in}{5.062756in}}%
\pgfpathlineto{\pgfqpoint{8.661602in}{5.070259in}}%
\pgfpathlineto{\pgfqpoint{8.677035in}{5.071032in}}%
\pgfpathlineto{\pgfqpoint{8.692469in}{5.065133in}}%
\pgfpathlineto{\pgfqpoint{8.707903in}{5.052867in}}%
\pgfpathlineto{\pgfqpoint{8.723336in}{5.034775in}}%
\pgfpathlineto{\pgfqpoint{8.738770in}{5.011608in}}%
\pgfpathlineto{\pgfqpoint{8.754203in}{4.984309in}}%
\pgfpathlineto{\pgfqpoint{8.785071in}{4.921827in}}%
\pgfpathlineto{\pgfqpoint{8.815938in}{4.857333in}}%
\pgfpathlineto{\pgfqpoint{8.831372in}{4.827674in}}%
\pgfpathlineto{\pgfqpoint{8.846805in}{4.801490in}}%
\pgfpathlineto{\pgfqpoint{8.862239in}{4.780002in}}%
\pgfpathlineto{\pgfqpoint{8.877673in}{4.764316in}}%
\pgfpathlineto{\pgfqpoint{8.893106in}{4.755388in}}%
\pgfpathlineto{\pgfqpoint{8.908540in}{4.753998in}}%
\pgfpathlineto{\pgfqpoint{8.923973in}{4.760728in}}%
\pgfpathlineto{\pgfqpoint{8.939407in}{4.775947in}}%
\pgfpathlineto{\pgfqpoint{8.954841in}{4.799800in}}%
\pgfpathlineto{\pgfqpoint{8.970274in}{4.832206in}}%
\pgfpathlineto{\pgfqpoint{8.985708in}{4.872867in}}%
\pgfpathlineto{\pgfqpoint{9.001142in}{4.921270in}}%
\pgfpathlineto{\pgfqpoint{9.016575in}{4.976712in}}%
\pgfpathlineto{\pgfqpoint{9.032009in}{5.038318in}}%
\pgfpathlineto{\pgfqpoint{9.062876in}{5.175826in}}%
\pgfpathlineto{\pgfqpoint{9.155478in}{5.614996in}}%
\pgfpathlineto{\pgfqpoint{9.170912in}{5.679449in}}%
\pgfpathlineto{\pgfqpoint{9.186345in}{5.738693in}}%
\pgfpathlineto{\pgfqpoint{9.201779in}{5.792005in}}%
\pgfpathlineto{\pgfqpoint{9.217212in}{5.838822in}}%
\pgfpathlineto{\pgfqpoint{9.232646in}{5.878751in}}%
\pgfpathlineto{\pgfqpoint{9.248080in}{5.911568in}}%
\pgfpathlineto{\pgfqpoint{9.263513in}{5.937219in}}%
\pgfpathlineto{\pgfqpoint{9.278947in}{5.955804in}}%
\pgfpathlineto{\pgfqpoint{9.294381in}{5.967570in}}%
\pgfpathlineto{\pgfqpoint{9.309814in}{5.972887in}}%
\pgfpathlineto{\pgfqpoint{9.325248in}{5.972233in}}%
\pgfpathlineto{\pgfqpoint{9.340682in}{5.966168in}}%
\pgfpathlineto{\pgfqpoint{9.356115in}{5.955310in}}%
\pgfpathlineto{\pgfqpoint{9.371549in}{5.940309in}}%
\pgfpathlineto{\pgfqpoint{9.386982in}{5.921823in}}%
\pgfpathlineto{\pgfqpoint{9.402416in}{5.900492in}}%
\pgfpathlineto{\pgfqpoint{9.433283in}{5.851654in}}%
\pgfpathlineto{\pgfqpoint{9.479584in}{5.769981in}}%
\pgfpathlineto{\pgfqpoint{9.541319in}{5.655784in}}%
\pgfpathlineto{\pgfqpoint{9.587620in}{5.565792in}}%
\pgfpathlineto{\pgfqpoint{9.618487in}{5.500985in}}%
\pgfpathlineto{\pgfqpoint{9.649354in}{5.430380in}}%
\pgfpathlineto{\pgfqpoint{9.680222in}{5.353025in}}%
\pgfpathlineto{\pgfqpoint{9.711089in}{5.269180in}}%
\pgfpathlineto{\pgfqpoint{9.741956in}{5.180619in}}%
\pgfpathlineto{\pgfqpoint{9.741956in}{5.180619in}}%
\pgfusepath{stroke}%
\end{pgfscope}%
\begin{pgfscope}%
\pgfpathrectangle{\pgfqpoint{5.706832in}{3.881603in}}{\pgfqpoint{4.227273in}{2.800000in}} %
\pgfusepath{clip}%
\pgfsetrectcap%
\pgfsetroundjoin%
\pgfsetlinewidth{0.501875pt}%
\definecolor{currentstroke}{rgb}{0.021569,0.682749,0.930229}%
\pgfsetstrokecolor{currentstroke}%
\pgfsetdash{}{0pt}%
\pgfpathmoveto{\pgfqpoint{5.898981in}{5.550174in}}%
\pgfpathlineto{\pgfqpoint{5.914415in}{5.555350in}}%
\pgfpathlineto{\pgfqpoint{5.929848in}{5.556175in}}%
\pgfpathlineto{\pgfqpoint{5.945282in}{5.552152in}}%
\pgfpathlineto{\pgfqpoint{5.960715in}{5.542923in}}%
\pgfpathlineto{\pgfqpoint{5.976149in}{5.528282in}}%
\pgfpathlineto{\pgfqpoint{5.991583in}{5.508179in}}%
\pgfpathlineto{\pgfqpoint{6.007016in}{5.482724in}}%
\pgfpathlineto{\pgfqpoint{6.022450in}{5.452183in}}%
\pgfpathlineto{\pgfqpoint{6.037884in}{5.416971in}}%
\pgfpathlineto{\pgfqpoint{6.053317in}{5.377642in}}%
\pgfpathlineto{\pgfqpoint{6.084185in}{5.289441in}}%
\pgfpathlineto{\pgfqpoint{6.161353in}{5.054275in}}%
\pgfpathlineto{\pgfqpoint{6.176786in}{5.012315in}}%
\pgfpathlineto{\pgfqpoint{6.192220in}{4.974168in}}%
\pgfpathlineto{\pgfqpoint{6.207654in}{4.940617in}}%
\pgfpathlineto{\pgfqpoint{6.223087in}{4.912345in}}%
\pgfpathlineto{\pgfqpoint{6.238521in}{4.889925in}}%
\pgfpathlineto{\pgfqpoint{6.253955in}{4.873804in}}%
\pgfpathlineto{\pgfqpoint{6.269388in}{4.864298in}}%
\pgfpathlineto{\pgfqpoint{6.284822in}{4.861588in}}%
\pgfpathlineto{\pgfqpoint{6.300255in}{4.865717in}}%
\pgfpathlineto{\pgfqpoint{6.315689in}{4.876598in}}%
\pgfpathlineto{\pgfqpoint{6.331123in}{4.894016in}}%
\pgfpathlineto{\pgfqpoint{6.346556in}{4.917641in}}%
\pgfpathlineto{\pgfqpoint{6.361990in}{4.947041in}}%
\pgfpathlineto{\pgfqpoint{6.377424in}{4.981692in}}%
\pgfpathlineto{\pgfqpoint{6.392857in}{5.021002in}}%
\pgfpathlineto{\pgfqpoint{6.408291in}{5.064321in}}%
\pgfpathlineto{\pgfqpoint{6.439158in}{5.160221in}}%
\pgfpathlineto{\pgfqpoint{6.485459in}{5.316725in}}%
\pgfpathlineto{\pgfqpoint{6.531760in}{5.472945in}}%
\pgfpathlineto{\pgfqpoint{6.562627in}{5.570003in}}%
\pgfpathlineto{\pgfqpoint{6.593494in}{5.658049in}}%
\pgfpathlineto{\pgfqpoint{6.624362in}{5.735406in}}%
\pgfpathlineto{\pgfqpoint{6.639795in}{5.769801in}}%
\pgfpathlineto{\pgfqpoint{6.655229in}{5.801284in}}%
\pgfpathlineto{\pgfqpoint{6.670663in}{5.829856in}}%
\pgfpathlineto{\pgfqpoint{6.686096in}{5.855538in}}%
\pgfpathlineto{\pgfqpoint{6.701530in}{5.878362in}}%
\pgfpathlineto{\pgfqpoint{6.716964in}{5.898367in}}%
\pgfpathlineto{\pgfqpoint{6.732397in}{5.915586in}}%
\pgfpathlineto{\pgfqpoint{6.747831in}{5.930043in}}%
\pgfpathlineto{\pgfqpoint{6.763264in}{5.941748in}}%
\pgfpathlineto{\pgfqpoint{6.778698in}{5.950691in}}%
\pgfpathlineto{\pgfqpoint{6.794132in}{5.956846in}}%
\pgfpathlineto{\pgfqpoint{6.809565in}{5.960166in}}%
\pgfpathlineto{\pgfqpoint{6.824999in}{5.960591in}}%
\pgfpathlineto{\pgfqpoint{6.840433in}{5.958047in}}%
\pgfpathlineto{\pgfqpoint{6.855866in}{5.952457in}}%
\pgfpathlineto{\pgfqpoint{6.871300in}{5.943741in}}%
\pgfpathlineto{\pgfqpoint{6.886734in}{5.931832in}}%
\pgfpathlineto{\pgfqpoint{6.902167in}{5.916677in}}%
\pgfpathlineto{\pgfqpoint{6.917601in}{5.898249in}}%
\pgfpathlineto{\pgfqpoint{6.933034in}{5.876553in}}%
\pgfpathlineto{\pgfqpoint{6.948468in}{5.851635in}}%
\pgfpathlineto{\pgfqpoint{6.963902in}{5.823588in}}%
\pgfpathlineto{\pgfqpoint{6.979335in}{5.792554in}}%
\pgfpathlineto{\pgfqpoint{7.010203in}{5.722376in}}%
\pgfpathlineto{\pgfqpoint{7.041070in}{5.643342in}}%
\pgfpathlineto{\pgfqpoint{7.087371in}{5.515033in}}%
\pgfpathlineto{\pgfqpoint{7.133672in}{5.386348in}}%
\pgfpathlineto{\pgfqpoint{7.164539in}{5.306860in}}%
\pgfpathlineto{\pgfqpoint{7.195406in}{5.236398in}}%
\pgfpathlineto{\pgfqpoint{7.210840in}{5.205356in}}%
\pgfpathlineto{\pgfqpoint{7.226274in}{5.177392in}}%
\pgfpathlineto{\pgfqpoint{7.241707in}{5.152618in}}%
\pgfpathlineto{\pgfqpoint{7.257141in}{5.131053in}}%
\pgfpathlineto{\pgfqpoint{7.272574in}{5.112630in}}%
\pgfpathlineto{\pgfqpoint{7.288008in}{5.097193in}}%
\pgfpathlineto{\pgfqpoint{7.303442in}{5.084499in}}%
\pgfpathlineto{\pgfqpoint{7.318875in}{5.074231in}}%
\pgfpathlineto{\pgfqpoint{7.334309in}{5.066003in}}%
\pgfpathlineto{\pgfqpoint{7.365176in}{5.053867in}}%
\pgfpathlineto{\pgfqpoint{7.411477in}{5.039003in}}%
\pgfpathlineto{\pgfqpoint{7.426911in}{5.032961in}}%
\pgfpathlineto{\pgfqpoint{7.442344in}{5.025640in}}%
\pgfpathlineto{\pgfqpoint{7.457778in}{5.016686in}}%
\pgfpathlineto{\pgfqpoint{7.473212in}{5.005827in}}%
\pgfpathlineto{\pgfqpoint{7.488645in}{4.992884in}}%
\pgfpathlineto{\pgfqpoint{7.504079in}{4.977783in}}%
\pgfpathlineto{\pgfqpoint{7.519513in}{4.960561in}}%
\pgfpathlineto{\pgfqpoint{7.550380in}{4.920479in}}%
\pgfpathlineto{\pgfqpoint{7.596681in}{4.851897in}}%
\pgfpathlineto{\pgfqpoint{7.627548in}{4.807162in}}%
\pgfpathlineto{\pgfqpoint{7.642982in}{4.787213in}}%
\pgfpathlineto{\pgfqpoint{7.658415in}{4.769905in}}%
\pgfpathlineto{\pgfqpoint{7.673849in}{4.756008in}}%
\pgfpathlineto{\pgfqpoint{7.689283in}{4.746261in}}%
\pgfpathlineto{\pgfqpoint{7.704716in}{4.741352in}}%
\pgfpathlineto{\pgfqpoint{7.720150in}{4.741890in}}%
\pgfpathlineto{\pgfqpoint{7.735584in}{4.748383in}}%
\pgfpathlineto{\pgfqpoint{7.751017in}{4.761221in}}%
\pgfpathlineto{\pgfqpoint{7.766451in}{4.780657in}}%
\pgfpathlineto{\pgfqpoint{7.781884in}{4.806793in}}%
\pgfpathlineto{\pgfqpoint{7.797318in}{4.839570in}}%
\pgfpathlineto{\pgfqpoint{7.812752in}{4.878763in}}%
\pgfpathlineto{\pgfqpoint{7.828185in}{4.923980in}}%
\pgfpathlineto{\pgfqpoint{7.843619in}{4.974666in}}%
\pgfpathlineto{\pgfqpoint{7.874486in}{5.089459in}}%
\pgfpathlineto{\pgfqpoint{7.920787in}{5.280660in}}%
\pgfpathlineto{\pgfqpoint{7.951654in}{5.407432in}}%
\pgfpathlineto{\pgfqpoint{7.982522in}{5.522221in}}%
\pgfpathlineto{\pgfqpoint{7.997955in}{5.572184in}}%
\pgfpathlineto{\pgfqpoint{8.013389in}{5.615781in}}%
\pgfpathlineto{\pgfqpoint{8.028823in}{5.652097in}}%
\pgfpathlineto{\pgfqpoint{8.044256in}{5.680366in}}%
\pgfpathlineto{\pgfqpoint{8.059690in}{5.699984in}}%
\pgfpathlineto{\pgfqpoint{8.075123in}{5.710528in}}%
\pgfpathlineto{\pgfqpoint{8.090557in}{5.711767in}}%
\pgfpathlineto{\pgfqpoint{8.105991in}{5.703669in}}%
\pgfpathlineto{\pgfqpoint{8.121424in}{5.686401in}}%
\pgfpathlineto{\pgfqpoint{8.136858in}{5.660324in}}%
\pgfpathlineto{\pgfqpoint{8.152292in}{5.625986in}}%
\pgfpathlineto{\pgfqpoint{8.167725in}{5.584106in}}%
\pgfpathlineto{\pgfqpoint{8.183159in}{5.535553in}}%
\pgfpathlineto{\pgfqpoint{8.198593in}{5.481331in}}%
\pgfpathlineto{\pgfqpoint{8.229460in}{5.360375in}}%
\pgfpathlineto{\pgfqpoint{8.306628in}{5.041542in}}%
\pgfpathlineto{\pgfqpoint{8.322062in}{4.984245in}}%
\pgfpathlineto{\pgfqpoint{8.337495in}{4.931391in}}%
\pgfpathlineto{\pgfqpoint{8.352929in}{4.883703in}}%
\pgfpathlineto{\pgfqpoint{8.368363in}{4.841745in}}%
\pgfpathlineto{\pgfqpoint{8.383796in}{4.805904in}}%
\pgfpathlineto{\pgfqpoint{8.399230in}{4.776396in}}%
\pgfpathlineto{\pgfqpoint{8.414663in}{4.753257in}}%
\pgfpathlineto{\pgfqpoint{8.430097in}{4.736350in}}%
\pgfpathlineto{\pgfqpoint{8.445531in}{4.725378in}}%
\pgfpathlineto{\pgfqpoint{8.460964in}{4.719892in}}%
\pgfpathlineto{\pgfqpoint{8.476398in}{4.719315in}}%
\pgfpathlineto{\pgfqpoint{8.491832in}{4.722960in}}%
\pgfpathlineto{\pgfqpoint{8.507265in}{4.730056in}}%
\pgfpathlineto{\pgfqpoint{8.522699in}{4.739771in}}%
\pgfpathlineto{\pgfqpoint{8.553566in}{4.763617in}}%
\pgfpathlineto{\pgfqpoint{8.584433in}{4.787746in}}%
\pgfpathlineto{\pgfqpoint{8.599867in}{4.798005in}}%
\pgfpathlineto{\pgfqpoint{8.615301in}{4.806209in}}%
\pgfpathlineto{\pgfqpoint{8.630734in}{4.811859in}}%
\pgfpathlineto{\pgfqpoint{8.646168in}{4.814587in}}%
\pgfpathlineto{\pgfqpoint{8.661602in}{4.814163in}}%
\pgfpathlineto{\pgfqpoint{8.677035in}{4.810504in}}%
\pgfpathlineto{\pgfqpoint{8.692469in}{4.803674in}}%
\pgfpathlineto{\pgfqpoint{8.707903in}{4.793881in}}%
\pgfpathlineto{\pgfqpoint{8.723336in}{4.781468in}}%
\pgfpathlineto{\pgfqpoint{8.738770in}{4.766900in}}%
\pgfpathlineto{\pgfqpoint{8.769637in}{4.733691in}}%
\pgfpathlineto{\pgfqpoint{8.800504in}{4.699791in}}%
\pgfpathlineto{\pgfqpoint{8.815938in}{4.684524in}}%
\pgfpathlineto{\pgfqpoint{8.831372in}{4.671437in}}%
\pgfpathlineto{\pgfqpoint{8.846805in}{4.661289in}}%
\pgfpathlineto{\pgfqpoint{8.862239in}{4.654791in}}%
\pgfpathlineto{\pgfqpoint{8.877673in}{4.652575in}}%
\pgfpathlineto{\pgfqpoint{8.893106in}{4.655183in}}%
\pgfpathlineto{\pgfqpoint{8.908540in}{4.663044in}}%
\pgfpathlineto{\pgfqpoint{8.923973in}{4.676466in}}%
\pgfpathlineto{\pgfqpoint{8.939407in}{4.695625in}}%
\pgfpathlineto{\pgfqpoint{8.954841in}{4.720558in}}%
\pgfpathlineto{\pgfqpoint{8.970274in}{4.751164in}}%
\pgfpathlineto{\pgfqpoint{8.985708in}{4.787210in}}%
\pgfpathlineto{\pgfqpoint{9.001142in}{4.828329in}}%
\pgfpathlineto{\pgfqpoint{9.016575in}{4.874038in}}%
\pgfpathlineto{\pgfqpoint{9.047443in}{4.976790in}}%
\pgfpathlineto{\pgfqpoint{9.078310in}{5.089812in}}%
\pgfpathlineto{\pgfqpoint{9.140044in}{5.320744in}}%
\pgfpathlineto{\pgfqpoint{9.170912in}{5.426016in}}%
\pgfpathlineto{\pgfqpoint{9.186345in}{5.473730in}}%
\pgfpathlineto{\pgfqpoint{9.201779in}{5.517470in}}%
\pgfpathlineto{\pgfqpoint{9.217212in}{5.556842in}}%
\pgfpathlineto{\pgfqpoint{9.232646in}{5.591554in}}%
\pgfpathlineto{\pgfqpoint{9.248080in}{5.621419in}}%
\pgfpathlineto{\pgfqpoint{9.263513in}{5.646355in}}%
\pgfpathlineto{\pgfqpoint{9.278947in}{5.666376in}}%
\pgfpathlineto{\pgfqpoint{9.294381in}{5.681588in}}%
\pgfpathlineto{\pgfqpoint{9.309814in}{5.692176in}}%
\pgfpathlineto{\pgfqpoint{9.325248in}{5.698392in}}%
\pgfpathlineto{\pgfqpoint{9.340682in}{5.700543in}}%
\pgfpathlineto{\pgfqpoint{9.356115in}{5.698976in}}%
\pgfpathlineto{\pgfqpoint{9.371549in}{5.694060in}}%
\pgfpathlineto{\pgfqpoint{9.386982in}{5.686174in}}%
\pgfpathlineto{\pgfqpoint{9.402416in}{5.675694in}}%
\pgfpathlineto{\pgfqpoint{9.417850in}{5.662976in}}%
\pgfpathlineto{\pgfqpoint{9.433283in}{5.648347in}}%
\pgfpathlineto{\pgfqpoint{9.464151in}{5.614467in}}%
\pgfpathlineto{\pgfqpoint{9.495018in}{5.575785in}}%
\pgfpathlineto{\pgfqpoint{9.525885in}{5.533213in}}%
\pgfpathlineto{\pgfqpoint{9.556752in}{5.486898in}}%
\pgfpathlineto{\pgfqpoint{9.587620in}{5.436420in}}%
\pgfpathlineto{\pgfqpoint{9.618487in}{5.381103in}}%
\pgfpathlineto{\pgfqpoint{9.649354in}{5.320374in}}%
\pgfpathlineto{\pgfqpoint{9.680222in}{5.254127in}}%
\pgfpathlineto{\pgfqpoint{9.711089in}{5.183006in}}%
\pgfpathlineto{\pgfqpoint{9.741956in}{5.108583in}}%
\pgfpathlineto{\pgfqpoint{9.741956in}{5.108583in}}%
\pgfusepath{stroke}%
\end{pgfscope}%
\begin{pgfscope}%
\pgfpathrectangle{\pgfqpoint{5.706832in}{3.881603in}}{\pgfqpoint{4.227273in}{2.800000in}} %
\pgfusepath{clip}%
\pgfsetrectcap%
\pgfsetroundjoin%
\pgfsetlinewidth{0.501875pt}%
\definecolor{currentstroke}{rgb}{0.056863,0.767363,0.905873}%
\pgfsetstrokecolor{currentstroke}%
\pgfsetdash{}{0pt}%
\pgfpathmoveto{\pgfqpoint{5.898981in}{5.620234in}}%
\pgfpathlineto{\pgfqpoint{5.914415in}{5.618178in}}%
\pgfpathlineto{\pgfqpoint{5.929848in}{5.611890in}}%
\pgfpathlineto{\pgfqpoint{5.945282in}{5.601097in}}%
\pgfpathlineto{\pgfqpoint{5.960715in}{5.585639in}}%
\pgfpathlineto{\pgfqpoint{5.976149in}{5.565481in}}%
\pgfpathlineto{\pgfqpoint{5.991583in}{5.540714in}}%
\pgfpathlineto{\pgfqpoint{6.007016in}{5.511552in}}%
\pgfpathlineto{\pgfqpoint{6.022450in}{5.478333in}}%
\pgfpathlineto{\pgfqpoint{6.037884in}{5.441508in}}%
\pgfpathlineto{\pgfqpoint{6.068751in}{5.359353in}}%
\pgfpathlineto{\pgfqpoint{6.161353in}{5.098688in}}%
\pgfpathlineto{\pgfqpoint{6.176786in}{5.061747in}}%
\pgfpathlineto{\pgfqpoint{6.192220in}{5.028740in}}%
\pgfpathlineto{\pgfqpoint{6.207654in}{5.000321in}}%
\pgfpathlineto{\pgfqpoint{6.223087in}{4.977056in}}%
\pgfpathlineto{\pgfqpoint{6.238521in}{4.959415in}}%
\pgfpathlineto{\pgfqpoint{6.253955in}{4.947756in}}%
\pgfpathlineto{\pgfqpoint{6.269388in}{4.942327in}}%
\pgfpathlineto{\pgfqpoint{6.284822in}{4.943258in}}%
\pgfpathlineto{\pgfqpoint{6.300255in}{4.950560in}}%
\pgfpathlineto{\pgfqpoint{6.315689in}{4.964132in}}%
\pgfpathlineto{\pgfqpoint{6.331123in}{4.983761in}}%
\pgfpathlineto{\pgfqpoint{6.346556in}{5.009133in}}%
\pgfpathlineto{\pgfqpoint{6.361990in}{5.039844in}}%
\pgfpathlineto{\pgfqpoint{6.377424in}{5.075407in}}%
\pgfpathlineto{\pgfqpoint{6.392857in}{5.115269in}}%
\pgfpathlineto{\pgfqpoint{6.423724in}{5.205430in}}%
\pgfpathlineto{\pgfqpoint{6.454592in}{5.305104in}}%
\pgfpathlineto{\pgfqpoint{6.531760in}{5.561391in}}%
\pgfpathlineto{\pgfqpoint{6.562627in}{5.654600in}}%
\pgfpathlineto{\pgfqpoint{6.593494in}{5.737180in}}%
\pgfpathlineto{\pgfqpoint{6.608928in}{5.773741in}}%
\pgfpathlineto{\pgfqpoint{6.624362in}{5.806867in}}%
\pgfpathlineto{\pgfqpoint{6.639795in}{5.836415in}}%
\pgfpathlineto{\pgfqpoint{6.655229in}{5.862288in}}%
\pgfpathlineto{\pgfqpoint{6.670663in}{5.884429in}}%
\pgfpathlineto{\pgfqpoint{6.686096in}{5.902820in}}%
\pgfpathlineto{\pgfqpoint{6.701530in}{5.917471in}}%
\pgfpathlineto{\pgfqpoint{6.716964in}{5.928414in}}%
\pgfpathlineto{\pgfqpoint{6.732397in}{5.935703in}}%
\pgfpathlineto{\pgfqpoint{6.747831in}{5.939401in}}%
\pgfpathlineto{\pgfqpoint{6.763264in}{5.939584in}}%
\pgfpathlineto{\pgfqpoint{6.778698in}{5.936331in}}%
\pgfpathlineto{\pgfqpoint{6.794132in}{5.929729in}}%
\pgfpathlineto{\pgfqpoint{6.809565in}{5.919864in}}%
\pgfpathlineto{\pgfqpoint{6.824999in}{5.906828in}}%
\pgfpathlineto{\pgfqpoint{6.840433in}{5.890718in}}%
\pgfpathlineto{\pgfqpoint{6.855866in}{5.871636in}}%
\pgfpathlineto{\pgfqpoint{6.871300in}{5.849695in}}%
\pgfpathlineto{\pgfqpoint{6.886734in}{5.825021in}}%
\pgfpathlineto{\pgfqpoint{6.902167in}{5.797754in}}%
\pgfpathlineto{\pgfqpoint{6.933034in}{5.736112in}}%
\pgfpathlineto{\pgfqpoint{6.963902in}{5.666352in}}%
\pgfpathlineto{\pgfqpoint{6.994769in}{5.590502in}}%
\pgfpathlineto{\pgfqpoint{7.102804in}{5.316137in}}%
\pgfpathlineto{\pgfqpoint{7.133672in}{5.247911in}}%
\pgfpathlineto{\pgfqpoint{7.149105in}{5.217340in}}%
\pgfpathlineto{\pgfqpoint{7.164539in}{5.189519in}}%
\pgfpathlineto{\pgfqpoint{7.179973in}{5.164679in}}%
\pgfpathlineto{\pgfqpoint{7.195406in}{5.142991in}}%
\pgfpathlineto{\pgfqpoint{7.210840in}{5.124559in}}%
\pgfpathlineto{\pgfqpoint{7.226274in}{5.109417in}}%
\pgfpathlineto{\pgfqpoint{7.241707in}{5.097525in}}%
\pgfpathlineto{\pgfqpoint{7.257141in}{5.088767in}}%
\pgfpathlineto{\pgfqpoint{7.272574in}{5.082951in}}%
\pgfpathlineto{\pgfqpoint{7.288008in}{5.079817in}}%
\pgfpathlineto{\pgfqpoint{7.303442in}{5.079035in}}%
\pgfpathlineto{\pgfqpoint{7.318875in}{5.080219in}}%
\pgfpathlineto{\pgfqpoint{7.349743in}{5.086692in}}%
\pgfpathlineto{\pgfqpoint{7.396044in}{5.099237in}}%
\pgfpathlineto{\pgfqpoint{7.411477in}{5.102152in}}%
\pgfpathlineto{\pgfqpoint{7.426911in}{5.103644in}}%
\pgfpathlineto{\pgfqpoint{7.442344in}{5.103303in}}%
\pgfpathlineto{\pgfqpoint{7.457778in}{5.100778in}}%
\pgfpathlineto{\pgfqpoint{7.473212in}{5.095795in}}%
\pgfpathlineto{\pgfqpoint{7.488645in}{5.088163in}}%
\pgfpathlineto{\pgfqpoint{7.504079in}{5.077781in}}%
\pgfpathlineto{\pgfqpoint{7.519513in}{5.064651in}}%
\pgfpathlineto{\pgfqpoint{7.534946in}{5.048872in}}%
\pgfpathlineto{\pgfqpoint{7.550380in}{5.030648in}}%
\pgfpathlineto{\pgfqpoint{7.581247in}{4.988160in}}%
\pgfpathlineto{\pgfqpoint{7.673849in}{4.849786in}}%
\pgfpathlineto{\pgfqpoint{7.689283in}{4.831843in}}%
\pgfpathlineto{\pgfqpoint{7.704716in}{4.817134in}}%
\pgfpathlineto{\pgfqpoint{7.720150in}{4.806253in}}%
\pgfpathlineto{\pgfqpoint{7.735584in}{4.799725in}}%
\pgfpathlineto{\pgfqpoint{7.751017in}{4.797989in}}%
\pgfpathlineto{\pgfqpoint{7.766451in}{4.801385in}}%
\pgfpathlineto{\pgfqpoint{7.781884in}{4.810141in}}%
\pgfpathlineto{\pgfqpoint{7.797318in}{4.824363in}}%
\pgfpathlineto{\pgfqpoint{7.812752in}{4.844031in}}%
\pgfpathlineto{\pgfqpoint{7.828185in}{4.868992in}}%
\pgfpathlineto{\pgfqpoint{7.843619in}{4.898968in}}%
\pgfpathlineto{\pgfqpoint{7.859053in}{4.933552in}}%
\pgfpathlineto{\pgfqpoint{7.874486in}{4.972221in}}%
\pgfpathlineto{\pgfqpoint{7.905353in}{5.059204in}}%
\pgfpathlineto{\pgfqpoint{7.997955in}{5.338194in}}%
\pgfpathlineto{\pgfqpoint{8.013389in}{5.378087in}}%
\pgfpathlineto{\pgfqpoint{8.028823in}{5.413902in}}%
\pgfpathlineto{\pgfqpoint{8.044256in}{5.444974in}}%
\pgfpathlineto{\pgfqpoint{8.059690in}{5.470737in}}%
\pgfpathlineto{\pgfqpoint{8.075123in}{5.490743in}}%
\pgfpathlineto{\pgfqpoint{8.090557in}{5.504672in}}%
\pgfpathlineto{\pgfqpoint{8.105991in}{5.512334in}}%
\pgfpathlineto{\pgfqpoint{8.121424in}{5.513675in}}%
\pgfpathlineto{\pgfqpoint{8.136858in}{5.508776in}}%
\pgfpathlineto{\pgfqpoint{8.152292in}{5.497849in}}%
\pgfpathlineto{\pgfqpoint{8.167725in}{5.481226in}}%
\pgfpathlineto{\pgfqpoint{8.183159in}{5.459353in}}%
\pgfpathlineto{\pgfqpoint{8.198593in}{5.432774in}}%
\pgfpathlineto{\pgfqpoint{8.214026in}{5.402113in}}%
\pgfpathlineto{\pgfqpoint{8.244893in}{5.331339in}}%
\pgfpathlineto{\pgfqpoint{8.291194in}{5.212750in}}%
\pgfpathlineto{\pgfqpoint{8.322062in}{5.133902in}}%
\pgfpathlineto{\pgfqpoint{8.352929in}{5.061046in}}%
\pgfpathlineto{\pgfqpoint{8.368363in}{5.028037in}}%
\pgfpathlineto{\pgfqpoint{8.383796in}{4.997740in}}%
\pgfpathlineto{\pgfqpoint{8.399230in}{4.970346in}}%
\pgfpathlineto{\pgfqpoint{8.414663in}{4.945942in}}%
\pgfpathlineto{\pgfqpoint{8.430097in}{4.924512in}}%
\pgfpathlineto{\pgfqpoint{8.445531in}{4.905938in}}%
\pgfpathlineto{\pgfqpoint{8.460964in}{4.890018in}}%
\pgfpathlineto{\pgfqpoint{8.476398in}{4.876469in}}%
\pgfpathlineto{\pgfqpoint{8.491832in}{4.864944in}}%
\pgfpathlineto{\pgfqpoint{8.522699in}{4.846357in}}%
\pgfpathlineto{\pgfqpoint{8.599867in}{4.806001in}}%
\pgfpathlineto{\pgfqpoint{8.630734in}{4.784722in}}%
\pgfpathlineto{\pgfqpoint{8.661602in}{4.758164in}}%
\pgfpathlineto{\pgfqpoint{8.692469in}{4.726606in}}%
\pgfpathlineto{\pgfqpoint{8.785071in}{4.624914in}}%
\pgfpathlineto{\pgfqpoint{8.800504in}{4.611758in}}%
\pgfpathlineto{\pgfqpoint{8.815938in}{4.601108in}}%
\pgfpathlineto{\pgfqpoint{8.831372in}{4.593485in}}%
\pgfpathlineto{\pgfqpoint{8.846805in}{4.589367in}}%
\pgfpathlineto{\pgfqpoint{8.862239in}{4.589181in}}%
\pgfpathlineto{\pgfqpoint{8.877673in}{4.593281in}}%
\pgfpathlineto{\pgfqpoint{8.893106in}{4.601939in}}%
\pgfpathlineto{\pgfqpoint{8.908540in}{4.615336in}}%
\pgfpathlineto{\pgfqpoint{8.923973in}{4.633552in}}%
\pgfpathlineto{\pgfqpoint{8.939407in}{4.656565in}}%
\pgfpathlineto{\pgfqpoint{8.954841in}{4.684246in}}%
\pgfpathlineto{\pgfqpoint{8.970274in}{4.716362in}}%
\pgfpathlineto{\pgfqpoint{8.985708in}{4.752580in}}%
\pgfpathlineto{\pgfqpoint{9.001142in}{4.792475in}}%
\pgfpathlineto{\pgfqpoint{9.032009in}{4.881189in}}%
\pgfpathlineto{\pgfqpoint{9.078310in}{5.027107in}}%
\pgfpathlineto{\pgfqpoint{9.124611in}{5.171760in}}%
\pgfpathlineto{\pgfqpoint{9.155478in}{5.258482in}}%
\pgfpathlineto{\pgfqpoint{9.170912in}{5.297185in}}%
\pgfpathlineto{\pgfqpoint{9.186345in}{5.332166in}}%
\pgfpathlineto{\pgfqpoint{9.201779in}{5.363069in}}%
\pgfpathlineto{\pgfqpoint{9.217212in}{5.389624in}}%
\pgfpathlineto{\pgfqpoint{9.232646in}{5.411659in}}%
\pgfpathlineto{\pgfqpoint{9.248080in}{5.429093in}}%
\pgfpathlineto{\pgfqpoint{9.263513in}{5.441938in}}%
\pgfpathlineto{\pgfqpoint{9.278947in}{5.450290in}}%
\pgfpathlineto{\pgfqpoint{9.294381in}{5.454328in}}%
\pgfpathlineto{\pgfqpoint{9.309814in}{5.454300in}}%
\pgfpathlineto{\pgfqpoint{9.325248in}{5.450514in}}%
\pgfpathlineto{\pgfqpoint{9.340682in}{5.443329in}}%
\pgfpathlineto{\pgfqpoint{9.356115in}{5.433139in}}%
\pgfpathlineto{\pgfqpoint{9.371549in}{5.420361in}}%
\pgfpathlineto{\pgfqpoint{9.386982in}{5.405424in}}%
\pgfpathlineto{\pgfqpoint{9.417850in}{5.370766in}}%
\pgfpathlineto{\pgfqpoint{9.464151in}{5.312610in}}%
\pgfpathlineto{\pgfqpoint{9.510452in}{5.254302in}}%
\pgfpathlineto{\pgfqpoint{9.541319in}{5.217809in}}%
\pgfpathlineto{\pgfqpoint{9.572186in}{5.183835in}}%
\pgfpathlineto{\pgfqpoint{9.618487in}{5.137153in}}%
\pgfpathlineto{\pgfqpoint{9.664788in}{5.093909in}}%
\pgfpathlineto{\pgfqpoint{9.726522in}{5.038868in}}%
\pgfpathlineto{\pgfqpoint{9.741956in}{5.025566in}}%
\pgfpathlineto{\pgfqpoint{9.741956in}{5.025566in}}%
\pgfusepath{stroke}%
\end{pgfscope}%
\begin{pgfscope}%
\pgfpathrectangle{\pgfqpoint{5.706832in}{3.881603in}}{\pgfqpoint{4.227273in}{2.800000in}} %
\pgfusepath{clip}%
\pgfsetrectcap%
\pgfsetroundjoin%
\pgfsetlinewidth{0.501875pt}%
\definecolor{currentstroke}{rgb}{0.135294,0.840344,0.878081}%
\pgfsetstrokecolor{currentstroke}%
\pgfsetdash{}{0pt}%
\pgfpathmoveto{\pgfqpoint{5.898981in}{5.610880in}}%
\pgfpathlineto{\pgfqpoint{5.914415in}{5.603499in}}%
\pgfpathlineto{\pgfqpoint{5.929848in}{5.593746in}}%
\pgfpathlineto{\pgfqpoint{5.945282in}{5.581635in}}%
\pgfpathlineto{\pgfqpoint{5.960715in}{5.567208in}}%
\pgfpathlineto{\pgfqpoint{5.976149in}{5.550533in}}%
\pgfpathlineto{\pgfqpoint{5.991583in}{5.531702in}}%
\pgfpathlineto{\pgfqpoint{6.007016in}{5.510837in}}%
\pgfpathlineto{\pgfqpoint{6.037884in}{5.463624in}}%
\pgfpathlineto{\pgfqpoint{6.068751in}{5.410450in}}%
\pgfpathlineto{\pgfqpoint{6.115052in}{5.324226in}}%
\pgfpathlineto{\pgfqpoint{6.161353in}{5.238797in}}%
\pgfpathlineto{\pgfqpoint{6.192220in}{5.188073in}}%
\pgfpathlineto{\pgfqpoint{6.207654in}{5.166027in}}%
\pgfpathlineto{\pgfqpoint{6.223087in}{5.146851in}}%
\pgfpathlineto{\pgfqpoint{6.238521in}{5.131029in}}%
\pgfpathlineto{\pgfqpoint{6.253955in}{5.119025in}}%
\pgfpathlineto{\pgfqpoint{6.269388in}{5.111265in}}%
\pgfpathlineto{\pgfqpoint{6.284822in}{5.108131in}}%
\pgfpathlineto{\pgfqpoint{6.300255in}{5.109947in}}%
\pgfpathlineto{\pgfqpoint{6.315689in}{5.116964in}}%
\pgfpathlineto{\pgfqpoint{6.331123in}{5.129354in}}%
\pgfpathlineto{\pgfqpoint{6.346556in}{5.147197in}}%
\pgfpathlineto{\pgfqpoint{6.361990in}{5.170480in}}%
\pgfpathlineto{\pgfqpoint{6.377424in}{5.199082in}}%
\pgfpathlineto{\pgfqpoint{6.392857in}{5.232782in}}%
\pgfpathlineto{\pgfqpoint{6.408291in}{5.271253in}}%
\pgfpathlineto{\pgfqpoint{6.423724in}{5.314066in}}%
\pgfpathlineto{\pgfqpoint{6.454592in}{5.410542in}}%
\pgfpathlineto{\pgfqpoint{6.485459in}{5.517063in}}%
\pgfpathlineto{\pgfqpoint{6.547194in}{5.735429in}}%
\pgfpathlineto{\pgfqpoint{6.578061in}{5.834397in}}%
\pgfpathlineto{\pgfqpoint{6.593494in}{5.878736in}}%
\pgfpathlineto{\pgfqpoint{6.608928in}{5.918826in}}%
\pgfpathlineto{\pgfqpoint{6.624362in}{5.954165in}}%
\pgfpathlineto{\pgfqpoint{6.639795in}{5.984352in}}%
\pgfpathlineto{\pgfqpoint{6.655229in}{6.009092in}}%
\pgfpathlineto{\pgfqpoint{6.670663in}{6.028204in}}%
\pgfpathlineto{\pgfqpoint{6.686096in}{6.041618in}}%
\pgfpathlineto{\pgfqpoint{6.701530in}{6.049371in}}%
\pgfpathlineto{\pgfqpoint{6.716964in}{6.051605in}}%
\pgfpathlineto{\pgfqpoint{6.732397in}{6.048553in}}%
\pgfpathlineto{\pgfqpoint{6.747831in}{6.040532in}}%
\pgfpathlineto{\pgfqpoint{6.763264in}{6.027927in}}%
\pgfpathlineto{\pgfqpoint{6.778698in}{6.011174in}}%
\pgfpathlineto{\pgfqpoint{6.794132in}{5.990750in}}%
\pgfpathlineto{\pgfqpoint{6.809565in}{5.967152in}}%
\pgfpathlineto{\pgfqpoint{6.840433in}{5.912439in}}%
\pgfpathlineto{\pgfqpoint{6.871300in}{5.850875in}}%
\pgfpathlineto{\pgfqpoint{6.933034in}{5.719677in}}%
\pgfpathlineto{\pgfqpoint{6.994769in}{5.590412in}}%
\pgfpathlineto{\pgfqpoint{7.041070in}{5.498236in}}%
\pgfpathlineto{\pgfqpoint{7.087371in}{5.410588in}}%
\pgfpathlineto{\pgfqpoint{7.133672in}{5.328613in}}%
\pgfpathlineto{\pgfqpoint{7.164539in}{5.278642in}}%
\pgfpathlineto{\pgfqpoint{7.195406in}{5.234106in}}%
\pgfpathlineto{\pgfqpoint{7.210840in}{5.214402in}}%
\pgfpathlineto{\pgfqpoint{7.226274in}{5.196687in}}%
\pgfpathlineto{\pgfqpoint{7.241707in}{5.181144in}}%
\pgfpathlineto{\pgfqpoint{7.257141in}{5.167929in}}%
\pgfpathlineto{\pgfqpoint{7.272574in}{5.157151in}}%
\pgfpathlineto{\pgfqpoint{7.288008in}{5.148864in}}%
\pgfpathlineto{\pgfqpoint{7.303442in}{5.143059in}}%
\pgfpathlineto{\pgfqpoint{7.318875in}{5.139658in}}%
\pgfpathlineto{\pgfqpoint{7.334309in}{5.138504in}}%
\pgfpathlineto{\pgfqpoint{7.349743in}{5.139367in}}%
\pgfpathlineto{\pgfqpoint{7.365176in}{5.141939in}}%
\pgfpathlineto{\pgfqpoint{7.396044in}{5.150646in}}%
\pgfpathlineto{\pgfqpoint{7.442344in}{5.165375in}}%
\pgfpathlineto{\pgfqpoint{7.457778in}{5.168601in}}%
\pgfpathlineto{\pgfqpoint{7.473212in}{5.170097in}}%
\pgfpathlineto{\pgfqpoint{7.488645in}{5.169377in}}%
\pgfpathlineto{\pgfqpoint{7.504079in}{5.166014in}}%
\pgfpathlineto{\pgfqpoint{7.519513in}{5.159656in}}%
\pgfpathlineto{\pgfqpoint{7.534946in}{5.150046in}}%
\pgfpathlineto{\pgfqpoint{7.550380in}{5.137037in}}%
\pgfpathlineto{\pgfqpoint{7.565814in}{5.120598in}}%
\pgfpathlineto{\pgfqpoint{7.581247in}{5.100826in}}%
\pgfpathlineto{\pgfqpoint{7.596681in}{5.077944in}}%
\pgfpathlineto{\pgfqpoint{7.612114in}{5.052307in}}%
\pgfpathlineto{\pgfqpoint{7.642982in}{4.994764in}}%
\pgfpathlineto{\pgfqpoint{7.704716in}{4.873867in}}%
\pgfpathlineto{\pgfqpoint{7.720150in}{4.847114in}}%
\pgfpathlineto{\pgfqpoint{7.735584in}{4.823407in}}%
\pgfpathlineto{\pgfqpoint{7.751017in}{4.803516in}}%
\pgfpathlineto{\pgfqpoint{7.766451in}{4.788132in}}%
\pgfpathlineto{\pgfqpoint{7.781884in}{4.777844in}}%
\pgfpathlineto{\pgfqpoint{7.797318in}{4.773120in}}%
\pgfpathlineto{\pgfqpoint{7.812752in}{4.774287in}}%
\pgfpathlineto{\pgfqpoint{7.828185in}{4.781514in}}%
\pgfpathlineto{\pgfqpoint{7.843619in}{4.794813in}}%
\pgfpathlineto{\pgfqpoint{7.859053in}{4.814024in}}%
\pgfpathlineto{\pgfqpoint{7.874486in}{4.838827in}}%
\pgfpathlineto{\pgfqpoint{7.889920in}{4.868741in}}%
\pgfpathlineto{\pgfqpoint{7.905353in}{4.903142in}}%
\pgfpathlineto{\pgfqpoint{7.936221in}{4.982290in}}%
\pgfpathlineto{\pgfqpoint{8.013389in}{5.196340in}}%
\pgfpathlineto{\pgfqpoint{8.028823in}{5.234022in}}%
\pgfpathlineto{\pgfqpoint{8.044256in}{5.267900in}}%
\pgfpathlineto{\pgfqpoint{8.059690in}{5.297260in}}%
\pgfpathlineto{\pgfqpoint{8.075123in}{5.321524in}}%
\pgfpathlineto{\pgfqpoint{8.090557in}{5.340256in}}%
\pgfpathlineto{\pgfqpoint{8.105991in}{5.353177in}}%
\pgfpathlineto{\pgfqpoint{8.121424in}{5.360169in}}%
\pgfpathlineto{\pgfqpoint{8.136858in}{5.361270in}}%
\pgfpathlineto{\pgfqpoint{8.152292in}{5.356672in}}%
\pgfpathlineto{\pgfqpoint{8.167725in}{5.346707in}}%
\pgfpathlineto{\pgfqpoint{8.183159in}{5.331832in}}%
\pgfpathlineto{\pgfqpoint{8.198593in}{5.312610in}}%
\pgfpathlineto{\pgfqpoint{8.214026in}{5.289684in}}%
\pgfpathlineto{\pgfqpoint{8.229460in}{5.263757in}}%
\pgfpathlineto{\pgfqpoint{8.260327in}{5.205839in}}%
\pgfpathlineto{\pgfqpoint{8.322062in}{5.085226in}}%
\pgfpathlineto{\pgfqpoint{8.352929in}{5.031376in}}%
\pgfpathlineto{\pgfqpoint{8.383796in}{4.985138in}}%
\pgfpathlineto{\pgfqpoint{8.399230in}{4.965018in}}%
\pgfpathlineto{\pgfqpoint{8.430097in}{4.930139in}}%
\pgfpathlineto{\pgfqpoint{8.460964in}{4.900583in}}%
\pgfpathlineto{\pgfqpoint{8.522699in}{4.845539in}}%
\pgfpathlineto{\pgfqpoint{8.553566in}{4.814192in}}%
\pgfpathlineto{\pgfqpoint{8.584433in}{4.777707in}}%
\pgfpathlineto{\pgfqpoint{8.615301in}{4.735825in}}%
\pgfpathlineto{\pgfqpoint{8.661602in}{4.666187in}}%
\pgfpathlineto{\pgfqpoint{8.692469in}{4.619752in}}%
\pgfpathlineto{\pgfqpoint{8.723336in}{4.578234in}}%
\pgfpathlineto{\pgfqpoint{8.738770in}{4.560770in}}%
\pgfpathlineto{\pgfqpoint{8.754203in}{4.546260in}}%
\pgfpathlineto{\pgfqpoint{8.769637in}{4.535223in}}%
\pgfpathlineto{\pgfqpoint{8.785071in}{4.528117in}}%
\pgfpathlineto{\pgfqpoint{8.800504in}{4.525326in}}%
\pgfpathlineto{\pgfqpoint{8.815938in}{4.527143in}}%
\pgfpathlineto{\pgfqpoint{8.831372in}{4.533758in}}%
\pgfpathlineto{\pgfqpoint{8.846805in}{4.545256in}}%
\pgfpathlineto{\pgfqpoint{8.862239in}{4.561609in}}%
\pgfpathlineto{\pgfqpoint{8.877673in}{4.582679in}}%
\pgfpathlineto{\pgfqpoint{8.893106in}{4.608222in}}%
\pgfpathlineto{\pgfqpoint{8.908540in}{4.637894in}}%
\pgfpathlineto{\pgfqpoint{8.923973in}{4.671264in}}%
\pgfpathlineto{\pgfqpoint{8.954841in}{4.747013in}}%
\pgfpathlineto{\pgfqpoint{8.985708in}{4.830814in}}%
\pgfpathlineto{\pgfqpoint{9.047443in}{5.002569in}}%
\pgfpathlineto{\pgfqpoint{9.078310in}{5.081447in}}%
\pgfpathlineto{\pgfqpoint{9.109177in}{5.151075in}}%
\pgfpathlineto{\pgfqpoint{9.124611in}{5.181758in}}%
\pgfpathlineto{\pgfqpoint{9.140044in}{5.209489in}}%
\pgfpathlineto{\pgfqpoint{9.155478in}{5.234214in}}%
\pgfpathlineto{\pgfqpoint{9.170912in}{5.255947in}}%
\pgfpathlineto{\pgfqpoint{9.186345in}{5.274761in}}%
\pgfpathlineto{\pgfqpoint{9.201779in}{5.290781in}}%
\pgfpathlineto{\pgfqpoint{9.217212in}{5.304170in}}%
\pgfpathlineto{\pgfqpoint{9.232646in}{5.315120in}}%
\pgfpathlineto{\pgfqpoint{9.248080in}{5.323842in}}%
\pgfpathlineto{\pgfqpoint{9.263513in}{5.330554in}}%
\pgfpathlineto{\pgfqpoint{9.278947in}{5.335469in}}%
\pgfpathlineto{\pgfqpoint{9.294381in}{5.338794in}}%
\pgfpathlineto{\pgfqpoint{9.309814in}{5.340714in}}%
\pgfpathlineto{\pgfqpoint{9.340682in}{5.340976in}}%
\pgfpathlineto{\pgfqpoint{9.371549in}{5.337267in}}%
\pgfpathlineto{\pgfqpoint{9.402416in}{5.330203in}}%
\pgfpathlineto{\pgfqpoint{9.433283in}{5.320101in}}%
\pgfpathlineto{\pgfqpoint{9.464151in}{5.307151in}}%
\pgfpathlineto{\pgfqpoint{9.495018in}{5.291609in}}%
\pgfpathlineto{\pgfqpoint{9.541319in}{5.264603in}}%
\pgfpathlineto{\pgfqpoint{9.603053in}{5.227008in}}%
\pgfpathlineto{\pgfqpoint{9.633921in}{5.210710in}}%
\pgfpathlineto{\pgfqpoint{9.664788in}{5.198131in}}%
\pgfpathlineto{\pgfqpoint{9.680222in}{5.193655in}}%
\pgfpathlineto{\pgfqpoint{9.695655in}{5.190559in}}%
\pgfpathlineto{\pgfqpoint{9.711089in}{5.188925in}}%
\pgfpathlineto{\pgfqpoint{9.726522in}{5.188795in}}%
\pgfpathlineto{\pgfqpoint{9.741956in}{5.190164in}}%
\pgfpathlineto{\pgfqpoint{9.741956in}{5.190164in}}%
\pgfusepath{stroke}%
\end{pgfscope}%
\begin{pgfscope}%
\pgfpathrectangle{\pgfqpoint{5.706832in}{3.881603in}}{\pgfqpoint{4.227273in}{2.800000in}} %
\pgfusepath{clip}%
\pgfsetrectcap%
\pgfsetroundjoin%
\pgfsetlinewidth{0.501875pt}%
\definecolor{currentstroke}{rgb}{0.221569,0.905873,0.843667}%
\pgfsetstrokecolor{currentstroke}%
\pgfsetdash{}{0pt}%
\pgfpathmoveto{\pgfqpoint{5.898981in}{5.694076in}}%
\pgfpathlineto{\pgfqpoint{5.914415in}{5.688561in}}%
\pgfpathlineto{\pgfqpoint{5.929848in}{5.678417in}}%
\pgfpathlineto{\pgfqpoint{5.945282in}{5.663621in}}%
\pgfpathlineto{\pgfqpoint{5.960715in}{5.644234in}}%
\pgfpathlineto{\pgfqpoint{5.976149in}{5.620403in}}%
\pgfpathlineto{\pgfqpoint{5.991583in}{5.592363in}}%
\pgfpathlineto{\pgfqpoint{6.007016in}{5.560432in}}%
\pgfpathlineto{\pgfqpoint{6.022450in}{5.525008in}}%
\pgfpathlineto{\pgfqpoint{6.053317in}{5.445655in}}%
\pgfpathlineto{\pgfqpoint{6.099618in}{5.314409in}}%
\pgfpathlineto{\pgfqpoint{6.130485in}{5.226894in}}%
\pgfpathlineto{\pgfqpoint{6.161353in}{5.146369in}}%
\pgfpathlineto{\pgfqpoint{6.176786in}{5.110593in}}%
\pgfpathlineto{\pgfqpoint{6.192220in}{5.078752in}}%
\pgfpathlineto{\pgfqpoint{6.207654in}{5.051493in}}%
\pgfpathlineto{\pgfqpoint{6.223087in}{5.029404in}}%
\pgfpathlineto{\pgfqpoint{6.238521in}{5.012997in}}%
\pgfpathlineto{\pgfqpoint{6.253955in}{5.002697in}}%
\pgfpathlineto{\pgfqpoint{6.269388in}{4.998836in}}%
\pgfpathlineto{\pgfqpoint{6.284822in}{5.001637in}}%
\pgfpathlineto{\pgfqpoint{6.300255in}{5.011215in}}%
\pgfpathlineto{\pgfqpoint{6.315689in}{5.027568in}}%
\pgfpathlineto{\pgfqpoint{6.331123in}{5.050577in}}%
\pgfpathlineto{\pgfqpoint{6.346556in}{5.080008in}}%
\pgfpathlineto{\pgfqpoint{6.361990in}{5.115509in}}%
\pgfpathlineto{\pgfqpoint{6.377424in}{5.156622in}}%
\pgfpathlineto{\pgfqpoint{6.392857in}{5.202788in}}%
\pgfpathlineto{\pgfqpoint{6.423724in}{5.307606in}}%
\pgfpathlineto{\pgfqpoint{6.454592in}{5.423918in}}%
\pgfpathlineto{\pgfqpoint{6.516326in}{5.663611in}}%
\pgfpathlineto{\pgfqpoint{6.547194in}{5.773315in}}%
\pgfpathlineto{\pgfqpoint{6.562627in}{5.822941in}}%
\pgfpathlineto{\pgfqpoint{6.578061in}{5.868240in}}%
\pgfpathlineto{\pgfqpoint{6.593494in}{5.908693in}}%
\pgfpathlineto{\pgfqpoint{6.608928in}{5.943879in}}%
\pgfpathlineto{\pgfqpoint{6.624362in}{5.973481in}}%
\pgfpathlineto{\pgfqpoint{6.639795in}{5.997289in}}%
\pgfpathlineto{\pgfqpoint{6.655229in}{6.015199in}}%
\pgfpathlineto{\pgfqpoint{6.670663in}{6.027208in}}%
\pgfpathlineto{\pgfqpoint{6.686096in}{6.033408in}}%
\pgfpathlineto{\pgfqpoint{6.701530in}{6.033980in}}%
\pgfpathlineto{\pgfqpoint{6.716964in}{6.029182in}}%
\pgfpathlineto{\pgfqpoint{6.732397in}{6.019339in}}%
\pgfpathlineto{\pgfqpoint{6.747831in}{6.004826in}}%
\pgfpathlineto{\pgfqpoint{6.763264in}{5.986060in}}%
\pgfpathlineto{\pgfqpoint{6.778698in}{5.963484in}}%
\pgfpathlineto{\pgfqpoint{6.794132in}{5.937550in}}%
\pgfpathlineto{\pgfqpoint{6.809565in}{5.908715in}}%
\pgfpathlineto{\pgfqpoint{6.840433in}{5.844091in}}%
\pgfpathlineto{\pgfqpoint{6.871300in}{5.772869in}}%
\pgfpathlineto{\pgfqpoint{6.917601in}{5.659472in}}%
\pgfpathlineto{\pgfqpoint{7.010203in}{5.430810in}}%
\pgfpathlineto{\pgfqpoint{7.056504in}{5.323717in}}%
\pgfpathlineto{\pgfqpoint{7.087371in}{5.257895in}}%
\pgfpathlineto{\pgfqpoint{7.118238in}{5.198567in}}%
\pgfpathlineto{\pgfqpoint{7.133672in}{5.171989in}}%
\pgfpathlineto{\pgfqpoint{7.149105in}{5.147840in}}%
\pgfpathlineto{\pgfqpoint{7.164539in}{5.126405in}}%
\pgfpathlineto{\pgfqpoint{7.179973in}{5.107956in}}%
\pgfpathlineto{\pgfqpoint{7.195406in}{5.092748in}}%
\pgfpathlineto{\pgfqpoint{7.210840in}{5.080998in}}%
\pgfpathlineto{\pgfqpoint{7.226274in}{5.072879in}}%
\pgfpathlineto{\pgfqpoint{7.241707in}{5.068507in}}%
\pgfpathlineto{\pgfqpoint{7.257141in}{5.067926in}}%
\pgfpathlineto{\pgfqpoint{7.272574in}{5.071101in}}%
\pgfpathlineto{\pgfqpoint{7.288008in}{5.077912in}}%
\pgfpathlineto{\pgfqpoint{7.303442in}{5.088143in}}%
\pgfpathlineto{\pgfqpoint{7.318875in}{5.101485in}}%
\pgfpathlineto{\pgfqpoint{7.334309in}{5.117534in}}%
\pgfpathlineto{\pgfqpoint{7.365176in}{5.155700in}}%
\pgfpathlineto{\pgfqpoint{7.426911in}{5.238000in}}%
\pgfpathlineto{\pgfqpoint{7.442344in}{5.255520in}}%
\pgfpathlineto{\pgfqpoint{7.457778in}{5.270320in}}%
\pgfpathlineto{\pgfqpoint{7.473212in}{5.281706in}}%
\pgfpathlineto{\pgfqpoint{7.488645in}{5.289057in}}%
\pgfpathlineto{\pgfqpoint{7.504079in}{5.291842in}}%
\pgfpathlineto{\pgfqpoint{7.519513in}{5.289642in}}%
\pgfpathlineto{\pgfqpoint{7.534946in}{5.282164in}}%
\pgfpathlineto{\pgfqpoint{7.550380in}{5.269257in}}%
\pgfpathlineto{\pgfqpoint{7.565814in}{5.250924in}}%
\pgfpathlineto{\pgfqpoint{7.581247in}{5.227322in}}%
\pgfpathlineto{\pgfqpoint{7.596681in}{5.198767in}}%
\pgfpathlineto{\pgfqpoint{7.612114in}{5.165728in}}%
\pgfpathlineto{\pgfqpoint{7.627548in}{5.128819in}}%
\pgfpathlineto{\pgfqpoint{7.658415in}{5.046482in}}%
\pgfpathlineto{\pgfqpoint{7.720150in}{4.874397in}}%
\pgfpathlineto{\pgfqpoint{7.735584in}{4.835886in}}%
\pgfpathlineto{\pgfqpoint{7.751017in}{4.801218in}}%
\pgfpathlineto{\pgfqpoint{7.766451in}{4.771303in}}%
\pgfpathlineto{\pgfqpoint{7.781884in}{4.746945in}}%
\pgfpathlineto{\pgfqpoint{7.797318in}{4.728818in}}%
\pgfpathlineto{\pgfqpoint{7.812752in}{4.717446in}}%
\pgfpathlineto{\pgfqpoint{7.828185in}{4.713188in}}%
\pgfpathlineto{\pgfqpoint{7.843619in}{4.716225in}}%
\pgfpathlineto{\pgfqpoint{7.859053in}{4.726558in}}%
\pgfpathlineto{\pgfqpoint{7.874486in}{4.744004in}}%
\pgfpathlineto{\pgfqpoint{7.889920in}{4.768207in}}%
\pgfpathlineto{\pgfqpoint{7.905353in}{4.798641in}}%
\pgfpathlineto{\pgfqpoint{7.920787in}{4.834632in}}%
\pgfpathlineto{\pgfqpoint{7.936221in}{4.875377in}}%
\pgfpathlineto{\pgfqpoint{7.967088in}{4.967401in}}%
\pgfpathlineto{\pgfqpoint{8.044256in}{5.210686in}}%
\pgfpathlineto{\pgfqpoint{8.059690in}{5.253569in}}%
\pgfpathlineto{\pgfqpoint{8.075123in}{5.292531in}}%
\pgfpathlineto{\pgfqpoint{8.090557in}{5.326939in}}%
\pgfpathlineto{\pgfqpoint{8.105991in}{5.356298in}}%
\pgfpathlineto{\pgfqpoint{8.121424in}{5.380260in}}%
\pgfpathlineto{\pgfqpoint{8.136858in}{5.398626in}}%
\pgfpathlineto{\pgfqpoint{8.152292in}{5.411344in}}%
\pgfpathlineto{\pgfqpoint{8.167725in}{5.418502in}}%
\pgfpathlineto{\pgfqpoint{8.183159in}{5.420314in}}%
\pgfpathlineto{\pgfqpoint{8.198593in}{5.417112in}}%
\pgfpathlineto{\pgfqpoint{8.214026in}{5.409318in}}%
\pgfpathlineto{\pgfqpoint{8.229460in}{5.397433in}}%
\pgfpathlineto{\pgfqpoint{8.244893in}{5.382007in}}%
\pgfpathlineto{\pgfqpoint{8.260327in}{5.363618in}}%
\pgfpathlineto{\pgfqpoint{8.291194in}{5.320279in}}%
\pgfpathlineto{\pgfqpoint{8.337495in}{5.246754in}}%
\pgfpathlineto{\pgfqpoint{8.399230in}{5.148085in}}%
\pgfpathlineto{\pgfqpoint{8.522699in}{4.956309in}}%
\pgfpathlineto{\pgfqpoint{8.553566in}{4.901547in}}%
\pgfpathlineto{\pgfqpoint{8.584433in}{4.841970in}}%
\pgfpathlineto{\pgfqpoint{8.630734in}{4.745602in}}%
\pgfpathlineto{\pgfqpoint{8.677035in}{4.648841in}}%
\pgfpathlineto{\pgfqpoint{8.707903in}{4.590991in}}%
\pgfpathlineto{\pgfqpoint{8.723336in}{4.565906in}}%
\pgfpathlineto{\pgfqpoint{8.738770in}{4.544220in}}%
\pgfpathlineto{\pgfqpoint{8.754203in}{4.526530in}}%
\pgfpathlineto{\pgfqpoint{8.769637in}{4.513382in}}%
\pgfpathlineto{\pgfqpoint{8.785071in}{4.505255in}}%
\pgfpathlineto{\pgfqpoint{8.800504in}{4.502541in}}%
\pgfpathlineto{\pgfqpoint{8.815938in}{4.505529in}}%
\pgfpathlineto{\pgfqpoint{8.831372in}{4.514393in}}%
\pgfpathlineto{\pgfqpoint{8.846805in}{4.529186in}}%
\pgfpathlineto{\pgfqpoint{8.862239in}{4.549829in}}%
\pgfpathlineto{\pgfqpoint{8.877673in}{4.576118in}}%
\pgfpathlineto{\pgfqpoint{8.893106in}{4.607716in}}%
\pgfpathlineto{\pgfqpoint{8.908540in}{4.644171in}}%
\pgfpathlineto{\pgfqpoint{8.923973in}{4.684917in}}%
\pgfpathlineto{\pgfqpoint{8.954841in}{4.776558in}}%
\pgfpathlineto{\pgfqpoint{9.001142in}{4.927462in}}%
\pgfpathlineto{\pgfqpoint{9.032009in}{5.027094in}}%
\pgfpathlineto{\pgfqpoint{9.062876in}{5.118307in}}%
\pgfpathlineto{\pgfqpoint{9.078310in}{5.158984in}}%
\pgfpathlineto{\pgfqpoint{9.093743in}{5.195597in}}%
\pgfpathlineto{\pgfqpoint{9.109177in}{5.227696in}}%
\pgfpathlineto{\pgfqpoint{9.124611in}{5.254939in}}%
\pgfpathlineto{\pgfqpoint{9.140044in}{5.277105in}}%
\pgfpathlineto{\pgfqpoint{9.155478in}{5.294091in}}%
\pgfpathlineto{\pgfqpoint{9.170912in}{5.305911in}}%
\pgfpathlineto{\pgfqpoint{9.186345in}{5.312691in}}%
\pgfpathlineto{\pgfqpoint{9.201779in}{5.314661in}}%
\pgfpathlineto{\pgfqpoint{9.217212in}{5.312144in}}%
\pgfpathlineto{\pgfqpoint{9.232646in}{5.305544in}}%
\pgfpathlineto{\pgfqpoint{9.248080in}{5.295332in}}%
\pgfpathlineto{\pgfqpoint{9.263513in}{5.282030in}}%
\pgfpathlineto{\pgfqpoint{9.278947in}{5.266197in}}%
\pgfpathlineto{\pgfqpoint{9.309814in}{5.229254in}}%
\pgfpathlineto{\pgfqpoint{9.371549in}{5.149993in}}%
\pgfpathlineto{\pgfqpoint{9.402416in}{5.115479in}}%
\pgfpathlineto{\pgfqpoint{9.417850in}{5.100812in}}%
\pgfpathlineto{\pgfqpoint{9.433283in}{5.088210in}}%
\pgfpathlineto{\pgfqpoint{9.448717in}{5.077848in}}%
\pgfpathlineto{\pgfqpoint{9.464151in}{5.069841in}}%
\pgfpathlineto{\pgfqpoint{9.479584in}{5.064244in}}%
\pgfpathlineto{\pgfqpoint{9.495018in}{5.061062in}}%
\pgfpathlineto{\pgfqpoint{9.510452in}{5.060251in}}%
\pgfpathlineto{\pgfqpoint{9.525885in}{5.061726in}}%
\pgfpathlineto{\pgfqpoint{9.541319in}{5.065368in}}%
\pgfpathlineto{\pgfqpoint{9.556752in}{5.071029in}}%
\pgfpathlineto{\pgfqpoint{9.572186in}{5.078540in}}%
\pgfpathlineto{\pgfqpoint{9.587620in}{5.087715in}}%
\pgfpathlineto{\pgfqpoint{9.618487in}{5.110269in}}%
\pgfpathlineto{\pgfqpoint{9.649354in}{5.137085in}}%
\pgfpathlineto{\pgfqpoint{9.695655in}{5.181842in}}%
\pgfpathlineto{\pgfqpoint{9.741956in}{5.227648in}}%
\pgfpathlineto{\pgfqpoint{9.741956in}{5.227648in}}%
\pgfusepath{stroke}%
\end{pgfscope}%
\begin{pgfscope}%
\pgfpathrectangle{\pgfqpoint{5.706832in}{3.881603in}}{\pgfqpoint{4.227273in}{2.800000in}} %
\pgfusepath{clip}%
\pgfsetrectcap%
\pgfsetroundjoin%
\pgfsetlinewidth{0.501875pt}%
\definecolor{currentstroke}{rgb}{0.300000,0.951057,0.809017}%
\pgfsetstrokecolor{currentstroke}%
\pgfsetdash{}{0pt}%
\pgfpathmoveto{\pgfqpoint{5.898981in}{5.654468in}}%
\pgfpathlineto{\pgfqpoint{5.914415in}{5.655330in}}%
\pgfpathlineto{\pgfqpoint{5.929848in}{5.652277in}}%
\pgfpathlineto{\pgfqpoint{5.945282in}{5.645176in}}%
\pgfpathlineto{\pgfqpoint{5.960715in}{5.633965in}}%
\pgfpathlineto{\pgfqpoint{5.976149in}{5.618653in}}%
\pgfpathlineto{\pgfqpoint{5.991583in}{5.599325in}}%
\pgfpathlineto{\pgfqpoint{6.007016in}{5.576141in}}%
\pgfpathlineto{\pgfqpoint{6.022450in}{5.549341in}}%
\pgfpathlineto{\pgfqpoint{6.037884in}{5.519236in}}%
\pgfpathlineto{\pgfqpoint{6.068751in}{5.450727in}}%
\pgfpathlineto{\pgfqpoint{6.099618in}{5.374478in}}%
\pgfpathlineto{\pgfqpoint{6.161353in}{5.218303in}}%
\pgfpathlineto{\pgfqpoint{6.192220in}{5.149246in}}%
\pgfpathlineto{\pgfqpoint{6.207654in}{5.119365in}}%
\pgfpathlineto{\pgfqpoint{6.223087in}{5.093428in}}%
\pgfpathlineto{\pgfqpoint{6.238521in}{5.072009in}}%
\pgfpathlineto{\pgfqpoint{6.253955in}{5.055622in}}%
\pgfpathlineto{\pgfqpoint{6.269388in}{5.044705in}}%
\pgfpathlineto{\pgfqpoint{6.284822in}{5.039612in}}%
\pgfpathlineto{\pgfqpoint{6.300255in}{5.040606in}}%
\pgfpathlineto{\pgfqpoint{6.315689in}{5.047849in}}%
\pgfpathlineto{\pgfqpoint{6.331123in}{5.061400in}}%
\pgfpathlineto{\pgfqpoint{6.346556in}{5.081210in}}%
\pgfpathlineto{\pgfqpoint{6.361990in}{5.107121in}}%
\pgfpathlineto{\pgfqpoint{6.377424in}{5.138871in}}%
\pgfpathlineto{\pgfqpoint{6.392857in}{5.176092in}}%
\pgfpathlineto{\pgfqpoint{6.408291in}{5.218323in}}%
\pgfpathlineto{\pgfqpoint{6.423724in}{5.265011in}}%
\pgfpathlineto{\pgfqpoint{6.454592in}{5.369177in}}%
\pgfpathlineto{\pgfqpoint{6.500893in}{5.541260in}}%
\pgfpathlineto{\pgfqpoint{6.547194in}{5.713116in}}%
\pgfpathlineto{\pgfqpoint{6.578061in}{5.817135in}}%
\pgfpathlineto{\pgfqpoint{6.593494in}{5.863901in}}%
\pgfpathlineto{\pgfqpoint{6.608928in}{5.906399in}}%
\pgfpathlineto{\pgfqpoint{6.624362in}{5.944151in}}%
\pgfpathlineto{\pgfqpoint{6.639795in}{5.976763in}}%
\pgfpathlineto{\pgfqpoint{6.655229in}{6.003936in}}%
\pgfpathlineto{\pgfqpoint{6.670663in}{6.025462in}}%
\pgfpathlineto{\pgfqpoint{6.686096in}{6.041228in}}%
\pgfpathlineto{\pgfqpoint{6.701530in}{6.051209in}}%
\pgfpathlineto{\pgfqpoint{6.716964in}{6.055467in}}%
\pgfpathlineto{\pgfqpoint{6.732397in}{6.054143in}}%
\pgfpathlineto{\pgfqpoint{6.747831in}{6.047448in}}%
\pgfpathlineto{\pgfqpoint{6.763264in}{6.035658in}}%
\pgfpathlineto{\pgfqpoint{6.778698in}{6.019102in}}%
\pgfpathlineto{\pgfqpoint{6.794132in}{5.998151in}}%
\pgfpathlineto{\pgfqpoint{6.809565in}{5.973210in}}%
\pgfpathlineto{\pgfqpoint{6.824999in}{5.944706in}}%
\pgfpathlineto{\pgfqpoint{6.840433in}{5.913080in}}%
\pgfpathlineto{\pgfqpoint{6.871300in}{5.842243in}}%
\pgfpathlineto{\pgfqpoint{6.902167in}{5.764192in}}%
\pgfpathlineto{\pgfqpoint{7.025636in}{5.440037in}}%
\pgfpathlineto{\pgfqpoint{7.056504in}{5.368087in}}%
\pgfpathlineto{\pgfqpoint{7.087371in}{5.303591in}}%
\pgfpathlineto{\pgfqpoint{7.102804in}{5.274623in}}%
\pgfpathlineto{\pgfqpoint{7.118238in}{5.248076in}}%
\pgfpathlineto{\pgfqpoint{7.133672in}{5.224108in}}%
\pgfpathlineto{\pgfqpoint{7.149105in}{5.202856in}}%
\pgfpathlineto{\pgfqpoint{7.164539in}{5.184441in}}%
\pgfpathlineto{\pgfqpoint{7.179973in}{5.168953in}}%
\pgfpathlineto{\pgfqpoint{7.195406in}{5.156454in}}%
\pgfpathlineto{\pgfqpoint{7.210840in}{5.146971in}}%
\pgfpathlineto{\pgfqpoint{7.226274in}{5.140490in}}%
\pgfpathlineto{\pgfqpoint{7.241707in}{5.136951in}}%
\pgfpathlineto{\pgfqpoint{7.257141in}{5.136247in}}%
\pgfpathlineto{\pgfqpoint{7.272574in}{5.138216in}}%
\pgfpathlineto{\pgfqpoint{7.288008in}{5.142645in}}%
\pgfpathlineto{\pgfqpoint{7.303442in}{5.149265in}}%
\pgfpathlineto{\pgfqpoint{7.318875in}{5.157757in}}%
\pgfpathlineto{\pgfqpoint{7.349743in}{5.178823in}}%
\pgfpathlineto{\pgfqpoint{7.411477in}{5.224545in}}%
\pgfpathlineto{\pgfqpoint{7.426911in}{5.233835in}}%
\pgfpathlineto{\pgfqpoint{7.442344in}{5.241286in}}%
\pgfpathlineto{\pgfqpoint{7.457778in}{5.246450in}}%
\pgfpathlineto{\pgfqpoint{7.473212in}{5.248927in}}%
\pgfpathlineto{\pgfqpoint{7.488645in}{5.248378in}}%
\pgfpathlineto{\pgfqpoint{7.504079in}{5.244538in}}%
\pgfpathlineto{\pgfqpoint{7.519513in}{5.237224in}}%
\pgfpathlineto{\pgfqpoint{7.534946in}{5.226348in}}%
\pgfpathlineto{\pgfqpoint{7.550380in}{5.211917in}}%
\pgfpathlineto{\pgfqpoint{7.565814in}{5.194044in}}%
\pgfpathlineto{\pgfqpoint{7.581247in}{5.172942in}}%
\pgfpathlineto{\pgfqpoint{7.596681in}{5.148927in}}%
\pgfpathlineto{\pgfqpoint{7.627548in}{5.093883in}}%
\pgfpathlineto{\pgfqpoint{7.720150in}{4.915996in}}%
\pgfpathlineto{\pgfqpoint{7.735584in}{4.891675in}}%
\pgfpathlineto{\pgfqpoint{7.751017in}{4.870646in}}%
\pgfpathlineto{\pgfqpoint{7.766451in}{4.853460in}}%
\pgfpathlineto{\pgfqpoint{7.781884in}{4.840581in}}%
\pgfpathlineto{\pgfqpoint{7.797318in}{4.832373in}}%
\pgfpathlineto{\pgfqpoint{7.812752in}{4.829088in}}%
\pgfpathlineto{\pgfqpoint{7.828185in}{4.830860in}}%
\pgfpathlineto{\pgfqpoint{7.843619in}{4.837695in}}%
\pgfpathlineto{\pgfqpoint{7.859053in}{4.849478in}}%
\pgfpathlineto{\pgfqpoint{7.874486in}{4.865967in}}%
\pgfpathlineto{\pgfqpoint{7.889920in}{4.886805in}}%
\pgfpathlineto{\pgfqpoint{7.905353in}{4.911528in}}%
\pgfpathlineto{\pgfqpoint{7.920787in}{4.939578in}}%
\pgfpathlineto{\pgfqpoint{7.951654in}{5.003047in}}%
\pgfpathlineto{\pgfqpoint{8.028823in}{5.170397in}}%
\pgfpathlineto{\pgfqpoint{8.044256in}{5.199543in}}%
\pgfpathlineto{\pgfqpoint{8.059690in}{5.225778in}}%
\pgfpathlineto{\pgfqpoint{8.075123in}{5.248624in}}%
\pgfpathlineto{\pgfqpoint{8.090557in}{5.267703in}}%
\pgfpathlineto{\pgfqpoint{8.105991in}{5.282741in}}%
\pgfpathlineto{\pgfqpoint{8.121424in}{5.293574in}}%
\pgfpathlineto{\pgfqpoint{8.136858in}{5.300146in}}%
\pgfpathlineto{\pgfqpoint{8.152292in}{5.302508in}}%
\pgfpathlineto{\pgfqpoint{8.167725in}{5.300807in}}%
\pgfpathlineto{\pgfqpoint{8.183159in}{5.295283in}}%
\pgfpathlineto{\pgfqpoint{8.198593in}{5.286248in}}%
\pgfpathlineto{\pgfqpoint{8.214026in}{5.274081in}}%
\pgfpathlineto{\pgfqpoint{8.229460in}{5.259206in}}%
\pgfpathlineto{\pgfqpoint{8.244893in}{5.242075in}}%
\pgfpathlineto{\pgfqpoint{8.275761in}{5.202911in}}%
\pgfpathlineto{\pgfqpoint{8.383796in}{5.055640in}}%
\pgfpathlineto{\pgfqpoint{8.414663in}{5.018385in}}%
\pgfpathlineto{\pgfqpoint{8.460964in}{4.966997in}}%
\pgfpathlineto{\pgfqpoint{8.522699in}{4.900186in}}%
\pgfpathlineto{\pgfqpoint{8.553566in}{4.864513in}}%
\pgfpathlineto{\pgfqpoint{8.599867in}{4.806592in}}%
\pgfpathlineto{\pgfqpoint{8.677035in}{4.707260in}}%
\pgfpathlineto{\pgfqpoint{8.692469in}{4.689669in}}%
\pgfpathlineto{\pgfqpoint{8.707903in}{4.673792in}}%
\pgfpathlineto{\pgfqpoint{8.723336in}{4.660079in}}%
\pgfpathlineto{\pgfqpoint{8.738770in}{4.648973in}}%
\pgfpathlineto{\pgfqpoint{8.754203in}{4.640892in}}%
\pgfpathlineto{\pgfqpoint{8.769637in}{4.636222in}}%
\pgfpathlineto{\pgfqpoint{8.785071in}{4.635295in}}%
\pgfpathlineto{\pgfqpoint{8.800504in}{4.638380in}}%
\pgfpathlineto{\pgfqpoint{8.815938in}{4.645670in}}%
\pgfpathlineto{\pgfqpoint{8.831372in}{4.657276in}}%
\pgfpathlineto{\pgfqpoint{8.846805in}{4.673216in}}%
\pgfpathlineto{\pgfqpoint{8.862239in}{4.693415in}}%
\pgfpathlineto{\pgfqpoint{8.877673in}{4.717701in}}%
\pgfpathlineto{\pgfqpoint{8.893106in}{4.745812in}}%
\pgfpathlineto{\pgfqpoint{8.908540in}{4.777395in}}%
\pgfpathlineto{\pgfqpoint{8.939407in}{4.849171in}}%
\pgfpathlineto{\pgfqpoint{8.970274in}{4.928786in}}%
\pgfpathlineto{\pgfqpoint{9.032009in}{5.091504in}}%
\pgfpathlineto{\pgfqpoint{9.062876in}{5.164557in}}%
\pgfpathlineto{\pgfqpoint{9.078310in}{5.197067in}}%
\pgfpathlineto{\pgfqpoint{9.093743in}{5.226323in}}%
\pgfpathlineto{\pgfqpoint{9.109177in}{5.251989in}}%
\pgfpathlineto{\pgfqpoint{9.124611in}{5.273816in}}%
\pgfpathlineto{\pgfqpoint{9.140044in}{5.291646in}}%
\pgfpathlineto{\pgfqpoint{9.155478in}{5.305413in}}%
\pgfpathlineto{\pgfqpoint{9.170912in}{5.315142in}}%
\pgfpathlineto{\pgfqpoint{9.186345in}{5.320943in}}%
\pgfpathlineto{\pgfqpoint{9.201779in}{5.323009in}}%
\pgfpathlineto{\pgfqpoint{9.217212in}{5.321605in}}%
\pgfpathlineto{\pgfqpoint{9.232646in}{5.317060in}}%
\pgfpathlineto{\pgfqpoint{9.248080in}{5.309755in}}%
\pgfpathlineto{\pgfqpoint{9.263513in}{5.300115in}}%
\pgfpathlineto{\pgfqpoint{9.278947in}{5.288592in}}%
\pgfpathlineto{\pgfqpoint{9.309814in}{5.261774in}}%
\pgfpathlineto{\pgfqpoint{9.371549in}{5.205842in}}%
\pgfpathlineto{\pgfqpoint{9.402416in}{5.183071in}}%
\pgfpathlineto{\pgfqpoint{9.417850in}{5.174042in}}%
\pgfpathlineto{\pgfqpoint{9.433283in}{5.166851in}}%
\pgfpathlineto{\pgfqpoint{9.448717in}{5.161629in}}%
\pgfpathlineto{\pgfqpoint{9.464151in}{5.158453in}}%
\pgfpathlineto{\pgfqpoint{9.479584in}{5.157349in}}%
\pgfpathlineto{\pgfqpoint{9.495018in}{5.158294in}}%
\pgfpathlineto{\pgfqpoint{9.510452in}{5.161223in}}%
\pgfpathlineto{\pgfqpoint{9.525885in}{5.166029in}}%
\pgfpathlineto{\pgfqpoint{9.541319in}{5.172573in}}%
\pgfpathlineto{\pgfqpoint{9.556752in}{5.180690in}}%
\pgfpathlineto{\pgfqpoint{9.587620in}{5.200872in}}%
\pgfpathlineto{\pgfqpoint{9.618487in}{5.224918in}}%
\pgfpathlineto{\pgfqpoint{9.726522in}{5.315288in}}%
\pgfpathlineto{\pgfqpoint{9.741956in}{5.326567in}}%
\pgfpathlineto{\pgfqpoint{9.741956in}{5.326567in}}%
\pgfusepath{stroke}%
\end{pgfscope}%
\begin{pgfscope}%
\pgfpathrectangle{\pgfqpoint{5.706832in}{3.881603in}}{\pgfqpoint{4.227273in}{2.800000in}} %
\pgfusepath{clip}%
\pgfsetrectcap%
\pgfsetroundjoin%
\pgfsetlinewidth{0.501875pt}%
\definecolor{currentstroke}{rgb}{0.378431,0.981823,0.771298}%
\pgfsetstrokecolor{currentstroke}%
\pgfsetdash{}{0pt}%
\pgfpathmoveto{\pgfqpoint{5.898981in}{5.588802in}}%
\pgfpathlineto{\pgfqpoint{5.914415in}{5.596817in}}%
\pgfpathlineto{\pgfqpoint{5.929848in}{5.601210in}}%
\pgfpathlineto{\pgfqpoint{5.945282in}{5.601798in}}%
\pgfpathlineto{\pgfqpoint{5.960715in}{5.598463in}}%
\pgfpathlineto{\pgfqpoint{5.976149in}{5.591156in}}%
\pgfpathlineto{\pgfqpoint{5.991583in}{5.579903in}}%
\pgfpathlineto{\pgfqpoint{6.007016in}{5.564802in}}%
\pgfpathlineto{\pgfqpoint{6.022450in}{5.546025in}}%
\pgfpathlineto{\pgfqpoint{6.037884in}{5.523820in}}%
\pgfpathlineto{\pgfqpoint{6.053317in}{5.498505in}}%
\pgfpathlineto{\pgfqpoint{6.084185in}{5.440150in}}%
\pgfpathlineto{\pgfqpoint{6.115052in}{5.374756in}}%
\pgfpathlineto{\pgfqpoint{6.176786in}{5.241562in}}%
\pgfpathlineto{\pgfqpoint{6.192220in}{5.211475in}}%
\pgfpathlineto{\pgfqpoint{6.207654in}{5.183945in}}%
\pgfpathlineto{\pgfqpoint{6.223087in}{5.159576in}}%
\pgfpathlineto{\pgfqpoint{6.238521in}{5.138926in}}%
\pgfpathlineto{\pgfqpoint{6.253955in}{5.122506in}}%
\pgfpathlineto{\pgfqpoint{6.269388in}{5.110763in}}%
\pgfpathlineto{\pgfqpoint{6.284822in}{5.104074in}}%
\pgfpathlineto{\pgfqpoint{6.300255in}{5.102735in}}%
\pgfpathlineto{\pgfqpoint{6.315689in}{5.106956in}}%
\pgfpathlineto{\pgfqpoint{6.331123in}{5.116855in}}%
\pgfpathlineto{\pgfqpoint{6.346556in}{5.132456in}}%
\pgfpathlineto{\pgfqpoint{6.361990in}{5.153683in}}%
\pgfpathlineto{\pgfqpoint{6.377424in}{5.180366in}}%
\pgfpathlineto{\pgfqpoint{6.392857in}{5.212238in}}%
\pgfpathlineto{\pgfqpoint{6.408291in}{5.248945in}}%
\pgfpathlineto{\pgfqpoint{6.423724in}{5.290045in}}%
\pgfpathlineto{\pgfqpoint{6.454592in}{5.383292in}}%
\pgfpathlineto{\pgfqpoint{6.485459in}{5.487103in}}%
\pgfpathlineto{\pgfqpoint{6.562627in}{5.755579in}}%
\pgfpathlineto{\pgfqpoint{6.593494in}{5.851472in}}%
\pgfpathlineto{\pgfqpoint{6.608928in}{5.894359in}}%
\pgfpathlineto{\pgfqpoint{6.624362in}{5.933162in}}%
\pgfpathlineto{\pgfqpoint{6.639795in}{5.967438in}}%
\pgfpathlineto{\pgfqpoint{6.655229in}{5.996815in}}%
\pgfpathlineto{\pgfqpoint{6.670663in}{6.021005in}}%
\pgfpathlineto{\pgfqpoint{6.686096in}{6.039799in}}%
\pgfpathlineto{\pgfqpoint{6.701530in}{6.053073in}}%
\pgfpathlineto{\pgfqpoint{6.716964in}{6.060782in}}%
\pgfpathlineto{\pgfqpoint{6.732397in}{6.062957in}}%
\pgfpathlineto{\pgfqpoint{6.747831in}{6.059707in}}%
\pgfpathlineto{\pgfqpoint{6.763264in}{6.051205in}}%
\pgfpathlineto{\pgfqpoint{6.778698in}{6.037686in}}%
\pgfpathlineto{\pgfqpoint{6.794132in}{6.019442in}}%
\pgfpathlineto{\pgfqpoint{6.809565in}{5.996808in}}%
\pgfpathlineto{\pgfqpoint{6.824999in}{5.970159in}}%
\pgfpathlineto{\pgfqpoint{6.840433in}{5.939903in}}%
\pgfpathlineto{\pgfqpoint{6.855866in}{5.906466in}}%
\pgfpathlineto{\pgfqpoint{6.886734in}{5.831839in}}%
\pgfpathlineto{\pgfqpoint{6.917601in}{5.749896in}}%
\pgfpathlineto{\pgfqpoint{7.025636in}{5.454991in}}%
\pgfpathlineto{\pgfqpoint{7.056504in}{5.380697in}}%
\pgfpathlineto{\pgfqpoint{7.087371in}{5.315185in}}%
\pgfpathlineto{\pgfqpoint{7.102804in}{5.286265in}}%
\pgfpathlineto{\pgfqpoint{7.118238in}{5.260132in}}%
\pgfpathlineto{\pgfqpoint{7.133672in}{5.236920in}}%
\pgfpathlineto{\pgfqpoint{7.149105in}{5.216725in}}%
\pgfpathlineto{\pgfqpoint{7.164539in}{5.199610in}}%
\pgfpathlineto{\pgfqpoint{7.179973in}{5.185598in}}%
\pgfpathlineto{\pgfqpoint{7.195406in}{5.174672in}}%
\pgfpathlineto{\pgfqpoint{7.210840in}{5.166774in}}%
\pgfpathlineto{\pgfqpoint{7.226274in}{5.161798in}}%
\pgfpathlineto{\pgfqpoint{7.241707in}{5.159597in}}%
\pgfpathlineto{\pgfqpoint{7.257141in}{5.159978in}}%
\pgfpathlineto{\pgfqpoint{7.272574in}{5.162700in}}%
\pgfpathlineto{\pgfqpoint{7.288008in}{5.167483in}}%
\pgfpathlineto{\pgfqpoint{7.303442in}{5.174004in}}%
\pgfpathlineto{\pgfqpoint{7.334309in}{5.190786in}}%
\pgfpathlineto{\pgfqpoint{7.396044in}{5.227594in}}%
\pgfpathlineto{\pgfqpoint{7.411477in}{5.234899in}}%
\pgfpathlineto{\pgfqpoint{7.426911in}{5.240584in}}%
\pgfpathlineto{\pgfqpoint{7.442344in}{5.244261in}}%
\pgfpathlineto{\pgfqpoint{7.457778in}{5.245585in}}%
\pgfpathlineto{\pgfqpoint{7.473212in}{5.244260in}}%
\pgfpathlineto{\pgfqpoint{7.488645in}{5.240051in}}%
\pgfpathlineto{\pgfqpoint{7.504079in}{5.232791in}}%
\pgfpathlineto{\pgfqpoint{7.519513in}{5.222393in}}%
\pgfpathlineto{\pgfqpoint{7.534946in}{5.208846in}}%
\pgfpathlineto{\pgfqpoint{7.550380in}{5.192226in}}%
\pgfpathlineto{\pgfqpoint{7.565814in}{5.172694in}}%
\pgfpathlineto{\pgfqpoint{7.581247in}{5.150495in}}%
\pgfpathlineto{\pgfqpoint{7.612114in}{5.099472in}}%
\pgfpathlineto{\pgfqpoint{7.658415in}{5.013320in}}%
\pgfpathlineto{\pgfqpoint{7.689283in}{4.956000in}}%
\pgfpathlineto{\pgfqpoint{7.720150in}{4.904438in}}%
\pgfpathlineto{\pgfqpoint{7.735584in}{4.882287in}}%
\pgfpathlineto{\pgfqpoint{7.751017in}{4.863263in}}%
\pgfpathlineto{\pgfqpoint{7.766451in}{4.847825in}}%
\pgfpathlineto{\pgfqpoint{7.781884in}{4.836356in}}%
\pgfpathlineto{\pgfqpoint{7.797318in}{4.829152in}}%
\pgfpathlineto{\pgfqpoint{7.812752in}{4.826418in}}%
\pgfpathlineto{\pgfqpoint{7.828185in}{4.828255in}}%
\pgfpathlineto{\pgfqpoint{7.843619in}{4.834662in}}%
\pgfpathlineto{\pgfqpoint{7.859053in}{4.845531in}}%
\pgfpathlineto{\pgfqpoint{7.874486in}{4.860653in}}%
\pgfpathlineto{\pgfqpoint{7.889920in}{4.879717in}}%
\pgfpathlineto{\pgfqpoint{7.905353in}{4.902326in}}%
\pgfpathlineto{\pgfqpoint{7.920787in}{4.927999in}}%
\pgfpathlineto{\pgfqpoint{7.951654in}{4.986291in}}%
\pgfpathlineto{\pgfqpoint{8.028823in}{5.142629in}}%
\pgfpathlineto{\pgfqpoint{8.044256in}{5.170538in}}%
\pgfpathlineto{\pgfqpoint{8.059690in}{5.195972in}}%
\pgfpathlineto{\pgfqpoint{8.075123in}{5.218463in}}%
\pgfpathlineto{\pgfqpoint{8.090557in}{5.237621in}}%
\pgfpathlineto{\pgfqpoint{8.105991in}{5.253140in}}%
\pgfpathlineto{\pgfqpoint{8.121424in}{5.264803in}}%
\pgfpathlineto{\pgfqpoint{8.136858in}{5.272484in}}%
\pgfpathlineto{\pgfqpoint{8.152292in}{5.276144in}}%
\pgfpathlineto{\pgfqpoint{8.167725in}{5.275835in}}%
\pgfpathlineto{\pgfqpoint{8.183159in}{5.271685in}}%
\pgfpathlineto{\pgfqpoint{8.198593in}{5.263900in}}%
\pgfpathlineto{\pgfqpoint{8.214026in}{5.252747in}}%
\pgfpathlineto{\pgfqpoint{8.229460in}{5.238546in}}%
\pgfpathlineto{\pgfqpoint{8.244893in}{5.221659in}}%
\pgfpathlineto{\pgfqpoint{8.260327in}{5.202477in}}%
\pgfpathlineto{\pgfqpoint{8.291194in}{5.158856in}}%
\pgfpathlineto{\pgfqpoint{8.337495in}{5.086259in}}%
\pgfpathlineto{\pgfqpoint{8.383796in}{5.013318in}}%
\pgfpathlineto{\pgfqpoint{8.414663in}{4.967804in}}%
\pgfpathlineto{\pgfqpoint{8.445531in}{4.926067in}}%
\pgfpathlineto{\pgfqpoint{8.476398in}{4.888500in}}%
\pgfpathlineto{\pgfqpoint{8.507265in}{4.855053in}}%
\pgfpathlineto{\pgfqpoint{8.538133in}{4.825458in}}%
\pgfpathlineto{\pgfqpoint{8.569000in}{4.799467in}}%
\pgfpathlineto{\pgfqpoint{8.599867in}{4.777048in}}%
\pgfpathlineto{\pgfqpoint{8.630734in}{4.758512in}}%
\pgfpathlineto{\pgfqpoint{8.646168in}{4.750903in}}%
\pgfpathlineto{\pgfqpoint{8.661602in}{4.744553in}}%
\pgfpathlineto{\pgfqpoint{8.677035in}{4.739598in}}%
\pgfpathlineto{\pgfqpoint{8.692469in}{4.736186in}}%
\pgfpathlineto{\pgfqpoint{8.707903in}{4.734469in}}%
\pgfpathlineto{\pgfqpoint{8.723336in}{4.734599in}}%
\pgfpathlineto{\pgfqpoint{8.738770in}{4.736716in}}%
\pgfpathlineto{\pgfqpoint{8.754203in}{4.740946in}}%
\pgfpathlineto{\pgfqpoint{8.769637in}{4.747388in}}%
\pgfpathlineto{\pgfqpoint{8.785071in}{4.756111in}}%
\pgfpathlineto{\pgfqpoint{8.800504in}{4.767148in}}%
\pgfpathlineto{\pgfqpoint{8.815938in}{4.780487in}}%
\pgfpathlineto{\pgfqpoint{8.831372in}{4.796073in}}%
\pgfpathlineto{\pgfqpoint{8.846805in}{4.813801in}}%
\pgfpathlineto{\pgfqpoint{8.877673in}{4.855027in}}%
\pgfpathlineto{\pgfqpoint{8.908540in}{4.902397in}}%
\pgfpathlineto{\pgfqpoint{8.970274in}{5.005549in}}%
\pgfpathlineto{\pgfqpoint{9.001142in}{5.055536in}}%
\pgfpathlineto{\pgfqpoint{9.032009in}{5.100652in}}%
\pgfpathlineto{\pgfqpoint{9.047443in}{5.120617in}}%
\pgfpathlineto{\pgfqpoint{9.062876in}{5.138527in}}%
\pgfpathlineto{\pgfqpoint{9.078310in}{5.154195in}}%
\pgfpathlineto{\pgfqpoint{9.093743in}{5.167494in}}%
\pgfpathlineto{\pgfqpoint{9.109177in}{5.178353in}}%
\pgfpathlineto{\pgfqpoint{9.124611in}{5.186763in}}%
\pgfpathlineto{\pgfqpoint{9.140044in}{5.192774in}}%
\pgfpathlineto{\pgfqpoint{9.155478in}{5.196495in}}%
\pgfpathlineto{\pgfqpoint{9.170912in}{5.198089in}}%
\pgfpathlineto{\pgfqpoint{9.186345in}{5.197771in}}%
\pgfpathlineto{\pgfqpoint{9.201779in}{5.195795in}}%
\pgfpathlineto{\pgfqpoint{9.232646in}{5.188072in}}%
\pgfpathlineto{\pgfqpoint{9.309814in}{5.162476in}}%
\pgfpathlineto{\pgfqpoint{9.325248in}{5.158945in}}%
\pgfpathlineto{\pgfqpoint{9.340682in}{5.156622in}}%
\pgfpathlineto{\pgfqpoint{9.356115in}{5.155731in}}%
\pgfpathlineto{\pgfqpoint{9.371549in}{5.156453in}}%
\pgfpathlineto{\pgfqpoint{9.386982in}{5.158920in}}%
\pgfpathlineto{\pgfqpoint{9.402416in}{5.163218in}}%
\pgfpathlineto{\pgfqpoint{9.417850in}{5.169383in}}%
\pgfpathlineto{\pgfqpoint{9.433283in}{5.177400in}}%
\pgfpathlineto{\pgfqpoint{9.448717in}{5.187208in}}%
\pgfpathlineto{\pgfqpoint{9.464151in}{5.198705in}}%
\pgfpathlineto{\pgfqpoint{9.495018in}{5.226147in}}%
\pgfpathlineto{\pgfqpoint{9.525885in}{5.258184in}}%
\pgfpathlineto{\pgfqpoint{9.633921in}{5.377782in}}%
\pgfpathlineto{\pgfqpoint{9.664788in}{5.405962in}}%
\pgfpathlineto{\pgfqpoint{9.680222in}{5.418049in}}%
\pgfpathlineto{\pgfqpoint{9.695655in}{5.428621in}}%
\pgfpathlineto{\pgfqpoint{9.711089in}{5.437582in}}%
\pgfpathlineto{\pgfqpoint{9.726522in}{5.444863in}}%
\pgfpathlineto{\pgfqpoint{9.741956in}{5.450427in}}%
\pgfpathlineto{\pgfqpoint{9.741956in}{5.450427in}}%
\pgfusepath{stroke}%
\end{pgfscope}%
\begin{pgfscope}%
\pgfpathrectangle{\pgfqpoint{5.706832in}{3.881603in}}{\pgfqpoint{4.227273in}{2.800000in}} %
\pgfusepath{clip}%
\pgfsetrectcap%
\pgfsetroundjoin%
\pgfsetlinewidth{0.501875pt}%
\definecolor{currentstroke}{rgb}{0.456863,0.997705,0.730653}%
\pgfsetstrokecolor{currentstroke}%
\pgfsetdash{}{0pt}%
\pgfpathmoveto{\pgfqpoint{5.898981in}{5.597250in}}%
\pgfpathlineto{\pgfqpoint{5.914415in}{5.600055in}}%
\pgfpathlineto{\pgfqpoint{5.929848in}{5.599369in}}%
\pgfpathlineto{\pgfqpoint{5.945282in}{5.595067in}}%
\pgfpathlineto{\pgfqpoint{5.960715in}{5.587086in}}%
\pgfpathlineto{\pgfqpoint{5.976149in}{5.575422in}}%
\pgfpathlineto{\pgfqpoint{5.991583in}{5.560137in}}%
\pgfpathlineto{\pgfqpoint{6.007016in}{5.541357in}}%
\pgfpathlineto{\pgfqpoint{6.022450in}{5.519274in}}%
\pgfpathlineto{\pgfqpoint{6.037884in}{5.494144in}}%
\pgfpathlineto{\pgfqpoint{6.053317in}{5.466285in}}%
\pgfpathlineto{\pgfqpoint{6.084185in}{5.403942in}}%
\pgfpathlineto{\pgfqpoint{6.130485in}{5.301001in}}%
\pgfpathlineto{\pgfqpoint{6.161353in}{5.232488in}}%
\pgfpathlineto{\pgfqpoint{6.192220in}{5.169581in}}%
\pgfpathlineto{\pgfqpoint{6.207654in}{5.141703in}}%
\pgfpathlineto{\pgfqpoint{6.223087in}{5.116955in}}%
\pgfpathlineto{\pgfqpoint{6.238521in}{5.095855in}}%
\pgfpathlineto{\pgfqpoint{6.253955in}{5.078875in}}%
\pgfpathlineto{\pgfqpoint{6.269388in}{5.066430in}}%
\pgfpathlineto{\pgfqpoint{6.284822in}{5.058874in}}%
\pgfpathlineto{\pgfqpoint{6.300255in}{5.056487in}}%
\pgfpathlineto{\pgfqpoint{6.315689in}{5.059475in}}%
\pgfpathlineto{\pgfqpoint{6.331123in}{5.067959in}}%
\pgfpathlineto{\pgfqpoint{6.346556in}{5.081977in}}%
\pgfpathlineto{\pgfqpoint{6.361990in}{5.101477in}}%
\pgfpathlineto{\pgfqpoint{6.377424in}{5.126322in}}%
\pgfpathlineto{\pgfqpoint{6.392857in}{5.156289in}}%
\pgfpathlineto{\pgfqpoint{6.408291in}{5.191072in}}%
\pgfpathlineto{\pgfqpoint{6.423724in}{5.230286in}}%
\pgfpathlineto{\pgfqpoint{6.439158in}{5.273475in}}%
\pgfpathlineto{\pgfqpoint{6.470025in}{5.369648in}}%
\pgfpathlineto{\pgfqpoint{6.500893in}{5.474828in}}%
\pgfpathlineto{\pgfqpoint{6.562627in}{5.690785in}}%
\pgfpathlineto{\pgfqpoint{6.593494in}{5.790741in}}%
\pgfpathlineto{\pgfqpoint{6.608928in}{5.836518in}}%
\pgfpathlineto{\pgfqpoint{6.624362in}{5.878770in}}%
\pgfpathlineto{\pgfqpoint{6.639795in}{5.917023in}}%
\pgfpathlineto{\pgfqpoint{6.655229in}{5.950869in}}%
\pgfpathlineto{\pgfqpoint{6.670663in}{5.979969in}}%
\pgfpathlineto{\pgfqpoint{6.686096in}{6.004053in}}%
\pgfpathlineto{\pgfqpoint{6.701530in}{6.022927in}}%
\pgfpathlineto{\pgfqpoint{6.716964in}{6.036470in}}%
\pgfpathlineto{\pgfqpoint{6.732397in}{6.044633in}}%
\pgfpathlineto{\pgfqpoint{6.747831in}{6.047440in}}%
\pgfpathlineto{\pgfqpoint{6.763264in}{6.044980in}}%
\pgfpathlineto{\pgfqpoint{6.778698in}{6.037403in}}%
\pgfpathlineto{\pgfqpoint{6.794132in}{6.024920in}}%
\pgfpathlineto{\pgfqpoint{6.809565in}{6.007791in}}%
\pgfpathlineto{\pgfqpoint{6.824999in}{5.986321in}}%
\pgfpathlineto{\pgfqpoint{6.840433in}{5.960856in}}%
\pgfpathlineto{\pgfqpoint{6.855866in}{5.931769in}}%
\pgfpathlineto{\pgfqpoint{6.871300in}{5.899463in}}%
\pgfpathlineto{\pgfqpoint{6.902167in}{5.826879in}}%
\pgfpathlineto{\pgfqpoint{6.933034in}{5.746563in}}%
\pgfpathlineto{\pgfqpoint{7.041070in}{5.453106in}}%
\pgfpathlineto{\pgfqpoint{7.071937in}{5.377919in}}%
\pgfpathlineto{\pgfqpoint{7.102804in}{5.310846in}}%
\pgfpathlineto{\pgfqpoint{7.118238in}{5.280855in}}%
\pgfpathlineto{\pgfqpoint{7.133672in}{5.253431in}}%
\pgfpathlineto{\pgfqpoint{7.149105in}{5.228677in}}%
\pgfpathlineto{\pgfqpoint{7.164539in}{5.206665in}}%
\pgfpathlineto{\pgfqpoint{7.179973in}{5.187424in}}%
\pgfpathlineto{\pgfqpoint{7.195406in}{5.170948in}}%
\pgfpathlineto{\pgfqpoint{7.210840in}{5.157191in}}%
\pgfpathlineto{\pgfqpoint{7.226274in}{5.146068in}}%
\pgfpathlineto{\pgfqpoint{7.241707in}{5.137455in}}%
\pgfpathlineto{\pgfqpoint{7.257141in}{5.131193in}}%
\pgfpathlineto{\pgfqpoint{7.272574in}{5.127084in}}%
\pgfpathlineto{\pgfqpoint{7.288008in}{5.124901in}}%
\pgfpathlineto{\pgfqpoint{7.303442in}{5.124383in}}%
\pgfpathlineto{\pgfqpoint{7.334309in}{5.127184in}}%
\pgfpathlineto{\pgfqpoint{7.380610in}{5.136178in}}%
\pgfpathlineto{\pgfqpoint{7.411477in}{5.141511in}}%
\pgfpathlineto{\pgfqpoint{7.426911in}{5.143029in}}%
\pgfpathlineto{\pgfqpoint{7.442344in}{5.143409in}}%
\pgfpathlineto{\pgfqpoint{7.457778in}{5.142411in}}%
\pgfpathlineto{\pgfqpoint{7.473212in}{5.139837in}}%
\pgfpathlineto{\pgfqpoint{7.488645in}{5.135537in}}%
\pgfpathlineto{\pgfqpoint{7.504079in}{5.129411in}}%
\pgfpathlineto{\pgfqpoint{7.519513in}{5.121417in}}%
\pgfpathlineto{\pgfqpoint{7.534946in}{5.111572in}}%
\pgfpathlineto{\pgfqpoint{7.550380in}{5.099953in}}%
\pgfpathlineto{\pgfqpoint{7.581247in}{5.071999in}}%
\pgfpathlineto{\pgfqpoint{7.612114in}{5.039331in}}%
\pgfpathlineto{\pgfqpoint{7.673849in}{4.970633in}}%
\pgfpathlineto{\pgfqpoint{7.704716in}{4.941060in}}%
\pgfpathlineto{\pgfqpoint{7.720150in}{4.928913in}}%
\pgfpathlineto{\pgfqpoint{7.735584in}{4.919028in}}%
\pgfpathlineto{\pgfqpoint{7.751017in}{4.911734in}}%
\pgfpathlineto{\pgfqpoint{7.766451in}{4.907309in}}%
\pgfpathlineto{\pgfqpoint{7.781884in}{4.905974in}}%
\pgfpathlineto{\pgfqpoint{7.797318in}{4.907882in}}%
\pgfpathlineto{\pgfqpoint{7.812752in}{4.913115in}}%
\pgfpathlineto{\pgfqpoint{7.828185in}{4.921679in}}%
\pgfpathlineto{\pgfqpoint{7.843619in}{4.933506in}}%
\pgfpathlineto{\pgfqpoint{7.859053in}{4.948445in}}%
\pgfpathlineto{\pgfqpoint{7.874486in}{4.966278in}}%
\pgfpathlineto{\pgfqpoint{7.889920in}{4.986711in}}%
\pgfpathlineto{\pgfqpoint{7.920787in}{5.033901in}}%
\pgfpathlineto{\pgfqpoint{7.967088in}{5.113712in}}%
\pgfpathlineto{\pgfqpoint{7.997955in}{5.167007in}}%
\pgfpathlineto{\pgfqpoint{8.028823in}{5.215555in}}%
\pgfpathlineto{\pgfqpoint{8.044256in}{5.236862in}}%
\pgfpathlineto{\pgfqpoint{8.059690in}{5.255639in}}%
\pgfpathlineto{\pgfqpoint{8.075123in}{5.271541in}}%
\pgfpathlineto{\pgfqpoint{8.090557in}{5.284283in}}%
\pgfpathlineto{\pgfqpoint{8.105991in}{5.293648in}}%
\pgfpathlineto{\pgfqpoint{8.121424in}{5.299491in}}%
\pgfpathlineto{\pgfqpoint{8.136858in}{5.301737in}}%
\pgfpathlineto{\pgfqpoint{8.152292in}{5.300383in}}%
\pgfpathlineto{\pgfqpoint{8.167725in}{5.295496in}}%
\pgfpathlineto{\pgfqpoint{8.183159in}{5.287206in}}%
\pgfpathlineto{\pgfqpoint{8.198593in}{5.275701in}}%
\pgfpathlineto{\pgfqpoint{8.214026in}{5.261221in}}%
\pgfpathlineto{\pgfqpoint{8.229460in}{5.244050in}}%
\pgfpathlineto{\pgfqpoint{8.244893in}{5.224505in}}%
\pgfpathlineto{\pgfqpoint{8.275761in}{5.179678in}}%
\pgfpathlineto{\pgfqpoint{8.306628in}{5.129600in}}%
\pgfpathlineto{\pgfqpoint{8.399230in}{4.974437in}}%
\pgfpathlineto{\pgfqpoint{8.430097in}{4.927890in}}%
\pgfpathlineto{\pgfqpoint{8.460964in}{4.886023in}}%
\pgfpathlineto{\pgfqpoint{8.491832in}{4.849360in}}%
\pgfpathlineto{\pgfqpoint{8.522699in}{4.818092in}}%
\pgfpathlineto{\pgfqpoint{8.553566in}{4.792235in}}%
\pgfpathlineto{\pgfqpoint{8.584433in}{4.771784in}}%
\pgfpathlineto{\pgfqpoint{8.599867in}{4.763609in}}%
\pgfpathlineto{\pgfqpoint{8.615301in}{4.756833in}}%
\pgfpathlineto{\pgfqpoint{8.630734in}{4.751494in}}%
\pgfpathlineto{\pgfqpoint{8.646168in}{4.747640in}}%
\pgfpathlineto{\pgfqpoint{8.661602in}{4.745330in}}%
\pgfpathlineto{\pgfqpoint{8.677035in}{4.744628in}}%
\pgfpathlineto{\pgfqpoint{8.692469in}{4.745604in}}%
\pgfpathlineto{\pgfqpoint{8.707903in}{4.748328in}}%
\pgfpathlineto{\pgfqpoint{8.723336in}{4.752862in}}%
\pgfpathlineto{\pgfqpoint{8.738770in}{4.759263in}}%
\pgfpathlineto{\pgfqpoint{8.754203in}{4.767572in}}%
\pgfpathlineto{\pgfqpoint{8.769637in}{4.777810in}}%
\pgfpathlineto{\pgfqpoint{8.785071in}{4.789976in}}%
\pgfpathlineto{\pgfqpoint{8.800504in}{4.804042in}}%
\pgfpathlineto{\pgfqpoint{8.815938in}{4.819951in}}%
\pgfpathlineto{\pgfqpoint{8.846805in}{4.856897in}}%
\pgfpathlineto{\pgfqpoint{8.877673in}{4.899688in}}%
\pgfpathlineto{\pgfqpoint{8.908540in}{4.946682in}}%
\pgfpathlineto{\pgfqpoint{8.985708in}{5.068373in}}%
\pgfpathlineto{\pgfqpoint{9.016575in}{5.112493in}}%
\pgfpathlineto{\pgfqpoint{9.047443in}{5.150692in}}%
\pgfpathlineto{\pgfqpoint{9.062876in}{5.167028in}}%
\pgfpathlineto{\pgfqpoint{9.078310in}{5.181302in}}%
\pgfpathlineto{\pgfqpoint{9.093743in}{5.193406in}}%
\pgfpathlineto{\pgfqpoint{9.109177in}{5.203274in}}%
\pgfpathlineto{\pgfqpoint{9.124611in}{5.210888in}}%
\pgfpathlineto{\pgfqpoint{9.140044in}{5.216278in}}%
\pgfpathlineto{\pgfqpoint{9.155478in}{5.219518in}}%
\pgfpathlineto{\pgfqpoint{9.170912in}{5.220728in}}%
\pgfpathlineto{\pgfqpoint{9.186345in}{5.220067in}}%
\pgfpathlineto{\pgfqpoint{9.201779in}{5.217734in}}%
\pgfpathlineto{\pgfqpoint{9.217212in}{5.213954in}}%
\pgfpathlineto{\pgfqpoint{9.248080in}{5.203091in}}%
\pgfpathlineto{\pgfqpoint{9.340682in}{5.164432in}}%
\pgfpathlineto{\pgfqpoint{9.356115in}{5.159953in}}%
\pgfpathlineto{\pgfqpoint{9.371549in}{5.156643in}}%
\pgfpathlineto{\pgfqpoint{9.386982in}{5.154666in}}%
\pgfpathlineto{\pgfqpoint{9.402416in}{5.154152in}}%
\pgfpathlineto{\pgfqpoint{9.417850in}{5.155195in}}%
\pgfpathlineto{\pgfqpoint{9.433283in}{5.157855in}}%
\pgfpathlineto{\pgfqpoint{9.448717in}{5.162153in}}%
\pgfpathlineto{\pgfqpoint{9.464151in}{5.168074in}}%
\pgfpathlineto{\pgfqpoint{9.479584in}{5.175573in}}%
\pgfpathlineto{\pgfqpoint{9.495018in}{5.184570in}}%
\pgfpathlineto{\pgfqpoint{9.525885in}{5.206611in}}%
\pgfpathlineto{\pgfqpoint{9.556752in}{5.233068in}}%
\pgfpathlineto{\pgfqpoint{9.603053in}{5.277916in}}%
\pgfpathlineto{\pgfqpoint{9.664788in}{5.339129in}}%
\pgfpathlineto{\pgfqpoint{9.695655in}{5.366951in}}%
\pgfpathlineto{\pgfqpoint{9.726522in}{5.391200in}}%
\pgfpathlineto{\pgfqpoint{9.741956in}{5.401663in}}%
\pgfpathlineto{\pgfqpoint{9.741956in}{5.401663in}}%
\pgfusepath{stroke}%
\end{pgfscope}%
\begin{pgfscope}%
\pgfpathrectangle{\pgfqpoint{5.706832in}{3.881603in}}{\pgfqpoint{4.227273in}{2.800000in}} %
\pgfusepath{clip}%
\pgfsetrectcap%
\pgfsetroundjoin%
\pgfsetlinewidth{0.501875pt}%
\definecolor{currentstroke}{rgb}{0.543137,0.997705,0.682749}%
\pgfsetstrokecolor{currentstroke}%
\pgfsetdash{}{0pt}%
\pgfpathmoveto{\pgfqpoint{5.898981in}{5.535440in}}%
\pgfpathlineto{\pgfqpoint{5.914415in}{5.538855in}}%
\pgfpathlineto{\pgfqpoint{5.929848in}{5.539467in}}%
\pgfpathlineto{\pgfqpoint{5.945282in}{5.537072in}}%
\pgfpathlineto{\pgfqpoint{5.960715in}{5.531520in}}%
\pgfpathlineto{\pgfqpoint{5.976149in}{5.522711in}}%
\pgfpathlineto{\pgfqpoint{5.991583in}{5.510610in}}%
\pgfpathlineto{\pgfqpoint{6.007016in}{5.495244in}}%
\pgfpathlineto{\pgfqpoint{6.022450in}{5.476709in}}%
\pgfpathlineto{\pgfqpoint{6.037884in}{5.455173in}}%
\pgfpathlineto{\pgfqpoint{6.053317in}{5.430869in}}%
\pgfpathlineto{\pgfqpoint{6.084185in}{5.375242in}}%
\pgfpathlineto{\pgfqpoint{6.115052in}{5.313011in}}%
\pgfpathlineto{\pgfqpoint{6.176786in}{5.185470in}}%
\pgfpathlineto{\pgfqpoint{6.207654in}{5.129663in}}%
\pgfpathlineto{\pgfqpoint{6.223087in}{5.105860in}}%
\pgfpathlineto{\pgfqpoint{6.238521in}{5.085540in}}%
\pgfpathlineto{\pgfqpoint{6.253955in}{5.069212in}}%
\pgfpathlineto{\pgfqpoint{6.269388in}{5.057331in}}%
\pgfpathlineto{\pgfqpoint{6.284822in}{5.050284in}}%
\pgfpathlineto{\pgfqpoint{6.300255in}{5.048384in}}%
\pgfpathlineto{\pgfqpoint{6.315689in}{5.051863in}}%
\pgfpathlineto{\pgfqpoint{6.331123in}{5.060862in}}%
\pgfpathlineto{\pgfqpoint{6.346556in}{5.075434in}}%
\pgfpathlineto{\pgfqpoint{6.361990in}{5.095533in}}%
\pgfpathlineto{\pgfqpoint{6.377424in}{5.121021in}}%
\pgfpathlineto{\pgfqpoint{6.392857in}{5.151668in}}%
\pgfpathlineto{\pgfqpoint{6.408291in}{5.187154in}}%
\pgfpathlineto{\pgfqpoint{6.423724in}{5.227074in}}%
\pgfpathlineto{\pgfqpoint{6.439158in}{5.270951in}}%
\pgfpathlineto{\pgfqpoint{6.470025in}{5.368330in}}%
\pgfpathlineto{\pgfqpoint{6.516326in}{5.528798in}}%
\pgfpathlineto{\pgfqpoint{6.562627in}{5.689830in}}%
\pgfpathlineto{\pgfqpoint{6.593494in}{5.788354in}}%
\pgfpathlineto{\pgfqpoint{6.608928in}{5.833114in}}%
\pgfpathlineto{\pgfqpoint{6.624362in}{5.874168in}}%
\pgfpathlineto{\pgfqpoint{6.639795in}{5.911067in}}%
\pgfpathlineto{\pgfqpoint{6.655229in}{5.943431in}}%
\pgfpathlineto{\pgfqpoint{6.670663in}{5.970955in}}%
\pgfpathlineto{\pgfqpoint{6.686096in}{5.993409in}}%
\pgfpathlineto{\pgfqpoint{6.701530in}{6.010643in}}%
\pgfpathlineto{\pgfqpoint{6.716964in}{6.022578in}}%
\pgfpathlineto{\pgfqpoint{6.732397in}{6.029211in}}%
\pgfpathlineto{\pgfqpoint{6.747831in}{6.030610in}}%
\pgfpathlineto{\pgfqpoint{6.763264in}{6.026907in}}%
\pgfpathlineto{\pgfqpoint{6.778698in}{6.018291in}}%
\pgfpathlineto{\pgfqpoint{6.794132in}{6.005007in}}%
\pgfpathlineto{\pgfqpoint{6.809565in}{5.987345in}}%
\pgfpathlineto{\pgfqpoint{6.824999in}{5.965634in}}%
\pgfpathlineto{\pgfqpoint{6.840433in}{5.940234in}}%
\pgfpathlineto{\pgfqpoint{6.855866in}{5.911530in}}%
\pgfpathlineto{\pgfqpoint{6.886734in}{5.845825in}}%
\pgfpathlineto{\pgfqpoint{6.917601in}{5.771819in}}%
\pgfpathlineto{\pgfqpoint{6.963902in}{5.652384in}}%
\pgfpathlineto{\pgfqpoint{7.025636in}{5.493016in}}%
\pgfpathlineto{\pgfqpoint{7.056504in}{5.418951in}}%
\pgfpathlineto{\pgfqpoint{7.087371in}{5.351265in}}%
\pgfpathlineto{\pgfqpoint{7.118238in}{5.291388in}}%
\pgfpathlineto{\pgfqpoint{7.133672in}{5.264700in}}%
\pgfpathlineto{\pgfqpoint{7.149105in}{5.240292in}}%
\pgfpathlineto{\pgfqpoint{7.164539in}{5.218211in}}%
\pgfpathlineto{\pgfqpoint{7.179973in}{5.198473in}}%
\pgfpathlineto{\pgfqpoint{7.195406in}{5.181065in}}%
\pgfpathlineto{\pgfqpoint{7.210840in}{5.165940in}}%
\pgfpathlineto{\pgfqpoint{7.226274in}{5.153022in}}%
\pgfpathlineto{\pgfqpoint{7.241707in}{5.142205in}}%
\pgfpathlineto{\pgfqpoint{7.257141in}{5.133353in}}%
\pgfpathlineto{\pgfqpoint{7.272574in}{5.126302in}}%
\pgfpathlineto{\pgfqpoint{7.288008in}{5.120865in}}%
\pgfpathlineto{\pgfqpoint{7.303442in}{5.116828in}}%
\pgfpathlineto{\pgfqpoint{7.334309in}{5.112013in}}%
\pgfpathlineto{\pgfqpoint{7.380610in}{5.109095in}}%
\pgfpathlineto{\pgfqpoint{7.411477in}{5.106980in}}%
\pgfpathlineto{\pgfqpoint{7.442344in}{5.102504in}}%
\pgfpathlineto{\pgfqpoint{7.473212in}{5.094186in}}%
\pgfpathlineto{\pgfqpoint{7.488645in}{5.088276in}}%
\pgfpathlineto{\pgfqpoint{7.504079in}{5.081121in}}%
\pgfpathlineto{\pgfqpoint{7.534946in}{5.063124in}}%
\pgfpathlineto{\pgfqpoint{7.565814in}{5.040794in}}%
\pgfpathlineto{\pgfqpoint{7.612114in}{5.002369in}}%
\pgfpathlineto{\pgfqpoint{7.642982in}{4.976642in}}%
\pgfpathlineto{\pgfqpoint{7.673849in}{4.953907in}}%
\pgfpathlineto{\pgfqpoint{7.689283in}{4.944529in}}%
\pgfpathlineto{\pgfqpoint{7.704716in}{4.936922in}}%
\pgfpathlineto{\pgfqpoint{7.720150in}{4.931390in}}%
\pgfpathlineto{\pgfqpoint{7.735584in}{4.928204in}}%
\pgfpathlineto{\pgfqpoint{7.751017in}{4.927591in}}%
\pgfpathlineto{\pgfqpoint{7.766451in}{4.929726in}}%
\pgfpathlineto{\pgfqpoint{7.781884in}{4.934726in}}%
\pgfpathlineto{\pgfqpoint{7.797318in}{4.942645in}}%
\pgfpathlineto{\pgfqpoint{7.812752in}{4.953472in}}%
\pgfpathlineto{\pgfqpoint{7.828185in}{4.967124in}}%
\pgfpathlineto{\pgfqpoint{7.843619in}{4.983453in}}%
\pgfpathlineto{\pgfqpoint{7.859053in}{5.002243in}}%
\pgfpathlineto{\pgfqpoint{7.889920in}{5.046041in}}%
\pgfpathlineto{\pgfqpoint{7.920787in}{5.095666in}}%
\pgfpathlineto{\pgfqpoint{7.982522in}{5.198167in}}%
\pgfpathlineto{\pgfqpoint{8.013389in}{5.243541in}}%
\pgfpathlineto{\pgfqpoint{8.028823in}{5.263241in}}%
\pgfpathlineto{\pgfqpoint{8.044256in}{5.280469in}}%
\pgfpathlineto{\pgfqpoint{8.059690in}{5.294931in}}%
\pgfpathlineto{\pgfqpoint{8.075123in}{5.306395in}}%
\pgfpathlineto{\pgfqpoint{8.090557in}{5.314693in}}%
\pgfpathlineto{\pgfqpoint{8.105991in}{5.319727in}}%
\pgfpathlineto{\pgfqpoint{8.121424in}{5.321467in}}%
\pgfpathlineto{\pgfqpoint{8.136858in}{5.319950in}}%
\pgfpathlineto{\pgfqpoint{8.152292in}{5.315280in}}%
\pgfpathlineto{\pgfqpoint{8.167725in}{5.307617in}}%
\pgfpathlineto{\pgfqpoint{8.183159in}{5.297177in}}%
\pgfpathlineto{\pgfqpoint{8.198593in}{5.284220in}}%
\pgfpathlineto{\pgfqpoint{8.214026in}{5.269043in}}%
\pgfpathlineto{\pgfqpoint{8.244893in}{5.233342in}}%
\pgfpathlineto{\pgfqpoint{8.275761in}{5.192806in}}%
\pgfpathlineto{\pgfqpoint{8.368363in}{5.067192in}}%
\pgfpathlineto{\pgfqpoint{8.399230in}{5.029930in}}%
\pgfpathlineto{\pgfqpoint{8.430097in}{4.996348in}}%
\pgfpathlineto{\pgfqpoint{8.460964in}{4.966353in}}%
\pgfpathlineto{\pgfqpoint{8.491832in}{4.939434in}}%
\pgfpathlineto{\pgfqpoint{8.538133in}{4.903273in}}%
\pgfpathlineto{\pgfqpoint{8.584433in}{4.870418in}}%
\pgfpathlineto{\pgfqpoint{8.630734in}{4.840478in}}%
\pgfpathlineto{\pgfqpoint{8.661602in}{4.823020in}}%
\pgfpathlineto{\pgfqpoint{8.692469in}{4.808847in}}%
\pgfpathlineto{\pgfqpoint{8.707903in}{4.803442in}}%
\pgfpathlineto{\pgfqpoint{8.723336in}{4.799416in}}%
\pgfpathlineto{\pgfqpoint{8.738770in}{4.796965in}}%
\pgfpathlineto{\pgfqpoint{8.754203in}{4.796277in}}%
\pgfpathlineto{\pgfqpoint{8.769637in}{4.797522in}}%
\pgfpathlineto{\pgfqpoint{8.785071in}{4.800846in}}%
\pgfpathlineto{\pgfqpoint{8.800504in}{4.806365in}}%
\pgfpathlineto{\pgfqpoint{8.815938in}{4.814161in}}%
\pgfpathlineto{\pgfqpoint{8.831372in}{4.824270in}}%
\pgfpathlineto{\pgfqpoint{8.846805in}{4.836686in}}%
\pgfpathlineto{\pgfqpoint{8.862239in}{4.851356in}}%
\pgfpathlineto{\pgfqpoint{8.877673in}{4.868176in}}%
\pgfpathlineto{\pgfqpoint{8.893106in}{4.886997in}}%
\pgfpathlineto{\pgfqpoint{8.923973in}{4.929814in}}%
\pgfpathlineto{\pgfqpoint{8.954841in}{4.977762in}}%
\pgfpathlineto{\pgfqpoint{9.032009in}{5.102872in}}%
\pgfpathlineto{\pgfqpoint{9.062876in}{5.147775in}}%
\pgfpathlineto{\pgfqpoint{9.078310in}{5.167887in}}%
\pgfpathlineto{\pgfqpoint{9.093743in}{5.186117in}}%
\pgfpathlineto{\pgfqpoint{9.109177in}{5.202280in}}%
\pgfpathlineto{\pgfqpoint{9.124611in}{5.216236in}}%
\pgfpathlineto{\pgfqpoint{9.140044in}{5.227899in}}%
\pgfpathlineto{\pgfqpoint{9.155478in}{5.237231in}}%
\pgfpathlineto{\pgfqpoint{9.170912in}{5.244249in}}%
\pgfpathlineto{\pgfqpoint{9.186345in}{5.249016in}}%
\pgfpathlineto{\pgfqpoint{9.201779in}{5.251646in}}%
\pgfpathlineto{\pgfqpoint{9.217212in}{5.252294in}}%
\pgfpathlineto{\pgfqpoint{9.232646in}{5.251153in}}%
\pgfpathlineto{\pgfqpoint{9.248080in}{5.248451in}}%
\pgfpathlineto{\pgfqpoint{9.278947in}{5.239385in}}%
\pgfpathlineto{\pgfqpoint{9.325248in}{5.220835in}}%
\pgfpathlineto{\pgfqpoint{9.356115in}{5.208448in}}%
\pgfpathlineto{\pgfqpoint{9.386982in}{5.198413in}}%
\pgfpathlineto{\pgfqpoint{9.402416in}{5.194797in}}%
\pgfpathlineto{\pgfqpoint{9.417850in}{5.192335in}}%
\pgfpathlineto{\pgfqpoint{9.433283in}{5.191147in}}%
\pgfpathlineto{\pgfqpoint{9.448717in}{5.191323in}}%
\pgfpathlineto{\pgfqpoint{9.464151in}{5.192914in}}%
\pgfpathlineto{\pgfqpoint{9.479584in}{5.195941in}}%
\pgfpathlineto{\pgfqpoint{9.495018in}{5.200391in}}%
\pgfpathlineto{\pgfqpoint{9.510452in}{5.206221in}}%
\pgfpathlineto{\pgfqpoint{9.541319in}{5.221718in}}%
\pgfpathlineto{\pgfqpoint{9.572186in}{5.241602in}}%
\pgfpathlineto{\pgfqpoint{9.603053in}{5.264766in}}%
\pgfpathlineto{\pgfqpoint{9.664788in}{5.315799in}}%
\pgfpathlineto{\pgfqpoint{9.711089in}{5.353048in}}%
\pgfpathlineto{\pgfqpoint{9.741956in}{5.375149in}}%
\pgfpathlineto{\pgfqpoint{9.741956in}{5.375149in}}%
\pgfusepath{stroke}%
\end{pgfscope}%
\begin{pgfscope}%
\pgfpathrectangle{\pgfqpoint{5.706832in}{3.881603in}}{\pgfqpoint{4.227273in}{2.800000in}} %
\pgfusepath{clip}%
\pgfsetrectcap%
\pgfsetroundjoin%
\pgfsetlinewidth{0.501875pt}%
\definecolor{currentstroke}{rgb}{0.621569,0.981823,0.636474}%
\pgfsetstrokecolor{currentstroke}%
\pgfsetdash{}{0pt}%
\pgfpathmoveto{\pgfqpoint{5.898981in}{5.524308in}}%
\pgfpathlineto{\pgfqpoint{5.914415in}{5.523786in}}%
\pgfpathlineto{\pgfqpoint{5.929848in}{5.520251in}}%
\pgfpathlineto{\pgfqpoint{5.945282in}{5.513583in}}%
\pgfpathlineto{\pgfqpoint{5.960715in}{5.503715in}}%
\pgfpathlineto{\pgfqpoint{5.976149in}{5.490635in}}%
\pgfpathlineto{\pgfqpoint{5.991583in}{5.474395in}}%
\pgfpathlineto{\pgfqpoint{6.007016in}{5.455108in}}%
\pgfpathlineto{\pgfqpoint{6.022450in}{5.432949in}}%
\pgfpathlineto{\pgfqpoint{6.037884in}{5.408158in}}%
\pgfpathlineto{\pgfqpoint{6.068751in}{5.351933in}}%
\pgfpathlineto{\pgfqpoint{6.099618in}{5.289484in}}%
\pgfpathlineto{\pgfqpoint{6.161353in}{5.161578in}}%
\pgfpathlineto{\pgfqpoint{6.192220in}{5.104909in}}%
\pgfpathlineto{\pgfqpoint{6.207654in}{5.080339in}}%
\pgfpathlineto{\pgfqpoint{6.223087in}{5.058973in}}%
\pgfpathlineto{\pgfqpoint{6.238521in}{5.041285in}}%
\pgfpathlineto{\pgfqpoint{6.253955in}{5.027701in}}%
\pgfpathlineto{\pgfqpoint{6.269388in}{5.018591in}}%
\pgfpathlineto{\pgfqpoint{6.284822in}{5.014264in}}%
\pgfpathlineto{\pgfqpoint{6.300255in}{5.014954in}}%
\pgfpathlineto{\pgfqpoint{6.315689in}{5.020822in}}%
\pgfpathlineto{\pgfqpoint{6.331123in}{5.031951in}}%
\pgfpathlineto{\pgfqpoint{6.346556in}{5.048340in}}%
\pgfpathlineto{\pgfqpoint{6.361990in}{5.069908in}}%
\pgfpathlineto{\pgfqpoint{6.377424in}{5.096493in}}%
\pgfpathlineto{\pgfqpoint{6.392857in}{5.127852in}}%
\pgfpathlineto{\pgfqpoint{6.408291in}{5.163670in}}%
\pgfpathlineto{\pgfqpoint{6.423724in}{5.203563in}}%
\pgfpathlineto{\pgfqpoint{6.454592in}{5.293728in}}%
\pgfpathlineto{\pgfqpoint{6.485459in}{5.394164in}}%
\pgfpathlineto{\pgfqpoint{6.578061in}{5.708480in}}%
\pgfpathlineto{\pgfqpoint{6.608928in}{5.801354in}}%
\pgfpathlineto{\pgfqpoint{6.624362in}{5.843118in}}%
\pgfpathlineto{\pgfqpoint{6.639795in}{5.881187in}}%
\pgfpathlineto{\pgfqpoint{6.655229in}{5.915193in}}%
\pgfpathlineto{\pgfqpoint{6.670663in}{5.944830in}}%
\pgfpathlineto{\pgfqpoint{6.686096in}{5.969854in}}%
\pgfpathlineto{\pgfqpoint{6.701530in}{5.990082in}}%
\pgfpathlineto{\pgfqpoint{6.716964in}{6.005393in}}%
\pgfpathlineto{\pgfqpoint{6.732397in}{6.015731in}}%
\pgfpathlineto{\pgfqpoint{6.747831in}{6.021094in}}%
\pgfpathlineto{\pgfqpoint{6.763264in}{6.021541in}}%
\pgfpathlineto{\pgfqpoint{6.778698in}{6.017180in}}%
\pgfpathlineto{\pgfqpoint{6.794132in}{6.008169in}}%
\pgfpathlineto{\pgfqpoint{6.809565in}{5.994711in}}%
\pgfpathlineto{\pgfqpoint{6.824999in}{5.977044in}}%
\pgfpathlineto{\pgfqpoint{6.840433in}{5.955445in}}%
\pgfpathlineto{\pgfqpoint{6.855866in}{5.930216in}}%
\pgfpathlineto{\pgfqpoint{6.871300in}{5.901683in}}%
\pgfpathlineto{\pgfqpoint{6.886734in}{5.870194in}}%
\pgfpathlineto{\pgfqpoint{6.917601in}{5.799793in}}%
\pgfpathlineto{\pgfqpoint{6.948468in}{5.721983in}}%
\pgfpathlineto{\pgfqpoint{7.010203in}{5.556049in}}%
\pgfpathlineto{\pgfqpoint{7.056504in}{5.433602in}}%
\pgfpathlineto{\pgfqpoint{7.087371in}{5.357548in}}%
\pgfpathlineto{\pgfqpoint{7.118238in}{5.288287in}}%
\pgfpathlineto{\pgfqpoint{7.149105in}{5.227331in}}%
\pgfpathlineto{\pgfqpoint{7.164539in}{5.200306in}}%
\pgfpathlineto{\pgfqpoint{7.179973in}{5.175697in}}%
\pgfpathlineto{\pgfqpoint{7.195406in}{5.153548in}}%
\pgfpathlineto{\pgfqpoint{7.210840in}{5.133863in}}%
\pgfpathlineto{\pgfqpoint{7.226274in}{5.116615in}}%
\pgfpathlineto{\pgfqpoint{7.241707in}{5.101739in}}%
\pgfpathlineto{\pgfqpoint{7.257141in}{5.089135in}}%
\pgfpathlineto{\pgfqpoint{7.272574in}{5.078672in}}%
\pgfpathlineto{\pgfqpoint{7.288008in}{5.070186in}}%
\pgfpathlineto{\pgfqpoint{7.303442in}{5.063489in}}%
\pgfpathlineto{\pgfqpoint{7.318875in}{5.058367in}}%
\pgfpathlineto{\pgfqpoint{7.349743in}{5.051894in}}%
\pgfpathlineto{\pgfqpoint{7.380610in}{5.048758in}}%
\pgfpathlineto{\pgfqpoint{7.442344in}{5.044411in}}%
\pgfpathlineto{\pgfqpoint{7.473212in}{5.039736in}}%
\pgfpathlineto{\pgfqpoint{7.504079in}{5.031850in}}%
\pgfpathlineto{\pgfqpoint{7.534946in}{5.020386in}}%
\pgfpathlineto{\pgfqpoint{7.565814in}{5.005709in}}%
\pgfpathlineto{\pgfqpoint{7.658415in}{4.956207in}}%
\pgfpathlineto{\pgfqpoint{7.673849in}{4.949872in}}%
\pgfpathlineto{\pgfqpoint{7.689283in}{4.944874in}}%
\pgfpathlineto{\pgfqpoint{7.704716in}{4.941499in}}%
\pgfpathlineto{\pgfqpoint{7.720150in}{4.940016in}}%
\pgfpathlineto{\pgfqpoint{7.735584in}{4.940662in}}%
\pgfpathlineto{\pgfqpoint{7.751017in}{4.943634in}}%
\pgfpathlineto{\pgfqpoint{7.766451in}{4.949088in}}%
\pgfpathlineto{\pgfqpoint{7.781884in}{4.957126in}}%
\pgfpathlineto{\pgfqpoint{7.797318in}{4.967794in}}%
\pgfpathlineto{\pgfqpoint{7.812752in}{4.981080in}}%
\pgfpathlineto{\pgfqpoint{7.828185in}{4.996912in}}%
\pgfpathlineto{\pgfqpoint{7.843619in}{5.015153in}}%
\pgfpathlineto{\pgfqpoint{7.859053in}{5.035610in}}%
\pgfpathlineto{\pgfqpoint{7.889920in}{5.082115in}}%
\pgfpathlineto{\pgfqpoint{7.920787in}{5.133842in}}%
\pgfpathlineto{\pgfqpoint{7.982522in}{5.239905in}}%
\pgfpathlineto{\pgfqpoint{8.013389in}{5.287252in}}%
\pgfpathlineto{\pgfqpoint{8.028823in}{5.308058in}}%
\pgfpathlineto{\pgfqpoint{8.044256in}{5.326481in}}%
\pgfpathlineto{\pgfqpoint{8.059690in}{5.342223in}}%
\pgfpathlineto{\pgfqpoint{8.075123in}{5.355039in}}%
\pgfpathlineto{\pgfqpoint{8.090557in}{5.364741in}}%
\pgfpathlineto{\pgfqpoint{8.105991in}{5.371200in}}%
\pgfpathlineto{\pgfqpoint{8.121424in}{5.374350in}}%
\pgfpathlineto{\pgfqpoint{8.136858in}{5.374187in}}%
\pgfpathlineto{\pgfqpoint{8.152292in}{5.370766in}}%
\pgfpathlineto{\pgfqpoint{8.167725in}{5.364199in}}%
\pgfpathlineto{\pgfqpoint{8.183159in}{5.354651in}}%
\pgfpathlineto{\pgfqpoint{8.198593in}{5.342332in}}%
\pgfpathlineto{\pgfqpoint{8.214026in}{5.327490in}}%
\pgfpathlineto{\pgfqpoint{8.229460in}{5.310407in}}%
\pgfpathlineto{\pgfqpoint{8.260327in}{5.270748in}}%
\pgfpathlineto{\pgfqpoint{8.291194in}{5.225913in}}%
\pgfpathlineto{\pgfqpoint{8.399230in}{5.062081in}}%
\pgfpathlineto{\pgfqpoint{8.430097in}{5.020314in}}%
\pgfpathlineto{\pgfqpoint{8.460964in}{4.982403in}}%
\pgfpathlineto{\pgfqpoint{8.491832in}{4.948492in}}%
\pgfpathlineto{\pgfqpoint{8.522699in}{4.918462in}}%
\pgfpathlineto{\pgfqpoint{8.553566in}{4.892085in}}%
\pgfpathlineto{\pgfqpoint{8.584433in}{4.869196in}}%
\pgfpathlineto{\pgfqpoint{8.615301in}{4.849814in}}%
\pgfpathlineto{\pgfqpoint{8.646168in}{4.834217in}}%
\pgfpathlineto{\pgfqpoint{8.677035in}{4.822958in}}%
\pgfpathlineto{\pgfqpoint{8.692469in}{4.819190in}}%
\pgfpathlineto{\pgfqpoint{8.707903in}{4.816806in}}%
\pgfpathlineto{\pgfqpoint{8.723336in}{4.815918in}}%
\pgfpathlineto{\pgfqpoint{8.738770in}{4.816635in}}%
\pgfpathlineto{\pgfqpoint{8.754203in}{4.819057in}}%
\pgfpathlineto{\pgfqpoint{8.769637in}{4.823272in}}%
\pgfpathlineto{\pgfqpoint{8.785071in}{4.829348in}}%
\pgfpathlineto{\pgfqpoint{8.800504in}{4.837331in}}%
\pgfpathlineto{\pgfqpoint{8.815938in}{4.847237in}}%
\pgfpathlineto{\pgfqpoint{8.831372in}{4.859052in}}%
\pgfpathlineto{\pgfqpoint{8.846805in}{4.872729in}}%
\pgfpathlineto{\pgfqpoint{8.862239in}{4.888182in}}%
\pgfpathlineto{\pgfqpoint{8.893106in}{4.923905in}}%
\pgfpathlineto{\pgfqpoint{8.923973in}{4.964844in}}%
\pgfpathlineto{\pgfqpoint{8.970274in}{5.031847in}}%
\pgfpathlineto{\pgfqpoint{9.016575in}{5.098727in}}%
\pgfpathlineto{\pgfqpoint{9.047443in}{5.139332in}}%
\pgfpathlineto{\pgfqpoint{9.078310in}{5.174322in}}%
\pgfpathlineto{\pgfqpoint{9.093743in}{5.189194in}}%
\pgfpathlineto{\pgfqpoint{9.109177in}{5.202118in}}%
\pgfpathlineto{\pgfqpoint{9.124611in}{5.212994in}}%
\pgfpathlineto{\pgfqpoint{9.140044in}{5.221769in}}%
\pgfpathlineto{\pgfqpoint{9.155478in}{5.228436in}}%
\pgfpathlineto{\pgfqpoint{9.170912in}{5.233033in}}%
\pgfpathlineto{\pgfqpoint{9.186345in}{5.235644in}}%
\pgfpathlineto{\pgfqpoint{9.201779in}{5.236394in}}%
\pgfpathlineto{\pgfqpoint{9.217212in}{5.235446in}}%
\pgfpathlineto{\pgfqpoint{9.232646in}{5.233000in}}%
\pgfpathlineto{\pgfqpoint{9.263513in}{5.224542in}}%
\pgfpathlineto{\pgfqpoint{9.309814in}{5.206915in}}%
\pgfpathlineto{\pgfqpoint{9.340682in}{5.195098in}}%
\pgfpathlineto{\pgfqpoint{9.371549in}{5.185665in}}%
\pgfpathlineto{\pgfqpoint{9.386982in}{5.182401in}}%
\pgfpathlineto{\pgfqpoint{9.402416in}{5.180347in}}%
\pgfpathlineto{\pgfqpoint{9.417850in}{5.179644in}}%
\pgfpathlineto{\pgfqpoint{9.433283in}{5.180397in}}%
\pgfpathlineto{\pgfqpoint{9.448717in}{5.182674in}}%
\pgfpathlineto{\pgfqpoint{9.464151in}{5.186506in}}%
\pgfpathlineto{\pgfqpoint{9.479584in}{5.191885in}}%
\pgfpathlineto{\pgfqpoint{9.495018in}{5.198770in}}%
\pgfpathlineto{\pgfqpoint{9.510452in}{5.207085in}}%
\pgfpathlineto{\pgfqpoint{9.541319in}{5.227552in}}%
\pgfpathlineto{\pgfqpoint{9.572186in}{5.252126in}}%
\pgfpathlineto{\pgfqpoint{9.618487in}{5.293435in}}%
\pgfpathlineto{\pgfqpoint{9.664788in}{5.335151in}}%
\pgfpathlineto{\pgfqpoint{9.695655in}{5.360535in}}%
\pgfpathlineto{\pgfqpoint{9.726522in}{5.382382in}}%
\pgfpathlineto{\pgfqpoint{9.741956in}{5.391640in}}%
\pgfpathlineto{\pgfqpoint{9.741956in}{5.391640in}}%
\pgfusepath{stroke}%
\end{pgfscope}%
\begin{pgfscope}%
\pgfpathrectangle{\pgfqpoint{5.706832in}{3.881603in}}{\pgfqpoint{4.227273in}{2.800000in}} %
\pgfusepath{clip}%
\pgfsetrectcap%
\pgfsetroundjoin%
\pgfsetlinewidth{0.501875pt}%
\definecolor{currentstroke}{rgb}{0.700000,0.951057,0.587785}%
\pgfsetstrokecolor{currentstroke}%
\pgfsetdash{}{0pt}%
\pgfpathmoveto{\pgfqpoint{5.898981in}{5.534796in}}%
\pgfpathlineto{\pgfqpoint{5.914415in}{5.536158in}}%
\pgfpathlineto{\pgfqpoint{5.929848in}{5.534456in}}%
\pgfpathlineto{\pgfqpoint{5.945282in}{5.529579in}}%
\pgfpathlineto{\pgfqpoint{5.960715in}{5.521471in}}%
\pgfpathlineto{\pgfqpoint{5.976149in}{5.510136in}}%
\pgfpathlineto{\pgfqpoint{5.991583in}{5.495638in}}%
\pgfpathlineto{\pgfqpoint{6.007016in}{5.478104in}}%
\pgfpathlineto{\pgfqpoint{6.022450in}{5.457726in}}%
\pgfpathlineto{\pgfqpoint{6.037884in}{5.434753in}}%
\pgfpathlineto{\pgfqpoint{6.068751in}{5.382312in}}%
\pgfpathlineto{\pgfqpoint{6.099618in}{5.323863in}}%
\pgfpathlineto{\pgfqpoint{6.161353in}{5.204321in}}%
\pgfpathlineto{\pgfqpoint{6.192220in}{5.151653in}}%
\pgfpathlineto{\pgfqpoint{6.207654in}{5.128919in}}%
\pgfpathlineto{\pgfqpoint{6.223087in}{5.109222in}}%
\pgfpathlineto{\pgfqpoint{6.238521in}{5.092992in}}%
\pgfpathlineto{\pgfqpoint{6.253955in}{5.080610in}}%
\pgfpathlineto{\pgfqpoint{6.269388in}{5.072403in}}%
\pgfpathlineto{\pgfqpoint{6.284822in}{5.068635in}}%
\pgfpathlineto{\pgfqpoint{6.300255in}{5.069505in}}%
\pgfpathlineto{\pgfqpoint{6.315689in}{5.075139in}}%
\pgfpathlineto{\pgfqpoint{6.331123in}{5.085591in}}%
\pgfpathlineto{\pgfqpoint{6.346556in}{5.100843in}}%
\pgfpathlineto{\pgfqpoint{6.361990in}{5.120800in}}%
\pgfpathlineto{\pgfqpoint{6.377424in}{5.145296in}}%
\pgfpathlineto{\pgfqpoint{6.392857in}{5.174095in}}%
\pgfpathlineto{\pgfqpoint{6.408291in}{5.206898in}}%
\pgfpathlineto{\pgfqpoint{6.423724in}{5.243342in}}%
\pgfpathlineto{\pgfqpoint{6.454592in}{5.325452in}}%
\pgfpathlineto{\pgfqpoint{6.485459in}{5.416594in}}%
\pgfpathlineto{\pgfqpoint{6.578061in}{5.700323in}}%
\pgfpathlineto{\pgfqpoint{6.608928in}{5.783774in}}%
\pgfpathlineto{\pgfqpoint{6.624362in}{5.821225in}}%
\pgfpathlineto{\pgfqpoint{6.639795in}{5.855305in}}%
\pgfpathlineto{\pgfqpoint{6.655229in}{5.885683in}}%
\pgfpathlineto{\pgfqpoint{6.670663in}{5.912080in}}%
\pgfpathlineto{\pgfqpoint{6.686096in}{5.934274in}}%
\pgfpathlineto{\pgfqpoint{6.701530in}{5.952094in}}%
\pgfpathlineto{\pgfqpoint{6.716964in}{5.965428in}}%
\pgfpathlineto{\pgfqpoint{6.732397in}{5.974217in}}%
\pgfpathlineto{\pgfqpoint{6.747831in}{5.978455in}}%
\pgfpathlineto{\pgfqpoint{6.763264in}{5.978187in}}%
\pgfpathlineto{\pgfqpoint{6.778698in}{5.973506in}}%
\pgfpathlineto{\pgfqpoint{6.794132in}{5.964549in}}%
\pgfpathlineto{\pgfqpoint{6.809565in}{5.951493in}}%
\pgfpathlineto{\pgfqpoint{6.824999in}{5.934553in}}%
\pgfpathlineto{\pgfqpoint{6.840433in}{5.913974in}}%
\pgfpathlineto{\pgfqpoint{6.855866in}{5.890031in}}%
\pgfpathlineto{\pgfqpoint{6.871300in}{5.863022in}}%
\pgfpathlineto{\pgfqpoint{6.886734in}{5.833264in}}%
\pgfpathlineto{\pgfqpoint{6.917601in}{5.766838in}}%
\pgfpathlineto{\pgfqpoint{6.948468in}{5.693513in}}%
\pgfpathlineto{\pgfqpoint{7.010203in}{5.537383in}}%
\pgfpathlineto{\pgfqpoint{7.056504in}{5.422487in}}%
\pgfpathlineto{\pgfqpoint{7.087371in}{5.351381in}}%
\pgfpathlineto{\pgfqpoint{7.118238in}{5.286923in}}%
\pgfpathlineto{\pgfqpoint{7.149105in}{5.230584in}}%
\pgfpathlineto{\pgfqpoint{7.164539in}{5.205790in}}%
\pgfpathlineto{\pgfqpoint{7.179973in}{5.183357in}}%
\pgfpathlineto{\pgfqpoint{7.195406in}{5.163328in}}%
\pgfpathlineto{\pgfqpoint{7.210840in}{5.145709in}}%
\pgfpathlineto{\pgfqpoint{7.226274in}{5.130471in}}%
\pgfpathlineto{\pgfqpoint{7.241707in}{5.117551in}}%
\pgfpathlineto{\pgfqpoint{7.257141in}{5.106852in}}%
\pgfpathlineto{\pgfqpoint{7.272574in}{5.098243in}}%
\pgfpathlineto{\pgfqpoint{7.288008in}{5.091562in}}%
\pgfpathlineto{\pgfqpoint{7.303442in}{5.086622in}}%
\pgfpathlineto{\pgfqpoint{7.318875in}{5.083209in}}%
\pgfpathlineto{\pgfqpoint{7.349743in}{5.080012in}}%
\pgfpathlineto{\pgfqpoint{7.380610in}{5.079943in}}%
\pgfpathlineto{\pgfqpoint{7.442344in}{5.080908in}}%
\pgfpathlineto{\pgfqpoint{7.473212in}{5.078302in}}%
\pgfpathlineto{\pgfqpoint{7.504079in}{5.071925in}}%
\pgfpathlineto{\pgfqpoint{7.534946in}{5.061254in}}%
\pgfpathlineto{\pgfqpoint{7.565814in}{5.046493in}}%
\pgfpathlineto{\pgfqpoint{7.596681in}{5.028566in}}%
\pgfpathlineto{\pgfqpoint{7.673849in}{4.981420in}}%
\pgfpathlineto{\pgfqpoint{7.689283in}{4.973828in}}%
\pgfpathlineto{\pgfqpoint{7.704716in}{4.967541in}}%
\pgfpathlineto{\pgfqpoint{7.720150in}{4.962835in}}%
\pgfpathlineto{\pgfqpoint{7.735584in}{4.959961in}}%
\pgfpathlineto{\pgfqpoint{7.751017in}{4.959137in}}%
\pgfpathlineto{\pgfqpoint{7.766451in}{4.960541in}}%
\pgfpathlineto{\pgfqpoint{7.781884in}{4.964302in}}%
\pgfpathlineto{\pgfqpoint{7.797318in}{4.970500in}}%
\pgfpathlineto{\pgfqpoint{7.812752in}{4.979155in}}%
\pgfpathlineto{\pgfqpoint{7.828185in}{4.990233in}}%
\pgfpathlineto{\pgfqpoint{7.843619in}{5.003638in}}%
\pgfpathlineto{\pgfqpoint{7.859053in}{5.019218in}}%
\pgfpathlineto{\pgfqpoint{7.874486in}{5.036766in}}%
\pgfpathlineto{\pgfqpoint{7.905353in}{5.076687in}}%
\pgfpathlineto{\pgfqpoint{7.951654in}{5.143557in}}%
\pgfpathlineto{\pgfqpoint{7.997955in}{5.209458in}}%
\pgfpathlineto{\pgfqpoint{8.028823in}{5.247585in}}%
\pgfpathlineto{\pgfqpoint{8.044256in}{5.263847in}}%
\pgfpathlineto{\pgfqpoint{8.059690in}{5.277852in}}%
\pgfpathlineto{\pgfqpoint{8.075123in}{5.289369in}}%
\pgfpathlineto{\pgfqpoint{8.090557in}{5.298216in}}%
\pgfpathlineto{\pgfqpoint{8.105991in}{5.304266in}}%
\pgfpathlineto{\pgfqpoint{8.121424in}{5.307452in}}%
\pgfpathlineto{\pgfqpoint{8.136858in}{5.307759in}}%
\pgfpathlineto{\pgfqpoint{8.152292in}{5.305231in}}%
\pgfpathlineto{\pgfqpoint{8.167725in}{5.299961in}}%
\pgfpathlineto{\pgfqpoint{8.183159in}{5.292091in}}%
\pgfpathlineto{\pgfqpoint{8.198593in}{5.281803in}}%
\pgfpathlineto{\pgfqpoint{8.214026in}{5.269316in}}%
\pgfpathlineto{\pgfqpoint{8.229460in}{5.254876in}}%
\pgfpathlineto{\pgfqpoint{8.260327in}{5.221221in}}%
\pgfpathlineto{\pgfqpoint{8.291194in}{5.183080in}}%
\pgfpathlineto{\pgfqpoint{8.399230in}{5.043531in}}%
\pgfpathlineto{\pgfqpoint{8.430097in}{5.007875in}}%
\pgfpathlineto{\pgfqpoint{8.460964in}{4.975416in}}%
\pgfpathlineto{\pgfqpoint{8.491832in}{4.946265in}}%
\pgfpathlineto{\pgfqpoint{8.522699in}{4.920342in}}%
\pgfpathlineto{\pgfqpoint{8.553566in}{4.897521in}}%
\pgfpathlineto{\pgfqpoint{8.584433in}{4.877779in}}%
\pgfpathlineto{\pgfqpoint{8.615301in}{4.861296in}}%
\pgfpathlineto{\pgfqpoint{8.646168in}{4.848502in}}%
\pgfpathlineto{\pgfqpoint{8.661602in}{4.843691in}}%
\pgfpathlineto{\pgfqpoint{8.677035in}{4.840070in}}%
\pgfpathlineto{\pgfqpoint{8.692469in}{4.837749in}}%
\pgfpathlineto{\pgfqpoint{8.707903in}{4.836839in}}%
\pgfpathlineto{\pgfqpoint{8.723336in}{4.837451in}}%
\pgfpathlineto{\pgfqpoint{8.738770in}{4.839690in}}%
\pgfpathlineto{\pgfqpoint{8.754203in}{4.843647in}}%
\pgfpathlineto{\pgfqpoint{8.769637in}{4.849397in}}%
\pgfpathlineto{\pgfqpoint{8.785071in}{4.856991in}}%
\pgfpathlineto{\pgfqpoint{8.800504in}{4.866456in}}%
\pgfpathlineto{\pgfqpoint{8.815938in}{4.877788in}}%
\pgfpathlineto{\pgfqpoint{8.831372in}{4.890948in}}%
\pgfpathlineto{\pgfqpoint{8.846805in}{4.905865in}}%
\pgfpathlineto{\pgfqpoint{8.877673in}{4.940496in}}%
\pgfpathlineto{\pgfqpoint{8.908540in}{4.980387in}}%
\pgfpathlineto{\pgfqpoint{8.954841in}{5.046022in}}%
\pgfpathlineto{\pgfqpoint{9.001142in}{5.111784in}}%
\pgfpathlineto{\pgfqpoint{9.032009in}{5.151676in}}%
\pgfpathlineto{\pgfqpoint{9.062876in}{5.185821in}}%
\pgfpathlineto{\pgfqpoint{9.078310in}{5.200168in}}%
\pgfpathlineto{\pgfqpoint{9.093743in}{5.212462in}}%
\pgfpathlineto{\pgfqpoint{9.109177in}{5.222580in}}%
\pgfpathlineto{\pgfqpoint{9.124611in}{5.230443in}}%
\pgfpathlineto{\pgfqpoint{9.140044in}{5.236019in}}%
\pgfpathlineto{\pgfqpoint{9.155478in}{5.239323in}}%
\pgfpathlineto{\pgfqpoint{9.170912in}{5.240421in}}%
\pgfpathlineto{\pgfqpoint{9.186345in}{5.239423in}}%
\pgfpathlineto{\pgfqpoint{9.201779in}{5.236482in}}%
\pgfpathlineto{\pgfqpoint{9.217212in}{5.231792in}}%
\pgfpathlineto{\pgfqpoint{9.232646in}{5.225583in}}%
\pgfpathlineto{\pgfqpoint{9.263513in}{5.209665in}}%
\pgfpathlineto{\pgfqpoint{9.356115in}{5.155492in}}%
\pgfpathlineto{\pgfqpoint{9.371549in}{5.148639in}}%
\pgfpathlineto{\pgfqpoint{9.386982in}{5.143091in}}%
\pgfpathlineto{\pgfqpoint{9.402416in}{5.139038in}}%
\pgfpathlineto{\pgfqpoint{9.417850in}{5.136636in}}%
\pgfpathlineto{\pgfqpoint{9.433283in}{5.135996in}}%
\pgfpathlineto{\pgfqpoint{9.448717in}{5.137190in}}%
\pgfpathlineto{\pgfqpoint{9.464151in}{5.140248in}}%
\pgfpathlineto{\pgfqpoint{9.479584in}{5.145155in}}%
\pgfpathlineto{\pgfqpoint{9.495018in}{5.151857in}}%
\pgfpathlineto{\pgfqpoint{9.510452in}{5.160260in}}%
\pgfpathlineto{\pgfqpoint{9.525885in}{5.170236in}}%
\pgfpathlineto{\pgfqpoint{9.556752in}{5.194233in}}%
\pgfpathlineto{\pgfqpoint{9.587620in}{5.222266in}}%
\pgfpathlineto{\pgfqpoint{9.680222in}{5.311812in}}%
\pgfpathlineto{\pgfqpoint{9.711089in}{5.337454in}}%
\pgfpathlineto{\pgfqpoint{9.741956in}{5.358627in}}%
\pgfpathlineto{\pgfqpoint{9.741956in}{5.358627in}}%
\pgfusepath{stroke}%
\end{pgfscope}%
\begin{pgfscope}%
\pgfpathrectangle{\pgfqpoint{5.706832in}{3.881603in}}{\pgfqpoint{4.227273in}{2.800000in}} %
\pgfusepath{clip}%
\pgfsetrectcap%
\pgfsetroundjoin%
\pgfsetlinewidth{0.501875pt}%
\definecolor{currentstroke}{rgb}{0.778431,0.905873,0.536867}%
\pgfsetstrokecolor{currentstroke}%
\pgfsetdash{}{0pt}%
\pgfpathmoveto{\pgfqpoint{5.898981in}{5.531406in}}%
\pgfpathlineto{\pgfqpoint{5.914415in}{5.532180in}}%
\pgfpathlineto{\pgfqpoint{5.929848in}{5.529985in}}%
\pgfpathlineto{\pgfqpoint{5.945282in}{5.524733in}}%
\pgfpathlineto{\pgfqpoint{5.960715in}{5.516394in}}%
\pgfpathlineto{\pgfqpoint{5.976149in}{5.504996in}}%
\pgfpathlineto{\pgfqpoint{5.991583in}{5.490628in}}%
\pgfpathlineto{\pgfqpoint{6.007016in}{5.473439in}}%
\pgfpathlineto{\pgfqpoint{6.022450in}{5.453638in}}%
\pgfpathlineto{\pgfqpoint{6.037884in}{5.431489in}}%
\pgfpathlineto{\pgfqpoint{6.068751in}{5.381460in}}%
\pgfpathlineto{\pgfqpoint{6.115052in}{5.298118in}}%
\pgfpathlineto{\pgfqpoint{6.145919in}{5.242406in}}%
\pgfpathlineto{\pgfqpoint{6.176786in}{5.191165in}}%
\pgfpathlineto{\pgfqpoint{6.192220in}{5.168433in}}%
\pgfpathlineto{\pgfqpoint{6.207654in}{5.148235in}}%
\pgfpathlineto{\pgfqpoint{6.223087in}{5.130990in}}%
\pgfpathlineto{\pgfqpoint{6.238521in}{5.117079in}}%
\pgfpathlineto{\pgfqpoint{6.253955in}{5.106836in}}%
\pgfpathlineto{\pgfqpoint{6.269388in}{5.100543in}}%
\pgfpathlineto{\pgfqpoint{6.284822in}{5.098427in}}%
\pgfpathlineto{\pgfqpoint{6.300255in}{5.100656in}}%
\pgfpathlineto{\pgfqpoint{6.315689in}{5.107333in}}%
\pgfpathlineto{\pgfqpoint{6.331123in}{5.118497in}}%
\pgfpathlineto{\pgfqpoint{6.346556in}{5.134121in}}%
\pgfpathlineto{\pgfqpoint{6.361990in}{5.154114in}}%
\pgfpathlineto{\pgfqpoint{6.377424in}{5.178319in}}%
\pgfpathlineto{\pgfqpoint{6.392857in}{5.206521in}}%
\pgfpathlineto{\pgfqpoint{6.408291in}{5.238443in}}%
\pgfpathlineto{\pgfqpoint{6.423724in}{5.273759in}}%
\pgfpathlineto{\pgfqpoint{6.454592in}{5.353018in}}%
\pgfpathlineto{\pgfqpoint{6.485459in}{5.440808in}}%
\pgfpathlineto{\pgfqpoint{6.578061in}{5.714914in}}%
\pgfpathlineto{\pgfqpoint{6.608928in}{5.796181in}}%
\pgfpathlineto{\pgfqpoint{6.624362in}{5.832815in}}%
\pgfpathlineto{\pgfqpoint{6.639795in}{5.866262in}}%
\pgfpathlineto{\pgfqpoint{6.655229in}{5.896188in}}%
\pgfpathlineto{\pgfqpoint{6.670663in}{5.922305in}}%
\pgfpathlineto{\pgfqpoint{6.686096in}{5.944373in}}%
\pgfpathlineto{\pgfqpoint{6.701530in}{5.962207in}}%
\pgfpathlineto{\pgfqpoint{6.716964in}{5.975671in}}%
\pgfpathlineto{\pgfqpoint{6.732397in}{5.984686in}}%
\pgfpathlineto{\pgfqpoint{6.747831in}{5.989220in}}%
\pgfpathlineto{\pgfqpoint{6.763264in}{5.989297in}}%
\pgfpathlineto{\pgfqpoint{6.778698in}{5.984987in}}%
\pgfpathlineto{\pgfqpoint{6.794132in}{5.976406in}}%
\pgfpathlineto{\pgfqpoint{6.809565in}{5.963714in}}%
\pgfpathlineto{\pgfqpoint{6.824999in}{5.947108in}}%
\pgfpathlineto{\pgfqpoint{6.840433in}{5.926824in}}%
\pgfpathlineto{\pgfqpoint{6.855866in}{5.903125in}}%
\pgfpathlineto{\pgfqpoint{6.871300in}{5.876303in}}%
\pgfpathlineto{\pgfqpoint{6.886734in}{5.846671in}}%
\pgfpathlineto{\pgfqpoint{6.917601in}{5.780318in}}%
\pgfpathlineto{\pgfqpoint{6.948468in}{5.706845in}}%
\pgfpathlineto{\pgfqpoint{7.010203in}{5.549931in}}%
\pgfpathlineto{\pgfqpoint{7.056504in}{5.434194in}}%
\pgfpathlineto{\pgfqpoint{7.087371in}{5.362453in}}%
\pgfpathlineto{\pgfqpoint{7.118238in}{5.297299in}}%
\pgfpathlineto{\pgfqpoint{7.149105in}{5.240186in}}%
\pgfpathlineto{\pgfqpoint{7.164539in}{5.214966in}}%
\pgfpathlineto{\pgfqpoint{7.179973in}{5.192079in}}%
\pgfpathlineto{\pgfqpoint{7.195406in}{5.171562in}}%
\pgfpathlineto{\pgfqpoint{7.210840in}{5.153420in}}%
\pgfpathlineto{\pgfqpoint{7.226274in}{5.137622in}}%
\pgfpathlineto{\pgfqpoint{7.241707in}{5.124105in}}%
\pgfpathlineto{\pgfqpoint{7.257141in}{5.112770in}}%
\pgfpathlineto{\pgfqpoint{7.272574in}{5.103489in}}%
\pgfpathlineto{\pgfqpoint{7.288008in}{5.096102in}}%
\pgfpathlineto{\pgfqpoint{7.303442in}{5.090423in}}%
\pgfpathlineto{\pgfqpoint{7.318875in}{5.086240in}}%
\pgfpathlineto{\pgfqpoint{7.349743in}{5.081428in}}%
\pgfpathlineto{\pgfqpoint{7.380610in}{5.079660in}}%
\pgfpathlineto{\pgfqpoint{7.442344in}{5.077016in}}%
\pgfpathlineto{\pgfqpoint{7.473212in}{5.072485in}}%
\pgfpathlineto{\pgfqpoint{7.504079in}{5.064070in}}%
\pgfpathlineto{\pgfqpoint{7.534946in}{5.051221in}}%
\pgfpathlineto{\pgfqpoint{7.565814in}{5.034116in}}%
\pgfpathlineto{\pgfqpoint{7.596681in}{5.013683in}}%
\pgfpathlineto{\pgfqpoint{7.673849in}{4.959910in}}%
\pgfpathlineto{\pgfqpoint{7.689283in}{4.951041in}}%
\pgfpathlineto{\pgfqpoint{7.704716in}{4.943540in}}%
\pgfpathlineto{\pgfqpoint{7.720150in}{4.937707in}}%
\pgfpathlineto{\pgfqpoint{7.735584in}{4.933819in}}%
\pgfpathlineto{\pgfqpoint{7.751017in}{4.932122in}}%
\pgfpathlineto{\pgfqpoint{7.766451in}{4.932817in}}%
\pgfpathlineto{\pgfqpoint{7.781884in}{4.936062in}}%
\pgfpathlineto{\pgfqpoint{7.797318in}{4.941957in}}%
\pgfpathlineto{\pgfqpoint{7.812752in}{4.950546in}}%
\pgfpathlineto{\pgfqpoint{7.828185in}{4.961809in}}%
\pgfpathlineto{\pgfqpoint{7.843619in}{4.975667in}}%
\pgfpathlineto{\pgfqpoint{7.859053in}{4.991975in}}%
\pgfpathlineto{\pgfqpoint{7.874486in}{5.010529in}}%
\pgfpathlineto{\pgfqpoint{7.905353in}{5.053283in}}%
\pgfpathlineto{\pgfqpoint{7.936221in}{5.101275in}}%
\pgfpathlineto{\pgfqpoint{7.997955in}{5.199799in}}%
\pgfpathlineto{\pgfqpoint{8.028823in}{5.243394in}}%
\pgfpathlineto{\pgfqpoint{8.044256in}{5.262366in}}%
\pgfpathlineto{\pgfqpoint{8.059690in}{5.279003in}}%
\pgfpathlineto{\pgfqpoint{8.075123in}{5.293029in}}%
\pgfpathlineto{\pgfqpoint{8.090557in}{5.304216in}}%
\pgfpathlineto{\pgfqpoint{8.105991in}{5.312400in}}%
\pgfpathlineto{\pgfqpoint{8.121424in}{5.317471in}}%
\pgfpathlineto{\pgfqpoint{8.136858in}{5.319384in}}%
\pgfpathlineto{\pgfqpoint{8.152292in}{5.318153in}}%
\pgfpathlineto{\pgfqpoint{8.167725in}{5.313849in}}%
\pgfpathlineto{\pgfqpoint{8.183159in}{5.306595in}}%
\pgfpathlineto{\pgfqpoint{8.198593in}{5.296564in}}%
\pgfpathlineto{\pgfqpoint{8.214026in}{5.283971in}}%
\pgfpathlineto{\pgfqpoint{8.229460in}{5.269063in}}%
\pgfpathlineto{\pgfqpoint{8.244893in}{5.252116in}}%
\pgfpathlineto{\pgfqpoint{8.275761in}{5.213288in}}%
\pgfpathlineto{\pgfqpoint{8.306628in}{5.169919in}}%
\pgfpathlineto{\pgfqpoint{8.399230in}{5.034662in}}%
\pgfpathlineto{\pgfqpoint{8.430097in}{4.993416in}}%
\pgfpathlineto{\pgfqpoint{8.460964in}{4.955752in}}%
\pgfpathlineto{\pgfqpoint{8.491832in}{4.922070in}}%
\pgfpathlineto{\pgfqpoint{8.522699in}{4.892514in}}%
\pgfpathlineto{\pgfqpoint{8.553566in}{4.867110in}}%
\pgfpathlineto{\pgfqpoint{8.584433in}{4.845894in}}%
\pgfpathlineto{\pgfqpoint{8.615301in}{4.829010in}}%
\pgfpathlineto{\pgfqpoint{8.646168in}{4.816756in}}%
\pgfpathlineto{\pgfqpoint{8.661602in}{4.812501in}}%
\pgfpathlineto{\pgfqpoint{8.677035in}{4.809583in}}%
\pgfpathlineto{\pgfqpoint{8.692469in}{4.808072in}}%
\pgfpathlineto{\pgfqpoint{8.707903in}{4.808039in}}%
\pgfpathlineto{\pgfqpoint{8.723336in}{4.809551in}}%
\pgfpathlineto{\pgfqpoint{8.738770in}{4.812667in}}%
\pgfpathlineto{\pgfqpoint{8.754203in}{4.817433in}}%
\pgfpathlineto{\pgfqpoint{8.769637in}{4.823882in}}%
\pgfpathlineto{\pgfqpoint{8.785071in}{4.832024in}}%
\pgfpathlineto{\pgfqpoint{8.800504in}{4.841849in}}%
\pgfpathlineto{\pgfqpoint{8.815938in}{4.853319in}}%
\pgfpathlineto{\pgfqpoint{8.831372in}{4.866371in}}%
\pgfpathlineto{\pgfqpoint{8.862239in}{4.896821in}}%
\pgfpathlineto{\pgfqpoint{8.893106in}{4.932124in}}%
\pgfpathlineto{\pgfqpoint{8.939407in}{4.990840in}}%
\pgfpathlineto{\pgfqpoint{9.001142in}{5.069983in}}%
\pgfpathlineto{\pgfqpoint{9.032009in}{5.105494in}}%
\pgfpathlineto{\pgfqpoint{9.062876in}{5.135980in}}%
\pgfpathlineto{\pgfqpoint{9.078310in}{5.148923in}}%
\pgfpathlineto{\pgfqpoint{9.093743in}{5.160175in}}%
\pgfpathlineto{\pgfqpoint{9.109177in}{5.169658in}}%
\pgfpathlineto{\pgfqpoint{9.124611in}{5.177337in}}%
\pgfpathlineto{\pgfqpoint{9.140044in}{5.183211in}}%
\pgfpathlineto{\pgfqpoint{9.155478in}{5.187320in}}%
\pgfpathlineto{\pgfqpoint{9.170912in}{5.189738in}}%
\pgfpathlineto{\pgfqpoint{9.186345in}{5.190576in}}%
\pgfpathlineto{\pgfqpoint{9.201779in}{5.189976in}}%
\pgfpathlineto{\pgfqpoint{9.232646in}{5.185175in}}%
\pgfpathlineto{\pgfqpoint{9.263513in}{5.176966in}}%
\pgfpathlineto{\pgfqpoint{9.340682in}{5.153880in}}%
\pgfpathlineto{\pgfqpoint{9.371549in}{5.148392in}}%
\pgfpathlineto{\pgfqpoint{9.386982in}{5.147211in}}%
\pgfpathlineto{\pgfqpoint{9.402416in}{5.147250in}}%
\pgfpathlineto{\pgfqpoint{9.417850in}{5.148609in}}%
\pgfpathlineto{\pgfqpoint{9.433283in}{5.151350in}}%
\pgfpathlineto{\pgfqpoint{9.448717in}{5.155505in}}%
\pgfpathlineto{\pgfqpoint{9.464151in}{5.161070in}}%
\pgfpathlineto{\pgfqpoint{9.479584in}{5.168010in}}%
\pgfpathlineto{\pgfqpoint{9.510452in}{5.185711in}}%
\pgfpathlineto{\pgfqpoint{9.541319in}{5.207725in}}%
\pgfpathlineto{\pgfqpoint{9.572186in}{5.232793in}}%
\pgfpathlineto{\pgfqpoint{9.664788in}{5.310849in}}%
\pgfpathlineto{\pgfqpoint{9.695655in}{5.332709in}}%
\pgfpathlineto{\pgfqpoint{9.726522in}{5.350435in}}%
\pgfpathlineto{\pgfqpoint{9.741956in}{5.357508in}}%
\pgfpathlineto{\pgfqpoint{9.741956in}{5.357508in}}%
\pgfusepath{stroke}%
\end{pgfscope}%
\begin{pgfscope}%
\pgfpathrectangle{\pgfqpoint{5.706832in}{3.881603in}}{\pgfqpoint{4.227273in}{2.800000in}} %
\pgfusepath{clip}%
\pgfsetrectcap%
\pgfsetroundjoin%
\pgfsetlinewidth{0.501875pt}%
\definecolor{currentstroke}{rgb}{0.864706,0.840344,0.478512}%
\pgfsetstrokecolor{currentstroke}%
\pgfsetdash{}{0pt}%
\pgfpathmoveto{\pgfqpoint{5.898981in}{5.544844in}}%
\pgfpathlineto{\pgfqpoint{5.914415in}{5.545306in}}%
\pgfpathlineto{\pgfqpoint{5.929848in}{5.542942in}}%
\pgfpathlineto{\pgfqpoint{5.945282in}{5.537651in}}%
\pgfpathlineto{\pgfqpoint{5.960715in}{5.529380in}}%
\pgfpathlineto{\pgfqpoint{5.976149in}{5.518138in}}%
\pgfpathlineto{\pgfqpoint{5.991583in}{5.503989in}}%
\pgfpathlineto{\pgfqpoint{6.007016in}{5.487058in}}%
\pgfpathlineto{\pgfqpoint{6.022450in}{5.467526in}}%
\pgfpathlineto{\pgfqpoint{6.037884in}{5.445631in}}%
\pgfpathlineto{\pgfqpoint{6.068751in}{5.395960in}}%
\pgfpathlineto{\pgfqpoint{6.099618in}{5.340918in}}%
\pgfpathlineto{\pgfqpoint{6.161353in}{5.228919in}}%
\pgfpathlineto{\pgfqpoint{6.192220in}{5.179713in}}%
\pgfpathlineto{\pgfqpoint{6.207654in}{5.158491in}}%
\pgfpathlineto{\pgfqpoint{6.223087in}{5.140115in}}%
\pgfpathlineto{\pgfqpoint{6.238521in}{5.124987in}}%
\pgfpathlineto{\pgfqpoint{6.253955in}{5.113468in}}%
\pgfpathlineto{\pgfqpoint{6.269388in}{5.105872in}}%
\pgfpathlineto{\pgfqpoint{6.284822in}{5.102456in}}%
\pgfpathlineto{\pgfqpoint{6.300255in}{5.103419in}}%
\pgfpathlineto{\pgfqpoint{6.315689in}{5.108897in}}%
\pgfpathlineto{\pgfqpoint{6.331123in}{5.118959in}}%
\pgfpathlineto{\pgfqpoint{6.346556in}{5.133605in}}%
\pgfpathlineto{\pgfqpoint{6.361990in}{5.152766in}}%
\pgfpathlineto{\pgfqpoint{6.377424in}{5.176303in}}%
\pgfpathlineto{\pgfqpoint{6.392857in}{5.204010in}}%
\pgfpathlineto{\pgfqpoint{6.408291in}{5.235616in}}%
\pgfpathlineto{\pgfqpoint{6.423724in}{5.270791in}}%
\pgfpathlineto{\pgfqpoint{6.454592in}{5.350246in}}%
\pgfpathlineto{\pgfqpoint{6.485459in}{5.438728in}}%
\pgfpathlineto{\pgfqpoint{6.578061in}{5.715349in}}%
\pgfpathlineto{\pgfqpoint{6.608928in}{5.796773in}}%
\pgfpathlineto{\pgfqpoint{6.624362in}{5.833271in}}%
\pgfpathlineto{\pgfqpoint{6.639795in}{5.866438in}}%
\pgfpathlineto{\pgfqpoint{6.655229in}{5.895950in}}%
\pgfpathlineto{\pgfqpoint{6.670663in}{5.921537in}}%
\pgfpathlineto{\pgfqpoint{6.686096in}{5.942987in}}%
\pgfpathlineto{\pgfqpoint{6.701530in}{5.960148in}}%
\pgfpathlineto{\pgfqpoint{6.716964in}{5.972925in}}%
\pgfpathlineto{\pgfqpoint{6.732397in}{5.981279in}}%
\pgfpathlineto{\pgfqpoint{6.747831in}{5.985228in}}%
\pgfpathlineto{\pgfqpoint{6.763264in}{5.984838in}}%
\pgfpathlineto{\pgfqpoint{6.778698in}{5.980223in}}%
\pgfpathlineto{\pgfqpoint{6.794132in}{5.971540in}}%
\pgfpathlineto{\pgfqpoint{6.809565in}{5.958979in}}%
\pgfpathlineto{\pgfqpoint{6.824999in}{5.942764in}}%
\pgfpathlineto{\pgfqpoint{6.840433in}{5.923145in}}%
\pgfpathlineto{\pgfqpoint{6.855866in}{5.900391in}}%
\pgfpathlineto{\pgfqpoint{6.871300in}{5.874788in}}%
\pgfpathlineto{\pgfqpoint{6.902167in}{5.816223in}}%
\pgfpathlineto{\pgfqpoint{6.933034in}{5.749876in}}%
\pgfpathlineto{\pgfqpoint{6.963902in}{5.678180in}}%
\pgfpathlineto{\pgfqpoint{7.087371in}{5.383321in}}%
\pgfpathlineto{\pgfqpoint{7.118238in}{5.317699in}}%
\pgfpathlineto{\pgfqpoint{7.149105in}{5.258691in}}%
\pgfpathlineto{\pgfqpoint{7.179973in}{5.207515in}}%
\pgfpathlineto{\pgfqpoint{7.195406in}{5.185142in}}%
\pgfpathlineto{\pgfqpoint{7.210840in}{5.165005in}}%
\pgfpathlineto{\pgfqpoint{7.226274in}{5.147137in}}%
\pgfpathlineto{\pgfqpoint{7.241707in}{5.131534in}}%
\pgfpathlineto{\pgfqpoint{7.257141in}{5.118161in}}%
\pgfpathlineto{\pgfqpoint{7.272574in}{5.106945in}}%
\pgfpathlineto{\pgfqpoint{7.288008in}{5.097779in}}%
\pgfpathlineto{\pgfqpoint{7.303442in}{5.090524in}}%
\pgfpathlineto{\pgfqpoint{7.318875in}{5.085004in}}%
\pgfpathlineto{\pgfqpoint{7.334309in}{5.081018in}}%
\pgfpathlineto{\pgfqpoint{7.365176in}{5.076711in}}%
\pgfpathlineto{\pgfqpoint{7.396044in}{5.075567in}}%
\pgfpathlineto{\pgfqpoint{7.442344in}{5.075051in}}%
\pgfpathlineto{\pgfqpoint{7.473212in}{5.072588in}}%
\pgfpathlineto{\pgfqpoint{7.504079in}{5.066611in}}%
\pgfpathlineto{\pgfqpoint{7.534946in}{5.056257in}}%
\pgfpathlineto{\pgfqpoint{7.565814in}{5.041431in}}%
\pgfpathlineto{\pgfqpoint{7.596681in}{5.022851in}}%
\pgfpathlineto{\pgfqpoint{7.689283in}{4.962313in}}%
\pgfpathlineto{\pgfqpoint{7.704716in}{4.954671in}}%
\pgfpathlineto{\pgfqpoint{7.720150in}{4.948581in}}%
\pgfpathlineto{\pgfqpoint{7.735584in}{4.944334in}}%
\pgfpathlineto{\pgfqpoint{7.751017in}{4.942187in}}%
\pgfpathlineto{\pgfqpoint{7.766451in}{4.942358in}}%
\pgfpathlineto{\pgfqpoint{7.781884in}{4.945014in}}%
\pgfpathlineto{\pgfqpoint{7.797318in}{4.950267in}}%
\pgfpathlineto{\pgfqpoint{7.812752in}{4.958170in}}%
\pgfpathlineto{\pgfqpoint{7.828185in}{4.968710in}}%
\pgfpathlineto{\pgfqpoint{7.843619in}{4.981812in}}%
\pgfpathlineto{\pgfqpoint{7.859053in}{4.997335in}}%
\pgfpathlineto{\pgfqpoint{7.874486in}{5.015073in}}%
\pgfpathlineto{\pgfqpoint{7.905353in}{5.056092in}}%
\pgfpathlineto{\pgfqpoint{7.936221in}{5.102172in}}%
\pgfpathlineto{\pgfqpoint{7.997955in}{5.196079in}}%
\pgfpathlineto{\pgfqpoint{8.028823in}{5.236843in}}%
\pgfpathlineto{\pgfqpoint{8.044256in}{5.254252in}}%
\pgfpathlineto{\pgfqpoint{8.059690in}{5.269229in}}%
\pgfpathlineto{\pgfqpoint{8.075123in}{5.281501in}}%
\pgfpathlineto{\pgfqpoint{8.090557in}{5.290854in}}%
\pgfpathlineto{\pgfqpoint{8.105991in}{5.297132in}}%
\pgfpathlineto{\pgfqpoint{8.121424in}{5.300248in}}%
\pgfpathlineto{\pgfqpoint{8.136858in}{5.300178in}}%
\pgfpathlineto{\pgfqpoint{8.152292in}{5.296963in}}%
\pgfpathlineto{\pgfqpoint{8.167725in}{5.290705in}}%
\pgfpathlineto{\pgfqpoint{8.183159in}{5.281563in}}%
\pgfpathlineto{\pgfqpoint{8.198593in}{5.269744in}}%
\pgfpathlineto{\pgfqpoint{8.214026in}{5.255503in}}%
\pgfpathlineto{\pgfqpoint{8.229460in}{5.239125in}}%
\pgfpathlineto{\pgfqpoint{8.260327in}{5.201226in}}%
\pgfpathlineto{\pgfqpoint{8.291194in}{5.158690in}}%
\pgfpathlineto{\pgfqpoint{8.368363in}{5.048388in}}%
\pgfpathlineto{\pgfqpoint{8.399230in}{5.007752in}}%
\pgfpathlineto{\pgfqpoint{8.430097in}{4.970829in}}%
\pgfpathlineto{\pgfqpoint{8.460964in}{4.937974in}}%
\pgfpathlineto{\pgfqpoint{8.491832in}{4.909093in}}%
\pgfpathlineto{\pgfqpoint{8.522699in}{4.883830in}}%
\pgfpathlineto{\pgfqpoint{8.553566in}{4.861769in}}%
\pgfpathlineto{\pgfqpoint{8.584433in}{4.842626in}}%
\pgfpathlineto{\pgfqpoint{8.615301in}{4.826394in}}%
\pgfpathlineto{\pgfqpoint{8.646168in}{4.813418in}}%
\pgfpathlineto{\pgfqpoint{8.677035in}{4.804380in}}%
\pgfpathlineto{\pgfqpoint{8.692469in}{4.801625in}}%
\pgfpathlineto{\pgfqpoint{8.707903in}{4.800215in}}%
\pgfpathlineto{\pgfqpoint{8.723336in}{4.800280in}}%
\pgfpathlineto{\pgfqpoint{8.738770in}{4.801942in}}%
\pgfpathlineto{\pgfqpoint{8.754203in}{4.805311in}}%
\pgfpathlineto{\pgfqpoint{8.769637in}{4.810476in}}%
\pgfpathlineto{\pgfqpoint{8.785071in}{4.817501in}}%
\pgfpathlineto{\pgfqpoint{8.800504in}{4.826420in}}%
\pgfpathlineto{\pgfqpoint{8.815938in}{4.837230in}}%
\pgfpathlineto{\pgfqpoint{8.831372in}{4.849892in}}%
\pgfpathlineto{\pgfqpoint{8.846805in}{4.864327in}}%
\pgfpathlineto{\pgfqpoint{8.877673in}{4.897997in}}%
\pgfpathlineto{\pgfqpoint{8.908540in}{4.936833in}}%
\pgfpathlineto{\pgfqpoint{8.954841in}{5.000474in}}%
\pgfpathlineto{\pgfqpoint{9.001142in}{5.063669in}}%
\pgfpathlineto{\pgfqpoint{9.032009in}{5.101734in}}%
\pgfpathlineto{\pgfqpoint{9.062876in}{5.134307in}}%
\pgfpathlineto{\pgfqpoint{9.078310in}{5.148098in}}%
\pgfpathlineto{\pgfqpoint{9.093743in}{5.160075in}}%
\pgfpathlineto{\pgfqpoint{9.109177in}{5.170179in}}%
\pgfpathlineto{\pgfqpoint{9.124611in}{5.178397in}}%
\pgfpathlineto{\pgfqpoint{9.140044in}{5.184764in}}%
\pgfpathlineto{\pgfqpoint{9.155478in}{5.189353in}}%
\pgfpathlineto{\pgfqpoint{9.170912in}{5.192279in}}%
\pgfpathlineto{\pgfqpoint{9.186345in}{5.193693in}}%
\pgfpathlineto{\pgfqpoint{9.201779in}{5.193774in}}%
\pgfpathlineto{\pgfqpoint{9.232646in}{5.190781in}}%
\pgfpathlineto{\pgfqpoint{9.278947in}{5.181895in}}%
\pgfpathlineto{\pgfqpoint{9.325248in}{5.173353in}}%
\pgfpathlineto{\pgfqpoint{9.356115in}{5.170564in}}%
\pgfpathlineto{\pgfqpoint{9.386982in}{5.171462in}}%
\pgfpathlineto{\pgfqpoint{9.417850in}{5.176679in}}%
\pgfpathlineto{\pgfqpoint{9.448717in}{5.186348in}}%
\pgfpathlineto{\pgfqpoint{9.479584in}{5.200125in}}%
\pgfpathlineto{\pgfqpoint{9.510452in}{5.217261in}}%
\pgfpathlineto{\pgfqpoint{9.556752in}{5.246896in}}%
\pgfpathlineto{\pgfqpoint{9.618487in}{5.287046in}}%
\pgfpathlineto{\pgfqpoint{9.649354in}{5.304573in}}%
\pgfpathlineto{\pgfqpoint{9.680222in}{5.319024in}}%
\pgfpathlineto{\pgfqpoint{9.711089in}{5.329706in}}%
\pgfpathlineto{\pgfqpoint{9.741956in}{5.336270in}}%
\pgfpathlineto{\pgfqpoint{9.741956in}{5.336270in}}%
\pgfusepath{stroke}%
\end{pgfscope}%
\begin{pgfscope}%
\pgfpathrectangle{\pgfqpoint{5.706832in}{3.881603in}}{\pgfqpoint{4.227273in}{2.800000in}} %
\pgfusepath{clip}%
\pgfsetrectcap%
\pgfsetroundjoin%
\pgfsetlinewidth{0.501875pt}%
\definecolor{currentstroke}{rgb}{0.943137,0.767363,0.423549}%
\pgfsetstrokecolor{currentstroke}%
\pgfsetdash{}{0pt}%
\pgfpathmoveto{\pgfqpoint{5.898981in}{5.525401in}}%
\pgfpathlineto{\pgfqpoint{5.914415in}{5.524288in}}%
\pgfpathlineto{\pgfqpoint{5.929848in}{5.520573in}}%
\pgfpathlineto{\pgfqpoint{5.945282in}{5.514171in}}%
\pgfpathlineto{\pgfqpoint{5.960715in}{5.505046in}}%
\pgfpathlineto{\pgfqpoint{5.976149in}{5.493213in}}%
\pgfpathlineto{\pgfqpoint{5.991583in}{5.478740in}}%
\pgfpathlineto{\pgfqpoint{6.007016in}{5.461747in}}%
\pgfpathlineto{\pgfqpoint{6.022450in}{5.442408in}}%
\pgfpathlineto{\pgfqpoint{6.053317in}{5.397639in}}%
\pgfpathlineto{\pgfqpoint{6.084185in}{5.346789in}}%
\pgfpathlineto{\pgfqpoint{6.161353in}{5.213811in}}%
\pgfpathlineto{\pgfqpoint{6.192220in}{5.167757in}}%
\pgfpathlineto{\pgfqpoint{6.207654in}{5.148091in}}%
\pgfpathlineto{\pgfqpoint{6.223087in}{5.131232in}}%
\pgfpathlineto{\pgfqpoint{6.238521in}{5.117565in}}%
\pgfpathlineto{\pgfqpoint{6.253955in}{5.107438in}}%
\pgfpathlineto{\pgfqpoint{6.269388in}{5.101152in}}%
\pgfpathlineto{\pgfqpoint{6.284822in}{5.098956in}}%
\pgfpathlineto{\pgfqpoint{6.300255in}{5.101044in}}%
\pgfpathlineto{\pgfqpoint{6.315689in}{5.107545in}}%
\pgfpathlineto{\pgfqpoint{6.331123in}{5.118526in}}%
\pgfpathlineto{\pgfqpoint{6.346556in}{5.133986in}}%
\pgfpathlineto{\pgfqpoint{6.361990in}{5.153857in}}%
\pgfpathlineto{\pgfqpoint{6.377424in}{5.178002in}}%
\pgfpathlineto{\pgfqpoint{6.392857in}{5.206219in}}%
\pgfpathlineto{\pgfqpoint{6.408291in}{5.238241in}}%
\pgfpathlineto{\pgfqpoint{6.423724in}{5.273743in}}%
\pgfpathlineto{\pgfqpoint{6.454592in}{5.353613in}}%
\pgfpathlineto{\pgfqpoint{6.485459in}{5.442233in}}%
\pgfpathlineto{\pgfqpoint{6.578061in}{5.718196in}}%
\pgfpathlineto{\pgfqpoint{6.608928in}{5.799229in}}%
\pgfpathlineto{\pgfqpoint{6.624362in}{5.835525in}}%
\pgfpathlineto{\pgfqpoint{6.639795in}{5.868494in}}%
\pgfpathlineto{\pgfqpoint{6.655229in}{5.897814in}}%
\pgfpathlineto{\pgfqpoint{6.670663in}{5.923220in}}%
\pgfpathlineto{\pgfqpoint{6.686096in}{5.944501in}}%
\pgfpathlineto{\pgfqpoint{6.701530in}{5.961507in}}%
\pgfpathlineto{\pgfqpoint{6.716964in}{5.974144in}}%
\pgfpathlineto{\pgfqpoint{6.732397in}{5.982375in}}%
\pgfpathlineto{\pgfqpoint{6.747831in}{5.986218in}}%
\pgfpathlineto{\pgfqpoint{6.763264in}{5.985738in}}%
\pgfpathlineto{\pgfqpoint{6.778698in}{5.981050in}}%
\pgfpathlineto{\pgfqpoint{6.794132in}{5.972308in}}%
\pgfpathlineto{\pgfqpoint{6.809565in}{5.959703in}}%
\pgfpathlineto{\pgfqpoint{6.824999in}{5.943456in}}%
\pgfpathlineto{\pgfqpoint{6.840433in}{5.923816in}}%
\pgfpathlineto{\pgfqpoint{6.855866in}{5.901050in}}%
\pgfpathlineto{\pgfqpoint{6.871300in}{5.875441in}}%
\pgfpathlineto{\pgfqpoint{6.902167in}{5.816875in}}%
\pgfpathlineto{\pgfqpoint{6.933034in}{5.750522in}}%
\pgfpathlineto{\pgfqpoint{6.963902in}{5.678786in}}%
\pgfpathlineto{\pgfqpoint{7.087371in}{5.382814in}}%
\pgfpathlineto{\pgfqpoint{7.118238in}{5.316536in}}%
\pgfpathlineto{\pgfqpoint{7.149105in}{5.256701in}}%
\pgfpathlineto{\pgfqpoint{7.179973in}{5.204553in}}%
\pgfpathlineto{\pgfqpoint{7.195406in}{5.181655in}}%
\pgfpathlineto{\pgfqpoint{7.210840in}{5.160977in}}%
\pgfpathlineto{\pgfqpoint{7.226274in}{5.142562in}}%
\pgfpathlineto{\pgfqpoint{7.241707in}{5.126417in}}%
\pgfpathlineto{\pgfqpoint{7.257141in}{5.112520in}}%
\pgfpathlineto{\pgfqpoint{7.272574in}{5.100810in}}%
\pgfpathlineto{\pgfqpoint{7.288008in}{5.091192in}}%
\pgfpathlineto{\pgfqpoint{7.303442in}{5.083538in}}%
\pgfpathlineto{\pgfqpoint{7.318875in}{5.077685in}}%
\pgfpathlineto{\pgfqpoint{7.334309in}{5.073440in}}%
\pgfpathlineto{\pgfqpoint{7.349743in}{5.070584in}}%
\pgfpathlineto{\pgfqpoint{7.380610in}{5.068049in}}%
\pgfpathlineto{\pgfqpoint{7.473212in}{5.066598in}}%
\pgfpathlineto{\pgfqpoint{7.504079in}{5.061690in}}%
\pgfpathlineto{\pgfqpoint{7.519513in}{5.057637in}}%
\pgfpathlineto{\pgfqpoint{7.550380in}{5.046056in}}%
\pgfpathlineto{\pgfqpoint{7.581247in}{5.030071in}}%
\pgfpathlineto{\pgfqpoint{7.612114in}{5.010682in}}%
\pgfpathlineto{\pgfqpoint{7.673849in}{4.969226in}}%
\pgfpathlineto{\pgfqpoint{7.704716in}{4.952093in}}%
\pgfpathlineto{\pgfqpoint{7.720150in}{4.945581in}}%
\pgfpathlineto{\pgfqpoint{7.735584in}{4.940851in}}%
\pgfpathlineto{\pgfqpoint{7.751017in}{4.938178in}}%
\pgfpathlineto{\pgfqpoint{7.766451in}{4.937796in}}%
\pgfpathlineto{\pgfqpoint{7.781884in}{4.939891in}}%
\pgfpathlineto{\pgfqpoint{7.797318in}{4.944595in}}%
\pgfpathlineto{\pgfqpoint{7.812752in}{4.951979in}}%
\pgfpathlineto{\pgfqpoint{7.828185in}{4.962049in}}%
\pgfpathlineto{\pgfqpoint{7.843619in}{4.974746in}}%
\pgfpathlineto{\pgfqpoint{7.859053in}{4.989941in}}%
\pgfpathlineto{\pgfqpoint{7.874486in}{5.007446in}}%
\pgfpathlineto{\pgfqpoint{7.905353in}{5.048313in}}%
\pgfpathlineto{\pgfqpoint{7.936221in}{5.094704in}}%
\pgfpathlineto{\pgfqpoint{7.997955in}{5.190643in}}%
\pgfpathlineto{\pgfqpoint{8.028823in}{5.233057in}}%
\pgfpathlineto{\pgfqpoint{8.044256in}{5.251422in}}%
\pgfpathlineto{\pgfqpoint{8.059690in}{5.267430in}}%
\pgfpathlineto{\pgfqpoint{8.075123in}{5.280798in}}%
\pgfpathlineto{\pgfqpoint{8.090557in}{5.291299in}}%
\pgfpathlineto{\pgfqpoint{8.105991in}{5.298768in}}%
\pgfpathlineto{\pgfqpoint{8.121424in}{5.303107in}}%
\pgfpathlineto{\pgfqpoint{8.136858in}{5.304278in}}%
\pgfpathlineto{\pgfqpoint{8.152292in}{5.302313in}}%
\pgfpathlineto{\pgfqpoint{8.167725in}{5.297301in}}%
\pgfpathlineto{\pgfqpoint{8.183159in}{5.289389in}}%
\pgfpathlineto{\pgfqpoint{8.198593in}{5.278773in}}%
\pgfpathlineto{\pgfqpoint{8.214026in}{5.265693in}}%
\pgfpathlineto{\pgfqpoint{8.229460in}{5.250425in}}%
\pgfpathlineto{\pgfqpoint{8.260327in}{5.214539in}}%
\pgfpathlineto{\pgfqpoint{8.291194in}{5.173656in}}%
\pgfpathlineto{\pgfqpoint{8.399230in}{5.024524in}}%
\pgfpathlineto{\pgfqpoint{8.430097in}{4.986803in}}%
\pgfpathlineto{\pgfqpoint{8.460964in}{4.952585in}}%
\pgfpathlineto{\pgfqpoint{8.491832in}{4.921847in}}%
\pgfpathlineto{\pgfqpoint{8.522699in}{4.894332in}}%
\pgfpathlineto{\pgfqpoint{8.553566in}{4.869741in}}%
\pgfpathlineto{\pgfqpoint{8.584433in}{4.847906in}}%
\pgfpathlineto{\pgfqpoint{8.615301in}{4.828928in}}%
\pgfpathlineto{\pgfqpoint{8.646168in}{4.813232in}}%
\pgfpathlineto{\pgfqpoint{8.677035in}{4.801550in}}%
\pgfpathlineto{\pgfqpoint{8.692469in}{4.797506in}}%
\pgfpathlineto{\pgfqpoint{8.707903in}{4.794828in}}%
\pgfpathlineto{\pgfqpoint{8.723336in}{4.793644in}}%
\pgfpathlineto{\pgfqpoint{8.738770in}{4.794071in}}%
\pgfpathlineto{\pgfqpoint{8.754203in}{4.796215in}}%
\pgfpathlineto{\pgfqpoint{8.769637in}{4.800160in}}%
\pgfpathlineto{\pgfqpoint{8.785071in}{4.805965in}}%
\pgfpathlineto{\pgfqpoint{8.800504in}{4.813660in}}%
\pgfpathlineto{\pgfqpoint{8.815938in}{4.823241in}}%
\pgfpathlineto{\pgfqpoint{8.831372in}{4.834669in}}%
\pgfpathlineto{\pgfqpoint{8.846805in}{4.847865in}}%
\pgfpathlineto{\pgfqpoint{8.877673in}{4.879070in}}%
\pgfpathlineto{\pgfqpoint{8.908540in}{4.915519in}}%
\pgfpathlineto{\pgfqpoint{8.954841in}{4.975993in}}%
\pgfpathlineto{\pgfqpoint{9.001142in}{5.037008in}}%
\pgfpathlineto{\pgfqpoint{9.032009in}{5.074473in}}%
\pgfpathlineto{\pgfqpoint{9.062876in}{5.107331in}}%
\pgfpathlineto{\pgfqpoint{9.078310in}{5.121640in}}%
\pgfpathlineto{\pgfqpoint{9.093743in}{5.134402in}}%
\pgfpathlineto{\pgfqpoint{9.109177in}{5.145568in}}%
\pgfpathlineto{\pgfqpoint{9.124611in}{5.155128in}}%
\pgfpathlineto{\pgfqpoint{9.140044in}{5.163116in}}%
\pgfpathlineto{\pgfqpoint{9.155478in}{5.169601in}}%
\pgfpathlineto{\pgfqpoint{9.170912in}{5.174687in}}%
\pgfpathlineto{\pgfqpoint{9.201779in}{5.181231in}}%
\pgfpathlineto{\pgfqpoint{9.232646in}{5.184101in}}%
\pgfpathlineto{\pgfqpoint{9.278947in}{5.184990in}}%
\pgfpathlineto{\pgfqpoint{9.325248in}{5.186460in}}%
\pgfpathlineto{\pgfqpoint{9.356115in}{5.189897in}}%
\pgfpathlineto{\pgfqpoint{9.386982in}{5.196254in}}%
\pgfpathlineto{\pgfqpoint{9.417850in}{5.205848in}}%
\pgfpathlineto{\pgfqpoint{9.448717in}{5.218548in}}%
\pgfpathlineto{\pgfqpoint{9.479584in}{5.233826in}}%
\pgfpathlineto{\pgfqpoint{9.525885in}{5.259690in}}%
\pgfpathlineto{\pgfqpoint{9.587620in}{5.294054in}}%
\pgfpathlineto{\pgfqpoint{9.618487in}{5.308783in}}%
\pgfpathlineto{\pgfqpoint{9.649354in}{5.320713in}}%
\pgfpathlineto{\pgfqpoint{9.680222in}{5.329272in}}%
\pgfpathlineto{\pgfqpoint{9.711089in}{5.334191in}}%
\pgfpathlineto{\pgfqpoint{9.741956in}{5.335510in}}%
\pgfpathlineto{\pgfqpoint{9.741956in}{5.335510in}}%
\pgfusepath{stroke}%
\end{pgfscope}%
\begin{pgfscope}%
\pgfpathrectangle{\pgfqpoint{5.706832in}{3.881603in}}{\pgfqpoint{4.227273in}{2.800000in}} %
\pgfusepath{clip}%
\pgfsetrectcap%
\pgfsetroundjoin%
\pgfsetlinewidth{0.501875pt}%
\definecolor{currentstroke}{rgb}{1.000000,0.682749,0.366979}%
\pgfsetstrokecolor{currentstroke}%
\pgfsetdash{}{0pt}%
\pgfpathmoveto{\pgfqpoint{5.898981in}{5.529830in}}%
\pgfpathlineto{\pgfqpoint{5.914415in}{5.528190in}}%
\pgfpathlineto{\pgfqpoint{5.929848in}{5.523947in}}%
\pgfpathlineto{\pgfqpoint{5.945282in}{5.517031in}}%
\pgfpathlineto{\pgfqpoint{5.960715in}{5.507420in}}%
\pgfpathlineto{\pgfqpoint{5.976149in}{5.495143in}}%
\pgfpathlineto{\pgfqpoint{5.991583in}{5.480278in}}%
\pgfpathlineto{\pgfqpoint{6.007016in}{5.462954in}}%
\pgfpathlineto{\pgfqpoint{6.022450in}{5.443350in}}%
\pgfpathlineto{\pgfqpoint{6.053317in}{5.398252in}}%
\pgfpathlineto{\pgfqpoint{6.084185in}{5.347318in}}%
\pgfpathlineto{\pgfqpoint{6.161353in}{5.214627in}}%
\pgfpathlineto{\pgfqpoint{6.192220in}{5.168562in}}%
\pgfpathlineto{\pgfqpoint{6.207654in}{5.148802in}}%
\pgfpathlineto{\pgfqpoint{6.223087in}{5.131777in}}%
\pgfpathlineto{\pgfqpoint{6.238521in}{5.117866in}}%
\pgfpathlineto{\pgfqpoint{6.253955in}{5.107414in}}%
\pgfpathlineto{\pgfqpoint{6.269388in}{5.100726in}}%
\pgfpathlineto{\pgfqpoint{6.284822in}{5.098056in}}%
\pgfpathlineto{\pgfqpoint{6.300255in}{5.099610in}}%
\pgfpathlineto{\pgfqpoint{6.315689in}{5.105534in}}%
\pgfpathlineto{\pgfqpoint{6.331123in}{5.115913in}}%
\pgfpathlineto{\pgfqpoint{6.346556in}{5.130769in}}%
\pgfpathlineto{\pgfqpoint{6.361990in}{5.150058in}}%
\pgfpathlineto{\pgfqpoint{6.377424in}{5.173668in}}%
\pgfpathlineto{\pgfqpoint{6.392857in}{5.201423in}}%
\pgfpathlineto{\pgfqpoint{6.408291in}{5.233080in}}%
\pgfpathlineto{\pgfqpoint{6.423724in}{5.268335in}}%
\pgfpathlineto{\pgfqpoint{6.454592in}{5.348133in}}%
\pgfpathlineto{\pgfqpoint{6.485459in}{5.437312in}}%
\pgfpathlineto{\pgfqpoint{6.578061in}{5.718293in}}%
\pgfpathlineto{\pgfqpoint{6.608928in}{5.801550in}}%
\pgfpathlineto{\pgfqpoint{6.624362in}{5.838924in}}%
\pgfpathlineto{\pgfqpoint{6.639795in}{5.872905in}}%
\pgfpathlineto{\pgfqpoint{6.655229in}{5.903141in}}%
\pgfpathlineto{\pgfqpoint{6.670663in}{5.929337in}}%
\pgfpathlineto{\pgfqpoint{6.686096in}{5.951260in}}%
\pgfpathlineto{\pgfqpoint{6.701530in}{5.968740in}}%
\pgfpathlineto{\pgfqpoint{6.716964in}{5.981670in}}%
\pgfpathlineto{\pgfqpoint{6.732397in}{5.990003in}}%
\pgfpathlineto{\pgfqpoint{6.747831in}{5.993755in}}%
\pgfpathlineto{\pgfqpoint{6.763264in}{5.992994in}}%
\pgfpathlineto{\pgfqpoint{6.778698in}{5.987843in}}%
\pgfpathlineto{\pgfqpoint{6.794132in}{5.978469in}}%
\pgfpathlineto{\pgfqpoint{6.809565in}{5.965081in}}%
\pgfpathlineto{\pgfqpoint{6.824999in}{5.947922in}}%
\pgfpathlineto{\pgfqpoint{6.840433in}{5.927264in}}%
\pgfpathlineto{\pgfqpoint{6.855866in}{5.903401in}}%
\pgfpathlineto{\pgfqpoint{6.871300in}{5.876643in}}%
\pgfpathlineto{\pgfqpoint{6.902167in}{5.815737in}}%
\pgfpathlineto{\pgfqpoint{6.933034in}{5.747167in}}%
\pgfpathlineto{\pgfqpoint{6.979335in}{5.635602in}}%
\pgfpathlineto{\pgfqpoint{7.056504in}{5.446304in}}%
\pgfpathlineto{\pgfqpoint{7.087371in}{5.375564in}}%
\pgfpathlineto{\pgfqpoint{7.118238in}{5.310350in}}%
\pgfpathlineto{\pgfqpoint{7.149105in}{5.252098in}}%
\pgfpathlineto{\pgfqpoint{7.179973in}{5.201916in}}%
\pgfpathlineto{\pgfqpoint{7.195406in}{5.180091in}}%
\pgfpathlineto{\pgfqpoint{7.210840in}{5.160517in}}%
\pgfpathlineto{\pgfqpoint{7.226274in}{5.143210in}}%
\pgfpathlineto{\pgfqpoint{7.241707in}{5.128152in}}%
\pgfpathlineto{\pgfqpoint{7.257141in}{5.115294in}}%
\pgfpathlineto{\pgfqpoint{7.272574in}{5.104551in}}%
\pgfpathlineto{\pgfqpoint{7.288008in}{5.095805in}}%
\pgfpathlineto{\pgfqpoint{7.303442in}{5.088906in}}%
\pgfpathlineto{\pgfqpoint{7.318875in}{5.083673in}}%
\pgfpathlineto{\pgfqpoint{7.334309in}{5.079898in}}%
\pgfpathlineto{\pgfqpoint{7.365176in}{5.075781in}}%
\pgfpathlineto{\pgfqpoint{7.411477in}{5.074309in}}%
\pgfpathlineto{\pgfqpoint{7.442344in}{5.073400in}}%
\pgfpathlineto{\pgfqpoint{7.473212in}{5.070352in}}%
\pgfpathlineto{\pgfqpoint{7.504079in}{5.063767in}}%
\pgfpathlineto{\pgfqpoint{7.534946in}{5.052874in}}%
\pgfpathlineto{\pgfqpoint{7.565814in}{5.037647in}}%
\pgfpathlineto{\pgfqpoint{7.596681in}{5.018832in}}%
\pgfpathlineto{\pgfqpoint{7.689283in}{4.958193in}}%
\pgfpathlineto{\pgfqpoint{7.704716in}{4.950504in}}%
\pgfpathlineto{\pgfqpoint{7.720150in}{4.944324in}}%
\pgfpathlineto{\pgfqpoint{7.735584in}{4.939931in}}%
\pgfpathlineto{\pgfqpoint{7.751017in}{4.937573in}}%
\pgfpathlineto{\pgfqpoint{7.766451in}{4.937462in}}%
\pgfpathlineto{\pgfqpoint{7.781884in}{4.939762in}}%
\pgfpathlineto{\pgfqpoint{7.797318in}{4.944589in}}%
\pgfpathlineto{\pgfqpoint{7.812752in}{4.952001in}}%
\pgfpathlineto{\pgfqpoint{7.828185in}{4.961997in}}%
\pgfpathlineto{\pgfqpoint{7.843619in}{4.974519in}}%
\pgfpathlineto{\pgfqpoint{7.859053in}{4.989444in}}%
\pgfpathlineto{\pgfqpoint{7.874486in}{5.006593in}}%
\pgfpathlineto{\pgfqpoint{7.905353in}{5.046572in}}%
\pgfpathlineto{\pgfqpoint{7.936221in}{5.091999in}}%
\pgfpathlineto{\pgfqpoint{7.997955in}{5.186694in}}%
\pgfpathlineto{\pgfqpoint{8.028823in}{5.229244in}}%
\pgfpathlineto{\pgfqpoint{8.044256in}{5.247927in}}%
\pgfpathlineto{\pgfqpoint{8.059690in}{5.264428in}}%
\pgfpathlineto{\pgfqpoint{8.075123in}{5.278462in}}%
\pgfpathlineto{\pgfqpoint{8.090557in}{5.289794in}}%
\pgfpathlineto{\pgfqpoint{8.105991in}{5.298243in}}%
\pgfpathlineto{\pgfqpoint{8.121424in}{5.303688in}}%
\pgfpathlineto{\pgfqpoint{8.136858in}{5.306069in}}%
\pgfpathlineto{\pgfqpoint{8.152292in}{5.305384in}}%
\pgfpathlineto{\pgfqpoint{8.167725in}{5.301689in}}%
\pgfpathlineto{\pgfqpoint{8.183159in}{5.295095in}}%
\pgfpathlineto{\pgfqpoint{8.198593in}{5.285763in}}%
\pgfpathlineto{\pgfqpoint{8.214026in}{5.273897in}}%
\pgfpathlineto{\pgfqpoint{8.229460in}{5.259735in}}%
\pgfpathlineto{\pgfqpoint{8.244893in}{5.243548in}}%
\pgfpathlineto{\pgfqpoint{8.275761in}{5.206260in}}%
\pgfpathlineto{\pgfqpoint{8.306628in}{5.164434in}}%
\pgfpathlineto{\pgfqpoint{8.414663in}{5.012915in}}%
\pgfpathlineto{\pgfqpoint{8.445531in}{4.974189in}}%
\pgfpathlineto{\pgfqpoint{8.476398in}{4.938775in}}%
\pgfpathlineto{\pgfqpoint{8.507265in}{4.906777in}}%
\pgfpathlineto{\pgfqpoint{8.538133in}{4.878150in}}%
\pgfpathlineto{\pgfqpoint{8.569000in}{4.852869in}}%
\pgfpathlineto{\pgfqpoint{8.599867in}{4.831068in}}%
\pgfpathlineto{\pgfqpoint{8.630734in}{4.813118in}}%
\pgfpathlineto{\pgfqpoint{8.661602in}{4.799643in}}%
\pgfpathlineto{\pgfqpoint{8.677035in}{4.794833in}}%
\pgfpathlineto{\pgfqpoint{8.692469in}{4.791457in}}%
\pgfpathlineto{\pgfqpoint{8.707903in}{4.789624in}}%
\pgfpathlineto{\pgfqpoint{8.723336in}{4.789442in}}%
\pgfpathlineto{\pgfqpoint{8.738770in}{4.791002in}}%
\pgfpathlineto{\pgfqpoint{8.754203in}{4.794382in}}%
\pgfpathlineto{\pgfqpoint{8.769637in}{4.799636in}}%
\pgfpathlineto{\pgfqpoint{8.785071in}{4.806793in}}%
\pgfpathlineto{\pgfqpoint{8.800504in}{4.815849in}}%
\pgfpathlineto{\pgfqpoint{8.815938in}{4.826768in}}%
\pgfpathlineto{\pgfqpoint{8.831372in}{4.839480in}}%
\pgfpathlineto{\pgfqpoint{8.846805in}{4.853879in}}%
\pgfpathlineto{\pgfqpoint{8.877673in}{4.887139in}}%
\pgfpathlineto{\pgfqpoint{8.908540in}{4.925055in}}%
\pgfpathlineto{\pgfqpoint{9.016575in}{5.065839in}}%
\pgfpathlineto{\pgfqpoint{9.047443in}{5.100155in}}%
\pgfpathlineto{\pgfqpoint{9.078310in}{5.129053in}}%
\pgfpathlineto{\pgfqpoint{9.093743in}{5.141228in}}%
\pgfpathlineto{\pgfqpoint{9.109177in}{5.151834in}}%
\pgfpathlineto{\pgfqpoint{9.124611in}{5.160879in}}%
\pgfpathlineto{\pgfqpoint{9.140044in}{5.168404in}}%
\pgfpathlineto{\pgfqpoint{9.155478in}{5.174489in}}%
\pgfpathlineto{\pgfqpoint{9.170912in}{5.179237in}}%
\pgfpathlineto{\pgfqpoint{9.201779in}{5.185279in}}%
\pgfpathlineto{\pgfqpoint{9.232646in}{5.187811in}}%
\pgfpathlineto{\pgfqpoint{9.278947in}{5.188205in}}%
\pgfpathlineto{\pgfqpoint{9.325248in}{5.188725in}}%
\pgfpathlineto{\pgfqpoint{9.356115in}{5.191055in}}%
\pgfpathlineto{\pgfqpoint{9.386982in}{5.195847in}}%
\pgfpathlineto{\pgfqpoint{9.417850in}{5.203427in}}%
\pgfpathlineto{\pgfqpoint{9.448717in}{5.213752in}}%
\pgfpathlineto{\pgfqpoint{9.479584in}{5.226443in}}%
\pgfpathlineto{\pgfqpoint{9.525885in}{5.248454in}}%
\pgfpathlineto{\pgfqpoint{9.603053in}{5.285831in}}%
\pgfpathlineto{\pgfqpoint{9.633921in}{5.298465in}}%
\pgfpathlineto{\pgfqpoint{9.664788in}{5.308712in}}%
\pgfpathlineto{\pgfqpoint{9.695655in}{5.316136in}}%
\pgfpathlineto{\pgfqpoint{9.726522in}{5.320551in}}%
\pgfpathlineto{\pgfqpoint{9.741956in}{5.321642in}}%
\pgfpathlineto{\pgfqpoint{9.741956in}{5.321642in}}%
\pgfusepath{stroke}%
\end{pgfscope}%
\begin{pgfscope}%
\pgfpathrectangle{\pgfqpoint{5.706832in}{3.881603in}}{\pgfqpoint{4.227273in}{2.800000in}} %
\pgfusepath{clip}%
\pgfsetrectcap%
\pgfsetroundjoin%
\pgfsetlinewidth{0.501875pt}%
\definecolor{currentstroke}{rgb}{1.000000,0.587785,0.309017}%
\pgfsetstrokecolor{currentstroke}%
\pgfsetdash{}{0pt}%
\pgfpathmoveto{\pgfqpoint{5.898981in}{5.540206in}}%
\pgfpathlineto{\pgfqpoint{5.914415in}{5.538181in}}%
\pgfpathlineto{\pgfqpoint{5.929848in}{5.533432in}}%
\pgfpathlineto{\pgfqpoint{5.945282in}{5.525900in}}%
\pgfpathlineto{\pgfqpoint{5.960715in}{5.515576in}}%
\pgfpathlineto{\pgfqpoint{5.976149in}{5.502500in}}%
\pgfpathlineto{\pgfqpoint{5.991583in}{5.486762in}}%
\pgfpathlineto{\pgfqpoint{6.007016in}{5.468503in}}%
\pgfpathlineto{\pgfqpoint{6.022450in}{5.447912in}}%
\pgfpathlineto{\pgfqpoint{6.053317in}{5.400722in}}%
\pgfpathlineto{\pgfqpoint{6.084185in}{5.347587in}}%
\pgfpathlineto{\pgfqpoint{6.161353in}{5.209189in}}%
\pgfpathlineto{\pgfqpoint{6.192220in}{5.160843in}}%
\pgfpathlineto{\pgfqpoint{6.207654in}{5.139963in}}%
\pgfpathlineto{\pgfqpoint{6.223087in}{5.121837in}}%
\pgfpathlineto{\pgfqpoint{6.238521in}{5.106852in}}%
\pgfpathlineto{\pgfqpoint{6.253955in}{5.095364in}}%
\pgfpathlineto{\pgfqpoint{6.269388in}{5.087684in}}%
\pgfpathlineto{\pgfqpoint{6.284822in}{5.084078in}}%
\pgfpathlineto{\pgfqpoint{6.300255in}{5.084763in}}%
\pgfpathlineto{\pgfqpoint{6.315689in}{5.089896in}}%
\pgfpathlineto{\pgfqpoint{6.331123in}{5.099573in}}%
\pgfpathlineto{\pgfqpoint{6.346556in}{5.113829in}}%
\pgfpathlineto{\pgfqpoint{6.361990in}{5.132630in}}%
\pgfpathlineto{\pgfqpoint{6.377424in}{5.155875in}}%
\pgfpathlineto{\pgfqpoint{6.392857in}{5.183395in}}%
\pgfpathlineto{\pgfqpoint{6.408291in}{5.214955in}}%
\pgfpathlineto{\pgfqpoint{6.423724in}{5.250254in}}%
\pgfpathlineto{\pgfqpoint{6.454592in}{5.330575in}}%
\pgfpathlineto{\pgfqpoint{6.485459in}{5.420836in}}%
\pgfpathlineto{\pgfqpoint{6.593494in}{5.751503in}}%
\pgfpathlineto{\pgfqpoint{6.624362in}{5.831355in}}%
\pgfpathlineto{\pgfqpoint{6.639795in}{5.866294in}}%
\pgfpathlineto{\pgfqpoint{6.655229in}{5.897386in}}%
\pgfpathlineto{\pgfqpoint{6.670663in}{5.924311in}}%
\pgfpathlineto{\pgfqpoint{6.686096in}{5.946813in}}%
\pgfpathlineto{\pgfqpoint{6.701530in}{5.964702in}}%
\pgfpathlineto{\pgfqpoint{6.716964in}{5.977851in}}%
\pgfpathlineto{\pgfqpoint{6.732397in}{5.986202in}}%
\pgfpathlineto{\pgfqpoint{6.747831in}{5.989758in}}%
\pgfpathlineto{\pgfqpoint{6.763264in}{5.988585in}}%
\pgfpathlineto{\pgfqpoint{6.778698in}{5.982803in}}%
\pgfpathlineto{\pgfqpoint{6.794132in}{5.972584in}}%
\pgfpathlineto{\pgfqpoint{6.809565in}{5.958146in}}%
\pgfpathlineto{\pgfqpoint{6.824999in}{5.939747in}}%
\pgfpathlineto{\pgfqpoint{6.840433in}{5.917677in}}%
\pgfpathlineto{\pgfqpoint{6.855866in}{5.892254in}}%
\pgfpathlineto{\pgfqpoint{6.871300in}{5.863815in}}%
\pgfpathlineto{\pgfqpoint{6.902167in}{5.799312in}}%
\pgfpathlineto{\pgfqpoint{6.933034in}{5.727072in}}%
\pgfpathlineto{\pgfqpoint{6.979335in}{5.610531in}}%
\pgfpathlineto{\pgfqpoint{7.041070in}{5.454041in}}%
\pgfpathlineto{\pgfqpoint{7.071937in}{5.380702in}}%
\pgfpathlineto{\pgfqpoint{7.102804in}{5.313306in}}%
\pgfpathlineto{\pgfqpoint{7.133672in}{5.253439in}}%
\pgfpathlineto{\pgfqpoint{7.149105in}{5.226719in}}%
\pgfpathlineto{\pgfqpoint{7.164539in}{5.202296in}}%
\pgfpathlineto{\pgfqpoint{7.179973in}{5.180247in}}%
\pgfpathlineto{\pgfqpoint{7.195406in}{5.160618in}}%
\pgfpathlineto{\pgfqpoint{7.210840in}{5.143424in}}%
\pgfpathlineto{\pgfqpoint{7.226274in}{5.128641in}}%
\pgfpathlineto{\pgfqpoint{7.241707in}{5.116214in}}%
\pgfpathlineto{\pgfqpoint{7.257141in}{5.106050in}}%
\pgfpathlineto{\pgfqpoint{7.272574in}{5.098022in}}%
\pgfpathlineto{\pgfqpoint{7.288008in}{5.091969in}}%
\pgfpathlineto{\pgfqpoint{7.303442in}{5.087698in}}%
\pgfpathlineto{\pgfqpoint{7.318875in}{5.084991in}}%
\pgfpathlineto{\pgfqpoint{7.349743in}{5.083265in}}%
\pgfpathlineto{\pgfqpoint{7.380610in}{5.084634in}}%
\pgfpathlineto{\pgfqpoint{7.426911in}{5.087532in}}%
\pgfpathlineto{\pgfqpoint{7.457778in}{5.086953in}}%
\pgfpathlineto{\pgfqpoint{7.473212in}{5.085257in}}%
\pgfpathlineto{\pgfqpoint{7.488645in}{5.082413in}}%
\pgfpathlineto{\pgfqpoint{7.504079in}{5.078317in}}%
\pgfpathlineto{\pgfqpoint{7.519513in}{5.072911in}}%
\pgfpathlineto{\pgfqpoint{7.534946in}{5.066187in}}%
\pgfpathlineto{\pgfqpoint{7.565814in}{5.049013in}}%
\pgfpathlineto{\pgfqpoint{7.596681in}{5.027735in}}%
\pgfpathlineto{\pgfqpoint{7.689283in}{4.958477in}}%
\pgfpathlineto{\pgfqpoint{7.704716in}{4.949383in}}%
\pgfpathlineto{\pgfqpoint{7.720150in}{4.941849in}}%
\pgfpathlineto{\pgfqpoint{7.735584in}{4.936166in}}%
\pgfpathlineto{\pgfqpoint{7.751017in}{4.932595in}}%
\pgfpathlineto{\pgfqpoint{7.766451in}{4.931356in}}%
\pgfpathlineto{\pgfqpoint{7.781884in}{4.932624in}}%
\pgfpathlineto{\pgfqpoint{7.797318in}{4.936524in}}%
\pgfpathlineto{\pgfqpoint{7.812752in}{4.943123in}}%
\pgfpathlineto{\pgfqpoint{7.828185in}{4.952430in}}%
\pgfpathlineto{\pgfqpoint{7.843619in}{4.964394in}}%
\pgfpathlineto{\pgfqpoint{7.859053in}{4.978901in}}%
\pgfpathlineto{\pgfqpoint{7.874486in}{4.995782in}}%
\pgfpathlineto{\pgfqpoint{7.889920in}{5.014808in}}%
\pgfpathlineto{\pgfqpoint{7.920787in}{5.058136in}}%
\pgfpathlineto{\pgfqpoint{7.967088in}{5.130916in}}%
\pgfpathlineto{\pgfqpoint{8.013389in}{5.203137in}}%
\pgfpathlineto{\pgfqpoint{8.044256in}{5.245375in}}%
\pgfpathlineto{\pgfqpoint{8.059690in}{5.263581in}}%
\pgfpathlineto{\pgfqpoint{8.075123in}{5.279419in}}%
\pgfpathlineto{\pgfqpoint{8.090557in}{5.292629in}}%
\pgfpathlineto{\pgfqpoint{8.105991in}{5.303002in}}%
\pgfpathlineto{\pgfqpoint{8.121424in}{5.310385in}}%
\pgfpathlineto{\pgfqpoint{8.136858in}{5.314682in}}%
\pgfpathlineto{\pgfqpoint{8.152292in}{5.315857in}}%
\pgfpathlineto{\pgfqpoint{8.167725in}{5.313930in}}%
\pgfpathlineto{\pgfqpoint{8.183159in}{5.308978in}}%
\pgfpathlineto{\pgfqpoint{8.198593in}{5.301125in}}%
\pgfpathlineto{\pgfqpoint{8.214026in}{5.290546in}}%
\pgfpathlineto{\pgfqpoint{8.229460in}{5.277451in}}%
\pgfpathlineto{\pgfqpoint{8.244893in}{5.262084in}}%
\pgfpathlineto{\pgfqpoint{8.260327in}{5.244715in}}%
\pgfpathlineto{\pgfqpoint{8.291194in}{5.205129in}}%
\pgfpathlineto{\pgfqpoint{8.322062in}{5.161052in}}%
\pgfpathlineto{\pgfqpoint{8.430097in}{5.001538in}}%
\pgfpathlineto{\pgfqpoint{8.460964in}{4.960611in}}%
\pgfpathlineto{\pgfqpoint{8.491832in}{4.923261in}}%
\pgfpathlineto{\pgfqpoint{8.522699in}{4.889805in}}%
\pgfpathlineto{\pgfqpoint{8.553566in}{4.860434in}}%
\pgfpathlineto{\pgfqpoint{8.584433in}{4.835346in}}%
\pgfpathlineto{\pgfqpoint{8.615301in}{4.814841in}}%
\pgfpathlineto{\pgfqpoint{8.630734in}{4.806444in}}%
\pgfpathlineto{\pgfqpoint{8.646168in}{4.799374in}}%
\pgfpathlineto{\pgfqpoint{8.661602in}{4.793710in}}%
\pgfpathlineto{\pgfqpoint{8.677035in}{4.789532in}}%
\pgfpathlineto{\pgfqpoint{8.692469in}{4.786924in}}%
\pgfpathlineto{\pgfqpoint{8.707903in}{4.785964in}}%
\pgfpathlineto{\pgfqpoint{8.723336in}{4.786723in}}%
\pgfpathlineto{\pgfqpoint{8.738770in}{4.789260in}}%
\pgfpathlineto{\pgfqpoint{8.754203in}{4.793617in}}%
\pgfpathlineto{\pgfqpoint{8.769637in}{4.799812in}}%
\pgfpathlineto{\pgfqpoint{8.785071in}{4.807844in}}%
\pgfpathlineto{\pgfqpoint{8.800504in}{4.817678in}}%
\pgfpathlineto{\pgfqpoint{8.815938in}{4.829255in}}%
\pgfpathlineto{\pgfqpoint{8.831372in}{4.842481in}}%
\pgfpathlineto{\pgfqpoint{8.862239in}{4.873361in}}%
\pgfpathlineto{\pgfqpoint{8.893106in}{4.908998in}}%
\pgfpathlineto{\pgfqpoint{8.939407in}{4.967591in}}%
\pgfpathlineto{\pgfqpoint{8.985708in}{5.026498in}}%
\pgfpathlineto{\pgfqpoint{9.016575in}{5.062788in}}%
\pgfpathlineto{\pgfqpoint{9.047443in}{5.094871in}}%
\pgfpathlineto{\pgfqpoint{9.078310in}{5.121685in}}%
\pgfpathlineto{\pgfqpoint{9.093743in}{5.132930in}}%
\pgfpathlineto{\pgfqpoint{9.109177in}{5.142705in}}%
\pgfpathlineto{\pgfqpoint{9.124611in}{5.151031in}}%
\pgfpathlineto{\pgfqpoint{9.140044in}{5.157962in}}%
\pgfpathlineto{\pgfqpoint{9.155478in}{5.163582in}}%
\pgfpathlineto{\pgfqpoint{9.186345in}{5.171342in}}%
\pgfpathlineto{\pgfqpoint{9.217212in}{5.175419in}}%
\pgfpathlineto{\pgfqpoint{9.263513in}{5.177502in}}%
\pgfpathlineto{\pgfqpoint{9.325248in}{5.179561in}}%
\pgfpathlineto{\pgfqpoint{9.356115in}{5.182656in}}%
\pgfpathlineto{\pgfqpoint{9.386982in}{5.188004in}}%
\pgfpathlineto{\pgfqpoint{9.417850in}{5.195835in}}%
\pgfpathlineto{\pgfqpoint{9.448717in}{5.206053in}}%
\pgfpathlineto{\pgfqpoint{9.479584in}{5.218268in}}%
\pgfpathlineto{\pgfqpoint{9.541319in}{5.246087in}}%
\pgfpathlineto{\pgfqpoint{9.587620in}{5.266770in}}%
\pgfpathlineto{\pgfqpoint{9.618487in}{5.278961in}}%
\pgfpathlineto{\pgfqpoint{9.649354in}{5.289079in}}%
\pgfpathlineto{\pgfqpoint{9.680222in}{5.296627in}}%
\pgfpathlineto{\pgfqpoint{9.711089in}{5.301319in}}%
\pgfpathlineto{\pgfqpoint{9.741956in}{5.303094in}}%
\pgfpathlineto{\pgfqpoint{9.741956in}{5.303094in}}%
\pgfusepath{stroke}%
\end{pgfscope}%
\begin{pgfscope}%
\pgfpathrectangle{\pgfqpoint{5.706832in}{3.881603in}}{\pgfqpoint{4.227273in}{2.800000in}} %
\pgfusepath{clip}%
\pgfsetrectcap%
\pgfsetroundjoin%
\pgfsetlinewidth{0.501875pt}%
\definecolor{currentstroke}{rgb}{1.000000,0.473094,0.243914}%
\pgfsetstrokecolor{currentstroke}%
\pgfsetdash{}{0pt}%
\pgfpathmoveto{\pgfqpoint{5.898981in}{5.539938in}}%
\pgfpathlineto{\pgfqpoint{5.914415in}{5.537144in}}%
\pgfpathlineto{\pgfqpoint{5.929848in}{5.531634in}}%
\pgfpathlineto{\pgfqpoint{5.945282in}{5.523359in}}%
\pgfpathlineto{\pgfqpoint{5.960715in}{5.512323in}}%
\pgfpathlineto{\pgfqpoint{5.976149in}{5.498579in}}%
\pgfpathlineto{\pgfqpoint{5.991583in}{5.482232in}}%
\pgfpathlineto{\pgfqpoint{6.007016in}{5.463438in}}%
\pgfpathlineto{\pgfqpoint{6.022450in}{5.442397in}}%
\pgfpathlineto{\pgfqpoint{6.053317in}{5.394603in}}%
\pgfpathlineto{\pgfqpoint{6.084185in}{5.341294in}}%
\pgfpathlineto{\pgfqpoint{6.161353in}{5.204171in}}%
\pgfpathlineto{\pgfqpoint{6.192220in}{5.156820in}}%
\pgfpathlineto{\pgfqpoint{6.207654in}{5.136494in}}%
\pgfpathlineto{\pgfqpoint{6.223087in}{5.118949in}}%
\pgfpathlineto{\pgfqpoint{6.238521in}{5.104562in}}%
\pgfpathlineto{\pgfqpoint{6.253955in}{5.093681in}}%
\pgfpathlineto{\pgfqpoint{6.269388in}{5.086611in}}%
\pgfpathlineto{\pgfqpoint{6.284822in}{5.083613in}}%
\pgfpathlineto{\pgfqpoint{6.300255in}{5.084894in}}%
\pgfpathlineto{\pgfqpoint{6.315689in}{5.090604in}}%
\pgfpathlineto{\pgfqpoint{6.331123in}{5.100832in}}%
\pgfpathlineto{\pgfqpoint{6.346556in}{5.115600in}}%
\pgfpathlineto{\pgfqpoint{6.361990in}{5.134861in}}%
\pgfpathlineto{\pgfqpoint{6.377424in}{5.158502in}}%
\pgfpathlineto{\pgfqpoint{6.392857in}{5.186339in}}%
\pgfpathlineto{\pgfqpoint{6.408291in}{5.218122in}}%
\pgfpathlineto{\pgfqpoint{6.423724in}{5.253536in}}%
\pgfpathlineto{\pgfqpoint{6.454592in}{5.333719in}}%
\pgfpathlineto{\pgfqpoint{6.485459in}{5.423322in}}%
\pgfpathlineto{\pgfqpoint{6.578061in}{5.705802in}}%
\pgfpathlineto{\pgfqpoint{6.608928in}{5.789868in}}%
\pgfpathlineto{\pgfqpoint{6.624362in}{5.827757in}}%
\pgfpathlineto{\pgfqpoint{6.639795in}{5.862337in}}%
\pgfpathlineto{\pgfqpoint{6.655229in}{5.893256in}}%
\pgfpathlineto{\pgfqpoint{6.670663in}{5.920215in}}%
\pgfpathlineto{\pgfqpoint{6.686096in}{5.942972in}}%
\pgfpathlineto{\pgfqpoint{6.701530in}{5.961338in}}%
\pgfpathlineto{\pgfqpoint{6.716964in}{5.975183in}}%
\pgfpathlineto{\pgfqpoint{6.732397in}{5.984436in}}%
\pgfpathlineto{\pgfqpoint{6.747831in}{5.989077in}}%
\pgfpathlineto{\pgfqpoint{6.763264in}{5.989146in}}%
\pgfpathlineto{\pgfqpoint{6.778698in}{5.984729in}}%
\pgfpathlineto{\pgfqpoint{6.794132in}{5.975963in}}%
\pgfpathlineto{\pgfqpoint{6.809565in}{5.963031in}}%
\pgfpathlineto{\pgfqpoint{6.824999in}{5.946152in}}%
\pgfpathlineto{\pgfqpoint{6.840433in}{5.925584in}}%
\pgfpathlineto{\pgfqpoint{6.855866in}{5.901613in}}%
\pgfpathlineto{\pgfqpoint{6.871300in}{5.874551in}}%
\pgfpathlineto{\pgfqpoint{6.902167in}{5.812491in}}%
\pgfpathlineto{\pgfqpoint{6.933034in}{5.742187in}}%
\pgfpathlineto{\pgfqpoint{6.979335in}{5.627483in}}%
\pgfpathlineto{\pgfqpoint{7.056504in}{5.433756in}}%
\pgfpathlineto{\pgfqpoint{7.087371in}{5.362086in}}%
\pgfpathlineto{\pgfqpoint{7.118238in}{5.296537in}}%
\pgfpathlineto{\pgfqpoint{7.149105in}{5.238517in}}%
\pgfpathlineto{\pgfqpoint{7.164539in}{5.212670in}}%
\pgfpathlineto{\pgfqpoint{7.179973in}{5.189064in}}%
\pgfpathlineto{\pgfqpoint{7.195406in}{5.167760in}}%
\pgfpathlineto{\pgfqpoint{7.210840in}{5.148793in}}%
\pgfpathlineto{\pgfqpoint{7.226274in}{5.132162in}}%
\pgfpathlineto{\pgfqpoint{7.241707in}{5.117837in}}%
\pgfpathlineto{\pgfqpoint{7.257141in}{5.105749in}}%
\pgfpathlineto{\pgfqpoint{7.272574in}{5.095798in}}%
\pgfpathlineto{\pgfqpoint{7.288008in}{5.087846in}}%
\pgfpathlineto{\pgfqpoint{7.303442in}{5.081725in}}%
\pgfpathlineto{\pgfqpoint{7.318875in}{5.077235in}}%
\pgfpathlineto{\pgfqpoint{7.334309in}{5.074149in}}%
\pgfpathlineto{\pgfqpoint{7.365176in}{5.071181in}}%
\pgfpathlineto{\pgfqpoint{7.457778in}{5.068537in}}%
\pgfpathlineto{\pgfqpoint{7.488645in}{5.063399in}}%
\pgfpathlineto{\pgfqpoint{7.519513in}{5.054130in}}%
\pgfpathlineto{\pgfqpoint{7.550380in}{5.040563in}}%
\pgfpathlineto{\pgfqpoint{7.581247in}{5.023291in}}%
\pgfpathlineto{\pgfqpoint{7.627548in}{4.993413in}}%
\pgfpathlineto{\pgfqpoint{7.658415in}{4.973670in}}%
\pgfpathlineto{\pgfqpoint{7.689283in}{4.956694in}}%
\pgfpathlineto{\pgfqpoint{7.704716in}{4.949962in}}%
\pgfpathlineto{\pgfqpoint{7.720150in}{4.944761in}}%
\pgfpathlineto{\pgfqpoint{7.735584in}{4.941337in}}%
\pgfpathlineto{\pgfqpoint{7.751017in}{4.939907in}}%
\pgfpathlineto{\pgfqpoint{7.766451in}{4.940652in}}%
\pgfpathlineto{\pgfqpoint{7.781884in}{4.943713in}}%
\pgfpathlineto{\pgfqpoint{7.797318in}{4.949190in}}%
\pgfpathlineto{\pgfqpoint{7.812752in}{4.957128in}}%
\pgfpathlineto{\pgfqpoint{7.828185in}{4.967528in}}%
\pgfpathlineto{\pgfqpoint{7.843619in}{4.980332in}}%
\pgfpathlineto{\pgfqpoint{7.859053in}{4.995432in}}%
\pgfpathlineto{\pgfqpoint{7.874486in}{5.012665in}}%
\pgfpathlineto{\pgfqpoint{7.905353in}{5.052628in}}%
\pgfpathlineto{\pgfqpoint{7.936221in}{5.097962in}}%
\pgfpathlineto{\pgfqpoint{8.013389in}{5.215219in}}%
\pgfpathlineto{\pgfqpoint{8.044256in}{5.255194in}}%
\pgfpathlineto{\pgfqpoint{8.059690in}{5.272191in}}%
\pgfpathlineto{\pgfqpoint{8.075123in}{5.286786in}}%
\pgfpathlineto{\pgfqpoint{8.090557in}{5.298735in}}%
\pgfpathlineto{\pgfqpoint{8.105991in}{5.307845in}}%
\pgfpathlineto{\pgfqpoint{8.121424in}{5.313983in}}%
\pgfpathlineto{\pgfqpoint{8.136858in}{5.317075in}}%
\pgfpathlineto{\pgfqpoint{8.152292in}{5.317106in}}%
\pgfpathlineto{\pgfqpoint{8.167725in}{5.314118in}}%
\pgfpathlineto{\pgfqpoint{8.183159in}{5.308210in}}%
\pgfpathlineto{\pgfqpoint{8.198593in}{5.299526in}}%
\pgfpathlineto{\pgfqpoint{8.214026in}{5.288256in}}%
\pgfpathlineto{\pgfqpoint{8.229460in}{5.274624in}}%
\pgfpathlineto{\pgfqpoint{8.244893in}{5.258880in}}%
\pgfpathlineto{\pgfqpoint{8.275761in}{5.222152in}}%
\pgfpathlineto{\pgfqpoint{8.306628in}{5.180319in}}%
\pgfpathlineto{\pgfqpoint{8.368363in}{5.089877in}}%
\pgfpathlineto{\pgfqpoint{8.414663in}{5.022954in}}%
\pgfpathlineto{\pgfqpoint{8.445531in}{4.981080in}}%
\pgfpathlineto{\pgfqpoint{8.476398in}{4.942284in}}%
\pgfpathlineto{\pgfqpoint{8.507265in}{4.907043in}}%
\pgfpathlineto{\pgfqpoint{8.538133in}{4.875696in}}%
\pgfpathlineto{\pgfqpoint{8.569000in}{4.848533in}}%
\pgfpathlineto{\pgfqpoint{8.599867in}{4.825867in}}%
\pgfpathlineto{\pgfqpoint{8.630734in}{4.808084in}}%
\pgfpathlineto{\pgfqpoint{8.646168in}{4.801166in}}%
\pgfpathlineto{\pgfqpoint{8.661602in}{4.795650in}}%
\pgfpathlineto{\pgfqpoint{8.677035in}{4.791602in}}%
\pgfpathlineto{\pgfqpoint{8.692469in}{4.789091in}}%
\pgfpathlineto{\pgfqpoint{8.707903in}{4.788181in}}%
\pgfpathlineto{\pgfqpoint{8.723336in}{4.788934in}}%
\pgfpathlineto{\pgfqpoint{8.738770in}{4.791399in}}%
\pgfpathlineto{\pgfqpoint{8.754203in}{4.795618in}}%
\pgfpathlineto{\pgfqpoint{8.769637in}{4.801612in}}%
\pgfpathlineto{\pgfqpoint{8.785071in}{4.809387in}}%
\pgfpathlineto{\pgfqpoint{8.800504in}{4.818925in}}%
\pgfpathlineto{\pgfqpoint{8.815938in}{4.830183in}}%
\pgfpathlineto{\pgfqpoint{8.831372in}{4.843090in}}%
\pgfpathlineto{\pgfqpoint{8.862239in}{4.873430in}}%
\pgfpathlineto{\pgfqpoint{8.893106in}{4.908813in}}%
\pgfpathlineto{\pgfqpoint{8.939407in}{4.967870in}}%
\pgfpathlineto{\pgfqpoint{9.001142in}{5.047681in}}%
\pgfpathlineto{\pgfqpoint{9.032009in}{5.083694in}}%
\pgfpathlineto{\pgfqpoint{9.062876in}{5.114996in}}%
\pgfpathlineto{\pgfqpoint{9.078310in}{5.128543in}}%
\pgfpathlineto{\pgfqpoint{9.093743in}{5.140579in}}%
\pgfpathlineto{\pgfqpoint{9.109177in}{5.151069in}}%
\pgfpathlineto{\pgfqpoint{9.124611in}{5.160016in}}%
\pgfpathlineto{\pgfqpoint{9.140044in}{5.167456in}}%
\pgfpathlineto{\pgfqpoint{9.155478in}{5.173459in}}%
\pgfpathlineto{\pgfqpoint{9.170912in}{5.178124in}}%
\pgfpathlineto{\pgfqpoint{9.201779in}{5.183952in}}%
\pgfpathlineto{\pgfqpoint{9.232646in}{5.186135in}}%
\pgfpathlineto{\pgfqpoint{9.278947in}{5.185549in}}%
\pgfpathlineto{\pgfqpoint{9.340682in}{5.184426in}}%
\pgfpathlineto{\pgfqpoint{9.371549in}{5.186114in}}%
\pgfpathlineto{\pgfqpoint{9.402416in}{5.190236in}}%
\pgfpathlineto{\pgfqpoint{9.433283in}{5.197105in}}%
\pgfpathlineto{\pgfqpoint{9.464151in}{5.206707in}}%
\pgfpathlineto{\pgfqpoint{9.495018in}{5.218722in}}%
\pgfpathlineto{\pgfqpoint{9.541319in}{5.239886in}}%
\pgfpathlineto{\pgfqpoint{9.618487in}{5.276034in}}%
\pgfpathlineto{\pgfqpoint{9.649354in}{5.287974in}}%
\pgfpathlineto{\pgfqpoint{9.680222in}{5.297261in}}%
\pgfpathlineto{\pgfqpoint{9.711089in}{5.303420in}}%
\pgfpathlineto{\pgfqpoint{9.741956in}{5.306285in}}%
\pgfpathlineto{\pgfqpoint{9.741956in}{5.306285in}}%
\pgfusepath{stroke}%
\end{pgfscope}%
\begin{pgfscope}%
\pgfpathrectangle{\pgfqpoint{5.706832in}{3.881603in}}{\pgfqpoint{4.227273in}{2.800000in}} %
\pgfusepath{clip}%
\pgfsetrectcap%
\pgfsetroundjoin%
\pgfsetlinewidth{0.501875pt}%
\definecolor{currentstroke}{rgb}{1.000000,0.361242,0.183750}%
\pgfsetstrokecolor{currentstroke}%
\pgfsetdash{}{0pt}%
\pgfpathmoveto{\pgfqpoint{5.898981in}{5.548757in}}%
\pgfpathlineto{\pgfqpoint{5.914415in}{5.545935in}}%
\pgfpathlineto{\pgfqpoint{5.929848in}{5.540314in}}%
\pgfpathlineto{\pgfqpoint{5.945282in}{5.531850in}}%
\pgfpathlineto{\pgfqpoint{5.960715in}{5.520550in}}%
\pgfpathlineto{\pgfqpoint{5.976149in}{5.506470in}}%
\pgfpathlineto{\pgfqpoint{5.991583in}{5.489720in}}%
\pgfpathlineto{\pgfqpoint{6.007016in}{5.470456in}}%
\pgfpathlineto{\pgfqpoint{6.022450in}{5.448884in}}%
\pgfpathlineto{\pgfqpoint{6.053317in}{5.399850in}}%
\pgfpathlineto{\pgfqpoint{6.084185in}{5.345077in}}%
\pgfpathlineto{\pgfqpoint{6.161353in}{5.203474in}}%
\pgfpathlineto{\pgfqpoint{6.192220in}{5.154125in}}%
\pgfpathlineto{\pgfqpoint{6.207654in}{5.132792in}}%
\pgfpathlineto{\pgfqpoint{6.223087in}{5.114251in}}%
\pgfpathlineto{\pgfqpoint{6.238521in}{5.098892in}}%
\pgfpathlineto{\pgfqpoint{6.253955in}{5.087075in}}%
\pgfpathlineto{\pgfqpoint{6.269388in}{5.079120in}}%
\pgfpathlineto{\pgfqpoint{6.284822in}{5.075300in}}%
\pgfpathlineto{\pgfqpoint{6.300255in}{5.075836in}}%
\pgfpathlineto{\pgfqpoint{6.315689in}{5.080889in}}%
\pgfpathlineto{\pgfqpoint{6.331123in}{5.090559in}}%
\pgfpathlineto{\pgfqpoint{6.346556in}{5.104876in}}%
\pgfpathlineto{\pgfqpoint{6.361990in}{5.123800in}}%
\pgfpathlineto{\pgfqpoint{6.377424in}{5.147223in}}%
\pgfpathlineto{\pgfqpoint{6.392857in}{5.174962in}}%
\pgfpathlineto{\pgfqpoint{6.408291in}{5.206769in}}%
\pgfpathlineto{\pgfqpoint{6.423724in}{5.242326in}}%
\pgfpathlineto{\pgfqpoint{6.454592in}{5.323133in}}%
\pgfpathlineto{\pgfqpoint{6.485459in}{5.413757in}}%
\pgfpathlineto{\pgfqpoint{6.593494in}{5.744685in}}%
\pgfpathlineto{\pgfqpoint{6.624362in}{5.824779in}}%
\pgfpathlineto{\pgfqpoint{6.639795in}{5.859967in}}%
\pgfpathlineto{\pgfqpoint{6.655229in}{5.891413in}}%
\pgfpathlineto{\pgfqpoint{6.670663in}{5.918807in}}%
\pgfpathlineto{\pgfqpoint{6.686096in}{5.941897in}}%
\pgfpathlineto{\pgfqpoint{6.701530in}{5.960489in}}%
\pgfpathlineto{\pgfqpoint{6.716964in}{5.974450in}}%
\pgfpathlineto{\pgfqpoint{6.732397in}{5.983704in}}%
\pgfpathlineto{\pgfqpoint{6.747831in}{5.988239in}}%
\pgfpathlineto{\pgfqpoint{6.763264in}{5.988094in}}%
\pgfpathlineto{\pgfqpoint{6.778698in}{5.983365in}}%
\pgfpathlineto{\pgfqpoint{6.794132in}{5.974198in}}%
\pgfpathlineto{\pgfqpoint{6.809565in}{5.960786in}}%
\pgfpathlineto{\pgfqpoint{6.824999in}{5.943362in}}%
\pgfpathlineto{\pgfqpoint{6.840433in}{5.922197in}}%
\pgfpathlineto{\pgfqpoint{6.855866in}{5.897589in}}%
\pgfpathlineto{\pgfqpoint{6.871300in}{5.869865in}}%
\pgfpathlineto{\pgfqpoint{6.902167in}{5.806457in}}%
\pgfpathlineto{\pgfqpoint{6.933034in}{5.734850in}}%
\pgfpathlineto{\pgfqpoint{6.979335in}{5.618434in}}%
\pgfpathlineto{\pgfqpoint{7.041070in}{5.460725in}}%
\pgfpathlineto{\pgfqpoint{7.071937in}{5.386216in}}%
\pgfpathlineto{\pgfqpoint{7.102804in}{5.317282in}}%
\pgfpathlineto{\pgfqpoint{7.133672in}{5.255533in}}%
\pgfpathlineto{\pgfqpoint{7.164539in}{5.202222in}}%
\pgfpathlineto{\pgfqpoint{7.179973in}{5.179012in}}%
\pgfpathlineto{\pgfqpoint{7.195406in}{5.158193in}}%
\pgfpathlineto{\pgfqpoint{7.210840in}{5.139795in}}%
\pgfpathlineto{\pgfqpoint{7.226274in}{5.123816in}}%
\pgfpathlineto{\pgfqpoint{7.241707in}{5.110217in}}%
\pgfpathlineto{\pgfqpoint{7.257141in}{5.098923in}}%
\pgfpathlineto{\pgfqpoint{7.272574in}{5.089821in}}%
\pgfpathlineto{\pgfqpoint{7.288008in}{5.082765in}}%
\pgfpathlineto{\pgfqpoint{7.303442in}{5.077571in}}%
\pgfpathlineto{\pgfqpoint{7.318875in}{5.074025in}}%
\pgfpathlineto{\pgfqpoint{7.334309in}{5.071886in}}%
\pgfpathlineto{\pgfqpoint{7.365176in}{5.070755in}}%
\pgfpathlineto{\pgfqpoint{7.442344in}{5.073038in}}%
\pgfpathlineto{\pgfqpoint{7.473212in}{5.070554in}}%
\pgfpathlineto{\pgfqpoint{7.488645in}{5.067812in}}%
\pgfpathlineto{\pgfqpoint{7.504079in}{5.063932in}}%
\pgfpathlineto{\pgfqpoint{7.519513in}{5.058871in}}%
\pgfpathlineto{\pgfqpoint{7.550380in}{5.045280in}}%
\pgfpathlineto{\pgfqpoint{7.581247in}{5.027675in}}%
\pgfpathlineto{\pgfqpoint{7.627548in}{4.996831in}}%
\pgfpathlineto{\pgfqpoint{7.658415in}{4.976225in}}%
\pgfpathlineto{\pgfqpoint{7.689283in}{4.958297in}}%
\pgfpathlineto{\pgfqpoint{7.704716in}{4.951083in}}%
\pgfpathlineto{\pgfqpoint{7.720150in}{4.945412in}}%
\pgfpathlineto{\pgfqpoint{7.735584in}{4.941542in}}%
\pgfpathlineto{\pgfqpoint{7.751017in}{4.939703in}}%
\pgfpathlineto{\pgfqpoint{7.766451in}{4.940091in}}%
\pgfpathlineto{\pgfqpoint{7.781884in}{4.942860in}}%
\pgfpathlineto{\pgfqpoint{7.797318in}{4.948121in}}%
\pgfpathlineto{\pgfqpoint{7.812752in}{4.955932in}}%
\pgfpathlineto{\pgfqpoint{7.828185in}{4.966302in}}%
\pgfpathlineto{\pgfqpoint{7.843619in}{4.979182in}}%
\pgfpathlineto{\pgfqpoint{7.859053in}{4.994469in}}%
\pgfpathlineto{\pgfqpoint{7.874486in}{5.012000in}}%
\pgfpathlineto{\pgfqpoint{7.905353in}{5.052887in}}%
\pgfpathlineto{\pgfqpoint{7.936221in}{5.099529in}}%
\pgfpathlineto{\pgfqpoint{8.013389in}{5.220963in}}%
\pgfpathlineto{\pgfqpoint{8.044256in}{5.262571in}}%
\pgfpathlineto{\pgfqpoint{8.059690in}{5.280301in}}%
\pgfpathlineto{\pgfqpoint{8.075123in}{5.295554in}}%
\pgfpathlineto{\pgfqpoint{8.090557in}{5.308074in}}%
\pgfpathlineto{\pgfqpoint{8.105991in}{5.317660in}}%
\pgfpathlineto{\pgfqpoint{8.121424in}{5.324173in}}%
\pgfpathlineto{\pgfqpoint{8.136858in}{5.327537in}}%
\pgfpathlineto{\pgfqpoint{8.152292in}{5.327737in}}%
\pgfpathlineto{\pgfqpoint{8.167725in}{5.324820in}}%
\pgfpathlineto{\pgfqpoint{8.183159in}{5.318886in}}%
\pgfpathlineto{\pgfqpoint{8.198593in}{5.310089in}}%
\pgfpathlineto{\pgfqpoint{8.214026in}{5.298623in}}%
\pgfpathlineto{\pgfqpoint{8.229460in}{5.284720in}}%
\pgfpathlineto{\pgfqpoint{8.244893in}{5.268637in}}%
\pgfpathlineto{\pgfqpoint{8.275761in}{5.231052in}}%
\pgfpathlineto{\pgfqpoint{8.306628in}{5.188161in}}%
\pgfpathlineto{\pgfqpoint{8.352929in}{5.118685in}}%
\pgfpathlineto{\pgfqpoint{8.414663in}{5.026006in}}%
\pgfpathlineto{\pgfqpoint{8.445531in}{4.982655in}}%
\pgfpathlineto{\pgfqpoint{8.476398in}{4.942482in}}%
\pgfpathlineto{\pgfqpoint{8.507265in}{4.906073in}}%
\pgfpathlineto{\pgfqpoint{8.538133in}{4.873875in}}%
\pgfpathlineto{\pgfqpoint{8.569000in}{4.846263in}}%
\pgfpathlineto{\pgfqpoint{8.599867in}{4.823596in}}%
\pgfpathlineto{\pgfqpoint{8.615301in}{4.814235in}}%
\pgfpathlineto{\pgfqpoint{8.630734in}{4.806254in}}%
\pgfpathlineto{\pgfqpoint{8.646168in}{4.799707in}}%
\pgfpathlineto{\pgfqpoint{8.661602in}{4.794647in}}%
\pgfpathlineto{\pgfqpoint{8.677035in}{4.791128in}}%
\pgfpathlineto{\pgfqpoint{8.692469in}{4.789203in}}%
\pgfpathlineto{\pgfqpoint{8.707903in}{4.788922in}}%
\pgfpathlineto{\pgfqpoint{8.723336in}{4.790329in}}%
\pgfpathlineto{\pgfqpoint{8.738770in}{4.793461in}}%
\pgfpathlineto{\pgfqpoint{8.754203in}{4.798345in}}%
\pgfpathlineto{\pgfqpoint{8.769637in}{4.804992in}}%
\pgfpathlineto{\pgfqpoint{8.785071in}{4.813398in}}%
\pgfpathlineto{\pgfqpoint{8.800504in}{4.823536in}}%
\pgfpathlineto{\pgfqpoint{8.815938in}{4.835359in}}%
\pgfpathlineto{\pgfqpoint{8.831372in}{4.848794in}}%
\pgfpathlineto{\pgfqpoint{8.862239in}{4.880063in}}%
\pgfpathlineto{\pgfqpoint{8.893106in}{4.916204in}}%
\pgfpathlineto{\pgfqpoint{8.939407in}{4.976036in}}%
\pgfpathlineto{\pgfqpoint{8.985708in}{5.036748in}}%
\pgfpathlineto{\pgfqpoint{9.016575in}{5.074309in}}%
\pgfpathlineto{\pgfqpoint{9.047443in}{5.107432in}}%
\pgfpathlineto{\pgfqpoint{9.062876in}{5.121894in}}%
\pgfpathlineto{\pgfqpoint{9.078310in}{5.134796in}}%
\pgfpathlineto{\pgfqpoint{9.093743in}{5.146066in}}%
\pgfpathlineto{\pgfqpoint{9.109177in}{5.155673in}}%
\pgfpathlineto{\pgfqpoint{9.124611in}{5.163627in}}%
\pgfpathlineto{\pgfqpoint{9.140044in}{5.169979in}}%
\pgfpathlineto{\pgfqpoint{9.155478in}{5.174814in}}%
\pgfpathlineto{\pgfqpoint{9.170912in}{5.178250in}}%
\pgfpathlineto{\pgfqpoint{9.201779in}{5.181526in}}%
\pgfpathlineto{\pgfqpoint{9.232646in}{5.181184in}}%
\pgfpathlineto{\pgfqpoint{9.278947in}{5.177232in}}%
\pgfpathlineto{\pgfqpoint{9.325248in}{5.173463in}}%
\pgfpathlineto{\pgfqpoint{9.356115in}{5.172972in}}%
\pgfpathlineto{\pgfqpoint{9.386982in}{5.174988in}}%
\pgfpathlineto{\pgfqpoint{9.417850in}{5.179885in}}%
\pgfpathlineto{\pgfqpoint{9.448717in}{5.187711in}}%
\pgfpathlineto{\pgfqpoint{9.479584in}{5.198213in}}%
\pgfpathlineto{\pgfqpoint{9.510452in}{5.210885in}}%
\pgfpathlineto{\pgfqpoint{9.572186in}{5.239750in}}%
\pgfpathlineto{\pgfqpoint{9.618487in}{5.260983in}}%
\pgfpathlineto{\pgfqpoint{9.649354in}{5.273288in}}%
\pgfpathlineto{\pgfqpoint{9.680222in}{5.283275in}}%
\pgfpathlineto{\pgfqpoint{9.711089in}{5.290475in}}%
\pgfpathlineto{\pgfqpoint{9.741956in}{5.294686in}}%
\pgfpathlineto{\pgfqpoint{9.741956in}{5.294686in}}%
\pgfusepath{stroke}%
\end{pgfscope}%
\begin{pgfscope}%
\pgfpathrectangle{\pgfqpoint{5.706832in}{3.881603in}}{\pgfqpoint{4.227273in}{2.800000in}} %
\pgfusepath{clip}%
\pgfsetrectcap%
\pgfsetroundjoin%
\pgfsetlinewidth{0.501875pt}%
\definecolor{currentstroke}{rgb}{1.000000,0.243914,0.122888}%
\pgfsetstrokecolor{currentstroke}%
\pgfsetdash{}{0pt}%
\pgfpathmoveto{\pgfqpoint{5.898981in}{5.547402in}}%
\pgfpathlineto{\pgfqpoint{5.914415in}{5.544436in}}%
\pgfpathlineto{\pgfqpoint{5.929848in}{5.538790in}}%
\pgfpathlineto{\pgfqpoint{5.945282in}{5.530421in}}%
\pgfpathlineto{\pgfqpoint{5.960715in}{5.519329in}}%
\pgfpathlineto{\pgfqpoint{5.976149in}{5.505565in}}%
\pgfpathlineto{\pgfqpoint{5.991583in}{5.489225in}}%
\pgfpathlineto{\pgfqpoint{6.007016in}{5.470453in}}%
\pgfpathlineto{\pgfqpoint{6.022450in}{5.449440in}}%
\pgfpathlineto{\pgfqpoint{6.053317in}{5.401662in}}%
\pgfpathlineto{\pgfqpoint{6.084185in}{5.348229in}}%
\pgfpathlineto{\pgfqpoint{6.161353in}{5.209727in}}%
\pgfpathlineto{\pgfqpoint{6.192220in}{5.161354in}}%
\pgfpathlineto{\pgfqpoint{6.207654in}{5.140437in}}%
\pgfpathlineto{\pgfqpoint{6.223087in}{5.122256in}}%
\pgfpathlineto{\pgfqpoint{6.238521in}{5.107204in}}%
\pgfpathlineto{\pgfqpoint{6.253955in}{5.095637in}}%
\pgfpathlineto{\pgfqpoint{6.269388in}{5.087873in}}%
\pgfpathlineto{\pgfqpoint{6.284822in}{5.084184in}}%
\pgfpathlineto{\pgfqpoint{6.300255in}{5.084790in}}%
\pgfpathlineto{\pgfqpoint{6.315689in}{5.089851in}}%
\pgfpathlineto{\pgfqpoint{6.331123in}{5.099466in}}%
\pgfpathlineto{\pgfqpoint{6.346556in}{5.113665in}}%
\pgfpathlineto{\pgfqpoint{6.361990in}{5.132409in}}%
\pgfpathlineto{\pgfqpoint{6.377424in}{5.155590in}}%
\pgfpathlineto{\pgfqpoint{6.392857in}{5.183027in}}%
\pgfpathlineto{\pgfqpoint{6.408291in}{5.214474in}}%
\pgfpathlineto{\pgfqpoint{6.423724in}{5.249616in}}%
\pgfpathlineto{\pgfqpoint{6.454592in}{5.329435in}}%
\pgfpathlineto{\pgfqpoint{6.485459in}{5.418879in}}%
\pgfpathlineto{\pgfqpoint{6.593494in}{5.744528in}}%
\pgfpathlineto{\pgfqpoint{6.624362in}{5.822925in}}%
\pgfpathlineto{\pgfqpoint{6.639795in}{5.857268in}}%
\pgfpathlineto{\pgfqpoint{6.655229in}{5.887888in}}%
\pgfpathlineto{\pgfqpoint{6.670663in}{5.914487in}}%
\pgfpathlineto{\pgfqpoint{6.686096in}{5.936826in}}%
\pgfpathlineto{\pgfqpoint{6.701530in}{5.954729in}}%
\pgfpathlineto{\pgfqpoint{6.716964in}{5.968076in}}%
\pgfpathlineto{\pgfqpoint{6.732397in}{5.976810in}}%
\pgfpathlineto{\pgfqpoint{6.747831in}{5.980929in}}%
\pgfpathlineto{\pgfqpoint{6.763264in}{5.980488in}}%
\pgfpathlineto{\pgfqpoint{6.778698in}{5.975594in}}%
\pgfpathlineto{\pgfqpoint{6.794132in}{5.966399in}}%
\pgfpathlineto{\pgfqpoint{6.809565in}{5.953101in}}%
\pgfpathlineto{\pgfqpoint{6.824999in}{5.935934in}}%
\pgfpathlineto{\pgfqpoint{6.840433in}{5.915163in}}%
\pgfpathlineto{\pgfqpoint{6.855866in}{5.891080in}}%
\pgfpathlineto{\pgfqpoint{6.871300in}{5.863998in}}%
\pgfpathlineto{\pgfqpoint{6.902167in}{5.802160in}}%
\pgfpathlineto{\pgfqpoint{6.933034in}{5.732359in}}%
\pgfpathlineto{\pgfqpoint{6.979335in}{5.618670in}}%
\pgfpathlineto{\pgfqpoint{7.056504in}{5.426204in}}%
\pgfpathlineto{\pgfqpoint{7.087371in}{5.354627in}}%
\pgfpathlineto{\pgfqpoint{7.118238in}{5.288921in}}%
\pgfpathlineto{\pgfqpoint{7.149105in}{5.230566in}}%
\pgfpathlineto{\pgfqpoint{7.164539in}{5.204519in}}%
\pgfpathlineto{\pgfqpoint{7.179973in}{5.180710in}}%
\pgfpathlineto{\pgfqpoint{7.195406in}{5.159221in}}%
\pgfpathlineto{\pgfqpoint{7.210840in}{5.140101in}}%
\pgfpathlineto{\pgfqpoint{7.226274in}{5.123368in}}%
\pgfpathlineto{\pgfqpoint{7.241707in}{5.109005in}}%
\pgfpathlineto{\pgfqpoint{7.257141in}{5.096959in}}%
\pgfpathlineto{\pgfqpoint{7.272574in}{5.087141in}}%
\pgfpathlineto{\pgfqpoint{7.288008in}{5.079424in}}%
\pgfpathlineto{\pgfqpoint{7.303442in}{5.073647in}}%
\pgfpathlineto{\pgfqpoint{7.318875in}{5.069614in}}%
\pgfpathlineto{\pgfqpoint{7.334309in}{5.067100in}}%
\pgfpathlineto{\pgfqpoint{7.365176in}{5.065612in}}%
\pgfpathlineto{\pgfqpoint{7.411477in}{5.068040in}}%
\pgfpathlineto{\pgfqpoint{7.442344in}{5.069490in}}%
\pgfpathlineto{\pgfqpoint{7.473212in}{5.068498in}}%
\pgfpathlineto{\pgfqpoint{7.504079in}{5.063638in}}%
\pgfpathlineto{\pgfqpoint{7.519513in}{5.059497in}}%
\pgfpathlineto{\pgfqpoint{7.550380in}{5.047699in}}%
\pgfpathlineto{\pgfqpoint{7.581247in}{5.031666in}}%
\pgfpathlineto{\pgfqpoint{7.612114in}{5.012599in}}%
\pgfpathlineto{\pgfqpoint{7.673849in}{4.973119in}}%
\pgfpathlineto{\pgfqpoint{7.704716in}{4.957430in}}%
\pgfpathlineto{\pgfqpoint{7.720150in}{4.951663in}}%
\pgfpathlineto{\pgfqpoint{7.735584in}{4.947651in}}%
\pgfpathlineto{\pgfqpoint{7.751017in}{4.945638in}}%
\pgfpathlineto{\pgfqpoint{7.766451in}{4.945833in}}%
\pgfpathlineto{\pgfqpoint{7.781884in}{4.948402in}}%
\pgfpathlineto{\pgfqpoint{7.797318in}{4.953460in}}%
\pgfpathlineto{\pgfqpoint{7.812752in}{4.961072in}}%
\pgfpathlineto{\pgfqpoint{7.828185in}{4.971245in}}%
\pgfpathlineto{\pgfqpoint{7.843619in}{4.983927in}}%
\pgfpathlineto{\pgfqpoint{7.859053in}{4.999009in}}%
\pgfpathlineto{\pgfqpoint{7.874486in}{5.016320in}}%
\pgfpathlineto{\pgfqpoint{7.905353in}{5.056673in}}%
\pgfpathlineto{\pgfqpoint{7.936221in}{5.102585in}}%
\pgfpathlineto{\pgfqpoint{7.997955in}{5.198560in}}%
\pgfpathlineto{\pgfqpoint{8.028823in}{5.241784in}}%
\pgfpathlineto{\pgfqpoint{8.044256in}{5.260775in}}%
\pgfpathlineto{\pgfqpoint{8.059690in}{5.277551in}}%
\pgfpathlineto{\pgfqpoint{8.075123in}{5.291817in}}%
\pgfpathlineto{\pgfqpoint{8.090557in}{5.303333in}}%
\pgfpathlineto{\pgfqpoint{8.105991in}{5.311918in}}%
\pgfpathlineto{\pgfqpoint{8.121424in}{5.317451in}}%
\pgfpathlineto{\pgfqpoint{8.136858in}{5.319876in}}%
\pgfpathlineto{\pgfqpoint{8.152292in}{5.319198in}}%
\pgfpathlineto{\pgfqpoint{8.167725in}{5.315482in}}%
\pgfpathlineto{\pgfqpoint{8.183159in}{5.308850in}}%
\pgfpathlineto{\pgfqpoint{8.198593in}{5.299471in}}%
\pgfpathlineto{\pgfqpoint{8.214026in}{5.287558in}}%
\pgfpathlineto{\pgfqpoint{8.229460in}{5.273358in}}%
\pgfpathlineto{\pgfqpoint{8.244893in}{5.257142in}}%
\pgfpathlineto{\pgfqpoint{8.275761in}{5.219829in}}%
\pgfpathlineto{\pgfqpoint{8.306628in}{5.177981in}}%
\pgfpathlineto{\pgfqpoint{8.414663in}{5.025217in}}%
\pgfpathlineto{\pgfqpoint{8.445531in}{4.985609in}}%
\pgfpathlineto{\pgfqpoint{8.476398in}{4.949179in}}%
\pgfpathlineto{\pgfqpoint{8.507265in}{4.916190in}}%
\pgfpathlineto{\pgfqpoint{8.538133in}{4.886773in}}%
\pgfpathlineto{\pgfqpoint{8.569000in}{4.861041in}}%
\pgfpathlineto{\pgfqpoint{8.599867in}{4.839186in}}%
\pgfpathlineto{\pgfqpoint{8.630734in}{4.821537in}}%
\pgfpathlineto{\pgfqpoint{8.646168in}{4.814434in}}%
\pgfpathlineto{\pgfqpoint{8.661602in}{4.808574in}}%
\pgfpathlineto{\pgfqpoint{8.677035in}{4.804036in}}%
\pgfpathlineto{\pgfqpoint{8.692469in}{4.800904in}}%
\pgfpathlineto{\pgfqpoint{8.707903in}{4.799259in}}%
\pgfpathlineto{\pgfqpoint{8.723336in}{4.799184in}}%
\pgfpathlineto{\pgfqpoint{8.738770in}{4.800751in}}%
\pgfpathlineto{\pgfqpoint{8.754203in}{4.804022in}}%
\pgfpathlineto{\pgfqpoint{8.769637in}{4.809044in}}%
\pgfpathlineto{\pgfqpoint{8.785071in}{4.815844in}}%
\pgfpathlineto{\pgfqpoint{8.800504in}{4.824424in}}%
\pgfpathlineto{\pgfqpoint{8.815938in}{4.834760in}}%
\pgfpathlineto{\pgfqpoint{8.831372in}{4.846799in}}%
\pgfpathlineto{\pgfqpoint{8.846805in}{4.860455in}}%
\pgfpathlineto{\pgfqpoint{8.877673in}{4.892117in}}%
\pgfpathlineto{\pgfqpoint{8.908540in}{4.928433in}}%
\pgfpathlineto{\pgfqpoint{8.954841in}{4.987721in}}%
\pgfpathlineto{\pgfqpoint{9.001142in}{5.046548in}}%
\pgfpathlineto{\pgfqpoint{9.032009in}{5.082082in}}%
\pgfpathlineto{\pgfqpoint{9.062876in}{5.112691in}}%
\pgfpathlineto{\pgfqpoint{9.078310in}{5.125773in}}%
\pgfpathlineto{\pgfqpoint{9.093743in}{5.137253in}}%
\pgfpathlineto{\pgfqpoint{9.109177in}{5.147090in}}%
\pgfpathlineto{\pgfqpoint{9.124611in}{5.155286in}}%
\pgfpathlineto{\pgfqpoint{9.140044in}{5.161884in}}%
\pgfpathlineto{\pgfqpoint{9.155478in}{5.166964in}}%
\pgfpathlineto{\pgfqpoint{9.170912in}{5.170639in}}%
\pgfpathlineto{\pgfqpoint{9.201779in}{5.174364in}}%
\pgfpathlineto{\pgfqpoint{9.232646in}{5.174437in}}%
\pgfpathlineto{\pgfqpoint{9.278947in}{5.171091in}}%
\pgfpathlineto{\pgfqpoint{9.325248in}{5.168014in}}%
\pgfpathlineto{\pgfqpoint{9.356115in}{5.168077in}}%
\pgfpathlineto{\pgfqpoint{9.386982in}{5.170713in}}%
\pgfpathlineto{\pgfqpoint{9.417850in}{5.176256in}}%
\pgfpathlineto{\pgfqpoint{9.448717in}{5.184680in}}%
\pgfpathlineto{\pgfqpoint{9.479584in}{5.195650in}}%
\pgfpathlineto{\pgfqpoint{9.525885in}{5.215554in}}%
\pgfpathlineto{\pgfqpoint{9.618487in}{5.257764in}}%
\pgfpathlineto{\pgfqpoint{9.649354in}{5.269560in}}%
\pgfpathlineto{\pgfqpoint{9.680222in}{5.279159in}}%
\pgfpathlineto{\pgfqpoint{9.711089in}{5.286220in}}%
\pgfpathlineto{\pgfqpoint{9.741956in}{5.290640in}}%
\pgfpathlineto{\pgfqpoint{9.741956in}{5.290640in}}%
\pgfusepath{stroke}%
\end{pgfscope}%
\begin{pgfscope}%
\pgfpathrectangle{\pgfqpoint{5.706832in}{3.881603in}}{\pgfqpoint{4.227273in}{2.800000in}} %
\pgfusepath{clip}%
\pgfsetrectcap%
\pgfsetroundjoin%
\pgfsetlinewidth{0.501875pt}%
\definecolor{currentstroke}{rgb}{1.000000,0.122888,0.061561}%
\pgfsetstrokecolor{currentstroke}%
\pgfsetdash{}{0pt}%
\pgfpathmoveto{\pgfqpoint{5.898981in}{5.550754in}}%
\pgfpathlineto{\pgfqpoint{5.914415in}{5.548040in}}%
\pgfpathlineto{\pgfqpoint{5.929848in}{5.542683in}}%
\pgfpathlineto{\pgfqpoint{5.945282in}{5.534638in}}%
\pgfpathlineto{\pgfqpoint{5.960715in}{5.523909in}}%
\pgfpathlineto{\pgfqpoint{5.976149in}{5.510541in}}%
\pgfpathlineto{\pgfqpoint{5.991583in}{5.494629in}}%
\pgfpathlineto{\pgfqpoint{6.007016in}{5.476312in}}%
\pgfpathlineto{\pgfqpoint{6.022450in}{5.455773in}}%
\pgfpathlineto{\pgfqpoint{6.053317in}{5.408975in}}%
\pgfpathlineto{\pgfqpoint{6.084185in}{5.356507in}}%
\pgfpathlineto{\pgfqpoint{6.161353in}{5.219852in}}%
\pgfpathlineto{\pgfqpoint{6.192220in}{5.171806in}}%
\pgfpathlineto{\pgfqpoint{6.207654in}{5.150933in}}%
\pgfpathlineto{\pgfqpoint{6.223087in}{5.132713in}}%
\pgfpathlineto{\pgfqpoint{6.238521in}{5.117532in}}%
\pgfpathlineto{\pgfqpoint{6.253955in}{5.105747in}}%
\pgfpathlineto{\pgfqpoint{6.269388in}{5.097678in}}%
\pgfpathlineto{\pgfqpoint{6.284822in}{5.093601in}}%
\pgfpathlineto{\pgfqpoint{6.300255in}{5.093740in}}%
\pgfpathlineto{\pgfqpoint{6.315689in}{5.098264in}}%
\pgfpathlineto{\pgfqpoint{6.331123in}{5.107283in}}%
\pgfpathlineto{\pgfqpoint{6.346556in}{5.120839in}}%
\pgfpathlineto{\pgfqpoint{6.361990in}{5.138907in}}%
\pgfpathlineto{\pgfqpoint{6.377424in}{5.161394in}}%
\pgfpathlineto{\pgfqpoint{6.392857in}{5.188136in}}%
\pgfpathlineto{\pgfqpoint{6.408291in}{5.218900in}}%
\pgfpathlineto{\pgfqpoint{6.423724in}{5.253387in}}%
\pgfpathlineto{\pgfqpoint{6.439158in}{5.291237in}}%
\pgfpathlineto{\pgfqpoint{6.470025in}{5.375305in}}%
\pgfpathlineto{\pgfqpoint{6.500893in}{5.467214in}}%
\pgfpathlineto{\pgfqpoint{6.578061in}{5.701769in}}%
\pgfpathlineto{\pgfqpoint{6.608928in}{5.785565in}}%
\pgfpathlineto{\pgfqpoint{6.624362in}{5.823247in}}%
\pgfpathlineto{\pgfqpoint{6.639795in}{5.857547in}}%
\pgfpathlineto{\pgfqpoint{6.655229in}{5.888102in}}%
\pgfpathlineto{\pgfqpoint{6.670663in}{5.914606in}}%
\pgfpathlineto{\pgfqpoint{6.686096in}{5.936816in}}%
\pgfpathlineto{\pgfqpoint{6.701530in}{5.954550in}}%
\pgfpathlineto{\pgfqpoint{6.716964in}{5.967691in}}%
\pgfpathlineto{\pgfqpoint{6.732397in}{5.976181in}}%
\pgfpathlineto{\pgfqpoint{6.747831in}{5.980024in}}%
\pgfpathlineto{\pgfqpoint{6.763264in}{5.979283in}}%
\pgfpathlineto{\pgfqpoint{6.778698in}{5.974070in}}%
\pgfpathlineto{\pgfqpoint{6.794132in}{5.964549in}}%
\pgfpathlineto{\pgfqpoint{6.809565in}{5.950926in}}%
\pgfpathlineto{\pgfqpoint{6.824999in}{5.933444in}}%
\pgfpathlineto{\pgfqpoint{6.840433in}{5.912377in}}%
\pgfpathlineto{\pgfqpoint{6.855866in}{5.888025in}}%
\pgfpathlineto{\pgfqpoint{6.871300in}{5.860708in}}%
\pgfpathlineto{\pgfqpoint{6.902167in}{5.798515in}}%
\pgfpathlineto{\pgfqpoint{6.933034in}{5.728533in}}%
\pgfpathlineto{\pgfqpoint{6.979335in}{5.614896in}}%
\pgfpathlineto{\pgfqpoint{7.056504in}{5.423266in}}%
\pgfpathlineto{\pgfqpoint{7.087371in}{5.352245in}}%
\pgfpathlineto{\pgfqpoint{7.118238in}{5.287205in}}%
\pgfpathlineto{\pgfqpoint{7.149105in}{5.229609in}}%
\pgfpathlineto{\pgfqpoint{7.164539in}{5.203970in}}%
\pgfpathlineto{\pgfqpoint{7.179973in}{5.180582in}}%
\pgfpathlineto{\pgfqpoint{7.195406in}{5.159523in}}%
\pgfpathlineto{\pgfqpoint{7.210840in}{5.140835in}}%
\pgfpathlineto{\pgfqpoint{7.226274in}{5.124532in}}%
\pgfpathlineto{\pgfqpoint{7.241707in}{5.110590in}}%
\pgfpathlineto{\pgfqpoint{7.257141in}{5.098950in}}%
\pgfpathlineto{\pgfqpoint{7.272574in}{5.089516in}}%
\pgfpathlineto{\pgfqpoint{7.288008in}{5.082158in}}%
\pgfpathlineto{\pgfqpoint{7.303442in}{5.076708in}}%
\pgfpathlineto{\pgfqpoint{7.318875in}{5.072969in}}%
\pgfpathlineto{\pgfqpoint{7.334309in}{5.070712in}}%
\pgfpathlineto{\pgfqpoint{7.365176in}{5.069626in}}%
\pgfpathlineto{\pgfqpoint{7.411477in}{5.072392in}}%
\pgfpathlineto{\pgfqpoint{7.442344in}{5.073926in}}%
\pgfpathlineto{\pgfqpoint{7.473212in}{5.072937in}}%
\pgfpathlineto{\pgfqpoint{7.504079in}{5.068030in}}%
\pgfpathlineto{\pgfqpoint{7.519513in}{5.063854in}}%
\pgfpathlineto{\pgfqpoint{7.550380in}{5.051966in}}%
\pgfpathlineto{\pgfqpoint{7.581247in}{5.035821in}}%
\pgfpathlineto{\pgfqpoint{7.612114in}{5.016612in}}%
\pgfpathlineto{\pgfqpoint{7.673849in}{4.976680in}}%
\pgfpathlineto{\pgfqpoint{7.704716in}{4.960626in}}%
\pgfpathlineto{\pgfqpoint{7.720150in}{4.954627in}}%
\pgfpathlineto{\pgfqpoint{7.735584in}{4.950345in}}%
\pgfpathlineto{\pgfqpoint{7.751017in}{4.948019in}}%
\pgfpathlineto{\pgfqpoint{7.766451in}{4.947853in}}%
\pgfpathlineto{\pgfqpoint{7.781884in}{4.950009in}}%
\pgfpathlineto{\pgfqpoint{7.797318in}{4.954601in}}%
\pgfpathlineto{\pgfqpoint{7.812752in}{4.961689in}}%
\pgfpathlineto{\pgfqpoint{7.828185in}{4.971281in}}%
\pgfpathlineto{\pgfqpoint{7.843619in}{4.983325in}}%
\pgfpathlineto{\pgfqpoint{7.859053in}{4.997713in}}%
\pgfpathlineto{\pgfqpoint{7.874486in}{5.014281in}}%
\pgfpathlineto{\pgfqpoint{7.905353in}{5.053025in}}%
\pgfpathlineto{\pgfqpoint{7.936221in}{5.097232in}}%
\pgfpathlineto{\pgfqpoint{8.013389in}{5.211646in}}%
\pgfpathlineto{\pgfqpoint{8.044256in}{5.250391in}}%
\pgfpathlineto{\pgfqpoint{8.059690in}{5.266758in}}%
\pgfpathlineto{\pgfqpoint{8.075123in}{5.280723in}}%
\pgfpathlineto{\pgfqpoint{8.090557in}{5.292049in}}%
\pgfpathlineto{\pgfqpoint{8.105991in}{5.300551in}}%
\pgfpathlineto{\pgfqpoint{8.121424in}{5.306106in}}%
\pgfpathlineto{\pgfqpoint{8.136858in}{5.308649in}}%
\pgfpathlineto{\pgfqpoint{8.152292in}{5.308176in}}%
\pgfpathlineto{\pgfqpoint{8.167725in}{5.304738in}}%
\pgfpathlineto{\pgfqpoint{8.183159in}{5.298442in}}%
\pgfpathlineto{\pgfqpoint{8.198593in}{5.289443in}}%
\pgfpathlineto{\pgfqpoint{8.214026in}{5.277938in}}%
\pgfpathlineto{\pgfqpoint{8.229460in}{5.264160in}}%
\pgfpathlineto{\pgfqpoint{8.244893in}{5.248369in}}%
\pgfpathlineto{\pgfqpoint{8.275761in}{5.211870in}}%
\pgfpathlineto{\pgfqpoint{8.306628in}{5.170749in}}%
\pgfpathlineto{\pgfqpoint{8.414663in}{5.019974in}}%
\pgfpathlineto{\pgfqpoint{8.445531in}{4.980965in}}%
\pgfpathlineto{\pgfqpoint{8.476398in}{4.945226in}}%
\pgfpathlineto{\pgfqpoint{8.507265in}{4.913034in}}%
\pgfpathlineto{\pgfqpoint{8.538133in}{4.884509in}}%
\pgfpathlineto{\pgfqpoint{8.569000in}{4.859743in}}%
\pgfpathlineto{\pgfqpoint{8.599867in}{4.838900in}}%
\pgfpathlineto{\pgfqpoint{8.630734in}{4.822278in}}%
\pgfpathlineto{\pgfqpoint{8.646168in}{4.815687in}}%
\pgfpathlineto{\pgfqpoint{8.661602in}{4.810332in}}%
\pgfpathlineto{\pgfqpoint{8.677035in}{4.806289in}}%
\pgfpathlineto{\pgfqpoint{8.692469in}{4.803636in}}%
\pgfpathlineto{\pgfqpoint{8.707903in}{4.802454in}}%
\pgfpathlineto{\pgfqpoint{8.723336in}{4.802814in}}%
\pgfpathlineto{\pgfqpoint{8.738770in}{4.804784in}}%
\pgfpathlineto{\pgfqpoint{8.754203in}{4.808417in}}%
\pgfpathlineto{\pgfqpoint{8.769637in}{4.813750in}}%
\pgfpathlineto{\pgfqpoint{8.785071in}{4.820799in}}%
\pgfpathlineto{\pgfqpoint{8.800504in}{4.829556in}}%
\pgfpathlineto{\pgfqpoint{8.815938in}{4.839985in}}%
\pgfpathlineto{\pgfqpoint{8.831372in}{4.852021in}}%
\pgfpathlineto{\pgfqpoint{8.862239in}{4.880504in}}%
\pgfpathlineto{\pgfqpoint{8.893106in}{4.913880in}}%
\pgfpathlineto{\pgfqpoint{8.939407in}{4.969607in}}%
\pgfpathlineto{\pgfqpoint{8.985708in}{5.026316in}}%
\pgfpathlineto{\pgfqpoint{9.016575in}{5.061356in}}%
\pgfpathlineto{\pgfqpoint{9.047443in}{5.092176in}}%
\pgfpathlineto{\pgfqpoint{9.078310in}{5.117549in}}%
\pgfpathlineto{\pgfqpoint{9.093743in}{5.127967in}}%
\pgfpathlineto{\pgfqpoint{9.109177in}{5.136831in}}%
\pgfpathlineto{\pgfqpoint{9.124611in}{5.144158in}}%
\pgfpathlineto{\pgfqpoint{9.140044in}{5.150004in}}%
\pgfpathlineto{\pgfqpoint{9.155478in}{5.154458in}}%
\pgfpathlineto{\pgfqpoint{9.186345in}{5.159698in}}%
\pgfpathlineto{\pgfqpoint{9.217212in}{5.161117in}}%
\pgfpathlineto{\pgfqpoint{9.263513in}{5.159306in}}%
\pgfpathlineto{\pgfqpoint{9.309814in}{5.157163in}}%
\pgfpathlineto{\pgfqpoint{9.340682in}{5.157585in}}%
\pgfpathlineto{\pgfqpoint{9.371549in}{5.160445in}}%
\pgfpathlineto{\pgfqpoint{9.402416in}{5.166141in}}%
\pgfpathlineto{\pgfqpoint{9.433283in}{5.174701in}}%
\pgfpathlineto{\pgfqpoint{9.464151in}{5.185831in}}%
\pgfpathlineto{\pgfqpoint{9.510452in}{5.206089in}}%
\pgfpathlineto{\pgfqpoint{9.618487in}{5.256385in}}%
\pgfpathlineto{\pgfqpoint{9.649354in}{5.268120in}}%
\pgfpathlineto{\pgfqpoint{9.680222in}{5.277663in}}%
\pgfpathlineto{\pgfqpoint{9.711089in}{5.284757in}}%
\pgfpathlineto{\pgfqpoint{9.741956in}{5.289344in}}%
\pgfpathlineto{\pgfqpoint{9.741956in}{5.289344in}}%
\pgfusepath{stroke}%
\end{pgfscope}%
\begin{pgfscope}%
\pgfpathrectangle{\pgfqpoint{5.706832in}{3.881603in}}{\pgfqpoint{4.227273in}{2.800000in}} %
\pgfusepath{clip}%
\pgfsetrectcap%
\pgfsetroundjoin%
\pgfsetlinewidth{0.501875pt}%
\definecolor{currentstroke}{rgb}{0.000000,0.000000,0.000000}%
\pgfsetstrokecolor{currentstroke}%
\pgfsetdash{}{0pt}%
\pgfpathmoveto{\pgfqpoint{5.898981in}{5.549443in}}%
\pgfpathlineto{\pgfqpoint{5.914415in}{5.547213in}}%
\pgfpathlineto{\pgfqpoint{5.929848in}{5.542336in}}%
\pgfpathlineto{\pgfqpoint{5.945282in}{5.534750in}}%
\pgfpathlineto{\pgfqpoint{5.960715in}{5.524438in}}%
\pgfpathlineto{\pgfqpoint{5.976149in}{5.511431in}}%
\pgfpathlineto{\pgfqpoint{5.991583in}{5.495809in}}%
\pgfpathlineto{\pgfqpoint{6.007016in}{5.477704in}}%
\pgfpathlineto{\pgfqpoint{6.022450in}{5.457295in}}%
\pgfpathlineto{\pgfqpoint{6.053317in}{5.410514in}}%
\pgfpathlineto{\pgfqpoint{6.084185in}{5.357785in}}%
\pgfpathlineto{\pgfqpoint{6.161353in}{5.219897in}}%
\pgfpathlineto{\pgfqpoint{6.192220in}{5.171371in}}%
\pgfpathlineto{\pgfqpoint{6.207654in}{5.150289in}}%
\pgfpathlineto{\pgfqpoint{6.223087in}{5.131882in}}%
\pgfpathlineto{\pgfqpoint{6.238521in}{5.116536in}}%
\pgfpathlineto{\pgfqpoint{6.253955in}{5.104603in}}%
\pgfpathlineto{\pgfqpoint{6.269388in}{5.096399in}}%
\pgfpathlineto{\pgfqpoint{6.284822in}{5.092193in}}%
\pgfpathlineto{\pgfqpoint{6.300255in}{5.092205in}}%
\pgfpathlineto{\pgfqpoint{6.315689in}{5.096600in}}%
\pgfpathlineto{\pgfqpoint{6.331123in}{5.105482in}}%
\pgfpathlineto{\pgfqpoint{6.346556in}{5.118893in}}%
\pgfpathlineto{\pgfqpoint{6.361990in}{5.136807in}}%
\pgfpathlineto{\pgfqpoint{6.377424in}{5.159131in}}%
\pgfpathlineto{\pgfqpoint{6.392857in}{5.185705in}}%
\pgfpathlineto{\pgfqpoint{6.408291in}{5.216298in}}%
\pgfpathlineto{\pgfqpoint{6.423724in}{5.250616in}}%
\pgfpathlineto{\pgfqpoint{6.439158in}{5.288305in}}%
\pgfpathlineto{\pgfqpoint{6.470025in}{5.372089in}}%
\pgfpathlineto{\pgfqpoint{6.500893in}{5.463786in}}%
\pgfpathlineto{\pgfqpoint{6.578061in}{5.698155in}}%
\pgfpathlineto{\pgfqpoint{6.608928in}{5.781976in}}%
\pgfpathlineto{\pgfqpoint{6.624362in}{5.819685in}}%
\pgfpathlineto{\pgfqpoint{6.639795in}{5.854021in}}%
\pgfpathlineto{\pgfqpoint{6.655229in}{5.884625in}}%
\pgfpathlineto{\pgfqpoint{6.670663in}{5.911196in}}%
\pgfpathlineto{\pgfqpoint{6.686096in}{5.933495in}}%
\pgfpathlineto{\pgfqpoint{6.701530in}{5.951349in}}%
\pgfpathlineto{\pgfqpoint{6.716964in}{5.964648in}}%
\pgfpathlineto{\pgfqpoint{6.732397in}{5.973344in}}%
\pgfpathlineto{\pgfqpoint{6.747831in}{5.977450in}}%
\pgfpathlineto{\pgfqpoint{6.763264in}{5.977035in}}%
\pgfpathlineto{\pgfqpoint{6.778698in}{5.972220in}}%
\pgfpathlineto{\pgfqpoint{6.794132in}{5.963172in}}%
\pgfpathlineto{\pgfqpoint{6.809565in}{5.950099in}}%
\pgfpathlineto{\pgfqpoint{6.824999in}{5.933241in}}%
\pgfpathlineto{\pgfqpoint{6.840433in}{5.912869in}}%
\pgfpathlineto{\pgfqpoint{6.855866in}{5.889273in}}%
\pgfpathlineto{\pgfqpoint{6.871300in}{5.862760in}}%
\pgfpathlineto{\pgfqpoint{6.902167in}{5.802257in}}%
\pgfpathlineto{\pgfqpoint{6.933034in}{5.733944in}}%
\pgfpathlineto{\pgfqpoint{6.979335in}{5.622393in}}%
\pgfpathlineto{\pgfqpoint{7.056504in}{5.431988in}}%
\pgfpathlineto{\pgfqpoint{7.087371in}{5.360494in}}%
\pgfpathlineto{\pgfqpoint{7.118238in}{5.294485in}}%
\pgfpathlineto{\pgfqpoint{7.149105in}{5.235533in}}%
\pgfpathlineto{\pgfqpoint{7.164539in}{5.209112in}}%
\pgfpathlineto{\pgfqpoint{7.179973in}{5.184900in}}%
\pgfpathlineto{\pgfqpoint{7.195406in}{5.162988in}}%
\pgfpathlineto{\pgfqpoint{7.210840in}{5.143439in}}%
\pgfpathlineto{\pgfqpoint{7.226274in}{5.126279in}}%
\pgfpathlineto{\pgfqpoint{7.241707in}{5.111498in}}%
\pgfpathlineto{\pgfqpoint{7.257141in}{5.099048in}}%
\pgfpathlineto{\pgfqpoint{7.272574in}{5.088842in}}%
\pgfpathlineto{\pgfqpoint{7.288008in}{5.080753in}}%
\pgfpathlineto{\pgfqpoint{7.303442in}{5.074621in}}%
\pgfpathlineto{\pgfqpoint{7.318875in}{5.070249in}}%
\pgfpathlineto{\pgfqpoint{7.334309in}{5.067413in}}%
\pgfpathlineto{\pgfqpoint{7.365176in}{5.065324in}}%
\pgfpathlineto{\pgfqpoint{7.411477in}{5.067003in}}%
\pgfpathlineto{\pgfqpoint{7.442344in}{5.068117in}}%
\pgfpathlineto{\pgfqpoint{7.473212in}{5.066999in}}%
\pgfpathlineto{\pgfqpoint{7.504079in}{5.062301in}}%
\pgfpathlineto{\pgfqpoint{7.534946in}{5.053352in}}%
\pgfpathlineto{\pgfqpoint{7.565814in}{5.040233in}}%
\pgfpathlineto{\pgfqpoint{7.596681in}{5.023767in}}%
\pgfpathlineto{\pgfqpoint{7.673849in}{4.978810in}}%
\pgfpathlineto{\pgfqpoint{7.704716in}{4.964962in}}%
\pgfpathlineto{\pgfqpoint{7.720150in}{4.960057in}}%
\pgfpathlineto{\pgfqpoint{7.735584in}{4.956843in}}%
\pgfpathlineto{\pgfqpoint{7.751017in}{4.955542in}}%
\pgfpathlineto{\pgfqpoint{7.766451in}{4.956347in}}%
\pgfpathlineto{\pgfqpoint{7.781884in}{4.959407in}}%
\pgfpathlineto{\pgfqpoint{7.797318in}{4.964824in}}%
\pgfpathlineto{\pgfqpoint{7.812752in}{4.972652in}}%
\pgfpathlineto{\pgfqpoint{7.828185in}{4.982890in}}%
\pgfpathlineto{\pgfqpoint{7.843619in}{4.995482in}}%
\pgfpathlineto{\pgfqpoint{7.859053in}{5.010314in}}%
\pgfpathlineto{\pgfqpoint{7.874486in}{5.027218in}}%
\pgfpathlineto{\pgfqpoint{7.905353in}{5.066298in}}%
\pgfpathlineto{\pgfqpoint{7.936221in}{5.110375in}}%
\pgfpathlineto{\pgfqpoint{7.997955in}{5.201378in}}%
\pgfpathlineto{\pgfqpoint{8.028823in}{5.241709in}}%
\pgfpathlineto{\pgfqpoint{8.044256in}{5.259210in}}%
\pgfpathlineto{\pgfqpoint{8.059690in}{5.274491in}}%
\pgfpathlineto{\pgfqpoint{8.075123in}{5.287273in}}%
\pgfpathlineto{\pgfqpoint{8.090557in}{5.297334in}}%
\pgfpathlineto{\pgfqpoint{8.105991in}{5.304507in}}%
\pgfpathlineto{\pgfqpoint{8.121424in}{5.308688in}}%
\pgfpathlineto{\pgfqpoint{8.136858in}{5.309836in}}%
\pgfpathlineto{\pgfqpoint{8.152292in}{5.307969in}}%
\pgfpathlineto{\pgfqpoint{8.167725in}{5.303167in}}%
\pgfpathlineto{\pgfqpoint{8.183159in}{5.295561in}}%
\pgfpathlineto{\pgfqpoint{8.198593in}{5.285335in}}%
\pgfpathlineto{\pgfqpoint{8.214026in}{5.272711in}}%
\pgfpathlineto{\pgfqpoint{8.229460in}{5.257944in}}%
\pgfpathlineto{\pgfqpoint{8.260327in}{5.223120in}}%
\pgfpathlineto{\pgfqpoint{8.291194in}{5.183238in}}%
\pgfpathlineto{\pgfqpoint{8.414663in}{5.015565in}}%
\pgfpathlineto{\pgfqpoint{8.445531in}{4.978842in}}%
\pgfpathlineto{\pgfqpoint{8.476398in}{4.945552in}}%
\pgfpathlineto{\pgfqpoint{8.507265in}{4.915791in}}%
\pgfpathlineto{\pgfqpoint{8.538133in}{4.889521in}}%
\pgfpathlineto{\pgfqpoint{8.569000in}{4.866708in}}%
\pgfpathlineto{\pgfqpoint{8.599867in}{4.847428in}}%
\pgfpathlineto{\pgfqpoint{8.630734in}{4.831936in}}%
\pgfpathlineto{\pgfqpoint{8.661602in}{4.820683in}}%
\pgfpathlineto{\pgfqpoint{8.677035in}{4.816835in}}%
\pgfpathlineto{\pgfqpoint{8.692469in}{4.814288in}}%
\pgfpathlineto{\pgfqpoint{8.707903in}{4.813130in}}%
\pgfpathlineto{\pgfqpoint{8.723336in}{4.813449in}}%
\pgfpathlineto{\pgfqpoint{8.738770in}{4.815325in}}%
\pgfpathlineto{\pgfqpoint{8.754203in}{4.818827in}}%
\pgfpathlineto{\pgfqpoint{8.769637in}{4.824006in}}%
\pgfpathlineto{\pgfqpoint{8.785071in}{4.830895in}}%
\pgfpathlineto{\pgfqpoint{8.800504in}{4.839499in}}%
\pgfpathlineto{\pgfqpoint{8.815938in}{4.849798in}}%
\pgfpathlineto{\pgfqpoint{8.831372in}{4.861738in}}%
\pgfpathlineto{\pgfqpoint{8.862239in}{4.890169in}}%
\pgfpathlineto{\pgfqpoint{8.893106in}{4.923717in}}%
\pgfpathlineto{\pgfqpoint{8.923973in}{4.960830in}}%
\pgfpathlineto{\pgfqpoint{9.001142in}{5.056268in}}%
\pgfpathlineto{\pgfqpoint{9.032009in}{5.090166in}}%
\pgfpathlineto{\pgfqpoint{9.062876in}{5.119059in}}%
\pgfpathlineto{\pgfqpoint{9.078310in}{5.131279in}}%
\pgfpathlineto{\pgfqpoint{9.093743in}{5.141903in}}%
\pgfpathlineto{\pgfqpoint{9.109177in}{5.150900in}}%
\pgfpathlineto{\pgfqpoint{9.124611in}{5.158278in}}%
\pgfpathlineto{\pgfqpoint{9.140044in}{5.164084in}}%
\pgfpathlineto{\pgfqpoint{9.155478in}{5.168400in}}%
\pgfpathlineto{\pgfqpoint{9.170912in}{5.171342in}}%
\pgfpathlineto{\pgfqpoint{9.201779in}{5.173693in}}%
\pgfpathlineto{\pgfqpoint{9.232646in}{5.172509in}}%
\pgfpathlineto{\pgfqpoint{9.294381in}{5.165703in}}%
\pgfpathlineto{\pgfqpoint{9.325248in}{5.162994in}}%
\pgfpathlineto{\pgfqpoint{9.356115in}{5.162306in}}%
\pgfpathlineto{\pgfqpoint{9.386982in}{5.164390in}}%
\pgfpathlineto{\pgfqpoint{9.417850in}{5.169615in}}%
\pgfpathlineto{\pgfqpoint{9.448717in}{5.177984in}}%
\pgfpathlineto{\pgfqpoint{9.479584in}{5.189164in}}%
\pgfpathlineto{\pgfqpoint{9.510452in}{5.202558in}}%
\pgfpathlineto{\pgfqpoint{9.572186in}{5.232746in}}%
\pgfpathlineto{\pgfqpoint{9.618487in}{5.254888in}}%
\pgfpathlineto{\pgfqpoint{9.649354in}{5.267914in}}%
\pgfpathlineto{\pgfqpoint{9.680222in}{5.278867in}}%
\pgfpathlineto{\pgfqpoint{9.711089in}{5.287404in}}%
\pgfpathlineto{\pgfqpoint{9.741956in}{5.293410in}}%
\pgfpathlineto{\pgfqpoint{9.741956in}{5.293410in}}%
\pgfusepath{stroke}%
\end{pgfscope}%
\begin{pgfscope}%
\pgfsetrectcap%
\pgfsetmiterjoin%
\pgfsetlinewidth{0.803000pt}%
\definecolor{currentstroke}{rgb}{0.000000,0.000000,0.000000}%
\pgfsetstrokecolor{currentstroke}%
\pgfsetdash{}{0pt}%
\pgfpathmoveto{\pgfqpoint{5.706832in}{3.881603in}}%
\pgfpathlineto{\pgfqpoint{5.706832in}{6.681603in}}%
\pgfusepath{stroke}%
\end{pgfscope}%
\begin{pgfscope}%
\pgfsetrectcap%
\pgfsetmiterjoin%
\pgfsetlinewidth{0.803000pt}%
\definecolor{currentstroke}{rgb}{0.000000,0.000000,0.000000}%
\pgfsetstrokecolor{currentstroke}%
\pgfsetdash{}{0pt}%
\pgfpathmoveto{\pgfqpoint{9.934105in}{3.881603in}}%
\pgfpathlineto{\pgfqpoint{9.934105in}{6.681603in}}%
\pgfusepath{stroke}%
\end{pgfscope}%
\begin{pgfscope}%
\pgfsetrectcap%
\pgfsetmiterjoin%
\pgfsetlinewidth{0.803000pt}%
\definecolor{currentstroke}{rgb}{0.000000,0.000000,0.000000}%
\pgfsetstrokecolor{currentstroke}%
\pgfsetdash{}{0pt}%
\pgfpathmoveto{\pgfqpoint{5.706832in}{3.881603in}}%
\pgfpathlineto{\pgfqpoint{9.934105in}{3.881603in}}%
\pgfusepath{stroke}%
\end{pgfscope}%
\begin{pgfscope}%
\pgfsetrectcap%
\pgfsetmiterjoin%
\pgfsetlinewidth{0.803000pt}%
\definecolor{currentstroke}{rgb}{0.000000,0.000000,0.000000}%
\pgfsetstrokecolor{currentstroke}%
\pgfsetdash{}{0pt}%
\pgfpathmoveto{\pgfqpoint{5.706832in}{6.681603in}}%
\pgfpathlineto{\pgfqpoint{9.934105in}{6.681603in}}%
\pgfusepath{stroke}%
\end{pgfscope}%
\begin{pgfscope}%
\pgfsetbuttcap%
\pgfsetmiterjoin%
\definecolor{currentfill}{rgb}{1.000000,1.000000,1.000000}%
\pgfsetfillcolor{currentfill}%
\pgfsetfillopacity{0.800000}%
\pgfsetlinewidth{1.003750pt}%
\definecolor{currentstroke}{rgb}{0.800000,0.800000,0.800000}%
\pgfsetstrokecolor{currentstroke}%
\pgfsetstrokeopacity{0.800000}%
\pgfsetdash{}{0pt}%
\pgfpathmoveto{\pgfqpoint{9.288678in}{3.951048in}}%
\pgfpathlineto{\pgfqpoint{9.836883in}{3.951048in}}%
\pgfpathquadraticcurveto{\pgfqpoint{9.864660in}{3.951048in}}{\pgfqpoint{9.864660in}{3.978826in}}%
\pgfpathlineto{\pgfqpoint{9.864660in}{4.168794in}}%
\pgfpathquadraticcurveto{\pgfqpoint{9.864660in}{4.196572in}}{\pgfqpoint{9.836883in}{4.196572in}}%
\pgfpathlineto{\pgfqpoint{9.288678in}{4.196572in}}%
\pgfpathquadraticcurveto{\pgfqpoint{9.260901in}{4.196572in}}{\pgfqpoint{9.260901in}{4.168794in}}%
\pgfpathlineto{\pgfqpoint{9.260901in}{3.978826in}}%
\pgfpathquadraticcurveto{\pgfqpoint{9.260901in}{3.951048in}}{\pgfqpoint{9.288678in}{3.951048in}}%
\pgfpathclose%
\pgfusepath{stroke,fill}%
\end{pgfscope}%
\begin{pgfscope}%
\pgfsetrectcap%
\pgfsetroundjoin%
\pgfsetlinewidth{0.501875pt}%
\definecolor{currentstroke}{rgb}{0.000000,0.000000,0.000000}%
\pgfsetstrokecolor{currentstroke}%
\pgfsetdash{}{0pt}%
\pgfpathmoveto{\pgfqpoint{9.316456in}{4.084104in}}%
\pgfpathlineto{\pgfqpoint{9.594234in}{4.084104in}}%
\pgfusepath{stroke}%
\end{pgfscope}%
\begin{pgfscope}%
\pgftext[x=9.705345in,y=4.035493in,left,base]{\rmfamily\fontsize{10.000000}{12.000000}\selectfont K}%
\end{pgfscope}%
\begin{pgfscope}%
\pgfsetbuttcap%
\pgfsetmiterjoin%
\definecolor{currentfill}{rgb}{1.000000,1.000000,1.000000}%
\pgfsetfillcolor{currentfill}%
\pgfsetlinewidth{0.000000pt}%
\definecolor{currentstroke}{rgb}{0.000000,0.000000,0.000000}%
\pgfsetstrokecolor{currentstroke}%
\pgfsetstrokeopacity{0.000000}%
\pgfsetdash{}{0pt}%
\pgfpathmoveto{\pgfqpoint{0.634105in}{0.521603in}}%
\pgfpathlineto{\pgfqpoint{4.861378in}{0.521603in}}%
\pgfpathlineto{\pgfqpoint{4.861378in}{3.321603in}}%
\pgfpathlineto{\pgfqpoint{0.634105in}{3.321603in}}%
\pgfpathclose%
\pgfusepath{fill}%
\end{pgfscope}%
\begin{pgfscope}%
\pgfsetbuttcap%
\pgfsetroundjoin%
\definecolor{currentfill}{rgb}{0.000000,0.000000,0.000000}%
\pgfsetfillcolor{currentfill}%
\pgfsetlinewidth{0.803000pt}%
\definecolor{currentstroke}{rgb}{0.000000,0.000000,0.000000}%
\pgfsetstrokecolor{currentstroke}%
\pgfsetdash{}{0pt}%
\pgfsys@defobject{currentmarker}{\pgfqpoint{0.000000in}{-0.048611in}}{\pgfqpoint{0.000000in}{0.000000in}}{%
\pgfpathmoveto{\pgfqpoint{0.000000in}{0.000000in}}%
\pgfpathlineto{\pgfqpoint{0.000000in}{-0.048611in}}%
\pgfusepath{stroke,fill}%
}%
\begin{pgfscope}%
\pgfsys@transformshift{0.826254in}{0.521603in}%
\pgfsys@useobject{currentmarker}{}%
\end{pgfscope}%
\end{pgfscope}%
\begin{pgfscope}%
\pgftext[x=0.826254in,y=0.424381in,,top]{\rmfamily\fontsize{10.000000}{12.000000}\selectfont \(\displaystyle -1.00\)}%
\end{pgfscope}%
\begin{pgfscope}%
\pgfsetbuttcap%
\pgfsetroundjoin%
\definecolor{currentfill}{rgb}{0.000000,0.000000,0.000000}%
\pgfsetfillcolor{currentfill}%
\pgfsetlinewidth{0.803000pt}%
\definecolor{currentstroke}{rgb}{0.000000,0.000000,0.000000}%
\pgfsetstrokecolor{currentstroke}%
\pgfsetdash{}{0pt}%
\pgfsys@defobject{currentmarker}{\pgfqpoint{0.000000in}{-0.048611in}}{\pgfqpoint{0.000000in}{0.000000in}}{%
\pgfpathmoveto{\pgfqpoint{0.000000in}{0.000000in}}%
\pgfpathlineto{\pgfqpoint{0.000000in}{-0.048611in}}%
\pgfusepath{stroke,fill}%
}%
\begin{pgfscope}%
\pgfsys@transformshift{1.306626in}{0.521603in}%
\pgfsys@useobject{currentmarker}{}%
\end{pgfscope}%
\end{pgfscope}%
\begin{pgfscope}%
\pgftext[x=1.306626in,y=0.424381in,,top]{\rmfamily\fontsize{10.000000}{12.000000}\selectfont \(\displaystyle -0.75\)}%
\end{pgfscope}%
\begin{pgfscope}%
\pgfsetbuttcap%
\pgfsetroundjoin%
\definecolor{currentfill}{rgb}{0.000000,0.000000,0.000000}%
\pgfsetfillcolor{currentfill}%
\pgfsetlinewidth{0.803000pt}%
\definecolor{currentstroke}{rgb}{0.000000,0.000000,0.000000}%
\pgfsetstrokecolor{currentstroke}%
\pgfsetdash{}{0pt}%
\pgfsys@defobject{currentmarker}{\pgfqpoint{0.000000in}{-0.048611in}}{\pgfqpoint{0.000000in}{0.000000in}}{%
\pgfpathmoveto{\pgfqpoint{0.000000in}{0.000000in}}%
\pgfpathlineto{\pgfqpoint{0.000000in}{-0.048611in}}%
\pgfusepath{stroke,fill}%
}%
\begin{pgfscope}%
\pgfsys@transformshift{1.786997in}{0.521603in}%
\pgfsys@useobject{currentmarker}{}%
\end{pgfscope}%
\end{pgfscope}%
\begin{pgfscope}%
\pgftext[x=1.786997in,y=0.424381in,,top]{\rmfamily\fontsize{10.000000}{12.000000}\selectfont \(\displaystyle -0.50\)}%
\end{pgfscope}%
\begin{pgfscope}%
\pgfsetbuttcap%
\pgfsetroundjoin%
\definecolor{currentfill}{rgb}{0.000000,0.000000,0.000000}%
\pgfsetfillcolor{currentfill}%
\pgfsetlinewidth{0.803000pt}%
\definecolor{currentstroke}{rgb}{0.000000,0.000000,0.000000}%
\pgfsetstrokecolor{currentstroke}%
\pgfsetdash{}{0pt}%
\pgfsys@defobject{currentmarker}{\pgfqpoint{0.000000in}{-0.048611in}}{\pgfqpoint{0.000000in}{0.000000in}}{%
\pgfpathmoveto{\pgfqpoint{0.000000in}{0.000000in}}%
\pgfpathlineto{\pgfqpoint{0.000000in}{-0.048611in}}%
\pgfusepath{stroke,fill}%
}%
\begin{pgfscope}%
\pgfsys@transformshift{2.267369in}{0.521603in}%
\pgfsys@useobject{currentmarker}{}%
\end{pgfscope}%
\end{pgfscope}%
\begin{pgfscope}%
\pgftext[x=2.267369in,y=0.424381in,,top]{\rmfamily\fontsize{10.000000}{12.000000}\selectfont \(\displaystyle -0.25\)}%
\end{pgfscope}%
\begin{pgfscope}%
\pgfsetbuttcap%
\pgfsetroundjoin%
\definecolor{currentfill}{rgb}{0.000000,0.000000,0.000000}%
\pgfsetfillcolor{currentfill}%
\pgfsetlinewidth{0.803000pt}%
\definecolor{currentstroke}{rgb}{0.000000,0.000000,0.000000}%
\pgfsetstrokecolor{currentstroke}%
\pgfsetdash{}{0pt}%
\pgfsys@defobject{currentmarker}{\pgfqpoint{0.000000in}{-0.048611in}}{\pgfqpoint{0.000000in}{0.000000in}}{%
\pgfpathmoveto{\pgfqpoint{0.000000in}{0.000000in}}%
\pgfpathlineto{\pgfqpoint{0.000000in}{-0.048611in}}%
\pgfusepath{stroke,fill}%
}%
\begin{pgfscope}%
\pgfsys@transformshift{2.747741in}{0.521603in}%
\pgfsys@useobject{currentmarker}{}%
\end{pgfscope}%
\end{pgfscope}%
\begin{pgfscope}%
\pgftext[x=2.747741in,y=0.424381in,,top]{\rmfamily\fontsize{10.000000}{12.000000}\selectfont \(\displaystyle 0.00\)}%
\end{pgfscope}%
\begin{pgfscope}%
\pgfsetbuttcap%
\pgfsetroundjoin%
\definecolor{currentfill}{rgb}{0.000000,0.000000,0.000000}%
\pgfsetfillcolor{currentfill}%
\pgfsetlinewidth{0.803000pt}%
\definecolor{currentstroke}{rgb}{0.000000,0.000000,0.000000}%
\pgfsetstrokecolor{currentstroke}%
\pgfsetdash{}{0pt}%
\pgfsys@defobject{currentmarker}{\pgfqpoint{0.000000in}{-0.048611in}}{\pgfqpoint{0.000000in}{0.000000in}}{%
\pgfpathmoveto{\pgfqpoint{0.000000in}{0.000000in}}%
\pgfpathlineto{\pgfqpoint{0.000000in}{-0.048611in}}%
\pgfusepath{stroke,fill}%
}%
\begin{pgfscope}%
\pgfsys@transformshift{3.228113in}{0.521603in}%
\pgfsys@useobject{currentmarker}{}%
\end{pgfscope}%
\end{pgfscope}%
\begin{pgfscope}%
\pgftext[x=3.228113in,y=0.424381in,,top]{\rmfamily\fontsize{10.000000}{12.000000}\selectfont \(\displaystyle 0.25\)}%
\end{pgfscope}%
\begin{pgfscope}%
\pgfsetbuttcap%
\pgfsetroundjoin%
\definecolor{currentfill}{rgb}{0.000000,0.000000,0.000000}%
\pgfsetfillcolor{currentfill}%
\pgfsetlinewidth{0.803000pt}%
\definecolor{currentstroke}{rgb}{0.000000,0.000000,0.000000}%
\pgfsetstrokecolor{currentstroke}%
\pgfsetdash{}{0pt}%
\pgfsys@defobject{currentmarker}{\pgfqpoint{0.000000in}{-0.048611in}}{\pgfqpoint{0.000000in}{0.000000in}}{%
\pgfpathmoveto{\pgfqpoint{0.000000in}{0.000000in}}%
\pgfpathlineto{\pgfqpoint{0.000000in}{-0.048611in}}%
\pgfusepath{stroke,fill}%
}%
\begin{pgfscope}%
\pgfsys@transformshift{3.708485in}{0.521603in}%
\pgfsys@useobject{currentmarker}{}%
\end{pgfscope}%
\end{pgfscope}%
\begin{pgfscope}%
\pgftext[x=3.708485in,y=0.424381in,,top]{\rmfamily\fontsize{10.000000}{12.000000}\selectfont \(\displaystyle 0.50\)}%
\end{pgfscope}%
\begin{pgfscope}%
\pgfsetbuttcap%
\pgfsetroundjoin%
\definecolor{currentfill}{rgb}{0.000000,0.000000,0.000000}%
\pgfsetfillcolor{currentfill}%
\pgfsetlinewidth{0.803000pt}%
\definecolor{currentstroke}{rgb}{0.000000,0.000000,0.000000}%
\pgfsetstrokecolor{currentstroke}%
\pgfsetdash{}{0pt}%
\pgfsys@defobject{currentmarker}{\pgfqpoint{0.000000in}{-0.048611in}}{\pgfqpoint{0.000000in}{0.000000in}}{%
\pgfpathmoveto{\pgfqpoint{0.000000in}{0.000000in}}%
\pgfpathlineto{\pgfqpoint{0.000000in}{-0.048611in}}%
\pgfusepath{stroke,fill}%
}%
\begin{pgfscope}%
\pgfsys@transformshift{4.188857in}{0.521603in}%
\pgfsys@useobject{currentmarker}{}%
\end{pgfscope}%
\end{pgfscope}%
\begin{pgfscope}%
\pgftext[x=4.188857in,y=0.424381in,,top]{\rmfamily\fontsize{10.000000}{12.000000}\selectfont \(\displaystyle 0.75\)}%
\end{pgfscope}%
\begin{pgfscope}%
\pgfsetbuttcap%
\pgfsetroundjoin%
\definecolor{currentfill}{rgb}{0.000000,0.000000,0.000000}%
\pgfsetfillcolor{currentfill}%
\pgfsetlinewidth{0.803000pt}%
\definecolor{currentstroke}{rgb}{0.000000,0.000000,0.000000}%
\pgfsetstrokecolor{currentstroke}%
\pgfsetdash{}{0pt}%
\pgfsys@defobject{currentmarker}{\pgfqpoint{0.000000in}{-0.048611in}}{\pgfqpoint{0.000000in}{0.000000in}}{%
\pgfpathmoveto{\pgfqpoint{0.000000in}{0.000000in}}%
\pgfpathlineto{\pgfqpoint{0.000000in}{-0.048611in}}%
\pgfusepath{stroke,fill}%
}%
\begin{pgfscope}%
\pgfsys@transformshift{4.669229in}{0.521603in}%
\pgfsys@useobject{currentmarker}{}%
\end{pgfscope}%
\end{pgfscope}%
\begin{pgfscope}%
\pgftext[x=4.669229in,y=0.424381in,,top]{\rmfamily\fontsize{10.000000}{12.000000}\selectfont \(\displaystyle 1.00\)}%
\end{pgfscope}%
\begin{pgfscope}%
\pgftext[x=2.747741in,y=0.234413in,,top]{\rmfamily\fontsize{10.000000}{12.000000}\selectfont \(\displaystyle x\)}%
\end{pgfscope}%
\begin{pgfscope}%
\pgfsetbuttcap%
\pgfsetroundjoin%
\definecolor{currentfill}{rgb}{0.000000,0.000000,0.000000}%
\pgfsetfillcolor{currentfill}%
\pgfsetlinewidth{0.803000pt}%
\definecolor{currentstroke}{rgb}{0.000000,0.000000,0.000000}%
\pgfsetstrokecolor{currentstroke}%
\pgfsetdash{}{0pt}%
\pgfsys@defobject{currentmarker}{\pgfqpoint{-0.048611in}{0.000000in}}{\pgfqpoint{0.000000in}{0.000000in}}{%
\pgfpathmoveto{\pgfqpoint{0.000000in}{0.000000in}}%
\pgfpathlineto{\pgfqpoint{-0.048611in}{0.000000in}}%
\pgfusepath{stroke,fill}%
}%
\begin{pgfscope}%
\pgfsys@transformshift{0.634105in}{0.975954in}%
\pgfsys@useobject{currentmarker}{}%
\end{pgfscope}%
\end{pgfscope}%
\begin{pgfscope}%
\pgftext[x=0.289968in,y=0.923193in,left,base]{\rmfamily\fontsize{10.000000}{12.000000}\selectfont \(\displaystyle -20\)}%
\end{pgfscope}%
\begin{pgfscope}%
\pgfsetbuttcap%
\pgfsetroundjoin%
\definecolor{currentfill}{rgb}{0.000000,0.000000,0.000000}%
\pgfsetfillcolor{currentfill}%
\pgfsetlinewidth{0.803000pt}%
\definecolor{currentstroke}{rgb}{0.000000,0.000000,0.000000}%
\pgfsetstrokecolor{currentstroke}%
\pgfsetdash{}{0pt}%
\pgfsys@defobject{currentmarker}{\pgfqpoint{-0.048611in}{0.000000in}}{\pgfqpoint{0.000000in}{0.000000in}}{%
\pgfpathmoveto{\pgfqpoint{0.000000in}{0.000000in}}%
\pgfpathlineto{\pgfqpoint{-0.048611in}{0.000000in}}%
\pgfusepath{stroke,fill}%
}%
\begin{pgfscope}%
\pgfsys@transformshift{0.634105in}{1.445404in}%
\pgfsys@useobject{currentmarker}{}%
\end{pgfscope}%
\end{pgfscope}%
\begin{pgfscope}%
\pgftext[x=0.289968in,y=1.392642in,left,base]{\rmfamily\fontsize{10.000000}{12.000000}\selectfont \(\displaystyle -10\)}%
\end{pgfscope}%
\begin{pgfscope}%
\pgfsetbuttcap%
\pgfsetroundjoin%
\definecolor{currentfill}{rgb}{0.000000,0.000000,0.000000}%
\pgfsetfillcolor{currentfill}%
\pgfsetlinewidth{0.803000pt}%
\definecolor{currentstroke}{rgb}{0.000000,0.000000,0.000000}%
\pgfsetstrokecolor{currentstroke}%
\pgfsetdash{}{0pt}%
\pgfsys@defobject{currentmarker}{\pgfqpoint{-0.048611in}{0.000000in}}{\pgfqpoint{0.000000in}{0.000000in}}{%
\pgfpathmoveto{\pgfqpoint{0.000000in}{0.000000in}}%
\pgfpathlineto{\pgfqpoint{-0.048611in}{0.000000in}}%
\pgfusepath{stroke,fill}%
}%
\begin{pgfscope}%
\pgfsys@transformshift{0.634105in}{1.914853in}%
\pgfsys@useobject{currentmarker}{}%
\end{pgfscope}%
\end{pgfscope}%
\begin{pgfscope}%
\pgftext[x=0.467438in,y=1.862091in,left,base]{\rmfamily\fontsize{10.000000}{12.000000}\selectfont \(\displaystyle 0\)}%
\end{pgfscope}%
\begin{pgfscope}%
\pgfsetbuttcap%
\pgfsetroundjoin%
\definecolor{currentfill}{rgb}{0.000000,0.000000,0.000000}%
\pgfsetfillcolor{currentfill}%
\pgfsetlinewidth{0.803000pt}%
\definecolor{currentstroke}{rgb}{0.000000,0.000000,0.000000}%
\pgfsetstrokecolor{currentstroke}%
\pgfsetdash{}{0pt}%
\pgfsys@defobject{currentmarker}{\pgfqpoint{-0.048611in}{0.000000in}}{\pgfqpoint{0.000000in}{0.000000in}}{%
\pgfpathmoveto{\pgfqpoint{0.000000in}{0.000000in}}%
\pgfpathlineto{\pgfqpoint{-0.048611in}{0.000000in}}%
\pgfusepath{stroke,fill}%
}%
\begin{pgfscope}%
\pgfsys@transformshift{0.634105in}{2.384302in}%
\pgfsys@useobject{currentmarker}{}%
\end{pgfscope}%
\end{pgfscope}%
\begin{pgfscope}%
\pgftext[x=0.397993in,y=2.331541in,left,base]{\rmfamily\fontsize{10.000000}{12.000000}\selectfont \(\displaystyle 10\)}%
\end{pgfscope}%
\begin{pgfscope}%
\pgfsetbuttcap%
\pgfsetroundjoin%
\definecolor{currentfill}{rgb}{0.000000,0.000000,0.000000}%
\pgfsetfillcolor{currentfill}%
\pgfsetlinewidth{0.803000pt}%
\definecolor{currentstroke}{rgb}{0.000000,0.000000,0.000000}%
\pgfsetstrokecolor{currentstroke}%
\pgfsetdash{}{0pt}%
\pgfsys@defobject{currentmarker}{\pgfqpoint{-0.048611in}{0.000000in}}{\pgfqpoint{0.000000in}{0.000000in}}{%
\pgfpathmoveto{\pgfqpoint{0.000000in}{0.000000in}}%
\pgfpathlineto{\pgfqpoint{-0.048611in}{0.000000in}}%
\pgfusepath{stroke,fill}%
}%
\begin{pgfscope}%
\pgfsys@transformshift{0.634105in}{2.853751in}%
\pgfsys@useobject{currentmarker}{}%
\end{pgfscope}%
\end{pgfscope}%
\begin{pgfscope}%
\pgftext[x=0.397993in,y=2.800990in,left,base]{\rmfamily\fontsize{10.000000}{12.000000}\selectfont \(\displaystyle 20\)}%
\end{pgfscope}%
\begin{pgfscope}%
\pgfsetbuttcap%
\pgfsetroundjoin%
\definecolor{currentfill}{rgb}{0.000000,0.000000,0.000000}%
\pgfsetfillcolor{currentfill}%
\pgfsetlinewidth{0.803000pt}%
\definecolor{currentstroke}{rgb}{0.000000,0.000000,0.000000}%
\pgfsetstrokecolor{currentstroke}%
\pgfsetdash{}{0pt}%
\pgfsys@defobject{currentmarker}{\pgfqpoint{-0.048611in}{0.000000in}}{\pgfqpoint{0.000000in}{0.000000in}}{%
\pgfpathmoveto{\pgfqpoint{0.000000in}{0.000000in}}%
\pgfpathlineto{\pgfqpoint{-0.048611in}{0.000000in}}%
\pgfusepath{stroke,fill}%
}%
\begin{pgfscope}%
\pgfsys@transformshift{0.634105in}{3.323200in}%
\pgfsys@useobject{currentmarker}{}%
\end{pgfscope}%
\end{pgfscope}%
\begin{pgfscope}%
\pgftext[x=0.397993in,y=3.270439in,left,base]{\rmfamily\fontsize{10.000000}{12.000000}\selectfont \(\displaystyle 30\)}%
\end{pgfscope}%
\begin{pgfscope}%
\pgftext[x=0.234413in,y=1.921603in,,bottom,rotate=90.000000]{\rmfamily\fontsize{10.000000}{12.000000}\selectfont \(\displaystyle y_1\)}%
\end{pgfscope}%
\begin{pgfscope}%
\pgfpathrectangle{\pgfqpoint{0.634105in}{0.521603in}}{\pgfqpoint{4.227273in}{2.800000in}} %
\pgfusepath{clip}%
\pgfsetrectcap%
\pgfsetroundjoin%
\pgfsetlinewidth{0.501875pt}%
\definecolor{currentstroke}{rgb}{0.500000,0.000000,1.000000}%
\pgfsetstrokecolor{currentstroke}%
\pgfsetdash{}{0pt}%
\pgfpathmoveto{\pgfqpoint{0.826254in}{1.886826in}}%
\pgfpathlineto{\pgfqpoint{0.965156in}{2.113340in}}%
\pgfpathlineto{\pgfqpoint{1.026891in}{2.209805in}}%
\pgfpathlineto{\pgfqpoint{1.073192in}{2.279043in}}%
\pgfpathlineto{\pgfqpoint{1.119493in}{2.344898in}}%
\pgfpathlineto{\pgfqpoint{1.165794in}{2.406757in}}%
\pgfpathlineto{\pgfqpoint{1.212095in}{2.464046in}}%
\pgfpathlineto{\pgfqpoint{1.242962in}{2.499436in}}%
\pgfpathlineto{\pgfqpoint{1.273829in}{2.532410in}}%
\pgfpathlineto{\pgfqpoint{1.304696in}{2.562834in}}%
\pgfpathlineto{\pgfqpoint{1.335564in}{2.590581in}}%
\pgfpathlineto{\pgfqpoint{1.366431in}{2.615536in}}%
\pgfpathlineto{\pgfqpoint{1.397298in}{2.637598in}}%
\pgfpathlineto{\pgfqpoint{1.428165in}{2.656673in}}%
\pgfpathlineto{\pgfqpoint{1.459033in}{2.672685in}}%
\pgfpathlineto{\pgfqpoint{1.489900in}{2.685566in}}%
\pgfpathlineto{\pgfqpoint{1.520767in}{2.695263in}}%
\pgfpathlineto{\pgfqpoint{1.551634in}{2.701737in}}%
\pgfpathlineto{\pgfqpoint{1.582502in}{2.704960in}}%
\pgfpathlineto{\pgfqpoint{1.613369in}{2.704919in}}%
\pgfpathlineto{\pgfqpoint{1.644236in}{2.701615in}}%
\pgfpathlineto{\pgfqpoint{1.675104in}{2.695061in}}%
\pgfpathlineto{\pgfqpoint{1.705971in}{2.685284in}}%
\pgfpathlineto{\pgfqpoint{1.736838in}{2.672324in}}%
\pgfpathlineto{\pgfqpoint{1.767705in}{2.656236in}}%
\pgfpathlineto{\pgfqpoint{1.798573in}{2.637085in}}%
\pgfpathlineto{\pgfqpoint{1.829440in}{2.614951in}}%
\pgfpathlineto{\pgfqpoint{1.860307in}{2.589924in}}%
\pgfpathlineto{\pgfqpoint{1.891174in}{2.562110in}}%
\pgfpathlineto{\pgfqpoint{1.922042in}{2.531621in}}%
\pgfpathlineto{\pgfqpoint{1.952909in}{2.498585in}}%
\pgfpathlineto{\pgfqpoint{1.983776in}{2.463137in}}%
\pgfpathlineto{\pgfqpoint{2.030077in}{2.405767in}}%
\pgfpathlineto{\pgfqpoint{2.076378in}{2.343837in}}%
\pgfpathlineto{\pgfqpoint{2.122679in}{2.277921in}}%
\pgfpathlineto{\pgfqpoint{2.168980in}{2.208633in}}%
\pgfpathlineto{\pgfqpoint{2.230714in}{2.112117in}}%
\pgfpathlineto{\pgfqpoint{2.307883in}{1.987086in}}%
\pgfpathlineto{\pgfqpoint{2.477653in}{1.710068in}}%
\pgfpathlineto{\pgfqpoint{2.539387in}{1.613868in}}%
\pgfpathlineto{\pgfqpoint{2.585688in}{1.544894in}}%
\pgfpathlineto{\pgfqpoint{2.631989in}{1.479358in}}%
\pgfpathlineto{\pgfqpoint{2.678290in}{1.417867in}}%
\pgfpathlineto{\pgfqpoint{2.724591in}{1.360994in}}%
\pgfpathlineto{\pgfqpoint{2.755458in}{1.325906in}}%
\pgfpathlineto{\pgfqpoint{2.786325in}{1.293251in}}%
\pgfpathlineto{\pgfqpoint{2.817193in}{1.263163in}}%
\pgfpathlineto{\pgfqpoint{2.848060in}{1.235768in}}%
\pgfpathlineto{\pgfqpoint{2.878927in}{1.211178in}}%
\pgfpathlineto{\pgfqpoint{2.909794in}{1.189494in}}%
\pgfpathlineto{\pgfqpoint{2.940662in}{1.170807in}}%
\pgfpathlineto{\pgfqpoint{2.971529in}{1.155194in}}%
\pgfpathlineto{\pgfqpoint{3.002396in}{1.142718in}}%
\pgfpathlineto{\pgfqpoint{3.033263in}{1.133432in}}%
\pgfpathlineto{\pgfqpoint{3.064131in}{1.127374in}}%
\pgfpathlineto{\pgfqpoint{3.094998in}{1.124568in}}%
\pgfpathlineto{\pgfqpoint{3.125865in}{1.125028in}}%
\pgfpathlineto{\pgfqpoint{3.156733in}{1.128749in}}%
\pgfpathlineto{\pgfqpoint{3.187600in}{1.135718in}}%
\pgfpathlineto{\pgfqpoint{3.218467in}{1.145906in}}%
\pgfpathlineto{\pgfqpoint{3.249334in}{1.159270in}}%
\pgfpathlineto{\pgfqpoint{3.280202in}{1.175755in}}%
\pgfpathlineto{\pgfqpoint{3.311069in}{1.195293in}}%
\pgfpathlineto{\pgfqpoint{3.341936in}{1.217803in}}%
\pgfpathlineto{\pgfqpoint{3.372803in}{1.243193in}}%
\pgfpathlineto{\pgfqpoint{3.403671in}{1.271357in}}%
\pgfpathlineto{\pgfqpoint{3.434538in}{1.302179in}}%
\pgfpathlineto{\pgfqpoint{3.465405in}{1.335533in}}%
\pgfpathlineto{\pgfqpoint{3.496273in}{1.371279in}}%
\pgfpathlineto{\pgfqpoint{3.542573in}{1.429060in}}%
\pgfpathlineto{\pgfqpoint{3.588874in}{1.491354in}}%
\pgfpathlineto{\pgfqpoint{3.635175in}{1.557582in}}%
\pgfpathlineto{\pgfqpoint{3.696910in}{1.650939in}}%
\pgfpathlineto{\pgfqpoint{3.758644in}{1.748651in}}%
\pgfpathlineto{\pgfqpoint{3.851246in}{1.899847in}}%
\pgfpathlineto{\pgfqpoint{3.974715in}{2.101332in}}%
\pgfpathlineto{\pgfqpoint{4.036450in}{2.198282in}}%
\pgfpathlineto{\pgfqpoint{4.098184in}{2.290554in}}%
\pgfpathlineto{\pgfqpoint{4.144485in}{2.355768in}}%
\pgfpathlineto{\pgfqpoint{4.190786in}{2.416886in}}%
\pgfpathlineto{\pgfqpoint{4.237087in}{2.473340in}}%
\pgfpathlineto{\pgfqpoint{4.267954in}{2.508124in}}%
\pgfpathlineto{\pgfqpoint{4.298822in}{2.540458in}}%
\pgfpathlineto{\pgfqpoint{4.329689in}{2.570206in}}%
\pgfpathlineto{\pgfqpoint{4.360556in}{2.597248in}}%
\pgfpathlineto{\pgfqpoint{4.391423in}{2.621471in}}%
\pgfpathlineto{\pgfqpoint{4.422291in}{2.642775in}}%
\pgfpathlineto{\pgfqpoint{4.453158in}{2.661073in}}%
\pgfpathlineto{\pgfqpoint{4.484025in}{2.676287in}}%
\pgfpathlineto{\pgfqpoint{4.514892in}{2.688357in}}%
\pgfpathlineto{\pgfqpoint{4.545760in}{2.697231in}}%
\pgfpathlineto{\pgfqpoint{4.576627in}{2.702874in}}%
\pgfpathlineto{\pgfqpoint{4.607494in}{2.705261in}}%
\pgfpathlineto{\pgfqpoint{4.638362in}{2.704383in}}%
\pgfpathlineto{\pgfqpoint{4.669229in}{2.700244in}}%
\pgfpathlineto{\pgfqpoint{4.669229in}{2.700244in}}%
\pgfusepath{stroke}%
\end{pgfscope}%
\begin{pgfscope}%
\pgfpathrectangle{\pgfqpoint{0.634105in}{0.521603in}}{\pgfqpoint{4.227273in}{2.800000in}} %
\pgfusepath{clip}%
\pgfsetrectcap%
\pgfsetroundjoin%
\pgfsetlinewidth{0.501875pt}%
\definecolor{currentstroke}{rgb}{0.421569,0.122888,0.998103}%
\pgfsetstrokecolor{currentstroke}%
\pgfsetdash{}{0pt}%
\pgfpathmoveto{\pgfqpoint{0.826254in}{1.886826in}}%
\pgfpathlineto{\pgfqpoint{0.965156in}{2.113340in}}%
\pgfpathlineto{\pgfqpoint{1.026891in}{2.209805in}}%
\pgfpathlineto{\pgfqpoint{1.073192in}{2.279043in}}%
\pgfpathlineto{\pgfqpoint{1.119493in}{2.344898in}}%
\pgfpathlineto{\pgfqpoint{1.165794in}{2.406757in}}%
\pgfpathlineto{\pgfqpoint{1.212095in}{2.464046in}}%
\pgfpathlineto{\pgfqpoint{1.242962in}{2.499436in}}%
\pgfpathlineto{\pgfqpoint{1.273829in}{2.532410in}}%
\pgfpathlineto{\pgfqpoint{1.304696in}{2.562834in}}%
\pgfpathlineto{\pgfqpoint{1.335564in}{2.590581in}}%
\pgfpathlineto{\pgfqpoint{1.366431in}{2.615536in}}%
\pgfpathlineto{\pgfqpoint{1.397298in}{2.637598in}}%
\pgfpathlineto{\pgfqpoint{1.428165in}{2.656673in}}%
\pgfpathlineto{\pgfqpoint{1.459033in}{2.672685in}}%
\pgfpathlineto{\pgfqpoint{1.489900in}{2.685566in}}%
\pgfpathlineto{\pgfqpoint{1.520767in}{2.695263in}}%
\pgfpathlineto{\pgfqpoint{1.551634in}{2.701737in}}%
\pgfpathlineto{\pgfqpoint{1.582502in}{2.704960in}}%
\pgfpathlineto{\pgfqpoint{1.613369in}{2.704919in}}%
\pgfpathlineto{\pgfqpoint{1.644236in}{2.701615in}}%
\pgfpathlineto{\pgfqpoint{1.675104in}{2.695061in}}%
\pgfpathlineto{\pgfqpoint{1.705971in}{2.685284in}}%
\pgfpathlineto{\pgfqpoint{1.736838in}{2.672324in}}%
\pgfpathlineto{\pgfqpoint{1.767705in}{2.656236in}}%
\pgfpathlineto{\pgfqpoint{1.798573in}{2.637085in}}%
\pgfpathlineto{\pgfqpoint{1.829440in}{2.614951in}}%
\pgfpathlineto{\pgfqpoint{1.860307in}{2.589924in}}%
\pgfpathlineto{\pgfqpoint{1.891174in}{2.562110in}}%
\pgfpathlineto{\pgfqpoint{1.922042in}{2.531621in}}%
\pgfpathlineto{\pgfqpoint{1.952909in}{2.498585in}}%
\pgfpathlineto{\pgfqpoint{1.983776in}{2.463137in}}%
\pgfpathlineto{\pgfqpoint{2.030077in}{2.405767in}}%
\pgfpathlineto{\pgfqpoint{2.076378in}{2.343837in}}%
\pgfpathlineto{\pgfqpoint{2.122679in}{2.277921in}}%
\pgfpathlineto{\pgfqpoint{2.168980in}{2.208633in}}%
\pgfpathlineto{\pgfqpoint{2.230714in}{2.112117in}}%
\pgfpathlineto{\pgfqpoint{2.307883in}{1.987086in}}%
\pgfpathlineto{\pgfqpoint{2.477653in}{1.710068in}}%
\pgfpathlineto{\pgfqpoint{2.539387in}{1.613868in}}%
\pgfpathlineto{\pgfqpoint{2.585688in}{1.544894in}}%
\pgfpathlineto{\pgfqpoint{2.631989in}{1.479358in}}%
\pgfpathlineto{\pgfqpoint{2.678290in}{1.417867in}}%
\pgfpathlineto{\pgfqpoint{2.724591in}{1.360994in}}%
\pgfpathlineto{\pgfqpoint{2.755458in}{1.325906in}}%
\pgfpathlineto{\pgfqpoint{2.786325in}{1.293251in}}%
\pgfpathlineto{\pgfqpoint{2.817193in}{1.263163in}}%
\pgfpathlineto{\pgfqpoint{2.848060in}{1.235768in}}%
\pgfpathlineto{\pgfqpoint{2.878927in}{1.211178in}}%
\pgfpathlineto{\pgfqpoint{2.909794in}{1.189494in}}%
\pgfpathlineto{\pgfqpoint{2.940662in}{1.170807in}}%
\pgfpathlineto{\pgfqpoint{2.971529in}{1.155194in}}%
\pgfpathlineto{\pgfqpoint{3.002396in}{1.142718in}}%
\pgfpathlineto{\pgfqpoint{3.033263in}{1.133432in}}%
\pgfpathlineto{\pgfqpoint{3.064131in}{1.127374in}}%
\pgfpathlineto{\pgfqpoint{3.094998in}{1.124568in}}%
\pgfpathlineto{\pgfqpoint{3.125865in}{1.125028in}}%
\pgfpathlineto{\pgfqpoint{3.156733in}{1.128749in}}%
\pgfpathlineto{\pgfqpoint{3.187600in}{1.135718in}}%
\pgfpathlineto{\pgfqpoint{3.218467in}{1.145906in}}%
\pgfpathlineto{\pgfqpoint{3.249334in}{1.159270in}}%
\pgfpathlineto{\pgfqpoint{3.280202in}{1.175755in}}%
\pgfpathlineto{\pgfqpoint{3.311069in}{1.195293in}}%
\pgfpathlineto{\pgfqpoint{3.341936in}{1.217803in}}%
\pgfpathlineto{\pgfqpoint{3.372803in}{1.243193in}}%
\pgfpathlineto{\pgfqpoint{3.403671in}{1.271357in}}%
\pgfpathlineto{\pgfqpoint{3.434538in}{1.302179in}}%
\pgfpathlineto{\pgfqpoint{3.465405in}{1.335533in}}%
\pgfpathlineto{\pgfqpoint{3.496273in}{1.371279in}}%
\pgfpathlineto{\pgfqpoint{3.542573in}{1.429060in}}%
\pgfpathlineto{\pgfqpoint{3.588874in}{1.491354in}}%
\pgfpathlineto{\pgfqpoint{3.635175in}{1.557582in}}%
\pgfpathlineto{\pgfqpoint{3.696910in}{1.650939in}}%
\pgfpathlineto{\pgfqpoint{3.758644in}{1.748651in}}%
\pgfpathlineto{\pgfqpoint{3.851246in}{1.899847in}}%
\pgfpathlineto{\pgfqpoint{3.974715in}{2.101332in}}%
\pgfpathlineto{\pgfqpoint{4.036450in}{2.198282in}}%
\pgfpathlineto{\pgfqpoint{4.098184in}{2.290554in}}%
\pgfpathlineto{\pgfqpoint{4.144485in}{2.355768in}}%
\pgfpathlineto{\pgfqpoint{4.190786in}{2.416886in}}%
\pgfpathlineto{\pgfqpoint{4.237087in}{2.473340in}}%
\pgfpathlineto{\pgfqpoint{4.267954in}{2.508124in}}%
\pgfpathlineto{\pgfqpoint{4.298822in}{2.540458in}}%
\pgfpathlineto{\pgfqpoint{4.329689in}{2.570206in}}%
\pgfpathlineto{\pgfqpoint{4.360556in}{2.597248in}}%
\pgfpathlineto{\pgfqpoint{4.391423in}{2.621471in}}%
\pgfpathlineto{\pgfqpoint{4.422291in}{2.642775in}}%
\pgfpathlineto{\pgfqpoint{4.453158in}{2.661073in}}%
\pgfpathlineto{\pgfqpoint{4.484025in}{2.676287in}}%
\pgfpathlineto{\pgfqpoint{4.514892in}{2.688357in}}%
\pgfpathlineto{\pgfqpoint{4.545760in}{2.697231in}}%
\pgfpathlineto{\pgfqpoint{4.576627in}{2.702874in}}%
\pgfpathlineto{\pgfqpoint{4.607494in}{2.705261in}}%
\pgfpathlineto{\pgfqpoint{4.638362in}{2.704383in}}%
\pgfpathlineto{\pgfqpoint{4.669229in}{2.700244in}}%
\pgfpathlineto{\pgfqpoint{4.669229in}{2.700244in}}%
\pgfusepath{stroke}%
\end{pgfscope}%
\begin{pgfscope}%
\pgfpathrectangle{\pgfqpoint{0.634105in}{0.521603in}}{\pgfqpoint{4.227273in}{2.800000in}} %
\pgfusepath{clip}%
\pgfsetrectcap%
\pgfsetroundjoin%
\pgfsetlinewidth{0.501875pt}%
\definecolor{currentstroke}{rgb}{0.343137,0.243914,0.992421}%
\pgfsetstrokecolor{currentstroke}%
\pgfsetdash{}{0pt}%
\pgfpathmoveto{\pgfqpoint{0.826254in}{2.646020in}}%
\pgfpathlineto{\pgfqpoint{0.841687in}{2.680811in}}%
\pgfpathlineto{\pgfqpoint{0.857121in}{2.694276in}}%
\pgfpathlineto{\pgfqpoint{0.872555in}{2.685090in}}%
\pgfpathlineto{\pgfqpoint{0.887988in}{2.652643in}}%
\pgfpathlineto{\pgfqpoint{0.903422in}{2.597067in}}%
\pgfpathlineto{\pgfqpoint{0.918855in}{2.519238in}}%
\pgfpathlineto{\pgfqpoint{0.934289in}{2.420751in}}%
\pgfpathlineto{\pgfqpoint{0.949723in}{2.303875in}}%
\pgfpathlineto{\pgfqpoint{0.965156in}{2.171487in}}%
\pgfpathlineto{\pgfqpoint{0.996024in}{1.874174in}}%
\pgfpathlineto{\pgfqpoint{1.042325in}{1.407716in}}%
\pgfpathlineto{\pgfqpoint{1.057758in}{1.263799in}}%
\pgfpathlineto{\pgfqpoint{1.073192in}{1.132478in}}%
\pgfpathlineto{\pgfqpoint{1.088625in}{1.017385in}}%
\pgfpathlineto{\pgfqpoint{1.104059in}{0.921682in}}%
\pgfpathlineto{\pgfqpoint{1.119493in}{0.847964in}}%
\pgfpathlineto{\pgfqpoint{1.134926in}{0.798178in}}%
\pgfpathlineto{\pgfqpoint{1.150360in}{0.773564in}}%
\pgfpathlineto{\pgfqpoint{1.165794in}{0.774615in}}%
\pgfpathlineto{\pgfqpoint{1.181227in}{0.801064in}}%
\pgfpathlineto{\pgfqpoint{1.196661in}{0.851894in}}%
\pgfpathlineto{\pgfqpoint{1.212095in}{0.925372in}}%
\pgfpathlineto{\pgfqpoint{1.227528in}{1.019107in}}%
\pgfpathlineto{\pgfqpoint{1.242962in}{1.130124in}}%
\pgfpathlineto{\pgfqpoint{1.258395in}{1.254962in}}%
\pgfpathlineto{\pgfqpoint{1.289263in}{1.530505in}}%
\pgfpathlineto{\pgfqpoint{1.320130in}{1.812802in}}%
\pgfpathlineto{\pgfqpoint{1.335564in}{1.946127in}}%
\pgfpathlineto{\pgfqpoint{1.350997in}{2.069104in}}%
\pgfpathlineto{\pgfqpoint{1.366431in}{2.178344in}}%
\pgfpathlineto{\pgfqpoint{1.381864in}{2.270956in}}%
\pgfpathlineto{\pgfqpoint{1.397298in}{2.344636in}}%
\pgfpathlineto{\pgfqpoint{1.412732in}{2.397740in}}%
\pgfpathlineto{\pgfqpoint{1.428165in}{2.429326in}}%
\pgfpathlineto{\pgfqpoint{1.443599in}{2.439186in}}%
\pgfpathlineto{\pgfqpoint{1.459033in}{2.427847in}}%
\pgfpathlineto{\pgfqpoint{1.474466in}{2.396544in}}%
\pgfpathlineto{\pgfqpoint{1.489900in}{2.347180in}}%
\pgfpathlineto{\pgfqpoint{1.505334in}{2.282257in}}%
\pgfpathlineto{\pgfqpoint{1.520767in}{2.204789in}}%
\pgfpathlineto{\pgfqpoint{1.551634in}{2.026205in}}%
\pgfpathlineto{\pgfqpoint{1.582502in}{1.841575in}}%
\pgfpathlineto{\pgfqpoint{1.597935in}{1.756684in}}%
\pgfpathlineto{\pgfqpoint{1.613369in}{1.681617in}}%
\pgfpathlineto{\pgfqpoint{1.628803in}{1.619627in}}%
\pgfpathlineto{\pgfqpoint{1.644236in}{1.573511in}}%
\pgfpathlineto{\pgfqpoint{1.659670in}{1.545512in}}%
\pgfpathlineto{\pgfqpoint{1.675104in}{1.537247in}}%
\pgfpathlineto{\pgfqpoint{1.690537in}{1.549645in}}%
\pgfpathlineto{\pgfqpoint{1.705971in}{1.582914in}}%
\pgfpathlineto{\pgfqpoint{1.721404in}{1.636535in}}%
\pgfpathlineto{\pgfqpoint{1.736838in}{1.709270in}}%
\pgfpathlineto{\pgfqpoint{1.752272in}{1.799200in}}%
\pgfpathlineto{\pgfqpoint{1.767705in}{1.903787in}}%
\pgfpathlineto{\pgfqpoint{1.798573in}{2.144170in}}%
\pgfpathlineto{\pgfqpoint{1.844874in}{2.525705in}}%
\pgfpathlineto{\pgfqpoint{1.860307in}{2.642182in}}%
\pgfpathlineto{\pgfqpoint{1.875741in}{2.746688in}}%
\pgfpathlineto{\pgfqpoint{1.891174in}{2.835643in}}%
\pgfpathlineto{\pgfqpoint{1.906608in}{2.905905in}}%
\pgfpathlineto{\pgfqpoint{1.922042in}{2.954866in}}%
\pgfpathlineto{\pgfqpoint{1.937475in}{2.980538in}}%
\pgfpathlineto{\pgfqpoint{1.952909in}{2.981624in}}%
\pgfpathlineto{\pgfqpoint{1.968343in}{2.957559in}}%
\pgfpathlineto{\pgfqpoint{1.983776in}{2.908536in}}%
\pgfpathlineto{\pgfqpoint{1.999210in}{2.835497in}}%
\pgfpathlineto{\pgfqpoint{2.014644in}{2.740111in}}%
\pgfpathlineto{\pgfqpoint{2.030077in}{2.624724in}}%
\pgfpathlineto{\pgfqpoint{2.045511in}{2.492288in}}%
\pgfpathlineto{\pgfqpoint{2.076378in}{2.190533in}}%
\pgfpathlineto{\pgfqpoint{2.138113in}{1.555266in}}%
\pgfpathlineto{\pgfqpoint{2.153546in}{1.414543in}}%
\pgfpathlineto{\pgfqpoint{2.168980in}{1.288786in}}%
\pgfpathlineto{\pgfqpoint{2.184414in}{1.181145in}}%
\pgfpathlineto{\pgfqpoint{2.199847in}{1.094201in}}%
\pgfpathlineto{\pgfqpoint{2.215281in}{1.029882in}}%
\pgfpathlineto{\pgfqpoint{2.230714in}{0.989411in}}%
\pgfpathlineto{\pgfqpoint{2.246148in}{0.973268in}}%
\pgfpathlineto{\pgfqpoint{2.261582in}{0.981184in}}%
\pgfpathlineto{\pgfqpoint{2.277015in}{1.012150in}}%
\pgfpathlineto{\pgfqpoint{2.292449in}{1.064455in}}%
\pgfpathlineto{\pgfqpoint{2.307883in}{1.135751in}}%
\pgfpathlineto{\pgfqpoint{2.323316in}{1.223125in}}%
\pgfpathlineto{\pgfqpoint{2.338750in}{1.323202in}}%
\pgfpathlineto{\pgfqpoint{2.369617in}{1.546335in}}%
\pgfpathlineto{\pgfqpoint{2.400484in}{1.773399in}}%
\pgfpathlineto{\pgfqpoint{2.415918in}{1.878525in}}%
\pgfpathlineto{\pgfqpoint{2.431352in}{1.973210in}}%
\pgfpathlineto{\pgfqpoint{2.446785in}{2.054310in}}%
\pgfpathlineto{\pgfqpoint{2.462219in}{2.119193in}}%
\pgfpathlineto{\pgfqpoint{2.477653in}{2.165822in}}%
\pgfpathlineto{\pgfqpoint{2.493086in}{2.192821in}}%
\pgfpathlineto{\pgfqpoint{2.508520in}{2.199521in}}%
\pgfpathlineto{\pgfqpoint{2.523954in}{2.185979in}}%
\pgfpathlineto{\pgfqpoint{2.539387in}{2.152972in}}%
\pgfpathlineto{\pgfqpoint{2.554821in}{2.101973in}}%
\pgfpathlineto{\pgfqpoint{2.570254in}{2.035102in}}%
\pgfpathlineto{\pgfqpoint{2.585688in}{1.955050in}}%
\pgfpathlineto{\pgfqpoint{2.616555in}{1.768481in}}%
\pgfpathlineto{\pgfqpoint{2.647423in}{1.571470in}}%
\pgfpathlineto{\pgfqpoint{2.662856in}{1.478853in}}%
\pgfpathlineto{\pgfqpoint{2.678290in}{1.395281in}}%
\pgfpathlineto{\pgfqpoint{2.693724in}{1.324303in}}%
\pgfpathlineto{\pgfqpoint{2.709157in}{1.269082in}}%
\pgfpathlineto{\pgfqpoint{2.724591in}{1.232295in}}%
\pgfpathlineto{\pgfqpoint{2.740024in}{1.216039in}}%
\pgfpathlineto{\pgfqpoint{2.755458in}{1.221756in}}%
\pgfpathlineto{\pgfqpoint{2.770892in}{1.250182in}}%
\pgfpathlineto{\pgfqpoint{2.786325in}{1.301317in}}%
\pgfpathlineto{\pgfqpoint{2.801759in}{1.374420in}}%
\pgfpathlineto{\pgfqpoint{2.817193in}{1.468027in}}%
\pgfpathlineto{\pgfqpoint{2.832626in}{1.579992in}}%
\pgfpathlineto{\pgfqpoint{2.848060in}{1.707554in}}%
\pgfpathlineto{\pgfqpoint{2.878927in}{1.995848in}}%
\pgfpathlineto{\pgfqpoint{2.925228in}{2.451314in}}%
\pgfpathlineto{\pgfqpoint{2.940662in}{2.592349in}}%
\pgfpathlineto{\pgfqpoint{2.956095in}{2.721161in}}%
\pgfpathlineto{\pgfqpoint{2.971529in}{2.834077in}}%
\pgfpathlineto{\pgfqpoint{2.986963in}{2.927876in}}%
\pgfpathlineto{\pgfqpoint{3.002396in}{2.999890in}}%
\pgfpathlineto{\pgfqpoint{3.017830in}{3.048086in}}%
\pgfpathlineto{\pgfqpoint{3.033263in}{3.071133in}}%
\pgfpathlineto{\pgfqpoint{3.048697in}{3.068439in}}%
\pgfpathlineto{\pgfqpoint{3.064131in}{3.040174in}}%
\pgfpathlineto{\pgfqpoint{3.079564in}{2.987262in}}%
\pgfpathlineto{\pgfqpoint{3.094998in}{2.911348in}}%
\pgfpathlineto{\pgfqpoint{3.110432in}{2.814753in}}%
\pgfpathlineto{\pgfqpoint{3.125865in}{2.700392in}}%
\pgfpathlineto{\pgfqpoint{3.141299in}{2.571684in}}%
\pgfpathlineto{\pgfqpoint{3.172166in}{2.286763in}}%
\pgfpathlineto{\pgfqpoint{3.218467in}{1.853326in}}%
\pgfpathlineto{\pgfqpoint{3.233901in}{1.723659in}}%
\pgfpathlineto{\pgfqpoint{3.249334in}{1.607515in}}%
\pgfpathlineto{\pgfqpoint{3.264768in}{1.507906in}}%
\pgfpathlineto{\pgfqpoint{3.280202in}{1.427272in}}%
\pgfpathlineto{\pgfqpoint{3.295635in}{1.367401in}}%
\pgfpathlineto{\pgfqpoint{3.311069in}{1.329380in}}%
\pgfpathlineto{\pgfqpoint{3.326503in}{1.313565in}}%
\pgfpathlineto{\pgfqpoint{3.341936in}{1.319574in}}%
\pgfpathlineto{\pgfqpoint{3.357370in}{1.346304in}}%
\pgfpathlineto{\pgfqpoint{3.372803in}{1.391974in}}%
\pgfpathlineto{\pgfqpoint{3.388237in}{1.454190in}}%
\pgfpathlineto{\pgfqpoint{3.403671in}{1.530020in}}%
\pgfpathlineto{\pgfqpoint{3.419104in}{1.616106in}}%
\pgfpathlineto{\pgfqpoint{3.480839in}{1.987306in}}%
\pgfpathlineto{\pgfqpoint{3.496273in}{2.067581in}}%
\pgfpathlineto{\pgfqpoint{3.511706in}{2.135751in}}%
\pgfpathlineto{\pgfqpoint{3.527140in}{2.188906in}}%
\pgfpathlineto{\pgfqpoint{3.542573in}{2.224673in}}%
\pgfpathlineto{\pgfqpoint{3.558007in}{2.241292in}}%
\pgfpathlineto{\pgfqpoint{3.573441in}{2.237679in}}%
\pgfpathlineto{\pgfqpoint{3.588874in}{2.213463in}}%
\pgfpathlineto{\pgfqpoint{3.604308in}{2.169001in}}%
\pgfpathlineto{\pgfqpoint{3.619742in}{2.105370in}}%
\pgfpathlineto{\pgfqpoint{3.635175in}{2.024334in}}%
\pgfpathlineto{\pgfqpoint{3.650609in}{1.928290in}}%
\pgfpathlineto{\pgfqpoint{3.666043in}{1.820189in}}%
\pgfpathlineto{\pgfqpoint{3.696910in}{1.581819in}}%
\pgfpathlineto{\pgfqpoint{3.727777in}{1.340011in}}%
\pgfpathlineto{\pgfqpoint{3.743211in}{1.227985in}}%
\pgfpathlineto{\pgfqpoint{3.758644in}{1.127129in}}%
\pgfpathlineto{\pgfqpoint{3.774078in}{1.041048in}}%
\pgfpathlineto{\pgfqpoint{3.789512in}{0.972935in}}%
\pgfpathlineto{\pgfqpoint{3.804945in}{0.925466in}}%
\pgfpathlineto{\pgfqpoint{3.820379in}{0.900709in}}%
\pgfpathlineto{\pgfqpoint{3.835813in}{0.900059in}}%
\pgfpathlineto{\pgfqpoint{3.851246in}{0.924181in}}%
\pgfpathlineto{\pgfqpoint{3.866680in}{0.972993in}}%
\pgfpathlineto{\pgfqpoint{3.882113in}{1.045659in}}%
\pgfpathlineto{\pgfqpoint{3.897547in}{1.140612in}}%
\pgfpathlineto{\pgfqpoint{3.912981in}{1.255602in}}%
\pgfpathlineto{\pgfqpoint{3.928414in}{1.387763in}}%
\pgfpathlineto{\pgfqpoint{3.959282in}{1.689583in}}%
\pgfpathlineto{\pgfqpoint{4.021016in}{2.328205in}}%
\pgfpathlineto{\pgfqpoint{4.036450in}{2.470354in}}%
\pgfpathlineto{\pgfqpoint{4.051883in}{2.597665in}}%
\pgfpathlineto{\pgfqpoint{4.067317in}{2.706901in}}%
\pgfpathlineto{\pgfqpoint{4.082751in}{2.795387in}}%
\pgfpathlineto{\pgfqpoint{4.098184in}{2.861086in}}%
\pgfpathlineto{\pgfqpoint{4.113618in}{2.902662in}}%
\pgfpathlineto{\pgfqpoint{4.129052in}{2.919522in}}%
\pgfpathlineto{\pgfqpoint{4.144485in}{2.911824in}}%
\pgfpathlineto{\pgfqpoint{4.159919in}{2.880472in}}%
\pgfpathlineto{\pgfqpoint{4.175352in}{2.827080in}}%
\pgfpathlineto{\pgfqpoint{4.190786in}{2.753919in}}%
\pgfpathlineto{\pgfqpoint{4.206220in}{2.663838in}}%
\pgfpathlineto{\pgfqpoint{4.221653in}{2.560171in}}%
\pgfpathlineto{\pgfqpoint{4.252521in}{2.327152in}}%
\pgfpathlineto{\pgfqpoint{4.283388in}{2.086751in}}%
\pgfpathlineto{\pgfqpoint{4.298822in}{1.973813in}}%
\pgfpathlineto{\pgfqpoint{4.314255in}{1.870679in}}%
\pgfpathlineto{\pgfqpoint{4.329689in}{1.780614in}}%
\pgfpathlineto{\pgfqpoint{4.345122in}{1.706386in}}%
\pgfpathlineto{\pgfqpoint{4.360556in}{1.650186in}}%
\pgfpathlineto{\pgfqpoint{4.375990in}{1.613546in}}%
\pgfpathlineto{\pgfqpoint{4.391423in}{1.597304in}}%
\pgfpathlineto{\pgfqpoint{4.406857in}{1.601570in}}%
\pgfpathlineto{\pgfqpoint{4.422291in}{1.625728in}}%
\pgfpathlineto{\pgfqpoint{4.437724in}{1.668459in}}%
\pgfpathlineto{\pgfqpoint{4.453158in}{1.727789in}}%
\pgfpathlineto{\pgfqpoint{4.468592in}{1.801152in}}%
\pgfpathlineto{\pgfqpoint{4.484025in}{1.885481in}}%
\pgfpathlineto{\pgfqpoint{4.514892in}{2.072881in}}%
\pgfpathlineto{\pgfqpoint{4.545760in}{2.259620in}}%
\pgfpathlineto{\pgfqpoint{4.561193in}{2.343029in}}%
\pgfpathlineto{\pgfqpoint{4.576627in}{2.414931in}}%
\pgfpathlineto{\pgfqpoint{4.592061in}{2.472097in}}%
\pgfpathlineto{\pgfqpoint{4.607494in}{2.511760in}}%
\pgfpathlineto{\pgfqpoint{4.622928in}{2.531718in}}%
\pgfpathlineto{\pgfqpoint{4.638362in}{2.530409in}}%
\pgfpathlineto{\pgfqpoint{4.653795in}{2.506965in}}%
\pgfpathlineto{\pgfqpoint{4.669229in}{2.461248in}}%
\pgfpathlineto{\pgfqpoint{4.669229in}{2.461248in}}%
\pgfusepath{stroke}%
\end{pgfscope}%
\begin{pgfscope}%
\pgfpathrectangle{\pgfqpoint{0.634105in}{0.521603in}}{\pgfqpoint{4.227273in}{2.800000in}} %
\pgfusepath{clip}%
\pgfsetrectcap%
\pgfsetroundjoin%
\pgfsetlinewidth{0.501875pt}%
\definecolor{currentstroke}{rgb}{0.264706,0.361242,0.982973}%
\pgfsetstrokecolor{currentstroke}%
\pgfsetdash{}{0pt}%
\pgfpathmoveto{\pgfqpoint{0.826254in}{2.497047in}}%
\pgfpathlineto{\pgfqpoint{0.841687in}{2.563504in}}%
\pgfpathlineto{\pgfqpoint{0.857121in}{2.612614in}}%
\pgfpathlineto{\pgfqpoint{0.872555in}{2.642595in}}%
\pgfpathlineto{\pgfqpoint{0.887988in}{2.652229in}}%
\pgfpathlineto{\pgfqpoint{0.903422in}{2.640904in}}%
\pgfpathlineto{\pgfqpoint{0.918855in}{2.608629in}}%
\pgfpathlineto{\pgfqpoint{0.934289in}{2.556028in}}%
\pgfpathlineto{\pgfqpoint{0.949723in}{2.484323in}}%
\pgfpathlineto{\pgfqpoint{0.965156in}{2.395295in}}%
\pgfpathlineto{\pgfqpoint{0.980590in}{2.291226in}}%
\pgfpathlineto{\pgfqpoint{0.996024in}{2.174835in}}%
\pgfpathlineto{\pgfqpoint{1.026891in}{1.917609in}}%
\pgfpathlineto{\pgfqpoint{1.073192in}{1.522305in}}%
\pgfpathlineto{\pgfqpoint{1.088625in}{1.401879in}}%
\pgfpathlineto{\pgfqpoint{1.104059in}{1.292466in}}%
\pgfpathlineto{\pgfqpoint{1.119493in}{1.196825in}}%
\pgfpathlineto{\pgfqpoint{1.134926in}{1.117298in}}%
\pgfpathlineto{\pgfqpoint{1.150360in}{1.055755in}}%
\pgfpathlineto{\pgfqpoint{1.165794in}{1.013539in}}%
\pgfpathlineto{\pgfqpoint{1.181227in}{0.991440in}}%
\pgfpathlineto{\pgfqpoint{1.196661in}{0.989671in}}%
\pgfpathlineto{\pgfqpoint{1.212095in}{1.007881in}}%
\pgfpathlineto{\pgfqpoint{1.227528in}{1.045166in}}%
\pgfpathlineto{\pgfqpoint{1.242962in}{1.100111in}}%
\pgfpathlineto{\pgfqpoint{1.258395in}{1.170838in}}%
\pgfpathlineto{\pgfqpoint{1.273829in}{1.255069in}}%
\pgfpathlineto{\pgfqpoint{1.289263in}{1.350207in}}%
\pgfpathlineto{\pgfqpoint{1.320130in}{1.561706in}}%
\pgfpathlineto{\pgfqpoint{1.366431in}{1.886959in}}%
\pgfpathlineto{\pgfqpoint{1.381864in}{1.986004in}}%
\pgfpathlineto{\pgfqpoint{1.397298in}{2.076176in}}%
\pgfpathlineto{\pgfqpoint{1.412732in}{2.155456in}}%
\pgfpathlineto{\pgfqpoint{1.428165in}{2.222231in}}%
\pgfpathlineto{\pgfqpoint{1.443599in}{2.275332in}}%
\pgfpathlineto{\pgfqpoint{1.459033in}{2.314065in}}%
\pgfpathlineto{\pgfqpoint{1.474466in}{2.338211in}}%
\pgfpathlineto{\pgfqpoint{1.489900in}{2.348027in}}%
\pgfpathlineto{\pgfqpoint{1.505334in}{2.344219in}}%
\pgfpathlineto{\pgfqpoint{1.520767in}{2.327908in}}%
\pgfpathlineto{\pgfqpoint{1.536201in}{2.300580in}}%
\pgfpathlineto{\pgfqpoint{1.551634in}{2.264029in}}%
\pgfpathlineto{\pgfqpoint{1.567068in}{2.220287in}}%
\pgfpathlineto{\pgfqpoint{1.597935in}{2.120087in}}%
\pgfpathlineto{\pgfqpoint{1.628803in}{2.018066in}}%
\pgfpathlineto{\pgfqpoint{1.644236in}{1.971774in}}%
\pgfpathlineto{\pgfqpoint{1.659670in}{1.931158in}}%
\pgfpathlineto{\pgfqpoint{1.675104in}{1.897788in}}%
\pgfpathlineto{\pgfqpoint{1.690537in}{1.872906in}}%
\pgfpathlineto{\pgfqpoint{1.705971in}{1.857397in}}%
\pgfpathlineto{\pgfqpoint{1.721404in}{1.851759in}}%
\pgfpathlineto{\pgfqpoint{1.736838in}{1.856096in}}%
\pgfpathlineto{\pgfqpoint{1.752272in}{1.870120in}}%
\pgfpathlineto{\pgfqpoint{1.767705in}{1.893170in}}%
\pgfpathlineto{\pgfqpoint{1.783139in}{1.924241in}}%
\pgfpathlineto{\pgfqpoint{1.798573in}{1.962027in}}%
\pgfpathlineto{\pgfqpoint{1.814006in}{2.004968in}}%
\pgfpathlineto{\pgfqpoint{1.875741in}{2.191863in}}%
\pgfpathlineto{\pgfqpoint{1.891174in}{2.232891in}}%
\pgfpathlineto{\pgfqpoint{1.906608in}{2.268090in}}%
\pgfpathlineto{\pgfqpoint{1.922042in}{2.295983in}}%
\pgfpathlineto{\pgfqpoint{1.937475in}{2.315354in}}%
\pgfpathlineto{\pgfqpoint{1.952909in}{2.325290in}}%
\pgfpathlineto{\pgfqpoint{1.968343in}{2.325210in}}%
\pgfpathlineto{\pgfqpoint{1.983776in}{2.314884in}}%
\pgfpathlineto{\pgfqpoint{1.999210in}{2.294441in}}%
\pgfpathlineto{\pgfqpoint{2.014644in}{2.264363in}}%
\pgfpathlineto{\pgfqpoint{2.030077in}{2.225462in}}%
\pgfpathlineto{\pgfqpoint{2.045511in}{2.178855in}}%
\pgfpathlineto{\pgfqpoint{2.060944in}{2.125917in}}%
\pgfpathlineto{\pgfqpoint{2.091812in}{2.007549in}}%
\pgfpathlineto{\pgfqpoint{2.138113in}{1.825818in}}%
\pgfpathlineto{\pgfqpoint{2.153546in}{1.771325in}}%
\pgfpathlineto{\pgfqpoint{2.168980in}{1.722565in}}%
\pgfpathlineto{\pgfqpoint{2.184414in}{1.680853in}}%
\pgfpathlineto{\pgfqpoint{2.199847in}{1.647236in}}%
\pgfpathlineto{\pgfqpoint{2.215281in}{1.622453in}}%
\pgfpathlineto{\pgfqpoint{2.230714in}{1.606915in}}%
\pgfpathlineto{\pgfqpoint{2.246148in}{1.600685in}}%
\pgfpathlineto{\pgfqpoint{2.261582in}{1.603483in}}%
\pgfpathlineto{\pgfqpoint{2.277015in}{1.614692in}}%
\pgfpathlineto{\pgfqpoint{2.292449in}{1.633384in}}%
\pgfpathlineto{\pgfqpoint{2.307883in}{1.658356in}}%
\pgfpathlineto{\pgfqpoint{2.323316in}{1.688173in}}%
\pgfpathlineto{\pgfqpoint{2.354184in}{1.755775in}}%
\pgfpathlineto{\pgfqpoint{2.385051in}{1.822292in}}%
\pgfpathlineto{\pgfqpoint{2.400484in}{1.850812in}}%
\pgfpathlineto{\pgfqpoint{2.415918in}{1.874084in}}%
\pgfpathlineto{\pgfqpoint{2.431352in}{1.890789in}}%
\pgfpathlineto{\pgfqpoint{2.446785in}{1.899870in}}%
\pgfpathlineto{\pgfqpoint{2.462219in}{1.900579in}}%
\pgfpathlineto{\pgfqpoint{2.477653in}{1.892513in}}%
\pgfpathlineto{\pgfqpoint{2.493086in}{1.875629in}}%
\pgfpathlineto{\pgfqpoint{2.508520in}{1.850263in}}%
\pgfpathlineto{\pgfqpoint{2.523954in}{1.817119in}}%
\pgfpathlineto{\pgfqpoint{2.539387in}{1.777253in}}%
\pgfpathlineto{\pgfqpoint{2.554821in}{1.732046in}}%
\pgfpathlineto{\pgfqpoint{2.631989in}{1.490614in}}%
\pgfpathlineto{\pgfqpoint{2.647423in}{1.453875in}}%
\pgfpathlineto{\pgfqpoint{2.662856in}{1.425813in}}%
\pgfpathlineto{\pgfqpoint{2.678290in}{1.408208in}}%
\pgfpathlineto{\pgfqpoint{2.693724in}{1.402565in}}%
\pgfpathlineto{\pgfqpoint{2.709157in}{1.410054in}}%
\pgfpathlineto{\pgfqpoint{2.724591in}{1.431462in}}%
\pgfpathlineto{\pgfqpoint{2.740024in}{1.467154in}}%
\pgfpathlineto{\pgfqpoint{2.755458in}{1.517043in}}%
\pgfpathlineto{\pgfqpoint{2.770892in}{1.580582in}}%
\pgfpathlineto{\pgfqpoint{2.786325in}{1.656770in}}%
\pgfpathlineto{\pgfqpoint{2.801759in}{1.744166in}}%
\pgfpathlineto{\pgfqpoint{2.817193in}{1.840923in}}%
\pgfpathlineto{\pgfqpoint{2.848060in}{2.053417in}}%
\pgfpathlineto{\pgfqpoint{2.894361in}{2.379197in}}%
\pgfpathlineto{\pgfqpoint{2.909794in}{2.478117in}}%
\pgfpathlineto{\pgfqpoint{2.925228in}{2.567469in}}%
\pgfpathlineto{\pgfqpoint{2.940662in}{2.644674in}}%
\pgfpathlineto{\pgfqpoint{2.956095in}{2.707439in}}%
\pgfpathlineto{\pgfqpoint{2.971529in}{2.753838in}}%
\pgfpathlineto{\pgfqpoint{2.986963in}{2.782376in}}%
\pgfpathlineto{\pgfqpoint{3.002396in}{2.792040in}}%
\pgfpathlineto{\pgfqpoint{3.017830in}{2.782338in}}%
\pgfpathlineto{\pgfqpoint{3.033263in}{2.753328in}}%
\pgfpathlineto{\pgfqpoint{3.048697in}{2.705614in}}%
\pgfpathlineto{\pgfqpoint{3.064131in}{2.640346in}}%
\pgfpathlineto{\pgfqpoint{3.079564in}{2.559186in}}%
\pgfpathlineto{\pgfqpoint{3.094998in}{2.464267in}}%
\pgfpathlineto{\pgfqpoint{3.110432in}{2.358136in}}%
\pgfpathlineto{\pgfqpoint{3.141299in}{2.124034in}}%
\pgfpathlineto{\pgfqpoint{3.187600in}{1.767318in}}%
\pgfpathlineto{\pgfqpoint{3.203033in}{1.660190in}}%
\pgfpathlineto{\pgfqpoint{3.218467in}{1.564086in}}%
\pgfpathlineto{\pgfqpoint{3.233901in}{1.481656in}}%
\pgfpathlineto{\pgfqpoint{3.249334in}{1.415147in}}%
\pgfpathlineto{\pgfqpoint{3.264768in}{1.366324in}}%
\pgfpathlineto{\pgfqpoint{3.280202in}{1.336424in}}%
\pgfpathlineto{\pgfqpoint{3.295635in}{1.326108in}}%
\pgfpathlineto{\pgfqpoint{3.311069in}{1.335438in}}%
\pgfpathlineto{\pgfqpoint{3.326503in}{1.363876in}}%
\pgfpathlineto{\pgfqpoint{3.341936in}{1.410296in}}%
\pgfpathlineto{\pgfqpoint{3.357370in}{1.473013in}}%
\pgfpathlineto{\pgfqpoint{3.372803in}{1.549834in}}%
\pgfpathlineto{\pgfqpoint{3.388237in}{1.638120in}}%
\pgfpathlineto{\pgfqpoint{3.419104in}{1.836809in}}%
\pgfpathlineto{\pgfqpoint{3.449972in}{2.042368in}}%
\pgfpathlineto{\pgfqpoint{3.465405in}{2.138983in}}%
\pgfpathlineto{\pgfqpoint{3.480839in}{2.226988in}}%
\pgfpathlineto{\pgfqpoint{3.496273in}{2.303294in}}%
\pgfpathlineto{\pgfqpoint{3.511706in}{2.365160in}}%
\pgfpathlineto{\pgfqpoint{3.527140in}{2.410275in}}%
\pgfpathlineto{\pgfqpoint{3.542573in}{2.436838in}}%
\pgfpathlineto{\pgfqpoint{3.558007in}{2.443610in}}%
\pgfpathlineto{\pgfqpoint{3.573441in}{2.429962in}}%
\pgfpathlineto{\pgfqpoint{3.588874in}{2.395895in}}%
\pgfpathlineto{\pgfqpoint{3.604308in}{2.342045in}}%
\pgfpathlineto{\pgfqpoint{3.619742in}{2.269671in}}%
\pgfpathlineto{\pgfqpoint{3.635175in}{2.180614in}}%
\pgfpathlineto{\pgfqpoint{3.650609in}{2.077255in}}%
\pgfpathlineto{\pgfqpoint{3.681476in}{1.839394in}}%
\pgfpathlineto{\pgfqpoint{3.743211in}{1.337446in}}%
\pgfpathlineto{\pgfqpoint{3.758644in}{1.228108in}}%
\pgfpathlineto{\pgfqpoint{3.774078in}{1.132231in}}%
\pgfpathlineto{\pgfqpoint{3.789512in}{1.052801in}}%
\pgfpathlineto{\pgfqpoint{3.804945in}{0.992362in}}%
\pgfpathlineto{\pgfqpoint{3.820379in}{0.952931in}}%
\pgfpathlineto{\pgfqpoint{3.835813in}{0.935946in}}%
\pgfpathlineto{\pgfqpoint{3.851246in}{0.942219in}}%
\pgfpathlineto{\pgfqpoint{3.866680in}{0.971907in}}%
\pgfpathlineto{\pgfqpoint{3.882113in}{1.024512in}}%
\pgfpathlineto{\pgfqpoint{3.897547in}{1.098891in}}%
\pgfpathlineto{\pgfqpoint{3.912981in}{1.193291in}}%
\pgfpathlineto{\pgfqpoint{3.928414in}{1.305396in}}%
\pgfpathlineto{\pgfqpoint{3.943848in}{1.432401in}}%
\pgfpathlineto{\pgfqpoint{3.974715in}{1.717915in}}%
\pgfpathlineto{\pgfqpoint{4.021016in}{2.169348in}}%
\pgfpathlineto{\pgfqpoint{4.036450in}{2.310749in}}%
\pgfpathlineto{\pgfqpoint{4.051883in}{2.441589in}}%
\pgfpathlineto{\pgfqpoint{4.067317in}{2.558681in}}%
\pgfpathlineto{\pgfqpoint{4.082751in}{2.659247in}}%
\pgfpathlineto{\pgfqpoint{4.098184in}{2.740996in}}%
\pgfpathlineto{\pgfqpoint{4.113618in}{2.802190in}}%
\pgfpathlineto{\pgfqpoint{4.129052in}{2.841681in}}%
\pgfpathlineto{\pgfqpoint{4.144485in}{2.858945in}}%
\pgfpathlineto{\pgfqpoint{4.159919in}{2.854086in}}%
\pgfpathlineto{\pgfqpoint{4.175352in}{2.827823in}}%
\pgfpathlineto{\pgfqpoint{4.190786in}{2.781466in}}%
\pgfpathlineto{\pgfqpoint{4.206220in}{2.716865in}}%
\pgfpathlineto{\pgfqpoint{4.221653in}{2.636351in}}%
\pgfpathlineto{\pgfqpoint{4.237087in}{2.542657in}}%
\pgfpathlineto{\pgfqpoint{4.267954in}{2.328147in}}%
\pgfpathlineto{\pgfqpoint{4.314255in}{1.988795in}}%
\pgfpathlineto{\pgfqpoint{4.329689in}{1.884207in}}%
\pgfpathlineto{\pgfqpoint{4.345122in}{1.788859in}}%
\pgfpathlineto{\pgfqpoint{4.360556in}{1.705247in}}%
\pgfpathlineto{\pgfqpoint{4.375990in}{1.635440in}}%
\pgfpathlineto{\pgfqpoint{4.391423in}{1.581022in}}%
\pgfpathlineto{\pgfqpoint{4.406857in}{1.543054in}}%
\pgfpathlineto{\pgfqpoint{4.422291in}{1.522050in}}%
\pgfpathlineto{\pgfqpoint{4.437724in}{1.517974in}}%
\pgfpathlineto{\pgfqpoint{4.453158in}{1.530246in}}%
\pgfpathlineto{\pgfqpoint{4.468592in}{1.557777in}}%
\pgfpathlineto{\pgfqpoint{4.484025in}{1.599009in}}%
\pgfpathlineto{\pgfqpoint{4.499459in}{1.651972in}}%
\pgfpathlineto{\pgfqpoint{4.514892in}{1.714360in}}%
\pgfpathlineto{\pgfqpoint{4.545760in}{1.856957in}}%
\pgfpathlineto{\pgfqpoint{4.576627in}{2.004709in}}%
\pgfpathlineto{\pgfqpoint{4.592061in}{2.073573in}}%
\pgfpathlineto{\pgfqpoint{4.607494in}{2.135659in}}%
\pgfpathlineto{\pgfqpoint{4.622928in}{2.188700in}}%
\pgfpathlineto{\pgfqpoint{4.638362in}{2.230764in}}%
\pgfpathlineto{\pgfqpoint{4.653795in}{2.260308in}}%
\pgfpathlineto{\pgfqpoint{4.669229in}{2.276224in}}%
\pgfpathlineto{\pgfqpoint{4.669229in}{2.276224in}}%
\pgfusepath{stroke}%
\end{pgfscope}%
\begin{pgfscope}%
\pgfpathrectangle{\pgfqpoint{0.634105in}{0.521603in}}{\pgfqpoint{4.227273in}{2.800000in}} %
\pgfusepath{clip}%
\pgfsetrectcap%
\pgfsetroundjoin%
\pgfsetlinewidth{0.501875pt}%
\definecolor{currentstroke}{rgb}{0.186275,0.473094,0.969797}%
\pgfsetstrokecolor{currentstroke}%
\pgfsetdash{}{0pt}%
\pgfpathmoveto{\pgfqpoint{0.826254in}{2.270167in}}%
\pgfpathlineto{\pgfqpoint{0.841687in}{2.268219in}}%
\pgfpathlineto{\pgfqpoint{0.857121in}{2.263274in}}%
\pgfpathlineto{\pgfqpoint{0.872555in}{2.254852in}}%
\pgfpathlineto{\pgfqpoint{0.887988in}{2.242543in}}%
\pgfpathlineto{\pgfqpoint{0.903422in}{2.226022in}}%
\pgfpathlineto{\pgfqpoint{0.918855in}{2.205055in}}%
\pgfpathlineto{\pgfqpoint{0.934289in}{2.179513in}}%
\pgfpathlineto{\pgfqpoint{0.949723in}{2.149373in}}%
\pgfpathlineto{\pgfqpoint{0.965156in}{2.114726in}}%
\pgfpathlineto{\pgfqpoint{0.980590in}{2.075774in}}%
\pgfpathlineto{\pgfqpoint{0.996024in}{2.032832in}}%
\pgfpathlineto{\pgfqpoint{1.026891in}{1.936772in}}%
\pgfpathlineto{\pgfqpoint{1.057758in}{1.831081in}}%
\pgfpathlineto{\pgfqpoint{1.119493in}{1.614899in}}%
\pgfpathlineto{\pgfqpoint{1.150360in}{1.518004in}}%
\pgfpathlineto{\pgfqpoint{1.165794in}{1.475338in}}%
\pgfpathlineto{\pgfqpoint{1.181227in}{1.437570in}}%
\pgfpathlineto{\pgfqpoint{1.196661in}{1.405400in}}%
\pgfpathlineto{\pgfqpoint{1.212095in}{1.379447in}}%
\pgfpathlineto{\pgfqpoint{1.227528in}{1.360227in}}%
\pgfpathlineto{\pgfqpoint{1.242962in}{1.348151in}}%
\pgfpathlineto{\pgfqpoint{1.258395in}{1.343509in}}%
\pgfpathlineto{\pgfqpoint{1.273829in}{1.346466in}}%
\pgfpathlineto{\pgfqpoint{1.289263in}{1.357056in}}%
\pgfpathlineto{\pgfqpoint{1.304696in}{1.375181in}}%
\pgfpathlineto{\pgfqpoint{1.320130in}{1.400612in}}%
\pgfpathlineto{\pgfqpoint{1.335564in}{1.432992in}}%
\pgfpathlineto{\pgfqpoint{1.350997in}{1.471844in}}%
\pgfpathlineto{\pgfqpoint{1.366431in}{1.516578in}}%
\pgfpathlineto{\pgfqpoint{1.381864in}{1.566501in}}%
\pgfpathlineto{\pgfqpoint{1.412732in}{1.678729in}}%
\pgfpathlineto{\pgfqpoint{1.459033in}{1.864496in}}%
\pgfpathlineto{\pgfqpoint{1.505334in}{2.048253in}}%
\pgfpathlineto{\pgfqpoint{1.536201in}{2.157462in}}%
\pgfpathlineto{\pgfqpoint{1.551634in}{2.205710in}}%
\pgfpathlineto{\pgfqpoint{1.567068in}{2.248866in}}%
\pgfpathlineto{\pgfqpoint{1.582502in}{2.286422in}}%
\pgfpathlineto{\pgfqpoint{1.597935in}{2.317985in}}%
\pgfpathlineto{\pgfqpoint{1.613369in}{2.343281in}}%
\pgfpathlineto{\pgfqpoint{1.628803in}{2.362162in}}%
\pgfpathlineto{\pgfqpoint{1.644236in}{2.374603in}}%
\pgfpathlineto{\pgfqpoint{1.659670in}{2.380700in}}%
\pgfpathlineto{\pgfqpoint{1.675104in}{2.380670in}}%
\pgfpathlineto{\pgfqpoint{1.690537in}{2.374836in}}%
\pgfpathlineto{\pgfqpoint{1.705971in}{2.363624in}}%
\pgfpathlineto{\pgfqpoint{1.721404in}{2.347546in}}%
\pgfpathlineto{\pgfqpoint{1.736838in}{2.327190in}}%
\pgfpathlineto{\pgfqpoint{1.752272in}{2.303206in}}%
\pgfpathlineto{\pgfqpoint{1.783139in}{2.247146in}}%
\pgfpathlineto{\pgfqpoint{1.875741in}{2.065131in}}%
\pgfpathlineto{\pgfqpoint{1.891174in}{2.039342in}}%
\pgfpathlineto{\pgfqpoint{1.906608in}{2.016092in}}%
\pgfpathlineto{\pgfqpoint{1.922042in}{1.995648in}}%
\pgfpathlineto{\pgfqpoint{1.937475in}{1.978174in}}%
\pgfpathlineto{\pgfqpoint{1.952909in}{1.963733in}}%
\pgfpathlineto{\pgfqpoint{1.968343in}{1.952279in}}%
\pgfpathlineto{\pgfqpoint{1.983776in}{1.943670in}}%
\pgfpathlineto{\pgfqpoint{1.999210in}{1.937665in}}%
\pgfpathlineto{\pgfqpoint{2.014644in}{1.933936in}}%
\pgfpathlineto{\pgfqpoint{2.030077in}{1.932076in}}%
\pgfpathlineto{\pgfqpoint{2.060944in}{1.932015in}}%
\pgfpathlineto{\pgfqpoint{2.091812in}{1.933141in}}%
\pgfpathlineto{\pgfqpoint{2.107245in}{1.932678in}}%
\pgfpathlineto{\pgfqpoint{2.122679in}{1.930747in}}%
\pgfpathlineto{\pgfqpoint{2.138113in}{1.926787in}}%
\pgfpathlineto{\pgfqpoint{2.153546in}{1.920279in}}%
\pgfpathlineto{\pgfqpoint{2.168980in}{1.910763in}}%
\pgfpathlineto{\pgfqpoint{2.184414in}{1.897847in}}%
\pgfpathlineto{\pgfqpoint{2.199847in}{1.881222in}}%
\pgfpathlineto{\pgfqpoint{2.215281in}{1.860672in}}%
\pgfpathlineto{\pgfqpoint{2.230714in}{1.836082in}}%
\pgfpathlineto{\pgfqpoint{2.246148in}{1.807442in}}%
\pgfpathlineto{\pgfqpoint{2.261582in}{1.774853in}}%
\pgfpathlineto{\pgfqpoint{2.277015in}{1.738525in}}%
\pgfpathlineto{\pgfqpoint{2.307883in}{1.656030in}}%
\pgfpathlineto{\pgfqpoint{2.338750in}{1.563705in}}%
\pgfpathlineto{\pgfqpoint{2.415918in}{1.325837in}}%
\pgfpathlineto{\pgfqpoint{2.431352in}{1.283451in}}%
\pgfpathlineto{\pgfqpoint{2.446785in}{1.244711in}}%
\pgfpathlineto{\pgfqpoint{2.462219in}{1.210389in}}%
\pgfpathlineto{\pgfqpoint{2.477653in}{1.181198in}}%
\pgfpathlineto{\pgfqpoint{2.493086in}{1.157781in}}%
\pgfpathlineto{\pgfqpoint{2.508520in}{1.140690in}}%
\pgfpathlineto{\pgfqpoint{2.523954in}{1.130381in}}%
\pgfpathlineto{\pgfqpoint{2.539387in}{1.127196in}}%
\pgfpathlineto{\pgfqpoint{2.554821in}{1.131357in}}%
\pgfpathlineto{\pgfqpoint{2.570254in}{1.142961in}}%
\pgfpathlineto{\pgfqpoint{2.585688in}{1.161976in}}%
\pgfpathlineto{\pgfqpoint{2.601122in}{1.188237in}}%
\pgfpathlineto{\pgfqpoint{2.616555in}{1.221453in}}%
\pgfpathlineto{\pgfqpoint{2.631989in}{1.261205in}}%
\pgfpathlineto{\pgfqpoint{2.647423in}{1.306960in}}%
\pgfpathlineto{\pgfqpoint{2.662856in}{1.358075in}}%
\pgfpathlineto{\pgfqpoint{2.693724in}{1.473355in}}%
\pgfpathlineto{\pgfqpoint{2.724591in}{1.600251in}}%
\pgfpathlineto{\pgfqpoint{2.786325in}{1.858372in}}%
\pgfpathlineto{\pgfqpoint{2.817193in}{1.974412in}}%
\pgfpathlineto{\pgfqpoint{2.832626in}{2.026186in}}%
\pgfpathlineto{\pgfqpoint{2.848060in}{2.072854in}}%
\pgfpathlineto{\pgfqpoint{2.863493in}{2.113835in}}%
\pgfpathlineto{\pgfqpoint{2.878927in}{2.148662in}}%
\pgfpathlineto{\pgfqpoint{2.894361in}{2.176983in}}%
\pgfpathlineto{\pgfqpoint{2.909794in}{2.198576in}}%
\pgfpathlineto{\pgfqpoint{2.925228in}{2.213340in}}%
\pgfpathlineto{\pgfqpoint{2.940662in}{2.221306in}}%
\pgfpathlineto{\pgfqpoint{2.956095in}{2.222626in}}%
\pgfpathlineto{\pgfqpoint{2.971529in}{2.217571in}}%
\pgfpathlineto{\pgfqpoint{2.986963in}{2.206520in}}%
\pgfpathlineto{\pgfqpoint{3.002396in}{2.189954in}}%
\pgfpathlineto{\pgfqpoint{3.017830in}{2.168440in}}%
\pgfpathlineto{\pgfqpoint{3.033263in}{2.142618in}}%
\pgfpathlineto{\pgfqpoint{3.048697in}{2.113188in}}%
\pgfpathlineto{\pgfqpoint{3.079564in}{2.046481in}}%
\pgfpathlineto{\pgfqpoint{3.156733in}{1.869046in}}%
\pgfpathlineto{\pgfqpoint{3.187600in}{1.807852in}}%
\pgfpathlineto{\pgfqpoint{3.203033in}{1.781375in}}%
\pgfpathlineto{\pgfqpoint{3.218467in}{1.758052in}}%
\pgfpathlineto{\pgfqpoint{3.233901in}{1.738057in}}%
\pgfpathlineto{\pgfqpoint{3.249334in}{1.721460in}}%
\pgfpathlineto{\pgfqpoint{3.264768in}{1.708225in}}%
\pgfpathlineto{\pgfqpoint{3.280202in}{1.698213in}}%
\pgfpathlineto{\pgfqpoint{3.295635in}{1.691190in}}%
\pgfpathlineto{\pgfqpoint{3.311069in}{1.686831in}}%
\pgfpathlineto{\pgfqpoint{3.326503in}{1.684733in}}%
\pgfpathlineto{\pgfqpoint{3.341936in}{1.684426in}}%
\pgfpathlineto{\pgfqpoint{3.372803in}{1.687058in}}%
\pgfpathlineto{\pgfqpoint{3.403671in}{1.690179in}}%
\pgfpathlineto{\pgfqpoint{3.419104in}{1.690459in}}%
\pgfpathlineto{\pgfqpoint{3.434538in}{1.689140in}}%
\pgfpathlineto{\pgfqpoint{3.449972in}{1.685710in}}%
\pgfpathlineto{\pgfqpoint{3.465405in}{1.679715in}}%
\pgfpathlineto{\pgfqpoint{3.480839in}{1.670769in}}%
\pgfpathlineto{\pgfqpoint{3.496273in}{1.658571in}}%
\pgfpathlineto{\pgfqpoint{3.511706in}{1.642909in}}%
\pgfpathlineto{\pgfqpoint{3.527140in}{1.623672in}}%
\pgfpathlineto{\pgfqpoint{3.542573in}{1.600852in}}%
\pgfpathlineto{\pgfqpoint{3.558007in}{1.574550in}}%
\pgfpathlineto{\pgfqpoint{3.573441in}{1.544972in}}%
\pgfpathlineto{\pgfqpoint{3.604308in}{1.477342in}}%
\pgfpathlineto{\pgfqpoint{3.635175in}{1.401615in}}%
\pgfpathlineto{\pgfqpoint{3.696910in}{1.246658in}}%
\pgfpathlineto{\pgfqpoint{3.712343in}{1.211605in}}%
\pgfpathlineto{\pgfqpoint{3.727777in}{1.179604in}}%
\pgfpathlineto{\pgfqpoint{3.743211in}{1.151415in}}%
\pgfpathlineto{\pgfqpoint{3.758644in}{1.127757in}}%
\pgfpathlineto{\pgfqpoint{3.774078in}{1.109288in}}%
\pgfpathlineto{\pgfqpoint{3.789512in}{1.096590in}}%
\pgfpathlineto{\pgfqpoint{3.804945in}{1.090157in}}%
\pgfpathlineto{\pgfqpoint{3.820379in}{1.090380in}}%
\pgfpathlineto{\pgfqpoint{3.835813in}{1.097538in}}%
\pgfpathlineto{\pgfqpoint{3.851246in}{1.111790in}}%
\pgfpathlineto{\pgfqpoint{3.866680in}{1.133170in}}%
\pgfpathlineto{\pgfqpoint{3.882113in}{1.161585in}}%
\pgfpathlineto{\pgfqpoint{3.897547in}{1.196811in}}%
\pgfpathlineto{\pgfqpoint{3.912981in}{1.238501in}}%
\pgfpathlineto{\pgfqpoint{3.928414in}{1.286185in}}%
\pgfpathlineto{\pgfqpoint{3.943848in}{1.339282in}}%
\pgfpathlineto{\pgfqpoint{3.974715in}{1.458892in}}%
\pgfpathlineto{\pgfqpoint{4.005583in}{1.590868in}}%
\pgfpathlineto{\pgfqpoint{4.067317in}{1.861903in}}%
\pgfpathlineto{\pgfqpoint{4.098184in}{1.985589in}}%
\pgfpathlineto{\pgfqpoint{4.113618in}{2.041358in}}%
\pgfpathlineto{\pgfqpoint{4.129052in}{2.092066in}}%
\pgfpathlineto{\pgfqpoint{4.144485in}{2.137069in}}%
\pgfpathlineto{\pgfqpoint{4.159919in}{2.175833in}}%
\pgfpathlineto{\pgfqpoint{4.175352in}{2.207940in}}%
\pgfpathlineto{\pgfqpoint{4.190786in}{2.233092in}}%
\pgfpathlineto{\pgfqpoint{4.206220in}{2.251123in}}%
\pgfpathlineto{\pgfqpoint{4.221653in}{2.261995in}}%
\pgfpathlineto{\pgfqpoint{4.237087in}{2.265797in}}%
\pgfpathlineto{\pgfqpoint{4.252521in}{2.262743in}}%
\pgfpathlineto{\pgfqpoint{4.267954in}{2.253167in}}%
\pgfpathlineto{\pgfqpoint{4.283388in}{2.237508in}}%
\pgfpathlineto{\pgfqpoint{4.298822in}{2.216307in}}%
\pgfpathlineto{\pgfqpoint{4.314255in}{2.190189in}}%
\pgfpathlineto{\pgfqpoint{4.329689in}{2.159847in}}%
\pgfpathlineto{\pgfqpoint{4.345122in}{2.126033in}}%
\pgfpathlineto{\pgfqpoint{4.375990in}{2.051151in}}%
\pgfpathlineto{\pgfqpoint{4.437724in}{1.894671in}}%
\pgfpathlineto{\pgfqpoint{4.468592in}{1.824669in}}%
\pgfpathlineto{\pgfqpoint{4.484025in}{1.793777in}}%
\pgfpathlineto{\pgfqpoint{4.499459in}{1.766174in}}%
\pgfpathlineto{\pgfqpoint{4.514892in}{1.742145in}}%
\pgfpathlineto{\pgfqpoint{4.530326in}{1.721869in}}%
\pgfpathlineto{\pgfqpoint{4.545760in}{1.705416in}}%
\pgfpathlineto{\pgfqpoint{4.561193in}{1.692746in}}%
\pgfpathlineto{\pgfqpoint{4.576627in}{1.683718in}}%
\pgfpathlineto{\pgfqpoint{4.592061in}{1.678092in}}%
\pgfpathlineto{\pgfqpoint{4.607494in}{1.675536in}}%
\pgfpathlineto{\pgfqpoint{4.622928in}{1.675643in}}%
\pgfpathlineto{\pgfqpoint{4.638362in}{1.677936in}}%
\pgfpathlineto{\pgfqpoint{4.669229in}{1.686923in}}%
\pgfpathlineto{\pgfqpoint{4.669229in}{1.686923in}}%
\pgfusepath{stroke}%
\end{pgfscope}%
\begin{pgfscope}%
\pgfpathrectangle{\pgfqpoint{0.634105in}{0.521603in}}{\pgfqpoint{4.227273in}{2.800000in}} %
\pgfusepath{clip}%
\pgfsetrectcap%
\pgfsetroundjoin%
\pgfsetlinewidth{0.501875pt}%
\definecolor{currentstroke}{rgb}{0.100000,0.587785,0.951057}%
\pgfsetstrokecolor{currentstroke}%
\pgfsetdash{}{0pt}%
\pgfpathmoveto{\pgfqpoint{0.826254in}{2.365297in}}%
\pgfpathlineto{\pgfqpoint{0.841687in}{2.341280in}}%
\pgfpathlineto{\pgfqpoint{0.857121in}{2.313978in}}%
\pgfpathlineto{\pgfqpoint{0.872555in}{2.283671in}}%
\pgfpathlineto{\pgfqpoint{0.903422in}{2.215358in}}%
\pgfpathlineto{\pgfqpoint{0.934289in}{2.139305in}}%
\pgfpathlineto{\pgfqpoint{1.042325in}{1.862951in}}%
\pgfpathlineto{\pgfqpoint{1.073192in}{1.796052in}}%
\pgfpathlineto{\pgfqpoint{1.088625in}{1.766700in}}%
\pgfpathlineto{\pgfqpoint{1.104059in}{1.740538in}}%
\pgfpathlineto{\pgfqpoint{1.119493in}{1.717864in}}%
\pgfpathlineto{\pgfqpoint{1.134926in}{1.698931in}}%
\pgfpathlineto{\pgfqpoint{1.150360in}{1.683946in}}%
\pgfpathlineto{\pgfqpoint{1.165794in}{1.673070in}}%
\pgfpathlineto{\pgfqpoint{1.181227in}{1.666408in}}%
\pgfpathlineto{\pgfqpoint{1.196661in}{1.664017in}}%
\pgfpathlineto{\pgfqpoint{1.212095in}{1.665899in}}%
\pgfpathlineto{\pgfqpoint{1.227528in}{1.672002in}}%
\pgfpathlineto{\pgfqpoint{1.242962in}{1.682223in}}%
\pgfpathlineto{\pgfqpoint{1.258395in}{1.696407in}}%
\pgfpathlineto{\pgfqpoint{1.273829in}{1.714350in}}%
\pgfpathlineto{\pgfqpoint{1.289263in}{1.735800in}}%
\pgfpathlineto{\pgfqpoint{1.304696in}{1.760464in}}%
\pgfpathlineto{\pgfqpoint{1.320130in}{1.788007in}}%
\pgfpathlineto{\pgfqpoint{1.350997in}{1.850227in}}%
\pgfpathlineto{\pgfqpoint{1.381864in}{1.919171in}}%
\pgfpathlineto{\pgfqpoint{1.459033in}{2.097005in}}%
\pgfpathlineto{\pgfqpoint{1.489900in}{2.160682in}}%
\pgfpathlineto{\pgfqpoint{1.505334in}{2.189213in}}%
\pgfpathlineto{\pgfqpoint{1.520767in}{2.215045in}}%
\pgfpathlineto{\pgfqpoint{1.536201in}{2.237847in}}%
\pgfpathlineto{\pgfqpoint{1.551634in}{2.257333in}}%
\pgfpathlineto{\pgfqpoint{1.567068in}{2.273257in}}%
\pgfpathlineto{\pgfqpoint{1.582502in}{2.285417in}}%
\pgfpathlineto{\pgfqpoint{1.597935in}{2.293663in}}%
\pgfpathlineto{\pgfqpoint{1.613369in}{2.297893in}}%
\pgfpathlineto{\pgfqpoint{1.628803in}{2.298055in}}%
\pgfpathlineto{\pgfqpoint{1.644236in}{2.294149in}}%
\pgfpathlineto{\pgfqpoint{1.659670in}{2.286224in}}%
\pgfpathlineto{\pgfqpoint{1.675104in}{2.274381in}}%
\pgfpathlineto{\pgfqpoint{1.690537in}{2.258766in}}%
\pgfpathlineto{\pgfqpoint{1.705971in}{2.239573in}}%
\pgfpathlineto{\pgfqpoint{1.721404in}{2.217037in}}%
\pgfpathlineto{\pgfqpoint{1.736838in}{2.191432in}}%
\pgfpathlineto{\pgfqpoint{1.752272in}{2.163068in}}%
\pgfpathlineto{\pgfqpoint{1.783139in}{2.099447in}}%
\pgfpathlineto{\pgfqpoint{1.814006in}{2.029172in}}%
\pgfpathlineto{\pgfqpoint{1.906608in}{1.810993in}}%
\pgfpathlineto{\pgfqpoint{1.937475in}{1.746291in}}%
\pgfpathlineto{\pgfqpoint{1.952909in}{1.716983in}}%
\pgfpathlineto{\pgfqpoint{1.968343in}{1.690062in}}%
\pgfpathlineto{\pgfqpoint{1.983776in}{1.665746in}}%
\pgfpathlineto{\pgfqpoint{1.999210in}{1.644215in}}%
\pgfpathlineto{\pgfqpoint{2.014644in}{1.625605in}}%
\pgfpathlineto{\pgfqpoint{2.030077in}{1.610005in}}%
\pgfpathlineto{\pgfqpoint{2.045511in}{1.597460in}}%
\pgfpathlineto{\pgfqpoint{2.060944in}{1.587969in}}%
\pgfpathlineto{\pgfqpoint{2.076378in}{1.581485in}}%
\pgfpathlineto{\pgfqpoint{2.091812in}{1.577916in}}%
\pgfpathlineto{\pgfqpoint{2.107245in}{1.577130in}}%
\pgfpathlineto{\pgfqpoint{2.122679in}{1.578952in}}%
\pgfpathlineto{\pgfqpoint{2.138113in}{1.583170in}}%
\pgfpathlineto{\pgfqpoint{2.153546in}{1.589539in}}%
\pgfpathlineto{\pgfqpoint{2.168980in}{1.597783in}}%
\pgfpathlineto{\pgfqpoint{2.199847in}{1.618668in}}%
\pgfpathlineto{\pgfqpoint{2.246148in}{1.655934in}}%
\pgfpathlineto{\pgfqpoint{2.277015in}{1.680619in}}%
\pgfpathlineto{\pgfqpoint{2.307883in}{1.701936in}}%
\pgfpathlineto{\pgfqpoint{2.323316in}{1.710531in}}%
\pgfpathlineto{\pgfqpoint{2.338750in}{1.717365in}}%
\pgfpathlineto{\pgfqpoint{2.354184in}{1.722186in}}%
\pgfpathlineto{\pgfqpoint{2.369617in}{1.724774in}}%
\pgfpathlineto{\pgfqpoint{2.385051in}{1.724944in}}%
\pgfpathlineto{\pgfqpoint{2.400484in}{1.722551in}}%
\pgfpathlineto{\pgfqpoint{2.415918in}{1.717492in}}%
\pgfpathlineto{\pgfqpoint{2.431352in}{1.709704in}}%
\pgfpathlineto{\pgfqpoint{2.446785in}{1.699171in}}%
\pgfpathlineto{\pgfqpoint{2.462219in}{1.685918in}}%
\pgfpathlineto{\pgfqpoint{2.477653in}{1.670016in}}%
\pgfpathlineto{\pgfqpoint{2.493086in}{1.651579in}}%
\pgfpathlineto{\pgfqpoint{2.508520in}{1.630763in}}%
\pgfpathlineto{\pgfqpoint{2.539387in}{1.582806in}}%
\pgfpathlineto{\pgfqpoint{2.570254in}{1.528126in}}%
\pgfpathlineto{\pgfqpoint{2.616555in}{1.438972in}}%
\pgfpathlineto{\pgfqpoint{2.662856in}{1.349846in}}%
\pgfpathlineto{\pgfqpoint{2.693724in}{1.295414in}}%
\pgfpathlineto{\pgfqpoint{2.709157in}{1.270773in}}%
\pgfpathlineto{\pgfqpoint{2.724591in}{1.248288in}}%
\pgfpathlineto{\pgfqpoint{2.740024in}{1.228261in}}%
\pgfpathlineto{\pgfqpoint{2.755458in}{1.210962in}}%
\pgfpathlineto{\pgfqpoint{2.770892in}{1.196627in}}%
\pgfpathlineto{\pgfqpoint{2.786325in}{1.185456in}}%
\pgfpathlineto{\pgfqpoint{2.801759in}{1.177606in}}%
\pgfpathlineto{\pgfqpoint{2.817193in}{1.173191in}}%
\pgfpathlineto{\pgfqpoint{2.832626in}{1.172283in}}%
\pgfpathlineto{\pgfqpoint{2.848060in}{1.174902in}}%
\pgfpathlineto{\pgfqpoint{2.863493in}{1.181025in}}%
\pgfpathlineto{\pgfqpoint{2.878927in}{1.190580in}}%
\pgfpathlineto{\pgfqpoint{2.894361in}{1.203447in}}%
\pgfpathlineto{\pgfqpoint{2.909794in}{1.219462in}}%
\pgfpathlineto{\pgfqpoint{2.925228in}{1.238415in}}%
\pgfpathlineto{\pgfqpoint{2.940662in}{1.260058in}}%
\pgfpathlineto{\pgfqpoint{2.956095in}{1.284102in}}%
\pgfpathlineto{\pgfqpoint{2.986963in}{1.338067in}}%
\pgfpathlineto{\pgfqpoint{3.017830in}{1.397389in}}%
\pgfpathlineto{\pgfqpoint{3.079564in}{1.518896in}}%
\pgfpathlineto{\pgfqpoint{3.110432in}{1.574246in}}%
\pgfpathlineto{\pgfqpoint{3.125865in}{1.599128in}}%
\pgfpathlineto{\pgfqpoint{3.141299in}{1.621647in}}%
\pgfpathlineto{\pgfqpoint{3.156733in}{1.641460in}}%
\pgfpathlineto{\pgfqpoint{3.172166in}{1.658260in}}%
\pgfpathlineto{\pgfqpoint{3.187600in}{1.671775in}}%
\pgfpathlineto{\pgfqpoint{3.203033in}{1.681775in}}%
\pgfpathlineto{\pgfqpoint{3.218467in}{1.688077in}}%
\pgfpathlineto{\pgfqpoint{3.233901in}{1.690541in}}%
\pgfpathlineto{\pgfqpoint{3.249334in}{1.689081in}}%
\pgfpathlineto{\pgfqpoint{3.264768in}{1.683658in}}%
\pgfpathlineto{\pgfqpoint{3.280202in}{1.674288in}}%
\pgfpathlineto{\pgfqpoint{3.295635in}{1.661037in}}%
\pgfpathlineto{\pgfqpoint{3.311069in}{1.644022in}}%
\pgfpathlineto{\pgfqpoint{3.326503in}{1.623412in}}%
\pgfpathlineto{\pgfqpoint{3.341936in}{1.599421in}}%
\pgfpathlineto{\pgfqpoint{3.357370in}{1.572313in}}%
\pgfpathlineto{\pgfqpoint{3.372803in}{1.542389in}}%
\pgfpathlineto{\pgfqpoint{3.403671in}{1.475502in}}%
\pgfpathlineto{\pgfqpoint{3.434538in}{1.401884in}}%
\pgfpathlineto{\pgfqpoint{3.511706in}{1.211940in}}%
\pgfpathlineto{\pgfqpoint{3.542573in}{1.143386in}}%
\pgfpathlineto{\pgfqpoint{3.558007in}{1.112484in}}%
\pgfpathlineto{\pgfqpoint{3.573441in}{1.084364in}}%
\pgfpathlineto{\pgfqpoint{3.588874in}{1.059387in}}%
\pgfpathlineto{\pgfqpoint{3.604308in}{1.037876in}}%
\pgfpathlineto{\pgfqpoint{3.619742in}{1.020114in}}%
\pgfpathlineto{\pgfqpoint{3.635175in}{1.006339in}}%
\pgfpathlineto{\pgfqpoint{3.650609in}{0.996741in}}%
\pgfpathlineto{\pgfqpoint{3.666043in}{0.991460in}}%
\pgfpathlineto{\pgfqpoint{3.681476in}{0.990584in}}%
\pgfpathlineto{\pgfqpoint{3.696910in}{0.994149in}}%
\pgfpathlineto{\pgfqpoint{3.712343in}{1.002137in}}%
\pgfpathlineto{\pgfqpoint{3.727777in}{1.014478in}}%
\pgfpathlineto{\pgfqpoint{3.743211in}{1.031049in}}%
\pgfpathlineto{\pgfqpoint{3.758644in}{1.051677in}}%
\pgfpathlineto{\pgfqpoint{3.774078in}{1.076141in}}%
\pgfpathlineto{\pgfqpoint{3.789512in}{1.104176in}}%
\pgfpathlineto{\pgfqpoint{3.804945in}{1.135474in}}%
\pgfpathlineto{\pgfqpoint{3.835813in}{1.206447in}}%
\pgfpathlineto{\pgfqpoint{3.866680in}{1.285933in}}%
\pgfpathlineto{\pgfqpoint{3.928414in}{1.456312in}}%
\pgfpathlineto{\pgfqpoint{3.959282in}{1.539869in}}%
\pgfpathlineto{\pgfqpoint{3.990149in}{1.617642in}}%
\pgfpathlineto{\pgfqpoint{4.021016in}{1.686521in}}%
\pgfpathlineto{\pgfqpoint{4.036450in}{1.716793in}}%
\pgfpathlineto{\pgfqpoint{4.051883in}{1.743918in}}%
\pgfpathlineto{\pgfqpoint{4.067317in}{1.767674in}}%
\pgfpathlineto{\pgfqpoint{4.082751in}{1.787885in}}%
\pgfpathlineto{\pgfqpoint{4.098184in}{1.804423in}}%
\pgfpathlineto{\pgfqpoint{4.113618in}{1.817209in}}%
\pgfpathlineto{\pgfqpoint{4.129052in}{1.826213in}}%
\pgfpathlineto{\pgfqpoint{4.144485in}{1.831457in}}%
\pgfpathlineto{\pgfqpoint{4.159919in}{1.833008in}}%
\pgfpathlineto{\pgfqpoint{4.175352in}{1.830985in}}%
\pgfpathlineto{\pgfqpoint{4.190786in}{1.825550in}}%
\pgfpathlineto{\pgfqpoint{4.206220in}{1.816906in}}%
\pgfpathlineto{\pgfqpoint{4.221653in}{1.805299in}}%
\pgfpathlineto{\pgfqpoint{4.237087in}{1.791008in}}%
\pgfpathlineto{\pgfqpoint{4.252521in}{1.774346in}}%
\pgfpathlineto{\pgfqpoint{4.283388in}{1.735284in}}%
\pgfpathlineto{\pgfqpoint{4.314255in}{1.691054in}}%
\pgfpathlineto{\pgfqpoint{4.375990in}{1.599596in}}%
\pgfpathlineto{\pgfqpoint{4.406857in}{1.558458in}}%
\pgfpathlineto{\pgfqpoint{4.422291in}{1.540246in}}%
\pgfpathlineto{\pgfqpoint{4.437724in}{1.523997in}}%
\pgfpathlineto{\pgfqpoint{4.453158in}{1.509966in}}%
\pgfpathlineto{\pgfqpoint{4.468592in}{1.498367in}}%
\pgfpathlineto{\pgfqpoint{4.484025in}{1.489379in}}%
\pgfpathlineto{\pgfqpoint{4.499459in}{1.483135in}}%
\pgfpathlineto{\pgfqpoint{4.514892in}{1.479726in}}%
\pgfpathlineto{\pgfqpoint{4.530326in}{1.479198in}}%
\pgfpathlineto{\pgfqpoint{4.545760in}{1.481552in}}%
\pgfpathlineto{\pgfqpoint{4.561193in}{1.486744in}}%
\pgfpathlineto{\pgfqpoint{4.576627in}{1.494686in}}%
\pgfpathlineto{\pgfqpoint{4.592061in}{1.505249in}}%
\pgfpathlineto{\pgfqpoint{4.607494in}{1.518261in}}%
\pgfpathlineto{\pgfqpoint{4.622928in}{1.533516in}}%
\pgfpathlineto{\pgfqpoint{4.638362in}{1.550772in}}%
\pgfpathlineto{\pgfqpoint{4.669229in}{1.590171in}}%
\pgfpathlineto{\pgfqpoint{4.669229in}{1.590171in}}%
\pgfusepath{stroke}%
\end{pgfscope}%
\begin{pgfscope}%
\pgfpathrectangle{\pgfqpoint{0.634105in}{0.521603in}}{\pgfqpoint{4.227273in}{2.800000in}} %
\pgfusepath{clip}%
\pgfsetrectcap%
\pgfsetroundjoin%
\pgfsetlinewidth{0.501875pt}%
\definecolor{currentstroke}{rgb}{0.021569,0.682749,0.930229}%
\pgfsetstrokecolor{currentstroke}%
\pgfsetdash{}{0pt}%
\pgfpathmoveto{\pgfqpoint{0.826254in}{2.548800in}}%
\pgfpathlineto{\pgfqpoint{0.841687in}{2.533615in}}%
\pgfpathlineto{\pgfqpoint{0.857121in}{2.513903in}}%
\pgfpathlineto{\pgfqpoint{0.872555in}{2.489931in}}%
\pgfpathlineto{\pgfqpoint{0.887988in}{2.462026in}}%
\pgfpathlineto{\pgfqpoint{0.903422in}{2.430570in}}%
\pgfpathlineto{\pgfqpoint{0.934289in}{2.358767in}}%
\pgfpathlineto{\pgfqpoint{0.965156in}{2.278447in}}%
\pgfpathlineto{\pgfqpoint{1.042325in}{2.069752in}}%
\pgfpathlineto{\pgfqpoint{1.073192in}{1.995138in}}%
\pgfpathlineto{\pgfqpoint{1.088625in}{1.961878in}}%
\pgfpathlineto{\pgfqpoint{1.104059in}{1.931937in}}%
\pgfpathlineto{\pgfqpoint{1.119493in}{1.905727in}}%
\pgfpathlineto{\pgfqpoint{1.134926in}{1.883605in}}%
\pgfpathlineto{\pgfqpoint{1.150360in}{1.865873in}}%
\pgfpathlineto{\pgfqpoint{1.165794in}{1.852772in}}%
\pgfpathlineto{\pgfqpoint{1.181227in}{1.844477in}}%
\pgfpathlineto{\pgfqpoint{1.196661in}{1.841096in}}%
\pgfpathlineto{\pgfqpoint{1.212095in}{1.842670in}}%
\pgfpathlineto{\pgfqpoint{1.227528in}{1.849169in}}%
\pgfpathlineto{\pgfqpoint{1.242962in}{1.860495in}}%
\pgfpathlineto{\pgfqpoint{1.258395in}{1.876480in}}%
\pgfpathlineto{\pgfqpoint{1.273829in}{1.896895in}}%
\pgfpathlineto{\pgfqpoint{1.289263in}{1.921444in}}%
\pgfpathlineto{\pgfqpoint{1.304696in}{1.949775in}}%
\pgfpathlineto{\pgfqpoint{1.320130in}{1.981484in}}%
\pgfpathlineto{\pgfqpoint{1.350997in}{2.053178in}}%
\pgfpathlineto{\pgfqpoint{1.381864in}{2.132442in}}%
\pgfpathlineto{\pgfqpoint{1.443599in}{2.295405in}}%
\pgfpathlineto{\pgfqpoint{1.474466in}{2.369759in}}%
\pgfpathlineto{\pgfqpoint{1.489900in}{2.403207in}}%
\pgfpathlineto{\pgfqpoint{1.505334in}{2.433503in}}%
\pgfpathlineto{\pgfqpoint{1.520767in}{2.460196in}}%
\pgfpathlineto{\pgfqpoint{1.536201in}{2.482886in}}%
\pgfpathlineto{\pgfqpoint{1.551634in}{2.501226in}}%
\pgfpathlineto{\pgfqpoint{1.567068in}{2.514925in}}%
\pgfpathlineto{\pgfqpoint{1.582502in}{2.523757in}}%
\pgfpathlineto{\pgfqpoint{1.597935in}{2.527562in}}%
\pgfpathlineto{\pgfqpoint{1.613369in}{2.526245in}}%
\pgfpathlineto{\pgfqpoint{1.628803in}{2.519780in}}%
\pgfpathlineto{\pgfqpoint{1.644236in}{2.508212in}}%
\pgfpathlineto{\pgfqpoint{1.659670in}{2.491652in}}%
\pgfpathlineto{\pgfqpoint{1.675104in}{2.470277in}}%
\pgfpathlineto{\pgfqpoint{1.690537in}{2.444330in}}%
\pgfpathlineto{\pgfqpoint{1.705971in}{2.414112in}}%
\pgfpathlineto{\pgfqpoint{1.721404in}{2.379981in}}%
\pgfpathlineto{\pgfqpoint{1.736838in}{2.342345in}}%
\pgfpathlineto{\pgfqpoint{1.767705in}{2.258414in}}%
\pgfpathlineto{\pgfqpoint{1.798573in}{2.166362in}}%
\pgfpathlineto{\pgfqpoint{1.875741in}{1.930037in}}%
\pgfpathlineto{\pgfqpoint{1.906608in}{1.844685in}}%
\pgfpathlineto{\pgfqpoint{1.922042in}{1.805997in}}%
\pgfpathlineto{\pgfqpoint{1.937475in}{1.770532in}}%
\pgfpathlineto{\pgfqpoint{1.952909in}{1.738654in}}%
\pgfpathlineto{\pgfqpoint{1.968343in}{1.710672in}}%
\pgfpathlineto{\pgfqpoint{1.983776in}{1.686836in}}%
\pgfpathlineto{\pgfqpoint{1.999210in}{1.667334in}}%
\pgfpathlineto{\pgfqpoint{2.014644in}{1.652286in}}%
\pgfpathlineto{\pgfqpoint{2.030077in}{1.641751in}}%
\pgfpathlineto{\pgfqpoint{2.045511in}{1.635717in}}%
\pgfpathlineto{\pgfqpoint{2.060944in}{1.634107in}}%
\pgfpathlineto{\pgfqpoint{2.076378in}{1.636777in}}%
\pgfpathlineto{\pgfqpoint{2.091812in}{1.643521in}}%
\pgfpathlineto{\pgfqpoint{2.107245in}{1.654073in}}%
\pgfpathlineto{\pgfqpoint{2.122679in}{1.668109in}}%
\pgfpathlineto{\pgfqpoint{2.138113in}{1.685258in}}%
\pgfpathlineto{\pgfqpoint{2.153546in}{1.705097in}}%
\pgfpathlineto{\pgfqpoint{2.184414in}{1.750977in}}%
\pgfpathlineto{\pgfqpoint{2.277015in}{1.900460in}}%
\pgfpathlineto{\pgfqpoint{2.292449in}{1.921444in}}%
\pgfpathlineto{\pgfqpoint{2.307883in}{1.939945in}}%
\pgfpathlineto{\pgfqpoint{2.323316in}{1.955517in}}%
\pgfpathlineto{\pgfqpoint{2.338750in}{1.967756in}}%
\pgfpathlineto{\pgfqpoint{2.354184in}{1.976306in}}%
\pgfpathlineto{\pgfqpoint{2.369617in}{1.980865in}}%
\pgfpathlineto{\pgfqpoint{2.385051in}{1.981186in}}%
\pgfpathlineto{\pgfqpoint{2.400484in}{1.977087in}}%
\pgfpathlineto{\pgfqpoint{2.415918in}{1.968448in}}%
\pgfpathlineto{\pgfqpoint{2.431352in}{1.955217in}}%
\pgfpathlineto{\pgfqpoint{2.446785in}{1.937406in}}%
\pgfpathlineto{\pgfqpoint{2.462219in}{1.915097in}}%
\pgfpathlineto{\pgfqpoint{2.477653in}{1.888437in}}%
\pgfpathlineto{\pgfqpoint{2.493086in}{1.857638in}}%
\pgfpathlineto{\pgfqpoint{2.508520in}{1.822974in}}%
\pgfpathlineto{\pgfqpoint{2.523954in}{1.784774in}}%
\pgfpathlineto{\pgfqpoint{2.554821in}{1.699356in}}%
\pgfpathlineto{\pgfqpoint{2.585688in}{1.605001in}}%
\pgfpathlineto{\pgfqpoint{2.678290in}{1.311560in}}%
\pgfpathlineto{\pgfqpoint{2.709157in}{1.225375in}}%
\pgfpathlineto{\pgfqpoint{2.724591in}{1.186860in}}%
\pgfpathlineto{\pgfqpoint{2.740024in}{1.152004in}}%
\pgfpathlineto{\pgfqpoint{2.755458in}{1.121207in}}%
\pgfpathlineto{\pgfqpoint{2.770892in}{1.094818in}}%
\pgfpathlineto{\pgfqpoint{2.786325in}{1.073127in}}%
\pgfpathlineto{\pgfqpoint{2.801759in}{1.056363in}}%
\pgfpathlineto{\pgfqpoint{2.817193in}{1.044690in}}%
\pgfpathlineto{\pgfqpoint{2.832626in}{1.038208in}}%
\pgfpathlineto{\pgfqpoint{2.848060in}{1.036944in}}%
\pgfpathlineto{\pgfqpoint{2.863493in}{1.040861in}}%
\pgfpathlineto{\pgfqpoint{2.878927in}{1.049851in}}%
\pgfpathlineto{\pgfqpoint{2.894361in}{1.063739in}}%
\pgfpathlineto{\pgfqpoint{2.909794in}{1.082288in}}%
\pgfpathlineto{\pgfqpoint{2.925228in}{1.105197in}}%
\pgfpathlineto{\pgfqpoint{2.940662in}{1.132110in}}%
\pgfpathlineto{\pgfqpoint{2.956095in}{1.162617in}}%
\pgfpathlineto{\pgfqpoint{2.986963in}{1.232554in}}%
\pgfpathlineto{\pgfqpoint{3.017830in}{1.310924in}}%
\pgfpathlineto{\pgfqpoint{3.094998in}{1.513880in}}%
\pgfpathlineto{\pgfqpoint{3.125865in}{1.585886in}}%
\pgfpathlineto{\pgfqpoint{3.141299in}{1.617822in}}%
\pgfpathlineto{\pgfqpoint{3.156733in}{1.646448in}}%
\pgfpathlineto{\pgfqpoint{3.172166in}{1.671378in}}%
\pgfpathlineto{\pgfqpoint{3.187600in}{1.692277in}}%
\pgfpathlineto{\pgfqpoint{3.203033in}{1.708869in}}%
\pgfpathlineto{\pgfqpoint{3.218467in}{1.720942in}}%
\pgfpathlineto{\pgfqpoint{3.233901in}{1.728348in}}%
\pgfpathlineto{\pgfqpoint{3.249334in}{1.731006in}}%
\pgfpathlineto{\pgfqpoint{3.264768in}{1.728904in}}%
\pgfpathlineto{\pgfqpoint{3.280202in}{1.722099in}}%
\pgfpathlineto{\pgfqpoint{3.295635in}{1.710713in}}%
\pgfpathlineto{\pgfqpoint{3.311069in}{1.694937in}}%
\pgfpathlineto{\pgfqpoint{3.326503in}{1.675021in}}%
\pgfpathlineto{\pgfqpoint{3.341936in}{1.651277in}}%
\pgfpathlineto{\pgfqpoint{3.357370in}{1.624070in}}%
\pgfpathlineto{\pgfqpoint{3.372803in}{1.593817in}}%
\pgfpathlineto{\pgfqpoint{3.403671in}{1.526045in}}%
\pgfpathlineto{\pgfqpoint{3.449972in}{1.414073in}}%
\pgfpathlineto{\pgfqpoint{3.496273in}{1.303179in}}%
\pgfpathlineto{\pgfqpoint{3.527140in}{1.237295in}}%
\pgfpathlineto{\pgfqpoint{3.542573in}{1.208319in}}%
\pgfpathlineto{\pgfqpoint{3.558007in}{1.182602in}}%
\pgfpathlineto{\pgfqpoint{3.573441in}{1.160548in}}%
\pgfpathlineto{\pgfqpoint{3.588874in}{1.142515in}}%
\pgfpathlineto{\pgfqpoint{3.604308in}{1.128802in}}%
\pgfpathlineto{\pgfqpoint{3.619742in}{1.119652in}}%
\pgfpathlineto{\pgfqpoint{3.635175in}{1.115241in}}%
\pgfpathlineto{\pgfqpoint{3.650609in}{1.115686in}}%
\pgfpathlineto{\pgfqpoint{3.666043in}{1.121032in}}%
\pgfpathlineto{\pgfqpoint{3.681476in}{1.131261in}}%
\pgfpathlineto{\pgfqpoint{3.696910in}{1.146285in}}%
\pgfpathlineto{\pgfqpoint{3.712343in}{1.165954in}}%
\pgfpathlineto{\pgfqpoint{3.727777in}{1.190053in}}%
\pgfpathlineto{\pgfqpoint{3.743211in}{1.218307in}}%
\pgfpathlineto{\pgfqpoint{3.758644in}{1.250386in}}%
\pgfpathlineto{\pgfqpoint{3.774078in}{1.285909in}}%
\pgfpathlineto{\pgfqpoint{3.804945in}{1.365535in}}%
\pgfpathlineto{\pgfqpoint{3.835813in}{1.453331in}}%
\pgfpathlineto{\pgfqpoint{3.912981in}{1.680217in}}%
\pgfpathlineto{\pgfqpoint{3.943848in}{1.762511in}}%
\pgfpathlineto{\pgfqpoint{3.959282in}{1.799842in}}%
\pgfpathlineto{\pgfqpoint{3.974715in}{1.834058in}}%
\pgfpathlineto{\pgfqpoint{3.990149in}{1.864791in}}%
\pgfpathlineto{\pgfqpoint{4.005583in}{1.891720in}}%
\pgfpathlineto{\pgfqpoint{4.021016in}{1.914585in}}%
\pgfpathlineto{\pgfqpoint{4.036450in}{1.933184in}}%
\pgfpathlineto{\pgfqpoint{4.051883in}{1.947376in}}%
\pgfpathlineto{\pgfqpoint{4.067317in}{1.957086in}}%
\pgfpathlineto{\pgfqpoint{4.082751in}{1.962304in}}%
\pgfpathlineto{\pgfqpoint{4.098184in}{1.963085in}}%
\pgfpathlineto{\pgfqpoint{4.113618in}{1.959550in}}%
\pgfpathlineto{\pgfqpoint{4.129052in}{1.951880in}}%
\pgfpathlineto{\pgfqpoint{4.144485in}{1.940317in}}%
\pgfpathlineto{\pgfqpoint{4.159919in}{1.925160in}}%
\pgfpathlineto{\pgfqpoint{4.175352in}{1.906759in}}%
\pgfpathlineto{\pgfqpoint{4.190786in}{1.885511in}}%
\pgfpathlineto{\pgfqpoint{4.221653in}{1.836258in}}%
\pgfpathlineto{\pgfqpoint{4.267954in}{1.752951in}}%
\pgfpathlineto{\pgfqpoint{4.298822in}{1.697303in}}%
\pgfpathlineto{\pgfqpoint{4.329689in}{1.646523in}}%
\pgfpathlineto{\pgfqpoint{4.345122in}{1.624212in}}%
\pgfpathlineto{\pgfqpoint{4.360556in}{1.604568in}}%
\pgfpathlineto{\pgfqpoint{4.375990in}{1.588004in}}%
\pgfpathlineto{\pgfqpoint{4.391423in}{1.574888in}}%
\pgfpathlineto{\pgfqpoint{4.406857in}{1.565540in}}%
\pgfpathlineto{\pgfqpoint{4.422291in}{1.560223in}}%
\pgfpathlineto{\pgfqpoint{4.437724in}{1.559145in}}%
\pgfpathlineto{\pgfqpoint{4.453158in}{1.562450in}}%
\pgfpathlineto{\pgfqpoint{4.468592in}{1.570218in}}%
\pgfpathlineto{\pgfqpoint{4.484025in}{1.582467in}}%
\pgfpathlineto{\pgfqpoint{4.499459in}{1.599146in}}%
\pgfpathlineto{\pgfqpoint{4.514892in}{1.620143in}}%
\pgfpathlineto{\pgfqpoint{4.530326in}{1.645280in}}%
\pgfpathlineto{\pgfqpoint{4.545760in}{1.674320in}}%
\pgfpathlineto{\pgfqpoint{4.561193in}{1.706968in}}%
\pgfpathlineto{\pgfqpoint{4.576627in}{1.742874in}}%
\pgfpathlineto{\pgfqpoint{4.607494in}{1.822832in}}%
\pgfpathlineto{\pgfqpoint{4.638362in}{1.910533in}}%
\pgfpathlineto{\pgfqpoint{4.669229in}{2.001834in}}%
\pgfpathlineto{\pgfqpoint{4.669229in}{2.001834in}}%
\pgfusepath{stroke}%
\end{pgfscope}%
\begin{pgfscope}%
\pgfpathrectangle{\pgfqpoint{0.634105in}{0.521603in}}{\pgfqpoint{4.227273in}{2.800000in}} %
\pgfusepath{clip}%
\pgfsetrectcap%
\pgfsetroundjoin%
\pgfsetlinewidth{0.501875pt}%
\definecolor{currentstroke}{rgb}{0.056863,0.767363,0.905873}%
\pgfsetstrokecolor{currentstroke}%
\pgfsetdash{}{0pt}%
\pgfpathmoveto{\pgfqpoint{0.826254in}{2.426000in}}%
\pgfpathlineto{\pgfqpoint{0.841687in}{2.426736in}}%
\pgfpathlineto{\pgfqpoint{0.857121in}{2.421564in}}%
\pgfpathlineto{\pgfqpoint{0.872555in}{2.410296in}}%
\pgfpathlineto{\pgfqpoint{0.887988in}{2.392885in}}%
\pgfpathlineto{\pgfqpoint{0.903422in}{2.369435in}}%
\pgfpathlineto{\pgfqpoint{0.918855in}{2.340199in}}%
\pgfpathlineto{\pgfqpoint{0.934289in}{2.305580in}}%
\pgfpathlineto{\pgfqpoint{0.949723in}{2.266123in}}%
\pgfpathlineto{\pgfqpoint{0.965156in}{2.222507in}}%
\pgfpathlineto{\pgfqpoint{0.996024in}{2.126074in}}%
\pgfpathlineto{\pgfqpoint{1.057758in}{1.923689in}}%
\pgfpathlineto{\pgfqpoint{1.073192in}{1.877181in}}%
\pgfpathlineto{\pgfqpoint{1.088625in}{1.834333in}}%
\pgfpathlineto{\pgfqpoint{1.104059in}{1.796051in}}%
\pgfpathlineto{\pgfqpoint{1.119493in}{1.763139in}}%
\pgfpathlineto{\pgfqpoint{1.134926in}{1.736281in}}%
\pgfpathlineto{\pgfqpoint{1.150360in}{1.716022in}}%
\pgfpathlineto{\pgfqpoint{1.165794in}{1.702753in}}%
\pgfpathlineto{\pgfqpoint{1.181227in}{1.696705in}}%
\pgfpathlineto{\pgfqpoint{1.196661in}{1.697942in}}%
\pgfpathlineto{\pgfqpoint{1.212095in}{1.706357in}}%
\pgfpathlineto{\pgfqpoint{1.227528in}{1.721681in}}%
\pgfpathlineto{\pgfqpoint{1.242962in}{1.743492in}}%
\pgfpathlineto{\pgfqpoint{1.258395in}{1.771220in}}%
\pgfpathlineto{\pgfqpoint{1.273829in}{1.804171in}}%
\pgfpathlineto{\pgfqpoint{1.289263in}{1.841546in}}%
\pgfpathlineto{\pgfqpoint{1.320130in}{1.925957in}}%
\pgfpathlineto{\pgfqpoint{1.397298in}{2.148190in}}%
\pgfpathlineto{\pgfqpoint{1.412732in}{2.187187in}}%
\pgfpathlineto{\pgfqpoint{1.428165in}{2.222568in}}%
\pgfpathlineto{\pgfqpoint{1.443599in}{2.253770in}}%
\pgfpathlineto{\pgfqpoint{1.459033in}{2.280358in}}%
\pgfpathlineto{\pgfqpoint{1.474466in}{2.302028in}}%
\pgfpathlineto{\pgfqpoint{1.489900in}{2.318614in}}%
\pgfpathlineto{\pgfqpoint{1.505334in}{2.330081in}}%
\pgfpathlineto{\pgfqpoint{1.520767in}{2.336524in}}%
\pgfpathlineto{\pgfqpoint{1.536201in}{2.338154in}}%
\pgfpathlineto{\pgfqpoint{1.551634in}{2.335289in}}%
\pgfpathlineto{\pgfqpoint{1.567068in}{2.328335in}}%
\pgfpathlineto{\pgfqpoint{1.582502in}{2.317769in}}%
\pgfpathlineto{\pgfqpoint{1.597935in}{2.304123in}}%
\pgfpathlineto{\pgfqpoint{1.613369in}{2.287959in}}%
\pgfpathlineto{\pgfqpoint{1.644236in}{2.250367in}}%
\pgfpathlineto{\pgfqpoint{1.721404in}{2.149357in}}%
\pgfpathlineto{\pgfqpoint{1.752272in}{2.113657in}}%
\pgfpathlineto{\pgfqpoint{1.783139in}{2.082351in}}%
\pgfpathlineto{\pgfqpoint{1.814006in}{2.054847in}}%
\pgfpathlineto{\pgfqpoint{1.906608in}{1.977863in}}%
\pgfpathlineto{\pgfqpoint{1.937475in}{1.947804in}}%
\pgfpathlineto{\pgfqpoint{1.968343in}{1.913837in}}%
\pgfpathlineto{\pgfqpoint{1.999210in}{1.876509in}}%
\pgfpathlineto{\pgfqpoint{2.060944in}{1.799874in}}%
\pgfpathlineto{\pgfqpoint{2.091812in}{1.766914in}}%
\pgfpathlineto{\pgfqpoint{2.107245in}{1.753388in}}%
\pgfpathlineto{\pgfqpoint{2.122679in}{1.742434in}}%
\pgfpathlineto{\pgfqpoint{2.138113in}{1.734455in}}%
\pgfpathlineto{\pgfqpoint{2.153546in}{1.729793in}}%
\pgfpathlineto{\pgfqpoint{2.168980in}{1.728714in}}%
\pgfpathlineto{\pgfqpoint{2.184414in}{1.731392in}}%
\pgfpathlineto{\pgfqpoint{2.199847in}{1.737896in}}%
\pgfpathlineto{\pgfqpoint{2.215281in}{1.748183in}}%
\pgfpathlineto{\pgfqpoint{2.230714in}{1.762089in}}%
\pgfpathlineto{\pgfqpoint{2.246148in}{1.779333in}}%
\pgfpathlineto{\pgfqpoint{2.261582in}{1.799513in}}%
\pgfpathlineto{\pgfqpoint{2.292449in}{1.846528in}}%
\pgfpathlineto{\pgfqpoint{2.354184in}{1.947254in}}%
\pgfpathlineto{\pgfqpoint{2.369617in}{1.969057in}}%
\pgfpathlineto{\pgfqpoint{2.385051in}{1.987816in}}%
\pgfpathlineto{\pgfqpoint{2.400484in}{2.002739in}}%
\pgfpathlineto{\pgfqpoint{2.415918in}{2.013102in}}%
\pgfpathlineto{\pgfqpoint{2.431352in}{2.018274in}}%
\pgfpathlineto{\pgfqpoint{2.446785in}{2.017735in}}%
\pgfpathlineto{\pgfqpoint{2.462219in}{2.011097in}}%
\pgfpathlineto{\pgfqpoint{2.477653in}{1.998117in}}%
\pgfpathlineto{\pgfqpoint{2.493086in}{1.978707in}}%
\pgfpathlineto{\pgfqpoint{2.508520in}{1.952943in}}%
\pgfpathlineto{\pgfqpoint{2.523954in}{1.921060in}}%
\pgfpathlineto{\pgfqpoint{2.539387in}{1.883458in}}%
\pgfpathlineto{\pgfqpoint{2.554821in}{1.840687in}}%
\pgfpathlineto{\pgfqpoint{2.570254in}{1.793437in}}%
\pgfpathlineto{\pgfqpoint{2.601122in}{1.688870in}}%
\pgfpathlineto{\pgfqpoint{2.678290in}{1.415853in}}%
\pgfpathlineto{\pgfqpoint{2.693724in}{1.367743in}}%
\pgfpathlineto{\pgfqpoint{2.709157in}{1.324100in}}%
\pgfpathlineto{\pgfqpoint{2.724591in}{1.285740in}}%
\pgfpathlineto{\pgfqpoint{2.740024in}{1.253354in}}%
\pgfpathlineto{\pgfqpoint{2.755458in}{1.227489in}}%
\pgfpathlineto{\pgfqpoint{2.770892in}{1.208540in}}%
\pgfpathlineto{\pgfqpoint{2.786325in}{1.196734in}}%
\pgfpathlineto{\pgfqpoint{2.801759in}{1.192134in}}%
\pgfpathlineto{\pgfqpoint{2.817193in}{1.194634in}}%
\pgfpathlineto{\pgfqpoint{2.832626in}{1.203966in}}%
\pgfpathlineto{\pgfqpoint{2.848060in}{1.219714in}}%
\pgfpathlineto{\pgfqpoint{2.863493in}{1.241320in}}%
\pgfpathlineto{\pgfqpoint{2.878927in}{1.268112in}}%
\pgfpathlineto{\pgfqpoint{2.894361in}{1.299314in}}%
\pgfpathlineto{\pgfqpoint{2.925228in}{1.371503in}}%
\pgfpathlineto{\pgfqpoint{3.002396in}{1.566680in}}%
\pgfpathlineto{\pgfqpoint{3.017830in}{1.601484in}}%
\pgfpathlineto{\pgfqpoint{3.033263in}{1.633283in}}%
\pgfpathlineto{\pgfqpoint{3.048697in}{1.661602in}}%
\pgfpathlineto{\pgfqpoint{3.064131in}{1.686090in}}%
\pgfpathlineto{\pgfqpoint{3.079564in}{1.706519in}}%
\pgfpathlineto{\pgfqpoint{3.094998in}{1.722788in}}%
\pgfpathlineto{\pgfqpoint{3.110432in}{1.734917in}}%
\pgfpathlineto{\pgfqpoint{3.125865in}{1.743037in}}%
\pgfpathlineto{\pgfqpoint{3.141299in}{1.747379in}}%
\pgfpathlineto{\pgfqpoint{3.156733in}{1.748261in}}%
\pgfpathlineto{\pgfqpoint{3.172166in}{1.746068in}}%
\pgfpathlineto{\pgfqpoint{3.187600in}{1.741239in}}%
\pgfpathlineto{\pgfqpoint{3.203033in}{1.734239in}}%
\pgfpathlineto{\pgfqpoint{3.233901in}{1.715646in}}%
\pgfpathlineto{\pgfqpoint{3.326503in}{1.652153in}}%
\pgfpathlineto{\pgfqpoint{3.357370in}{1.634821in}}%
\pgfpathlineto{\pgfqpoint{3.403671in}{1.612515in}}%
\pgfpathlineto{\pgfqpoint{3.449972in}{1.589781in}}%
\pgfpathlineto{\pgfqpoint{3.480839in}{1.571029in}}%
\pgfpathlineto{\pgfqpoint{3.511706in}{1.547481in}}%
\pgfpathlineto{\pgfqpoint{3.542573in}{1.518468in}}%
\pgfpathlineto{\pgfqpoint{3.573441in}{1.484563in}}%
\pgfpathlineto{\pgfqpoint{3.650609in}{1.394258in}}%
\pgfpathlineto{\pgfqpoint{3.666043in}{1.379080in}}%
\pgfpathlineto{\pgfqpoint{3.681476in}{1.366210in}}%
\pgfpathlineto{\pgfqpoint{3.696910in}{1.356243in}}%
\pgfpathlineto{\pgfqpoint{3.712343in}{1.349747in}}%
\pgfpathlineto{\pgfqpoint{3.727777in}{1.347242in}}%
\pgfpathlineto{\pgfqpoint{3.743211in}{1.349180in}}%
\pgfpathlineto{\pgfqpoint{3.758644in}{1.355929in}}%
\pgfpathlineto{\pgfqpoint{3.774078in}{1.367749in}}%
\pgfpathlineto{\pgfqpoint{3.789512in}{1.384790in}}%
\pgfpathlineto{\pgfqpoint{3.804945in}{1.407070in}}%
\pgfpathlineto{\pgfqpoint{3.820379in}{1.434476in}}%
\pgfpathlineto{\pgfqpoint{3.835813in}{1.466758in}}%
\pgfpathlineto{\pgfqpoint{3.851246in}{1.503529in}}%
\pgfpathlineto{\pgfqpoint{3.866680in}{1.544274in}}%
\pgfpathlineto{\pgfqpoint{3.897547in}{1.635041in}}%
\pgfpathlineto{\pgfqpoint{3.974715in}{1.876402in}}%
\pgfpathlineto{\pgfqpoint{3.990149in}{1.919631in}}%
\pgfpathlineto{\pgfqpoint{4.005583in}{1.959037in}}%
\pgfpathlineto{\pgfqpoint{4.021016in}{1.993822in}}%
\pgfpathlineto{\pgfqpoint{4.036450in}{2.023293in}}%
\pgfpathlineto{\pgfqpoint{4.051883in}{2.046877in}}%
\pgfpathlineto{\pgfqpoint{4.067317in}{2.064144in}}%
\pgfpathlineto{\pgfqpoint{4.082751in}{2.074813in}}%
\pgfpathlineto{\pgfqpoint{4.098184in}{2.078767in}}%
\pgfpathlineto{\pgfqpoint{4.113618in}{2.076052in}}%
\pgfpathlineto{\pgfqpoint{4.129052in}{2.066882in}}%
\pgfpathlineto{\pgfqpoint{4.144485in}{2.051627in}}%
\pgfpathlineto{\pgfqpoint{4.159919in}{2.030811in}}%
\pgfpathlineto{\pgfqpoint{4.175352in}{2.005091in}}%
\pgfpathlineto{\pgfqpoint{4.190786in}{1.975245in}}%
\pgfpathlineto{\pgfqpoint{4.221653in}{1.906754in}}%
\pgfpathlineto{\pgfqpoint{4.283388in}{1.762331in}}%
\pgfpathlineto{\pgfqpoint{4.298822in}{1.730467in}}%
\pgfpathlineto{\pgfqpoint{4.314255in}{1.702099in}}%
\pgfpathlineto{\pgfqpoint{4.329689in}{1.677971in}}%
\pgfpathlineto{\pgfqpoint{4.345122in}{1.658707in}}%
\pgfpathlineto{\pgfqpoint{4.360556in}{1.644795in}}%
\pgfpathlineto{\pgfqpoint{4.375990in}{1.636580in}}%
\pgfpathlineto{\pgfqpoint{4.391423in}{1.634253in}}%
\pgfpathlineto{\pgfqpoint{4.406857in}{1.637854in}}%
\pgfpathlineto{\pgfqpoint{4.422291in}{1.647274in}}%
\pgfpathlineto{\pgfqpoint{4.437724in}{1.662258in}}%
\pgfpathlineto{\pgfqpoint{4.453158in}{1.682421in}}%
\pgfpathlineto{\pgfqpoint{4.468592in}{1.707260in}}%
\pgfpathlineto{\pgfqpoint{4.484025in}{1.736173in}}%
\pgfpathlineto{\pgfqpoint{4.499459in}{1.768475in}}%
\pgfpathlineto{\pgfqpoint{4.530326in}{1.840246in}}%
\pgfpathlineto{\pgfqpoint{4.592061in}{1.990710in}}%
\pgfpathlineto{\pgfqpoint{4.622928in}{2.058068in}}%
\pgfpathlineto{\pgfqpoint{4.638362in}{2.087824in}}%
\pgfpathlineto{\pgfqpoint{4.653795in}{2.114470in}}%
\pgfpathlineto{\pgfqpoint{4.669229in}{2.137766in}}%
\pgfpathlineto{\pgfqpoint{4.669229in}{2.137766in}}%
\pgfusepath{stroke}%
\end{pgfscope}%
\begin{pgfscope}%
\pgfpathrectangle{\pgfqpoint{0.634105in}{0.521603in}}{\pgfqpoint{4.227273in}{2.800000in}} %
\pgfusepath{clip}%
\pgfsetrectcap%
\pgfsetroundjoin%
\pgfsetlinewidth{0.501875pt}%
\definecolor{currentstroke}{rgb}{0.135294,0.840344,0.878081}%
\pgfsetstrokecolor{currentstroke}%
\pgfsetdash{}{0pt}%
\pgfpathmoveto{\pgfqpoint{0.826254in}{2.569212in}}%
\pgfpathlineto{\pgfqpoint{0.841687in}{2.564917in}}%
\pgfpathlineto{\pgfqpoint{0.857121in}{2.553516in}}%
\pgfpathlineto{\pgfqpoint{0.872555in}{2.534882in}}%
\pgfpathlineto{\pgfqpoint{0.887988in}{2.509026in}}%
\pgfpathlineto{\pgfqpoint{0.903422in}{2.476103in}}%
\pgfpathlineto{\pgfqpoint{0.918855in}{2.436410in}}%
\pgfpathlineto{\pgfqpoint{0.934289in}{2.390389in}}%
\pgfpathlineto{\pgfqpoint{0.949723in}{2.338621in}}%
\pgfpathlineto{\pgfqpoint{0.980590in}{2.220810in}}%
\pgfpathlineto{\pgfqpoint{1.011457in}{2.089991in}}%
\pgfpathlineto{\pgfqpoint{1.073192in}{1.823439in}}%
\pgfpathlineto{\pgfqpoint{1.104059in}{1.705661in}}%
\pgfpathlineto{\pgfqpoint{1.119493in}{1.654338in}}%
\pgfpathlineto{\pgfqpoint{1.134926in}{1.609260in}}%
\pgfpathlineto{\pgfqpoint{1.150360in}{1.571188in}}%
\pgfpathlineto{\pgfqpoint{1.165794in}{1.540747in}}%
\pgfpathlineto{\pgfqpoint{1.181227in}{1.518419in}}%
\pgfpathlineto{\pgfqpoint{1.196661in}{1.504528in}}%
\pgfpathlineto{\pgfqpoint{1.212095in}{1.499238in}}%
\pgfpathlineto{\pgfqpoint{1.227528in}{1.502552in}}%
\pgfpathlineto{\pgfqpoint{1.242962in}{1.514310in}}%
\pgfpathlineto{\pgfqpoint{1.258395in}{1.534197in}}%
\pgfpathlineto{\pgfqpoint{1.273829in}{1.561755in}}%
\pgfpathlineto{\pgfqpoint{1.289263in}{1.596390in}}%
\pgfpathlineto{\pgfqpoint{1.304696in}{1.637391in}}%
\pgfpathlineto{\pgfqpoint{1.320130in}{1.683948in}}%
\pgfpathlineto{\pgfqpoint{1.350997in}{1.790100in}}%
\pgfpathlineto{\pgfqpoint{1.397298in}{1.967210in}}%
\pgfpathlineto{\pgfqpoint{1.428165in}{2.085662in}}%
\pgfpathlineto{\pgfqpoint{1.459033in}{2.195931in}}%
\pgfpathlineto{\pgfqpoint{1.474466in}{2.246069in}}%
\pgfpathlineto{\pgfqpoint{1.489900in}{2.292033in}}%
\pgfpathlineto{\pgfqpoint{1.505334in}{2.333316in}}%
\pgfpathlineto{\pgfqpoint{1.520767in}{2.369523in}}%
\pgfpathlineto{\pgfqpoint{1.536201in}{2.400363in}}%
\pgfpathlineto{\pgfqpoint{1.551634in}{2.425656in}}%
\pgfpathlineto{\pgfqpoint{1.567068in}{2.445320in}}%
\pgfpathlineto{\pgfqpoint{1.582502in}{2.459370in}}%
\pgfpathlineto{\pgfqpoint{1.597935in}{2.467908in}}%
\pgfpathlineto{\pgfqpoint{1.613369in}{2.471112in}}%
\pgfpathlineto{\pgfqpoint{1.628803in}{2.469224in}}%
\pgfpathlineto{\pgfqpoint{1.644236in}{2.462542in}}%
\pgfpathlineto{\pgfqpoint{1.659670in}{2.451403in}}%
\pgfpathlineto{\pgfqpoint{1.675104in}{2.436174in}}%
\pgfpathlineto{\pgfqpoint{1.690537in}{2.417242in}}%
\pgfpathlineto{\pgfqpoint{1.705971in}{2.395000in}}%
\pgfpathlineto{\pgfqpoint{1.721404in}{2.369839in}}%
\pgfpathlineto{\pgfqpoint{1.752272in}{2.312277in}}%
\pgfpathlineto{\pgfqpoint{1.783139in}{2.247409in}}%
\pgfpathlineto{\pgfqpoint{1.829440in}{2.141781in}}%
\pgfpathlineto{\pgfqpoint{1.922042in}{1.924592in}}%
\pgfpathlineto{\pgfqpoint{1.952909in}{1.856647in}}%
\pgfpathlineto{\pgfqpoint{1.983776in}{1.793839in}}%
\pgfpathlineto{\pgfqpoint{2.014644in}{1.738121in}}%
\pgfpathlineto{\pgfqpoint{2.030077in}{1.713561in}}%
\pgfpathlineto{\pgfqpoint{2.045511in}{1.691542in}}%
\pgfpathlineto{\pgfqpoint{2.060944in}{1.672313in}}%
\pgfpathlineto{\pgfqpoint{2.076378in}{1.656105in}}%
\pgfpathlineto{\pgfqpoint{2.091812in}{1.643126in}}%
\pgfpathlineto{\pgfqpoint{2.107245in}{1.633553in}}%
\pgfpathlineto{\pgfqpoint{2.122679in}{1.627520in}}%
\pgfpathlineto{\pgfqpoint{2.138113in}{1.625116in}}%
\pgfpathlineto{\pgfqpoint{2.153546in}{1.626373in}}%
\pgfpathlineto{\pgfqpoint{2.168980in}{1.631261in}}%
\pgfpathlineto{\pgfqpoint{2.184414in}{1.639686in}}%
\pgfpathlineto{\pgfqpoint{2.199847in}{1.651480in}}%
\pgfpathlineto{\pgfqpoint{2.215281in}{1.666407in}}%
\pgfpathlineto{\pgfqpoint{2.230714in}{1.684158in}}%
\pgfpathlineto{\pgfqpoint{2.246148in}{1.704355in}}%
\pgfpathlineto{\pgfqpoint{2.277015in}{1.750266in}}%
\pgfpathlineto{\pgfqpoint{2.338750in}{1.848726in}}%
\pgfpathlineto{\pgfqpoint{2.354184in}{1.871191in}}%
\pgfpathlineto{\pgfqpoint{2.369617in}{1.891572in}}%
\pgfpathlineto{\pgfqpoint{2.385051in}{1.909298in}}%
\pgfpathlineto{\pgfqpoint{2.400484in}{1.923844in}}%
\pgfpathlineto{\pgfqpoint{2.415918in}{1.934747in}}%
\pgfpathlineto{\pgfqpoint{2.431352in}{1.941618in}}%
\pgfpathlineto{\pgfqpoint{2.446785in}{1.944150in}}%
\pgfpathlineto{\pgfqpoint{2.462219in}{1.942135in}}%
\pgfpathlineto{\pgfqpoint{2.477653in}{1.935461in}}%
\pgfpathlineto{\pgfqpoint{2.493086in}{1.924125in}}%
\pgfpathlineto{\pgfqpoint{2.508520in}{1.908230in}}%
\pgfpathlineto{\pgfqpoint{2.523954in}{1.887984in}}%
\pgfpathlineto{\pgfqpoint{2.539387in}{1.863698in}}%
\pgfpathlineto{\pgfqpoint{2.554821in}{1.835778in}}%
\pgfpathlineto{\pgfqpoint{2.570254in}{1.804718in}}%
\pgfpathlineto{\pgfqpoint{2.601122in}{1.735508in}}%
\pgfpathlineto{\pgfqpoint{2.678290in}{1.552733in}}%
\pgfpathlineto{\pgfqpoint{2.693724in}{1.520079in}}%
\pgfpathlineto{\pgfqpoint{2.709157in}{1.490216in}}%
\pgfpathlineto{\pgfqpoint{2.724591in}{1.463667in}}%
\pgfpathlineto{\pgfqpoint{2.740024in}{1.440876in}}%
\pgfpathlineto{\pgfqpoint{2.755458in}{1.422194in}}%
\pgfpathlineto{\pgfqpoint{2.770892in}{1.407876in}}%
\pgfpathlineto{\pgfqpoint{2.786325in}{1.398072in}}%
\pgfpathlineto{\pgfqpoint{2.801759in}{1.392829in}}%
\pgfpathlineto{\pgfqpoint{2.817193in}{1.392092in}}%
\pgfpathlineto{\pgfqpoint{2.832626in}{1.395706in}}%
\pgfpathlineto{\pgfqpoint{2.848060in}{1.403423in}}%
\pgfpathlineto{\pgfqpoint{2.863493in}{1.414912in}}%
\pgfpathlineto{\pgfqpoint{2.878927in}{1.429769in}}%
\pgfpathlineto{\pgfqpoint{2.894361in}{1.447531in}}%
\pgfpathlineto{\pgfqpoint{2.909794in}{1.467688in}}%
\pgfpathlineto{\pgfqpoint{2.940662in}{1.513006in}}%
\pgfpathlineto{\pgfqpoint{3.002396in}{1.608307in}}%
\pgfpathlineto{\pgfqpoint{3.033263in}{1.650439in}}%
\pgfpathlineto{\pgfqpoint{3.048697in}{1.668783in}}%
\pgfpathlineto{\pgfqpoint{3.064131in}{1.684970in}}%
\pgfpathlineto{\pgfqpoint{3.079564in}{1.698826in}}%
\pgfpathlineto{\pgfqpoint{3.094998in}{1.710251in}}%
\pgfpathlineto{\pgfqpoint{3.110432in}{1.719208in}}%
\pgfpathlineto{\pgfqpoint{3.125865in}{1.725721in}}%
\pgfpathlineto{\pgfqpoint{3.141299in}{1.729870in}}%
\pgfpathlineto{\pgfqpoint{3.156733in}{1.731779in}}%
\pgfpathlineto{\pgfqpoint{3.172166in}{1.731612in}}%
\pgfpathlineto{\pgfqpoint{3.187600in}{1.729558in}}%
\pgfpathlineto{\pgfqpoint{3.203033in}{1.725824in}}%
\pgfpathlineto{\pgfqpoint{3.218467in}{1.720623in}}%
\pgfpathlineto{\pgfqpoint{3.249334in}{1.706653in}}%
\pgfpathlineto{\pgfqpoint{3.280202in}{1.689155in}}%
\pgfpathlineto{\pgfqpoint{3.311069in}{1.669249in}}%
\pgfpathlineto{\pgfqpoint{3.357370in}{1.636102in}}%
\pgfpathlineto{\pgfqpoint{3.388237in}{1.611875in}}%
\pgfpathlineto{\pgfqpoint{3.419104in}{1.585602in}}%
\pgfpathlineto{\pgfqpoint{3.449972in}{1.556951in}}%
\pgfpathlineto{\pgfqpoint{3.480839in}{1.525859in}}%
\pgfpathlineto{\pgfqpoint{3.542573in}{1.458934in}}%
\pgfpathlineto{\pgfqpoint{3.573441in}{1.426279in}}%
\pgfpathlineto{\pgfqpoint{3.604308in}{1.397568in}}%
\pgfpathlineto{\pgfqpoint{3.619742in}{1.385709in}}%
\pgfpathlineto{\pgfqpoint{3.635175in}{1.376099in}}%
\pgfpathlineto{\pgfqpoint{3.650609in}{1.369186in}}%
\pgfpathlineto{\pgfqpoint{3.666043in}{1.365405in}}%
\pgfpathlineto{\pgfqpoint{3.681476in}{1.365164in}}%
\pgfpathlineto{\pgfqpoint{3.696910in}{1.368834in}}%
\pgfpathlineto{\pgfqpoint{3.712343in}{1.376727in}}%
\pgfpathlineto{\pgfqpoint{3.727777in}{1.389090in}}%
\pgfpathlineto{\pgfqpoint{3.743211in}{1.406086in}}%
\pgfpathlineto{\pgfqpoint{3.758644in}{1.427786in}}%
\pgfpathlineto{\pgfqpoint{3.774078in}{1.454161in}}%
\pgfpathlineto{\pgfqpoint{3.789512in}{1.485074in}}%
\pgfpathlineto{\pgfqpoint{3.804945in}{1.520277in}}%
\pgfpathlineto{\pgfqpoint{3.820379in}{1.559411in}}%
\pgfpathlineto{\pgfqpoint{3.851246in}{1.647500in}}%
\pgfpathlineto{\pgfqpoint{3.882113in}{1.744432in}}%
\pgfpathlineto{\pgfqpoint{3.928414in}{1.892658in}}%
\pgfpathlineto{\pgfqpoint{3.959282in}{1.983143in}}%
\pgfpathlineto{\pgfqpoint{3.974715in}{2.023260in}}%
\pgfpathlineto{\pgfqpoint{3.990149in}{2.058887in}}%
\pgfpathlineto{\pgfqpoint{4.005583in}{2.089308in}}%
\pgfpathlineto{\pgfqpoint{4.021016in}{2.113907in}}%
\pgfpathlineto{\pgfqpoint{4.036450in}{2.132181in}}%
\pgfpathlineto{\pgfqpoint{4.051883in}{2.143757in}}%
\pgfpathlineto{\pgfqpoint{4.067317in}{2.148402in}}%
\pgfpathlineto{\pgfqpoint{4.082751in}{2.146031in}}%
\pgfpathlineto{\pgfqpoint{4.098184in}{2.136714in}}%
\pgfpathlineto{\pgfqpoint{4.113618in}{2.120674in}}%
\pgfpathlineto{\pgfqpoint{4.129052in}{2.098283in}}%
\pgfpathlineto{\pgfqpoint{4.144485in}{2.070058in}}%
\pgfpathlineto{\pgfqpoint{4.159919in}{2.036648in}}%
\pgfpathlineto{\pgfqpoint{4.175352in}{1.998820in}}%
\pgfpathlineto{\pgfqpoint{4.206220in}{1.913457in}}%
\pgfpathlineto{\pgfqpoint{4.267954in}{1.731937in}}%
\pgfpathlineto{\pgfqpoint{4.283388in}{1.690314in}}%
\pgfpathlineto{\pgfqpoint{4.298822in}{1.652133in}}%
\pgfpathlineto{\pgfqpoint{4.314255in}{1.618241in}}%
\pgfpathlineto{\pgfqpoint{4.329689in}{1.589377in}}%
\pgfpathlineto{\pgfqpoint{4.345122in}{1.566164in}}%
\pgfpathlineto{\pgfqpoint{4.360556in}{1.549090in}}%
\pgfpathlineto{\pgfqpoint{4.375990in}{1.538498in}}%
\pgfpathlineto{\pgfqpoint{4.391423in}{1.534586in}}%
\pgfpathlineto{\pgfqpoint{4.406857in}{1.537399in}}%
\pgfpathlineto{\pgfqpoint{4.422291in}{1.546834in}}%
\pgfpathlineto{\pgfqpoint{4.437724in}{1.562647in}}%
\pgfpathlineto{\pgfqpoint{4.453158in}{1.584458in}}%
\pgfpathlineto{\pgfqpoint{4.468592in}{1.611768in}}%
\pgfpathlineto{\pgfqpoint{4.484025in}{1.643971in}}%
\pgfpathlineto{\pgfqpoint{4.499459in}{1.680375in}}%
\pgfpathlineto{\pgfqpoint{4.530326in}{1.762696in}}%
\pgfpathlineto{\pgfqpoint{4.622928in}{2.026872in}}%
\pgfpathlineto{\pgfqpoint{4.638362in}{2.065496in}}%
\pgfpathlineto{\pgfqpoint{4.653795in}{2.100984in}}%
\pgfpathlineto{\pgfqpoint{4.669229in}{2.132938in}}%
\pgfpathlineto{\pgfqpoint{4.669229in}{2.132938in}}%
\pgfusepath{stroke}%
\end{pgfscope}%
\begin{pgfscope}%
\pgfpathrectangle{\pgfqpoint{0.634105in}{0.521603in}}{\pgfqpoint{4.227273in}{2.800000in}} %
\pgfusepath{clip}%
\pgfsetrectcap%
\pgfsetroundjoin%
\pgfsetlinewidth{0.501875pt}%
\definecolor{currentstroke}{rgb}{0.221569,0.905873,0.843667}%
\pgfsetstrokecolor{currentstroke}%
\pgfsetdash{}{0pt}%
\pgfpathmoveto{\pgfqpoint{0.826254in}{2.305588in}}%
\pgfpathlineto{\pgfqpoint{0.841687in}{2.317868in}}%
\pgfpathlineto{\pgfqpoint{0.857121in}{2.326476in}}%
\pgfpathlineto{\pgfqpoint{0.872555in}{2.331154in}}%
\pgfpathlineto{\pgfqpoint{0.887988in}{2.331705in}}%
\pgfpathlineto{\pgfqpoint{0.903422in}{2.327993in}}%
\pgfpathlineto{\pgfqpoint{0.918855in}{2.319952in}}%
\pgfpathlineto{\pgfqpoint{0.934289in}{2.307588in}}%
\pgfpathlineto{\pgfqpoint{0.949723in}{2.290980in}}%
\pgfpathlineto{\pgfqpoint{0.965156in}{2.270284in}}%
\pgfpathlineto{\pgfqpoint{0.980590in}{2.245732in}}%
\pgfpathlineto{\pgfqpoint{0.996024in}{2.217628in}}%
\pgfpathlineto{\pgfqpoint{1.011457in}{2.186346in}}%
\pgfpathlineto{\pgfqpoint{1.042325in}{2.116067in}}%
\pgfpathlineto{\pgfqpoint{1.088625in}{1.999539in}}%
\pgfpathlineto{\pgfqpoint{1.119493in}{1.921656in}}%
\pgfpathlineto{\pgfqpoint{1.150360in}{1.849695in}}%
\pgfpathlineto{\pgfqpoint{1.165794in}{1.817524in}}%
\pgfpathlineto{\pgfqpoint{1.181227in}{1.788683in}}%
\pgfpathlineto{\pgfqpoint{1.196661in}{1.763708in}}%
\pgfpathlineto{\pgfqpoint{1.212095in}{1.743079in}}%
\pgfpathlineto{\pgfqpoint{1.227528in}{1.727208in}}%
\pgfpathlineto{\pgfqpoint{1.242962in}{1.716435in}}%
\pgfpathlineto{\pgfqpoint{1.258395in}{1.711017in}}%
\pgfpathlineto{\pgfqpoint{1.273829in}{1.711124in}}%
\pgfpathlineto{\pgfqpoint{1.289263in}{1.716837in}}%
\pgfpathlineto{\pgfqpoint{1.304696in}{1.728144in}}%
\pgfpathlineto{\pgfqpoint{1.320130in}{1.744938in}}%
\pgfpathlineto{\pgfqpoint{1.335564in}{1.767023in}}%
\pgfpathlineto{\pgfqpoint{1.350997in}{1.794114in}}%
\pgfpathlineto{\pgfqpoint{1.366431in}{1.825838in}}%
\pgfpathlineto{\pgfqpoint{1.381864in}{1.861750in}}%
\pgfpathlineto{\pgfqpoint{1.397298in}{1.901331in}}%
\pgfpathlineto{\pgfqpoint{1.428165in}{1.989132in}}%
\pgfpathlineto{\pgfqpoint{1.474466in}{2.132448in}}%
\pgfpathlineto{\pgfqpoint{1.505334in}{2.227487in}}%
\pgfpathlineto{\pgfqpoint{1.536201in}{2.315604in}}%
\pgfpathlineto{\pgfqpoint{1.551634in}{2.355462in}}%
\pgfpathlineto{\pgfqpoint{1.567068in}{2.391760in}}%
\pgfpathlineto{\pgfqpoint{1.582502in}{2.424004in}}%
\pgfpathlineto{\pgfqpoint{1.597935in}{2.451773in}}%
\pgfpathlineto{\pgfqpoint{1.613369in}{2.474722in}}%
\pgfpathlineto{\pgfqpoint{1.628803in}{2.492587in}}%
\pgfpathlineto{\pgfqpoint{1.644236in}{2.505193in}}%
\pgfpathlineto{\pgfqpoint{1.659670in}{2.512449in}}%
\pgfpathlineto{\pgfqpoint{1.675104in}{2.514351in}}%
\pgfpathlineto{\pgfqpoint{1.690537in}{2.510981in}}%
\pgfpathlineto{\pgfqpoint{1.705971in}{2.502504in}}%
\pgfpathlineto{\pgfqpoint{1.721404in}{2.489160in}}%
\pgfpathlineto{\pgfqpoint{1.736838in}{2.471260in}}%
\pgfpathlineto{\pgfqpoint{1.752272in}{2.449180in}}%
\pgfpathlineto{\pgfqpoint{1.767705in}{2.423349in}}%
\pgfpathlineto{\pgfqpoint{1.783139in}{2.394245in}}%
\pgfpathlineto{\pgfqpoint{1.814006in}{2.328283in}}%
\pgfpathlineto{\pgfqpoint{1.844874in}{2.255627in}}%
\pgfpathlineto{\pgfqpoint{1.906608in}{2.107561in}}%
\pgfpathlineto{\pgfqpoint{1.937475in}{2.039877in}}%
\pgfpathlineto{\pgfqpoint{1.968343in}{1.980439in}}%
\pgfpathlineto{\pgfqpoint{1.983776in}{1.954442in}}%
\pgfpathlineto{\pgfqpoint{1.999210in}{1.931138in}}%
\pgfpathlineto{\pgfqpoint{2.014644in}{1.910608in}}%
\pgfpathlineto{\pgfqpoint{2.030077in}{1.892871in}}%
\pgfpathlineto{\pgfqpoint{2.045511in}{1.877887in}}%
\pgfpathlineto{\pgfqpoint{2.060944in}{1.865561in}}%
\pgfpathlineto{\pgfqpoint{2.076378in}{1.855747in}}%
\pgfpathlineto{\pgfqpoint{2.091812in}{1.848253in}}%
\pgfpathlineto{\pgfqpoint{2.107245in}{1.842849in}}%
\pgfpathlineto{\pgfqpoint{2.122679in}{1.839273in}}%
\pgfpathlineto{\pgfqpoint{2.138113in}{1.837241in}}%
\pgfpathlineto{\pgfqpoint{2.168980in}{1.836582in}}%
\pgfpathlineto{\pgfqpoint{2.261582in}{1.840158in}}%
\pgfpathlineto{\pgfqpoint{2.292449in}{1.836429in}}%
\pgfpathlineto{\pgfqpoint{2.307883in}{1.832866in}}%
\pgfpathlineto{\pgfqpoint{2.323316in}{1.828067in}}%
\pgfpathlineto{\pgfqpoint{2.338750in}{1.822002in}}%
\pgfpathlineto{\pgfqpoint{2.369617in}{1.806159in}}%
\pgfpathlineto{\pgfqpoint{2.400484in}{1.785859in}}%
\pgfpathlineto{\pgfqpoint{2.431352in}{1.762116in}}%
\pgfpathlineto{\pgfqpoint{2.539387in}{1.672995in}}%
\pgfpathlineto{\pgfqpoint{2.570254in}{1.652013in}}%
\pgfpathlineto{\pgfqpoint{2.601122in}{1.635171in}}%
\pgfpathlineto{\pgfqpoint{2.616555in}{1.628496in}}%
\pgfpathlineto{\pgfqpoint{2.631989in}{1.623032in}}%
\pgfpathlineto{\pgfqpoint{2.647423in}{1.618777in}}%
\pgfpathlineto{\pgfqpoint{2.678290in}{1.613745in}}%
\pgfpathlineto{\pgfqpoint{2.709157in}{1.612844in}}%
\pgfpathlineto{\pgfqpoint{2.740024in}{1.615155in}}%
\pgfpathlineto{\pgfqpoint{2.786325in}{1.622024in}}%
\pgfpathlineto{\pgfqpoint{2.832626in}{1.628985in}}%
\pgfpathlineto{\pgfqpoint{2.863493in}{1.631551in}}%
\pgfpathlineto{\pgfqpoint{2.894361in}{1.631222in}}%
\pgfpathlineto{\pgfqpoint{2.925228in}{1.627251in}}%
\pgfpathlineto{\pgfqpoint{2.956095in}{1.619224in}}%
\pgfpathlineto{\pgfqpoint{2.986963in}{1.607070in}}%
\pgfpathlineto{\pgfqpoint{3.017830in}{1.591050in}}%
\pgfpathlineto{\pgfqpoint{3.048697in}{1.571717in}}%
\pgfpathlineto{\pgfqpoint{3.079564in}{1.549849in}}%
\pgfpathlineto{\pgfqpoint{3.141299in}{1.502332in}}%
\pgfpathlineto{\pgfqpoint{3.187600in}{1.467399in}}%
\pgfpathlineto{\pgfqpoint{3.218467in}{1.446268in}}%
\pgfpathlineto{\pgfqpoint{3.249334in}{1.427848in}}%
\pgfpathlineto{\pgfqpoint{3.280202in}{1.412848in}}%
\pgfpathlineto{\pgfqpoint{3.311069in}{1.401850in}}%
\pgfpathlineto{\pgfqpoint{3.326503in}{1.398001in}}%
\pgfpathlineto{\pgfqpoint{3.341936in}{1.395320in}}%
\pgfpathlineto{\pgfqpoint{3.357370in}{1.393850in}}%
\pgfpathlineto{\pgfqpoint{3.372803in}{1.393627in}}%
\pgfpathlineto{\pgfqpoint{3.388237in}{1.394685in}}%
\pgfpathlineto{\pgfqpoint{3.403671in}{1.397053in}}%
\pgfpathlineto{\pgfqpoint{3.419104in}{1.400755in}}%
\pgfpathlineto{\pgfqpoint{3.434538in}{1.405810in}}%
\pgfpathlineto{\pgfqpoint{3.449972in}{1.412234in}}%
\pgfpathlineto{\pgfqpoint{3.465405in}{1.420039in}}%
\pgfpathlineto{\pgfqpoint{3.480839in}{1.429228in}}%
\pgfpathlineto{\pgfqpoint{3.496273in}{1.439800in}}%
\pgfpathlineto{\pgfqpoint{3.527140in}{1.465054in}}%
\pgfpathlineto{\pgfqpoint{3.558007in}{1.495626in}}%
\pgfpathlineto{\pgfqpoint{3.588874in}{1.531173in}}%
\pgfpathlineto{\pgfqpoint{3.619742in}{1.571144in}}%
\pgfpathlineto{\pgfqpoint{3.650609in}{1.614753in}}%
\pgfpathlineto{\pgfqpoint{3.696910in}{1.684656in}}%
\pgfpathlineto{\pgfqpoint{3.758644in}{1.778991in}}%
\pgfpathlineto{\pgfqpoint{3.789512in}{1.823022in}}%
\pgfpathlineto{\pgfqpoint{3.820379in}{1.862706in}}%
\pgfpathlineto{\pgfqpoint{3.851246in}{1.896449in}}%
\pgfpathlineto{\pgfqpoint{3.866680in}{1.910655in}}%
\pgfpathlineto{\pgfqpoint{3.882113in}{1.922889in}}%
\pgfpathlineto{\pgfqpoint{3.897547in}{1.933037in}}%
\pgfpathlineto{\pgfqpoint{3.912981in}{1.941014in}}%
\pgfpathlineto{\pgfqpoint{3.928414in}{1.946767in}}%
\pgfpathlineto{\pgfqpoint{3.943848in}{1.950279in}}%
\pgfpathlineto{\pgfqpoint{3.959282in}{1.951569in}}%
\pgfpathlineto{\pgfqpoint{3.974715in}{1.950691in}}%
\pgfpathlineto{\pgfqpoint{3.990149in}{1.947743in}}%
\pgfpathlineto{\pgfqpoint{4.005583in}{1.942855in}}%
\pgfpathlineto{\pgfqpoint{4.021016in}{1.936200in}}%
\pgfpathlineto{\pgfqpoint{4.036450in}{1.927983in}}%
\pgfpathlineto{\pgfqpoint{4.067317in}{1.907846in}}%
\pgfpathlineto{\pgfqpoint{4.159919in}{1.839741in}}%
\pgfpathlineto{\pgfqpoint{4.175352in}{1.830770in}}%
\pgfpathlineto{\pgfqpoint{4.190786in}{1.823357in}}%
\pgfpathlineto{\pgfqpoint{4.206220in}{1.817795in}}%
\pgfpathlineto{\pgfqpoint{4.221653in}{1.814349in}}%
\pgfpathlineto{\pgfqpoint{4.237087in}{1.813251in}}%
\pgfpathlineto{\pgfqpoint{4.252521in}{1.814691in}}%
\pgfpathlineto{\pgfqpoint{4.267954in}{1.818817in}}%
\pgfpathlineto{\pgfqpoint{4.283388in}{1.825727in}}%
\pgfpathlineto{\pgfqpoint{4.298822in}{1.835472in}}%
\pgfpathlineto{\pgfqpoint{4.314255in}{1.848047in}}%
\pgfpathlineto{\pgfqpoint{4.329689in}{1.863396in}}%
\pgfpathlineto{\pgfqpoint{4.345122in}{1.881407in}}%
\pgfpathlineto{\pgfqpoint{4.360556in}{1.901919in}}%
\pgfpathlineto{\pgfqpoint{4.375990in}{1.924720in}}%
\pgfpathlineto{\pgfqpoint{4.406857in}{1.976110in}}%
\pgfpathlineto{\pgfqpoint{4.437724in}{2.033035in}}%
\pgfpathlineto{\pgfqpoint{4.514892in}{2.179058in}}%
\pgfpathlineto{\pgfqpoint{4.545760in}{2.230315in}}%
\pgfpathlineto{\pgfqpoint{4.561193in}{2.252877in}}%
\pgfpathlineto{\pgfqpoint{4.576627in}{2.272966in}}%
\pgfpathlineto{\pgfqpoint{4.592061in}{2.290313in}}%
\pgfpathlineto{\pgfqpoint{4.607494in}{2.304694in}}%
\pgfpathlineto{\pgfqpoint{4.622928in}{2.315939in}}%
\pgfpathlineto{\pgfqpoint{4.638362in}{2.323930in}}%
\pgfpathlineto{\pgfqpoint{4.653795in}{2.328607in}}%
\pgfpathlineto{\pgfqpoint{4.669229in}{2.329967in}}%
\pgfpathlineto{\pgfqpoint{4.669229in}{2.329967in}}%
\pgfusepath{stroke}%
\end{pgfscope}%
\begin{pgfscope}%
\pgfpathrectangle{\pgfqpoint{0.634105in}{0.521603in}}{\pgfqpoint{4.227273in}{2.800000in}} %
\pgfusepath{clip}%
\pgfsetrectcap%
\pgfsetroundjoin%
\pgfsetlinewidth{0.501875pt}%
\definecolor{currentstroke}{rgb}{0.300000,0.951057,0.809017}%
\pgfsetstrokecolor{currentstroke}%
\pgfsetdash{}{0pt}%
\pgfpathmoveto{\pgfqpoint{0.826254in}{2.359767in}}%
\pgfpathlineto{\pgfqpoint{0.841687in}{2.360285in}}%
\pgfpathlineto{\pgfqpoint{0.857121in}{2.355948in}}%
\pgfpathlineto{\pgfqpoint{0.872555in}{2.346688in}}%
\pgfpathlineto{\pgfqpoint{0.887988in}{2.332506in}}%
\pgfpathlineto{\pgfqpoint{0.903422in}{2.313476in}}%
\pgfpathlineto{\pgfqpoint{0.918855in}{2.289746in}}%
\pgfpathlineto{\pgfqpoint{0.934289in}{2.261535in}}%
\pgfpathlineto{\pgfqpoint{0.949723in}{2.229134in}}%
\pgfpathlineto{\pgfqpoint{0.965156in}{2.192900in}}%
\pgfpathlineto{\pgfqpoint{0.996024in}{2.110680in}}%
\pgfpathlineto{\pgfqpoint{1.026891in}{2.018909in}}%
\pgfpathlineto{\pgfqpoint{1.104059in}{1.779874in}}%
\pgfpathlineto{\pgfqpoint{1.134926in}{1.694223in}}%
\pgfpathlineto{\pgfqpoint{1.150360in}{1.656090in}}%
\pgfpathlineto{\pgfqpoint{1.165794in}{1.621849in}}%
\pgfpathlineto{\pgfqpoint{1.181227in}{1.592013in}}%
\pgfpathlineto{\pgfqpoint{1.196661in}{1.567039in}}%
\pgfpathlineto{\pgfqpoint{1.212095in}{1.547317in}}%
\pgfpathlineto{\pgfqpoint{1.227528in}{1.533163in}}%
\pgfpathlineto{\pgfqpoint{1.242962in}{1.524821in}}%
\pgfpathlineto{\pgfqpoint{1.258395in}{1.522448in}}%
\pgfpathlineto{\pgfqpoint{1.273829in}{1.526122in}}%
\pgfpathlineto{\pgfqpoint{1.289263in}{1.535832in}}%
\pgfpathlineto{\pgfqpoint{1.304696in}{1.551483in}}%
\pgfpathlineto{\pgfqpoint{1.320130in}{1.572895in}}%
\pgfpathlineto{\pgfqpoint{1.335564in}{1.599805in}}%
\pgfpathlineto{\pgfqpoint{1.350997in}{1.631873in}}%
\pgfpathlineto{\pgfqpoint{1.366431in}{1.668686in}}%
\pgfpathlineto{\pgfqpoint{1.381864in}{1.709761in}}%
\pgfpathlineto{\pgfqpoint{1.412732in}{1.802488in}}%
\pgfpathlineto{\pgfqpoint{1.443599in}{1.905163in}}%
\pgfpathlineto{\pgfqpoint{1.520767in}{2.169438in}}%
\pgfpathlineto{\pgfqpoint{1.551634in}{2.263821in}}%
\pgfpathlineto{\pgfqpoint{1.567068in}{2.306095in}}%
\pgfpathlineto{\pgfqpoint{1.582502in}{2.344398in}}%
\pgfpathlineto{\pgfqpoint{1.597935in}{2.378284in}}%
\pgfpathlineto{\pgfqpoint{1.613369in}{2.407376in}}%
\pgfpathlineto{\pgfqpoint{1.628803in}{2.431370in}}%
\pgfpathlineto{\pgfqpoint{1.644236in}{2.450040in}}%
\pgfpathlineto{\pgfqpoint{1.659670in}{2.463236in}}%
\pgfpathlineto{\pgfqpoint{1.675104in}{2.470890in}}%
\pgfpathlineto{\pgfqpoint{1.690537in}{2.473012in}}%
\pgfpathlineto{\pgfqpoint{1.705971in}{2.469690in}}%
\pgfpathlineto{\pgfqpoint{1.721404in}{2.461086in}}%
\pgfpathlineto{\pgfqpoint{1.736838in}{2.447436in}}%
\pgfpathlineto{\pgfqpoint{1.752272in}{2.429038in}}%
\pgfpathlineto{\pgfqpoint{1.767705in}{2.406252in}}%
\pgfpathlineto{\pgfqpoint{1.783139in}{2.379489in}}%
\pgfpathlineto{\pgfqpoint{1.798573in}{2.349208in}}%
\pgfpathlineto{\pgfqpoint{1.829440in}{2.280093in}}%
\pgfpathlineto{\pgfqpoint{1.860307in}{2.203141in}}%
\pgfpathlineto{\pgfqpoint{1.937475in}{2.004923in}}%
\pgfpathlineto{\pgfqpoint{1.968343in}{1.933652in}}%
\pgfpathlineto{\pgfqpoint{1.983776in}{1.901350in}}%
\pgfpathlineto{\pgfqpoint{1.999210in}{1.871662in}}%
\pgfpathlineto{\pgfqpoint{2.014644in}{1.844813in}}%
\pgfpathlineto{\pgfqpoint{2.030077in}{1.820967in}}%
\pgfpathlineto{\pgfqpoint{2.045511in}{1.800231in}}%
\pgfpathlineto{\pgfqpoint{2.060944in}{1.782649in}}%
\pgfpathlineto{\pgfqpoint{2.076378in}{1.768210in}}%
\pgfpathlineto{\pgfqpoint{2.091812in}{1.756846in}}%
\pgfpathlineto{\pgfqpoint{2.107245in}{1.748435in}}%
\pgfpathlineto{\pgfqpoint{2.122679in}{1.742809in}}%
\pgfpathlineto{\pgfqpoint{2.138113in}{1.739756in}}%
\pgfpathlineto{\pgfqpoint{2.153546in}{1.739026in}}%
\pgfpathlineto{\pgfqpoint{2.168980in}{1.740335in}}%
\pgfpathlineto{\pgfqpoint{2.184414in}{1.743379in}}%
\pgfpathlineto{\pgfqpoint{2.215281in}{1.753351in}}%
\pgfpathlineto{\pgfqpoint{2.307883in}{1.790401in}}%
\pgfpathlineto{\pgfqpoint{2.323316in}{1.794451in}}%
\pgfpathlineto{\pgfqpoint{2.338750in}{1.797256in}}%
\pgfpathlineto{\pgfqpoint{2.354184in}{1.798654in}}%
\pgfpathlineto{\pgfqpoint{2.369617in}{1.798526in}}%
\pgfpathlineto{\pgfqpoint{2.385051in}{1.796794in}}%
\pgfpathlineto{\pgfqpoint{2.400484in}{1.793420in}}%
\pgfpathlineto{\pgfqpoint{2.415918in}{1.788412in}}%
\pgfpathlineto{\pgfqpoint{2.431352in}{1.781818in}}%
\pgfpathlineto{\pgfqpoint{2.446785in}{1.773724in}}%
\pgfpathlineto{\pgfqpoint{2.477653in}{1.753565in}}%
\pgfpathlineto{\pgfqpoint{2.508520in}{1.729295in}}%
\pgfpathlineto{\pgfqpoint{2.601122in}{1.650337in}}%
\pgfpathlineto{\pgfqpoint{2.631989in}{1.628591in}}%
\pgfpathlineto{\pgfqpoint{2.647423in}{1.619590in}}%
\pgfpathlineto{\pgfqpoint{2.662856in}{1.612071in}}%
\pgfpathlineto{\pgfqpoint{2.678290in}{1.606171in}}%
\pgfpathlineto{\pgfqpoint{2.693724in}{1.601997in}}%
\pgfpathlineto{\pgfqpoint{2.709157in}{1.599619in}}%
\pgfpathlineto{\pgfqpoint{2.724591in}{1.599070in}}%
\pgfpathlineto{\pgfqpoint{2.740024in}{1.600347in}}%
\pgfpathlineto{\pgfqpoint{2.755458in}{1.603409in}}%
\pgfpathlineto{\pgfqpoint{2.770892in}{1.608176in}}%
\pgfpathlineto{\pgfqpoint{2.786325in}{1.614536in}}%
\pgfpathlineto{\pgfqpoint{2.801759in}{1.622339in}}%
\pgfpathlineto{\pgfqpoint{2.832626in}{1.641535in}}%
\pgfpathlineto{\pgfqpoint{2.863493in}{1.664020in}}%
\pgfpathlineto{\pgfqpoint{2.925228in}{1.710449in}}%
\pgfpathlineto{\pgfqpoint{2.956095in}{1.729925in}}%
\pgfpathlineto{\pgfqpoint{2.971529in}{1.737787in}}%
\pgfpathlineto{\pgfqpoint{2.986963in}{1.744080in}}%
\pgfpathlineto{\pgfqpoint{3.002396in}{1.748591in}}%
\pgfpathlineto{\pgfqpoint{3.017830in}{1.751138in}}%
\pgfpathlineto{\pgfqpoint{3.033263in}{1.751567in}}%
\pgfpathlineto{\pgfqpoint{3.048697in}{1.749761in}}%
\pgfpathlineto{\pgfqpoint{3.064131in}{1.745637in}}%
\pgfpathlineto{\pgfqpoint{3.079564in}{1.739152in}}%
\pgfpathlineto{\pgfqpoint{3.094998in}{1.730303in}}%
\pgfpathlineto{\pgfqpoint{3.110432in}{1.719125in}}%
\pgfpathlineto{\pgfqpoint{3.125865in}{1.705696in}}%
\pgfpathlineto{\pgfqpoint{3.141299in}{1.690130in}}%
\pgfpathlineto{\pgfqpoint{3.156733in}{1.672581in}}%
\pgfpathlineto{\pgfqpoint{3.187600in}{1.632325in}}%
\pgfpathlineto{\pgfqpoint{3.218467in}{1.586826in}}%
\pgfpathlineto{\pgfqpoint{3.311069in}{1.443379in}}%
\pgfpathlineto{\pgfqpoint{3.341936in}{1.402260in}}%
\pgfpathlineto{\pgfqpoint{3.357370in}{1.384448in}}%
\pgfpathlineto{\pgfqpoint{3.372803in}{1.368864in}}%
\pgfpathlineto{\pgfqpoint{3.388237in}{1.355773in}}%
\pgfpathlineto{\pgfqpoint{3.403671in}{1.345411in}}%
\pgfpathlineto{\pgfqpoint{3.419104in}{1.337983in}}%
\pgfpathlineto{\pgfqpoint{3.434538in}{1.333657in}}%
\pgfpathlineto{\pgfqpoint{3.449972in}{1.332563in}}%
\pgfpathlineto{\pgfqpoint{3.465405in}{1.334789in}}%
\pgfpathlineto{\pgfqpoint{3.480839in}{1.340383in}}%
\pgfpathlineto{\pgfqpoint{3.496273in}{1.349348in}}%
\pgfpathlineto{\pgfqpoint{3.511706in}{1.361645in}}%
\pgfpathlineto{\pgfqpoint{3.527140in}{1.377191in}}%
\pgfpathlineto{\pgfqpoint{3.542573in}{1.395863in}}%
\pgfpathlineto{\pgfqpoint{3.558007in}{1.417495in}}%
\pgfpathlineto{\pgfqpoint{3.573441in}{1.441883in}}%
\pgfpathlineto{\pgfqpoint{3.588874in}{1.468786in}}%
\pgfpathlineto{\pgfqpoint{3.619742in}{1.529017in}}%
\pgfpathlineto{\pgfqpoint{3.650609in}{1.595672in}}%
\pgfpathlineto{\pgfqpoint{3.758644in}{1.837040in}}%
\pgfpathlineto{\pgfqpoint{3.789512in}{1.895973in}}%
\pgfpathlineto{\pgfqpoint{3.804945in}{1.922081in}}%
\pgfpathlineto{\pgfqpoint{3.820379in}{1.945590in}}%
\pgfpathlineto{\pgfqpoint{3.835813in}{1.966271in}}%
\pgfpathlineto{\pgfqpoint{3.851246in}{1.983935in}}%
\pgfpathlineto{\pgfqpoint{3.866680in}{1.998433in}}%
\pgfpathlineto{\pgfqpoint{3.882113in}{2.009656in}}%
\pgfpathlineto{\pgfqpoint{3.897547in}{2.017543in}}%
\pgfpathlineto{\pgfqpoint{3.912981in}{2.022074in}}%
\pgfpathlineto{\pgfqpoint{3.928414in}{2.023276in}}%
\pgfpathlineto{\pgfqpoint{3.943848in}{2.021221in}}%
\pgfpathlineto{\pgfqpoint{3.959282in}{2.016025in}}%
\pgfpathlineto{\pgfqpoint{3.974715in}{2.007845in}}%
\pgfpathlineto{\pgfqpoint{3.990149in}{1.996879in}}%
\pgfpathlineto{\pgfqpoint{4.005583in}{1.983364in}}%
\pgfpathlineto{\pgfqpoint{4.021016in}{1.967569in}}%
\pgfpathlineto{\pgfqpoint{4.051883in}{1.930367in}}%
\pgfpathlineto{\pgfqpoint{4.082751in}{1.887971in}}%
\pgfpathlineto{\pgfqpoint{4.144485in}{1.799571in}}%
\pgfpathlineto{\pgfqpoint{4.175352in}{1.759657in}}%
\pgfpathlineto{\pgfqpoint{4.190786in}{1.742031in}}%
\pgfpathlineto{\pgfqpoint{4.206220in}{1.726385in}}%
\pgfpathlineto{\pgfqpoint{4.221653in}{1.713006in}}%
\pgfpathlineto{\pgfqpoint{4.237087in}{1.702145in}}%
\pgfpathlineto{\pgfqpoint{4.252521in}{1.694019in}}%
\pgfpathlineto{\pgfqpoint{4.267954in}{1.688804in}}%
\pgfpathlineto{\pgfqpoint{4.283388in}{1.686633in}}%
\pgfpathlineto{\pgfqpoint{4.298822in}{1.687597in}}%
\pgfpathlineto{\pgfqpoint{4.314255in}{1.691740in}}%
\pgfpathlineto{\pgfqpoint{4.329689in}{1.699060in}}%
\pgfpathlineto{\pgfqpoint{4.345122in}{1.709509in}}%
\pgfpathlineto{\pgfqpoint{4.360556in}{1.722995in}}%
\pgfpathlineto{\pgfqpoint{4.375990in}{1.739380in}}%
\pgfpathlineto{\pgfqpoint{4.391423in}{1.758487in}}%
\pgfpathlineto{\pgfqpoint{4.406857in}{1.780097in}}%
\pgfpathlineto{\pgfqpoint{4.437724in}{1.829778in}}%
\pgfpathlineto{\pgfqpoint{4.468592in}{1.886031in}}%
\pgfpathlineto{\pgfqpoint{4.576627in}{2.093295in}}%
\pgfpathlineto{\pgfqpoint{4.607494in}{2.143403in}}%
\pgfpathlineto{\pgfqpoint{4.622928in}{2.165296in}}%
\pgfpathlineto{\pgfqpoint{4.638362in}{2.184735in}}%
\pgfpathlineto{\pgfqpoint{4.653795in}{2.201503in}}%
\pgfpathlineto{\pgfqpoint{4.669229in}{2.215423in}}%
\pgfpathlineto{\pgfqpoint{4.669229in}{2.215423in}}%
\pgfusepath{stroke}%
\end{pgfscope}%
\begin{pgfscope}%
\pgfpathrectangle{\pgfqpoint{0.634105in}{0.521603in}}{\pgfqpoint{4.227273in}{2.800000in}} %
\pgfusepath{clip}%
\pgfsetrectcap%
\pgfsetroundjoin%
\pgfsetlinewidth{0.501875pt}%
\definecolor{currentstroke}{rgb}{0.378431,0.981823,0.771298}%
\pgfsetstrokecolor{currentstroke}%
\pgfsetdash{}{0pt}%
\pgfpathmoveto{\pgfqpoint{0.826254in}{2.288891in}}%
\pgfpathlineto{\pgfqpoint{0.841687in}{2.287801in}}%
\pgfpathlineto{\pgfqpoint{0.857121in}{2.282969in}}%
\pgfpathlineto{\pgfqpoint{0.872555in}{2.274223in}}%
\pgfpathlineto{\pgfqpoint{0.887988in}{2.261448in}}%
\pgfpathlineto{\pgfqpoint{0.903422in}{2.244598in}}%
\pgfpathlineto{\pgfqpoint{0.918855in}{2.223694in}}%
\pgfpathlineto{\pgfqpoint{0.934289in}{2.198829in}}%
\pgfpathlineto{\pgfqpoint{0.949723in}{2.170168in}}%
\pgfpathlineto{\pgfqpoint{0.965156in}{2.137951in}}%
\pgfpathlineto{\pgfqpoint{0.980590in}{2.102488in}}%
\pgfpathlineto{\pgfqpoint{1.011457in}{2.023397in}}%
\pgfpathlineto{\pgfqpoint{1.042325in}{1.936645in}}%
\pgfpathlineto{\pgfqpoint{1.104059in}{1.758825in}}%
\pgfpathlineto{\pgfqpoint{1.134926in}{1.678015in}}%
\pgfpathlineto{\pgfqpoint{1.150360in}{1.641870in}}%
\pgfpathlineto{\pgfqpoint{1.165794in}{1.609351in}}%
\pgfpathlineto{\pgfqpoint{1.181227in}{1.580994in}}%
\pgfpathlineto{\pgfqpoint{1.196661in}{1.557276in}}%
\pgfpathlineto{\pgfqpoint{1.212095in}{1.538608in}}%
\pgfpathlineto{\pgfqpoint{1.227528in}{1.525327in}}%
\pgfpathlineto{\pgfqpoint{1.242962in}{1.517690in}}%
\pgfpathlineto{\pgfqpoint{1.258395in}{1.515866in}}%
\pgfpathlineto{\pgfqpoint{1.273829in}{1.519933in}}%
\pgfpathlineto{\pgfqpoint{1.289263in}{1.529879in}}%
\pgfpathlineto{\pgfqpoint{1.304696in}{1.545597in}}%
\pgfpathlineto{\pgfqpoint{1.320130in}{1.566888in}}%
\pgfpathlineto{\pgfqpoint{1.335564in}{1.593465in}}%
\pgfpathlineto{\pgfqpoint{1.350997in}{1.624958in}}%
\pgfpathlineto{\pgfqpoint{1.366431in}{1.660916in}}%
\pgfpathlineto{\pgfqpoint{1.381864in}{1.700821in}}%
\pgfpathlineto{\pgfqpoint{1.412732in}{1.790099in}}%
\pgfpathlineto{\pgfqpoint{1.459033in}{1.937678in}}%
\pgfpathlineto{\pgfqpoint{1.505334in}{2.084661in}}%
\pgfpathlineto{\pgfqpoint{1.536201in}{2.173174in}}%
\pgfpathlineto{\pgfqpoint{1.551634in}{2.212764in}}%
\pgfpathlineto{\pgfqpoint{1.567068in}{2.248563in}}%
\pgfpathlineto{\pgfqpoint{1.582502in}{2.280147in}}%
\pgfpathlineto{\pgfqpoint{1.597935in}{2.307171in}}%
\pgfpathlineto{\pgfqpoint{1.613369in}{2.329369in}}%
\pgfpathlineto{\pgfqpoint{1.628803in}{2.346561in}}%
\pgfpathlineto{\pgfqpoint{1.644236in}{2.358655in}}%
\pgfpathlineto{\pgfqpoint{1.659670in}{2.365641in}}%
\pgfpathlineto{\pgfqpoint{1.675104in}{2.367596in}}%
\pgfpathlineto{\pgfqpoint{1.690537in}{2.364677in}}%
\pgfpathlineto{\pgfqpoint{1.705971in}{2.357116in}}%
\pgfpathlineto{\pgfqpoint{1.721404in}{2.345213in}}%
\pgfpathlineto{\pgfqpoint{1.736838in}{2.329330in}}%
\pgfpathlineto{\pgfqpoint{1.752272in}{2.309882in}}%
\pgfpathlineto{\pgfqpoint{1.767705in}{2.287326in}}%
\pgfpathlineto{\pgfqpoint{1.798573in}{2.234876in}}%
\pgfpathlineto{\pgfqpoint{1.829440in}{2.176103in}}%
\pgfpathlineto{\pgfqpoint{1.891174in}{2.055825in}}%
\pgfpathlineto{\pgfqpoint{1.922042in}{2.001379in}}%
\pgfpathlineto{\pgfqpoint{1.952909in}{1.954230in}}%
\pgfpathlineto{\pgfqpoint{1.968343in}{1.933894in}}%
\pgfpathlineto{\pgfqpoint{1.983776in}{1.915862in}}%
\pgfpathlineto{\pgfqpoint{1.999210in}{1.900163in}}%
\pgfpathlineto{\pgfqpoint{2.014644in}{1.886770in}}%
\pgfpathlineto{\pgfqpoint{2.030077in}{1.875598in}}%
\pgfpathlineto{\pgfqpoint{2.045511in}{1.866511in}}%
\pgfpathlineto{\pgfqpoint{2.060944in}{1.859327in}}%
\pgfpathlineto{\pgfqpoint{2.076378in}{1.853826in}}%
\pgfpathlineto{\pgfqpoint{2.091812in}{1.849755in}}%
\pgfpathlineto{\pgfqpoint{2.122679in}{1.844782in}}%
\pgfpathlineto{\pgfqpoint{2.199847in}{1.837187in}}%
\pgfpathlineto{\pgfqpoint{2.230714in}{1.830567in}}%
\pgfpathlineto{\pgfqpoint{2.246148in}{1.825711in}}%
\pgfpathlineto{\pgfqpoint{2.261582in}{1.819680in}}%
\pgfpathlineto{\pgfqpoint{2.277015in}{1.812425in}}%
\pgfpathlineto{\pgfqpoint{2.307883in}{1.794272in}}%
\pgfpathlineto{\pgfqpoint{2.338750in}{1.771764in}}%
\pgfpathlineto{\pgfqpoint{2.369617in}{1.746077in}}%
\pgfpathlineto{\pgfqpoint{2.446785in}{1.679960in}}%
\pgfpathlineto{\pgfqpoint{2.477653in}{1.658312in}}%
\pgfpathlineto{\pgfqpoint{2.493086in}{1.649538in}}%
\pgfpathlineto{\pgfqpoint{2.508520in}{1.642412in}}%
\pgfpathlineto{\pgfqpoint{2.523954in}{1.637107in}}%
\pgfpathlineto{\pgfqpoint{2.539387in}{1.633758in}}%
\pgfpathlineto{\pgfqpoint{2.554821in}{1.632459in}}%
\pgfpathlineto{\pgfqpoint{2.570254in}{1.633257in}}%
\pgfpathlineto{\pgfqpoint{2.585688in}{1.636157in}}%
\pgfpathlineto{\pgfqpoint{2.601122in}{1.641113in}}%
\pgfpathlineto{\pgfqpoint{2.616555in}{1.648036in}}%
\pgfpathlineto{\pgfqpoint{2.631989in}{1.656789in}}%
\pgfpathlineto{\pgfqpoint{2.647423in}{1.667195in}}%
\pgfpathlineto{\pgfqpoint{2.678290in}{1.692065in}}%
\pgfpathlineto{\pgfqpoint{2.724591in}{1.735365in}}%
\pgfpathlineto{\pgfqpoint{2.770892in}{1.778294in}}%
\pgfpathlineto{\pgfqpoint{2.801759in}{1.802446in}}%
\pgfpathlineto{\pgfqpoint{2.817193in}{1.812315in}}%
\pgfpathlineto{\pgfqpoint{2.832626in}{1.820391in}}%
\pgfpathlineto{\pgfqpoint{2.848060in}{1.826476in}}%
\pgfpathlineto{\pgfqpoint{2.863493in}{1.830415in}}%
\pgfpathlineto{\pgfqpoint{2.878927in}{1.832095in}}%
\pgfpathlineto{\pgfqpoint{2.894361in}{1.831449in}}%
\pgfpathlineto{\pgfqpoint{2.909794in}{1.828461in}}%
\pgfpathlineto{\pgfqpoint{2.925228in}{1.823159in}}%
\pgfpathlineto{\pgfqpoint{2.940662in}{1.815620in}}%
\pgfpathlineto{\pgfqpoint{2.956095in}{1.805965in}}%
\pgfpathlineto{\pgfqpoint{2.971529in}{1.794357in}}%
\pgfpathlineto{\pgfqpoint{2.986963in}{1.780995in}}%
\pgfpathlineto{\pgfqpoint{3.017830in}{1.749977in}}%
\pgfpathlineto{\pgfqpoint{3.048697in}{1.715074in}}%
\pgfpathlineto{\pgfqpoint{3.110432in}{1.643255in}}%
\pgfpathlineto{\pgfqpoint{3.141299in}{1.611014in}}%
\pgfpathlineto{\pgfqpoint{3.172166in}{1.583836in}}%
\pgfpathlineto{\pgfqpoint{3.187600in}{1.572590in}}%
\pgfpathlineto{\pgfqpoint{3.203033in}{1.563069in}}%
\pgfpathlineto{\pgfqpoint{3.218467in}{1.555344in}}%
\pgfpathlineto{\pgfqpoint{3.233901in}{1.549445in}}%
\pgfpathlineto{\pgfqpoint{3.249334in}{1.545362in}}%
\pgfpathlineto{\pgfqpoint{3.264768in}{1.543045in}}%
\pgfpathlineto{\pgfqpoint{3.280202in}{1.542408in}}%
\pgfpathlineto{\pgfqpoint{3.295635in}{1.543330in}}%
\pgfpathlineto{\pgfqpoint{3.311069in}{1.545663in}}%
\pgfpathlineto{\pgfqpoint{3.341936in}{1.553841in}}%
\pgfpathlineto{\pgfqpoint{3.372803in}{1.565350in}}%
\pgfpathlineto{\pgfqpoint{3.465405in}{1.603206in}}%
\pgfpathlineto{\pgfqpoint{3.496273in}{1.612420in}}%
\pgfpathlineto{\pgfqpoint{3.527140in}{1.618705in}}%
\pgfpathlineto{\pgfqpoint{3.558007in}{1.622123in}}%
\pgfpathlineto{\pgfqpoint{3.588874in}{1.623282in}}%
\pgfpathlineto{\pgfqpoint{3.666043in}{1.624253in}}%
\pgfpathlineto{\pgfqpoint{3.696910in}{1.627967in}}%
\pgfpathlineto{\pgfqpoint{3.712343in}{1.631341in}}%
\pgfpathlineto{\pgfqpoint{3.727777in}{1.635963in}}%
\pgfpathlineto{\pgfqpoint{3.743211in}{1.641982in}}%
\pgfpathlineto{\pgfqpoint{3.758644in}{1.649519in}}%
\pgfpathlineto{\pgfqpoint{3.774078in}{1.658661in}}%
\pgfpathlineto{\pgfqpoint{3.789512in}{1.669461in}}%
\pgfpathlineto{\pgfqpoint{3.804945in}{1.681931in}}%
\pgfpathlineto{\pgfqpoint{3.820379in}{1.696042in}}%
\pgfpathlineto{\pgfqpoint{3.851246in}{1.728867in}}%
\pgfpathlineto{\pgfqpoint{3.882113in}{1.766887in}}%
\pgfpathlineto{\pgfqpoint{3.928414in}{1.829920in}}%
\pgfpathlineto{\pgfqpoint{3.974715in}{1.893412in}}%
\pgfpathlineto{\pgfqpoint{4.005583in}{1.931813in}}%
\pgfpathlineto{\pgfqpoint{4.021016in}{1.948907in}}%
\pgfpathlineto{\pgfqpoint{4.036450in}{1.964222in}}%
\pgfpathlineto{\pgfqpoint{4.051883in}{1.977509in}}%
\pgfpathlineto{\pgfqpoint{4.067317in}{1.988556in}}%
\pgfpathlineto{\pgfqpoint{4.082751in}{1.997188in}}%
\pgfpathlineto{\pgfqpoint{4.098184in}{2.003276in}}%
\pgfpathlineto{\pgfqpoint{4.113618in}{2.006738in}}%
\pgfpathlineto{\pgfqpoint{4.129052in}{2.007545in}}%
\pgfpathlineto{\pgfqpoint{4.144485in}{2.005717in}}%
\pgfpathlineto{\pgfqpoint{4.159919in}{2.001329in}}%
\pgfpathlineto{\pgfqpoint{4.175352in}{1.994508in}}%
\pgfpathlineto{\pgfqpoint{4.190786in}{1.985431in}}%
\pgfpathlineto{\pgfqpoint{4.206220in}{1.974323in}}%
\pgfpathlineto{\pgfqpoint{4.221653in}{1.961451in}}%
\pgfpathlineto{\pgfqpoint{4.252521in}{1.931675in}}%
\pgfpathlineto{\pgfqpoint{4.329689in}{1.850904in}}%
\pgfpathlineto{\pgfqpoint{4.345122in}{1.836772in}}%
\pgfpathlineto{\pgfqpoint{4.360556in}{1.824183in}}%
\pgfpathlineto{\pgfqpoint{4.375990in}{1.813462in}}%
\pgfpathlineto{\pgfqpoint{4.391423in}{1.804898in}}%
\pgfpathlineto{\pgfqpoint{4.406857in}{1.798739in}}%
\pgfpathlineto{\pgfqpoint{4.422291in}{1.795182in}}%
\pgfpathlineto{\pgfqpoint{4.437724in}{1.794372in}}%
\pgfpathlineto{\pgfqpoint{4.453158in}{1.796401in}}%
\pgfpathlineto{\pgfqpoint{4.468592in}{1.801298in}}%
\pgfpathlineto{\pgfqpoint{4.484025in}{1.809035in}}%
\pgfpathlineto{\pgfqpoint{4.499459in}{1.819524in}}%
\pgfpathlineto{\pgfqpoint{4.514892in}{1.832621in}}%
\pgfpathlineto{\pgfqpoint{4.530326in}{1.848124in}}%
\pgfpathlineto{\pgfqpoint{4.545760in}{1.865782in}}%
\pgfpathlineto{\pgfqpoint{4.576627in}{1.906323in}}%
\pgfpathlineto{\pgfqpoint{4.622928in}{1.974639in}}%
\pgfpathlineto{\pgfqpoint{4.653795in}{2.020355in}}%
\pgfpathlineto{\pgfqpoint{4.669229in}{2.041980in}}%
\pgfpathlineto{\pgfqpoint{4.669229in}{2.041980in}}%
\pgfusepath{stroke}%
\end{pgfscope}%
\begin{pgfscope}%
\pgfpathrectangle{\pgfqpoint{0.634105in}{0.521603in}}{\pgfqpoint{4.227273in}{2.800000in}} %
\pgfusepath{clip}%
\pgfsetrectcap%
\pgfsetroundjoin%
\pgfsetlinewidth{0.501875pt}%
\definecolor{currentstroke}{rgb}{0.456863,0.997705,0.730653}%
\pgfsetstrokecolor{currentstroke}%
\pgfsetdash{}{0pt}%
\pgfpathmoveto{\pgfqpoint{0.826254in}{2.296489in}}%
\pgfpathlineto{\pgfqpoint{0.841687in}{2.293988in}}%
\pgfpathlineto{\pgfqpoint{0.857121in}{2.287130in}}%
\pgfpathlineto{\pgfqpoint{0.872555in}{2.275734in}}%
\pgfpathlineto{\pgfqpoint{0.887988in}{2.259699in}}%
\pgfpathlineto{\pgfqpoint{0.903422in}{2.239006in}}%
\pgfpathlineto{\pgfqpoint{0.918855in}{2.213724in}}%
\pgfpathlineto{\pgfqpoint{0.934289in}{2.184009in}}%
\pgfpathlineto{\pgfqpoint{0.949723in}{2.150111in}}%
\pgfpathlineto{\pgfqpoint{0.965156in}{2.112366in}}%
\pgfpathlineto{\pgfqpoint{0.996024in}{2.027097in}}%
\pgfpathlineto{\pgfqpoint{1.026891in}{1.932469in}}%
\pgfpathlineto{\pgfqpoint{1.088625in}{1.736662in}}%
\pgfpathlineto{\pgfqpoint{1.119493in}{1.647279in}}%
\pgfpathlineto{\pgfqpoint{1.134926in}{1.607255in}}%
\pgfpathlineto{\pgfqpoint{1.150360in}{1.571217in}}%
\pgfpathlineto{\pgfqpoint{1.165794in}{1.539747in}}%
\pgfpathlineto{\pgfqpoint{1.181227in}{1.513353in}}%
\pgfpathlineto{\pgfqpoint{1.196661in}{1.492460in}}%
\pgfpathlineto{\pgfqpoint{1.212095in}{1.477403in}}%
\pgfpathlineto{\pgfqpoint{1.227528in}{1.468415in}}%
\pgfpathlineto{\pgfqpoint{1.242962in}{1.465630in}}%
\pgfpathlineto{\pgfqpoint{1.258395in}{1.469078in}}%
\pgfpathlineto{\pgfqpoint{1.273829in}{1.478683in}}%
\pgfpathlineto{\pgfqpoint{1.289263in}{1.494269in}}%
\pgfpathlineto{\pgfqpoint{1.304696in}{1.515564in}}%
\pgfpathlineto{\pgfqpoint{1.320130in}{1.542203in}}%
\pgfpathlineto{\pgfqpoint{1.335564in}{1.573744in}}%
\pgfpathlineto{\pgfqpoint{1.350997in}{1.609667in}}%
\pgfpathlineto{\pgfqpoint{1.366431in}{1.649395in}}%
\pgfpathlineto{\pgfqpoint{1.397298in}{1.737726in}}%
\pgfpathlineto{\pgfqpoint{1.443599in}{1.882213in}}%
\pgfpathlineto{\pgfqpoint{1.489900in}{2.024814in}}%
\pgfpathlineto{\pgfqpoint{1.520767in}{2.110681in}}%
\pgfpathlineto{\pgfqpoint{1.536201in}{2.149344in}}%
\pgfpathlineto{\pgfqpoint{1.551634in}{2.184635in}}%
\pgfpathlineto{\pgfqpoint{1.567068in}{2.216243in}}%
\pgfpathlineto{\pgfqpoint{1.582502in}{2.243930in}}%
\pgfpathlineto{\pgfqpoint{1.597935in}{2.267529in}}%
\pgfpathlineto{\pgfqpoint{1.613369in}{2.286940in}}%
\pgfpathlineto{\pgfqpoint{1.628803in}{2.302129in}}%
\pgfpathlineto{\pgfqpoint{1.644236in}{2.313126in}}%
\pgfpathlineto{\pgfqpoint{1.659670in}{2.320015in}}%
\pgfpathlineto{\pgfqpoint{1.675104in}{2.322933in}}%
\pgfpathlineto{\pgfqpoint{1.690537in}{2.322061in}}%
\pgfpathlineto{\pgfqpoint{1.705971in}{2.317616in}}%
\pgfpathlineto{\pgfqpoint{1.721404in}{2.309849in}}%
\pgfpathlineto{\pgfqpoint{1.736838in}{2.299035in}}%
\pgfpathlineto{\pgfqpoint{1.752272in}{2.285464in}}%
\pgfpathlineto{\pgfqpoint{1.767705in}{2.269441in}}%
\pgfpathlineto{\pgfqpoint{1.783139in}{2.251275in}}%
\pgfpathlineto{\pgfqpoint{1.814006in}{2.209749in}}%
\pgfpathlineto{\pgfqpoint{1.844874in}{2.163301in}}%
\pgfpathlineto{\pgfqpoint{1.968343in}{1.969667in}}%
\pgfpathlineto{\pgfqpoint{1.999210in}{1.927407in}}%
\pgfpathlineto{\pgfqpoint{2.030077in}{1.889928in}}%
\pgfpathlineto{\pgfqpoint{2.060944in}{1.857942in}}%
\pgfpathlineto{\pgfqpoint{2.076378in}{1.844158in}}%
\pgfpathlineto{\pgfqpoint{2.091812in}{1.831895in}}%
\pgfpathlineto{\pgfqpoint{2.107245in}{1.821162in}}%
\pgfpathlineto{\pgfqpoint{2.122679in}{1.811953in}}%
\pgfpathlineto{\pgfqpoint{2.138113in}{1.804242in}}%
\pgfpathlineto{\pgfqpoint{2.153546in}{1.797983in}}%
\pgfpathlineto{\pgfqpoint{2.168980in}{1.793110in}}%
\pgfpathlineto{\pgfqpoint{2.184414in}{1.789536in}}%
\pgfpathlineto{\pgfqpoint{2.215281in}{1.785850in}}%
\pgfpathlineto{\pgfqpoint{2.246148in}{1.785878in}}%
\pgfpathlineto{\pgfqpoint{2.277015in}{1.788350in}}%
\pgfpathlineto{\pgfqpoint{2.354184in}{1.795993in}}%
\pgfpathlineto{\pgfqpoint{2.385051in}{1.796189in}}%
\pgfpathlineto{\pgfqpoint{2.415918in}{1.793288in}}%
\pgfpathlineto{\pgfqpoint{2.446785in}{1.786873in}}%
\pgfpathlineto{\pgfqpoint{2.477653in}{1.777028in}}%
\pgfpathlineto{\pgfqpoint{2.508520in}{1.764364in}}%
\pgfpathlineto{\pgfqpoint{2.601122in}{1.721903in}}%
\pgfpathlineto{\pgfqpoint{2.631989in}{1.711544in}}%
\pgfpathlineto{\pgfqpoint{2.647423in}{1.707951in}}%
\pgfpathlineto{\pgfqpoint{2.662856in}{1.705619in}}%
\pgfpathlineto{\pgfqpoint{2.678290in}{1.704662in}}%
\pgfpathlineto{\pgfqpoint{2.693724in}{1.705160in}}%
\pgfpathlineto{\pgfqpoint{2.709157in}{1.707152in}}%
\pgfpathlineto{\pgfqpoint{2.724591in}{1.710634in}}%
\pgfpathlineto{\pgfqpoint{2.740024in}{1.715559in}}%
\pgfpathlineto{\pgfqpoint{2.755458in}{1.721837in}}%
\pgfpathlineto{\pgfqpoint{2.786325in}{1.737878in}}%
\pgfpathlineto{\pgfqpoint{2.817193in}{1.757235in}}%
\pgfpathlineto{\pgfqpoint{2.878927in}{1.797561in}}%
\pgfpathlineto{\pgfqpoint{2.909794in}{1.813899in}}%
\pgfpathlineto{\pgfqpoint{2.925228in}{1.820144in}}%
\pgfpathlineto{\pgfqpoint{2.940662in}{1.824796in}}%
\pgfpathlineto{\pgfqpoint{2.956095in}{1.827666in}}%
\pgfpathlineto{\pgfqpoint{2.971529in}{1.828608in}}%
\pgfpathlineto{\pgfqpoint{2.986963in}{1.827521in}}%
\pgfpathlineto{\pgfqpoint{3.002396in}{1.824354in}}%
\pgfpathlineto{\pgfqpoint{3.017830in}{1.819108in}}%
\pgfpathlineto{\pgfqpoint{3.033263in}{1.811836in}}%
\pgfpathlineto{\pgfqpoint{3.048697in}{1.802641in}}%
\pgfpathlineto{\pgfqpoint{3.064131in}{1.791675in}}%
\pgfpathlineto{\pgfqpoint{3.094998in}{1.765245in}}%
\pgfpathlineto{\pgfqpoint{3.125865in}{1.734526in}}%
\pgfpathlineto{\pgfqpoint{3.203033in}{1.654716in}}%
\pgfpathlineto{\pgfqpoint{3.233901in}{1.627736in}}%
\pgfpathlineto{\pgfqpoint{3.249334in}{1.616257in}}%
\pgfpathlineto{\pgfqpoint{3.264768in}{1.606328in}}%
\pgfpathlineto{\pgfqpoint{3.280202in}{1.598047in}}%
\pgfpathlineto{\pgfqpoint{3.295635in}{1.591466in}}%
\pgfpathlineto{\pgfqpoint{3.311069in}{1.586592in}}%
\pgfpathlineto{\pgfqpoint{3.326503in}{1.583382in}}%
\pgfpathlineto{\pgfqpoint{3.341936in}{1.581753in}}%
\pgfpathlineto{\pgfqpoint{3.357370in}{1.581582in}}%
\pgfpathlineto{\pgfqpoint{3.372803in}{1.582714in}}%
\pgfpathlineto{\pgfqpoint{3.403671in}{1.588140in}}%
\pgfpathlineto{\pgfqpoint{3.434538in}{1.596393in}}%
\pgfpathlineto{\pgfqpoint{3.511706in}{1.618914in}}%
\pgfpathlineto{\pgfqpoint{3.542573in}{1.625694in}}%
\pgfpathlineto{\pgfqpoint{3.573441in}{1.630645in}}%
\pgfpathlineto{\pgfqpoint{3.666043in}{1.642466in}}%
\pgfpathlineto{\pgfqpoint{3.696910in}{1.650265in}}%
\pgfpathlineto{\pgfqpoint{3.712343in}{1.655835in}}%
\pgfpathlineto{\pgfqpoint{3.727777in}{1.662770in}}%
\pgfpathlineto{\pgfqpoint{3.743211in}{1.671221in}}%
\pgfpathlineto{\pgfqpoint{3.758644in}{1.681306in}}%
\pgfpathlineto{\pgfqpoint{3.774078in}{1.693098in}}%
\pgfpathlineto{\pgfqpoint{3.789512in}{1.706626in}}%
\pgfpathlineto{\pgfqpoint{3.804945in}{1.721864in}}%
\pgfpathlineto{\pgfqpoint{3.835813in}{1.757120in}}%
\pgfpathlineto{\pgfqpoint{3.866680in}{1.797633in}}%
\pgfpathlineto{\pgfqpoint{3.974715in}{1.947476in}}%
\pgfpathlineto{\pgfqpoint{3.990149in}{1.965171in}}%
\pgfpathlineto{\pgfqpoint{4.005583in}{1.980888in}}%
\pgfpathlineto{\pgfqpoint{4.021016in}{1.994340in}}%
\pgfpathlineto{\pgfqpoint{4.036450in}{2.005283in}}%
\pgfpathlineto{\pgfqpoint{4.051883in}{2.013528in}}%
\pgfpathlineto{\pgfqpoint{4.067317in}{2.018941in}}%
\pgfpathlineto{\pgfqpoint{4.082751in}{2.021453in}}%
\pgfpathlineto{\pgfqpoint{4.098184in}{2.021058in}}%
\pgfpathlineto{\pgfqpoint{4.113618in}{2.017816in}}%
\pgfpathlineto{\pgfqpoint{4.129052in}{2.011852in}}%
\pgfpathlineto{\pgfqpoint{4.144485in}{2.003356in}}%
\pgfpathlineto{\pgfqpoint{4.159919in}{1.992572in}}%
\pgfpathlineto{\pgfqpoint{4.175352in}{1.979801in}}%
\pgfpathlineto{\pgfqpoint{4.206220in}{1.949710in}}%
\pgfpathlineto{\pgfqpoint{4.283388in}{1.867160in}}%
\pgfpathlineto{\pgfqpoint{4.298822in}{1.852784in}}%
\pgfpathlineto{\pgfqpoint{4.314255in}{1.840028in}}%
\pgfpathlineto{\pgfqpoint{4.329689in}{1.829214in}}%
\pgfpathlineto{\pgfqpoint{4.345122in}{1.820615in}}%
\pgfpathlineto{\pgfqpoint{4.360556in}{1.814448in}}%
\pgfpathlineto{\pgfqpoint{4.375990in}{1.810873in}}%
\pgfpathlineto{\pgfqpoint{4.391423in}{1.809987in}}%
\pgfpathlineto{\pgfqpoint{4.406857in}{1.811822in}}%
\pgfpathlineto{\pgfqpoint{4.422291in}{1.816349in}}%
\pgfpathlineto{\pgfqpoint{4.437724in}{1.823475in}}%
\pgfpathlineto{\pgfqpoint{4.453158in}{1.833049in}}%
\pgfpathlineto{\pgfqpoint{4.468592in}{1.844865in}}%
\pgfpathlineto{\pgfqpoint{4.484025in}{1.858670in}}%
\pgfpathlineto{\pgfqpoint{4.514892in}{1.891027in}}%
\pgfpathlineto{\pgfqpoint{4.561193in}{1.946151in}}%
\pgfpathlineto{\pgfqpoint{4.592061in}{1.982902in}}%
\pgfpathlineto{\pgfqpoint{4.622928in}{2.016252in}}%
\pgfpathlineto{\pgfqpoint{4.638362in}{2.030846in}}%
\pgfpathlineto{\pgfqpoint{4.653795in}{2.043688in}}%
\pgfpathlineto{\pgfqpoint{4.669229in}{2.054556in}}%
\pgfpathlineto{\pgfqpoint{4.669229in}{2.054556in}}%
\pgfusepath{stroke}%
\end{pgfscope}%
\begin{pgfscope}%
\pgfpathrectangle{\pgfqpoint{0.634105in}{0.521603in}}{\pgfqpoint{4.227273in}{2.800000in}} %
\pgfusepath{clip}%
\pgfsetrectcap%
\pgfsetroundjoin%
\pgfsetlinewidth{0.501875pt}%
\definecolor{currentstroke}{rgb}{0.543137,0.997705,0.682749}%
\pgfsetstrokecolor{currentstroke}%
\pgfsetdash{}{0pt}%
\pgfpathmoveto{\pgfqpoint{0.826254in}{2.293806in}}%
\pgfpathlineto{\pgfqpoint{0.841687in}{2.290419in}}%
\pgfpathlineto{\pgfqpoint{0.857121in}{2.282814in}}%
\pgfpathlineto{\pgfqpoint{0.872555in}{2.270934in}}%
\pgfpathlineto{\pgfqpoint{0.887988in}{2.254792in}}%
\pgfpathlineto{\pgfqpoint{0.903422in}{2.234474in}}%
\pgfpathlineto{\pgfqpoint{0.918855in}{2.210137in}}%
\pgfpathlineto{\pgfqpoint{0.934289in}{2.182010in}}%
\pgfpathlineto{\pgfqpoint{0.949723in}{2.150395in}}%
\pgfpathlineto{\pgfqpoint{0.980590in}{2.078229in}}%
\pgfpathlineto{\pgfqpoint{1.011457in}{1.997277in}}%
\pgfpathlineto{\pgfqpoint{1.088625in}{1.786366in}}%
\pgfpathlineto{\pgfqpoint{1.119493in}{1.711477in}}%
\pgfpathlineto{\pgfqpoint{1.134926in}{1.678414in}}%
\pgfpathlineto{\pgfqpoint{1.150360in}{1.648950in}}%
\pgfpathlineto{\pgfqpoint{1.165794in}{1.623529in}}%
\pgfpathlineto{\pgfqpoint{1.181227in}{1.602528in}}%
\pgfpathlineto{\pgfqpoint{1.196661in}{1.586253in}}%
\pgfpathlineto{\pgfqpoint{1.212095in}{1.574934in}}%
\pgfpathlineto{\pgfqpoint{1.227528in}{1.568722in}}%
\pgfpathlineto{\pgfqpoint{1.242962in}{1.567685in}}%
\pgfpathlineto{\pgfqpoint{1.258395in}{1.571811in}}%
\pgfpathlineto{\pgfqpoint{1.273829in}{1.581005in}}%
\pgfpathlineto{\pgfqpoint{1.289263in}{1.595095in}}%
\pgfpathlineto{\pgfqpoint{1.304696in}{1.613835in}}%
\pgfpathlineto{\pgfqpoint{1.320130in}{1.636913in}}%
\pgfpathlineto{\pgfqpoint{1.335564in}{1.663954in}}%
\pgfpathlineto{\pgfqpoint{1.350997in}{1.694534in}}%
\pgfpathlineto{\pgfqpoint{1.381864in}{1.764395in}}%
\pgfpathlineto{\pgfqpoint{1.412732in}{1.842381in}}%
\pgfpathlineto{\pgfqpoint{1.489900in}{2.044544in}}%
\pgfpathlineto{\pgfqpoint{1.520767in}{2.117715in}}%
\pgfpathlineto{\pgfqpoint{1.536201in}{2.150943in}}%
\pgfpathlineto{\pgfqpoint{1.551634in}{2.181486in}}%
\pgfpathlineto{\pgfqpoint{1.567068in}{2.209071in}}%
\pgfpathlineto{\pgfqpoint{1.582502in}{2.233472in}}%
\pgfpathlineto{\pgfqpoint{1.597935in}{2.254512in}}%
\pgfpathlineto{\pgfqpoint{1.613369in}{2.272063in}}%
\pgfpathlineto{\pgfqpoint{1.628803in}{2.286043in}}%
\pgfpathlineto{\pgfqpoint{1.644236in}{2.296413in}}%
\pgfpathlineto{\pgfqpoint{1.659670in}{2.303179in}}%
\pgfpathlineto{\pgfqpoint{1.675104in}{2.306382in}}%
\pgfpathlineto{\pgfqpoint{1.690537in}{2.306104in}}%
\pgfpathlineto{\pgfqpoint{1.705971in}{2.302455in}}%
\pgfpathlineto{\pgfqpoint{1.721404in}{2.295579in}}%
\pgfpathlineto{\pgfqpoint{1.736838in}{2.285645in}}%
\pgfpathlineto{\pgfqpoint{1.752272in}{2.272844in}}%
\pgfpathlineto{\pgfqpoint{1.767705in}{2.257392in}}%
\pgfpathlineto{\pgfqpoint{1.783139in}{2.239520in}}%
\pgfpathlineto{\pgfqpoint{1.798573in}{2.219478in}}%
\pgfpathlineto{\pgfqpoint{1.829440in}{2.173939in}}%
\pgfpathlineto{\pgfqpoint{1.860307in}{2.122996in}}%
\pgfpathlineto{\pgfqpoint{1.968343in}{1.936002in}}%
\pgfpathlineto{\pgfqpoint{1.999210in}{1.889451in}}%
\pgfpathlineto{\pgfqpoint{2.030077in}{1.849547in}}%
\pgfpathlineto{\pgfqpoint{2.045511in}{1.832540in}}%
\pgfpathlineto{\pgfqpoint{2.060944in}{1.817675in}}%
\pgfpathlineto{\pgfqpoint{2.076378in}{1.805047in}}%
\pgfpathlineto{\pgfqpoint{2.091812in}{1.794711in}}%
\pgfpathlineto{\pgfqpoint{2.107245in}{1.786685in}}%
\pgfpathlineto{\pgfqpoint{2.122679in}{1.780946in}}%
\pgfpathlineto{\pgfqpoint{2.138113in}{1.777433in}}%
\pgfpathlineto{\pgfqpoint{2.153546in}{1.776043in}}%
\pgfpathlineto{\pgfqpoint{2.168980in}{1.776639in}}%
\pgfpathlineto{\pgfqpoint{2.184414in}{1.779043in}}%
\pgfpathlineto{\pgfqpoint{2.199847in}{1.783049in}}%
\pgfpathlineto{\pgfqpoint{2.230714in}{1.794891in}}%
\pgfpathlineto{\pgfqpoint{2.261582in}{1.810009in}}%
\pgfpathlineto{\pgfqpoint{2.307883in}{1.833633in}}%
\pgfpathlineto{\pgfqpoint{2.338750in}{1.846598in}}%
\pgfpathlineto{\pgfqpoint{2.354184in}{1.851454in}}%
\pgfpathlineto{\pgfqpoint{2.369617in}{1.854931in}}%
\pgfpathlineto{\pgfqpoint{2.385051in}{1.856851in}}%
\pgfpathlineto{\pgfqpoint{2.400484in}{1.857074in}}%
\pgfpathlineto{\pgfqpoint{2.415918in}{1.855501in}}%
\pgfpathlineto{\pgfqpoint{2.431352in}{1.852076in}}%
\pgfpathlineto{\pgfqpoint{2.446785in}{1.846785in}}%
\pgfpathlineto{\pgfqpoint{2.462219in}{1.839663in}}%
\pgfpathlineto{\pgfqpoint{2.477653in}{1.830783in}}%
\pgfpathlineto{\pgfqpoint{2.493086in}{1.820263in}}%
\pgfpathlineto{\pgfqpoint{2.523954in}{1.794954in}}%
\pgfpathlineto{\pgfqpoint{2.554821in}{1.765357in}}%
\pgfpathlineto{\pgfqpoint{2.647423in}{1.671745in}}%
\pgfpathlineto{\pgfqpoint{2.678290in}{1.646102in}}%
\pgfpathlineto{\pgfqpoint{2.693724in}{1.635368in}}%
\pgfpathlineto{\pgfqpoint{2.709157in}{1.626249in}}%
\pgfpathlineto{\pgfqpoint{2.724591in}{1.618868in}}%
\pgfpathlineto{\pgfqpoint{2.740024in}{1.613310in}}%
\pgfpathlineto{\pgfqpoint{2.755458in}{1.609621in}}%
\pgfpathlineto{\pgfqpoint{2.770892in}{1.607806in}}%
\pgfpathlineto{\pgfqpoint{2.786325in}{1.607830in}}%
\pgfpathlineto{\pgfqpoint{2.801759in}{1.609617in}}%
\pgfpathlineto{\pgfqpoint{2.817193in}{1.613056in}}%
\pgfpathlineto{\pgfqpoint{2.832626in}{1.617999in}}%
\pgfpathlineto{\pgfqpoint{2.863493in}{1.631657in}}%
\pgfpathlineto{\pgfqpoint{2.894361in}{1.648862in}}%
\pgfpathlineto{\pgfqpoint{2.956095in}{1.685744in}}%
\pgfpathlineto{\pgfqpoint{2.986963in}{1.701230in}}%
\pgfpathlineto{\pgfqpoint{3.002396in}{1.707393in}}%
\pgfpathlineto{\pgfqpoint{3.017830in}{1.712234in}}%
\pgfpathlineto{\pgfqpoint{3.033263in}{1.715585in}}%
\pgfpathlineto{\pgfqpoint{3.048697in}{1.717312in}}%
\pgfpathlineto{\pgfqpoint{3.064131in}{1.717313in}}%
\pgfpathlineto{\pgfqpoint{3.079564in}{1.715520in}}%
\pgfpathlineto{\pgfqpoint{3.094998in}{1.711906in}}%
\pgfpathlineto{\pgfqpoint{3.110432in}{1.706478in}}%
\pgfpathlineto{\pgfqpoint{3.125865in}{1.699278in}}%
\pgfpathlineto{\pgfqpoint{3.141299in}{1.690388in}}%
\pgfpathlineto{\pgfqpoint{3.156733in}{1.679918in}}%
\pgfpathlineto{\pgfqpoint{3.187600in}{1.654833in}}%
\pgfpathlineto{\pgfqpoint{3.218467in}{1.625455in}}%
\pgfpathlineto{\pgfqpoint{3.280202in}{1.560864in}}%
\pgfpathlineto{\pgfqpoint{3.311069in}{1.529387in}}%
\pgfpathlineto{\pgfqpoint{3.341936in}{1.500803in}}%
\pgfpathlineto{\pgfqpoint{3.372803in}{1.476587in}}%
\pgfpathlineto{\pgfqpoint{3.388237in}{1.466491in}}%
\pgfpathlineto{\pgfqpoint{3.403671in}{1.457891in}}%
\pgfpathlineto{\pgfqpoint{3.419104in}{1.450875in}}%
\pgfpathlineto{\pgfqpoint{3.434538in}{1.445511in}}%
\pgfpathlineto{\pgfqpoint{3.449972in}{1.441843in}}%
\pgfpathlineto{\pgfqpoint{3.465405in}{1.439899in}}%
\pgfpathlineto{\pgfqpoint{3.480839in}{1.439687in}}%
\pgfpathlineto{\pgfqpoint{3.496273in}{1.441203in}}%
\pgfpathlineto{\pgfqpoint{3.511706in}{1.444429in}}%
\pgfpathlineto{\pgfqpoint{3.527140in}{1.449338in}}%
\pgfpathlineto{\pgfqpoint{3.542573in}{1.455892in}}%
\pgfpathlineto{\pgfqpoint{3.558007in}{1.464051in}}%
\pgfpathlineto{\pgfqpoint{3.573441in}{1.473769in}}%
\pgfpathlineto{\pgfqpoint{3.588874in}{1.484998in}}%
\pgfpathlineto{\pgfqpoint{3.619742in}{1.511781in}}%
\pgfpathlineto{\pgfqpoint{3.650609in}{1.543976in}}%
\pgfpathlineto{\pgfqpoint{3.681476in}{1.581110in}}%
\pgfpathlineto{\pgfqpoint{3.712343in}{1.622625in}}%
\pgfpathlineto{\pgfqpoint{3.743211in}{1.667813in}}%
\pgfpathlineto{\pgfqpoint{3.789512in}{1.740387in}}%
\pgfpathlineto{\pgfqpoint{3.866680in}{1.862680in}}%
\pgfpathlineto{\pgfqpoint{3.897547in}{1.906911in}}%
\pgfpathlineto{\pgfqpoint{3.928414in}{1.945462in}}%
\pgfpathlineto{\pgfqpoint{3.943848in}{1.961991in}}%
\pgfpathlineto{\pgfqpoint{3.959282in}{1.976388in}}%
\pgfpathlineto{\pgfqpoint{3.974715in}{1.988455in}}%
\pgfpathlineto{\pgfqpoint{3.990149in}{1.998024in}}%
\pgfpathlineto{\pgfqpoint{4.005583in}{2.004964in}}%
\pgfpathlineto{\pgfqpoint{4.021016in}{2.009185in}}%
\pgfpathlineto{\pgfqpoint{4.036450in}{2.010641in}}%
\pgfpathlineto{\pgfqpoint{4.051883in}{2.009336in}}%
\pgfpathlineto{\pgfqpoint{4.067317in}{2.005327in}}%
\pgfpathlineto{\pgfqpoint{4.082751in}{1.998722in}}%
\pgfpathlineto{\pgfqpoint{4.098184in}{1.989683in}}%
\pgfpathlineto{\pgfqpoint{4.113618in}{1.978421in}}%
\pgfpathlineto{\pgfqpoint{4.129052in}{1.965198in}}%
\pgfpathlineto{\pgfqpoint{4.159919in}{1.934128in}}%
\pgfpathlineto{\pgfqpoint{4.252521in}{1.831633in}}%
\pgfpathlineto{\pgfqpoint{4.267954in}{1.817489in}}%
\pgfpathlineto{\pgfqpoint{4.283388in}{1.805204in}}%
\pgfpathlineto{\pgfqpoint{4.298822in}{1.795090in}}%
\pgfpathlineto{\pgfqpoint{4.314255in}{1.787412in}}%
\pgfpathlineto{\pgfqpoint{4.329689in}{1.782383in}}%
\pgfpathlineto{\pgfqpoint{4.345122in}{1.780156in}}%
\pgfpathlineto{\pgfqpoint{4.360556in}{1.780822in}}%
\pgfpathlineto{\pgfqpoint{4.375990in}{1.784408in}}%
\pgfpathlineto{\pgfqpoint{4.391423in}{1.790876in}}%
\pgfpathlineto{\pgfqpoint{4.406857in}{1.800124in}}%
\pgfpathlineto{\pgfqpoint{4.422291in}{1.811990in}}%
\pgfpathlineto{\pgfqpoint{4.437724in}{1.826254in}}%
\pgfpathlineto{\pgfqpoint{4.453158in}{1.842643in}}%
\pgfpathlineto{\pgfqpoint{4.484025in}{1.880493in}}%
\pgfpathlineto{\pgfqpoint{4.530326in}{1.944261in}}%
\pgfpathlineto{\pgfqpoint{4.561193in}{1.986682in}}%
\pgfpathlineto{\pgfqpoint{4.592061in}{2.025325in}}%
\pgfpathlineto{\pgfqpoint{4.607494in}{2.042350in}}%
\pgfpathlineto{\pgfqpoint{4.622928in}{2.057446in}}%
\pgfpathlineto{\pgfqpoint{4.638362in}{2.070372in}}%
\pgfpathlineto{\pgfqpoint{4.653795in}{2.080934in}}%
\pgfpathlineto{\pgfqpoint{4.669229in}{2.088995in}}%
\pgfpathlineto{\pgfqpoint{4.669229in}{2.088995in}}%
\pgfusepath{stroke}%
\end{pgfscope}%
\begin{pgfscope}%
\pgfpathrectangle{\pgfqpoint{0.634105in}{0.521603in}}{\pgfqpoint{4.227273in}{2.800000in}} %
\pgfusepath{clip}%
\pgfsetrectcap%
\pgfsetroundjoin%
\pgfsetlinewidth{0.501875pt}%
\definecolor{currentstroke}{rgb}{0.621569,0.981823,0.636474}%
\pgfsetstrokecolor{currentstroke}%
\pgfsetdash{}{0pt}%
\pgfpathmoveto{\pgfqpoint{0.826254in}{2.316596in}}%
\pgfpathlineto{\pgfqpoint{0.841687in}{2.320491in}}%
\pgfpathlineto{\pgfqpoint{0.857121in}{2.320276in}}%
\pgfpathlineto{\pgfqpoint{0.872555in}{2.315734in}}%
\pgfpathlineto{\pgfqpoint{0.887988in}{2.306727in}}%
\pgfpathlineto{\pgfqpoint{0.903422in}{2.293197in}}%
\pgfpathlineto{\pgfqpoint{0.918855in}{2.275172in}}%
\pgfpathlineto{\pgfqpoint{0.934289in}{2.252770in}}%
\pgfpathlineto{\pgfqpoint{0.949723in}{2.226199in}}%
\pgfpathlineto{\pgfqpoint{0.965156in}{2.195755in}}%
\pgfpathlineto{\pgfqpoint{0.980590in}{2.161818in}}%
\pgfpathlineto{\pgfqpoint{1.011457in}{2.085384in}}%
\pgfpathlineto{\pgfqpoint{1.057758in}{1.958153in}}%
\pgfpathlineto{\pgfqpoint{1.104059in}{1.832063in}}%
\pgfpathlineto{\pgfqpoint{1.134926in}{1.757931in}}%
\pgfpathlineto{\pgfqpoint{1.150360in}{1.725772in}}%
\pgfpathlineto{\pgfqpoint{1.165794in}{1.697620in}}%
\pgfpathlineto{\pgfqpoint{1.181227in}{1.673938in}}%
\pgfpathlineto{\pgfqpoint{1.196661in}{1.655105in}}%
\pgfpathlineto{\pgfqpoint{1.212095in}{1.641412in}}%
\pgfpathlineto{\pgfqpoint{1.227528in}{1.633054in}}%
\pgfpathlineto{\pgfqpoint{1.242962in}{1.630127in}}%
\pgfpathlineto{\pgfqpoint{1.258395in}{1.632631in}}%
\pgfpathlineto{\pgfqpoint{1.273829in}{1.640467in}}%
\pgfpathlineto{\pgfqpoint{1.289263in}{1.653445in}}%
\pgfpathlineto{\pgfqpoint{1.304696in}{1.671289in}}%
\pgfpathlineto{\pgfqpoint{1.320130in}{1.693642in}}%
\pgfpathlineto{\pgfqpoint{1.335564in}{1.720081in}}%
\pgfpathlineto{\pgfqpoint{1.350997in}{1.750121in}}%
\pgfpathlineto{\pgfqpoint{1.381864in}{1.818852in}}%
\pgfpathlineto{\pgfqpoint{1.412732in}{1.895248in}}%
\pgfpathlineto{\pgfqpoint{1.474466in}{2.052185in}}%
\pgfpathlineto{\pgfqpoint{1.505334in}{2.124076in}}%
\pgfpathlineto{\pgfqpoint{1.520767in}{2.156781in}}%
\pgfpathlineto{\pgfqpoint{1.536201in}{2.186846in}}%
\pgfpathlineto{\pgfqpoint{1.551634in}{2.213980in}}%
\pgfpathlineto{\pgfqpoint{1.567068in}{2.237946in}}%
\pgfpathlineto{\pgfqpoint{1.582502in}{2.258560in}}%
\pgfpathlineto{\pgfqpoint{1.597935in}{2.275692in}}%
\pgfpathlineto{\pgfqpoint{1.613369in}{2.289257in}}%
\pgfpathlineto{\pgfqpoint{1.628803in}{2.299221in}}%
\pgfpathlineto{\pgfqpoint{1.644236in}{2.305589in}}%
\pgfpathlineto{\pgfqpoint{1.659670in}{2.308405in}}%
\pgfpathlineto{\pgfqpoint{1.675104in}{2.307748in}}%
\pgfpathlineto{\pgfqpoint{1.690537in}{2.303729in}}%
\pgfpathlineto{\pgfqpoint{1.705971in}{2.296483in}}%
\pgfpathlineto{\pgfqpoint{1.721404in}{2.286171in}}%
\pgfpathlineto{\pgfqpoint{1.736838in}{2.272975in}}%
\pgfpathlineto{\pgfqpoint{1.752272in}{2.257095in}}%
\pgfpathlineto{\pgfqpoint{1.767705in}{2.238748in}}%
\pgfpathlineto{\pgfqpoint{1.783139in}{2.218167in}}%
\pgfpathlineto{\pgfqpoint{1.814006in}{2.171302in}}%
\pgfpathlineto{\pgfqpoint{1.844874in}{2.118621in}}%
\pgfpathlineto{\pgfqpoint{1.906608in}{2.005123in}}%
\pgfpathlineto{\pgfqpoint{1.952909in}{1.922592in}}%
\pgfpathlineto{\pgfqpoint{1.983776in}{1.873401in}}%
\pgfpathlineto{\pgfqpoint{1.999210in}{1.851401in}}%
\pgfpathlineto{\pgfqpoint{2.014644in}{1.831459in}}%
\pgfpathlineto{\pgfqpoint{2.030077in}{1.813788in}}%
\pgfpathlineto{\pgfqpoint{2.045511in}{1.798565in}}%
\pgfpathlineto{\pgfqpoint{2.060944in}{1.785927in}}%
\pgfpathlineto{\pgfqpoint{2.076378in}{1.775969in}}%
\pgfpathlineto{\pgfqpoint{2.091812in}{1.768739in}}%
\pgfpathlineto{\pgfqpoint{2.107245in}{1.764235in}}%
\pgfpathlineto{\pgfqpoint{2.122679in}{1.762404in}}%
\pgfpathlineto{\pgfqpoint{2.138113in}{1.763142in}}%
\pgfpathlineto{\pgfqpoint{2.153546in}{1.766292in}}%
\pgfpathlineto{\pgfqpoint{2.168980in}{1.771650in}}%
\pgfpathlineto{\pgfqpoint{2.184414in}{1.778964in}}%
\pgfpathlineto{\pgfqpoint{2.199847in}{1.787942in}}%
\pgfpathlineto{\pgfqpoint{2.230714in}{1.809541in}}%
\pgfpathlineto{\pgfqpoint{2.292449in}{1.856649in}}%
\pgfpathlineto{\pgfqpoint{2.307883in}{1.866935in}}%
\pgfpathlineto{\pgfqpoint{2.323316in}{1.875870in}}%
\pgfpathlineto{\pgfqpoint{2.338750in}{1.883117in}}%
\pgfpathlineto{\pgfqpoint{2.354184in}{1.888380in}}%
\pgfpathlineto{\pgfqpoint{2.369617in}{1.891404in}}%
\pgfpathlineto{\pgfqpoint{2.385051in}{1.891985in}}%
\pgfpathlineto{\pgfqpoint{2.400484in}{1.889976in}}%
\pgfpathlineto{\pgfqpoint{2.415918in}{1.885284in}}%
\pgfpathlineto{\pgfqpoint{2.431352in}{1.877880in}}%
\pgfpathlineto{\pgfqpoint{2.446785in}{1.867795in}}%
\pgfpathlineto{\pgfqpoint{2.462219in}{1.855117in}}%
\pgfpathlineto{\pgfqpoint{2.477653in}{1.839996in}}%
\pgfpathlineto{\pgfqpoint{2.493086in}{1.822636in}}%
\pgfpathlineto{\pgfqpoint{2.523954in}{1.782250in}}%
\pgfpathlineto{\pgfqpoint{2.554821in}{1.736470in}}%
\pgfpathlineto{\pgfqpoint{2.631989in}{1.618444in}}%
\pgfpathlineto{\pgfqpoint{2.662856in}{1.577985in}}%
\pgfpathlineto{\pgfqpoint{2.678290in}{1.560592in}}%
\pgfpathlineto{\pgfqpoint{2.693724in}{1.545452in}}%
\pgfpathlineto{\pgfqpoint{2.709157in}{1.532783in}}%
\pgfpathlineto{\pgfqpoint{2.724591in}{1.522748in}}%
\pgfpathlineto{\pgfqpoint{2.740024in}{1.515461in}}%
\pgfpathlineto{\pgfqpoint{2.755458in}{1.510976in}}%
\pgfpathlineto{\pgfqpoint{2.770892in}{1.509293in}}%
\pgfpathlineto{\pgfqpoint{2.786325in}{1.510355in}}%
\pgfpathlineto{\pgfqpoint{2.801759in}{1.514051in}}%
\pgfpathlineto{\pgfqpoint{2.817193in}{1.520215in}}%
\pgfpathlineto{\pgfqpoint{2.832626in}{1.528636in}}%
\pgfpathlineto{\pgfqpoint{2.848060in}{1.539055in}}%
\pgfpathlineto{\pgfqpoint{2.878927in}{1.564666in}}%
\pgfpathlineto{\pgfqpoint{2.909794in}{1.594318in}}%
\pgfpathlineto{\pgfqpoint{2.956095in}{1.639720in}}%
\pgfpathlineto{\pgfqpoint{2.986963in}{1.666161in}}%
\pgfpathlineto{\pgfqpoint{3.002396in}{1.677183in}}%
\pgfpathlineto{\pgfqpoint{3.017830in}{1.686342in}}%
\pgfpathlineto{\pgfqpoint{3.033263in}{1.693385in}}%
\pgfpathlineto{\pgfqpoint{3.048697in}{1.698108in}}%
\pgfpathlineto{\pgfqpoint{3.064131in}{1.700356in}}%
\pgfpathlineto{\pgfqpoint{3.079564in}{1.700027in}}%
\pgfpathlineto{\pgfqpoint{3.094998in}{1.697073in}}%
\pgfpathlineto{\pgfqpoint{3.110432in}{1.691504in}}%
\pgfpathlineto{\pgfqpoint{3.125865in}{1.683383in}}%
\pgfpathlineto{\pgfqpoint{3.141299in}{1.672831in}}%
\pgfpathlineto{\pgfqpoint{3.156733in}{1.660014in}}%
\pgfpathlineto{\pgfqpoint{3.172166in}{1.645151in}}%
\pgfpathlineto{\pgfqpoint{3.203033in}{1.610350in}}%
\pgfpathlineto{\pgfqpoint{3.233901in}{1.570878in}}%
\pgfpathlineto{\pgfqpoint{3.295635in}{1.489167in}}%
\pgfpathlineto{\pgfqpoint{3.326503in}{1.452517in}}%
\pgfpathlineto{\pgfqpoint{3.341936in}{1.436329in}}%
\pgfpathlineto{\pgfqpoint{3.357370in}{1.421900in}}%
\pgfpathlineto{\pgfqpoint{3.372803in}{1.409430in}}%
\pgfpathlineto{\pgfqpoint{3.388237in}{1.399079in}}%
\pgfpathlineto{\pgfqpoint{3.403671in}{1.390965in}}%
\pgfpathlineto{\pgfqpoint{3.419104in}{1.385159in}}%
\pgfpathlineto{\pgfqpoint{3.434538in}{1.381693in}}%
\pgfpathlineto{\pgfqpoint{3.449972in}{1.380556in}}%
\pgfpathlineto{\pgfqpoint{3.465405in}{1.381704in}}%
\pgfpathlineto{\pgfqpoint{3.480839in}{1.385056in}}%
\pgfpathlineto{\pgfqpoint{3.496273in}{1.390506in}}%
\pgfpathlineto{\pgfqpoint{3.511706in}{1.397925in}}%
\pgfpathlineto{\pgfqpoint{3.527140in}{1.407168in}}%
\pgfpathlineto{\pgfqpoint{3.542573in}{1.418076in}}%
\pgfpathlineto{\pgfqpoint{3.573441in}{1.444242in}}%
\pgfpathlineto{\pgfqpoint{3.604308in}{1.475145in}}%
\pgfpathlineto{\pgfqpoint{3.635175in}{1.509626in}}%
\pgfpathlineto{\pgfqpoint{3.681476in}{1.566019in}}%
\pgfpathlineto{\pgfqpoint{3.743211in}{1.646457in}}%
\pgfpathlineto{\pgfqpoint{3.851246in}{1.789258in}}%
\pgfpathlineto{\pgfqpoint{3.882113in}{1.826652in}}%
\pgfpathlineto{\pgfqpoint{3.912981in}{1.860320in}}%
\pgfpathlineto{\pgfqpoint{3.943848in}{1.888804in}}%
\pgfpathlineto{\pgfqpoint{3.959282in}{1.900654in}}%
\pgfpathlineto{\pgfqpoint{3.974715in}{1.910695in}}%
\pgfpathlineto{\pgfqpoint{3.990149in}{1.918788in}}%
\pgfpathlineto{\pgfqpoint{4.005583in}{1.924827in}}%
\pgfpathlineto{\pgfqpoint{4.021016in}{1.928737in}}%
\pgfpathlineto{\pgfqpoint{4.036450in}{1.930480in}}%
\pgfpathlineto{\pgfqpoint{4.051883in}{1.930066in}}%
\pgfpathlineto{\pgfqpoint{4.067317in}{1.927548in}}%
\pgfpathlineto{\pgfqpoint{4.082751in}{1.923025in}}%
\pgfpathlineto{\pgfqpoint{4.098184in}{1.916649in}}%
\pgfpathlineto{\pgfqpoint{4.113618in}{1.908614in}}%
\pgfpathlineto{\pgfqpoint{4.144485in}{1.888563in}}%
\pgfpathlineto{\pgfqpoint{4.190786in}{1.853161in}}%
\pgfpathlineto{\pgfqpoint{4.221653in}{1.830088in}}%
\pgfpathlineto{\pgfqpoint{4.252521in}{1.810751in}}%
\pgfpathlineto{\pgfqpoint{4.267954in}{1.803286in}}%
\pgfpathlineto{\pgfqpoint{4.283388in}{1.797652in}}%
\pgfpathlineto{\pgfqpoint{4.298822in}{1.794062in}}%
\pgfpathlineto{\pgfqpoint{4.314255in}{1.792678in}}%
\pgfpathlineto{\pgfqpoint{4.329689in}{1.793604in}}%
\pgfpathlineto{\pgfqpoint{4.345122in}{1.796882in}}%
\pgfpathlineto{\pgfqpoint{4.360556in}{1.802494in}}%
\pgfpathlineto{\pgfqpoint{4.375990in}{1.810357in}}%
\pgfpathlineto{\pgfqpoint{4.391423in}{1.820332in}}%
\pgfpathlineto{\pgfqpoint{4.406857in}{1.832222in}}%
\pgfpathlineto{\pgfqpoint{4.437724in}{1.860728in}}%
\pgfpathlineto{\pgfqpoint{4.468592in}{1.893460in}}%
\pgfpathlineto{\pgfqpoint{4.530326in}{1.960345in}}%
\pgfpathlineto{\pgfqpoint{4.561193in}{1.989026in}}%
\pgfpathlineto{\pgfqpoint{4.576627in}{2.001175in}}%
\pgfpathlineto{\pgfqpoint{4.592061in}{2.011589in}}%
\pgfpathlineto{\pgfqpoint{4.607494in}{2.020125in}}%
\pgfpathlineto{\pgfqpoint{4.622928in}{2.026694in}}%
\pgfpathlineto{\pgfqpoint{4.638362in}{2.031261in}}%
\pgfpathlineto{\pgfqpoint{4.653795in}{2.033847in}}%
\pgfpathlineto{\pgfqpoint{4.669229in}{2.034521in}}%
\pgfpathlineto{\pgfqpoint{4.669229in}{2.034521in}}%
\pgfusepath{stroke}%
\end{pgfscope}%
\begin{pgfscope}%
\pgfpathrectangle{\pgfqpoint{0.634105in}{0.521603in}}{\pgfqpoint{4.227273in}{2.800000in}} %
\pgfusepath{clip}%
\pgfsetrectcap%
\pgfsetroundjoin%
\pgfsetlinewidth{0.501875pt}%
\definecolor{currentstroke}{rgb}{0.700000,0.951057,0.587785}%
\pgfsetstrokecolor{currentstroke}%
\pgfsetdash{}{0pt}%
\pgfpathmoveto{\pgfqpoint{0.826254in}{2.312937in}}%
\pgfpathlineto{\pgfqpoint{0.841687in}{2.317345in}}%
\pgfpathlineto{\pgfqpoint{0.857121in}{2.318043in}}%
\pgfpathlineto{\pgfqpoint{0.872555in}{2.314753in}}%
\pgfpathlineto{\pgfqpoint{0.887988in}{2.307269in}}%
\pgfpathlineto{\pgfqpoint{0.903422in}{2.295466in}}%
\pgfpathlineto{\pgfqpoint{0.918855in}{2.279305in}}%
\pgfpathlineto{\pgfqpoint{0.934289in}{2.258841in}}%
\pgfpathlineto{\pgfqpoint{0.949723in}{2.234221in}}%
\pgfpathlineto{\pgfqpoint{0.965156in}{2.205688in}}%
\pgfpathlineto{\pgfqpoint{0.980590in}{2.173575in}}%
\pgfpathlineto{\pgfqpoint{1.011457in}{2.100370in}}%
\pgfpathlineto{\pgfqpoint{1.042325in}{2.018854in}}%
\pgfpathlineto{\pgfqpoint{1.104059in}{1.852006in}}%
\pgfpathlineto{\pgfqpoint{1.134926in}{1.777931in}}%
\pgfpathlineto{\pgfqpoint{1.150360in}{1.745564in}}%
\pgfpathlineto{\pgfqpoint{1.165794in}{1.717076in}}%
\pgfpathlineto{\pgfqpoint{1.181227in}{1.692953in}}%
\pgfpathlineto{\pgfqpoint{1.196661in}{1.673600in}}%
\pgfpathlineto{\pgfqpoint{1.212095in}{1.659331in}}%
\pgfpathlineto{\pgfqpoint{1.227528in}{1.650364in}}%
\pgfpathlineto{\pgfqpoint{1.242962in}{1.646816in}}%
\pgfpathlineto{\pgfqpoint{1.258395in}{1.648707in}}%
\pgfpathlineto{\pgfqpoint{1.273829in}{1.655954in}}%
\pgfpathlineto{\pgfqpoint{1.289263in}{1.668381in}}%
\pgfpathlineto{\pgfqpoint{1.304696in}{1.685720in}}%
\pgfpathlineto{\pgfqpoint{1.320130in}{1.707622in}}%
\pgfpathlineto{\pgfqpoint{1.335564in}{1.733668in}}%
\pgfpathlineto{\pgfqpoint{1.350997in}{1.763374in}}%
\pgfpathlineto{\pgfqpoint{1.381864in}{1.831596in}}%
\pgfpathlineto{\pgfqpoint{1.412732in}{1.907656in}}%
\pgfpathlineto{\pgfqpoint{1.474466in}{2.064180in}}%
\pgfpathlineto{\pgfqpoint{1.505334in}{2.135894in}}%
\pgfpathlineto{\pgfqpoint{1.520767in}{2.168513in}}%
\pgfpathlineto{\pgfqpoint{1.536201in}{2.198502in}}%
\pgfpathlineto{\pgfqpoint{1.551634in}{2.225581in}}%
\pgfpathlineto{\pgfqpoint{1.567068in}{2.249526in}}%
\pgfpathlineto{\pgfqpoint{1.582502in}{2.270172in}}%
\pgfpathlineto{\pgfqpoint{1.597935in}{2.287409in}}%
\pgfpathlineto{\pgfqpoint{1.613369in}{2.301177in}}%
\pgfpathlineto{\pgfqpoint{1.628803in}{2.311464in}}%
\pgfpathlineto{\pgfqpoint{1.644236in}{2.318302in}}%
\pgfpathlineto{\pgfqpoint{1.659670in}{2.321759in}}%
\pgfpathlineto{\pgfqpoint{1.675104in}{2.321933in}}%
\pgfpathlineto{\pgfqpoint{1.690537in}{2.318953in}}%
\pgfpathlineto{\pgfqpoint{1.705971in}{2.312966in}}%
\pgfpathlineto{\pgfqpoint{1.721404in}{2.304137in}}%
\pgfpathlineto{\pgfqpoint{1.736838in}{2.292647in}}%
\pgfpathlineto{\pgfqpoint{1.752272in}{2.278683in}}%
\pgfpathlineto{\pgfqpoint{1.767705in}{2.262443in}}%
\pgfpathlineto{\pgfqpoint{1.783139in}{2.244128in}}%
\pgfpathlineto{\pgfqpoint{1.814006in}{2.202099in}}%
\pgfpathlineto{\pgfqpoint{1.844874in}{2.154288in}}%
\pgfpathlineto{\pgfqpoint{1.891174in}{2.075580in}}%
\pgfpathlineto{\pgfqpoint{1.968343in}{1.941324in}}%
\pgfpathlineto{\pgfqpoint{1.999210in}{1.892221in}}%
\pgfpathlineto{\pgfqpoint{2.030077in}{1.848588in}}%
\pgfpathlineto{\pgfqpoint{2.045511in}{1.829348in}}%
\pgfpathlineto{\pgfqpoint{2.060944in}{1.812063in}}%
\pgfpathlineto{\pgfqpoint{2.076378in}{1.796878in}}%
\pgfpathlineto{\pgfqpoint{2.091812in}{1.783902in}}%
\pgfpathlineto{\pgfqpoint{2.107245in}{1.773210in}}%
\pgfpathlineto{\pgfqpoint{2.122679in}{1.764833in}}%
\pgfpathlineto{\pgfqpoint{2.138113in}{1.758759in}}%
\pgfpathlineto{\pgfqpoint{2.153546in}{1.754929in}}%
\pgfpathlineto{\pgfqpoint{2.168980in}{1.753241in}}%
\pgfpathlineto{\pgfqpoint{2.184414in}{1.753542in}}%
\pgfpathlineto{\pgfqpoint{2.199847in}{1.755638in}}%
\pgfpathlineto{\pgfqpoint{2.215281in}{1.759294in}}%
\pgfpathlineto{\pgfqpoint{2.246148in}{1.770162in}}%
\pgfpathlineto{\pgfqpoint{2.323316in}{1.802620in}}%
\pgfpathlineto{\pgfqpoint{2.338750in}{1.807250in}}%
\pgfpathlineto{\pgfqpoint{2.354184in}{1.810508in}}%
\pgfpathlineto{\pgfqpoint{2.369617in}{1.812127in}}%
\pgfpathlineto{\pgfqpoint{2.385051in}{1.811883in}}%
\pgfpathlineto{\pgfqpoint{2.400484in}{1.809600in}}%
\pgfpathlineto{\pgfqpoint{2.415918in}{1.805153in}}%
\pgfpathlineto{\pgfqpoint{2.431352in}{1.798477in}}%
\pgfpathlineto{\pgfqpoint{2.446785in}{1.789563in}}%
\pgfpathlineto{\pgfqpoint{2.462219in}{1.778462in}}%
\pgfpathlineto{\pgfqpoint{2.477653in}{1.765284in}}%
\pgfpathlineto{\pgfqpoint{2.493086in}{1.750195in}}%
\pgfpathlineto{\pgfqpoint{2.523954in}{1.715203in}}%
\pgfpathlineto{\pgfqpoint{2.554821in}{1.675754in}}%
\pgfpathlineto{\pgfqpoint{2.616555in}{1.594888in}}%
\pgfpathlineto{\pgfqpoint{2.647423in}{1.559506in}}%
\pgfpathlineto{\pgfqpoint{2.662856in}{1.544305in}}%
\pgfpathlineto{\pgfqpoint{2.678290in}{1.531155in}}%
\pgfpathlineto{\pgfqpoint{2.693724in}{1.520296in}}%
\pgfpathlineto{\pgfqpoint{2.709157in}{1.511919in}}%
\pgfpathlineto{\pgfqpoint{2.724591in}{1.506163in}}%
\pgfpathlineto{\pgfqpoint{2.740024in}{1.503107in}}%
\pgfpathlineto{\pgfqpoint{2.755458in}{1.502775in}}%
\pgfpathlineto{\pgfqpoint{2.770892in}{1.505128in}}%
\pgfpathlineto{\pgfqpoint{2.786325in}{1.510071in}}%
\pgfpathlineto{\pgfqpoint{2.801759in}{1.517452in}}%
\pgfpathlineto{\pgfqpoint{2.817193in}{1.527064in}}%
\pgfpathlineto{\pgfqpoint{2.832626in}{1.538654in}}%
\pgfpathlineto{\pgfqpoint{2.863493in}{1.566537in}}%
\pgfpathlineto{\pgfqpoint{2.909794in}{1.614706in}}%
\pgfpathlineto{\pgfqpoint{2.940662in}{1.646448in}}%
\pgfpathlineto{\pgfqpoint{2.971529in}{1.674229in}}%
\pgfpathlineto{\pgfqpoint{2.986963in}{1.685743in}}%
\pgfpathlineto{\pgfqpoint{3.002396in}{1.695261in}}%
\pgfpathlineto{\pgfqpoint{3.017830in}{1.702530in}}%
\pgfpathlineto{\pgfqpoint{3.033263in}{1.707344in}}%
\pgfpathlineto{\pgfqpoint{3.048697in}{1.709552in}}%
\pgfpathlineto{\pgfqpoint{3.064131in}{1.709060in}}%
\pgfpathlineto{\pgfqpoint{3.079564in}{1.705836in}}%
\pgfpathlineto{\pgfqpoint{3.094998in}{1.699905in}}%
\pgfpathlineto{\pgfqpoint{3.110432in}{1.691355in}}%
\pgfpathlineto{\pgfqpoint{3.125865in}{1.680328in}}%
\pgfpathlineto{\pgfqpoint{3.141299in}{1.667023in}}%
\pgfpathlineto{\pgfqpoint{3.156733in}{1.651686in}}%
\pgfpathlineto{\pgfqpoint{3.187600in}{1.616110in}}%
\pgfpathlineto{\pgfqpoint{3.233901in}{1.555730in}}%
\pgfpathlineto{\pgfqpoint{3.280202in}{1.495936in}}%
\pgfpathlineto{\pgfqpoint{3.311069in}{1.461118in}}%
\pgfpathlineto{\pgfqpoint{3.326503in}{1.446104in}}%
\pgfpathlineto{\pgfqpoint{3.341936in}{1.432985in}}%
\pgfpathlineto{\pgfqpoint{3.357370in}{1.421924in}}%
\pgfpathlineto{\pgfqpoint{3.372803in}{1.413036in}}%
\pgfpathlineto{\pgfqpoint{3.388237in}{1.406384in}}%
\pgfpathlineto{\pgfqpoint{3.403671in}{1.401982in}}%
\pgfpathlineto{\pgfqpoint{3.419104in}{1.399796in}}%
\pgfpathlineto{\pgfqpoint{3.434538in}{1.399750in}}%
\pgfpathlineto{\pgfqpoint{3.449972in}{1.401729in}}%
\pgfpathlineto{\pgfqpoint{3.465405in}{1.405587in}}%
\pgfpathlineto{\pgfqpoint{3.480839in}{1.411154in}}%
\pgfpathlineto{\pgfqpoint{3.496273in}{1.418240in}}%
\pgfpathlineto{\pgfqpoint{3.527140in}{1.436175in}}%
\pgfpathlineto{\pgfqpoint{3.558007in}{1.457823in}}%
\pgfpathlineto{\pgfqpoint{3.604308in}{1.494287in}}%
\pgfpathlineto{\pgfqpoint{3.681476in}{1.559009in}}%
\pgfpathlineto{\pgfqpoint{3.743211in}{1.612982in}}%
\pgfpathlineto{\pgfqpoint{3.789512in}{1.655796in}}%
\pgfpathlineto{\pgfqpoint{3.912981in}{1.773669in}}%
\pgfpathlineto{\pgfqpoint{3.943848in}{1.798837in}}%
\pgfpathlineto{\pgfqpoint{3.974715in}{1.819572in}}%
\pgfpathlineto{\pgfqpoint{3.990149in}{1.827889in}}%
\pgfpathlineto{\pgfqpoint{4.005583in}{1.834680in}}%
\pgfpathlineto{\pgfqpoint{4.021016in}{1.839867in}}%
\pgfpathlineto{\pgfqpoint{4.036450in}{1.843410in}}%
\pgfpathlineto{\pgfqpoint{4.051883in}{1.845315in}}%
\pgfpathlineto{\pgfqpoint{4.067317in}{1.845631in}}%
\pgfpathlineto{\pgfqpoint{4.082751in}{1.844456in}}%
\pgfpathlineto{\pgfqpoint{4.098184in}{1.841931in}}%
\pgfpathlineto{\pgfqpoint{4.129052in}{1.833615in}}%
\pgfpathlineto{\pgfqpoint{4.221653in}{1.801911in}}%
\pgfpathlineto{\pgfqpoint{4.237087in}{1.798786in}}%
\pgfpathlineto{\pgfqpoint{4.252521in}{1.797023in}}%
\pgfpathlineto{\pgfqpoint{4.267954in}{1.796835in}}%
\pgfpathlineto{\pgfqpoint{4.283388in}{1.798395in}}%
\pgfpathlineto{\pgfqpoint{4.298822in}{1.801822in}}%
\pgfpathlineto{\pgfqpoint{4.314255in}{1.807181in}}%
\pgfpathlineto{\pgfqpoint{4.329689in}{1.814482in}}%
\pgfpathlineto{\pgfqpoint{4.345122in}{1.823672in}}%
\pgfpathlineto{\pgfqpoint{4.360556in}{1.834644in}}%
\pgfpathlineto{\pgfqpoint{4.391423in}{1.861238in}}%
\pgfpathlineto{\pgfqpoint{4.422291in}{1.892403in}}%
\pgfpathlineto{\pgfqpoint{4.499459in}{1.973753in}}%
\pgfpathlineto{\pgfqpoint{4.530326in}{2.000713in}}%
\pgfpathlineto{\pgfqpoint{4.545760in}{2.011851in}}%
\pgfpathlineto{\pgfqpoint{4.561193in}{2.021156in}}%
\pgfpathlineto{\pgfqpoint{4.576627in}{2.028488in}}%
\pgfpathlineto{\pgfqpoint{4.592061in}{2.033759in}}%
\pgfpathlineto{\pgfqpoint{4.607494in}{2.036940in}}%
\pgfpathlineto{\pgfqpoint{4.622928in}{2.038057in}}%
\pgfpathlineto{\pgfqpoint{4.638362in}{2.037192in}}%
\pgfpathlineto{\pgfqpoint{4.653795in}{2.034476in}}%
\pgfpathlineto{\pgfqpoint{4.669229in}{2.030088in}}%
\pgfpathlineto{\pgfqpoint{4.669229in}{2.030088in}}%
\pgfusepath{stroke}%
\end{pgfscope}%
\begin{pgfscope}%
\pgfpathrectangle{\pgfqpoint{0.634105in}{0.521603in}}{\pgfqpoint{4.227273in}{2.800000in}} %
\pgfusepath{clip}%
\pgfsetrectcap%
\pgfsetroundjoin%
\pgfsetlinewidth{0.501875pt}%
\definecolor{currentstroke}{rgb}{0.778431,0.905873,0.536867}%
\pgfsetstrokecolor{currentstroke}%
\pgfsetdash{}{0pt}%
\pgfpathmoveto{\pgfqpoint{0.826254in}{2.262500in}}%
\pgfpathlineto{\pgfqpoint{0.841687in}{2.263855in}}%
\pgfpathlineto{\pgfqpoint{0.857121in}{2.261707in}}%
\pgfpathlineto{\pgfqpoint{0.872555in}{2.255833in}}%
\pgfpathlineto{\pgfqpoint{0.887988in}{2.246070in}}%
\pgfpathlineto{\pgfqpoint{0.903422in}{2.232331in}}%
\pgfpathlineto{\pgfqpoint{0.918855in}{2.214602in}}%
\pgfpathlineto{\pgfqpoint{0.934289in}{2.192953in}}%
\pgfpathlineto{\pgfqpoint{0.949723in}{2.167537in}}%
\pgfpathlineto{\pgfqpoint{0.965156in}{2.138589in}}%
\pgfpathlineto{\pgfqpoint{0.980590in}{2.106427in}}%
\pgfpathlineto{\pgfqpoint{1.011457in}{2.034106in}}%
\pgfpathlineto{\pgfqpoint{1.042325in}{1.954524in}}%
\pgfpathlineto{\pgfqpoint{1.104059in}{1.793147in}}%
\pgfpathlineto{\pgfqpoint{1.134926in}{1.721840in}}%
\pgfpathlineto{\pgfqpoint{1.150360in}{1.690727in}}%
\pgfpathlineto{\pgfqpoint{1.165794in}{1.663371in}}%
\pgfpathlineto{\pgfqpoint{1.181227in}{1.640245in}}%
\pgfpathlineto{\pgfqpoint{1.196661in}{1.621745in}}%
\pgfpathlineto{\pgfqpoint{1.212095in}{1.608188in}}%
\pgfpathlineto{\pgfqpoint{1.227528in}{1.599802in}}%
\pgfpathlineto{\pgfqpoint{1.242962in}{1.596724in}}%
\pgfpathlineto{\pgfqpoint{1.258395in}{1.598995in}}%
\pgfpathlineto{\pgfqpoint{1.273829in}{1.606562in}}%
\pgfpathlineto{\pgfqpoint{1.289263in}{1.619284in}}%
\pgfpathlineto{\pgfqpoint{1.304696in}{1.636927in}}%
\pgfpathlineto{\pgfqpoint{1.320130in}{1.659179in}}%
\pgfpathlineto{\pgfqpoint{1.335564in}{1.685654in}}%
\pgfpathlineto{\pgfqpoint{1.350997in}{1.715900in}}%
\pgfpathlineto{\pgfqpoint{1.366431in}{1.749413in}}%
\pgfpathlineto{\pgfqpoint{1.397298in}{1.824013in}}%
\pgfpathlineto{\pgfqpoint{1.443599in}{1.945965in}}%
\pgfpathlineto{\pgfqpoint{1.489900in}{2.065887in}}%
\pgfpathlineto{\pgfqpoint{1.520767in}{2.137728in}}%
\pgfpathlineto{\pgfqpoint{1.536201in}{2.169942in}}%
\pgfpathlineto{\pgfqpoint{1.551634in}{2.199246in}}%
\pgfpathlineto{\pgfqpoint{1.567068in}{2.225385in}}%
\pgfpathlineto{\pgfqpoint{1.582502in}{2.248165in}}%
\pgfpathlineto{\pgfqpoint{1.597935in}{2.267449in}}%
\pgfpathlineto{\pgfqpoint{1.613369in}{2.283158in}}%
\pgfpathlineto{\pgfqpoint{1.628803in}{2.295264in}}%
\pgfpathlineto{\pgfqpoint{1.644236in}{2.303790in}}%
\pgfpathlineto{\pgfqpoint{1.659670in}{2.308800in}}%
\pgfpathlineto{\pgfqpoint{1.675104in}{2.310399in}}%
\pgfpathlineto{\pgfqpoint{1.690537in}{2.308722in}}%
\pgfpathlineto{\pgfqpoint{1.705971in}{2.303931in}}%
\pgfpathlineto{\pgfqpoint{1.721404in}{2.296211in}}%
\pgfpathlineto{\pgfqpoint{1.736838in}{2.285762in}}%
\pgfpathlineto{\pgfqpoint{1.752272in}{2.272794in}}%
\pgfpathlineto{\pgfqpoint{1.767705in}{2.257528in}}%
\pgfpathlineto{\pgfqpoint{1.783139in}{2.240186in}}%
\pgfpathlineto{\pgfqpoint{1.814006in}{2.200172in}}%
\pgfpathlineto{\pgfqpoint{1.844874in}{2.154534in}}%
\pgfpathlineto{\pgfqpoint{1.891174in}{2.079348in}}%
\pgfpathlineto{\pgfqpoint{1.968343in}{1.950223in}}%
\pgfpathlineto{\pgfqpoint{1.999210in}{1.902087in}}%
\pgfpathlineto{\pgfqpoint{2.030077in}{1.858354in}}%
\pgfpathlineto{\pgfqpoint{2.060944in}{1.820450in}}%
\pgfpathlineto{\pgfqpoint{2.076378in}{1.804066in}}%
\pgfpathlineto{\pgfqpoint{2.091812in}{1.789555in}}%
\pgfpathlineto{\pgfqpoint{2.107245in}{1.777008in}}%
\pgfpathlineto{\pgfqpoint{2.122679in}{1.766483in}}%
\pgfpathlineto{\pgfqpoint{2.138113in}{1.758005in}}%
\pgfpathlineto{\pgfqpoint{2.153546in}{1.751564in}}%
\pgfpathlineto{\pgfqpoint{2.168980in}{1.747113in}}%
\pgfpathlineto{\pgfqpoint{2.184414in}{1.744564in}}%
\pgfpathlineto{\pgfqpoint{2.199847in}{1.743793in}}%
\pgfpathlineto{\pgfqpoint{2.215281in}{1.744637in}}%
\pgfpathlineto{\pgfqpoint{2.230714in}{1.746898in}}%
\pgfpathlineto{\pgfqpoint{2.261582in}{1.754727in}}%
\pgfpathlineto{\pgfqpoint{2.354184in}{1.784245in}}%
\pgfpathlineto{\pgfqpoint{2.369617in}{1.786968in}}%
\pgfpathlineto{\pgfqpoint{2.385051in}{1.788368in}}%
\pgfpathlineto{\pgfqpoint{2.400484in}{1.788249in}}%
\pgfpathlineto{\pgfqpoint{2.415918in}{1.786458in}}%
\pgfpathlineto{\pgfqpoint{2.431352in}{1.782887in}}%
\pgfpathlineto{\pgfqpoint{2.446785in}{1.777481in}}%
\pgfpathlineto{\pgfqpoint{2.462219in}{1.770235in}}%
\pgfpathlineto{\pgfqpoint{2.477653in}{1.761197in}}%
\pgfpathlineto{\pgfqpoint{2.493086in}{1.750470in}}%
\pgfpathlineto{\pgfqpoint{2.523954in}{1.724595in}}%
\pgfpathlineto{\pgfqpoint{2.554821in}{1.694351in}}%
\pgfpathlineto{\pgfqpoint{2.631989in}{1.615429in}}%
\pgfpathlineto{\pgfqpoint{2.662856in}{1.589419in}}%
\pgfpathlineto{\pgfqpoint{2.678290in}{1.578801in}}%
\pgfpathlineto{\pgfqpoint{2.693724in}{1.570086in}}%
\pgfpathlineto{\pgfqpoint{2.709157in}{1.563451in}}%
\pgfpathlineto{\pgfqpoint{2.724591in}{1.559024in}}%
\pgfpathlineto{\pgfqpoint{2.740024in}{1.556879in}}%
\pgfpathlineto{\pgfqpoint{2.755458in}{1.557037in}}%
\pgfpathlineto{\pgfqpoint{2.770892in}{1.559458in}}%
\pgfpathlineto{\pgfqpoint{2.786325in}{1.564052in}}%
\pgfpathlineto{\pgfqpoint{2.801759in}{1.570670in}}%
\pgfpathlineto{\pgfqpoint{2.817193in}{1.579117in}}%
\pgfpathlineto{\pgfqpoint{2.848060in}{1.600493in}}%
\pgfpathlineto{\pgfqpoint{2.878927in}{1.625822in}}%
\pgfpathlineto{\pgfqpoint{2.925228in}{1.665192in}}%
\pgfpathlineto{\pgfqpoint{2.956095in}{1.688242in}}%
\pgfpathlineto{\pgfqpoint{2.971529in}{1.697833in}}%
\pgfpathlineto{\pgfqpoint{2.986963in}{1.705767in}}%
\pgfpathlineto{\pgfqpoint{3.002396in}{1.711820in}}%
\pgfpathlineto{\pgfqpoint{3.017830in}{1.715811in}}%
\pgfpathlineto{\pgfqpoint{3.033263in}{1.717613in}}%
\pgfpathlineto{\pgfqpoint{3.048697in}{1.717147in}}%
\pgfpathlineto{\pgfqpoint{3.064131in}{1.714393in}}%
\pgfpathlineto{\pgfqpoint{3.079564in}{1.709384in}}%
\pgfpathlineto{\pgfqpoint{3.094998in}{1.702204in}}%
\pgfpathlineto{\pgfqpoint{3.110432in}{1.692990in}}%
\pgfpathlineto{\pgfqpoint{3.125865in}{1.681923in}}%
\pgfpathlineto{\pgfqpoint{3.156733in}{1.655152in}}%
\pgfpathlineto{\pgfqpoint{3.187600in}{1.624040in}}%
\pgfpathlineto{\pgfqpoint{3.264768in}{1.543458in}}%
\pgfpathlineto{\pgfqpoint{3.295635in}{1.516224in}}%
\pgfpathlineto{\pgfqpoint{3.311069in}{1.504610in}}%
\pgfpathlineto{\pgfqpoint{3.326503in}{1.494534in}}%
\pgfpathlineto{\pgfqpoint{3.341936in}{1.486097in}}%
\pgfpathlineto{\pgfqpoint{3.357370in}{1.479357in}}%
\pgfpathlineto{\pgfqpoint{3.372803in}{1.474329in}}%
\pgfpathlineto{\pgfqpoint{3.388237in}{1.470990in}}%
\pgfpathlineto{\pgfqpoint{3.403671in}{1.469276in}}%
\pgfpathlineto{\pgfqpoint{3.419104in}{1.469095in}}%
\pgfpathlineto{\pgfqpoint{3.434538in}{1.470327in}}%
\pgfpathlineto{\pgfqpoint{3.465405in}{1.476446in}}%
\pgfpathlineto{\pgfqpoint{3.496273in}{1.486372in}}%
\pgfpathlineto{\pgfqpoint{3.527140in}{1.498823in}}%
\pgfpathlineto{\pgfqpoint{3.588874in}{1.527232in}}%
\pgfpathlineto{\pgfqpoint{3.666043in}{1.565095in}}%
\pgfpathlineto{\pgfqpoint{3.712343in}{1.590543in}}%
\pgfpathlineto{\pgfqpoint{3.743211in}{1.609927in}}%
\pgfpathlineto{\pgfqpoint{3.774078in}{1.631861in}}%
\pgfpathlineto{\pgfqpoint{3.804945in}{1.656544in}}%
\pgfpathlineto{\pgfqpoint{3.835813in}{1.683746in}}%
\pgfpathlineto{\pgfqpoint{3.897547in}{1.742511in}}%
\pgfpathlineto{\pgfqpoint{3.943848in}{1.785280in}}%
\pgfpathlineto{\pgfqpoint{3.974715in}{1.810321in}}%
\pgfpathlineto{\pgfqpoint{4.005583in}{1.830988in}}%
\pgfpathlineto{\pgfqpoint{4.021016in}{1.839368in}}%
\pgfpathlineto{\pgfqpoint{4.036450in}{1.846343in}}%
\pgfpathlineto{\pgfqpoint{4.051883in}{1.851886in}}%
\pgfpathlineto{\pgfqpoint{4.067317in}{1.856011in}}%
\pgfpathlineto{\pgfqpoint{4.082751in}{1.858771in}}%
\pgfpathlineto{\pgfqpoint{4.098184in}{1.860265in}}%
\pgfpathlineto{\pgfqpoint{4.129052in}{1.860025in}}%
\pgfpathlineto{\pgfqpoint{4.159919in}{1.856761in}}%
\pgfpathlineto{\pgfqpoint{4.221653in}{1.848705in}}%
\pgfpathlineto{\pgfqpoint{4.252521in}{1.847786in}}%
\pgfpathlineto{\pgfqpoint{4.267954in}{1.848833in}}%
\pgfpathlineto{\pgfqpoint{4.283388in}{1.851083in}}%
\pgfpathlineto{\pgfqpoint{4.298822in}{1.854635in}}%
\pgfpathlineto{\pgfqpoint{4.314255in}{1.859543in}}%
\pgfpathlineto{\pgfqpoint{4.329689in}{1.865816in}}%
\pgfpathlineto{\pgfqpoint{4.360556in}{1.882246in}}%
\pgfpathlineto{\pgfqpoint{4.391423in}{1.903064in}}%
\pgfpathlineto{\pgfqpoint{4.437724in}{1.939117in}}%
\pgfpathlineto{\pgfqpoint{4.484025in}{1.974875in}}%
\pgfpathlineto{\pgfqpoint{4.514892in}{1.995000in}}%
\pgfpathlineto{\pgfqpoint{4.530326in}{2.003238in}}%
\pgfpathlineto{\pgfqpoint{4.545760in}{2.010009in}}%
\pgfpathlineto{\pgfqpoint{4.561193in}{2.015173in}}%
\pgfpathlineto{\pgfqpoint{4.576627in}{2.018631in}}%
\pgfpathlineto{\pgfqpoint{4.592061in}{2.020331in}}%
\pgfpathlineto{\pgfqpoint{4.607494in}{2.020268in}}%
\pgfpathlineto{\pgfqpoint{4.622928in}{2.018482in}}%
\pgfpathlineto{\pgfqpoint{4.638362in}{2.015059in}}%
\pgfpathlineto{\pgfqpoint{4.653795in}{2.010128in}}%
\pgfpathlineto{\pgfqpoint{4.669229in}{2.003853in}}%
\pgfpathlineto{\pgfqpoint{4.669229in}{2.003853in}}%
\pgfusepath{stroke}%
\end{pgfscope}%
\begin{pgfscope}%
\pgfpathrectangle{\pgfqpoint{0.634105in}{0.521603in}}{\pgfqpoint{4.227273in}{2.800000in}} %
\pgfusepath{clip}%
\pgfsetrectcap%
\pgfsetroundjoin%
\pgfsetlinewidth{0.501875pt}%
\definecolor{currentstroke}{rgb}{0.864706,0.840344,0.478512}%
\pgfsetstrokecolor{currentstroke}%
\pgfsetdash{}{0pt}%
\pgfpathmoveto{\pgfqpoint{0.826254in}{2.299488in}}%
\pgfpathlineto{\pgfqpoint{0.841687in}{2.298375in}}%
\pgfpathlineto{\pgfqpoint{0.857121in}{2.293504in}}%
\pgfpathlineto{\pgfqpoint{0.872555in}{2.284749in}}%
\pgfpathlineto{\pgfqpoint{0.887988in}{2.272049in}}%
\pgfpathlineto{\pgfqpoint{0.903422in}{2.255416in}}%
\pgfpathlineto{\pgfqpoint{0.918855in}{2.234931in}}%
\pgfpathlineto{\pgfqpoint{0.934289in}{2.210750in}}%
\pgfpathlineto{\pgfqpoint{0.949723in}{2.183103in}}%
\pgfpathlineto{\pgfqpoint{0.965156in}{2.152289in}}%
\pgfpathlineto{\pgfqpoint{0.996024in}{2.082685in}}%
\pgfpathlineto{\pgfqpoint{1.026891in}{2.005545in}}%
\pgfpathlineto{\pgfqpoint{1.088625in}{1.846347in}}%
\pgfpathlineto{\pgfqpoint{1.119493in}{1.773794in}}%
\pgfpathlineto{\pgfqpoint{1.134926in}{1.741301in}}%
\pgfpathlineto{\pgfqpoint{1.150360in}{1.712022in}}%
\pgfpathlineto{\pgfqpoint{1.165794in}{1.686419in}}%
\pgfpathlineto{\pgfqpoint{1.181227in}{1.664898in}}%
\pgfpathlineto{\pgfqpoint{1.196661in}{1.647799in}}%
\pgfpathlineto{\pgfqpoint{1.212095in}{1.635393in}}%
\pgfpathlineto{\pgfqpoint{1.227528in}{1.627880in}}%
\pgfpathlineto{\pgfqpoint{1.242962in}{1.625378in}}%
\pgfpathlineto{\pgfqpoint{1.258395in}{1.627930in}}%
\pgfpathlineto{\pgfqpoint{1.273829in}{1.635497in}}%
\pgfpathlineto{\pgfqpoint{1.289263in}{1.647963in}}%
\pgfpathlineto{\pgfqpoint{1.304696in}{1.665136in}}%
\pgfpathlineto{\pgfqpoint{1.320130in}{1.686753in}}%
\pgfpathlineto{\pgfqpoint{1.335564in}{1.712485in}}%
\pgfpathlineto{\pgfqpoint{1.350997in}{1.741940in}}%
\pgfpathlineto{\pgfqpoint{1.366431in}{1.774675in}}%
\pgfpathlineto{\pgfqpoint{1.397298in}{1.847998in}}%
\pgfpathlineto{\pgfqpoint{1.443599in}{1.969397in}}%
\pgfpathlineto{\pgfqpoint{1.489900in}{2.090844in}}%
\pgfpathlineto{\pgfqpoint{1.520767in}{2.164632in}}%
\pgfpathlineto{\pgfqpoint{1.536201in}{2.197957in}}%
\pgfpathlineto{\pgfqpoint{1.551634in}{2.228377in}}%
\pgfpathlineto{\pgfqpoint{1.567068in}{2.255562in}}%
\pgfpathlineto{\pgfqpoint{1.582502in}{2.279242in}}%
\pgfpathlineto{\pgfqpoint{1.597935in}{2.299204in}}%
\pgfpathlineto{\pgfqpoint{1.613369in}{2.315294in}}%
\pgfpathlineto{\pgfqpoint{1.628803in}{2.327418in}}%
\pgfpathlineto{\pgfqpoint{1.644236in}{2.335540in}}%
\pgfpathlineto{\pgfqpoint{1.659670in}{2.339679in}}%
\pgfpathlineto{\pgfqpoint{1.675104in}{2.339906in}}%
\pgfpathlineto{\pgfqpoint{1.690537in}{2.336338in}}%
\pgfpathlineto{\pgfqpoint{1.705971in}{2.329137in}}%
\pgfpathlineto{\pgfqpoint{1.721404in}{2.318502in}}%
\pgfpathlineto{\pgfqpoint{1.736838in}{2.304664in}}%
\pgfpathlineto{\pgfqpoint{1.752272in}{2.287882in}}%
\pgfpathlineto{\pgfqpoint{1.767705in}{2.268438in}}%
\pgfpathlineto{\pgfqpoint{1.783139in}{2.246629in}}%
\pgfpathlineto{\pgfqpoint{1.814006in}{2.197168in}}%
\pgfpathlineto{\pgfqpoint{1.844874in}{2.142060in}}%
\pgfpathlineto{\pgfqpoint{1.952909in}{1.941138in}}%
\pgfpathlineto{\pgfqpoint{1.983776in}{1.890936in}}%
\pgfpathlineto{\pgfqpoint{2.014644in}{1.847477in}}%
\pgfpathlineto{\pgfqpoint{2.030077in}{1.828729in}}%
\pgfpathlineto{\pgfqpoint{2.045511in}{1.812155in}}%
\pgfpathlineto{\pgfqpoint{2.060944in}{1.797860in}}%
\pgfpathlineto{\pgfqpoint{2.076378in}{1.785913in}}%
\pgfpathlineto{\pgfqpoint{2.091812in}{1.776347in}}%
\pgfpathlineto{\pgfqpoint{2.107245in}{1.769159in}}%
\pgfpathlineto{\pgfqpoint{2.122679in}{1.764305in}}%
\pgfpathlineto{\pgfqpoint{2.138113in}{1.761701in}}%
\pgfpathlineto{\pgfqpoint{2.153546in}{1.761222in}}%
\pgfpathlineto{\pgfqpoint{2.168980in}{1.762705in}}%
\pgfpathlineto{\pgfqpoint{2.184414in}{1.765947in}}%
\pgfpathlineto{\pgfqpoint{2.199847in}{1.770712in}}%
\pgfpathlineto{\pgfqpoint{2.230714in}{1.783711in}}%
\pgfpathlineto{\pgfqpoint{2.307883in}{1.821604in}}%
\pgfpathlineto{\pgfqpoint{2.323316in}{1.827467in}}%
\pgfpathlineto{\pgfqpoint{2.338750in}{1.832036in}}%
\pgfpathlineto{\pgfqpoint{2.354184in}{1.835055in}}%
\pgfpathlineto{\pgfqpoint{2.369617in}{1.836300in}}%
\pgfpathlineto{\pgfqpoint{2.385051in}{1.835591in}}%
\pgfpathlineto{\pgfqpoint{2.400484in}{1.832797in}}%
\pgfpathlineto{\pgfqpoint{2.415918in}{1.827834in}}%
\pgfpathlineto{\pgfqpoint{2.431352in}{1.820674in}}%
\pgfpathlineto{\pgfqpoint{2.446785in}{1.811339in}}%
\pgfpathlineto{\pgfqpoint{2.462219in}{1.799905in}}%
\pgfpathlineto{\pgfqpoint{2.477653in}{1.786498in}}%
\pgfpathlineto{\pgfqpoint{2.493086in}{1.771293in}}%
\pgfpathlineto{\pgfqpoint{2.523954in}{1.736391in}}%
\pgfpathlineto{\pgfqpoint{2.554821in}{1.697345in}}%
\pgfpathlineto{\pgfqpoint{2.616555in}{1.617127in}}%
\pgfpathlineto{\pgfqpoint{2.647423in}{1.581262in}}%
\pgfpathlineto{\pgfqpoint{2.662856in}{1.565458in}}%
\pgfpathlineto{\pgfqpoint{2.678290in}{1.551411in}}%
\pgfpathlineto{\pgfqpoint{2.693724in}{1.539335in}}%
\pgfpathlineto{\pgfqpoint{2.709157in}{1.529400in}}%
\pgfpathlineto{\pgfqpoint{2.724591in}{1.521736in}}%
\pgfpathlineto{\pgfqpoint{2.740024in}{1.516429in}}%
\pgfpathlineto{\pgfqpoint{2.755458in}{1.513518in}}%
\pgfpathlineto{\pgfqpoint{2.770892in}{1.512995in}}%
\pgfpathlineto{\pgfqpoint{2.786325in}{1.514808in}}%
\pgfpathlineto{\pgfqpoint{2.801759in}{1.518858in}}%
\pgfpathlineto{\pgfqpoint{2.817193in}{1.525005in}}%
\pgfpathlineto{\pgfqpoint{2.832626in}{1.533070in}}%
\pgfpathlineto{\pgfqpoint{2.848060in}{1.542837in}}%
\pgfpathlineto{\pgfqpoint{2.878927in}{1.566459in}}%
\pgfpathlineto{\pgfqpoint{2.925228in}{1.607711in}}%
\pgfpathlineto{\pgfqpoint{2.971529in}{1.648324in}}%
\pgfpathlineto{\pgfqpoint{3.002396in}{1.670928in}}%
\pgfpathlineto{\pgfqpoint{3.017830in}{1.680080in}}%
\pgfpathlineto{\pgfqpoint{3.033263in}{1.687508in}}%
\pgfpathlineto{\pgfqpoint{3.048697in}{1.693038in}}%
\pgfpathlineto{\pgfqpoint{3.064131in}{1.696542in}}%
\pgfpathlineto{\pgfqpoint{3.079564in}{1.697931in}}%
\pgfpathlineto{\pgfqpoint{3.094998in}{1.697166in}}%
\pgfpathlineto{\pgfqpoint{3.110432in}{1.694251in}}%
\pgfpathlineto{\pgfqpoint{3.125865in}{1.689238in}}%
\pgfpathlineto{\pgfqpoint{3.141299in}{1.682222in}}%
\pgfpathlineto{\pgfqpoint{3.156733in}{1.673338in}}%
\pgfpathlineto{\pgfqpoint{3.172166in}{1.662757in}}%
\pgfpathlineto{\pgfqpoint{3.203033in}{1.637352in}}%
\pgfpathlineto{\pgfqpoint{3.233901in}{1.607920in}}%
\pgfpathlineto{\pgfqpoint{3.311069in}{1.530641in}}%
\pgfpathlineto{\pgfqpoint{3.341936in}{1.503417in}}%
\pgfpathlineto{\pgfqpoint{3.372803in}{1.480584in}}%
\pgfpathlineto{\pgfqpoint{3.388237in}{1.471117in}}%
\pgfpathlineto{\pgfqpoint{3.403671in}{1.463052in}}%
\pgfpathlineto{\pgfqpoint{3.419104in}{1.456430in}}%
\pgfpathlineto{\pgfqpoint{3.434538in}{1.451267in}}%
\pgfpathlineto{\pgfqpoint{3.449972in}{1.447557in}}%
\pgfpathlineto{\pgfqpoint{3.465405in}{1.445275in}}%
\pgfpathlineto{\pgfqpoint{3.480839in}{1.444382in}}%
\pgfpathlineto{\pgfqpoint{3.496273in}{1.444828in}}%
\pgfpathlineto{\pgfqpoint{3.511706in}{1.446555in}}%
\pgfpathlineto{\pgfqpoint{3.527140in}{1.449505in}}%
\pgfpathlineto{\pgfqpoint{3.558007in}{1.458831in}}%
\pgfpathlineto{\pgfqpoint{3.588874in}{1.472370in}}%
\pgfpathlineto{\pgfqpoint{3.619742in}{1.489770in}}%
\pgfpathlineto{\pgfqpoint{3.650609in}{1.510771in}}%
\pgfpathlineto{\pgfqpoint{3.681476in}{1.535166in}}%
\pgfpathlineto{\pgfqpoint{3.712343in}{1.562732in}}%
\pgfpathlineto{\pgfqpoint{3.743211in}{1.593148in}}%
\pgfpathlineto{\pgfqpoint{3.789512in}{1.642995in}}%
\pgfpathlineto{\pgfqpoint{3.897547in}{1.763496in}}%
\pgfpathlineto{\pgfqpoint{3.928414in}{1.793767in}}%
\pgfpathlineto{\pgfqpoint{3.959282in}{1.819993in}}%
\pgfpathlineto{\pgfqpoint{3.990149in}{1.841242in}}%
\pgfpathlineto{\pgfqpoint{4.005583in}{1.849797in}}%
\pgfpathlineto{\pgfqpoint{4.021016in}{1.856915in}}%
\pgfpathlineto{\pgfqpoint{4.036450in}{1.862585in}}%
\pgfpathlineto{\pgfqpoint{4.051883in}{1.866825in}}%
\pgfpathlineto{\pgfqpoint{4.067317in}{1.869682in}}%
\pgfpathlineto{\pgfqpoint{4.082751in}{1.871233in}}%
\pgfpathlineto{\pgfqpoint{4.113618in}{1.870851in}}%
\pgfpathlineto{\pgfqpoint{4.144485in}{1.866788in}}%
\pgfpathlineto{\pgfqpoint{4.190786in}{1.856949in}}%
\pgfpathlineto{\pgfqpoint{4.237087in}{1.847480in}}%
\pgfpathlineto{\pgfqpoint{4.267954in}{1.843993in}}%
\pgfpathlineto{\pgfqpoint{4.298822in}{1.844276in}}%
\pgfpathlineto{\pgfqpoint{4.314255in}{1.846117in}}%
\pgfpathlineto{\pgfqpoint{4.329689in}{1.849185in}}%
\pgfpathlineto{\pgfqpoint{4.345122in}{1.853507in}}%
\pgfpathlineto{\pgfqpoint{4.360556in}{1.859080in}}%
\pgfpathlineto{\pgfqpoint{4.391423in}{1.873778in}}%
\pgfpathlineto{\pgfqpoint{4.422291in}{1.892559in}}%
\pgfpathlineto{\pgfqpoint{4.453158in}{1.914236in}}%
\pgfpathlineto{\pgfqpoint{4.530326in}{1.970646in}}%
\pgfpathlineto{\pgfqpoint{4.561193in}{1.989679in}}%
\pgfpathlineto{\pgfqpoint{4.592061in}{2.004540in}}%
\pgfpathlineto{\pgfqpoint{4.607494in}{2.010107in}}%
\pgfpathlineto{\pgfqpoint{4.622928in}{2.014335in}}%
\pgfpathlineto{\pgfqpoint{4.638362in}{2.017197in}}%
\pgfpathlineto{\pgfqpoint{4.653795in}{2.018698in}}%
\pgfpathlineto{\pgfqpoint{4.669229in}{2.018881in}}%
\pgfpathlineto{\pgfqpoint{4.669229in}{2.018881in}}%
\pgfusepath{stroke}%
\end{pgfscope}%
\begin{pgfscope}%
\pgfpathrectangle{\pgfqpoint{0.634105in}{0.521603in}}{\pgfqpoint{4.227273in}{2.800000in}} %
\pgfusepath{clip}%
\pgfsetrectcap%
\pgfsetroundjoin%
\pgfsetlinewidth{0.501875pt}%
\definecolor{currentstroke}{rgb}{0.943137,0.767363,0.423549}%
\pgfsetstrokecolor{currentstroke}%
\pgfsetdash{}{0pt}%
\pgfpathmoveto{\pgfqpoint{0.826254in}{2.294650in}}%
\pgfpathlineto{\pgfqpoint{0.841687in}{2.292536in}}%
\pgfpathlineto{\pgfqpoint{0.857121in}{2.286471in}}%
\pgfpathlineto{\pgfqpoint{0.872555in}{2.276359in}}%
\pgfpathlineto{\pgfqpoint{0.887988in}{2.262169in}}%
\pgfpathlineto{\pgfqpoint{0.903422in}{2.243941in}}%
\pgfpathlineto{\pgfqpoint{0.918855in}{2.221787in}}%
\pgfpathlineto{\pgfqpoint{0.934289in}{2.195888in}}%
\pgfpathlineto{\pgfqpoint{0.949723in}{2.166499in}}%
\pgfpathlineto{\pgfqpoint{0.965156in}{2.133939in}}%
\pgfpathlineto{\pgfqpoint{0.996024in}{2.060893in}}%
\pgfpathlineto{\pgfqpoint{1.026891in}{1.980445in}}%
\pgfpathlineto{\pgfqpoint{1.088625in}{1.815129in}}%
\pgfpathlineto{\pgfqpoint{1.119493in}{1.739740in}}%
\pgfpathlineto{\pgfqpoint{1.134926in}{1.705875in}}%
\pgfpathlineto{\pgfqpoint{1.150360in}{1.675255in}}%
\pgfpathlineto{\pgfqpoint{1.165794in}{1.648345in}}%
\pgfpathlineto{\pgfqpoint{1.181227in}{1.625553in}}%
\pgfpathlineto{\pgfqpoint{1.196661in}{1.607229in}}%
\pgfpathlineto{\pgfqpoint{1.212095in}{1.593656in}}%
\pgfpathlineto{\pgfqpoint{1.227528in}{1.585045in}}%
\pgfpathlineto{\pgfqpoint{1.242962in}{1.581533in}}%
\pgfpathlineto{\pgfqpoint{1.258395in}{1.583180in}}%
\pgfpathlineto{\pgfqpoint{1.273829in}{1.589970in}}%
\pgfpathlineto{\pgfqpoint{1.289263in}{1.601808in}}%
\pgfpathlineto{\pgfqpoint{1.304696in}{1.618525in}}%
\pgfpathlineto{\pgfqpoint{1.320130in}{1.639882in}}%
\pgfpathlineto{\pgfqpoint{1.335564in}{1.665569in}}%
\pgfpathlineto{\pgfqpoint{1.350997in}{1.695215in}}%
\pgfpathlineto{\pgfqpoint{1.366431in}{1.728394in}}%
\pgfpathlineto{\pgfqpoint{1.397298in}{1.803403in}}%
\pgfpathlineto{\pgfqpoint{1.428165in}{1.886349in}}%
\pgfpathlineto{\pgfqpoint{1.489900in}{2.057495in}}%
\pgfpathlineto{\pgfqpoint{1.520767in}{2.136530in}}%
\pgfpathlineto{\pgfqpoint{1.536201in}{2.172584in}}%
\pgfpathlineto{\pgfqpoint{1.551634in}{2.205733in}}%
\pgfpathlineto{\pgfqpoint{1.567068in}{2.235594in}}%
\pgfpathlineto{\pgfqpoint{1.582502in}{2.261837in}}%
\pgfpathlineto{\pgfqpoint{1.597935in}{2.284193in}}%
\pgfpathlineto{\pgfqpoint{1.613369in}{2.302454in}}%
\pgfpathlineto{\pgfqpoint{1.628803in}{2.316475in}}%
\pgfpathlineto{\pgfqpoint{1.644236in}{2.326174in}}%
\pgfpathlineto{\pgfqpoint{1.659670in}{2.331529in}}%
\pgfpathlineto{\pgfqpoint{1.675104in}{2.332581in}}%
\pgfpathlineto{\pgfqpoint{1.690537in}{2.329426in}}%
\pgfpathlineto{\pgfqpoint{1.705971in}{2.322214in}}%
\pgfpathlineto{\pgfqpoint{1.721404in}{2.311145in}}%
\pgfpathlineto{\pgfqpoint{1.736838in}{2.296461in}}%
\pgfpathlineto{\pgfqpoint{1.752272in}{2.278445in}}%
\pgfpathlineto{\pgfqpoint{1.767705in}{2.257413in}}%
\pgfpathlineto{\pgfqpoint{1.783139in}{2.233707in}}%
\pgfpathlineto{\pgfqpoint{1.814006in}{2.179752in}}%
\pgfpathlineto{\pgfqpoint{1.844874in}{2.119662in}}%
\pgfpathlineto{\pgfqpoint{1.937475in}{1.933778in}}%
\pgfpathlineto{\pgfqpoint{1.968343in}{1.879462in}}%
\pgfpathlineto{\pgfqpoint{1.983776in}{1.855081in}}%
\pgfpathlineto{\pgfqpoint{1.999210in}{1.832855in}}%
\pgfpathlineto{\pgfqpoint{2.014644in}{1.812965in}}%
\pgfpathlineto{\pgfqpoint{2.030077in}{1.795556in}}%
\pgfpathlineto{\pgfqpoint{2.045511in}{1.780733in}}%
\pgfpathlineto{\pgfqpoint{2.060944in}{1.768559in}}%
\pgfpathlineto{\pgfqpoint{2.076378in}{1.759053in}}%
\pgfpathlineto{\pgfqpoint{2.091812in}{1.752195in}}%
\pgfpathlineto{\pgfqpoint{2.107245in}{1.747916in}}%
\pgfpathlineto{\pgfqpoint{2.122679in}{1.746107in}}%
\pgfpathlineto{\pgfqpoint{2.138113in}{1.746617in}}%
\pgfpathlineto{\pgfqpoint{2.153546in}{1.749255in}}%
\pgfpathlineto{\pgfqpoint{2.168980in}{1.753790in}}%
\pgfpathlineto{\pgfqpoint{2.184414in}{1.759961in}}%
\pgfpathlineto{\pgfqpoint{2.215281in}{1.776019in}}%
\pgfpathlineto{\pgfqpoint{2.292449in}{1.822152in}}%
\pgfpathlineto{\pgfqpoint{2.307883in}{1.829652in}}%
\pgfpathlineto{\pgfqpoint{2.323316in}{1.835831in}}%
\pgfpathlineto{\pgfqpoint{2.338750in}{1.840419in}}%
\pgfpathlineto{\pgfqpoint{2.354184in}{1.843181in}}%
\pgfpathlineto{\pgfqpoint{2.369617in}{1.843926in}}%
\pgfpathlineto{\pgfqpoint{2.385051in}{1.842510in}}%
\pgfpathlineto{\pgfqpoint{2.400484in}{1.838837in}}%
\pgfpathlineto{\pgfqpoint{2.415918in}{1.832862in}}%
\pgfpathlineto{\pgfqpoint{2.431352in}{1.824597in}}%
\pgfpathlineto{\pgfqpoint{2.446785in}{1.814102in}}%
\pgfpathlineto{\pgfqpoint{2.462219in}{1.801493in}}%
\pgfpathlineto{\pgfqpoint{2.477653in}{1.786930in}}%
\pgfpathlineto{\pgfqpoint{2.508520in}{1.752812in}}%
\pgfpathlineto{\pgfqpoint{2.539387in}{1.713851in}}%
\pgfpathlineto{\pgfqpoint{2.616555in}{1.612416in}}%
\pgfpathlineto{\pgfqpoint{2.647423in}{1.577359in}}%
\pgfpathlineto{\pgfqpoint{2.662856in}{1.562243in}}%
\pgfpathlineto{\pgfqpoint{2.678290in}{1.549066in}}%
\pgfpathlineto{\pgfqpoint{2.693724in}{1.538026in}}%
\pgfpathlineto{\pgfqpoint{2.709157in}{1.529284in}}%
\pgfpathlineto{\pgfqpoint{2.724591in}{1.522951in}}%
\pgfpathlineto{\pgfqpoint{2.740024in}{1.519096in}}%
\pgfpathlineto{\pgfqpoint{2.755458in}{1.517738in}}%
\pgfpathlineto{\pgfqpoint{2.770892in}{1.518849in}}%
\pgfpathlineto{\pgfqpoint{2.786325in}{1.522354in}}%
\pgfpathlineto{\pgfqpoint{2.801759in}{1.528135in}}%
\pgfpathlineto{\pgfqpoint{2.817193in}{1.536028in}}%
\pgfpathlineto{\pgfqpoint{2.832626in}{1.545834in}}%
\pgfpathlineto{\pgfqpoint{2.848060in}{1.557316in}}%
\pgfpathlineto{\pgfqpoint{2.878927in}{1.584219in}}%
\pgfpathlineto{\pgfqpoint{2.925228in}{1.629784in}}%
\pgfpathlineto{\pgfqpoint{2.956095in}{1.659798in}}%
\pgfpathlineto{\pgfqpoint{2.986963in}{1.686508in}}%
\pgfpathlineto{\pgfqpoint{3.002396in}{1.697892in}}%
\pgfpathlineto{\pgfqpoint{3.017830in}{1.707624in}}%
\pgfpathlineto{\pgfqpoint{3.033263in}{1.715489in}}%
\pgfpathlineto{\pgfqpoint{3.048697in}{1.721312in}}%
\pgfpathlineto{\pgfqpoint{3.064131in}{1.724959in}}%
\pgfpathlineto{\pgfqpoint{3.079564in}{1.726337in}}%
\pgfpathlineto{\pgfqpoint{3.094998in}{1.725402in}}%
\pgfpathlineto{\pgfqpoint{3.110432in}{1.722152in}}%
\pgfpathlineto{\pgfqpoint{3.125865in}{1.716632in}}%
\pgfpathlineto{\pgfqpoint{3.141299in}{1.708928in}}%
\pgfpathlineto{\pgfqpoint{3.156733in}{1.699169in}}%
\pgfpathlineto{\pgfqpoint{3.172166in}{1.687523in}}%
\pgfpathlineto{\pgfqpoint{3.203033in}{1.659389in}}%
\pgfpathlineto{\pgfqpoint{3.233901in}{1.626429in}}%
\pgfpathlineto{\pgfqpoint{3.341936in}{1.504624in}}%
\pgfpathlineto{\pgfqpoint{3.372803in}{1.476343in}}%
\pgfpathlineto{\pgfqpoint{3.388237in}{1.464271in}}%
\pgfpathlineto{\pgfqpoint{3.403671in}{1.453743in}}%
\pgfpathlineto{\pgfqpoint{3.419104in}{1.444849in}}%
\pgfpathlineto{\pgfqpoint{3.434538in}{1.437648in}}%
\pgfpathlineto{\pgfqpoint{3.449972in}{1.432176in}}%
\pgfpathlineto{\pgfqpoint{3.465405in}{1.428444in}}%
\pgfpathlineto{\pgfqpoint{3.480839in}{1.426442in}}%
\pgfpathlineto{\pgfqpoint{3.496273in}{1.426142in}}%
\pgfpathlineto{\pgfqpoint{3.511706in}{1.427500in}}%
\pgfpathlineto{\pgfqpoint{3.527140in}{1.430460in}}%
\pgfpathlineto{\pgfqpoint{3.542573in}{1.434958in}}%
\pgfpathlineto{\pgfqpoint{3.558007in}{1.440921in}}%
\pgfpathlineto{\pgfqpoint{3.573441in}{1.448271in}}%
\pgfpathlineto{\pgfqpoint{3.604308in}{1.466816in}}%
\pgfpathlineto{\pgfqpoint{3.635175in}{1.489940in}}%
\pgfpathlineto{\pgfqpoint{3.666043in}{1.516988in}}%
\pgfpathlineto{\pgfqpoint{3.696910in}{1.547294in}}%
\pgfpathlineto{\pgfqpoint{3.743211in}{1.597326in}}%
\pgfpathlineto{\pgfqpoint{3.882113in}{1.753758in}}%
\pgfpathlineto{\pgfqpoint{3.912981in}{1.783560in}}%
\pgfpathlineto{\pgfqpoint{3.943848in}{1.809461in}}%
\pgfpathlineto{\pgfqpoint{3.974715in}{1.830724in}}%
\pgfpathlineto{\pgfqpoint{3.990149in}{1.839452in}}%
\pgfpathlineto{\pgfqpoint{4.005583in}{1.846860in}}%
\pgfpathlineto{\pgfqpoint{4.021016in}{1.852936in}}%
\pgfpathlineto{\pgfqpoint{4.036450in}{1.857689in}}%
\pgfpathlineto{\pgfqpoint{4.051883in}{1.861150in}}%
\pgfpathlineto{\pgfqpoint{4.067317in}{1.863373in}}%
\pgfpathlineto{\pgfqpoint{4.098184in}{1.864424in}}%
\pgfpathlineto{\pgfqpoint{4.129052in}{1.861679in}}%
\pgfpathlineto{\pgfqpoint{4.159919in}{1.856244in}}%
\pgfpathlineto{\pgfqpoint{4.252521in}{1.837040in}}%
\pgfpathlineto{\pgfqpoint{4.283388in}{1.834048in}}%
\pgfpathlineto{\pgfqpoint{4.314255in}{1.834529in}}%
\pgfpathlineto{\pgfqpoint{4.345122in}{1.839082in}}%
\pgfpathlineto{\pgfqpoint{4.375990in}{1.847899in}}%
\pgfpathlineto{\pgfqpoint{4.406857in}{1.860744in}}%
\pgfpathlineto{\pgfqpoint{4.437724in}{1.876967in}}%
\pgfpathlineto{\pgfqpoint{4.484025in}{1.905369in}}%
\pgfpathlineto{\pgfqpoint{4.545760in}{1.943913in}}%
\pgfpathlineto{\pgfqpoint{4.576627in}{1.960403in}}%
\pgfpathlineto{\pgfqpoint{4.607494in}{1.973481in}}%
\pgfpathlineto{\pgfqpoint{4.638362in}{1.982349in}}%
\pgfpathlineto{\pgfqpoint{4.653795in}{1.985059in}}%
\pgfpathlineto{\pgfqpoint{4.669229in}{1.986593in}}%
\pgfpathlineto{\pgfqpoint{4.669229in}{1.986593in}}%
\pgfusepath{stroke}%
\end{pgfscope}%
\begin{pgfscope}%
\pgfpathrectangle{\pgfqpoint{0.634105in}{0.521603in}}{\pgfqpoint{4.227273in}{2.800000in}} %
\pgfusepath{clip}%
\pgfsetrectcap%
\pgfsetroundjoin%
\pgfsetlinewidth{0.501875pt}%
\definecolor{currentstroke}{rgb}{1.000000,0.682749,0.366979}%
\pgfsetstrokecolor{currentstroke}%
\pgfsetdash{}{0pt}%
\pgfpathmoveto{\pgfqpoint{0.826254in}{2.296092in}}%
\pgfpathlineto{\pgfqpoint{0.841687in}{2.293237in}}%
\pgfpathlineto{\pgfqpoint{0.857121in}{2.286373in}}%
\pgfpathlineto{\pgfqpoint{0.872555in}{2.275395in}}%
\pgfpathlineto{\pgfqpoint{0.887988in}{2.260270in}}%
\pgfpathlineto{\pgfqpoint{0.903422in}{2.241038in}}%
\pgfpathlineto{\pgfqpoint{0.918855in}{2.217811in}}%
\pgfpathlineto{\pgfqpoint{0.934289in}{2.190777in}}%
\pgfpathlineto{\pgfqpoint{0.949723in}{2.160196in}}%
\pgfpathlineto{\pgfqpoint{0.965156in}{2.126399in}}%
\pgfpathlineto{\pgfqpoint{0.996024in}{2.050801in}}%
\pgfpathlineto{\pgfqpoint{1.026891in}{1.967821in}}%
\pgfpathlineto{\pgfqpoint{1.088625in}{1.798216in}}%
\pgfpathlineto{\pgfqpoint{1.119493in}{1.721436in}}%
\pgfpathlineto{\pgfqpoint{1.134926in}{1.687132in}}%
\pgfpathlineto{\pgfqpoint{1.150360in}{1.656262in}}%
\pgfpathlineto{\pgfqpoint{1.165794in}{1.629295in}}%
\pgfpathlineto{\pgfqpoint{1.181227in}{1.606644in}}%
\pgfpathlineto{\pgfqpoint{1.196661in}{1.588654in}}%
\pgfpathlineto{\pgfqpoint{1.212095in}{1.575597in}}%
\pgfpathlineto{\pgfqpoint{1.227528in}{1.567669in}}%
\pgfpathlineto{\pgfqpoint{1.242962in}{1.564985in}}%
\pgfpathlineto{\pgfqpoint{1.258395in}{1.567579in}}%
\pgfpathlineto{\pgfqpoint{1.273829in}{1.575404in}}%
\pgfpathlineto{\pgfqpoint{1.289263in}{1.588333in}}%
\pgfpathlineto{\pgfqpoint{1.304696in}{1.606160in}}%
\pgfpathlineto{\pgfqpoint{1.320130in}{1.628609in}}%
\pgfpathlineto{\pgfqpoint{1.335564in}{1.655335in}}%
\pgfpathlineto{\pgfqpoint{1.350997in}{1.685935in}}%
\pgfpathlineto{\pgfqpoint{1.366431in}{1.719949in}}%
\pgfpathlineto{\pgfqpoint{1.397298in}{1.796182in}}%
\pgfpathlineto{\pgfqpoint{1.443599in}{1.922665in}}%
\pgfpathlineto{\pgfqpoint{1.489900in}{2.049853in}}%
\pgfpathlineto{\pgfqpoint{1.520767in}{2.127737in}}%
\pgfpathlineto{\pgfqpoint{1.536201in}{2.163167in}}%
\pgfpathlineto{\pgfqpoint{1.551634in}{2.195716in}}%
\pgfpathlineto{\pgfqpoint{1.567068in}{2.225043in}}%
\pgfpathlineto{\pgfqpoint{1.582502in}{2.250861in}}%
\pgfpathlineto{\pgfqpoint{1.597935in}{2.272943in}}%
\pgfpathlineto{\pgfqpoint{1.613369in}{2.291117in}}%
\pgfpathlineto{\pgfqpoint{1.628803in}{2.305268in}}%
\pgfpathlineto{\pgfqpoint{1.644236in}{2.315339in}}%
\pgfpathlineto{\pgfqpoint{1.659670in}{2.321329in}}%
\pgfpathlineto{\pgfqpoint{1.675104in}{2.323287in}}%
\pgfpathlineto{\pgfqpoint{1.690537in}{2.321312in}}%
\pgfpathlineto{\pgfqpoint{1.705971in}{2.315548in}}%
\pgfpathlineto{\pgfqpoint{1.721404in}{2.306180in}}%
\pgfpathlineto{\pgfqpoint{1.736838in}{2.293429in}}%
\pgfpathlineto{\pgfqpoint{1.752272in}{2.277548in}}%
\pgfpathlineto{\pgfqpoint{1.767705in}{2.258818in}}%
\pgfpathlineto{\pgfqpoint{1.783139in}{2.237541in}}%
\pgfpathlineto{\pgfqpoint{1.814006in}{2.188641in}}%
\pgfpathlineto{\pgfqpoint{1.844874in}{2.133545in}}%
\pgfpathlineto{\pgfqpoint{1.952909in}{1.931322in}}%
\pgfpathlineto{\pgfqpoint{1.983776in}{1.881144in}}%
\pgfpathlineto{\pgfqpoint{2.014644in}{1.838046in}}%
\pgfpathlineto{\pgfqpoint{2.030077in}{1.819593in}}%
\pgfpathlineto{\pgfqpoint{2.045511in}{1.803374in}}%
\pgfpathlineto{\pgfqpoint{2.060944in}{1.789474in}}%
\pgfpathlineto{\pgfqpoint{2.076378in}{1.777942in}}%
\pgfpathlineto{\pgfqpoint{2.091812in}{1.768789in}}%
\pgfpathlineto{\pgfqpoint{2.107245in}{1.761990in}}%
\pgfpathlineto{\pgfqpoint{2.122679in}{1.757478in}}%
\pgfpathlineto{\pgfqpoint{2.138113in}{1.755152in}}%
\pgfpathlineto{\pgfqpoint{2.153546in}{1.754873in}}%
\pgfpathlineto{\pgfqpoint{2.168980in}{1.756465in}}%
\pgfpathlineto{\pgfqpoint{2.184414in}{1.759723in}}%
\pgfpathlineto{\pgfqpoint{2.199847in}{1.764410in}}%
\pgfpathlineto{\pgfqpoint{2.230714in}{1.777005in}}%
\pgfpathlineto{\pgfqpoint{2.307883in}{1.813334in}}%
\pgfpathlineto{\pgfqpoint{2.323316in}{1.818993in}}%
\pgfpathlineto{\pgfqpoint{2.338750in}{1.823459in}}%
\pgfpathlineto{\pgfqpoint{2.354184in}{1.826496in}}%
\pgfpathlineto{\pgfqpoint{2.369617in}{1.827901in}}%
\pgfpathlineto{\pgfqpoint{2.385051in}{1.827509in}}%
\pgfpathlineto{\pgfqpoint{2.400484in}{1.825197in}}%
\pgfpathlineto{\pgfqpoint{2.415918in}{1.820889in}}%
\pgfpathlineto{\pgfqpoint{2.431352in}{1.814553in}}%
\pgfpathlineto{\pgfqpoint{2.446785in}{1.806210in}}%
\pgfpathlineto{\pgfqpoint{2.462219in}{1.795925in}}%
\pgfpathlineto{\pgfqpoint{2.477653in}{1.783814in}}%
\pgfpathlineto{\pgfqpoint{2.508520in}{1.754792in}}%
\pgfpathlineto{\pgfqpoint{2.539387in}{1.720892in}}%
\pgfpathlineto{\pgfqpoint{2.631989in}{1.613757in}}%
\pgfpathlineto{\pgfqpoint{2.662856in}{1.584763in}}%
\pgfpathlineto{\pgfqpoint{2.678290in}{1.572820in}}%
\pgfpathlineto{\pgfqpoint{2.693724in}{1.562871in}}%
\pgfpathlineto{\pgfqpoint{2.709157in}{1.555082in}}%
\pgfpathlineto{\pgfqpoint{2.724591in}{1.549576in}}%
\pgfpathlineto{\pgfqpoint{2.740024in}{1.546427in}}%
\pgfpathlineto{\pgfqpoint{2.755458in}{1.545661in}}%
\pgfpathlineto{\pgfqpoint{2.770892in}{1.547252in}}%
\pgfpathlineto{\pgfqpoint{2.786325in}{1.551126in}}%
\pgfpathlineto{\pgfqpoint{2.801759in}{1.557160in}}%
\pgfpathlineto{\pgfqpoint{2.817193in}{1.565188in}}%
\pgfpathlineto{\pgfqpoint{2.832626in}{1.575002in}}%
\pgfpathlineto{\pgfqpoint{2.863493in}{1.598971in}}%
\pgfpathlineto{\pgfqpoint{2.894361in}{1.626769in}}%
\pgfpathlineto{\pgfqpoint{2.956095in}{1.683405in}}%
\pgfpathlineto{\pgfqpoint{2.986963in}{1.707066in}}%
\pgfpathlineto{\pgfqpoint{3.002396in}{1.716729in}}%
\pgfpathlineto{\pgfqpoint{3.017830in}{1.724639in}}%
\pgfpathlineto{\pgfqpoint{3.033263in}{1.730614in}}%
\pgfpathlineto{\pgfqpoint{3.048697in}{1.734511in}}%
\pgfpathlineto{\pgfqpoint{3.064131in}{1.736232in}}%
\pgfpathlineto{\pgfqpoint{3.079564in}{1.735724in}}%
\pgfpathlineto{\pgfqpoint{3.094998in}{1.732979in}}%
\pgfpathlineto{\pgfqpoint{3.110432in}{1.728034in}}%
\pgfpathlineto{\pgfqpoint{3.125865in}{1.720969in}}%
\pgfpathlineto{\pgfqpoint{3.141299in}{1.711904in}}%
\pgfpathlineto{\pgfqpoint{3.156733in}{1.700995in}}%
\pgfpathlineto{\pgfqpoint{3.172166in}{1.688431in}}%
\pgfpathlineto{\pgfqpoint{3.203033in}{1.659221in}}%
\pgfpathlineto{\pgfqpoint{3.233901in}{1.626228in}}%
\pgfpathlineto{\pgfqpoint{3.311069in}{1.540963in}}%
\pgfpathlineto{\pgfqpoint{3.341936in}{1.510818in}}%
\pgfpathlineto{\pgfqpoint{3.372803in}{1.485220in}}%
\pgfpathlineto{\pgfqpoint{3.388237in}{1.474447in}}%
\pgfpathlineto{\pgfqpoint{3.403671in}{1.465141in}}%
\pgfpathlineto{\pgfqpoint{3.419104in}{1.457358in}}%
\pgfpathlineto{\pgfqpoint{3.434538in}{1.451130in}}%
\pgfpathlineto{\pgfqpoint{3.449972in}{1.446466in}}%
\pgfpathlineto{\pgfqpoint{3.465405in}{1.443358in}}%
\pgfpathlineto{\pgfqpoint{3.480839in}{1.441781in}}%
\pgfpathlineto{\pgfqpoint{3.496273in}{1.441697in}}%
\pgfpathlineto{\pgfqpoint{3.511706in}{1.443058in}}%
\pgfpathlineto{\pgfqpoint{3.527140in}{1.445812in}}%
\pgfpathlineto{\pgfqpoint{3.542573in}{1.449899in}}%
\pgfpathlineto{\pgfqpoint{3.558007in}{1.455262in}}%
\pgfpathlineto{\pgfqpoint{3.588874in}{1.469580in}}%
\pgfpathlineto{\pgfqpoint{3.619742in}{1.488327in}}%
\pgfpathlineto{\pgfqpoint{3.650609in}{1.511114in}}%
\pgfpathlineto{\pgfqpoint{3.681476in}{1.537584in}}%
\pgfpathlineto{\pgfqpoint{3.712343in}{1.567374in}}%
\pgfpathlineto{\pgfqpoint{3.743211in}{1.600052in}}%
\pgfpathlineto{\pgfqpoint{3.789512in}{1.653211in}}%
\pgfpathlineto{\pgfqpoint{3.897547in}{1.780653in}}%
\pgfpathlineto{\pgfqpoint{3.928414in}{1.812572in}}%
\pgfpathlineto{\pgfqpoint{3.959282in}{1.840139in}}%
\pgfpathlineto{\pgfqpoint{3.974715in}{1.851941in}}%
\pgfpathlineto{\pgfqpoint{3.990149in}{1.862271in}}%
\pgfpathlineto{\pgfqpoint{4.005583in}{1.871042in}}%
\pgfpathlineto{\pgfqpoint{4.021016in}{1.878194in}}%
\pgfpathlineto{\pgfqpoint{4.036450in}{1.883695in}}%
\pgfpathlineto{\pgfqpoint{4.051883in}{1.887545in}}%
\pgfpathlineto{\pgfqpoint{4.067317in}{1.889775in}}%
\pgfpathlineto{\pgfqpoint{4.082751in}{1.890447in}}%
\pgfpathlineto{\pgfqpoint{4.098184in}{1.889656in}}%
\pgfpathlineto{\pgfqpoint{4.113618in}{1.887526in}}%
\pgfpathlineto{\pgfqpoint{4.144485in}{1.879886in}}%
\pgfpathlineto{\pgfqpoint{4.175352in}{1.869027in}}%
\pgfpathlineto{\pgfqpoint{4.252521in}{1.839640in}}%
\pgfpathlineto{\pgfqpoint{4.283388in}{1.831741in}}%
\pgfpathlineto{\pgfqpoint{4.298822in}{1.829393in}}%
\pgfpathlineto{\pgfqpoint{4.314255in}{1.828301in}}%
\pgfpathlineto{\pgfqpoint{4.329689in}{1.828566in}}%
\pgfpathlineto{\pgfqpoint{4.345122in}{1.830251in}}%
\pgfpathlineto{\pgfqpoint{4.360556in}{1.833388in}}%
\pgfpathlineto{\pgfqpoint{4.375990in}{1.837966in}}%
\pgfpathlineto{\pgfqpoint{4.391423in}{1.843941in}}%
\pgfpathlineto{\pgfqpoint{4.422291in}{1.859718in}}%
\pgfpathlineto{\pgfqpoint{4.453158in}{1.879679in}}%
\pgfpathlineto{\pgfqpoint{4.499459in}{1.914172in}}%
\pgfpathlineto{\pgfqpoint{4.545760in}{1.948770in}}%
\pgfpathlineto{\pgfqpoint{4.576627in}{1.968901in}}%
\pgfpathlineto{\pgfqpoint{4.607494in}{1.984890in}}%
\pgfpathlineto{\pgfqpoint{4.622928in}{1.990966in}}%
\pgfpathlineto{\pgfqpoint{4.638362in}{1.995628in}}%
\pgfpathlineto{\pgfqpoint{4.653795in}{1.998816in}}%
\pgfpathlineto{\pgfqpoint{4.669229in}{2.000502in}}%
\pgfpathlineto{\pgfqpoint{4.669229in}{2.000502in}}%
\pgfusepath{stroke}%
\end{pgfscope}%
\begin{pgfscope}%
\pgfpathrectangle{\pgfqpoint{0.634105in}{0.521603in}}{\pgfqpoint{4.227273in}{2.800000in}} %
\pgfusepath{clip}%
\pgfsetrectcap%
\pgfsetroundjoin%
\pgfsetlinewidth{0.501875pt}%
\definecolor{currentstroke}{rgb}{1.000000,0.587785,0.309017}%
\pgfsetstrokecolor{currentstroke}%
\pgfsetdash{}{0pt}%
\pgfpathmoveto{\pgfqpoint{0.826254in}{2.303555in}}%
\pgfpathlineto{\pgfqpoint{0.841687in}{2.300806in}}%
\pgfpathlineto{\pgfqpoint{0.857121in}{2.294160in}}%
\pgfpathlineto{\pgfqpoint{0.872555in}{2.283514in}}%
\pgfpathlineto{\pgfqpoint{0.887988in}{2.268826in}}%
\pgfpathlineto{\pgfqpoint{0.903422in}{2.250122in}}%
\pgfpathlineto{\pgfqpoint{0.918855in}{2.227495in}}%
\pgfpathlineto{\pgfqpoint{0.934289in}{2.201106in}}%
\pgfpathlineto{\pgfqpoint{0.949723in}{2.171183in}}%
\pgfpathlineto{\pgfqpoint{0.965156in}{2.138021in}}%
\pgfpathlineto{\pgfqpoint{0.996024in}{2.063461in}}%
\pgfpathlineto{\pgfqpoint{1.026891in}{1.980963in}}%
\pgfpathlineto{\pgfqpoint{1.104059in}{1.769500in}}%
\pgfpathlineto{\pgfqpoint{1.134926in}{1.695854in}}%
\pgfpathlineto{\pgfqpoint{1.150360in}{1.663805in}}%
\pgfpathlineto{\pgfqpoint{1.165794in}{1.635654in}}%
\pgfpathlineto{\pgfqpoint{1.181227in}{1.611864in}}%
\pgfpathlineto{\pgfqpoint{1.196661in}{1.592832in}}%
\pgfpathlineto{\pgfqpoint{1.212095in}{1.578873in}}%
\pgfpathlineto{\pgfqpoint{1.227528in}{1.570217in}}%
\pgfpathlineto{\pgfqpoint{1.242962in}{1.567002in}}%
\pgfpathlineto{\pgfqpoint{1.258395in}{1.569272in}}%
\pgfpathlineto{\pgfqpoint{1.273829in}{1.576975in}}%
\pgfpathlineto{\pgfqpoint{1.289263in}{1.589968in}}%
\pgfpathlineto{\pgfqpoint{1.304696in}{1.608020in}}%
\pgfpathlineto{\pgfqpoint{1.320130in}{1.630818in}}%
\pgfpathlineto{\pgfqpoint{1.335564in}{1.657978in}}%
\pgfpathlineto{\pgfqpoint{1.350997in}{1.689053in}}%
\pgfpathlineto{\pgfqpoint{1.366431in}{1.723546in}}%
\pgfpathlineto{\pgfqpoint{1.397298in}{1.800614in}}%
\pgfpathlineto{\pgfqpoint{1.443599in}{1.927788in}}%
\pgfpathlineto{\pgfqpoint{1.489900in}{2.055133in}}%
\pgfpathlineto{\pgfqpoint{1.520767in}{2.133140in}}%
\pgfpathlineto{\pgfqpoint{1.536201in}{2.168716in}}%
\pgfpathlineto{\pgfqpoint{1.551634in}{2.201492in}}%
\pgfpathlineto{\pgfqpoint{1.567068in}{2.231133in}}%
\pgfpathlineto{\pgfqpoint{1.582502in}{2.257352in}}%
\pgfpathlineto{\pgfqpoint{1.597935in}{2.279903in}}%
\pgfpathlineto{\pgfqpoint{1.613369in}{2.298594in}}%
\pgfpathlineto{\pgfqpoint{1.628803in}{2.313280in}}%
\pgfpathlineto{\pgfqpoint{1.644236in}{2.323868in}}%
\pgfpathlineto{\pgfqpoint{1.659670in}{2.330318in}}%
\pgfpathlineto{\pgfqpoint{1.675104in}{2.332645in}}%
\pgfpathlineto{\pgfqpoint{1.690537in}{2.330918in}}%
\pgfpathlineto{\pgfqpoint{1.705971in}{2.325259in}}%
\pgfpathlineto{\pgfqpoint{1.721404in}{2.315843in}}%
\pgfpathlineto{\pgfqpoint{1.736838in}{2.302893in}}%
\pgfpathlineto{\pgfqpoint{1.752272in}{2.286678in}}%
\pgfpathlineto{\pgfqpoint{1.767705in}{2.267506in}}%
\pgfpathlineto{\pgfqpoint{1.783139in}{2.245716in}}%
\pgfpathlineto{\pgfqpoint{1.814006in}{2.195767in}}%
\pgfpathlineto{\pgfqpoint{1.844874in}{2.139916in}}%
\pgfpathlineto{\pgfqpoint{1.937475in}{1.966695in}}%
\pgfpathlineto{\pgfqpoint{1.968343in}{1.915290in}}%
\pgfpathlineto{\pgfqpoint{1.999210in}{1.870026in}}%
\pgfpathlineto{\pgfqpoint{2.014644in}{1.850062in}}%
\pgfpathlineto{\pgfqpoint{2.030077in}{1.832024in}}%
\pgfpathlineto{\pgfqpoint{2.045511in}{1.815997in}}%
\pgfpathlineto{\pgfqpoint{2.060944in}{1.802044in}}%
\pgfpathlineto{\pgfqpoint{2.076378in}{1.790209in}}%
\pgfpathlineto{\pgfqpoint{2.091812in}{1.780516in}}%
\pgfpathlineto{\pgfqpoint{2.107245in}{1.772964in}}%
\pgfpathlineto{\pgfqpoint{2.122679in}{1.767528in}}%
\pgfpathlineto{\pgfqpoint{2.138113in}{1.764155in}}%
\pgfpathlineto{\pgfqpoint{2.153546in}{1.762760in}}%
\pgfpathlineto{\pgfqpoint{2.168980in}{1.763225in}}%
\pgfpathlineto{\pgfqpoint{2.184414in}{1.765395in}}%
\pgfpathlineto{\pgfqpoint{2.199847in}{1.769081in}}%
\pgfpathlineto{\pgfqpoint{2.230714in}{1.780063in}}%
\pgfpathlineto{\pgfqpoint{2.277015in}{1.801270in}}%
\pgfpathlineto{\pgfqpoint{2.307883in}{1.814800in}}%
\pgfpathlineto{\pgfqpoint{2.338750in}{1.824813in}}%
\pgfpathlineto{\pgfqpoint{2.354184in}{1.827784in}}%
\pgfpathlineto{\pgfqpoint{2.369617in}{1.829087in}}%
\pgfpathlineto{\pgfqpoint{2.385051in}{1.828550in}}%
\pgfpathlineto{\pgfqpoint{2.400484in}{1.826052in}}%
\pgfpathlineto{\pgfqpoint{2.415918in}{1.821534in}}%
\pgfpathlineto{\pgfqpoint{2.431352in}{1.814991in}}%
\pgfpathlineto{\pgfqpoint{2.446785in}{1.806478in}}%
\pgfpathlineto{\pgfqpoint{2.462219in}{1.796104in}}%
\pgfpathlineto{\pgfqpoint{2.477653in}{1.784027in}}%
\pgfpathlineto{\pgfqpoint{2.508520in}{1.755609in}}%
\pgfpathlineto{\pgfqpoint{2.539387in}{1.723219in}}%
\pgfpathlineto{\pgfqpoint{2.601122in}{1.655828in}}%
\pgfpathlineto{\pgfqpoint{2.631989in}{1.625454in}}%
\pgfpathlineto{\pgfqpoint{2.662856in}{1.600009in}}%
\pgfpathlineto{\pgfqpoint{2.678290in}{1.589649in}}%
\pgfpathlineto{\pgfqpoint{2.693724in}{1.581084in}}%
\pgfpathlineto{\pgfqpoint{2.709157in}{1.574442in}}%
\pgfpathlineto{\pgfqpoint{2.724591in}{1.569820in}}%
\pgfpathlineto{\pgfqpoint{2.740024in}{1.567279in}}%
\pgfpathlineto{\pgfqpoint{2.755458in}{1.566846in}}%
\pgfpathlineto{\pgfqpoint{2.770892in}{1.568511in}}%
\pgfpathlineto{\pgfqpoint{2.786325in}{1.572222in}}%
\pgfpathlineto{\pgfqpoint{2.801759in}{1.577887in}}%
\pgfpathlineto{\pgfqpoint{2.817193in}{1.585375in}}%
\pgfpathlineto{\pgfqpoint{2.832626in}{1.594510in}}%
\pgfpathlineto{\pgfqpoint{2.863493in}{1.616834in}}%
\pgfpathlineto{\pgfqpoint{2.894361in}{1.642735in}}%
\pgfpathlineto{\pgfqpoint{2.940662in}{1.682672in}}%
\pgfpathlineto{\pgfqpoint{2.971529in}{1.706066in}}%
\pgfpathlineto{\pgfqpoint{2.986963in}{1.715842in}}%
\pgfpathlineto{\pgfqpoint{3.002396in}{1.723976in}}%
\pgfpathlineto{\pgfqpoint{3.017830in}{1.730250in}}%
\pgfpathlineto{\pgfqpoint{3.033263in}{1.734492in}}%
\pgfpathlineto{\pgfqpoint{3.048697in}{1.736581in}}%
\pgfpathlineto{\pgfqpoint{3.064131in}{1.736452in}}%
\pgfpathlineto{\pgfqpoint{3.079564in}{1.734093in}}%
\pgfpathlineto{\pgfqpoint{3.094998in}{1.729545in}}%
\pgfpathlineto{\pgfqpoint{3.110432in}{1.722896in}}%
\pgfpathlineto{\pgfqpoint{3.125865in}{1.714280in}}%
\pgfpathlineto{\pgfqpoint{3.141299in}{1.703867in}}%
\pgfpathlineto{\pgfqpoint{3.172166in}{1.678484in}}%
\pgfpathlineto{\pgfqpoint{3.203033in}{1.648594in}}%
\pgfpathlineto{\pgfqpoint{3.311069in}{1.536511in}}%
\pgfpathlineto{\pgfqpoint{3.341936in}{1.509279in}}%
\pgfpathlineto{\pgfqpoint{3.372803in}{1.486460in}}%
\pgfpathlineto{\pgfqpoint{3.388237in}{1.476958in}}%
\pgfpathlineto{\pgfqpoint{3.403671in}{1.468825in}}%
\pgfpathlineto{\pgfqpoint{3.419104in}{1.462114in}}%
\pgfpathlineto{\pgfqpoint{3.434538in}{1.456859in}}%
\pgfpathlineto{\pgfqpoint{3.449972in}{1.453077in}}%
\pgfpathlineto{\pgfqpoint{3.465405in}{1.450770in}}%
\pgfpathlineto{\pgfqpoint{3.480839in}{1.449925in}}%
\pgfpathlineto{\pgfqpoint{3.496273in}{1.450514in}}%
\pgfpathlineto{\pgfqpoint{3.511706in}{1.452499in}}%
\pgfpathlineto{\pgfqpoint{3.527140in}{1.455830in}}%
\pgfpathlineto{\pgfqpoint{3.542573in}{1.460450in}}%
\pgfpathlineto{\pgfqpoint{3.573441in}{1.473315in}}%
\pgfpathlineto{\pgfqpoint{3.604308in}{1.490601in}}%
\pgfpathlineto{\pgfqpoint{3.635175in}{1.511862in}}%
\pgfpathlineto{\pgfqpoint{3.666043in}{1.536745in}}%
\pgfpathlineto{\pgfqpoint{3.696910in}{1.564988in}}%
\pgfpathlineto{\pgfqpoint{3.727777in}{1.596360in}}%
\pgfpathlineto{\pgfqpoint{3.758644in}{1.630555in}}%
\pgfpathlineto{\pgfqpoint{3.804945in}{1.685973in}}%
\pgfpathlineto{\pgfqpoint{3.882113in}{1.781064in}}%
\pgfpathlineto{\pgfqpoint{3.912981in}{1.815921in}}%
\pgfpathlineto{\pgfqpoint{3.943848in}{1.846562in}}%
\pgfpathlineto{\pgfqpoint{3.959282in}{1.859845in}}%
\pgfpathlineto{\pgfqpoint{3.974715in}{1.871568in}}%
\pgfpathlineto{\pgfqpoint{3.990149in}{1.881613in}}%
\pgfpathlineto{\pgfqpoint{4.005583in}{1.889896in}}%
\pgfpathlineto{\pgfqpoint{4.021016in}{1.896367in}}%
\pgfpathlineto{\pgfqpoint{4.036450in}{1.901012in}}%
\pgfpathlineto{\pgfqpoint{4.051883in}{1.903853in}}%
\pgfpathlineto{\pgfqpoint{4.067317in}{1.904947in}}%
\pgfpathlineto{\pgfqpoint{4.082751in}{1.904386in}}%
\pgfpathlineto{\pgfqpoint{4.098184in}{1.902291in}}%
\pgfpathlineto{\pgfqpoint{4.113618in}{1.898811in}}%
\pgfpathlineto{\pgfqpoint{4.144485in}{1.888404in}}%
\pgfpathlineto{\pgfqpoint{4.175352in}{1.874764in}}%
\pgfpathlineto{\pgfqpoint{4.252521in}{1.838448in}}%
\pgfpathlineto{\pgfqpoint{4.283388in}{1.827829in}}%
\pgfpathlineto{\pgfqpoint{4.298822in}{1.824180in}}%
\pgfpathlineto{\pgfqpoint{4.314255in}{1.821860in}}%
\pgfpathlineto{\pgfqpoint{4.329689in}{1.820998in}}%
\pgfpathlineto{\pgfqpoint{4.345122in}{1.821695in}}%
\pgfpathlineto{\pgfqpoint{4.360556in}{1.824013in}}%
\pgfpathlineto{\pgfqpoint{4.375990in}{1.827975in}}%
\pgfpathlineto{\pgfqpoint{4.391423in}{1.833564in}}%
\pgfpathlineto{\pgfqpoint{4.406857in}{1.840717in}}%
\pgfpathlineto{\pgfqpoint{4.422291in}{1.849329in}}%
\pgfpathlineto{\pgfqpoint{4.453158in}{1.870305in}}%
\pgfpathlineto{\pgfqpoint{4.484025in}{1.894893in}}%
\pgfpathlineto{\pgfqpoint{4.545760in}{1.946642in}}%
\pgfpathlineto{\pgfqpoint{4.576627in}{1.969546in}}%
\pgfpathlineto{\pgfqpoint{4.607494in}{1.988104in}}%
\pgfpathlineto{\pgfqpoint{4.622928in}{1.995387in}}%
\pgfpathlineto{\pgfqpoint{4.638362in}{2.001227in}}%
\pgfpathlineto{\pgfqpoint{4.653795in}{2.005594in}}%
\pgfpathlineto{\pgfqpoint{4.669229in}{2.008496in}}%
\pgfpathlineto{\pgfqpoint{4.669229in}{2.008496in}}%
\pgfusepath{stroke}%
\end{pgfscope}%
\begin{pgfscope}%
\pgfpathrectangle{\pgfqpoint{0.634105in}{0.521603in}}{\pgfqpoint{4.227273in}{2.800000in}} %
\pgfusepath{clip}%
\pgfsetrectcap%
\pgfsetroundjoin%
\pgfsetlinewidth{0.501875pt}%
\definecolor{currentstroke}{rgb}{1.000000,0.473094,0.243914}%
\pgfsetstrokecolor{currentstroke}%
\pgfsetdash{}{0pt}%
\pgfpathmoveto{\pgfqpoint{0.826254in}{2.316601in}}%
\pgfpathlineto{\pgfqpoint{0.841687in}{2.314736in}}%
\pgfpathlineto{\pgfqpoint{0.857121in}{2.308949in}}%
\pgfpathlineto{\pgfqpoint{0.872555in}{2.299109in}}%
\pgfpathlineto{\pgfqpoint{0.887988in}{2.285154in}}%
\pgfpathlineto{\pgfqpoint{0.903422in}{2.267086in}}%
\pgfpathlineto{\pgfqpoint{0.918855in}{2.244981in}}%
\pgfpathlineto{\pgfqpoint{0.934289in}{2.218989in}}%
\pgfpathlineto{\pgfqpoint{0.949723in}{2.189331in}}%
\pgfpathlineto{\pgfqpoint{0.965156in}{2.156304in}}%
\pgfpathlineto{\pgfqpoint{0.996024in}{2.081667in}}%
\pgfpathlineto{\pgfqpoint{1.026891in}{1.998751in}}%
\pgfpathlineto{\pgfqpoint{1.104059in}{1.785960in}}%
\pgfpathlineto{\pgfqpoint{1.134926in}{1.712018in}}%
\pgfpathlineto{\pgfqpoint{1.150360in}{1.679885in}}%
\pgfpathlineto{\pgfqpoint{1.165794in}{1.651677in}}%
\pgfpathlineto{\pgfqpoint{1.181227in}{1.627845in}}%
\pgfpathlineto{\pgfqpoint{1.196661in}{1.608763in}}%
\pgfpathlineto{\pgfqpoint{1.212095in}{1.594729in}}%
\pgfpathlineto{\pgfqpoint{1.227528in}{1.585953in}}%
\pgfpathlineto{\pgfqpoint{1.242962in}{1.582557in}}%
\pgfpathlineto{\pgfqpoint{1.258395in}{1.584571in}}%
\pgfpathlineto{\pgfqpoint{1.273829in}{1.591934in}}%
\pgfpathlineto{\pgfqpoint{1.289263in}{1.604500in}}%
\pgfpathlineto{\pgfqpoint{1.304696in}{1.622037in}}%
\pgfpathlineto{\pgfqpoint{1.320130in}{1.644237in}}%
\pgfpathlineto{\pgfqpoint{1.335564in}{1.670724in}}%
\pgfpathlineto{\pgfqpoint{1.350997in}{1.701059in}}%
\pgfpathlineto{\pgfqpoint{1.366431in}{1.734755in}}%
\pgfpathlineto{\pgfqpoint{1.397298in}{1.810087in}}%
\pgfpathlineto{\pgfqpoint{1.443599in}{1.934353in}}%
\pgfpathlineto{\pgfqpoint{1.489900in}{2.058341in}}%
\pgfpathlineto{\pgfqpoint{1.520767in}{2.133775in}}%
\pgfpathlineto{\pgfqpoint{1.536201in}{2.167958in}}%
\pgfpathlineto{\pgfqpoint{1.551634in}{2.199277in}}%
\pgfpathlineto{\pgfqpoint{1.567068in}{2.227414in}}%
\pgfpathlineto{\pgfqpoint{1.582502in}{2.252103in}}%
\pgfpathlineto{\pgfqpoint{1.597935in}{2.273129in}}%
\pgfpathlineto{\pgfqpoint{1.613369in}{2.290331in}}%
\pgfpathlineto{\pgfqpoint{1.628803in}{2.303604in}}%
\pgfpathlineto{\pgfqpoint{1.644236in}{2.312893in}}%
\pgfpathlineto{\pgfqpoint{1.659670in}{2.318197in}}%
\pgfpathlineto{\pgfqpoint{1.675104in}{2.319565in}}%
\pgfpathlineto{\pgfqpoint{1.690537in}{2.317091in}}%
\pgfpathlineto{\pgfqpoint{1.705971in}{2.310918in}}%
\pgfpathlineto{\pgfqpoint{1.721404in}{2.301226in}}%
\pgfpathlineto{\pgfqpoint{1.736838in}{2.288232in}}%
\pgfpathlineto{\pgfqpoint{1.752272in}{2.272189in}}%
\pgfpathlineto{\pgfqpoint{1.767705in}{2.253372in}}%
\pgfpathlineto{\pgfqpoint{1.783139in}{2.232082in}}%
\pgfpathlineto{\pgfqpoint{1.814006in}{2.183365in}}%
\pgfpathlineto{\pgfqpoint{1.844874in}{2.128689in}}%
\pgfpathlineto{\pgfqpoint{1.952909in}{1.928129in}}%
\pgfpathlineto{\pgfqpoint{1.983776in}{1.877763in}}%
\pgfpathlineto{\pgfqpoint{2.014644in}{1.833898in}}%
\pgfpathlineto{\pgfqpoint{2.030077in}{1.814829in}}%
\pgfpathlineto{\pgfqpoint{2.045511in}{1.797847in}}%
\pgfpathlineto{\pgfqpoint{2.060944in}{1.783055in}}%
\pgfpathlineto{\pgfqpoint{2.076378in}{1.770526in}}%
\pgfpathlineto{\pgfqpoint{2.091812in}{1.760303in}}%
\pgfpathlineto{\pgfqpoint{2.107245in}{1.752400in}}%
\pgfpathlineto{\pgfqpoint{2.122679in}{1.746794in}}%
\pgfpathlineto{\pgfqpoint{2.138113in}{1.743428in}}%
\pgfpathlineto{\pgfqpoint{2.153546in}{1.742209in}}%
\pgfpathlineto{\pgfqpoint{2.168980in}{1.743007in}}%
\pgfpathlineto{\pgfqpoint{2.184414in}{1.745654in}}%
\pgfpathlineto{\pgfqpoint{2.199847in}{1.749948in}}%
\pgfpathlineto{\pgfqpoint{2.230714in}{1.762504in}}%
\pgfpathlineto{\pgfqpoint{2.261582in}{1.778456in}}%
\pgfpathlineto{\pgfqpoint{2.307883in}{1.803194in}}%
\pgfpathlineto{\pgfqpoint{2.338750in}{1.816449in}}%
\pgfpathlineto{\pgfqpoint{2.354184in}{1.821213in}}%
\pgfpathlineto{\pgfqpoint{2.369617in}{1.824402in}}%
\pgfpathlineto{\pgfqpoint{2.385051in}{1.825816in}}%
\pgfpathlineto{\pgfqpoint{2.400484in}{1.825297in}}%
\pgfpathlineto{\pgfqpoint{2.415918in}{1.822738in}}%
\pgfpathlineto{\pgfqpoint{2.431352in}{1.818086in}}%
\pgfpathlineto{\pgfqpoint{2.446785in}{1.811340in}}%
\pgfpathlineto{\pgfqpoint{2.462219in}{1.802550in}}%
\pgfpathlineto{\pgfqpoint{2.477653in}{1.791822in}}%
\pgfpathlineto{\pgfqpoint{2.493086in}{1.779307in}}%
\pgfpathlineto{\pgfqpoint{2.523954in}{1.749747in}}%
\pgfpathlineto{\pgfqpoint{2.554821in}{1.715882in}}%
\pgfpathlineto{\pgfqpoint{2.616555in}{1.645141in}}%
\pgfpathlineto{\pgfqpoint{2.647423in}{1.613447in}}%
\pgfpathlineto{\pgfqpoint{2.662856in}{1.599572in}}%
\pgfpathlineto{\pgfqpoint{2.678290in}{1.587348in}}%
\pgfpathlineto{\pgfqpoint{2.693724in}{1.576987in}}%
\pgfpathlineto{\pgfqpoint{2.709157in}{1.568660in}}%
\pgfpathlineto{\pgfqpoint{2.724591in}{1.562495in}}%
\pgfpathlineto{\pgfqpoint{2.740024in}{1.558574in}}%
\pgfpathlineto{\pgfqpoint{2.755458in}{1.556929in}}%
\pgfpathlineto{\pgfqpoint{2.770892in}{1.557546in}}%
\pgfpathlineto{\pgfqpoint{2.786325in}{1.560359in}}%
\pgfpathlineto{\pgfqpoint{2.801759in}{1.565255in}}%
\pgfpathlineto{\pgfqpoint{2.817193in}{1.572073in}}%
\pgfpathlineto{\pgfqpoint{2.832626in}{1.580611in}}%
\pgfpathlineto{\pgfqpoint{2.863493in}{1.601843in}}%
\pgfpathlineto{\pgfqpoint{2.909794in}{1.639571in}}%
\pgfpathlineto{\pgfqpoint{2.940662in}{1.664764in}}%
\pgfpathlineto{\pgfqpoint{2.971529in}{1.686882in}}%
\pgfpathlineto{\pgfqpoint{2.986963in}{1.696037in}}%
\pgfpathlineto{\pgfqpoint{3.002396in}{1.703581in}}%
\pgfpathlineto{\pgfqpoint{3.017830in}{1.709305in}}%
\pgfpathlineto{\pgfqpoint{3.033263in}{1.713046in}}%
\pgfpathlineto{\pgfqpoint{3.048697in}{1.714686in}}%
\pgfpathlineto{\pgfqpoint{3.064131in}{1.714160in}}%
\pgfpathlineto{\pgfqpoint{3.079564in}{1.711451in}}%
\pgfpathlineto{\pgfqpoint{3.094998in}{1.706591in}}%
\pgfpathlineto{\pgfqpoint{3.110432in}{1.699659in}}%
\pgfpathlineto{\pgfqpoint{3.125865in}{1.690781in}}%
\pgfpathlineto{\pgfqpoint{3.141299in}{1.680118in}}%
\pgfpathlineto{\pgfqpoint{3.172166in}{1.654255in}}%
\pgfpathlineto{\pgfqpoint{3.203033in}{1.623945in}}%
\pgfpathlineto{\pgfqpoint{3.295635in}{1.527454in}}%
\pgfpathlineto{\pgfqpoint{3.326503in}{1.499942in}}%
\pgfpathlineto{\pgfqpoint{3.357370in}{1.477247in}}%
\pgfpathlineto{\pgfqpoint{3.372803in}{1.467993in}}%
\pgfpathlineto{\pgfqpoint{3.388237in}{1.460234in}}%
\pgfpathlineto{\pgfqpoint{3.403671in}{1.454010in}}%
\pgfpathlineto{\pgfqpoint{3.419104in}{1.449334in}}%
\pgfpathlineto{\pgfqpoint{3.434538in}{1.446200in}}%
\pgfpathlineto{\pgfqpoint{3.449972in}{1.444579in}}%
\pgfpathlineto{\pgfqpoint{3.465405in}{1.444425in}}%
\pgfpathlineto{\pgfqpoint{3.480839in}{1.445677in}}%
\pgfpathlineto{\pgfqpoint{3.496273in}{1.448265in}}%
\pgfpathlineto{\pgfqpoint{3.511706in}{1.452110in}}%
\pgfpathlineto{\pgfqpoint{3.542573in}{1.463237in}}%
\pgfpathlineto{\pgfqpoint{3.573441in}{1.478400in}}%
\pgfpathlineto{\pgfqpoint{3.604308in}{1.497013in}}%
\pgfpathlineto{\pgfqpoint{3.635175in}{1.518615in}}%
\pgfpathlineto{\pgfqpoint{3.666043in}{1.542891in}}%
\pgfpathlineto{\pgfqpoint{3.696910in}{1.569645in}}%
\pgfpathlineto{\pgfqpoint{3.727777in}{1.598731in}}%
\pgfpathlineto{\pgfqpoint{3.774078in}{1.646290in}}%
\pgfpathlineto{\pgfqpoint{3.835813in}{1.714690in}}%
\pgfpathlineto{\pgfqpoint{3.882113in}{1.766064in}}%
\pgfpathlineto{\pgfqpoint{3.912981in}{1.798066in}}%
\pgfpathlineto{\pgfqpoint{3.943848in}{1.826629in}}%
\pgfpathlineto{\pgfqpoint{3.974715in}{1.850400in}}%
\pgfpathlineto{\pgfqpoint{3.990149in}{1.860125in}}%
\pgfpathlineto{\pgfqpoint{4.005583in}{1.868260in}}%
\pgfpathlineto{\pgfqpoint{4.021016in}{1.874727in}}%
\pgfpathlineto{\pgfqpoint{4.036450in}{1.879482in}}%
\pgfpathlineto{\pgfqpoint{4.051883in}{1.882516in}}%
\pgfpathlineto{\pgfqpoint{4.067317in}{1.883853in}}%
\pgfpathlineto{\pgfqpoint{4.082751in}{1.883559in}}%
\pgfpathlineto{\pgfqpoint{4.098184in}{1.881734in}}%
\pgfpathlineto{\pgfqpoint{4.113618in}{1.878511in}}%
\pgfpathlineto{\pgfqpoint{4.144485in}{1.868571in}}%
\pgfpathlineto{\pgfqpoint{4.175352in}{1.855387in}}%
\pgfpathlineto{\pgfqpoint{4.237087in}{1.827169in}}%
\pgfpathlineto{\pgfqpoint{4.267954in}{1.816151in}}%
\pgfpathlineto{\pgfqpoint{4.283388in}{1.812199in}}%
\pgfpathlineto{\pgfqpoint{4.298822in}{1.809519in}}%
\pgfpathlineto{\pgfqpoint{4.314255in}{1.808249in}}%
\pgfpathlineto{\pgfqpoint{4.329689in}{1.808486in}}%
\pgfpathlineto{\pgfqpoint{4.345122in}{1.810289in}}%
\pgfpathlineto{\pgfqpoint{4.360556in}{1.813675in}}%
\pgfpathlineto{\pgfqpoint{4.375990in}{1.818621in}}%
\pgfpathlineto{\pgfqpoint{4.391423in}{1.825059in}}%
\pgfpathlineto{\pgfqpoint{4.406857in}{1.832887in}}%
\pgfpathlineto{\pgfqpoint{4.437724in}{1.852112in}}%
\pgfpathlineto{\pgfqpoint{4.468592in}{1.874815in}}%
\pgfpathlineto{\pgfqpoint{4.545760in}{1.934627in}}%
\pgfpathlineto{\pgfqpoint{4.576627in}{1.954955in}}%
\pgfpathlineto{\pgfqpoint{4.607494in}{1.971047in}}%
\pgfpathlineto{\pgfqpoint{4.622928in}{1.977233in}}%
\pgfpathlineto{\pgfqpoint{4.638362in}{1.982104in}}%
\pgfpathlineto{\pgfqpoint{4.653795in}{1.985648in}}%
\pgfpathlineto{\pgfqpoint{4.669229in}{1.987886in}}%
\pgfpathlineto{\pgfqpoint{4.669229in}{1.987886in}}%
\pgfusepath{stroke}%
\end{pgfscope}%
\begin{pgfscope}%
\pgfpathrectangle{\pgfqpoint{0.634105in}{0.521603in}}{\pgfqpoint{4.227273in}{2.800000in}} %
\pgfusepath{clip}%
\pgfsetrectcap%
\pgfsetroundjoin%
\pgfsetlinewidth{0.501875pt}%
\definecolor{currentstroke}{rgb}{1.000000,0.361242,0.183750}%
\pgfsetstrokecolor{currentstroke}%
\pgfsetdash{}{0pt}%
\pgfpathmoveto{\pgfqpoint{0.826254in}{2.309857in}}%
\pgfpathlineto{\pgfqpoint{0.841687in}{2.307809in}}%
\pgfpathlineto{\pgfqpoint{0.857121in}{2.302049in}}%
\pgfpathlineto{\pgfqpoint{0.872555in}{2.292449in}}%
\pgfpathlineto{\pgfqpoint{0.887988in}{2.278947in}}%
\pgfpathlineto{\pgfqpoint{0.903422in}{2.261548in}}%
\pgfpathlineto{\pgfqpoint{0.918855in}{2.240326in}}%
\pgfpathlineto{\pgfqpoint{0.934289in}{2.215431in}}%
\pgfpathlineto{\pgfqpoint{0.949723in}{2.187081in}}%
\pgfpathlineto{\pgfqpoint{0.965156in}{2.155567in}}%
\pgfpathlineto{\pgfqpoint{0.996024in}{2.084531in}}%
\pgfpathlineto{\pgfqpoint{1.026891in}{2.005863in}}%
\pgfpathlineto{\pgfqpoint{1.088625in}{1.843164in}}%
\pgfpathlineto{\pgfqpoint{1.119493in}{1.768624in}}%
\pgfpathlineto{\pgfqpoint{1.134926in}{1.735084in}}%
\pgfpathlineto{\pgfqpoint{1.150360in}{1.704729in}}%
\pgfpathlineto{\pgfqpoint{1.165794in}{1.678025in}}%
\pgfpathlineto{\pgfqpoint{1.181227in}{1.655380in}}%
\pgfpathlineto{\pgfqpoint{1.196661in}{1.637141in}}%
\pgfpathlineto{\pgfqpoint{1.212095in}{1.623582in}}%
\pgfpathlineto{\pgfqpoint{1.227528in}{1.614905in}}%
\pgfpathlineto{\pgfqpoint{1.242962in}{1.611233in}}%
\pgfpathlineto{\pgfqpoint{1.258395in}{1.612610in}}%
\pgfpathlineto{\pgfqpoint{1.273829in}{1.619001in}}%
\pgfpathlineto{\pgfqpoint{1.289263in}{1.630293in}}%
\pgfpathlineto{\pgfqpoint{1.304696in}{1.646297in}}%
\pgfpathlineto{\pgfqpoint{1.320130in}{1.666754in}}%
\pgfpathlineto{\pgfqpoint{1.335564in}{1.691337in}}%
\pgfpathlineto{\pgfqpoint{1.350997in}{1.719661in}}%
\pgfpathlineto{\pgfqpoint{1.366431in}{1.751291in}}%
\pgfpathlineto{\pgfqpoint{1.397298in}{1.822504in}}%
\pgfpathlineto{\pgfqpoint{1.428165in}{1.900766in}}%
\pgfpathlineto{\pgfqpoint{1.489900in}{2.060596in}}%
\pgfpathlineto{\pgfqpoint{1.520767in}{2.133635in}}%
\pgfpathlineto{\pgfqpoint{1.536201in}{2.166792in}}%
\pgfpathlineto{\pgfqpoint{1.551634in}{2.197191in}}%
\pgfpathlineto{\pgfqpoint{1.567068in}{2.224511in}}%
\pgfpathlineto{\pgfqpoint{1.582502in}{2.248484in}}%
\pgfpathlineto{\pgfqpoint{1.597935in}{2.268899in}}%
\pgfpathlineto{\pgfqpoint{1.613369in}{2.285604in}}%
\pgfpathlineto{\pgfqpoint{1.628803in}{2.298500in}}%
\pgfpathlineto{\pgfqpoint{1.644236in}{2.307548in}}%
\pgfpathlineto{\pgfqpoint{1.659670in}{2.312761in}}%
\pgfpathlineto{\pgfqpoint{1.675104in}{2.314199in}}%
\pgfpathlineto{\pgfqpoint{1.690537in}{2.311971in}}%
\pgfpathlineto{\pgfqpoint{1.705971in}{2.306226in}}%
\pgfpathlineto{\pgfqpoint{1.721404in}{2.297150in}}%
\pgfpathlineto{\pgfqpoint{1.736838in}{2.284959in}}%
\pgfpathlineto{\pgfqpoint{1.752272in}{2.269896in}}%
\pgfpathlineto{\pgfqpoint{1.767705in}{2.252224in}}%
\pgfpathlineto{\pgfqpoint{1.783139in}{2.232222in}}%
\pgfpathlineto{\pgfqpoint{1.814006in}{2.186396in}}%
\pgfpathlineto{\pgfqpoint{1.844874in}{2.134799in}}%
\pgfpathlineto{\pgfqpoint{1.906608in}{2.023737in}}%
\pgfpathlineto{\pgfqpoint{1.952909in}{1.942250in}}%
\pgfpathlineto{\pgfqpoint{1.983776in}{1.892478in}}%
\pgfpathlineto{\pgfqpoint{2.014644in}{1.848284in}}%
\pgfpathlineto{\pgfqpoint{2.030077in}{1.828710in}}%
\pgfpathlineto{\pgfqpoint{2.045511in}{1.811008in}}%
\pgfpathlineto{\pgfqpoint{2.060944in}{1.795288in}}%
\pgfpathlineto{\pgfqpoint{2.076378in}{1.781638in}}%
\pgfpathlineto{\pgfqpoint{2.091812in}{1.770115in}}%
\pgfpathlineto{\pgfqpoint{2.107245in}{1.760747in}}%
\pgfpathlineto{\pgfqpoint{2.122679in}{1.753529in}}%
\pgfpathlineto{\pgfqpoint{2.138113in}{1.748421in}}%
\pgfpathlineto{\pgfqpoint{2.153546in}{1.745351in}}%
\pgfpathlineto{\pgfqpoint{2.168980in}{1.744210in}}%
\pgfpathlineto{\pgfqpoint{2.184414in}{1.744855in}}%
\pgfpathlineto{\pgfqpoint{2.199847in}{1.747112in}}%
\pgfpathlineto{\pgfqpoint{2.215281in}{1.750771in}}%
\pgfpathlineto{\pgfqpoint{2.246148in}{1.761340in}}%
\pgfpathlineto{\pgfqpoint{2.338750in}{1.798808in}}%
\pgfpathlineto{\pgfqpoint{2.354184in}{1.802874in}}%
\pgfpathlineto{\pgfqpoint{2.369617in}{1.805633in}}%
\pgfpathlineto{\pgfqpoint{2.385051in}{1.806890in}}%
\pgfpathlineto{\pgfqpoint{2.400484in}{1.806486in}}%
\pgfpathlineto{\pgfqpoint{2.415918in}{1.804307in}}%
\pgfpathlineto{\pgfqpoint{2.431352in}{1.800284in}}%
\pgfpathlineto{\pgfqpoint{2.446785in}{1.794397in}}%
\pgfpathlineto{\pgfqpoint{2.462219in}{1.786674in}}%
\pgfpathlineto{\pgfqpoint{2.477653in}{1.777191in}}%
\pgfpathlineto{\pgfqpoint{2.493086in}{1.766071in}}%
\pgfpathlineto{\pgfqpoint{2.523954in}{1.739620in}}%
\pgfpathlineto{\pgfqpoint{2.554821in}{1.709083in}}%
\pgfpathlineto{\pgfqpoint{2.631989in}{1.629721in}}%
\pgfpathlineto{\pgfqpoint{2.662856in}{1.602952in}}%
\pgfpathlineto{\pgfqpoint{2.678290in}{1.591677in}}%
\pgfpathlineto{\pgfqpoint{2.693724in}{1.582074in}}%
\pgfpathlineto{\pgfqpoint{2.709157in}{1.574292in}}%
\pgfpathlineto{\pgfqpoint{2.724591in}{1.568440in}}%
\pgfpathlineto{\pgfqpoint{2.740024in}{1.564582in}}%
\pgfpathlineto{\pgfqpoint{2.755458in}{1.562736in}}%
\pgfpathlineto{\pgfqpoint{2.770892in}{1.562874in}}%
\pgfpathlineto{\pgfqpoint{2.786325in}{1.564922in}}%
\pgfpathlineto{\pgfqpoint{2.801759in}{1.568766in}}%
\pgfpathlineto{\pgfqpoint{2.817193in}{1.574250in}}%
\pgfpathlineto{\pgfqpoint{2.832626in}{1.581181in}}%
\pgfpathlineto{\pgfqpoint{2.863493in}{1.598466in}}%
\pgfpathlineto{\pgfqpoint{2.909794in}{1.628937in}}%
\pgfpathlineto{\pgfqpoint{2.940662in}{1.648919in}}%
\pgfpathlineto{\pgfqpoint{2.971529in}{1.666008in}}%
\pgfpathlineto{\pgfqpoint{2.986963in}{1.672841in}}%
\pgfpathlineto{\pgfqpoint{3.002396in}{1.678252in}}%
\pgfpathlineto{\pgfqpoint{3.017830in}{1.682069in}}%
\pgfpathlineto{\pgfqpoint{3.033263in}{1.684160in}}%
\pgfpathlineto{\pgfqpoint{3.048697in}{1.684431in}}%
\pgfpathlineto{\pgfqpoint{3.064131in}{1.682831in}}%
\pgfpathlineto{\pgfqpoint{3.079564in}{1.679351in}}%
\pgfpathlineto{\pgfqpoint{3.094998in}{1.674025in}}%
\pgfpathlineto{\pgfqpoint{3.110432in}{1.666926in}}%
\pgfpathlineto{\pgfqpoint{3.125865in}{1.658169in}}%
\pgfpathlineto{\pgfqpoint{3.156733in}{1.636305in}}%
\pgfpathlineto{\pgfqpoint{3.187600in}{1.609970in}}%
\pgfpathlineto{\pgfqpoint{3.295635in}{1.510327in}}%
\pgfpathlineto{\pgfqpoint{3.326503in}{1.487332in}}%
\pgfpathlineto{\pgfqpoint{3.341936in}{1.477636in}}%
\pgfpathlineto{\pgfqpoint{3.357370in}{1.469280in}}%
\pgfpathlineto{\pgfqpoint{3.372803in}{1.462329in}}%
\pgfpathlineto{\pgfqpoint{3.388237in}{1.456819in}}%
\pgfpathlineto{\pgfqpoint{3.403671in}{1.452758in}}%
\pgfpathlineto{\pgfqpoint{3.419104in}{1.450128in}}%
\pgfpathlineto{\pgfqpoint{3.434538in}{1.448889in}}%
\pgfpathlineto{\pgfqpoint{3.449972in}{1.448983in}}%
\pgfpathlineto{\pgfqpoint{3.465405in}{1.450336in}}%
\pgfpathlineto{\pgfqpoint{3.496273in}{1.456480in}}%
\pgfpathlineto{\pgfqpoint{3.527140in}{1.466613in}}%
\pgfpathlineto{\pgfqpoint{3.558007in}{1.480056in}}%
\pgfpathlineto{\pgfqpoint{3.588874in}{1.496261in}}%
\pgfpathlineto{\pgfqpoint{3.619742in}{1.514858in}}%
\pgfpathlineto{\pgfqpoint{3.650609in}{1.535659in}}%
\pgfpathlineto{\pgfqpoint{3.681476in}{1.558614in}}%
\pgfpathlineto{\pgfqpoint{3.712343in}{1.583731in}}%
\pgfpathlineto{\pgfqpoint{3.743211in}{1.610975in}}%
\pgfpathlineto{\pgfqpoint{3.789512in}{1.655385in}}%
\pgfpathlineto{\pgfqpoint{3.912981in}{1.779034in}}%
\pgfpathlineto{\pgfqpoint{3.943848in}{1.805397in}}%
\pgfpathlineto{\pgfqpoint{3.974715in}{1.827539in}}%
\pgfpathlineto{\pgfqpoint{3.990149in}{1.836716in}}%
\pgfpathlineto{\pgfqpoint{4.005583in}{1.844506in}}%
\pgfpathlineto{\pgfqpoint{4.021016in}{1.850846in}}%
\pgfpathlineto{\pgfqpoint{4.036450in}{1.855703in}}%
\pgfpathlineto{\pgfqpoint{4.051883in}{1.859075in}}%
\pgfpathlineto{\pgfqpoint{4.067317in}{1.860992in}}%
\pgfpathlineto{\pgfqpoint{4.082751in}{1.861517in}}%
\pgfpathlineto{\pgfqpoint{4.098184in}{1.860746in}}%
\pgfpathlineto{\pgfqpoint{4.129052in}{1.855842in}}%
\pgfpathlineto{\pgfqpoint{4.159919in}{1.847602in}}%
\pgfpathlineto{\pgfqpoint{4.237087in}{1.823570in}}%
\pgfpathlineto{\pgfqpoint{4.267954in}{1.817241in}}%
\pgfpathlineto{\pgfqpoint{4.283388in}{1.815561in}}%
\pgfpathlineto{\pgfqpoint{4.298822in}{1.815062in}}%
\pgfpathlineto{\pgfqpoint{4.314255in}{1.815850in}}%
\pgfpathlineto{\pgfqpoint{4.329689in}{1.817994in}}%
\pgfpathlineto{\pgfqpoint{4.345122in}{1.821525in}}%
\pgfpathlineto{\pgfqpoint{4.360556in}{1.826440in}}%
\pgfpathlineto{\pgfqpoint{4.375990in}{1.832694in}}%
\pgfpathlineto{\pgfqpoint{4.406857in}{1.848875in}}%
\pgfpathlineto{\pgfqpoint{4.437724in}{1.869056in}}%
\pgfpathlineto{\pgfqpoint{4.484025in}{1.903648in}}%
\pgfpathlineto{\pgfqpoint{4.530326in}{1.938363in}}%
\pgfpathlineto{\pgfqpoint{4.561193in}{1.958845in}}%
\pgfpathlineto{\pgfqpoint{4.592061in}{1.975650in}}%
\pgfpathlineto{\pgfqpoint{4.622928in}{1.987910in}}%
\pgfpathlineto{\pgfqpoint{4.638362in}{1.992193in}}%
\pgfpathlineto{\pgfqpoint{4.653795in}{1.995230in}}%
\pgfpathlineto{\pgfqpoint{4.669229in}{1.997048in}}%
\pgfpathlineto{\pgfqpoint{4.669229in}{1.997048in}}%
\pgfusepath{stroke}%
\end{pgfscope}%
\begin{pgfscope}%
\pgfpathrectangle{\pgfqpoint{0.634105in}{0.521603in}}{\pgfqpoint{4.227273in}{2.800000in}} %
\pgfusepath{clip}%
\pgfsetrectcap%
\pgfsetroundjoin%
\pgfsetlinewidth{0.501875pt}%
\definecolor{currentstroke}{rgb}{1.000000,0.243914,0.122888}%
\pgfsetstrokecolor{currentstroke}%
\pgfsetdash{}{0pt}%
\pgfpathmoveto{\pgfqpoint{0.826254in}{2.313546in}}%
\pgfpathlineto{\pgfqpoint{0.841687in}{2.312691in}}%
\pgfpathlineto{\pgfqpoint{0.857121in}{2.308016in}}%
\pgfpathlineto{\pgfqpoint{0.872555in}{2.299372in}}%
\pgfpathlineto{\pgfqpoint{0.887988in}{2.286674in}}%
\pgfpathlineto{\pgfqpoint{0.903422in}{2.269910in}}%
\pgfpathlineto{\pgfqpoint{0.918855in}{2.249138in}}%
\pgfpathlineto{\pgfqpoint{0.934289in}{2.224494in}}%
\pgfpathlineto{\pgfqpoint{0.949723in}{2.196191in}}%
\pgfpathlineto{\pgfqpoint{0.965156in}{2.164516in}}%
\pgfpathlineto{\pgfqpoint{0.996024in}{2.092547in}}%
\pgfpathlineto{\pgfqpoint{1.026891in}{2.012206in}}%
\pgfpathlineto{\pgfqpoint{1.104059in}{1.804898in}}%
\pgfpathlineto{\pgfqpoint{1.134926in}{1.732404in}}%
\pgfpathlineto{\pgfqpoint{1.150360in}{1.700750in}}%
\pgfpathlineto{\pgfqpoint{1.165794in}{1.672831in}}%
\pgfpathlineto{\pgfqpoint{1.181227in}{1.649079in}}%
\pgfpathlineto{\pgfqpoint{1.196661in}{1.629860in}}%
\pgfpathlineto{\pgfqpoint{1.212095in}{1.615465in}}%
\pgfpathlineto{\pgfqpoint{1.227528in}{1.606109in}}%
\pgfpathlineto{\pgfqpoint{1.242962in}{1.601921in}}%
\pgfpathlineto{\pgfqpoint{1.258395in}{1.602952in}}%
\pgfpathlineto{\pgfqpoint{1.273829in}{1.609165in}}%
\pgfpathlineto{\pgfqpoint{1.289263in}{1.620446in}}%
\pgfpathlineto{\pgfqpoint{1.304696in}{1.636601in}}%
\pgfpathlineto{\pgfqpoint{1.320130in}{1.657362in}}%
\pgfpathlineto{\pgfqpoint{1.335564in}{1.682394in}}%
\pgfpathlineto{\pgfqpoint{1.350997in}{1.711301in}}%
\pgfpathlineto{\pgfqpoint{1.366431in}{1.743633in}}%
\pgfpathlineto{\pgfqpoint{1.397298in}{1.816560in}}%
\pgfpathlineto{\pgfqpoint{1.428165in}{1.896850in}}%
\pgfpathlineto{\pgfqpoint{1.489900in}{2.061182in}}%
\pgfpathlineto{\pgfqpoint{1.520767in}{2.136435in}}%
\pgfpathlineto{\pgfqpoint{1.536201in}{2.170629in}}%
\pgfpathlineto{\pgfqpoint{1.551634in}{2.201999in}}%
\pgfpathlineto{\pgfqpoint{1.567068in}{2.230207in}}%
\pgfpathlineto{\pgfqpoint{1.582502in}{2.254969in}}%
\pgfpathlineto{\pgfqpoint{1.597935in}{2.276064in}}%
\pgfpathlineto{\pgfqpoint{1.613369in}{2.293326in}}%
\pgfpathlineto{\pgfqpoint{1.628803in}{2.306649in}}%
\pgfpathlineto{\pgfqpoint{1.644236in}{2.315989in}}%
\pgfpathlineto{\pgfqpoint{1.659670in}{2.321353in}}%
\pgfpathlineto{\pgfqpoint{1.675104in}{2.322804in}}%
\pgfpathlineto{\pgfqpoint{1.690537in}{2.320452in}}%
\pgfpathlineto{\pgfqpoint{1.705971in}{2.314455in}}%
\pgfpathlineto{\pgfqpoint{1.721404in}{2.305006in}}%
\pgfpathlineto{\pgfqpoint{1.736838in}{2.292334in}}%
\pgfpathlineto{\pgfqpoint{1.752272in}{2.276696in}}%
\pgfpathlineto{\pgfqpoint{1.767705in}{2.258371in}}%
\pgfpathlineto{\pgfqpoint{1.783139in}{2.237656in}}%
\pgfpathlineto{\pgfqpoint{1.814006in}{2.190294in}}%
\pgfpathlineto{\pgfqpoint{1.844874in}{2.137135in}}%
\pgfpathlineto{\pgfqpoint{1.968343in}{1.914779in}}%
\pgfpathlineto{\pgfqpoint{1.999210in}{1.867345in}}%
\pgfpathlineto{\pgfqpoint{2.030077in}{1.826583in}}%
\pgfpathlineto{\pgfqpoint{2.045511in}{1.809087in}}%
\pgfpathlineto{\pgfqpoint{2.060944in}{1.793672in}}%
\pgfpathlineto{\pgfqpoint{2.076378in}{1.780422in}}%
\pgfpathlineto{\pgfqpoint{2.091812in}{1.769393in}}%
\pgfpathlineto{\pgfqpoint{2.107245in}{1.760608in}}%
\pgfpathlineto{\pgfqpoint{2.122679in}{1.754055in}}%
\pgfpathlineto{\pgfqpoint{2.138113in}{1.749690in}}%
\pgfpathlineto{\pgfqpoint{2.153546in}{1.747427in}}%
\pgfpathlineto{\pgfqpoint{2.168980in}{1.747147in}}%
\pgfpathlineto{\pgfqpoint{2.184414in}{1.748691in}}%
\pgfpathlineto{\pgfqpoint{2.199847in}{1.751868in}}%
\pgfpathlineto{\pgfqpoint{2.230714in}{1.762186in}}%
\pgfpathlineto{\pgfqpoint{2.261582in}{1.775967in}}%
\pgfpathlineto{\pgfqpoint{2.307883in}{1.797777in}}%
\pgfpathlineto{\pgfqpoint{2.338750in}{1.809430in}}%
\pgfpathlineto{\pgfqpoint{2.354184in}{1.813528in}}%
\pgfpathlineto{\pgfqpoint{2.369617in}{1.816151in}}%
\pgfpathlineto{\pgfqpoint{2.385051in}{1.817101in}}%
\pgfpathlineto{\pgfqpoint{2.400484in}{1.816228in}}%
\pgfpathlineto{\pgfqpoint{2.415918in}{1.813424in}}%
\pgfpathlineto{\pgfqpoint{2.431352in}{1.808634in}}%
\pgfpathlineto{\pgfqpoint{2.446785in}{1.801856in}}%
\pgfpathlineto{\pgfqpoint{2.462219in}{1.793138in}}%
\pgfpathlineto{\pgfqpoint{2.477653in}{1.782579in}}%
\pgfpathlineto{\pgfqpoint{2.493086in}{1.770328in}}%
\pgfpathlineto{\pgfqpoint{2.523954in}{1.741551in}}%
\pgfpathlineto{\pgfqpoint{2.554821in}{1.708770in}}%
\pgfpathlineto{\pgfqpoint{2.616555in}{1.640871in}}%
\pgfpathlineto{\pgfqpoint{2.647423in}{1.610776in}}%
\pgfpathlineto{\pgfqpoint{2.662856in}{1.597696in}}%
\pgfpathlineto{\pgfqpoint{2.678290in}{1.586243in}}%
\pgfpathlineto{\pgfqpoint{2.693724in}{1.576607in}}%
\pgfpathlineto{\pgfqpoint{2.709157in}{1.568940in}}%
\pgfpathlineto{\pgfqpoint{2.724591in}{1.563348in}}%
\pgfpathlineto{\pgfqpoint{2.740024in}{1.559892in}}%
\pgfpathlineto{\pgfqpoint{2.755458in}{1.558585in}}%
\pgfpathlineto{\pgfqpoint{2.770892in}{1.559395in}}%
\pgfpathlineto{\pgfqpoint{2.786325in}{1.562242in}}%
\pgfpathlineto{\pgfqpoint{2.801759in}{1.567003in}}%
\pgfpathlineto{\pgfqpoint{2.817193in}{1.573512in}}%
\pgfpathlineto{\pgfqpoint{2.832626in}{1.581567in}}%
\pgfpathlineto{\pgfqpoint{2.863493in}{1.601335in}}%
\pgfpathlineto{\pgfqpoint{2.909794in}{1.635857in}}%
\pgfpathlineto{\pgfqpoint{2.940662in}{1.658494in}}%
\pgfpathlineto{\pgfqpoint{2.971529in}{1.677955in}}%
\pgfpathlineto{\pgfqpoint{2.986963in}{1.685799in}}%
\pgfpathlineto{\pgfqpoint{3.002396in}{1.692067in}}%
\pgfpathlineto{\pgfqpoint{3.017830in}{1.696565in}}%
\pgfpathlineto{\pgfqpoint{3.033263in}{1.699135in}}%
\pgfpathlineto{\pgfqpoint{3.048697in}{1.699667in}}%
\pgfpathlineto{\pgfqpoint{3.064131in}{1.698095in}}%
\pgfpathlineto{\pgfqpoint{3.079564in}{1.694399in}}%
\pgfpathlineto{\pgfqpoint{3.094998in}{1.688611in}}%
\pgfpathlineto{\pgfqpoint{3.110432in}{1.680805in}}%
\pgfpathlineto{\pgfqpoint{3.125865in}{1.671102in}}%
\pgfpathlineto{\pgfqpoint{3.141299in}{1.659664in}}%
\pgfpathlineto{\pgfqpoint{3.172166in}{1.632406in}}%
\pgfpathlineto{\pgfqpoint{3.203033in}{1.600955in}}%
\pgfpathlineto{\pgfqpoint{3.280202in}{1.518652in}}%
\pgfpathlineto{\pgfqpoint{3.311069in}{1.489913in}}%
\pgfpathlineto{\pgfqpoint{3.341936in}{1.466159in}}%
\pgfpathlineto{\pgfqpoint{3.357370in}{1.456507in}}%
\pgfpathlineto{\pgfqpoint{3.372803in}{1.448461in}}%
\pgfpathlineto{\pgfqpoint{3.388237in}{1.442073in}}%
\pgfpathlineto{\pgfqpoint{3.403671in}{1.437358in}}%
\pgfpathlineto{\pgfqpoint{3.419104in}{1.434309in}}%
\pgfpathlineto{\pgfqpoint{3.434538in}{1.432887in}}%
\pgfpathlineto{\pgfqpoint{3.449972in}{1.433036in}}%
\pgfpathlineto{\pgfqpoint{3.465405in}{1.434681in}}%
\pgfpathlineto{\pgfqpoint{3.480839in}{1.437733in}}%
\pgfpathlineto{\pgfqpoint{3.496273in}{1.442096in}}%
\pgfpathlineto{\pgfqpoint{3.527140in}{1.454354in}}%
\pgfpathlineto{\pgfqpoint{3.558007in}{1.470670in}}%
\pgfpathlineto{\pgfqpoint{3.588874in}{1.490352in}}%
\pgfpathlineto{\pgfqpoint{3.619742in}{1.512856in}}%
\pgfpathlineto{\pgfqpoint{3.650609in}{1.537792in}}%
\pgfpathlineto{\pgfqpoint{3.681476in}{1.564889in}}%
\pgfpathlineto{\pgfqpoint{3.727777in}{1.609095in}}%
\pgfpathlineto{\pgfqpoint{3.774078in}{1.656680in}}%
\pgfpathlineto{\pgfqpoint{3.866680in}{1.754105in}}%
\pgfpathlineto{\pgfqpoint{3.897547in}{1.783865in}}%
\pgfpathlineto{\pgfqpoint{3.928414in}{1.810383in}}%
\pgfpathlineto{\pgfqpoint{3.959282in}{1.832553in}}%
\pgfpathlineto{\pgfqpoint{3.974715in}{1.841709in}}%
\pgfpathlineto{\pgfqpoint{3.990149in}{1.849457in}}%
\pgfpathlineto{\pgfqpoint{4.005583in}{1.855731in}}%
\pgfpathlineto{\pgfqpoint{4.021016in}{1.860494in}}%
\pgfpathlineto{\pgfqpoint{4.036450in}{1.863737in}}%
\pgfpathlineto{\pgfqpoint{4.051883in}{1.865483in}}%
\pgfpathlineto{\pgfqpoint{4.067317in}{1.865785in}}%
\pgfpathlineto{\pgfqpoint{4.082751in}{1.864729in}}%
\pgfpathlineto{\pgfqpoint{4.113618in}{1.859035in}}%
\pgfpathlineto{\pgfqpoint{4.144485in}{1.849664in}}%
\pgfpathlineto{\pgfqpoint{4.237087in}{1.816680in}}%
\pgfpathlineto{\pgfqpoint{4.267954in}{1.810162in}}%
\pgfpathlineto{\pgfqpoint{4.283388in}{1.808642in}}%
\pgfpathlineto{\pgfqpoint{4.298822in}{1.808468in}}%
\pgfpathlineto{\pgfqpoint{4.314255in}{1.809738in}}%
\pgfpathlineto{\pgfqpoint{4.329689in}{1.812511in}}%
\pgfpathlineto{\pgfqpoint{4.345122in}{1.816809in}}%
\pgfpathlineto{\pgfqpoint{4.360556in}{1.822613in}}%
\pgfpathlineto{\pgfqpoint{4.375990in}{1.829864in}}%
\pgfpathlineto{\pgfqpoint{4.406857in}{1.848285in}}%
\pgfpathlineto{\pgfqpoint{4.437724in}{1.870902in}}%
\pgfpathlineto{\pgfqpoint{4.484025in}{1.909133in}}%
\pgfpathlineto{\pgfqpoint{4.530326in}{1.946953in}}%
\pgfpathlineto{\pgfqpoint{4.561193in}{1.968964in}}%
\pgfpathlineto{\pgfqpoint{4.592061in}{1.986745in}}%
\pgfpathlineto{\pgfqpoint{4.607494in}{1.993740in}}%
\pgfpathlineto{\pgfqpoint{4.622928in}{1.999371in}}%
\pgfpathlineto{\pgfqpoint{4.638362in}{2.003607in}}%
\pgfpathlineto{\pgfqpoint{4.653795in}{2.006448in}}%
\pgfpathlineto{\pgfqpoint{4.669229in}{2.007926in}}%
\pgfpathlineto{\pgfqpoint{4.669229in}{2.007926in}}%
\pgfusepath{stroke}%
\end{pgfscope}%
\begin{pgfscope}%
\pgfpathrectangle{\pgfqpoint{0.634105in}{0.521603in}}{\pgfqpoint{4.227273in}{2.800000in}} %
\pgfusepath{clip}%
\pgfsetrectcap%
\pgfsetroundjoin%
\pgfsetlinewidth{0.501875pt}%
\definecolor{currentstroke}{rgb}{1.000000,0.122888,0.061561}%
\pgfsetstrokecolor{currentstroke}%
\pgfsetdash{}{0pt}%
\pgfpathmoveto{\pgfqpoint{0.826254in}{2.310355in}}%
\pgfpathlineto{\pgfqpoint{0.841687in}{2.309189in}}%
\pgfpathlineto{\pgfqpoint{0.857121in}{2.304256in}}%
\pgfpathlineto{\pgfqpoint{0.872555in}{2.295405in}}%
\pgfpathlineto{\pgfqpoint{0.887988in}{2.282547in}}%
\pgfpathlineto{\pgfqpoint{0.903422in}{2.265666in}}%
\pgfpathlineto{\pgfqpoint{0.918855in}{2.244815in}}%
\pgfpathlineto{\pgfqpoint{0.934289in}{2.220126in}}%
\pgfpathlineto{\pgfqpoint{0.949723in}{2.191803in}}%
\pgfpathlineto{\pgfqpoint{0.965156in}{2.160130in}}%
\pgfpathlineto{\pgfqpoint{0.996024in}{2.088217in}}%
\pgfpathlineto{\pgfqpoint{1.026891in}{2.007996in}}%
\pgfpathlineto{\pgfqpoint{1.104059in}{1.801500in}}%
\pgfpathlineto{\pgfqpoint{1.134926in}{1.729716in}}%
\pgfpathlineto{\pgfqpoint{1.150360in}{1.698532in}}%
\pgfpathlineto{\pgfqpoint{1.165794in}{1.671166in}}%
\pgfpathlineto{\pgfqpoint{1.181227in}{1.648049in}}%
\pgfpathlineto{\pgfqpoint{1.196661in}{1.629544in}}%
\pgfpathlineto{\pgfqpoint{1.212095in}{1.615933in}}%
\pgfpathlineto{\pgfqpoint{1.227528in}{1.607418in}}%
\pgfpathlineto{\pgfqpoint{1.242962in}{1.604113in}}%
\pgfpathlineto{\pgfqpoint{1.258395in}{1.606050in}}%
\pgfpathlineto{\pgfqpoint{1.273829in}{1.613173in}}%
\pgfpathlineto{\pgfqpoint{1.289263in}{1.625346in}}%
\pgfpathlineto{\pgfqpoint{1.304696in}{1.642352in}}%
\pgfpathlineto{\pgfqpoint{1.320130in}{1.663904in}}%
\pgfpathlineto{\pgfqpoint{1.335564in}{1.689650in}}%
\pgfpathlineto{\pgfqpoint{1.350997in}{1.719178in}}%
\pgfpathlineto{\pgfqpoint{1.366431in}{1.752027in}}%
\pgfpathlineto{\pgfqpoint{1.397298in}{1.825647in}}%
\pgfpathlineto{\pgfqpoint{1.443599in}{1.947592in}}%
\pgfpathlineto{\pgfqpoint{1.489900in}{2.069751in}}%
\pgfpathlineto{\pgfqpoint{1.520767in}{2.144185in}}%
\pgfpathlineto{\pgfqpoint{1.536201in}{2.177902in}}%
\pgfpathlineto{\pgfqpoint{1.551634in}{2.208764in}}%
\pgfpathlineto{\pgfqpoint{1.567068in}{2.236446in}}%
\pgfpathlineto{\pgfqpoint{1.582502in}{2.260678in}}%
\pgfpathlineto{\pgfqpoint{1.597935in}{2.281248in}}%
\pgfpathlineto{\pgfqpoint{1.613369in}{2.298006in}}%
\pgfpathlineto{\pgfqpoint{1.628803in}{2.310858in}}%
\pgfpathlineto{\pgfqpoint{1.644236in}{2.319769in}}%
\pgfpathlineto{\pgfqpoint{1.659670in}{2.324763in}}%
\pgfpathlineto{\pgfqpoint{1.675104in}{2.325913in}}%
\pgfpathlineto{\pgfqpoint{1.690537in}{2.323341in}}%
\pgfpathlineto{\pgfqpoint{1.705971in}{2.317214in}}%
\pgfpathlineto{\pgfqpoint{1.721404in}{2.307734in}}%
\pgfpathlineto{\pgfqpoint{1.736838in}{2.295137in}}%
\pgfpathlineto{\pgfqpoint{1.752272in}{2.279683in}}%
\pgfpathlineto{\pgfqpoint{1.767705in}{2.261654in}}%
\pgfpathlineto{\pgfqpoint{1.783139in}{2.241342in}}%
\pgfpathlineto{\pgfqpoint{1.814006in}{2.195089in}}%
\pgfpathlineto{\pgfqpoint{1.844874in}{2.143368in}}%
\pgfpathlineto{\pgfqpoint{1.968343in}{1.927272in}}%
\pgfpathlineto{\pgfqpoint{1.999210in}{1.880770in}}%
\pgfpathlineto{\pgfqpoint{2.030077in}{1.840462in}}%
\pgfpathlineto{\pgfqpoint{2.045511in}{1.823002in}}%
\pgfpathlineto{\pgfqpoint{2.060944in}{1.807497in}}%
\pgfpathlineto{\pgfqpoint{2.076378in}{1.794035in}}%
\pgfpathlineto{\pgfqpoint{2.091812in}{1.782682in}}%
\pgfpathlineto{\pgfqpoint{2.107245in}{1.773472in}}%
\pgfpathlineto{\pgfqpoint{2.122679in}{1.766409in}}%
\pgfpathlineto{\pgfqpoint{2.138113in}{1.761464in}}%
\pgfpathlineto{\pgfqpoint{2.153546in}{1.758571in}}%
\pgfpathlineto{\pgfqpoint{2.168980in}{1.757629in}}%
\pgfpathlineto{\pgfqpoint{2.184414in}{1.758497in}}%
\pgfpathlineto{\pgfqpoint{2.199847in}{1.761001in}}%
\pgfpathlineto{\pgfqpoint{2.215281in}{1.764929in}}%
\pgfpathlineto{\pgfqpoint{2.246148in}{1.776060in}}%
\pgfpathlineto{\pgfqpoint{2.323316in}{1.809161in}}%
\pgfpathlineto{\pgfqpoint{2.338750in}{1.814188in}}%
\pgfpathlineto{\pgfqpoint{2.354184in}{1.818003in}}%
\pgfpathlineto{\pgfqpoint{2.369617in}{1.820365in}}%
\pgfpathlineto{\pgfqpoint{2.385051in}{1.821072in}}%
\pgfpathlineto{\pgfqpoint{2.400484in}{1.819965in}}%
\pgfpathlineto{\pgfqpoint{2.415918in}{1.816937in}}%
\pgfpathlineto{\pgfqpoint{2.431352in}{1.811931in}}%
\pgfpathlineto{\pgfqpoint{2.446785in}{1.804945in}}%
\pgfpathlineto{\pgfqpoint{2.462219in}{1.796030in}}%
\pgfpathlineto{\pgfqpoint{2.477653in}{1.785288in}}%
\pgfpathlineto{\pgfqpoint{2.493086in}{1.772872in}}%
\pgfpathlineto{\pgfqpoint{2.523954in}{1.743833in}}%
\pgfpathlineto{\pgfqpoint{2.554821in}{1.710898in}}%
\pgfpathlineto{\pgfqpoint{2.616555in}{1.642976in}}%
\pgfpathlineto{\pgfqpoint{2.647423in}{1.612945in}}%
\pgfpathlineto{\pgfqpoint{2.662856in}{1.599899in}}%
\pgfpathlineto{\pgfqpoint{2.678290in}{1.588476in}}%
\pgfpathlineto{\pgfqpoint{2.693724in}{1.578864in}}%
\pgfpathlineto{\pgfqpoint{2.709157in}{1.571217in}}%
\pgfpathlineto{\pgfqpoint{2.724591in}{1.565642in}}%
\pgfpathlineto{\pgfqpoint{2.740024in}{1.562205in}}%
\pgfpathlineto{\pgfqpoint{2.755458in}{1.560926in}}%
\pgfpathlineto{\pgfqpoint{2.770892in}{1.561778in}}%
\pgfpathlineto{\pgfqpoint{2.786325in}{1.564690in}}%
\pgfpathlineto{\pgfqpoint{2.801759in}{1.569544in}}%
\pgfpathlineto{\pgfqpoint{2.817193in}{1.576181in}}%
\pgfpathlineto{\pgfqpoint{2.832626in}{1.584401in}}%
\pgfpathlineto{\pgfqpoint{2.863493in}{1.604622in}}%
\pgfpathlineto{\pgfqpoint{2.909794in}{1.640078in}}%
\pgfpathlineto{\pgfqpoint{2.940662in}{1.663429in}}%
\pgfpathlineto{\pgfqpoint{2.971529in}{1.683588in}}%
\pgfpathlineto{\pgfqpoint{2.986963in}{1.691753in}}%
\pgfpathlineto{\pgfqpoint{3.002396in}{1.698316in}}%
\pgfpathlineto{\pgfqpoint{3.017830in}{1.703074in}}%
\pgfpathlineto{\pgfqpoint{3.033263in}{1.705873in}}%
\pgfpathlineto{\pgfqpoint{3.048697in}{1.706598in}}%
\pgfpathlineto{\pgfqpoint{3.064131in}{1.705188in}}%
\pgfpathlineto{\pgfqpoint{3.079564in}{1.701628in}}%
\pgfpathlineto{\pgfqpoint{3.094998in}{1.695955in}}%
\pgfpathlineto{\pgfqpoint{3.110432in}{1.688251in}}%
\pgfpathlineto{\pgfqpoint{3.125865in}{1.678646in}}%
\pgfpathlineto{\pgfqpoint{3.141299in}{1.667311in}}%
\pgfpathlineto{\pgfqpoint{3.172166in}{1.640305in}}%
\pgfpathlineto{\pgfqpoint{3.203033in}{1.609219in}}%
\pgfpathlineto{\pgfqpoint{3.280202in}{1.528460in}}%
\pgfpathlineto{\pgfqpoint{3.311069in}{1.500507in}}%
\pgfpathlineto{\pgfqpoint{3.341936in}{1.477495in}}%
\pgfpathlineto{\pgfqpoint{3.357370in}{1.468154in}}%
\pgfpathlineto{\pgfqpoint{3.372803in}{1.460353in}}%
\pgfpathlineto{\pgfqpoint{3.388237in}{1.454120in}}%
\pgfpathlineto{\pgfqpoint{3.403671in}{1.449456in}}%
\pgfpathlineto{\pgfqpoint{3.419104in}{1.446330in}}%
\pgfpathlineto{\pgfqpoint{3.434538in}{1.444691in}}%
\pgfpathlineto{\pgfqpoint{3.449972in}{1.444468in}}%
\pgfpathlineto{\pgfqpoint{3.465405in}{1.445580in}}%
\pgfpathlineto{\pgfqpoint{3.480839in}{1.447933in}}%
\pgfpathlineto{\pgfqpoint{3.511706in}{1.455990in}}%
\pgfpathlineto{\pgfqpoint{3.542573in}{1.467929in}}%
\pgfpathlineto{\pgfqpoint{3.573441in}{1.483167in}}%
\pgfpathlineto{\pgfqpoint{3.604308in}{1.501312in}}%
\pgfpathlineto{\pgfqpoint{3.635175in}{1.522165in}}%
\pgfpathlineto{\pgfqpoint{3.666043in}{1.545666in}}%
\pgfpathlineto{\pgfqpoint{3.696910in}{1.571819in}}%
\pgfpathlineto{\pgfqpoint{3.727777in}{1.600586in}}%
\pgfpathlineto{\pgfqpoint{3.758644in}{1.631791in}}%
\pgfpathlineto{\pgfqpoint{3.804945in}{1.682229in}}%
\pgfpathlineto{\pgfqpoint{3.897547in}{1.785480in}}%
\pgfpathlineto{\pgfqpoint{3.928414in}{1.816117in}}%
\pgfpathlineto{\pgfqpoint{3.959282in}{1.842595in}}%
\pgfpathlineto{\pgfqpoint{3.974715in}{1.853895in}}%
\pgfpathlineto{\pgfqpoint{3.990149in}{1.863734in}}%
\pgfpathlineto{\pgfqpoint{4.005583in}{1.872017in}}%
\pgfpathlineto{\pgfqpoint{4.021016in}{1.878676in}}%
\pgfpathlineto{\pgfqpoint{4.036450in}{1.883678in}}%
\pgfpathlineto{\pgfqpoint{4.051883in}{1.887020in}}%
\pgfpathlineto{\pgfqpoint{4.067317in}{1.888736in}}%
\pgfpathlineto{\pgfqpoint{4.082751in}{1.888895in}}%
\pgfpathlineto{\pgfqpoint{4.098184in}{1.887600in}}%
\pgfpathlineto{\pgfqpoint{4.113618in}{1.884988in}}%
\pgfpathlineto{\pgfqpoint{4.144485in}{1.876503in}}%
\pgfpathlineto{\pgfqpoint{4.175352in}{1.865058in}}%
\pgfpathlineto{\pgfqpoint{4.237087in}{1.840915in}}%
\pgfpathlineto{\pgfqpoint{4.267954in}{1.832049in}}%
\pgfpathlineto{\pgfqpoint{4.283388in}{1.829158in}}%
\pgfpathlineto{\pgfqpoint{4.298822in}{1.827514in}}%
\pgfpathlineto{\pgfqpoint{4.314255in}{1.827241in}}%
\pgfpathlineto{\pgfqpoint{4.329689in}{1.828428in}}%
\pgfpathlineto{\pgfqpoint{4.345122in}{1.831126in}}%
\pgfpathlineto{\pgfqpoint{4.360556in}{1.835344in}}%
\pgfpathlineto{\pgfqpoint{4.375990in}{1.841054in}}%
\pgfpathlineto{\pgfqpoint{4.391423in}{1.848186in}}%
\pgfpathlineto{\pgfqpoint{4.422291in}{1.866251in}}%
\pgfpathlineto{\pgfqpoint{4.453158in}{1.888288in}}%
\pgfpathlineto{\pgfqpoint{4.561193in}{1.971283in}}%
\pgfpathlineto{\pgfqpoint{4.592061in}{1.989387in}}%
\pgfpathlineto{\pgfqpoint{4.607494in}{1.996663in}}%
\pgfpathlineto{\pgfqpoint{4.622928in}{2.002639in}}%
\pgfpathlineto{\pgfqpoint{4.638362in}{2.007275in}}%
\pgfpathlineto{\pgfqpoint{4.653795in}{2.010564in}}%
\pgfpathlineto{\pgfqpoint{4.669229in}{2.012532in}}%
\pgfpathlineto{\pgfqpoint{4.669229in}{2.012532in}}%
\pgfusepath{stroke}%
\end{pgfscope}%
\begin{pgfscope}%
\pgfpathrectangle{\pgfqpoint{0.634105in}{0.521603in}}{\pgfqpoint{4.227273in}{2.800000in}} %
\pgfusepath{clip}%
\pgfsetrectcap%
\pgfsetroundjoin%
\pgfsetlinewidth{0.501875pt}%
\definecolor{currentstroke}{rgb}{0.000000,0.000000,0.000000}%
\pgfsetstrokecolor{currentstroke}%
\pgfsetdash{}{0pt}%
\pgfpathmoveto{\pgfqpoint{0.826254in}{2.309386in}}%
\pgfpathlineto{\pgfqpoint{0.841687in}{2.308002in}}%
\pgfpathlineto{\pgfqpoint{0.857121in}{2.302788in}}%
\pgfpathlineto{\pgfqpoint{0.872555in}{2.293596in}}%
\pgfpathlineto{\pgfqpoint{0.887988in}{2.280346in}}%
\pgfpathlineto{\pgfqpoint{0.903422in}{2.263029in}}%
\pgfpathlineto{\pgfqpoint{0.918855in}{2.241710in}}%
\pgfpathlineto{\pgfqpoint{0.934289in}{2.216530in}}%
\pgfpathlineto{\pgfqpoint{0.949723in}{2.187708in}}%
\pgfpathlineto{\pgfqpoint{0.965156in}{2.155536in}}%
\pgfpathlineto{\pgfqpoint{0.996024in}{2.082665in}}%
\pgfpathlineto{\pgfqpoint{1.026891in}{2.001582in}}%
\pgfpathlineto{\pgfqpoint{1.104059in}{1.793340in}}%
\pgfpathlineto{\pgfqpoint{1.134926in}{1.720920in}}%
\pgfpathlineto{\pgfqpoint{1.150360in}{1.689410in}}%
\pgfpathlineto{\pgfqpoint{1.165794in}{1.661709in}}%
\pgfpathlineto{\pgfqpoint{1.181227in}{1.638254in}}%
\pgfpathlineto{\pgfqpoint{1.196661in}{1.619409in}}%
\pgfpathlineto{\pgfqpoint{1.212095in}{1.605466in}}%
\pgfpathlineto{\pgfqpoint{1.227528in}{1.596637in}}%
\pgfpathlineto{\pgfqpoint{1.242962in}{1.593050in}}%
\pgfpathlineto{\pgfqpoint{1.258395in}{1.594749in}}%
\pgfpathlineto{\pgfqpoint{1.273829in}{1.601693in}}%
\pgfpathlineto{\pgfqpoint{1.289263in}{1.613757in}}%
\pgfpathlineto{\pgfqpoint{1.304696in}{1.630737in}}%
\pgfpathlineto{\pgfqpoint{1.320130in}{1.652355in}}%
\pgfpathlineto{\pgfqpoint{1.335564in}{1.678262in}}%
\pgfpathlineto{\pgfqpoint{1.350997in}{1.708049in}}%
\pgfpathlineto{\pgfqpoint{1.366431in}{1.741252in}}%
\pgfpathlineto{\pgfqpoint{1.397298in}{1.815843in}}%
\pgfpathlineto{\pgfqpoint{1.443599in}{1.939732in}}%
\pgfpathlineto{\pgfqpoint{1.489900in}{2.064111in}}%
\pgfpathlineto{\pgfqpoint{1.520767in}{2.140017in}}%
\pgfpathlineto{\pgfqpoint{1.536201in}{2.174440in}}%
\pgfpathlineto{\pgfqpoint{1.551634in}{2.205981in}}%
\pgfpathlineto{\pgfqpoint{1.567068in}{2.234309in}}%
\pgfpathlineto{\pgfqpoint{1.582502in}{2.259149in}}%
\pgfpathlineto{\pgfqpoint{1.597935in}{2.280284in}}%
\pgfpathlineto{\pgfqpoint{1.613369in}{2.297558in}}%
\pgfpathlineto{\pgfqpoint{1.628803in}{2.310874in}}%
\pgfpathlineto{\pgfqpoint{1.644236in}{2.320191in}}%
\pgfpathlineto{\pgfqpoint{1.659670in}{2.325524in}}%
\pgfpathlineto{\pgfqpoint{1.675104in}{2.326943in}}%
\pgfpathlineto{\pgfqpoint{1.690537in}{2.324564in}}%
\pgfpathlineto{\pgfqpoint{1.705971in}{2.318549in}}%
\pgfpathlineto{\pgfqpoint{1.721404in}{2.309099in}}%
\pgfpathlineto{\pgfqpoint{1.736838in}{2.296447in}}%
\pgfpathlineto{\pgfqpoint{1.752272in}{2.280856in}}%
\pgfpathlineto{\pgfqpoint{1.767705in}{2.262612in}}%
\pgfpathlineto{\pgfqpoint{1.783139in}{2.242015in}}%
\pgfpathlineto{\pgfqpoint{1.814006in}{2.195023in}}%
\pgfpathlineto{\pgfqpoint{1.844874in}{2.142429in}}%
\pgfpathlineto{\pgfqpoint{1.968343in}{1.923707in}}%
\pgfpathlineto{\pgfqpoint{1.999210in}{1.877076in}}%
\pgfpathlineto{\pgfqpoint{2.030077in}{1.836836in}}%
\pgfpathlineto{\pgfqpoint{2.045511in}{1.819467in}}%
\pgfpathlineto{\pgfqpoint{2.060944in}{1.804078in}}%
\pgfpathlineto{\pgfqpoint{2.076378in}{1.790753in}}%
\pgfpathlineto{\pgfqpoint{2.091812in}{1.779547in}}%
\pgfpathlineto{\pgfqpoint{2.107245in}{1.770490in}}%
\pgfpathlineto{\pgfqpoint{2.122679in}{1.763578in}}%
\pgfpathlineto{\pgfqpoint{2.138113in}{1.758778in}}%
\pgfpathlineto{\pgfqpoint{2.153546in}{1.756020in}}%
\pgfpathlineto{\pgfqpoint{2.168980in}{1.755198in}}%
\pgfpathlineto{\pgfqpoint{2.184414in}{1.756171in}}%
\pgfpathlineto{\pgfqpoint{2.199847in}{1.758762in}}%
\pgfpathlineto{\pgfqpoint{2.215281in}{1.762759in}}%
\pgfpathlineto{\pgfqpoint{2.246148in}{1.773971in}}%
\pgfpathlineto{\pgfqpoint{2.323316in}{1.807018in}}%
\pgfpathlineto{\pgfqpoint{2.338750in}{1.812016in}}%
\pgfpathlineto{\pgfqpoint{2.354184in}{1.815808in}}%
\pgfpathlineto{\pgfqpoint{2.369617in}{1.818159in}}%
\pgfpathlineto{\pgfqpoint{2.385051in}{1.818876in}}%
\pgfpathlineto{\pgfqpoint{2.400484in}{1.817807in}}%
\pgfpathlineto{\pgfqpoint{2.415918in}{1.814852in}}%
\pgfpathlineto{\pgfqpoint{2.431352in}{1.809963in}}%
\pgfpathlineto{\pgfqpoint{2.446785in}{1.803144in}}%
\pgfpathlineto{\pgfqpoint{2.462219in}{1.794453in}}%
\pgfpathlineto{\pgfqpoint{2.477653in}{1.783998in}}%
\pgfpathlineto{\pgfqpoint{2.493086in}{1.771935in}}%
\pgfpathlineto{\pgfqpoint{2.523954in}{1.743809in}}%
\pgfpathlineto{\pgfqpoint{2.554821in}{1.712055in}}%
\pgfpathlineto{\pgfqpoint{2.616555in}{1.647103in}}%
\pgfpathlineto{\pgfqpoint{2.647423in}{1.618684in}}%
\pgfpathlineto{\pgfqpoint{2.662856in}{1.606425in}}%
\pgfpathlineto{\pgfqpoint{2.678290in}{1.595755in}}%
\pgfpathlineto{\pgfqpoint{2.693724in}{1.586850in}}%
\pgfpathlineto{\pgfqpoint{2.709157in}{1.579845in}}%
\pgfpathlineto{\pgfqpoint{2.724591in}{1.574836in}}%
\pgfpathlineto{\pgfqpoint{2.740024in}{1.571879in}}%
\pgfpathlineto{\pgfqpoint{2.755458in}{1.570985in}}%
\pgfpathlineto{\pgfqpoint{2.770892in}{1.572121in}}%
\pgfpathlineto{\pgfqpoint{2.786325in}{1.575212in}}%
\pgfpathlineto{\pgfqpoint{2.801759in}{1.580141in}}%
\pgfpathlineto{\pgfqpoint{2.817193in}{1.586751in}}%
\pgfpathlineto{\pgfqpoint{2.832626in}{1.594848in}}%
\pgfpathlineto{\pgfqpoint{2.863493in}{1.614562in}}%
\pgfpathlineto{\pgfqpoint{2.956095in}{1.681167in}}%
\pgfpathlineto{\pgfqpoint{2.971529in}{1.690158in}}%
\pgfpathlineto{\pgfqpoint{2.986963in}{1.697793in}}%
\pgfpathlineto{\pgfqpoint{3.002396in}{1.703844in}}%
\pgfpathlineto{\pgfqpoint{3.017830in}{1.708124in}}%
\pgfpathlineto{\pgfqpoint{3.033263in}{1.710482in}}%
\pgfpathlineto{\pgfqpoint{3.048697in}{1.710816in}}%
\pgfpathlineto{\pgfqpoint{3.064131in}{1.709071in}}%
\pgfpathlineto{\pgfqpoint{3.079564in}{1.705242in}}%
\pgfpathlineto{\pgfqpoint{3.094998in}{1.699371in}}%
\pgfpathlineto{\pgfqpoint{3.110432in}{1.691545in}}%
\pgfpathlineto{\pgfqpoint{3.125865in}{1.681898in}}%
\pgfpathlineto{\pgfqpoint{3.141299in}{1.670601in}}%
\pgfpathlineto{\pgfqpoint{3.172166in}{1.643903in}}%
\pgfpathlineto{\pgfqpoint{3.203033in}{1.613387in}}%
\pgfpathlineto{\pgfqpoint{3.280202in}{1.534517in}}%
\pgfpathlineto{\pgfqpoint{3.311069in}{1.507173in}}%
\pgfpathlineto{\pgfqpoint{3.341936in}{1.484533in}}%
\pgfpathlineto{\pgfqpoint{3.357370in}{1.475277in}}%
\pgfpathlineto{\pgfqpoint{3.372803in}{1.467495in}}%
\pgfpathlineto{\pgfqpoint{3.388237in}{1.461221in}}%
\pgfpathlineto{\pgfqpoint{3.403671in}{1.456463in}}%
\pgfpathlineto{\pgfqpoint{3.419104in}{1.453202in}}%
\pgfpathlineto{\pgfqpoint{3.434538in}{1.451398in}}%
\pgfpathlineto{\pgfqpoint{3.449972in}{1.450993in}}%
\pgfpathlineto{\pgfqpoint{3.465405in}{1.451915in}}%
\pgfpathlineto{\pgfqpoint{3.480839in}{1.454082in}}%
\pgfpathlineto{\pgfqpoint{3.511706in}{1.461801in}}%
\pgfpathlineto{\pgfqpoint{3.542573in}{1.473466in}}%
\pgfpathlineto{\pgfqpoint{3.573441in}{1.488482in}}%
\pgfpathlineto{\pgfqpoint{3.604308in}{1.506423in}}%
\pgfpathlineto{\pgfqpoint{3.635175in}{1.527058in}}%
\pgfpathlineto{\pgfqpoint{3.666043in}{1.550322in}}%
\pgfpathlineto{\pgfqpoint{3.696910in}{1.576245in}}%
\pgfpathlineto{\pgfqpoint{3.727777in}{1.604839in}}%
\pgfpathlineto{\pgfqpoint{3.758644in}{1.635983in}}%
\pgfpathlineto{\pgfqpoint{3.804945in}{1.686603in}}%
\pgfpathlineto{\pgfqpoint{3.897547in}{1.790737in}}%
\pgfpathlineto{\pgfqpoint{3.928414in}{1.821526in}}%
\pgfpathlineto{\pgfqpoint{3.959282in}{1.847968in}}%
\pgfpathlineto{\pgfqpoint{3.974715in}{1.859172in}}%
\pgfpathlineto{\pgfqpoint{3.990149in}{1.868864in}}%
\pgfpathlineto{\pgfqpoint{4.005583in}{1.876951in}}%
\pgfpathlineto{\pgfqpoint{4.021016in}{1.883374in}}%
\pgfpathlineto{\pgfqpoint{4.036450in}{1.888100in}}%
\pgfpathlineto{\pgfqpoint{4.051883in}{1.891136in}}%
\pgfpathlineto{\pgfqpoint{4.067317in}{1.892520in}}%
\pgfpathlineto{\pgfqpoint{4.082751in}{1.892325in}}%
\pgfpathlineto{\pgfqpoint{4.098184in}{1.890658in}}%
\pgfpathlineto{\pgfqpoint{4.113618in}{1.887657in}}%
\pgfpathlineto{\pgfqpoint{4.144485in}{1.878342in}}%
\pgfpathlineto{\pgfqpoint{4.175352in}{1.865992in}}%
\pgfpathlineto{\pgfqpoint{4.237087in}{1.839791in}}%
\pgfpathlineto{\pgfqpoint{4.267954in}{1.829814in}}%
\pgfpathlineto{\pgfqpoint{4.283388in}{1.826369in}}%
\pgfpathlineto{\pgfqpoint{4.298822in}{1.824186in}}%
\pgfpathlineto{\pgfqpoint{4.314255in}{1.823399in}}%
\pgfpathlineto{\pgfqpoint{4.329689in}{1.824110in}}%
\pgfpathlineto{\pgfqpoint{4.345122in}{1.826379in}}%
\pgfpathlineto{\pgfqpoint{4.360556in}{1.830227in}}%
\pgfpathlineto{\pgfqpoint{4.375990in}{1.835633in}}%
\pgfpathlineto{\pgfqpoint{4.391423in}{1.842535in}}%
\pgfpathlineto{\pgfqpoint{4.422291in}{1.860377in}}%
\pgfpathlineto{\pgfqpoint{4.453158in}{1.882514in}}%
\pgfpathlineto{\pgfqpoint{4.499459in}{1.920007in}}%
\pgfpathlineto{\pgfqpoint{4.545760in}{1.956805in}}%
\pgfpathlineto{\pgfqpoint{4.576627in}{1.977898in}}%
\pgfpathlineto{\pgfqpoint{4.607494in}{1.994594in}}%
\pgfpathlineto{\pgfqpoint{4.622928in}{2.001009in}}%
\pgfpathlineto{\pgfqpoint{4.638362in}{2.006053in}}%
\pgfpathlineto{\pgfqpoint{4.653795in}{2.009712in}}%
\pgfpathlineto{\pgfqpoint{4.669229in}{2.012003in}}%
\pgfpathlineto{\pgfqpoint{4.669229in}{2.012003in}}%
\pgfusepath{stroke}%
\end{pgfscope}%
\begin{pgfscope}%
\pgfsetrectcap%
\pgfsetmiterjoin%
\pgfsetlinewidth{0.803000pt}%
\definecolor{currentstroke}{rgb}{0.000000,0.000000,0.000000}%
\pgfsetstrokecolor{currentstroke}%
\pgfsetdash{}{0pt}%
\pgfpathmoveto{\pgfqpoint{0.634105in}{0.521603in}}%
\pgfpathlineto{\pgfqpoint{0.634105in}{3.321603in}}%
\pgfusepath{stroke}%
\end{pgfscope}%
\begin{pgfscope}%
\pgfsetrectcap%
\pgfsetmiterjoin%
\pgfsetlinewidth{0.803000pt}%
\definecolor{currentstroke}{rgb}{0.000000,0.000000,0.000000}%
\pgfsetstrokecolor{currentstroke}%
\pgfsetdash{}{0pt}%
\pgfpathmoveto{\pgfqpoint{4.861378in}{0.521603in}}%
\pgfpathlineto{\pgfqpoint{4.861378in}{3.321603in}}%
\pgfusepath{stroke}%
\end{pgfscope}%
\begin{pgfscope}%
\pgfsetrectcap%
\pgfsetmiterjoin%
\pgfsetlinewidth{0.803000pt}%
\definecolor{currentstroke}{rgb}{0.000000,0.000000,0.000000}%
\pgfsetstrokecolor{currentstroke}%
\pgfsetdash{}{0pt}%
\pgfpathmoveto{\pgfqpoint{0.634105in}{0.521603in}}%
\pgfpathlineto{\pgfqpoint{4.861378in}{0.521603in}}%
\pgfusepath{stroke}%
\end{pgfscope}%
\begin{pgfscope}%
\pgfsetrectcap%
\pgfsetmiterjoin%
\pgfsetlinewidth{0.803000pt}%
\definecolor{currentstroke}{rgb}{0.000000,0.000000,0.000000}%
\pgfsetstrokecolor{currentstroke}%
\pgfsetdash{}{0pt}%
\pgfpathmoveto{\pgfqpoint{0.634105in}{3.321603in}}%
\pgfpathlineto{\pgfqpoint{4.861378in}{3.321603in}}%
\pgfusepath{stroke}%
\end{pgfscope}%
\begin{pgfscope}%
\pgftext[x=5.284105in,y=3.404937in,,base]{\rmfamily\fontsize{12.000000}{14.400000}\selectfont \(\displaystyle \widetilde{K}u \approx Ku\), realization 2}%
\end{pgfscope}%
\begin{pgfscope}%
\pgfsetbuttcap%
\pgfsetmiterjoin%
\definecolor{currentfill}{rgb}{1.000000,1.000000,1.000000}%
\pgfsetfillcolor{currentfill}%
\pgfsetfillopacity{0.800000}%
\pgfsetlinewidth{1.003750pt}%
\definecolor{currentstroke}{rgb}{0.800000,0.800000,0.800000}%
\pgfsetstrokecolor{currentstroke}%
\pgfsetstrokeopacity{0.800000}%
\pgfsetdash{}{0pt}%
\pgfpathmoveto{\pgfqpoint{4.215951in}{0.591048in}}%
\pgfpathlineto{\pgfqpoint{4.764155in}{0.591048in}}%
\pgfpathquadraticcurveto{\pgfqpoint{4.791933in}{0.591048in}}{\pgfqpoint{4.791933in}{0.618826in}}%
\pgfpathlineto{\pgfqpoint{4.791933in}{0.808794in}}%
\pgfpathquadraticcurveto{\pgfqpoint{4.791933in}{0.836572in}}{\pgfqpoint{4.764155in}{0.836572in}}%
\pgfpathlineto{\pgfqpoint{4.215951in}{0.836572in}}%
\pgfpathquadraticcurveto{\pgfqpoint{4.188173in}{0.836572in}}{\pgfqpoint{4.188173in}{0.808794in}}%
\pgfpathlineto{\pgfqpoint{4.188173in}{0.618826in}}%
\pgfpathquadraticcurveto{\pgfqpoint{4.188173in}{0.591048in}}{\pgfqpoint{4.215951in}{0.591048in}}%
\pgfpathclose%
\pgfusepath{stroke,fill}%
\end{pgfscope}%
\begin{pgfscope}%
\pgfsetrectcap%
\pgfsetroundjoin%
\pgfsetlinewidth{0.501875pt}%
\definecolor{currentstroke}{rgb}{0.000000,0.000000,0.000000}%
\pgfsetstrokecolor{currentstroke}%
\pgfsetdash{}{0pt}%
\pgfpathmoveto{\pgfqpoint{4.243729in}{0.724104in}}%
\pgfpathlineto{\pgfqpoint{4.521507in}{0.724104in}}%
\pgfusepath{stroke}%
\end{pgfscope}%
\begin{pgfscope}%
\pgftext[x=4.632618in,y=0.675493in,left,base]{\rmfamily\fontsize{10.000000}{12.000000}\selectfont K}%
\end{pgfscope}%
\begin{pgfscope}%
\pgfsetbuttcap%
\pgfsetmiterjoin%
\definecolor{currentfill}{rgb}{1.000000,1.000000,1.000000}%
\pgfsetfillcolor{currentfill}%
\pgfsetlinewidth{0.000000pt}%
\definecolor{currentstroke}{rgb}{0.000000,0.000000,0.000000}%
\pgfsetstrokecolor{currentstroke}%
\pgfsetstrokeopacity{0.000000}%
\pgfsetdash{}{0pt}%
\pgfpathmoveto{\pgfqpoint{5.706832in}{0.521603in}}%
\pgfpathlineto{\pgfqpoint{9.934105in}{0.521603in}}%
\pgfpathlineto{\pgfqpoint{9.934105in}{3.321603in}}%
\pgfpathlineto{\pgfqpoint{5.706832in}{3.321603in}}%
\pgfpathclose%
\pgfusepath{fill}%
\end{pgfscope}%
\begin{pgfscope}%
\pgfsetbuttcap%
\pgfsetroundjoin%
\definecolor{currentfill}{rgb}{0.000000,0.000000,0.000000}%
\pgfsetfillcolor{currentfill}%
\pgfsetlinewidth{0.803000pt}%
\definecolor{currentstroke}{rgb}{0.000000,0.000000,0.000000}%
\pgfsetstrokecolor{currentstroke}%
\pgfsetdash{}{0pt}%
\pgfsys@defobject{currentmarker}{\pgfqpoint{0.000000in}{-0.048611in}}{\pgfqpoint{0.000000in}{0.000000in}}{%
\pgfpathmoveto{\pgfqpoint{0.000000in}{0.000000in}}%
\pgfpathlineto{\pgfqpoint{0.000000in}{-0.048611in}}%
\pgfusepath{stroke,fill}%
}%
\begin{pgfscope}%
\pgfsys@transformshift{5.898981in}{0.521603in}%
\pgfsys@useobject{currentmarker}{}%
\end{pgfscope}%
\end{pgfscope}%
\begin{pgfscope}%
\pgftext[x=5.898981in,y=0.424381in,,top]{\rmfamily\fontsize{10.000000}{12.000000}\selectfont \(\displaystyle -1.00\)}%
\end{pgfscope}%
\begin{pgfscope}%
\pgfsetbuttcap%
\pgfsetroundjoin%
\definecolor{currentfill}{rgb}{0.000000,0.000000,0.000000}%
\pgfsetfillcolor{currentfill}%
\pgfsetlinewidth{0.803000pt}%
\definecolor{currentstroke}{rgb}{0.000000,0.000000,0.000000}%
\pgfsetstrokecolor{currentstroke}%
\pgfsetdash{}{0pt}%
\pgfsys@defobject{currentmarker}{\pgfqpoint{0.000000in}{-0.048611in}}{\pgfqpoint{0.000000in}{0.000000in}}{%
\pgfpathmoveto{\pgfqpoint{0.000000in}{0.000000in}}%
\pgfpathlineto{\pgfqpoint{0.000000in}{-0.048611in}}%
\pgfusepath{stroke,fill}%
}%
\begin{pgfscope}%
\pgfsys@transformshift{6.379353in}{0.521603in}%
\pgfsys@useobject{currentmarker}{}%
\end{pgfscope}%
\end{pgfscope}%
\begin{pgfscope}%
\pgftext[x=6.379353in,y=0.424381in,,top]{\rmfamily\fontsize{10.000000}{12.000000}\selectfont \(\displaystyle -0.75\)}%
\end{pgfscope}%
\begin{pgfscope}%
\pgfsetbuttcap%
\pgfsetroundjoin%
\definecolor{currentfill}{rgb}{0.000000,0.000000,0.000000}%
\pgfsetfillcolor{currentfill}%
\pgfsetlinewidth{0.803000pt}%
\definecolor{currentstroke}{rgb}{0.000000,0.000000,0.000000}%
\pgfsetstrokecolor{currentstroke}%
\pgfsetdash{}{0pt}%
\pgfsys@defobject{currentmarker}{\pgfqpoint{0.000000in}{-0.048611in}}{\pgfqpoint{0.000000in}{0.000000in}}{%
\pgfpathmoveto{\pgfqpoint{0.000000in}{0.000000in}}%
\pgfpathlineto{\pgfqpoint{0.000000in}{-0.048611in}}%
\pgfusepath{stroke,fill}%
}%
\begin{pgfscope}%
\pgfsys@transformshift{6.859725in}{0.521603in}%
\pgfsys@useobject{currentmarker}{}%
\end{pgfscope}%
\end{pgfscope}%
\begin{pgfscope}%
\pgftext[x=6.859725in,y=0.424381in,,top]{\rmfamily\fontsize{10.000000}{12.000000}\selectfont \(\displaystyle -0.50\)}%
\end{pgfscope}%
\begin{pgfscope}%
\pgfsetbuttcap%
\pgfsetroundjoin%
\definecolor{currentfill}{rgb}{0.000000,0.000000,0.000000}%
\pgfsetfillcolor{currentfill}%
\pgfsetlinewidth{0.803000pt}%
\definecolor{currentstroke}{rgb}{0.000000,0.000000,0.000000}%
\pgfsetstrokecolor{currentstroke}%
\pgfsetdash{}{0pt}%
\pgfsys@defobject{currentmarker}{\pgfqpoint{0.000000in}{-0.048611in}}{\pgfqpoint{0.000000in}{0.000000in}}{%
\pgfpathmoveto{\pgfqpoint{0.000000in}{0.000000in}}%
\pgfpathlineto{\pgfqpoint{0.000000in}{-0.048611in}}%
\pgfusepath{stroke,fill}%
}%
\begin{pgfscope}%
\pgfsys@transformshift{7.340097in}{0.521603in}%
\pgfsys@useobject{currentmarker}{}%
\end{pgfscope}%
\end{pgfscope}%
\begin{pgfscope}%
\pgftext[x=7.340097in,y=0.424381in,,top]{\rmfamily\fontsize{10.000000}{12.000000}\selectfont \(\displaystyle -0.25\)}%
\end{pgfscope}%
\begin{pgfscope}%
\pgfsetbuttcap%
\pgfsetroundjoin%
\definecolor{currentfill}{rgb}{0.000000,0.000000,0.000000}%
\pgfsetfillcolor{currentfill}%
\pgfsetlinewidth{0.803000pt}%
\definecolor{currentstroke}{rgb}{0.000000,0.000000,0.000000}%
\pgfsetstrokecolor{currentstroke}%
\pgfsetdash{}{0pt}%
\pgfsys@defobject{currentmarker}{\pgfqpoint{0.000000in}{-0.048611in}}{\pgfqpoint{0.000000in}{0.000000in}}{%
\pgfpathmoveto{\pgfqpoint{0.000000in}{0.000000in}}%
\pgfpathlineto{\pgfqpoint{0.000000in}{-0.048611in}}%
\pgfusepath{stroke,fill}%
}%
\begin{pgfscope}%
\pgfsys@transformshift{7.820468in}{0.521603in}%
\pgfsys@useobject{currentmarker}{}%
\end{pgfscope}%
\end{pgfscope}%
\begin{pgfscope}%
\pgftext[x=7.820468in,y=0.424381in,,top]{\rmfamily\fontsize{10.000000}{12.000000}\selectfont \(\displaystyle 0.00\)}%
\end{pgfscope}%
\begin{pgfscope}%
\pgfsetbuttcap%
\pgfsetroundjoin%
\definecolor{currentfill}{rgb}{0.000000,0.000000,0.000000}%
\pgfsetfillcolor{currentfill}%
\pgfsetlinewidth{0.803000pt}%
\definecolor{currentstroke}{rgb}{0.000000,0.000000,0.000000}%
\pgfsetstrokecolor{currentstroke}%
\pgfsetdash{}{0pt}%
\pgfsys@defobject{currentmarker}{\pgfqpoint{0.000000in}{-0.048611in}}{\pgfqpoint{0.000000in}{0.000000in}}{%
\pgfpathmoveto{\pgfqpoint{0.000000in}{0.000000in}}%
\pgfpathlineto{\pgfqpoint{0.000000in}{-0.048611in}}%
\pgfusepath{stroke,fill}%
}%
\begin{pgfscope}%
\pgfsys@transformshift{8.300840in}{0.521603in}%
\pgfsys@useobject{currentmarker}{}%
\end{pgfscope}%
\end{pgfscope}%
\begin{pgfscope}%
\pgftext[x=8.300840in,y=0.424381in,,top]{\rmfamily\fontsize{10.000000}{12.000000}\selectfont \(\displaystyle 0.25\)}%
\end{pgfscope}%
\begin{pgfscope}%
\pgfsetbuttcap%
\pgfsetroundjoin%
\definecolor{currentfill}{rgb}{0.000000,0.000000,0.000000}%
\pgfsetfillcolor{currentfill}%
\pgfsetlinewidth{0.803000pt}%
\definecolor{currentstroke}{rgb}{0.000000,0.000000,0.000000}%
\pgfsetstrokecolor{currentstroke}%
\pgfsetdash{}{0pt}%
\pgfsys@defobject{currentmarker}{\pgfqpoint{0.000000in}{-0.048611in}}{\pgfqpoint{0.000000in}{0.000000in}}{%
\pgfpathmoveto{\pgfqpoint{0.000000in}{0.000000in}}%
\pgfpathlineto{\pgfqpoint{0.000000in}{-0.048611in}}%
\pgfusepath{stroke,fill}%
}%
\begin{pgfscope}%
\pgfsys@transformshift{8.781212in}{0.521603in}%
\pgfsys@useobject{currentmarker}{}%
\end{pgfscope}%
\end{pgfscope}%
\begin{pgfscope}%
\pgftext[x=8.781212in,y=0.424381in,,top]{\rmfamily\fontsize{10.000000}{12.000000}\selectfont \(\displaystyle 0.50\)}%
\end{pgfscope}%
\begin{pgfscope}%
\pgfsetbuttcap%
\pgfsetroundjoin%
\definecolor{currentfill}{rgb}{0.000000,0.000000,0.000000}%
\pgfsetfillcolor{currentfill}%
\pgfsetlinewidth{0.803000pt}%
\definecolor{currentstroke}{rgb}{0.000000,0.000000,0.000000}%
\pgfsetstrokecolor{currentstroke}%
\pgfsetdash{}{0pt}%
\pgfsys@defobject{currentmarker}{\pgfqpoint{0.000000in}{-0.048611in}}{\pgfqpoint{0.000000in}{0.000000in}}{%
\pgfpathmoveto{\pgfqpoint{0.000000in}{0.000000in}}%
\pgfpathlineto{\pgfqpoint{0.000000in}{-0.048611in}}%
\pgfusepath{stroke,fill}%
}%
\begin{pgfscope}%
\pgfsys@transformshift{9.261584in}{0.521603in}%
\pgfsys@useobject{currentmarker}{}%
\end{pgfscope}%
\end{pgfscope}%
\begin{pgfscope}%
\pgftext[x=9.261584in,y=0.424381in,,top]{\rmfamily\fontsize{10.000000}{12.000000}\selectfont \(\displaystyle 0.75\)}%
\end{pgfscope}%
\begin{pgfscope}%
\pgfsetbuttcap%
\pgfsetroundjoin%
\definecolor{currentfill}{rgb}{0.000000,0.000000,0.000000}%
\pgfsetfillcolor{currentfill}%
\pgfsetlinewidth{0.803000pt}%
\definecolor{currentstroke}{rgb}{0.000000,0.000000,0.000000}%
\pgfsetstrokecolor{currentstroke}%
\pgfsetdash{}{0pt}%
\pgfsys@defobject{currentmarker}{\pgfqpoint{0.000000in}{-0.048611in}}{\pgfqpoint{0.000000in}{0.000000in}}{%
\pgfpathmoveto{\pgfqpoint{0.000000in}{0.000000in}}%
\pgfpathlineto{\pgfqpoint{0.000000in}{-0.048611in}}%
\pgfusepath{stroke,fill}%
}%
\begin{pgfscope}%
\pgfsys@transformshift{9.741956in}{0.521603in}%
\pgfsys@useobject{currentmarker}{}%
\end{pgfscope}%
\end{pgfscope}%
\begin{pgfscope}%
\pgftext[x=9.741956in,y=0.424381in,,top]{\rmfamily\fontsize{10.000000}{12.000000}\selectfont \(\displaystyle 1.00\)}%
\end{pgfscope}%
\begin{pgfscope}%
\pgftext[x=7.820468in,y=0.234413in,,top]{\rmfamily\fontsize{10.000000}{12.000000}\selectfont \(\displaystyle x\)}%
\end{pgfscope}%
\begin{pgfscope}%
\pgfsetbuttcap%
\pgfsetroundjoin%
\definecolor{currentfill}{rgb}{0.000000,0.000000,0.000000}%
\pgfsetfillcolor{currentfill}%
\pgfsetlinewidth{0.803000pt}%
\definecolor{currentstroke}{rgb}{0.000000,0.000000,0.000000}%
\pgfsetstrokecolor{currentstroke}%
\pgfsetdash{}{0pt}%
\pgfsys@defobject{currentmarker}{\pgfqpoint{-0.048611in}{0.000000in}}{\pgfqpoint{0.000000in}{0.000000in}}{%
\pgfpathmoveto{\pgfqpoint{0.000000in}{0.000000in}}%
\pgfpathlineto{\pgfqpoint{-0.048611in}{0.000000in}}%
\pgfusepath{stroke,fill}%
}%
\begin{pgfscope}%
\pgfsys@transformshift{5.706832in}{0.975954in}%
\pgfsys@useobject{currentmarker}{}%
\end{pgfscope}%
\end{pgfscope}%
\begin{pgfscope}%
\pgfsetbuttcap%
\pgfsetroundjoin%
\definecolor{currentfill}{rgb}{0.000000,0.000000,0.000000}%
\pgfsetfillcolor{currentfill}%
\pgfsetlinewidth{0.803000pt}%
\definecolor{currentstroke}{rgb}{0.000000,0.000000,0.000000}%
\pgfsetstrokecolor{currentstroke}%
\pgfsetdash{}{0pt}%
\pgfsys@defobject{currentmarker}{\pgfqpoint{-0.048611in}{0.000000in}}{\pgfqpoint{0.000000in}{0.000000in}}{%
\pgfpathmoveto{\pgfqpoint{0.000000in}{0.000000in}}%
\pgfpathlineto{\pgfqpoint{-0.048611in}{0.000000in}}%
\pgfusepath{stroke,fill}%
}%
\begin{pgfscope}%
\pgfsys@transformshift{5.706832in}{1.445404in}%
\pgfsys@useobject{currentmarker}{}%
\end{pgfscope}%
\end{pgfscope}%
\begin{pgfscope}%
\pgfsetbuttcap%
\pgfsetroundjoin%
\definecolor{currentfill}{rgb}{0.000000,0.000000,0.000000}%
\pgfsetfillcolor{currentfill}%
\pgfsetlinewidth{0.803000pt}%
\definecolor{currentstroke}{rgb}{0.000000,0.000000,0.000000}%
\pgfsetstrokecolor{currentstroke}%
\pgfsetdash{}{0pt}%
\pgfsys@defobject{currentmarker}{\pgfqpoint{-0.048611in}{0.000000in}}{\pgfqpoint{0.000000in}{0.000000in}}{%
\pgfpathmoveto{\pgfqpoint{0.000000in}{0.000000in}}%
\pgfpathlineto{\pgfqpoint{-0.048611in}{0.000000in}}%
\pgfusepath{stroke,fill}%
}%
\begin{pgfscope}%
\pgfsys@transformshift{5.706832in}{1.914853in}%
\pgfsys@useobject{currentmarker}{}%
\end{pgfscope}%
\end{pgfscope}%
\begin{pgfscope}%
\pgfsetbuttcap%
\pgfsetroundjoin%
\definecolor{currentfill}{rgb}{0.000000,0.000000,0.000000}%
\pgfsetfillcolor{currentfill}%
\pgfsetlinewidth{0.803000pt}%
\definecolor{currentstroke}{rgb}{0.000000,0.000000,0.000000}%
\pgfsetstrokecolor{currentstroke}%
\pgfsetdash{}{0pt}%
\pgfsys@defobject{currentmarker}{\pgfqpoint{-0.048611in}{0.000000in}}{\pgfqpoint{0.000000in}{0.000000in}}{%
\pgfpathmoveto{\pgfqpoint{0.000000in}{0.000000in}}%
\pgfpathlineto{\pgfqpoint{-0.048611in}{0.000000in}}%
\pgfusepath{stroke,fill}%
}%
\begin{pgfscope}%
\pgfsys@transformshift{5.706832in}{2.384302in}%
\pgfsys@useobject{currentmarker}{}%
\end{pgfscope}%
\end{pgfscope}%
\begin{pgfscope}%
\pgfsetbuttcap%
\pgfsetroundjoin%
\definecolor{currentfill}{rgb}{0.000000,0.000000,0.000000}%
\pgfsetfillcolor{currentfill}%
\pgfsetlinewidth{0.803000pt}%
\definecolor{currentstroke}{rgb}{0.000000,0.000000,0.000000}%
\pgfsetstrokecolor{currentstroke}%
\pgfsetdash{}{0pt}%
\pgfsys@defobject{currentmarker}{\pgfqpoint{-0.048611in}{0.000000in}}{\pgfqpoint{0.000000in}{0.000000in}}{%
\pgfpathmoveto{\pgfqpoint{0.000000in}{0.000000in}}%
\pgfpathlineto{\pgfqpoint{-0.048611in}{0.000000in}}%
\pgfusepath{stroke,fill}%
}%
\begin{pgfscope}%
\pgfsys@transformshift{5.706832in}{2.853751in}%
\pgfsys@useobject{currentmarker}{}%
\end{pgfscope}%
\end{pgfscope}%
\begin{pgfscope}%
\pgfsetbuttcap%
\pgfsetroundjoin%
\definecolor{currentfill}{rgb}{0.000000,0.000000,0.000000}%
\pgfsetfillcolor{currentfill}%
\pgfsetlinewidth{0.803000pt}%
\definecolor{currentstroke}{rgb}{0.000000,0.000000,0.000000}%
\pgfsetstrokecolor{currentstroke}%
\pgfsetdash{}{0pt}%
\pgfsys@defobject{currentmarker}{\pgfqpoint{-0.048611in}{0.000000in}}{\pgfqpoint{0.000000in}{0.000000in}}{%
\pgfpathmoveto{\pgfqpoint{0.000000in}{0.000000in}}%
\pgfpathlineto{\pgfqpoint{-0.048611in}{0.000000in}}%
\pgfusepath{stroke,fill}%
}%
\begin{pgfscope}%
\pgfsys@transformshift{5.706832in}{3.323200in}%
\pgfsys@useobject{currentmarker}{}%
\end{pgfscope}%
\end{pgfscope}%
\begin{pgfscope}%
\pgftext[x=5.651277in,y=1.921603in,,bottom,rotate=90.000000]{\rmfamily\fontsize{10.000000}{12.000000}\selectfont \(\displaystyle y_2\)}%
\end{pgfscope}%
\begin{pgfscope}%
\pgfpathrectangle{\pgfqpoint{5.706832in}{0.521603in}}{\pgfqpoint{4.227273in}{2.800000in}} %
\pgfusepath{clip}%
\pgfsetrectcap%
\pgfsetroundjoin%
\pgfsetlinewidth{0.501875pt}%
\definecolor{currentstroke}{rgb}{0.500000,0.000000,1.000000}%
\pgfsetstrokecolor{currentstroke}%
\pgfsetdash{}{0pt}%
\pgfpathmoveto{\pgfqpoint{5.898981in}{1.713883in}}%
\pgfpathlineto{\pgfqpoint{5.991583in}{1.880761in}}%
\pgfpathlineto{\pgfqpoint{6.130485in}{2.131496in}}%
\pgfpathlineto{\pgfqpoint{6.192220in}{2.238359in}}%
\pgfpathlineto{\pgfqpoint{6.238521in}{2.315095in}}%
\pgfpathlineto{\pgfqpoint{6.284822in}{2.388112in}}%
\pgfpathlineto{\pgfqpoint{6.331123in}{2.456733in}}%
\pgfpathlineto{\pgfqpoint{6.377424in}{2.520320in}}%
\pgfpathlineto{\pgfqpoint{6.408291in}{2.559621in}}%
\pgfpathlineto{\pgfqpoint{6.439158in}{2.596258in}}%
\pgfpathlineto{\pgfqpoint{6.470025in}{2.630080in}}%
\pgfpathlineto{\pgfqpoint{6.500893in}{2.660948in}}%
\pgfpathlineto{\pgfqpoint{6.531760in}{2.688734in}}%
\pgfpathlineto{\pgfqpoint{6.562627in}{2.713323in}}%
\pgfpathlineto{\pgfqpoint{6.593494in}{2.734614in}}%
\pgfpathlineto{\pgfqpoint{6.624362in}{2.752519in}}%
\pgfpathlineto{\pgfqpoint{6.655229in}{2.766963in}}%
\pgfpathlineto{\pgfqpoint{6.686096in}{2.777888in}}%
\pgfpathlineto{\pgfqpoint{6.716964in}{2.785248in}}%
\pgfpathlineto{\pgfqpoint{6.747831in}{2.789012in}}%
\pgfpathlineto{\pgfqpoint{6.778698in}{2.789165in}}%
\pgfpathlineto{\pgfqpoint{6.809565in}{2.785706in}}%
\pgfpathlineto{\pgfqpoint{6.840433in}{2.778651in}}%
\pgfpathlineto{\pgfqpoint{6.871300in}{2.768027in}}%
\pgfpathlineto{\pgfqpoint{6.902167in}{2.753879in}}%
\pgfpathlineto{\pgfqpoint{6.933034in}{2.736265in}}%
\pgfpathlineto{\pgfqpoint{6.963902in}{2.715258in}}%
\pgfpathlineto{\pgfqpoint{6.994769in}{2.690944in}}%
\pgfpathlineto{\pgfqpoint{7.025636in}{2.663425in}}%
\pgfpathlineto{\pgfqpoint{7.056504in}{2.632813in}}%
\pgfpathlineto{\pgfqpoint{7.087371in}{2.599236in}}%
\pgfpathlineto{\pgfqpoint{7.118238in}{2.562832in}}%
\pgfpathlineto{\pgfqpoint{7.149105in}{2.523751in}}%
\pgfpathlineto{\pgfqpoint{7.195406in}{2.460467in}}%
\pgfpathlineto{\pgfqpoint{7.241707in}{2.392114in}}%
\pgfpathlineto{\pgfqpoint{7.288008in}{2.319327in}}%
\pgfpathlineto{\pgfqpoint{7.334309in}{2.242782in}}%
\pgfpathlineto{\pgfqpoint{7.396044in}{2.136111in}}%
\pgfpathlineto{\pgfqpoint{7.473212in}{1.997848in}}%
\pgfpathlineto{\pgfqpoint{7.642982in}{1.691235in}}%
\pgfpathlineto{\pgfqpoint{7.704716in}{1.584662in}}%
\pgfpathlineto{\pgfqpoint{7.751017in}{1.508217in}}%
\pgfpathlineto{\pgfqpoint{7.797318in}{1.435549in}}%
\pgfpathlineto{\pgfqpoint{7.843619in}{1.367334in}}%
\pgfpathlineto{\pgfqpoint{7.889920in}{1.304205in}}%
\pgfpathlineto{\pgfqpoint{7.920787in}{1.265237in}}%
\pgfpathlineto{\pgfqpoint{7.951654in}{1.228952in}}%
\pgfpathlineto{\pgfqpoint{7.982522in}{1.195501in}}%
\pgfpathlineto{\pgfqpoint{8.013389in}{1.165021in}}%
\pgfpathlineto{\pgfqpoint{8.044256in}{1.137639in}}%
\pgfpathlineto{\pgfqpoint{8.075123in}{1.113467in}}%
\pgfpathlineto{\pgfqpoint{8.105991in}{1.092606in}}%
\pgfpathlineto{\pgfqpoint{8.136858in}{1.075141in}}%
\pgfpathlineto{\pgfqpoint{8.167725in}{1.061144in}}%
\pgfpathlineto{\pgfqpoint{8.198593in}{1.050674in}}%
\pgfpathlineto{\pgfqpoint{8.229460in}{1.043774in}}%
\pgfpathlineto{\pgfqpoint{8.260327in}{1.040472in}}%
\pgfpathlineto{\pgfqpoint{8.291194in}{1.040782in}}%
\pgfpathlineto{\pgfqpoint{8.322062in}{1.044703in}}%
\pgfpathlineto{\pgfqpoint{8.352929in}{1.052218in}}%
\pgfpathlineto{\pgfqpoint{8.383796in}{1.063296in}}%
\pgfpathlineto{\pgfqpoint{8.414663in}{1.077892in}}%
\pgfpathlineto{\pgfqpoint{8.445531in}{1.095946in}}%
\pgfpathlineto{\pgfqpoint{8.476398in}{1.117382in}}%
\pgfpathlineto{\pgfqpoint{8.507265in}{1.142112in}}%
\pgfpathlineto{\pgfqpoint{8.538133in}{1.170034in}}%
\pgfpathlineto{\pgfqpoint{8.569000in}{1.201033in}}%
\pgfpathlineto{\pgfqpoint{8.599867in}{1.234981in}}%
\pgfpathlineto{\pgfqpoint{8.630734in}{1.271737in}}%
\pgfpathlineto{\pgfqpoint{8.661602in}{1.311149in}}%
\pgfpathlineto{\pgfqpoint{8.707903in}{1.374890in}}%
\pgfpathlineto{\pgfqpoint{8.754203in}{1.443647in}}%
\pgfpathlineto{\pgfqpoint{8.800504in}{1.516782in}}%
\pgfpathlineto{\pgfqpoint{8.846805in}{1.593616in}}%
\pgfpathlineto{\pgfqpoint{8.908540in}{1.700575in}}%
\pgfpathlineto{\pgfqpoint{8.985708in}{1.839034in}}%
\pgfpathlineto{\pgfqpoint{9.155478in}{2.145431in}}%
\pgfpathlineto{\pgfqpoint{9.217212in}{2.251706in}}%
\pgfpathlineto{\pgfqpoint{9.263513in}{2.327856in}}%
\pgfpathlineto{\pgfqpoint{9.309814in}{2.400169in}}%
\pgfpathlineto{\pgfqpoint{9.356115in}{2.467974in}}%
\pgfpathlineto{\pgfqpoint{9.402416in}{2.530640in}}%
\pgfpathlineto{\pgfqpoint{9.433283in}{2.569272in}}%
\pgfpathlineto{\pgfqpoint{9.464151in}{2.605202in}}%
\pgfpathlineto{\pgfqpoint{9.495018in}{2.638280in}}%
\pgfpathlineto{\pgfqpoint{9.525885in}{2.668369in}}%
\pgfpathlineto{\pgfqpoint{9.556752in}{2.695346in}}%
\pgfpathlineto{\pgfqpoint{9.587620in}{2.719099in}}%
\pgfpathlineto{\pgfqpoint{9.618487in}{2.739530in}}%
\pgfpathlineto{\pgfqpoint{9.649354in}{2.756554in}}%
\pgfpathlineto{\pgfqpoint{9.680222in}{2.770102in}}%
\pgfpathlineto{\pgfqpoint{9.711089in}{2.780116in}}%
\pgfpathlineto{\pgfqpoint{9.741956in}{2.786556in}}%
\pgfpathlineto{\pgfqpoint{9.741956in}{2.786556in}}%
\pgfusepath{stroke}%
\end{pgfscope}%
\begin{pgfscope}%
\pgfpathrectangle{\pgfqpoint{5.706832in}{0.521603in}}{\pgfqpoint{4.227273in}{2.800000in}} %
\pgfusepath{clip}%
\pgfsetrectcap%
\pgfsetroundjoin%
\pgfsetlinewidth{0.501875pt}%
\definecolor{currentstroke}{rgb}{0.421569,0.122888,0.998103}%
\pgfsetstrokecolor{currentstroke}%
\pgfsetdash{}{0pt}%
\pgfpathmoveto{\pgfqpoint{5.898981in}{1.713883in}}%
\pgfpathlineto{\pgfqpoint{5.991583in}{1.880761in}}%
\pgfpathlineto{\pgfqpoint{6.130485in}{2.131496in}}%
\pgfpathlineto{\pgfqpoint{6.192220in}{2.238359in}}%
\pgfpathlineto{\pgfqpoint{6.238521in}{2.315095in}}%
\pgfpathlineto{\pgfqpoint{6.284822in}{2.388112in}}%
\pgfpathlineto{\pgfqpoint{6.331123in}{2.456733in}}%
\pgfpathlineto{\pgfqpoint{6.377424in}{2.520320in}}%
\pgfpathlineto{\pgfqpoint{6.408291in}{2.559621in}}%
\pgfpathlineto{\pgfqpoint{6.439158in}{2.596258in}}%
\pgfpathlineto{\pgfqpoint{6.470025in}{2.630080in}}%
\pgfpathlineto{\pgfqpoint{6.500893in}{2.660948in}}%
\pgfpathlineto{\pgfqpoint{6.531760in}{2.688734in}}%
\pgfpathlineto{\pgfqpoint{6.562627in}{2.713323in}}%
\pgfpathlineto{\pgfqpoint{6.593494in}{2.734614in}}%
\pgfpathlineto{\pgfqpoint{6.624362in}{2.752519in}}%
\pgfpathlineto{\pgfqpoint{6.655229in}{2.766963in}}%
\pgfpathlineto{\pgfqpoint{6.686096in}{2.777888in}}%
\pgfpathlineto{\pgfqpoint{6.716964in}{2.785248in}}%
\pgfpathlineto{\pgfqpoint{6.747831in}{2.789012in}}%
\pgfpathlineto{\pgfqpoint{6.778698in}{2.789165in}}%
\pgfpathlineto{\pgfqpoint{6.809565in}{2.785706in}}%
\pgfpathlineto{\pgfqpoint{6.840433in}{2.778651in}}%
\pgfpathlineto{\pgfqpoint{6.871300in}{2.768027in}}%
\pgfpathlineto{\pgfqpoint{6.902167in}{2.753879in}}%
\pgfpathlineto{\pgfqpoint{6.933034in}{2.736265in}}%
\pgfpathlineto{\pgfqpoint{6.963902in}{2.715258in}}%
\pgfpathlineto{\pgfqpoint{6.994769in}{2.690944in}}%
\pgfpathlineto{\pgfqpoint{7.025636in}{2.663425in}}%
\pgfpathlineto{\pgfqpoint{7.056504in}{2.632813in}}%
\pgfpathlineto{\pgfqpoint{7.087371in}{2.599236in}}%
\pgfpathlineto{\pgfqpoint{7.118238in}{2.562832in}}%
\pgfpathlineto{\pgfqpoint{7.149105in}{2.523751in}}%
\pgfpathlineto{\pgfqpoint{7.195406in}{2.460467in}}%
\pgfpathlineto{\pgfqpoint{7.241707in}{2.392114in}}%
\pgfpathlineto{\pgfqpoint{7.288008in}{2.319327in}}%
\pgfpathlineto{\pgfqpoint{7.334309in}{2.242782in}}%
\pgfpathlineto{\pgfqpoint{7.396044in}{2.136111in}}%
\pgfpathlineto{\pgfqpoint{7.473212in}{1.997848in}}%
\pgfpathlineto{\pgfqpoint{7.642982in}{1.691235in}}%
\pgfpathlineto{\pgfqpoint{7.704716in}{1.584662in}}%
\pgfpathlineto{\pgfqpoint{7.751017in}{1.508217in}}%
\pgfpathlineto{\pgfqpoint{7.797318in}{1.435549in}}%
\pgfpathlineto{\pgfqpoint{7.843619in}{1.367334in}}%
\pgfpathlineto{\pgfqpoint{7.889920in}{1.304205in}}%
\pgfpathlineto{\pgfqpoint{7.920787in}{1.265237in}}%
\pgfpathlineto{\pgfqpoint{7.951654in}{1.228952in}}%
\pgfpathlineto{\pgfqpoint{7.982522in}{1.195501in}}%
\pgfpathlineto{\pgfqpoint{8.013389in}{1.165021in}}%
\pgfpathlineto{\pgfqpoint{8.044256in}{1.137639in}}%
\pgfpathlineto{\pgfqpoint{8.075123in}{1.113467in}}%
\pgfpathlineto{\pgfqpoint{8.105991in}{1.092606in}}%
\pgfpathlineto{\pgfqpoint{8.136858in}{1.075141in}}%
\pgfpathlineto{\pgfqpoint{8.167725in}{1.061144in}}%
\pgfpathlineto{\pgfqpoint{8.198593in}{1.050674in}}%
\pgfpathlineto{\pgfqpoint{8.229460in}{1.043774in}}%
\pgfpathlineto{\pgfqpoint{8.260327in}{1.040472in}}%
\pgfpathlineto{\pgfqpoint{8.291194in}{1.040782in}}%
\pgfpathlineto{\pgfqpoint{8.322062in}{1.044703in}}%
\pgfpathlineto{\pgfqpoint{8.352929in}{1.052218in}}%
\pgfpathlineto{\pgfqpoint{8.383796in}{1.063296in}}%
\pgfpathlineto{\pgfqpoint{8.414663in}{1.077892in}}%
\pgfpathlineto{\pgfqpoint{8.445531in}{1.095946in}}%
\pgfpathlineto{\pgfqpoint{8.476398in}{1.117382in}}%
\pgfpathlineto{\pgfqpoint{8.507265in}{1.142112in}}%
\pgfpathlineto{\pgfqpoint{8.538133in}{1.170034in}}%
\pgfpathlineto{\pgfqpoint{8.569000in}{1.201033in}}%
\pgfpathlineto{\pgfqpoint{8.599867in}{1.234981in}}%
\pgfpathlineto{\pgfqpoint{8.630734in}{1.271737in}}%
\pgfpathlineto{\pgfqpoint{8.661602in}{1.311149in}}%
\pgfpathlineto{\pgfqpoint{8.707903in}{1.374890in}}%
\pgfpathlineto{\pgfqpoint{8.754203in}{1.443647in}}%
\pgfpathlineto{\pgfqpoint{8.800504in}{1.516782in}}%
\pgfpathlineto{\pgfqpoint{8.846805in}{1.593616in}}%
\pgfpathlineto{\pgfqpoint{8.908540in}{1.700575in}}%
\pgfpathlineto{\pgfqpoint{8.985708in}{1.839034in}}%
\pgfpathlineto{\pgfqpoint{9.155478in}{2.145431in}}%
\pgfpathlineto{\pgfqpoint{9.217212in}{2.251706in}}%
\pgfpathlineto{\pgfqpoint{9.263513in}{2.327856in}}%
\pgfpathlineto{\pgfqpoint{9.309814in}{2.400169in}}%
\pgfpathlineto{\pgfqpoint{9.356115in}{2.467974in}}%
\pgfpathlineto{\pgfqpoint{9.402416in}{2.530640in}}%
\pgfpathlineto{\pgfqpoint{9.433283in}{2.569272in}}%
\pgfpathlineto{\pgfqpoint{9.464151in}{2.605202in}}%
\pgfpathlineto{\pgfqpoint{9.495018in}{2.638280in}}%
\pgfpathlineto{\pgfqpoint{9.525885in}{2.668369in}}%
\pgfpathlineto{\pgfqpoint{9.556752in}{2.695346in}}%
\pgfpathlineto{\pgfqpoint{9.587620in}{2.719099in}}%
\pgfpathlineto{\pgfqpoint{9.618487in}{2.739530in}}%
\pgfpathlineto{\pgfqpoint{9.649354in}{2.756554in}}%
\pgfpathlineto{\pgfqpoint{9.680222in}{2.770102in}}%
\pgfpathlineto{\pgfqpoint{9.711089in}{2.780116in}}%
\pgfpathlineto{\pgfqpoint{9.741956in}{2.786556in}}%
\pgfpathlineto{\pgfqpoint{9.741956in}{2.786556in}}%
\pgfusepath{stroke}%
\end{pgfscope}%
\begin{pgfscope}%
\pgfpathrectangle{\pgfqpoint{5.706832in}{0.521603in}}{\pgfqpoint{4.227273in}{2.800000in}} %
\pgfusepath{clip}%
\pgfsetrectcap%
\pgfsetroundjoin%
\pgfsetlinewidth{0.501875pt}%
\definecolor{currentstroke}{rgb}{0.343137,0.243914,0.992421}%
\pgfsetstrokecolor{currentstroke}%
\pgfsetdash{}{0pt}%
\pgfpathmoveto{\pgfqpoint{5.898981in}{2.577533in}}%
\pgfpathlineto{\pgfqpoint{5.914415in}{2.558951in}}%
\pgfpathlineto{\pgfqpoint{5.929848in}{2.522965in}}%
\pgfpathlineto{\pgfqpoint{5.945282in}{2.469229in}}%
\pgfpathlineto{\pgfqpoint{5.960715in}{2.397955in}}%
\pgfpathlineto{\pgfqpoint{5.976149in}{2.309923in}}%
\pgfpathlineto{\pgfqpoint{5.991583in}{2.206461in}}%
\pgfpathlineto{\pgfqpoint{6.007016in}{2.089410in}}%
\pgfpathlineto{\pgfqpoint{6.037884in}{1.824170in}}%
\pgfpathlineto{\pgfqpoint{6.099618in}{1.254223in}}%
\pgfpathlineto{\pgfqpoint{6.115052in}{1.123002in}}%
\pgfpathlineto{\pgfqpoint{6.130485in}{1.002851in}}%
\pgfpathlineto{\pgfqpoint{6.145919in}{0.896689in}}%
\pgfpathlineto{\pgfqpoint{6.161353in}{0.807073in}}%
\pgfpathlineto{\pgfqpoint{6.176786in}{0.736132in}}%
\pgfpathlineto{\pgfqpoint{6.192220in}{0.685493in}}%
\pgfpathlineto{\pgfqpoint{6.207654in}{0.656240in}}%
\pgfpathlineto{\pgfqpoint{6.223087in}{0.648876in}}%
\pgfpathlineto{\pgfqpoint{6.238521in}{0.663316in}}%
\pgfpathlineto{\pgfqpoint{6.253955in}{0.698887in}}%
\pgfpathlineto{\pgfqpoint{6.269388in}{0.754352in}}%
\pgfpathlineto{\pgfqpoint{6.284822in}{0.827953in}}%
\pgfpathlineto{\pgfqpoint{6.300255in}{0.917462in}}%
\pgfpathlineto{\pgfqpoint{6.315689in}{1.020260in}}%
\pgfpathlineto{\pgfqpoint{6.346556in}{1.253767in}}%
\pgfpathlineto{\pgfqpoint{6.408291in}{1.742024in}}%
\pgfpathlineto{\pgfqpoint{6.423724in}{1.850388in}}%
\pgfpathlineto{\pgfqpoint{6.439158in}{1.947989in}}%
\pgfpathlineto{\pgfqpoint{6.454592in}{2.032854in}}%
\pgfpathlineto{\pgfqpoint{6.470025in}{2.103493in}}%
\pgfpathlineto{\pgfqpoint{6.485459in}{2.158943in}}%
\pgfpathlineto{\pgfqpoint{6.500893in}{2.198785in}}%
\pgfpathlineto{\pgfqpoint{6.516326in}{2.223152in}}%
\pgfpathlineto{\pgfqpoint{6.531760in}{2.232719in}}%
\pgfpathlineto{\pgfqpoint{6.547194in}{2.228664in}}%
\pgfpathlineto{\pgfqpoint{6.562627in}{2.212631in}}%
\pgfpathlineto{\pgfqpoint{6.578061in}{2.186659in}}%
\pgfpathlineto{\pgfqpoint{6.593494in}{2.153111in}}%
\pgfpathlineto{\pgfqpoint{6.624362in}{2.073835in}}%
\pgfpathlineto{\pgfqpoint{6.639795in}{2.033641in}}%
\pgfpathlineto{\pgfqpoint{6.655229in}{1.996744in}}%
\pgfpathlineto{\pgfqpoint{6.670663in}{1.965729in}}%
\pgfpathlineto{\pgfqpoint{6.686096in}{1.942942in}}%
\pgfpathlineto{\pgfqpoint{6.701530in}{1.930399in}}%
\pgfpathlineto{\pgfqpoint{6.716964in}{1.929716in}}%
\pgfpathlineto{\pgfqpoint{6.732397in}{1.942053in}}%
\pgfpathlineto{\pgfqpoint{6.747831in}{1.968061in}}%
\pgfpathlineto{\pgfqpoint{6.763264in}{2.007864in}}%
\pgfpathlineto{\pgfqpoint{6.778698in}{2.061040in}}%
\pgfpathlineto{\pgfqpoint{6.794132in}{2.126639in}}%
\pgfpathlineto{\pgfqpoint{6.809565in}{2.203198in}}%
\pgfpathlineto{\pgfqpoint{6.824999in}{2.288795in}}%
\pgfpathlineto{\pgfqpoint{6.855866in}{2.477449in}}%
\pgfpathlineto{\pgfqpoint{6.886734in}{2.670483in}}%
\pgfpathlineto{\pgfqpoint{6.902167in}{2.760976in}}%
\pgfpathlineto{\pgfqpoint{6.917601in}{2.843331in}}%
\pgfpathlineto{\pgfqpoint{6.933034in}{2.914612in}}%
\pgfpathlineto{\pgfqpoint{6.948468in}{2.972124in}}%
\pgfpathlineto{\pgfqpoint{6.963902in}{3.013501in}}%
\pgfpathlineto{\pgfqpoint{6.979335in}{3.036789in}}%
\pgfpathlineto{\pgfqpoint{6.994769in}{3.040509in}}%
\pgfpathlineto{\pgfqpoint{7.010203in}{3.023713in}}%
\pgfpathlineto{\pgfqpoint{7.025636in}{2.986018in}}%
\pgfpathlineto{\pgfqpoint{7.041070in}{2.927621in}}%
\pgfpathlineto{\pgfqpoint{7.056504in}{2.849306in}}%
\pgfpathlineto{\pgfqpoint{7.071937in}{2.752419in}}%
\pgfpathlineto{\pgfqpoint{7.087371in}{2.638834in}}%
\pgfpathlineto{\pgfqpoint{7.102804in}{2.510902in}}%
\pgfpathlineto{\pgfqpoint{7.133672in}{2.223357in}}%
\pgfpathlineto{\pgfqpoint{7.195406in}{1.614097in}}%
\pgfpathlineto{\pgfqpoint{7.210840in}{1.474589in}}%
\pgfpathlineto{\pgfqpoint{7.226274in}{1.346526in}}%
\pgfpathlineto{\pgfqpoint{7.241707in}{1.232547in}}%
\pgfpathlineto{\pgfqpoint{7.257141in}{1.134862in}}%
\pgfpathlineto{\pgfqpoint{7.272574in}{1.055190in}}%
\pgfpathlineto{\pgfqpoint{7.288008in}{0.994714in}}%
\pgfpathlineto{\pgfqpoint{7.303442in}{0.954046in}}%
\pgfpathlineto{\pgfqpoint{7.318875in}{0.933224in}}%
\pgfpathlineto{\pgfqpoint{7.334309in}{0.931711in}}%
\pgfpathlineto{\pgfqpoint{7.349743in}{0.948422in}}%
\pgfpathlineto{\pgfqpoint{7.365176in}{0.981770in}}%
\pgfpathlineto{\pgfqpoint{7.380610in}{1.029718in}}%
\pgfpathlineto{\pgfqpoint{7.396044in}{1.089854in}}%
\pgfpathlineto{\pgfqpoint{7.411477in}{1.159474in}}%
\pgfpathlineto{\pgfqpoint{7.442344in}{1.315445in}}%
\pgfpathlineto{\pgfqpoint{7.473212in}{1.473766in}}%
\pgfpathlineto{\pgfqpoint{7.488645in}{1.546687in}}%
\pgfpathlineto{\pgfqpoint{7.504079in}{1.612109in}}%
\pgfpathlineto{\pgfqpoint{7.519513in}{1.667959in}}%
\pgfpathlineto{\pgfqpoint{7.534946in}{1.712595in}}%
\pgfpathlineto{\pgfqpoint{7.550380in}{1.744854in}}%
\pgfpathlineto{\pgfqpoint{7.565814in}{1.764090in}}%
\pgfpathlineto{\pgfqpoint{7.581247in}{1.770193in}}%
\pgfpathlineto{\pgfqpoint{7.596681in}{1.763587in}}%
\pgfpathlineto{\pgfqpoint{7.612114in}{1.745216in}}%
\pgfpathlineto{\pgfqpoint{7.627548in}{1.716506in}}%
\pgfpathlineto{\pgfqpoint{7.642982in}{1.679313in}}%
\pgfpathlineto{\pgfqpoint{7.658415in}{1.635863in}}%
\pgfpathlineto{\pgfqpoint{7.704716in}{1.493986in}}%
\pgfpathlineto{\pgfqpoint{7.720150in}{1.452140in}}%
\pgfpathlineto{\pgfqpoint{7.735584in}{1.417628in}}%
\pgfpathlineto{\pgfqpoint{7.751017in}{1.392989in}}%
\pgfpathlineto{\pgfqpoint{7.766451in}{1.380481in}}%
\pgfpathlineto{\pgfqpoint{7.781884in}{1.382001in}}%
\pgfpathlineto{\pgfqpoint{7.797318in}{1.399011in}}%
\pgfpathlineto{\pgfqpoint{7.812752in}{1.432484in}}%
\pgfpathlineto{\pgfqpoint{7.828185in}{1.482863in}}%
\pgfpathlineto{\pgfqpoint{7.843619in}{1.550035in}}%
\pgfpathlineto{\pgfqpoint{7.859053in}{1.633329in}}%
\pgfpathlineto{\pgfqpoint{7.874486in}{1.731523in}}%
\pgfpathlineto{\pgfqpoint{7.889920in}{1.842882in}}%
\pgfpathlineto{\pgfqpoint{7.920787in}{2.095856in}}%
\pgfpathlineto{\pgfqpoint{7.982522in}{2.639920in}}%
\pgfpathlineto{\pgfqpoint{7.997955in}{2.764691in}}%
\pgfpathlineto{\pgfqpoint{8.013389in}{2.878444in}}%
\pgfpathlineto{\pgfqpoint{8.028823in}{2.978262in}}%
\pgfpathlineto{\pgfqpoint{8.044256in}{3.061571in}}%
\pgfpathlineto{\pgfqpoint{8.059690in}{3.126220in}}%
\pgfpathlineto{\pgfqpoint{8.075123in}{3.170548in}}%
\pgfpathlineto{\pgfqpoint{8.090557in}{3.193434in}}%
\pgfpathlineto{\pgfqpoint{8.105991in}{3.194331in}}%
\pgfpathlineto{\pgfqpoint{8.121424in}{3.173283in}}%
\pgfpathlineto{\pgfqpoint{8.136858in}{3.130926in}}%
\pgfpathlineto{\pgfqpoint{8.152292in}{3.068461in}}%
\pgfpathlineto{\pgfqpoint{8.167725in}{2.987622in}}%
\pgfpathlineto{\pgfqpoint{8.183159in}{2.890621in}}%
\pgfpathlineto{\pgfqpoint{8.198593in}{2.780077in}}%
\pgfpathlineto{\pgfqpoint{8.229460in}{2.530374in}}%
\pgfpathlineto{\pgfqpoint{8.291194in}{2.008220in}}%
\pgfpathlineto{\pgfqpoint{8.306628in}{1.891516in}}%
\pgfpathlineto{\pgfqpoint{8.322062in}{1.785833in}}%
\pgfpathlineto{\pgfqpoint{8.337495in}{1.693262in}}%
\pgfpathlineto{\pgfqpoint{8.352929in}{1.615414in}}%
\pgfpathlineto{\pgfqpoint{8.368363in}{1.553379in}}%
\pgfpathlineto{\pgfqpoint{8.383796in}{1.507698in}}%
\pgfpathlineto{\pgfqpoint{8.399230in}{1.478352in}}%
\pgfpathlineto{\pgfqpoint{8.414663in}{1.464779in}}%
\pgfpathlineto{\pgfqpoint{8.430097in}{1.465894in}}%
\pgfpathlineto{\pgfqpoint{8.445531in}{1.480136in}}%
\pgfpathlineto{\pgfqpoint{8.460964in}{1.505532in}}%
\pgfpathlineto{\pgfqpoint{8.476398in}{1.539763in}}%
\pgfpathlineto{\pgfqpoint{8.491832in}{1.580254in}}%
\pgfpathlineto{\pgfqpoint{8.538133in}{1.711614in}}%
\pgfpathlineto{\pgfqpoint{8.553566in}{1.749528in}}%
\pgfpathlineto{\pgfqpoint{8.569000in}{1.780290in}}%
\pgfpathlineto{\pgfqpoint{8.584433in}{1.801777in}}%
\pgfpathlineto{\pgfqpoint{8.599867in}{1.812254in}}%
\pgfpathlineto{\pgfqpoint{8.615301in}{1.810431in}}%
\pgfpathlineto{\pgfqpoint{8.630734in}{1.795511in}}%
\pgfpathlineto{\pgfqpoint{8.646168in}{1.767222in}}%
\pgfpathlineto{\pgfqpoint{8.661602in}{1.725832in}}%
\pgfpathlineto{\pgfqpoint{8.677035in}{1.672141in}}%
\pgfpathlineto{\pgfqpoint{8.692469in}{1.607460in}}%
\pgfpathlineto{\pgfqpoint{8.707903in}{1.533573in}}%
\pgfpathlineto{\pgfqpoint{8.738770in}{1.367322in}}%
\pgfpathlineto{\pgfqpoint{8.769637in}{1.194625in}}%
\pgfpathlineto{\pgfqpoint{8.785071in}{1.113324in}}%
\pgfpathlineto{\pgfqpoint{8.800504in}{1.039439in}}%
\pgfpathlineto{\pgfqpoint{8.815938in}{0.975877in}}%
\pgfpathlineto{\pgfqpoint{8.831372in}{0.925322in}}%
\pgfpathlineto{\pgfqpoint{8.846805in}{0.890143in}}%
\pgfpathlineto{\pgfqpoint{8.862239in}{0.872319in}}%
\pgfpathlineto{\pgfqpoint{8.877673in}{0.873363in}}%
\pgfpathlineto{\pgfqpoint{8.893106in}{0.894268in}}%
\pgfpathlineto{\pgfqpoint{8.908540in}{0.935474in}}%
\pgfpathlineto{\pgfqpoint{8.923973in}{0.996841in}}%
\pgfpathlineto{\pgfqpoint{8.939407in}{1.077650in}}%
\pgfpathlineto{\pgfqpoint{8.954841in}{1.176616in}}%
\pgfpathlineto{\pgfqpoint{8.970274in}{1.291923in}}%
\pgfpathlineto{\pgfqpoint{8.985708in}{1.421273in}}%
\pgfpathlineto{\pgfqpoint{9.016575in}{1.710907in}}%
\pgfpathlineto{\pgfqpoint{9.078310in}{2.321858in}}%
\pgfpathlineto{\pgfqpoint{9.093743in}{2.461223in}}%
\pgfpathlineto{\pgfqpoint{9.109177in}{2.588850in}}%
\pgfpathlineto{\pgfqpoint{9.124611in}{2.702047in}}%
\pgfpathlineto{\pgfqpoint{9.140044in}{2.798544in}}%
\pgfpathlineto{\pgfqpoint{9.155478in}{2.876556in}}%
\pgfpathlineto{\pgfqpoint{9.170912in}{2.934834in}}%
\pgfpathlineto{\pgfqpoint{9.186345in}{2.972696in}}%
\pgfpathlineto{\pgfqpoint{9.201779in}{2.990041in}}%
\pgfpathlineto{\pgfqpoint{9.217212in}{2.987348in}}%
\pgfpathlineto{\pgfqpoint{9.232646in}{2.965653in}}%
\pgfpathlineto{\pgfqpoint{9.248080in}{2.926507in}}%
\pgfpathlineto{\pgfqpoint{9.263513in}{2.871926in}}%
\pgfpathlineto{\pgfqpoint{9.278947in}{2.804317in}}%
\pgfpathlineto{\pgfqpoint{9.294381in}{2.726395in}}%
\pgfpathlineto{\pgfqpoint{9.325248in}{2.551482in}}%
\pgfpathlineto{\pgfqpoint{9.356115in}{2.371538in}}%
\pgfpathlineto{\pgfqpoint{9.371549in}{2.287033in}}%
\pgfpathlineto{\pgfqpoint{9.386982in}{2.209668in}}%
\pgfpathlineto{\pgfqpoint{9.402416in}{2.141652in}}%
\pgfpathlineto{\pgfqpoint{9.417850in}{2.084771in}}%
\pgfpathlineto{\pgfqpoint{9.433283in}{2.040340in}}%
\pgfpathlineto{\pgfqpoint{9.448717in}{2.009161in}}%
\pgfpathlineto{\pgfqpoint{9.464151in}{1.991498in}}%
\pgfpathlineto{\pgfqpoint{9.479584in}{1.987079in}}%
\pgfpathlineto{\pgfqpoint{9.495018in}{1.995107in}}%
\pgfpathlineto{\pgfqpoint{9.510452in}{2.014289in}}%
\pgfpathlineto{\pgfqpoint{9.525885in}{2.042889in}}%
\pgfpathlineto{\pgfqpoint{9.541319in}{2.078789in}}%
\pgfpathlineto{\pgfqpoint{9.572186in}{2.162561in}}%
\pgfpathlineto{\pgfqpoint{9.587620in}{2.205017in}}%
\pgfpathlineto{\pgfqpoint{9.603053in}{2.244126in}}%
\pgfpathlineto{\pgfqpoint{9.618487in}{2.277161in}}%
\pgfpathlineto{\pgfqpoint{9.633921in}{2.301557in}}%
\pgfpathlineto{\pgfqpoint{9.649354in}{2.315015in}}%
\pgfpathlineto{\pgfqpoint{9.664788in}{2.315575in}}%
\pgfpathlineto{\pgfqpoint{9.680222in}{2.301697in}}%
\pgfpathlineto{\pgfqpoint{9.695655in}{2.272315in}}%
\pgfpathlineto{\pgfqpoint{9.711089in}{2.226882in}}%
\pgfpathlineto{\pgfqpoint{9.726522in}{2.165399in}}%
\pgfpathlineto{\pgfqpoint{9.741956in}{2.088421in}}%
\pgfpathlineto{\pgfqpoint{9.741956in}{2.088421in}}%
\pgfusepath{stroke}%
\end{pgfscope}%
\begin{pgfscope}%
\pgfpathrectangle{\pgfqpoint{5.706832in}{0.521603in}}{\pgfqpoint{4.227273in}{2.800000in}} %
\pgfusepath{clip}%
\pgfsetrectcap%
\pgfsetroundjoin%
\pgfsetlinewidth{0.501875pt}%
\definecolor{currentstroke}{rgb}{0.264706,0.361242,0.982973}%
\pgfsetstrokecolor{currentstroke}%
\pgfsetdash{}{0pt}%
\pgfpathmoveto{\pgfqpoint{5.898981in}{2.339506in}}%
\pgfpathlineto{\pgfqpoint{5.914415in}{2.365801in}}%
\pgfpathlineto{\pgfqpoint{5.929848in}{2.379932in}}%
\pgfpathlineto{\pgfqpoint{5.945282in}{2.380812in}}%
\pgfpathlineto{\pgfqpoint{5.960715in}{2.367759in}}%
\pgfpathlineto{\pgfqpoint{5.976149in}{2.340519in}}%
\pgfpathlineto{\pgfqpoint{5.991583in}{2.299272in}}%
\pgfpathlineto{\pgfqpoint{6.007016in}{2.244626in}}%
\pgfpathlineto{\pgfqpoint{6.022450in}{2.177596in}}%
\pgfpathlineto{\pgfqpoint{6.037884in}{2.099571in}}%
\pgfpathlineto{\pgfqpoint{6.053317in}{2.012277in}}%
\pgfpathlineto{\pgfqpoint{6.084185in}{1.818111in}}%
\pgfpathlineto{\pgfqpoint{6.130485in}{1.513107in}}%
\pgfpathlineto{\pgfqpoint{6.161353in}{1.328852in}}%
\pgfpathlineto{\pgfqpoint{6.176786in}{1.249175in}}%
\pgfpathlineto{\pgfqpoint{6.192220in}{1.180308in}}%
\pgfpathlineto{\pgfqpoint{6.207654in}{1.123766in}}%
\pgfpathlineto{\pgfqpoint{6.223087in}{1.080721in}}%
\pgfpathlineto{\pgfqpoint{6.238521in}{1.051978in}}%
\pgfpathlineto{\pgfqpoint{6.253955in}{1.037955in}}%
\pgfpathlineto{\pgfqpoint{6.269388in}{1.038679in}}%
\pgfpathlineto{\pgfqpoint{6.284822in}{1.053796in}}%
\pgfpathlineto{\pgfqpoint{6.300255in}{1.082589in}}%
\pgfpathlineto{\pgfqpoint{6.315689in}{1.124009in}}%
\pgfpathlineto{\pgfqpoint{6.331123in}{1.176713in}}%
\pgfpathlineto{\pgfqpoint{6.346556in}{1.239112in}}%
\pgfpathlineto{\pgfqpoint{6.361990in}{1.309429in}}%
\pgfpathlineto{\pgfqpoint{6.392857in}{1.466097in}}%
\pgfpathlineto{\pgfqpoint{6.439158in}{1.711701in}}%
\pgfpathlineto{\pgfqpoint{6.470025in}{1.861549in}}%
\pgfpathlineto{\pgfqpoint{6.485459in}{1.927979in}}%
\pgfpathlineto{\pgfqpoint{6.500893in}{1.987403in}}%
\pgfpathlineto{\pgfqpoint{6.516326in}{2.039173in}}%
\pgfpathlineto{\pgfqpoint{6.531760in}{2.082939in}}%
\pgfpathlineto{\pgfqpoint{6.547194in}{2.118649in}}%
\pgfpathlineto{\pgfqpoint{6.562627in}{2.146534in}}%
\pgfpathlineto{\pgfqpoint{6.578061in}{2.167088in}}%
\pgfpathlineto{\pgfqpoint{6.593494in}{2.181034in}}%
\pgfpathlineto{\pgfqpoint{6.608928in}{2.189290in}}%
\pgfpathlineto{\pgfqpoint{6.624362in}{2.192922in}}%
\pgfpathlineto{\pgfqpoint{6.639795in}{2.193097in}}%
\pgfpathlineto{\pgfqpoint{6.670663in}{2.187944in}}%
\pgfpathlineto{\pgfqpoint{6.686096in}{2.184998in}}%
\pgfpathlineto{\pgfqpoint{6.701530in}{2.183262in}}%
\pgfpathlineto{\pgfqpoint{6.716964in}{2.183664in}}%
\pgfpathlineto{\pgfqpoint{6.732397in}{2.186952in}}%
\pgfpathlineto{\pgfqpoint{6.747831in}{2.193673in}}%
\pgfpathlineto{\pgfqpoint{6.763264in}{2.204144in}}%
\pgfpathlineto{\pgfqpoint{6.778698in}{2.218442in}}%
\pgfpathlineto{\pgfqpoint{6.794132in}{2.236407in}}%
\pgfpathlineto{\pgfqpoint{6.809565in}{2.257641in}}%
\pgfpathlineto{\pgfqpoint{6.840433in}{2.307259in}}%
\pgfpathlineto{\pgfqpoint{6.886734in}{2.385208in}}%
\pgfpathlineto{\pgfqpoint{6.902167in}{2.407559in}}%
\pgfpathlineto{\pgfqpoint{6.917601in}{2.426087in}}%
\pgfpathlineto{\pgfqpoint{6.933034in}{2.439652in}}%
\pgfpathlineto{\pgfqpoint{6.948468in}{2.447217in}}%
\pgfpathlineto{\pgfqpoint{6.963902in}{2.447896in}}%
\pgfpathlineto{\pgfqpoint{6.979335in}{2.440990in}}%
\pgfpathlineto{\pgfqpoint{6.994769in}{2.426017in}}%
\pgfpathlineto{\pgfqpoint{7.010203in}{2.402735in}}%
\pgfpathlineto{\pgfqpoint{7.025636in}{2.371159in}}%
\pgfpathlineto{\pgfqpoint{7.041070in}{2.331567in}}%
\pgfpathlineto{\pgfqpoint{7.056504in}{2.284496in}}%
\pgfpathlineto{\pgfqpoint{7.071937in}{2.230727in}}%
\pgfpathlineto{\pgfqpoint{7.087371in}{2.171267in}}%
\pgfpathlineto{\pgfqpoint{7.118238in}{2.040243in}}%
\pgfpathlineto{\pgfqpoint{7.164539in}{1.835275in}}%
\pgfpathlineto{\pgfqpoint{7.195406in}{1.710837in}}%
\pgfpathlineto{\pgfqpoint{7.210840in}{1.656537in}}%
\pgfpathlineto{\pgfqpoint{7.226274in}{1.609078in}}%
\pgfpathlineto{\pgfqpoint{7.241707in}{1.569366in}}%
\pgfpathlineto{\pgfqpoint{7.257141in}{1.538062in}}%
\pgfpathlineto{\pgfqpoint{7.272574in}{1.515552in}}%
\pgfpathlineto{\pgfqpoint{7.288008in}{1.501939in}}%
\pgfpathlineto{\pgfqpoint{7.303442in}{1.497031in}}%
\pgfpathlineto{\pgfqpoint{7.318875in}{1.500355in}}%
\pgfpathlineto{\pgfqpoint{7.334309in}{1.511166in}}%
\pgfpathlineto{\pgfqpoint{7.349743in}{1.528478in}}%
\pgfpathlineto{\pgfqpoint{7.365176in}{1.551092in}}%
\pgfpathlineto{\pgfqpoint{7.380610in}{1.577642in}}%
\pgfpathlineto{\pgfqpoint{7.442344in}{1.692792in}}%
\pgfpathlineto{\pgfqpoint{7.457778in}{1.716213in}}%
\pgfpathlineto{\pgfqpoint{7.473212in}{1.734759in}}%
\pgfpathlineto{\pgfqpoint{7.488645in}{1.747366in}}%
\pgfpathlineto{\pgfqpoint{7.504079in}{1.753221in}}%
\pgfpathlineto{\pgfqpoint{7.519513in}{1.751788in}}%
\pgfpathlineto{\pgfqpoint{7.534946in}{1.742839in}}%
\pgfpathlineto{\pgfqpoint{7.550380in}{1.726466in}}%
\pgfpathlineto{\pgfqpoint{7.565814in}{1.703088in}}%
\pgfpathlineto{\pgfqpoint{7.581247in}{1.673442in}}%
\pgfpathlineto{\pgfqpoint{7.596681in}{1.638568in}}%
\pgfpathlineto{\pgfqpoint{7.627548in}{1.558610in}}%
\pgfpathlineto{\pgfqpoint{7.658415in}{1.476215in}}%
\pgfpathlineto{\pgfqpoint{7.673849in}{1.438781in}}%
\pgfpathlineto{\pgfqpoint{7.689283in}{1.406441in}}%
\pgfpathlineto{\pgfqpoint{7.704716in}{1.381075in}}%
\pgfpathlineto{\pgfqpoint{7.720150in}{1.364436in}}%
\pgfpathlineto{\pgfqpoint{7.735584in}{1.358090in}}%
\pgfpathlineto{\pgfqpoint{7.751017in}{1.363354in}}%
\pgfpathlineto{\pgfqpoint{7.766451in}{1.381244in}}%
\pgfpathlineto{\pgfqpoint{7.781884in}{1.412435in}}%
\pgfpathlineto{\pgfqpoint{7.797318in}{1.457224in}}%
\pgfpathlineto{\pgfqpoint{7.812752in}{1.515505in}}%
\pgfpathlineto{\pgfqpoint{7.828185in}{1.586766in}}%
\pgfpathlineto{\pgfqpoint{7.843619in}{1.670086in}}%
\pgfpathlineto{\pgfqpoint{7.859053in}{1.764151in}}%
\pgfpathlineto{\pgfqpoint{7.889920in}{1.977475in}}%
\pgfpathlineto{\pgfqpoint{7.967088in}{2.548580in}}%
\pgfpathlineto{\pgfqpoint{7.982522in}{2.648262in}}%
\pgfpathlineto{\pgfqpoint{7.997955in}{2.737073in}}%
\pgfpathlineto{\pgfqpoint{8.013389in}{2.812773in}}%
\pgfpathlineto{\pgfqpoint{8.028823in}{2.873423in}}%
\pgfpathlineto{\pgfqpoint{8.044256in}{2.917444in}}%
\pgfpathlineto{\pgfqpoint{8.059690in}{2.943662in}}%
\pgfpathlineto{\pgfqpoint{8.075123in}{2.951354in}}%
\pgfpathlineto{\pgfqpoint{8.090557in}{2.940268in}}%
\pgfpathlineto{\pgfqpoint{8.105991in}{2.910633in}}%
\pgfpathlineto{\pgfqpoint{8.121424in}{2.863165in}}%
\pgfpathlineto{\pgfqpoint{8.136858in}{2.799039in}}%
\pgfpathlineto{\pgfqpoint{8.152292in}{2.719870in}}%
\pgfpathlineto{\pgfqpoint{8.167725in}{2.627660in}}%
\pgfpathlineto{\pgfqpoint{8.183159in}{2.524749in}}%
\pgfpathlineto{\pgfqpoint{8.214026in}{2.297456in}}%
\pgfpathlineto{\pgfqpoint{8.260327in}{1.946182in}}%
\pgfpathlineto{\pgfqpoint{8.275761in}{1.837888in}}%
\pgfpathlineto{\pgfqpoint{8.291194in}{1.738417in}}%
\pgfpathlineto{\pgfqpoint{8.306628in}{1.650035in}}%
\pgfpathlineto{\pgfqpoint{8.322062in}{1.574658in}}%
\pgfpathlineto{\pgfqpoint{8.337495in}{1.513793in}}%
\pgfpathlineto{\pgfqpoint{8.352929in}{1.468497in}}%
\pgfpathlineto{\pgfqpoint{8.368363in}{1.439353in}}%
\pgfpathlineto{\pgfqpoint{8.383796in}{1.426448in}}%
\pgfpathlineto{\pgfqpoint{8.399230in}{1.429378in}}%
\pgfpathlineto{\pgfqpoint{8.414663in}{1.447264in}}%
\pgfpathlineto{\pgfqpoint{8.430097in}{1.478779in}}%
\pgfpathlineto{\pgfqpoint{8.445531in}{1.522193in}}%
\pgfpathlineto{\pgfqpoint{8.460964in}{1.575426in}}%
\pgfpathlineto{\pgfqpoint{8.476398in}{1.636120in}}%
\pgfpathlineto{\pgfqpoint{8.538133in}{1.900705in}}%
\pgfpathlineto{\pgfqpoint{8.553566in}{1.958887in}}%
\pgfpathlineto{\pgfqpoint{8.569000in}{2.008897in}}%
\pgfpathlineto{\pgfqpoint{8.584433in}{2.048656in}}%
\pgfpathlineto{\pgfqpoint{8.599867in}{2.076430in}}%
\pgfpathlineto{\pgfqpoint{8.615301in}{2.090887in}}%
\pgfpathlineto{\pgfqpoint{8.630734in}{2.091140in}}%
\pgfpathlineto{\pgfqpoint{8.646168in}{2.076772in}}%
\pgfpathlineto{\pgfqpoint{8.661602in}{2.047856in}}%
\pgfpathlineto{\pgfqpoint{8.677035in}{2.004948in}}%
\pgfpathlineto{\pgfqpoint{8.692469in}{1.949075in}}%
\pgfpathlineto{\pgfqpoint{8.707903in}{1.881706in}}%
\pgfpathlineto{\pgfqpoint{8.723336in}{1.804707in}}%
\pgfpathlineto{\pgfqpoint{8.754203in}{1.630922in}}%
\pgfpathlineto{\pgfqpoint{8.800504in}{1.360556in}}%
\pgfpathlineto{\pgfqpoint{8.815938in}{1.279063in}}%
\pgfpathlineto{\pgfqpoint{8.831372in}{1.206415in}}%
\pgfpathlineto{\pgfqpoint{8.846805in}{1.145060in}}%
\pgfpathlineto{\pgfqpoint{8.862239in}{1.097160in}}%
\pgfpathlineto{\pgfqpoint{8.877673in}{1.064526in}}%
\pgfpathlineto{\pgfqpoint{8.893106in}{1.048564in}}%
\pgfpathlineto{\pgfqpoint{8.908540in}{1.050230in}}%
\pgfpathlineto{\pgfqpoint{8.923973in}{1.070007in}}%
\pgfpathlineto{\pgfqpoint{8.939407in}{1.107884in}}%
\pgfpathlineto{\pgfqpoint{8.954841in}{1.163359in}}%
\pgfpathlineto{\pgfqpoint{8.970274in}{1.235452in}}%
\pgfpathlineto{\pgfqpoint{8.985708in}{1.322735in}}%
\pgfpathlineto{\pgfqpoint{9.001142in}{1.423373in}}%
\pgfpathlineto{\pgfqpoint{9.016575in}{1.535174in}}%
\pgfpathlineto{\pgfqpoint{9.047443in}{1.782117in}}%
\pgfpathlineto{\pgfqpoint{9.093743in}{2.168678in}}%
\pgfpathlineto{\pgfqpoint{9.124611in}{2.404001in}}%
\pgfpathlineto{\pgfqpoint{9.140044in}{2.507190in}}%
\pgfpathlineto{\pgfqpoint{9.155478in}{2.597884in}}%
\pgfpathlineto{\pgfqpoint{9.170912in}{2.674384in}}%
\pgfpathlineto{\pgfqpoint{9.186345in}{2.735396in}}%
\pgfpathlineto{\pgfqpoint{9.201779in}{2.780056in}}%
\pgfpathlineto{\pgfqpoint{9.217212in}{2.807946in}}%
\pgfpathlineto{\pgfqpoint{9.232646in}{2.819096in}}%
\pgfpathlineto{\pgfqpoint{9.248080in}{2.813973in}}%
\pgfpathlineto{\pgfqpoint{9.263513in}{2.793459in}}%
\pgfpathlineto{\pgfqpoint{9.278947in}{2.758808in}}%
\pgfpathlineto{\pgfqpoint{9.294381in}{2.711608in}}%
\pgfpathlineto{\pgfqpoint{9.309814in}{2.653719in}}%
\pgfpathlineto{\pgfqpoint{9.325248in}{2.587212in}}%
\pgfpathlineto{\pgfqpoint{9.356115in}{2.437277in}}%
\pgfpathlineto{\pgfqpoint{9.402416in}{2.203985in}}%
\pgfpathlineto{\pgfqpoint{9.417850in}{2.132374in}}%
\pgfpathlineto{\pgfqpoint{9.433283in}{2.066774in}}%
\pgfpathlineto{\pgfqpoint{9.448717in}{2.008538in}}%
\pgfpathlineto{\pgfqpoint{9.464151in}{1.958694in}}%
\pgfpathlineto{\pgfqpoint{9.479584in}{1.917925in}}%
\pgfpathlineto{\pgfqpoint{9.495018in}{1.886557in}}%
\pgfpathlineto{\pgfqpoint{9.510452in}{1.864563in}}%
\pgfpathlineto{\pgfqpoint{9.525885in}{1.851574in}}%
\pgfpathlineto{\pgfqpoint{9.541319in}{1.846906in}}%
\pgfpathlineto{\pgfqpoint{9.556752in}{1.849589in}}%
\pgfpathlineto{\pgfqpoint{9.572186in}{1.858417in}}%
\pgfpathlineto{\pgfqpoint{9.587620in}{1.871993in}}%
\pgfpathlineto{\pgfqpoint{9.618487in}{1.907186in}}%
\pgfpathlineto{\pgfqpoint{9.649354in}{1.942362in}}%
\pgfpathlineto{\pgfqpoint{9.664788in}{1.956078in}}%
\pgfpathlineto{\pgfqpoint{9.680222in}{1.965392in}}%
\pgfpathlineto{\pgfqpoint{9.695655in}{1.969177in}}%
\pgfpathlineto{\pgfqpoint{9.711089in}{1.966542in}}%
\pgfpathlineto{\pgfqpoint{9.726522in}{1.956864in}}%
\pgfpathlineto{\pgfqpoint{9.741956in}{1.939806in}}%
\pgfpathlineto{\pgfqpoint{9.741956in}{1.939806in}}%
\pgfusepath{stroke}%
\end{pgfscope}%
\begin{pgfscope}%
\pgfpathrectangle{\pgfqpoint{5.706832in}{0.521603in}}{\pgfqpoint{4.227273in}{2.800000in}} %
\pgfusepath{clip}%
\pgfsetrectcap%
\pgfsetroundjoin%
\pgfsetlinewidth{0.501875pt}%
\definecolor{currentstroke}{rgb}{0.186275,0.473094,0.969797}%
\pgfsetstrokecolor{currentstroke}%
\pgfsetdash{}{0pt}%
\pgfpathmoveto{\pgfqpoint{5.898981in}{2.247423in}}%
\pgfpathlineto{\pgfqpoint{5.929848in}{2.189149in}}%
\pgfpathlineto{\pgfqpoint{5.960715in}{2.126425in}}%
\pgfpathlineto{\pgfqpoint{5.991583in}{2.057194in}}%
\pgfpathlineto{\pgfqpoint{6.022450in}{1.980145in}}%
\pgfpathlineto{\pgfqpoint{6.053317in}{1.895005in}}%
\pgfpathlineto{\pgfqpoint{6.084185in}{1.802725in}}%
\pgfpathlineto{\pgfqpoint{6.192220in}{1.466444in}}%
\pgfpathlineto{\pgfqpoint{6.207654in}{1.424224in}}%
\pgfpathlineto{\pgfqpoint{6.223087in}{1.385475in}}%
\pgfpathlineto{\pgfqpoint{6.238521in}{1.350891in}}%
\pgfpathlineto{\pgfqpoint{6.253955in}{1.321144in}}%
\pgfpathlineto{\pgfqpoint{6.269388in}{1.296870in}}%
\pgfpathlineto{\pgfqpoint{6.284822in}{1.278653in}}%
\pgfpathlineto{\pgfqpoint{6.300255in}{1.267011in}}%
\pgfpathlineto{\pgfqpoint{6.315689in}{1.262387in}}%
\pgfpathlineto{\pgfqpoint{6.331123in}{1.265131in}}%
\pgfpathlineto{\pgfqpoint{6.346556in}{1.275497in}}%
\pgfpathlineto{\pgfqpoint{6.361990in}{1.293633in}}%
\pgfpathlineto{\pgfqpoint{6.377424in}{1.319573in}}%
\pgfpathlineto{\pgfqpoint{6.392857in}{1.353236in}}%
\pgfpathlineto{\pgfqpoint{6.408291in}{1.394421in}}%
\pgfpathlineto{\pgfqpoint{6.423724in}{1.442814in}}%
\pgfpathlineto{\pgfqpoint{6.439158in}{1.497984in}}%
\pgfpathlineto{\pgfqpoint{6.454592in}{1.559394in}}%
\pgfpathlineto{\pgfqpoint{6.485459in}{1.698285in}}%
\pgfpathlineto{\pgfqpoint{6.516326in}{1.853352in}}%
\pgfpathlineto{\pgfqpoint{6.608928in}{2.341298in}}%
\pgfpathlineto{\pgfqpoint{6.639795in}{2.485911in}}%
\pgfpathlineto{\pgfqpoint{6.655229in}{2.550899in}}%
\pgfpathlineto{\pgfqpoint{6.670663in}{2.610064in}}%
\pgfpathlineto{\pgfqpoint{6.686096in}{2.662816in}}%
\pgfpathlineto{\pgfqpoint{6.701530in}{2.708668in}}%
\pgfpathlineto{\pgfqpoint{6.716964in}{2.747245in}}%
\pgfpathlineto{\pgfqpoint{6.732397in}{2.778285in}}%
\pgfpathlineto{\pgfqpoint{6.747831in}{2.801644in}}%
\pgfpathlineto{\pgfqpoint{6.763264in}{2.817293in}}%
\pgfpathlineto{\pgfqpoint{6.778698in}{2.825317in}}%
\pgfpathlineto{\pgfqpoint{6.794132in}{2.825911in}}%
\pgfpathlineto{\pgfqpoint{6.809565in}{2.819368in}}%
\pgfpathlineto{\pgfqpoint{6.824999in}{2.806076in}}%
\pgfpathlineto{\pgfqpoint{6.840433in}{2.786506in}}%
\pgfpathlineto{\pgfqpoint{6.855866in}{2.761197in}}%
\pgfpathlineto{\pgfqpoint{6.871300in}{2.730747in}}%
\pgfpathlineto{\pgfqpoint{6.886734in}{2.695794in}}%
\pgfpathlineto{\pgfqpoint{6.902167in}{2.657007in}}%
\pgfpathlineto{\pgfqpoint{6.933034in}{2.570662in}}%
\pgfpathlineto{\pgfqpoint{6.979335in}{2.429130in}}%
\pgfpathlineto{\pgfqpoint{7.025636in}{2.287033in}}%
\pgfpathlineto{\pgfqpoint{7.056504in}{2.197697in}}%
\pgfpathlineto{\pgfqpoint{7.087371in}{2.114931in}}%
\pgfpathlineto{\pgfqpoint{7.118238in}{2.039244in}}%
\pgfpathlineto{\pgfqpoint{7.149105in}{1.969987in}}%
\pgfpathlineto{\pgfqpoint{7.195406in}{1.874509in}}%
\pgfpathlineto{\pgfqpoint{7.272574in}{1.718486in}}%
\pgfpathlineto{\pgfqpoint{7.303442in}{1.650867in}}%
\pgfpathlineto{\pgfqpoint{7.334309in}{1.578425in}}%
\pgfpathlineto{\pgfqpoint{7.380610in}{1.461407in}}%
\pgfpathlineto{\pgfqpoint{7.457778in}{1.261736in}}%
\pgfpathlineto{\pgfqpoint{7.488645in}{1.191050in}}%
\pgfpathlineto{\pgfqpoint{7.504079in}{1.160034in}}%
\pgfpathlineto{\pgfqpoint{7.519513in}{1.132745in}}%
\pgfpathlineto{\pgfqpoint{7.534946in}{1.109803in}}%
\pgfpathlineto{\pgfqpoint{7.550380in}{1.091804in}}%
\pgfpathlineto{\pgfqpoint{7.565814in}{1.079305in}}%
\pgfpathlineto{\pgfqpoint{7.581247in}{1.072806in}}%
\pgfpathlineto{\pgfqpoint{7.596681in}{1.072744in}}%
\pgfpathlineto{\pgfqpoint{7.612114in}{1.079475in}}%
\pgfpathlineto{\pgfqpoint{7.627548in}{1.093266in}}%
\pgfpathlineto{\pgfqpoint{7.642982in}{1.114286in}}%
\pgfpathlineto{\pgfqpoint{7.658415in}{1.142600in}}%
\pgfpathlineto{\pgfqpoint{7.673849in}{1.178161in}}%
\pgfpathlineto{\pgfqpoint{7.689283in}{1.220811in}}%
\pgfpathlineto{\pgfqpoint{7.704716in}{1.270275in}}%
\pgfpathlineto{\pgfqpoint{7.720150in}{1.326169in}}%
\pgfpathlineto{\pgfqpoint{7.735584in}{1.387999in}}%
\pgfpathlineto{\pgfqpoint{7.766451in}{1.526990in}}%
\pgfpathlineto{\pgfqpoint{7.797318in}{1.681421in}}%
\pgfpathlineto{\pgfqpoint{7.889920in}{2.164752in}}%
\pgfpathlineto{\pgfqpoint{7.920787in}{2.306945in}}%
\pgfpathlineto{\pgfqpoint{7.936221in}{2.370475in}}%
\pgfpathlineto{\pgfqpoint{7.951654in}{2.427976in}}%
\pgfpathlineto{\pgfqpoint{7.967088in}{2.478819in}}%
\pgfpathlineto{\pgfqpoint{7.982522in}{2.522479in}}%
\pgfpathlineto{\pgfqpoint{7.997955in}{2.558543in}}%
\pgfpathlineto{\pgfqpoint{8.013389in}{2.586710in}}%
\pgfpathlineto{\pgfqpoint{8.028823in}{2.606801in}}%
\pgfpathlineto{\pgfqpoint{8.044256in}{2.618754in}}%
\pgfpathlineto{\pgfqpoint{8.059690in}{2.622629in}}%
\pgfpathlineto{\pgfqpoint{8.075123in}{2.618598in}}%
\pgfpathlineto{\pgfqpoint{8.090557in}{2.606945in}}%
\pgfpathlineto{\pgfqpoint{8.105991in}{2.588052in}}%
\pgfpathlineto{\pgfqpoint{8.121424in}{2.562395in}}%
\pgfpathlineto{\pgfqpoint{8.136858in}{2.530532in}}%
\pgfpathlineto{\pgfqpoint{8.152292in}{2.493084in}}%
\pgfpathlineto{\pgfqpoint{8.167725in}{2.450730in}}%
\pgfpathlineto{\pgfqpoint{8.198593in}{2.354198in}}%
\pgfpathlineto{\pgfqpoint{8.229460in}{2.246874in}}%
\pgfpathlineto{\pgfqpoint{8.322062in}{1.915662in}}%
\pgfpathlineto{\pgfqpoint{8.352929in}{1.816648in}}%
\pgfpathlineto{\pgfqpoint{8.383796in}{1.727622in}}%
\pgfpathlineto{\pgfqpoint{8.414663in}{1.649177in}}%
\pgfpathlineto{\pgfqpoint{8.445531in}{1.580688in}}%
\pgfpathlineto{\pgfqpoint{8.476398in}{1.520538in}}%
\pgfpathlineto{\pgfqpoint{8.507265in}{1.466441in}}%
\pgfpathlineto{\pgfqpoint{8.569000in}{1.366221in}}%
\pgfpathlineto{\pgfqpoint{8.615301in}{1.289617in}}%
\pgfpathlineto{\pgfqpoint{8.661602in}{1.208063in}}%
\pgfpathlineto{\pgfqpoint{8.723336in}{1.096883in}}%
\pgfpathlineto{\pgfqpoint{8.754203in}{1.046050in}}%
\pgfpathlineto{\pgfqpoint{8.769637in}{1.023452in}}%
\pgfpathlineto{\pgfqpoint{8.785071in}{1.003456in}}%
\pgfpathlineto{\pgfqpoint{8.800504in}{0.986621in}}%
\pgfpathlineto{\pgfqpoint{8.815938in}{0.973512in}}%
\pgfpathlineto{\pgfqpoint{8.831372in}{0.964674in}}%
\pgfpathlineto{\pgfqpoint{8.846805in}{0.960631in}}%
\pgfpathlineto{\pgfqpoint{8.862239in}{0.961861in}}%
\pgfpathlineto{\pgfqpoint{8.877673in}{0.968790in}}%
\pgfpathlineto{\pgfqpoint{8.893106in}{0.981777in}}%
\pgfpathlineto{\pgfqpoint{8.908540in}{1.001101in}}%
\pgfpathlineto{\pgfqpoint{8.923973in}{1.026954in}}%
\pgfpathlineto{\pgfqpoint{8.939407in}{1.059434in}}%
\pgfpathlineto{\pgfqpoint{8.954841in}{1.098532in}}%
\pgfpathlineto{\pgfqpoint{8.970274in}{1.144137in}}%
\pgfpathlineto{\pgfqpoint{8.985708in}{1.196025in}}%
\pgfpathlineto{\pgfqpoint{9.001142in}{1.253865in}}%
\pgfpathlineto{\pgfqpoint{9.016575in}{1.317220in}}%
\pgfpathlineto{\pgfqpoint{9.047443in}{1.458217in}}%
\pgfpathlineto{\pgfqpoint{9.078310in}{1.613611in}}%
\pgfpathlineto{\pgfqpoint{9.170912in}{2.096682in}}%
\pgfpathlineto{\pgfqpoint{9.201779in}{2.238229in}}%
\pgfpathlineto{\pgfqpoint{9.217212in}{2.301324in}}%
\pgfpathlineto{\pgfqpoint{9.232646in}{2.358290in}}%
\pgfpathlineto{\pgfqpoint{9.248080in}{2.408474in}}%
\pgfpathlineto{\pgfqpoint{9.263513in}{2.451324in}}%
\pgfpathlineto{\pgfqpoint{9.278947in}{2.486395in}}%
\pgfpathlineto{\pgfqpoint{9.294381in}{2.513355in}}%
\pgfpathlineto{\pgfqpoint{9.309814in}{2.531993in}}%
\pgfpathlineto{\pgfqpoint{9.325248in}{2.542221in}}%
\pgfpathlineto{\pgfqpoint{9.340682in}{2.544068in}}%
\pgfpathlineto{\pgfqpoint{9.356115in}{2.537689in}}%
\pgfpathlineto{\pgfqpoint{9.371549in}{2.523348in}}%
\pgfpathlineto{\pgfqpoint{9.386982in}{2.501422in}}%
\pgfpathlineto{\pgfqpoint{9.402416in}{2.472385in}}%
\pgfpathlineto{\pgfqpoint{9.417850in}{2.436801in}}%
\pgfpathlineto{\pgfqpoint{9.433283in}{2.395311in}}%
\pgfpathlineto{\pgfqpoint{9.448717in}{2.348621in}}%
\pgfpathlineto{\pgfqpoint{9.479584in}{2.242688in}}%
\pgfpathlineto{\pgfqpoint{9.510452in}{2.125356in}}%
\pgfpathlineto{\pgfqpoint{9.587620in}{1.823080in}}%
\pgfpathlineto{\pgfqpoint{9.618487in}{1.712587in}}%
\pgfpathlineto{\pgfqpoint{9.649354in}{1.613772in}}%
\pgfpathlineto{\pgfqpoint{9.664788in}{1.569407in}}%
\pgfpathlineto{\pgfqpoint{9.680222in}{1.528576in}}%
\pgfpathlineto{\pgfqpoint{9.695655in}{1.491303in}}%
\pgfpathlineto{\pgfqpoint{9.711089in}{1.457528in}}%
\pgfpathlineto{\pgfqpoint{9.726522in}{1.427116in}}%
\pgfpathlineto{\pgfqpoint{9.741956in}{1.399869in}}%
\pgfpathlineto{\pgfqpoint{9.741956in}{1.399869in}}%
\pgfusepath{stroke}%
\end{pgfscope}%
\begin{pgfscope}%
\pgfpathrectangle{\pgfqpoint{5.706832in}{0.521603in}}{\pgfqpoint{4.227273in}{2.800000in}} %
\pgfusepath{clip}%
\pgfsetrectcap%
\pgfsetroundjoin%
\pgfsetlinewidth{0.501875pt}%
\definecolor{currentstroke}{rgb}{0.100000,0.587785,0.951057}%
\pgfsetstrokecolor{currentstroke}%
\pgfsetdash{}{0pt}%
\pgfpathmoveto{\pgfqpoint{5.898981in}{2.492288in}}%
\pgfpathlineto{\pgfqpoint{5.914415in}{2.464385in}}%
\pgfpathlineto{\pgfqpoint{5.929848in}{2.433387in}}%
\pgfpathlineto{\pgfqpoint{5.960715in}{2.363311in}}%
\pgfpathlineto{\pgfqpoint{5.991583in}{2.284835in}}%
\pgfpathlineto{\pgfqpoint{6.037884in}{2.158623in}}%
\pgfpathlineto{\pgfqpoint{6.084185in}{2.033246in}}%
\pgfpathlineto{\pgfqpoint{6.115052in}{1.956406in}}%
\pgfpathlineto{\pgfqpoint{6.130485in}{1.921367in}}%
\pgfpathlineto{\pgfqpoint{6.145919in}{1.889141in}}%
\pgfpathlineto{\pgfqpoint{6.161353in}{1.860109in}}%
\pgfpathlineto{\pgfqpoint{6.176786in}{1.834620in}}%
\pgfpathlineto{\pgfqpoint{6.192220in}{1.812983in}}%
\pgfpathlineto{\pgfqpoint{6.207654in}{1.795465in}}%
\pgfpathlineto{\pgfqpoint{6.223087in}{1.782286in}}%
\pgfpathlineto{\pgfqpoint{6.238521in}{1.773617in}}%
\pgfpathlineto{\pgfqpoint{6.253955in}{1.769576in}}%
\pgfpathlineto{\pgfqpoint{6.269388in}{1.770230in}}%
\pgfpathlineto{\pgfqpoint{6.284822in}{1.775591in}}%
\pgfpathlineto{\pgfqpoint{6.300255in}{1.785614in}}%
\pgfpathlineto{\pgfqpoint{6.315689in}{1.800202in}}%
\pgfpathlineto{\pgfqpoint{6.331123in}{1.819205in}}%
\pgfpathlineto{\pgfqpoint{6.346556in}{1.842419in}}%
\pgfpathlineto{\pgfqpoint{6.361990in}{1.869592in}}%
\pgfpathlineto{\pgfqpoint{6.377424in}{1.900427in}}%
\pgfpathlineto{\pgfqpoint{6.392857in}{1.934581in}}%
\pgfpathlineto{\pgfqpoint{6.423724in}{2.011297in}}%
\pgfpathlineto{\pgfqpoint{6.454592in}{2.096318in}}%
\pgfpathlineto{\pgfqpoint{6.547194in}{2.362416in}}%
\pgfpathlineto{\pgfqpoint{6.578061in}{2.441637in}}%
\pgfpathlineto{\pgfqpoint{6.593494in}{2.477405in}}%
\pgfpathlineto{\pgfqpoint{6.608928in}{2.510085in}}%
\pgfpathlineto{\pgfqpoint{6.624362in}{2.539329in}}%
\pgfpathlineto{\pgfqpoint{6.639795in}{2.564829in}}%
\pgfpathlineto{\pgfqpoint{6.655229in}{2.586323in}}%
\pgfpathlineto{\pgfqpoint{6.670663in}{2.603596in}}%
\pgfpathlineto{\pgfqpoint{6.686096in}{2.616483in}}%
\pgfpathlineto{\pgfqpoint{6.701530in}{2.624870in}}%
\pgfpathlineto{\pgfqpoint{6.716964in}{2.628695in}}%
\pgfpathlineto{\pgfqpoint{6.732397in}{2.627949in}}%
\pgfpathlineto{\pgfqpoint{6.747831in}{2.622674in}}%
\pgfpathlineto{\pgfqpoint{6.763264in}{2.612964in}}%
\pgfpathlineto{\pgfqpoint{6.778698in}{2.598961in}}%
\pgfpathlineto{\pgfqpoint{6.794132in}{2.580854in}}%
\pgfpathlineto{\pgfqpoint{6.809565in}{2.558877in}}%
\pgfpathlineto{\pgfqpoint{6.824999in}{2.533303in}}%
\pgfpathlineto{\pgfqpoint{6.840433in}{2.504441in}}%
\pgfpathlineto{\pgfqpoint{6.855866in}{2.472632in}}%
\pgfpathlineto{\pgfqpoint{6.886734in}{2.401673in}}%
\pgfpathlineto{\pgfqpoint{6.917601in}{2.323607in}}%
\pgfpathlineto{\pgfqpoint{7.010203in}{2.080241in}}%
\pgfpathlineto{\pgfqpoint{7.041070in}{2.006558in}}%
\pgfpathlineto{\pgfqpoint{7.071937in}{1.940958in}}%
\pgfpathlineto{\pgfqpoint{7.087371in}{1.911784in}}%
\pgfpathlineto{\pgfqpoint{7.102804in}{1.885264in}}%
\pgfpathlineto{\pgfqpoint{7.118238in}{1.861523in}}%
\pgfpathlineto{\pgfqpoint{7.133672in}{1.840643in}}%
\pgfpathlineto{\pgfqpoint{7.149105in}{1.822660in}}%
\pgfpathlineto{\pgfqpoint{7.164539in}{1.807569in}}%
\pgfpathlineto{\pgfqpoint{7.179973in}{1.795321in}}%
\pgfpathlineto{\pgfqpoint{7.195406in}{1.785827in}}%
\pgfpathlineto{\pgfqpoint{7.210840in}{1.778958in}}%
\pgfpathlineto{\pgfqpoint{7.226274in}{1.774551in}}%
\pgfpathlineto{\pgfqpoint{7.241707in}{1.772406in}}%
\pgfpathlineto{\pgfqpoint{7.257141in}{1.772298in}}%
\pgfpathlineto{\pgfqpoint{7.272574in}{1.773974in}}%
\pgfpathlineto{\pgfqpoint{7.288008in}{1.777159in}}%
\pgfpathlineto{\pgfqpoint{7.318875in}{1.786883in}}%
\pgfpathlineto{\pgfqpoint{7.396044in}{1.816429in}}%
\pgfpathlineto{\pgfqpoint{7.426911in}{1.824138in}}%
\pgfpathlineto{\pgfqpoint{7.442344in}{1.826079in}}%
\pgfpathlineto{\pgfqpoint{7.457778in}{1.826473in}}%
\pgfpathlineto{\pgfqpoint{7.473212in}{1.825154in}}%
\pgfpathlineto{\pgfqpoint{7.488645in}{1.821990in}}%
\pgfpathlineto{\pgfqpoint{7.504079in}{1.816886in}}%
\pgfpathlineto{\pgfqpoint{7.519513in}{1.809784in}}%
\pgfpathlineto{\pgfqpoint{7.534946in}{1.800664in}}%
\pgfpathlineto{\pgfqpoint{7.550380in}{1.789545in}}%
\pgfpathlineto{\pgfqpoint{7.565814in}{1.776485in}}%
\pgfpathlineto{\pgfqpoint{7.581247in}{1.761577in}}%
\pgfpathlineto{\pgfqpoint{7.612114in}{1.726778in}}%
\pgfpathlineto{\pgfqpoint{7.642982in}{1.686583in}}%
\pgfpathlineto{\pgfqpoint{7.689283in}{1.620400in}}%
\pgfpathlineto{\pgfqpoint{7.735584in}{1.554039in}}%
\pgfpathlineto{\pgfqpoint{7.766451in}{1.513783in}}%
\pgfpathlineto{\pgfqpoint{7.781884in}{1.495749in}}%
\pgfpathlineto{\pgfqpoint{7.797318in}{1.479482in}}%
\pgfpathlineto{\pgfqpoint{7.812752in}{1.465233in}}%
\pgfpathlineto{\pgfqpoint{7.828185in}{1.453231in}}%
\pgfpathlineto{\pgfqpoint{7.843619in}{1.443673in}}%
\pgfpathlineto{\pgfqpoint{7.859053in}{1.436726in}}%
\pgfpathlineto{\pgfqpoint{7.874486in}{1.432521in}}%
\pgfpathlineto{\pgfqpoint{7.889920in}{1.431149in}}%
\pgfpathlineto{\pgfqpoint{7.905353in}{1.432664in}}%
\pgfpathlineto{\pgfqpoint{7.920787in}{1.437076in}}%
\pgfpathlineto{\pgfqpoint{7.936221in}{1.444354in}}%
\pgfpathlineto{\pgfqpoint{7.951654in}{1.454423in}}%
\pgfpathlineto{\pgfqpoint{7.967088in}{1.467168in}}%
\pgfpathlineto{\pgfqpoint{7.982522in}{1.482430in}}%
\pgfpathlineto{\pgfqpoint{7.997955in}{1.500012in}}%
\pgfpathlineto{\pgfqpoint{8.013389in}{1.519678in}}%
\pgfpathlineto{\pgfqpoint{8.044256in}{1.564151in}}%
\pgfpathlineto{\pgfqpoint{8.075123in}{1.613329in}}%
\pgfpathlineto{\pgfqpoint{8.136858in}{1.713970in}}%
\pgfpathlineto{\pgfqpoint{8.167725in}{1.759108in}}%
\pgfpathlineto{\pgfqpoint{8.183159in}{1.779009in}}%
\pgfpathlineto{\pgfqpoint{8.198593in}{1.796640in}}%
\pgfpathlineto{\pgfqpoint{8.214026in}{1.811660in}}%
\pgfpathlineto{\pgfqpoint{8.229460in}{1.823756in}}%
\pgfpathlineto{\pgfqpoint{8.244893in}{1.832646in}}%
\pgfpathlineto{\pgfqpoint{8.260327in}{1.838085in}}%
\pgfpathlineto{\pgfqpoint{8.275761in}{1.839869in}}%
\pgfpathlineto{\pgfqpoint{8.291194in}{1.837835in}}%
\pgfpathlineto{\pgfqpoint{8.306628in}{1.831867in}}%
\pgfpathlineto{\pgfqpoint{8.322062in}{1.821899in}}%
\pgfpathlineto{\pgfqpoint{8.337495in}{1.807913in}}%
\pgfpathlineto{\pgfqpoint{8.352929in}{1.789941in}}%
\pgfpathlineto{\pgfqpoint{8.368363in}{1.768067in}}%
\pgfpathlineto{\pgfqpoint{8.383796in}{1.742426in}}%
\pgfpathlineto{\pgfqpoint{8.399230in}{1.713200in}}%
\pgfpathlineto{\pgfqpoint{8.414663in}{1.680621in}}%
\pgfpathlineto{\pgfqpoint{8.445531in}{1.606552in}}%
\pgfpathlineto{\pgfqpoint{8.476398in}{1.522916in}}%
\pgfpathlineto{\pgfqpoint{8.522699in}{1.386744in}}%
\pgfpathlineto{\pgfqpoint{8.569000in}{1.248924in}}%
\pgfpathlineto{\pgfqpoint{8.599867in}{1.162638in}}%
\pgfpathlineto{\pgfqpoint{8.630734in}{1.085251in}}%
\pgfpathlineto{\pgfqpoint{8.646168in}{1.050987in}}%
\pgfpathlineto{\pgfqpoint{8.661602in}{1.020201in}}%
\pgfpathlineto{\pgfqpoint{8.677035in}{0.993242in}}%
\pgfpathlineto{\pgfqpoint{8.692469in}{0.970420in}}%
\pgfpathlineto{\pgfqpoint{8.707903in}{0.951998in}}%
\pgfpathlineto{\pgfqpoint{8.723336in}{0.938193in}}%
\pgfpathlineto{\pgfqpoint{8.738770in}{0.929170in}}%
\pgfpathlineto{\pgfqpoint{8.754203in}{0.925042in}}%
\pgfpathlineto{\pgfqpoint{8.769637in}{0.925867in}}%
\pgfpathlineto{\pgfqpoint{8.785071in}{0.931650in}}%
\pgfpathlineto{\pgfqpoint{8.800504in}{0.942339in}}%
\pgfpathlineto{\pgfqpoint{8.815938in}{0.957828in}}%
\pgfpathlineto{\pgfqpoint{8.831372in}{0.977959in}}%
\pgfpathlineto{\pgfqpoint{8.846805in}{1.002521in}}%
\pgfpathlineto{\pgfqpoint{8.862239in}{1.031256in}}%
\pgfpathlineto{\pgfqpoint{8.877673in}{1.063860in}}%
\pgfpathlineto{\pgfqpoint{8.893106in}{1.099988in}}%
\pgfpathlineto{\pgfqpoint{8.923973in}{1.181250in}}%
\pgfpathlineto{\pgfqpoint{8.954841in}{1.271619in}}%
\pgfpathlineto{\pgfqpoint{9.062876in}{1.604092in}}%
\pgfpathlineto{\pgfqpoint{9.093743in}{1.687974in}}%
\pgfpathlineto{\pgfqpoint{9.109177in}{1.725938in}}%
\pgfpathlineto{\pgfqpoint{9.124611in}{1.760794in}}%
\pgfpathlineto{\pgfqpoint{9.140044in}{1.792253in}}%
\pgfpathlineto{\pgfqpoint{9.155478in}{1.820070in}}%
\pgfpathlineto{\pgfqpoint{9.170912in}{1.844044in}}%
\pgfpathlineto{\pgfqpoint{9.186345in}{1.864027in}}%
\pgfpathlineto{\pgfqpoint{9.201779in}{1.879918in}}%
\pgfpathlineto{\pgfqpoint{9.217212in}{1.891667in}}%
\pgfpathlineto{\pgfqpoint{9.232646in}{1.899277in}}%
\pgfpathlineto{\pgfqpoint{9.248080in}{1.902798in}}%
\pgfpathlineto{\pgfqpoint{9.263513in}{1.902329in}}%
\pgfpathlineto{\pgfqpoint{9.278947in}{1.898016in}}%
\pgfpathlineto{\pgfqpoint{9.294381in}{1.890048in}}%
\pgfpathlineto{\pgfqpoint{9.309814in}{1.878655in}}%
\pgfpathlineto{\pgfqpoint{9.325248in}{1.864104in}}%
\pgfpathlineto{\pgfqpoint{9.340682in}{1.846694in}}%
\pgfpathlineto{\pgfqpoint{9.356115in}{1.826753in}}%
\pgfpathlineto{\pgfqpoint{9.386982in}{1.780704in}}%
\pgfpathlineto{\pgfqpoint{9.417850in}{1.728959in}}%
\pgfpathlineto{\pgfqpoint{9.495018in}{1.595047in}}%
\pgfpathlineto{\pgfqpoint{9.525885in}{1.547289in}}%
\pgfpathlineto{\pgfqpoint{9.541319in}{1.525903in}}%
\pgfpathlineto{\pgfqpoint{9.556752in}{1.506504in}}%
\pgfpathlineto{\pgfqpoint{9.572186in}{1.489294in}}%
\pgfpathlineto{\pgfqpoint{9.587620in}{1.474433in}}%
\pgfpathlineto{\pgfqpoint{9.603053in}{1.462042in}}%
\pgfpathlineto{\pgfqpoint{9.618487in}{1.452201in}}%
\pgfpathlineto{\pgfqpoint{9.633921in}{1.444947in}}%
\pgfpathlineto{\pgfqpoint{9.649354in}{1.440275in}}%
\pgfpathlineto{\pgfqpoint{9.664788in}{1.438140in}}%
\pgfpathlineto{\pgfqpoint{9.680222in}{1.438457in}}%
\pgfpathlineto{\pgfqpoint{9.695655in}{1.441103in}}%
\pgfpathlineto{\pgfqpoint{9.711089in}{1.445921in}}%
\pgfpathlineto{\pgfqpoint{9.726522in}{1.452721in}}%
\pgfpathlineto{\pgfqpoint{9.741956in}{1.461284in}}%
\pgfpathlineto{\pgfqpoint{9.741956in}{1.461284in}}%
\pgfusepath{stroke}%
\end{pgfscope}%
\begin{pgfscope}%
\pgfpathrectangle{\pgfqpoint{5.706832in}{0.521603in}}{\pgfqpoint{4.227273in}{2.800000in}} %
\pgfusepath{clip}%
\pgfsetrectcap%
\pgfsetroundjoin%
\pgfsetlinewidth{0.501875pt}%
\definecolor{currentstroke}{rgb}{0.021569,0.682749,0.930229}%
\pgfsetstrokecolor{currentstroke}%
\pgfsetdash{}{0pt}%
\pgfpathmoveto{\pgfqpoint{5.898981in}{2.407150in}}%
\pgfpathlineto{\pgfqpoint{5.914415in}{2.397080in}}%
\pgfpathlineto{\pgfqpoint{5.929848in}{2.383336in}}%
\pgfpathlineto{\pgfqpoint{5.945282in}{2.366131in}}%
\pgfpathlineto{\pgfqpoint{5.960715in}{2.345736in}}%
\pgfpathlineto{\pgfqpoint{5.976149in}{2.322469in}}%
\pgfpathlineto{\pgfqpoint{6.007016in}{2.268828in}}%
\pgfpathlineto{\pgfqpoint{6.037884in}{2.208604in}}%
\pgfpathlineto{\pgfqpoint{6.099618in}{2.084134in}}%
\pgfpathlineto{\pgfqpoint{6.130485in}{2.028155in}}%
\pgfpathlineto{\pgfqpoint{6.145919in}{2.003489in}}%
\pgfpathlineto{\pgfqpoint{6.161353in}{1.981655in}}%
\pgfpathlineto{\pgfqpoint{6.176786in}{1.963071in}}%
\pgfpathlineto{\pgfqpoint{6.192220in}{1.948118in}}%
\pgfpathlineto{\pgfqpoint{6.207654in}{1.937127in}}%
\pgfpathlineto{\pgfqpoint{6.223087in}{1.930380in}}%
\pgfpathlineto{\pgfqpoint{6.238521in}{1.928101in}}%
\pgfpathlineto{\pgfqpoint{6.253955in}{1.930457in}}%
\pgfpathlineto{\pgfqpoint{6.269388in}{1.937551in}}%
\pgfpathlineto{\pgfqpoint{6.284822in}{1.949422in}}%
\pgfpathlineto{\pgfqpoint{6.300255in}{1.966044in}}%
\pgfpathlineto{\pgfqpoint{6.315689in}{1.987325in}}%
\pgfpathlineto{\pgfqpoint{6.331123in}{2.013109in}}%
\pgfpathlineto{\pgfqpoint{6.346556in}{2.043178in}}%
\pgfpathlineto{\pgfqpoint{6.361990in}{2.077250in}}%
\pgfpathlineto{\pgfqpoint{6.377424in}{2.114990in}}%
\pgfpathlineto{\pgfqpoint{6.408291in}{2.199865in}}%
\pgfpathlineto{\pgfqpoint{6.439158in}{2.294138in}}%
\pgfpathlineto{\pgfqpoint{6.531760in}{2.589370in}}%
\pgfpathlineto{\pgfqpoint{6.562627in}{2.676427in}}%
\pgfpathlineto{\pgfqpoint{6.578061in}{2.715328in}}%
\pgfpathlineto{\pgfqpoint{6.593494in}{2.750495in}}%
\pgfpathlineto{\pgfqpoint{6.608928in}{2.781494in}}%
\pgfpathlineto{\pgfqpoint{6.624362in}{2.807938in}}%
\pgfpathlineto{\pgfqpoint{6.639795in}{2.829496in}}%
\pgfpathlineto{\pgfqpoint{6.655229in}{2.845896in}}%
\pgfpathlineto{\pgfqpoint{6.670663in}{2.856926in}}%
\pgfpathlineto{\pgfqpoint{6.686096in}{2.862439in}}%
\pgfpathlineto{\pgfqpoint{6.701530in}{2.862357in}}%
\pgfpathlineto{\pgfqpoint{6.716964in}{2.856666in}}%
\pgfpathlineto{\pgfqpoint{6.732397in}{2.845420in}}%
\pgfpathlineto{\pgfqpoint{6.747831in}{2.828741in}}%
\pgfpathlineto{\pgfqpoint{6.763264in}{2.806814in}}%
\pgfpathlineto{\pgfqpoint{6.778698in}{2.779887in}}%
\pgfpathlineto{\pgfqpoint{6.794132in}{2.748266in}}%
\pgfpathlineto{\pgfqpoint{6.809565in}{2.712310in}}%
\pgfpathlineto{\pgfqpoint{6.824999in}{2.672429in}}%
\pgfpathlineto{\pgfqpoint{6.855866in}{2.582738in}}%
\pgfpathlineto{\pgfqpoint{6.886734in}{2.483222in}}%
\pgfpathlineto{\pgfqpoint{6.994769in}{2.121590in}}%
\pgfpathlineto{\pgfqpoint{7.025636in}{2.032053in}}%
\pgfpathlineto{\pgfqpoint{7.041070in}{1.991850in}}%
\pgfpathlineto{\pgfqpoint{7.056504in}{1.955137in}}%
\pgfpathlineto{\pgfqpoint{7.071937in}{1.922183in}}%
\pgfpathlineto{\pgfqpoint{7.087371in}{1.893197in}}%
\pgfpathlineto{\pgfqpoint{7.102804in}{1.868328in}}%
\pgfpathlineto{\pgfqpoint{7.118238in}{1.847661in}}%
\pgfpathlineto{\pgfqpoint{7.133672in}{1.831222in}}%
\pgfpathlineto{\pgfqpoint{7.149105in}{1.818969in}}%
\pgfpathlineto{\pgfqpoint{7.164539in}{1.810804in}}%
\pgfpathlineto{\pgfqpoint{7.179973in}{1.806567in}}%
\pgfpathlineto{\pgfqpoint{7.195406in}{1.806041in}}%
\pgfpathlineto{\pgfqpoint{7.210840in}{1.808957in}}%
\pgfpathlineto{\pgfqpoint{7.226274in}{1.815000in}}%
\pgfpathlineto{\pgfqpoint{7.241707in}{1.823807in}}%
\pgfpathlineto{\pgfqpoint{7.257141in}{1.834983in}}%
\pgfpathlineto{\pgfqpoint{7.272574in}{1.848096in}}%
\pgfpathlineto{\pgfqpoint{7.303442in}{1.878306in}}%
\pgfpathlineto{\pgfqpoint{7.349743in}{1.926389in}}%
\pgfpathlineto{\pgfqpoint{7.380610in}{1.954726in}}%
\pgfpathlineto{\pgfqpoint{7.396044in}{1.966433in}}%
\pgfpathlineto{\pgfqpoint{7.411477in}{1.975962in}}%
\pgfpathlineto{\pgfqpoint{7.426911in}{1.982957in}}%
\pgfpathlineto{\pgfqpoint{7.442344in}{1.987104in}}%
\pgfpathlineto{\pgfqpoint{7.457778in}{1.988135in}}%
\pgfpathlineto{\pgfqpoint{7.473212in}{1.985836in}}%
\pgfpathlineto{\pgfqpoint{7.488645in}{1.980048in}}%
\pgfpathlineto{\pgfqpoint{7.504079in}{1.970668in}}%
\pgfpathlineto{\pgfqpoint{7.519513in}{1.957652in}}%
\pgfpathlineto{\pgfqpoint{7.534946in}{1.941016in}}%
\pgfpathlineto{\pgfqpoint{7.550380in}{1.920838in}}%
\pgfpathlineto{\pgfqpoint{7.565814in}{1.897250in}}%
\pgfpathlineto{\pgfqpoint{7.581247in}{1.870444in}}%
\pgfpathlineto{\pgfqpoint{7.596681in}{1.840666in}}%
\pgfpathlineto{\pgfqpoint{7.627548in}{1.773420in}}%
\pgfpathlineto{\pgfqpoint{7.658415in}{1.698403in}}%
\pgfpathlineto{\pgfqpoint{7.751017in}{1.462947in}}%
\pgfpathlineto{\pgfqpoint{7.781884in}{1.394008in}}%
\pgfpathlineto{\pgfqpoint{7.797318in}{1.363440in}}%
\pgfpathlineto{\pgfqpoint{7.812752in}{1.336019in}}%
\pgfpathlineto{\pgfqpoint{7.828185in}{1.312109in}}%
\pgfpathlineto{\pgfqpoint{7.843619in}{1.292025in}}%
\pgfpathlineto{\pgfqpoint{7.859053in}{1.276036in}}%
\pgfpathlineto{\pgfqpoint{7.874486in}{1.264356in}}%
\pgfpathlineto{\pgfqpoint{7.889920in}{1.257140in}}%
\pgfpathlineto{\pgfqpoint{7.905353in}{1.254485in}}%
\pgfpathlineto{\pgfqpoint{7.920787in}{1.256425in}}%
\pgfpathlineto{\pgfqpoint{7.936221in}{1.262932in}}%
\pgfpathlineto{\pgfqpoint{7.951654in}{1.273914in}}%
\pgfpathlineto{\pgfqpoint{7.967088in}{1.289217in}}%
\pgfpathlineto{\pgfqpoint{7.982522in}{1.308629in}}%
\pgfpathlineto{\pgfqpoint{7.997955in}{1.331876in}}%
\pgfpathlineto{\pgfqpoint{8.013389in}{1.358633in}}%
\pgfpathlineto{\pgfqpoint{8.028823in}{1.388524in}}%
\pgfpathlineto{\pgfqpoint{8.059690in}{1.455981in}}%
\pgfpathlineto{\pgfqpoint{8.090557in}{1.530431in}}%
\pgfpathlineto{\pgfqpoint{8.152292in}{1.683130in}}%
\pgfpathlineto{\pgfqpoint{8.183159in}{1.752521in}}%
\pgfpathlineto{\pgfqpoint{8.198593in}{1.783616in}}%
\pgfpathlineto{\pgfqpoint{8.214026in}{1.811669in}}%
\pgfpathlineto{\pgfqpoint{8.229460in}{1.836242in}}%
\pgfpathlineto{\pgfqpoint{8.244893in}{1.856944in}}%
\pgfpathlineto{\pgfqpoint{8.260327in}{1.873434in}}%
\pgfpathlineto{\pgfqpoint{8.275761in}{1.885423in}}%
\pgfpathlineto{\pgfqpoint{8.291194in}{1.892682in}}%
\pgfpathlineto{\pgfqpoint{8.306628in}{1.895044in}}%
\pgfpathlineto{\pgfqpoint{8.322062in}{1.892406in}}%
\pgfpathlineto{\pgfqpoint{8.337495in}{1.884732in}}%
\pgfpathlineto{\pgfqpoint{8.352929in}{1.872049in}}%
\pgfpathlineto{\pgfqpoint{8.368363in}{1.854453in}}%
\pgfpathlineto{\pgfqpoint{8.383796in}{1.832105in}}%
\pgfpathlineto{\pgfqpoint{8.399230in}{1.805229in}}%
\pgfpathlineto{\pgfqpoint{8.414663in}{1.774108in}}%
\pgfpathlineto{\pgfqpoint{8.430097in}{1.739081in}}%
\pgfpathlineto{\pgfqpoint{8.445531in}{1.700542in}}%
\pgfpathlineto{\pgfqpoint{8.476398in}{1.614721in}}%
\pgfpathlineto{\pgfqpoint{8.507265in}{1.520609in}}%
\pgfpathlineto{\pgfqpoint{8.584433in}{1.278431in}}%
\pgfpathlineto{\pgfqpoint{8.615301in}{1.190782in}}%
\pgfpathlineto{\pgfqpoint{8.630734in}{1.151075in}}%
\pgfpathlineto{\pgfqpoint{8.646168in}{1.114737in}}%
\pgfpathlineto{\pgfqpoint{8.661602in}{1.082183in}}%
\pgfpathlineto{\pgfqpoint{8.677035in}{1.053779in}}%
\pgfpathlineto{\pgfqpoint{8.692469in}{1.029837in}}%
\pgfpathlineto{\pgfqpoint{8.707903in}{1.010612in}}%
\pgfpathlineto{\pgfqpoint{8.723336in}{0.996297in}}%
\pgfpathlineto{\pgfqpoint{8.738770in}{0.987023in}}%
\pgfpathlineto{\pgfqpoint{8.754203in}{0.982856in}}%
\pgfpathlineto{\pgfqpoint{8.769637in}{0.983799in}}%
\pgfpathlineto{\pgfqpoint{8.785071in}{0.989789in}}%
\pgfpathlineto{\pgfqpoint{8.800504in}{1.000700in}}%
\pgfpathlineto{\pgfqpoint{8.815938in}{1.016346in}}%
\pgfpathlineto{\pgfqpoint{8.831372in}{1.036481in}}%
\pgfpathlineto{\pgfqpoint{8.846805in}{1.060807in}}%
\pgfpathlineto{\pgfqpoint{8.862239in}{1.088976in}}%
\pgfpathlineto{\pgfqpoint{8.877673in}{1.120594in}}%
\pgfpathlineto{\pgfqpoint{8.908540in}{1.192419in}}%
\pgfpathlineto{\pgfqpoint{8.939407in}{1.272478in}}%
\pgfpathlineto{\pgfqpoint{9.032009in}{1.520906in}}%
\pgfpathlineto{\pgfqpoint{9.062876in}{1.593329in}}%
\pgfpathlineto{\pgfqpoint{9.078310in}{1.625687in}}%
\pgfpathlineto{\pgfqpoint{9.093743in}{1.655042in}}%
\pgfpathlineto{\pgfqpoint{9.109177in}{1.681133in}}%
\pgfpathlineto{\pgfqpoint{9.124611in}{1.703751in}}%
\pgfpathlineto{\pgfqpoint{9.140044in}{1.722745in}}%
\pgfpathlineto{\pgfqpoint{9.155478in}{1.738024in}}%
\pgfpathlineto{\pgfqpoint{9.170912in}{1.749553in}}%
\pgfpathlineto{\pgfqpoint{9.186345in}{1.757360in}}%
\pgfpathlineto{\pgfqpoint{9.201779in}{1.761529in}}%
\pgfpathlineto{\pgfqpoint{9.217212in}{1.762201in}}%
\pgfpathlineto{\pgfqpoint{9.232646in}{1.759573in}}%
\pgfpathlineto{\pgfqpoint{9.248080in}{1.753892in}}%
\pgfpathlineto{\pgfqpoint{9.263513in}{1.745451in}}%
\pgfpathlineto{\pgfqpoint{9.278947in}{1.734586in}}%
\pgfpathlineto{\pgfqpoint{9.294381in}{1.721671in}}%
\pgfpathlineto{\pgfqpoint{9.325248in}{1.691331in}}%
\pgfpathlineto{\pgfqpoint{9.402416in}{1.610070in}}%
\pgfpathlineto{\pgfqpoint{9.417850in}{1.596523in}}%
\pgfpathlineto{\pgfqpoint{9.433283in}{1.584879in}}%
\pgfpathlineto{\pgfqpoint{9.448717in}{1.575517in}}%
\pgfpathlineto{\pgfqpoint{9.464151in}{1.568783in}}%
\pgfpathlineto{\pgfqpoint{9.479584in}{1.564980in}}%
\pgfpathlineto{\pgfqpoint{9.495018in}{1.564365in}}%
\pgfpathlineto{\pgfqpoint{9.510452in}{1.567146in}}%
\pgfpathlineto{\pgfqpoint{9.525885in}{1.573478in}}%
\pgfpathlineto{\pgfqpoint{9.541319in}{1.583460in}}%
\pgfpathlineto{\pgfqpoint{9.556752in}{1.597135in}}%
\pgfpathlineto{\pgfqpoint{9.572186in}{1.614484in}}%
\pgfpathlineto{\pgfqpoint{9.587620in}{1.635434in}}%
\pgfpathlineto{\pgfqpoint{9.603053in}{1.659849in}}%
\pgfpathlineto{\pgfqpoint{9.618487in}{1.687542in}}%
\pgfpathlineto{\pgfqpoint{9.633921in}{1.718269in}}%
\pgfpathlineto{\pgfqpoint{9.664788in}{1.787604in}}%
\pgfpathlineto{\pgfqpoint{9.695655in}{1.864973in}}%
\pgfpathlineto{\pgfqpoint{9.741956in}{1.988395in}}%
\pgfpathlineto{\pgfqpoint{9.741956in}{1.988395in}}%
\pgfusepath{stroke}%
\end{pgfscope}%
\begin{pgfscope}%
\pgfpathrectangle{\pgfqpoint{5.706832in}{0.521603in}}{\pgfqpoint{4.227273in}{2.800000in}} %
\pgfusepath{clip}%
\pgfsetrectcap%
\pgfsetroundjoin%
\pgfsetlinewidth{0.501875pt}%
\definecolor{currentstroke}{rgb}{0.056863,0.767363,0.905873}%
\pgfsetstrokecolor{currentstroke}%
\pgfsetdash{}{0pt}%
\pgfpathmoveto{\pgfqpoint{5.898981in}{2.187122in}}%
\pgfpathlineto{\pgfqpoint{5.914415in}{2.179841in}}%
\pgfpathlineto{\pgfqpoint{5.929848in}{2.170053in}}%
\pgfpathlineto{\pgfqpoint{5.945282in}{2.157898in}}%
\pgfpathlineto{\pgfqpoint{5.960715in}{2.143534in}}%
\pgfpathlineto{\pgfqpoint{5.976149in}{2.127144in}}%
\pgfpathlineto{\pgfqpoint{6.007016in}{2.089107in}}%
\pgfpathlineto{\pgfqpoint{6.037884in}{2.045623in}}%
\pgfpathlineto{\pgfqpoint{6.145919in}{1.885135in}}%
\pgfpathlineto{\pgfqpoint{6.161353in}{1.866087in}}%
\pgfpathlineto{\pgfqpoint{6.176786in}{1.849289in}}%
\pgfpathlineto{\pgfqpoint{6.192220in}{1.835197in}}%
\pgfpathlineto{\pgfqpoint{6.207654in}{1.824259in}}%
\pgfpathlineto{\pgfqpoint{6.223087in}{1.816909in}}%
\pgfpathlineto{\pgfqpoint{6.238521in}{1.813553in}}%
\pgfpathlineto{\pgfqpoint{6.253955in}{1.814559in}}%
\pgfpathlineto{\pgfqpoint{6.269388in}{1.820238in}}%
\pgfpathlineto{\pgfqpoint{6.284822in}{1.830839in}}%
\pgfpathlineto{\pgfqpoint{6.300255in}{1.846532in}}%
\pgfpathlineto{\pgfqpoint{6.315689in}{1.867397in}}%
\pgfpathlineto{\pgfqpoint{6.331123in}{1.893416in}}%
\pgfpathlineto{\pgfqpoint{6.346556in}{1.924467in}}%
\pgfpathlineto{\pgfqpoint{6.361990in}{1.960314in}}%
\pgfpathlineto{\pgfqpoint{6.377424in}{2.000614in}}%
\pgfpathlineto{\pgfqpoint{6.392857in}{2.044911in}}%
\pgfpathlineto{\pgfqpoint{6.423724in}{2.143163in}}%
\pgfpathlineto{\pgfqpoint{6.470025in}{2.303703in}}%
\pgfpathlineto{\pgfqpoint{6.500893in}{2.409660in}}%
\pgfpathlineto{\pgfqpoint{6.531760in}{2.506565in}}%
\pgfpathlineto{\pgfqpoint{6.547194in}{2.549576in}}%
\pgfpathlineto{\pgfqpoint{6.562627in}{2.588007in}}%
\pgfpathlineto{\pgfqpoint{6.578061in}{2.621269in}}%
\pgfpathlineto{\pgfqpoint{6.593494in}{2.648888in}}%
\pgfpathlineto{\pgfqpoint{6.608928in}{2.670522in}}%
\pgfpathlineto{\pgfqpoint{6.624362in}{2.685967in}}%
\pgfpathlineto{\pgfqpoint{6.639795in}{2.695163in}}%
\pgfpathlineto{\pgfqpoint{6.655229in}{2.698192in}}%
\pgfpathlineto{\pgfqpoint{6.670663in}{2.695272in}}%
\pgfpathlineto{\pgfqpoint{6.686096in}{2.686750in}}%
\pgfpathlineto{\pgfqpoint{6.701530in}{2.673086in}}%
\pgfpathlineto{\pgfqpoint{6.716964in}{2.654834in}}%
\pgfpathlineto{\pgfqpoint{6.732397in}{2.632626in}}%
\pgfpathlineto{\pgfqpoint{6.747831in}{2.607144in}}%
\pgfpathlineto{\pgfqpoint{6.778698in}{2.549195in}}%
\pgfpathlineto{\pgfqpoint{6.871300in}{2.365202in}}%
\pgfpathlineto{\pgfqpoint{6.902167in}{2.311863in}}%
\pgfpathlineto{\pgfqpoint{6.933034in}{2.264079in}}%
\pgfpathlineto{\pgfqpoint{6.979335in}{2.199244in}}%
\pgfpathlineto{\pgfqpoint{7.025636in}{2.134509in}}%
\pgfpathlineto{\pgfqpoint{7.056504in}{2.086884in}}%
\pgfpathlineto{\pgfqpoint{7.087371in}{2.033748in}}%
\pgfpathlineto{\pgfqpoint{7.118238in}{1.975126in}}%
\pgfpathlineto{\pgfqpoint{7.210840in}{1.792218in}}%
\pgfpathlineto{\pgfqpoint{7.226274in}{1.766493in}}%
\pgfpathlineto{\pgfqpoint{7.241707in}{1.743838in}}%
\pgfpathlineto{\pgfqpoint{7.257141in}{1.724844in}}%
\pgfpathlineto{\pgfqpoint{7.272574in}{1.710026in}}%
\pgfpathlineto{\pgfqpoint{7.288008in}{1.699793in}}%
\pgfpathlineto{\pgfqpoint{7.303442in}{1.694428in}}%
\pgfpathlineto{\pgfqpoint{7.318875in}{1.694067in}}%
\pgfpathlineto{\pgfqpoint{7.334309in}{1.698686in}}%
\pgfpathlineto{\pgfqpoint{7.349743in}{1.708094in}}%
\pgfpathlineto{\pgfqpoint{7.365176in}{1.721926in}}%
\pgfpathlineto{\pgfqpoint{7.380610in}{1.739656in}}%
\pgfpathlineto{\pgfqpoint{7.396044in}{1.760600in}}%
\pgfpathlineto{\pgfqpoint{7.426911in}{1.808738in}}%
\pgfpathlineto{\pgfqpoint{7.457778in}{1.858595in}}%
\pgfpathlineto{\pgfqpoint{7.473212in}{1.881493in}}%
\pgfpathlineto{\pgfqpoint{7.488645in}{1.901609in}}%
\pgfpathlineto{\pgfqpoint{7.504079in}{1.917932in}}%
\pgfpathlineto{\pgfqpoint{7.519513in}{1.929550in}}%
\pgfpathlineto{\pgfqpoint{7.534946in}{1.935677in}}%
\pgfpathlineto{\pgfqpoint{7.550380in}{1.935690in}}%
\pgfpathlineto{\pgfqpoint{7.565814in}{1.929152in}}%
\pgfpathlineto{\pgfqpoint{7.581247in}{1.915829in}}%
\pgfpathlineto{\pgfqpoint{7.596681in}{1.895710in}}%
\pgfpathlineto{\pgfqpoint{7.612114in}{1.869006in}}%
\pgfpathlineto{\pgfqpoint{7.627548in}{1.836151in}}%
\pgfpathlineto{\pgfqpoint{7.642982in}{1.797792in}}%
\pgfpathlineto{\pgfqpoint{7.658415in}{1.754773in}}%
\pgfpathlineto{\pgfqpoint{7.689283in}{1.658952in}}%
\pgfpathlineto{\pgfqpoint{7.735584in}{1.509430in}}%
\pgfpathlineto{\pgfqpoint{7.751017in}{1.463360in}}%
\pgfpathlineto{\pgfqpoint{7.766451in}{1.421333in}}%
\pgfpathlineto{\pgfqpoint{7.781884in}{1.384512in}}%
\pgfpathlineto{\pgfqpoint{7.797318in}{1.353920in}}%
\pgfpathlineto{\pgfqpoint{7.812752in}{1.330405in}}%
\pgfpathlineto{\pgfqpoint{7.828185in}{1.314615in}}%
\pgfpathlineto{\pgfqpoint{7.843619in}{1.306974in}}%
\pgfpathlineto{\pgfqpoint{7.859053in}{1.307677in}}%
\pgfpathlineto{\pgfqpoint{7.874486in}{1.316678in}}%
\pgfpathlineto{\pgfqpoint{7.889920in}{1.333694in}}%
\pgfpathlineto{\pgfqpoint{7.905353in}{1.358223in}}%
\pgfpathlineto{\pgfqpoint{7.920787in}{1.389555in}}%
\pgfpathlineto{\pgfqpoint{7.936221in}{1.426801in}}%
\pgfpathlineto{\pgfqpoint{7.951654in}{1.468927in}}%
\pgfpathlineto{\pgfqpoint{7.982522in}{1.563150in}}%
\pgfpathlineto{\pgfqpoint{8.028823in}{1.710874in}}%
\pgfpathlineto{\pgfqpoint{8.059690in}{1.799952in}}%
\pgfpathlineto{\pgfqpoint{8.075123in}{1.838745in}}%
\pgfpathlineto{\pgfqpoint{8.090557in}{1.872758in}}%
\pgfpathlineto{\pgfqpoint{8.105991in}{1.901511in}}%
\pgfpathlineto{\pgfqpoint{8.121424in}{1.924713in}}%
\pgfpathlineto{\pgfqpoint{8.136858in}{1.942268in}}%
\pgfpathlineto{\pgfqpoint{8.152292in}{1.954270in}}%
\pgfpathlineto{\pgfqpoint{8.167725in}{1.960981in}}%
\pgfpathlineto{\pgfqpoint{8.183159in}{1.962821in}}%
\pgfpathlineto{\pgfqpoint{8.198593in}{1.960335in}}%
\pgfpathlineto{\pgfqpoint{8.214026in}{1.954166in}}%
\pgfpathlineto{\pgfqpoint{8.229460in}{1.945025in}}%
\pgfpathlineto{\pgfqpoint{8.244893in}{1.933651in}}%
\pgfpathlineto{\pgfqpoint{8.275761in}{1.907124in}}%
\pgfpathlineto{\pgfqpoint{8.306628in}{1.879872in}}%
\pgfpathlineto{\pgfqpoint{8.337495in}{1.855721in}}%
\pgfpathlineto{\pgfqpoint{8.368363in}{1.836428in}}%
\pgfpathlineto{\pgfqpoint{8.399230in}{1.821465in}}%
\pgfpathlineto{\pgfqpoint{8.445531in}{1.800998in}}%
\pgfpathlineto{\pgfqpoint{8.460964in}{1.792589in}}%
\pgfpathlineto{\pgfqpoint{8.476398in}{1.782389in}}%
\pgfpathlineto{\pgfqpoint{8.491832in}{1.769811in}}%
\pgfpathlineto{\pgfqpoint{8.507265in}{1.754324in}}%
\pgfpathlineto{\pgfqpoint{8.522699in}{1.735482in}}%
\pgfpathlineto{\pgfqpoint{8.538133in}{1.712946in}}%
\pgfpathlineto{\pgfqpoint{8.553566in}{1.686513in}}%
\pgfpathlineto{\pgfqpoint{8.569000in}{1.656127in}}%
\pgfpathlineto{\pgfqpoint{8.584433in}{1.621898in}}%
\pgfpathlineto{\pgfqpoint{8.599867in}{1.584101in}}%
\pgfpathlineto{\pgfqpoint{8.630734in}{1.499718in}}%
\pgfpathlineto{\pgfqpoint{8.707903in}{1.274082in}}%
\pgfpathlineto{\pgfqpoint{8.723336in}{1.234270in}}%
\pgfpathlineto{\pgfqpoint{8.738770in}{1.198644in}}%
\pgfpathlineto{\pgfqpoint{8.754203in}{1.168147in}}%
\pgfpathlineto{\pgfqpoint{8.769637in}{1.143621in}}%
\pgfpathlineto{\pgfqpoint{8.785071in}{1.125771in}}%
\pgfpathlineto{\pgfqpoint{8.800504in}{1.115141in}}%
\pgfpathlineto{\pgfqpoint{8.815938in}{1.112097in}}%
\pgfpathlineto{\pgfqpoint{8.831372in}{1.116811in}}%
\pgfpathlineto{\pgfqpoint{8.846805in}{1.129252in}}%
\pgfpathlineto{\pgfqpoint{8.862239in}{1.149188in}}%
\pgfpathlineto{\pgfqpoint{8.877673in}{1.176192in}}%
\pgfpathlineto{\pgfqpoint{8.893106in}{1.209653in}}%
\pgfpathlineto{\pgfqpoint{8.908540in}{1.248795in}}%
\pgfpathlineto{\pgfqpoint{8.923973in}{1.292700in}}%
\pgfpathlineto{\pgfqpoint{8.954841in}{1.390610in}}%
\pgfpathlineto{\pgfqpoint{9.016575in}{1.594885in}}%
\pgfpathlineto{\pgfqpoint{9.032009in}{1.641252in}}%
\pgfpathlineto{\pgfqpoint{9.047443in}{1.683821in}}%
\pgfpathlineto{\pgfqpoint{9.062876in}{1.721851in}}%
\pgfpathlineto{\pgfqpoint{9.078310in}{1.754758in}}%
\pgfpathlineto{\pgfqpoint{9.093743in}{1.782130in}}%
\pgfpathlineto{\pgfqpoint{9.109177in}{1.803727in}}%
\pgfpathlineto{\pgfqpoint{9.124611in}{1.819488in}}%
\pgfpathlineto{\pgfqpoint{9.140044in}{1.829524in}}%
\pgfpathlineto{\pgfqpoint{9.155478in}{1.834104in}}%
\pgfpathlineto{\pgfqpoint{9.170912in}{1.833645in}}%
\pgfpathlineto{\pgfqpoint{9.186345in}{1.828686in}}%
\pgfpathlineto{\pgfqpoint{9.201779in}{1.819871in}}%
\pgfpathlineto{\pgfqpoint{9.217212in}{1.807919in}}%
\pgfpathlineto{\pgfqpoint{9.232646in}{1.793600in}}%
\pgfpathlineto{\pgfqpoint{9.278947in}{1.744330in}}%
\pgfpathlineto{\pgfqpoint{9.309814in}{1.713638in}}%
\pgfpathlineto{\pgfqpoint{9.325248in}{1.700842in}}%
\pgfpathlineto{\pgfqpoint{9.340682in}{1.690391in}}%
\pgfpathlineto{\pgfqpoint{9.356115in}{1.682637in}}%
\pgfpathlineto{\pgfqpoint{9.371549in}{1.677828in}}%
\pgfpathlineto{\pgfqpoint{9.386982in}{1.676104in}}%
\pgfpathlineto{\pgfqpoint{9.402416in}{1.677503in}}%
\pgfpathlineto{\pgfqpoint{9.417850in}{1.681970in}}%
\pgfpathlineto{\pgfqpoint{9.433283in}{1.689370in}}%
\pgfpathlineto{\pgfqpoint{9.448717in}{1.699496in}}%
\pgfpathlineto{\pgfqpoint{9.464151in}{1.712091in}}%
\pgfpathlineto{\pgfqpoint{9.479584in}{1.726858in}}%
\pgfpathlineto{\pgfqpoint{9.510452in}{1.761624in}}%
\pgfpathlineto{\pgfqpoint{9.541319in}{1.801245in}}%
\pgfpathlineto{\pgfqpoint{9.603053in}{1.886516in}}%
\pgfpathlineto{\pgfqpoint{9.680222in}{1.992314in}}%
\pgfpathlineto{\pgfqpoint{9.741956in}{2.073379in}}%
\pgfpathlineto{\pgfqpoint{9.741956in}{2.073379in}}%
\pgfusepath{stroke}%
\end{pgfscope}%
\begin{pgfscope}%
\pgfpathrectangle{\pgfqpoint{5.706832in}{0.521603in}}{\pgfqpoint{4.227273in}{2.800000in}} %
\pgfusepath{clip}%
\pgfsetrectcap%
\pgfsetroundjoin%
\pgfsetlinewidth{0.501875pt}%
\definecolor{currentstroke}{rgb}{0.135294,0.840344,0.878081}%
\pgfsetstrokecolor{currentstroke}%
\pgfsetdash{}{0pt}%
\pgfpathmoveto{\pgfqpoint{5.898981in}{2.430733in}}%
\pgfpathlineto{\pgfqpoint{5.914415in}{2.425218in}}%
\pgfpathlineto{\pgfqpoint{5.929848in}{2.413894in}}%
\pgfpathlineto{\pgfqpoint{5.945282in}{2.396730in}}%
\pgfpathlineto{\pgfqpoint{5.960715in}{2.373806in}}%
\pgfpathlineto{\pgfqpoint{5.976149in}{2.345318in}}%
\pgfpathlineto{\pgfqpoint{5.991583in}{2.311575in}}%
\pgfpathlineto{\pgfqpoint{6.007016in}{2.273001in}}%
\pgfpathlineto{\pgfqpoint{6.022450in}{2.230125in}}%
\pgfpathlineto{\pgfqpoint{6.053317in}{2.134070in}}%
\pgfpathlineto{\pgfqpoint{6.099618in}{1.976026in}}%
\pgfpathlineto{\pgfqpoint{6.130485in}{1.871686in}}%
\pgfpathlineto{\pgfqpoint{6.161353in}{1.776727in}}%
\pgfpathlineto{\pgfqpoint{6.176786in}{1.734942in}}%
\pgfpathlineto{\pgfqpoint{6.192220in}{1.698001in}}%
\pgfpathlineto{\pgfqpoint{6.207654in}{1.666593in}}%
\pgfpathlineto{\pgfqpoint{6.223087in}{1.641308in}}%
\pgfpathlineto{\pgfqpoint{6.238521in}{1.622629in}}%
\pgfpathlineto{\pgfqpoint{6.253955in}{1.610925in}}%
\pgfpathlineto{\pgfqpoint{6.269388in}{1.606437in}}%
\pgfpathlineto{\pgfqpoint{6.284822in}{1.609283in}}%
\pgfpathlineto{\pgfqpoint{6.300255in}{1.619450in}}%
\pgfpathlineto{\pgfqpoint{6.315689in}{1.636803in}}%
\pgfpathlineto{\pgfqpoint{6.331123in}{1.661084in}}%
\pgfpathlineto{\pgfqpoint{6.346556in}{1.691920in}}%
\pgfpathlineto{\pgfqpoint{6.361990in}{1.728838in}}%
\pgfpathlineto{\pgfqpoint{6.377424in}{1.771270in}}%
\pgfpathlineto{\pgfqpoint{6.392857in}{1.818570in}}%
\pgfpathlineto{\pgfqpoint{6.423724in}{1.924881in}}%
\pgfpathlineto{\pgfqpoint{6.454592in}{2.041591in}}%
\pgfpathlineto{\pgfqpoint{6.516326in}{2.280569in}}%
\pgfpathlineto{\pgfqpoint{6.547194in}{2.390923in}}%
\pgfpathlineto{\pgfqpoint{6.578061in}{2.488572in}}%
\pgfpathlineto{\pgfqpoint{6.593494in}{2.531455in}}%
\pgfpathlineto{\pgfqpoint{6.608928in}{2.569896in}}%
\pgfpathlineto{\pgfqpoint{6.624362in}{2.603628in}}%
\pgfpathlineto{\pgfqpoint{6.639795in}{2.632456in}}%
\pgfpathlineto{\pgfqpoint{6.655229in}{2.656253in}}%
\pgfpathlineto{\pgfqpoint{6.670663in}{2.674953in}}%
\pgfpathlineto{\pgfqpoint{6.686096in}{2.688553in}}%
\pgfpathlineto{\pgfqpoint{6.701530in}{2.697096in}}%
\pgfpathlineto{\pgfqpoint{6.716964in}{2.700675in}}%
\pgfpathlineto{\pgfqpoint{6.732397in}{2.699419in}}%
\pgfpathlineto{\pgfqpoint{6.747831in}{2.693489in}}%
\pgfpathlineto{\pgfqpoint{6.763264in}{2.683073in}}%
\pgfpathlineto{\pgfqpoint{6.778698in}{2.668380in}}%
\pgfpathlineto{\pgfqpoint{6.794132in}{2.649636in}}%
\pgfpathlineto{\pgfqpoint{6.809565in}{2.627077in}}%
\pgfpathlineto{\pgfqpoint{6.824999in}{2.600951in}}%
\pgfpathlineto{\pgfqpoint{6.840433in}{2.571514in}}%
\pgfpathlineto{\pgfqpoint{6.855866in}{2.539028in}}%
\pgfpathlineto{\pgfqpoint{6.886734in}{2.465993in}}%
\pgfpathlineto{\pgfqpoint{6.917601in}{2.384095in}}%
\pgfpathlineto{\pgfqpoint{6.948468in}{2.295736in}}%
\pgfpathlineto{\pgfqpoint{7.056504in}{1.975174in}}%
\pgfpathlineto{\pgfqpoint{7.087371in}{1.892918in}}%
\pgfpathlineto{\pgfqpoint{7.118238in}{1.820553in}}%
\pgfpathlineto{\pgfqpoint{7.133672in}{1.788973in}}%
\pgfpathlineto{\pgfqpoint{7.149105in}{1.760878in}}%
\pgfpathlineto{\pgfqpoint{7.164539in}{1.736530in}}%
\pgfpathlineto{\pgfqpoint{7.179973in}{1.716139in}}%
\pgfpathlineto{\pgfqpoint{7.195406in}{1.699860in}}%
\pgfpathlineto{\pgfqpoint{7.210840in}{1.687782in}}%
\pgfpathlineto{\pgfqpoint{7.226274in}{1.679928in}}%
\pgfpathlineto{\pgfqpoint{7.241707in}{1.676245in}}%
\pgfpathlineto{\pgfqpoint{7.257141in}{1.676609in}}%
\pgfpathlineto{\pgfqpoint{7.272574in}{1.680819in}}%
\pgfpathlineto{\pgfqpoint{7.288008in}{1.688599in}}%
\pgfpathlineto{\pgfqpoint{7.303442in}{1.699606in}}%
\pgfpathlineto{\pgfqpoint{7.318875in}{1.713427in}}%
\pgfpathlineto{\pgfqpoint{7.334309in}{1.729595in}}%
\pgfpathlineto{\pgfqpoint{7.365176in}{1.766857in}}%
\pgfpathlineto{\pgfqpoint{7.411477in}{1.826374in}}%
\pgfpathlineto{\pgfqpoint{7.442344in}{1.861560in}}%
\pgfpathlineto{\pgfqpoint{7.457778in}{1.876140in}}%
\pgfpathlineto{\pgfqpoint{7.473212in}{1.888072in}}%
\pgfpathlineto{\pgfqpoint{7.488645in}{1.896955in}}%
\pgfpathlineto{\pgfqpoint{7.504079in}{1.902462in}}%
\pgfpathlineto{\pgfqpoint{7.519513in}{1.904353in}}%
\pgfpathlineto{\pgfqpoint{7.534946in}{1.902477in}}%
\pgfpathlineto{\pgfqpoint{7.550380in}{1.896776in}}%
\pgfpathlineto{\pgfqpoint{7.565814in}{1.887292in}}%
\pgfpathlineto{\pgfqpoint{7.581247in}{1.874162in}}%
\pgfpathlineto{\pgfqpoint{7.596681in}{1.857614in}}%
\pgfpathlineto{\pgfqpoint{7.612114in}{1.837968in}}%
\pgfpathlineto{\pgfqpoint{7.627548in}{1.815619in}}%
\pgfpathlineto{\pgfqpoint{7.658415in}{1.764743in}}%
\pgfpathlineto{\pgfqpoint{7.735584in}{1.628340in}}%
\pgfpathlineto{\pgfqpoint{7.751017in}{1.604300in}}%
\pgfpathlineto{\pgfqpoint{7.766451in}{1.582650in}}%
\pgfpathlineto{\pgfqpoint{7.781884in}{1.563860in}}%
\pgfpathlineto{\pgfqpoint{7.797318in}{1.548329in}}%
\pgfpathlineto{\pgfqpoint{7.812752in}{1.536382in}}%
\pgfpathlineto{\pgfqpoint{7.828185in}{1.528256in}}%
\pgfpathlineto{\pgfqpoint{7.843619in}{1.524100in}}%
\pgfpathlineto{\pgfqpoint{7.859053in}{1.523974in}}%
\pgfpathlineto{\pgfqpoint{7.874486in}{1.527843in}}%
\pgfpathlineto{\pgfqpoint{7.889920in}{1.535587in}}%
\pgfpathlineto{\pgfqpoint{7.905353in}{1.546999in}}%
\pgfpathlineto{\pgfqpoint{7.920787in}{1.561797in}}%
\pgfpathlineto{\pgfqpoint{7.936221in}{1.579632in}}%
\pgfpathlineto{\pgfqpoint{7.951654in}{1.600094in}}%
\pgfpathlineto{\pgfqpoint{7.982522in}{1.647054in}}%
\pgfpathlineto{\pgfqpoint{8.075123in}{1.800133in}}%
\pgfpathlineto{\pgfqpoint{8.105991in}{1.842843in}}%
\pgfpathlineto{\pgfqpoint{8.121424in}{1.861010in}}%
\pgfpathlineto{\pgfqpoint{8.136858in}{1.876748in}}%
\pgfpathlineto{\pgfqpoint{8.152292in}{1.889903in}}%
\pgfpathlineto{\pgfqpoint{8.167725in}{1.900379in}}%
\pgfpathlineto{\pgfqpoint{8.183159in}{1.908137in}}%
\pgfpathlineto{\pgfqpoint{8.198593in}{1.913186in}}%
\pgfpathlineto{\pgfqpoint{8.214026in}{1.915581in}}%
\pgfpathlineto{\pgfqpoint{8.229460in}{1.915415in}}%
\pgfpathlineto{\pgfqpoint{8.244893in}{1.912810in}}%
\pgfpathlineto{\pgfqpoint{8.260327in}{1.907911in}}%
\pgfpathlineto{\pgfqpoint{8.275761in}{1.900875in}}%
\pgfpathlineto{\pgfqpoint{8.291194in}{1.891865in}}%
\pgfpathlineto{\pgfqpoint{8.306628in}{1.881044in}}%
\pgfpathlineto{\pgfqpoint{8.322062in}{1.868564in}}%
\pgfpathlineto{\pgfqpoint{8.352929in}{1.839172in}}%
\pgfpathlineto{\pgfqpoint{8.383796in}{1.804584in}}%
\pgfpathlineto{\pgfqpoint{8.414663in}{1.765371in}}%
\pgfpathlineto{\pgfqpoint{8.445531in}{1.721815in}}%
\pgfpathlineto{\pgfqpoint{8.476398in}{1.674046in}}%
\pgfpathlineto{\pgfqpoint{8.507265in}{1.622229in}}%
\pgfpathlineto{\pgfqpoint{8.538133in}{1.566791in}}%
\pgfpathlineto{\pgfqpoint{8.599867in}{1.449304in}}%
\pgfpathlineto{\pgfqpoint{8.646168in}{1.363216in}}%
\pgfpathlineto{\pgfqpoint{8.677035in}{1.312229in}}%
\pgfpathlineto{\pgfqpoint{8.692469in}{1.289982in}}%
\pgfpathlineto{\pgfqpoint{8.707903in}{1.270517in}}%
\pgfpathlineto{\pgfqpoint{8.723336in}{1.254295in}}%
\pgfpathlineto{\pgfqpoint{8.738770in}{1.241754in}}%
\pgfpathlineto{\pgfqpoint{8.754203in}{1.233302in}}%
\pgfpathlineto{\pgfqpoint{8.769637in}{1.229298in}}%
\pgfpathlineto{\pgfqpoint{8.785071in}{1.230046in}}%
\pgfpathlineto{\pgfqpoint{8.800504in}{1.235776in}}%
\pgfpathlineto{\pgfqpoint{8.815938in}{1.246640in}}%
\pgfpathlineto{\pgfqpoint{8.831372in}{1.262697in}}%
\pgfpathlineto{\pgfqpoint{8.846805in}{1.283912in}}%
\pgfpathlineto{\pgfqpoint{8.862239in}{1.310147in}}%
\pgfpathlineto{\pgfqpoint{8.877673in}{1.341158in}}%
\pgfpathlineto{\pgfqpoint{8.893106in}{1.376601in}}%
\pgfpathlineto{\pgfqpoint{8.908540in}{1.416030in}}%
\pgfpathlineto{\pgfqpoint{8.939407in}{1.504603in}}%
\pgfpathlineto{\pgfqpoint{8.985708in}{1.651349in}}%
\pgfpathlineto{\pgfqpoint{9.016575in}{1.749177in}}%
\pgfpathlineto{\pgfqpoint{9.047443in}{1.839254in}}%
\pgfpathlineto{\pgfqpoint{9.062876in}{1.879416in}}%
\pgfpathlineto{\pgfqpoint{9.078310in}{1.915387in}}%
\pgfpathlineto{\pgfqpoint{9.093743in}{1.946561in}}%
\pgfpathlineto{\pgfqpoint{9.109177in}{1.972429in}}%
\pgfpathlineto{\pgfqpoint{9.124611in}{1.992592in}}%
\pgfpathlineto{\pgfqpoint{9.140044in}{2.006771in}}%
\pgfpathlineto{\pgfqpoint{9.155478in}{2.014817in}}%
\pgfpathlineto{\pgfqpoint{9.170912in}{2.016709in}}%
\pgfpathlineto{\pgfqpoint{9.186345in}{2.012563in}}%
\pgfpathlineto{\pgfqpoint{9.201779in}{2.002620in}}%
\pgfpathlineto{\pgfqpoint{9.217212in}{1.987249in}}%
\pgfpathlineto{\pgfqpoint{9.232646in}{1.966935in}}%
\pgfpathlineto{\pgfqpoint{9.248080in}{1.942267in}}%
\pgfpathlineto{\pgfqpoint{9.263513in}{1.913922in}}%
\pgfpathlineto{\pgfqpoint{9.294381in}{1.849273in}}%
\pgfpathlineto{\pgfqpoint{9.356115in}{1.711662in}}%
\pgfpathlineto{\pgfqpoint{9.371549in}{1.680439in}}%
\pgfpathlineto{\pgfqpoint{9.386982in}{1.652036in}}%
\pgfpathlineto{\pgfqpoint{9.402416in}{1.627112in}}%
\pgfpathlineto{\pgfqpoint{9.417850in}{1.606235in}}%
\pgfpathlineto{\pgfqpoint{9.433283in}{1.589875in}}%
\pgfpathlineto{\pgfqpoint{9.448717in}{1.578389in}}%
\pgfpathlineto{\pgfqpoint{9.464151in}{1.572023in}}%
\pgfpathlineto{\pgfqpoint{9.479584in}{1.570904in}}%
\pgfpathlineto{\pgfqpoint{9.495018in}{1.575042in}}%
\pgfpathlineto{\pgfqpoint{9.510452in}{1.584334in}}%
\pgfpathlineto{\pgfqpoint{9.525885in}{1.598570in}}%
\pgfpathlineto{\pgfqpoint{9.541319in}{1.617442in}}%
\pgfpathlineto{\pgfqpoint{9.556752in}{1.640552in}}%
\pgfpathlineto{\pgfqpoint{9.572186in}{1.667431in}}%
\pgfpathlineto{\pgfqpoint{9.587620in}{1.697545in}}%
\pgfpathlineto{\pgfqpoint{9.618487in}{1.765138in}}%
\pgfpathlineto{\pgfqpoint{9.711089in}{1.983107in}}%
\pgfpathlineto{\pgfqpoint{9.741956in}{2.046381in}}%
\pgfpathlineto{\pgfqpoint{9.741956in}{2.046381in}}%
\pgfusepath{stroke}%
\end{pgfscope}%
\begin{pgfscope}%
\pgfpathrectangle{\pgfqpoint{5.706832in}{0.521603in}}{\pgfqpoint{4.227273in}{2.800000in}} %
\pgfusepath{clip}%
\pgfsetrectcap%
\pgfsetroundjoin%
\pgfsetlinewidth{0.501875pt}%
\definecolor{currentstroke}{rgb}{0.221569,0.905873,0.843667}%
\pgfsetstrokecolor{currentstroke}%
\pgfsetdash{}{0pt}%
\pgfpathmoveto{\pgfqpoint{5.898981in}{2.143898in}}%
\pgfpathlineto{\pgfqpoint{5.914415in}{2.156193in}}%
\pgfpathlineto{\pgfqpoint{5.929848in}{2.166150in}}%
\pgfpathlineto{\pgfqpoint{5.945282in}{2.173561in}}%
\pgfpathlineto{\pgfqpoint{5.960715in}{2.178257in}}%
\pgfpathlineto{\pgfqpoint{5.976149in}{2.180108in}}%
\pgfpathlineto{\pgfqpoint{5.991583in}{2.179028in}}%
\pgfpathlineto{\pgfqpoint{6.007016in}{2.174985in}}%
\pgfpathlineto{\pgfqpoint{6.022450in}{2.167993in}}%
\pgfpathlineto{\pgfqpoint{6.037884in}{2.158126in}}%
\pgfpathlineto{\pgfqpoint{6.053317in}{2.145509in}}%
\pgfpathlineto{\pgfqpoint{6.068751in}{2.130324in}}%
\pgfpathlineto{\pgfqpoint{6.084185in}{2.112807in}}%
\pgfpathlineto{\pgfqpoint{6.115052in}{2.071971in}}%
\pgfpathlineto{\pgfqpoint{6.145919in}{2.025845in}}%
\pgfpathlineto{\pgfqpoint{6.207654in}{1.931907in}}%
\pgfpathlineto{\pgfqpoint{6.223087in}{1.910911in}}%
\pgfpathlineto{\pgfqpoint{6.238521in}{1.891895in}}%
\pgfpathlineto{\pgfqpoint{6.253955in}{1.875328in}}%
\pgfpathlineto{\pgfqpoint{6.269388in}{1.861654in}}%
\pgfpathlineto{\pgfqpoint{6.284822in}{1.851276in}}%
\pgfpathlineto{\pgfqpoint{6.300255in}{1.844558in}}%
\pgfpathlineto{\pgfqpoint{6.315689in}{1.841807in}}%
\pgfpathlineto{\pgfqpoint{6.331123in}{1.843276in}}%
\pgfpathlineto{\pgfqpoint{6.346556in}{1.849152in}}%
\pgfpathlineto{\pgfqpoint{6.361990in}{1.859552in}}%
\pgfpathlineto{\pgfqpoint{6.377424in}{1.874523in}}%
\pgfpathlineto{\pgfqpoint{6.392857in}{1.894037in}}%
\pgfpathlineto{\pgfqpoint{6.408291in}{1.917991in}}%
\pgfpathlineto{\pgfqpoint{6.423724in}{1.946206in}}%
\pgfpathlineto{\pgfqpoint{6.439158in}{1.978433in}}%
\pgfpathlineto{\pgfqpoint{6.454592in}{2.014350in}}%
\pgfpathlineto{\pgfqpoint{6.485459in}{2.095660in}}%
\pgfpathlineto{\pgfqpoint{6.516326in}{2.186387in}}%
\pgfpathlineto{\pgfqpoint{6.608928in}{2.468812in}}%
\pgfpathlineto{\pgfqpoint{6.639795in}{2.550144in}}%
\pgfpathlineto{\pgfqpoint{6.655229in}{2.585870in}}%
\pgfpathlineto{\pgfqpoint{6.670663in}{2.617680in}}%
\pgfpathlineto{\pgfqpoint{6.686096in}{2.645179in}}%
\pgfpathlineto{\pgfqpoint{6.701530in}{2.668035in}}%
\pgfpathlineto{\pgfqpoint{6.716964in}{2.685986in}}%
\pgfpathlineto{\pgfqpoint{6.732397in}{2.698844in}}%
\pgfpathlineto{\pgfqpoint{6.747831in}{2.706493in}}%
\pgfpathlineto{\pgfqpoint{6.763264in}{2.708896in}}%
\pgfpathlineto{\pgfqpoint{6.778698in}{2.706090in}}%
\pgfpathlineto{\pgfqpoint{6.794132in}{2.698186in}}%
\pgfpathlineto{\pgfqpoint{6.809565in}{2.685365in}}%
\pgfpathlineto{\pgfqpoint{6.824999in}{2.667875in}}%
\pgfpathlineto{\pgfqpoint{6.840433in}{2.646024in}}%
\pgfpathlineto{\pgfqpoint{6.855866in}{2.620174in}}%
\pgfpathlineto{\pgfqpoint{6.871300in}{2.590733in}}%
\pgfpathlineto{\pgfqpoint{6.886734in}{2.558148in}}%
\pgfpathlineto{\pgfqpoint{6.917601in}{2.485474in}}%
\pgfpathlineto{\pgfqpoint{6.963902in}{2.365277in}}%
\pgfpathlineto{\pgfqpoint{7.010203in}{2.243527in}}%
\pgfpathlineto{\pgfqpoint{7.041070in}{2.167327in}}%
\pgfpathlineto{\pgfqpoint{7.071937in}{2.098265in}}%
\pgfpathlineto{\pgfqpoint{7.102804in}{2.038152in}}%
\pgfpathlineto{\pgfqpoint{7.118238in}{2.011777in}}%
\pgfpathlineto{\pgfqpoint{7.133672in}{1.987919in}}%
\pgfpathlineto{\pgfqpoint{7.149105in}{1.966561in}}%
\pgfpathlineto{\pgfqpoint{7.164539in}{1.947636in}}%
\pgfpathlineto{\pgfqpoint{7.179973in}{1.931032in}}%
\pgfpathlineto{\pgfqpoint{7.195406in}{1.916596in}}%
\pgfpathlineto{\pgfqpoint{7.210840in}{1.904141in}}%
\pgfpathlineto{\pgfqpoint{7.226274in}{1.893451in}}%
\pgfpathlineto{\pgfqpoint{7.257141in}{1.876400in}}%
\pgfpathlineto{\pgfqpoint{7.288008in}{1.863398in}}%
\pgfpathlineto{\pgfqpoint{7.380610in}{1.828976in}}%
\pgfpathlineto{\pgfqpoint{7.411477in}{1.813970in}}%
\pgfpathlineto{\pgfqpoint{7.442344in}{1.795864in}}%
\pgfpathlineto{\pgfqpoint{7.473212in}{1.774694in}}%
\pgfpathlineto{\pgfqpoint{7.504079in}{1.751030in}}%
\pgfpathlineto{\pgfqpoint{7.596681in}{1.676702in}}%
\pgfpathlineto{\pgfqpoint{7.627548in}{1.655691in}}%
\pgfpathlineto{\pgfqpoint{7.658415in}{1.638950in}}%
\pgfpathlineto{\pgfqpoint{7.673849in}{1.632536in}}%
\pgfpathlineto{\pgfqpoint{7.689283in}{1.627570in}}%
\pgfpathlineto{\pgfqpoint{7.704716in}{1.624125in}}%
\pgfpathlineto{\pgfqpoint{7.720150in}{1.622246in}}%
\pgfpathlineto{\pgfqpoint{7.735584in}{1.621947in}}%
\pgfpathlineto{\pgfqpoint{7.751017in}{1.623211in}}%
\pgfpathlineto{\pgfqpoint{7.766451in}{1.625991in}}%
\pgfpathlineto{\pgfqpoint{7.781884in}{1.630210in}}%
\pgfpathlineto{\pgfqpoint{7.797318in}{1.635763in}}%
\pgfpathlineto{\pgfqpoint{7.828185in}{1.650321in}}%
\pgfpathlineto{\pgfqpoint{7.859053in}{1.668355in}}%
\pgfpathlineto{\pgfqpoint{7.951654in}{1.727008in}}%
\pgfpathlineto{\pgfqpoint{7.982522in}{1.742465in}}%
\pgfpathlineto{\pgfqpoint{7.997955in}{1.748577in}}%
\pgfpathlineto{\pgfqpoint{8.013389in}{1.753414in}}%
\pgfpathlineto{\pgfqpoint{8.028823in}{1.756855in}}%
\pgfpathlineto{\pgfqpoint{8.044256in}{1.758802in}}%
\pgfpathlineto{\pgfqpoint{8.059690in}{1.759184in}}%
\pgfpathlineto{\pgfqpoint{8.075123in}{1.757953in}}%
\pgfpathlineto{\pgfqpoint{8.090557in}{1.755090in}}%
\pgfpathlineto{\pgfqpoint{8.105991in}{1.750598in}}%
\pgfpathlineto{\pgfqpoint{8.121424in}{1.744508in}}%
\pgfpathlineto{\pgfqpoint{8.136858in}{1.736871in}}%
\pgfpathlineto{\pgfqpoint{8.152292in}{1.727763in}}%
\pgfpathlineto{\pgfqpoint{8.183159in}{1.705528in}}%
\pgfpathlineto{\pgfqpoint{8.214026in}{1.678758in}}%
\pgfpathlineto{\pgfqpoint{8.244893in}{1.648613in}}%
\pgfpathlineto{\pgfqpoint{8.306628in}{1.583401in}}%
\pgfpathlineto{\pgfqpoint{8.352929in}{1.535361in}}%
\pgfpathlineto{\pgfqpoint{8.383796in}{1.506033in}}%
\pgfpathlineto{\pgfqpoint{8.414663in}{1.480078in}}%
\pgfpathlineto{\pgfqpoint{8.445531in}{1.458350in}}%
\pgfpathlineto{\pgfqpoint{8.460964in}{1.449291in}}%
\pgfpathlineto{\pgfqpoint{8.476398in}{1.441529in}}%
\pgfpathlineto{\pgfqpoint{8.491832in}{1.435122in}}%
\pgfpathlineto{\pgfqpoint{8.507265in}{1.430115in}}%
\pgfpathlineto{\pgfqpoint{8.522699in}{1.426546in}}%
\pgfpathlineto{\pgfqpoint{8.538133in}{1.424438in}}%
\pgfpathlineto{\pgfqpoint{8.553566in}{1.423808in}}%
\pgfpathlineto{\pgfqpoint{8.569000in}{1.424661in}}%
\pgfpathlineto{\pgfqpoint{8.584433in}{1.426993in}}%
\pgfpathlineto{\pgfqpoint{8.599867in}{1.430791in}}%
\pgfpathlineto{\pgfqpoint{8.615301in}{1.436029in}}%
\pgfpathlineto{\pgfqpoint{8.630734in}{1.442677in}}%
\pgfpathlineto{\pgfqpoint{8.646168in}{1.450690in}}%
\pgfpathlineto{\pgfqpoint{8.677035in}{1.470597in}}%
\pgfpathlineto{\pgfqpoint{8.707903in}{1.495215in}}%
\pgfpathlineto{\pgfqpoint{8.738770in}{1.523852in}}%
\pgfpathlineto{\pgfqpoint{8.769637in}{1.555667in}}%
\pgfpathlineto{\pgfqpoint{8.815938in}{1.607169in}}%
\pgfpathlineto{\pgfqpoint{8.893106in}{1.693591in}}%
\pgfpathlineto{\pgfqpoint{8.923973in}{1.724909in}}%
\pgfpathlineto{\pgfqpoint{8.954841in}{1.752719in}}%
\pgfpathlineto{\pgfqpoint{8.985708in}{1.776142in}}%
\pgfpathlineto{\pgfqpoint{9.016575in}{1.794551in}}%
\pgfpathlineto{\pgfqpoint{9.032009in}{1.801765in}}%
\pgfpathlineto{\pgfqpoint{9.047443in}{1.807637in}}%
\pgfpathlineto{\pgfqpoint{9.062876in}{1.812188in}}%
\pgfpathlineto{\pgfqpoint{9.078310in}{1.815460in}}%
\pgfpathlineto{\pgfqpoint{9.109177in}{1.818467in}}%
\pgfpathlineto{\pgfqpoint{9.140044in}{1.817492in}}%
\pgfpathlineto{\pgfqpoint{9.170912in}{1.813722in}}%
\pgfpathlineto{\pgfqpoint{9.248080in}{1.802159in}}%
\pgfpathlineto{\pgfqpoint{9.278947in}{1.801142in}}%
\pgfpathlineto{\pgfqpoint{9.294381in}{1.802236in}}%
\pgfpathlineto{\pgfqpoint{9.309814in}{1.804640in}}%
\pgfpathlineto{\pgfqpoint{9.325248in}{1.808513in}}%
\pgfpathlineto{\pgfqpoint{9.340682in}{1.813985in}}%
\pgfpathlineto{\pgfqpoint{9.356115in}{1.821164in}}%
\pgfpathlineto{\pgfqpoint{9.371549in}{1.830124in}}%
\pgfpathlineto{\pgfqpoint{9.386982in}{1.840907in}}%
\pgfpathlineto{\pgfqpoint{9.402416in}{1.853521in}}%
\pgfpathlineto{\pgfqpoint{9.417850in}{1.867939in}}%
\pgfpathlineto{\pgfqpoint{9.433283in}{1.884098in}}%
\pgfpathlineto{\pgfqpoint{9.464151in}{1.921206in}}%
\pgfpathlineto{\pgfqpoint{9.495018in}{1.963652in}}%
\pgfpathlineto{\pgfqpoint{9.541319in}{2.033570in}}%
\pgfpathlineto{\pgfqpoint{9.587620in}{2.104597in}}%
\pgfpathlineto{\pgfqpoint{9.618487in}{2.148714in}}%
\pgfpathlineto{\pgfqpoint{9.649354in}{2.187636in}}%
\pgfpathlineto{\pgfqpoint{9.664788in}{2.204529in}}%
\pgfpathlineto{\pgfqpoint{9.680222in}{2.219426in}}%
\pgfpathlineto{\pgfqpoint{9.695655in}{2.232155in}}%
\pgfpathlineto{\pgfqpoint{9.711089in}{2.242579in}}%
\pgfpathlineto{\pgfqpoint{9.726522in}{2.250594in}}%
\pgfpathlineto{\pgfqpoint{9.741956in}{2.256141in}}%
\pgfpathlineto{\pgfqpoint{9.741956in}{2.256141in}}%
\pgfusepath{stroke}%
\end{pgfscope}%
\begin{pgfscope}%
\pgfpathrectangle{\pgfqpoint{5.706832in}{0.521603in}}{\pgfqpoint{4.227273in}{2.800000in}} %
\pgfusepath{clip}%
\pgfsetrectcap%
\pgfsetroundjoin%
\pgfsetlinewidth{0.501875pt}%
\definecolor{currentstroke}{rgb}{0.300000,0.951057,0.809017}%
\pgfsetstrokecolor{currentstroke}%
\pgfsetdash{}{0pt}%
\pgfpathmoveto{\pgfqpoint{5.898981in}{2.204660in}}%
\pgfpathlineto{\pgfqpoint{5.914415in}{2.207822in}}%
\pgfpathlineto{\pgfqpoint{5.929848in}{2.207404in}}%
\pgfpathlineto{\pgfqpoint{5.945282in}{2.203320in}}%
\pgfpathlineto{\pgfqpoint{5.960715in}{2.195537in}}%
\pgfpathlineto{\pgfqpoint{5.976149in}{2.184082in}}%
\pgfpathlineto{\pgfqpoint{5.991583in}{2.169039in}}%
\pgfpathlineto{\pgfqpoint{6.007016in}{2.150551in}}%
\pgfpathlineto{\pgfqpoint{6.022450in}{2.128820in}}%
\pgfpathlineto{\pgfqpoint{6.037884in}{2.104106in}}%
\pgfpathlineto{\pgfqpoint{6.053317in}{2.076721in}}%
\pgfpathlineto{\pgfqpoint{6.084185in}{2.015443in}}%
\pgfpathlineto{\pgfqpoint{6.130485in}{1.913981in}}%
\pgfpathlineto{\pgfqpoint{6.176786in}{1.813400in}}%
\pgfpathlineto{\pgfqpoint{6.207654in}{1.754104in}}%
\pgfpathlineto{\pgfqpoint{6.223087in}{1.728382in}}%
\pgfpathlineto{\pgfqpoint{6.238521in}{1.705935in}}%
\pgfpathlineto{\pgfqpoint{6.253955in}{1.687212in}}%
\pgfpathlineto{\pgfqpoint{6.269388in}{1.672618in}}%
\pgfpathlineto{\pgfqpoint{6.284822in}{1.662507in}}%
\pgfpathlineto{\pgfqpoint{6.300255in}{1.657174in}}%
\pgfpathlineto{\pgfqpoint{6.315689in}{1.656854in}}%
\pgfpathlineto{\pgfqpoint{6.331123in}{1.661711in}}%
\pgfpathlineto{\pgfqpoint{6.346556in}{1.671842in}}%
\pgfpathlineto{\pgfqpoint{6.361990in}{1.687269in}}%
\pgfpathlineto{\pgfqpoint{6.377424in}{1.707941in}}%
\pgfpathlineto{\pgfqpoint{6.392857in}{1.733731in}}%
\pgfpathlineto{\pgfqpoint{6.408291in}{1.764442in}}%
\pgfpathlineto{\pgfqpoint{6.423724in}{1.799802in}}%
\pgfpathlineto{\pgfqpoint{6.439158in}{1.839475in}}%
\pgfpathlineto{\pgfqpoint{6.454592in}{1.883060in}}%
\pgfpathlineto{\pgfqpoint{6.485459in}{1.980080in}}%
\pgfpathlineto{\pgfqpoint{6.516326in}{2.086621in}}%
\pgfpathlineto{\pgfqpoint{6.593494in}{2.362302in}}%
\pgfpathlineto{\pgfqpoint{6.624362in}{2.462902in}}%
\pgfpathlineto{\pgfqpoint{6.639795in}{2.508712in}}%
\pgfpathlineto{\pgfqpoint{6.655229in}{2.550823in}}%
\pgfpathlineto{\pgfqpoint{6.670663in}{2.588764in}}%
\pgfpathlineto{\pgfqpoint{6.686096in}{2.622120in}}%
\pgfpathlineto{\pgfqpoint{6.701530in}{2.650538in}}%
\pgfpathlineto{\pgfqpoint{6.716964in}{2.673733in}}%
\pgfpathlineto{\pgfqpoint{6.732397in}{2.691485in}}%
\pgfpathlineto{\pgfqpoint{6.747831in}{2.703648in}}%
\pgfpathlineto{\pgfqpoint{6.763264in}{2.710147in}}%
\pgfpathlineto{\pgfqpoint{6.778698in}{2.710982in}}%
\pgfpathlineto{\pgfqpoint{6.794132in}{2.706220in}}%
\pgfpathlineto{\pgfqpoint{6.809565in}{2.696002in}}%
\pgfpathlineto{\pgfqpoint{6.824999in}{2.680530in}}%
\pgfpathlineto{\pgfqpoint{6.840433in}{2.660070in}}%
\pgfpathlineto{\pgfqpoint{6.855866in}{2.634945in}}%
\pgfpathlineto{\pgfqpoint{6.871300in}{2.605527in}}%
\pgfpathlineto{\pgfqpoint{6.886734in}{2.572232in}}%
\pgfpathlineto{\pgfqpoint{6.902167in}{2.535512in}}%
\pgfpathlineto{\pgfqpoint{6.933034in}{2.453746in}}%
\pgfpathlineto{\pgfqpoint{6.963902in}{2.364283in}}%
\pgfpathlineto{\pgfqpoint{7.041070in}{2.133586in}}%
\pgfpathlineto{\pgfqpoint{7.071937in}{2.048041in}}%
\pgfpathlineto{\pgfqpoint{7.102804in}{1.970645in}}%
\pgfpathlineto{\pgfqpoint{7.133672in}{1.903251in}}%
\pgfpathlineto{\pgfqpoint{7.149105in}{1.873659in}}%
\pgfpathlineto{\pgfqpoint{7.164539in}{1.846889in}}%
\pgfpathlineto{\pgfqpoint{7.179973in}{1.822943in}}%
\pgfpathlineto{\pgfqpoint{7.195406in}{1.801776in}}%
\pgfpathlineto{\pgfqpoint{7.210840in}{1.783298in}}%
\pgfpathlineto{\pgfqpoint{7.226274in}{1.767381in}}%
\pgfpathlineto{\pgfqpoint{7.241707in}{1.753859in}}%
\pgfpathlineto{\pgfqpoint{7.257141in}{1.742540in}}%
\pgfpathlineto{\pgfqpoint{7.272574in}{1.733204in}}%
\pgfpathlineto{\pgfqpoint{7.288008in}{1.725613in}}%
\pgfpathlineto{\pgfqpoint{7.303442in}{1.719517in}}%
\pgfpathlineto{\pgfqpoint{7.334309in}{1.710780in}}%
\pgfpathlineto{\pgfqpoint{7.365176in}{1.704962in}}%
\pgfpathlineto{\pgfqpoint{7.426911in}{1.694927in}}%
\pgfpathlineto{\pgfqpoint{7.457778in}{1.688053in}}%
\pgfpathlineto{\pgfqpoint{7.488645in}{1.678974in}}%
\pgfpathlineto{\pgfqpoint{7.519513in}{1.667639in}}%
\pgfpathlineto{\pgfqpoint{7.565814in}{1.647556in}}%
\pgfpathlineto{\pgfqpoint{7.612114in}{1.626962in}}%
\pgfpathlineto{\pgfqpoint{7.642982in}{1.615314in}}%
\pgfpathlineto{\pgfqpoint{7.673849in}{1.607156in}}%
\pgfpathlineto{\pgfqpoint{7.689283in}{1.604856in}}%
\pgfpathlineto{\pgfqpoint{7.704716in}{1.603969in}}%
\pgfpathlineto{\pgfqpoint{7.720150in}{1.604641in}}%
\pgfpathlineto{\pgfqpoint{7.735584in}{1.606996in}}%
\pgfpathlineto{\pgfqpoint{7.751017in}{1.611130in}}%
\pgfpathlineto{\pgfqpoint{7.766451in}{1.617110in}}%
\pgfpathlineto{\pgfqpoint{7.781884in}{1.624971in}}%
\pgfpathlineto{\pgfqpoint{7.797318in}{1.634713in}}%
\pgfpathlineto{\pgfqpoint{7.812752in}{1.646301in}}%
\pgfpathlineto{\pgfqpoint{7.828185in}{1.659664in}}%
\pgfpathlineto{\pgfqpoint{7.859053in}{1.691260in}}%
\pgfpathlineto{\pgfqpoint{7.889920in}{1.728249in}}%
\pgfpathlineto{\pgfqpoint{7.936221in}{1.789942in}}%
\pgfpathlineto{\pgfqpoint{7.982522in}{1.852417in}}%
\pgfpathlineto{\pgfqpoint{8.013389in}{1.890413in}}%
\pgfpathlineto{\pgfqpoint{8.028823in}{1.907390in}}%
\pgfpathlineto{\pgfqpoint{8.044256in}{1.922629in}}%
\pgfpathlineto{\pgfqpoint{8.059690in}{1.935861in}}%
\pgfpathlineto{\pgfqpoint{8.075123in}{1.946836in}}%
\pgfpathlineto{\pgfqpoint{8.090557in}{1.955335in}}%
\pgfpathlineto{\pgfqpoint{8.105991in}{1.961167in}}%
\pgfpathlineto{\pgfqpoint{8.121424in}{1.964176in}}%
\pgfpathlineto{\pgfqpoint{8.136858in}{1.964242in}}%
\pgfpathlineto{\pgfqpoint{8.152292in}{1.961286in}}%
\pgfpathlineto{\pgfqpoint{8.167725in}{1.955269in}}%
\pgfpathlineto{\pgfqpoint{8.183159in}{1.946193in}}%
\pgfpathlineto{\pgfqpoint{8.198593in}{1.934104in}}%
\pgfpathlineto{\pgfqpoint{8.214026in}{1.919089in}}%
\pgfpathlineto{\pgfqpoint{8.229460in}{1.901276in}}%
\pgfpathlineto{\pgfqpoint{8.244893in}{1.880832in}}%
\pgfpathlineto{\pgfqpoint{8.260327in}{1.857963in}}%
\pgfpathlineto{\pgfqpoint{8.291194in}{1.805939in}}%
\pgfpathlineto{\pgfqpoint{8.322062in}{1.747474in}}%
\pgfpathlineto{\pgfqpoint{8.430097in}{1.531983in}}%
\pgfpathlineto{\pgfqpoint{8.460964in}{1.479167in}}%
\pgfpathlineto{\pgfqpoint{8.476398in}{1.455874in}}%
\pgfpathlineto{\pgfqpoint{8.491832in}{1.435021in}}%
\pgfpathlineto{\pgfqpoint{8.507265in}{1.416840in}}%
\pgfpathlineto{\pgfqpoint{8.522699in}{1.401532in}}%
\pgfpathlineto{\pgfqpoint{8.538133in}{1.389258in}}%
\pgfpathlineto{\pgfqpoint{8.553566in}{1.380142in}}%
\pgfpathlineto{\pgfqpoint{8.569000in}{1.374267in}}%
\pgfpathlineto{\pgfqpoint{8.584433in}{1.371674in}}%
\pgfpathlineto{\pgfqpoint{8.599867in}{1.372363in}}%
\pgfpathlineto{\pgfqpoint{8.615301in}{1.376294in}}%
\pgfpathlineto{\pgfqpoint{8.630734in}{1.383384in}}%
\pgfpathlineto{\pgfqpoint{8.646168in}{1.393515in}}%
\pgfpathlineto{\pgfqpoint{8.661602in}{1.406528in}}%
\pgfpathlineto{\pgfqpoint{8.677035in}{1.422233in}}%
\pgfpathlineto{\pgfqpoint{8.692469in}{1.440407in}}%
\pgfpathlineto{\pgfqpoint{8.707903in}{1.460799in}}%
\pgfpathlineto{\pgfqpoint{8.738770in}{1.507114in}}%
\pgfpathlineto{\pgfqpoint{8.769637in}{1.558750in}}%
\pgfpathlineto{\pgfqpoint{8.862239in}{1.719315in}}%
\pgfpathlineto{\pgfqpoint{8.893106in}{1.766302in}}%
\pgfpathlineto{\pgfqpoint{8.908540in}{1.787338in}}%
\pgfpathlineto{\pgfqpoint{8.923973in}{1.806450in}}%
\pgfpathlineto{\pgfqpoint{8.939407in}{1.823463in}}%
\pgfpathlineto{\pgfqpoint{8.954841in}{1.838236in}}%
\pgfpathlineto{\pgfqpoint{8.970274in}{1.850662in}}%
\pgfpathlineto{\pgfqpoint{8.985708in}{1.860669in}}%
\pgfpathlineto{\pgfqpoint{9.001142in}{1.868225in}}%
\pgfpathlineto{\pgfqpoint{9.016575in}{1.873332in}}%
\pgfpathlineto{\pgfqpoint{9.032009in}{1.876028in}}%
\pgfpathlineto{\pgfqpoint{9.047443in}{1.876390in}}%
\pgfpathlineto{\pgfqpoint{9.062876in}{1.874526in}}%
\pgfpathlineto{\pgfqpoint{9.078310in}{1.870578in}}%
\pgfpathlineto{\pgfqpoint{9.093743in}{1.864717in}}%
\pgfpathlineto{\pgfqpoint{9.109177in}{1.857142in}}%
\pgfpathlineto{\pgfqpoint{9.140044in}{1.837765in}}%
\pgfpathlineto{\pgfqpoint{9.170912in}{1.814462in}}%
\pgfpathlineto{\pgfqpoint{9.232646in}{1.765078in}}%
\pgfpathlineto{\pgfqpoint{9.263513in}{1.743567in}}%
\pgfpathlineto{\pgfqpoint{9.278947in}{1.734548in}}%
\pgfpathlineto{\pgfqpoint{9.294381in}{1.727005in}}%
\pgfpathlineto{\pgfqpoint{9.309814in}{1.721148in}}%
\pgfpathlineto{\pgfqpoint{9.325248in}{1.717163in}}%
\pgfpathlineto{\pgfqpoint{9.340682in}{1.715207in}}%
\pgfpathlineto{\pgfqpoint{9.356115in}{1.715408in}}%
\pgfpathlineto{\pgfqpoint{9.371549in}{1.717859in}}%
\pgfpathlineto{\pgfqpoint{9.386982in}{1.722622in}}%
\pgfpathlineto{\pgfqpoint{9.402416in}{1.729722in}}%
\pgfpathlineto{\pgfqpoint{9.417850in}{1.739151in}}%
\pgfpathlineto{\pgfqpoint{9.433283in}{1.750864in}}%
\pgfpathlineto{\pgfqpoint{9.448717in}{1.764784in}}%
\pgfpathlineto{\pgfqpoint{9.464151in}{1.780799in}}%
\pgfpathlineto{\pgfqpoint{9.479584in}{1.798768in}}%
\pgfpathlineto{\pgfqpoint{9.510452in}{1.839847in}}%
\pgfpathlineto{\pgfqpoint{9.541319in}{1.886339in}}%
\pgfpathlineto{\pgfqpoint{9.587620in}{1.961750in}}%
\pgfpathlineto{\pgfqpoint{9.633921in}{2.037166in}}%
\pgfpathlineto{\pgfqpoint{9.664788in}{2.083619in}}%
\pgfpathlineto{\pgfqpoint{9.695655in}{2.124527in}}%
\pgfpathlineto{\pgfqpoint{9.711089in}{2.142321in}}%
\pgfpathlineto{\pgfqpoint{9.726522in}{2.158076in}}%
\pgfpathlineto{\pgfqpoint{9.741956in}{2.171626in}}%
\pgfpathlineto{\pgfqpoint{9.741956in}{2.171626in}}%
\pgfusepath{stroke}%
\end{pgfscope}%
\begin{pgfscope}%
\pgfpathrectangle{\pgfqpoint{5.706832in}{0.521603in}}{\pgfqpoint{4.227273in}{2.800000in}} %
\pgfusepath{clip}%
\pgfsetrectcap%
\pgfsetroundjoin%
\pgfsetlinewidth{0.501875pt}%
\definecolor{currentstroke}{rgb}{0.378431,0.981823,0.771298}%
\pgfsetstrokecolor{currentstroke}%
\pgfsetdash{}{0pt}%
\pgfpathmoveto{\pgfqpoint{5.898981in}{2.105682in}}%
\pgfpathlineto{\pgfqpoint{5.914415in}{2.103648in}}%
\pgfpathlineto{\pgfqpoint{5.929848in}{2.099521in}}%
\pgfpathlineto{\pgfqpoint{5.945282in}{2.093158in}}%
\pgfpathlineto{\pgfqpoint{5.960715in}{2.084451in}}%
\pgfpathlineto{\pgfqpoint{5.976149in}{2.073331in}}%
\pgfpathlineto{\pgfqpoint{5.991583in}{2.059773in}}%
\pgfpathlineto{\pgfqpoint{6.007016in}{2.043800in}}%
\pgfpathlineto{\pgfqpoint{6.022450in}{2.025482in}}%
\pgfpathlineto{\pgfqpoint{6.037884in}{2.004943in}}%
\pgfpathlineto{\pgfqpoint{6.068751in}{1.957957in}}%
\pgfpathlineto{\pgfqpoint{6.099618in}{1.904856in}}%
\pgfpathlineto{\pgfqpoint{6.176786in}{1.764931in}}%
\pgfpathlineto{\pgfqpoint{6.207654in}{1.715669in}}%
\pgfpathlineto{\pgfqpoint{6.223087in}{1.694426in}}%
\pgfpathlineto{\pgfqpoint{6.238521in}{1.676074in}}%
\pgfpathlineto{\pgfqpoint{6.253955in}{1.661057in}}%
\pgfpathlineto{\pgfqpoint{6.269388in}{1.649782in}}%
\pgfpathlineto{\pgfqpoint{6.284822in}{1.642617in}}%
\pgfpathlineto{\pgfqpoint{6.300255in}{1.639874in}}%
\pgfpathlineto{\pgfqpoint{6.315689in}{1.641810in}}%
\pgfpathlineto{\pgfqpoint{6.331123in}{1.648617in}}%
\pgfpathlineto{\pgfqpoint{6.346556in}{1.660416in}}%
\pgfpathlineto{\pgfqpoint{6.361990in}{1.677257in}}%
\pgfpathlineto{\pgfqpoint{6.377424in}{1.699112in}}%
\pgfpathlineto{\pgfqpoint{6.392857in}{1.725876in}}%
\pgfpathlineto{\pgfqpoint{6.408291in}{1.757369in}}%
\pgfpathlineto{\pgfqpoint{6.423724in}{1.793333in}}%
\pgfpathlineto{\pgfqpoint{6.439158in}{1.833438in}}%
\pgfpathlineto{\pgfqpoint{6.470025in}{1.924415in}}%
\pgfpathlineto{\pgfqpoint{6.500893in}{2.026404in}}%
\pgfpathlineto{\pgfqpoint{6.608928in}{2.399340in}}%
\pgfpathlineto{\pgfqpoint{6.639795in}{2.489265in}}%
\pgfpathlineto{\pgfqpoint{6.655229in}{2.528539in}}%
\pgfpathlineto{\pgfqpoint{6.670663in}{2.563406in}}%
\pgfpathlineto{\pgfqpoint{6.686096in}{2.593485in}}%
\pgfpathlineto{\pgfqpoint{6.701530in}{2.618464in}}%
\pgfpathlineto{\pgfqpoint{6.716964in}{2.638106in}}%
\pgfpathlineto{\pgfqpoint{6.732397in}{2.652250in}}%
\pgfpathlineto{\pgfqpoint{6.747831in}{2.660815in}}%
\pgfpathlineto{\pgfqpoint{6.763264in}{2.663797in}}%
\pgfpathlineto{\pgfqpoint{6.778698in}{2.661270in}}%
\pgfpathlineto{\pgfqpoint{6.794132in}{2.653380in}}%
\pgfpathlineto{\pgfqpoint{6.809565in}{2.640345in}}%
\pgfpathlineto{\pgfqpoint{6.824999in}{2.622446in}}%
\pgfpathlineto{\pgfqpoint{6.840433in}{2.600021in}}%
\pgfpathlineto{\pgfqpoint{6.855866in}{2.573462in}}%
\pgfpathlineto{\pgfqpoint{6.871300in}{2.543199in}}%
\pgfpathlineto{\pgfqpoint{6.886734in}{2.509699in}}%
\pgfpathlineto{\pgfqpoint{6.917601in}{2.434965in}}%
\pgfpathlineto{\pgfqpoint{6.948468in}{2.353308in}}%
\pgfpathlineto{\pgfqpoint{7.025636in}{2.144247in}}%
\pgfpathlineto{\pgfqpoint{7.056504in}{2.067144in}}%
\pgfpathlineto{\pgfqpoint{7.087371in}{1.997177in}}%
\pgfpathlineto{\pgfqpoint{7.118238in}{1.935483in}}%
\pgfpathlineto{\pgfqpoint{7.149105in}{1.882371in}}%
\pgfpathlineto{\pgfqpoint{7.179973in}{1.837406in}}%
\pgfpathlineto{\pgfqpoint{7.210840in}{1.799555in}}%
\pgfpathlineto{\pgfqpoint{7.241707in}{1.767381in}}%
\pgfpathlineto{\pgfqpoint{7.272574in}{1.739258in}}%
\pgfpathlineto{\pgfqpoint{7.318875in}{1.701230in}}%
\pgfpathlineto{\pgfqpoint{7.488645in}{1.566871in}}%
\pgfpathlineto{\pgfqpoint{7.519513in}{1.546838in}}%
\pgfpathlineto{\pgfqpoint{7.534946in}{1.538501in}}%
\pgfpathlineto{\pgfqpoint{7.550380in}{1.531574in}}%
\pgfpathlineto{\pgfqpoint{7.565814in}{1.526270in}}%
\pgfpathlineto{\pgfqpoint{7.581247in}{1.522789in}}%
\pgfpathlineto{\pgfqpoint{7.596681in}{1.521315in}}%
\pgfpathlineto{\pgfqpoint{7.612114in}{1.522008in}}%
\pgfpathlineto{\pgfqpoint{7.627548in}{1.524999in}}%
\pgfpathlineto{\pgfqpoint{7.642982in}{1.530388in}}%
\pgfpathlineto{\pgfqpoint{7.658415in}{1.538236in}}%
\pgfpathlineto{\pgfqpoint{7.673849in}{1.548566in}}%
\pgfpathlineto{\pgfqpoint{7.689283in}{1.561362in}}%
\pgfpathlineto{\pgfqpoint{7.704716in}{1.576562in}}%
\pgfpathlineto{\pgfqpoint{7.720150in}{1.594064in}}%
\pgfpathlineto{\pgfqpoint{7.735584in}{1.613722in}}%
\pgfpathlineto{\pgfqpoint{7.766451in}{1.658731in}}%
\pgfpathlineto{\pgfqpoint{7.797318in}{1.709685in}}%
\pgfpathlineto{\pgfqpoint{7.905353in}{1.898396in}}%
\pgfpathlineto{\pgfqpoint{7.936221in}{1.944136in}}%
\pgfpathlineto{\pgfqpoint{7.951654in}{1.964110in}}%
\pgfpathlineto{\pgfqpoint{7.967088in}{1.981837in}}%
\pgfpathlineto{\pgfqpoint{7.982522in}{1.997125in}}%
\pgfpathlineto{\pgfqpoint{7.997955in}{2.009819in}}%
\pgfpathlineto{\pgfqpoint{8.013389in}{2.019811in}}%
\pgfpathlineto{\pgfqpoint{8.028823in}{2.027030in}}%
\pgfpathlineto{\pgfqpoint{8.044256in}{2.031454in}}%
\pgfpathlineto{\pgfqpoint{8.059690in}{2.033101in}}%
\pgfpathlineto{\pgfqpoint{8.075123in}{2.032030in}}%
\pgfpathlineto{\pgfqpoint{8.090557in}{2.028339in}}%
\pgfpathlineto{\pgfqpoint{8.105991in}{2.022165in}}%
\pgfpathlineto{\pgfqpoint{8.121424in}{2.013673in}}%
\pgfpathlineto{\pgfqpoint{8.136858in}{2.003060in}}%
\pgfpathlineto{\pgfqpoint{8.152292in}{1.990543in}}%
\pgfpathlineto{\pgfqpoint{8.183159in}{1.960756in}}%
\pgfpathlineto{\pgfqpoint{8.214026in}{1.926324in}}%
\pgfpathlineto{\pgfqpoint{8.275761in}{1.851549in}}%
\pgfpathlineto{\pgfqpoint{8.322062in}{1.797090in}}%
\pgfpathlineto{\pgfqpoint{8.352929in}{1.763881in}}%
\pgfpathlineto{\pgfqpoint{8.383796in}{1.733899in}}%
\pgfpathlineto{\pgfqpoint{8.414663in}{1.707291in}}%
\pgfpathlineto{\pgfqpoint{8.445531in}{1.683827in}}%
\pgfpathlineto{\pgfqpoint{8.476398in}{1.662993in}}%
\pgfpathlineto{\pgfqpoint{8.522699in}{1.635178in}}%
\pgfpathlineto{\pgfqpoint{8.599867in}{1.592739in}}%
\pgfpathlineto{\pgfqpoint{8.677035in}{1.552275in}}%
\pgfpathlineto{\pgfqpoint{8.707903in}{1.538001in}}%
\pgfpathlineto{\pgfqpoint{8.738770in}{1.526221in}}%
\pgfpathlineto{\pgfqpoint{8.769637in}{1.518141in}}%
\pgfpathlineto{\pgfqpoint{8.785071in}{1.515879in}}%
\pgfpathlineto{\pgfqpoint{8.800504in}{1.515005in}}%
\pgfpathlineto{\pgfqpoint{8.815938in}{1.515656in}}%
\pgfpathlineto{\pgfqpoint{8.831372in}{1.517954in}}%
\pgfpathlineto{\pgfqpoint{8.846805in}{1.522003in}}%
\pgfpathlineto{\pgfqpoint{8.862239in}{1.527883in}}%
\pgfpathlineto{\pgfqpoint{8.877673in}{1.535644in}}%
\pgfpathlineto{\pgfqpoint{8.893106in}{1.545310in}}%
\pgfpathlineto{\pgfqpoint{8.908540in}{1.556872in}}%
\pgfpathlineto{\pgfqpoint{8.923973in}{1.570285in}}%
\pgfpathlineto{\pgfqpoint{8.939407in}{1.585473in}}%
\pgfpathlineto{\pgfqpoint{8.970274in}{1.620691in}}%
\pgfpathlineto{\pgfqpoint{9.001142in}{1.661245in}}%
\pgfpathlineto{\pgfqpoint{9.047443in}{1.728167in}}%
\pgfpathlineto{\pgfqpoint{9.093743in}{1.795895in}}%
\pgfpathlineto{\pgfqpoint{9.124611in}{1.837653in}}%
\pgfpathlineto{\pgfqpoint{9.155478in}{1.874163in}}%
\pgfpathlineto{\pgfqpoint{9.170912in}{1.889867in}}%
\pgfpathlineto{\pgfqpoint{9.186345in}{1.903618in}}%
\pgfpathlineto{\pgfqpoint{9.201779in}{1.915269in}}%
\pgfpathlineto{\pgfqpoint{9.217212in}{1.924713in}}%
\pgfpathlineto{\pgfqpoint{9.232646in}{1.931883in}}%
\pgfpathlineto{\pgfqpoint{9.248080in}{1.936757in}}%
\pgfpathlineto{\pgfqpoint{9.263513in}{1.939358in}}%
\pgfpathlineto{\pgfqpoint{9.278947in}{1.939750in}}%
\pgfpathlineto{\pgfqpoint{9.294381in}{1.938044in}}%
\pgfpathlineto{\pgfqpoint{9.309814in}{1.934390in}}%
\pgfpathlineto{\pgfqpoint{9.325248in}{1.928977in}}%
\pgfpathlineto{\pgfqpoint{9.340682in}{1.922030in}}%
\pgfpathlineto{\pgfqpoint{9.371549in}{1.904576in}}%
\pgfpathlineto{\pgfqpoint{9.464151in}{1.845581in}}%
\pgfpathlineto{\pgfqpoint{9.479584in}{1.838045in}}%
\pgfpathlineto{\pgfqpoint{9.495018in}{1.831918in}}%
\pgfpathlineto{\pgfqpoint{9.510452in}{1.827426in}}%
\pgfpathlineto{\pgfqpoint{9.525885in}{1.824759in}}%
\pgfpathlineto{\pgfqpoint{9.541319in}{1.824067in}}%
\pgfpathlineto{\pgfqpoint{9.556752in}{1.825457in}}%
\pgfpathlineto{\pgfqpoint{9.572186in}{1.828990in}}%
\pgfpathlineto{\pgfqpoint{9.587620in}{1.834681in}}%
\pgfpathlineto{\pgfqpoint{9.603053in}{1.842494in}}%
\pgfpathlineto{\pgfqpoint{9.618487in}{1.852350in}}%
\pgfpathlineto{\pgfqpoint{9.633921in}{1.864120in}}%
\pgfpathlineto{\pgfqpoint{9.649354in}{1.877635in}}%
\pgfpathlineto{\pgfqpoint{9.680222in}{1.909022in}}%
\pgfpathlineto{\pgfqpoint{9.711089in}{1.944428in}}%
\pgfpathlineto{\pgfqpoint{9.741956in}{1.981374in}}%
\pgfpathlineto{\pgfqpoint{9.741956in}{1.981374in}}%
\pgfusepath{stroke}%
\end{pgfscope}%
\begin{pgfscope}%
\pgfpathrectangle{\pgfqpoint{5.706832in}{0.521603in}}{\pgfqpoint{4.227273in}{2.800000in}} %
\pgfusepath{clip}%
\pgfsetrectcap%
\pgfsetroundjoin%
\pgfsetlinewidth{0.501875pt}%
\definecolor{currentstroke}{rgb}{0.456863,0.997705,0.730653}%
\pgfsetstrokecolor{currentstroke}%
\pgfsetdash{}{0pt}%
\pgfpathmoveto{\pgfqpoint{5.898981in}{2.102450in}}%
\pgfpathlineto{\pgfqpoint{5.914415in}{2.098765in}}%
\pgfpathlineto{\pgfqpoint{5.929848in}{2.092350in}}%
\pgfpathlineto{\pgfqpoint{5.945282in}{2.083070in}}%
\pgfpathlineto{\pgfqpoint{5.960715in}{2.070844in}}%
\pgfpathlineto{\pgfqpoint{5.976149in}{2.055646in}}%
\pgfpathlineto{\pgfqpoint{5.991583in}{2.037512in}}%
\pgfpathlineto{\pgfqpoint{6.007016in}{2.016542in}}%
\pgfpathlineto{\pgfqpoint{6.022450in}{1.992901in}}%
\pgfpathlineto{\pgfqpoint{6.053317in}{1.938593in}}%
\pgfpathlineto{\pgfqpoint{6.084185in}{1.877178in}}%
\pgfpathlineto{\pgfqpoint{6.161353in}{1.716856in}}%
\pgfpathlineto{\pgfqpoint{6.176786in}{1.687969in}}%
\pgfpathlineto{\pgfqpoint{6.192220in}{1.661458in}}%
\pgfpathlineto{\pgfqpoint{6.207654in}{1.637862in}}%
\pgfpathlineto{\pgfqpoint{6.223087in}{1.617691in}}%
\pgfpathlineto{\pgfqpoint{6.238521in}{1.601407in}}%
\pgfpathlineto{\pgfqpoint{6.253955in}{1.589423in}}%
\pgfpathlineto{\pgfqpoint{6.269388in}{1.582090in}}%
\pgfpathlineto{\pgfqpoint{6.284822in}{1.579690in}}%
\pgfpathlineto{\pgfqpoint{6.300255in}{1.582431in}}%
\pgfpathlineto{\pgfqpoint{6.315689in}{1.590440in}}%
\pgfpathlineto{\pgfqpoint{6.331123in}{1.603764in}}%
\pgfpathlineto{\pgfqpoint{6.346556in}{1.622368in}}%
\pgfpathlineto{\pgfqpoint{6.361990in}{1.646132in}}%
\pgfpathlineto{\pgfqpoint{6.377424in}{1.674857in}}%
\pgfpathlineto{\pgfqpoint{6.392857in}{1.708268in}}%
\pgfpathlineto{\pgfqpoint{6.408291in}{1.746018in}}%
\pgfpathlineto{\pgfqpoint{6.423724in}{1.787696in}}%
\pgfpathlineto{\pgfqpoint{6.454592in}{1.880918in}}%
\pgfpathlineto{\pgfqpoint{6.485459in}{1.983668in}}%
\pgfpathlineto{\pgfqpoint{6.578061in}{2.301402in}}%
\pgfpathlineto{\pgfqpoint{6.608928in}{2.395054in}}%
\pgfpathlineto{\pgfqpoint{6.624362in}{2.437322in}}%
\pgfpathlineto{\pgfqpoint{6.639795in}{2.476026in}}%
\pgfpathlineto{\pgfqpoint{6.655229in}{2.510838in}}%
\pgfpathlineto{\pgfqpoint{6.670663in}{2.541484in}}%
\pgfpathlineto{\pgfqpoint{6.686096in}{2.567745in}}%
\pgfpathlineto{\pgfqpoint{6.701530in}{2.589460in}}%
\pgfpathlineto{\pgfqpoint{6.716964in}{2.606519in}}%
\pgfpathlineto{\pgfqpoint{6.732397in}{2.618868in}}%
\pgfpathlineto{\pgfqpoint{6.747831in}{2.626500in}}%
\pgfpathlineto{\pgfqpoint{6.763264in}{2.629457in}}%
\pgfpathlineto{\pgfqpoint{6.778698in}{2.627823in}}%
\pgfpathlineto{\pgfqpoint{6.794132in}{2.621724in}}%
\pgfpathlineto{\pgfqpoint{6.809565in}{2.611321in}}%
\pgfpathlineto{\pgfqpoint{6.824999in}{2.596808in}}%
\pgfpathlineto{\pgfqpoint{6.840433in}{2.578407in}}%
\pgfpathlineto{\pgfqpoint{6.855866in}{2.556364in}}%
\pgfpathlineto{\pgfqpoint{6.871300in}{2.530947in}}%
\pgfpathlineto{\pgfqpoint{6.886734in}{2.502442in}}%
\pgfpathlineto{\pgfqpoint{6.917601in}{2.437381in}}%
\pgfpathlineto{\pgfqpoint{6.948468in}{2.363705in}}%
\pgfpathlineto{\pgfqpoint{6.979335in}{2.284041in}}%
\pgfpathlineto{\pgfqpoint{7.102804in}{1.957186in}}%
\pgfpathlineto{\pgfqpoint{7.133672in}{1.885056in}}%
\pgfpathlineto{\pgfqpoint{7.164539in}{1.820441in}}%
\pgfpathlineto{\pgfqpoint{7.195406in}{1.764469in}}%
\pgfpathlineto{\pgfqpoint{7.210840in}{1.739938in}}%
\pgfpathlineto{\pgfqpoint{7.226274in}{1.717756in}}%
\pgfpathlineto{\pgfqpoint{7.241707in}{1.697915in}}%
\pgfpathlineto{\pgfqpoint{7.257141in}{1.680375in}}%
\pgfpathlineto{\pgfqpoint{7.272574in}{1.665061in}}%
\pgfpathlineto{\pgfqpoint{7.288008in}{1.651869in}}%
\pgfpathlineto{\pgfqpoint{7.303442in}{1.640666in}}%
\pgfpathlineto{\pgfqpoint{7.318875in}{1.631297in}}%
\pgfpathlineto{\pgfqpoint{7.334309in}{1.623584in}}%
\pgfpathlineto{\pgfqpoint{7.349743in}{1.617335in}}%
\pgfpathlineto{\pgfqpoint{7.380610in}{1.608417in}}%
\pgfpathlineto{\pgfqpoint{7.411477in}{1.602900in}}%
\pgfpathlineto{\pgfqpoint{7.457778in}{1.597722in}}%
\pgfpathlineto{\pgfqpoint{7.534946in}{1.589341in}}%
\pgfpathlineto{\pgfqpoint{7.596681in}{1.582491in}}%
\pgfpathlineto{\pgfqpoint{7.627548in}{1.581513in}}%
\pgfpathlineto{\pgfqpoint{7.642982in}{1.582300in}}%
\pgfpathlineto{\pgfqpoint{7.658415in}{1.584224in}}%
\pgfpathlineto{\pgfqpoint{7.673849in}{1.587494in}}%
\pgfpathlineto{\pgfqpoint{7.689283in}{1.592313in}}%
\pgfpathlineto{\pgfqpoint{7.704716in}{1.598862in}}%
\pgfpathlineto{\pgfqpoint{7.720150in}{1.607300in}}%
\pgfpathlineto{\pgfqpoint{7.735584in}{1.617752in}}%
\pgfpathlineto{\pgfqpoint{7.751017in}{1.630305in}}%
\pgfpathlineto{\pgfqpoint{7.766451in}{1.645004in}}%
\pgfpathlineto{\pgfqpoint{7.781884in}{1.661846in}}%
\pgfpathlineto{\pgfqpoint{7.797318in}{1.680778in}}%
\pgfpathlineto{\pgfqpoint{7.828185in}{1.724443in}}%
\pgfpathlineto{\pgfqpoint{7.859053in}{1.774563in}}%
\pgfpathlineto{\pgfqpoint{7.905353in}{1.856936in}}%
\pgfpathlineto{\pgfqpoint{7.951654in}{1.939308in}}%
\pgfpathlineto{\pgfqpoint{7.982522in}{1.988996in}}%
\pgfpathlineto{\pgfqpoint{7.997955in}{2.011142in}}%
\pgfpathlineto{\pgfqpoint{8.013389in}{2.031043in}}%
\pgfpathlineto{\pgfqpoint{8.028823in}{2.048414in}}%
\pgfpathlineto{\pgfqpoint{8.044256in}{2.063013in}}%
\pgfpathlineto{\pgfqpoint{8.059690in}{2.074650in}}%
\pgfpathlineto{\pgfqpoint{8.075123in}{2.083191in}}%
\pgfpathlineto{\pgfqpoint{8.090557in}{2.088557in}}%
\pgfpathlineto{\pgfqpoint{8.105991in}{2.090731in}}%
\pgfpathlineto{\pgfqpoint{8.121424in}{2.089752in}}%
\pgfpathlineto{\pgfqpoint{8.136858in}{2.085715in}}%
\pgfpathlineto{\pgfqpoint{8.152292in}{2.078770in}}%
\pgfpathlineto{\pgfqpoint{8.167725in}{2.069110in}}%
\pgfpathlineto{\pgfqpoint{8.183159in}{2.056975in}}%
\pgfpathlineto{\pgfqpoint{8.198593in}{2.042634in}}%
\pgfpathlineto{\pgfqpoint{8.229460in}{2.008544in}}%
\pgfpathlineto{\pgfqpoint{8.260327in}{1.969389in}}%
\pgfpathlineto{\pgfqpoint{8.352929in}{1.845645in}}%
\pgfpathlineto{\pgfqpoint{8.383796in}{1.808350in}}%
\pgfpathlineto{\pgfqpoint{8.414663in}{1.774588in}}%
\pgfpathlineto{\pgfqpoint{8.445531in}{1.744366in}}%
\pgfpathlineto{\pgfqpoint{8.476398in}{1.717203in}}%
\pgfpathlineto{\pgfqpoint{8.522699in}{1.680435in}}%
\pgfpathlineto{\pgfqpoint{8.615301in}{1.611956in}}%
\pgfpathlineto{\pgfqpoint{8.661602in}{1.579041in}}%
\pgfpathlineto{\pgfqpoint{8.692469in}{1.559098in}}%
\pgfpathlineto{\pgfqpoint{8.723336in}{1.542370in}}%
\pgfpathlineto{\pgfqpoint{8.738770in}{1.535725in}}%
\pgfpathlineto{\pgfqpoint{8.754203in}{1.530509in}}%
\pgfpathlineto{\pgfqpoint{8.769637in}{1.526929in}}%
\pgfpathlineto{\pgfqpoint{8.785071in}{1.525175in}}%
\pgfpathlineto{\pgfqpoint{8.800504in}{1.525416in}}%
\pgfpathlineto{\pgfqpoint{8.815938in}{1.527793in}}%
\pgfpathlineto{\pgfqpoint{8.831372in}{1.532412in}}%
\pgfpathlineto{\pgfqpoint{8.846805in}{1.539337in}}%
\pgfpathlineto{\pgfqpoint{8.862239in}{1.548590in}}%
\pgfpathlineto{\pgfqpoint{8.877673in}{1.560144in}}%
\pgfpathlineto{\pgfqpoint{8.893106in}{1.573925in}}%
\pgfpathlineto{\pgfqpoint{8.908540in}{1.589809in}}%
\pgfpathlineto{\pgfqpoint{8.923973in}{1.607625in}}%
\pgfpathlineto{\pgfqpoint{8.954841in}{1.648144in}}%
\pgfpathlineto{\pgfqpoint{8.985708in}{1.693298in}}%
\pgfpathlineto{\pgfqpoint{9.047443in}{1.786723in}}%
\pgfpathlineto{\pgfqpoint{9.078310in}{1.829336in}}%
\pgfpathlineto{\pgfqpoint{9.093743in}{1.848476in}}%
\pgfpathlineto{\pgfqpoint{9.109177in}{1.865812in}}%
\pgfpathlineto{\pgfqpoint{9.124611in}{1.881124in}}%
\pgfpathlineto{\pgfqpoint{9.140044in}{1.894241in}}%
\pgfpathlineto{\pgfqpoint{9.155478in}{1.905043in}}%
\pgfpathlineto{\pgfqpoint{9.170912in}{1.913462in}}%
\pgfpathlineto{\pgfqpoint{9.186345in}{1.919486in}}%
\pgfpathlineto{\pgfqpoint{9.201779in}{1.923156in}}%
\pgfpathlineto{\pgfqpoint{9.217212in}{1.924566in}}%
\pgfpathlineto{\pgfqpoint{9.232646in}{1.923860in}}%
\pgfpathlineto{\pgfqpoint{9.248080in}{1.921226in}}%
\pgfpathlineto{\pgfqpoint{9.263513in}{1.916892in}}%
\pgfpathlineto{\pgfqpoint{9.294381in}{1.904200in}}%
\pgfpathlineto{\pgfqpoint{9.325248in}{1.888143in}}%
\pgfpathlineto{\pgfqpoint{9.371549in}{1.863270in}}%
\pgfpathlineto{\pgfqpoint{9.402416in}{1.849632in}}%
\pgfpathlineto{\pgfqpoint{9.417850in}{1.844456in}}%
\pgfpathlineto{\pgfqpoint{9.433283in}{1.840637in}}%
\pgfpathlineto{\pgfqpoint{9.448717in}{1.838323in}}%
\pgfpathlineto{\pgfqpoint{9.464151in}{1.837620in}}%
\pgfpathlineto{\pgfqpoint{9.479584in}{1.838590in}}%
\pgfpathlineto{\pgfqpoint{9.495018in}{1.841253in}}%
\pgfpathlineto{\pgfqpoint{9.510452in}{1.845587in}}%
\pgfpathlineto{\pgfqpoint{9.525885in}{1.851528in}}%
\pgfpathlineto{\pgfqpoint{9.541319in}{1.858977in}}%
\pgfpathlineto{\pgfqpoint{9.572186in}{1.877833in}}%
\pgfpathlineto{\pgfqpoint{9.603053in}{1.900773in}}%
\pgfpathlineto{\pgfqpoint{9.664788in}{1.952038in}}%
\pgfpathlineto{\pgfqpoint{9.695655in}{1.976789in}}%
\pgfpathlineto{\pgfqpoint{9.726522in}{1.998781in}}%
\pgfpathlineto{\pgfqpoint{9.741956in}{2.008329in}}%
\pgfpathlineto{\pgfqpoint{9.741956in}{2.008329in}}%
\pgfusepath{stroke}%
\end{pgfscope}%
\begin{pgfscope}%
\pgfpathrectangle{\pgfqpoint{5.706832in}{0.521603in}}{\pgfqpoint{4.227273in}{2.800000in}} %
\pgfusepath{clip}%
\pgfsetrectcap%
\pgfsetroundjoin%
\pgfsetlinewidth{0.501875pt}%
\definecolor{currentstroke}{rgb}{0.543137,0.997705,0.682749}%
\pgfsetstrokecolor{currentstroke}%
\pgfsetdash{}{0pt}%
\pgfpathmoveto{\pgfqpoint{5.898981in}{2.155187in}}%
\pgfpathlineto{\pgfqpoint{5.914415in}{2.153932in}}%
\pgfpathlineto{\pgfqpoint{5.929848in}{2.149488in}}%
\pgfpathlineto{\pgfqpoint{5.945282in}{2.141853in}}%
\pgfpathlineto{\pgfqpoint{5.960715in}{2.131082in}}%
\pgfpathlineto{\pgfqpoint{5.976149in}{2.117288in}}%
\pgfpathlineto{\pgfqpoint{5.991583in}{2.100641in}}%
\pgfpathlineto{\pgfqpoint{6.007016in}{2.081367in}}%
\pgfpathlineto{\pgfqpoint{6.022450in}{2.059742in}}%
\pgfpathlineto{\pgfqpoint{6.053317in}{2.010771in}}%
\pgfpathlineto{\pgfqpoint{6.099618in}{1.928959in}}%
\pgfpathlineto{\pgfqpoint{6.145919in}{1.847821in}}%
\pgfpathlineto{\pgfqpoint{6.176786in}{1.800216in}}%
\pgfpathlineto{\pgfqpoint{6.192220in}{1.779656in}}%
\pgfpathlineto{\pgfqpoint{6.207654in}{1.761773in}}%
\pgfpathlineto{\pgfqpoint{6.223087in}{1.746910in}}%
\pgfpathlineto{\pgfqpoint{6.238521in}{1.735368in}}%
\pgfpathlineto{\pgfqpoint{6.253955in}{1.727401in}}%
\pgfpathlineto{\pgfqpoint{6.269388in}{1.723215in}}%
\pgfpathlineto{\pgfqpoint{6.284822in}{1.722965in}}%
\pgfpathlineto{\pgfqpoint{6.300255in}{1.726752in}}%
\pgfpathlineto{\pgfqpoint{6.315689in}{1.734624in}}%
\pgfpathlineto{\pgfqpoint{6.331123in}{1.746574in}}%
\pgfpathlineto{\pgfqpoint{6.346556in}{1.762543in}}%
\pgfpathlineto{\pgfqpoint{6.361990in}{1.782420in}}%
\pgfpathlineto{\pgfqpoint{6.377424in}{1.806044in}}%
\pgfpathlineto{\pgfqpoint{6.392857in}{1.833208in}}%
\pgfpathlineto{\pgfqpoint{6.408291in}{1.863659in}}%
\pgfpathlineto{\pgfqpoint{6.423724in}{1.897108in}}%
\pgfpathlineto{\pgfqpoint{6.454592in}{1.971651in}}%
\pgfpathlineto{\pgfqpoint{6.485459in}{2.053860in}}%
\pgfpathlineto{\pgfqpoint{6.593494in}{2.352368in}}%
\pgfpathlineto{\pgfqpoint{6.624362in}{2.426296in}}%
\pgfpathlineto{\pgfqpoint{6.639795in}{2.459348in}}%
\pgfpathlineto{\pgfqpoint{6.655229in}{2.489341in}}%
\pgfpathlineto{\pgfqpoint{6.670663in}{2.515974in}}%
\pgfpathlineto{\pgfqpoint{6.686096in}{2.538988in}}%
\pgfpathlineto{\pgfqpoint{6.701530in}{2.558160in}}%
\pgfpathlineto{\pgfqpoint{6.716964in}{2.573312in}}%
\pgfpathlineto{\pgfqpoint{6.732397in}{2.584308in}}%
\pgfpathlineto{\pgfqpoint{6.747831in}{2.591059in}}%
\pgfpathlineto{\pgfqpoint{6.763264in}{2.593521in}}%
\pgfpathlineto{\pgfqpoint{6.778698in}{2.591698in}}%
\pgfpathlineto{\pgfqpoint{6.794132in}{2.585639in}}%
\pgfpathlineto{\pgfqpoint{6.809565in}{2.575436in}}%
\pgfpathlineto{\pgfqpoint{6.824999in}{2.561228in}}%
\pgfpathlineto{\pgfqpoint{6.840433in}{2.543193in}}%
\pgfpathlineto{\pgfqpoint{6.855866in}{2.521549in}}%
\pgfpathlineto{\pgfqpoint{6.871300in}{2.496548in}}%
\pgfpathlineto{\pgfqpoint{6.886734in}{2.468475in}}%
\pgfpathlineto{\pgfqpoint{6.917601in}{2.404387in}}%
\pgfpathlineto{\pgfqpoint{6.948468in}{2.332044in}}%
\pgfpathlineto{\pgfqpoint{6.994769in}{2.214603in}}%
\pgfpathlineto{\pgfqpoint{7.056504in}{2.057146in}}%
\pgfpathlineto{\pgfqpoint{7.087371in}{1.984230in}}%
\pgfpathlineto{\pgfqpoint{7.118238in}{1.918188in}}%
\pgfpathlineto{\pgfqpoint{7.149105in}{1.860646in}}%
\pgfpathlineto{\pgfqpoint{7.164539in}{1.835411in}}%
\pgfpathlineto{\pgfqpoint{7.179973in}{1.812641in}}%
\pgfpathlineto{\pgfqpoint{7.195406in}{1.792370in}}%
\pgfpathlineto{\pgfqpoint{7.210840in}{1.774593in}}%
\pgfpathlineto{\pgfqpoint{7.226274in}{1.759264in}}%
\pgfpathlineto{\pgfqpoint{7.241707in}{1.746305in}}%
\pgfpathlineto{\pgfqpoint{7.257141in}{1.735601in}}%
\pgfpathlineto{\pgfqpoint{7.272574in}{1.727007in}}%
\pgfpathlineto{\pgfqpoint{7.288008in}{1.720350in}}%
\pgfpathlineto{\pgfqpoint{7.303442in}{1.715430in}}%
\pgfpathlineto{\pgfqpoint{7.318875in}{1.712030in}}%
\pgfpathlineto{\pgfqpoint{7.349743in}{1.708840in}}%
\pgfpathlineto{\pgfqpoint{7.380610in}{1.708806in}}%
\pgfpathlineto{\pgfqpoint{7.442344in}{1.710450in}}%
\pgfpathlineto{\pgfqpoint{7.473212in}{1.708829in}}%
\pgfpathlineto{\pgfqpoint{7.504079in}{1.704061in}}%
\pgfpathlineto{\pgfqpoint{7.534946in}{1.695694in}}%
\pgfpathlineto{\pgfqpoint{7.565814in}{1.683888in}}%
\pgfpathlineto{\pgfqpoint{7.596681in}{1.669413in}}%
\pgfpathlineto{\pgfqpoint{7.673849in}{1.631083in}}%
\pgfpathlineto{\pgfqpoint{7.704716in}{1.619789in}}%
\pgfpathlineto{\pgfqpoint{7.720150in}{1.615980in}}%
\pgfpathlineto{\pgfqpoint{7.735584in}{1.613686in}}%
\pgfpathlineto{\pgfqpoint{7.751017in}{1.613095in}}%
\pgfpathlineto{\pgfqpoint{7.766451in}{1.614364in}}%
\pgfpathlineto{\pgfqpoint{7.781884in}{1.617613in}}%
\pgfpathlineto{\pgfqpoint{7.797318in}{1.622922in}}%
\pgfpathlineto{\pgfqpoint{7.812752in}{1.630324in}}%
\pgfpathlineto{\pgfqpoint{7.828185in}{1.639811in}}%
\pgfpathlineto{\pgfqpoint{7.843619in}{1.651325in}}%
\pgfpathlineto{\pgfqpoint{7.859053in}{1.664762in}}%
\pgfpathlineto{\pgfqpoint{7.874486in}{1.679975in}}%
\pgfpathlineto{\pgfqpoint{7.905353in}{1.714925in}}%
\pgfpathlineto{\pgfqpoint{7.936221in}{1.754222in}}%
\pgfpathlineto{\pgfqpoint{8.013389in}{1.855281in}}%
\pgfpathlineto{\pgfqpoint{8.044256in}{1.890021in}}%
\pgfpathlineto{\pgfqpoint{8.059690in}{1.904988in}}%
\pgfpathlineto{\pgfqpoint{8.075123in}{1.918033in}}%
\pgfpathlineto{\pgfqpoint{8.090557in}{1.928959in}}%
\pgfpathlineto{\pgfqpoint{8.105991in}{1.937608in}}%
\pgfpathlineto{\pgfqpoint{8.121424in}{1.943864in}}%
\pgfpathlineto{\pgfqpoint{8.136858in}{1.947652in}}%
\pgfpathlineto{\pgfqpoint{8.152292in}{1.948937in}}%
\pgfpathlineto{\pgfqpoint{8.167725in}{1.947729in}}%
\pgfpathlineto{\pgfqpoint{8.183159in}{1.944073in}}%
\pgfpathlineto{\pgfqpoint{8.198593in}{1.938049in}}%
\pgfpathlineto{\pgfqpoint{8.214026in}{1.929773in}}%
\pgfpathlineto{\pgfqpoint{8.229460in}{1.919385in}}%
\pgfpathlineto{\pgfqpoint{8.244893in}{1.907052in}}%
\pgfpathlineto{\pgfqpoint{8.260327in}{1.892956in}}%
\pgfpathlineto{\pgfqpoint{8.291194in}{1.860277in}}%
\pgfpathlineto{\pgfqpoint{8.322062in}{1.823016in}}%
\pgfpathlineto{\pgfqpoint{8.368363in}{1.762178in}}%
\pgfpathlineto{\pgfqpoint{8.445531in}{1.659919in}}%
\pgfpathlineto{\pgfqpoint{8.476398in}{1.622230in}}%
\pgfpathlineto{\pgfqpoint{8.507265in}{1.587799in}}%
\pgfpathlineto{\pgfqpoint{8.538133in}{1.557402in}}%
\pgfpathlineto{\pgfqpoint{8.569000in}{1.531751in}}%
\pgfpathlineto{\pgfqpoint{8.584433in}{1.520918in}}%
\pgfpathlineto{\pgfqpoint{8.599867in}{1.511523in}}%
\pgfpathlineto{\pgfqpoint{8.615301in}{1.503645in}}%
\pgfpathlineto{\pgfqpoint{8.630734in}{1.497363in}}%
\pgfpathlineto{\pgfqpoint{8.646168in}{1.492750in}}%
\pgfpathlineto{\pgfqpoint{8.661602in}{1.489873in}}%
\pgfpathlineto{\pgfqpoint{8.677035in}{1.488792in}}%
\pgfpathlineto{\pgfqpoint{8.692469in}{1.489558in}}%
\pgfpathlineto{\pgfqpoint{8.707903in}{1.492208in}}%
\pgfpathlineto{\pgfqpoint{8.723336in}{1.496765in}}%
\pgfpathlineto{\pgfqpoint{8.738770in}{1.503236in}}%
\pgfpathlineto{\pgfqpoint{8.754203in}{1.511608in}}%
\pgfpathlineto{\pgfqpoint{8.769637in}{1.521845in}}%
\pgfpathlineto{\pgfqpoint{8.785071in}{1.533890in}}%
\pgfpathlineto{\pgfqpoint{8.800504in}{1.547661in}}%
\pgfpathlineto{\pgfqpoint{8.831372in}{1.579926in}}%
\pgfpathlineto{\pgfqpoint{8.862239in}{1.617480in}}%
\pgfpathlineto{\pgfqpoint{8.893106in}{1.658784in}}%
\pgfpathlineto{\pgfqpoint{8.970274in}{1.765778in}}%
\pgfpathlineto{\pgfqpoint{9.001142in}{1.804513in}}%
\pgfpathlineto{\pgfqpoint{9.032009in}{1.837860in}}%
\pgfpathlineto{\pgfqpoint{9.047443in}{1.851968in}}%
\pgfpathlineto{\pgfqpoint{9.062876in}{1.864127in}}%
\pgfpathlineto{\pgfqpoint{9.078310in}{1.874195in}}%
\pgfpathlineto{\pgfqpoint{9.093743in}{1.882067in}}%
\pgfpathlineto{\pgfqpoint{9.109177in}{1.887676in}}%
\pgfpathlineto{\pgfqpoint{9.124611in}{1.890995in}}%
\pgfpathlineto{\pgfqpoint{9.140044in}{1.892040in}}%
\pgfpathlineto{\pgfqpoint{9.155478in}{1.890865in}}%
\pgfpathlineto{\pgfqpoint{9.170912in}{1.887570in}}%
\pgfpathlineto{\pgfqpoint{9.186345in}{1.882295in}}%
\pgfpathlineto{\pgfqpoint{9.201779in}{1.875215in}}%
\pgfpathlineto{\pgfqpoint{9.217212in}{1.866546in}}%
\pgfpathlineto{\pgfqpoint{9.248080in}{1.845445in}}%
\pgfpathlineto{\pgfqpoint{9.294381in}{1.808792in}}%
\pgfpathlineto{\pgfqpoint{9.325248in}{1.784754in}}%
\pgfpathlineto{\pgfqpoint{9.356115in}{1.764033in}}%
\pgfpathlineto{\pgfqpoint{9.371549in}{1.755656in}}%
\pgfpathlineto{\pgfqpoint{9.386982in}{1.748948in}}%
\pgfpathlineto{\pgfqpoint{9.402416in}{1.744127in}}%
\pgfpathlineto{\pgfqpoint{9.417850in}{1.741375in}}%
\pgfpathlineto{\pgfqpoint{9.433283in}{1.740829in}}%
\pgfpathlineto{\pgfqpoint{9.448717in}{1.742583in}}%
\pgfpathlineto{\pgfqpoint{9.464151in}{1.746679in}}%
\pgfpathlineto{\pgfqpoint{9.479584in}{1.753114in}}%
\pgfpathlineto{\pgfqpoint{9.495018in}{1.761833in}}%
\pgfpathlineto{\pgfqpoint{9.510452in}{1.772732in}}%
\pgfpathlineto{\pgfqpoint{9.525885in}{1.785662in}}%
\pgfpathlineto{\pgfqpoint{9.541319in}{1.800430in}}%
\pgfpathlineto{\pgfqpoint{9.572186in}{1.834519in}}%
\pgfpathlineto{\pgfqpoint{9.603053in}{1.872776in}}%
\pgfpathlineto{\pgfqpoint{9.664788in}{1.951464in}}%
\pgfpathlineto{\pgfqpoint{9.695655in}{1.986723in}}%
\pgfpathlineto{\pgfqpoint{9.711089in}{2.002329in}}%
\pgfpathlineto{\pgfqpoint{9.726522in}{2.016281in}}%
\pgfpathlineto{\pgfqpoint{9.741956in}{2.028402in}}%
\pgfpathlineto{\pgfqpoint{9.741956in}{2.028402in}}%
\pgfusepath{stroke}%
\end{pgfscope}%
\begin{pgfscope}%
\pgfpathrectangle{\pgfqpoint{5.706832in}{0.521603in}}{\pgfqpoint{4.227273in}{2.800000in}} %
\pgfusepath{clip}%
\pgfsetrectcap%
\pgfsetroundjoin%
\pgfsetlinewidth{0.501875pt}%
\definecolor{currentstroke}{rgb}{0.621569,0.981823,0.636474}%
\pgfsetstrokecolor{currentstroke}%
\pgfsetdash{}{0pt}%
\pgfpathmoveto{\pgfqpoint{5.898981in}{2.196274in}}%
\pgfpathlineto{\pgfqpoint{5.914415in}{2.201244in}}%
\pgfpathlineto{\pgfqpoint{5.929848in}{2.202948in}}%
\pgfpathlineto{\pgfqpoint{5.945282in}{2.201294in}}%
\pgfpathlineto{\pgfqpoint{5.960715in}{2.196256in}}%
\pgfpathlineto{\pgfqpoint{5.976149in}{2.187877in}}%
\pgfpathlineto{\pgfqpoint{5.991583in}{2.176267in}}%
\pgfpathlineto{\pgfqpoint{6.007016in}{2.161597in}}%
\pgfpathlineto{\pgfqpoint{6.022450in}{2.144106in}}%
\pgfpathlineto{\pgfqpoint{6.037884in}{2.124085in}}%
\pgfpathlineto{\pgfqpoint{6.068751in}{2.077876in}}%
\pgfpathlineto{\pgfqpoint{6.099618in}{2.026212in}}%
\pgfpathlineto{\pgfqpoint{6.145919in}{1.946644in}}%
\pgfpathlineto{\pgfqpoint{6.176786in}{1.897870in}}%
\pgfpathlineto{\pgfqpoint{6.192220in}{1.876166in}}%
\pgfpathlineto{\pgfqpoint{6.207654in}{1.856821in}}%
\pgfpathlineto{\pgfqpoint{6.223087in}{1.840219in}}%
\pgfpathlineto{\pgfqpoint{6.238521in}{1.826707in}}%
\pgfpathlineto{\pgfqpoint{6.253955in}{1.816582in}}%
\pgfpathlineto{\pgfqpoint{6.269388in}{1.810094in}}%
\pgfpathlineto{\pgfqpoint{6.284822in}{1.807441in}}%
\pgfpathlineto{\pgfqpoint{6.300255in}{1.808764in}}%
\pgfpathlineto{\pgfqpoint{6.315689in}{1.814149in}}%
\pgfpathlineto{\pgfqpoint{6.331123in}{1.823623in}}%
\pgfpathlineto{\pgfqpoint{6.346556in}{1.837157in}}%
\pgfpathlineto{\pgfqpoint{6.361990in}{1.854667in}}%
\pgfpathlineto{\pgfqpoint{6.377424in}{1.876012in}}%
\pgfpathlineto{\pgfqpoint{6.392857in}{1.900999in}}%
\pgfpathlineto{\pgfqpoint{6.408291in}{1.929386in}}%
\pgfpathlineto{\pgfqpoint{6.423724in}{1.960885in}}%
\pgfpathlineto{\pgfqpoint{6.454592in}{2.031850in}}%
\pgfpathlineto{\pgfqpoint{6.485459in}{2.110816in}}%
\pgfpathlineto{\pgfqpoint{6.593494in}{2.397419in}}%
\pgfpathlineto{\pgfqpoint{6.624362in}{2.466710in}}%
\pgfpathlineto{\pgfqpoint{6.639795in}{2.497106in}}%
\pgfpathlineto{\pgfqpoint{6.655229in}{2.524200in}}%
\pgfpathlineto{\pgfqpoint{6.670663in}{2.547691in}}%
\pgfpathlineto{\pgfqpoint{6.686096in}{2.567326in}}%
\pgfpathlineto{\pgfqpoint{6.701530in}{2.582898in}}%
\pgfpathlineto{\pgfqpoint{6.716964in}{2.594253in}}%
\pgfpathlineto{\pgfqpoint{6.732397in}{2.601290in}}%
\pgfpathlineto{\pgfqpoint{6.747831in}{2.603961in}}%
\pgfpathlineto{\pgfqpoint{6.763264in}{2.602269in}}%
\pgfpathlineto{\pgfqpoint{6.778698in}{2.596273in}}%
\pgfpathlineto{\pgfqpoint{6.794132in}{2.586081in}}%
\pgfpathlineto{\pgfqpoint{6.809565in}{2.571847in}}%
\pgfpathlineto{\pgfqpoint{6.824999in}{2.553773in}}%
\pgfpathlineto{\pgfqpoint{6.840433in}{2.532099in}}%
\pgfpathlineto{\pgfqpoint{6.855866in}{2.507103in}}%
\pgfpathlineto{\pgfqpoint{6.871300in}{2.479092in}}%
\pgfpathlineto{\pgfqpoint{6.902167in}{2.415390in}}%
\pgfpathlineto{\pgfqpoint{6.933034in}{2.343891in}}%
\pgfpathlineto{\pgfqpoint{6.979335in}{2.228714in}}%
\pgfpathlineto{\pgfqpoint{7.025636in}{2.113025in}}%
\pgfpathlineto{\pgfqpoint{7.056504in}{2.040164in}}%
\pgfpathlineto{\pgfqpoint{7.087371in}{1.973394in}}%
\pgfpathlineto{\pgfqpoint{7.118238in}{1.914505in}}%
\pgfpathlineto{\pgfqpoint{7.133672in}{1.888437in}}%
\pgfpathlineto{\pgfqpoint{7.149105in}{1.864779in}}%
\pgfpathlineto{\pgfqpoint{7.164539in}{1.843605in}}%
\pgfpathlineto{\pgfqpoint{7.179973in}{1.824956in}}%
\pgfpathlineto{\pgfqpoint{7.195406in}{1.808834in}}%
\pgfpathlineto{\pgfqpoint{7.210840in}{1.795208in}}%
\pgfpathlineto{\pgfqpoint{7.226274in}{1.784013in}}%
\pgfpathlineto{\pgfqpoint{7.241707in}{1.775148in}}%
\pgfpathlineto{\pgfqpoint{7.257141in}{1.768478in}}%
\pgfpathlineto{\pgfqpoint{7.272574in}{1.763838in}}%
\pgfpathlineto{\pgfqpoint{7.288008in}{1.761033in}}%
\pgfpathlineto{\pgfqpoint{7.303442in}{1.759842in}}%
\pgfpathlineto{\pgfqpoint{7.334309in}{1.761293in}}%
\pgfpathlineto{\pgfqpoint{7.365176in}{1.766009in}}%
\pgfpathlineto{\pgfqpoint{7.411477in}{1.774050in}}%
\pgfpathlineto{\pgfqpoint{7.442344in}{1.776693in}}%
\pgfpathlineto{\pgfqpoint{7.457778in}{1.776430in}}%
\pgfpathlineto{\pgfqpoint{7.473212in}{1.774830in}}%
\pgfpathlineto{\pgfqpoint{7.488645in}{1.771732in}}%
\pgfpathlineto{\pgfqpoint{7.504079in}{1.767015in}}%
\pgfpathlineto{\pgfqpoint{7.519513in}{1.760609in}}%
\pgfpathlineto{\pgfqpoint{7.534946in}{1.752493in}}%
\pgfpathlineto{\pgfqpoint{7.550380in}{1.742700in}}%
\pgfpathlineto{\pgfqpoint{7.565814in}{1.731318in}}%
\pgfpathlineto{\pgfqpoint{7.596681in}{1.704408in}}%
\pgfpathlineto{\pgfqpoint{7.627548in}{1.673504in}}%
\pgfpathlineto{\pgfqpoint{7.689283in}{1.609916in}}%
\pgfpathlineto{\pgfqpoint{7.720150in}{1.583227in}}%
\pgfpathlineto{\pgfqpoint{7.735584in}{1.572479in}}%
\pgfpathlineto{\pgfqpoint{7.751017in}{1.563914in}}%
\pgfpathlineto{\pgfqpoint{7.766451in}{1.557822in}}%
\pgfpathlineto{\pgfqpoint{7.781884in}{1.554446in}}%
\pgfpathlineto{\pgfqpoint{7.797318in}{1.553971in}}%
\pgfpathlineto{\pgfqpoint{7.812752in}{1.556519in}}%
\pgfpathlineto{\pgfqpoint{7.828185in}{1.562144in}}%
\pgfpathlineto{\pgfqpoint{7.843619in}{1.570832in}}%
\pgfpathlineto{\pgfqpoint{7.859053in}{1.582493in}}%
\pgfpathlineto{\pgfqpoint{7.874486in}{1.596970in}}%
\pgfpathlineto{\pgfqpoint{7.889920in}{1.614036in}}%
\pgfpathlineto{\pgfqpoint{7.905353in}{1.633404in}}%
\pgfpathlineto{\pgfqpoint{7.936221in}{1.677612in}}%
\pgfpathlineto{\pgfqpoint{7.982522in}{1.751200in}}%
\pgfpathlineto{\pgfqpoint{8.013389in}{1.799637in}}%
\pgfpathlineto{\pgfqpoint{8.044256in}{1.843239in}}%
\pgfpathlineto{\pgfqpoint{8.059690in}{1.862185in}}%
\pgfpathlineto{\pgfqpoint{8.075123in}{1.878757in}}%
\pgfpathlineto{\pgfqpoint{8.090557in}{1.892667in}}%
\pgfpathlineto{\pgfqpoint{8.105991in}{1.903685in}}%
\pgfpathlineto{\pgfqpoint{8.121424in}{1.911646in}}%
\pgfpathlineto{\pgfqpoint{8.136858in}{1.916449in}}%
\pgfpathlineto{\pgfqpoint{8.152292in}{1.918056in}}%
\pgfpathlineto{\pgfqpoint{8.167725in}{1.916494in}}%
\pgfpathlineto{\pgfqpoint{8.183159in}{1.911848in}}%
\pgfpathlineto{\pgfqpoint{8.198593in}{1.904258in}}%
\pgfpathlineto{\pgfqpoint{8.214026in}{1.893911in}}%
\pgfpathlineto{\pgfqpoint{8.229460in}{1.881037in}}%
\pgfpathlineto{\pgfqpoint{8.244893in}{1.865897in}}%
\pgfpathlineto{\pgfqpoint{8.260327in}{1.848778in}}%
\pgfpathlineto{\pgfqpoint{8.291194in}{1.809825in}}%
\pgfpathlineto{\pgfqpoint{8.322062in}{1.766663in}}%
\pgfpathlineto{\pgfqpoint{8.399230in}{1.655268in}}%
\pgfpathlineto{\pgfqpoint{8.430097in}{1.614057in}}%
\pgfpathlineto{\pgfqpoint{8.460964in}{1.576614in}}%
\pgfpathlineto{\pgfqpoint{8.491832in}{1.543655in}}%
\pgfpathlineto{\pgfqpoint{8.522699in}{1.515627in}}%
\pgfpathlineto{\pgfqpoint{8.553566in}{1.492814in}}%
\pgfpathlineto{\pgfqpoint{8.569000in}{1.483434in}}%
\pgfpathlineto{\pgfqpoint{8.584433in}{1.475441in}}%
\pgfpathlineto{\pgfqpoint{8.599867in}{1.468863in}}%
\pgfpathlineto{\pgfqpoint{8.615301in}{1.463733in}}%
\pgfpathlineto{\pgfqpoint{8.630734in}{1.460082in}}%
\pgfpathlineto{\pgfqpoint{8.646168in}{1.457944in}}%
\pgfpathlineto{\pgfqpoint{8.661602in}{1.457352in}}%
\pgfpathlineto{\pgfqpoint{8.677035in}{1.458334in}}%
\pgfpathlineto{\pgfqpoint{8.692469in}{1.460914in}}%
\pgfpathlineto{\pgfqpoint{8.707903in}{1.465105in}}%
\pgfpathlineto{\pgfqpoint{8.723336in}{1.470910in}}%
\pgfpathlineto{\pgfqpoint{8.738770in}{1.478316in}}%
\pgfpathlineto{\pgfqpoint{8.754203in}{1.487293in}}%
\pgfpathlineto{\pgfqpoint{8.769637in}{1.497791in}}%
\pgfpathlineto{\pgfqpoint{8.800504in}{1.523039in}}%
\pgfpathlineto{\pgfqpoint{8.831372in}{1.553205in}}%
\pgfpathlineto{\pgfqpoint{8.862239in}{1.587069in}}%
\pgfpathlineto{\pgfqpoint{8.970274in}{1.710843in}}%
\pgfpathlineto{\pgfqpoint{9.001142in}{1.740315in}}%
\pgfpathlineto{\pgfqpoint{9.016575in}{1.753066in}}%
\pgfpathlineto{\pgfqpoint{9.032009in}{1.764299in}}%
\pgfpathlineto{\pgfqpoint{9.047443in}{1.773904in}}%
\pgfpathlineto{\pgfqpoint{9.062876in}{1.781803in}}%
\pgfpathlineto{\pgfqpoint{9.078310in}{1.787955in}}%
\pgfpathlineto{\pgfqpoint{9.093743in}{1.792350in}}%
\pgfpathlineto{\pgfqpoint{9.109177in}{1.795015in}}%
\pgfpathlineto{\pgfqpoint{9.124611in}{1.796013in}}%
\pgfpathlineto{\pgfqpoint{9.140044in}{1.795438in}}%
\pgfpathlineto{\pgfqpoint{9.155478in}{1.793414in}}%
\pgfpathlineto{\pgfqpoint{9.186345in}{1.785657in}}%
\pgfpathlineto{\pgfqpoint{9.217212in}{1.774240in}}%
\pgfpathlineto{\pgfqpoint{9.309814in}{1.735975in}}%
\pgfpathlineto{\pgfqpoint{9.340682in}{1.727827in}}%
\pgfpathlineto{\pgfqpoint{9.356115in}{1.725471in}}%
\pgfpathlineto{\pgfqpoint{9.371549in}{1.724442in}}%
\pgfpathlineto{\pgfqpoint{9.386982in}{1.724843in}}%
\pgfpathlineto{\pgfqpoint{9.402416in}{1.726745in}}%
\pgfpathlineto{\pgfqpoint{9.417850in}{1.730190in}}%
\pgfpathlineto{\pgfqpoint{9.433283in}{1.735187in}}%
\pgfpathlineto{\pgfqpoint{9.448717in}{1.741713in}}%
\pgfpathlineto{\pgfqpoint{9.464151in}{1.749714in}}%
\pgfpathlineto{\pgfqpoint{9.495018in}{1.769781in}}%
\pgfpathlineto{\pgfqpoint{9.525885in}{1.794394in}}%
\pgfpathlineto{\pgfqpoint{9.556752in}{1.822201in}}%
\pgfpathlineto{\pgfqpoint{9.649354in}{1.908406in}}%
\pgfpathlineto{\pgfqpoint{9.680222in}{1.932682in}}%
\pgfpathlineto{\pgfqpoint{9.711089in}{1.952584in}}%
\pgfpathlineto{\pgfqpoint{9.726522in}{1.960658in}}%
\pgfpathlineto{\pgfqpoint{9.741956in}{1.967418in}}%
\pgfpathlineto{\pgfqpoint{9.741956in}{1.967418in}}%
\pgfusepath{stroke}%
\end{pgfscope}%
\begin{pgfscope}%
\pgfpathrectangle{\pgfqpoint{5.706832in}{0.521603in}}{\pgfqpoint{4.227273in}{2.800000in}} %
\pgfusepath{clip}%
\pgfsetrectcap%
\pgfsetroundjoin%
\pgfsetlinewidth{0.501875pt}%
\definecolor{currentstroke}{rgb}{0.700000,0.951057,0.587785}%
\pgfsetstrokecolor{currentstroke}%
\pgfsetdash{}{0pt}%
\pgfpathmoveto{\pgfqpoint{5.898981in}{2.199507in}}%
\pgfpathlineto{\pgfqpoint{5.914415in}{2.200379in}}%
\pgfpathlineto{\pgfqpoint{5.929848in}{2.198553in}}%
\pgfpathlineto{\pgfqpoint{5.945282in}{2.193934in}}%
\pgfpathlineto{\pgfqpoint{5.960715in}{2.186478in}}%
\pgfpathlineto{\pgfqpoint{5.976149in}{2.176197in}}%
\pgfpathlineto{\pgfqpoint{5.991583in}{2.163162in}}%
\pgfpathlineto{\pgfqpoint{6.007016in}{2.147497in}}%
\pgfpathlineto{\pgfqpoint{6.022450in}{2.129381in}}%
\pgfpathlineto{\pgfqpoint{6.037884in}{2.109043in}}%
\pgfpathlineto{\pgfqpoint{6.068751in}{2.062849in}}%
\pgfpathlineto{\pgfqpoint{6.099618in}{2.011600in}}%
\pgfpathlineto{\pgfqpoint{6.161353in}{1.906976in}}%
\pgfpathlineto{\pgfqpoint{6.192220in}{1.860668in}}%
\pgfpathlineto{\pgfqpoint{6.207654in}{1.840553in}}%
\pgfpathlineto{\pgfqpoint{6.223087in}{1.823014in}}%
\pgfpathlineto{\pgfqpoint{6.238521in}{1.808427in}}%
\pgfpathlineto{\pgfqpoint{6.253955in}{1.797139in}}%
\pgfpathlineto{\pgfqpoint{6.269388in}{1.789453in}}%
\pgfpathlineto{\pgfqpoint{6.284822in}{1.785632in}}%
\pgfpathlineto{\pgfqpoint{6.300255in}{1.785885in}}%
\pgfpathlineto{\pgfqpoint{6.315689in}{1.790367in}}%
\pgfpathlineto{\pgfqpoint{6.331123in}{1.799174in}}%
\pgfpathlineto{\pgfqpoint{6.346556in}{1.812342in}}%
\pgfpathlineto{\pgfqpoint{6.361990in}{1.829839in}}%
\pgfpathlineto{\pgfqpoint{6.377424in}{1.851570in}}%
\pgfpathlineto{\pgfqpoint{6.392857in}{1.877373in}}%
\pgfpathlineto{\pgfqpoint{6.408291in}{1.907021in}}%
\pgfpathlineto{\pgfqpoint{6.423724in}{1.940225in}}%
\pgfpathlineto{\pgfqpoint{6.454592in}{2.015849in}}%
\pgfpathlineto{\pgfqpoint{6.485459in}{2.100835in}}%
\pgfpathlineto{\pgfqpoint{6.593494in}{2.410698in}}%
\pgfpathlineto{\pgfqpoint{6.624362in}{2.484810in}}%
\pgfpathlineto{\pgfqpoint{6.639795in}{2.517085in}}%
\pgfpathlineto{\pgfqpoint{6.655229in}{2.545704in}}%
\pgfpathlineto{\pgfqpoint{6.670663in}{2.570390in}}%
\pgfpathlineto{\pgfqpoint{6.686096in}{2.590929in}}%
\pgfpathlineto{\pgfqpoint{6.701530in}{2.607173in}}%
\pgfpathlineto{\pgfqpoint{6.716964in}{2.619036in}}%
\pgfpathlineto{\pgfqpoint{6.732397in}{2.626495in}}%
\pgfpathlineto{\pgfqpoint{6.747831in}{2.629587in}}%
\pgfpathlineto{\pgfqpoint{6.763264in}{2.628401in}}%
\pgfpathlineto{\pgfqpoint{6.778698in}{2.623077in}}%
\pgfpathlineto{\pgfqpoint{6.794132in}{2.613795in}}%
\pgfpathlineto{\pgfqpoint{6.809565in}{2.600772in}}%
\pgfpathlineto{\pgfqpoint{6.824999in}{2.584254in}}%
\pgfpathlineto{\pgfqpoint{6.840433in}{2.564506in}}%
\pgfpathlineto{\pgfqpoint{6.855866in}{2.541812in}}%
\pgfpathlineto{\pgfqpoint{6.871300in}{2.516461in}}%
\pgfpathlineto{\pgfqpoint{6.902167in}{2.458964in}}%
\pgfpathlineto{\pgfqpoint{6.933034in}{2.394329in}}%
\pgfpathlineto{\pgfqpoint{6.979335in}{2.288769in}}%
\pgfpathlineto{\pgfqpoint{7.087371in}{2.036038in}}%
\pgfpathlineto{\pgfqpoint{7.118238in}{1.970013in}}%
\pgfpathlineto{\pgfqpoint{7.149105in}{1.909673in}}%
\pgfpathlineto{\pgfqpoint{7.179973in}{1.856415in}}%
\pgfpathlineto{\pgfqpoint{7.195406in}{1.832811in}}%
\pgfpathlineto{\pgfqpoint{7.210840in}{1.811375in}}%
\pgfpathlineto{\pgfqpoint{7.226274in}{1.792187in}}%
\pgfpathlineto{\pgfqpoint{7.241707in}{1.775294in}}%
\pgfpathlineto{\pgfqpoint{7.257141in}{1.760705in}}%
\pgfpathlineto{\pgfqpoint{7.272574in}{1.748393in}}%
\pgfpathlineto{\pgfqpoint{7.288008in}{1.738287in}}%
\pgfpathlineto{\pgfqpoint{7.303442in}{1.730278in}}%
\pgfpathlineto{\pgfqpoint{7.318875in}{1.724213in}}%
\pgfpathlineto{\pgfqpoint{7.334309in}{1.719902in}}%
\pgfpathlineto{\pgfqpoint{7.349743in}{1.717118in}}%
\pgfpathlineto{\pgfqpoint{7.380610in}{1.715076in}}%
\pgfpathlineto{\pgfqpoint{7.473212in}{1.715603in}}%
\pgfpathlineto{\pgfqpoint{7.488645in}{1.713633in}}%
\pgfpathlineto{\pgfqpoint{7.504079in}{1.710466in}}%
\pgfpathlineto{\pgfqpoint{7.519513in}{1.705971in}}%
\pgfpathlineto{\pgfqpoint{7.534946in}{1.700069in}}%
\pgfpathlineto{\pgfqpoint{7.550380in}{1.692742in}}%
\pgfpathlineto{\pgfqpoint{7.565814in}{1.684032in}}%
\pgfpathlineto{\pgfqpoint{7.596681in}{1.662941in}}%
\pgfpathlineto{\pgfqpoint{7.627548in}{1.638322in}}%
\pgfpathlineto{\pgfqpoint{7.673849in}{1.600030in}}%
\pgfpathlineto{\pgfqpoint{7.704716in}{1.577888in}}%
\pgfpathlineto{\pgfqpoint{7.720150in}{1.568993in}}%
\pgfpathlineto{\pgfqpoint{7.735584in}{1.562038in}}%
\pgfpathlineto{\pgfqpoint{7.751017in}{1.557351in}}%
\pgfpathlineto{\pgfqpoint{7.766451in}{1.555216in}}%
\pgfpathlineto{\pgfqpoint{7.781884in}{1.555863in}}%
\pgfpathlineto{\pgfqpoint{7.797318in}{1.559464in}}%
\pgfpathlineto{\pgfqpoint{7.812752in}{1.566118in}}%
\pgfpathlineto{\pgfqpoint{7.828185in}{1.575853in}}%
\pgfpathlineto{\pgfqpoint{7.843619in}{1.588621in}}%
\pgfpathlineto{\pgfqpoint{7.859053in}{1.604297in}}%
\pgfpathlineto{\pgfqpoint{7.874486in}{1.622678in}}%
\pgfpathlineto{\pgfqpoint{7.889920in}{1.643492in}}%
\pgfpathlineto{\pgfqpoint{7.920787in}{1.691006in}}%
\pgfpathlineto{\pgfqpoint{7.967088in}{1.770345in}}%
\pgfpathlineto{\pgfqpoint{7.997955in}{1.822722in}}%
\pgfpathlineto{\pgfqpoint{8.028823in}{1.869844in}}%
\pgfpathlineto{\pgfqpoint{8.044256in}{1.890235in}}%
\pgfpathlineto{\pgfqpoint{8.059690in}{1.907957in}}%
\pgfpathlineto{\pgfqpoint{8.075123in}{1.922663in}}%
\pgfpathlineto{\pgfqpoint{8.090557in}{1.934071in}}%
\pgfpathlineto{\pgfqpoint{8.105991in}{1.941974in}}%
\pgfpathlineto{\pgfqpoint{8.121424in}{1.946238in}}%
\pgfpathlineto{\pgfqpoint{8.136858in}{1.946812in}}%
\pgfpathlineto{\pgfqpoint{8.152292in}{1.943718in}}%
\pgfpathlineto{\pgfqpoint{8.167725in}{1.937056in}}%
\pgfpathlineto{\pgfqpoint{8.183159in}{1.926996in}}%
\pgfpathlineto{\pgfqpoint{8.198593in}{1.913773in}}%
\pgfpathlineto{\pgfqpoint{8.214026in}{1.897677in}}%
\pgfpathlineto{\pgfqpoint{8.229460in}{1.879046in}}%
\pgfpathlineto{\pgfqpoint{8.260327in}{1.835687in}}%
\pgfpathlineto{\pgfqpoint{8.291194in}{1.786892in}}%
\pgfpathlineto{\pgfqpoint{8.368363in}{1.661347in}}%
\pgfpathlineto{\pgfqpoint{8.399230in}{1.616097in}}%
\pgfpathlineto{\pgfqpoint{8.430097in}{1.575886in}}%
\pgfpathlineto{\pgfqpoint{8.460964in}{1.541153in}}%
\pgfpathlineto{\pgfqpoint{8.491832in}{1.511731in}}%
\pgfpathlineto{\pgfqpoint{8.522699in}{1.487068in}}%
\pgfpathlineto{\pgfqpoint{8.553566in}{1.466472in}}%
\pgfpathlineto{\pgfqpoint{8.584433in}{1.449364in}}%
\pgfpathlineto{\pgfqpoint{8.615301in}{1.435458in}}%
\pgfpathlineto{\pgfqpoint{8.646168in}{1.424872in}}%
\pgfpathlineto{\pgfqpoint{8.677035in}{1.418118in}}%
\pgfpathlineto{\pgfqpoint{8.692469in}{1.416423in}}%
\pgfpathlineto{\pgfqpoint{8.707903in}{1.416003in}}%
\pgfpathlineto{\pgfqpoint{8.723336in}{1.416974in}}%
\pgfpathlineto{\pgfqpoint{8.738770in}{1.419446in}}%
\pgfpathlineto{\pgfqpoint{8.754203in}{1.423512in}}%
\pgfpathlineto{\pgfqpoint{8.769637in}{1.429246in}}%
\pgfpathlineto{\pgfqpoint{8.785071in}{1.436690in}}%
\pgfpathlineto{\pgfqpoint{8.800504in}{1.445855in}}%
\pgfpathlineto{\pgfqpoint{8.815938in}{1.456712in}}%
\pgfpathlineto{\pgfqpoint{8.831372in}{1.469193in}}%
\pgfpathlineto{\pgfqpoint{8.862239in}{1.498547in}}%
\pgfpathlineto{\pgfqpoint{8.893106in}{1.532569in}}%
\pgfpathlineto{\pgfqpoint{9.001142in}{1.659130in}}%
\pgfpathlineto{\pgfqpoint{9.032009in}{1.688694in}}%
\pgfpathlineto{\pgfqpoint{9.047443in}{1.701350in}}%
\pgfpathlineto{\pgfqpoint{9.062876in}{1.712454in}}%
\pgfpathlineto{\pgfqpoint{9.078310in}{1.721966in}}%
\pgfpathlineto{\pgfqpoint{9.093743in}{1.729893in}}%
\pgfpathlineto{\pgfqpoint{9.109177in}{1.736294in}}%
\pgfpathlineto{\pgfqpoint{9.124611in}{1.741268in}}%
\pgfpathlineto{\pgfqpoint{9.155478in}{1.747538in}}%
\pgfpathlineto{\pgfqpoint{9.186345in}{1.750212in}}%
\pgfpathlineto{\pgfqpoint{9.263513in}{1.753267in}}%
\pgfpathlineto{\pgfqpoint{9.294381in}{1.757589in}}%
\pgfpathlineto{\pgfqpoint{9.325248in}{1.765551in}}%
\pgfpathlineto{\pgfqpoint{9.356115in}{1.777671in}}%
\pgfpathlineto{\pgfqpoint{9.386982in}{1.793887in}}%
\pgfpathlineto{\pgfqpoint{9.417850in}{1.813618in}}%
\pgfpathlineto{\pgfqpoint{9.464151in}{1.847555in}}%
\pgfpathlineto{\pgfqpoint{9.525885in}{1.894041in}}%
\pgfpathlineto{\pgfqpoint{9.556752in}{1.914828in}}%
\pgfpathlineto{\pgfqpoint{9.587620in}{1.932476in}}%
\pgfpathlineto{\pgfqpoint{9.618487in}{1.946178in}}%
\pgfpathlineto{\pgfqpoint{9.649354in}{1.955434in}}%
\pgfpathlineto{\pgfqpoint{9.664788in}{1.958330in}}%
\pgfpathlineto{\pgfqpoint{9.680222in}{1.960075in}}%
\pgfpathlineto{\pgfqpoint{9.711089in}{1.960250in}}%
\pgfpathlineto{\pgfqpoint{9.741956in}{1.956412in}}%
\pgfpathlineto{\pgfqpoint{9.741956in}{1.956412in}}%
\pgfusepath{stroke}%
\end{pgfscope}%
\begin{pgfscope}%
\pgfpathrectangle{\pgfqpoint{5.706832in}{0.521603in}}{\pgfqpoint{4.227273in}{2.800000in}} %
\pgfusepath{clip}%
\pgfsetrectcap%
\pgfsetroundjoin%
\pgfsetlinewidth{0.501875pt}%
\definecolor{currentstroke}{rgb}{0.778431,0.905873,0.536867}%
\pgfsetstrokecolor{currentstroke}%
\pgfsetdash{}{0pt}%
\pgfpathmoveto{\pgfqpoint{5.898981in}{2.152483in}}%
\pgfpathlineto{\pgfqpoint{5.914415in}{2.149719in}}%
\pgfpathlineto{\pgfqpoint{5.929848in}{2.144672in}}%
\pgfpathlineto{\pgfqpoint{5.945282in}{2.137253in}}%
\pgfpathlineto{\pgfqpoint{5.960715in}{2.127413in}}%
\pgfpathlineto{\pgfqpoint{5.976149in}{2.115143in}}%
\pgfpathlineto{\pgfqpoint{5.991583in}{2.100484in}}%
\pgfpathlineto{\pgfqpoint{6.007016in}{2.083518in}}%
\pgfpathlineto{\pgfqpoint{6.022450in}{2.064377in}}%
\pgfpathlineto{\pgfqpoint{6.053317in}{2.020335in}}%
\pgfpathlineto{\pgfqpoint{6.084185in}{1.970327in}}%
\pgfpathlineto{\pgfqpoint{6.176786in}{1.813204in}}%
\pgfpathlineto{\pgfqpoint{6.192220in}{1.790465in}}%
\pgfpathlineto{\pgfqpoint{6.207654in}{1.769911in}}%
\pgfpathlineto{\pgfqpoint{6.223087in}{1.751978in}}%
\pgfpathlineto{\pgfqpoint{6.238521in}{1.737075in}}%
\pgfpathlineto{\pgfqpoint{6.253955in}{1.725582in}}%
\pgfpathlineto{\pgfqpoint{6.269388in}{1.717837in}}%
\pgfpathlineto{\pgfqpoint{6.284822in}{1.714132in}}%
\pgfpathlineto{\pgfqpoint{6.300255in}{1.714707in}}%
\pgfpathlineto{\pgfqpoint{6.315689in}{1.719737in}}%
\pgfpathlineto{\pgfqpoint{6.331123in}{1.729335in}}%
\pgfpathlineto{\pgfqpoint{6.346556in}{1.743544in}}%
\pgfpathlineto{\pgfqpoint{6.361990in}{1.762333in}}%
\pgfpathlineto{\pgfqpoint{6.377424in}{1.785598in}}%
\pgfpathlineto{\pgfqpoint{6.392857in}{1.813161in}}%
\pgfpathlineto{\pgfqpoint{6.408291in}{1.844771in}}%
\pgfpathlineto{\pgfqpoint{6.423724in}{1.880108in}}%
\pgfpathlineto{\pgfqpoint{6.454592in}{1.960373in}}%
\pgfpathlineto{\pgfqpoint{6.485459in}{2.050233in}}%
\pgfpathlineto{\pgfqpoint{6.578061in}{2.332204in}}%
\pgfpathlineto{\pgfqpoint{6.608928in}{2.414977in}}%
\pgfpathlineto{\pgfqpoint{6.624362in}{2.451966in}}%
\pgfpathlineto{\pgfqpoint{6.639795in}{2.485497in}}%
\pgfpathlineto{\pgfqpoint{6.655229in}{2.515249in}}%
\pgfpathlineto{\pgfqpoint{6.670663in}{2.540968in}}%
\pgfpathlineto{\pgfqpoint{6.686096in}{2.562465in}}%
\pgfpathlineto{\pgfqpoint{6.701530in}{2.579617in}}%
\pgfpathlineto{\pgfqpoint{6.716964in}{2.592364in}}%
\pgfpathlineto{\pgfqpoint{6.732397in}{2.600711in}}%
\pgfpathlineto{\pgfqpoint{6.747831in}{2.604715in}}%
\pgfpathlineto{\pgfqpoint{6.763264in}{2.604486in}}%
\pgfpathlineto{\pgfqpoint{6.778698in}{2.600180in}}%
\pgfpathlineto{\pgfqpoint{6.794132in}{2.591987in}}%
\pgfpathlineto{\pgfqpoint{6.809565in}{2.580131in}}%
\pgfpathlineto{\pgfqpoint{6.824999in}{2.564855in}}%
\pgfpathlineto{\pgfqpoint{6.840433in}{2.546421in}}%
\pgfpathlineto{\pgfqpoint{6.855866in}{2.525100in}}%
\pgfpathlineto{\pgfqpoint{6.871300in}{2.501165in}}%
\pgfpathlineto{\pgfqpoint{6.902167in}{2.446545in}}%
\pgfpathlineto{\pgfqpoint{6.933034in}{2.384671in}}%
\pgfpathlineto{\pgfqpoint{6.963902in}{2.317514in}}%
\pgfpathlineto{\pgfqpoint{7.010203in}{2.210811in}}%
\pgfpathlineto{\pgfqpoint{7.087371in}{2.030960in}}%
\pgfpathlineto{\pgfqpoint{7.118238in}{1.963109in}}%
\pgfpathlineto{\pgfqpoint{7.149105in}{1.900078in}}%
\pgfpathlineto{\pgfqpoint{7.179973in}{1.843447in}}%
\pgfpathlineto{\pgfqpoint{7.210840in}{1.794597in}}%
\pgfpathlineto{\pgfqpoint{7.226274in}{1.773432in}}%
\pgfpathlineto{\pgfqpoint{7.241707in}{1.754566in}}%
\pgfpathlineto{\pgfqpoint{7.257141in}{1.738049in}}%
\pgfpathlineto{\pgfqpoint{7.272574in}{1.723894in}}%
\pgfpathlineto{\pgfqpoint{7.288008in}{1.712069in}}%
\pgfpathlineto{\pgfqpoint{7.303442in}{1.702503in}}%
\pgfpathlineto{\pgfqpoint{7.318875in}{1.695080in}}%
\pgfpathlineto{\pgfqpoint{7.334309in}{1.689641in}}%
\pgfpathlineto{\pgfqpoint{7.349743in}{1.685990in}}%
\pgfpathlineto{\pgfqpoint{7.365176in}{1.683896in}}%
\pgfpathlineto{\pgfqpoint{7.396044in}{1.683309in}}%
\pgfpathlineto{\pgfqpoint{7.442344in}{1.686999in}}%
\pgfpathlineto{\pgfqpoint{7.473212in}{1.689041in}}%
\pgfpathlineto{\pgfqpoint{7.504079in}{1.688347in}}%
\pgfpathlineto{\pgfqpoint{7.519513in}{1.686506in}}%
\pgfpathlineto{\pgfqpoint{7.534946in}{1.683514in}}%
\pgfpathlineto{\pgfqpoint{7.550380in}{1.679325in}}%
\pgfpathlineto{\pgfqpoint{7.581247in}{1.667461in}}%
\pgfpathlineto{\pgfqpoint{7.612114in}{1.651709in}}%
\pgfpathlineto{\pgfqpoint{7.689283in}{1.607784in}}%
\pgfpathlineto{\pgfqpoint{7.704716in}{1.600790in}}%
\pgfpathlineto{\pgfqpoint{7.720150in}{1.595204in}}%
\pgfpathlineto{\pgfqpoint{7.735584in}{1.591355in}}%
\pgfpathlineto{\pgfqpoint{7.751017in}{1.589536in}}%
\pgfpathlineto{\pgfqpoint{7.766451in}{1.590003in}}%
\pgfpathlineto{\pgfqpoint{7.781884in}{1.592960in}}%
\pgfpathlineto{\pgfqpoint{7.797318in}{1.598550in}}%
\pgfpathlineto{\pgfqpoint{7.812752in}{1.606854in}}%
\pgfpathlineto{\pgfqpoint{7.828185in}{1.617882in}}%
\pgfpathlineto{\pgfqpoint{7.843619in}{1.631572in}}%
\pgfpathlineto{\pgfqpoint{7.859053in}{1.647790in}}%
\pgfpathlineto{\pgfqpoint{7.874486in}{1.666333in}}%
\pgfpathlineto{\pgfqpoint{7.905353in}{1.709250in}}%
\pgfpathlineto{\pgfqpoint{7.936221in}{1.757486in}}%
\pgfpathlineto{\pgfqpoint{7.997955in}{1.855528in}}%
\pgfpathlineto{\pgfqpoint{8.028823in}{1.897727in}}%
\pgfpathlineto{\pgfqpoint{8.044256in}{1.915587in}}%
\pgfpathlineto{\pgfqpoint{8.059690in}{1.930811in}}%
\pgfpathlineto{\pgfqpoint{8.075123in}{1.943109in}}%
\pgfpathlineto{\pgfqpoint{8.090557in}{1.952261in}}%
\pgfpathlineto{\pgfqpoint{8.105991in}{1.958116in}}%
\pgfpathlineto{\pgfqpoint{8.121424in}{1.960593in}}%
\pgfpathlineto{\pgfqpoint{8.136858in}{1.959688in}}%
\pgfpathlineto{\pgfqpoint{8.152292in}{1.955469in}}%
\pgfpathlineto{\pgfqpoint{8.167725in}{1.948068in}}%
\pgfpathlineto{\pgfqpoint{8.183159in}{1.937684in}}%
\pgfpathlineto{\pgfqpoint{8.198593in}{1.924569in}}%
\pgfpathlineto{\pgfqpoint{8.214026in}{1.909022in}}%
\pgfpathlineto{\pgfqpoint{8.229460in}{1.891378in}}%
\pgfpathlineto{\pgfqpoint{8.260327in}{1.851254in}}%
\pgfpathlineto{\pgfqpoint{8.306628in}{1.784585in}}%
\pgfpathlineto{\pgfqpoint{8.352929in}{1.718275in}}%
\pgfpathlineto{\pgfqpoint{8.383796in}{1.677675in}}%
\pgfpathlineto{\pgfqpoint{8.414663in}{1.641290in}}%
\pgfpathlineto{\pgfqpoint{8.445531in}{1.609420in}}%
\pgfpathlineto{\pgfqpoint{8.476398in}{1.581750in}}%
\pgfpathlineto{\pgfqpoint{8.507265in}{1.557551in}}%
\pgfpathlineto{\pgfqpoint{8.538133in}{1.535951in}}%
\pgfpathlineto{\pgfqpoint{8.584433in}{1.506852in}}%
\pgfpathlineto{\pgfqpoint{8.630734in}{1.480992in}}%
\pgfpathlineto{\pgfqpoint{8.661602in}{1.466195in}}%
\pgfpathlineto{\pgfqpoint{8.692469in}{1.454499in}}%
\pgfpathlineto{\pgfqpoint{8.707903in}{1.450222in}}%
\pgfpathlineto{\pgfqpoint{8.723336in}{1.447228in}}%
\pgfpathlineto{\pgfqpoint{8.738770in}{1.445690in}}%
\pgfpathlineto{\pgfqpoint{8.754203in}{1.445767in}}%
\pgfpathlineto{\pgfqpoint{8.769637in}{1.447599in}}%
\pgfpathlineto{\pgfqpoint{8.785071in}{1.451296in}}%
\pgfpathlineto{\pgfqpoint{8.800504in}{1.456935in}}%
\pgfpathlineto{\pgfqpoint{8.815938in}{1.464552in}}%
\pgfpathlineto{\pgfqpoint{8.831372in}{1.474141in}}%
\pgfpathlineto{\pgfqpoint{8.846805in}{1.485647in}}%
\pgfpathlineto{\pgfqpoint{8.862239in}{1.498972in}}%
\pgfpathlineto{\pgfqpoint{8.893106in}{1.530463in}}%
\pgfpathlineto{\pgfqpoint{8.923973in}{1.566998in}}%
\pgfpathlineto{\pgfqpoint{9.032009in}{1.703055in}}%
\pgfpathlineto{\pgfqpoint{9.062876in}{1.735566in}}%
\pgfpathlineto{\pgfqpoint{9.093743in}{1.762640in}}%
\pgfpathlineto{\pgfqpoint{9.109177in}{1.774002in}}%
\pgfpathlineto{\pgfqpoint{9.124611in}{1.783936in}}%
\pgfpathlineto{\pgfqpoint{9.155478in}{1.799832in}}%
\pgfpathlineto{\pgfqpoint{9.186345in}{1.811314in}}%
\pgfpathlineto{\pgfqpoint{9.217212in}{1.819787in}}%
\pgfpathlineto{\pgfqpoint{9.294381in}{1.838057in}}%
\pgfpathlineto{\pgfqpoint{9.325248in}{1.847632in}}%
\pgfpathlineto{\pgfqpoint{9.356115in}{1.859609in}}%
\pgfpathlineto{\pgfqpoint{9.386982in}{1.874000in}}%
\pgfpathlineto{\pgfqpoint{9.433283in}{1.898886in}}%
\pgfpathlineto{\pgfqpoint{9.495018in}{1.932705in}}%
\pgfpathlineto{\pgfqpoint{9.525885in}{1.947059in}}%
\pgfpathlineto{\pgfqpoint{9.556752in}{1.958161in}}%
\pgfpathlineto{\pgfqpoint{9.587620in}{1.965177in}}%
\pgfpathlineto{\pgfqpoint{9.603053in}{1.966990in}}%
\pgfpathlineto{\pgfqpoint{9.618487in}{1.967640in}}%
\pgfpathlineto{\pgfqpoint{9.633921in}{1.967132in}}%
\pgfpathlineto{\pgfqpoint{9.664788in}{1.962795in}}%
\pgfpathlineto{\pgfqpoint{9.695655in}{1.954514in}}%
\pgfpathlineto{\pgfqpoint{9.726522in}{1.943137in}}%
\pgfpathlineto{\pgfqpoint{9.741956in}{1.936614in}}%
\pgfpathlineto{\pgfqpoint{9.741956in}{1.936614in}}%
\pgfusepath{stroke}%
\end{pgfscope}%
\begin{pgfscope}%
\pgfpathrectangle{\pgfqpoint{5.706832in}{0.521603in}}{\pgfqpoint{4.227273in}{2.800000in}} %
\pgfusepath{clip}%
\pgfsetrectcap%
\pgfsetroundjoin%
\pgfsetlinewidth{0.501875pt}%
\definecolor{currentstroke}{rgb}{0.864706,0.840344,0.478512}%
\pgfsetstrokecolor{currentstroke}%
\pgfsetdash{}{0pt}%
\pgfpathmoveto{\pgfqpoint{5.898981in}{2.174443in}}%
\pgfpathlineto{\pgfqpoint{5.914415in}{2.171583in}}%
\pgfpathlineto{\pgfqpoint{5.929848in}{2.166149in}}%
\pgfpathlineto{\pgfqpoint{5.945282in}{2.158096in}}%
\pgfpathlineto{\pgfqpoint{5.960715in}{2.147427in}}%
\pgfpathlineto{\pgfqpoint{5.976149in}{2.134190in}}%
\pgfpathlineto{\pgfqpoint{5.991583in}{2.118485in}}%
\pgfpathlineto{\pgfqpoint{6.007016in}{2.100457in}}%
\pgfpathlineto{\pgfqpoint{6.022450in}{2.080301in}}%
\pgfpathlineto{\pgfqpoint{6.053317in}{2.034595in}}%
\pgfpathlineto{\pgfqpoint{6.084185in}{1.983752in}}%
\pgfpathlineto{\pgfqpoint{6.145919in}{1.878658in}}%
\pgfpathlineto{\pgfqpoint{6.176786in}{1.831121in}}%
\pgfpathlineto{\pgfqpoint{6.192220in}{1.810105in}}%
\pgfpathlineto{\pgfqpoint{6.207654in}{1.791471in}}%
\pgfpathlineto{\pgfqpoint{6.223087in}{1.775603in}}%
\pgfpathlineto{\pgfqpoint{6.238521in}{1.762857in}}%
\pgfpathlineto{\pgfqpoint{6.253955in}{1.753555in}}%
\pgfpathlineto{\pgfqpoint{6.269388in}{1.747979in}}%
\pgfpathlineto{\pgfqpoint{6.284822in}{1.746365in}}%
\pgfpathlineto{\pgfqpoint{6.300255in}{1.748900in}}%
\pgfpathlineto{\pgfqpoint{6.315689in}{1.755717in}}%
\pgfpathlineto{\pgfqpoint{6.331123in}{1.766891in}}%
\pgfpathlineto{\pgfqpoint{6.346556in}{1.782437in}}%
\pgfpathlineto{\pgfqpoint{6.361990in}{1.802307in}}%
\pgfpathlineto{\pgfqpoint{6.377424in}{1.826393in}}%
\pgfpathlineto{\pgfqpoint{6.392857in}{1.854520in}}%
\pgfpathlineto{\pgfqpoint{6.408291in}{1.886455in}}%
\pgfpathlineto{\pgfqpoint{6.423724in}{1.921901in}}%
\pgfpathlineto{\pgfqpoint{6.454592in}{2.001874in}}%
\pgfpathlineto{\pgfqpoint{6.485459in}{2.091041in}}%
\pgfpathlineto{\pgfqpoint{6.578061in}{2.371742in}}%
\pgfpathlineto{\pgfqpoint{6.608928in}{2.454940in}}%
\pgfpathlineto{\pgfqpoint{6.624362in}{2.492275in}}%
\pgfpathlineto{\pgfqpoint{6.639795in}{2.526196in}}%
\pgfpathlineto{\pgfqpoint{6.655229in}{2.556336in}}%
\pgfpathlineto{\pgfqpoint{6.670663in}{2.582387in}}%
\pgfpathlineto{\pgfqpoint{6.686096in}{2.604100in}}%
\pgfpathlineto{\pgfqpoint{6.701530in}{2.621290in}}%
\pgfpathlineto{\pgfqpoint{6.716964in}{2.633834in}}%
\pgfpathlineto{\pgfqpoint{6.732397in}{2.641674in}}%
\pgfpathlineto{\pgfqpoint{6.747831in}{2.644811in}}%
\pgfpathlineto{\pgfqpoint{6.763264in}{2.643308in}}%
\pgfpathlineto{\pgfqpoint{6.778698in}{2.637280in}}%
\pgfpathlineto{\pgfqpoint{6.794132in}{2.626893in}}%
\pgfpathlineto{\pgfqpoint{6.809565in}{2.612359in}}%
\pgfpathlineto{\pgfqpoint{6.824999in}{2.593927in}}%
\pgfpathlineto{\pgfqpoint{6.840433in}{2.571879in}}%
\pgfpathlineto{\pgfqpoint{6.855866in}{2.546523in}}%
\pgfpathlineto{\pgfqpoint{6.871300in}{2.518188in}}%
\pgfpathlineto{\pgfqpoint{6.902167in}{2.453966in}}%
\pgfpathlineto{\pgfqpoint{6.933034in}{2.382040in}}%
\pgfpathlineto{\pgfqpoint{6.979335in}{2.265878in}}%
\pgfpathlineto{\pgfqpoint{7.041070in}{2.109390in}}%
\pgfpathlineto{\pgfqpoint{7.071937in}{2.035722in}}%
\pgfpathlineto{\pgfqpoint{7.102804in}{1.967734in}}%
\pgfpathlineto{\pgfqpoint{7.133672in}{1.906997in}}%
\pgfpathlineto{\pgfqpoint{7.164539in}{1.854708in}}%
\pgfpathlineto{\pgfqpoint{7.179973in}{1.831988in}}%
\pgfpathlineto{\pgfqpoint{7.195406in}{1.811630in}}%
\pgfpathlineto{\pgfqpoint{7.210840in}{1.793651in}}%
\pgfpathlineto{\pgfqpoint{7.226274in}{1.778038in}}%
\pgfpathlineto{\pgfqpoint{7.241707in}{1.764741in}}%
\pgfpathlineto{\pgfqpoint{7.257141in}{1.753677in}}%
\pgfpathlineto{\pgfqpoint{7.272574in}{1.744728in}}%
\pgfpathlineto{\pgfqpoint{7.288008in}{1.737745in}}%
\pgfpathlineto{\pgfqpoint{7.303442in}{1.732547in}}%
\pgfpathlineto{\pgfqpoint{7.318875in}{1.728924in}}%
\pgfpathlineto{\pgfqpoint{7.349743in}{1.725458in}}%
\pgfpathlineto{\pgfqpoint{7.380610in}{1.725281in}}%
\pgfpathlineto{\pgfqpoint{7.442344in}{1.726089in}}%
\pgfpathlineto{\pgfqpoint{7.473212in}{1.723120in}}%
\pgfpathlineto{\pgfqpoint{7.488645in}{1.720120in}}%
\pgfpathlineto{\pgfqpoint{7.504079in}{1.715950in}}%
\pgfpathlineto{\pgfqpoint{7.519513in}{1.710542in}}%
\pgfpathlineto{\pgfqpoint{7.534946in}{1.703880in}}%
\pgfpathlineto{\pgfqpoint{7.565814in}{1.686955in}}%
\pgfpathlineto{\pgfqpoint{7.596681in}{1.665982in}}%
\pgfpathlineto{\pgfqpoint{7.689283in}{1.596470in}}%
\pgfpathlineto{\pgfqpoint{7.704716in}{1.586989in}}%
\pgfpathlineto{\pgfqpoint{7.720150in}{1.578941in}}%
\pgfpathlineto{\pgfqpoint{7.735584in}{1.572615in}}%
\pgfpathlineto{\pgfqpoint{7.751017in}{1.568272in}}%
\pgfpathlineto{\pgfqpoint{7.766451in}{1.566138in}}%
\pgfpathlineto{\pgfqpoint{7.781884in}{1.566399in}}%
\pgfpathlineto{\pgfqpoint{7.797318in}{1.569191in}}%
\pgfpathlineto{\pgfqpoint{7.812752in}{1.574600in}}%
\pgfpathlineto{\pgfqpoint{7.828185in}{1.582655in}}%
\pgfpathlineto{\pgfqpoint{7.843619in}{1.593327in}}%
\pgfpathlineto{\pgfqpoint{7.859053in}{1.606529in}}%
\pgfpathlineto{\pgfqpoint{7.874486in}{1.622114in}}%
\pgfpathlineto{\pgfqpoint{7.889920in}{1.639881in}}%
\pgfpathlineto{\pgfqpoint{7.920787in}{1.680905in}}%
\pgfpathlineto{\pgfqpoint{7.951654in}{1.727053in}}%
\pgfpathlineto{\pgfqpoint{8.013389in}{1.822190in}}%
\pgfpathlineto{\pgfqpoint{8.044256in}{1.864594in}}%
\pgfpathlineto{\pgfqpoint{8.059690in}{1.883161in}}%
\pgfpathlineto{\pgfqpoint{8.075123in}{1.899541in}}%
\pgfpathlineto{\pgfqpoint{8.090557in}{1.913465in}}%
\pgfpathlineto{\pgfqpoint{8.105991in}{1.924712in}}%
\pgfpathlineto{\pgfqpoint{8.121424in}{1.933117in}}%
\pgfpathlineto{\pgfqpoint{8.136858in}{1.938571in}}%
\pgfpathlineto{\pgfqpoint{8.152292in}{1.941021in}}%
\pgfpathlineto{\pgfqpoint{8.167725in}{1.940470in}}%
\pgfpathlineto{\pgfqpoint{8.183159in}{1.936976in}}%
\pgfpathlineto{\pgfqpoint{8.198593in}{1.930646in}}%
\pgfpathlineto{\pgfqpoint{8.214026in}{1.921633in}}%
\pgfpathlineto{\pgfqpoint{8.229460in}{1.910127in}}%
\pgfpathlineto{\pgfqpoint{8.244893in}{1.896353in}}%
\pgfpathlineto{\pgfqpoint{8.260327in}{1.880558in}}%
\pgfpathlineto{\pgfqpoint{8.291194in}{1.843974in}}%
\pgfpathlineto{\pgfqpoint{8.322062in}{1.802561in}}%
\pgfpathlineto{\pgfqpoint{8.383796in}{1.713425in}}%
\pgfpathlineto{\pgfqpoint{8.430097in}{1.647435in}}%
\pgfpathlineto{\pgfqpoint{8.460964in}{1.605909in}}%
\pgfpathlineto{\pgfqpoint{8.491832in}{1.567095in}}%
\pgfpathlineto{\pgfqpoint{8.522699in}{1.531396in}}%
\pgfpathlineto{\pgfqpoint{8.553566in}{1.499157in}}%
\pgfpathlineto{\pgfqpoint{8.584433in}{1.470776in}}%
\pgfpathlineto{\pgfqpoint{8.615301in}{1.446785in}}%
\pgfpathlineto{\pgfqpoint{8.630734in}{1.436644in}}%
\pgfpathlineto{\pgfqpoint{8.646168in}{1.427866in}}%
\pgfpathlineto{\pgfqpoint{8.661602in}{1.420552in}}%
\pgfpathlineto{\pgfqpoint{8.677035in}{1.414804in}}%
\pgfpathlineto{\pgfqpoint{8.692469in}{1.410719in}}%
\pgfpathlineto{\pgfqpoint{8.707903in}{1.408388in}}%
\pgfpathlineto{\pgfqpoint{8.723336in}{1.407890in}}%
\pgfpathlineto{\pgfqpoint{8.738770in}{1.409284in}}%
\pgfpathlineto{\pgfqpoint{8.754203in}{1.412612in}}%
\pgfpathlineto{\pgfqpoint{8.769637in}{1.417887in}}%
\pgfpathlineto{\pgfqpoint{8.785071in}{1.425098in}}%
\pgfpathlineto{\pgfqpoint{8.800504in}{1.434203in}}%
\pgfpathlineto{\pgfqpoint{8.815938in}{1.445128in}}%
\pgfpathlineto{\pgfqpoint{8.831372in}{1.457769in}}%
\pgfpathlineto{\pgfqpoint{8.862239in}{1.487637in}}%
\pgfpathlineto{\pgfqpoint{8.893106in}{1.522423in}}%
\pgfpathlineto{\pgfqpoint{8.939407in}{1.580024in}}%
\pgfpathlineto{\pgfqpoint{9.001142in}{1.657084in}}%
\pgfpathlineto{\pgfqpoint{9.032009in}{1.691879in}}%
\pgfpathlineto{\pgfqpoint{9.062876in}{1.722492in}}%
\pgfpathlineto{\pgfqpoint{9.093743in}{1.748216in}}%
\pgfpathlineto{\pgfqpoint{9.124611in}{1.768835in}}%
\pgfpathlineto{\pgfqpoint{9.155478in}{1.784598in}}%
\pgfpathlineto{\pgfqpoint{9.186345in}{1.796144in}}%
\pgfpathlineto{\pgfqpoint{9.217212in}{1.804398in}}%
\pgfpathlineto{\pgfqpoint{9.263513in}{1.813015in}}%
\pgfpathlineto{\pgfqpoint{9.325248in}{1.823239in}}%
\pgfpathlineto{\pgfqpoint{9.356115in}{1.829861in}}%
\pgfpathlineto{\pgfqpoint{9.386982in}{1.838278in}}%
\pgfpathlineto{\pgfqpoint{9.417850in}{1.848715in}}%
\pgfpathlineto{\pgfqpoint{9.448717in}{1.861099in}}%
\pgfpathlineto{\pgfqpoint{9.495018in}{1.882482in}}%
\pgfpathlineto{\pgfqpoint{9.587620in}{1.926704in}}%
\pgfpathlineto{\pgfqpoint{9.618487in}{1.938872in}}%
\pgfpathlineto{\pgfqpoint{9.649354in}{1.948546in}}%
\pgfpathlineto{\pgfqpoint{9.680222in}{1.955276in}}%
\pgfpathlineto{\pgfqpoint{9.711089in}{1.958867in}}%
\pgfpathlineto{\pgfqpoint{9.741956in}{1.959402in}}%
\pgfpathlineto{\pgfqpoint{9.741956in}{1.959402in}}%
\pgfusepath{stroke}%
\end{pgfscope}%
\begin{pgfscope}%
\pgfpathrectangle{\pgfqpoint{5.706832in}{0.521603in}}{\pgfqpoint{4.227273in}{2.800000in}} %
\pgfusepath{clip}%
\pgfsetrectcap%
\pgfsetroundjoin%
\pgfsetlinewidth{0.501875pt}%
\definecolor{currentstroke}{rgb}{0.943137,0.767363,0.423549}%
\pgfsetstrokecolor{currentstroke}%
\pgfsetdash{}{0pt}%
\pgfpathmoveto{\pgfqpoint{5.898981in}{2.185515in}}%
\pgfpathlineto{\pgfqpoint{5.914415in}{2.181746in}}%
\pgfpathlineto{\pgfqpoint{5.929848in}{2.175300in}}%
\pgfpathlineto{\pgfqpoint{5.945282in}{2.166153in}}%
\pgfpathlineto{\pgfqpoint{5.960715in}{2.154324in}}%
\pgfpathlineto{\pgfqpoint{5.976149in}{2.139876in}}%
\pgfpathlineto{\pgfqpoint{5.991583in}{2.122918in}}%
\pgfpathlineto{\pgfqpoint{6.007016in}{2.103600in}}%
\pgfpathlineto{\pgfqpoint{6.022450in}{2.082119in}}%
\pgfpathlineto{\pgfqpoint{6.053317in}{2.033641in}}%
\pgfpathlineto{\pgfqpoint{6.084185in}{1.979805in}}%
\pgfpathlineto{\pgfqpoint{6.161353in}{1.841155in}}%
\pgfpathlineto{\pgfqpoint{6.192220in}{1.792787in}}%
\pgfpathlineto{\pgfqpoint{6.207654in}{1.771820in}}%
\pgfpathlineto{\pgfqpoint{6.223087in}{1.753537in}}%
\pgfpathlineto{\pgfqpoint{6.238521in}{1.738314in}}%
\pgfpathlineto{\pgfqpoint{6.253955in}{1.726499in}}%
\pgfpathlineto{\pgfqpoint{6.269388in}{1.718403in}}%
\pgfpathlineto{\pgfqpoint{6.284822in}{1.714298in}}%
\pgfpathlineto{\pgfqpoint{6.300255in}{1.714408in}}%
\pgfpathlineto{\pgfqpoint{6.315689in}{1.718906in}}%
\pgfpathlineto{\pgfqpoint{6.331123in}{1.727906in}}%
\pgfpathlineto{\pgfqpoint{6.346556in}{1.741464in}}%
\pgfpathlineto{\pgfqpoint{6.361990in}{1.759570in}}%
\pgfpathlineto{\pgfqpoint{6.377424in}{1.782151in}}%
\pgfpathlineto{\pgfqpoint{6.392857in}{1.809065in}}%
\pgfpathlineto{\pgfqpoint{6.408291in}{1.840102in}}%
\pgfpathlineto{\pgfqpoint{6.423724in}{1.874989in}}%
\pgfpathlineto{\pgfqpoint{6.439158in}{1.913389in}}%
\pgfpathlineto{\pgfqpoint{6.470025in}{1.999085in}}%
\pgfpathlineto{\pgfqpoint{6.500893in}{2.093409in}}%
\pgfpathlineto{\pgfqpoint{6.578061in}{2.337173in}}%
\pgfpathlineto{\pgfqpoint{6.608928in}{2.425350in}}%
\pgfpathlineto{\pgfqpoint{6.624362in}{2.465195in}}%
\pgfpathlineto{\pgfqpoint{6.639795in}{2.501570in}}%
\pgfpathlineto{\pgfqpoint{6.655229in}{2.534057in}}%
\pgfpathlineto{\pgfqpoint{6.670663in}{2.562298in}}%
\pgfpathlineto{\pgfqpoint{6.686096in}{2.585999in}}%
\pgfpathlineto{\pgfqpoint{6.701530in}{2.604936in}}%
\pgfpathlineto{\pgfqpoint{6.716964in}{2.618952in}}%
\pgfpathlineto{\pgfqpoint{6.732397in}{2.627960in}}%
\pgfpathlineto{\pgfqpoint{6.747831in}{2.631944in}}%
\pgfpathlineto{\pgfqpoint{6.763264in}{2.630951in}}%
\pgfpathlineto{\pgfqpoint{6.778698in}{2.625094in}}%
\pgfpathlineto{\pgfqpoint{6.794132in}{2.614542in}}%
\pgfpathlineto{\pgfqpoint{6.809565in}{2.599520in}}%
\pgfpathlineto{\pgfqpoint{6.824999in}{2.580296in}}%
\pgfpathlineto{\pgfqpoint{6.840433in}{2.557179in}}%
\pgfpathlineto{\pgfqpoint{6.855866in}{2.530514in}}%
\pgfpathlineto{\pgfqpoint{6.871300in}{2.500669in}}%
\pgfpathlineto{\pgfqpoint{6.902167in}{2.433006in}}%
\pgfpathlineto{\pgfqpoint{6.933034in}{2.357423in}}%
\pgfpathlineto{\pgfqpoint{6.994769in}{2.195405in}}%
\pgfpathlineto{\pgfqpoint{7.041070in}{2.076127in}}%
\pgfpathlineto{\pgfqpoint{7.071937in}{2.002470in}}%
\pgfpathlineto{\pgfqpoint{7.102804in}{1.935861in}}%
\pgfpathlineto{\pgfqpoint{7.133672in}{1.877809in}}%
\pgfpathlineto{\pgfqpoint{7.149105in}{1.852329in}}%
\pgfpathlineto{\pgfqpoint{7.164539in}{1.829325in}}%
\pgfpathlineto{\pgfqpoint{7.179973in}{1.808841in}}%
\pgfpathlineto{\pgfqpoint{7.195406in}{1.790883in}}%
\pgfpathlineto{\pgfqpoint{7.210840in}{1.775424in}}%
\pgfpathlineto{\pgfqpoint{7.226274in}{1.762400in}}%
\pgfpathlineto{\pgfqpoint{7.241707in}{1.751713in}}%
\pgfpathlineto{\pgfqpoint{7.257141in}{1.743229in}}%
\pgfpathlineto{\pgfqpoint{7.272574in}{1.736783in}}%
\pgfpathlineto{\pgfqpoint{7.288008in}{1.732183in}}%
\pgfpathlineto{\pgfqpoint{7.303442in}{1.729205in}}%
\pgfpathlineto{\pgfqpoint{7.334309in}{1.727126in}}%
\pgfpathlineto{\pgfqpoint{7.365176in}{1.728404in}}%
\pgfpathlineto{\pgfqpoint{7.426911in}{1.732080in}}%
\pgfpathlineto{\pgfqpoint{7.457778in}{1.730413in}}%
\pgfpathlineto{\pgfqpoint{7.473212in}{1.728003in}}%
\pgfpathlineto{\pgfqpoint{7.488645in}{1.724370in}}%
\pgfpathlineto{\pgfqpoint{7.504079in}{1.719437in}}%
\pgfpathlineto{\pgfqpoint{7.519513in}{1.713175in}}%
\pgfpathlineto{\pgfqpoint{7.534946in}{1.705603in}}%
\pgfpathlineto{\pgfqpoint{7.565814in}{1.686843in}}%
\pgfpathlineto{\pgfqpoint{7.596681in}{1.664243in}}%
\pgfpathlineto{\pgfqpoint{7.673849in}{1.603553in}}%
\pgfpathlineto{\pgfqpoint{7.704716in}{1.583961in}}%
\pgfpathlineto{\pgfqpoint{7.720150in}{1.576428in}}%
\pgfpathlineto{\pgfqpoint{7.735584in}{1.570784in}}%
\pgfpathlineto{\pgfqpoint{7.751017in}{1.567278in}}%
\pgfpathlineto{\pgfqpoint{7.766451in}{1.566120in}}%
\pgfpathlineto{\pgfqpoint{7.781884in}{1.567480in}}%
\pgfpathlineto{\pgfqpoint{7.797318in}{1.571474in}}%
\pgfpathlineto{\pgfqpoint{7.812752in}{1.578168in}}%
\pgfpathlineto{\pgfqpoint{7.828185in}{1.587571in}}%
\pgfpathlineto{\pgfqpoint{7.843619in}{1.599637in}}%
\pgfpathlineto{\pgfqpoint{7.859053in}{1.614258in}}%
\pgfpathlineto{\pgfqpoint{7.874486in}{1.631274in}}%
\pgfpathlineto{\pgfqpoint{7.889920in}{1.650469in}}%
\pgfpathlineto{\pgfqpoint{7.920787in}{1.694290in}}%
\pgfpathlineto{\pgfqpoint{7.951654in}{1.743103in}}%
\pgfpathlineto{\pgfqpoint{8.013389in}{1.843032in}}%
\pgfpathlineto{\pgfqpoint{8.044256in}{1.887516in}}%
\pgfpathlineto{\pgfqpoint{8.059690in}{1.907032in}}%
\pgfpathlineto{\pgfqpoint{8.075123in}{1.924291in}}%
\pgfpathlineto{\pgfqpoint{8.090557in}{1.939016in}}%
\pgfpathlineto{\pgfqpoint{8.105991in}{1.950974in}}%
\pgfpathlineto{\pgfqpoint{8.121424in}{1.959988in}}%
\pgfpathlineto{\pgfqpoint{8.136858in}{1.965934in}}%
\pgfpathlineto{\pgfqpoint{8.152292in}{1.968742in}}%
\pgfpathlineto{\pgfqpoint{8.167725in}{1.968399in}}%
\pgfpathlineto{\pgfqpoint{8.183159in}{1.964944in}}%
\pgfpathlineto{\pgfqpoint{8.198593in}{1.958469in}}%
\pgfpathlineto{\pgfqpoint{8.214026in}{1.949111in}}%
\pgfpathlineto{\pgfqpoint{8.229460in}{1.937047in}}%
\pgfpathlineto{\pgfqpoint{8.244893in}{1.922493in}}%
\pgfpathlineto{\pgfqpoint{8.260327in}{1.905690in}}%
\pgfpathlineto{\pgfqpoint{8.291194in}{1.866412in}}%
\pgfpathlineto{\pgfqpoint{8.322062in}{1.821467in}}%
\pgfpathlineto{\pgfqpoint{8.368363in}{1.748351in}}%
\pgfpathlineto{\pgfqpoint{8.430097in}{1.650536in}}%
\pgfpathlineto{\pgfqpoint{8.460964in}{1.604889in}}%
\pgfpathlineto{\pgfqpoint{8.491832in}{1.562800in}}%
\pgfpathlineto{\pgfqpoint{8.522699in}{1.524969in}}%
\pgfpathlineto{\pgfqpoint{8.553566in}{1.491932in}}%
\pgfpathlineto{\pgfqpoint{8.584433in}{1.464161in}}%
\pgfpathlineto{\pgfqpoint{8.599867in}{1.452395in}}%
\pgfpathlineto{\pgfqpoint{8.615301in}{1.442123in}}%
\pgfpathlineto{\pgfqpoint{8.630734in}{1.433409in}}%
\pgfpathlineto{\pgfqpoint{8.646168in}{1.426316in}}%
\pgfpathlineto{\pgfqpoint{8.661602in}{1.420907in}}%
\pgfpathlineto{\pgfqpoint{8.677035in}{1.417242in}}%
\pgfpathlineto{\pgfqpoint{8.692469in}{1.415374in}}%
\pgfpathlineto{\pgfqpoint{8.707903in}{1.415347in}}%
\pgfpathlineto{\pgfqpoint{8.723336in}{1.417193in}}%
\pgfpathlineto{\pgfqpoint{8.738770in}{1.420928in}}%
\pgfpathlineto{\pgfqpoint{8.754203in}{1.426548in}}%
\pgfpathlineto{\pgfqpoint{8.769637in}{1.434028in}}%
\pgfpathlineto{\pgfqpoint{8.785071in}{1.443321in}}%
\pgfpathlineto{\pgfqpoint{8.800504in}{1.454352in}}%
\pgfpathlineto{\pgfqpoint{8.815938in}{1.467021in}}%
\pgfpathlineto{\pgfqpoint{8.846805in}{1.496750in}}%
\pgfpathlineto{\pgfqpoint{8.877673in}{1.531225in}}%
\pgfpathlineto{\pgfqpoint{8.923973in}{1.588298in}}%
\pgfpathlineto{\pgfqpoint{8.985708in}{1.665006in}}%
\pgfpathlineto{\pgfqpoint{9.016575in}{1.699840in}}%
\pgfpathlineto{\pgfqpoint{9.047443in}{1.730585in}}%
\pgfpathlineto{\pgfqpoint{9.078310in}{1.756451in}}%
\pgfpathlineto{\pgfqpoint{9.109177in}{1.777119in}}%
\pgfpathlineto{\pgfqpoint{9.140044in}{1.792733in}}%
\pgfpathlineto{\pgfqpoint{9.170912in}{1.803829in}}%
\pgfpathlineto{\pgfqpoint{9.201779in}{1.811245in}}%
\pgfpathlineto{\pgfqpoint{9.232646in}{1.815993in}}%
\pgfpathlineto{\pgfqpoint{9.278947in}{1.820435in}}%
\pgfpathlineto{\pgfqpoint{9.340682in}{1.826232in}}%
\pgfpathlineto{\pgfqpoint{9.386982in}{1.833270in}}%
\pgfpathlineto{\pgfqpoint{9.433283in}{1.843538in}}%
\pgfpathlineto{\pgfqpoint{9.479584in}{1.856649in}}%
\pgfpathlineto{\pgfqpoint{9.618487in}{1.899186in}}%
\pgfpathlineto{\pgfqpoint{9.649354in}{1.906203in}}%
\pgfpathlineto{\pgfqpoint{9.680222in}{1.911493in}}%
\pgfpathlineto{\pgfqpoint{9.711089in}{1.914849in}}%
\pgfpathlineto{\pgfqpoint{9.741956in}{1.916192in}}%
\pgfpathlineto{\pgfqpoint{9.741956in}{1.916192in}}%
\pgfusepath{stroke}%
\end{pgfscope}%
\begin{pgfscope}%
\pgfpathrectangle{\pgfqpoint{5.706832in}{0.521603in}}{\pgfqpoint{4.227273in}{2.800000in}} %
\pgfusepath{clip}%
\pgfsetrectcap%
\pgfsetroundjoin%
\pgfsetlinewidth{0.501875pt}%
\definecolor{currentstroke}{rgb}{1.000000,0.682749,0.366979}%
\pgfsetstrokecolor{currentstroke}%
\pgfsetdash{}{0pt}%
\pgfpathmoveto{\pgfqpoint{5.898981in}{2.180923in}}%
\pgfpathlineto{\pgfqpoint{5.914415in}{2.176522in}}%
\pgfpathlineto{\pgfqpoint{5.929848in}{2.169357in}}%
\pgfpathlineto{\pgfqpoint{5.945282in}{2.159394in}}%
\pgfpathlineto{\pgfqpoint{5.960715in}{2.146642in}}%
\pgfpathlineto{\pgfqpoint{5.976149in}{2.131163in}}%
\pgfpathlineto{\pgfqpoint{5.991583in}{2.113067in}}%
\pgfpathlineto{\pgfqpoint{6.007016in}{2.092512in}}%
\pgfpathlineto{\pgfqpoint{6.022450in}{2.069709in}}%
\pgfpathlineto{\pgfqpoint{6.053317in}{2.018413in}}%
\pgfpathlineto{\pgfqpoint{6.084185in}{1.961713in}}%
\pgfpathlineto{\pgfqpoint{6.145919in}{1.844859in}}%
\pgfpathlineto{\pgfqpoint{6.176786in}{1.791854in}}%
\pgfpathlineto{\pgfqpoint{6.192220in}{1.768305in}}%
\pgfpathlineto{\pgfqpoint{6.207654in}{1.747308in}}%
\pgfpathlineto{\pgfqpoint{6.223087in}{1.729271in}}%
\pgfpathlineto{\pgfqpoint{6.238521in}{1.714572in}}%
\pgfpathlineto{\pgfqpoint{6.253955in}{1.703549in}}%
\pgfpathlineto{\pgfqpoint{6.269388in}{1.696498in}}%
\pgfpathlineto{\pgfqpoint{6.284822in}{1.693664in}}%
\pgfpathlineto{\pgfqpoint{6.300255in}{1.695237in}}%
\pgfpathlineto{\pgfqpoint{6.315689in}{1.701349in}}%
\pgfpathlineto{\pgfqpoint{6.331123in}{1.712070in}}%
\pgfpathlineto{\pgfqpoint{6.346556in}{1.727406in}}%
\pgfpathlineto{\pgfqpoint{6.361990in}{1.747295in}}%
\pgfpathlineto{\pgfqpoint{6.377424in}{1.771612in}}%
\pgfpathlineto{\pgfqpoint{6.392857in}{1.800165in}}%
\pgfpathlineto{\pgfqpoint{6.408291in}{1.832700in}}%
\pgfpathlineto{\pgfqpoint{6.423724in}{1.868901in}}%
\pgfpathlineto{\pgfqpoint{6.454592in}{1.950773in}}%
\pgfpathlineto{\pgfqpoint{6.485459in}{2.042253in}}%
\pgfpathlineto{\pgfqpoint{6.593494in}{2.376263in}}%
\pgfpathlineto{\pgfqpoint{6.624362in}{2.457561in}}%
\pgfpathlineto{\pgfqpoint{6.639795in}{2.493408in}}%
\pgfpathlineto{\pgfqpoint{6.655229in}{2.525543in}}%
\pgfpathlineto{\pgfqpoint{6.670663in}{2.553648in}}%
\pgfpathlineto{\pgfqpoint{6.686096in}{2.577464in}}%
\pgfpathlineto{\pgfqpoint{6.701530in}{2.596786in}}%
\pgfpathlineto{\pgfqpoint{6.716964in}{2.611472in}}%
\pgfpathlineto{\pgfqpoint{6.732397in}{2.621438in}}%
\pgfpathlineto{\pgfqpoint{6.747831in}{2.626660in}}%
\pgfpathlineto{\pgfqpoint{6.763264in}{2.627167in}}%
\pgfpathlineto{\pgfqpoint{6.778698in}{2.623045in}}%
\pgfpathlineto{\pgfqpoint{6.794132in}{2.614429in}}%
\pgfpathlineto{\pgfqpoint{6.809565in}{2.601497in}}%
\pgfpathlineto{\pgfqpoint{6.824999in}{2.584472in}}%
\pgfpathlineto{\pgfqpoint{6.840433in}{2.563610in}}%
\pgfpathlineto{\pgfqpoint{6.855866in}{2.539201in}}%
\pgfpathlineto{\pgfqpoint{6.871300in}{2.511558in}}%
\pgfpathlineto{\pgfqpoint{6.886734in}{2.481017in}}%
\pgfpathlineto{\pgfqpoint{6.917601in}{2.412655in}}%
\pgfpathlineto{\pgfqpoint{6.948468in}{2.337034in}}%
\pgfpathlineto{\pgfqpoint{7.010203in}{2.175802in}}%
\pgfpathlineto{\pgfqpoint{7.056504in}{2.057093in}}%
\pgfpathlineto{\pgfqpoint{7.087371in}{1.983617in}}%
\pgfpathlineto{\pgfqpoint{7.118238in}{1.916998in}}%
\pgfpathlineto{\pgfqpoint{7.149105in}{1.858749in}}%
\pgfpathlineto{\pgfqpoint{7.164539in}{1.833102in}}%
\pgfpathlineto{\pgfqpoint{7.179973in}{1.809887in}}%
\pgfpathlineto{\pgfqpoint{7.195406in}{1.789146in}}%
\pgfpathlineto{\pgfqpoint{7.210840in}{1.770884in}}%
\pgfpathlineto{\pgfqpoint{7.226274in}{1.755070in}}%
\pgfpathlineto{\pgfqpoint{7.241707in}{1.741637in}}%
\pgfpathlineto{\pgfqpoint{7.257141in}{1.730483in}}%
\pgfpathlineto{\pgfqpoint{7.272574in}{1.721471in}}%
\pgfpathlineto{\pgfqpoint{7.288008in}{1.714433in}}%
\pgfpathlineto{\pgfqpoint{7.303442in}{1.709175in}}%
\pgfpathlineto{\pgfqpoint{7.318875in}{1.705474in}}%
\pgfpathlineto{\pgfqpoint{7.349743in}{1.701768in}}%
\pgfpathlineto{\pgfqpoint{7.380610in}{1.701235in}}%
\pgfpathlineto{\pgfqpoint{7.442344in}{1.701281in}}%
\pgfpathlineto{\pgfqpoint{7.473212in}{1.698212in}}%
\pgfpathlineto{\pgfqpoint{7.504079in}{1.691394in}}%
\pgfpathlineto{\pgfqpoint{7.534946in}{1.680340in}}%
\pgfpathlineto{\pgfqpoint{7.565814in}{1.665281in}}%
\pgfpathlineto{\pgfqpoint{7.596681in}{1.647159in}}%
\pgfpathlineto{\pgfqpoint{7.658415in}{1.608500in}}%
\pgfpathlineto{\pgfqpoint{7.689283in}{1.592335in}}%
\pgfpathlineto{\pgfqpoint{7.704716in}{1.586071in}}%
\pgfpathlineto{\pgfqpoint{7.720150in}{1.581392in}}%
\pgfpathlineto{\pgfqpoint{7.735584in}{1.578552in}}%
\pgfpathlineto{\pgfqpoint{7.751017in}{1.577773in}}%
\pgfpathlineto{\pgfqpoint{7.766451in}{1.579241in}}%
\pgfpathlineto{\pgfqpoint{7.781884in}{1.583100in}}%
\pgfpathlineto{\pgfqpoint{7.797318in}{1.589442in}}%
\pgfpathlineto{\pgfqpoint{7.812752in}{1.598313in}}%
\pgfpathlineto{\pgfqpoint{7.828185in}{1.609700in}}%
\pgfpathlineto{\pgfqpoint{7.843619in}{1.623538in}}%
\pgfpathlineto{\pgfqpoint{7.859053in}{1.639706in}}%
\pgfpathlineto{\pgfqpoint{7.874486in}{1.658029in}}%
\pgfpathlineto{\pgfqpoint{7.905353in}{1.700197in}}%
\pgfpathlineto{\pgfqpoint{7.936221in}{1.747726in}}%
\pgfpathlineto{\pgfqpoint{8.013389in}{1.870587in}}%
\pgfpathlineto{\pgfqpoint{8.044256in}{1.913081in}}%
\pgfpathlineto{\pgfqpoint{8.059690in}{1.931443in}}%
\pgfpathlineto{\pgfqpoint{8.075123in}{1.947480in}}%
\pgfpathlineto{\pgfqpoint{8.090557in}{1.960946in}}%
\pgfpathlineto{\pgfqpoint{8.105991in}{1.971636in}}%
\pgfpathlineto{\pgfqpoint{8.121424in}{1.979400in}}%
\pgfpathlineto{\pgfqpoint{8.136858in}{1.984137in}}%
\pgfpathlineto{\pgfqpoint{8.152292in}{1.985801in}}%
\pgfpathlineto{\pgfqpoint{8.167725in}{1.984397in}}%
\pgfpathlineto{\pgfqpoint{8.183159in}{1.979978in}}%
\pgfpathlineto{\pgfqpoint{8.198593in}{1.972647in}}%
\pgfpathlineto{\pgfqpoint{8.214026in}{1.962547in}}%
\pgfpathlineto{\pgfqpoint{8.229460in}{1.949860in}}%
\pgfpathlineto{\pgfqpoint{8.244893in}{1.934799in}}%
\pgfpathlineto{\pgfqpoint{8.260327in}{1.917603in}}%
\pgfpathlineto{\pgfqpoint{8.291194in}{1.877854in}}%
\pgfpathlineto{\pgfqpoint{8.322062in}{1.832790in}}%
\pgfpathlineto{\pgfqpoint{8.368363in}{1.759966in}}%
\pgfpathlineto{\pgfqpoint{8.430097in}{1.662959in}}%
\pgfpathlineto{\pgfqpoint{8.460964in}{1.617735in}}%
\pgfpathlineto{\pgfqpoint{8.491832in}{1.576007in}}%
\pgfpathlineto{\pgfqpoint{8.522699in}{1.538419in}}%
\pgfpathlineto{\pgfqpoint{8.553566in}{1.505449in}}%
\pgfpathlineto{\pgfqpoint{8.584433in}{1.477504in}}%
\pgfpathlineto{\pgfqpoint{8.599867in}{1.465540in}}%
\pgfpathlineto{\pgfqpoint{8.615301in}{1.454987in}}%
\pgfpathlineto{\pgfqpoint{8.630734in}{1.445901in}}%
\pgfpathlineto{\pgfqpoint{8.646168in}{1.438344in}}%
\pgfpathlineto{\pgfqpoint{8.661602in}{1.432375in}}%
\pgfpathlineto{\pgfqpoint{8.677035in}{1.428055in}}%
\pgfpathlineto{\pgfqpoint{8.692469in}{1.425445in}}%
\pgfpathlineto{\pgfqpoint{8.707903in}{1.424596in}}%
\pgfpathlineto{\pgfqpoint{8.723336in}{1.425553in}}%
\pgfpathlineto{\pgfqpoint{8.738770in}{1.428349in}}%
\pgfpathlineto{\pgfqpoint{8.754203in}{1.433004in}}%
\pgfpathlineto{\pgfqpoint{8.769637in}{1.439517in}}%
\pgfpathlineto{\pgfqpoint{8.785071in}{1.447869in}}%
\pgfpathlineto{\pgfqpoint{8.800504in}{1.458016in}}%
\pgfpathlineto{\pgfqpoint{8.815938in}{1.469891in}}%
\pgfpathlineto{\pgfqpoint{8.831372in}{1.483402in}}%
\pgfpathlineto{\pgfqpoint{8.862239in}{1.514830in}}%
\pgfpathlineto{\pgfqpoint{8.893106in}{1.551061in}}%
\pgfpathlineto{\pgfqpoint{8.939407in}{1.610908in}}%
\pgfpathlineto{\pgfqpoint{9.001142in}{1.691325in}}%
\pgfpathlineto{\pgfqpoint{9.032009in}{1.727743in}}%
\pgfpathlineto{\pgfqpoint{9.062876in}{1.759644in}}%
\pgfpathlineto{\pgfqpoint{9.093743in}{1.786028in}}%
\pgfpathlineto{\pgfqpoint{9.109177in}{1.796972in}}%
\pgfpathlineto{\pgfqpoint{9.124611in}{1.806387in}}%
\pgfpathlineto{\pgfqpoint{9.140044in}{1.814287in}}%
\pgfpathlineto{\pgfqpoint{9.155478in}{1.820722in}}%
\pgfpathlineto{\pgfqpoint{9.170912in}{1.825765in}}%
\pgfpathlineto{\pgfqpoint{9.186345in}{1.829518in}}%
\pgfpathlineto{\pgfqpoint{9.217212in}{1.833663in}}%
\pgfpathlineto{\pgfqpoint{9.248080in}{1.834337in}}%
\pgfpathlineto{\pgfqpoint{9.294381in}{1.831778in}}%
\pgfpathlineto{\pgfqpoint{9.340682in}{1.828894in}}%
\pgfpathlineto{\pgfqpoint{9.371549in}{1.828648in}}%
\pgfpathlineto{\pgfqpoint{9.402416in}{1.830644in}}%
\pgfpathlineto{\pgfqpoint{9.433283in}{1.835275in}}%
\pgfpathlineto{\pgfqpoint{9.464151in}{1.842580in}}%
\pgfpathlineto{\pgfqpoint{9.495018in}{1.852274in}}%
\pgfpathlineto{\pgfqpoint{9.541319in}{1.870000in}}%
\pgfpathlineto{\pgfqpoint{9.618487in}{1.901110in}}%
\pgfpathlineto{\pgfqpoint{9.649354in}{1.911511in}}%
\pgfpathlineto{\pgfqpoint{9.680222in}{1.919599in}}%
\pgfpathlineto{\pgfqpoint{9.711089in}{1.924858in}}%
\pgfpathlineto{\pgfqpoint{9.741956in}{1.927014in}}%
\pgfpathlineto{\pgfqpoint{9.741956in}{1.927014in}}%
\pgfusepath{stroke}%
\end{pgfscope}%
\begin{pgfscope}%
\pgfpathrectangle{\pgfqpoint{5.706832in}{0.521603in}}{\pgfqpoint{4.227273in}{2.800000in}} %
\pgfusepath{clip}%
\pgfsetrectcap%
\pgfsetroundjoin%
\pgfsetlinewidth{0.501875pt}%
\definecolor{currentstroke}{rgb}{1.000000,0.587785,0.309017}%
\pgfsetstrokecolor{currentstroke}%
\pgfsetdash{}{0pt}%
\pgfpathmoveto{\pgfqpoint{5.898981in}{2.182610in}}%
\pgfpathlineto{\pgfqpoint{5.914415in}{2.178834in}}%
\pgfpathlineto{\pgfqpoint{5.929848in}{2.172289in}}%
\pgfpathlineto{\pgfqpoint{5.945282in}{2.162940in}}%
\pgfpathlineto{\pgfqpoint{5.960715in}{2.150805in}}%
\pgfpathlineto{\pgfqpoint{5.976149in}{2.135954in}}%
\pgfpathlineto{\pgfqpoint{5.991583in}{2.118504in}}%
\pgfpathlineto{\pgfqpoint{6.007016in}{2.098620in}}%
\pgfpathlineto{\pgfqpoint{6.022450in}{2.076510in}}%
\pgfpathlineto{\pgfqpoint{6.053317in}{2.026653in}}%
\pgfpathlineto{\pgfqpoint{6.084185in}{1.971356in}}%
\pgfpathlineto{\pgfqpoint{6.161353in}{1.829151in}}%
\pgfpathlineto{\pgfqpoint{6.192220in}{1.779600in}}%
\pgfpathlineto{\pgfqpoint{6.207654in}{1.758154in}}%
\pgfpathlineto{\pgfqpoint{6.223087in}{1.739494in}}%
\pgfpathlineto{\pgfqpoint{6.238521in}{1.724020in}}%
\pgfpathlineto{\pgfqpoint{6.253955in}{1.712103in}}%
\pgfpathlineto{\pgfqpoint{6.269388in}{1.704075in}}%
\pgfpathlineto{\pgfqpoint{6.284822in}{1.700225in}}%
\pgfpathlineto{\pgfqpoint{6.300255in}{1.700786in}}%
\pgfpathlineto{\pgfqpoint{6.315689in}{1.705933in}}%
\pgfpathlineto{\pgfqpoint{6.331123in}{1.715771in}}%
\pgfpathlineto{\pgfqpoint{6.346556in}{1.730339in}}%
\pgfpathlineto{\pgfqpoint{6.361990in}{1.749596in}}%
\pgfpathlineto{\pgfqpoint{6.377424in}{1.773431in}}%
\pgfpathlineto{\pgfqpoint{6.392857in}{1.801653in}}%
\pgfpathlineto{\pgfqpoint{6.408291in}{1.834001in}}%
\pgfpathlineto{\pgfqpoint{6.423724in}{1.870146in}}%
\pgfpathlineto{\pgfqpoint{6.454592in}{1.952200in}}%
\pgfpathlineto{\pgfqpoint{6.485459in}{2.044072in}}%
\pgfpathlineto{\pgfqpoint{6.593494in}{2.378863in}}%
\pgfpathlineto{\pgfqpoint{6.624362in}{2.460159in}}%
\pgfpathlineto{\pgfqpoint{6.639795in}{2.496024in}}%
\pgfpathlineto{\pgfqpoint{6.655229in}{2.528206in}}%
\pgfpathlineto{\pgfqpoint{6.670663in}{2.556395in}}%
\pgfpathlineto{\pgfqpoint{6.686096in}{2.580334in}}%
\pgfpathlineto{\pgfqpoint{6.701530in}{2.599817in}}%
\pgfpathlineto{\pgfqpoint{6.716964in}{2.614694in}}%
\pgfpathlineto{\pgfqpoint{6.732397in}{2.624872in}}%
\pgfpathlineto{\pgfqpoint{6.747831in}{2.630309in}}%
\pgfpathlineto{\pgfqpoint{6.763264in}{2.631021in}}%
\pgfpathlineto{\pgfqpoint{6.778698in}{2.627079in}}%
\pgfpathlineto{\pgfqpoint{6.794132in}{2.618603in}}%
\pgfpathlineto{\pgfqpoint{6.809565in}{2.605764in}}%
\pgfpathlineto{\pgfqpoint{6.824999in}{2.588781in}}%
\pgfpathlineto{\pgfqpoint{6.840433in}{2.567912in}}%
\pgfpathlineto{\pgfqpoint{6.855866in}{2.543454in}}%
\pgfpathlineto{\pgfqpoint{6.871300in}{2.515732in}}%
\pgfpathlineto{\pgfqpoint{6.886734in}{2.485101in}}%
\pgfpathlineto{\pgfqpoint{6.917601in}{2.416603in}}%
\pgfpathlineto{\pgfqpoint{6.948468in}{2.341031in}}%
\pgfpathlineto{\pgfqpoint{7.071937in}{2.026035in}}%
\pgfpathlineto{\pgfqpoint{7.102804in}{1.956196in}}%
\pgfpathlineto{\pgfqpoint{7.133672in}{1.893551in}}%
\pgfpathlineto{\pgfqpoint{7.164539in}{1.839291in}}%
\pgfpathlineto{\pgfqpoint{7.179973in}{1.815574in}}%
\pgfpathlineto{\pgfqpoint{7.195406in}{1.794223in}}%
\pgfpathlineto{\pgfqpoint{7.210840in}{1.775270in}}%
\pgfpathlineto{\pgfqpoint{7.226274in}{1.758712in}}%
\pgfpathlineto{\pgfqpoint{7.241707in}{1.744514in}}%
\pgfpathlineto{\pgfqpoint{7.257141in}{1.732605in}}%
\pgfpathlineto{\pgfqpoint{7.272574in}{1.722875in}}%
\pgfpathlineto{\pgfqpoint{7.288008in}{1.715179in}}%
\pgfpathlineto{\pgfqpoint{7.303442in}{1.709337in}}%
\pgfpathlineto{\pgfqpoint{7.318875in}{1.705136in}}%
\pgfpathlineto{\pgfqpoint{7.334309in}{1.702333in}}%
\pgfpathlineto{\pgfqpoint{7.365176in}{1.699857in}}%
\pgfpathlineto{\pgfqpoint{7.457778in}{1.697525in}}%
\pgfpathlineto{\pgfqpoint{7.488645in}{1.692247in}}%
\pgfpathlineto{\pgfqpoint{7.519513in}{1.683004in}}%
\pgfpathlineto{\pgfqpoint{7.550380in}{1.669868in}}%
\pgfpathlineto{\pgfqpoint{7.581247in}{1.653643in}}%
\pgfpathlineto{\pgfqpoint{7.658415in}{1.609770in}}%
\pgfpathlineto{\pgfqpoint{7.689283in}{1.596065in}}%
\pgfpathlineto{\pgfqpoint{7.704716in}{1.591060in}}%
\pgfpathlineto{\pgfqpoint{7.720150in}{1.587600in}}%
\pgfpathlineto{\pgfqpoint{7.735584in}{1.585900in}}%
\pgfpathlineto{\pgfqpoint{7.751017in}{1.586148in}}%
\pgfpathlineto{\pgfqpoint{7.766451in}{1.588503in}}%
\pgfpathlineto{\pgfqpoint{7.781884in}{1.593091in}}%
\pgfpathlineto{\pgfqpoint{7.797318in}{1.600003in}}%
\pgfpathlineto{\pgfqpoint{7.812752in}{1.609286in}}%
\pgfpathlineto{\pgfqpoint{7.828185in}{1.620944in}}%
\pgfpathlineto{\pgfqpoint{7.843619in}{1.634932in}}%
\pgfpathlineto{\pgfqpoint{7.859053in}{1.651155in}}%
\pgfpathlineto{\pgfqpoint{7.874486in}{1.669466in}}%
\pgfpathlineto{\pgfqpoint{7.905353in}{1.711510in}}%
\pgfpathlineto{\pgfqpoint{7.936221in}{1.758880in}}%
\pgfpathlineto{\pgfqpoint{7.997955in}{1.857690in}}%
\pgfpathlineto{\pgfqpoint{8.028823in}{1.902359in}}%
\pgfpathlineto{\pgfqpoint{8.044256in}{1.922054in}}%
\pgfpathlineto{\pgfqpoint{8.059690in}{1.939506in}}%
\pgfpathlineto{\pgfqpoint{8.075123in}{1.954408in}}%
\pgfpathlineto{\pgfqpoint{8.090557in}{1.966510in}}%
\pgfpathlineto{\pgfqpoint{8.105991in}{1.975624in}}%
\pgfpathlineto{\pgfqpoint{8.121424in}{1.981629in}}%
\pgfpathlineto{\pgfqpoint{8.136858in}{1.984469in}}%
\pgfpathlineto{\pgfqpoint{8.152292in}{1.984155in}}%
\pgfpathlineto{\pgfqpoint{8.167725in}{1.980760in}}%
\pgfpathlineto{\pgfqpoint{8.183159in}{1.974411in}}%
\pgfpathlineto{\pgfqpoint{8.198593in}{1.965286in}}%
\pgfpathlineto{\pgfqpoint{8.214026in}{1.953601in}}%
\pgfpathlineto{\pgfqpoint{8.229460in}{1.939602in}}%
\pgfpathlineto{\pgfqpoint{8.244893in}{1.923554in}}%
\pgfpathlineto{\pgfqpoint{8.275761in}{1.886416in}}%
\pgfpathlineto{\pgfqpoint{8.306628in}{1.844367in}}%
\pgfpathlineto{\pgfqpoint{8.352929in}{1.776361in}}%
\pgfpathlineto{\pgfqpoint{8.430097in}{1.662301in}}%
\pgfpathlineto{\pgfqpoint{8.460964in}{1.619601in}}%
\pgfpathlineto{\pgfqpoint{8.491832in}{1.579837in}}%
\pgfpathlineto{\pgfqpoint{8.522699in}{1.543751in}}%
\pgfpathlineto{\pgfqpoint{8.553566in}{1.512036in}}%
\pgfpathlineto{\pgfqpoint{8.584433in}{1.485314in}}%
\pgfpathlineto{\pgfqpoint{8.599867in}{1.473996in}}%
\pgfpathlineto{\pgfqpoint{8.615301in}{1.464117in}}%
\pgfpathlineto{\pgfqpoint{8.630734in}{1.455726in}}%
\pgfpathlineto{\pgfqpoint{8.646168in}{1.448865in}}%
\pgfpathlineto{\pgfqpoint{8.661602in}{1.443572in}}%
\pgfpathlineto{\pgfqpoint{8.677035in}{1.439876in}}%
\pgfpathlineto{\pgfqpoint{8.692469in}{1.437807in}}%
\pgfpathlineto{\pgfqpoint{8.707903in}{1.437386in}}%
\pgfpathlineto{\pgfqpoint{8.723336in}{1.438634in}}%
\pgfpathlineto{\pgfqpoint{8.738770in}{1.441566in}}%
\pgfpathlineto{\pgfqpoint{8.754203in}{1.446194in}}%
\pgfpathlineto{\pgfqpoint{8.769637in}{1.452522in}}%
\pgfpathlineto{\pgfqpoint{8.785071in}{1.460545in}}%
\pgfpathlineto{\pgfqpoint{8.800504in}{1.470245in}}%
\pgfpathlineto{\pgfqpoint{8.815938in}{1.481589in}}%
\pgfpathlineto{\pgfqpoint{8.831372in}{1.494527in}}%
\pgfpathlineto{\pgfqpoint{8.862239in}{1.524855in}}%
\pgfpathlineto{\pgfqpoint{8.893106in}{1.560313in}}%
\pgfpathlineto{\pgfqpoint{8.923973in}{1.599530in}}%
\pgfpathlineto{\pgfqpoint{9.016575in}{1.720625in}}%
\pgfpathlineto{\pgfqpoint{9.047443in}{1.755198in}}%
\pgfpathlineto{\pgfqpoint{9.062876in}{1.770382in}}%
\pgfpathlineto{\pgfqpoint{9.078310in}{1.783964in}}%
\pgfpathlineto{\pgfqpoint{9.093743in}{1.795848in}}%
\pgfpathlineto{\pgfqpoint{9.109177in}{1.805981in}}%
\pgfpathlineto{\pgfqpoint{9.124611in}{1.814357in}}%
\pgfpathlineto{\pgfqpoint{9.140044in}{1.821014in}}%
\pgfpathlineto{\pgfqpoint{9.155478in}{1.826032in}}%
\pgfpathlineto{\pgfqpoint{9.170912in}{1.829524in}}%
\pgfpathlineto{\pgfqpoint{9.186345in}{1.831636in}}%
\pgfpathlineto{\pgfqpoint{9.217212in}{1.832411in}}%
\pgfpathlineto{\pgfqpoint{9.248080in}{1.829872in}}%
\pgfpathlineto{\pgfqpoint{9.356115in}{1.816329in}}%
\pgfpathlineto{\pgfqpoint{9.386982in}{1.815887in}}%
\pgfpathlineto{\pgfqpoint{9.417850in}{1.818249in}}%
\pgfpathlineto{\pgfqpoint{9.448717in}{1.823651in}}%
\pgfpathlineto{\pgfqpoint{9.479584in}{1.832060in}}%
\pgfpathlineto{\pgfqpoint{9.510452in}{1.843178in}}%
\pgfpathlineto{\pgfqpoint{9.541319in}{1.856453in}}%
\pgfpathlineto{\pgfqpoint{9.664788in}{1.913618in}}%
\pgfpathlineto{\pgfqpoint{9.695655in}{1.924210in}}%
\pgfpathlineto{\pgfqpoint{9.726522in}{1.931932in}}%
\pgfpathlineto{\pgfqpoint{9.741956in}{1.934639in}}%
\pgfpathlineto{\pgfqpoint{9.741956in}{1.934639in}}%
\pgfusepath{stroke}%
\end{pgfscope}%
\begin{pgfscope}%
\pgfpathrectangle{\pgfqpoint{5.706832in}{0.521603in}}{\pgfqpoint{4.227273in}{2.800000in}} %
\pgfusepath{clip}%
\pgfsetrectcap%
\pgfsetroundjoin%
\pgfsetlinewidth{0.501875pt}%
\definecolor{currentstroke}{rgb}{1.000000,0.473094,0.243914}%
\pgfsetstrokecolor{currentstroke}%
\pgfsetdash{}{0pt}%
\pgfpathmoveto{\pgfqpoint{5.898981in}{2.201692in}}%
\pgfpathlineto{\pgfqpoint{5.914415in}{2.198786in}}%
\pgfpathlineto{\pgfqpoint{5.929848in}{2.193170in}}%
\pgfpathlineto{\pgfqpoint{5.945282in}{2.184804in}}%
\pgfpathlineto{\pgfqpoint{5.960715in}{2.173695in}}%
\pgfpathlineto{\pgfqpoint{5.976149in}{2.159891in}}%
\pgfpathlineto{\pgfqpoint{5.991583in}{2.143489in}}%
\pgfpathlineto{\pgfqpoint{6.007016in}{2.124630in}}%
\pgfpathlineto{\pgfqpoint{6.022450in}{2.103497in}}%
\pgfpathlineto{\pgfqpoint{6.053317in}{2.055361in}}%
\pgfpathlineto{\pgfqpoint{6.084185in}{2.001365in}}%
\pgfpathlineto{\pgfqpoint{6.161353in}{1.860308in}}%
\pgfpathlineto{\pgfqpoint{6.192220in}{1.810452in}}%
\pgfpathlineto{\pgfqpoint{6.207654in}{1.788710in}}%
\pgfpathlineto{\pgfqpoint{6.223087in}{1.769661in}}%
\pgfpathlineto{\pgfqpoint{6.238521in}{1.753705in}}%
\pgfpathlineto{\pgfqpoint{6.253955in}{1.741208in}}%
\pgfpathlineto{\pgfqpoint{6.269388in}{1.732498in}}%
\pgfpathlineto{\pgfqpoint{6.284822in}{1.727856in}}%
\pgfpathlineto{\pgfqpoint{6.300255in}{1.727513in}}%
\pgfpathlineto{\pgfqpoint{6.315689in}{1.731638in}}%
\pgfpathlineto{\pgfqpoint{6.331123in}{1.740338in}}%
\pgfpathlineto{\pgfqpoint{6.346556in}{1.753654in}}%
\pgfpathlineto{\pgfqpoint{6.361990in}{1.771553in}}%
\pgfpathlineto{\pgfqpoint{6.377424in}{1.793934in}}%
\pgfpathlineto{\pgfqpoint{6.392857in}{1.820625in}}%
\pgfpathlineto{\pgfqpoint{6.408291in}{1.851384in}}%
\pgfpathlineto{\pgfqpoint{6.423724in}{1.885900in}}%
\pgfpathlineto{\pgfqpoint{6.439158in}{1.923806in}}%
\pgfpathlineto{\pgfqpoint{6.470025in}{2.008027in}}%
\pgfpathlineto{\pgfqpoint{6.500893in}{2.100099in}}%
\pgfpathlineto{\pgfqpoint{6.578061in}{2.334960in}}%
\pgfpathlineto{\pgfqpoint{6.608928in}{2.418838in}}%
\pgfpathlineto{\pgfqpoint{6.624362in}{2.456557in}}%
\pgfpathlineto{\pgfqpoint{6.639795in}{2.490893in}}%
\pgfpathlineto{\pgfqpoint{6.655229in}{2.521481in}}%
\pgfpathlineto{\pgfqpoint{6.670663in}{2.548017in}}%
\pgfpathlineto{\pgfqpoint{6.686096in}{2.570254in}}%
\pgfpathlineto{\pgfqpoint{6.701530in}{2.588008in}}%
\pgfpathlineto{\pgfqpoint{6.716964in}{2.601158in}}%
\pgfpathlineto{\pgfqpoint{6.732397in}{2.609643in}}%
\pgfpathlineto{\pgfqpoint{6.747831in}{2.613466in}}%
\pgfpathlineto{\pgfqpoint{6.763264in}{2.612682in}}%
\pgfpathlineto{\pgfqpoint{6.778698in}{2.607404in}}%
\pgfpathlineto{\pgfqpoint{6.794132in}{2.597793in}}%
\pgfpathlineto{\pgfqpoint{6.809565in}{2.584052in}}%
\pgfpathlineto{\pgfqpoint{6.824999in}{2.566426in}}%
\pgfpathlineto{\pgfqpoint{6.840433in}{2.545188in}}%
\pgfpathlineto{\pgfqpoint{6.855866in}{2.520642in}}%
\pgfpathlineto{\pgfqpoint{6.871300in}{2.493109in}}%
\pgfpathlineto{\pgfqpoint{6.902167in}{2.430436in}}%
\pgfpathlineto{\pgfqpoint{6.933034in}{2.359941in}}%
\pgfpathlineto{\pgfqpoint{6.979335in}{2.245564in}}%
\pgfpathlineto{\pgfqpoint{7.041070in}{2.090427in}}%
\pgfpathlineto{\pgfqpoint{7.071937in}{2.016874in}}%
\pgfpathlineto{\pgfqpoint{7.102804in}{1.948613in}}%
\pgfpathlineto{\pgfqpoint{7.133672in}{1.887274in}}%
\pgfpathlineto{\pgfqpoint{7.164539in}{1.834177in}}%
\pgfpathlineto{\pgfqpoint{7.179973in}{1.811030in}}%
\pgfpathlineto{\pgfqpoint{7.195406in}{1.790263in}}%
\pgfpathlineto{\pgfqpoint{7.210840in}{1.771923in}}%
\pgfpathlineto{\pgfqpoint{7.226274in}{1.756021in}}%
\pgfpathlineto{\pgfqpoint{7.241707in}{1.742537in}}%
\pgfpathlineto{\pgfqpoint{7.257141in}{1.731411in}}%
\pgfpathlineto{\pgfqpoint{7.272574in}{1.722546in}}%
\pgfpathlineto{\pgfqpoint{7.288008in}{1.715811in}}%
\pgfpathlineto{\pgfqpoint{7.303442in}{1.711036in}}%
\pgfpathlineto{\pgfqpoint{7.318875in}{1.708021in}}%
\pgfpathlineto{\pgfqpoint{7.334309in}{1.706533in}}%
\pgfpathlineto{\pgfqpoint{7.365176in}{1.707100in}}%
\pgfpathlineto{\pgfqpoint{7.411477in}{1.712527in}}%
\pgfpathlineto{\pgfqpoint{7.442344in}{1.715851in}}%
\pgfpathlineto{\pgfqpoint{7.473212in}{1.716561in}}%
\pgfpathlineto{\pgfqpoint{7.488645in}{1.715444in}}%
\pgfpathlineto{\pgfqpoint{7.504079in}{1.713162in}}%
\pgfpathlineto{\pgfqpoint{7.519513in}{1.709635in}}%
\pgfpathlineto{\pgfqpoint{7.534946in}{1.704835in}}%
\pgfpathlineto{\pgfqpoint{7.565814in}{1.691546in}}%
\pgfpathlineto{\pgfqpoint{7.596681in}{1.674059in}}%
\pgfpathlineto{\pgfqpoint{7.642982in}{1.643398in}}%
\pgfpathlineto{\pgfqpoint{7.673849in}{1.623187in}}%
\pgfpathlineto{\pgfqpoint{7.704716in}{1.606093in}}%
\pgfpathlineto{\pgfqpoint{7.720150in}{1.599510in}}%
\pgfpathlineto{\pgfqpoint{7.735584in}{1.594632in}}%
\pgfpathlineto{\pgfqpoint{7.751017in}{1.591724in}}%
\pgfpathlineto{\pgfqpoint{7.766451in}{1.591014in}}%
\pgfpathlineto{\pgfqpoint{7.781884in}{1.592688in}}%
\pgfpathlineto{\pgfqpoint{7.797318in}{1.596881in}}%
\pgfpathlineto{\pgfqpoint{7.812752in}{1.603673in}}%
\pgfpathlineto{\pgfqpoint{7.828185in}{1.613085in}}%
\pgfpathlineto{\pgfqpoint{7.843619in}{1.625075in}}%
\pgfpathlineto{\pgfqpoint{7.859053in}{1.639541in}}%
\pgfpathlineto{\pgfqpoint{7.874486in}{1.656315in}}%
\pgfpathlineto{\pgfqpoint{7.889920in}{1.675169in}}%
\pgfpathlineto{\pgfqpoint{7.920787in}{1.717942in}}%
\pgfpathlineto{\pgfqpoint{7.967088in}{1.789181in}}%
\pgfpathlineto{\pgfqpoint{7.997955in}{1.836412in}}%
\pgfpathlineto{\pgfqpoint{8.028823in}{1.879320in}}%
\pgfpathlineto{\pgfqpoint{8.044256in}{1.898132in}}%
\pgfpathlineto{\pgfqpoint{8.059690in}{1.914710in}}%
\pgfpathlineto{\pgfqpoint{8.075123in}{1.928757in}}%
\pgfpathlineto{\pgfqpoint{8.090557in}{1.940036in}}%
\pgfpathlineto{\pgfqpoint{8.105991in}{1.948369in}}%
\pgfpathlineto{\pgfqpoint{8.121424in}{1.953642in}}%
\pgfpathlineto{\pgfqpoint{8.136858in}{1.955806in}}%
\pgfpathlineto{\pgfqpoint{8.152292in}{1.954876in}}%
\pgfpathlineto{\pgfqpoint{8.167725in}{1.950927in}}%
\pgfpathlineto{\pgfqpoint{8.183159in}{1.944090in}}%
\pgfpathlineto{\pgfqpoint{8.198593in}{1.934544in}}%
\pgfpathlineto{\pgfqpoint{8.214026in}{1.922511in}}%
\pgfpathlineto{\pgfqpoint{8.229460in}{1.908243in}}%
\pgfpathlineto{\pgfqpoint{8.244893in}{1.892019in}}%
\pgfpathlineto{\pgfqpoint{8.275761in}{1.854878in}}%
\pgfpathlineto{\pgfqpoint{8.306628in}{1.813455in}}%
\pgfpathlineto{\pgfqpoint{8.414663in}{1.663689in}}%
\pgfpathlineto{\pgfqpoint{8.445531in}{1.625171in}}%
\pgfpathlineto{\pgfqpoint{8.476398in}{1.589864in}}%
\pgfpathlineto{\pgfqpoint{8.507265in}{1.558011in}}%
\pgfpathlineto{\pgfqpoint{8.538133in}{1.529736in}}%
\pgfpathlineto{\pgfqpoint{8.569000in}{1.505150in}}%
\pgfpathlineto{\pgfqpoint{8.599867in}{1.484440in}}%
\pgfpathlineto{\pgfqpoint{8.630734in}{1.467916in}}%
\pgfpathlineto{\pgfqpoint{8.646168in}{1.461360in}}%
\pgfpathlineto{\pgfqpoint{8.661602in}{1.456028in}}%
\pgfpathlineto{\pgfqpoint{8.677035in}{1.451994in}}%
\pgfpathlineto{\pgfqpoint{8.692469in}{1.449335in}}%
\pgfpathlineto{\pgfqpoint{8.707903in}{1.448126in}}%
\pgfpathlineto{\pgfqpoint{8.723336in}{1.448442in}}%
\pgfpathlineto{\pgfqpoint{8.738770in}{1.450349in}}%
\pgfpathlineto{\pgfqpoint{8.754203in}{1.453903in}}%
\pgfpathlineto{\pgfqpoint{8.769637in}{1.459145in}}%
\pgfpathlineto{\pgfqpoint{8.785071in}{1.466097in}}%
\pgfpathlineto{\pgfqpoint{8.800504in}{1.474761in}}%
\pgfpathlineto{\pgfqpoint{8.815938in}{1.485109in}}%
\pgfpathlineto{\pgfqpoint{8.831372in}{1.497087in}}%
\pgfpathlineto{\pgfqpoint{8.862239in}{1.525565in}}%
\pgfpathlineto{\pgfqpoint{8.893106in}{1.559141in}}%
\pgfpathlineto{\pgfqpoint{8.923973in}{1.596285in}}%
\pgfpathlineto{\pgfqpoint{9.001142in}{1.691788in}}%
\pgfpathlineto{\pgfqpoint{9.032009in}{1.725568in}}%
\pgfpathlineto{\pgfqpoint{9.062876in}{1.754131in}}%
\pgfpathlineto{\pgfqpoint{9.078310in}{1.766079in}}%
\pgfpathlineto{\pgfqpoint{9.093743in}{1.776351in}}%
\pgfpathlineto{\pgfqpoint{9.109177in}{1.784910in}}%
\pgfpathlineto{\pgfqpoint{9.124611in}{1.791767in}}%
\pgfpathlineto{\pgfqpoint{9.140044in}{1.796971in}}%
\pgfpathlineto{\pgfqpoint{9.155478in}{1.800616in}}%
\pgfpathlineto{\pgfqpoint{9.170912in}{1.802827in}}%
\pgfpathlineto{\pgfqpoint{9.201779in}{1.803610in}}%
\pgfpathlineto{\pgfqpoint{9.232646in}{1.800843in}}%
\pgfpathlineto{\pgfqpoint{9.325248in}{1.787824in}}%
\pgfpathlineto{\pgfqpoint{9.356115in}{1.786554in}}%
\pgfpathlineto{\pgfqpoint{9.386982in}{1.788255in}}%
\pgfpathlineto{\pgfqpoint{9.417850in}{1.793160in}}%
\pgfpathlineto{\pgfqpoint{9.448717in}{1.801130in}}%
\pgfpathlineto{\pgfqpoint{9.479584in}{1.811726in}}%
\pgfpathlineto{\pgfqpoint{9.525885in}{1.831100in}}%
\pgfpathlineto{\pgfqpoint{9.633921in}{1.879170in}}%
\pgfpathlineto{\pgfqpoint{9.664788in}{1.890578in}}%
\pgfpathlineto{\pgfqpoint{9.695655in}{1.900096in}}%
\pgfpathlineto{\pgfqpoint{9.726522in}{1.907521in}}%
\pgfpathlineto{\pgfqpoint{9.741956in}{1.910434in}}%
\pgfpathlineto{\pgfqpoint{9.741956in}{1.910434in}}%
\pgfusepath{stroke}%
\end{pgfscope}%
\begin{pgfscope}%
\pgfpathrectangle{\pgfqpoint{5.706832in}{0.521603in}}{\pgfqpoint{4.227273in}{2.800000in}} %
\pgfusepath{clip}%
\pgfsetrectcap%
\pgfsetroundjoin%
\pgfsetlinewidth{0.501875pt}%
\definecolor{currentstroke}{rgb}{1.000000,0.361242,0.183750}%
\pgfsetstrokecolor{currentstroke}%
\pgfsetdash{}{0pt}%
\pgfpathmoveto{\pgfqpoint{5.898981in}{2.194518in}}%
\pgfpathlineto{\pgfqpoint{5.914415in}{2.191429in}}%
\pgfpathlineto{\pgfqpoint{5.929848in}{2.185872in}}%
\pgfpathlineto{\pgfqpoint{5.945282in}{2.177811in}}%
\pgfpathlineto{\pgfqpoint{5.960715in}{2.167252in}}%
\pgfpathlineto{\pgfqpoint{5.976149in}{2.154240in}}%
\pgfpathlineto{\pgfqpoint{5.991583in}{2.138860in}}%
\pgfpathlineto{\pgfqpoint{6.007016in}{2.121240in}}%
\pgfpathlineto{\pgfqpoint{6.022450in}{2.101549in}}%
\pgfpathlineto{\pgfqpoint{6.053317in}{2.056818in}}%
\pgfpathlineto{\pgfqpoint{6.084185in}{2.006764in}}%
\pgfpathlineto{\pgfqpoint{6.161353in}{1.876364in}}%
\pgfpathlineto{\pgfqpoint{6.192220in}{1.830393in}}%
\pgfpathlineto{\pgfqpoint{6.207654in}{1.810377in}}%
\pgfpathlineto{\pgfqpoint{6.223087in}{1.792869in}}%
\pgfpathlineto{\pgfqpoint{6.238521in}{1.778241in}}%
\pgfpathlineto{\pgfqpoint{6.253955in}{1.766835in}}%
\pgfpathlineto{\pgfqpoint{6.269388in}{1.758964in}}%
\pgfpathlineto{\pgfqpoint{6.284822in}{1.754894in}}%
\pgfpathlineto{\pgfqpoint{6.300255in}{1.754851in}}%
\pgfpathlineto{\pgfqpoint{6.315689in}{1.759005in}}%
\pgfpathlineto{\pgfqpoint{6.331123in}{1.767469in}}%
\pgfpathlineto{\pgfqpoint{6.346556in}{1.780297in}}%
\pgfpathlineto{\pgfqpoint{6.361990in}{1.797476in}}%
\pgfpathlineto{\pgfqpoint{6.377424in}{1.818929in}}%
\pgfpathlineto{\pgfqpoint{6.392857in}{1.844510in}}%
\pgfpathlineto{\pgfqpoint{6.408291in}{1.874006in}}%
\pgfpathlineto{\pgfqpoint{6.423724in}{1.907139in}}%
\pgfpathlineto{\pgfqpoint{6.439158in}{1.943569in}}%
\pgfpathlineto{\pgfqpoint{6.470025in}{2.024677in}}%
\pgfpathlineto{\pgfqpoint{6.500893in}{2.113575in}}%
\pgfpathlineto{\pgfqpoint{6.578061in}{2.340939in}}%
\pgfpathlineto{\pgfqpoint{6.608928in}{2.422101in}}%
\pgfpathlineto{\pgfqpoint{6.624362in}{2.458534in}}%
\pgfpathlineto{\pgfqpoint{6.639795in}{2.491637in}}%
\pgfpathlineto{\pgfqpoint{6.655229in}{2.521056in}}%
\pgfpathlineto{\pgfqpoint{6.670663in}{2.546498in}}%
\pgfpathlineto{\pgfqpoint{6.686096in}{2.567732in}}%
\pgfpathlineto{\pgfqpoint{6.701530in}{2.584594in}}%
\pgfpathlineto{\pgfqpoint{6.716964in}{2.596982in}}%
\pgfpathlineto{\pgfqpoint{6.732397in}{2.604858in}}%
\pgfpathlineto{\pgfqpoint{6.747831in}{2.608246in}}%
\pgfpathlineto{\pgfqpoint{6.763264in}{2.607224in}}%
\pgfpathlineto{\pgfqpoint{6.778698in}{2.601921in}}%
\pgfpathlineto{\pgfqpoint{6.794132in}{2.592512in}}%
\pgfpathlineto{\pgfqpoint{6.809565in}{2.579207in}}%
\pgfpathlineto{\pgfqpoint{6.824999in}{2.562253in}}%
\pgfpathlineto{\pgfqpoint{6.840433in}{2.541918in}}%
\pgfpathlineto{\pgfqpoint{6.855866in}{2.518492in}}%
\pgfpathlineto{\pgfqpoint{6.871300in}{2.492275in}}%
\pgfpathlineto{\pgfqpoint{6.902167in}{2.432712in}}%
\pgfpathlineto{\pgfqpoint{6.933034in}{2.365716in}}%
\pgfpathlineto{\pgfqpoint{6.979335in}{2.256577in}}%
\pgfpathlineto{\pgfqpoint{7.056504in}{2.070210in}}%
\pgfpathlineto{\pgfqpoint{7.087371in}{1.999989in}}%
\pgfpathlineto{\pgfqpoint{7.118238in}{1.934934in}}%
\pgfpathlineto{\pgfqpoint{7.149105in}{1.876580in}}%
\pgfpathlineto{\pgfqpoint{7.179973in}{1.826187in}}%
\pgfpathlineto{\pgfqpoint{7.195406in}{1.804273in}}%
\pgfpathlineto{\pgfqpoint{7.210840in}{1.784649in}}%
\pgfpathlineto{\pgfqpoint{7.226274in}{1.767354in}}%
\pgfpathlineto{\pgfqpoint{7.241707in}{1.752392in}}%
\pgfpathlineto{\pgfqpoint{7.257141in}{1.739730in}}%
\pgfpathlineto{\pgfqpoint{7.272574in}{1.729298in}}%
\pgfpathlineto{\pgfqpoint{7.288008in}{1.720991in}}%
\pgfpathlineto{\pgfqpoint{7.303442in}{1.714664in}}%
\pgfpathlineto{\pgfqpoint{7.318875in}{1.710143in}}%
\pgfpathlineto{\pgfqpoint{7.334309in}{1.707220in}}%
\pgfpathlineto{\pgfqpoint{7.365176in}{1.705210in}}%
\pgfpathlineto{\pgfqpoint{7.396044in}{1.706549in}}%
\pgfpathlineto{\pgfqpoint{7.457778in}{1.710500in}}%
\pgfpathlineto{\pgfqpoint{7.488645in}{1.709226in}}%
\pgfpathlineto{\pgfqpoint{7.519513in}{1.703971in}}%
\pgfpathlineto{\pgfqpoint{7.550380in}{1.694215in}}%
\pgfpathlineto{\pgfqpoint{7.581247in}{1.680217in}}%
\pgfpathlineto{\pgfqpoint{7.612114in}{1.662998in}}%
\pgfpathlineto{\pgfqpoint{7.673849in}{1.626082in}}%
\pgfpathlineto{\pgfqpoint{7.704716in}{1.610916in}}%
\pgfpathlineto{\pgfqpoint{7.720150in}{1.605189in}}%
\pgfpathlineto{\pgfqpoint{7.735584in}{1.601062in}}%
\pgfpathlineto{\pgfqpoint{7.751017in}{1.598773in}}%
\pgfpathlineto{\pgfqpoint{7.766451in}{1.598521in}}%
\pgfpathlineto{\pgfqpoint{7.781884in}{1.600465in}}%
\pgfpathlineto{\pgfqpoint{7.797318in}{1.604714in}}%
\pgfpathlineto{\pgfqpoint{7.812752in}{1.611323in}}%
\pgfpathlineto{\pgfqpoint{7.828185in}{1.620294in}}%
\pgfpathlineto{\pgfqpoint{7.843619in}{1.631570in}}%
\pgfpathlineto{\pgfqpoint{7.859053in}{1.645039in}}%
\pgfpathlineto{\pgfqpoint{7.874486in}{1.660532in}}%
\pgfpathlineto{\pgfqpoint{7.905353in}{1.696665in}}%
\pgfpathlineto{\pgfqpoint{7.936221in}{1.737681in}}%
\pgfpathlineto{\pgfqpoint{7.997955in}{1.822751in}}%
\pgfpathlineto{\pgfqpoint{8.028823in}{1.860591in}}%
\pgfpathlineto{\pgfqpoint{8.044256in}{1.877056in}}%
\pgfpathlineto{\pgfqpoint{8.059690in}{1.891469in}}%
\pgfpathlineto{\pgfqpoint{8.075123in}{1.903571in}}%
\pgfpathlineto{\pgfqpoint{8.090557in}{1.913149in}}%
\pgfpathlineto{\pgfqpoint{8.105991in}{1.920044in}}%
\pgfpathlineto{\pgfqpoint{8.121424in}{1.924152in}}%
\pgfpathlineto{\pgfqpoint{8.136858in}{1.925423in}}%
\pgfpathlineto{\pgfqpoint{8.152292in}{1.923867in}}%
\pgfpathlineto{\pgfqpoint{8.167725in}{1.919546in}}%
\pgfpathlineto{\pgfqpoint{8.183159in}{1.912572in}}%
\pgfpathlineto{\pgfqpoint{8.198593in}{1.903105in}}%
\pgfpathlineto{\pgfqpoint{8.214026in}{1.891346in}}%
\pgfpathlineto{\pgfqpoint{8.229460in}{1.877529in}}%
\pgfpathlineto{\pgfqpoint{8.260327in}{1.844777in}}%
\pgfpathlineto{\pgfqpoint{8.291194in}{1.807102in}}%
\pgfpathlineto{\pgfqpoint{8.414663in}{1.648893in}}%
\pgfpathlineto{\pgfqpoint{8.445531in}{1.614636in}}%
\pgfpathlineto{\pgfqpoint{8.476398in}{1.583712in}}%
\pgfpathlineto{\pgfqpoint{8.507265in}{1.556050in}}%
\pgfpathlineto{\pgfqpoint{8.538133in}{1.531426in}}%
\pgfpathlineto{\pgfqpoint{8.569000in}{1.509630in}}%
\pgfpathlineto{\pgfqpoint{8.599867in}{1.490619in}}%
\pgfpathlineto{\pgfqpoint{8.630734in}{1.474612in}}%
\pgfpathlineto{\pgfqpoint{8.661602in}{1.462115in}}%
\pgfpathlineto{\pgfqpoint{8.677035in}{1.457411in}}%
\pgfpathlineto{\pgfqpoint{8.692469in}{1.453877in}}%
\pgfpathlineto{\pgfqpoint{8.707903in}{1.451626in}}%
\pgfpathlineto{\pgfqpoint{8.723336in}{1.450769in}}%
\pgfpathlineto{\pgfqpoint{8.738770in}{1.451406in}}%
\pgfpathlineto{\pgfqpoint{8.754203in}{1.453626in}}%
\pgfpathlineto{\pgfqpoint{8.769637in}{1.457499in}}%
\pgfpathlineto{\pgfqpoint{8.785071in}{1.463069in}}%
\pgfpathlineto{\pgfqpoint{8.800504in}{1.470354in}}%
\pgfpathlineto{\pgfqpoint{8.815938in}{1.479338in}}%
\pgfpathlineto{\pgfqpoint{8.831372in}{1.489974in}}%
\pgfpathlineto{\pgfqpoint{8.846805in}{1.502177in}}%
\pgfpathlineto{\pgfqpoint{8.877673in}{1.530779in}}%
\pgfpathlineto{\pgfqpoint{8.908540in}{1.563816in}}%
\pgfpathlineto{\pgfqpoint{8.970274in}{1.635976in}}%
\pgfpathlineto{\pgfqpoint{9.001142in}{1.671084in}}%
\pgfpathlineto{\pgfqpoint{9.032009in}{1.703026in}}%
\pgfpathlineto{\pgfqpoint{9.062876in}{1.730336in}}%
\pgfpathlineto{\pgfqpoint{9.078310in}{1.741940in}}%
\pgfpathlineto{\pgfqpoint{9.093743in}{1.752082in}}%
\pgfpathlineto{\pgfqpoint{9.109177in}{1.760740in}}%
\pgfpathlineto{\pgfqpoint{9.124611in}{1.767933in}}%
\pgfpathlineto{\pgfqpoint{9.140044in}{1.773715in}}%
\pgfpathlineto{\pgfqpoint{9.155478in}{1.778174in}}%
\pgfpathlineto{\pgfqpoint{9.186345in}{1.783623in}}%
\pgfpathlineto{\pgfqpoint{9.217212in}{1.785502in}}%
\pgfpathlineto{\pgfqpoint{9.263513in}{1.784819in}}%
\pgfpathlineto{\pgfqpoint{9.309814in}{1.784266in}}%
\pgfpathlineto{\pgfqpoint{9.340682in}{1.785942in}}%
\pgfpathlineto{\pgfqpoint{9.371549in}{1.790134in}}%
\pgfpathlineto{\pgfqpoint{9.402416in}{1.797135in}}%
\pgfpathlineto{\pgfqpoint{9.433283in}{1.806848in}}%
\pgfpathlineto{\pgfqpoint{9.464151in}{1.818845in}}%
\pgfpathlineto{\pgfqpoint{9.510452in}{1.839633in}}%
\pgfpathlineto{\pgfqpoint{9.587620in}{1.875136in}}%
\pgfpathlineto{\pgfqpoint{9.618487in}{1.887586in}}%
\pgfpathlineto{\pgfqpoint{9.649354in}{1.898312in}}%
\pgfpathlineto{\pgfqpoint{9.680222in}{1.907081in}}%
\pgfpathlineto{\pgfqpoint{9.711089in}{1.913832in}}%
\pgfpathlineto{\pgfqpoint{9.741956in}{1.918646in}}%
\pgfpathlineto{\pgfqpoint{9.741956in}{1.918646in}}%
\pgfusepath{stroke}%
\end{pgfscope}%
\begin{pgfscope}%
\pgfpathrectangle{\pgfqpoint{5.706832in}{0.521603in}}{\pgfqpoint{4.227273in}{2.800000in}} %
\pgfusepath{clip}%
\pgfsetrectcap%
\pgfsetroundjoin%
\pgfsetlinewidth{0.501875pt}%
\definecolor{currentstroke}{rgb}{1.000000,0.243914,0.122888}%
\pgfsetstrokecolor{currentstroke}%
\pgfsetdash{}{0pt}%
\pgfpathmoveto{\pgfqpoint{5.898981in}{2.198339in}}%
\pgfpathlineto{\pgfqpoint{5.914415in}{2.196749in}}%
\pgfpathlineto{\pgfqpoint{5.929848in}{2.192556in}}%
\pgfpathlineto{\pgfqpoint{5.945282in}{2.185698in}}%
\pgfpathlineto{\pgfqpoint{5.960715in}{2.176156in}}%
\pgfpathlineto{\pgfqpoint{5.976149in}{2.163955in}}%
\pgfpathlineto{\pgfqpoint{5.991583in}{2.149168in}}%
\pgfpathlineto{\pgfqpoint{6.007016in}{2.131914in}}%
\pgfpathlineto{\pgfqpoint{6.022450in}{2.112361in}}%
\pgfpathlineto{\pgfqpoint{6.053317in}{2.067242in}}%
\pgfpathlineto{\pgfqpoint{6.084185in}{2.016000in}}%
\pgfpathlineto{\pgfqpoint{6.176786in}{1.855195in}}%
\pgfpathlineto{\pgfqpoint{6.192220in}{1.831804in}}%
\pgfpathlineto{\pgfqpoint{6.207654in}{1.810545in}}%
\pgfpathlineto{\pgfqpoint{6.223087in}{1.791831in}}%
\pgfpathlineto{\pgfqpoint{6.238521in}{1.776049in}}%
\pgfpathlineto{\pgfqpoint{6.253955in}{1.763558in}}%
\pgfpathlineto{\pgfqpoint{6.269388in}{1.754680in}}%
\pgfpathlineto{\pgfqpoint{6.284822in}{1.749695in}}%
\pgfpathlineto{\pgfqpoint{6.300255in}{1.748833in}}%
\pgfpathlineto{\pgfqpoint{6.315689in}{1.752272in}}%
\pgfpathlineto{\pgfqpoint{6.331123in}{1.760132in}}%
\pgfpathlineto{\pgfqpoint{6.346556in}{1.772467in}}%
\pgfpathlineto{\pgfqpoint{6.361990in}{1.789272in}}%
\pgfpathlineto{\pgfqpoint{6.377424in}{1.810469in}}%
\pgfpathlineto{\pgfqpoint{6.392857in}{1.835916in}}%
\pgfpathlineto{\pgfqpoint{6.408291in}{1.865400in}}%
\pgfpathlineto{\pgfqpoint{6.423724in}{1.898644in}}%
\pgfpathlineto{\pgfqpoint{6.439158in}{1.935307in}}%
\pgfpathlineto{\pgfqpoint{6.470025in}{2.017234in}}%
\pgfpathlineto{\pgfqpoint{6.500893in}{2.107381in}}%
\pgfpathlineto{\pgfqpoint{6.578061in}{2.339139in}}%
\pgfpathlineto{\pgfqpoint{6.608928in}{2.422206in}}%
\pgfpathlineto{\pgfqpoint{6.624362in}{2.459543in}}%
\pgfpathlineto{\pgfqpoint{6.639795in}{2.493493in}}%
\pgfpathlineto{\pgfqpoint{6.655229in}{2.523683in}}%
\pgfpathlineto{\pgfqpoint{6.670663in}{2.549807in}}%
\pgfpathlineto{\pgfqpoint{6.686096in}{2.571621in}}%
\pgfpathlineto{\pgfqpoint{6.701530in}{2.588951in}}%
\pgfpathlineto{\pgfqpoint{6.716964in}{2.601691in}}%
\pgfpathlineto{\pgfqpoint{6.732397in}{2.609802in}}%
\pgfpathlineto{\pgfqpoint{6.747831in}{2.613309in}}%
\pgfpathlineto{\pgfqpoint{6.763264in}{2.612294in}}%
\pgfpathlineto{\pgfqpoint{6.778698in}{2.606894in}}%
\pgfpathlineto{\pgfqpoint{6.794132in}{2.597293in}}%
\pgfpathlineto{\pgfqpoint{6.809565in}{2.583715in}}%
\pgfpathlineto{\pgfqpoint{6.824999in}{2.566419in}}%
\pgfpathlineto{\pgfqpoint{6.840433in}{2.545686in}}%
\pgfpathlineto{\pgfqpoint{6.855866in}{2.521819in}}%
\pgfpathlineto{\pgfqpoint{6.871300in}{2.495132in}}%
\pgfpathlineto{\pgfqpoint{6.902167in}{2.434588in}}%
\pgfpathlineto{\pgfqpoint{6.933034in}{2.366625in}}%
\pgfpathlineto{\pgfqpoint{6.979335in}{2.256200in}}%
\pgfpathlineto{\pgfqpoint{7.056504in}{2.068500in}}%
\pgfpathlineto{\pgfqpoint{7.087371in}{1.998139in}}%
\pgfpathlineto{\pgfqpoint{7.118238in}{1.933216in}}%
\pgfpathlineto{\pgfqpoint{7.149105in}{1.875283in}}%
\pgfpathlineto{\pgfqpoint{7.164539in}{1.849343in}}%
\pgfpathlineto{\pgfqpoint{7.179973in}{1.825591in}}%
\pgfpathlineto{\pgfqpoint{7.195406in}{1.804120in}}%
\pgfpathlineto{\pgfqpoint{7.210840in}{1.784992in}}%
\pgfpathlineto{\pgfqpoint{7.226274in}{1.768234in}}%
\pgfpathlineto{\pgfqpoint{7.241707in}{1.753838in}}%
\pgfpathlineto{\pgfqpoint{7.257141in}{1.741756in}}%
\pgfpathlineto{\pgfqpoint{7.272574in}{1.731904in}}%
\pgfpathlineto{\pgfqpoint{7.288008in}{1.724160in}}%
\pgfpathlineto{\pgfqpoint{7.303442in}{1.718365in}}%
\pgfpathlineto{\pgfqpoint{7.318875in}{1.714329in}}%
\pgfpathlineto{\pgfqpoint{7.334309in}{1.711831in}}%
\pgfpathlineto{\pgfqpoint{7.365176in}{1.710447in}}%
\pgfpathlineto{\pgfqpoint{7.396044in}{1.712063in}}%
\pgfpathlineto{\pgfqpoint{7.442344in}{1.715249in}}%
\pgfpathlineto{\pgfqpoint{7.473212in}{1.714952in}}%
\pgfpathlineto{\pgfqpoint{7.504079in}{1.710944in}}%
\pgfpathlineto{\pgfqpoint{7.519513in}{1.707264in}}%
\pgfpathlineto{\pgfqpoint{7.550380in}{1.696388in}}%
\pgfpathlineto{\pgfqpoint{7.581247in}{1.681165in}}%
\pgfpathlineto{\pgfqpoint{7.612114in}{1.662664in}}%
\pgfpathlineto{\pgfqpoint{7.673849in}{1.623175in}}%
\pgfpathlineto{\pgfqpoint{7.704716in}{1.606826in}}%
\pgfpathlineto{\pgfqpoint{7.720150in}{1.600569in}}%
\pgfpathlineto{\pgfqpoint{7.735584in}{1.595971in}}%
\pgfpathlineto{\pgfqpoint{7.751017in}{1.593284in}}%
\pgfpathlineto{\pgfqpoint{7.766451in}{1.592723in}}%
\pgfpathlineto{\pgfqpoint{7.781884in}{1.594464in}}%
\pgfpathlineto{\pgfqpoint{7.797318in}{1.598630in}}%
\pgfpathlineto{\pgfqpoint{7.812752in}{1.605291in}}%
\pgfpathlineto{\pgfqpoint{7.828185in}{1.614461in}}%
\pgfpathlineto{\pgfqpoint{7.843619in}{1.626090in}}%
\pgfpathlineto{\pgfqpoint{7.859053in}{1.640070in}}%
\pgfpathlineto{\pgfqpoint{7.874486in}{1.656232in}}%
\pgfpathlineto{\pgfqpoint{7.905353in}{1.694150in}}%
\pgfpathlineto{\pgfqpoint{7.936221in}{1.737444in}}%
\pgfpathlineto{\pgfqpoint{7.997955in}{1.827740in}}%
\pgfpathlineto{\pgfqpoint{8.028823in}{1.868014in}}%
\pgfpathlineto{\pgfqpoint{8.044256in}{1.885537in}}%
\pgfpathlineto{\pgfqpoint{8.059690in}{1.900864in}}%
\pgfpathlineto{\pgfqpoint{8.075123in}{1.913713in}}%
\pgfpathlineto{\pgfqpoint{8.090557in}{1.923852in}}%
\pgfpathlineto{\pgfqpoint{8.105991in}{1.931110in}}%
\pgfpathlineto{\pgfqpoint{8.121424in}{1.935373in}}%
\pgfpathlineto{\pgfqpoint{8.136858in}{1.936592in}}%
\pgfpathlineto{\pgfqpoint{8.152292in}{1.934778in}}%
\pgfpathlineto{\pgfqpoint{8.167725in}{1.930001in}}%
\pgfpathlineto{\pgfqpoint{8.183159in}{1.922385in}}%
\pgfpathlineto{\pgfqpoint{8.198593in}{1.912107in}}%
\pgfpathlineto{\pgfqpoint{8.214026in}{1.899383in}}%
\pgfpathlineto{\pgfqpoint{8.229460in}{1.884467in}}%
\pgfpathlineto{\pgfqpoint{8.260327in}{1.849202in}}%
\pgfpathlineto{\pgfqpoint{8.291194in}{1.808728in}}%
\pgfpathlineto{\pgfqpoint{8.399230in}{1.658952in}}%
\pgfpathlineto{\pgfqpoint{8.430097in}{1.620821in}}%
\pgfpathlineto{\pgfqpoint{8.460964in}{1.586442in}}%
\pgfpathlineto{\pgfqpoint{8.491832in}{1.556031in}}%
\pgfpathlineto{\pgfqpoint{8.522699in}{1.529552in}}%
\pgfpathlineto{\pgfqpoint{8.553566in}{1.506869in}}%
\pgfpathlineto{\pgfqpoint{8.584433in}{1.487889in}}%
\pgfpathlineto{\pgfqpoint{8.615301in}{1.472676in}}%
\pgfpathlineto{\pgfqpoint{8.646168in}{1.461499in}}%
\pgfpathlineto{\pgfqpoint{8.661602in}{1.457564in}}%
\pgfpathlineto{\pgfqpoint{8.677035in}{1.454826in}}%
\pgfpathlineto{\pgfqpoint{8.692469in}{1.453364in}}%
\pgfpathlineto{\pgfqpoint{8.707903in}{1.453255in}}%
\pgfpathlineto{\pgfqpoint{8.723336in}{1.454576in}}%
\pgfpathlineto{\pgfqpoint{8.738770in}{1.457392in}}%
\pgfpathlineto{\pgfqpoint{8.754203in}{1.461758in}}%
\pgfpathlineto{\pgfqpoint{8.769637in}{1.467708in}}%
\pgfpathlineto{\pgfqpoint{8.785071in}{1.475257in}}%
\pgfpathlineto{\pgfqpoint{8.800504in}{1.484390in}}%
\pgfpathlineto{\pgfqpoint{8.815938in}{1.495068in}}%
\pgfpathlineto{\pgfqpoint{8.846805in}{1.520737in}}%
\pgfpathlineto{\pgfqpoint{8.877673in}{1.551320in}}%
\pgfpathlineto{\pgfqpoint{8.908540in}{1.585409in}}%
\pgfpathlineto{\pgfqpoint{8.985708in}{1.673974in}}%
\pgfpathlineto{\pgfqpoint{9.016575in}{1.705678in}}%
\pgfpathlineto{\pgfqpoint{9.047443in}{1.732776in}}%
\pgfpathlineto{\pgfqpoint{9.062876in}{1.744247in}}%
\pgfpathlineto{\pgfqpoint{9.078310in}{1.754220in}}%
\pgfpathlineto{\pgfqpoint{9.093743in}{1.762659in}}%
\pgfpathlineto{\pgfqpoint{9.109177in}{1.769569in}}%
\pgfpathlineto{\pgfqpoint{9.124611in}{1.774993in}}%
\pgfpathlineto{\pgfqpoint{9.140044in}{1.779012in}}%
\pgfpathlineto{\pgfqpoint{9.155478in}{1.781741in}}%
\pgfpathlineto{\pgfqpoint{9.186345in}{1.783930in}}%
\pgfpathlineto{\pgfqpoint{9.217212in}{1.782962in}}%
\pgfpathlineto{\pgfqpoint{9.309814in}{1.776837in}}%
\pgfpathlineto{\pgfqpoint{9.340682in}{1.778220in}}%
\pgfpathlineto{\pgfqpoint{9.371549in}{1.782719in}}%
\pgfpathlineto{\pgfqpoint{9.402416in}{1.790538in}}%
\pgfpathlineto{\pgfqpoint{9.433283in}{1.801471in}}%
\pgfpathlineto{\pgfqpoint{9.464151in}{1.814983in}}%
\pgfpathlineto{\pgfqpoint{9.510452in}{1.838375in}}%
\pgfpathlineto{\pgfqpoint{9.587620in}{1.878317in}}%
\pgfpathlineto{\pgfqpoint{9.618487in}{1.892305in}}%
\pgfpathlineto{\pgfqpoint{9.649354in}{1.904286in}}%
\pgfpathlineto{\pgfqpoint{9.680222in}{1.913939in}}%
\pgfpathlineto{\pgfqpoint{9.711089in}{1.921131in}}%
\pgfpathlineto{\pgfqpoint{9.741956in}{1.925901in}}%
\pgfpathlineto{\pgfqpoint{9.741956in}{1.925901in}}%
\pgfusepath{stroke}%
\end{pgfscope}%
\begin{pgfscope}%
\pgfpathrectangle{\pgfqpoint{5.706832in}{0.521603in}}{\pgfqpoint{4.227273in}{2.800000in}} %
\pgfusepath{clip}%
\pgfsetrectcap%
\pgfsetroundjoin%
\pgfsetlinewidth{0.501875pt}%
\definecolor{currentstroke}{rgb}{1.000000,0.122888,0.061561}%
\pgfsetstrokecolor{currentstroke}%
\pgfsetdash{}{0pt}%
\pgfpathmoveto{\pgfqpoint{5.898981in}{2.190874in}}%
\pgfpathlineto{\pgfqpoint{5.914415in}{2.188965in}}%
\pgfpathlineto{\pgfqpoint{5.929848in}{2.184465in}}%
\pgfpathlineto{\pgfqpoint{5.945282in}{2.177304in}}%
\pgfpathlineto{\pgfqpoint{5.960715in}{2.167458in}}%
\pgfpathlineto{\pgfqpoint{5.976149in}{2.154947in}}%
\pgfpathlineto{\pgfqpoint{5.991583in}{2.139844in}}%
\pgfpathlineto{\pgfqpoint{6.007016in}{2.122266in}}%
\pgfpathlineto{\pgfqpoint{6.022450in}{2.102384in}}%
\pgfpathlineto{\pgfqpoint{6.053317in}{2.056621in}}%
\pgfpathlineto{\pgfqpoint{6.084185in}{2.004816in}}%
\pgfpathlineto{\pgfqpoint{6.161353in}{1.868770in}}%
\pgfpathlineto{\pgfqpoint{6.192220in}{1.820858in}}%
\pgfpathlineto{\pgfqpoint{6.207654in}{1.800073in}}%
\pgfpathlineto{\pgfqpoint{6.223087in}{1.781960in}}%
\pgfpathlineto{\pgfqpoint{6.238521in}{1.766904in}}%
\pgfpathlineto{\pgfqpoint{6.253955in}{1.755257in}}%
\pgfpathlineto{\pgfqpoint{6.269388in}{1.747329in}}%
\pgfpathlineto{\pgfqpoint{6.284822in}{1.743387in}}%
\pgfpathlineto{\pgfqpoint{6.300255in}{1.743643in}}%
\pgfpathlineto{\pgfqpoint{6.315689in}{1.748255in}}%
\pgfpathlineto{\pgfqpoint{6.331123in}{1.757321in}}%
\pgfpathlineto{\pgfqpoint{6.346556in}{1.770874in}}%
\pgfpathlineto{\pgfqpoint{6.361990in}{1.788882in}}%
\pgfpathlineto{\pgfqpoint{6.377424in}{1.811248in}}%
\pgfpathlineto{\pgfqpoint{6.392857in}{1.837806in}}%
\pgfpathlineto{\pgfqpoint{6.408291in}{1.868326in}}%
\pgfpathlineto{\pgfqpoint{6.423724in}{1.902511in}}%
\pgfpathlineto{\pgfqpoint{6.454592in}{1.980406in}}%
\pgfpathlineto{\pgfqpoint{6.485459in}{2.068024in}}%
\pgfpathlineto{\pgfqpoint{6.593494in}{2.388698in}}%
\pgfpathlineto{\pgfqpoint{6.624362in}{2.465878in}}%
\pgfpathlineto{\pgfqpoint{6.639795in}{2.499597in}}%
\pgfpathlineto{\pgfqpoint{6.655229in}{2.529577in}}%
\pgfpathlineto{\pgfqpoint{6.670663in}{2.555521in}}%
\pgfpathlineto{\pgfqpoint{6.686096in}{2.577199in}}%
\pgfpathlineto{\pgfqpoint{6.701530in}{2.594445in}}%
\pgfpathlineto{\pgfqpoint{6.716964in}{2.607161in}}%
\pgfpathlineto{\pgfqpoint{6.732397in}{2.615310in}}%
\pgfpathlineto{\pgfqpoint{6.747831in}{2.618916in}}%
\pgfpathlineto{\pgfqpoint{6.763264in}{2.618062in}}%
\pgfpathlineto{\pgfqpoint{6.778698in}{2.612879in}}%
\pgfpathlineto{\pgfqpoint{6.794132in}{2.603545in}}%
\pgfpathlineto{\pgfqpoint{6.809565in}{2.590278in}}%
\pgfpathlineto{\pgfqpoint{6.824999in}{2.573323in}}%
\pgfpathlineto{\pgfqpoint{6.840433in}{2.552955in}}%
\pgfpathlineto{\pgfqpoint{6.855866in}{2.529464in}}%
\pgfpathlineto{\pgfqpoint{6.871300in}{2.503154in}}%
\pgfpathlineto{\pgfqpoint{6.902167in}{2.443319in}}%
\pgfpathlineto{\pgfqpoint{6.933034in}{2.375935in}}%
\pgfpathlineto{\pgfqpoint{6.979335in}{2.265979in}}%
\pgfpathlineto{\pgfqpoint{7.071937in}{2.041544in}}%
\pgfpathlineto{\pgfqpoint{7.102804in}{1.972809in}}%
\pgfpathlineto{\pgfqpoint{7.133672in}{1.910119in}}%
\pgfpathlineto{\pgfqpoint{7.164539in}{1.854959in}}%
\pgfpathlineto{\pgfqpoint{7.179973in}{1.830577in}}%
\pgfpathlineto{\pgfqpoint{7.195406in}{1.808470in}}%
\pgfpathlineto{\pgfqpoint{7.210840in}{1.788709in}}%
\pgfpathlineto{\pgfqpoint{7.226274in}{1.771328in}}%
\pgfpathlineto{\pgfqpoint{7.241707in}{1.756322in}}%
\pgfpathlineto{\pgfqpoint{7.257141in}{1.743646in}}%
\pgfpathlineto{\pgfqpoint{7.272574in}{1.733219in}}%
\pgfpathlineto{\pgfqpoint{7.288008in}{1.724918in}}%
\pgfpathlineto{\pgfqpoint{7.303442in}{1.718582in}}%
\pgfpathlineto{\pgfqpoint{7.318875in}{1.714020in}}%
\pgfpathlineto{\pgfqpoint{7.334309in}{1.711007in}}%
\pgfpathlineto{\pgfqpoint{7.365176in}{1.708621in}}%
\pgfpathlineto{\pgfqpoint{7.396044in}{1.709263in}}%
\pgfpathlineto{\pgfqpoint{7.442344in}{1.711070in}}%
\pgfpathlineto{\pgfqpoint{7.473212in}{1.709967in}}%
\pgfpathlineto{\pgfqpoint{7.504079in}{1.705310in}}%
\pgfpathlineto{\pgfqpoint{7.534946in}{1.696342in}}%
\pgfpathlineto{\pgfqpoint{7.565814in}{1.683036in}}%
\pgfpathlineto{\pgfqpoint{7.596681in}{1.666108in}}%
\pgfpathlineto{\pgfqpoint{7.689283in}{1.610085in}}%
\pgfpathlineto{\pgfqpoint{7.704716in}{1.602876in}}%
\pgfpathlineto{\pgfqpoint{7.720150in}{1.597065in}}%
\pgfpathlineto{\pgfqpoint{7.735584in}{1.592925in}}%
\pgfpathlineto{\pgfqpoint{7.751017in}{1.590706in}}%
\pgfpathlineto{\pgfqpoint{7.766451in}{1.590620in}}%
\pgfpathlineto{\pgfqpoint{7.781884in}{1.592840in}}%
\pgfpathlineto{\pgfqpoint{7.797318in}{1.597489in}}%
\pgfpathlineto{\pgfqpoint{7.812752in}{1.604639in}}%
\pgfpathlineto{\pgfqpoint{7.828185in}{1.614302in}}%
\pgfpathlineto{\pgfqpoint{7.843619in}{1.626433in}}%
\pgfpathlineto{\pgfqpoint{7.859053in}{1.640923in}}%
\pgfpathlineto{\pgfqpoint{7.874486in}{1.657605in}}%
\pgfpathlineto{\pgfqpoint{7.905353in}{1.696587in}}%
\pgfpathlineto{\pgfqpoint{7.936221in}{1.740964in}}%
\pgfpathlineto{\pgfqpoint{7.997955in}{1.833320in}}%
\pgfpathlineto{\pgfqpoint{8.028823in}{1.874458in}}%
\pgfpathlineto{\pgfqpoint{8.044256in}{1.892349in}}%
\pgfpathlineto{\pgfqpoint{8.059690in}{1.907994in}}%
\pgfpathlineto{\pgfqpoint{8.075123in}{1.921108in}}%
\pgfpathlineto{\pgfqpoint{8.090557in}{1.931461in}}%
\pgfpathlineto{\pgfqpoint{8.105991in}{1.938882in}}%
\pgfpathlineto{\pgfqpoint{8.121424in}{1.943262in}}%
\pgfpathlineto{\pgfqpoint{8.136858in}{1.944557in}}%
\pgfpathlineto{\pgfqpoint{8.152292in}{1.942788in}}%
\pgfpathlineto{\pgfqpoint{8.167725in}{1.938033in}}%
\pgfpathlineto{\pgfqpoint{8.183159in}{1.930428in}}%
\pgfpathlineto{\pgfqpoint{8.198593in}{1.920157in}}%
\pgfpathlineto{\pgfqpoint{8.214026in}{1.907450in}}%
\pgfpathlineto{\pgfqpoint{8.229460in}{1.892568in}}%
\pgfpathlineto{\pgfqpoint{8.260327in}{1.857444in}}%
\pgfpathlineto{\pgfqpoint{8.291194in}{1.817225in}}%
\pgfpathlineto{\pgfqpoint{8.414663in}{1.648976in}}%
\pgfpathlineto{\pgfqpoint{8.445531in}{1.612409in}}%
\pgfpathlineto{\pgfqpoint{8.476398in}{1.579346in}}%
\pgfpathlineto{\pgfqpoint{8.507265in}{1.549829in}}%
\pgfpathlineto{\pgfqpoint{8.538133in}{1.523773in}}%
\pgfpathlineto{\pgfqpoint{8.569000in}{1.501112in}}%
\pgfpathlineto{\pgfqpoint{8.599867in}{1.481914in}}%
\pgfpathlineto{\pgfqpoint{8.630734in}{1.466448in}}%
\pgfpathlineto{\pgfqpoint{8.661602in}{1.455199in}}%
\pgfpathlineto{\pgfqpoint{8.677035in}{1.451355in}}%
\pgfpathlineto{\pgfqpoint{8.692469in}{1.448818in}}%
\pgfpathlineto{\pgfqpoint{8.707903in}{1.447679in}}%
\pgfpathlineto{\pgfqpoint{8.723336in}{1.448028in}}%
\pgfpathlineto{\pgfqpoint{8.738770in}{1.449944in}}%
\pgfpathlineto{\pgfqpoint{8.754203in}{1.453494in}}%
\pgfpathlineto{\pgfqpoint{8.769637in}{1.458730in}}%
\pgfpathlineto{\pgfqpoint{8.785071in}{1.465680in}}%
\pgfpathlineto{\pgfqpoint{8.800504in}{1.474346in}}%
\pgfpathlineto{\pgfqpoint{8.815938in}{1.484703in}}%
\pgfpathlineto{\pgfqpoint{8.831372in}{1.496694in}}%
\pgfpathlineto{\pgfqpoint{8.862239in}{1.525189in}}%
\pgfpathlineto{\pgfqpoint{8.893106in}{1.558736in}}%
\pgfpathlineto{\pgfqpoint{8.939407in}{1.615015in}}%
\pgfpathlineto{\pgfqpoint{9.001142in}{1.690776in}}%
\pgfpathlineto{\pgfqpoint{9.032009in}{1.724455in}}%
\pgfpathlineto{\pgfqpoint{9.062876in}{1.753135in}}%
\pgfpathlineto{\pgfqpoint{9.078310in}{1.765257in}}%
\pgfpathlineto{\pgfqpoint{9.093743in}{1.775792in}}%
\pgfpathlineto{\pgfqpoint{9.109177in}{1.784711in}}%
\pgfpathlineto{\pgfqpoint{9.124611in}{1.792026in}}%
\pgfpathlineto{\pgfqpoint{9.140044in}{1.797788in}}%
\pgfpathlineto{\pgfqpoint{9.155478in}{1.802083in}}%
\pgfpathlineto{\pgfqpoint{9.170912in}{1.805032in}}%
\pgfpathlineto{\pgfqpoint{9.201779in}{1.807510in}}%
\pgfpathlineto{\pgfqpoint{9.232646in}{1.806644in}}%
\pgfpathlineto{\pgfqpoint{9.325248in}{1.799594in}}%
\pgfpathlineto{\pgfqpoint{9.356115in}{1.800192in}}%
\pgfpathlineto{\pgfqpoint{9.386982in}{1.803662in}}%
\pgfpathlineto{\pgfqpoint{9.417850in}{1.810237in}}%
\pgfpathlineto{\pgfqpoint{9.448717in}{1.819767in}}%
\pgfpathlineto{\pgfqpoint{9.479584in}{1.831788in}}%
\pgfpathlineto{\pgfqpoint{9.525885in}{1.852937in}}%
\pgfpathlineto{\pgfqpoint{9.603053in}{1.889594in}}%
\pgfpathlineto{\pgfqpoint{9.633921in}{1.902494in}}%
\pgfpathlineto{\pgfqpoint{9.664788in}{1.913513in}}%
\pgfpathlineto{\pgfqpoint{9.695655in}{1.922318in}}%
\pgfpathlineto{\pgfqpoint{9.726522in}{1.928776in}}%
\pgfpathlineto{\pgfqpoint{9.741956in}{1.931136in}}%
\pgfpathlineto{\pgfqpoint{9.741956in}{1.931136in}}%
\pgfusepath{stroke}%
\end{pgfscope}%
\begin{pgfscope}%
\pgfpathrectangle{\pgfqpoint{5.706832in}{0.521603in}}{\pgfqpoint{4.227273in}{2.800000in}} %
\pgfusepath{clip}%
\pgfsetrectcap%
\pgfsetroundjoin%
\pgfsetlinewidth{0.501875pt}%
\definecolor{currentstroke}{rgb}{0.000000,0.000000,0.000000}%
\pgfsetstrokecolor{currentstroke}%
\pgfsetdash{}{0pt}%
\pgfpathmoveto{\pgfqpoint{5.898981in}{2.189443in}}%
\pgfpathlineto{\pgfqpoint{5.914415in}{2.187213in}}%
\pgfpathlineto{\pgfqpoint{5.929848in}{2.182336in}}%
\pgfpathlineto{\pgfqpoint{5.945282in}{2.174750in}}%
\pgfpathlineto{\pgfqpoint{5.960715in}{2.164438in}}%
\pgfpathlineto{\pgfqpoint{5.976149in}{2.151431in}}%
\pgfpathlineto{\pgfqpoint{5.991583in}{2.135809in}}%
\pgfpathlineto{\pgfqpoint{6.007016in}{2.117704in}}%
\pgfpathlineto{\pgfqpoint{6.022450in}{2.097295in}}%
\pgfpathlineto{\pgfqpoint{6.053317in}{2.050514in}}%
\pgfpathlineto{\pgfqpoint{6.084185in}{1.997785in}}%
\pgfpathlineto{\pgfqpoint{6.161353in}{1.859897in}}%
\pgfpathlineto{\pgfqpoint{6.192220in}{1.811371in}}%
\pgfpathlineto{\pgfqpoint{6.207654in}{1.790289in}}%
\pgfpathlineto{\pgfqpoint{6.223087in}{1.771882in}}%
\pgfpathlineto{\pgfqpoint{6.238521in}{1.756536in}}%
\pgfpathlineto{\pgfqpoint{6.253955in}{1.744603in}}%
\pgfpathlineto{\pgfqpoint{6.269388in}{1.736399in}}%
\pgfpathlineto{\pgfqpoint{6.284822in}{1.732193in}}%
\pgfpathlineto{\pgfqpoint{6.300255in}{1.732205in}}%
\pgfpathlineto{\pgfqpoint{6.315689in}{1.736600in}}%
\pgfpathlineto{\pgfqpoint{6.331123in}{1.745482in}}%
\pgfpathlineto{\pgfqpoint{6.346556in}{1.758893in}}%
\pgfpathlineto{\pgfqpoint{6.361990in}{1.776807in}}%
\pgfpathlineto{\pgfqpoint{6.377424in}{1.799131in}}%
\pgfpathlineto{\pgfqpoint{6.392857in}{1.825705in}}%
\pgfpathlineto{\pgfqpoint{6.408291in}{1.856298in}}%
\pgfpathlineto{\pgfqpoint{6.423724in}{1.890616in}}%
\pgfpathlineto{\pgfqpoint{6.439158in}{1.928305in}}%
\pgfpathlineto{\pgfqpoint{6.470025in}{2.012089in}}%
\pgfpathlineto{\pgfqpoint{6.500893in}{2.103786in}}%
\pgfpathlineto{\pgfqpoint{6.578061in}{2.338155in}}%
\pgfpathlineto{\pgfqpoint{6.608928in}{2.421976in}}%
\pgfpathlineto{\pgfqpoint{6.624362in}{2.459685in}}%
\pgfpathlineto{\pgfqpoint{6.639795in}{2.494021in}}%
\pgfpathlineto{\pgfqpoint{6.655229in}{2.524625in}}%
\pgfpathlineto{\pgfqpoint{6.670663in}{2.551196in}}%
\pgfpathlineto{\pgfqpoint{6.686096in}{2.573495in}}%
\pgfpathlineto{\pgfqpoint{6.701530in}{2.591349in}}%
\pgfpathlineto{\pgfqpoint{6.716964in}{2.604648in}}%
\pgfpathlineto{\pgfqpoint{6.732397in}{2.613344in}}%
\pgfpathlineto{\pgfqpoint{6.747831in}{2.617450in}}%
\pgfpathlineto{\pgfqpoint{6.763264in}{2.617035in}}%
\pgfpathlineto{\pgfqpoint{6.778698in}{2.612220in}}%
\pgfpathlineto{\pgfqpoint{6.794132in}{2.603172in}}%
\pgfpathlineto{\pgfqpoint{6.809565in}{2.590099in}}%
\pgfpathlineto{\pgfqpoint{6.824999in}{2.573241in}}%
\pgfpathlineto{\pgfqpoint{6.840433in}{2.552869in}}%
\pgfpathlineto{\pgfqpoint{6.855866in}{2.529273in}}%
\pgfpathlineto{\pgfqpoint{6.871300in}{2.502760in}}%
\pgfpathlineto{\pgfqpoint{6.902167in}{2.442257in}}%
\pgfpathlineto{\pgfqpoint{6.933034in}{2.373944in}}%
\pgfpathlineto{\pgfqpoint{6.979335in}{2.262393in}}%
\pgfpathlineto{\pgfqpoint{7.056504in}{2.071988in}}%
\pgfpathlineto{\pgfqpoint{7.087371in}{2.000494in}}%
\pgfpathlineto{\pgfqpoint{7.118238in}{1.934485in}}%
\pgfpathlineto{\pgfqpoint{7.149105in}{1.875533in}}%
\pgfpathlineto{\pgfqpoint{7.164539in}{1.849112in}}%
\pgfpathlineto{\pgfqpoint{7.179973in}{1.824900in}}%
\pgfpathlineto{\pgfqpoint{7.195406in}{1.802988in}}%
\pgfpathlineto{\pgfqpoint{7.210840in}{1.783439in}}%
\pgfpathlineto{\pgfqpoint{7.226274in}{1.766279in}}%
\pgfpathlineto{\pgfqpoint{7.241707in}{1.751498in}}%
\pgfpathlineto{\pgfqpoint{7.257141in}{1.739048in}}%
\pgfpathlineto{\pgfqpoint{7.272574in}{1.728842in}}%
\pgfpathlineto{\pgfqpoint{7.288008in}{1.720753in}}%
\pgfpathlineto{\pgfqpoint{7.303442in}{1.714621in}}%
\pgfpathlineto{\pgfqpoint{7.318875in}{1.710249in}}%
\pgfpathlineto{\pgfqpoint{7.334309in}{1.707413in}}%
\pgfpathlineto{\pgfqpoint{7.365176in}{1.705324in}}%
\pgfpathlineto{\pgfqpoint{7.411477in}{1.707003in}}%
\pgfpathlineto{\pgfqpoint{7.442344in}{1.708117in}}%
\pgfpathlineto{\pgfqpoint{7.473212in}{1.706999in}}%
\pgfpathlineto{\pgfqpoint{7.504079in}{1.702301in}}%
\pgfpathlineto{\pgfqpoint{7.534946in}{1.693352in}}%
\pgfpathlineto{\pgfqpoint{7.565814in}{1.680233in}}%
\pgfpathlineto{\pgfqpoint{7.596681in}{1.663767in}}%
\pgfpathlineto{\pgfqpoint{7.673849in}{1.618810in}}%
\pgfpathlineto{\pgfqpoint{7.704716in}{1.604962in}}%
\pgfpathlineto{\pgfqpoint{7.720150in}{1.600057in}}%
\pgfpathlineto{\pgfqpoint{7.735584in}{1.596843in}}%
\pgfpathlineto{\pgfqpoint{7.751017in}{1.595542in}}%
\pgfpathlineto{\pgfqpoint{7.766451in}{1.596347in}}%
\pgfpathlineto{\pgfqpoint{7.781884in}{1.599407in}}%
\pgfpathlineto{\pgfqpoint{7.797318in}{1.604824in}}%
\pgfpathlineto{\pgfqpoint{7.812752in}{1.612652in}}%
\pgfpathlineto{\pgfqpoint{7.828185in}{1.622890in}}%
\pgfpathlineto{\pgfqpoint{7.843619in}{1.635482in}}%
\pgfpathlineto{\pgfqpoint{7.859053in}{1.650314in}}%
\pgfpathlineto{\pgfqpoint{7.874486in}{1.667218in}}%
\pgfpathlineto{\pgfqpoint{7.905353in}{1.706298in}}%
\pgfpathlineto{\pgfqpoint{7.936221in}{1.750375in}}%
\pgfpathlineto{\pgfqpoint{7.997955in}{1.841378in}}%
\pgfpathlineto{\pgfqpoint{8.028823in}{1.881709in}}%
\pgfpathlineto{\pgfqpoint{8.044256in}{1.899210in}}%
\pgfpathlineto{\pgfqpoint{8.059690in}{1.914491in}}%
\pgfpathlineto{\pgfqpoint{8.075123in}{1.927273in}}%
\pgfpathlineto{\pgfqpoint{8.090557in}{1.937334in}}%
\pgfpathlineto{\pgfqpoint{8.105991in}{1.944507in}}%
\pgfpathlineto{\pgfqpoint{8.121424in}{1.948688in}}%
\pgfpathlineto{\pgfqpoint{8.136858in}{1.949836in}}%
\pgfpathlineto{\pgfqpoint{8.152292in}{1.947969in}}%
\pgfpathlineto{\pgfqpoint{8.167725in}{1.943167in}}%
\pgfpathlineto{\pgfqpoint{8.183159in}{1.935561in}}%
\pgfpathlineto{\pgfqpoint{8.198593in}{1.925335in}}%
\pgfpathlineto{\pgfqpoint{8.214026in}{1.912711in}}%
\pgfpathlineto{\pgfqpoint{8.229460in}{1.897944in}}%
\pgfpathlineto{\pgfqpoint{8.260327in}{1.863120in}}%
\pgfpathlineto{\pgfqpoint{8.291194in}{1.823238in}}%
\pgfpathlineto{\pgfqpoint{8.414663in}{1.655565in}}%
\pgfpathlineto{\pgfqpoint{8.445531in}{1.618842in}}%
\pgfpathlineto{\pgfqpoint{8.476398in}{1.585552in}}%
\pgfpathlineto{\pgfqpoint{8.507265in}{1.555791in}}%
\pgfpathlineto{\pgfqpoint{8.538133in}{1.529521in}}%
\pgfpathlineto{\pgfqpoint{8.569000in}{1.506708in}}%
\pgfpathlineto{\pgfqpoint{8.599867in}{1.487428in}}%
\pgfpathlineto{\pgfqpoint{8.630734in}{1.471936in}}%
\pgfpathlineto{\pgfqpoint{8.661602in}{1.460683in}}%
\pgfpathlineto{\pgfqpoint{8.677035in}{1.456835in}}%
\pgfpathlineto{\pgfqpoint{8.692469in}{1.454288in}}%
\pgfpathlineto{\pgfqpoint{8.707903in}{1.453130in}}%
\pgfpathlineto{\pgfqpoint{8.723336in}{1.453449in}}%
\pgfpathlineto{\pgfqpoint{8.738770in}{1.455325in}}%
\pgfpathlineto{\pgfqpoint{8.754203in}{1.458827in}}%
\pgfpathlineto{\pgfqpoint{8.769637in}{1.464006in}}%
\pgfpathlineto{\pgfqpoint{8.785071in}{1.470895in}}%
\pgfpathlineto{\pgfqpoint{8.800504in}{1.479499in}}%
\pgfpathlineto{\pgfqpoint{8.815938in}{1.489798in}}%
\pgfpathlineto{\pgfqpoint{8.831372in}{1.501738in}}%
\pgfpathlineto{\pgfqpoint{8.862239in}{1.530169in}}%
\pgfpathlineto{\pgfqpoint{8.893106in}{1.563717in}}%
\pgfpathlineto{\pgfqpoint{8.923973in}{1.600830in}}%
\pgfpathlineto{\pgfqpoint{9.001142in}{1.696268in}}%
\pgfpathlineto{\pgfqpoint{9.032009in}{1.730166in}}%
\pgfpathlineto{\pgfqpoint{9.062876in}{1.759059in}}%
\pgfpathlineto{\pgfqpoint{9.078310in}{1.771279in}}%
\pgfpathlineto{\pgfqpoint{9.093743in}{1.781903in}}%
\pgfpathlineto{\pgfqpoint{9.109177in}{1.790900in}}%
\pgfpathlineto{\pgfqpoint{9.124611in}{1.798278in}}%
\pgfpathlineto{\pgfqpoint{9.140044in}{1.804084in}}%
\pgfpathlineto{\pgfqpoint{9.155478in}{1.808400in}}%
\pgfpathlineto{\pgfqpoint{9.170912in}{1.811342in}}%
\pgfpathlineto{\pgfqpoint{9.201779in}{1.813693in}}%
\pgfpathlineto{\pgfqpoint{9.232646in}{1.812509in}}%
\pgfpathlineto{\pgfqpoint{9.294381in}{1.805703in}}%
\pgfpathlineto{\pgfqpoint{9.325248in}{1.802994in}}%
\pgfpathlineto{\pgfqpoint{9.356115in}{1.802306in}}%
\pgfpathlineto{\pgfqpoint{9.386982in}{1.804390in}}%
\pgfpathlineto{\pgfqpoint{9.417850in}{1.809615in}}%
\pgfpathlineto{\pgfqpoint{9.448717in}{1.817984in}}%
\pgfpathlineto{\pgfqpoint{9.479584in}{1.829164in}}%
\pgfpathlineto{\pgfqpoint{9.510452in}{1.842558in}}%
\pgfpathlineto{\pgfqpoint{9.572186in}{1.872746in}}%
\pgfpathlineto{\pgfqpoint{9.618487in}{1.894888in}}%
\pgfpathlineto{\pgfqpoint{9.649354in}{1.907914in}}%
\pgfpathlineto{\pgfqpoint{9.680222in}{1.918867in}}%
\pgfpathlineto{\pgfqpoint{9.711089in}{1.927404in}}%
\pgfpathlineto{\pgfqpoint{9.741956in}{1.933410in}}%
\pgfpathlineto{\pgfqpoint{9.741956in}{1.933410in}}%
\pgfusepath{stroke}%
\end{pgfscope}%
\begin{pgfscope}%
\pgfsetrectcap%
\pgfsetmiterjoin%
\pgfsetlinewidth{0.803000pt}%
\definecolor{currentstroke}{rgb}{0.000000,0.000000,0.000000}%
\pgfsetstrokecolor{currentstroke}%
\pgfsetdash{}{0pt}%
\pgfpathmoveto{\pgfqpoint{5.706832in}{0.521603in}}%
\pgfpathlineto{\pgfqpoint{5.706832in}{3.321603in}}%
\pgfusepath{stroke}%
\end{pgfscope}%
\begin{pgfscope}%
\pgfsetrectcap%
\pgfsetmiterjoin%
\pgfsetlinewidth{0.803000pt}%
\definecolor{currentstroke}{rgb}{0.000000,0.000000,0.000000}%
\pgfsetstrokecolor{currentstroke}%
\pgfsetdash{}{0pt}%
\pgfpathmoveto{\pgfqpoint{9.934105in}{0.521603in}}%
\pgfpathlineto{\pgfqpoint{9.934105in}{3.321603in}}%
\pgfusepath{stroke}%
\end{pgfscope}%
\begin{pgfscope}%
\pgfsetrectcap%
\pgfsetmiterjoin%
\pgfsetlinewidth{0.803000pt}%
\definecolor{currentstroke}{rgb}{0.000000,0.000000,0.000000}%
\pgfsetstrokecolor{currentstroke}%
\pgfsetdash{}{0pt}%
\pgfpathmoveto{\pgfqpoint{5.706832in}{0.521603in}}%
\pgfpathlineto{\pgfqpoint{9.934105in}{0.521603in}}%
\pgfusepath{stroke}%
\end{pgfscope}%
\begin{pgfscope}%
\pgfsetrectcap%
\pgfsetmiterjoin%
\pgfsetlinewidth{0.803000pt}%
\definecolor{currentstroke}{rgb}{0.000000,0.000000,0.000000}%
\pgfsetstrokecolor{currentstroke}%
\pgfsetdash{}{0pt}%
\pgfpathmoveto{\pgfqpoint{5.706832in}{3.321603in}}%
\pgfpathlineto{\pgfqpoint{9.934105in}{3.321603in}}%
\pgfusepath{stroke}%
\end{pgfscope}%
\begin{pgfscope}%
\pgfsetbuttcap%
\pgfsetmiterjoin%
\definecolor{currentfill}{rgb}{1.000000,1.000000,1.000000}%
\pgfsetfillcolor{currentfill}%
\pgfsetfillopacity{0.800000}%
\pgfsetlinewidth{1.003750pt}%
\definecolor{currentstroke}{rgb}{0.800000,0.800000,0.800000}%
\pgfsetstrokecolor{currentstroke}%
\pgfsetstrokeopacity{0.800000}%
\pgfsetdash{}{0pt}%
\pgfpathmoveto{\pgfqpoint{9.288678in}{0.591048in}}%
\pgfpathlineto{\pgfqpoint{9.836883in}{0.591048in}}%
\pgfpathquadraticcurveto{\pgfqpoint{9.864660in}{0.591048in}}{\pgfqpoint{9.864660in}{0.618826in}}%
\pgfpathlineto{\pgfqpoint{9.864660in}{0.808794in}}%
\pgfpathquadraticcurveto{\pgfqpoint{9.864660in}{0.836572in}}{\pgfqpoint{9.836883in}{0.836572in}}%
\pgfpathlineto{\pgfqpoint{9.288678in}{0.836572in}}%
\pgfpathquadraticcurveto{\pgfqpoint{9.260901in}{0.836572in}}{\pgfqpoint{9.260901in}{0.808794in}}%
\pgfpathlineto{\pgfqpoint{9.260901in}{0.618826in}}%
\pgfpathquadraticcurveto{\pgfqpoint{9.260901in}{0.591048in}}{\pgfqpoint{9.288678in}{0.591048in}}%
\pgfpathclose%
\pgfusepath{stroke,fill}%
\end{pgfscope}%
\begin{pgfscope}%
\pgfsetrectcap%
\pgfsetroundjoin%
\pgfsetlinewidth{0.501875pt}%
\definecolor{currentstroke}{rgb}{0.000000,0.000000,0.000000}%
\pgfsetstrokecolor{currentstroke}%
\pgfsetdash{}{0pt}%
\pgfpathmoveto{\pgfqpoint{9.316456in}{0.724104in}}%
\pgfpathlineto{\pgfqpoint{9.594234in}{0.724104in}}%
\pgfusepath{stroke}%
\end{pgfscope}%
\begin{pgfscope}%
\pgftext[x=9.705345in,y=0.675493in,left,base]{\rmfamily\fontsize{10.000000}{12.000000}\selectfont K}%
\end{pgfscope}%
\begin{pgfscope}%
\pgfpathrectangle{\pgfqpoint{10.174105in}{0.441603in}}{\pgfqpoint{0.120000in}{6.400000in}} %
\pgfusepath{clip}%
\pgfsetbuttcap%
\pgfsetmiterjoin%
\definecolor{currentfill}{rgb}{1.000000,1.000000,1.000000}%
\pgfsetfillcolor{currentfill}%
\pgfsetlinewidth{0.010037pt}%
\definecolor{currentstroke}{rgb}{1.000000,1.000000,1.000000}%
\pgfsetstrokecolor{currentstroke}%
\pgfsetdash{}{0pt}%
\pgfpathmoveto{\pgfqpoint{10.174105in}{0.441603in}}%
\pgfpathlineto{\pgfqpoint{10.174105in}{0.466603in}}%
\pgfpathlineto{\pgfqpoint{10.174105in}{6.816603in}}%
\pgfpathlineto{\pgfqpoint{10.174105in}{6.841603in}}%
\pgfpathlineto{\pgfqpoint{10.294105in}{6.841603in}}%
\pgfpathlineto{\pgfqpoint{10.294105in}{6.816603in}}%
\pgfpathlineto{\pgfqpoint{10.294105in}{0.466603in}}%
\pgfpathlineto{\pgfqpoint{10.294105in}{0.441603in}}%
\pgfpathclose%
\pgfusepath{stroke,fill}%
\end{pgfscope}%
\begin{pgfscope}%
\pgfsys@transformshift{10.170000in}{0.441564in}%
\pgftext[left,bottom]{\pgfimage[interpolate=true,width=0.120000in,height=6.400000in]{not_Mercer-img0.png}}%
\end{pgfscope}%
\begin{pgfscope}%
\pgfsetbuttcap%
\pgfsetroundjoin%
\definecolor{currentfill}{rgb}{0.000000,0.000000,0.000000}%
\pgfsetfillcolor{currentfill}%
\pgfsetlinewidth{0.803000pt}%
\definecolor{currentstroke}{rgb}{0.000000,0.000000,0.000000}%
\pgfsetstrokecolor{currentstroke}%
\pgfsetdash{}{0pt}%
\pgfsys@defobject{currentmarker}{\pgfqpoint{0.000000in}{0.000000in}}{\pgfqpoint{0.048611in}{0.000000in}}{%
\pgfpathmoveto{\pgfqpoint{0.000000in}{0.000000in}}%
\pgfpathlineto{\pgfqpoint{0.048611in}{0.000000in}}%
\pgfusepath{stroke,fill}%
}%
\begin{pgfscope}%
\pgfsys@transformshift{10.294105in}{0.441603in}%
\pgfsys@useobject{currentmarker}{}%
\end{pgfscope}%
\end{pgfscope}%
\begin{pgfscope}%
\pgftext[x=10.391327in,y=0.388842in,left,base]{\rmfamily\fontsize{10.000000}{12.000000}\selectfont \(\displaystyle 10^{0}\)}%
\end{pgfscope}%
\begin{pgfscope}%
\pgfsetbuttcap%
\pgfsetroundjoin%
\definecolor{currentfill}{rgb}{0.000000,0.000000,0.000000}%
\pgfsetfillcolor{currentfill}%
\pgfsetlinewidth{0.803000pt}%
\definecolor{currentstroke}{rgb}{0.000000,0.000000,0.000000}%
\pgfsetstrokecolor{currentstroke}%
\pgfsetdash{}{0pt}%
\pgfsys@defobject{currentmarker}{\pgfqpoint{0.000000in}{0.000000in}}{\pgfqpoint{0.048611in}{0.000000in}}{%
\pgfpathmoveto{\pgfqpoint{0.000000in}{0.000000in}}%
\pgfpathlineto{\pgfqpoint{0.048611in}{0.000000in}}%
\pgfusepath{stroke,fill}%
}%
\begin{pgfscope}%
\pgfsys@transformshift{10.294105in}{0.923251in}%
\pgfsys@useobject{currentmarker}{}%
\end{pgfscope}%
\end{pgfscope}%
\begin{pgfscope}%
\pgfsetbuttcap%
\pgfsetroundjoin%
\definecolor{currentfill}{rgb}{0.000000,0.000000,0.000000}%
\pgfsetfillcolor{currentfill}%
\pgfsetlinewidth{0.803000pt}%
\definecolor{currentstroke}{rgb}{0.000000,0.000000,0.000000}%
\pgfsetstrokecolor{currentstroke}%
\pgfsetdash{}{0pt}%
\pgfsys@defobject{currentmarker}{\pgfqpoint{0.000000in}{0.000000in}}{\pgfqpoint{0.048611in}{0.000000in}}{%
\pgfpathmoveto{\pgfqpoint{0.000000in}{0.000000in}}%
\pgfpathlineto{\pgfqpoint{0.048611in}{0.000000in}}%
\pgfusepath{stroke,fill}%
}%
\begin{pgfscope}%
\pgfsys@transformshift{10.294105in}{1.204997in}%
\pgfsys@useobject{currentmarker}{}%
\end{pgfscope}%
\end{pgfscope}%
\begin{pgfscope}%
\pgfsetbuttcap%
\pgfsetroundjoin%
\definecolor{currentfill}{rgb}{0.000000,0.000000,0.000000}%
\pgfsetfillcolor{currentfill}%
\pgfsetlinewidth{0.803000pt}%
\definecolor{currentstroke}{rgb}{0.000000,0.000000,0.000000}%
\pgfsetstrokecolor{currentstroke}%
\pgfsetdash{}{0pt}%
\pgfsys@defobject{currentmarker}{\pgfqpoint{0.000000in}{0.000000in}}{\pgfqpoint{0.048611in}{0.000000in}}{%
\pgfpathmoveto{\pgfqpoint{0.000000in}{0.000000in}}%
\pgfpathlineto{\pgfqpoint{0.048611in}{0.000000in}}%
\pgfusepath{stroke,fill}%
}%
\begin{pgfscope}%
\pgfsys@transformshift{10.294105in}{1.404899in}%
\pgfsys@useobject{currentmarker}{}%
\end{pgfscope}%
\end{pgfscope}%
\begin{pgfscope}%
\pgfsetbuttcap%
\pgfsetroundjoin%
\definecolor{currentfill}{rgb}{0.000000,0.000000,0.000000}%
\pgfsetfillcolor{currentfill}%
\pgfsetlinewidth{0.803000pt}%
\definecolor{currentstroke}{rgb}{0.000000,0.000000,0.000000}%
\pgfsetstrokecolor{currentstroke}%
\pgfsetdash{}{0pt}%
\pgfsys@defobject{currentmarker}{\pgfqpoint{0.000000in}{0.000000in}}{\pgfqpoint{0.048611in}{0.000000in}}{%
\pgfpathmoveto{\pgfqpoint{0.000000in}{0.000000in}}%
\pgfpathlineto{\pgfqpoint{0.048611in}{0.000000in}}%
\pgfusepath{stroke,fill}%
}%
\begin{pgfscope}%
\pgfsys@transformshift{10.294105in}{1.559955in}%
\pgfsys@useobject{currentmarker}{}%
\end{pgfscope}%
\end{pgfscope}%
\begin{pgfscope}%
\pgfsetbuttcap%
\pgfsetroundjoin%
\definecolor{currentfill}{rgb}{0.000000,0.000000,0.000000}%
\pgfsetfillcolor{currentfill}%
\pgfsetlinewidth{0.803000pt}%
\definecolor{currentstroke}{rgb}{0.000000,0.000000,0.000000}%
\pgfsetstrokecolor{currentstroke}%
\pgfsetdash{}{0pt}%
\pgfsys@defobject{currentmarker}{\pgfqpoint{0.000000in}{0.000000in}}{\pgfqpoint{0.048611in}{0.000000in}}{%
\pgfpathmoveto{\pgfqpoint{0.000000in}{0.000000in}}%
\pgfpathlineto{\pgfqpoint{0.048611in}{0.000000in}}%
\pgfusepath{stroke,fill}%
}%
\begin{pgfscope}%
\pgfsys@transformshift{10.294105in}{1.686645in}%
\pgfsys@useobject{currentmarker}{}%
\end{pgfscope}%
\end{pgfscope}%
\begin{pgfscope}%
\pgfsetbuttcap%
\pgfsetroundjoin%
\definecolor{currentfill}{rgb}{0.000000,0.000000,0.000000}%
\pgfsetfillcolor{currentfill}%
\pgfsetlinewidth{0.803000pt}%
\definecolor{currentstroke}{rgb}{0.000000,0.000000,0.000000}%
\pgfsetstrokecolor{currentstroke}%
\pgfsetdash{}{0pt}%
\pgfsys@defobject{currentmarker}{\pgfqpoint{0.000000in}{0.000000in}}{\pgfqpoint{0.048611in}{0.000000in}}{%
\pgfpathmoveto{\pgfqpoint{0.000000in}{0.000000in}}%
\pgfpathlineto{\pgfqpoint{0.048611in}{0.000000in}}%
\pgfusepath{stroke,fill}%
}%
\begin{pgfscope}%
\pgfsys@transformshift{10.294105in}{1.793760in}%
\pgfsys@useobject{currentmarker}{}%
\end{pgfscope}%
\end{pgfscope}%
\begin{pgfscope}%
\pgfsetbuttcap%
\pgfsetroundjoin%
\definecolor{currentfill}{rgb}{0.000000,0.000000,0.000000}%
\pgfsetfillcolor{currentfill}%
\pgfsetlinewidth{0.803000pt}%
\definecolor{currentstroke}{rgb}{0.000000,0.000000,0.000000}%
\pgfsetstrokecolor{currentstroke}%
\pgfsetdash{}{0pt}%
\pgfsys@defobject{currentmarker}{\pgfqpoint{0.000000in}{0.000000in}}{\pgfqpoint{0.048611in}{0.000000in}}{%
\pgfpathmoveto{\pgfqpoint{0.000000in}{0.000000in}}%
\pgfpathlineto{\pgfqpoint{0.048611in}{0.000000in}}%
\pgfusepath{stroke,fill}%
}%
\begin{pgfscope}%
\pgfsys@transformshift{10.294105in}{1.886547in}%
\pgfsys@useobject{currentmarker}{}%
\end{pgfscope}%
\end{pgfscope}%
\begin{pgfscope}%
\pgfsetbuttcap%
\pgfsetroundjoin%
\definecolor{currentfill}{rgb}{0.000000,0.000000,0.000000}%
\pgfsetfillcolor{currentfill}%
\pgfsetlinewidth{0.803000pt}%
\definecolor{currentstroke}{rgb}{0.000000,0.000000,0.000000}%
\pgfsetstrokecolor{currentstroke}%
\pgfsetdash{}{0pt}%
\pgfsys@defobject{currentmarker}{\pgfqpoint{0.000000in}{0.000000in}}{\pgfqpoint{0.048611in}{0.000000in}}{%
\pgfpathmoveto{\pgfqpoint{0.000000in}{0.000000in}}%
\pgfpathlineto{\pgfqpoint{0.048611in}{0.000000in}}%
\pgfusepath{stroke,fill}%
}%
\begin{pgfscope}%
\pgfsys@transformshift{10.294105in}{1.968391in}%
\pgfsys@useobject{currentmarker}{}%
\end{pgfscope}%
\end{pgfscope}%
\begin{pgfscope}%
\pgfsetbuttcap%
\pgfsetroundjoin%
\definecolor{currentfill}{rgb}{0.000000,0.000000,0.000000}%
\pgfsetfillcolor{currentfill}%
\pgfsetlinewidth{0.803000pt}%
\definecolor{currentstroke}{rgb}{0.000000,0.000000,0.000000}%
\pgfsetstrokecolor{currentstroke}%
\pgfsetdash{}{0pt}%
\pgfsys@defobject{currentmarker}{\pgfqpoint{0.000000in}{0.000000in}}{\pgfqpoint{0.048611in}{0.000000in}}{%
\pgfpathmoveto{\pgfqpoint{0.000000in}{0.000000in}}%
\pgfpathlineto{\pgfqpoint{0.048611in}{0.000000in}}%
\pgfusepath{stroke,fill}%
}%
\begin{pgfscope}%
\pgfsys@transformshift{10.294105in}{2.041603in}%
\pgfsys@useobject{currentmarker}{}%
\end{pgfscope}%
\end{pgfscope}%
\begin{pgfscope}%
\pgftext[x=10.391327in,y=1.988842in,left,base]{\rmfamily\fontsize{10.000000}{12.000000}\selectfont \(\displaystyle 10^{1}\)}%
\end{pgfscope}%
\begin{pgfscope}%
\pgfsetbuttcap%
\pgfsetroundjoin%
\definecolor{currentfill}{rgb}{0.000000,0.000000,0.000000}%
\pgfsetfillcolor{currentfill}%
\pgfsetlinewidth{0.803000pt}%
\definecolor{currentstroke}{rgb}{0.000000,0.000000,0.000000}%
\pgfsetstrokecolor{currentstroke}%
\pgfsetdash{}{0pt}%
\pgfsys@defobject{currentmarker}{\pgfqpoint{0.000000in}{0.000000in}}{\pgfqpoint{0.048611in}{0.000000in}}{%
\pgfpathmoveto{\pgfqpoint{0.000000in}{0.000000in}}%
\pgfpathlineto{\pgfqpoint{0.048611in}{0.000000in}}%
\pgfusepath{stroke,fill}%
}%
\begin{pgfscope}%
\pgfsys@transformshift{10.294105in}{2.523251in}%
\pgfsys@useobject{currentmarker}{}%
\end{pgfscope}%
\end{pgfscope}%
\begin{pgfscope}%
\pgfsetbuttcap%
\pgfsetroundjoin%
\definecolor{currentfill}{rgb}{0.000000,0.000000,0.000000}%
\pgfsetfillcolor{currentfill}%
\pgfsetlinewidth{0.803000pt}%
\definecolor{currentstroke}{rgb}{0.000000,0.000000,0.000000}%
\pgfsetstrokecolor{currentstroke}%
\pgfsetdash{}{0pt}%
\pgfsys@defobject{currentmarker}{\pgfqpoint{0.000000in}{0.000000in}}{\pgfqpoint{0.048611in}{0.000000in}}{%
\pgfpathmoveto{\pgfqpoint{0.000000in}{0.000000in}}%
\pgfpathlineto{\pgfqpoint{0.048611in}{0.000000in}}%
\pgfusepath{stroke,fill}%
}%
\begin{pgfscope}%
\pgfsys@transformshift{10.294105in}{2.804997in}%
\pgfsys@useobject{currentmarker}{}%
\end{pgfscope}%
\end{pgfscope}%
\begin{pgfscope}%
\pgfsetbuttcap%
\pgfsetroundjoin%
\definecolor{currentfill}{rgb}{0.000000,0.000000,0.000000}%
\pgfsetfillcolor{currentfill}%
\pgfsetlinewidth{0.803000pt}%
\definecolor{currentstroke}{rgb}{0.000000,0.000000,0.000000}%
\pgfsetstrokecolor{currentstroke}%
\pgfsetdash{}{0pt}%
\pgfsys@defobject{currentmarker}{\pgfqpoint{0.000000in}{0.000000in}}{\pgfqpoint{0.048611in}{0.000000in}}{%
\pgfpathmoveto{\pgfqpoint{0.000000in}{0.000000in}}%
\pgfpathlineto{\pgfqpoint{0.048611in}{0.000000in}}%
\pgfusepath{stroke,fill}%
}%
\begin{pgfscope}%
\pgfsys@transformshift{10.294105in}{3.004899in}%
\pgfsys@useobject{currentmarker}{}%
\end{pgfscope}%
\end{pgfscope}%
\begin{pgfscope}%
\pgfsetbuttcap%
\pgfsetroundjoin%
\definecolor{currentfill}{rgb}{0.000000,0.000000,0.000000}%
\pgfsetfillcolor{currentfill}%
\pgfsetlinewidth{0.803000pt}%
\definecolor{currentstroke}{rgb}{0.000000,0.000000,0.000000}%
\pgfsetstrokecolor{currentstroke}%
\pgfsetdash{}{0pt}%
\pgfsys@defobject{currentmarker}{\pgfqpoint{0.000000in}{0.000000in}}{\pgfqpoint{0.048611in}{0.000000in}}{%
\pgfpathmoveto{\pgfqpoint{0.000000in}{0.000000in}}%
\pgfpathlineto{\pgfqpoint{0.048611in}{0.000000in}}%
\pgfusepath{stroke,fill}%
}%
\begin{pgfscope}%
\pgfsys@transformshift{10.294105in}{3.159955in}%
\pgfsys@useobject{currentmarker}{}%
\end{pgfscope}%
\end{pgfscope}%
\begin{pgfscope}%
\pgfsetbuttcap%
\pgfsetroundjoin%
\definecolor{currentfill}{rgb}{0.000000,0.000000,0.000000}%
\pgfsetfillcolor{currentfill}%
\pgfsetlinewidth{0.803000pt}%
\definecolor{currentstroke}{rgb}{0.000000,0.000000,0.000000}%
\pgfsetstrokecolor{currentstroke}%
\pgfsetdash{}{0pt}%
\pgfsys@defobject{currentmarker}{\pgfqpoint{0.000000in}{0.000000in}}{\pgfqpoint{0.048611in}{0.000000in}}{%
\pgfpathmoveto{\pgfqpoint{0.000000in}{0.000000in}}%
\pgfpathlineto{\pgfqpoint{0.048611in}{0.000000in}}%
\pgfusepath{stroke,fill}%
}%
\begin{pgfscope}%
\pgfsys@transformshift{10.294105in}{3.286645in}%
\pgfsys@useobject{currentmarker}{}%
\end{pgfscope}%
\end{pgfscope}%
\begin{pgfscope}%
\pgfsetbuttcap%
\pgfsetroundjoin%
\definecolor{currentfill}{rgb}{0.000000,0.000000,0.000000}%
\pgfsetfillcolor{currentfill}%
\pgfsetlinewidth{0.803000pt}%
\definecolor{currentstroke}{rgb}{0.000000,0.000000,0.000000}%
\pgfsetstrokecolor{currentstroke}%
\pgfsetdash{}{0pt}%
\pgfsys@defobject{currentmarker}{\pgfqpoint{0.000000in}{0.000000in}}{\pgfqpoint{0.048611in}{0.000000in}}{%
\pgfpathmoveto{\pgfqpoint{0.000000in}{0.000000in}}%
\pgfpathlineto{\pgfqpoint{0.048611in}{0.000000in}}%
\pgfusepath{stroke,fill}%
}%
\begin{pgfscope}%
\pgfsys@transformshift{10.294105in}{3.393760in}%
\pgfsys@useobject{currentmarker}{}%
\end{pgfscope}%
\end{pgfscope}%
\begin{pgfscope}%
\pgfsetbuttcap%
\pgfsetroundjoin%
\definecolor{currentfill}{rgb}{0.000000,0.000000,0.000000}%
\pgfsetfillcolor{currentfill}%
\pgfsetlinewidth{0.803000pt}%
\definecolor{currentstroke}{rgb}{0.000000,0.000000,0.000000}%
\pgfsetstrokecolor{currentstroke}%
\pgfsetdash{}{0pt}%
\pgfsys@defobject{currentmarker}{\pgfqpoint{0.000000in}{0.000000in}}{\pgfqpoint{0.048611in}{0.000000in}}{%
\pgfpathmoveto{\pgfqpoint{0.000000in}{0.000000in}}%
\pgfpathlineto{\pgfqpoint{0.048611in}{0.000000in}}%
\pgfusepath{stroke,fill}%
}%
\begin{pgfscope}%
\pgfsys@transformshift{10.294105in}{3.486547in}%
\pgfsys@useobject{currentmarker}{}%
\end{pgfscope}%
\end{pgfscope}%
\begin{pgfscope}%
\pgfsetbuttcap%
\pgfsetroundjoin%
\definecolor{currentfill}{rgb}{0.000000,0.000000,0.000000}%
\pgfsetfillcolor{currentfill}%
\pgfsetlinewidth{0.803000pt}%
\definecolor{currentstroke}{rgb}{0.000000,0.000000,0.000000}%
\pgfsetstrokecolor{currentstroke}%
\pgfsetdash{}{0pt}%
\pgfsys@defobject{currentmarker}{\pgfqpoint{0.000000in}{0.000000in}}{\pgfqpoint{0.048611in}{0.000000in}}{%
\pgfpathmoveto{\pgfqpoint{0.000000in}{0.000000in}}%
\pgfpathlineto{\pgfqpoint{0.048611in}{0.000000in}}%
\pgfusepath{stroke,fill}%
}%
\begin{pgfscope}%
\pgfsys@transformshift{10.294105in}{3.568391in}%
\pgfsys@useobject{currentmarker}{}%
\end{pgfscope}%
\end{pgfscope}%
\begin{pgfscope}%
\pgfsetbuttcap%
\pgfsetroundjoin%
\definecolor{currentfill}{rgb}{0.000000,0.000000,0.000000}%
\pgfsetfillcolor{currentfill}%
\pgfsetlinewidth{0.803000pt}%
\definecolor{currentstroke}{rgb}{0.000000,0.000000,0.000000}%
\pgfsetstrokecolor{currentstroke}%
\pgfsetdash{}{0pt}%
\pgfsys@defobject{currentmarker}{\pgfqpoint{0.000000in}{0.000000in}}{\pgfqpoint{0.048611in}{0.000000in}}{%
\pgfpathmoveto{\pgfqpoint{0.000000in}{0.000000in}}%
\pgfpathlineto{\pgfqpoint{0.048611in}{0.000000in}}%
\pgfusepath{stroke,fill}%
}%
\begin{pgfscope}%
\pgfsys@transformshift{10.294105in}{3.641603in}%
\pgfsys@useobject{currentmarker}{}%
\end{pgfscope}%
\end{pgfscope}%
\begin{pgfscope}%
\pgftext[x=10.391327in,y=3.588842in,left,base]{\rmfamily\fontsize{10.000000}{12.000000}\selectfont \(\displaystyle 10^{2}\)}%
\end{pgfscope}%
\begin{pgfscope}%
\pgfsetbuttcap%
\pgfsetroundjoin%
\definecolor{currentfill}{rgb}{0.000000,0.000000,0.000000}%
\pgfsetfillcolor{currentfill}%
\pgfsetlinewidth{0.803000pt}%
\definecolor{currentstroke}{rgb}{0.000000,0.000000,0.000000}%
\pgfsetstrokecolor{currentstroke}%
\pgfsetdash{}{0pt}%
\pgfsys@defobject{currentmarker}{\pgfqpoint{0.000000in}{0.000000in}}{\pgfqpoint{0.048611in}{0.000000in}}{%
\pgfpathmoveto{\pgfqpoint{0.000000in}{0.000000in}}%
\pgfpathlineto{\pgfqpoint{0.048611in}{0.000000in}}%
\pgfusepath{stroke,fill}%
}%
\begin{pgfscope}%
\pgfsys@transformshift{10.294105in}{4.123251in}%
\pgfsys@useobject{currentmarker}{}%
\end{pgfscope}%
\end{pgfscope}%
\begin{pgfscope}%
\pgfsetbuttcap%
\pgfsetroundjoin%
\definecolor{currentfill}{rgb}{0.000000,0.000000,0.000000}%
\pgfsetfillcolor{currentfill}%
\pgfsetlinewidth{0.803000pt}%
\definecolor{currentstroke}{rgb}{0.000000,0.000000,0.000000}%
\pgfsetstrokecolor{currentstroke}%
\pgfsetdash{}{0pt}%
\pgfsys@defobject{currentmarker}{\pgfqpoint{0.000000in}{0.000000in}}{\pgfqpoint{0.048611in}{0.000000in}}{%
\pgfpathmoveto{\pgfqpoint{0.000000in}{0.000000in}}%
\pgfpathlineto{\pgfqpoint{0.048611in}{0.000000in}}%
\pgfusepath{stroke,fill}%
}%
\begin{pgfscope}%
\pgfsys@transformshift{10.294105in}{4.404997in}%
\pgfsys@useobject{currentmarker}{}%
\end{pgfscope}%
\end{pgfscope}%
\begin{pgfscope}%
\pgfsetbuttcap%
\pgfsetroundjoin%
\definecolor{currentfill}{rgb}{0.000000,0.000000,0.000000}%
\pgfsetfillcolor{currentfill}%
\pgfsetlinewidth{0.803000pt}%
\definecolor{currentstroke}{rgb}{0.000000,0.000000,0.000000}%
\pgfsetstrokecolor{currentstroke}%
\pgfsetdash{}{0pt}%
\pgfsys@defobject{currentmarker}{\pgfqpoint{0.000000in}{0.000000in}}{\pgfqpoint{0.048611in}{0.000000in}}{%
\pgfpathmoveto{\pgfqpoint{0.000000in}{0.000000in}}%
\pgfpathlineto{\pgfqpoint{0.048611in}{0.000000in}}%
\pgfusepath{stroke,fill}%
}%
\begin{pgfscope}%
\pgfsys@transformshift{10.294105in}{4.604899in}%
\pgfsys@useobject{currentmarker}{}%
\end{pgfscope}%
\end{pgfscope}%
\begin{pgfscope}%
\pgfsetbuttcap%
\pgfsetroundjoin%
\definecolor{currentfill}{rgb}{0.000000,0.000000,0.000000}%
\pgfsetfillcolor{currentfill}%
\pgfsetlinewidth{0.803000pt}%
\definecolor{currentstroke}{rgb}{0.000000,0.000000,0.000000}%
\pgfsetstrokecolor{currentstroke}%
\pgfsetdash{}{0pt}%
\pgfsys@defobject{currentmarker}{\pgfqpoint{0.000000in}{0.000000in}}{\pgfqpoint{0.048611in}{0.000000in}}{%
\pgfpathmoveto{\pgfqpoint{0.000000in}{0.000000in}}%
\pgfpathlineto{\pgfqpoint{0.048611in}{0.000000in}}%
\pgfusepath{stroke,fill}%
}%
\begin{pgfscope}%
\pgfsys@transformshift{10.294105in}{4.759955in}%
\pgfsys@useobject{currentmarker}{}%
\end{pgfscope}%
\end{pgfscope}%
\begin{pgfscope}%
\pgfsetbuttcap%
\pgfsetroundjoin%
\definecolor{currentfill}{rgb}{0.000000,0.000000,0.000000}%
\pgfsetfillcolor{currentfill}%
\pgfsetlinewidth{0.803000pt}%
\definecolor{currentstroke}{rgb}{0.000000,0.000000,0.000000}%
\pgfsetstrokecolor{currentstroke}%
\pgfsetdash{}{0pt}%
\pgfsys@defobject{currentmarker}{\pgfqpoint{0.000000in}{0.000000in}}{\pgfqpoint{0.048611in}{0.000000in}}{%
\pgfpathmoveto{\pgfqpoint{0.000000in}{0.000000in}}%
\pgfpathlineto{\pgfqpoint{0.048611in}{0.000000in}}%
\pgfusepath{stroke,fill}%
}%
\begin{pgfscope}%
\pgfsys@transformshift{10.294105in}{4.886645in}%
\pgfsys@useobject{currentmarker}{}%
\end{pgfscope}%
\end{pgfscope}%
\begin{pgfscope}%
\pgfsetbuttcap%
\pgfsetroundjoin%
\definecolor{currentfill}{rgb}{0.000000,0.000000,0.000000}%
\pgfsetfillcolor{currentfill}%
\pgfsetlinewidth{0.803000pt}%
\definecolor{currentstroke}{rgb}{0.000000,0.000000,0.000000}%
\pgfsetstrokecolor{currentstroke}%
\pgfsetdash{}{0pt}%
\pgfsys@defobject{currentmarker}{\pgfqpoint{0.000000in}{0.000000in}}{\pgfqpoint{0.048611in}{0.000000in}}{%
\pgfpathmoveto{\pgfqpoint{0.000000in}{0.000000in}}%
\pgfpathlineto{\pgfqpoint{0.048611in}{0.000000in}}%
\pgfusepath{stroke,fill}%
}%
\begin{pgfscope}%
\pgfsys@transformshift{10.294105in}{4.993760in}%
\pgfsys@useobject{currentmarker}{}%
\end{pgfscope}%
\end{pgfscope}%
\begin{pgfscope}%
\pgfsetbuttcap%
\pgfsetroundjoin%
\definecolor{currentfill}{rgb}{0.000000,0.000000,0.000000}%
\pgfsetfillcolor{currentfill}%
\pgfsetlinewidth{0.803000pt}%
\definecolor{currentstroke}{rgb}{0.000000,0.000000,0.000000}%
\pgfsetstrokecolor{currentstroke}%
\pgfsetdash{}{0pt}%
\pgfsys@defobject{currentmarker}{\pgfqpoint{0.000000in}{0.000000in}}{\pgfqpoint{0.048611in}{0.000000in}}{%
\pgfpathmoveto{\pgfqpoint{0.000000in}{0.000000in}}%
\pgfpathlineto{\pgfqpoint{0.048611in}{0.000000in}}%
\pgfusepath{stroke,fill}%
}%
\begin{pgfscope}%
\pgfsys@transformshift{10.294105in}{5.086547in}%
\pgfsys@useobject{currentmarker}{}%
\end{pgfscope}%
\end{pgfscope}%
\begin{pgfscope}%
\pgfsetbuttcap%
\pgfsetroundjoin%
\definecolor{currentfill}{rgb}{0.000000,0.000000,0.000000}%
\pgfsetfillcolor{currentfill}%
\pgfsetlinewidth{0.803000pt}%
\definecolor{currentstroke}{rgb}{0.000000,0.000000,0.000000}%
\pgfsetstrokecolor{currentstroke}%
\pgfsetdash{}{0pt}%
\pgfsys@defobject{currentmarker}{\pgfqpoint{0.000000in}{0.000000in}}{\pgfqpoint{0.048611in}{0.000000in}}{%
\pgfpathmoveto{\pgfqpoint{0.000000in}{0.000000in}}%
\pgfpathlineto{\pgfqpoint{0.048611in}{0.000000in}}%
\pgfusepath{stroke,fill}%
}%
\begin{pgfscope}%
\pgfsys@transformshift{10.294105in}{5.168391in}%
\pgfsys@useobject{currentmarker}{}%
\end{pgfscope}%
\end{pgfscope}%
\begin{pgfscope}%
\pgfsetbuttcap%
\pgfsetroundjoin%
\definecolor{currentfill}{rgb}{0.000000,0.000000,0.000000}%
\pgfsetfillcolor{currentfill}%
\pgfsetlinewidth{0.803000pt}%
\definecolor{currentstroke}{rgb}{0.000000,0.000000,0.000000}%
\pgfsetstrokecolor{currentstroke}%
\pgfsetdash{}{0pt}%
\pgfsys@defobject{currentmarker}{\pgfqpoint{0.000000in}{0.000000in}}{\pgfqpoint{0.048611in}{0.000000in}}{%
\pgfpathmoveto{\pgfqpoint{0.000000in}{0.000000in}}%
\pgfpathlineto{\pgfqpoint{0.048611in}{0.000000in}}%
\pgfusepath{stroke,fill}%
}%
\begin{pgfscope}%
\pgfsys@transformshift{10.294105in}{5.241603in}%
\pgfsys@useobject{currentmarker}{}%
\end{pgfscope}%
\end{pgfscope}%
\begin{pgfscope}%
\pgftext[x=10.391327in,y=5.188842in,left,base]{\rmfamily\fontsize{10.000000}{12.000000}\selectfont \(\displaystyle 10^{3}\)}%
\end{pgfscope}%
\begin{pgfscope}%
\pgfsetbuttcap%
\pgfsetroundjoin%
\definecolor{currentfill}{rgb}{0.000000,0.000000,0.000000}%
\pgfsetfillcolor{currentfill}%
\pgfsetlinewidth{0.803000pt}%
\definecolor{currentstroke}{rgb}{0.000000,0.000000,0.000000}%
\pgfsetstrokecolor{currentstroke}%
\pgfsetdash{}{0pt}%
\pgfsys@defobject{currentmarker}{\pgfqpoint{0.000000in}{0.000000in}}{\pgfqpoint{0.048611in}{0.000000in}}{%
\pgfpathmoveto{\pgfqpoint{0.000000in}{0.000000in}}%
\pgfpathlineto{\pgfqpoint{0.048611in}{0.000000in}}%
\pgfusepath{stroke,fill}%
}%
\begin{pgfscope}%
\pgfsys@transformshift{10.294105in}{5.723251in}%
\pgfsys@useobject{currentmarker}{}%
\end{pgfscope}%
\end{pgfscope}%
\begin{pgfscope}%
\pgfsetbuttcap%
\pgfsetroundjoin%
\definecolor{currentfill}{rgb}{0.000000,0.000000,0.000000}%
\pgfsetfillcolor{currentfill}%
\pgfsetlinewidth{0.803000pt}%
\definecolor{currentstroke}{rgb}{0.000000,0.000000,0.000000}%
\pgfsetstrokecolor{currentstroke}%
\pgfsetdash{}{0pt}%
\pgfsys@defobject{currentmarker}{\pgfqpoint{0.000000in}{0.000000in}}{\pgfqpoint{0.048611in}{0.000000in}}{%
\pgfpathmoveto{\pgfqpoint{0.000000in}{0.000000in}}%
\pgfpathlineto{\pgfqpoint{0.048611in}{0.000000in}}%
\pgfusepath{stroke,fill}%
}%
\begin{pgfscope}%
\pgfsys@transformshift{10.294105in}{6.004997in}%
\pgfsys@useobject{currentmarker}{}%
\end{pgfscope}%
\end{pgfscope}%
\begin{pgfscope}%
\pgfsetbuttcap%
\pgfsetroundjoin%
\definecolor{currentfill}{rgb}{0.000000,0.000000,0.000000}%
\pgfsetfillcolor{currentfill}%
\pgfsetlinewidth{0.803000pt}%
\definecolor{currentstroke}{rgb}{0.000000,0.000000,0.000000}%
\pgfsetstrokecolor{currentstroke}%
\pgfsetdash{}{0pt}%
\pgfsys@defobject{currentmarker}{\pgfqpoint{0.000000in}{0.000000in}}{\pgfqpoint{0.048611in}{0.000000in}}{%
\pgfpathmoveto{\pgfqpoint{0.000000in}{0.000000in}}%
\pgfpathlineto{\pgfqpoint{0.048611in}{0.000000in}}%
\pgfusepath{stroke,fill}%
}%
\begin{pgfscope}%
\pgfsys@transformshift{10.294105in}{6.204899in}%
\pgfsys@useobject{currentmarker}{}%
\end{pgfscope}%
\end{pgfscope}%
\begin{pgfscope}%
\pgfsetbuttcap%
\pgfsetroundjoin%
\definecolor{currentfill}{rgb}{0.000000,0.000000,0.000000}%
\pgfsetfillcolor{currentfill}%
\pgfsetlinewidth{0.803000pt}%
\definecolor{currentstroke}{rgb}{0.000000,0.000000,0.000000}%
\pgfsetstrokecolor{currentstroke}%
\pgfsetdash{}{0pt}%
\pgfsys@defobject{currentmarker}{\pgfqpoint{0.000000in}{0.000000in}}{\pgfqpoint{0.048611in}{0.000000in}}{%
\pgfpathmoveto{\pgfqpoint{0.000000in}{0.000000in}}%
\pgfpathlineto{\pgfqpoint{0.048611in}{0.000000in}}%
\pgfusepath{stroke,fill}%
}%
\begin{pgfscope}%
\pgfsys@transformshift{10.294105in}{6.359955in}%
\pgfsys@useobject{currentmarker}{}%
\end{pgfscope}%
\end{pgfscope}%
\begin{pgfscope}%
\pgfsetbuttcap%
\pgfsetroundjoin%
\definecolor{currentfill}{rgb}{0.000000,0.000000,0.000000}%
\pgfsetfillcolor{currentfill}%
\pgfsetlinewidth{0.803000pt}%
\definecolor{currentstroke}{rgb}{0.000000,0.000000,0.000000}%
\pgfsetstrokecolor{currentstroke}%
\pgfsetdash{}{0pt}%
\pgfsys@defobject{currentmarker}{\pgfqpoint{0.000000in}{0.000000in}}{\pgfqpoint{0.048611in}{0.000000in}}{%
\pgfpathmoveto{\pgfqpoint{0.000000in}{0.000000in}}%
\pgfpathlineto{\pgfqpoint{0.048611in}{0.000000in}}%
\pgfusepath{stroke,fill}%
}%
\begin{pgfscope}%
\pgfsys@transformshift{10.294105in}{6.486645in}%
\pgfsys@useobject{currentmarker}{}%
\end{pgfscope}%
\end{pgfscope}%
\begin{pgfscope}%
\pgfsetbuttcap%
\pgfsetroundjoin%
\definecolor{currentfill}{rgb}{0.000000,0.000000,0.000000}%
\pgfsetfillcolor{currentfill}%
\pgfsetlinewidth{0.803000pt}%
\definecolor{currentstroke}{rgb}{0.000000,0.000000,0.000000}%
\pgfsetstrokecolor{currentstroke}%
\pgfsetdash{}{0pt}%
\pgfsys@defobject{currentmarker}{\pgfqpoint{0.000000in}{0.000000in}}{\pgfqpoint{0.048611in}{0.000000in}}{%
\pgfpathmoveto{\pgfqpoint{0.000000in}{0.000000in}}%
\pgfpathlineto{\pgfqpoint{0.048611in}{0.000000in}}%
\pgfusepath{stroke,fill}%
}%
\begin{pgfscope}%
\pgfsys@transformshift{10.294105in}{6.593760in}%
\pgfsys@useobject{currentmarker}{}%
\end{pgfscope}%
\end{pgfscope}%
\begin{pgfscope}%
\pgfsetbuttcap%
\pgfsetroundjoin%
\definecolor{currentfill}{rgb}{0.000000,0.000000,0.000000}%
\pgfsetfillcolor{currentfill}%
\pgfsetlinewidth{0.803000pt}%
\definecolor{currentstroke}{rgb}{0.000000,0.000000,0.000000}%
\pgfsetstrokecolor{currentstroke}%
\pgfsetdash{}{0pt}%
\pgfsys@defobject{currentmarker}{\pgfqpoint{0.000000in}{0.000000in}}{\pgfqpoint{0.048611in}{0.000000in}}{%
\pgfpathmoveto{\pgfqpoint{0.000000in}{0.000000in}}%
\pgfpathlineto{\pgfqpoint{0.048611in}{0.000000in}}%
\pgfusepath{stroke,fill}%
}%
\begin{pgfscope}%
\pgfsys@transformshift{10.294105in}{6.686547in}%
\pgfsys@useobject{currentmarker}{}%
\end{pgfscope}%
\end{pgfscope}%
\begin{pgfscope}%
\pgfsetbuttcap%
\pgfsetroundjoin%
\definecolor{currentfill}{rgb}{0.000000,0.000000,0.000000}%
\pgfsetfillcolor{currentfill}%
\pgfsetlinewidth{0.803000pt}%
\definecolor{currentstroke}{rgb}{0.000000,0.000000,0.000000}%
\pgfsetstrokecolor{currentstroke}%
\pgfsetdash{}{0pt}%
\pgfsys@defobject{currentmarker}{\pgfqpoint{0.000000in}{0.000000in}}{\pgfqpoint{0.048611in}{0.000000in}}{%
\pgfpathmoveto{\pgfqpoint{0.000000in}{0.000000in}}%
\pgfpathlineto{\pgfqpoint{0.048611in}{0.000000in}}%
\pgfusepath{stroke,fill}%
}%
\begin{pgfscope}%
\pgfsys@transformshift{10.294105in}{6.768391in}%
\pgfsys@useobject{currentmarker}{}%
\end{pgfscope}%
\end{pgfscope}%
\begin{pgfscope}%
\pgfsetbuttcap%
\pgfsetroundjoin%
\definecolor{currentfill}{rgb}{0.000000,0.000000,0.000000}%
\pgfsetfillcolor{currentfill}%
\pgfsetlinewidth{0.803000pt}%
\definecolor{currentstroke}{rgb}{0.000000,0.000000,0.000000}%
\pgfsetstrokecolor{currentstroke}%
\pgfsetdash{}{0pt}%
\pgfsys@defobject{currentmarker}{\pgfqpoint{0.000000in}{0.000000in}}{\pgfqpoint{0.048611in}{0.000000in}}{%
\pgfpathmoveto{\pgfqpoint{0.000000in}{0.000000in}}%
\pgfpathlineto{\pgfqpoint{0.048611in}{0.000000in}}%
\pgfusepath{stroke,fill}%
}%
\begin{pgfscope}%
\pgfsys@transformshift{10.294105in}{6.841603in}%
\pgfsys@useobject{currentmarker}{}%
\end{pgfscope}%
\end{pgfscope}%
\begin{pgfscope}%
\pgftext[x=10.391327in,y=6.788842in,left,base]{\rmfamily\fontsize{10.000000}{12.000000}\selectfont \(\displaystyle 10^{4}\)}%
\end{pgfscope}%
\begin{pgfscope}%
\pgftext[x=10.648079in,y=3.641603in,,top,rotate=90.000000]{\rmfamily\fontsize{10.000000}{12.000000}\selectfont D=}%
\end{pgfscope}%
\begin{pgfscope}%
\pgfsetbuttcap%
\pgfsetmiterjoin%
\pgfsetlinewidth{0.803000pt}%
\definecolor{currentstroke}{rgb}{0.000000,0.000000,0.000000}%
\pgfsetstrokecolor{currentstroke}%
\pgfsetdash{}{0pt}%
\pgfpathmoveto{\pgfqpoint{10.174105in}{0.441603in}}%
\pgfpathlineto{\pgfqpoint{10.174105in}{0.466603in}}%
\pgfpathlineto{\pgfqpoint{10.174105in}{6.816603in}}%
\pgfpathlineto{\pgfqpoint{10.174105in}{6.841603in}}%
\pgfpathlineto{\pgfqpoint{10.294105in}{6.841603in}}%
\pgfpathlineto{\pgfqpoint{10.294105in}{6.816603in}}%
\pgfpathlineto{\pgfqpoint{10.294105in}{0.466603in}}%
\pgfpathlineto{\pgfqpoint{10.294105in}{0.441603in}}%
\pgfpathclose%
\pgfusepath{stroke}%
\end{pgfscope}%
\begin{pgfscope}%
\pgftext[x=10.234105in,y=6.924937in,,base]{\rmfamily\fontsize{12.000000}{14.400000}\selectfont \(\displaystyle \widetilde{K}\)}%
\end{pgfscope}%
\end{pgfpicture}%
\makeatother%
\endgroup%
}')}
\caption[Different outcomes of a Gaussian kernel approximation]{Different outcomes of a Gaussian kernel approximation. Top row and bottom row correspond to two different outcomes of $\tildeK{\omega}$, which are \emph{different} \acl{OVK}. However when $D$ tends to infinity, the different outcomes fo $\tildeK{\omega}$ yield the samel \acs{OVK}.}
\label{fig:not_Mercer}
\end{figure}

We computed the Gram Matrix of the Gaussian decomposable kernel
\begin{dmath*}
K(x,z)_{ij}=\exp\left(-\frac{1}{2(0.1)^2(x_i - x_j)^2}\right)\Gamma \condition{for $i$, $j\in\mathbb{N}_{250}$.}
\end{dmath*}
We computed a reference function (black line) defined as $(y_1, y_2)^T = f(x_i)=\sum_{j=1}^{250}K(x_i,x_j)u_j$ where $u_j\sim\mathcal{N}(0,1)$ \iid. We took $\Gamma=.5 I_2 + .5 1_2$ such that the outputs $y_1$ and $y_2$ share some similarities. Then we computed an approximate kernel matrix $\tildeK{\omega}\approx K$ for $25$ increasing values of $D$ ranging from $1$ to $10^4$. The two graphs on the top row shows that the more the number of features increase the closer the model $\widetilde{f}(x_i)=\sum_{j=1}^{250}\tildeK{\omega}(x_i,x_j)u_j$ is to $f$. The bottom row shows the same experiment but for a different realization of $\tildeK{\omega}$. When $D$ is small the curves of the bottom and top rows are very dissimilar --and sine wave like-- while they both converge to $f$ when $D$ increase.
\paragraph{}
In the same way we defined an \acs{ORFF}, we can define an approximate feature operator $\tildeW{\omega}$ which maps $\tildeH{\omega}$ onto $\mathcal{H}_{\tildeK{\omega}}$, where \begin{dmath*}
\tildeK{\omega}(x,z)=\tildePhi{\omega}(x)^\adjoint\tildePhi{\omega}(z)
\end{dmath*}
for all $x$, $z\in\mathcal{X}$.
\begin{definition}[Random Fourier feature operator] Let $\seq{\omega}=(\omega_j)_{j=1}^D\in\dual{\mathcal{X}}^D$ and let
\begin{dmath*}
\tildeK{\omega}_e=\frac{1}{D}\sum_{j=1}^D \conj{\pairing{\cdot,\omega_j}}B(\omega_j)B(\omega_j)^*.
\end{dmath*}
We call random Fourier feature operator the linear application $\tildeW{\omega}:\tildeH{\omega}\to \mathcal{H}_{\tildeK{\omega}}$ defined as
\begin{dmath*}
\left(\tildeW{\omega} \theta\right)(x) \colonequals \tildePhi{\omega}(x)^\adjoint \theta =\frac{1}{D}\sum_{j=1}^D \conj{\pairing{x,\omega_j}}B(\omega_j)g(\omega_j)
\end{dmath*}
where $\theta=\frac{1}{\sqrt{D}}\Vect_{j=1}^Dg(\omega_j)\in\tildeH{\omega}$. Then,
\begin{dmath*}
\left(\Ker \tildeW{\omega}\right)^\perp = \lspan\Set{\tildePhi{\omega}(x)y | \forall x\in\mathcal{X},\enskip \forall y\in\mathcal{Y}} \hiderel{\subseteq} \tildeH{\omega}.
\end{dmath*}
\end{definition}
The random Fourier feature operator is useful to show the relations between the random Fourier feature map with the functional feature map defined in \cref{pr:fourier_feature_map}. The relationship between the generic feature map (defined for all \acl{OVK}) the functional feature map (defining a shift-invariant $\mathcal{Y}$-Mercer \acl{OVK}) and the random Fourier feature map is presented in \cref{fig:rel_features}.
\begin{proposition}
\label{pr:phitilde_phi_rel}
For any $g\in \mathcal{H}=L^2(\mathcal{\dual{X}},\probability_{\dual{\Haar},\rho};\mathcal{Y}')$, let
\begin{dmath*}
\theta \colonequals \frac{1}{\sqrt{D}}\Vect_{j=1}^D g(\omega_j), \enskip \omega_j \sim \probability_{\dual{\Haar},\rho} \enskip\text{\iid}.
\end{dmath*}
Then
\begin{propenum}
\item \label{pr:cv_feature_map_1} $\left(\tildeW{\omega} \theta\right)(x)=\tildePhi{\omega}(x)^\adjoint \theta \converges{\asurely}{D\to\infty} \Phi_x^\adjoint g=(Wg)(x)$,
\item \label{pr:cv_feature_map_2} $\norm{\theta}_{\tildeH{\omega}}^2 \converges{\asurely}{D\to\infty} \norm{g}_{\mathcal{H}}^2$,
\end{propenum}
\end{proposition}
\begin{proof}[of \cref{pr:cv_feature_map_1}] since $(\omega_j)_{j=1}^D$ are \iid~random vectors, for all $y\in \mathcal{Y}$ and for all $y'\in\mathcal{Y}'$, $\inner{y, B(\cdot)y'}\in L^2(\dual{\mathcal{X}},\probability_{\dual{\Haar},\rho})$ and $g\in L^2(\dual{\mathcal{X}},\probability_{\dual{\Haar},\rho};\mathcal{Y}')$, from the strong law of large numbers
\begin{dmath*}
(\tildeW{\omega} \theta)(x)=\tildePhi{\omega}(x)^\adjoint \theta=\frac{1}{D}\sum_{j=1}^D \conj{\pairing{x,\omega_j}}B(\omega_j)g(\omega_j), \qquad \omega_j \hiderel{\sim} \probability_{\dual{\Haar},\rho} \enskip \text{\iid} \\
\converges{\asurely}{D\to\infty} \int_{\dual{\mathcal{X}}}\conj{\pairing{x,\omega}}B(\omega)g(\omega)d\probability_{\dual{\Haar},\rho}(\omega)
= (Wg)(x) \hiderel{\colonequals} \Phi_x^\adjoint g.\qquad\ensuremath{\Box}
\end{dmath*}
\end{proof}
\begin{proof}[of \cref{pr:cv_feature_map_2}] again, since $(\omega_j)_{j-1}^D$ are \iid~random vectors and $g\in L^2(\dual{\mathcal{X}},\probability_{\dual{\Haar},\rho};\mathcal{Y}')$, from the strong law of large numbers
\begin{dmath*}
\norm{\theta}^2_{\tildeH{\omega}}=\frac{1}{D}\sum_{j=1}^D\norm{g(\omega_j)}^2_{\mathcal{Y}'}, \qquad \omega_j \hiderel{\sim} \probability_{\dual{\Haar},\rho} \enskip \text{\iid} \\
\converges{\asurely}{D\to\infty} \int_{\dual{\mathcal{X}}} \norm{g(\omega)}_{\mathcal{Y}'}^2d\probability_{\dual{\Haar},\rho}(\omega) \\
= \norm{g}_{L^2\left(\dual{\mathcal{X}}, \probability_{\dual{\Haar},\rho}; \mathcal{Y}'\right)}^2.\qquad\ensuremath{\Box}
\end{dmath*}
\end{proof}
% Hence the sequence of function $\tildef{D}_j\colonequals (\tildePhi{\omega}_{1:j}(\cdot)^\adjoint \theta)_{j=1}^D$ converges almost surely to a function $f\in\mathcal{H}_{(A,\probability_{\dual{\Haar},\rho})}{\scriptstyle\implies} \mathcal{H}_K$.
% \paragraph{}
We write $\tildePhi{\omega}(x)^\adjoint \tildePhi{\omega}(x)\approx K(x,z)$ when $\tildePhi{\omega}(x)^\adjoint \tildePhi{\omega}(x)\converges{\asurely}{} K(x,z)$ in the weak operator topology when $D$ tends to infinity. With mild abuse of notation we say that $\tildePhi{\omega}(x)$ is an approximate feature map of $\Phi_x$ \ie~$\tildePhi{\omega}(x)\approx \Phi_x$, when for all $y'$, $y\in\mathcal{Y}$,
\begin{dmath*}
\inner{y, K(x,z)y'}_{\mathcal{Y}}=\inner{\Phi_x y, \Phi_z y'}_{\mathcal{L^2(\dual{\mathcal{X}},\probability_{\dual{\Haar},\rho};\mathcal{Y'})}}\approx \inner{\tildePhi{\omega}(x)y, \tildePhi{\omega}(x)y'}_{\tildeH{\omega}}\colonequals \inner{y, \tilde{K}(x,z)y'}_{\mathcal{Y}}
\end{dmath*}
where $\Phi_x$ is defined in the sense of \cref{pr:fourier_feature_map}. Then \cref{cr:ORFF-map-kernel} exhibit a construction of an \acs{ORFF} directly from an \acs{OVK}.
\begin{corollary}
\label{cr:ORFF-map-kernel}
If $K(x,z)$ is a shift-invariant $\mathcal{Y}$-Mercer kernel such that for all $y$, $y'\in\mathcal{Y}$, $\inner{y', K_e(\cdot)y}\in L^1(\mathcal{X},\Haar)$. Then
\begin{equation}
\tildePhi{\omega}(x)y= \frac{1}{\sqrt{D}}\Vect_{j=1}^D\pairing{x, \omega_j}B(\omega_j)^\adjoint y, \qquad \omega_j \sim \probability_{\dual{\Haar},\rho} \enskip\text{\iid},
\end{equation}
where $\inner{y, B(\omega)B(\omega)^\adjoint y'}\rho(\omega)=\FT{\inner{y', K_e(\cdot)y}}(\omega)$, is an approximated feature map of $K$.
\end{corollary}
\begin{proof}
Find $(A, \probability_{\dual{\Haar},\rho})$ from \cref{pr:spectral}, find a decomposition of $A(\omega)=B(\omega)B(\omega)^*$ for $\probability_{\dual{\Haar},\rho}$-almost all $\omega$ and apply \cref{pr:ORFF-map}.
\end{proof}
\begin{remark}
We find a decomposition such that for all $j=1, \ldots, D$, $A(\omega_j)=B(\omega_j)B(\omega_j)^\adjoint $ either by exhibiting an analytic closed-form or using a numerical decomposition.
\end{remark}
\Cref{cr:ORFF-map-kernel} allows us to define \cref{alg:ORFF_construction} for constructing \acs{ORFF} from an operator valued kernel.
\SetKwInOut{Input}{Input}
\SetKwInOut{Output}{Output}
\begin{center}
\begin{algorithm2e}[H]\label{alg:ORFF_construction}
	\SetAlgoLined
    \Input{$K(x, z)=K_e(\delta)$ a $\mathcal{Y}$-shift-invariant Mercer kernel such that $\forall y,y'\in\mathcal{Y},$ $\inner{y', K_e(\cdot)y}\in L^1(\mathbb{R}^d, \Haar)$ and $D$ the number of features.}
    \Output{A random feature $\tildePhi{\omega}(x)$ such that $\tildePhi{\omega}(x)^\adjoint \tildePhi{\omega}(z) \approx K(x,z)$}
    \BlankLine
	Define the pairing $\pairing{x, \omega}$ from the \acs{LCA} group $(\mathcal{X}, \groupop)$\;
	Find a decomposition $(B(\omega),\probability_{\dual{\Haar},\rho})$ such that
    \begin{dmath*}
    B(\omega)B(\omega)^\adjoint \rho(\omega)=\IFT{K_e}(\omega)\text{\;}
    \end{dmath*}
	\nl Draw $D$ random vectors $(\omega_j)_{j=1}^D$ \iid~from the probability law $\probability_{\dual{\Haar},\rho}$\;
    \nl \Return $\begin{cases}\tildePhi{\omega}(x)\in\mathcal{L}(\mathcal{Y},\tildeH{\omega}) &:  y \mapsto \frac{1}{\sqrt{D}}\Vect_{j=1}^D\pairing{x, \omega_j}B(\omega_j)^\adjoint y \\ \tildePhi{\omega}(x)^\adjoint \in\mathcal{L}(\tildeH{\omega}, \mathcal{Y}) &: \theta \mapsto \frac{1}{\sqrt{D}} \sum_{j=1}^D \pairing{x, \omega_j}B(\omega_j)\theta_j \end{cases}$\;
    \caption{Construction of \acs{ORFF} from \acs{OVK}}
\end{algorithm2e}
\end{center}

\afterpage{
\begin{landscape}
\begin{figure}[htb]
\centering
\resizebox{\textheight}{!}{%
\begin{tikzpicture}
  \tikzstyle{every node}=[font=\Huge]
  \matrix (m) [matrix of math nodes, nodes in empty cells, ampersand replacement=\&, row sep=3em, column sep=4em, minimum width=2em]
  {
     \Phi_x\in\mathcal{L}(\mathcal{Y}\text{;} \mathcal{H}) \& \& \mathcal{Y} \& \Phi_x\in\mathcal{L}\left(\mathcal{Y}\text{;} L^2\left(\dual{\mathcal{X}}, \probability_{\dual{\Haar},\rho}\text{;} \mathcal{Y}'\right)\right) \& \& \mathcal{Y} \& \tildePhi{\omega}(x)\in\mathcal{L}\left(\mathcal{Y}\text{;} \tildeH{\omega}\right) \& \& \mathcal{Y}\\
     \& \& \& \& \& \& \& \& \\
     x\in\mathcal{X} \& \& \& x\in\mathcal{X} \& \& \& x\in\mathcal{X} \& \& \\
      \& \& \& \& \& \& \& \& \\
  };
  \path[-stealth, very thick]
    (m-1-1) edge node [above] {$\Phi_x^\adjoint g$} (m-1-3)
    (m-3-1) edge node [below] {$f$} (m-1-3)
    (m-3-1) edge node [left] {$\Phi$} (m-1-1)

    (m-1-4) edge node [above] {$\Phi_x^\adjoint g$} (m-1-6)
    (m-3-4) edge node [below] {$f$} (m-1-6)
    (m-3-4) edge node [left] {$\Phi$} (m-1-4)

    (m-1-7) edge node [above] {$\tildePhi{\omega}(x)^\adjoint \theta$} (m-1-9)
    (m-3-7) edge node [below] {$\tildef{\omega}$} (m-1-9)
    (m-3-7) edge node [left] {$\tildePhi{\omega}$} (m-1-7)
    ;

    \node[rectangle,above delimiter=\}] (del-top-1) at ($0.5*(m-4-1.south west) + 0.5*(m-4-3.south east)$) {\tikz{\path (m-4-1.north) rectangle (m-4-3.north);}};
    \node[rectangle,above delimiter=\}] (del-top-2) at ($0.5*(m-4-4.south west) + 0.5*(m-4-6.south east)$) {\tikz{\path (m-4-4.north) rectangle (m-4-6.north);}};
    \node[rectangle,above delimiter=\}] (del-top-3) at ($0.5*(m-4-7.south west) + 0.5*(m-4-9.south east)$) {\tikz{\path (m-4-7.north) rectangle (m-4-9.north);}};

   \node[rectangle,above] (ker-top-1) at (-17, 5) {$\Phi_x^\adjoint \Phi_z=K(x, z)$};
   \node[rectangle,above] (ker-top-1) at (-1.5, 5) {$K_e\left(x\groupop z^{-1}\right)$};
   \node[rectangle,above] (ker-top-1) at (14.5, 5) {$\tildeK{\omega}_e\left(x\groupop z^{-1}\right)=\tildePhi{\omega}(x)^\adjoint \tildePhi{\omega}(x)$};

   \node[rectangle,above] (sym-top-1) at (-9.75, 5.25) {$=$};
   \node[rectangle,above] (sym-top-1) at (8.5, 5.25) {$\approx$};

   \draw[very thick] (-20.5,4.5) -- (20.5,4.5);

   \path[-stealth, very thick]
    (del-top-1.south) edge [bend right] node [below, text width=10cm] {\centering\huge Fourier, \\ $\quad\Phi_x(\omega)y=\pairing{x,\omega}B(\omega)^\adjoint y$.} (del-top-2.south)
    (del-top-2.south) edge [bend right] node [below, text width=14cm] {\centering\huge Monte-Carlo, \\ $\tildePhi{\omega}(x)y=\frac{1}{\sqrt{D}}\Vect_{j=1}^D(\Phi_x y)(\omega_j)$, $\omega_j\sim \probability_{\dual{\Haar},\rho}$ \ac{iid}.} (del-top-3.south)
    ;

\end{tikzpicture}
}

\caption[Relationships between feature-maps.]{Relationships between feature-maps. $\tildeH{\omega} = \Vect_{j=1}^D \mathcal{Y}'$.}
\label{fig:rel_features}
\end{figure}
\end{landscape}
}

\subsection{Examples of Operator Random Fourier Feature maps}
\label{subsec:examples_ORFF}
We now give two examples of operator-valued random Fourier feature map when. First we introduce the general form of an approximated feature map for a matrix-valued kernel on the additive group $(\mathbb{R}^d,+)$.
\begin{example}[Matrix-valued kernel on the additive group]\label{ex:additive_group}
In the following, $K(x,z)=K_0(x-z)$ is a $\mathcal{Y}$-Mercer matrix-valued kernel on $\mathcal{X}=\mathbb{R}^d$ invariant w.r.t. the group operation $+$. %
Then the function $\tildePhi{\omega}$ defined as follow is an \acl{ORFF} of $K_{0}$.
\begin{dmath*}
\tildePhi{\omega}(x)y=\frac{1}{\sqrt{D}}\Vect_{j=1}^D\begin{pmatrix}\cos{\inner{x,\omega_j}}B(\omega_j)^\adjoint y \\ \sin{\inner{x,\omega_j}}B(\omega_j)^\adjoint y\end{pmatrix}, \enskip \omega_j \hiderel{\sim} \probability_{\dual{\Haar},\rho} \enskip\text{\iid}.
\end{dmath*}
for all $y\in\mathcal{Y}$.
\end{example}
\begin{proof}
The (Pontryagin) dual of $\mathcal{X}=\mathbb{R}^d$
is $\dual{\mathcal{X}}\cong\mathbb{R}^d$, and the duality pairing is $\pairing{x-z,\omega}=\exp(i\inner{x-z, \omega})$. The kernel approximation yields:
\begin{dmath*}
\tilde{K}(x,z)=\tildePhi{\omega}(x)^\adjoint \tildePhi{\omega}(z)
= \frac{1}{D} \sum_{j=1}^D \begin{pmatrix} \cos{\inner{x,\omega_j}} & \sin{\inner{x,\omega_j}} \end{pmatrix}\begin{pmatrix}\cos{\inner{z,\omega_j}} \\ \sin{\inner{z,\omega_j}} \end{pmatrix} A(\omega_j)
= \frac{1}{D} \sum_{j=1}^D \cos{\inner{x-z,\omega_j}}A(\omega_j) \\
\converges{\asurely}{D\to\infty}\expectation_\rho\left[\cos{\inner{x-z,\omega}}A(\omega)\right]
\end{dmath*}
in the weak operator topology. Since for all $x\in\mathcal{X}$, $\sin\inner{x, \cdot}$ is an odd function and $A(\cdot)\rho(\cdot)$ is even,
\begin{dmath*}
\expectation_\rho\left[\cos{\inner{x-z,\omega}}A(\omega)\right]=\expectation_\rho\left[\exp(-i\inner{x-z,\omega})A(\omega)\right]\hiderel{=}K(x,z).
\end{dmath*}
Hence $\tilde{K}(x,z)\converges{\asurely}{D\to\infty}K(x,z)$.
\end{proof}
In particular we deduce the following features maps for the kernels proposed in \cref{subsec:dec_examples}.
\begin{itemize}
\item For the decomposable gaussian kernel $K_0^{dec,gauss}(\delta)=k_0^{gauss}(\delta)\Gamma$ for all $\delta\in\mathbb{R}^d$, let $BB^\adjoint=\Gamma$. A bounded --and unbounded-- \acs{ORFF} map is
\begin{dmath*}
\tildePhi{\omega}(x)y=\frac{1}{\sqrt{D}}\Vect_{j=1}^D\begin{pmatrix}\cos{\inner{x,\omega_j}}B^\adjoint y\\ \sin{\inner{x,\omega_j}}B^\adjoint y \end{pmatrix}
=(\tildePhi{\omega}(x)\otimes B^\adjoint)y,
\end{dmath*}
where $\omega_j \hiderel{\sim} \probability_{\mathcal{N}(0,\sigma^{-2}I_d)}$ \iid~and $\tildePhi{\omega}(x)=\frac{1}{\sqrt{D}}\Vect_{j=1}^D\begin{pmatrix}\cos{\inner{x,\omega_j}} \\ \sin{\inner{x,\omega_j}}\end{pmatrix}$ is a scalar \acs{RFF} map \citep{Rahimi2007}.
\item For the curl-free gaussian kernel, $K_0^{curl,gauss}=-\nabla\nabla^T k_0^{gauss}$ an unbounded \acs{ORFF} map is
\begin{dmath*}
\tildePhi{\omega}(x)y=\frac{1}{\sqrt{D}}\Vect_{j=1}^D\begin{pmatrix}\cos{\inner{x,\omega_j}}\omega_j^T y\\ \sin{\inner{x,\omega_j}}\omega_j^T y\end{pmatrix},
\end{dmath*}
$\omega_j \hiderel{\sim} \probability_{\mathcal{N}(0,\sigma^{-2}I_d)}$ \iid~and a bounded \acs{ORFF} map is
\begin{dmath*}
\tildePhi{\omega}(x) y=\frac{1}{\sqrt{D}}\Vect_{j=1}^D\begin{pmatrix}\cos{\inner{x,\omega_j}}\frac{\omega_j^T}{\norm{\omega_j}} y \\ \sin{\inner{x,\omega_j}}\frac{\omega_j^T}{\norm{\omega_j}} y \end{pmatrix}\condition{$\omega_j \hiderel{\sim} \probability_{\rho}$ \iid.}
\end{dmath*}
where $\rho(\omega)=\frac{\sigma^2\norm{\omega}^2}{d}\mathcal{N}(0,\sigma^{-2}I_d)(\omega)$ for all $\omega\in\mathbb{R}^d$.
\item For the divergence-free gaussian kernel $K_0^{div,gauss}(x,z)=(\nabla\nabla^T-\Delta I_d) k_0^{gauss}(x,z)$ an unbounded \acs{ORFF} map is
\begin{dmath*}
\tildePhi{\omega}(x) y=\frac{1}{\sqrt{D}}\Vect_{j=1}^D\begin{pmatrix}\cos{\inner{x,\omega_j}}B(\omega_j)^T y\\ \sin{\inner{x,\omega_j}}B(\omega_j)^T y\end{pmatrix} \condition{$\omega_j \hiderel{\sim} \probability_{\rho}$ \iid.}
\end{dmath*}
where $B(\omega)=\left(I_d-\frac{\omega^T\omega}{\norm{\omega}}\right)$ and $\rho=\mathcal{N}(0,\sigma^{-2}I_d)$ for all $\omega\in\mathbb{R}^d$. A bounded \acs{ORFF} map is
\begin{dmath*}
\tildePhi{\omega}(x) y=\frac{1}{\sqrt{D}}\Vect_{j=1}^D\begin{pmatrix}\cos{\inner{x,\omega_j}}B(\omega_j)^T y\\ \sin{\inner{x,\omega_j}}B(\omega_j)^T y\end{pmatrix}\condition{$\omega_j \hiderel{\sim} \probability_{\rho}$ \iid,}
\end{dmath*}
where $B(\omega)=\left(I_d-\frac{\omega^T\omega}{\norm{\omega}^2}\right)$ and $\rho(\omega)=\frac{\sigma^2\norm{\omega}^2}{d}\mathcal{N}(0,\sigma^{-2}I_d)$ for all $\omega\in\mathbb{R}^d$.
\end{itemize}
The second example extends scalar-valued Random Fourier Features on the skewed multiplicative group --described in \cref{subsec:characters} and \cref{subsubsec:skewedchi2}-- to the operator-valued case.
\begin{example}[Matrix-valued kernel on the skewed multiplicative group]
In the following, $K(x,z)=K_{1-c}(x\odot z^{-1})$ is a $\mathcal{Y}$-Mercer matrix-valued kernel on $\mathcal{X}=(-c;+\infty)^d$ invariant \wrt~the group operation\mpar{The group operation $\odot$ is defined in \cref{subsubsec:skewedchi2}.} $\odot$. Then the function $\tildePhi{\omega}$ defined as follow is an \acl{ORFF} of $K_{1-c}$.
\begin{dmath*}
\tildePhi{\omega}(x) y=\frac{1}{\sqrt{D}}\Vect_{j=1}^D\begin{pmatrix}\cos{\inner{\log(x+c),\omega_j}}B(\omega_j)^\adjoint y\\ \sin{\inner{\log(x+c),\omega_j}}B(\omega_j)^\adjoint y\end{pmatrix},
\end{dmath*}
$\omega_j \sim \probability_{\dual{\Haar},\rho}$ iid, for all $y\in\mathcal{Y}$.
\end{example}
\begin{proof}
The dual of $\mathcal{X}=(-c;+\infty)^d$
is $\dual{\mathcal{X}}\cong\mathbb{R}^d$, and the duality pairing is $\pairing{x \odot \myinv{z},\omega}=\exp(i\inner{\log(x\odot z^{-1}+c), \omega})$. Following the proof of \cref{ex:additive_group}, we have
\begin{equation*}
\begin{aligned}
\tilde{K}(x,z)&= \frac{1}{D} \sum_{j=1}^D \cos{\inner*{\log\left(\frac{x+c}{z+c}\right), \omega_j}}A(\omega_j).
\end{aligned}
\end{equation*}
which converges almost surely to
\begin{dmath*}
\expectation_{\rho}[\exp(-i\inner*{\log(x\odot z^{-1}+c)})A(\omega) ]\hiderel{=}\expectation_{\rho}[\conj{\pairing{x\odot z^{-1},\omega}}A(\omega)]\hiderel{=}K(x, z)
\end{dmath*}
when $D$ tends to infinity, in the weak operator topology.
\end{proof}
\begin{itemize}
\item For the skewed-$\chi^2$ decomposable kernel defined as $K_{1-c}^{dec,skewed}(\delta)=k_{1-c}^{skewed}(\delta)\Gamma$ for all $\delta\in\mathcal{X}$, let $BB^*=\Gamma$. A bounded --and unbounded-- \acs{ORFF} map is
\begin{dmath*}
\tildePhi{\omega}(x)y=\frac{1}{\sqrt{D}}\Vect_{j=1}^D\begin{pmatrix}\cos{\inner{\log(x+c),\omega_j}}B^\adjoint y\\ \sin{\inner{\log(x+c),\omega_j}}B^\adjoint y\end{pmatrix}, \enskip \omega_j \hiderel{\sim} \probability_{\rho} \enskip\text{\iid}
=(\tildePhi{\omega}(x) \otimes B^\adjoint)y,
\end{dmath*}
where $\rho=\mathcal{S}(0,2^{-1})$ and $\tildePhi{\omega}(x)=\frac{1}{\sqrt{D}}\Vect_{j=1}^D\begin{pmatrix}\cos{\inner{\log(x+c),\omega_j}} \\ \sin{\inner{\log(x+c),\omega_j}}\end{pmatrix}$ is a scalar \acs{RFF} map \citep{li2010random}.
\end{itemize}
\subsection{Regularization property}
We have shown so far that it is always possible to construct a feature map that allows to approximate a shift-invariant $\mathcal{Y}$-Mercer kernel. However we could also propose a construction of such map by studying the regularization induced with respect to the \acl{FT} of a target function $f\in \mathcal{H}_K$. In other words, what is the norm in $L^2(\dual{\mathcal{X}}, \dual{\Haar}; \mathcal{Y}')$ induced by $\norm{\cdot}_K$?
\begin{proposition}
Let $K$ be a shift-invariant $\mathcal{Y}$-Mercer Kernel such that for all $y$, $y'$ in $\mathcal{Y}$, $\inner{y', K_e(\cdot)y}\in L^1(\mathcal{X}, \Haar)$. Then for all $f\in\mathcal{H}_K$
\begin{dmath}
\norm{f}^2_K=\displaystyle\int_{\dual{\mathcal{X}}}\frac{\inner*{\FT{f}(\omega), A\left(\omega\right)^\dagger\FT{f}(\omega)}_{\mathcal{Y}}}{\rho(\omega)}d\dual{\Haar}(\omega).
\label{eq:reg_L2}
\end{dmath}
where $\inner{y', A(\omega)y}\rho(\omega)\colonequals\FT{\inner{y', K_e(\cdot)y}}(\omega)$.
\label{pr:regularization}
\end{proposition}
\begin{proof}
We first show how the \acl{FT} relates to the feature operator. Since $\mathcal{H}_K$ is embed into $\mathcal{H}=L^2(\dual{\mathcal{X}}, \probability_{\dual{\Haar},\rho}; \mathcal{Y})$ by mean of the feature operator $W$, we have for all $f\in\mathcal{H}_k$, for all $f\in\mathcal{H}$ and for all $x\in\mathcal{X}$
\begin{dgroup*}
\begin{dmath*}
\FT{\IFT{f}}(x)\hiderel{=}\int_{\dual{\mathcal{X}}}\overline{\pairing{x,\omega}}\IFT{f}(\omega)d\dual{\Haar}(\omega) = f(x)
\end{dmath*}
\begin{dmath*}
(Wg)(x)\hiderel{=}\int_{\dual{\mathcal{X}}}\conj{\pairing{x,\omega}}\rho(\omega)B(\omega)g(\omega)d\dual{\Haar}(\omega) = f(x).
\end{dmath*}
\end{dgroup*}
By injectivity of the \acl{FT}, $\IFT{f}(\omega)=\rho(\omega)B(\omega)g(\omega)$. From \cref{pr:feature_operator} we have
\begin{dmath*}
\norm{f}^2_{K} = \inf \Set{\norm{g}^2_{\mathcal{H}} | \forall g\hiderel{\in}\mathcal{H}, \enskip Wg\hiderel{=}f} = \inf \Set{\int_{\dual{\mathcal{X}}} \norm{g}^2_{\mathcal{Y}}d\probability_{\dual{\Haar},\rho} | \forall g\hiderel{\in}\mathcal{H},\enskip \IFT{f}\hiderel{=}\rho(\cdot)B(\cdot)g(\cdot)}.
\end{dmath*}
The pseudo inverse of the operator $B(\omega)$ -- noted $B(\omega)^\dagger$ -- is the unique solution of the system $\IFT{f}(\omega)=\rho(\omega)B(\omega)g(\omega)$ \wrt~$g(\omega)$ with minimal norm\mpar{Note that since $B(\omega)$ is bounded the pseudo inverse of $B(\omega)$ is well defined for $\dual{\Haar}$-almost all $\omega$. However if $B(\omega)$ is infinite dimensional, the pseudo inverse is continuous if and only if $A(\omega)$ has closed range. This is always true if $\mathcal{Y}$ is finite dimensional.}. Eventually,
\begin{dmath*}
\norm{f}^2_K = \int_{\dual{\mathcal{X}}} \frac{\norm{B(\omega)^\dagger\IFT{f}(\omega)}_{\mathcal{Y}}^2}{\rho(\omega)^2}d\probability_{\dual{\Haar},\rho}(\omega)
\end{dmath*}
Using the fact that $\IFT{\cdot}=\mathcal{F}\mathcal{R}[\cdot]$ and $\mathcal{F}^2[\cdot]=\mathcal{R}[\cdot]$,
\begin{dmath*}
\norm{f}^2_K= \displaystyle\int_{\dual{\mathcal{X}}} \frac{\norm{\mathcal{R}\left[B(\cdot)^\dagger\rho(\cdot)\right](\omega)\FT{f}(\omega)}^2_{\mathcal{Y}}}{\rho(\omega)^2}d\dual{\Haar}(\omega)
= \displaystyle\int_{\dual{\mathcal{X}}} \frac{\norm{B(\omega)^\dagger\rho(\omega)\FT{f}(\omega)}^2_{\mathcal{Y}}}{\rho(\omega)^2}d\dual{\Haar}(\omega)
= \displaystyle\int_{\dual{\mathcal{X}}} \frac{\inner{B(\omega)^\dagger\FT{f}(\omega),B(\omega)^\dagger\FT{f}(\omega)}_{\mathcal{Y}}}{\rho(\omega)}d\dual{\Haar}(\omega)
= \displaystyle\int_{\dual{\mathcal{X}}} \frac{\inner{\FT{f}(\omega),A(\omega)^\dagger\FT{f}(\omega)}_{\mathcal{Y}}}{\rho(\omega)}d\dual{\Haar}(\omega)
\end{dmath*}
\end{proof}
Note that if $K(x,z)=k(x,z)$ is a scalar kernel then for all $\omega$ in $\dual{\mathcal{X}}$, $A(\omega)=1$. Therefore we recover a well known results for kernels that is for any $f\in\mathcal{H}_k$ we have $\norm{f}_k=\int_{\dual{\mathcal{X}}}\FT{k_e}(\omega)^{-1}\FT{f}(\omega)^2d\dual{\Haar}(\omega)$ \citep{Yang2012,vertregularization,smola1998connection}. We also note that the regularization property in $\mathcal{H}_K$ does not depends (as expected) on the decomposition of $A(\omega)$ into $B(\omega)B(\omega)^\adjoint $. Therefore the decomposition should be chosen such that it optimizes the computation cost. For instance if $A(\omega)\in\mathcal{L}(\mathbb{R}^p)$ has rank $r$, one could find an operator $B(\omega)\in\mathcal{L}(\mathbb{R}^p, \mathbb{R}^r)$ such that $A(\omega)=B(\omega)B(\omega)^\adjoint$. Moreover, in light of \cref{pr:regularization} the regularization property of the kernel with respect to the \acl{FT}, it is also possible to define an approximate feature map of an \acl{OVK} from its regularization properties in the \acs{vv-RKHS} as proposed in \cref{alg:ORFF2_construction}.
\SetKwInOut{Input}{Input}
\SetKwInOut{Output}{Output}
\begin{center}
\begin{algorithm2e}\label{alg:ORFF2_construction}
    \SetAlgoLined
    \Input{
    \begin{itemize}
    \item The pairing $\pairing{x, \omega}$ of the \acs{LCA} group $(\mathcal{X}, \groupop)$.
    \item A probability measure $\probability_{\dual{\Haar},\rho}$ with density $\rho$ \wrt~the haar measure $\dual{\Haar}$ on $\dual{\mathcal{X}}$.
    \item An operator-valued function $B:\dual{\mathcal{X}}\to\mathcal{L}(\mathcal{Y},\mathcal{Y}')$ such that for all $y$ $y'\in\mathcal{Y}$, $\inner{y', B(\cdot)B(\cdot)^\adjoint y}\in L^1(\dual{\mathcal{X}},\probability_{\dual{\Haar},\rho})$.
    \item $D$ the number of features.
    \end{itemize}}
    \Output{A random feature $\tildePhi{\omega}(x)$ such that $\tildePhi{\omega}(x)^\adjoint \tildePhi{\omega}(z) \approx K(x,z)$.}
    \BlankLine
    Draw $D$ random vectors $(\omega_j)_{j=1}^D$ \iid~from the probability law $\probability_{\dual{\Haar},\rho}$\;
    \Return $\begin{cases}\tildePhi{\omega}(x) \in\mathcal{L}(\mathcal{Y}, \tildeH{\omega}) &:  y \mapsto \frac{1}{\sqrt{D}}\Vect_{j=1}^D\pairing{x, \omega_j}B(\omega_j)^\adjoint y \\ \tildePhi{\omega}(x)^\adjoint \in\mathcal{L}(\tildeH{\omega}, \mathcal{Y}) &: \theta \mapsto \frac{1}{\sqrt{D}} \sum_{j=1}^D \pairing{x, \omega_j}B(\omega_j)\theta_j \end{cases}$\;
   \caption{Construction of \acs{ORFF}}
\end{algorithm2e}
\end{center}
\subsection{Operator Random Feature engineering}
As in the scalar case, it is possible to construct. We list some examples in the following.
\subsubsection{Sum of kernels}
\begin{proposition}[Sum of kernels]
Let $I$ be a countable set and let $(K^i)_{i\in I}$ be a familly of $\mathcal{Y}$-reproducing kernels such that for all $y\in\mathcal{Y}$
\begin{dmath*}
\sum_{i\in I}\inner{y, K^i(x,x)y} < \infty.
\end{dmath*}
Given $x$, $z\in\mathcal{X}$, the serie $\sum_{i\in I}K^i(x,z)$ converges to a bounded operator $K(x,z)$ in the strong operator topology, and the map $K:\mathcal{X}\times\mathcal{X}\to\mathcal{L}(\mathcal{Y})$ defined by
\begin{dmath*}
K(x,z)y=\sum_{i\in I}K^{i}(x,z)y
\end{dmath*}
is a $\mathcal{Y}$-reproducing kernel. The corresponding space $\mathcal{H}_K$ is embedded in $\oplus_{i\in I} \mathcal{H}_{K^i}$ by means of the feature operator
\begin{dmath*}
(Wf)(x)=\sum_{i\in I} f_i
\end{dmath*}
where $f=\oplus_{i\in I} f_i$, $f_i\in\mathcal{H}_{K^i}$, and the sum converges in norm. Moreover if each $K^i$ is a Mercer kernel --resp. $\mathcal{C}_0$-kernel-- and $x\mapsto \sum_{i\in I}\norm{K^i(x,x)}_{\mathcal{Y},\mathcal{Y}}$ is locally bounded --resp. bounded-- then $K$ is Mercer --resp. $\mathcal{C}_0$.
\end{proposition}

% Let $\Omega=\Set{(\omega_j)_{j=1}^D | \omega_j \sim \probability_{\dual{\Haar}, \rho}\enskip \text{\iid}}$. Then
% \begin{dmath*}
% \inner{y', K(x,z)y}=\frac{1}{D}\sum_{\omega\in\Omega} \inner{\Phi_xy
% (\omega), \Phi_xy(\omega)}
% \end{dmath*}

% \begin{corollary}

% \end{corollary}

%----------------------------------------------------------------------------------------


% \afterpage{



% Since $\theta\in\tildeH{\omega}\left(\mathcal{Y}'\right)$ lives in a different space than $f\in\mathcal{H}_K$ is makes no sense to show the convergence of $\theta_*$ to $f_*$. However we can verify that $\tildef{1:D}$ converges to $f\in\mathcal{H}_K$ pointwise. We first introduce the following lemma and assumptions. First we characterize extremum estimators. Let $\mathcal{H}$ be any Hilbert space.
% \begin{assumption}[EE]
% \label{ass:ee}
% The parameter $\tilde{g}_*\in\mathcal{H}$ is an extremum estimator of $Q_D$ if $\tilde{g}_*=\argmin_{g\in\mathcal{H}} Q_D(g)$.
% \end{assumption}
% To derive consistency we must ensure that the stochastic objective $Q_D$ function converges uniformly-weakly to some target function $Q_0$. Let $B(g,\epsilon)$ be the open ball in $\mathcal{H}$ with radius $\epsilon$ and $\mathcal{H}\setminus{B(g,\epsilon)}$ denotes its complement in $\mathcal{H}$.
% \begin{assumption}[U-SCON]
% \label{ass:uwcon}
% A \emph{stochastic} function $Q_D$ converges uniformly-strongly to some \emph{non-stochastic} function $Q_0$ if \begin{dmath*}
% \sup_{g\in\mathcal{H}}\abs{Q_D(g)-Q_0(g)}\converges{\asurely}{D\to\infty}0
% \end{dmath*}
% \end{assumption}
% This is equivalent to say that $Q_D$ converges to $Q_0$ in probability in the uniform norm topology. Eventually the mimizers of $Q$ must be uniquely identifiable.
% \begin{assumption}[ID]
% \label{ass:id}
% A function has unique identifiable minimizer if there exist $g_*\in\mathcal{H}$ such that for all $\epsilon\ge 0$, $\inf_{g\in \mathcal{H}\setminus{B(g_*,\epsilon)}} Q(g)>Q(g_*)$.
% \end{assumption}
% Note that if $Q$ is a strictly convex function of $g$, then \cref{ass:id} holds. These three assumptions combined yields the consistency of an estimator.
% \begin{proposition} If
% \cref{ass:ee}, \cref{ass:uwcon} and \cref{ass:id} hold then $\tilde{g}_* \converges{\asurely}{D\to\infty} g_*.$
% \end{proposition}
% \begin{proof}

% \end{proof}
% We are now ready to prove the consistency of ORFF with respect to the minimization in the RKHS, that is if we replace the model $f(x)=\sum_{i=1}^NK(\cdot, x_i)c_i$ by a model $f(x)=\tildePhi{\omega}(x)^\adjoint \theta$ and find the minimizer of the ridge regression problem, then when $D$ tends to infinity, both model return the \say{same} function.
% \begin{proposition}[Consitensy of ORFF \wrt~\acs{vv-RKHS}] Let $K$ be a $\mathcal{Y}$-Mercer Kernel. If $\theta_*\in\tildeH{\omega}\left(\mathcal{Y}'\right)$ is the minimizer of \cref{eq:argmin_applied} and $f_*\in\mathcal{H}_K$ is the minimizer of \cref{eq:argmin_RKHS}. Then
% \begin{dmath}
% \tildePhi{\omega}(x)^\adjoint\theta_* \converges{\asurely}{D\to\infty}f_*(x),
% \end{dmath}
% for all $x\in\mathcal{X}$.
% \end{proposition}
% \begin{proof}
% % Let $(Wg)(x)=\Phi_x^\adjoint g$ be the feature operator associated to $\Phi$ and $\tildePhi{\omega}(x)y=\frac{1}{\sqrt{D}}\Vect_{j=1}^D(\Phi_x y)(\omega_j)$, where $\omega_j$ are \iid~random vectors following the law $\mu$. Since $W$ is a partial isometry

% % We have that for all $f\in\mathcal{H_K}$, there exists a $g\in\mathcal{H}$ such that $(W^\adjoint f)(x)=g(x)=\Phi(x)^\adjoint g$ where $W$ is a partial isometry with initial subspace $(\Ker W)^\perp\subset\mathcal{H}$ and final subspace $\mathcal{H}_K$.

% % \begin{dmath}
% % \label{eq:argmin_HS}
% % g_*=\argmin_{g\in(\Ker W)^\perp\subset\mathcal{H}} \frac{1}{N}\sum_{i=1}^N\norm{(Wg)(x_i)-y_i}_{\mathcal{Y}}^2 + \frac{\lambda}{2}\norm{g}^2_{\mathcal{H}}
% % \end{dmath}

% Define
% \begin{dmath}
% Q_0(g)=\frac{1}{N}\sum_{i=1}^N\norm{(Wg)(x_i)-y_i}_{\mathcal{Y}}^2 + \frac{\lambda}{2}\norm{g}^2_{\mathcal{H}}.
% \end{dmath}
% Moreover let $\tildePhi{\omega}(x_i)y=\Vect_{j=1}^D(\Phi_{x_i}y)(\omega_j)$ for all $y\in\mathcal{Y}$, and $\theta_g=\vect_{j=1}^Dg(\omega_j)$, where $\omega_j$ are \iid~random vectors following a law $\mu$. We define the empirical estimator of $Q_0$ as
% \begin{dmath}
% Q_D(g)=\frac{1}{N}\sum_{i=1}^N\norm{\tildePhi{\omega}(x_i)^\adjoint \theta_g-y_i}_{\mathcal{Y}}^2 + \frac{\lambda}{2D}\norm{\theta_g}^2_{\tildeH{\omega}\left(\mathcal{Y}'\right)}
% \end{dmath}
% Let $g_*=\argmin_{g\in\mathcal{H}}Q_0(g)$ and $\tilde{g}_*=\argmin_{g\in\mathcal{H}}Q_D(g)$ thus $\tilde{g}_*$ is an extremum estimator (\cref{ass:ee} holds). Notice that since $W$ is a bounded linear application, $Q_0$ is continuous strongly convex and coercive. By \cref{cor:unique_minimizer} attains an identifiable unique minimizer (\cref{ass:id} holds).

% Since W is bounded, $\Ker W$ is closed, so that we can perform the decomposition $\mathcal{H}=(\Ker W)^\perp\oplus \Ker W$. Then clearly by linearity of $W$ and the fact that for all $g\in\Ker W$, $Wg=0$,
% \begin{dmath*}
% g_*=\argmin_{g\in\mathcal{H}}Q_0(g)
% \hiderel{=}\argmin_{g\in(\Ker W)^\perp\oplus \Ker W}Q_0(g)
% \hiderel{=}\argmin_{g\in(\Ker W)^\perp}Q_0(g).
% \end{dmath*}
% Besides with similar arguments as in \cref{rk:rkhs_bound} we deduce that $\lambda\norm{g_*}_{\mathcal{H}}^2$ $\le$ $2\sigma^2_y$. As a result $g_*\in\mathcal{H}_\lambda\colonequals\Set{g\in(\Ker W)^\perp|\lambda\norm{g}_{\mathcal{H}}^2\le 2\sigma^2_y}$. Eventually
% \begin{dmath*}
% g_*=\argmin_{g\in\mathcal{H}}Q_0(g)\hiderel{=}\argmin_{g\in\mathcal{H}_\lambda}Q_0(g)
% \end{dmath*}
% \begin{enumerate}
% \item By assumption $\sup_{y\in\mathcal{Y}} \norm{y}\le \sigma_y$ thus from \cref{rk:rkhs_bound} we have that $\lambda\norm{f_*}^2_K$ is bounded by $2\sigma_y^2$ and so is $\lambda\norm{g_*}_{\mathcal{H}}^2$. We note $\mathcal{H}_K^{\lambda}=\Set{f\in\mathcal{H}_K|\lambda\norm{f}^2_K\le 2\sigma_y}$ and $\mathcal{H}^\lambda=\Set{g\in(\Ker W)^\perp|\lambda\norm{g}^2_{\mathcal{H}}\le 2\sigma_y}$.
% Then
% \begin{dmath*}
% f_*=\argmin_{f\in\mathcal{H}_K} \frac{1}{N}\displaystyle\sum_{i=1}^N\norm{f(x_i) - y_i}_{\mathcal{Y}}^2 + \frac{\lambda}{2}\norm{f}^2_{\mathcal{H}_K}
% =\argmin_{f\in\mathcal{H}_K^\lambda} \frac{1}{N}\displaystyle\sum_{i=1}^N\norm{f(x_i) - y_i}_{\mathcal{Y}}^2 + \frac{\lambda}{2}\norm{f}^2_{\mathcal{H}_K}
% \end{dmath*}
% and $g_*=\argmin_{g\in(Ker W)^\perp} Q_0(g)
% \hiderel{=}\argmin_{g\in\mathcal{H}^\lambda} Q_0(g)$.
% Besides the inclusion $\iota_K:\mathcal{H}_K\hookrightarrow\mathcal{C}(\mathcal{X};\mathcal{Y})$ is compact, thus the closed set $\mathcal{H}_K^{\lambda}=\Set{f\in\mathcal{H}_K|\lambda\norm{f}^2_K\le 2\sigma_y}$ is compact with respect to the compact-open topology. $W$ is a partial isometry with initial space $(\Ker W)^\perp$, the restriction $W|_{(\Ker W)^\perp}:(\Ker W)^\perp\to\mathcal{H}_K$ is an isometry. Thus the set $\mathcal{H}^\lambda=\Set{g\in(\Ker W)^\perp|\lambda\norm{g}^2_{\mathcal{H}}\le 2\sigma_y}=(W|_{(\Ker W)^\perp})^\adjoint \mathcal{H}_K^{\lambda}$ is compact.
% \item It is easy to note that $Q_0$ is continuous and measureable. Besides since $f_*$ is unique and $W|_{(\Ker W)^\perp}$ is an isometry, $Q_0$ has unique minimizer $g_*$ such that $W|_{(\Ker W)^\perp}g_*=f_*$.
% \item Suppose that $\norm{A(\omega)}_{\mathcal{Y},\mathcal{Y}}$ is bounded for all $\omega\in\dual{\mathcal{X}}$ as a consequence $\sup_{g\in\mathcal{H}^\lambda} \abs{Q_D(g)-Q_0(g)}$ is bounded thus from the strong law of large numbers $Q_D(g)\converges{\asurely}{D\to\infty}Q_0(g)$ uniformly.
% \end{enumerate}
% As a result, the extremum estimator $Q_D$ is consistent
% \begin{dmath}
% \label{eq:consitency1}
% \tilde{g}_* \converges{\proba}{D\to\infty} g_*
% \end{dmath}
% Let $\theta_{g,*}=\vect_{j=1}^D\tilde{g}_*(\omega_j)$. since $\tilde{g}_*$ minimizes $Q_D$, $\theta_{g,*}$ minimizes \cref{eq:argmin_applied}. Besides from \cref{pr:phitilde_phi_rel} we have
% \begin{dmath}
% \tilde{W}\theta_{g,*} \converges{\asurely}{D\to\infty} W\tilde{g}_*.
% \end{dmath}
% Finally \cref{eq:consitency1} shows that $W\tilde{g}_*\converges{\proba}{D\to\infty}Wg_*=f_*$, hence
% \begin{dmath*}
% \tildePhi{\omega}\theta_{g,*}\hiderel{=}\tilde{W}\theta_{g,*}\converges{\proba}{D\to\infty} f_*\hiderel{\in}\mathcal{H}_K.
% \end{dmath*}
% with $\theta_{g,*}=\theta_*$ defined in \cref{eq:argmin_RKHS}.
% \end{proof}

% \clearpage

%----------------------------------------------------------------------------------------
% \section{Consistency and generalization bounds}
% \label{sec:consistency and generalization bounds}
% \subsection{Operator-valued kernels}
% In this section are interested in finding a minimizer $f_*:\mathcal{X}\to\mathcal{Y}$, where $\mathcal{X}$ is a Polish space and $\mathcal{Y}$ a separable Hilbert space such that for all $x_i$ in $\mathcal{X}$ and all $y_i$ in $\mathcal{Y}$,
% \begin{dmath}
% f_* = \argmin_{f\in\mathcal{H}_K} \sum_{i=1}^NL_{y_i}(f(x_i)) \hiderel{=} \argmin_{f\in\mathcal{H}_K} \mathcal{R}_{emp}(f, L),
% \label{eq:pbOVK}
% \end{dmath}
% where $\mathcal{H}_K$ is a \acs{vv-RKHS} and $\forall y_i \in \mathcal{Y}$, $L_{y_i}$ a $L$-Lipschitz cost function. How does a function trained as in \cref{eq:pbOVK} generalizes on a test set?
% \begin{proposition} Suppose that $f\in\mathcal{H}_K$ a \acs{vv-RKHS} where $\sup_{x\in\mathcal{X}} \Tr[K(x,x)] < T$ and $\norm{f}_{\mathcal{H}_K}<B$. Moreover let $L:\mathcal{Y}\times \mathcal{Y}\to[0, C]$ be a $L$-Lipschitz cost function. Then if we are given $N$ training points, we have with at least probability $1-2\delta$ that
% \begin{dmath}
% \mathcal{R}_{true}(f, L) \le \mathcal{R}_{emp}(f, L)  + 2\sqrt{\frac{2}{N}}\left( LBT^{1/2} + \frac{3C}{4}\sqrt{\ln(1/\delta)}\right).
% \end{dmath}
% \label{pr:ovk_gen}
% \end{proposition}
% \begin{proof}
% This proof is due to Mauer \mpar{See \url{https://arxiv.org/pdf/1605.00251.pdf}}. We do not claim any originality for this proof. First let us introduce the notion of Rade\-macher complexity of a class of function $F$.
% \begin{definition}
% Let $\mathcal{X}$ be any set. Let $\epsilon_1$, $\hdots$, $\epsilon_N$ be $N$ independent Rade\-macher random variables, identically uniformly distributed on $\{-1;1\}$. For any class of functions $F:~\mathcal{X}\to\mathbb{R}$, then for all $x_1, \hdots x_N\in \mathcal{X}$ the quantity
% \begin{dmath}
% \mathcal{R}_N(F) \colonequals \expectation\left[ \sup_{f\in F} \sum_{i=1}^N \epsilon_i f(x_i) \right]
% \end{dmath}
% is called Rademacher complexity of the class $F$.
% \end{definition}
% In generalization bounds the Rademacher complexity of a class of function often involves a composition between a target function to be learn and a cost function, part of the risk we want to minimize. The idea is to bound the Rademacher complexity with a term that does not depends on the cost function, but only on the target function. Such a bound has be recently proposed by Mauer:
% \begin{proposition}
% \label{pr:radswap}
% Let $\mathcal{X}$ be any set, $x_1, \hdots, x_N$ in $\mathcal{X}$, let $F$ be a class of function $f:~\mathcal{X}\to\mathcal{Y}$ and $h_i:~\mathcal{Y}\to\mathbb{R}$ be a $L$-Lipschitz function, where $\mathcal{Y}$ is a second countable Hilbert space endowed with euclidean inner product. Then
% \begin{dmath}
% \expectation \sup_{f\in F} \sum_{i=1}^N \epsilon_i h_i(x_i) \le \sqrt{2}L\sup_{f\in F}\sum_{i=1, k}^{i=N} \epsilon_{ik}f_k(x_i),
% \end{dmath}
% where $\epsilon_{ik}$ is a doubly indexed independent Rademacher sequence and $f_k(x_i)$ is the $k$-th component of $f(x_i)$
% \end{proposition}
% From now on we consider functions $f\in\mathcal{H}_K$ a vv-RKHS. Then there exists an induced feature-map $\Phi:~\mathcal{X}\to \mathcal{L}(\mathcal{Y}, \mathcal{H})$ such that for all $y, y'\in\mathcal{Y}$ the kernel is given by
% \begin{dmath}
% \inner{y, K(x,z)y'} = \inner{\Phi_xy, \Phi_zy'}.
% \end{dmath}
% We say that the feature space $\mathcal{H}$ is embed into the RKHS $\mathcal{H}_K$ by mean of the \emph{feature operator} $(W\theta)(x):=(\Phi_x^\adjoint \theta)$. Indeed $W$ defines a partial isometry between $\mathcal{H}$ and $\mathcal{H}_K$. Remember that $\mathcal{Y}$ a is supposed to be a separable Hilbert space so it has a countable basis and let the class of $\mathcal{Y}$-valued functions $F$ be
% \begin{dmath}
% F=\Set{ f | f: x \mapsto \Phi_x^\adjoint  \theta, \enskip \norm{\theta} < B }\hiderel{\subset} \mathcal{H}_K.
% \end{dmath}
% Then from \cref{pr:radswap} and if $K$ is trace class, we have
% \begin{dmath}
% \label{eq:ker_bound}
% \expectation \sup_{\norm{\theta}<B} \sum_{i=1}^N \epsilon_i L_{y_i}(\Phi_{x_i}^\adjoint  \theta) \le \sqrt{2}L\expectation \sup_{\norm{\theta}<B} \sum_{i=1,k}^{i=N}\epsilon_{ik}\inner{\Phi_{x_i} \theta, e_k}
% = \sqrt{2}L\expectation \sup_{\norm{\theta}<B} \inner*{ \theta, \sum_{i,k}^{i=N}\epsilon_{ik}\Phi_{x_i}e_k}
% \le \sqrt{2}LB\expectation \norm{\sum_{i=1,k}^{i=N}\epsilon_{ik}\Phi_{x_i}e_k}
% \le \sqrt{2}LB\sqrt{\sum_{i=1,k}^{i=N} \norm{\Phi_{x_i}e_k}^2 }
% \le \sqrt{2}LB\sqrt{\sum_{i=1}^{i=N} \Tr\left[ K(x_i, x_i) \right] }.
% \end{dmath}
% From \url{http://www.jmlr.org/papers/volume3/bartlett02a/bartlett02a.pdf}:
% \begin{theorem}
% \label{th:gen_rad_bound}
% Let $\mathcal{X}$ be any set, $F$ a class of functions $f:\mathcal{X}\to[0, C]$ and let $X_1, \hdots, X_N$ be a sequence of iid random variable with value in $\mathcal{X}$. Then for $\delta > 0$, with probability at least $1-\delta$, we have for all $f\in F$ that
% \begin{dmath*}
% \expectation f(X) \le \frac{1}{N} \sum_{i=1}^Nf(X_i) + \frac{2}{N}\mathcal{R}_N(F) + C\sqrt{\frac{9\ln(2/\delta)}{2N}}
% \end{dmath*}
% \end{theorem}
% Conclude by plugin \cref{eq:ker_bound} in \cref{th:gen_rad_bound}.
% \end{proof}

% \subsection{Operator-valued Random Fourier Features}
% Suppose we want to learn an approximate function such that for all $x\in\mathcal{X}$, $\tilde{f}_*(x) \approx f_*(x)$, where $f_*$ is defined as in \cref{eq:pbOVK}. In this section we suppose that $K$ is a shift invariant $\mathcal{Y}$-Mercer kernel on the \acs{LCA} group $(\mathcal{X},\star)$. Brault et al. showed that such a Kernel can be written for all $y$, $y'$ in $\mathcal{Y}$
% \begin{dmath*}
% \inner{\tilde{\Phi}(x)y, \tilde{\Phi}(z)y'}\approx \inner{y, K(x,z)y'}
% \end{dmath*}
% where
% \begin{dmath*}
% \tilde{\Phi}(x)=\Vect_{j=1}^D \exp\left( -i\inner{x, \omega_j}B(\omega_j) \right), \enskip \omega_j \hiderel{\sim} \mu
% \label{eq:ORFF}
% \end{dmath*}
% such that $\inner{y, B(\omega)B(\omega)^\adjoint y'}\frac{d\mu}{d\omega}=\FT{\inner{y, K(\cdot y')})}(\omega)$. We are interested in solving \cref{eq:pbOVK} with the feature map defined from the kernel approximation defined in \cref{eq:ORFF}. Namely:
% \begin{dmath*}
% f_* = \argmin_{f\in\tilde{F}} \sum_{i=1}^NL_{y_i}(\tilde{f}(x_i))
% \hiderel{=} \argmin_{f\in\tilde{F}} \mathcal{R}_{emp}(\tilde{f}, L)
% = \argmin_{\theta\in\mathcal{Y}^D} \sum_{i=1}^NL_{y_i}(\tilde{\Phi}(x_i)^\adjoint \theta).
% \end{dmath*}
% \begin{proposition} Suppose that $\Tr[\tilde{K}_e(0)] \le T$ and $\norm{\theta}_2<\frac{B}{D}$. Let $L:\mathcal{Y}\times \mathcal{Y}\to[0, C]$ be a $L$-Lipschitz cost function. Then if we are given $N$ training points, we have with at least probability $1-2\delta$ that
% \begin{dmath*}
% \mathcal{R}_{true}(\tilde{f}, L) - \min_{f\in F}\mathcal{R}_{true}(f, L) \le O\left(\underbrace{\frac{1}{\sqrt{N}}\left(LBT^{1/2}+\frac{3C}{4}\sqrt{\ln\left(\frac{1}{\delta}\right)}\right)}_{\text{estimation error}}+\underbrace{\frac{LB}{\sqrt{D}}\left(1+\sqrt{\ln\left(\frac{1}{\delta}\right)}\right)}_{\text{approximation error}}\right)
% \end{dmath*}
% \label{pr:ovk_gen}
% \end{proposition}



% \begin{proof}
% We follow the proof of \url{https://people.eecs.berkeley.edu/~brecht/papers/08.rah.rec.nips.pdf}. It is an adaptation of the original proof in the light of the recent results of Mauer. We first define the two following sets:
% \begin{dmath}
% F=\Set{ f | f:\enskip x \mapsto \Phi_x^\adjoint  \theta, \enskip \norm{\theta(\omega)} < B\mu(\omega) }\hiderel{\subset} \mathcal{H}_K.
% \end{dmath}
% and
% \begin{dmath}
% \tilde{F}=\Set{ f | f: x \mapsto \tilde{\Phi}_x^\adjoint  \theta, \enskip \norm{\theta_j} < \frac{B}{D}}\hiderel{\subset} \mathcal{H}.
% \end{dmath}
% \begin{proposition}[Existence of approximation function] Let $\nu$ be a measure on $\mathcal{X}$, and $f_*$ a function in $F$. If $\omega_1, \hdots, \omega_D$ are drawn i.i.d. from $\mu$ then with probability at least $1 - \delta$, there exists a function $\tilde{f}\in \tilde{F}$ such that
% \begin{dmath}
% \sqrt{\int_{\mathcal{X}}\norm{f(x)-\tilde{f}(x)}^2_{\mathcal{Y}}d\mu(x)}\le \frac{B}{\sqrt{D}}\left(1 + \sqrt{2\log(1/\delta)} \right)
% \end{dmath}
% \label{pr:existence_app}
% \end{proposition}

% We use the following lemma of Rahimi and Recht:
% \begin{lemma}
% \label{lm:concentration_hilbert}
% Let $X_1, \hdots X_D$ be random variables with value in a ball $\mathcal{H}$ with radius $M$ centered around the origin in a Hilbert space. Denote the sample average $\bar{X}=\frac{1}{D}\sum_{j=1}^DX_j$. Then with probability $1-\delta$,
% \begin{dmath}
% \norm{\bar{X}-\expectation\bar{X}}\le \frac{M}{\sqrt{D}}\left(1 + \sqrt{2\log(1/\delta)} \right).
% \end{dmath}
% \end{lemma}
% \begin{proof}
% See \url{https://people.eecs.berkeley.edu/~brecht/papers/08.rah.rec.nips.pdf}, lemma 4 of the appendix.
% \end{proof}
% Since $f_*\in F$, we can write $f_*(x)=\int_{\omega\in\dual{\mathcal{X}}}\pairing{x,\omega}A(\omega)\theta(\omega)d\omega$. Construct the functions $f_j=\pairing{\cdot, \omega_j}A(\omega_j)\theta_j$, $j=1,\dots, D$ with $\theta_j:=\theta(\omega_j)/\mu(\omega_j)$. Let $\tilde{f}=\frac{1}{D}\sum_{j=1}^Df_j$ be the sample average of these functions. Then $\tilde{f}\in \tilde{F}$ because $\norm{\theta_j}/D\le B/D$. Also under the inner product $\inner{f,g}=\int_{\mathcal{X}} f(x)g(x)d\mu(x)$ we have that $\norm{\pairing{\cdot, \omega_j}A(\omega_j)\theta_j}\le B$. The claim follow by applying \cref{lm:concentration_hilbert} to $f_1, \hdots f_D$ under this inner product.
% \end{proof}

% \begin{proposition}[Bound on the approximation error] Suppose that $f\in F$ and $L_{y_i}$ is a $L$-Lipschitz cost function. With probability $1-\delta$ there exist a function $\tilde{f}\in \tilde{F}$ such that
% \begin{dmath}
% \mathcal{R}(\tilde{f}, L_{y_i})\le\mathcal{R}(f, L_{y_i}) + \frac{LB}{\sqrt{D}}\left(1 + \sqrt{2\log(1/\delta)} \right)
% \end{dmath}
% \label{pr:bound_app_error}
% \end{proposition}
% \begin{proof} For any two functions $f$ and $g$, the Lipschitz condition on $L_{y_i}$ followed by the concavity of square root gives
% \begin{dmath}
% \mathcal{R}(f, L) - \mathcal{R}(f, L) = \expectation \left[L(f(x)) - L(g(x))\right]
% \le \expectation \norm{L(f(x)) - L(g(x))}
% \le L \expectation \norm{f(x) - g(x)}
% \le L \sqrt{\expectation \norm{f(x) - g(x)}^2}
% \end{dmath}
% Eventually apply \cref{pr:existence_app} to conclude.
% \end{proof}
% \begin{proposition}[Bound on the estimation error]
% \label{pr:bound_est_error}
% Suppose $L: \mathcal{Y}\times\mathcal{Y}\to [0; C]$ is $L$-Lipschitz. Let $\omega_1, \hdots, \omega_D$ be fixed. If $\{(x_i, y_i)\}\subset(\mathcal{X}\times \mathcal{Y})^N$ are drawn i.i.d. from a fixed distribution, then for $\delta>0$ with probability $1-\delta$ over the dataset we have
% \begin{dmath}
% \forall\tilde{f}\in\mathcal{F}, \enskip \mathcal{R}_{true}(\tilde{f}, L) \le \mathcal{R}_{emp}(\tilde{f}, L) + \frac{1}{\sqrt{N}}\left( BL\sqrt{2\Tr\left[ \tilde{K}_e(0) \right]}+C\sqrt{\frac{9\ln(2/\delta)}{2}}\right)
% \end{dmath}
% \end{proposition}
% \begin{proof}
% Apply \cref{th:gen_rad_bound}. We get
% \begin{equation}
% \mathcal{R}_{true}(f, L) \le \mathcal{R}_{emp}(f, L) + \frac{2}{N}\mathcal{R}_N(F) + C\sqrt{\frac{9\ln(2/\delta)}{2N}}.
% \label{eq:est_bound}
% \end{dmath}
% Since $f\in F$, $\forall j\inrange{1}{D},\enskip\norm{\theta_j}_\mathcal{Y}<\frac{B}{D}$; and recall that $\tilde{\Phi}(x)=\vect_{j=1}^D\Phi_x(\omega_j)$. Now with similar arguments as in \cref{eq:ker_bound} we have
% \begin{dmath}
% \mathcal{R}_N(F) \le \sqrt{2}BL\sqrt{N\Tr\left[K(0)\right]}
% \end{dmath}
% Plugin back in \cref{eq:est_bound} yields the desired result.
% \end{proof}

%----------------------------------------------------------------------------------------
\section{Conclusions}
\label{sec:conclusions_construction}

\chapterend
