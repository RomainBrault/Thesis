% Chapter 1\chapter{Background} % Chapter title
\label{ch:background} % For referencing the chapter elsewhere, use \autoref{ch:introduction} 

%----------------------------------------------------------------------------------------
\section{Notations}
\label{sec:notations}
The euclidean inner product in $\mathbb{R}^d$ is denoted $\inner{\cdot, \cdot}$ and the euclidean norm is denoted $\norm{\cdot}$. The unit pure imaginary number $\sqrt{-1}$ is denoted $\iu$.
$\mathcal{B}(\mathbb{R}^d)$ is the Borel $\sigma$-algebra on $\mathbb{R}^d$.
If $\mathcal{X}$ and $\mathcal{Y}$ are two vector spaces, we denote by $\mathcal{F}(\mathcal{X};\mathcal{Y})$ the vector space of functions $f:\mathcal{X}\to\mathcal{Y}$ and $\mathcal{C}(\mathcal{X};\mathcal{Y})\subset\mathcal{F}(\mathcal{X};\mathcal{Y})$ the subspace of continuous functions.
If $\mathcal{H}$ is an Hilbert space we denote its scalar product by $\inner{.,.}_\mathcal{H}$ and its norm by $\norm{.}_\mathcal{H}$.
We set $\mathcal{L}(\mathcal{H})=\mathcal{L}(\mathcal{H};\mathcal{H})$ to be the space of linear operators from $\mathcal{H}$ to itself. If $W\in\mathcal{L}(\mathcal{H})$, $\Ker W$ denotes the nullspace, $\Ima W$ the image and $W^\adjoint \in \mathcal{L}(\mathcal{H})$ the adjoint operator (transpose when $W$ is a real matrix). All these notations are summarized in \cref{table:notations}.

\begin{table}[!ht]
\centering
\caption{Mathematical symbols used throughout the parper and their signification.}
\begin{tabularx}{\textwidth}{cX}
\toprule
Symbol & \multicolumn{1}{c}{Meaning} \\
\cmidrule{1-2}
$\iu$ & Unit pure imaginary number $\sqrt{-1}$. \\
$\ec$ & Euler constant. \\
$\inner{\cdot,\cdot}$ & Euclidean inner product. \\
$\norm{\cdot}$ & Euclidean norm. \\
$\mathcal{X}$ & Input space (LCA). \\
$\dual{\mathcal{X}}$ & The Pontryagin dual of $\mathcal{X}$. \\
$\mathcal{Y}$ & Output space (Hilbert space). \\
$\mathcal{H}$ & Feature space (Hilbert space). \\
$\inner{\cdot,\cdot}_{\mathcal{Y}}$ & The canonical inner product of the Hilbert space $\mathcal{Y}$. \\
$\norm{\cdot}_{\mathcal{Y}}$ & The canonical norm induced by the inner product of the Hilbert space $\mathcal{Y}$. \\
$\mathcal{F}(\mathcal{X};\mathcal{Y})$ & Vector space of function from $\mathcal{X}$ to $\mathcal{Y}$. \\
$\mathcal{C}(\mathcal{X};\mathcal{Y})$ & The vector subspace of $\mathcal{F}$ of continuous function from $\mathcal{X}$ to $\mathcal{Y}$. \\
$\mathcal{L}(\mathcal{H};\mathcal{Y})$ & The set of bounded linear operator from a Hilbert space $\mathcal{H}$ to a Hilbert space $\mathcal{Y}$. \\
$\mathcal{L}(\mathcal{Y})$ & The set of bounded linear operator from a Hilbert space $\mathcal{H}$ to itself. \\
$\mathcal{B}(\mathcal{X})$ & Borel $\sigma$-algebra on $\mathcal{X}$. \\
$\mu(\mathcal{X})$ & A scalar positive measure of $\mathcal{X}$. \\
$p_\mu(x)$ & The Radon-Nikodym derivative of $\mu$ w.r.t the Lebesgue measure. \\
$dx$, $d\omega$ & The canonical Haar measure of the LCA $(\mathcal{X},\mathcal{B}(\mathcal{X}))$. (resp. $(\dual{\mathcal{X}}, \mathcal{B}(\dual{\mathcal{X}})$). \\
$L^p(\mathcal{X},dx)$ & The Banach space of $\abs{\cdot}^p$-integrable function from $(\mathcal{X},\mathcal{B}(\mathcal{X},dx))$ to $\mathbb{C}$. \\
$L^p(\mathcal{X},dx;\mathcal{Y})$ & The Banach space of  $\norm{\cdot}_\mathcal{Y}^p$ (Bochner)-integrable function from $(\mathcal{X},\mathcal{B}(\mathcal{X}), dx)$ to $\mathcal{Y}$. \\
\bottomrule
\end{tabularx}
\label{table:notations}
\end{table}

%----------------------------------------------------------------------------------------
\section{About statistical learning}
\label{sec:about_statistical_learning}

%----------------------------------------------------------------------------------------
\section{On operator-valued kernels}
\label{sec:background_on_operator-valued_kernels}

%----------------------------------------------------------------------------------------
\section{On large-scale learning}
\label{sec:on_large-scale_learning}
