%----------------------------------------------------------------------------------------
\section{Motivations}
\label{sec:motivations}
Random Fourier Features have been proved useful to implement efficiently kernel methods in the scalar case, allowing to learn a linear model based on an approximated feature map. In this work, we are interested to construct approximated operator-valued feature maps to learn vector-valued functions. With an explicit (approximated) feature map, one converts the problem of learning a function $f$ in the vector-valued Reproducing Kernel Hilbert Space $\mathcal{H}_K$ into the learning of a linear model $\tilde{f}$ defined by:
 \begin{dmath*}
 \tilde{f}(x) = \tildePhi{\omega}(x)^\adjoint  \theta,
 \end{dmath*}
 where $\Phi: \mathcal{X} \to \mathcal{L}(\mathcal{H},\mathcal{Y})$ and $\theta \in \mathcal{H}$. The methodology we propose works for operator-valued kernels defined on any \acf{LCA} group, noted ($\mathcal{X}, \groupop)$, for some operation noted $\groupop$. This allows us to use the general context of Pontryagin duality for \acl{FT} of functions on \acs{LCA} groups. Building upon a generalization of Bochner's theorem for operator-valued measures, an operator-valued kernel is seen as the \emph{\acl{FT}} of an operator-valued positive measure. From that result, we extend the principle of Random Fourier Feature for scalar-valued kernels and derive a general methodology to build Operator Random Fourier Feature when operator-valued kernels are shift-invariant according to the chosen group operation.

\clearpage
%----------------------------------------------------------------------------------------
\section{Construction}
\label{sec:construction}
We present a construction of \acf{ORFF} such that $f: x\mapsto \tildePhi{\omega}(x)^\adjoint \theta$ is a continuous function that maps an arbitrary \acs{LCA} group $\mathcal{X}$ as input space to an arbitrary output Hilbert space $\mathcal{Y}$. First we define a functional \emph{Fourier feature map}, and then propose a Monte-Carlo sampling from this feature map to construct an approximation of a shift-invariant $\mathcal{Y}$-Mercer kernel.
Then, we prove the convergence of the kernel approximation $\tilde{K}(x,z)=\tildePhi{\omega}(x)^\adjoint \tildePhi{\omega}(z)$ with high probability on \emph{compact} subsets of the \acs{LCA} $\mathcal{X}$, when $\mathcal{Y}$ is \emph{finite dimensional}. Eventually we conclude with some numerical experiments.
\subsection{Theoretical study}
The following proposition of \citet{Zhang2012,Carmeli2010} extends Bochner's theorem to any shift-invariant $\mathcal{Y}$-Mercer kernel.
\begin{proposition}[Operator-valued Bochner's theorem \citep{Zhang2012,neeb1998operator}]\label{eq:bochner-gen}
If a continuous function $K$ from $\mathcal{X} \times \mathcal{X}$ to $\mathcal{Y}$ is a shift-invariant $\mathcal{Y}$-Mercer kernel on $\mathcal{X}$, then there exists a unique positive projection-valued measure $\dual{Q}: \mathcal{B}(\mathcal{X}) \to \mathcal{L}_+(\mathcal{Y})$ such that for all $x$, $z \in \mathcal{X}$,
\begin{dmath}
K(x, z) = \int_{\dual{\mathcal{X}}} \conj{\pairing{x \groupop \inv{z}, \omega}} d\dual{Q}(\omega),
\end{dmath}
where $\dual{Q}$ belongs to the set of all the projection-valued measures of bounded variation on the $\sigma$-algebra of Borel subsets of $\dual{\mathcal{X}}$. Conversely, from any positive operator-valued measure $M$, a shift-invariant kernel $K$ can be defined by \cref{eq:bochner-gen}.
\end{proposition}
Although this theorem is central to the spectral decomposition of shift-invariant $\mathcal{Y}$-Mercer \acs{OVK}, the following results proved by \citet{Carmeli2010} provides insights about this decomposition that are more relevant in practice. It first gives the necessary conditions to build shift-invariant $\mathcal{Y}$-Mercer kernel with  a pair $(A, \dual{\mu})$ where $A$ is an operator-valued function on $\dual{\mathcal{X}}$ and $\dual{\mu}$ is a real-valued positive measure on $\dual{\mathcal{X}}$. Note that obviously such a pair is not unique ad the choice of this paper may have an impact on theoretical properties as well as practical computations.
Secondly it also states that any \acs{OVK} have such a spectral decomposition when $\mathcal{Y}$ is finite dimensional or $\mathcal{X}$.

\begin{proposition}[\citet{Carmeli2010}]\label{pr:mercer_kernel_bochner}
Let $\dual{\mu}$ be a positive measure on $\mathcal{B}(\mathcal{\dual{\mathcal{X}}})$ and $A: \dual{\mathcal{X}}\to \mathcal{L}(\mathcal{Y})$ such that $\inner{A(\cdot)y,y'}\in L^1(\mathcal{X},\dual{\mu})$ for all $y,y'\in\mathcal{Y}$ and $A(\omega)\succcurlyeq 0$ for $\dual{\mu}$-almost all $\omega\in\dual{\mathcal{X}}$. Then, for all $\delta \in \mathcal{X}$,
\begin{dmath}
\label{eq:AK0}
K_e(\delta)=\int_{\dual{\mathcal{X}}}\conj{\pairing{\delta,\omega}}A(\omega)d\dual{\mu}(\omega)
\end{dmath}
is the kernel signature of a shift-invariant $\mathcal{Y}$-Mercer kernel $K$ such that $K(x,z)=K_e(x \groupop \inv{z})$. The \acs{vv-RKHS} $\mathcal{H}_K$ is embed in $L^2(\dual{\mathcal{X}},\dual{\mu};\mathcal{Y}')$ by mean of the feature operator
\begin{dmath}
\label{eq:feature_operator}
(Wg)(x)=\int_{\mathcal{\dual{X}}}\conj{\pairing{x,\omega}}B(\omega)g(\omega)d\dual{\mu}(\omega),
\end{dmath}
Where $B(\omega)B(\omega)^\adjoint=A(\omega)$ and both integral converges in the weak sense. If $\mathcal{Y}$ is finite dimensional or $\mathcal{X}$ is compact, any shift-invariant kernel is of the above form for some pair $(A, \dual{\mu})$.
\end{proposition}
\paragraph{}
When $p=1$ one can always assume $A$ is reduced to the scalar $1$, $\dual{\mu}$ is still a bounded positive measure and we retrieve the Bochner theorem applied to the scalar case (\cref{th:bochner-scalar}).
\paragraph{}
\Cref{pr:mercer_kernel_bochner} shows that a given pair $(A,\dual{\mu})$ characterize an \acs{OVK}. Namely given a measure $\dual{\mu}$ and a function $A$ such that $\inner{A(.)y,y'}\in L^1(\mathcal{X},\dual{\mu})$ for all $y,y'\in\mathcal{Y}$ and $A(\omega)\succcurlyeq 0$ for $\dual{\mu}$-almost all $\omega$, it gives rise to an \acs{OVK}. Since $(A,\dual{\mu})$ determine a unique kernel we can write $\mathcal{H}_{(A,\dual{\mu})}{\scriptstyle\implies}\mathcal{H}_K$ where $K$ is defined as in \cref{eq:AK0}. However the converse is to true: Given a $\mathcal{Y}$-Mercer shift invariant \acl{OVK}, there exist infinitely many pairs $(A,\dual{\mu})$ that characterize an \acs{OVK}.
\paragraph{}
The main difference between \cref{eq:bochner-gen} and \cref{pr:mercer_kernel_bochner} is that the first one characterize an \acs{OVK} by a unique \acf{POVM}, while the second one shows that the \acs{POVM} that uniquely characterize a $\mathcal{Y}$-Mercer \acs{OVK} has an operator-valued density with respect to a \emph{scalar} measure $\dual{\mu}$; and that this operator-valued density is not unique.
\paragraph{}
Finally \cref{pr:mercer_kernel_bochner} does not provide any \emph{constructive} way to obtain the pair $(A,\dual{\mu})$ that characterize an \acs{OVK}.
The following \cref{subsec:sufficient_conditions} is based on an other proposition of \citeauthor{carmeli2006vector} and show that if the kernel signature $K_e(\delta)$ of an $\acs{OVK}$ is in $L^1$ then it is possible to construct \emph{explicitly} a pair $(C,\dual{\Haar})$ from it. Additionally, we show that we can always extract a scalar-valued \emph{probability} density function from $C$ such that we obtain a pair $(A,\probability_{\dual{\mu},\rho})$ where $\probability_{\dual{\mu},\rho}$ is a \emph{probability} distribution absolutely continuous with respect to $\dual{\mu}$ and with associated probability density function (\pdf) $\rho$. Thus for all $\mathcal{Z}\subset\mathcal{B}(\dual{\mathcal{X}})$,
\begin{dmath*}
\probability_{\dual{\mu},\rho}(\mathcal{Z})=\int_{\mathcal{Z}} \rho(\omega)d\dual{\mu}(\omega).
\end{dmath*}
When the reference measure $\dual{\mu}$ is the Lebesgue measure, we note $\probability_{\dual{\mu},\rho}=\probability_\rho$.

\subsection{Sufficient conditions of existence}
\label{subsec:sufficient_conditions}
While \cref{pr:mercer_kernel_bochner} gives some insights on how to build an approximation of a $\mathcal{Y}$-Mercer kernel, we need a theorem that provides an explicit construction of the pair $(A, \probability_{\dual{\mu},\rho})$ from the kernel signature $K_e$. Proposition 14 in \citet{Carmeli2010} gives the solution, and also provide a sufficient condition for \cref{pr:mercer_kernel_bochner} to apply.
\begin{proposition}[\citet{Carmeli2010}]
\label{pr:inverse_ovk_Fourier_decomposition}
Let $K$ be a shift-invariant $\mathcal{Y}$-Mercer kernel. %
Suppose that for all $z \in \mathcal{X}$ and for all $y$, $y' \in\mathcal{Y}$, $\inner{K_e(.)y,y'}\in L^1(\mathcal{X},\Haar)$ where $\mathcal{X}$ is endowed with the group law $\groupop$. For all $\omega \in \dual{\mathcal{X}}$ and for all $y$, $y'$ in $\mathcal{Y}$, let
\begin{dmath}\label{eq:CK0}
\inner{y',C(\omega)y} = \int_{\mathcal{X}} \pairing{\delta, \omega}\inner{y', K_e(\delta)y}d\delta = \IFT{\inner{y', K_e(\cdot)y}}(\omega).
\end{dmath}
Then
\begin{propenum}
\item $C(\omega)$ is a bounded non-negative operator for all $\omega \in \dual{\mathcal{X}}$,
\item $\inner{y, C(\cdot)y'}\in L^1(\dual{\mathcal{X}},\dual{\Haar})$ for all $y,y'\in\mathcal{X}$,
\item for all $\delta\in\mathcal{X}$ and for all $y$, $y'$ in $\mathcal{Y}$,
\begin{dmath*}
\inner{y', K_e(\delta)y}= \int_{\dual{\mathcal{X}}}\conj{\pairing{\delta,\omega}}\inner{y', C(\omega)y}d\dual{\Haar}(\omega)
=\FT{\inner{y', C(\cdot)y}}(\delta).
\end{dmath*}
\end{propenum}
\end{proposition}
There have been a lot of confusion in the literature whether a kernel is the \acl{FT} or \acl{IFT} of a measure. However \cref{lm:C_characterization} clarify the relation between the \acl{FT} and \acl{IFT} for a translation invariant \acl{OVK}. Notice that in the real scalar case the \acl{FT} and \acl{IFT} of a shift-invariant kernel are the same, while the difference is significant for \acs{OVK}.
\paragraph{}
The following lemma is a direct consequence of the definition of $C(\omega)$ as the \acl{FT} of the adjoint of $K_e$ and also helps simplifying the definition of \acs{ORFF}.
\begin{lemma}
\label{lm:C_characterization}
Let $K_e$ be the signature of a shift-invariant $\mathcal{Y}$-Mercer kernel and let $\inner{y', C(\cdot)y}=\IFT{\inner{y', K_e(\cdot)y}}$ for all $y$. $y'\in\mathcal{Y}$. Then
\begin{propenum}
\item \label{lm:C_characterization_1} $C(\omega)$ is self-adjoint and $C$ is even.
\item \label{lm:C_characterization_2} $\IFT{\inner{y', K_e(\cdot)y}} = \FT{\inner{y', K_e(\cdot)y}}$.
\item \label{lm:C_characterization_3} $K_e(\delta)$ is self-adjoint and $K_e$ is even.
\end{propenum}
\end{lemma}
\begin{proof}
For any function $f$ on $(\mathcal{X},\groupop)$ define the flip operator $\mathcal{R}$ by
\begin{dmath*}
\mathcal{R}f(x) \colonequals f\left(\inv{x}\right).
\end{dmath*}
For any shift invariant $\mathcal{Y}$-Mercer kernel and for all $\delta\in\mathcal{X}$,  $K_e(\delta)=K_e\left(\inv{\delta}\right)^\adjoint$. Indeed,
\begin{dmath*}
\mathcal{R}\inner{y,K_e\left(x\groupop \inv{z}\right)y'}
=\inner{y,K_e\left(\inv{\left(x\groupop \inv{z}\right)}\right)y'}
=\inner{y,K_e\left(\inv{x}\groupop z\right)y'}
=\inner{y,K_e\left(x\groupop \inv{z}\right)^\adjoint y'}.
\end{dmath*}
\Cref{lm:C_characterization_1}: taking the \acl{FT} yields,
\begin{dmath*}
\IFT{\inner{y', K_e(\cdot)y}}=\mathcal{F}^{-1}\mathcal{R}\left[\inner{y', K_e(\cdot)y}\right]
\hiderel{=}\mathcal{R}\inner{y', C(\cdot)y}.
\end{dmath*}
Hence $C(\omega)=C\left(\inv{\omega}\right)^\adjoint$. Suppose that $\mathcal{Y}$ is a complex Hilbert space. Since for all $\omega\in\mathcal{\dual{X}}$, $C(\omega)$ is bounded and non-negative so $C(\omega)$ is self-adjoint. Besides we have $C(\omega)=C\left(\inv{\omega}\right)^\adjoint $ to $C$ must be pair. Suppose that $\mathcal{Y}$ is a real Hilbert space. Then we have the additional hypothesis that $K_e(\delta)=K_e(\delta)^\adjoint$. Taking the \acl{FT} yields that $C(\omega)=C(\omega)^\adjoint$. Since for any shift invariant $\mathcal{Y}$-Mercer kernel $C(\omega)=C\left(\inv{\omega}\right)^\adjoint$ we also conclude that $C\left(\inv{\omega}\right)=C(\omega)$.
\paragraph{}
\Cref{lm:C_characterization_2}: simply, for all $y$, $y'\in\mathcal{Y}$, $\inner{y, C(\inv{\omega})y'}$ $=$ $\inner{y', C(\omega)y}$ thus $\IFT{\inner{y', C(\cdot)y}}=\mathcal{F}\mathcal{R}\left[\inner{y', C(\cdot)y}\right]=\FT{\inner{y', C(\cdot)y}}$.
\paragraph{}
\Cref{lm:C_characterization_3}: from \cref{lm:C_characterization_2} we have $\IFT{\inner{y', K_e(\cdot)y}}$ $=$ $\mathcal{F}^{-1}\mathcal{R}{\inner{y', K_e(\cdot)y}}$. By injectivity of the \acl{FT}, $K_e$ is even. Since $K_e(\delta)=K_e(\inv{\delta})^\adjoint $, we must have $K_e(\delta)=K_e(\delta)^\adjoint $.
\end{proof}
While \cref{pr:inverse_ovk_Fourier_decomposition} gives an explicit form of the operator $C(\omega)$ defined as the \acl{FT} of the kernel $K$, it is not really convenient to work with the Haar measure $\dual{\Haar}$ on $\mathcal{B}(\dual{\mathcal{X}})$. However it is easily possible to turn $\dual{\Haar}$ into a probability measure to allow efficient integration over an infinite domain.
\paragraph{}
The following proposition allows to build a spectral decomposition of a shift-invariant $\mathcal{Y}$-Mercer kernel on a \acs{LCA} group $\mathcal{X}$ endowed with the group law $\groupop$ with respect to a scalar probability measure, by extracting a scalar probability density function from $C$.
\begin{proposition}[Shift-invariant $\mathcal{Y}$-Mercer kernel spectral decomposition]
\label{pr:spectral}
Let $K_e$ be the signature of a shift-invariant $\mathcal{Y}$-Mercer kernel. If for all $y$, $y' \in\mathcal{Y}$, $\inner{K_e(.)y,y'}\in L^1(\mathcal{X},\Haar)$ then there exists a positive probability measure $\probability_{\dual{\Haar},\rho}$ and an operator-valued function $A$ an such that for all $y,$ $y'\in\mathcal{Y}$,
\begin{dmath}
\label{eq:expectation_spec}
\inner{y', K_e(\delta)y}
=\expectation_{\dual{\Haar},\rho}\left[\conj{\pairing{\delta, \omega}}\inner{y', A(\omega)y}\right],
\end{dmath}
with
\begin{dmath}
\label{eq:comega}
\inner{y', A(\omega)y}\rho(\omega) = \FT{\inner{y', K_e(\cdot)y}}(\omega).
\end{dmath}
Moreover
\begin{propenum}
\item for all $y,$ $y'\in\mathcal{Y}$, $\inner{A(.)y,y'}\in L^1(\dual{\mathcal{X}}, \probability_{\dual{\Haar},\rho})$,
\item $A(\omega)$ is non-negative for $\probability_{\dual{\Haar},\rho}$-almost all $\omega\in\dual{\mathcal{X}}$,
\item $A(\cdot)$ and $\rho(\cdot)$ are even functions.
\end{propenum}
\end{proposition}
\begin{proof}
This is a simple consequence of \cref{pr:inverse_ovk_Fourier_decomposition} and \cref{lm:C_characterization}. By taking $\inner{y',C(\omega)y} = \IFT{\inner{y', K_e(\cdot)y}}(\omega)=\FT{\inner{y', K_e(\cdot)y}}(\omega)$ we can write the following equality concerning the \acs{OVK} signature $K_e$.
% Suppose that $\mu$ is absolutely continuous \wrt~$d\omega$. Then for all $\delta \in \mathcal{X}$ and for all $y,$ $y'$ in $\mathcal{Y}$
\begin{dmath*}
\inner{y', K_e(\delta)y}(\omega)=
\int_{\dual{\mathcal{X}}}\conj{\pairing{\delta, \omega}}\inner{y', C(\omega)y}d\dual{\Haar}(\omega)
=\int_{\dual{\mathcal{X}}}\conj{\pairing{\delta, \omega}}\inner*{y', \frac{1}{\rho(\omega)}C(\omega)y}\rho(\omega)d\dual{\Haar}(\omega).
\end{dmath*}
It is always possible to choose $\rho(\omega)$ such that $\int_{\dual{\mathcal{X}}}\rho(\omega)d\dual{\Haar}(\omega)=1$. For instance choose
\begin{dmath*}
\rho(\omega)=\frac{\norm{C(\omega)}_{\mathcal{Y},\mathcal{Y}}}{\int_{\dual{\mathcal{X}}}\norm{C(\omega)}_{\mathcal{Y},\mathcal{Y}}d\dual{\Haar}(\omega)}
\end{dmath*}
Since for all $y$, $y'\in\mathcal{Y}$, $\inner{y',C(\cdot)y}\in L^1(\dual{\mathcal{X}},\dual{\Haar})$ and $\mathcal{Y}$ is a separable Hilbert space, by pettis measurability theorem, $\int_{\dual{\mathcal{X}}}\norm{C(\omega)}_{\mathcal{Y},\mathcal{Y}}d\dual{\Haar}(\omega)$ is finite and so is $\norm{C(\omega)}_{\mathcal{Y},\mathcal{Y}}$ for all $\omega\in\dual{\mathcal{X}}$.
Therefore $\rho(\omega)$ is the density of a probability measure $\probability_{\dual{\Haar},\rho}$, \ie~conclude by taking
\begin{dmath*}
\probability_{\dual{\Haar},\rho}(\mathcal{Z}) = \int_{\mathcal{Z}}\rho(\omega)d\dual{\Haar}(\omega),
\end{dmath*}
for all $\mathcal{Z}\in\mathcal{B}(\dual{\mathcal{X}})$.
\end{proof}
\paragraph{}
In the case where $\mathcal{Y}=\mathbb{R}^p$, we rewrite \cref{eq:comega} coefficient-wise by choosing an orthonormal basis $\Set{e_j}_{j\in\mathbb{N}_p}$ of $\mathbb{R}^p$.
\begin{dmath}
\label{eq:operator_identification_real}
A(\omega)_{ij}\rho(\omega)\hiderel{=}\FT{K_e(\cdot)_{ij}}(\omega).
\end{dmath}
It follows that for all $i$ and $j$ in $\mathbb{N}_{p}$,
\begin{dmath}\label{eq:matrix-exp}
K_e(x\groupop \inv{z})_{ij} \hiderel{=} \FT{A(\cdot)_{ij}\rho(\cdot)}(x\groupop \inv{z})
\end{dmath}

\begin{remark}
Note that although the \acl{FT} of $K_e$ yields a unique operator-valued function $C(\cdot)$, the decomposition of $C(\cdot)$ into $A(\cdot)\rho(\cdot)$ is again not unique. The choice of the decomposition may be justified by the computational cost or by the nature of the constants involved in the uniform convergence of the estimator.
\end{remark}
Another difficulty arise from the fact that the quantity $\sup_{\omega\in\dual{\mathcal{X}}} \norm{A(\omega)}_{\mathcal{Y},\mathcal{Y}}$ obtained in \cref{pr:spectral} might not bounded. The unboundedness of $\norm{A(\cdot)}_{\mathcal{Y},\mathcal{Y}}$ forbid the use of the most simple concentrations inequalities -- which require the boundedness of the random variable to be controlled. Therefore in the context of \acl{OVK} concentration inequalities for unbounded random operators should be used. However, as pointed out by \citet{minh2016operator}, under some condition on the trace of $K_e(\delta)$, it is possible to turn $A(\cdot)$ into a bounded random operator for all $\omega$ in $\dual{\mathcal{X}}$. The idea is to define a sum measure $\rho=\sum_{j\in\mathbb{N}}\rho_{e_j}$, which gives rise to bounded operator $A(\omega)$ and is idependant of the $\Set{e_j}_{j\in\mathbb{N}}$ base, instead of constructing a measure from the operator norm as in \cref{pr:spectral}. Additionally with such construction the measure associated to $A(\cdot)$ is \emph{independant} from the basis of $\mathcal{Y}$. In this proof we relax the assumptions of \citet{minh2016operator} which requires $\int_{\mathcal{X}}\abs{\Tr{K_e(\delta)}}d\Haar(\delta)$ to be well defined. We only require $\Tr{K_e(e)}$ to be well defined.
\begin{proposition}[Bounded shift-invariant $\mathcal{Y}$-Mercer kernel spectral decomposition]
\label{pr:trace_measure}
Let $K_e$ be the signature of a shift-invariant $\mathcal{Y}$-Mercer kernel. If for all $y$ and $y'$ in $\mathcal{Y}$, $\inner{K_e(.)y,y'}\in L^1(\mathcal{X},\Haar)$ and $\Tr K_e(e)\in\mathbb{R}$, then
\begin{dmath}
\label{eq:expectation_tr}
\inner{y', K_e(\delta)y}
=\expectation_{\dual{\Haar},\rho_{\Tr}}\left[\conj{\pairing{\delta, \omega}}\inner{y, A_{\Tr}(\omega)y'}\right].
\end{dmath}
with
\begin{dgroup}
\begin{dmath}
\inner{y', C(\cdot)y}=\FT{\inner{y', K_e(\cdot)y}}
\end{dmath}
\begin{dmath}
c_{\Tr}=\Tr\left[K_e(e)\right]
\end{dmath}
\begin{dmath}
\label{eq:bounded_C}
A_{\Tr}(\omega)=c_{\Tr}\Tr\left[C(\omega)\right]^{-1}C(\omega)
\end{dmath}
\begin{dmath}
\rho_{\Tr}(\omega)=c_{\Tr}^{-1}\Tr\left[C(\omega)\right].
\end{dmath}

\end{dgroup}
\label{eq:bounded_mu}
Moreover
\begin{propenum}
\item For all $y$, $y'\in\mathcal{Y}$, $\inner{y, A_{\Tr}(\cdot)y'}\in L^1(\dual{\mathcal{X}}, \probability_{\dual{\Haar},\rho_{\Tr}})$.
\item $A_{\Tr}(\omega)$ is non-negative for all $\omega\in\dual{\mathcal{X}}$,
\item $\sup_{\omega\in\dual{\mathcal{X}}}\norm{A_{\Tr}(\omega)}_{\mathcal{Y},\mathcal{Y}}\le c_p$,
\item $A_{\Tr}(\cdot)$ and $\rho_{\Tr}$ are even functions.
\end{propenum}
\end{proposition}
\begin{proof}
Let $\Set{e_j}_{j\in\mathbb{N}}$ be an orthonormal basis of $\mathcal{Y}$. Notice that
\begin{dmath*}
\int_{\dual{\mathcal{X}}}\inner{e_j,C(\omega)e_j}d\dual{\Haar}(\omega)
=\int_{\dual{\mathcal{X}}}\underbrace{\conj{\pairing{e,\omega}}}_{=1}\inner{e_j,C(\omega)e_j}d\dual{\Haar}(\omega)
=\inner{e_j,K_e(e)e_j}.
\end{dmath*}
Since $C(\omega)$ is non-negative, all the $\inner{e_j,C(\omega)e_j}$. Thus using the monotone convergence theorem,
\begin{dmath*}
\int_{\dual{\mathcal{X}}}\Tr\left[C(\omega)\right] d\dual{\Haar}(\omega)
       =\int_{\dual{\mathcal{X}}}\sum_{j\in\mathbb{N}}\inner{e_j,C(\omega)e_j}d\dual{\Haar}(\omega)
       =\sum_{k\in\mathbb{N}}\inner{e_j,K_e(e)e_j}
       =\Tr\left[K_e(e)\right]\hiderel{=}c_{\Tr}\hiderel{<}\infty.
\end{dmath*}
Let $A_{\Tr}(\omega)$ and $\rho_{\Tr}(\omega)$ be defined respectively as in \cref{eq:bounded_C} and \cref{eq:bounded_mu}. By definition, $\int_{\dual{\mathcal{X}}}\rho_{\Tr}(\omega)d\dual{\Haar}(\omega)=1$ and $A_{\Tr}(\omega)\rho_{\Tr}(\omega)=C(\omega)$. Now it remains to check the finiteness of $\Tr\left[C(\omega)\right]$ for all $\omega\in\dual{\mathcal{X}}$. Since for all $\omega\in\dual{\mathcal{X}}$, $\Tr\left[C(\omega)\right]\ge 0$,
\begin{dmath*}
\Tr\left[C(\omega)\right] \le \int_{\dual{\mathcal{X}}} \Tr\left[C(\omega)\right] d\dual{\Haar}(\omega) \hiderel{=} \Tr\left[K_e(e)\right] \hiderel{<} \infty.
\end{dmath*}
Since $\Tr\left[C(\omega)\right]$ is positive and its integral is finite, $\rho_{\Tr}$ is a probability density function. The Schatten norms $\norm{\cdot}_p$ verifies $\Tr\left[\abs{\cdot}\right]=\norm{\cdot}_1\ge\norm{\cdot}_p\ge \norm{\cdot}_q \ge \norm{\cdot}_{\mathcal{Y},\mathcal{Y}}=\norm{\cdot}_{\infty}$ for all $p$, $q\in\mathbb{N}$ such that $1\le p \le q$. Therefore since for all $\omega\in\dual{\mathcal{X}}$, $C(\omega)$ is non-negative,
\begin{dmath*}
\norm{A_{\Tr}(\omega)}_{\mathcal{Y},\mathcal{Y}}=c_{\Tr}\Tr\left[C(\omega)\right]^{-1}\norm{C(\omega)}_{\infty}
\le c_{\Tr}\Tr\left[C(\omega)\right]^{-1}\norm{C(\omega)}_{1}
=c_{\Tr}\Tr\left[C(\omega)\right]^{-1}\Tr\left[\abs{C(\omega)}\right]
=c_{\Tr}\Tr\left[C(\omega)\right]^{-1}\Tr\left[C(\omega)\right]
\le c_{\Tr} \hiderel{<} \infty.
\end{dmath*}
Thus $\sup_{\omega\in\dual{\mathcal{X}}}\norm{A(\omega)}_{\mathcal{Y},\mathcal{Y}} \le c_{\Tr} < \infty$. As $C$ is an even function, so are $A_{\Tr}$ and $\rho_{\Tr}$. Eventually $\inner{y', C(\cdot)y}$ is in $L^1(\dual{\mathcal{X}}, \dual{\Haar})$, thus $\inner{y, A_{\Tr}(\cdot)\rho_{\Tr}(\cdot)y'}$ is in $L^1(\dual{\mathcal{X}}, \dual{\Haar})$, hence $\inner{y, A_{\Tr}(\cdot)y'}\in L^1(\dual{\mathcal{X}}, \probability_{\dual{\Haar},\rho_{\Tr}})$. Since the trace is idenpendent of the basis of $\mathcal{Y}$, so is $\rho_{\Tr}$.
\end{proof}
If $\mathcal{Y}$ is finite dimensional then $\Tr\left[K_e(e)\right]$ is well defined hence \cref{pr:trace_measure} is valid as long as $K_e(\cdot)_{ij}\in L^1(\mathcal{X}, \Haar)$ for all $i$, $j\in\mathbb{N}_{p}$, where $p$ is the dimension of $\mathcal{Y}$.

\subsection{Examples of spectral decomposition}
\label{subsec:dec_examples}
In this section we give exemple of spectral decomposition of various $\mathcal{Y}$-Mercer kernel, based on \cref{pr:spectral} and \cref{pr:trace_measure}.
\subsubsection{Gaussian decomposable kernel}
\label{par:gaussian_dec}
Recall that a decomposable $\mathbb{R}^p$-Mercer has the form $K(x,z)=k(x,z)\Gamma$, where $k(x,z)$ is a scalar Mercer kernel and $\Gamma\in\mathcal{L}(\mathbb{R}^p)$ is a non-negative operator. Let $K^{dec,gauss}_e(\cdot)=k_e^{gauss}(\cdot)\Gamma$ be the Gaussian decomposable kernel where $K_e$ and $k_e$ are respectively the signature of $K$ and $k$ on the additive group $\mathcal{X}=(\mathbb{R}^d,+)$ -- $\ie~\delta=x-z$ and $e=0$. The scalar Gaussian kernel reads for all $\delta\in\mathbb{R}^d$
\begin{dmath*}
k^{\text{gauss}}_0(\delta)\hiderel{=}\exp\left( -\frac{1}{2\sigma^2}\norm{\delta}^2_2\right)
\end{dmath*}
where $\sigma \in \mathbb{R}_+$ is an hyperparameter corresponding to the bandwith of the kernel. The --Pontryagin-- dual group of $\mathcal{X}=(\mathbb{R}^d,+)$ is $\dual{\mathcal{X}}\cong(\mathbb{R}^d,+)$ with the pairing
\begin{dmath*}
\pairing{\delta,\omega}=\exp\left(\iu\inner{\delta,\omega}\right)
\end{dmath*}
where $\delta$ and $\omega\in\mathbb{R}^d$. In this case the Haar measures on $\mathcal{X}$ and $\dual{\mathcal{X}}$ are in both case the Lebesgue measure. However in order to have the property that $\IFT{\FT{f}}=f$ and $\IFT{f}=\mathcal{R}\FT{f}$ one must normalize both measures by $\sqrt{2\pi}^{-d}$, \ie~for all $\mathcal{Z}\in\mathcal{B}\left(\mathbb{R}^d\right)$,
\begin{dgroup*}
\begin{dmath*}
\sqrt{2\pi}^{d}\Haar(\mathcal{Z}) = \Leb(\mathcal{Z}) \text{ and}
\end{dmath*}
\begin{dmath*}
\sqrt{2\pi}^{d}\dual{\Haar}(\mathcal{Z}) = \Leb(\mathcal{Z}).
\end{dmath*}
\end{dgroup*}
Then the \acl{FT} on $(\mathbb{R}^d,+)$ is
\begin{dmath*}
\FT{f}(\omega)
=\int_{\mathbb{R}^d}\exp\left(-\iu\inner{\delta,\omega}\right)f(x)d\Haar(\delta)
=\int_{\mathbb{R}^d}\exp\left(-\iu\inner{\delta,\omega}\right)f(x)\frac{d\Leb(\delta)}{\sqrt{2\pi}^d}.
\end{dmath*}
Since $k^{\text{gauss}}_0\in L^1$ and $\Gamma$ is bounded, it is possible to apply \cref{pr:spectral}, and obtain for all $i$, $j\in\mathbb{N}_p$,
\begin{dmath*}
C^{dec,gauss}(\omega)_{ij}=\FT{K^{dec,gauss}_0(\cdot)_{ij}}(\omega)
=\FT{k_0^{gauss}}(\omega)\Gamma_{ij}
=\int_{\mathbb{R}^d}\exp\left(-\iu\inner{\omega,x}\right)\exp\left( -\frac{\norm{\delta}^2_2}{2\sigma^2}\right)\frac{d\Leb(\delta)}{\sqrt{2\pi}^d} \Gamma_{ij}
=\frac{1}{\sqrt{2\pi\frac{1}{\sigma^2}}^d}\exp\left( -\frac{\sigma^2}{2}\norm{\omega}^2_2\right)\sqrt{2\pi}^d\Gamma_{ij}.
\end{dmath*}
Hence
\begin{dmath*}
C^{dec,gauss}(\omega)
=\underbrace{\frac{1}{\sqrt{2\pi\frac{1}{\sigma^2}}^d}\exp\left( -\frac{\sigma^2}{2}\norm{\omega}^2_2\right)\sqrt{2\pi}^d}_{\rho(\cdot)
=\mathcal{N}(0,\sigma^{-2}I_d)\sqrt{2\pi}^d}\underbrace{\Gamma}_{A(\cdot)=\Gamma}
\end{dmath*}
Therefore the canonical decomposition of $C^{dec,gauss}$ is $A^{dec,gauss}(\omega)=\Gamma$ and $\rho^{dec,gauss}=\mathcal{N}(0,\sigma^{-2}I_d)\sqrt{2\pi}^d$, where $\mathcal{N}$ is the Gaussian probability distribution. Note that this decomposition is done with respect to the \emph{normalized} Lebesgue measure $\dual{\Haar}$, meaning that for all $\mathcal{Z}\in\mathcal{B}(\dual{\mathcal{X}})$,
\begin{dmath*}
\probability_{\dual{\Haar},\mathcal{N}(0,\sigma^{-2}I_d)\sqrt{2\pi}^d}(\mathcal{Z})=\int_{\mathcal{Z}}\mathcal{N}(0,\sigma^{-2}I_d)\sqrt{2\pi}^dd\dual{\Haar}(\omega)
=\int_{\dual{\mathcal{X}}}\mathcal{N}(0,\sigma^{-2}I_d)d\Leb(\omega)
=\probability_{\mathcal{N}(0,\sigma^{-2}I_d)}(\mathcal{Z})
\end{dmath*}
Thus, the same decomposition with respect to the usual --non-normalized-- Lebesgue measure $\Leb$ yields
\begin{dgroup}
\begin{dmath}
A^{dec,gauss}(\cdot)=\Gamma
\end{dmath}
\begin{dmath}
\rho^{dec,gauss}=\mathcal{N}(0,\sigma^{-2}I_d)
\end{dmath}
\end{dgroup}
If $\Gamma$ is a trace class operator, applying \cref{pr:trace_measure} yields the same decomposition since $\Tr\left[K^{dec,gauss}_0(0)\right]=\Tr\left[\Gamma\right]$ and $\Tr\left[C^{dec,gauss}(\cdot)\right]=\mathcal{N}(0,\sigma^{-2}I_d)\sqrt{2\pi}^d\Tr\left[\Gamma\right]$.

\subsubsection{Skewed-$\chi^2$ decomposable kernel}
\label{subsubsec:skewedchi2}
The skewed-$\chi^2$ scalar kernel is defined on the \acs{LCA} product group $\mathcal{X}=((-c_k;+\infty)_{k=1}^d,\odot)$, with $c_k\in\mathbb{R}_{+}$. Let $(e_k)_{k=1}^d$ be the standard basis of $\mathcal{X}$ and ${}_k:x\mapsto \inner{x,e_k}$. The operator $\odot: \mathcal{X}\times\mathcal{X}\to\mathcal{X}$ is defined by
\begin{dmath*}
x\odot z = \left((x_k + c_k)(z_k + c_k) - c_k\right)_{k=1}^d.
\end{dmath*}
The identity element $e$ is $\left(1-c_k\right)_{k=1}^d$ since $(1-c) \odot x = x$. Thus the inverse element $x^{-1}$ is $((x_k+c_k)^{-1} - c_k)_{k=1}^d$. The skewed-$\chi^2$ scalar kernel reads
\begin{dmath}
k^{skewed}_{1-c}(\delta)=\prod_{k=1}^d\frac{2}{\sqrt{\delta_k+c_k}+\sqrt{\frac{1}{\delta_k+c_k}}}.
\end{dmath}
The dual of $\mathcal{X}$ is $\dual{\mathcal{X}}\cong\mathbb{R}$ with the pairing
\begin{dmath*}
\pairing{\delta,\omega}=\prod_{k=1}^d\exp\left(\iu\inner{\log(\delta_k+c_k),\omega_k}\right).
\end{dmath*}
The Haar measure are defined for all $\mathcal{Z}\in\mathcal{B}((-c;+\infty)^d)$ and all $\dual{\mathcal{Z}}\in\mathcal{B}(\mathbb{R}^d)$ by
\begin{dgroup*}
\begin{dmath*}
\sqrt{2\pi}^d\Haar(\mathcal{Z})=\int_{\mathcal{Z}}\prod_{k=1}^d\frac{1}{z_k+c_k}d\Leb(z)
\end{dmath*}
\begin{dmath*}
\sqrt{2\pi}^d\dual{\Haar}(\dual{\mathcal{Z}})=\Leb(\dual{\mathcal{Z}}).
\end{dmath*}
\end{dgroup*}
Thus the \acl{FT} is
\begin{dmath*}
\FT{f}(\omega)=\int_{(-c;+\infty)^d}\prod_{k=1}^d\frac{\exp\left(-\iu\inner{\log(\delta_k+c_k),\omega_k}\right)}{t_k + c_k}f(\delta)\frac{d\Leb(\delta)}{\sqrt{2\pi}^d}.
\end{dmath*}
Then, applying Fubini's theorem over product space,
\begin{dmath*}
\FT{k_0^{skewed}}(\omega)=\prod_{k=1}^d\int_{-c_k}^{\infty}\frac{2\exp\left(-\iu\inner{\log(\delta_k+c_k),\omega_k}\right)}{(t_k + c_k)\left(\sqrt{\delta_k+c_k}+\sqrt{\frac{1}{\delta_k+c_k}}\right)}\frac{d\Leb(\delta_k)}{\sqrt{2\pi}^d}
=\prod_{k=1}^d\int_{-c_k}^{\infty} \frac{2\exp\left(-\iu\inner{\delta_k,\omega_k}\right)}{\exp\left(\frac{1}{2}\delta_k\right)+\exp\left(-\frac{1}{2}\delta_k\right)} \frac{d\Leb(\delta_k)}{\sqrt{2\pi}^d}
=\sqrt{2\pi}^d\prod_{k=1}^d\sech(\pi\omega_k)
\end{dmath*}
Since $k^{\text{skewed}}_{1-c}\in L^1$ and $\Gamma$ is bounded, it is possible to apply \cref{pr:spectral}, and obtain for all $i$, $j\in\mathbb{N}_p$,
\begin{dmath*}
C^{dec,skewed}(\omega)
=\FT{k_{1-c}^{skewed}}(\omega)\Gamma
=\underbrace{\sqrt{2\pi}^d\prod_{k=1}^d\sech(\pi\omega_k)}_{\rho(\cdot)=\mathcal{S}(0,2^{-1})^d\sqrt{2\pi}^d}\underbrace{\Gamma}_{A(\cdot)}
\end{dmath*}
Hence the decomposition with respect to the usual --non-normalized-- Lebesgue measure $\Leb$ yields
\begin{dgroup}
\begin{dmath}
A^{dec,skewed}(\cdot)=\Gamma
\end{dmath}
\begin{dmath}
\rho^{dec,skewed}=\mathcal{S}(0,2^{-1})^d
\end{dmath}
\end{dgroup}
\subsubsection{Curl-free Gaussian kernel} The curl-free Gaussian kernel is defined as $K^{curl,gauss}_0=-\nabla\nabla^T k_0^{gauss}$. Here $\mathcal{X}=(\mathbb{R}^d, +)$ so the setting is the same than \cref{par:gaussian_dec}.
\begin{dmath*}
C^{curl,gauss}(\omega)_{ij}=
\FT{K^{curl,gauss}_{1-c}(\cdot)_{ij}}(\omega)
=\FT{-\frac{d^2}{d\delta_id\delta_j}k^{gauss}_0}(\omega)
=-(\iu\omega_i)(\iu\omega_j)\FT{k_0^{gauss}}(\omega)
=\omega_i\omega_j\FT{k_0^{gauss}}(\omega)
=\sqrt{2\pi\frac{1}{\sigma^2}}^d\exp\left( -\frac{\sigma^2}{2}\norm{\omega}^2_2\right)\sqrt{2\pi}^d\omega_i\omega_j.
\end{dmath*}
Hence
\begin{dmath*}
C^{curl,gauss}(\omega)=\underbrace{\frac{1}{\sqrt{2\pi\frac{1}{\sigma^2}}^d}\exp\left( -\frac{\sigma^2}{2}\norm{\omega}^2_2\right)\sqrt{2\pi}^d}_{\mu(\cdot)=\mathcal{N}(0,\sigma^{-2}I_d)\sqrt{2\pi}^d}\underbrace{\omega\omega^T}_{A(\omega)=\omega\omega^T}.
\end{dmath*}
Here a canonical decomposition is $A^{curl,gauss}(\omega)=\omega\omega^T$ for all $\omega\in\mathbb{R}^d$ and $\mu^{curl,gauss}=\mathcal{N}(0,\sigma^{-2}I_d)\sqrt{2\pi}^d$ with respect to the normalized Lebesgue measure $d\omega$. Again the decomposition with respect to the usual --non-normalized-- Lebesgue measure is for all $\omega\in\mathbb{R}^d$
\begin{dgroup}
\begin{dmath}
A^{curl,gauss}(\omega)=\omega\omega^T
\end{dmath}
\begin{dmath}
\mu^{curl,gauss}=\mathcal{N}(0,\sigma^{-2}I_d)
\end{dmath}
\end{dgroup}
Notice that in this case $\norm{A^{curl,gauss}(\cdot)}_{\mathbb{R}^d,\mathbb{R}^d}$ is not bounded. However applying \cref{pr:trace_measure} yields a different decomposition where the quantity $\norm{A^{curl,gauss}_{\Tr}(\cdot)}_{\mathbb{R}^d,\mathbb{R}^d}$ is bounded. First we have for all $\delta\in\mathbb{R}^d$ and for all $i$, $j\in\mathbb{N}_d$
\begin{dmath*}
\frac{d^2}{d\delta_id\delta_j}k^{gauss}_0(\delta)=\frac{\exp\left(-\frac{1}{2\sigma^2}\norm{\delta}^2_2\right)}{\sigma^2}\begin{cases}
\frac{\delta_i\delta_j}{\sigma^2} & \text{if } i\neq j \\
\left(1-\frac{\delta_i\delta_j}{\sigma^2}\right) & \text{otherwise.}
\end{cases}
\end{dmath*}
Hence
\begin{dmath*}
-\nabla\nabla^Tk^{gauss}_0(\delta)=\left(I_d -\frac{\delta\delta^T}{\sigma^2} \right)\frac{\exp\left(-\frac{1}{2\sigma^2}\norm{\delta}^2_2\right)}{\sigma^2}
\end{dmath*}
Thus $\Tr\left[K^{curl,gauss}_0(0)\right]=\Tr\left[\nabla\nabla^Tk^{gauss}_0(0)\right]=d\sigma^{-2}$ and $\Tr\left[C(\omega)\right]=\norm{\omega}_2^2\mathcal{N}(0,\sigma^{-2}I_d)\sqrt{2\pi}^d$. Apply \cref{pr:trace_measure} to obtain the decomposition $A^{curl,gauss}_{\Tr}(\omega)=\omega\omega^T\norm{\omega}_2^{-2}$ and the measure $\mu^{curl,gauss}_{\Tr}(\omega)=\sigma^2d^{-1}\norm{\omega}_2^2\mathcal{N}\left(0,\sigma^{-2}\right)\sqrt{2\pi}^d$ for all $\omega\in\mathbb{R}^d$, with respect to the normalized Lebesgue measure. Therefore the decomposition with respect to the usual non-normalized Lebesgue measure is
\begin{dgroup}
\begin{dmath}
A^{curl,gauss}_{\Tr}(\omega)=\frac{\omega\omega^T}{\norm{\omega}_2^{2}}
\end{dmath}
\begin{dmath}
\mu^{curl,gauss}_{\Tr}(\omega)=\frac{\sigma^2}{d}\norm{\omega}_2^2\mathcal{N}\left(0,\sigma^{-2}\right)(\omega)
\end{dmath}
\end{dgroup}
This example also illustrate that there exist many decomposition of $C(\omega)$ into $(A(\omega),\mu(\omega))$.
\subsubsection{Divergence-free kernel}
The divegence-free Gaussian kernel is defined as $K^{div,gauss}_0=(\nabla\nabla^T-\Delta)k_0^{gauss}$ on the group $\mathcal{X}=(\mathbb{R}^d, +)$. The setting is the same than \cref{par:gaussian_dec}. Hence
\begin{dmath*}
C^{div,gauss}(\omega)_{ij}=
\FT{K^{div,gauss}_0(\cdot)_{ij}}(\omega)
=\FT{\frac{d^2}{d\delta_id\delta_j}k^{gauss}_0-\delta_{i=j}\sum_{k=1}^d\frac{d^2}{d\delta_kd\delta_k}k^{gauss}_0}(\omega)
=\left(-(\iu\omega_i)(\iu\omega_j)-\delta_{i=j}\sum_{k=1}^d(\iu\omega_k)^2\right)\FT{k_0^{gauss}}=\left(\delta_{i=j}\sum_{k=1}^d\omega_k^2-\omega_i\omega_j\right)\FT{k_0^{gauss}}(\omega).
\end{dmath*}
Hence
\begin{dmath*}
C^{div,gauss}(\omega)=\underbrace{\frac{1}{\sqrt{2\pi\frac{1}{\sigma^2}}^d}\exp\left( -\frac{\sigma^2}{2}\norm{\omega}^2_2\right)\sqrt{2\pi}^d}_{\rho(\cdot)=\mathcal{N}(0,\sigma^{-2}I_d)\sqrt{2\pi}^d}\underbrace{\left(I_d\norm{\omega}_2^2-\omega\omega^T\right)}_{A(\omega)=I_d\norm{\omega}_2^2-\omega\omega^T}.
\end{dmath*}
Thus the canonical decomposition with respect to the normalized Lebesgue measure is $A^{div,gauss}(\omega)=I_d\norm{\omega}_2^2-\omega\omega^T$ and the measure $\rho^{div,gauss}=\mathcal{N}(0,\sigma^{-2}I_d)\sqrt{2\pi}^d$. The canonical decomposition with respect to the usual Lebesgue measure is
\begin{dgroup}
\begin{dmath}
A^{div,gauss}(\omega)=I_d\norm{\omega}_2^2-\omega\omega^T
\end{dmath}
\begin{dmath}
\rho^{div,gauss}=\mathcal{N}(0,\sigma^{-2}I_d).
\end{dmath}
\end{dgroup}
To obtain the bounded decomposition, again, apply \cref{pr:trace_measure}. For all $\delta\in\mathbb{R}^d$,
\begin{dmath*}
\sum_{k=1}^d\frac{d^2}{d\delta_k d\delta_k}k^{gauss}_0(\delta)=
\left(d-\frac{\norm{\delta}_2^2}{\sigma^2}\right)\frac{\exp\left(-\frac{1}{2\sigma^2}\norm{\delta}^2_2\right)}{\sigma^2}.
\end{dmath*}
Thus overall,
\begin{dmath*}
K^{div,gauss}_0(\delta)=\left(\frac{\delta\delta^T}{\sigma^2} + \left((d-1) - \frac{\norm{\delta}_2^2}{\sigma^2}\right)I_d \right)\frac{\exp\left(-\frac{1}{2\sigma^2}\norm{\delta}^2_2\right)}{\sigma^2},
\end{dmath*}
Eventually $\Tr\left[K^{div,gauss}_0(0)\right]=\Tr\left[(\nabla\nabla^T-\Delta)k^{gauss}_0(0)\right]=d(d-1)\sigma^{-2}$ and $\Tr\left[C(\omega)\right]=(d-1)\norm{\omega}_2^2\mathcal{N}(0,\sigma^2I_d)\sqrt{2\pi}^d$. As a result the decomposition with respect to the normalized Lebesgue measure is $A^{div,gauss}_{\Tr}(\omega)=(I_d-\omega\omega^T\norm{\omega}_2^{-2})$ and $\rho^{div,gauss}_{\Tr}(\omega)=d^{-1}\sigma^2\norm{\omega}_2^2\mathcal{N}(0,\sigma^2I_d)\sqrt{2\pi}^d$. The decomposition with respect to the normalized Lebesgue measure being
\begin{dgroup}
\begin{dmath}
A^{div,gauss}_{\Tr}(\omega)=I_d-\frac{\omega\omega^T}{\norm{\omega}_2^2}
\end{dmath}
\begin{dmath}
\rho^{div,gauss}_{\Tr}=\frac{\sigma^2}{d}\norm{\omega}_2^2\mathcal{N}(0,\sigma^{-2}I_d).
\end{dmath}
\end{dgroup}

\subsection{Functional Fourier feature map}
Let us introduce a functional feature map, we call here \emph{Fourier Feature map}, defined by the following proposition as a direct consequence of \cref{pr:mercer_kernel_bochner}.

\begin{proposition}[Functional Fourier feature map]\label{pr:fourier_feature_map}
Let $\mathcal{Y}$ and $\mathcal{Y}'$ be two Hilbert spaces. If there exist an operator-valued function $B:\dual{\mathcal{X}}\to\mathcal{L}(\mathcal{Y},\mathcal{Y}')$ such that for all $y$, $y'\in\mathcal{Y}$,
\begin{dmath*}
\inner{y, B(\omega)B(\omega)^\adjoint y'}_{\mathcal{Y}}=\inner{y', A(\omega)y}_{\mathcal{Y}}
\end{dmath*}
$\dual{\mu}$-almost everywhere and $\inner{y', A(\cdot)y}\in L^1(\dual{\mathcal{X}},\dual{\mu})$ then the operator $\Phi_x$ defined for all $y$ in $\mathcal{Y}$ by
\begin{dmath}
\label{eq:feature_shiftinv_map}
(\Phi_x y)(\omega)=\pairing{x,\omega}B(\omega)^\adjoint y,
\end{dmath}
is \emph{a feature map}\mpar{\Ie~it satisfies for all $x$, $z \in \mathcal{X}$, $\Phi_x^\adjoint \Phi_z=K(x,z)$ where $K$ is a $\mathcal{Y}$-Mercer \acs{OVK}.} of some shift-invariant $\mathcal{Y}$-Mercer kernel $K$.
\end{proposition}
\begin{proof}
For all $y$, $y'\in \mathcal{Y}$ and $x$, $z\in\mathcal{X}$,
\begin{dmath*}
\inner{y, \Phi_x^\adjoint \Phi_z y'}_{\mathcal{Y}} = \inner{\Phi_x y, \Phi_z y'}_{L^2(\dual{\mathcal{X}},\dual{\mu};\mathcal{Y}')}  \\
= \int_{\dual{\mathcal{X}}}\conj{\pairing{x,\omega}}\inner{y, B(\omega)\pairing{z,\omega}B(\omega)^\adjoint y'}d\dual{\mu}(\omega) \\
= \int_{\dual{\mathcal{X}}}\conj{\pairing{x \groupop \myinv{z},\omega}}\inner{y B(\omega)B(\omega)^\adjoint y'}d\dual{\mu}(\omega) \\
= \int_{\dual{\mathcal{X}}}\conj{\pairing{x \groupop \inv{z},\omega}}\inner{y,A(\omega)y'}d\dual{\mu}(\omega),
\end{dmath*}
which defines a $\mathcal{Y}$-Mercer according to \cref{pr:mercer_kernel_bochner} of \citet{Carmeli2010}.
\end{proof}
With this notation notice that $\Phi: \mathcal{X} \to \mathcal{L}(\mathcal{Y}; L^2(\dual{\mathcal{X}}, \dual{\mu}; \mathcal{Y}'))$ such that $\Phi_x\in \mathcal{L}(\mathcal{Y}; L^2(\dual{\mathcal{X}}, \dual{\mu}; \mathcal{Y}'))$ where $\Phi_x\colonequals\Phi(x)$.

\subsection{Building Operator-valued Random Fourier Features}
As shown in \cref{pr:spectral,pr:trace_measure} it is always possible to find a pair $(A, \probability_{\dual{\Haar},\rho})$ from a shift invariant $\mathcal{Y}$-Mercer \acl{OVK} $K_e$ such that $\probability_{\dual{\Haar},\rho}$ is a probability measure --\ie~$\int_{\dual{\mathcal{X}}} \rho d\dual{\Haar}=1$ where $\rho$ is the density of $\probability_{\dual{\Haar},\rho}$-- and $K_e(\delta)=\expectation_\rho{\conj{\pairing{\delta,\omega}}A(\omega)}$. In order to obtain an approximation of $K$ from a decomposition $(A, \probability_{\dual{\Haar},\rho})$ we turn our attention to a Monte-Carlo estimation of the expectations \cref{eq:expectation_tr} and \cref{eq:expectation_spec} characterizing a $\mathcal{Y}$-Mercer shift-invariant \acl{OVK}.
\paragraph{}
However, for efficient computations as motivated in the introduction, we are interested in finding an approximated \emph{feature map} instead of a kernel approximation. Indeed, an approximated feature map will allow to build linear models in regression tasks. The idea is to start from the Monte-Carlo approximation of the expectation and provide a systematic decomposition of the Monte-Carlo sample mean into an approximate feature map. The following proposition provides the general form of an \acl{ORFF}.
\begin{proposition}[ORFF]
\label{pr:ORFF-map}
Let $\mathcal{Y}$ and $\mathcal{Y}'$ be two Hilbert spaces. If one can find $B: \dual{\mathcal{X}} \to \mathcal{L}(\mathcal{Y},\mathcal{Y}')$ and a probability measure $\probability_{\dual{\Haar},\rho}$ on $\mathcal{B}(\dual{\mathcal{X}})$, such that for all $y\in\mathcal{Y}$ and all $y'\in\mathcal{Y}'$, $\inner{y, B(\cdot)y'} \in L^2(\dual{\mathcal{X}}, \probability_{\dual{\Haar},\rho})$, then the operator-valued function given for all $y\in\mathcal{Y}$ by
\begin{equation}
\label{eq:phitilde}
\tildePhi{\omega}(x)y= \frac{1}{\sqrt{D}}\Vect_{j=1}^D\pairing{x, \omega_j}B(\omega_j)^\adjoint y \condition{$\omega_j \sim \probability_{\dual{\Haar},\rho}$ \iid,}
\end{equation}
is an approximated feature map\mpar{\Ie~it satisfies $\tildePhi{\omega}(x)^\adjoint \tildePhi{\omega}(z)\converges{\asurely}{D\to\infty}K(x,z)$ where $K$ is a $\mathcal{Y}$-Mercer \acs{OVK}.} of a shift-invariant $\mathcal{Y}$-Mercer \acl{OVK}.
\end{proposition}
\begin{proof}
Let $(\omega_j)_{j=1}^D$ be a sequence of $D\in\mathbb{N}$ \iid~random vectors, each of them following the law $\probability_{\dual{\Haar},\rho}$. For all $x$, $z \in \mathcal{X}$ and all $y$, $y' \in \mathcal{Y}$,
\begin{dmath*}
\inner*{\tildePhi{\omega}(x)y',\tildePhi{\omega}(z)y}=\frac{1}{D}\inner*{\Vect_{j=1}^D\pairing{x, \omega_j}B(\omega_j)^\adjoint y',  \Vect_{j=1}^D\pairing{z, \omega_j}B(\omega_j)^\adjoint y}
= \frac{1}{D} \sum_{j=1}^D \inner*{y', \conj{\pairing{x, \omega_j}}B(\omega_j)\pairing{z, \omega_j}B(\omega_j)^\adjoint y}_{\mathcal{Y}}
= \inner*{y', \left(\frac{1}{D} \sum_{j=1}^D \conj{\pairing{x\groupop \inv{z}, \omega_j}}A(\omega_j)\right)y}_{\mathcal{Y}},
\end{dmath*}
where $A(\omega)=B(\omega)B(\omega)^\adjoint $. By assumption $\inner{y, A(\cdot)y'}\in L^1(\dual{\mathcal{X}},\probability_{\dual{\Haar},\rho})$ and $\omega_j$ are \iid~. Hence from the strong law of large numbers and \cref{pr:mercer_kernel_bochner} with $\dual{\mu}=\probability_{\dual{\Haar},\rho}$,
\begin{dmath*}
\frac{1}{D} \sum_{j=1}^D \conj{\pairing{x\groupop\inv{z},\omega_j}}A(\omega_j)\converges{\asurely}{D\to\infty}\expectation_{\rho}[\conj{\pairing{x\groupop z^{-1},\omega_j}}A(\omega)]=K_e(x\groupop\inv{z})
\end{dmath*}
where the integral converges in the weak operator topology.
\end{proof}
\begin{remark}
The approximate feature map proposed in \cref{pr:ORFF-map} has direct link with the functional Fourier feature map defined in \cref{pr:fourier_feature_map} since we have for all $y\in\mathcal{Y}$
\begin{dmath}
\tildePhi{\omega}(x)y = \frac{1}{\sqrt{D}}\Vect_{j=1}^D\pairing{x, \omega_j}B(\omega_j)^\adjoint y \condition{$\omega_j \hiderel{\sim} \probability_{\dual{\Haar},\rho}$ \iid}
= \frac{1}{\sqrt{D}} \Vect_{j=1}^D (\Phi_x y)(\omega_j).
\end{dmath}
\end{remark}
Therefore $\tildePhi{\omega}(x)$ can be seen as an \say{operator-valued vector} corresponding the \say{stacking} of $D$ \iid~operator-valued random variable $\pairing{x, \omega_j}B(\omega_j)^\adjoint$. Note that we consider $\omega$ to be a \emph{variable} in $\dual{\mathcal{X}}$ while $\omega_j$ are $\dual{\mathcal{X}}$-valued \emph{random variables}. \Ie~$\omega_j$ are $(\dual{\mathcal{X}}, \mathcal{B}(\dual{\mathcal{X}}))$-valued measurable function and $\probability_{\dual{\Haar},\rho}$ is the distribution of each $\omega_j$. Let
\begin{dmath*}
\tildeH{\omega} = \Vect_{j=1}^D\mathcal{Y}',
\end{dmath*}
be a Hilbert space. Let $\seq{\omega}\in\dual{\mathcal{X}}^D$ be a sequence where $\seq{\omega}=(\omega_j)_{j=1}^D$ (\ie~ an outcome of the random sequence $(\omega_j)_{j=1}^D$, $\omega_j \sim \probability_{\dual{\Haar},\rho}$). Notice that for any $x\in\mathcal{X}$ and $y\in\mathcal{Y}$, $\tildePhi{\omega}(x)y\in\tildeH{\omega}$ as we can take $g(\omega_j)=\pairing{x,\omega_j}B(\omega_j)^\adjoint y$. Thus we have for all $y$, $y'\in\mathcal{Y}$ and all $x$, $z\in\mathcal{X}$
\begin{dmath*}
\inner{y', \tildeK{\omega}(x, z)y}_{\mathcal{Y}} \hiderel{=} \inner{y', \tildePhi{\omega}(x)^\adjoint \tildePhi{\omega}(z)y}_{\mathcal{Y}} \hiderel{=} \inner{\tildePhi{\omega}(x)y', \tildePhi{\omega}(z)y}_{\tildeH{\omega}}.
\end{dmath*}
Thus $\tildeK{\omega}(x,z)=\tildePhi{\omega}(x)^\adjoint \tildePhi{\omega}(z)$ is a proper random shift-invariant \acl{OVK} as shown in the following proposition.
\begin{proposition} Let $\seq{\omega}\in\dual{\mathcal{X}}^D$. If for all $y$, $y'\in\mathcal{Y}$
\begin{dmath*}
\inner{y', \tildeK{\omega}_e\left(x\groupop z^{-1}\right)y}_{\mathcal{Y}}=\inner{\tildePhi{\omega}(x)y', \tildePhi{\omega}(z)y}_{\tildeH{\omega}}
=\inner*{y', \frac{1}{D}\sum_{j=1}^D \conj{\pairing{x\groupop z^{-1},\omega_j}}B(\omega_j)B(\omega_j)^*y}_{\mathcal{Y}},
\end{dmath*}
for all $x$,  $z\in\mathcal{X}$, then $\tildeK{\omega}$ is a shift-invariant \acl{OVK}.
\end{proposition}
\begin{proof} Apply \cref{pr:feature_operator} to $\tildePhi{\omega}$ considering the Hilbert space $\tildeH{\omega}$ to show that $\tildeK{\omega}$ is an \acs{OVK}. Then \cref{pr:kernel_signature} shows that $\tildeK{\omega}$ is shift-invariant since $\tildeK{\omega}(x,z)=\tildeK{\omega}_e\left(x,\groupop z^{-1}\right)$.
\end{proof}
We stress out that if $\seq{\omega}=(\omega_j)_{j=1}^D\sim \probability_{\dual{\Haar},\rho}$ \iid~is a \emph{random sequence} then
\begin{dmath*}
\tildeK{\omega}_e\left(x\groupop z^{-1}\right)=\tildePhi{\omega}(x)^{\adjoint}\tildePhi{\omega}(z)
\end{dmath*}
\emph{is not} an \acl{OVK} since it is a random variable and ($\tildeH{\omega}$ is no longer a Hilbert space, since \cref{eq:inner_Htilde} does not define a proper inner product). However this is not a problem since any outcome of the random sequence $\seq{\omega}$ gives birth to a (different) \acl{OVK}, and $\expectation_{\dual{\Haar},\rho}\tildeK{\omega}$ is an \acs{OVK}. This illustrated by \cref{fig:not_Mercer} where we represented the same function for different realization of $\tildeK{\omega}\approx K$. We generated $250$ points equally separated on the segment $(-1;1)$.
\begin{figure}[htb]
\centering
\resizebox{\textwidth}{!}{%
%% Creator: Matplotlib, PGF backend
%%
%% To include the figure in your LaTeX document, write
%%   \input{<filename>.pgf}
%%
%% Make sure the required packages are loaded in your preamble
%%   \usepackage{pgf}
%%
%% Figures using additional raster images can only be included by \input if
%% they are in the same directory as the main LaTeX file. For loading figures
%% from other directories you can use the `import` package
%%   \usepackage{import}
%% and then include the figures with
%%   \import{<path to file>}{<filename>.pgf}
%%
%% Matplotlib used the following preamble
%%   \usepackage{fontspec}
%%   \setmainfont{Times New Roman}
%%   \setsansfont{Lucida Grande}
%%   \setmonofont{Andale Mono}
%%
\begingroup%
\makeatletter%
\begin{pgfpicture}%
\pgfpathrectangle{\pgfpointorigin}{\pgfqpoint{10.865945in}{7.011001in}}%
\pgfusepath{use as bounding box, clip}%
\begin{pgfscope}%
\pgfsetbuttcap%
\pgfsetmiterjoin%
\definecolor{currentfill}{rgb}{1.000000,1.000000,1.000000}%
\pgfsetfillcolor{currentfill}%
\pgfsetlinewidth{0.000000pt}%
\definecolor{currentstroke}{rgb}{1.000000,1.000000,1.000000}%
\pgfsetstrokecolor{currentstroke}%
\pgfsetdash{}{0pt}%
\pgfpathmoveto{\pgfqpoint{0.000000in}{0.000000in}}%
\pgfpathlineto{\pgfqpoint{10.865945in}{0.000000in}}%
\pgfpathlineto{\pgfqpoint{10.865945in}{7.011001in}}%
\pgfpathlineto{\pgfqpoint{0.000000in}{7.011001in}}%
\pgfpathclose%
\pgfusepath{fill}%
\end{pgfscope}%
\begin{pgfscope}%
\pgfsetbuttcap%
\pgfsetmiterjoin%
\definecolor{currentfill}{rgb}{1.000000,1.000000,1.000000}%
\pgfsetfillcolor{currentfill}%
\pgfsetlinewidth{0.000000pt}%
\definecolor{currentstroke}{rgb}{0.000000,0.000000,0.000000}%
\pgfsetstrokecolor{currentstroke}%
\pgfsetstrokeopacity{0.000000}%
\pgfsetdash{}{0pt}%
\pgfpathmoveto{\pgfqpoint{0.625831in}{3.799602in}}%
\pgfpathlineto{\pgfqpoint{4.853104in}{3.799602in}}%
\pgfpathlineto{\pgfqpoint{4.853104in}{6.545056in}}%
\pgfpathlineto{\pgfqpoint{0.625831in}{6.545056in}}%
\pgfpathclose%
\pgfusepath{fill}%
\end{pgfscope}%
\begin{pgfscope}%
\pgfsetbuttcap%
\pgfsetroundjoin%
\definecolor{currentfill}{rgb}{0.000000,0.000000,0.000000}%
\pgfsetfillcolor{currentfill}%
\pgfsetlinewidth{0.803000pt}%
\definecolor{currentstroke}{rgb}{0.000000,0.000000,0.000000}%
\pgfsetstrokecolor{currentstroke}%
\pgfsetdash{}{0pt}%
\pgfsys@defobject{currentmarker}{\pgfqpoint{0.000000in}{-0.048611in}}{\pgfqpoint{0.000000in}{0.000000in}}{%
\pgfpathmoveto{\pgfqpoint{0.000000in}{0.000000in}}%
\pgfpathlineto{\pgfqpoint{0.000000in}{-0.048611in}}%
\pgfusepath{stroke,fill}%
}%
\begin{pgfscope}%
\pgfsys@transformshift{0.817980in}{3.799602in}%
\pgfsys@useobject{currentmarker}{}%
\end{pgfscope}%
\end{pgfscope}%
\begin{pgfscope}%
\pgfsetbuttcap%
\pgfsetroundjoin%
\definecolor{currentfill}{rgb}{0.000000,0.000000,0.000000}%
\pgfsetfillcolor{currentfill}%
\pgfsetlinewidth{0.803000pt}%
\definecolor{currentstroke}{rgb}{0.000000,0.000000,0.000000}%
\pgfsetstrokecolor{currentstroke}%
\pgfsetdash{}{0pt}%
\pgfsys@defobject{currentmarker}{\pgfqpoint{0.000000in}{-0.048611in}}{\pgfqpoint{0.000000in}{0.000000in}}{%
\pgfpathmoveto{\pgfqpoint{0.000000in}{0.000000in}}%
\pgfpathlineto{\pgfqpoint{0.000000in}{-0.048611in}}%
\pgfusepath{stroke,fill}%
}%
\begin{pgfscope}%
\pgfsys@transformshift{1.298352in}{3.799602in}%
\pgfsys@useobject{currentmarker}{}%
\end{pgfscope}%
\end{pgfscope}%
\begin{pgfscope}%
\pgfsetbuttcap%
\pgfsetroundjoin%
\definecolor{currentfill}{rgb}{0.000000,0.000000,0.000000}%
\pgfsetfillcolor{currentfill}%
\pgfsetlinewidth{0.803000pt}%
\definecolor{currentstroke}{rgb}{0.000000,0.000000,0.000000}%
\pgfsetstrokecolor{currentstroke}%
\pgfsetdash{}{0pt}%
\pgfsys@defobject{currentmarker}{\pgfqpoint{0.000000in}{-0.048611in}}{\pgfqpoint{0.000000in}{0.000000in}}{%
\pgfpathmoveto{\pgfqpoint{0.000000in}{0.000000in}}%
\pgfpathlineto{\pgfqpoint{0.000000in}{-0.048611in}}%
\pgfusepath{stroke,fill}%
}%
\begin{pgfscope}%
\pgfsys@transformshift{1.778724in}{3.799602in}%
\pgfsys@useobject{currentmarker}{}%
\end{pgfscope}%
\end{pgfscope}%
\begin{pgfscope}%
\pgfsetbuttcap%
\pgfsetroundjoin%
\definecolor{currentfill}{rgb}{0.000000,0.000000,0.000000}%
\pgfsetfillcolor{currentfill}%
\pgfsetlinewidth{0.803000pt}%
\definecolor{currentstroke}{rgb}{0.000000,0.000000,0.000000}%
\pgfsetstrokecolor{currentstroke}%
\pgfsetdash{}{0pt}%
\pgfsys@defobject{currentmarker}{\pgfqpoint{0.000000in}{-0.048611in}}{\pgfqpoint{0.000000in}{0.000000in}}{%
\pgfpathmoveto{\pgfqpoint{0.000000in}{0.000000in}}%
\pgfpathlineto{\pgfqpoint{0.000000in}{-0.048611in}}%
\pgfusepath{stroke,fill}%
}%
\begin{pgfscope}%
\pgfsys@transformshift{2.259096in}{3.799602in}%
\pgfsys@useobject{currentmarker}{}%
\end{pgfscope}%
\end{pgfscope}%
\begin{pgfscope}%
\pgfsetbuttcap%
\pgfsetroundjoin%
\definecolor{currentfill}{rgb}{0.000000,0.000000,0.000000}%
\pgfsetfillcolor{currentfill}%
\pgfsetlinewidth{0.803000pt}%
\definecolor{currentstroke}{rgb}{0.000000,0.000000,0.000000}%
\pgfsetstrokecolor{currentstroke}%
\pgfsetdash{}{0pt}%
\pgfsys@defobject{currentmarker}{\pgfqpoint{0.000000in}{-0.048611in}}{\pgfqpoint{0.000000in}{0.000000in}}{%
\pgfpathmoveto{\pgfqpoint{0.000000in}{0.000000in}}%
\pgfpathlineto{\pgfqpoint{0.000000in}{-0.048611in}}%
\pgfusepath{stroke,fill}%
}%
\begin{pgfscope}%
\pgfsys@transformshift{2.739468in}{3.799602in}%
\pgfsys@useobject{currentmarker}{}%
\end{pgfscope}%
\end{pgfscope}%
\begin{pgfscope}%
\pgfsetbuttcap%
\pgfsetroundjoin%
\definecolor{currentfill}{rgb}{0.000000,0.000000,0.000000}%
\pgfsetfillcolor{currentfill}%
\pgfsetlinewidth{0.803000pt}%
\definecolor{currentstroke}{rgb}{0.000000,0.000000,0.000000}%
\pgfsetstrokecolor{currentstroke}%
\pgfsetdash{}{0pt}%
\pgfsys@defobject{currentmarker}{\pgfqpoint{0.000000in}{-0.048611in}}{\pgfqpoint{0.000000in}{0.000000in}}{%
\pgfpathmoveto{\pgfqpoint{0.000000in}{0.000000in}}%
\pgfpathlineto{\pgfqpoint{0.000000in}{-0.048611in}}%
\pgfusepath{stroke,fill}%
}%
\begin{pgfscope}%
\pgfsys@transformshift{3.219840in}{3.799602in}%
\pgfsys@useobject{currentmarker}{}%
\end{pgfscope}%
\end{pgfscope}%
\begin{pgfscope}%
\pgfsetbuttcap%
\pgfsetroundjoin%
\definecolor{currentfill}{rgb}{0.000000,0.000000,0.000000}%
\pgfsetfillcolor{currentfill}%
\pgfsetlinewidth{0.803000pt}%
\definecolor{currentstroke}{rgb}{0.000000,0.000000,0.000000}%
\pgfsetstrokecolor{currentstroke}%
\pgfsetdash{}{0pt}%
\pgfsys@defobject{currentmarker}{\pgfqpoint{0.000000in}{-0.048611in}}{\pgfqpoint{0.000000in}{0.000000in}}{%
\pgfpathmoveto{\pgfqpoint{0.000000in}{0.000000in}}%
\pgfpathlineto{\pgfqpoint{0.000000in}{-0.048611in}}%
\pgfusepath{stroke,fill}%
}%
\begin{pgfscope}%
\pgfsys@transformshift{3.700211in}{3.799602in}%
\pgfsys@useobject{currentmarker}{}%
\end{pgfscope}%
\end{pgfscope}%
\begin{pgfscope}%
\pgfsetbuttcap%
\pgfsetroundjoin%
\definecolor{currentfill}{rgb}{0.000000,0.000000,0.000000}%
\pgfsetfillcolor{currentfill}%
\pgfsetlinewidth{0.803000pt}%
\definecolor{currentstroke}{rgb}{0.000000,0.000000,0.000000}%
\pgfsetstrokecolor{currentstroke}%
\pgfsetdash{}{0pt}%
\pgfsys@defobject{currentmarker}{\pgfqpoint{0.000000in}{-0.048611in}}{\pgfqpoint{0.000000in}{0.000000in}}{%
\pgfpathmoveto{\pgfqpoint{0.000000in}{0.000000in}}%
\pgfpathlineto{\pgfqpoint{0.000000in}{-0.048611in}}%
\pgfusepath{stroke,fill}%
}%
\begin{pgfscope}%
\pgfsys@transformshift{4.180583in}{3.799602in}%
\pgfsys@useobject{currentmarker}{}%
\end{pgfscope}%
\end{pgfscope}%
\begin{pgfscope}%
\pgfsetbuttcap%
\pgfsetroundjoin%
\definecolor{currentfill}{rgb}{0.000000,0.000000,0.000000}%
\pgfsetfillcolor{currentfill}%
\pgfsetlinewidth{0.803000pt}%
\definecolor{currentstroke}{rgb}{0.000000,0.000000,0.000000}%
\pgfsetstrokecolor{currentstroke}%
\pgfsetdash{}{0pt}%
\pgfsys@defobject{currentmarker}{\pgfqpoint{0.000000in}{-0.048611in}}{\pgfqpoint{0.000000in}{0.000000in}}{%
\pgfpathmoveto{\pgfqpoint{0.000000in}{0.000000in}}%
\pgfpathlineto{\pgfqpoint{0.000000in}{-0.048611in}}%
\pgfusepath{stroke,fill}%
}%
\begin{pgfscope}%
\pgfsys@transformshift{4.660955in}{3.799602in}%
\pgfsys@useobject{currentmarker}{}%
\end{pgfscope}%
\end{pgfscope}%
\begin{pgfscope}%
\pgfsetbuttcap%
\pgfsetroundjoin%
\definecolor{currentfill}{rgb}{0.000000,0.000000,0.000000}%
\pgfsetfillcolor{currentfill}%
\pgfsetlinewidth{0.803000pt}%
\definecolor{currentstroke}{rgb}{0.000000,0.000000,0.000000}%
\pgfsetstrokecolor{currentstroke}%
\pgfsetdash{}{0pt}%
\pgfsys@defobject{currentmarker}{\pgfqpoint{-0.048611in}{0.000000in}}{\pgfqpoint{0.000000in}{0.000000in}}{%
\pgfpathmoveto{\pgfqpoint{0.000000in}{0.000000in}}%
\pgfpathlineto{\pgfqpoint{-0.048611in}{0.000000in}}%
\pgfusepath{stroke,fill}%
}%
\begin{pgfscope}%
\pgfsys@transformshift{0.625831in}{4.245102in}%
\pgfsys@useobject{currentmarker}{}%
\end{pgfscope}%
\end{pgfscope}%
\begin{pgfscope}%
\pgftext[x=0.281695in,y=4.196884in,left,base]{\rmfamily\fontsize{10.000000}{12.000000}\selectfont \(\displaystyle -20\)}%
\end{pgfscope}%
\begin{pgfscope}%
\pgfsetbuttcap%
\pgfsetroundjoin%
\definecolor{currentfill}{rgb}{0.000000,0.000000,0.000000}%
\pgfsetfillcolor{currentfill}%
\pgfsetlinewidth{0.803000pt}%
\definecolor{currentstroke}{rgb}{0.000000,0.000000,0.000000}%
\pgfsetstrokecolor{currentstroke}%
\pgfsetdash{}{0pt}%
\pgfsys@defobject{currentmarker}{\pgfqpoint{-0.048611in}{0.000000in}}{\pgfqpoint{0.000000in}{0.000000in}}{%
\pgfpathmoveto{\pgfqpoint{0.000000in}{0.000000in}}%
\pgfpathlineto{\pgfqpoint{-0.048611in}{0.000000in}}%
\pgfusepath{stroke,fill}%
}%
\begin{pgfscope}%
\pgfsys@transformshift{0.625831in}{4.705406in}%
\pgfsys@useobject{currentmarker}{}%
\end{pgfscope}%
\end{pgfscope}%
\begin{pgfscope}%
\pgftext[x=0.281695in,y=4.657188in,left,base]{\rmfamily\fontsize{10.000000}{12.000000}\selectfont \(\displaystyle -10\)}%
\end{pgfscope}%
\begin{pgfscope}%
\pgfsetbuttcap%
\pgfsetroundjoin%
\definecolor{currentfill}{rgb}{0.000000,0.000000,0.000000}%
\pgfsetfillcolor{currentfill}%
\pgfsetlinewidth{0.803000pt}%
\definecolor{currentstroke}{rgb}{0.000000,0.000000,0.000000}%
\pgfsetstrokecolor{currentstroke}%
\pgfsetdash{}{0pt}%
\pgfsys@defobject{currentmarker}{\pgfqpoint{-0.048611in}{0.000000in}}{\pgfqpoint{0.000000in}{0.000000in}}{%
\pgfpathmoveto{\pgfqpoint{0.000000in}{0.000000in}}%
\pgfpathlineto{\pgfqpoint{-0.048611in}{0.000000in}}%
\pgfusepath{stroke,fill}%
}%
\begin{pgfscope}%
\pgfsys@transformshift{0.625831in}{5.165710in}%
\pgfsys@useobject{currentmarker}{}%
\end{pgfscope}%
\end{pgfscope}%
\begin{pgfscope}%
\pgftext[x=0.459164in,y=5.117492in,left,base]{\rmfamily\fontsize{10.000000}{12.000000}\selectfont \(\displaystyle 0\)}%
\end{pgfscope}%
\begin{pgfscope}%
\pgfsetbuttcap%
\pgfsetroundjoin%
\definecolor{currentfill}{rgb}{0.000000,0.000000,0.000000}%
\pgfsetfillcolor{currentfill}%
\pgfsetlinewidth{0.803000pt}%
\definecolor{currentstroke}{rgb}{0.000000,0.000000,0.000000}%
\pgfsetstrokecolor{currentstroke}%
\pgfsetdash{}{0pt}%
\pgfsys@defobject{currentmarker}{\pgfqpoint{-0.048611in}{0.000000in}}{\pgfqpoint{0.000000in}{0.000000in}}{%
\pgfpathmoveto{\pgfqpoint{0.000000in}{0.000000in}}%
\pgfpathlineto{\pgfqpoint{-0.048611in}{0.000000in}}%
\pgfusepath{stroke,fill}%
}%
\begin{pgfscope}%
\pgfsys@transformshift{0.625831in}{5.626014in}%
\pgfsys@useobject{currentmarker}{}%
\end{pgfscope}%
\end{pgfscope}%
\begin{pgfscope}%
\pgftext[x=0.389720in,y=5.577796in,left,base]{\rmfamily\fontsize{10.000000}{12.000000}\selectfont \(\displaystyle 10\)}%
\end{pgfscope}%
\begin{pgfscope}%
\pgfsetbuttcap%
\pgfsetroundjoin%
\definecolor{currentfill}{rgb}{0.000000,0.000000,0.000000}%
\pgfsetfillcolor{currentfill}%
\pgfsetlinewidth{0.803000pt}%
\definecolor{currentstroke}{rgb}{0.000000,0.000000,0.000000}%
\pgfsetstrokecolor{currentstroke}%
\pgfsetdash{}{0pt}%
\pgfsys@defobject{currentmarker}{\pgfqpoint{-0.048611in}{0.000000in}}{\pgfqpoint{0.000000in}{0.000000in}}{%
\pgfpathmoveto{\pgfqpoint{0.000000in}{0.000000in}}%
\pgfpathlineto{\pgfqpoint{-0.048611in}{0.000000in}}%
\pgfusepath{stroke,fill}%
}%
\begin{pgfscope}%
\pgfsys@transformshift{0.625831in}{6.086318in}%
\pgfsys@useobject{currentmarker}{}%
\end{pgfscope}%
\end{pgfscope}%
\begin{pgfscope}%
\pgftext[x=0.389720in,y=6.038100in,left,base]{\rmfamily\fontsize{10.000000}{12.000000}\selectfont \(\displaystyle 20\)}%
\end{pgfscope}%
\begin{pgfscope}%
\pgfsetbuttcap%
\pgfsetroundjoin%
\definecolor{currentfill}{rgb}{0.000000,0.000000,0.000000}%
\pgfsetfillcolor{currentfill}%
\pgfsetlinewidth{0.803000pt}%
\definecolor{currentstroke}{rgb}{0.000000,0.000000,0.000000}%
\pgfsetstrokecolor{currentstroke}%
\pgfsetdash{}{0pt}%
\pgfsys@defobject{currentmarker}{\pgfqpoint{-0.048611in}{0.000000in}}{\pgfqpoint{0.000000in}{0.000000in}}{%
\pgfpathmoveto{\pgfqpoint{0.000000in}{0.000000in}}%
\pgfpathlineto{\pgfqpoint{-0.048611in}{0.000000in}}%
\pgfusepath{stroke,fill}%
}%
\begin{pgfscope}%
\pgfsys@transformshift{0.625831in}{6.546622in}%
\pgfsys@useobject{currentmarker}{}%
\end{pgfscope}%
\end{pgfscope}%
\begin{pgfscope}%
\pgftext[x=0.389720in,y=6.498404in,left,base]{\rmfamily\fontsize{10.000000}{12.000000}\selectfont \(\displaystyle 30\)}%
\end{pgfscope}%
\begin{pgfscope}%
\pgftext[x=0.226139in,y=5.172329in,,bottom,rotate=90.000000]{\rmfamily\fontsize{10.000000}{12.000000}\selectfont \(\displaystyle y_1\)}%
\end{pgfscope}%
\begin{pgfscope}%
\pgfpathrectangle{\pgfqpoint{0.625831in}{3.799602in}}{\pgfqpoint{4.227273in}{2.745455in}} %
\pgfusepath{clip}%
\pgfsetrectcap%
\pgfsetroundjoin%
\pgfsetlinewidth{0.501875pt}%
\definecolor{currentstroke}{rgb}{0.500000,0.000000,1.000000}%
\pgfsetstrokecolor{currentstroke}%
\pgfsetdash{}{0pt}%
\pgfpathmoveto{\pgfqpoint{0.817980in}{5.494719in}}%
\pgfpathlineto{\pgfqpoint{0.833414in}{5.657542in}}%
\pgfpathlineto{\pgfqpoint{0.848847in}{5.810507in}}%
\pgfpathlineto{\pgfqpoint{0.864281in}{5.950549in}}%
\pgfpathlineto{\pgfqpoint{0.879715in}{6.074861in}}%
\pgfpathlineto{\pgfqpoint{0.895148in}{6.180950in}}%
\pgfpathlineto{\pgfqpoint{0.910582in}{6.266692in}}%
\pgfpathlineto{\pgfqpoint{0.926015in}{6.330366in}}%
\pgfpathlineto{\pgfqpoint{0.941449in}{6.370698in}}%
\pgfpathlineto{\pgfqpoint{0.956883in}{6.386878in}}%
\pgfpathlineto{\pgfqpoint{0.972316in}{6.378583in}}%
\pgfpathlineto{\pgfqpoint{0.987750in}{6.345979in}}%
\pgfpathlineto{\pgfqpoint{1.003184in}{6.289718in}}%
\pgfpathlineto{\pgfqpoint{1.018617in}{6.210930in}}%
\pgfpathlineto{\pgfqpoint{1.034051in}{6.111192in}}%
\pgfpathlineto{\pgfqpoint{1.049485in}{5.992504in}}%
\pgfpathlineto{\pgfqpoint{1.064918in}{5.857245in}}%
\pgfpathlineto{\pgfqpoint{1.095785in}{5.548135in}}%
\pgfpathlineto{\pgfqpoint{1.188387in}{4.546406in}}%
\pgfpathlineto{\pgfqpoint{1.203821in}{4.403932in}}%
\pgfpathlineto{\pgfqpoint{1.219255in}{4.276726in}}%
\pgfpathlineto{\pgfqpoint{1.234688in}{4.167337in}}%
\pgfpathlineto{\pgfqpoint{1.250122in}{4.077958in}}%
\pgfpathlineto{\pgfqpoint{1.265555in}{4.010381in}}%
\pgfpathlineto{\pgfqpoint{1.280989in}{3.965960in}}%
\pgfpathlineto{\pgfqpoint{1.296423in}{3.945586in}}%
\pgfpathlineto{\pgfqpoint{1.311856in}{3.949665in}}%
\pgfpathlineto{\pgfqpoint{1.327290in}{3.978118in}}%
\pgfpathlineto{\pgfqpoint{1.342724in}{4.030374in}}%
\pgfpathlineto{\pgfqpoint{1.358157in}{4.105384in}}%
\pgfpathlineto{\pgfqpoint{1.373591in}{4.201647in}}%
\pgfpathlineto{\pgfqpoint{1.389025in}{4.317232in}}%
\pgfpathlineto{\pgfqpoint{1.404458in}{4.449823in}}%
\pgfpathlineto{\pgfqpoint{1.435325in}{4.755104in}}%
\pgfpathlineto{\pgfqpoint{1.527927in}{5.759151in}}%
\pgfpathlineto{\pgfqpoint{1.543361in}{5.903973in}}%
\pgfpathlineto{\pgfqpoint{1.558795in}{6.033999in}}%
\pgfpathlineto{\pgfqpoint{1.574228in}{6.146622in}}%
\pgfpathlineto{\pgfqpoint{1.589662in}{6.239584in}}%
\pgfpathlineto{\pgfqpoint{1.605095in}{6.311023in}}%
\pgfpathlineto{\pgfqpoint{1.620529in}{6.359507in}}%
\pgfpathlineto{\pgfqpoint{1.635963in}{6.384064in}}%
\pgfpathlineto{\pgfqpoint{1.651396in}{6.384202in}}%
\pgfpathlineto{\pgfqpoint{1.666830in}{6.359918in}}%
\pgfpathlineto{\pgfqpoint{1.682264in}{6.311698in}}%
\pgfpathlineto{\pgfqpoint{1.697697in}{6.240510in}}%
\pgfpathlineto{\pgfqpoint{1.713131in}{6.147780in}}%
\pgfpathlineto{\pgfqpoint{1.728564in}{6.035366in}}%
\pgfpathlineto{\pgfqpoint{1.743998in}{5.905522in}}%
\pgfpathlineto{\pgfqpoint{1.774865in}{5.604251in}}%
\pgfpathlineto{\pgfqpoint{1.821166in}{5.095082in}}%
\pgfpathlineto{\pgfqpoint{1.852034in}{4.756938in}}%
\pgfpathlineto{\pgfqpoint{1.882901in}{4.451400in}}%
\pgfpathlineto{\pgfqpoint{1.898334in}{4.318633in}}%
\pgfpathlineto{\pgfqpoint{1.913768in}{4.202843in}}%
\pgfpathlineto{\pgfqpoint{1.929202in}{4.106352in}}%
\pgfpathlineto{\pgfqpoint{1.944635in}{4.031093in}}%
\pgfpathlineto{\pgfqpoint{1.960069in}{3.978575in}}%
\pgfpathlineto{\pgfqpoint{1.975503in}{3.949850in}}%
\pgfpathlineto{\pgfqpoint{1.990936in}{3.945495in}}%
\pgfpathlineto{\pgfqpoint{2.006370in}{3.965596in}}%
\pgfpathlineto{\pgfqpoint{2.021804in}{4.009751in}}%
\pgfpathlineto{\pgfqpoint{2.037237in}{4.077074in}}%
\pgfpathlineto{\pgfqpoint{2.052671in}{4.166217in}}%
\pgfpathlineto{\pgfqpoint{2.068104in}{4.275392in}}%
\pgfpathlineto{\pgfqpoint{2.083538in}{4.402412in}}%
\pgfpathlineto{\pgfqpoint{2.114405in}{4.699494in}}%
\pgfpathlineto{\pgfqpoint{2.160706in}{5.206575in}}%
\pgfpathlineto{\pgfqpoint{2.207007in}{5.706382in}}%
\pgfpathlineto{\pgfqpoint{2.237874in}{5.991071in}}%
\pgfpathlineto{\pgfqpoint{2.253308in}{6.109959in}}%
\pgfpathlineto{\pgfqpoint{2.268742in}{6.209922in}}%
\pgfpathlineto{\pgfqpoint{2.284175in}{6.288955in}}%
\pgfpathlineto{\pgfqpoint{2.299609in}{6.345476in}}%
\pgfpathlineto{\pgfqpoint{2.315043in}{6.378351in}}%
\pgfpathlineto{\pgfqpoint{2.330476in}{6.386921in}}%
\pgfpathlineto{\pgfqpoint{2.345910in}{6.371015in}}%
\pgfpathlineto{\pgfqpoint{2.361344in}{6.330951in}}%
\pgfpathlineto{\pgfqpoint{2.376777in}{6.267533in}}%
\pgfpathlineto{\pgfqpoint{2.392211in}{6.182031in}}%
\pgfpathlineto{\pgfqpoint{2.407644in}{6.076159in}}%
\pgfpathlineto{\pgfqpoint{2.423078in}{5.952039in}}%
\pgfpathlineto{\pgfqpoint{2.453945in}{5.659322in}}%
\pgfpathlineto{\pgfqpoint{2.484813in}{5.327231in}}%
\pgfpathlineto{\pgfqpoint{2.546547in}{4.651912in}}%
\pgfpathlineto{\pgfqpoint{2.577414in}{4.362555in}}%
\pgfpathlineto{\pgfqpoint{2.592848in}{4.240640in}}%
\pgfpathlineto{\pgfqpoint{2.608282in}{4.137266in}}%
\pgfpathlineto{\pgfqpoint{2.623715in}{4.054504in}}%
\pgfpathlineto{\pgfqpoint{2.639149in}{3.994014in}}%
\pgfpathlineto{\pgfqpoint{2.654583in}{3.957008in}}%
\pgfpathlineto{\pgfqpoint{2.670016in}{3.944228in}}%
\pgfpathlineto{\pgfqpoint{2.685450in}{3.955930in}}%
\pgfpathlineto{\pgfqpoint{2.700884in}{3.991879in}}%
\pgfpathlineto{\pgfqpoint{2.716317in}{4.051355in}}%
\pgfpathlineto{\pgfqpoint{2.731751in}{4.133165in}}%
\pgfpathlineto{\pgfqpoint{2.747184in}{4.235671in}}%
\pgfpathlineto{\pgfqpoint{2.762618in}{4.356817in}}%
\pgfpathlineto{\pgfqpoint{2.793485in}{4.644994in}}%
\pgfpathlineto{\pgfqpoint{2.824353in}{4.974707in}}%
\pgfpathlineto{\pgfqpoint{2.886087in}{5.652327in}}%
\pgfpathlineto{\pgfqpoint{2.916954in}{5.946180in}}%
\pgfpathlineto{\pgfqpoint{2.932388in}{6.071050in}}%
\pgfpathlineto{\pgfqpoint{2.947822in}{6.177774in}}%
\pgfpathlineto{\pgfqpoint{2.963255in}{6.264215in}}%
\pgfpathlineto{\pgfqpoint{2.978689in}{6.328638in}}%
\pgfpathlineto{\pgfqpoint{2.994123in}{6.369753in}}%
\pgfpathlineto{\pgfqpoint{3.009556in}{6.386735in}}%
\pgfpathlineto{\pgfqpoint{3.024990in}{6.379245in}}%
\pgfpathlineto{\pgfqpoint{3.040424in}{6.347432in}}%
\pgfpathlineto{\pgfqpoint{3.055857in}{6.291934in}}%
\pgfpathlineto{\pgfqpoint{3.071291in}{6.213864in}}%
\pgfpathlineto{\pgfqpoint{3.086724in}{6.114785in}}%
\pgfpathlineto{\pgfqpoint{3.102158in}{5.996685in}}%
\pgfpathlineto{\pgfqpoint{3.117592in}{5.861929in}}%
\pgfpathlineto{\pgfqpoint{3.148459in}{5.553537in}}%
\pgfpathlineto{\pgfqpoint{3.241061in}{4.551319in}}%
\pgfpathlineto{\pgfqpoint{3.256494in}{4.408389in}}%
\pgfpathlineto{\pgfqpoint{3.271928in}{4.280638in}}%
\pgfpathlineto{\pgfqpoint{3.287362in}{4.170627in}}%
\pgfpathlineto{\pgfqpoint{3.302795in}{4.080559in}}%
\pgfpathlineto{\pgfqpoint{3.318229in}{4.012241in}}%
\pgfpathlineto{\pgfqpoint{3.333663in}{3.967042in}}%
\pgfpathlineto{\pgfqpoint{3.349096in}{3.945868in}}%
\pgfpathlineto{\pgfqpoint{3.364530in}{3.949142in}}%
\pgfpathlineto{\pgfqpoint{3.379963in}{3.976800in}}%
\pgfpathlineto{\pgfqpoint{3.395397in}{4.028286in}}%
\pgfpathlineto{\pgfqpoint{3.410831in}{4.102570in}}%
\pgfpathlineto{\pgfqpoint{3.426264in}{4.198162in}}%
\pgfpathlineto{\pgfqpoint{3.441698in}{4.313147in}}%
\pgfpathlineto{\pgfqpoint{3.457132in}{4.445219in}}%
\pgfpathlineto{\pgfqpoint{3.487999in}{4.749749in}}%
\pgfpathlineto{\pgfqpoint{3.596034in}{5.899431in}}%
\pgfpathlineto{\pgfqpoint{3.611468in}{6.029986in}}%
\pgfpathlineto{\pgfqpoint{3.626902in}{6.143219in}}%
\pgfpathlineto{\pgfqpoint{3.642335in}{6.236860in}}%
\pgfpathlineto{\pgfqpoint{3.657769in}{6.309032in}}%
\pgfpathlineto{\pgfqpoint{3.673203in}{6.358289in}}%
\pgfpathlineto{\pgfqpoint{3.688636in}{6.383643in}}%
\pgfpathlineto{\pgfqpoint{3.704070in}{6.384587in}}%
\pgfpathlineto{\pgfqpoint{3.719503in}{6.361101in}}%
\pgfpathlineto{\pgfqpoint{3.734937in}{6.313656in}}%
\pgfpathlineto{\pgfqpoint{3.750371in}{6.243203in}}%
\pgfpathlineto{\pgfqpoint{3.765804in}{6.151153in}}%
\pgfpathlineto{\pgfqpoint{3.781238in}{6.039353in}}%
\pgfpathlineto{\pgfqpoint{3.796672in}{5.910043in}}%
\pgfpathlineto{\pgfqpoint{3.827539in}{5.609558in}}%
\pgfpathlineto{\pgfqpoint{3.873840in}{5.100765in}}%
\pgfpathlineto{\pgfqpoint{3.904707in}{4.762305in}}%
\pgfpathlineto{\pgfqpoint{3.935574in}{4.456025in}}%
\pgfpathlineto{\pgfqpoint{3.951008in}{4.322742in}}%
\pgfpathlineto{\pgfqpoint{3.966442in}{4.206355in}}%
\pgfpathlineto{\pgfqpoint{3.981875in}{4.109197in}}%
\pgfpathlineto{\pgfqpoint{3.997309in}{4.033213in}}%
\pgfpathlineto{\pgfqpoint{4.012743in}{3.979928in}}%
\pgfpathlineto{\pgfqpoint{4.028176in}{3.950409in}}%
\pgfpathlineto{\pgfqpoint{4.043610in}{3.945249in}}%
\pgfpathlineto{\pgfqpoint{4.059043in}{3.964549in}}%
\pgfpathlineto{\pgfqpoint{4.074477in}{4.007925in}}%
\pgfpathlineto{\pgfqpoint{4.089911in}{4.074505in}}%
\pgfpathlineto{\pgfqpoint{4.105344in}{4.162956in}}%
\pgfpathlineto{\pgfqpoint{4.120778in}{4.271505in}}%
\pgfpathlineto{\pgfqpoint{4.136212in}{4.397977in}}%
\pgfpathlineto{\pgfqpoint{4.167079in}{4.694239in}}%
\pgfpathlineto{\pgfqpoint{4.213380in}{5.200886in}}%
\pgfpathlineto{\pgfqpoint{4.259681in}{5.701273in}}%
\pgfpathlineto{\pgfqpoint{4.290548in}{5.986867in}}%
\pgfpathlineto{\pgfqpoint{4.305982in}{6.106338in}}%
\pgfpathlineto{\pgfqpoint{4.321415in}{6.206957in}}%
\pgfpathlineto{\pgfqpoint{4.336849in}{6.286707in}}%
\pgfpathlineto{\pgfqpoint{4.352283in}{6.343988in}}%
\pgfpathlineto{\pgfqpoint{4.367716in}{6.377654in}}%
\pgfpathlineto{\pgfqpoint{4.383150in}{6.387029in}}%
\pgfpathlineto{\pgfqpoint{4.398583in}{6.371925in}}%
\pgfpathlineto{\pgfqpoint{4.414017in}{6.332646in}}%
\pgfpathlineto{\pgfqpoint{4.429451in}{6.269978in}}%
\pgfpathlineto{\pgfqpoint{4.444884in}{6.185177in}}%
\pgfpathlineto{\pgfqpoint{4.460318in}{6.079943in}}%
\pgfpathlineto{\pgfqpoint{4.475752in}{5.956386in}}%
\pgfpathlineto{\pgfqpoint{4.506619in}{5.664523in}}%
\pgfpathlineto{\pgfqpoint{4.537486in}{5.332870in}}%
\pgfpathlineto{\pgfqpoint{4.599221in}{4.657081in}}%
\pgfpathlineto{\pgfqpoint{4.630088in}{4.366852in}}%
\pgfpathlineto{\pgfqpoint{4.645522in}{4.244367in}}%
\pgfpathlineto{\pgfqpoint{4.660955in}{4.140348in}}%
\pgfpathlineto{\pgfqpoint{4.660955in}{4.140348in}}%
\pgfusepath{stroke}%
\end{pgfscope}%
\begin{pgfscope}%
\pgfpathrectangle{\pgfqpoint{0.625831in}{3.799602in}}{\pgfqpoint{4.227273in}{2.745455in}} %
\pgfusepath{clip}%
\pgfsetrectcap%
\pgfsetroundjoin%
\pgfsetlinewidth{0.501875pt}%
\definecolor{currentstroke}{rgb}{0.421569,0.122888,0.998103}%
\pgfsetstrokecolor{currentstroke}%
\pgfsetdash{}{0pt}%
\pgfpathmoveto{\pgfqpoint{0.817980in}{5.494719in}}%
\pgfpathlineto{\pgfqpoint{0.833414in}{5.657542in}}%
\pgfpathlineto{\pgfqpoint{0.848847in}{5.810507in}}%
\pgfpathlineto{\pgfqpoint{0.864281in}{5.950549in}}%
\pgfpathlineto{\pgfqpoint{0.879715in}{6.074861in}}%
\pgfpathlineto{\pgfqpoint{0.895148in}{6.180950in}}%
\pgfpathlineto{\pgfqpoint{0.910582in}{6.266692in}}%
\pgfpathlineto{\pgfqpoint{0.926015in}{6.330366in}}%
\pgfpathlineto{\pgfqpoint{0.941449in}{6.370698in}}%
\pgfpathlineto{\pgfqpoint{0.956883in}{6.386878in}}%
\pgfpathlineto{\pgfqpoint{0.972316in}{6.378583in}}%
\pgfpathlineto{\pgfqpoint{0.987750in}{6.345979in}}%
\pgfpathlineto{\pgfqpoint{1.003184in}{6.289718in}}%
\pgfpathlineto{\pgfqpoint{1.018617in}{6.210930in}}%
\pgfpathlineto{\pgfqpoint{1.034051in}{6.111192in}}%
\pgfpathlineto{\pgfqpoint{1.049485in}{5.992504in}}%
\pgfpathlineto{\pgfqpoint{1.064918in}{5.857245in}}%
\pgfpathlineto{\pgfqpoint{1.095785in}{5.548135in}}%
\pgfpathlineto{\pgfqpoint{1.188387in}{4.546406in}}%
\pgfpathlineto{\pgfqpoint{1.203821in}{4.403932in}}%
\pgfpathlineto{\pgfqpoint{1.219255in}{4.276726in}}%
\pgfpathlineto{\pgfqpoint{1.234688in}{4.167337in}}%
\pgfpathlineto{\pgfqpoint{1.250122in}{4.077958in}}%
\pgfpathlineto{\pgfqpoint{1.265555in}{4.010381in}}%
\pgfpathlineto{\pgfqpoint{1.280989in}{3.965960in}}%
\pgfpathlineto{\pgfqpoint{1.296423in}{3.945586in}}%
\pgfpathlineto{\pgfqpoint{1.311856in}{3.949665in}}%
\pgfpathlineto{\pgfqpoint{1.327290in}{3.978118in}}%
\pgfpathlineto{\pgfqpoint{1.342724in}{4.030374in}}%
\pgfpathlineto{\pgfqpoint{1.358157in}{4.105384in}}%
\pgfpathlineto{\pgfqpoint{1.373591in}{4.201647in}}%
\pgfpathlineto{\pgfqpoint{1.389025in}{4.317232in}}%
\pgfpathlineto{\pgfqpoint{1.404458in}{4.449823in}}%
\pgfpathlineto{\pgfqpoint{1.435325in}{4.755104in}}%
\pgfpathlineto{\pgfqpoint{1.527927in}{5.759151in}}%
\pgfpathlineto{\pgfqpoint{1.543361in}{5.903973in}}%
\pgfpathlineto{\pgfqpoint{1.558795in}{6.033999in}}%
\pgfpathlineto{\pgfqpoint{1.574228in}{6.146622in}}%
\pgfpathlineto{\pgfqpoint{1.589662in}{6.239584in}}%
\pgfpathlineto{\pgfqpoint{1.605095in}{6.311023in}}%
\pgfpathlineto{\pgfqpoint{1.620529in}{6.359507in}}%
\pgfpathlineto{\pgfqpoint{1.635963in}{6.384064in}}%
\pgfpathlineto{\pgfqpoint{1.651396in}{6.384202in}}%
\pgfpathlineto{\pgfqpoint{1.666830in}{6.359918in}}%
\pgfpathlineto{\pgfqpoint{1.682264in}{6.311698in}}%
\pgfpathlineto{\pgfqpoint{1.697697in}{6.240510in}}%
\pgfpathlineto{\pgfqpoint{1.713131in}{6.147780in}}%
\pgfpathlineto{\pgfqpoint{1.728564in}{6.035366in}}%
\pgfpathlineto{\pgfqpoint{1.743998in}{5.905522in}}%
\pgfpathlineto{\pgfqpoint{1.774865in}{5.604251in}}%
\pgfpathlineto{\pgfqpoint{1.821166in}{5.095082in}}%
\pgfpathlineto{\pgfqpoint{1.852034in}{4.756938in}}%
\pgfpathlineto{\pgfqpoint{1.882901in}{4.451400in}}%
\pgfpathlineto{\pgfqpoint{1.898334in}{4.318633in}}%
\pgfpathlineto{\pgfqpoint{1.913768in}{4.202843in}}%
\pgfpathlineto{\pgfqpoint{1.929202in}{4.106352in}}%
\pgfpathlineto{\pgfqpoint{1.944635in}{4.031093in}}%
\pgfpathlineto{\pgfqpoint{1.960069in}{3.978575in}}%
\pgfpathlineto{\pgfqpoint{1.975503in}{3.949850in}}%
\pgfpathlineto{\pgfqpoint{1.990936in}{3.945495in}}%
\pgfpathlineto{\pgfqpoint{2.006370in}{3.965596in}}%
\pgfpathlineto{\pgfqpoint{2.021804in}{4.009751in}}%
\pgfpathlineto{\pgfqpoint{2.037237in}{4.077074in}}%
\pgfpathlineto{\pgfqpoint{2.052671in}{4.166217in}}%
\pgfpathlineto{\pgfqpoint{2.068104in}{4.275392in}}%
\pgfpathlineto{\pgfqpoint{2.083538in}{4.402412in}}%
\pgfpathlineto{\pgfqpoint{2.114405in}{4.699494in}}%
\pgfpathlineto{\pgfqpoint{2.160706in}{5.206575in}}%
\pgfpathlineto{\pgfqpoint{2.207007in}{5.706382in}}%
\pgfpathlineto{\pgfqpoint{2.237874in}{5.991071in}}%
\pgfpathlineto{\pgfqpoint{2.253308in}{6.109959in}}%
\pgfpathlineto{\pgfqpoint{2.268742in}{6.209922in}}%
\pgfpathlineto{\pgfqpoint{2.284175in}{6.288955in}}%
\pgfpathlineto{\pgfqpoint{2.299609in}{6.345476in}}%
\pgfpathlineto{\pgfqpoint{2.315043in}{6.378351in}}%
\pgfpathlineto{\pgfqpoint{2.330476in}{6.386921in}}%
\pgfpathlineto{\pgfqpoint{2.345910in}{6.371015in}}%
\pgfpathlineto{\pgfqpoint{2.361344in}{6.330951in}}%
\pgfpathlineto{\pgfqpoint{2.376777in}{6.267533in}}%
\pgfpathlineto{\pgfqpoint{2.392211in}{6.182031in}}%
\pgfpathlineto{\pgfqpoint{2.407644in}{6.076159in}}%
\pgfpathlineto{\pgfqpoint{2.423078in}{5.952039in}}%
\pgfpathlineto{\pgfqpoint{2.453945in}{5.659322in}}%
\pgfpathlineto{\pgfqpoint{2.484813in}{5.327231in}}%
\pgfpathlineto{\pgfqpoint{2.546547in}{4.651912in}}%
\pgfpathlineto{\pgfqpoint{2.577414in}{4.362555in}}%
\pgfpathlineto{\pgfqpoint{2.592848in}{4.240640in}}%
\pgfpathlineto{\pgfqpoint{2.608282in}{4.137266in}}%
\pgfpathlineto{\pgfqpoint{2.623715in}{4.054504in}}%
\pgfpathlineto{\pgfqpoint{2.639149in}{3.994014in}}%
\pgfpathlineto{\pgfqpoint{2.654583in}{3.957008in}}%
\pgfpathlineto{\pgfqpoint{2.670016in}{3.944228in}}%
\pgfpathlineto{\pgfqpoint{2.685450in}{3.955930in}}%
\pgfpathlineto{\pgfqpoint{2.700884in}{3.991879in}}%
\pgfpathlineto{\pgfqpoint{2.716317in}{4.051355in}}%
\pgfpathlineto{\pgfqpoint{2.731751in}{4.133165in}}%
\pgfpathlineto{\pgfqpoint{2.747184in}{4.235671in}}%
\pgfpathlineto{\pgfqpoint{2.762618in}{4.356817in}}%
\pgfpathlineto{\pgfqpoint{2.793485in}{4.644994in}}%
\pgfpathlineto{\pgfqpoint{2.824353in}{4.974707in}}%
\pgfpathlineto{\pgfqpoint{2.886087in}{5.652327in}}%
\pgfpathlineto{\pgfqpoint{2.916954in}{5.946180in}}%
\pgfpathlineto{\pgfqpoint{2.932388in}{6.071050in}}%
\pgfpathlineto{\pgfqpoint{2.947822in}{6.177774in}}%
\pgfpathlineto{\pgfqpoint{2.963255in}{6.264215in}}%
\pgfpathlineto{\pgfqpoint{2.978689in}{6.328638in}}%
\pgfpathlineto{\pgfqpoint{2.994123in}{6.369753in}}%
\pgfpathlineto{\pgfqpoint{3.009556in}{6.386735in}}%
\pgfpathlineto{\pgfqpoint{3.024990in}{6.379245in}}%
\pgfpathlineto{\pgfqpoint{3.040424in}{6.347432in}}%
\pgfpathlineto{\pgfqpoint{3.055857in}{6.291934in}}%
\pgfpathlineto{\pgfqpoint{3.071291in}{6.213864in}}%
\pgfpathlineto{\pgfqpoint{3.086724in}{6.114785in}}%
\pgfpathlineto{\pgfqpoint{3.102158in}{5.996685in}}%
\pgfpathlineto{\pgfqpoint{3.117592in}{5.861929in}}%
\pgfpathlineto{\pgfqpoint{3.148459in}{5.553537in}}%
\pgfpathlineto{\pgfqpoint{3.241061in}{4.551319in}}%
\pgfpathlineto{\pgfqpoint{3.256494in}{4.408389in}}%
\pgfpathlineto{\pgfqpoint{3.271928in}{4.280638in}}%
\pgfpathlineto{\pgfqpoint{3.287362in}{4.170627in}}%
\pgfpathlineto{\pgfqpoint{3.302795in}{4.080559in}}%
\pgfpathlineto{\pgfqpoint{3.318229in}{4.012241in}}%
\pgfpathlineto{\pgfqpoint{3.333663in}{3.967042in}}%
\pgfpathlineto{\pgfqpoint{3.349096in}{3.945868in}}%
\pgfpathlineto{\pgfqpoint{3.364530in}{3.949142in}}%
\pgfpathlineto{\pgfqpoint{3.379963in}{3.976800in}}%
\pgfpathlineto{\pgfqpoint{3.395397in}{4.028286in}}%
\pgfpathlineto{\pgfqpoint{3.410831in}{4.102570in}}%
\pgfpathlineto{\pgfqpoint{3.426264in}{4.198162in}}%
\pgfpathlineto{\pgfqpoint{3.441698in}{4.313147in}}%
\pgfpathlineto{\pgfqpoint{3.457132in}{4.445219in}}%
\pgfpathlineto{\pgfqpoint{3.487999in}{4.749749in}}%
\pgfpathlineto{\pgfqpoint{3.596034in}{5.899431in}}%
\pgfpathlineto{\pgfqpoint{3.611468in}{6.029986in}}%
\pgfpathlineto{\pgfqpoint{3.626902in}{6.143219in}}%
\pgfpathlineto{\pgfqpoint{3.642335in}{6.236860in}}%
\pgfpathlineto{\pgfqpoint{3.657769in}{6.309032in}}%
\pgfpathlineto{\pgfqpoint{3.673203in}{6.358289in}}%
\pgfpathlineto{\pgfqpoint{3.688636in}{6.383643in}}%
\pgfpathlineto{\pgfqpoint{3.704070in}{6.384587in}}%
\pgfpathlineto{\pgfqpoint{3.719503in}{6.361101in}}%
\pgfpathlineto{\pgfqpoint{3.734937in}{6.313656in}}%
\pgfpathlineto{\pgfqpoint{3.750371in}{6.243203in}}%
\pgfpathlineto{\pgfqpoint{3.765804in}{6.151153in}}%
\pgfpathlineto{\pgfqpoint{3.781238in}{6.039353in}}%
\pgfpathlineto{\pgfqpoint{3.796672in}{5.910043in}}%
\pgfpathlineto{\pgfqpoint{3.827539in}{5.609558in}}%
\pgfpathlineto{\pgfqpoint{3.873840in}{5.100765in}}%
\pgfpathlineto{\pgfqpoint{3.904707in}{4.762305in}}%
\pgfpathlineto{\pgfqpoint{3.935574in}{4.456025in}}%
\pgfpathlineto{\pgfqpoint{3.951008in}{4.322742in}}%
\pgfpathlineto{\pgfqpoint{3.966442in}{4.206355in}}%
\pgfpathlineto{\pgfqpoint{3.981875in}{4.109197in}}%
\pgfpathlineto{\pgfqpoint{3.997309in}{4.033213in}}%
\pgfpathlineto{\pgfqpoint{4.012743in}{3.979928in}}%
\pgfpathlineto{\pgfqpoint{4.028176in}{3.950409in}}%
\pgfpathlineto{\pgfqpoint{4.043610in}{3.945249in}}%
\pgfpathlineto{\pgfqpoint{4.059043in}{3.964549in}}%
\pgfpathlineto{\pgfqpoint{4.074477in}{4.007925in}}%
\pgfpathlineto{\pgfqpoint{4.089911in}{4.074505in}}%
\pgfpathlineto{\pgfqpoint{4.105344in}{4.162956in}}%
\pgfpathlineto{\pgfqpoint{4.120778in}{4.271505in}}%
\pgfpathlineto{\pgfqpoint{4.136212in}{4.397977in}}%
\pgfpathlineto{\pgfqpoint{4.167079in}{4.694239in}}%
\pgfpathlineto{\pgfqpoint{4.213380in}{5.200886in}}%
\pgfpathlineto{\pgfqpoint{4.259681in}{5.701273in}}%
\pgfpathlineto{\pgfqpoint{4.290548in}{5.986867in}}%
\pgfpathlineto{\pgfqpoint{4.305982in}{6.106338in}}%
\pgfpathlineto{\pgfqpoint{4.321415in}{6.206957in}}%
\pgfpathlineto{\pgfqpoint{4.336849in}{6.286707in}}%
\pgfpathlineto{\pgfqpoint{4.352283in}{6.343988in}}%
\pgfpathlineto{\pgfqpoint{4.367716in}{6.377654in}}%
\pgfpathlineto{\pgfqpoint{4.383150in}{6.387029in}}%
\pgfpathlineto{\pgfqpoint{4.398583in}{6.371925in}}%
\pgfpathlineto{\pgfqpoint{4.414017in}{6.332646in}}%
\pgfpathlineto{\pgfqpoint{4.429451in}{6.269978in}}%
\pgfpathlineto{\pgfqpoint{4.444884in}{6.185177in}}%
\pgfpathlineto{\pgfqpoint{4.460318in}{6.079943in}}%
\pgfpathlineto{\pgfqpoint{4.475752in}{5.956386in}}%
\pgfpathlineto{\pgfqpoint{4.506619in}{5.664523in}}%
\pgfpathlineto{\pgfqpoint{4.537486in}{5.332870in}}%
\pgfpathlineto{\pgfqpoint{4.599221in}{4.657081in}}%
\pgfpathlineto{\pgfqpoint{4.630088in}{4.366852in}}%
\pgfpathlineto{\pgfqpoint{4.645522in}{4.244367in}}%
\pgfpathlineto{\pgfqpoint{4.660955in}{4.140348in}}%
\pgfpathlineto{\pgfqpoint{4.660955in}{4.140348in}}%
\pgfusepath{stroke}%
\end{pgfscope}%
\begin{pgfscope}%
\pgfpathrectangle{\pgfqpoint{0.625831in}{3.799602in}}{\pgfqpoint{4.227273in}{2.745455in}} %
\pgfusepath{clip}%
\pgfsetrectcap%
\pgfsetroundjoin%
\pgfsetlinewidth{0.501875pt}%
\definecolor{currentstroke}{rgb}{0.343137,0.243914,0.992421}%
\pgfsetstrokecolor{currentstroke}%
\pgfsetdash{}{0pt}%
\pgfpathmoveto{\pgfqpoint{0.817980in}{5.316474in}}%
\pgfpathlineto{\pgfqpoint{0.833414in}{5.410339in}}%
\pgfpathlineto{\pgfqpoint{0.848847in}{5.499276in}}%
\pgfpathlineto{\pgfqpoint{0.879715in}{5.656315in}}%
\pgfpathlineto{\pgfqpoint{0.895148in}{5.721741in}}%
\pgfpathlineto{\pgfqpoint{0.910582in}{5.776943in}}%
\pgfpathlineto{\pgfqpoint{0.926015in}{5.821050in}}%
\pgfpathlineto{\pgfqpoint{0.941449in}{5.853409in}}%
\pgfpathlineto{\pgfqpoint{0.956883in}{5.873604in}}%
\pgfpathlineto{\pgfqpoint{0.972316in}{5.881462in}}%
\pgfpathlineto{\pgfqpoint{0.987750in}{5.877051in}}%
\pgfpathlineto{\pgfqpoint{1.003184in}{5.860688in}}%
\pgfpathlineto{\pgfqpoint{1.018617in}{5.832923in}}%
\pgfpathlineto{\pgfqpoint{1.034051in}{5.794534in}}%
\pgfpathlineto{\pgfqpoint{1.049485in}{5.746509in}}%
\pgfpathlineto{\pgfqpoint{1.064918in}{5.690025in}}%
\pgfpathlineto{\pgfqpoint{1.095785in}{5.557197in}}%
\pgfpathlineto{\pgfqpoint{1.188387in}{5.116209in}}%
\pgfpathlineto{\pgfqpoint{1.203821in}{5.054068in}}%
\pgfpathlineto{\pgfqpoint{1.219255in}{4.999284in}}%
\pgfpathlineto{\pgfqpoint{1.234688in}{4.953121in}}%
\pgfpathlineto{\pgfqpoint{1.250122in}{4.916667in}}%
\pgfpathlineto{\pgfqpoint{1.265555in}{4.890809in}}%
\pgfpathlineto{\pgfqpoint{1.280989in}{4.876217in}}%
\pgfpathlineto{\pgfqpoint{1.296423in}{4.873327in}}%
\pgfpathlineto{\pgfqpoint{1.311856in}{4.882336in}}%
\pgfpathlineto{\pgfqpoint{1.327290in}{4.903196in}}%
\pgfpathlineto{\pgfqpoint{1.342724in}{4.935616in}}%
\pgfpathlineto{\pgfqpoint{1.358157in}{4.979064in}}%
\pgfpathlineto{\pgfqpoint{1.373591in}{5.032783in}}%
\pgfpathlineto{\pgfqpoint{1.389025in}{5.095804in}}%
\pgfpathlineto{\pgfqpoint{1.419892in}{5.244921in}}%
\pgfpathlineto{\pgfqpoint{1.450759in}{5.415228in}}%
\pgfpathlineto{\pgfqpoint{1.512494in}{5.766676in}}%
\pgfpathlineto{\pgfqpoint{1.543361in}{5.920619in}}%
\pgfpathlineto{\pgfqpoint{1.558795in}{5.986622in}}%
\pgfpathlineto{\pgfqpoint{1.574228in}{6.043523in}}%
\pgfpathlineto{\pgfqpoint{1.589662in}{6.090195in}}%
\pgfpathlineto{\pgfqpoint{1.605095in}{6.125704in}}%
\pgfpathlineto{\pgfqpoint{1.620529in}{6.149336in}}%
\pgfpathlineto{\pgfqpoint{1.635963in}{6.160605in}}%
\pgfpathlineto{\pgfqpoint{1.651396in}{6.159266in}}%
\pgfpathlineto{\pgfqpoint{1.666830in}{6.145318in}}%
\pgfpathlineto{\pgfqpoint{1.682264in}{6.119008in}}%
\pgfpathlineto{\pgfqpoint{1.697697in}{6.080821in}}%
\pgfpathlineto{\pgfqpoint{1.713131in}{6.031473in}}%
\pgfpathlineto{\pgfqpoint{1.728564in}{5.971896in}}%
\pgfpathlineto{\pgfqpoint{1.743998in}{5.903220in}}%
\pgfpathlineto{\pgfqpoint{1.774865in}{5.743942in}}%
\pgfpathlineto{\pgfqpoint{1.821166in}{5.473626in}}%
\pgfpathlineto{\pgfqpoint{1.867467in}{5.206407in}}%
\pgfpathlineto{\pgfqpoint{1.898334in}{5.052182in}}%
\pgfpathlineto{\pgfqpoint{1.913768in}{4.986653in}}%
\pgfpathlineto{\pgfqpoint{1.929202in}{4.930461in}}%
\pgfpathlineto{\pgfqpoint{1.944635in}{4.884582in}}%
\pgfpathlineto{\pgfqpoint{1.960069in}{4.849777in}}%
\pgfpathlineto{\pgfqpoint{1.975503in}{4.826582in}}%
\pgfpathlineto{\pgfqpoint{1.990936in}{4.815294in}}%
\pgfpathlineto{\pgfqpoint{2.006370in}{4.815966in}}%
\pgfpathlineto{\pgfqpoint{2.021804in}{4.828406in}}%
\pgfpathlineto{\pgfqpoint{2.037237in}{4.852182in}}%
\pgfpathlineto{\pgfqpoint{2.052671in}{4.886629in}}%
\pgfpathlineto{\pgfqpoint{2.068104in}{4.930865in}}%
\pgfpathlineto{\pgfqpoint{2.083538in}{4.983805in}}%
\pgfpathlineto{\pgfqpoint{2.114405in}{5.110600in}}%
\pgfpathlineto{\pgfqpoint{2.160706in}{5.330171in}}%
\pgfpathlineto{\pgfqpoint{2.191574in}{5.476617in}}%
\pgfpathlineto{\pgfqpoint{2.222441in}{5.607386in}}%
\pgfpathlineto{\pgfqpoint{2.237874in}{5.662991in}}%
\pgfpathlineto{\pgfqpoint{2.253308in}{5.710238in}}%
\pgfpathlineto{\pgfqpoint{2.268742in}{5.747947in}}%
\pgfpathlineto{\pgfqpoint{2.284175in}{5.775129in}}%
\pgfpathlineto{\pgfqpoint{2.299609in}{5.791006in}}%
\pgfpathlineto{\pgfqpoint{2.315043in}{5.795023in}}%
\pgfpathlineto{\pgfqpoint{2.330476in}{5.786864in}}%
\pgfpathlineto{\pgfqpoint{2.345910in}{5.766456in}}%
\pgfpathlineto{\pgfqpoint{2.361344in}{5.733971in}}%
\pgfpathlineto{\pgfqpoint{2.376777in}{5.689824in}}%
\pgfpathlineto{\pgfqpoint{2.392211in}{5.634663in}}%
\pgfpathlineto{\pgfqpoint{2.407644in}{5.569357in}}%
\pgfpathlineto{\pgfqpoint{2.423078in}{5.494980in}}%
\pgfpathlineto{\pgfqpoint{2.453945in}{5.324199in}}%
\pgfpathlineto{\pgfqpoint{2.500246in}{5.035935in}}%
\pgfpathlineto{\pgfqpoint{2.546547in}{4.749811in}}%
\pgfpathlineto{\pgfqpoint{2.577414in}{4.582757in}}%
\pgfpathlineto{\pgfqpoint{2.592848in}{4.510887in}}%
\pgfpathlineto{\pgfqpoint{2.608282in}{4.448486in}}%
\pgfpathlineto{\pgfqpoint{2.623715in}{4.396601in}}%
\pgfpathlineto{\pgfqpoint{2.639149in}{4.356073in}}%
\pgfpathlineto{\pgfqpoint{2.654583in}{4.327518in}}%
\pgfpathlineto{\pgfqpoint{2.670016in}{4.311317in}}%
\pgfpathlineto{\pgfqpoint{2.685450in}{4.307609in}}%
\pgfpathlineto{\pgfqpoint{2.700884in}{4.316286in}}%
\pgfpathlineto{\pgfqpoint{2.716317in}{4.336997in}}%
\pgfpathlineto{\pgfqpoint{2.731751in}{4.369157in}}%
\pgfpathlineto{\pgfqpoint{2.747184in}{4.411953in}}%
\pgfpathlineto{\pgfqpoint{2.762618in}{4.464368in}}%
\pgfpathlineto{\pgfqpoint{2.778052in}{4.525196in}}%
\pgfpathlineto{\pgfqpoint{2.808919in}{4.666482in}}%
\pgfpathlineto{\pgfqpoint{2.901521in}{5.130074in}}%
\pgfpathlineto{\pgfqpoint{2.916954in}{5.195564in}}%
\pgfpathlineto{\pgfqpoint{2.932388in}{5.253604in}}%
\pgfpathlineto{\pgfqpoint{2.947822in}{5.302949in}}%
\pgfpathlineto{\pgfqpoint{2.963255in}{5.342532in}}%
\pgfpathlineto{\pgfqpoint{2.978689in}{5.371491in}}%
\pgfpathlineto{\pgfqpoint{2.994123in}{5.389185in}}%
\pgfpathlineto{\pgfqpoint{3.009556in}{5.395203in}}%
\pgfpathlineto{\pgfqpoint{3.024990in}{5.389378in}}%
\pgfpathlineto{\pgfqpoint{3.040424in}{5.371788in}}%
\pgfpathlineto{\pgfqpoint{3.055857in}{5.342753in}}%
\pgfpathlineto{\pgfqpoint{3.071291in}{5.302830in}}%
\pgfpathlineto{\pgfqpoint{3.086724in}{5.252803in}}%
\pgfpathlineto{\pgfqpoint{3.102158in}{5.193665in}}%
\pgfpathlineto{\pgfqpoint{3.133025in}{5.052958in}}%
\pgfpathlineto{\pgfqpoint{3.163893in}{4.892010in}}%
\pgfpathlineto{\pgfqpoint{3.225627in}{4.562235in}}%
\pgfpathlineto{\pgfqpoint{3.256494in}{4.420469in}}%
\pgfpathlineto{\pgfqpoint{3.271928in}{4.360824in}}%
\pgfpathlineto{\pgfqpoint{3.287362in}{4.310423in}}%
\pgfpathlineto{\pgfqpoint{3.302795in}{4.270363in}}%
\pgfpathlineto{\pgfqpoint{3.318229in}{4.241543in}}%
\pgfpathlineto{\pgfqpoint{3.333663in}{4.224641in}}%
\pgfpathlineto{\pgfqpoint{3.349096in}{4.220104in}}%
\pgfpathlineto{\pgfqpoint{3.364530in}{4.228138in}}%
\pgfpathlineto{\pgfqpoint{3.379963in}{4.248704in}}%
\pgfpathlineto{\pgfqpoint{3.395397in}{4.281518in}}%
\pgfpathlineto{\pgfqpoint{3.410831in}{4.326056in}}%
\pgfpathlineto{\pgfqpoint{3.426264in}{4.381567in}}%
\pgfpathlineto{\pgfqpoint{3.441698in}{4.447084in}}%
\pgfpathlineto{\pgfqpoint{3.472565in}{4.603324in}}%
\pgfpathlineto{\pgfqpoint{3.503433in}{4.783594in}}%
\pgfpathlineto{\pgfqpoint{3.580601in}{5.252315in}}%
\pgfpathlineto{\pgfqpoint{3.611468in}{5.411673in}}%
\pgfpathlineto{\pgfqpoint{3.626902in}{5.479309in}}%
\pgfpathlineto{\pgfqpoint{3.642335in}{5.537330in}}%
\pgfpathlineto{\pgfqpoint{3.657769in}{5.584785in}}%
\pgfpathlineto{\pgfqpoint{3.673203in}{5.620940in}}%
\pgfpathlineto{\pgfqpoint{3.688636in}{5.645290in}}%
\pgfpathlineto{\pgfqpoint{3.704070in}{5.657568in}}%
\pgfpathlineto{\pgfqpoint{3.719503in}{5.657753in}}%
\pgfpathlineto{\pgfqpoint{3.734937in}{5.646067in}}%
\pgfpathlineto{\pgfqpoint{3.750371in}{5.622974in}}%
\pgfpathlineto{\pgfqpoint{3.765804in}{5.589168in}}%
\pgfpathlineto{\pgfqpoint{3.781238in}{5.545557in}}%
\pgfpathlineto{\pgfqpoint{3.796672in}{5.493250in}}%
\pgfpathlineto{\pgfqpoint{3.827539in}{5.367829in}}%
\pgfpathlineto{\pgfqpoint{3.873840in}{5.150787in}}%
\pgfpathlineto{\pgfqpoint{3.904707in}{5.006391in}}%
\pgfpathlineto{\pgfqpoint{3.935574in}{4.877909in}}%
\pgfpathlineto{\pgfqpoint{3.951008in}{4.823499in}}%
\pgfpathlineto{\pgfqpoint{3.966442in}{4.777456in}}%
\pgfpathlineto{\pgfqpoint{3.981875in}{4.740932in}}%
\pgfpathlineto{\pgfqpoint{3.997309in}{4.714888in}}%
\pgfpathlineto{\pgfqpoint{4.012743in}{4.700075in}}%
\pgfpathlineto{\pgfqpoint{4.028176in}{4.697014in}}%
\pgfpathlineto{\pgfqpoint{4.043610in}{4.705987in}}%
\pgfpathlineto{\pgfqpoint{4.059043in}{4.727037in}}%
\pgfpathlineto{\pgfqpoint{4.074477in}{4.759956in}}%
\pgfpathlineto{\pgfqpoint{4.089911in}{4.804299in}}%
\pgfpathlineto{\pgfqpoint{4.105344in}{4.859387in}}%
\pgfpathlineto{\pgfqpoint{4.120778in}{4.924322in}}%
\pgfpathlineto{\pgfqpoint{4.151645in}{5.079161in}}%
\pgfpathlineto{\pgfqpoint{4.182513in}{5.258028in}}%
\pgfpathlineto{\pgfqpoint{4.259681in}{5.724348in}}%
\pgfpathlineto{\pgfqpoint{4.290548in}{5.882997in}}%
\pgfpathlineto{\pgfqpoint{4.305982in}{5.950187in}}%
\pgfpathlineto{\pgfqpoint{4.321415in}{6.007627in}}%
\pgfpathlineto{\pgfqpoint{4.336849in}{6.054300in}}%
\pgfpathlineto{\pgfqpoint{4.352283in}{6.089400in}}%
\pgfpathlineto{\pgfqpoint{4.367716in}{6.112347in}}%
\pgfpathlineto{\pgfqpoint{4.383150in}{6.122796in}}%
\pgfpathlineto{\pgfqpoint{4.398583in}{6.120648in}}%
\pgfpathlineto{\pgfqpoint{4.414017in}{6.106049in}}%
\pgfpathlineto{\pgfqpoint{4.429451in}{6.079387in}}%
\pgfpathlineto{\pgfqpoint{4.444884in}{6.041285in}}%
\pgfpathlineto{\pgfqpoint{4.460318in}{5.992589in}}%
\pgfpathlineto{\pgfqpoint{4.475752in}{5.934349in}}%
\pgfpathlineto{\pgfqpoint{4.491185in}{5.867799in}}%
\pgfpathlineto{\pgfqpoint{4.522052in}{5.715479in}}%
\pgfpathlineto{\pgfqpoint{4.614654in}{5.223020in}}%
\pgfpathlineto{\pgfqpoint{4.630088in}{5.153356in}}%
\pgfpathlineto{\pgfqpoint{4.645522in}{5.091298in}}%
\pgfpathlineto{\pgfqpoint{4.660955in}{5.038074in}}%
\pgfpathlineto{\pgfqpoint{4.660955in}{5.038074in}}%
\pgfusepath{stroke}%
\end{pgfscope}%
\begin{pgfscope}%
\pgfpathrectangle{\pgfqpoint{0.625831in}{3.799602in}}{\pgfqpoint{4.227273in}{2.745455in}} %
\pgfusepath{clip}%
\pgfsetrectcap%
\pgfsetroundjoin%
\pgfsetlinewidth{0.501875pt}%
\definecolor{currentstroke}{rgb}{0.264706,0.361242,0.982973}%
\pgfsetstrokecolor{currentstroke}%
\pgfsetdash{}{0pt}%
\pgfpathmoveto{\pgfqpoint{0.817980in}{5.373558in}}%
\pgfpathlineto{\pgfqpoint{0.833414in}{5.414389in}}%
\pgfpathlineto{\pgfqpoint{0.848847in}{5.451405in}}%
\pgfpathlineto{\pgfqpoint{0.864281in}{5.483715in}}%
\pgfpathlineto{\pgfqpoint{0.879715in}{5.510514in}}%
\pgfpathlineto{\pgfqpoint{0.895148in}{5.531105in}}%
\pgfpathlineto{\pgfqpoint{0.910582in}{5.544913in}}%
\pgfpathlineto{\pgfqpoint{0.926015in}{5.551497in}}%
\pgfpathlineto{\pgfqpoint{0.941449in}{5.550567in}}%
\pgfpathlineto{\pgfqpoint{0.956883in}{5.541981in}}%
\pgfpathlineto{\pgfqpoint{0.972316in}{5.525761in}}%
\pgfpathlineto{\pgfqpoint{0.987750in}{5.502086in}}%
\pgfpathlineto{\pgfqpoint{1.003184in}{5.471294in}}%
\pgfpathlineto{\pgfqpoint{1.018617in}{5.433875in}}%
\pgfpathlineto{\pgfqpoint{1.034051in}{5.390465in}}%
\pgfpathlineto{\pgfqpoint{1.064918in}{5.288863in}}%
\pgfpathlineto{\pgfqpoint{1.111219in}{5.114252in}}%
\pgfpathlineto{\pgfqpoint{1.157520in}{4.940187in}}%
\pgfpathlineto{\pgfqpoint{1.188387in}{4.840168in}}%
\pgfpathlineto{\pgfqpoint{1.203821in}{4.798286in}}%
\pgfpathlineto{\pgfqpoint{1.219255in}{4.763117in}}%
\pgfpathlineto{\pgfqpoint{1.234688in}{4.735498in}}%
\pgfpathlineto{\pgfqpoint{1.250122in}{4.716133in}}%
\pgfpathlineto{\pgfqpoint{1.265555in}{4.705583in}}%
\pgfpathlineto{\pgfqpoint{1.280989in}{4.704255in}}%
\pgfpathlineto{\pgfqpoint{1.296423in}{4.712386in}}%
\pgfpathlineto{\pgfqpoint{1.311856in}{4.730045in}}%
\pgfpathlineto{\pgfqpoint{1.327290in}{4.757130in}}%
\pgfpathlineto{\pgfqpoint{1.342724in}{4.793364in}}%
\pgfpathlineto{\pgfqpoint{1.358157in}{4.838303in}}%
\pgfpathlineto{\pgfqpoint{1.373591in}{4.891342in}}%
\pgfpathlineto{\pgfqpoint{1.389025in}{4.951730in}}%
\pgfpathlineto{\pgfqpoint{1.419892in}{5.090871in}}%
\pgfpathlineto{\pgfqpoint{1.450759in}{5.247289in}}%
\pgfpathlineto{\pgfqpoint{1.527927in}{5.648929in}}%
\pgfpathlineto{\pgfqpoint{1.558795in}{5.787429in}}%
\pgfpathlineto{\pgfqpoint{1.574228in}{5.847393in}}%
\pgfpathlineto{\pgfqpoint{1.589662in}{5.899941in}}%
\pgfpathlineto{\pgfqpoint{1.605095in}{5.944319in}}%
\pgfpathlineto{\pgfqpoint{1.620529in}{5.979922in}}%
\pgfpathlineto{\pgfqpoint{1.635963in}{6.006305in}}%
\pgfpathlineto{\pgfqpoint{1.651396in}{6.023187in}}%
\pgfpathlineto{\pgfqpoint{1.666830in}{6.030458in}}%
\pgfpathlineto{\pgfqpoint{1.682264in}{6.028180in}}%
\pgfpathlineto{\pgfqpoint{1.697697in}{6.016582in}}%
\pgfpathlineto{\pgfqpoint{1.713131in}{5.996056in}}%
\pgfpathlineto{\pgfqpoint{1.728564in}{5.967147in}}%
\pgfpathlineto{\pgfqpoint{1.743998in}{5.930542in}}%
\pgfpathlineto{\pgfqpoint{1.759432in}{5.887053in}}%
\pgfpathlineto{\pgfqpoint{1.790299in}{5.783211in}}%
\pgfpathlineto{\pgfqpoint{1.821166in}{5.663966in}}%
\pgfpathlineto{\pgfqpoint{1.882901in}{5.415731in}}%
\pgfpathlineto{\pgfqpoint{1.913768in}{5.304332in}}%
\pgfpathlineto{\pgfqpoint{1.929202in}{5.255159in}}%
\pgfpathlineto{\pgfqpoint{1.944635in}{5.211339in}}%
\pgfpathlineto{\pgfqpoint{1.960069in}{5.173463in}}%
\pgfpathlineto{\pgfqpoint{1.975503in}{5.141977in}}%
\pgfpathlineto{\pgfqpoint{1.990936in}{5.117180in}}%
\pgfpathlineto{\pgfqpoint{2.006370in}{5.099213in}}%
\pgfpathlineto{\pgfqpoint{2.021804in}{5.088062in}}%
\pgfpathlineto{\pgfqpoint{2.037237in}{5.083559in}}%
\pgfpathlineto{\pgfqpoint{2.052671in}{5.085386in}}%
\pgfpathlineto{\pgfqpoint{2.068104in}{5.093086in}}%
\pgfpathlineto{\pgfqpoint{2.083538in}{5.106070in}}%
\pgfpathlineto{\pgfqpoint{2.098972in}{5.123636in}}%
\pgfpathlineto{\pgfqpoint{2.114405in}{5.144980in}}%
\pgfpathlineto{\pgfqpoint{2.145273in}{5.195401in}}%
\pgfpathlineto{\pgfqpoint{2.191574in}{5.275692in}}%
\pgfpathlineto{\pgfqpoint{2.207007in}{5.299717in}}%
\pgfpathlineto{\pgfqpoint{2.222441in}{5.320788in}}%
\pgfpathlineto{\pgfqpoint{2.237874in}{5.338043in}}%
\pgfpathlineto{\pgfqpoint{2.253308in}{5.350710in}}%
\pgfpathlineto{\pgfqpoint{2.268742in}{5.358118in}}%
\pgfpathlineto{\pgfqpoint{2.284175in}{5.359717in}}%
\pgfpathlineto{\pgfqpoint{2.299609in}{5.355091in}}%
\pgfpathlineto{\pgfqpoint{2.315043in}{5.343963in}}%
\pgfpathlineto{\pgfqpoint{2.330476in}{5.326208in}}%
\pgfpathlineto{\pgfqpoint{2.345910in}{5.301854in}}%
\pgfpathlineto{\pgfqpoint{2.361344in}{5.271081in}}%
\pgfpathlineto{\pgfqpoint{2.376777in}{5.234222in}}%
\pgfpathlineto{\pgfqpoint{2.392211in}{5.191756in}}%
\pgfpathlineto{\pgfqpoint{2.423078in}{5.092585in}}%
\pgfpathlineto{\pgfqpoint{2.453945in}{4.979902in}}%
\pgfpathlineto{\pgfqpoint{2.515680in}{4.746150in}}%
\pgfpathlineto{\pgfqpoint{2.546547in}{4.642631in}}%
\pgfpathlineto{\pgfqpoint{2.561981in}{4.597955in}}%
\pgfpathlineto{\pgfqpoint{2.577414in}{4.559277in}}%
\pgfpathlineto{\pgfqpoint{2.592848in}{4.527434in}}%
\pgfpathlineto{\pgfqpoint{2.608282in}{4.503150in}}%
\pgfpathlineto{\pgfqpoint{2.623715in}{4.487016in}}%
\pgfpathlineto{\pgfqpoint{2.639149in}{4.479478in}}%
\pgfpathlineto{\pgfqpoint{2.654583in}{4.480827in}}%
\pgfpathlineto{\pgfqpoint{2.670016in}{4.491190in}}%
\pgfpathlineto{\pgfqpoint{2.685450in}{4.510531in}}%
\pgfpathlineto{\pgfqpoint{2.700884in}{4.538641in}}%
\pgfpathlineto{\pgfqpoint{2.716317in}{4.575151in}}%
\pgfpathlineto{\pgfqpoint{2.731751in}{4.619527in}}%
\pgfpathlineto{\pgfqpoint{2.747184in}{4.671088in}}%
\pgfpathlineto{\pgfqpoint{2.778052in}{4.792356in}}%
\pgfpathlineto{\pgfqpoint{2.808919in}{4.930995in}}%
\pgfpathlineto{\pgfqpoint{2.886087in}{5.290654in}}%
\pgfpathlineto{\pgfqpoint{2.916954in}{5.413639in}}%
\pgfpathlineto{\pgfqpoint{2.932388in}{5.466084in}}%
\pgfpathlineto{\pgfqpoint{2.947822in}{5.511239in}}%
\pgfpathlineto{\pgfqpoint{2.963255in}{5.548317in}}%
\pgfpathlineto{\pgfqpoint{2.978689in}{5.576677in}}%
\pgfpathlineto{\pgfqpoint{2.994123in}{5.595835in}}%
\pgfpathlineto{\pgfqpoint{3.009556in}{5.605473in}}%
\pgfpathlineto{\pgfqpoint{3.024990in}{5.605445in}}%
\pgfpathlineto{\pgfqpoint{3.040424in}{5.595778in}}%
\pgfpathlineto{\pgfqpoint{3.055857in}{5.576673in}}%
\pgfpathlineto{\pgfqpoint{3.071291in}{5.548500in}}%
\pgfpathlineto{\pgfqpoint{3.086724in}{5.511791in}}%
\pgfpathlineto{\pgfqpoint{3.102158in}{5.467230in}}%
\pgfpathlineto{\pgfqpoint{3.117592in}{5.415636in}}%
\pgfpathlineto{\pgfqpoint{3.148459in}{5.295225in}}%
\pgfpathlineto{\pgfqpoint{3.179326in}{5.159220in}}%
\pgfpathlineto{\pgfqpoint{3.241061in}{4.879201in}}%
\pgfpathlineto{\pgfqpoint{3.271928in}{4.754428in}}%
\pgfpathlineto{\pgfqpoint{3.287362in}{4.699636in}}%
\pgfpathlineto{\pgfqpoint{3.302795in}{4.651095in}}%
\pgfpathlineto{\pgfqpoint{3.318229in}{4.609539in}}%
\pgfpathlineto{\pgfqpoint{3.333663in}{4.575558in}}%
\pgfpathlineto{\pgfqpoint{3.349096in}{4.549591in}}%
\pgfpathlineto{\pgfqpoint{3.364530in}{4.531916in}}%
\pgfpathlineto{\pgfqpoint{3.379963in}{4.522650in}}%
\pgfpathlineto{\pgfqpoint{3.395397in}{4.521745in}}%
\pgfpathlineto{\pgfqpoint{3.410831in}{4.528993in}}%
\pgfpathlineto{\pgfqpoint{3.426264in}{4.544031in}}%
\pgfpathlineto{\pgfqpoint{3.441698in}{4.566351in}}%
\pgfpathlineto{\pgfqpoint{3.457132in}{4.595311in}}%
\pgfpathlineto{\pgfqpoint{3.472565in}{4.630151in}}%
\pgfpathlineto{\pgfqpoint{3.503433in}{4.713934in}}%
\pgfpathlineto{\pgfqpoint{3.534300in}{4.809923in}}%
\pgfpathlineto{\pgfqpoint{3.596034in}{5.004863in}}%
\pgfpathlineto{\pgfqpoint{3.626902in}{5.087831in}}%
\pgfpathlineto{\pgfqpoint{3.642335in}{5.122724in}}%
\pgfpathlineto{\pgfqpoint{3.657769in}{5.152368in}}%
\pgfpathlineto{\pgfqpoint{3.673203in}{5.176288in}}%
\pgfpathlineto{\pgfqpoint{3.688636in}{5.194147in}}%
\pgfpathlineto{\pgfqpoint{3.704070in}{5.205757in}}%
\pgfpathlineto{\pgfqpoint{3.719503in}{5.211082in}}%
\pgfpathlineto{\pgfqpoint{3.734937in}{5.210241in}}%
\pgfpathlineto{\pgfqpoint{3.750371in}{5.203497in}}%
\pgfpathlineto{\pgfqpoint{3.765804in}{5.191260in}}%
\pgfpathlineto{\pgfqpoint{3.781238in}{5.174074in}}%
\pgfpathlineto{\pgfqpoint{3.796672in}{5.152602in}}%
\pgfpathlineto{\pgfqpoint{3.827539in}{5.099985in}}%
\pgfpathlineto{\pgfqpoint{3.904707in}{4.956187in}}%
\pgfpathlineto{\pgfqpoint{3.920141in}{4.933357in}}%
\pgfpathlineto{\pgfqpoint{3.935574in}{4.914748in}}%
\pgfpathlineto{\pgfqpoint{3.951008in}{4.901198in}}%
\pgfpathlineto{\pgfqpoint{3.966442in}{4.893451in}}%
\pgfpathlineto{\pgfqpoint{3.981875in}{4.892133in}}%
\pgfpathlineto{\pgfqpoint{3.997309in}{4.897743in}}%
\pgfpathlineto{\pgfqpoint{4.012743in}{4.910639in}}%
\pgfpathlineto{\pgfqpoint{4.028176in}{4.931027in}}%
\pgfpathlineto{\pgfqpoint{4.043610in}{4.958958in}}%
\pgfpathlineto{\pgfqpoint{4.059043in}{4.994323in}}%
\pgfpathlineto{\pgfqpoint{4.074477in}{5.036854in}}%
\pgfpathlineto{\pgfqpoint{4.089911in}{5.086125in}}%
\pgfpathlineto{\pgfqpoint{4.120778in}{5.202455in}}%
\pgfpathlineto{\pgfqpoint{4.151645in}{5.337118in}}%
\pgfpathlineto{\pgfqpoint{4.244247in}{5.766201in}}%
\pgfpathlineto{\pgfqpoint{4.275114in}{5.886175in}}%
\pgfpathlineto{\pgfqpoint{4.290548in}{5.936782in}}%
\pgfpathlineto{\pgfqpoint{4.305982in}{5.979882in}}%
\pgfpathlineto{\pgfqpoint{4.321415in}{6.014684in}}%
\pgfpathlineto{\pgfqpoint{4.336849in}{6.040530in}}%
\pgfpathlineto{\pgfqpoint{4.352283in}{6.056917in}}%
\pgfpathlineto{\pgfqpoint{4.367716in}{6.063500in}}%
\pgfpathlineto{\pgfqpoint{4.383150in}{6.060105in}}%
\pgfpathlineto{\pgfqpoint{4.398583in}{6.046727in}}%
\pgfpathlineto{\pgfqpoint{4.414017in}{6.023540in}}%
\pgfpathlineto{\pgfqpoint{4.429451in}{5.990884in}}%
\pgfpathlineto{\pgfqpoint{4.444884in}{5.949267in}}%
\pgfpathlineto{\pgfqpoint{4.460318in}{5.899352in}}%
\pgfpathlineto{\pgfqpoint{4.475752in}{5.841949in}}%
\pgfpathlineto{\pgfqpoint{4.506619in}{5.708538in}}%
\pgfpathlineto{\pgfqpoint{4.537486in}{5.557766in}}%
\pgfpathlineto{\pgfqpoint{4.614654in}{5.170477in}}%
\pgfpathlineto{\pgfqpoint{4.645522in}{5.037489in}}%
\pgfpathlineto{\pgfqpoint{4.660955in}{4.980078in}}%
\pgfpathlineto{\pgfqpoint{4.660955in}{4.980078in}}%
\pgfusepath{stroke}%
\end{pgfscope}%
\begin{pgfscope}%
\pgfpathrectangle{\pgfqpoint{0.625831in}{3.799602in}}{\pgfqpoint{4.227273in}{2.745455in}} %
\pgfusepath{clip}%
\pgfsetrectcap%
\pgfsetroundjoin%
\pgfsetlinewidth{0.501875pt}%
\definecolor{currentstroke}{rgb}{0.186275,0.473094,0.969797}%
\pgfsetstrokecolor{currentstroke}%
\pgfsetdash{}{0pt}%
\pgfpathmoveto{\pgfqpoint{0.817980in}{5.599554in}}%
\pgfpathlineto{\pgfqpoint{0.833414in}{5.663543in}}%
\pgfpathlineto{\pgfqpoint{0.848847in}{5.714613in}}%
\pgfpathlineto{\pgfqpoint{0.864281in}{5.751341in}}%
\pgfpathlineto{\pgfqpoint{0.879715in}{5.772721in}}%
\pgfpathlineto{\pgfqpoint{0.895148in}{5.778188in}}%
\pgfpathlineto{\pgfqpoint{0.910582in}{5.767632in}}%
\pgfpathlineto{\pgfqpoint{0.926015in}{5.741401in}}%
\pgfpathlineto{\pgfqpoint{0.941449in}{5.700281in}}%
\pgfpathlineto{\pgfqpoint{0.956883in}{5.645474in}}%
\pgfpathlineto{\pgfqpoint{0.972316in}{5.578559in}}%
\pgfpathlineto{\pgfqpoint{0.987750in}{5.501438in}}%
\pgfpathlineto{\pgfqpoint{1.018617in}{5.325466in}}%
\pgfpathlineto{\pgfqpoint{1.080352in}{4.956076in}}%
\pgfpathlineto{\pgfqpoint{1.095785in}{4.874542in}}%
\pgfpathlineto{\pgfqpoint{1.111219in}{4.801766in}}%
\pgfpathlineto{\pgfqpoint{1.126653in}{4.739539in}}%
\pgfpathlineto{\pgfqpoint{1.142086in}{4.689323in}}%
\pgfpathlineto{\pgfqpoint{1.157520in}{4.652222in}}%
\pgfpathlineto{\pgfqpoint{1.172954in}{4.628952in}}%
\pgfpathlineto{\pgfqpoint{1.188387in}{4.619833in}}%
\pgfpathlineto{\pgfqpoint{1.203821in}{4.624787in}}%
\pgfpathlineto{\pgfqpoint{1.219255in}{4.643349in}}%
\pgfpathlineto{\pgfqpoint{1.234688in}{4.674694in}}%
\pgfpathlineto{\pgfqpoint{1.250122in}{4.717670in}}%
\pgfpathlineto{\pgfqpoint{1.265555in}{4.770840in}}%
\pgfpathlineto{\pgfqpoint{1.280989in}{4.832533in}}%
\pgfpathlineto{\pgfqpoint{1.311856in}{4.973992in}}%
\pgfpathlineto{\pgfqpoint{1.373591in}{5.273830in}}%
\pgfpathlineto{\pgfqpoint{1.404458in}{5.403075in}}%
\pgfpathlineto{\pgfqpoint{1.419892in}{5.457540in}}%
\pgfpathlineto{\pgfqpoint{1.435325in}{5.504018in}}%
\pgfpathlineto{\pgfqpoint{1.450759in}{5.541969in}}%
\pgfpathlineto{\pgfqpoint{1.466193in}{5.571164in}}%
\pgfpathlineto{\pgfqpoint{1.481626in}{5.591674in}}%
\pgfpathlineto{\pgfqpoint{1.497060in}{5.603859in}}%
\pgfpathlineto{\pgfqpoint{1.512494in}{5.608337in}}%
\pgfpathlineto{\pgfqpoint{1.527927in}{5.605957in}}%
\pgfpathlineto{\pgfqpoint{1.543361in}{5.597756in}}%
\pgfpathlineto{\pgfqpoint{1.558795in}{5.584911in}}%
\pgfpathlineto{\pgfqpoint{1.589662in}{5.550408in}}%
\pgfpathlineto{\pgfqpoint{1.620529in}{5.512786in}}%
\pgfpathlineto{\pgfqpoint{1.635963in}{5.495811in}}%
\pgfpathlineto{\pgfqpoint{1.651396in}{5.481415in}}%
\pgfpathlineto{\pgfqpoint{1.666830in}{5.470389in}}%
\pgfpathlineto{\pgfqpoint{1.682264in}{5.463312in}}%
\pgfpathlineto{\pgfqpoint{1.697697in}{5.460529in}}%
\pgfpathlineto{\pgfqpoint{1.713131in}{5.462144in}}%
\pgfpathlineto{\pgfqpoint{1.728564in}{5.468014in}}%
\pgfpathlineto{\pgfqpoint{1.743998in}{5.477765in}}%
\pgfpathlineto{\pgfqpoint{1.759432in}{5.490805in}}%
\pgfpathlineto{\pgfqpoint{1.790299in}{5.523463in}}%
\pgfpathlineto{\pgfqpoint{1.821166in}{5.558078in}}%
\pgfpathlineto{\pgfqpoint{1.836600in}{5.573289in}}%
\pgfpathlineto{\pgfqpoint{1.852034in}{5.585577in}}%
\pgfpathlineto{\pgfqpoint{1.867467in}{5.593873in}}%
\pgfpathlineto{\pgfqpoint{1.882901in}{5.597221in}}%
\pgfpathlineto{\pgfqpoint{1.898334in}{5.594819in}}%
\pgfpathlineto{\pgfqpoint{1.913768in}{5.586058in}}%
\pgfpathlineto{\pgfqpoint{1.929202in}{5.570548in}}%
\pgfpathlineto{\pgfqpoint{1.944635in}{5.548140in}}%
\pgfpathlineto{\pgfqpoint{1.960069in}{5.518940in}}%
\pgfpathlineto{\pgfqpoint{1.975503in}{5.483309in}}%
\pgfpathlineto{\pgfqpoint{1.990936in}{5.441856in}}%
\pgfpathlineto{\pgfqpoint{2.021804in}{5.345079in}}%
\pgfpathlineto{\pgfqpoint{2.098972in}{5.081357in}}%
\pgfpathlineto{\pgfqpoint{2.114405in}{5.036413in}}%
\pgfpathlineto{\pgfqpoint{2.129839in}{4.997314in}}%
\pgfpathlineto{\pgfqpoint{2.145273in}{4.965200in}}%
\pgfpathlineto{\pgfqpoint{2.160706in}{4.941005in}}%
\pgfpathlineto{\pgfqpoint{2.176140in}{4.925412in}}%
\pgfpathlineto{\pgfqpoint{2.191574in}{4.918831in}}%
\pgfpathlineto{\pgfqpoint{2.207007in}{4.921373in}}%
\pgfpathlineto{\pgfqpoint{2.222441in}{4.932841in}}%
\pgfpathlineto{\pgfqpoint{2.237874in}{4.952730in}}%
\pgfpathlineto{\pgfqpoint{2.253308in}{4.980234in}}%
\pgfpathlineto{\pgfqpoint{2.268742in}{5.014269in}}%
\pgfpathlineto{\pgfqpoint{2.299609in}{5.096396in}}%
\pgfpathlineto{\pgfqpoint{2.345910in}{5.229485in}}%
\pgfpathlineto{\pgfqpoint{2.361344in}{5.269171in}}%
\pgfpathlineto{\pgfqpoint{2.376777in}{5.303503in}}%
\pgfpathlineto{\pgfqpoint{2.392211in}{5.330837in}}%
\pgfpathlineto{\pgfqpoint{2.407644in}{5.349723in}}%
\pgfpathlineto{\pgfqpoint{2.423078in}{5.358955in}}%
\pgfpathlineto{\pgfqpoint{2.438512in}{5.357620in}}%
\pgfpathlineto{\pgfqpoint{2.453945in}{5.345134in}}%
\pgfpathlineto{\pgfqpoint{2.469379in}{5.321272in}}%
\pgfpathlineto{\pgfqpoint{2.484813in}{5.286179in}}%
\pgfpathlineto{\pgfqpoint{2.500246in}{5.240384in}}%
\pgfpathlineto{\pgfqpoint{2.515680in}{5.184784in}}%
\pgfpathlineto{\pgfqpoint{2.531114in}{5.120631in}}%
\pgfpathlineto{\pgfqpoint{2.561981in}{4.973240in}}%
\pgfpathlineto{\pgfqpoint{2.623715in}{4.660784in}}%
\pgfpathlineto{\pgfqpoint{2.639149in}{4.592545in}}%
\pgfpathlineto{\pgfqpoint{2.654583in}{4.532860in}}%
\pgfpathlineto{\pgfqpoint{2.670016in}{4.483788in}}%
\pgfpathlineto{\pgfqpoint{2.685450in}{4.447136in}}%
\pgfpathlineto{\pgfqpoint{2.700884in}{4.424402in}}%
\pgfpathlineto{\pgfqpoint{2.716317in}{4.416723in}}%
\pgfpathlineto{\pgfqpoint{2.731751in}{4.424833in}}%
\pgfpathlineto{\pgfqpoint{2.747184in}{4.449034in}}%
\pgfpathlineto{\pgfqpoint{2.762618in}{4.489178in}}%
\pgfpathlineto{\pgfqpoint{2.778052in}{4.544661in}}%
\pgfpathlineto{\pgfqpoint{2.793485in}{4.614437in}}%
\pgfpathlineto{\pgfqpoint{2.808919in}{4.697035in}}%
\pgfpathlineto{\pgfqpoint{2.839786in}{4.892937in}}%
\pgfpathlineto{\pgfqpoint{2.932388in}{5.541806in}}%
\pgfpathlineto{\pgfqpoint{2.947822in}{5.629630in}}%
\pgfpathlineto{\pgfqpoint{2.963255in}{5.704779in}}%
\pgfpathlineto{\pgfqpoint{2.978689in}{5.765245in}}%
\pgfpathlineto{\pgfqpoint{2.994123in}{5.809397in}}%
\pgfpathlineto{\pgfqpoint{3.009556in}{5.836035in}}%
\pgfpathlineto{\pgfqpoint{3.024990in}{5.844422in}}%
\pgfpathlineto{\pgfqpoint{3.040424in}{5.834306in}}%
\pgfpathlineto{\pgfqpoint{3.055857in}{5.805931in}}%
\pgfpathlineto{\pgfqpoint{3.071291in}{5.760029in}}%
\pgfpathlineto{\pgfqpoint{3.086724in}{5.697798in}}%
\pgfpathlineto{\pgfqpoint{3.102158in}{5.620872in}}%
\pgfpathlineto{\pgfqpoint{3.117592in}{5.531273in}}%
\pgfpathlineto{\pgfqpoint{3.148459in}{5.323723in}}%
\pgfpathlineto{\pgfqpoint{3.225627in}{4.770238in}}%
\pgfpathlineto{\pgfqpoint{3.241061in}{4.676221in}}%
\pgfpathlineto{\pgfqpoint{3.256494in}{4.593580in}}%
\pgfpathlineto{\pgfqpoint{3.271928in}{4.524226in}}%
\pgfpathlineto{\pgfqpoint{3.287362in}{4.469673in}}%
\pgfpathlineto{\pgfqpoint{3.302795in}{4.431003in}}%
\pgfpathlineto{\pgfqpoint{3.318229in}{4.408839in}}%
\pgfpathlineto{\pgfqpoint{3.333663in}{4.403334in}}%
\pgfpathlineto{\pgfqpoint{3.349096in}{4.414170in}}%
\pgfpathlineto{\pgfqpoint{3.364530in}{4.440578in}}%
\pgfpathlineto{\pgfqpoint{3.379963in}{4.481362in}}%
\pgfpathlineto{\pgfqpoint{3.395397in}{4.534946in}}%
\pgfpathlineto{\pgfqpoint{3.410831in}{4.599419in}}%
\pgfpathlineto{\pgfqpoint{3.441698in}{4.752116in}}%
\pgfpathlineto{\pgfqpoint{3.503433in}{5.082949in}}%
\pgfpathlineto{\pgfqpoint{3.518866in}{5.156571in}}%
\pgfpathlineto{\pgfqpoint{3.534300in}{5.222188in}}%
\pgfpathlineto{\pgfqpoint{3.549733in}{5.278067in}}%
\pgfpathlineto{\pgfqpoint{3.565167in}{5.322813in}}%
\pgfpathlineto{\pgfqpoint{3.580601in}{5.355409in}}%
\pgfpathlineto{\pgfqpoint{3.596034in}{5.375238in}}%
\pgfpathlineto{\pgfqpoint{3.611468in}{5.382094in}}%
\pgfpathlineto{\pgfqpoint{3.626902in}{5.376184in}}%
\pgfpathlineto{\pgfqpoint{3.642335in}{5.358107in}}%
\pgfpathlineto{\pgfqpoint{3.657769in}{5.328837in}}%
\pgfpathlineto{\pgfqpoint{3.673203in}{5.289680in}}%
\pgfpathlineto{\pgfqpoint{3.688636in}{5.242227in}}%
\pgfpathlineto{\pgfqpoint{3.719503in}{5.129910in}}%
\pgfpathlineto{\pgfqpoint{3.765804in}{4.949048in}}%
\pgfpathlineto{\pgfqpoint{3.781238in}{4.893849in}}%
\pgfpathlineto{\pgfqpoint{3.796672in}{4.844423in}}%
\pgfpathlineto{\pgfqpoint{3.812105in}{4.802426in}}%
\pgfpathlineto{\pgfqpoint{3.827539in}{4.769257in}}%
\pgfpathlineto{\pgfqpoint{3.842973in}{4.746019in}}%
\pgfpathlineto{\pgfqpoint{3.858406in}{4.733493in}}%
\pgfpathlineto{\pgfqpoint{3.873840in}{4.732118in}}%
\pgfpathlineto{\pgfqpoint{3.889273in}{4.741985in}}%
\pgfpathlineto{\pgfqpoint{3.904707in}{4.762846in}}%
\pgfpathlineto{\pgfqpoint{3.920141in}{4.794124in}}%
\pgfpathlineto{\pgfqpoint{3.935574in}{4.834944in}}%
\pgfpathlineto{\pgfqpoint{3.951008in}{4.884164in}}%
\pgfpathlineto{\pgfqpoint{3.981875in}{5.002184in}}%
\pgfpathlineto{\pgfqpoint{4.074477in}{5.394523in}}%
\pgfpathlineto{\pgfqpoint{4.089911in}{5.448904in}}%
\pgfpathlineto{\pgfqpoint{4.105344in}{5.496832in}}%
\pgfpathlineto{\pgfqpoint{4.120778in}{5.537565in}}%
\pgfpathlineto{\pgfqpoint{4.136212in}{5.570633in}}%
\pgfpathlineto{\pgfqpoint{4.151645in}{5.595840in}}%
\pgfpathlineto{\pgfqpoint{4.167079in}{5.613260in}}%
\pgfpathlineto{\pgfqpoint{4.182513in}{5.623229in}}%
\pgfpathlineto{\pgfqpoint{4.197946in}{5.626317in}}%
\pgfpathlineto{\pgfqpoint{4.213380in}{5.623302in}}%
\pgfpathlineto{\pgfqpoint{4.228813in}{5.615129in}}%
\pgfpathlineto{\pgfqpoint{4.244247in}{5.602873in}}%
\pgfpathlineto{\pgfqpoint{4.275114in}{5.570760in}}%
\pgfpathlineto{\pgfqpoint{4.305982in}{5.536292in}}%
\pgfpathlineto{\pgfqpoint{4.321415in}{5.520856in}}%
\pgfpathlineto{\pgfqpoint{4.336849in}{5.507802in}}%
\pgfpathlineto{\pgfqpoint{4.352283in}{5.497799in}}%
\pgfpathlineto{\pgfqpoint{4.367716in}{5.491310in}}%
\pgfpathlineto{\pgfqpoint{4.383150in}{5.488572in}}%
\pgfpathlineto{\pgfqpoint{4.398583in}{5.489589in}}%
\pgfpathlineto{\pgfqpoint{4.414017in}{5.494133in}}%
\pgfpathlineto{\pgfqpoint{4.429451in}{5.501752in}}%
\pgfpathlineto{\pgfqpoint{4.460318in}{5.523407in}}%
\pgfpathlineto{\pgfqpoint{4.491185in}{5.547373in}}%
\pgfpathlineto{\pgfqpoint{4.506619in}{5.557497in}}%
\pgfpathlineto{\pgfqpoint{4.522052in}{5.564847in}}%
\pgfpathlineto{\pgfqpoint{4.537486in}{5.568310in}}%
\pgfpathlineto{\pgfqpoint{4.552920in}{5.566856in}}%
\pgfpathlineto{\pgfqpoint{4.568353in}{5.559589in}}%
\pgfpathlineto{\pgfqpoint{4.583787in}{5.545788in}}%
\pgfpathlineto{\pgfqpoint{4.599221in}{5.524945in}}%
\pgfpathlineto{\pgfqpoint{4.614654in}{5.496789in}}%
\pgfpathlineto{\pgfqpoint{4.630088in}{5.461315in}}%
\pgfpathlineto{\pgfqpoint{4.645522in}{5.418793in}}%
\pgfpathlineto{\pgfqpoint{4.660955in}{5.369767in}}%
\pgfpathlineto{\pgfqpoint{4.660955in}{5.369767in}}%
\pgfusepath{stroke}%
\end{pgfscope}%
\begin{pgfscope}%
\pgfpathrectangle{\pgfqpoint{0.625831in}{3.799602in}}{\pgfqpoint{4.227273in}{2.745455in}} %
\pgfusepath{clip}%
\pgfsetrectcap%
\pgfsetroundjoin%
\pgfsetlinewidth{0.501875pt}%
\definecolor{currentstroke}{rgb}{0.100000,0.587785,0.951057}%
\pgfsetstrokecolor{currentstroke}%
\pgfsetdash{}{0pt}%
\pgfpathmoveto{\pgfqpoint{0.817980in}{5.555060in}}%
\pgfpathlineto{\pgfqpoint{0.833414in}{5.608057in}}%
\pgfpathlineto{\pgfqpoint{0.848847in}{5.650642in}}%
\pgfpathlineto{\pgfqpoint{0.864281in}{5.681495in}}%
\pgfpathlineto{\pgfqpoint{0.879715in}{5.699618in}}%
\pgfpathlineto{\pgfqpoint{0.895148in}{5.704361in}}%
\pgfpathlineto{\pgfqpoint{0.910582in}{5.695442in}}%
\pgfpathlineto{\pgfqpoint{0.926015in}{5.672946in}}%
\pgfpathlineto{\pgfqpoint{0.941449in}{5.637326in}}%
\pgfpathlineto{\pgfqpoint{0.956883in}{5.589387in}}%
\pgfpathlineto{\pgfqpoint{0.972316in}{5.530256in}}%
\pgfpathlineto{\pgfqpoint{0.987750in}{5.461357in}}%
\pgfpathlineto{\pgfqpoint{1.018617in}{5.301144in}}%
\pgfpathlineto{\pgfqpoint{1.095785in}{4.864707in}}%
\pgfpathlineto{\pgfqpoint{1.111219in}{4.787960in}}%
\pgfpathlineto{\pgfqpoint{1.126653in}{4.719138in}}%
\pgfpathlineto{\pgfqpoint{1.142086in}{4.659756in}}%
\pgfpathlineto{\pgfqpoint{1.157520in}{4.611071in}}%
\pgfpathlineto{\pgfqpoint{1.172954in}{4.574050in}}%
\pgfpathlineto{\pgfqpoint{1.188387in}{4.549360in}}%
\pgfpathlineto{\pgfqpoint{1.203821in}{4.537346in}}%
\pgfpathlineto{\pgfqpoint{1.219255in}{4.538041in}}%
\pgfpathlineto{\pgfqpoint{1.234688in}{4.551169in}}%
\pgfpathlineto{\pgfqpoint{1.250122in}{4.576161in}}%
\pgfpathlineto{\pgfqpoint{1.265555in}{4.612181in}}%
\pgfpathlineto{\pgfqpoint{1.280989in}{4.658159in}}%
\pgfpathlineto{\pgfqpoint{1.296423in}{4.712823in}}%
\pgfpathlineto{\pgfqpoint{1.327290in}{4.842379in}}%
\pgfpathlineto{\pgfqpoint{1.373591in}{5.063408in}}%
\pgfpathlineto{\pgfqpoint{1.404458in}{5.210101in}}%
\pgfpathlineto{\pgfqpoint{1.435325in}{5.343337in}}%
\pgfpathlineto{\pgfqpoint{1.450759in}{5.402217in}}%
\pgfpathlineto{\pgfqpoint{1.466193in}{5.454895in}}%
\pgfpathlineto{\pgfqpoint{1.481626in}{5.500854in}}%
\pgfpathlineto{\pgfqpoint{1.497060in}{5.539791in}}%
\pgfpathlineto{\pgfqpoint{1.512494in}{5.571614in}}%
\pgfpathlineto{\pgfqpoint{1.527927in}{5.596426in}}%
\pgfpathlineto{\pgfqpoint{1.543361in}{5.614511in}}%
\pgfpathlineto{\pgfqpoint{1.558795in}{5.626312in}}%
\pgfpathlineto{\pgfqpoint{1.574228in}{5.632398in}}%
\pgfpathlineto{\pgfqpoint{1.589662in}{5.633443in}}%
\pgfpathlineto{\pgfqpoint{1.605095in}{5.630189in}}%
\pgfpathlineto{\pgfqpoint{1.620529in}{5.623417in}}%
\pgfpathlineto{\pgfqpoint{1.635963in}{5.613913in}}%
\pgfpathlineto{\pgfqpoint{1.666830in}{5.589718in}}%
\pgfpathlineto{\pgfqpoint{1.728564in}{5.537622in}}%
\pgfpathlineto{\pgfqpoint{1.759432in}{5.515751in}}%
\pgfpathlineto{\pgfqpoint{1.790299in}{5.497588in}}%
\pgfpathlineto{\pgfqpoint{1.867467in}{5.456747in}}%
\pgfpathlineto{\pgfqpoint{1.898334in}{5.435101in}}%
\pgfpathlineto{\pgfqpoint{1.929202in}{5.406784in}}%
\pgfpathlineto{\pgfqpoint{1.960069in}{5.370768in}}%
\pgfpathlineto{\pgfqpoint{1.990936in}{5.327542in}}%
\pgfpathlineto{\pgfqpoint{2.037237in}{5.254064in}}%
\pgfpathlineto{\pgfqpoint{2.083538in}{5.181335in}}%
\pgfpathlineto{\pgfqpoint{2.114405in}{5.140468in}}%
\pgfpathlineto{\pgfqpoint{2.129839in}{5.123813in}}%
\pgfpathlineto{\pgfqpoint{2.145273in}{5.110190in}}%
\pgfpathlineto{\pgfqpoint{2.160706in}{5.099856in}}%
\pgfpathlineto{\pgfqpoint{2.176140in}{5.092956in}}%
\pgfpathlineto{\pgfqpoint{2.191574in}{5.089499in}}%
\pgfpathlineto{\pgfqpoint{2.207007in}{5.089361in}}%
\pgfpathlineto{\pgfqpoint{2.222441in}{5.092279in}}%
\pgfpathlineto{\pgfqpoint{2.237874in}{5.097853in}}%
\pgfpathlineto{\pgfqpoint{2.268742in}{5.114757in}}%
\pgfpathlineto{\pgfqpoint{2.315043in}{5.143704in}}%
\pgfpathlineto{\pgfqpoint{2.330476in}{5.151029in}}%
\pgfpathlineto{\pgfqpoint{2.345910in}{5.155792in}}%
\pgfpathlineto{\pgfqpoint{2.361344in}{5.157211in}}%
\pgfpathlineto{\pgfqpoint{2.376777in}{5.154587in}}%
\pgfpathlineto{\pgfqpoint{2.392211in}{5.147336in}}%
\pgfpathlineto{\pgfqpoint{2.407644in}{5.135016in}}%
\pgfpathlineto{\pgfqpoint{2.423078in}{5.117350in}}%
\pgfpathlineto{\pgfqpoint{2.438512in}{5.094245in}}%
\pgfpathlineto{\pgfqpoint{2.453945in}{5.065808in}}%
\pgfpathlineto{\pgfqpoint{2.469379in}{5.032350in}}%
\pgfpathlineto{\pgfqpoint{2.500246in}{4.952642in}}%
\pgfpathlineto{\pgfqpoint{2.546547in}{4.814548in}}%
\pgfpathlineto{\pgfqpoint{2.577414in}{4.724056in}}%
\pgfpathlineto{\pgfqpoint{2.592848in}{4.683416in}}%
\pgfpathlineto{\pgfqpoint{2.608282in}{4.647736in}}%
\pgfpathlineto{\pgfqpoint{2.623715in}{4.618376in}}%
\pgfpathlineto{\pgfqpoint{2.639149in}{4.596595in}}%
\pgfpathlineto{\pgfqpoint{2.654583in}{4.583512in}}%
\pgfpathlineto{\pgfqpoint{2.670016in}{4.580063in}}%
\pgfpathlineto{\pgfqpoint{2.685450in}{4.586966in}}%
\pgfpathlineto{\pgfqpoint{2.700884in}{4.604693in}}%
\pgfpathlineto{\pgfqpoint{2.716317in}{4.633443in}}%
\pgfpathlineto{\pgfqpoint{2.731751in}{4.673130in}}%
\pgfpathlineto{\pgfqpoint{2.747184in}{4.723369in}}%
\pgfpathlineto{\pgfqpoint{2.762618in}{4.783482in}}%
\pgfpathlineto{\pgfqpoint{2.778052in}{4.852505in}}%
\pgfpathlineto{\pgfqpoint{2.808919in}{5.012108in}}%
\pgfpathlineto{\pgfqpoint{2.901521in}{5.536410in}}%
\pgfpathlineto{\pgfqpoint{2.916954in}{5.609626in}}%
\pgfpathlineto{\pgfqpoint{2.932388in}{5.673699in}}%
\pgfpathlineto{\pgfqpoint{2.947822in}{5.727047in}}%
\pgfpathlineto{\pgfqpoint{2.963255in}{5.768334in}}%
\pgfpathlineto{\pgfqpoint{2.978689in}{5.796505in}}%
\pgfpathlineto{\pgfqpoint{2.994123in}{5.810822in}}%
\pgfpathlineto{\pgfqpoint{3.009556in}{5.810885in}}%
\pgfpathlineto{\pgfqpoint{3.024990in}{5.796648in}}%
\pgfpathlineto{\pgfqpoint{3.040424in}{5.768422in}}%
\pgfpathlineto{\pgfqpoint{3.055857in}{5.726871in}}%
\pgfpathlineto{\pgfqpoint{3.071291in}{5.672996in}}%
\pgfpathlineto{\pgfqpoint{3.086724in}{5.608106in}}%
\pgfpathlineto{\pgfqpoint{3.102158in}{5.533793in}}%
\pgfpathlineto{\pgfqpoint{3.133025in}{5.364384in}}%
\pgfpathlineto{\pgfqpoint{3.194760in}{5.002185in}}%
\pgfpathlineto{\pgfqpoint{3.225627in}{4.844042in}}%
\pgfpathlineto{\pgfqpoint{3.241061in}{4.777469in}}%
\pgfpathlineto{\pgfqpoint{3.256494in}{4.721238in}}%
\pgfpathlineto{\pgfqpoint{3.271928in}{4.676489in}}%
\pgfpathlineto{\pgfqpoint{3.287362in}{4.644035in}}%
\pgfpathlineto{\pgfqpoint{3.302795in}{4.624339in}}%
\pgfpathlineto{\pgfqpoint{3.318229in}{4.617503in}}%
\pgfpathlineto{\pgfqpoint{3.333663in}{4.623270in}}%
\pgfpathlineto{\pgfqpoint{3.349096in}{4.641034in}}%
\pgfpathlineto{\pgfqpoint{3.364530in}{4.669859in}}%
\pgfpathlineto{\pgfqpoint{3.379963in}{4.708509in}}%
\pgfpathlineto{\pgfqpoint{3.395397in}{4.755490in}}%
\pgfpathlineto{\pgfqpoint{3.426264in}{4.867436in}}%
\pgfpathlineto{\pgfqpoint{3.487999in}{5.108296in}}%
\pgfpathlineto{\pgfqpoint{3.503433in}{5.160589in}}%
\pgfpathlineto{\pgfqpoint{3.518866in}{5.206202in}}%
\pgfpathlineto{\pgfqpoint{3.534300in}{5.243725in}}%
\pgfpathlineto{\pgfqpoint{3.549733in}{5.272017in}}%
\pgfpathlineto{\pgfqpoint{3.565167in}{5.290237in}}%
\pgfpathlineto{\pgfqpoint{3.580601in}{5.297866in}}%
\pgfpathlineto{\pgfqpoint{3.596034in}{5.294722in}}%
\pgfpathlineto{\pgfqpoint{3.611468in}{5.280958in}}%
\pgfpathlineto{\pgfqpoint{3.626902in}{5.257063in}}%
\pgfpathlineto{\pgfqpoint{3.642335in}{5.223836in}}%
\pgfpathlineto{\pgfqpoint{3.657769in}{5.182368in}}%
\pgfpathlineto{\pgfqpoint{3.673203in}{5.134004in}}%
\pgfpathlineto{\pgfqpoint{3.704070in}{5.022982in}}%
\pgfpathlineto{\pgfqpoint{3.750371in}{4.847920in}}%
\pgfpathlineto{\pgfqpoint{3.765804in}{4.794797in}}%
\pgfpathlineto{\pgfqpoint{3.781238in}{4.747263in}}%
\pgfpathlineto{\pgfqpoint{3.796672in}{4.706896in}}%
\pgfpathlineto{\pgfqpoint{3.812105in}{4.675072in}}%
\pgfpathlineto{\pgfqpoint{3.827539in}{4.652929in}}%
\pgfpathlineto{\pgfqpoint{3.842973in}{4.641331in}}%
\pgfpathlineto{\pgfqpoint{3.858406in}{4.640850in}}%
\pgfpathlineto{\pgfqpoint{3.873840in}{4.651750in}}%
\pgfpathlineto{\pgfqpoint{3.889273in}{4.673983in}}%
\pgfpathlineto{\pgfqpoint{3.904707in}{4.707196in}}%
\pgfpathlineto{\pgfqpoint{3.920141in}{4.750741in}}%
\pgfpathlineto{\pgfqpoint{3.935574in}{4.803702in}}%
\pgfpathlineto{\pgfqpoint{3.951008in}{4.864919in}}%
\pgfpathlineto{\pgfqpoint{3.981875in}{5.006510in}}%
\pgfpathlineto{\pgfqpoint{4.074477in}{5.466228in}}%
\pgfpathlineto{\pgfqpoint{4.089911in}{5.530942in}}%
\pgfpathlineto{\pgfqpoint{4.105344in}{5.588662in}}%
\pgfpathlineto{\pgfqpoint{4.120778in}{5.638469in}}%
\pgfpathlineto{\pgfqpoint{4.136212in}{5.679684in}}%
\pgfpathlineto{\pgfqpoint{4.151645in}{5.711882in}}%
\pgfpathlineto{\pgfqpoint{4.167079in}{5.734892in}}%
\pgfpathlineto{\pgfqpoint{4.182513in}{5.748794in}}%
\pgfpathlineto{\pgfqpoint{4.197946in}{5.753908in}}%
\pgfpathlineto{\pgfqpoint{4.213380in}{5.750772in}}%
\pgfpathlineto{\pgfqpoint{4.228813in}{5.740119in}}%
\pgfpathlineto{\pgfqpoint{4.244247in}{5.722842in}}%
\pgfpathlineto{\pgfqpoint{4.259681in}{5.699964in}}%
\pgfpathlineto{\pgfqpoint{4.275114in}{5.672597in}}%
\pgfpathlineto{\pgfqpoint{4.305982in}{5.609047in}}%
\pgfpathlineto{\pgfqpoint{4.352283in}{5.508552in}}%
\pgfpathlineto{\pgfqpoint{4.383150in}{5.449243in}}%
\pgfpathlineto{\pgfqpoint{4.398583in}{5.423940in}}%
\pgfpathlineto{\pgfqpoint{4.414017in}{5.402050in}}%
\pgfpathlineto{\pgfqpoint{4.429451in}{5.383723in}}%
\pgfpathlineto{\pgfqpoint{4.444884in}{5.368931in}}%
\pgfpathlineto{\pgfqpoint{4.460318in}{5.357471in}}%
\pgfpathlineto{\pgfqpoint{4.475752in}{5.348984in}}%
\pgfpathlineto{\pgfqpoint{4.491185in}{5.342972in}}%
\pgfpathlineto{\pgfqpoint{4.522052in}{5.335870in}}%
\pgfpathlineto{\pgfqpoint{4.552920in}{5.330529in}}%
\pgfpathlineto{\pgfqpoint{4.568353in}{5.326660in}}%
\pgfpathlineto{\pgfqpoint{4.583787in}{5.321060in}}%
\pgfpathlineto{\pgfqpoint{4.599221in}{5.313123in}}%
\pgfpathlineto{\pgfqpoint{4.614654in}{5.302350in}}%
\pgfpathlineto{\pgfqpoint{4.630088in}{5.288370in}}%
\pgfpathlineto{\pgfqpoint{4.645522in}{5.270962in}}%
\pgfpathlineto{\pgfqpoint{4.660955in}{5.250060in}}%
\pgfpathlineto{\pgfqpoint{4.660955in}{5.250060in}}%
\pgfusepath{stroke}%
\end{pgfscope}%
\begin{pgfscope}%
\pgfpathrectangle{\pgfqpoint{0.625831in}{3.799602in}}{\pgfqpoint{4.227273in}{2.745455in}} %
\pgfusepath{clip}%
\pgfsetrectcap%
\pgfsetroundjoin%
\pgfsetlinewidth{0.501875pt}%
\definecolor{currentstroke}{rgb}{0.021569,0.682749,0.930229}%
\pgfsetstrokecolor{currentstroke}%
\pgfsetdash{}{0pt}%
\pgfpathmoveto{\pgfqpoint{0.817980in}{5.471221in}}%
\pgfpathlineto{\pgfqpoint{0.833414in}{5.497724in}}%
\pgfpathlineto{\pgfqpoint{0.848847in}{5.517894in}}%
\pgfpathlineto{\pgfqpoint{0.864281in}{5.530971in}}%
\pgfpathlineto{\pgfqpoint{0.879715in}{5.536390in}}%
\pgfpathlineto{\pgfqpoint{0.895148in}{5.533794in}}%
\pgfpathlineto{\pgfqpoint{0.910582in}{5.523046in}}%
\pgfpathlineto{\pgfqpoint{0.926015in}{5.504230in}}%
\pgfpathlineto{\pgfqpoint{0.941449in}{5.477650in}}%
\pgfpathlineto{\pgfqpoint{0.956883in}{5.443820in}}%
\pgfpathlineto{\pgfqpoint{0.972316in}{5.403448in}}%
\pgfpathlineto{\pgfqpoint{1.003184in}{5.306757in}}%
\pgfpathlineto{\pgfqpoint{1.049485in}{5.138939in}}%
\pgfpathlineto{\pgfqpoint{1.080352in}{5.026737in}}%
\pgfpathlineto{\pgfqpoint{1.111219in}{4.926140in}}%
\pgfpathlineto{\pgfqpoint{1.126653in}{4.883031in}}%
\pgfpathlineto{\pgfqpoint{1.142086in}{4.845987in}}%
\pgfpathlineto{\pgfqpoint{1.157520in}{4.815769in}}%
\pgfpathlineto{\pgfqpoint{1.172954in}{4.792962in}}%
\pgfpathlineto{\pgfqpoint{1.188387in}{4.777969in}}%
\pgfpathlineto{\pgfqpoint{1.203821in}{4.770996in}}%
\pgfpathlineto{\pgfqpoint{1.219255in}{4.772060in}}%
\pgfpathlineto{\pgfqpoint{1.234688in}{4.780990in}}%
\pgfpathlineto{\pgfqpoint{1.250122in}{4.797436in}}%
\pgfpathlineto{\pgfqpoint{1.265555in}{4.820886in}}%
\pgfpathlineto{\pgfqpoint{1.280989in}{4.850684in}}%
\pgfpathlineto{\pgfqpoint{1.296423in}{4.886051in}}%
\pgfpathlineto{\pgfqpoint{1.327290in}{4.969920in}}%
\pgfpathlineto{\pgfqpoint{1.373591in}{5.113895in}}%
\pgfpathlineto{\pgfqpoint{1.419892in}{5.256538in}}%
\pgfpathlineto{\pgfqpoint{1.450759in}{5.340007in}}%
\pgfpathlineto{\pgfqpoint{1.466193in}{5.376479in}}%
\pgfpathlineto{\pgfqpoint{1.481626in}{5.408938in}}%
\pgfpathlineto{\pgfqpoint{1.497060in}{5.437168in}}%
\pgfpathlineto{\pgfqpoint{1.512494in}{5.461077in}}%
\pgfpathlineto{\pgfqpoint{1.527927in}{5.480691in}}%
\pgfpathlineto{\pgfqpoint{1.543361in}{5.496146in}}%
\pgfpathlineto{\pgfqpoint{1.558795in}{5.507671in}}%
\pgfpathlineto{\pgfqpoint{1.574228in}{5.515573in}}%
\pgfpathlineto{\pgfqpoint{1.589662in}{5.520222in}}%
\pgfpathlineto{\pgfqpoint{1.605095in}{5.522030in}}%
\pgfpathlineto{\pgfqpoint{1.620529in}{5.521434in}}%
\pgfpathlineto{\pgfqpoint{1.651396in}{5.514783in}}%
\pgfpathlineto{\pgfqpoint{1.682264in}{5.503559in}}%
\pgfpathlineto{\pgfqpoint{1.805733in}{5.450507in}}%
\pgfpathlineto{\pgfqpoint{1.836600in}{5.435772in}}%
\pgfpathlineto{\pgfqpoint{1.867467in}{5.417950in}}%
\pgfpathlineto{\pgfqpoint{1.898334in}{5.395384in}}%
\pgfpathlineto{\pgfqpoint{1.929202in}{5.366838in}}%
\pgfpathlineto{\pgfqpoint{1.960069in}{5.331830in}}%
\pgfpathlineto{\pgfqpoint{1.990936in}{5.290827in}}%
\pgfpathlineto{\pgfqpoint{2.037237in}{5.221469in}}%
\pgfpathlineto{\pgfqpoint{2.098972in}{5.127545in}}%
\pgfpathlineto{\pgfqpoint{2.129839in}{5.086337in}}%
\pgfpathlineto{\pgfqpoint{2.160706in}{5.052109in}}%
\pgfpathlineto{\pgfqpoint{2.176140in}{5.038051in}}%
\pgfpathlineto{\pgfqpoint{2.191574in}{5.026114in}}%
\pgfpathlineto{\pgfqpoint{2.207007in}{5.016262in}}%
\pgfpathlineto{\pgfqpoint{2.222441in}{5.008374in}}%
\pgfpathlineto{\pgfqpoint{2.253308in}{4.997596in}}%
\pgfpathlineto{\pgfqpoint{2.284175in}{4.991286in}}%
\pgfpathlineto{\pgfqpoint{2.330476in}{4.982659in}}%
\pgfpathlineto{\pgfqpoint{2.361344in}{4.971943in}}%
\pgfpathlineto{\pgfqpoint{2.376777in}{4.963780in}}%
\pgfpathlineto{\pgfqpoint{2.392211in}{4.953282in}}%
\pgfpathlineto{\pgfqpoint{2.407644in}{4.940198in}}%
\pgfpathlineto{\pgfqpoint{2.423078in}{4.924377in}}%
\pgfpathlineto{\pgfqpoint{2.438512in}{4.905773in}}%
\pgfpathlineto{\pgfqpoint{2.469379in}{4.860624in}}%
\pgfpathlineto{\pgfqpoint{2.500246in}{4.806769in}}%
\pgfpathlineto{\pgfqpoint{2.577414in}{4.663025in}}%
\pgfpathlineto{\pgfqpoint{2.592848in}{4.638695in}}%
\pgfpathlineto{\pgfqpoint{2.608282in}{4.617762in}}%
\pgfpathlineto{\pgfqpoint{2.623715in}{4.601023in}}%
\pgfpathlineto{\pgfqpoint{2.639149in}{4.589210in}}%
\pgfpathlineto{\pgfqpoint{2.654583in}{4.582970in}}%
\pgfpathlineto{\pgfqpoint{2.670016in}{4.582838in}}%
\pgfpathlineto{\pgfqpoint{2.685450in}{4.589219in}}%
\pgfpathlineto{\pgfqpoint{2.700884in}{4.602366in}}%
\pgfpathlineto{\pgfqpoint{2.716317in}{4.622369in}}%
\pgfpathlineto{\pgfqpoint{2.731751in}{4.649143in}}%
\pgfpathlineto{\pgfqpoint{2.747184in}{4.682428in}}%
\pgfpathlineto{\pgfqpoint{2.762618in}{4.721782in}}%
\pgfpathlineto{\pgfqpoint{2.793485in}{4.816091in}}%
\pgfpathlineto{\pgfqpoint{2.824353in}{4.925336in}}%
\pgfpathlineto{\pgfqpoint{2.886087in}{5.152702in}}%
\pgfpathlineto{\pgfqpoint{2.916954in}{5.251008in}}%
\pgfpathlineto{\pgfqpoint{2.932388in}{5.292175in}}%
\pgfpathlineto{\pgfqpoint{2.947822in}{5.326678in}}%
\pgfpathlineto{\pgfqpoint{2.963255in}{5.353692in}}%
\pgfpathlineto{\pgfqpoint{2.978689in}{5.372563in}}%
\pgfpathlineto{\pgfqpoint{2.994123in}{5.382826in}}%
\pgfpathlineto{\pgfqpoint{3.009556in}{5.384227in}}%
\pgfpathlineto{\pgfqpoint{3.024990in}{5.376720in}}%
\pgfpathlineto{\pgfqpoint{3.040424in}{5.360481in}}%
\pgfpathlineto{\pgfqpoint{3.055857in}{5.335897in}}%
\pgfpathlineto{\pgfqpoint{3.071291in}{5.303560in}}%
\pgfpathlineto{\pgfqpoint{3.086724in}{5.264252in}}%
\pgfpathlineto{\pgfqpoint{3.102158in}{5.218924in}}%
\pgfpathlineto{\pgfqpoint{3.133025in}{5.114703in}}%
\pgfpathlineto{\pgfqpoint{3.210193in}{4.835254in}}%
\pgfpathlineto{\pgfqpoint{3.241061in}{4.742844in}}%
\pgfpathlineto{\pgfqpoint{3.256494in}{4.705123in}}%
\pgfpathlineto{\pgfqpoint{3.271928in}{4.674060in}}%
\pgfpathlineto{\pgfqpoint{3.287362in}{4.650170in}}%
\pgfpathlineto{\pgfqpoint{3.302795in}{4.633757in}}%
\pgfpathlineto{\pgfqpoint{3.318229in}{4.624912in}}%
\pgfpathlineto{\pgfqpoint{3.333663in}{4.623509in}}%
\pgfpathlineto{\pgfqpoint{3.349096in}{4.629215in}}%
\pgfpathlineto{\pgfqpoint{3.364530in}{4.641502in}}%
\pgfpathlineto{\pgfqpoint{3.379963in}{4.659661in}}%
\pgfpathlineto{\pgfqpoint{3.395397in}{4.682829in}}%
\pgfpathlineto{\pgfqpoint{3.426264in}{4.740119in}}%
\pgfpathlineto{\pgfqpoint{3.487999in}{4.866486in}}%
\pgfpathlineto{\pgfqpoint{3.503433in}{4.893851in}}%
\pgfpathlineto{\pgfqpoint{3.518866in}{4.917480in}}%
\pgfpathlineto{\pgfqpoint{3.534300in}{4.936558in}}%
\pgfpathlineto{\pgfqpoint{3.549733in}{4.950428in}}%
\pgfpathlineto{\pgfqpoint{3.565167in}{4.958612in}}%
\pgfpathlineto{\pgfqpoint{3.580601in}{4.960823in}}%
\pgfpathlineto{\pgfqpoint{3.596034in}{4.956975in}}%
\pgfpathlineto{\pgfqpoint{3.611468in}{4.947180in}}%
\pgfpathlineto{\pgfqpoint{3.626902in}{4.931750in}}%
\pgfpathlineto{\pgfqpoint{3.642335in}{4.911181in}}%
\pgfpathlineto{\pgfqpoint{3.657769in}{4.886142in}}%
\pgfpathlineto{\pgfqpoint{3.688636in}{4.826049in}}%
\pgfpathlineto{\pgfqpoint{3.750371in}{4.694930in}}%
\pgfpathlineto{\pgfqpoint{3.765804in}{4.666387in}}%
\pgfpathlineto{\pgfqpoint{3.781238in}{4.641689in}}%
\pgfpathlineto{\pgfqpoint{3.796672in}{4.621772in}}%
\pgfpathlineto{\pgfqpoint{3.812105in}{4.607449in}}%
\pgfpathlineto{\pgfqpoint{3.827539in}{4.599388in}}%
\pgfpathlineto{\pgfqpoint{3.842973in}{4.598089in}}%
\pgfpathlineto{\pgfqpoint{3.858406in}{4.603875in}}%
\pgfpathlineto{\pgfqpoint{3.873840in}{4.616882in}}%
\pgfpathlineto{\pgfqpoint{3.889273in}{4.637057in}}%
\pgfpathlineto{\pgfqpoint{3.904707in}{4.664162in}}%
\pgfpathlineto{\pgfqpoint{3.920141in}{4.697779in}}%
\pgfpathlineto{\pgfqpoint{3.935574in}{4.737327in}}%
\pgfpathlineto{\pgfqpoint{3.966442in}{4.831186in}}%
\pgfpathlineto{\pgfqpoint{3.997309in}{4.938588in}}%
\pgfpathlineto{\pgfqpoint{4.059043in}{5.160757in}}%
\pgfpathlineto{\pgfqpoint{4.089911in}{5.259338in}}%
\pgfpathlineto{\pgfqpoint{4.105344in}{5.302521in}}%
\pgfpathlineto{\pgfqpoint{4.120778in}{5.340804in}}%
\pgfpathlineto{\pgfqpoint{4.136212in}{5.373744in}}%
\pgfpathlineto{\pgfqpoint{4.151645in}{5.401051in}}%
\pgfpathlineto{\pgfqpoint{4.167079in}{5.422588in}}%
\pgfpathlineto{\pgfqpoint{4.182513in}{5.438371in}}%
\pgfpathlineto{\pgfqpoint{4.197946in}{5.448562in}}%
\pgfpathlineto{\pgfqpoint{4.213380in}{5.453453in}}%
\pgfpathlineto{\pgfqpoint{4.228813in}{5.453457in}}%
\pgfpathlineto{\pgfqpoint{4.244247in}{5.449085in}}%
\pgfpathlineto{\pgfqpoint{4.259681in}{5.440928in}}%
\pgfpathlineto{\pgfqpoint{4.275114in}{5.429632in}}%
\pgfpathlineto{\pgfqpoint{4.305982in}{5.400342in}}%
\pgfpathlineto{\pgfqpoint{4.383150in}{5.317795in}}%
\pgfpathlineto{\pgfqpoint{4.414017in}{5.291198in}}%
\pgfpathlineto{\pgfqpoint{4.444884in}{5.271010in}}%
\pgfpathlineto{\pgfqpoint{4.475752in}{5.256769in}}%
\pgfpathlineto{\pgfqpoint{4.506619in}{5.246570in}}%
\pgfpathlineto{\pgfqpoint{4.552920in}{5.232344in}}%
\pgfpathlineto{\pgfqpoint{4.583787in}{5.218738in}}%
\pgfpathlineto{\pgfqpoint{4.599221in}{5.209571in}}%
\pgfpathlineto{\pgfqpoint{4.614654in}{5.198450in}}%
\pgfpathlineto{\pgfqpoint{4.630088in}{5.185185in}}%
\pgfpathlineto{\pgfqpoint{4.645522in}{5.169677in}}%
\pgfpathlineto{\pgfqpoint{4.660955in}{5.151922in}}%
\pgfpathlineto{\pgfqpoint{4.660955in}{5.151922in}}%
\pgfusepath{stroke}%
\end{pgfscope}%
\begin{pgfscope}%
\pgfpathrectangle{\pgfqpoint{0.625831in}{3.799602in}}{\pgfqpoint{4.227273in}{2.745455in}} %
\pgfusepath{clip}%
\pgfsetrectcap%
\pgfsetroundjoin%
\pgfsetlinewidth{0.501875pt}%
\definecolor{currentstroke}{rgb}{0.056863,0.767363,0.905873}%
\pgfsetstrokecolor{currentstroke}%
\pgfsetdash{}{0pt}%
\pgfpathmoveto{\pgfqpoint{0.817980in}{5.541511in}}%
\pgfpathlineto{\pgfqpoint{0.833414in}{5.552750in}}%
\pgfpathlineto{\pgfqpoint{0.848847in}{5.558141in}}%
\pgfpathlineto{\pgfqpoint{0.864281in}{5.557240in}}%
\pgfpathlineto{\pgfqpoint{0.879715in}{5.549755in}}%
\pgfpathlineto{\pgfqpoint{0.895148in}{5.535560in}}%
\pgfpathlineto{\pgfqpoint{0.910582in}{5.514701in}}%
\pgfpathlineto{\pgfqpoint{0.926015in}{5.487391in}}%
\pgfpathlineto{\pgfqpoint{0.941449in}{5.454011in}}%
\pgfpathlineto{\pgfqpoint{0.956883in}{5.415097in}}%
\pgfpathlineto{\pgfqpoint{0.987750in}{5.323520in}}%
\pgfpathlineto{\pgfqpoint{1.018617in}{5.219509in}}%
\pgfpathlineto{\pgfqpoint{1.080352in}{5.007384in}}%
\pgfpathlineto{\pgfqpoint{1.111219in}{4.916274in}}%
\pgfpathlineto{\pgfqpoint{1.126653in}{4.877804in}}%
\pgfpathlineto{\pgfqpoint{1.142086in}{4.845079in}}%
\pgfpathlineto{\pgfqpoint{1.157520in}{4.818708in}}%
\pgfpathlineto{\pgfqpoint{1.172954in}{4.799150in}}%
\pgfpathlineto{\pgfqpoint{1.188387in}{4.786715in}}%
\pgfpathlineto{\pgfqpoint{1.203821in}{4.781554in}}%
\pgfpathlineto{\pgfqpoint{1.219255in}{4.783659in}}%
\pgfpathlineto{\pgfqpoint{1.234688in}{4.792868in}}%
\pgfpathlineto{\pgfqpoint{1.250122in}{4.808874in}}%
\pgfpathlineto{\pgfqpoint{1.265555in}{4.831233in}}%
\pgfpathlineto{\pgfqpoint{1.280989in}{4.859380in}}%
\pgfpathlineto{\pgfqpoint{1.296423in}{4.892646in}}%
\pgfpathlineto{\pgfqpoint{1.327290in}{4.971460in}}%
\pgfpathlineto{\pgfqpoint{1.373591in}{5.107637in}}%
\pgfpathlineto{\pgfqpoint{1.419892in}{5.244786in}}%
\pgfpathlineto{\pgfqpoint{1.450759in}{5.326650in}}%
\pgfpathlineto{\pgfqpoint{1.481626in}{5.395558in}}%
\pgfpathlineto{\pgfqpoint{1.497060in}{5.424218in}}%
\pgfpathlineto{\pgfqpoint{1.512494in}{5.448721in}}%
\pgfpathlineto{\pgfqpoint{1.527927in}{5.468971in}}%
\pgfpathlineto{\pgfqpoint{1.543361in}{5.484966in}}%
\pgfpathlineto{\pgfqpoint{1.558795in}{5.496791in}}%
\pgfpathlineto{\pgfqpoint{1.574228in}{5.504602in}}%
\pgfpathlineto{\pgfqpoint{1.589662in}{5.508622in}}%
\pgfpathlineto{\pgfqpoint{1.605095in}{5.509123in}}%
\pgfpathlineto{\pgfqpoint{1.620529in}{5.506416in}}%
\pgfpathlineto{\pgfqpoint{1.635963in}{5.500834in}}%
\pgfpathlineto{\pgfqpoint{1.651396in}{5.492724in}}%
\pgfpathlineto{\pgfqpoint{1.682264in}{5.470288in}}%
\pgfpathlineto{\pgfqpoint{1.713131in}{5.441676in}}%
\pgfpathlineto{\pgfqpoint{1.759432in}{5.391694in}}%
\pgfpathlineto{\pgfqpoint{1.821166in}{5.318517in}}%
\pgfpathlineto{\pgfqpoint{1.898334in}{5.221749in}}%
\pgfpathlineto{\pgfqpoint{1.990936in}{5.103767in}}%
\pgfpathlineto{\pgfqpoint{2.021804in}{5.067723in}}%
\pgfpathlineto{\pgfqpoint{2.052671in}{5.035866in}}%
\pgfpathlineto{\pgfqpoint{2.083538in}{5.009921in}}%
\pgfpathlineto{\pgfqpoint{2.098972in}{4.999652in}}%
\pgfpathlineto{\pgfqpoint{2.114405in}{4.991388in}}%
\pgfpathlineto{\pgfqpoint{2.129839in}{4.985234in}}%
\pgfpathlineto{\pgfqpoint{2.145273in}{4.981242in}}%
\pgfpathlineto{\pgfqpoint{2.160706in}{4.979408in}}%
\pgfpathlineto{\pgfqpoint{2.176140in}{4.979669in}}%
\pgfpathlineto{\pgfqpoint{2.191574in}{4.981895in}}%
\pgfpathlineto{\pgfqpoint{2.222441in}{4.991399in}}%
\pgfpathlineto{\pgfqpoint{2.253308in}{5.005620in}}%
\pgfpathlineto{\pgfqpoint{2.299609in}{5.028750in}}%
\pgfpathlineto{\pgfqpoint{2.330476in}{5.039896in}}%
\pgfpathlineto{\pgfqpoint{2.345910in}{5.042760in}}%
\pgfpathlineto{\pgfqpoint{2.361344in}{5.043240in}}%
\pgfpathlineto{\pgfqpoint{2.376777in}{5.040962in}}%
\pgfpathlineto{\pgfqpoint{2.392211in}{5.035616in}}%
\pgfpathlineto{\pgfqpoint{2.407644in}{5.026973in}}%
\pgfpathlineto{\pgfqpoint{2.423078in}{5.014890in}}%
\pgfpathlineto{\pgfqpoint{2.438512in}{4.999325in}}%
\pgfpathlineto{\pgfqpoint{2.453945in}{4.980335in}}%
\pgfpathlineto{\pgfqpoint{2.469379in}{4.958088in}}%
\pgfpathlineto{\pgfqpoint{2.500246in}{4.905015in}}%
\pgfpathlineto{\pgfqpoint{2.531114in}{4.843480in}}%
\pgfpathlineto{\pgfqpoint{2.592848in}{4.714944in}}%
\pgfpathlineto{\pgfqpoint{2.623715in}{4.659643in}}%
\pgfpathlineto{\pgfqpoint{2.639149in}{4.636805in}}%
\pgfpathlineto{\pgfqpoint{2.654583in}{4.618050in}}%
\pgfpathlineto{\pgfqpoint{2.670016in}{4.603935in}}%
\pgfpathlineto{\pgfqpoint{2.685450in}{4.594923in}}%
\pgfpathlineto{\pgfqpoint{2.700884in}{4.591364in}}%
\pgfpathlineto{\pgfqpoint{2.716317in}{4.593480in}}%
\pgfpathlineto{\pgfqpoint{2.731751in}{4.601363in}}%
\pgfpathlineto{\pgfqpoint{2.747184in}{4.614962in}}%
\pgfpathlineto{\pgfqpoint{2.762618in}{4.634088in}}%
\pgfpathlineto{\pgfqpoint{2.778052in}{4.658411in}}%
\pgfpathlineto{\pgfqpoint{2.793485in}{4.687471in}}%
\pgfpathlineto{\pgfqpoint{2.824353in}{4.757358in}}%
\pgfpathlineto{\pgfqpoint{2.855220in}{4.837881in}}%
\pgfpathlineto{\pgfqpoint{2.916954in}{5.002388in}}%
\pgfpathlineto{\pgfqpoint{2.947822in}{5.071856in}}%
\pgfpathlineto{\pgfqpoint{2.963255in}{5.100572in}}%
\pgfpathlineto{\pgfqpoint{2.978689in}{5.124437in}}%
\pgfpathlineto{\pgfqpoint{2.994123in}{5.142975in}}%
\pgfpathlineto{\pgfqpoint{3.009556in}{5.155844in}}%
\pgfpathlineto{\pgfqpoint{3.024990in}{5.162847in}}%
\pgfpathlineto{\pgfqpoint{3.040424in}{5.163937in}}%
\pgfpathlineto{\pgfqpoint{3.055857in}{5.159213in}}%
\pgfpathlineto{\pgfqpoint{3.071291in}{5.148923in}}%
\pgfpathlineto{\pgfqpoint{3.086724in}{5.133446in}}%
\pgfpathlineto{\pgfqpoint{3.102158in}{5.113290in}}%
\pgfpathlineto{\pgfqpoint{3.117592in}{5.089069in}}%
\pgfpathlineto{\pgfqpoint{3.148459in}{5.031315in}}%
\pgfpathlineto{\pgfqpoint{3.241061in}{4.841973in}}%
\pgfpathlineto{\pgfqpoint{3.256494in}{4.816161in}}%
\pgfpathlineto{\pgfqpoint{3.271928in}{4.793612in}}%
\pgfpathlineto{\pgfqpoint{3.287362in}{4.774701in}}%
\pgfpathlineto{\pgfqpoint{3.302795in}{4.759672in}}%
\pgfpathlineto{\pgfqpoint{3.318229in}{4.748636in}}%
\pgfpathlineto{\pgfqpoint{3.333663in}{4.741572in}}%
\pgfpathlineto{\pgfqpoint{3.349096in}{4.738327in}}%
\pgfpathlineto{\pgfqpoint{3.364530in}{4.738624in}}%
\pgfpathlineto{\pgfqpoint{3.379963in}{4.742076in}}%
\pgfpathlineto{\pgfqpoint{3.395397in}{4.748198in}}%
\pgfpathlineto{\pgfqpoint{3.426264in}{4.766134in}}%
\pgfpathlineto{\pgfqpoint{3.487999in}{4.806460in}}%
\pgfpathlineto{\pgfqpoint{3.503433in}{4.813697in}}%
\pgfpathlineto{\pgfqpoint{3.518866in}{4.818678in}}%
\pgfpathlineto{\pgfqpoint{3.534300in}{4.820983in}}%
\pgfpathlineto{\pgfqpoint{3.549733in}{4.820300in}}%
\pgfpathlineto{\pgfqpoint{3.565167in}{4.816431in}}%
\pgfpathlineto{\pgfqpoint{3.580601in}{4.809306in}}%
\pgfpathlineto{\pgfqpoint{3.596034in}{4.798979in}}%
\pgfpathlineto{\pgfqpoint{3.611468in}{4.785638in}}%
\pgfpathlineto{\pgfqpoint{3.626902in}{4.769589in}}%
\pgfpathlineto{\pgfqpoint{3.657769in}{4.731165in}}%
\pgfpathlineto{\pgfqpoint{3.719503in}{4.646406in}}%
\pgfpathlineto{\pgfqpoint{3.734937in}{4.627796in}}%
\pgfpathlineto{\pgfqpoint{3.750371in}{4.611716in}}%
\pgfpathlineto{\pgfqpoint{3.765804in}{4.598864in}}%
\pgfpathlineto{\pgfqpoint{3.781238in}{4.589874in}}%
\pgfpathlineto{\pgfqpoint{3.796672in}{4.585296in}}%
\pgfpathlineto{\pgfqpoint{3.812105in}{4.585575in}}%
\pgfpathlineto{\pgfqpoint{3.827539in}{4.591041in}}%
\pgfpathlineto{\pgfqpoint{3.842973in}{4.601897in}}%
\pgfpathlineto{\pgfqpoint{3.858406in}{4.618209in}}%
\pgfpathlineto{\pgfqpoint{3.873840in}{4.639906in}}%
\pgfpathlineto{\pgfqpoint{3.889273in}{4.666778in}}%
\pgfpathlineto{\pgfqpoint{3.904707in}{4.698479in}}%
\pgfpathlineto{\pgfqpoint{3.935574in}{4.774383in}}%
\pgfpathlineto{\pgfqpoint{3.966442in}{4.862599in}}%
\pgfpathlineto{\pgfqpoint{4.043610in}{5.094468in}}%
\pgfpathlineto{\pgfqpoint{4.074477in}{5.174122in}}%
\pgfpathlineto{\pgfqpoint{4.089911in}{5.208205in}}%
\pgfpathlineto{\pgfqpoint{4.105344in}{5.237731in}}%
\pgfpathlineto{\pgfqpoint{4.120778in}{5.262298in}}%
\pgfpathlineto{\pgfqpoint{4.136212in}{5.281625in}}%
\pgfpathlineto{\pgfqpoint{4.151645in}{5.295563in}}%
\pgfpathlineto{\pgfqpoint{4.167079in}{5.304090in}}%
\pgfpathlineto{\pgfqpoint{4.182513in}{5.307310in}}%
\pgfpathlineto{\pgfqpoint{4.197946in}{5.305449in}}%
\pgfpathlineto{\pgfqpoint{4.213380in}{5.298842in}}%
\pgfpathlineto{\pgfqpoint{4.228813in}{5.287924in}}%
\pgfpathlineto{\pgfqpoint{4.244247in}{5.273210in}}%
\pgfpathlineto{\pgfqpoint{4.259681in}{5.255284in}}%
\pgfpathlineto{\pgfqpoint{4.290548in}{5.212338in}}%
\pgfpathlineto{\pgfqpoint{4.383150in}{5.072874in}}%
\pgfpathlineto{\pgfqpoint{4.414017in}{5.037028in}}%
\pgfpathlineto{\pgfqpoint{4.429451in}{5.022619in}}%
\pgfpathlineto{\pgfqpoint{4.444884in}{5.010710in}}%
\pgfpathlineto{\pgfqpoint{4.460318in}{5.001303in}}%
\pgfpathlineto{\pgfqpoint{4.475752in}{4.994318in}}%
\pgfpathlineto{\pgfqpoint{4.491185in}{4.989599in}}%
\pgfpathlineto{\pgfqpoint{4.506619in}{4.986923in}}%
\pgfpathlineto{\pgfqpoint{4.537486in}{4.986555in}}%
\pgfpathlineto{\pgfqpoint{4.568353in}{4.990605in}}%
\pgfpathlineto{\pgfqpoint{4.630088in}{5.001082in}}%
\pgfpathlineto{\pgfqpoint{4.660955in}{5.003108in}}%
\pgfpathlineto{\pgfqpoint{4.660955in}{5.003108in}}%
\pgfusepath{stroke}%
\end{pgfscope}%
\begin{pgfscope}%
\pgfpathrectangle{\pgfqpoint{0.625831in}{3.799602in}}{\pgfqpoint{4.227273in}{2.745455in}} %
\pgfusepath{clip}%
\pgfsetrectcap%
\pgfsetroundjoin%
\pgfsetlinewidth{0.501875pt}%
\definecolor{currentstroke}{rgb}{0.135294,0.840344,0.878081}%
\pgfsetstrokecolor{currentstroke}%
\pgfsetdash{}{0pt}%
\pgfpathmoveto{\pgfqpoint{0.817980in}{5.606567in}}%
\pgfpathlineto{\pgfqpoint{0.833414in}{5.611944in}}%
\pgfpathlineto{\pgfqpoint{0.848847in}{5.612224in}}%
\pgfpathlineto{\pgfqpoint{0.864281in}{5.607175in}}%
\pgfpathlineto{\pgfqpoint{0.879715in}{5.596684in}}%
\pgfpathlineto{\pgfqpoint{0.895148in}{5.580761in}}%
\pgfpathlineto{\pgfqpoint{0.910582in}{5.559541in}}%
\pgfpathlineto{\pgfqpoint{0.926015in}{5.533285in}}%
\pgfpathlineto{\pgfqpoint{0.941449in}{5.502371in}}%
\pgfpathlineto{\pgfqpoint{0.972316in}{5.428641in}}%
\pgfpathlineto{\pgfqpoint{1.003184in}{5.343426in}}%
\pgfpathlineto{\pgfqpoint{1.080352in}{5.122352in}}%
\pgfpathlineto{\pgfqpoint{1.111219in}{5.048382in}}%
\pgfpathlineto{\pgfqpoint{1.126653in}{5.017539in}}%
\pgfpathlineto{\pgfqpoint{1.142086in}{4.991590in}}%
\pgfpathlineto{\pgfqpoint{1.157520in}{4.971011in}}%
\pgfpathlineto{\pgfqpoint{1.172954in}{4.956167in}}%
\pgfpathlineto{\pgfqpoint{1.188387in}{4.947304in}}%
\pgfpathlineto{\pgfqpoint{1.203821in}{4.944546in}}%
\pgfpathlineto{\pgfqpoint{1.219255in}{4.947894in}}%
\pgfpathlineto{\pgfqpoint{1.234688in}{4.957228in}}%
\pgfpathlineto{\pgfqpoint{1.250122in}{4.972313in}}%
\pgfpathlineto{\pgfqpoint{1.265555in}{4.992805in}}%
\pgfpathlineto{\pgfqpoint{1.280989in}{5.018264in}}%
\pgfpathlineto{\pgfqpoint{1.296423in}{5.048163in}}%
\pgfpathlineto{\pgfqpoint{1.327290in}{5.118826in}}%
\pgfpathlineto{\pgfqpoint{1.358157in}{5.199411in}}%
\pgfpathlineto{\pgfqpoint{1.435325in}{5.406724in}}%
\pgfpathlineto{\pgfqpoint{1.466193in}{5.478313in}}%
\pgfpathlineto{\pgfqpoint{1.497060in}{5.537394in}}%
\pgfpathlineto{\pgfqpoint{1.512494in}{5.561440in}}%
\pgfpathlineto{\pgfqpoint{1.527927in}{5.581553in}}%
\pgfpathlineto{\pgfqpoint{1.543361in}{5.597628in}}%
\pgfpathlineto{\pgfqpoint{1.558795in}{5.609632in}}%
\pgfpathlineto{\pgfqpoint{1.574228in}{5.617601in}}%
\pgfpathlineto{\pgfqpoint{1.589662in}{5.621632in}}%
\pgfpathlineto{\pgfqpoint{1.605095in}{5.621876in}}%
\pgfpathlineto{\pgfqpoint{1.620529in}{5.618528in}}%
\pgfpathlineto{\pgfqpoint{1.635963in}{5.611823in}}%
\pgfpathlineto{\pgfqpoint{1.651396in}{5.602018in}}%
\pgfpathlineto{\pgfqpoint{1.666830in}{5.589394in}}%
\pgfpathlineto{\pgfqpoint{1.682264in}{5.574242in}}%
\pgfpathlineto{\pgfqpoint{1.713131in}{5.537521in}}%
\pgfpathlineto{\pgfqpoint{1.743998in}{5.494143in}}%
\pgfpathlineto{\pgfqpoint{1.790299in}{5.421104in}}%
\pgfpathlineto{\pgfqpoint{1.929202in}{5.193803in}}%
\pgfpathlineto{\pgfqpoint{1.960069in}{5.149600in}}%
\pgfpathlineto{\pgfqpoint{1.990936in}{5.110237in}}%
\pgfpathlineto{\pgfqpoint{2.021804in}{5.076910in}}%
\pgfpathlineto{\pgfqpoint{2.052671in}{5.050734in}}%
\pgfpathlineto{\pgfqpoint{2.068104in}{5.040619in}}%
\pgfpathlineto{\pgfqpoint{2.083538in}{5.032600in}}%
\pgfpathlineto{\pgfqpoint{2.098972in}{5.026725in}}%
\pgfpathlineto{\pgfqpoint{2.114405in}{5.023009in}}%
\pgfpathlineto{\pgfqpoint{2.129839in}{5.021427in}}%
\pgfpathlineto{\pgfqpoint{2.145273in}{5.021910in}}%
\pgfpathlineto{\pgfqpoint{2.160706in}{5.024343in}}%
\pgfpathlineto{\pgfqpoint{2.191574in}{5.034354in}}%
\pgfpathlineto{\pgfqpoint{2.222441in}{5.049575in}}%
\pgfpathlineto{\pgfqpoint{2.299609in}{5.092144in}}%
\pgfpathlineto{\pgfqpoint{2.315043in}{5.098126in}}%
\pgfpathlineto{\pgfqpoint{2.330476in}{5.102270in}}%
\pgfpathlineto{\pgfqpoint{2.345910in}{5.104180in}}%
\pgfpathlineto{\pgfqpoint{2.361344in}{5.103500in}}%
\pgfpathlineto{\pgfqpoint{2.376777in}{5.099919in}}%
\pgfpathlineto{\pgfqpoint{2.392211in}{5.093188in}}%
\pgfpathlineto{\pgfqpoint{2.407644in}{5.083124in}}%
\pgfpathlineto{\pgfqpoint{2.423078in}{5.069621in}}%
\pgfpathlineto{\pgfqpoint{2.438512in}{5.052656in}}%
\pgfpathlineto{\pgfqpoint{2.453945in}{5.032291in}}%
\pgfpathlineto{\pgfqpoint{2.469379in}{5.008678in}}%
\pgfpathlineto{\pgfqpoint{2.500246in}{4.952758in}}%
\pgfpathlineto{\pgfqpoint{2.531114in}{4.887809in}}%
\pgfpathlineto{\pgfqpoint{2.623715in}{4.683031in}}%
\pgfpathlineto{\pgfqpoint{2.654583in}{4.628398in}}%
\pgfpathlineto{\pgfqpoint{2.670016in}{4.606338in}}%
\pgfpathlineto{\pgfqpoint{2.685450in}{4.588414in}}%
\pgfpathlineto{\pgfqpoint{2.700884in}{4.575000in}}%
\pgfpathlineto{\pgfqpoint{2.716317in}{4.566372in}}%
\pgfpathlineto{\pgfqpoint{2.731751in}{4.562707in}}%
\pgfpathlineto{\pgfqpoint{2.747184in}{4.564072in}}%
\pgfpathlineto{\pgfqpoint{2.762618in}{4.570425in}}%
\pgfpathlineto{\pgfqpoint{2.778052in}{4.581613in}}%
\pgfpathlineto{\pgfqpoint{2.793485in}{4.597374in}}%
\pgfpathlineto{\pgfqpoint{2.808919in}{4.617349in}}%
\pgfpathlineto{\pgfqpoint{2.824353in}{4.641082in}}%
\pgfpathlineto{\pgfqpoint{2.855220in}{4.697616in}}%
\pgfpathlineto{\pgfqpoint{2.901521in}{4.795340in}}%
\pgfpathlineto{\pgfqpoint{2.947822in}{4.891775in}}%
\pgfpathlineto{\pgfqpoint{2.978689in}{4.946546in}}%
\pgfpathlineto{\pgfqpoint{2.994123in}{4.969444in}}%
\pgfpathlineto{\pgfqpoint{3.009556in}{4.988808in}}%
\pgfpathlineto{\pgfqpoint{3.024990in}{5.004349in}}%
\pgfpathlineto{\pgfqpoint{3.040424in}{5.015876in}}%
\pgfpathlineto{\pgfqpoint{3.055857in}{5.023303in}}%
\pgfpathlineto{\pgfqpoint{3.071291in}{5.026645in}}%
\pgfpathlineto{\pgfqpoint{3.086724in}{5.026014in}}%
\pgfpathlineto{\pgfqpoint{3.102158in}{5.021615in}}%
\pgfpathlineto{\pgfqpoint{3.117592in}{5.013736in}}%
\pgfpathlineto{\pgfqpoint{3.133025in}{5.002739in}}%
\pgfpathlineto{\pgfqpoint{3.148459in}{4.989043in}}%
\pgfpathlineto{\pgfqpoint{3.179326in}{4.955445in}}%
\pgfpathlineto{\pgfqpoint{3.287362in}{4.823984in}}%
\pgfpathlineto{\pgfqpoint{3.318229in}{4.795553in}}%
\pgfpathlineto{\pgfqpoint{3.349096in}{4.774185in}}%
\pgfpathlineto{\pgfqpoint{3.379963in}{4.759641in}}%
\pgfpathlineto{\pgfqpoint{3.410831in}{4.750662in}}%
\pgfpathlineto{\pgfqpoint{3.457132in}{4.743139in}}%
\pgfpathlineto{\pgfqpoint{3.503433in}{4.735739in}}%
\pgfpathlineto{\pgfqpoint{3.534300in}{4.727541in}}%
\pgfpathlineto{\pgfqpoint{3.565167in}{4.715432in}}%
\pgfpathlineto{\pgfqpoint{3.596034in}{4.699435in}}%
\pgfpathlineto{\pgfqpoint{3.688636in}{4.644198in}}%
\pgfpathlineto{\pgfqpoint{3.704070in}{4.637557in}}%
\pgfpathlineto{\pgfqpoint{3.719503in}{4.632757in}}%
\pgfpathlineto{\pgfqpoint{3.734937in}{4.630213in}}%
\pgfpathlineto{\pgfqpoint{3.750371in}{4.630311in}}%
\pgfpathlineto{\pgfqpoint{3.765804in}{4.633396in}}%
\pgfpathlineto{\pgfqpoint{3.781238in}{4.639756in}}%
\pgfpathlineto{\pgfqpoint{3.796672in}{4.649615in}}%
\pgfpathlineto{\pgfqpoint{3.812105in}{4.663120in}}%
\pgfpathlineto{\pgfqpoint{3.827539in}{4.680334in}}%
\pgfpathlineto{\pgfqpoint{3.842973in}{4.701234in}}%
\pgfpathlineto{\pgfqpoint{3.858406in}{4.725706in}}%
\pgfpathlineto{\pgfqpoint{3.889273in}{4.784460in}}%
\pgfpathlineto{\pgfqpoint{3.920141in}{4.853946in}}%
\pgfpathlineto{\pgfqpoint{3.966442in}{4.969714in}}%
\pgfpathlineto{\pgfqpoint{4.012743in}{5.084996in}}%
\pgfpathlineto{\pgfqpoint{4.043610in}{5.153369in}}%
\pgfpathlineto{\pgfqpoint{4.059043in}{5.183344in}}%
\pgfpathlineto{\pgfqpoint{4.074477in}{5.209899in}}%
\pgfpathlineto{\pgfqpoint{4.089911in}{5.232670in}}%
\pgfpathlineto{\pgfqpoint{4.105344in}{5.251379in}}%
\pgfpathlineto{\pgfqpoint{4.120778in}{5.265838in}}%
\pgfpathlineto{\pgfqpoint{4.136212in}{5.275958in}}%
\pgfpathlineto{\pgfqpoint{4.151645in}{5.281746in}}%
\pgfpathlineto{\pgfqpoint{4.167079in}{5.283308in}}%
\pgfpathlineto{\pgfqpoint{4.182513in}{5.280840in}}%
\pgfpathlineto{\pgfqpoint{4.197946in}{5.274626in}}%
\pgfpathlineto{\pgfqpoint{4.213380in}{5.265027in}}%
\pgfpathlineto{\pgfqpoint{4.228813in}{5.252473in}}%
\pgfpathlineto{\pgfqpoint{4.259681in}{5.220475in}}%
\pgfpathlineto{\pgfqpoint{4.305982in}{5.163471in}}%
\pgfpathlineto{\pgfqpoint{4.336849in}{5.125950in}}%
\pgfpathlineto{\pgfqpoint{4.367716in}{5.093504in}}%
\pgfpathlineto{\pgfqpoint{4.383150in}{5.080200in}}%
\pgfpathlineto{\pgfqpoint{4.398583in}{5.069252in}}%
\pgfpathlineto{\pgfqpoint{4.414017in}{5.060877in}}%
\pgfpathlineto{\pgfqpoint{4.429451in}{5.055216in}}%
\pgfpathlineto{\pgfqpoint{4.444884in}{5.052329in}}%
\pgfpathlineto{\pgfqpoint{4.460318in}{5.052202in}}%
\pgfpathlineto{\pgfqpoint{4.475752in}{5.054750in}}%
\pgfpathlineto{\pgfqpoint{4.491185in}{5.059818in}}%
\pgfpathlineto{\pgfqpoint{4.506619in}{5.067196in}}%
\pgfpathlineto{\pgfqpoint{4.537486in}{5.087787in}}%
\pgfpathlineto{\pgfqpoint{4.568353in}{5.114005in}}%
\pgfpathlineto{\pgfqpoint{4.660955in}{5.198737in}}%
\pgfpathlineto{\pgfqpoint{4.660955in}{5.198737in}}%
\pgfusepath{stroke}%
\end{pgfscope}%
\begin{pgfscope}%
\pgfpathrectangle{\pgfqpoint{0.625831in}{3.799602in}}{\pgfqpoint{4.227273in}{2.745455in}} %
\pgfusepath{clip}%
\pgfsetrectcap%
\pgfsetroundjoin%
\pgfsetlinewidth{0.501875pt}%
\definecolor{currentstroke}{rgb}{0.221569,0.905873,0.843667}%
\pgfsetstrokecolor{currentstroke}%
\pgfsetdash{}{0pt}%
\pgfpathmoveto{\pgfqpoint{0.817980in}{5.699382in}}%
\pgfpathlineto{\pgfqpoint{0.833414in}{5.701926in}}%
\pgfpathlineto{\pgfqpoint{0.848847in}{5.697527in}}%
\pgfpathlineto{\pgfqpoint{0.864281in}{5.685962in}}%
\pgfpathlineto{\pgfqpoint{0.879715in}{5.667163in}}%
\pgfpathlineto{\pgfqpoint{0.895148in}{5.641223in}}%
\pgfpathlineto{\pgfqpoint{0.910582in}{5.608404in}}%
\pgfpathlineto{\pgfqpoint{0.926015in}{5.569127in}}%
\pgfpathlineto{\pgfqpoint{0.941449in}{5.523969in}}%
\pgfpathlineto{\pgfqpoint{0.972316in}{5.419021in}}%
\pgfpathlineto{\pgfqpoint{1.018617in}{5.239305in}}%
\pgfpathlineto{\pgfqpoint{1.064918in}{5.059408in}}%
\pgfpathlineto{\pgfqpoint{1.095785in}{4.954547in}}%
\pgfpathlineto{\pgfqpoint{1.111219in}{4.909739in}}%
\pgfpathlineto{\pgfqpoint{1.126653in}{4.871169in}}%
\pgfpathlineto{\pgfqpoint{1.142086in}{4.839542in}}%
\pgfpathlineto{\pgfqpoint{1.157520in}{4.815416in}}%
\pgfpathlineto{\pgfqpoint{1.172954in}{4.799190in}}%
\pgfpathlineto{\pgfqpoint{1.188387in}{4.791096in}}%
\pgfpathlineto{\pgfqpoint{1.203821in}{4.791194in}}%
\pgfpathlineto{\pgfqpoint{1.219255in}{4.799377in}}%
\pgfpathlineto{\pgfqpoint{1.234688in}{4.815373in}}%
\pgfpathlineto{\pgfqpoint{1.250122in}{4.838753in}}%
\pgfpathlineto{\pgfqpoint{1.265555in}{4.868950in}}%
\pgfpathlineto{\pgfqpoint{1.280989in}{4.905268in}}%
\pgfpathlineto{\pgfqpoint{1.311856in}{4.992972in}}%
\pgfpathlineto{\pgfqpoint{1.342724in}{5.094550in}}%
\pgfpathlineto{\pgfqpoint{1.419892in}{5.357435in}}%
\pgfpathlineto{\pgfqpoint{1.450759in}{5.447280in}}%
\pgfpathlineto{\pgfqpoint{1.466193in}{5.486080in}}%
\pgfpathlineto{\pgfqpoint{1.481626in}{5.520164in}}%
\pgfpathlineto{\pgfqpoint{1.497060in}{5.549199in}}%
\pgfpathlineto{\pgfqpoint{1.512494in}{5.572972in}}%
\pgfpathlineto{\pgfqpoint{1.527927in}{5.591385in}}%
\pgfpathlineto{\pgfqpoint{1.543361in}{5.604443in}}%
\pgfpathlineto{\pgfqpoint{1.558795in}{5.612256in}}%
\pgfpathlineto{\pgfqpoint{1.574228in}{5.615019in}}%
\pgfpathlineto{\pgfqpoint{1.589662in}{5.613003in}}%
\pgfpathlineto{\pgfqpoint{1.605095in}{5.606542in}}%
\pgfpathlineto{\pgfqpoint{1.620529in}{5.596016in}}%
\pgfpathlineto{\pgfqpoint{1.635963in}{5.581840in}}%
\pgfpathlineto{\pgfqpoint{1.651396in}{5.564447in}}%
\pgfpathlineto{\pgfqpoint{1.682264in}{5.521754in}}%
\pgfpathlineto{\pgfqpoint{1.713131in}{5.471299in}}%
\pgfpathlineto{\pgfqpoint{1.759432in}{5.387375in}}%
\pgfpathlineto{\pgfqpoint{1.882901in}{5.158285in}}%
\pgfpathlineto{\pgfqpoint{1.929202in}{5.080146in}}%
\pgfpathlineto{\pgfqpoint{1.960069in}{5.032921in}}%
\pgfpathlineto{\pgfqpoint{1.990936in}{4.991263in}}%
\pgfpathlineto{\pgfqpoint{2.021804in}{4.956988in}}%
\pgfpathlineto{\pgfqpoint{2.037237in}{4.943244in}}%
\pgfpathlineto{\pgfqpoint{2.052671in}{4.932099in}}%
\pgfpathlineto{\pgfqpoint{2.068104in}{4.923790in}}%
\pgfpathlineto{\pgfqpoint{2.083538in}{4.918530in}}%
\pgfpathlineto{\pgfqpoint{2.098972in}{4.916498in}}%
\pgfpathlineto{\pgfqpoint{2.114405in}{4.917825in}}%
\pgfpathlineto{\pgfqpoint{2.129839in}{4.922585in}}%
\pgfpathlineto{\pgfqpoint{2.145273in}{4.930786in}}%
\pgfpathlineto{\pgfqpoint{2.160706in}{4.942364in}}%
\pgfpathlineto{\pgfqpoint{2.176140in}{4.957171in}}%
\pgfpathlineto{\pgfqpoint{2.191574in}{4.974978in}}%
\pgfpathlineto{\pgfqpoint{2.222441in}{5.018242in}}%
\pgfpathlineto{\pgfqpoint{2.253308in}{5.068641in}}%
\pgfpathlineto{\pgfqpoint{2.299609in}{5.147171in}}%
\pgfpathlineto{\pgfqpoint{2.330476in}{5.193334in}}%
\pgfpathlineto{\pgfqpoint{2.345910in}{5.212387in}}%
\pgfpathlineto{\pgfqpoint{2.361344in}{5.227834in}}%
\pgfpathlineto{\pgfqpoint{2.376777in}{5.239050in}}%
\pgfpathlineto{\pgfqpoint{2.392211in}{5.245482in}}%
\pgfpathlineto{\pgfqpoint{2.407644in}{5.246669in}}%
\pgfpathlineto{\pgfqpoint{2.423078in}{5.242256in}}%
\pgfpathlineto{\pgfqpoint{2.438512in}{5.232007in}}%
\pgfpathlineto{\pgfqpoint{2.453945in}{5.215816in}}%
\pgfpathlineto{\pgfqpoint{2.469379in}{5.193716in}}%
\pgfpathlineto{\pgfqpoint{2.484813in}{5.165878in}}%
\pgfpathlineto{\pgfqpoint{2.500246in}{5.132616in}}%
\pgfpathlineto{\pgfqpoint{2.515680in}{5.094381in}}%
\pgfpathlineto{\pgfqpoint{2.546547in}{5.005424in}}%
\pgfpathlineto{\pgfqpoint{2.592848in}{4.852675in}}%
\pgfpathlineto{\pgfqpoint{2.639149in}{4.699461in}}%
\pgfpathlineto{\pgfqpoint{2.670016in}{4.610279in}}%
\pgfpathlineto{\pgfqpoint{2.685450in}{4.572314in}}%
\pgfpathlineto{\pgfqpoint{2.700884in}{4.539792in}}%
\pgfpathlineto{\pgfqpoint{2.716317in}{4.513341in}}%
\pgfpathlineto{\pgfqpoint{2.731751in}{4.493455in}}%
\pgfpathlineto{\pgfqpoint{2.747184in}{4.480484in}}%
\pgfpathlineto{\pgfqpoint{2.762618in}{4.474621in}}%
\pgfpathlineto{\pgfqpoint{2.778052in}{4.475905in}}%
\pgfpathlineto{\pgfqpoint{2.793485in}{4.484214in}}%
\pgfpathlineto{\pgfqpoint{2.808919in}{4.499275in}}%
\pgfpathlineto{\pgfqpoint{2.824353in}{4.520667in}}%
\pgfpathlineto{\pgfqpoint{2.839786in}{4.547837in}}%
\pgfpathlineto{\pgfqpoint{2.855220in}{4.580111in}}%
\pgfpathlineto{\pgfqpoint{2.886087in}{4.656803in}}%
\pgfpathlineto{\pgfqpoint{2.932388in}{4.788698in}}%
\pgfpathlineto{\pgfqpoint{2.978689in}{4.918457in}}%
\pgfpathlineto{\pgfqpoint{3.009556in}{4.992556in}}%
\pgfpathlineto{\pgfqpoint{3.024990in}{5.023899in}}%
\pgfpathlineto{\pgfqpoint{3.040424in}{5.050810in}}%
\pgfpathlineto{\pgfqpoint{3.055857in}{5.072976in}}%
\pgfpathlineto{\pgfqpoint{3.071291in}{5.090207in}}%
\pgfpathlineto{\pgfqpoint{3.086724in}{5.102442in}}%
\pgfpathlineto{\pgfqpoint{3.102158in}{5.109741in}}%
\pgfpathlineto{\pgfqpoint{3.117592in}{5.112278in}}%
\pgfpathlineto{\pgfqpoint{3.133025in}{5.110327in}}%
\pgfpathlineto{\pgfqpoint{3.148459in}{5.104253in}}%
\pgfpathlineto{\pgfqpoint{3.163893in}{5.094491in}}%
\pgfpathlineto{\pgfqpoint{3.179326in}{5.081531in}}%
\pgfpathlineto{\pgfqpoint{3.210193in}{5.048145in}}%
\pgfpathlineto{\pgfqpoint{3.241061in}{5.008411in}}%
\pgfpathlineto{\pgfqpoint{3.318229in}{4.905857in}}%
\pgfpathlineto{\pgfqpoint{3.349096in}{4.870009in}}%
\pgfpathlineto{\pgfqpoint{3.379963in}{4.838406in}}%
\pgfpathlineto{\pgfqpoint{3.426264in}{4.797358in}}%
\pgfpathlineto{\pgfqpoint{3.518866in}{4.719590in}}%
\pgfpathlineto{\pgfqpoint{3.565167in}{4.675434in}}%
\pgfpathlineto{\pgfqpoint{3.626902in}{4.615581in}}%
\pgfpathlineto{\pgfqpoint{3.657769in}{4.590991in}}%
\pgfpathlineto{\pgfqpoint{3.673203in}{4.581573in}}%
\pgfpathlineto{\pgfqpoint{3.688636in}{4.574718in}}%
\pgfpathlineto{\pgfqpoint{3.704070in}{4.570899in}}%
\pgfpathlineto{\pgfqpoint{3.719503in}{4.570559in}}%
\pgfpathlineto{\pgfqpoint{3.734937in}{4.574095in}}%
\pgfpathlineto{\pgfqpoint{3.750371in}{4.581841in}}%
\pgfpathlineto{\pgfqpoint{3.765804in}{4.594054in}}%
\pgfpathlineto{\pgfqpoint{3.781238in}{4.610897in}}%
\pgfpathlineto{\pgfqpoint{3.796672in}{4.632433in}}%
\pgfpathlineto{\pgfqpoint{3.812105in}{4.658611in}}%
\pgfpathlineto{\pgfqpoint{3.827539in}{4.689270in}}%
\pgfpathlineto{\pgfqpoint{3.858406in}{4.762785in}}%
\pgfpathlineto{\pgfqpoint{3.889273in}{4.849389in}}%
\pgfpathlineto{\pgfqpoint{3.997309in}{5.172276in}}%
\pgfpathlineto{\pgfqpoint{4.012743in}{5.209834in}}%
\pgfpathlineto{\pgfqpoint{4.028176in}{5.242900in}}%
\pgfpathlineto{\pgfqpoint{4.043610in}{5.270860in}}%
\pgfpathlineto{\pgfqpoint{4.059043in}{5.293211in}}%
\pgfpathlineto{\pgfqpoint{4.074477in}{5.309571in}}%
\pgfpathlineto{\pgfqpoint{4.089911in}{5.319693in}}%
\pgfpathlineto{\pgfqpoint{4.105344in}{5.323472in}}%
\pgfpathlineto{\pgfqpoint{4.120778in}{5.320945in}}%
\pgfpathlineto{\pgfqpoint{4.136212in}{5.312297in}}%
\pgfpathlineto{\pgfqpoint{4.151645in}{5.297850in}}%
\pgfpathlineto{\pgfqpoint{4.167079in}{5.278058in}}%
\pgfpathlineto{\pgfqpoint{4.182513in}{5.253497in}}%
\pgfpathlineto{\pgfqpoint{4.197946in}{5.224847in}}%
\pgfpathlineto{\pgfqpoint{4.228813in}{5.158419in}}%
\pgfpathlineto{\pgfqpoint{4.305982in}{4.980100in}}%
\pgfpathlineto{\pgfqpoint{4.336849in}{4.921908in}}%
\pgfpathlineto{\pgfqpoint{4.352283in}{4.898518in}}%
\pgfpathlineto{\pgfqpoint{4.367716in}{4.879614in}}%
\pgfpathlineto{\pgfqpoint{4.383150in}{4.865558in}}%
\pgfpathlineto{\pgfqpoint{4.398583in}{4.856585in}}%
\pgfpathlineto{\pgfqpoint{4.414017in}{4.852803in}}%
\pgfpathlineto{\pgfqpoint{4.429451in}{4.854188in}}%
\pgfpathlineto{\pgfqpoint{4.444884in}{4.860594in}}%
\pgfpathlineto{\pgfqpoint{4.460318in}{4.871758in}}%
\pgfpathlineto{\pgfqpoint{4.475752in}{4.887314in}}%
\pgfpathlineto{\pgfqpoint{4.491185in}{4.906802in}}%
\pgfpathlineto{\pgfqpoint{4.506619in}{4.929687in}}%
\pgfpathlineto{\pgfqpoint{4.537486in}{4.983234in}}%
\pgfpathlineto{\pgfqpoint{4.645522in}{5.185541in}}%
\pgfpathlineto{\pgfqpoint{4.660955in}{5.208894in}}%
\pgfpathlineto{\pgfqpoint{4.660955in}{5.208894in}}%
\pgfusepath{stroke}%
\end{pgfscope}%
\begin{pgfscope}%
\pgfpathrectangle{\pgfqpoint{0.625831in}{3.799602in}}{\pgfqpoint{4.227273in}{2.745455in}} %
\pgfusepath{clip}%
\pgfsetrectcap%
\pgfsetroundjoin%
\pgfsetlinewidth{0.501875pt}%
\definecolor{currentstroke}{rgb}{0.300000,0.951057,0.809017}%
\pgfsetstrokecolor{currentstroke}%
\pgfsetdash{}{0pt}%
\pgfpathmoveto{\pgfqpoint{0.817980in}{5.660382in}}%
\pgfpathlineto{\pgfqpoint{0.833414in}{5.668507in}}%
\pgfpathlineto{\pgfqpoint{0.848847in}{5.670691in}}%
\pgfpathlineto{\pgfqpoint{0.864281in}{5.666626in}}%
\pgfpathlineto{\pgfqpoint{0.879715in}{5.656135in}}%
\pgfpathlineto{\pgfqpoint{0.895148in}{5.639173in}}%
\pgfpathlineto{\pgfqpoint{0.910582in}{5.615835in}}%
\pgfpathlineto{\pgfqpoint{0.926015in}{5.586358in}}%
\pgfpathlineto{\pgfqpoint{0.941449in}{5.551117in}}%
\pgfpathlineto{\pgfqpoint{0.956883in}{5.510618in}}%
\pgfpathlineto{\pgfqpoint{0.987750in}{5.416479in}}%
\pgfpathlineto{\pgfqpoint{1.018617in}{5.310211in}}%
\pgfpathlineto{\pgfqpoint{1.080352in}{5.091857in}}%
\pgfpathlineto{\pgfqpoint{1.111219in}{4.995793in}}%
\pgfpathlineto{\pgfqpoint{1.126653in}{4.954308in}}%
\pgfpathlineto{\pgfqpoint{1.142086in}{4.918243in}}%
\pgfpathlineto{\pgfqpoint{1.157520in}{4.888260in}}%
\pgfpathlineto{\pgfqpoint{1.172954in}{4.864898in}}%
\pgfpathlineto{\pgfqpoint{1.188387in}{4.848568in}}%
\pgfpathlineto{\pgfqpoint{1.203821in}{4.839540in}}%
\pgfpathlineto{\pgfqpoint{1.219255in}{4.837943in}}%
\pgfpathlineto{\pgfqpoint{1.234688in}{4.843761in}}%
\pgfpathlineto{\pgfqpoint{1.250122in}{4.856836in}}%
\pgfpathlineto{\pgfqpoint{1.265555in}{4.876877in}}%
\pgfpathlineto{\pgfqpoint{1.280989in}{4.903464in}}%
\pgfpathlineto{\pgfqpoint{1.296423in}{4.936062in}}%
\pgfpathlineto{\pgfqpoint{1.311856in}{4.974036in}}%
\pgfpathlineto{\pgfqpoint{1.342724in}{5.063147in}}%
\pgfpathlineto{\pgfqpoint{1.373591in}{5.164299in}}%
\pgfpathlineto{\pgfqpoint{1.450759in}{5.424112in}}%
\pgfpathlineto{\pgfqpoint{1.481626in}{5.513897in}}%
\pgfpathlineto{\pgfqpoint{1.497060in}{5.553124in}}%
\pgfpathlineto{\pgfqpoint{1.512494in}{5.587941in}}%
\pgfpathlineto{\pgfqpoint{1.527927in}{5.617989in}}%
\pgfpathlineto{\pgfqpoint{1.543361in}{5.643004in}}%
\pgfpathlineto{\pgfqpoint{1.558795in}{5.662818in}}%
\pgfpathlineto{\pgfqpoint{1.574228in}{5.677352in}}%
\pgfpathlineto{\pgfqpoint{1.589662in}{5.686616in}}%
\pgfpathlineto{\pgfqpoint{1.605095in}{5.690697in}}%
\pgfpathlineto{\pgfqpoint{1.620529in}{5.689755in}}%
\pgfpathlineto{\pgfqpoint{1.635963in}{5.684016in}}%
\pgfpathlineto{\pgfqpoint{1.651396in}{5.673757in}}%
\pgfpathlineto{\pgfqpoint{1.666830in}{5.659302in}}%
\pgfpathlineto{\pgfqpoint{1.682264in}{5.641009in}}%
\pgfpathlineto{\pgfqpoint{1.697697in}{5.619262in}}%
\pgfpathlineto{\pgfqpoint{1.728564in}{5.567020in}}%
\pgfpathlineto{\pgfqpoint{1.759432in}{5.505854in}}%
\pgfpathlineto{\pgfqpoint{1.805733in}{5.404359in}}%
\pgfpathlineto{\pgfqpoint{1.882901in}{5.232844in}}%
\pgfpathlineto{\pgfqpoint{1.913768in}{5.170397in}}%
\pgfpathlineto{\pgfqpoint{1.944635in}{5.114427in}}%
\pgfpathlineto{\pgfqpoint{1.975503in}{5.066562in}}%
\pgfpathlineto{\pgfqpoint{2.006370in}{5.028176in}}%
\pgfpathlineto{\pgfqpoint{2.021804in}{5.012887in}}%
\pgfpathlineto{\pgfqpoint{2.037237in}{5.000345in}}%
\pgfpathlineto{\pgfqpoint{2.052671in}{4.990625in}}%
\pgfpathlineto{\pgfqpoint{2.068104in}{4.983774in}}%
\pgfpathlineto{\pgfqpoint{2.083538in}{4.979803in}}%
\pgfpathlineto{\pgfqpoint{2.098972in}{4.978688in}}%
\pgfpathlineto{\pgfqpoint{2.114405in}{4.980364in}}%
\pgfpathlineto{\pgfqpoint{2.129839in}{4.984723in}}%
\pgfpathlineto{\pgfqpoint{2.145273in}{4.991613in}}%
\pgfpathlineto{\pgfqpoint{2.160706in}{5.000834in}}%
\pgfpathlineto{\pgfqpoint{2.191574in}{5.025239in}}%
\pgfpathlineto{\pgfqpoint{2.222441in}{5.055436in}}%
\pgfpathlineto{\pgfqpoint{2.284175in}{5.120232in}}%
\pgfpathlineto{\pgfqpoint{2.315043in}{5.147487in}}%
\pgfpathlineto{\pgfqpoint{2.330476in}{5.158205in}}%
\pgfpathlineto{\pgfqpoint{2.345910in}{5.166416in}}%
\pgfpathlineto{\pgfqpoint{2.361344in}{5.171741in}}%
\pgfpathlineto{\pgfqpoint{2.376777in}{5.173854in}}%
\pgfpathlineto{\pgfqpoint{2.392211in}{5.172488in}}%
\pgfpathlineto{\pgfqpoint{2.407644in}{5.167448in}}%
\pgfpathlineto{\pgfqpoint{2.423078in}{5.158612in}}%
\pgfpathlineto{\pgfqpoint{2.438512in}{5.145944in}}%
\pgfpathlineto{\pgfqpoint{2.453945in}{5.129492in}}%
\pgfpathlineto{\pgfqpoint{2.469379in}{5.109391in}}%
\pgfpathlineto{\pgfqpoint{2.484813in}{5.085863in}}%
\pgfpathlineto{\pgfqpoint{2.515680in}{5.029831in}}%
\pgfpathlineto{\pgfqpoint{2.546547in}{4.964748in}}%
\pgfpathlineto{\pgfqpoint{2.623715in}{4.792252in}}%
\pgfpathlineto{\pgfqpoint{2.654583in}{4.732457in}}%
\pgfpathlineto{\pgfqpoint{2.670016in}{4.706882in}}%
\pgfpathlineto{\pgfqpoint{2.685450in}{4.684833in}}%
\pgfpathlineto{\pgfqpoint{2.700884in}{4.666700in}}%
\pgfpathlineto{\pgfqpoint{2.716317in}{4.652789in}}%
\pgfpathlineto{\pgfqpoint{2.731751in}{4.643314in}}%
\pgfpathlineto{\pgfqpoint{2.747184in}{4.638390in}}%
\pgfpathlineto{\pgfqpoint{2.762618in}{4.638034in}}%
\pgfpathlineto{\pgfqpoint{2.778052in}{4.642160in}}%
\pgfpathlineto{\pgfqpoint{2.793485in}{4.650585in}}%
\pgfpathlineto{\pgfqpoint{2.808919in}{4.663032in}}%
\pgfpathlineto{\pgfqpoint{2.824353in}{4.679142in}}%
\pgfpathlineto{\pgfqpoint{2.839786in}{4.698477in}}%
\pgfpathlineto{\pgfqpoint{2.870654in}{4.744770in}}%
\pgfpathlineto{\pgfqpoint{2.916954in}{4.824545in}}%
\pgfpathlineto{\pgfqpoint{2.963255in}{4.902305in}}%
\pgfpathlineto{\pgfqpoint{2.994123in}{4.945819in}}%
\pgfpathlineto{\pgfqpoint{3.009556in}{4.963813in}}%
\pgfpathlineto{\pgfqpoint{3.024990in}{4.978902in}}%
\pgfpathlineto{\pgfqpoint{3.040424in}{4.990894in}}%
\pgfpathlineto{\pgfqpoint{3.055857in}{4.999679in}}%
\pgfpathlineto{\pgfqpoint{3.071291in}{5.005230in}}%
\pgfpathlineto{\pgfqpoint{3.086724in}{5.007603in}}%
\pgfpathlineto{\pgfqpoint{3.102158in}{5.006926in}}%
\pgfpathlineto{\pgfqpoint{3.117592in}{5.003396in}}%
\pgfpathlineto{\pgfqpoint{3.133025in}{4.997270in}}%
\pgfpathlineto{\pgfqpoint{3.148459in}{4.988851in}}%
\pgfpathlineto{\pgfqpoint{3.179326in}{4.966517in}}%
\pgfpathlineto{\pgfqpoint{3.210193in}{4.939325in}}%
\pgfpathlineto{\pgfqpoint{3.287362in}{4.868237in}}%
\pgfpathlineto{\pgfqpoint{3.318229in}{4.843893in}}%
\pgfpathlineto{\pgfqpoint{3.349096in}{4.823330in}}%
\pgfpathlineto{\pgfqpoint{3.379963in}{4.806384in}}%
\pgfpathlineto{\pgfqpoint{3.426264in}{4.785856in}}%
\pgfpathlineto{\pgfqpoint{3.518866in}{4.747704in}}%
\pgfpathlineto{\pgfqpoint{3.626902in}{4.697650in}}%
\pgfpathlineto{\pgfqpoint{3.657769in}{4.689058in}}%
\pgfpathlineto{\pgfqpoint{3.673203in}{4.687232in}}%
\pgfpathlineto{\pgfqpoint{3.688636in}{4.687504in}}%
\pgfpathlineto{\pgfqpoint{3.704070in}{4.690192in}}%
\pgfpathlineto{\pgfqpoint{3.719503in}{4.695585in}}%
\pgfpathlineto{\pgfqpoint{3.734937in}{4.703927in}}%
\pgfpathlineto{\pgfqpoint{3.750371in}{4.715412in}}%
\pgfpathlineto{\pgfqpoint{3.765804in}{4.730168in}}%
\pgfpathlineto{\pgfqpoint{3.781238in}{4.748249in}}%
\pgfpathlineto{\pgfqpoint{3.796672in}{4.769633in}}%
\pgfpathlineto{\pgfqpoint{3.812105in}{4.794215in}}%
\pgfpathlineto{\pgfqpoint{3.842973in}{4.852125in}}%
\pgfpathlineto{\pgfqpoint{3.873840in}{4.919474in}}%
\pgfpathlineto{\pgfqpoint{3.981875in}{5.169692in}}%
\pgfpathlineto{\pgfqpoint{4.012743in}{5.225383in}}%
\pgfpathlineto{\pgfqpoint{4.028176in}{5.247814in}}%
\pgfpathlineto{\pgfqpoint{4.043610in}{5.266099in}}%
\pgfpathlineto{\pgfqpoint{4.059043in}{5.279945in}}%
\pgfpathlineto{\pgfqpoint{4.074477in}{5.289158in}}%
\pgfpathlineto{\pgfqpoint{4.089911in}{5.293650in}}%
\pgfpathlineto{\pgfqpoint{4.105344in}{5.293441in}}%
\pgfpathlineto{\pgfqpoint{4.120778in}{5.288657in}}%
\pgfpathlineto{\pgfqpoint{4.136212in}{5.279535in}}%
\pgfpathlineto{\pgfqpoint{4.151645in}{5.266410in}}%
\pgfpathlineto{\pgfqpoint{4.167079in}{5.249710in}}%
\pgfpathlineto{\pgfqpoint{4.197946in}{5.207699in}}%
\pgfpathlineto{\pgfqpoint{4.244247in}{5.132583in}}%
\pgfpathlineto{\pgfqpoint{4.275114in}{5.082421in}}%
\pgfpathlineto{\pgfqpoint{4.305982in}{5.038427in}}%
\pgfpathlineto{\pgfqpoint{4.321415in}{5.020208in}}%
\pgfpathlineto{\pgfqpoint{4.336849in}{5.005155in}}%
\pgfpathlineto{\pgfqpoint{4.352283in}{4.993646in}}%
\pgfpathlineto{\pgfqpoint{4.367716in}{4.985964in}}%
\pgfpathlineto{\pgfqpoint{4.383150in}{4.982292in}}%
\pgfpathlineto{\pgfqpoint{4.398583in}{4.982708in}}%
\pgfpathlineto{\pgfqpoint{4.414017in}{4.987187in}}%
\pgfpathlineto{\pgfqpoint{4.429451in}{4.995605in}}%
\pgfpathlineto{\pgfqpoint{4.444884in}{5.007739in}}%
\pgfpathlineto{\pgfqpoint{4.460318in}{5.023282in}}%
\pgfpathlineto{\pgfqpoint{4.475752in}{5.041847in}}%
\pgfpathlineto{\pgfqpoint{4.506619in}{5.086197in}}%
\pgfpathlineto{\pgfqpoint{4.552920in}{5.162813in}}%
\pgfpathlineto{\pgfqpoint{4.599221in}{5.238363in}}%
\pgfpathlineto{\pgfqpoint{4.630088in}{5.281616in}}%
\pgfpathlineto{\pgfqpoint{4.645522in}{5.299986in}}%
\pgfpathlineto{\pgfqpoint{4.660955in}{5.315845in}}%
\pgfpathlineto{\pgfqpoint{4.660955in}{5.315845in}}%
\pgfusepath{stroke}%
\end{pgfscope}%
\begin{pgfscope}%
\pgfpathrectangle{\pgfqpoint{0.625831in}{3.799602in}}{\pgfqpoint{4.227273in}{2.745455in}} %
\pgfusepath{clip}%
\pgfsetrectcap%
\pgfsetroundjoin%
\pgfsetlinewidth{0.501875pt}%
\definecolor{currentstroke}{rgb}{0.378431,0.981823,0.771298}%
\pgfsetstrokecolor{currentstroke}%
\pgfsetdash{}{0pt}%
\pgfpathmoveto{\pgfqpoint{0.817980in}{5.620530in}}%
\pgfpathlineto{\pgfqpoint{0.833414in}{5.632352in}}%
\pgfpathlineto{\pgfqpoint{0.848847in}{5.638720in}}%
\pgfpathlineto{\pgfqpoint{0.864281in}{5.639335in}}%
\pgfpathlineto{\pgfqpoint{0.879715in}{5.634012in}}%
\pgfpathlineto{\pgfqpoint{0.895148in}{5.622680in}}%
\pgfpathlineto{\pgfqpoint{0.910582in}{5.605391in}}%
\pgfpathlineto{\pgfqpoint{0.926015in}{5.582318in}}%
\pgfpathlineto{\pgfqpoint{0.941449in}{5.553756in}}%
\pgfpathlineto{\pgfqpoint{0.956883in}{5.520114in}}%
\pgfpathlineto{\pgfqpoint{0.987750in}{5.439768in}}%
\pgfpathlineto{\pgfqpoint{1.018617in}{5.346552in}}%
\pgfpathlineto{\pgfqpoint{1.095785in}{5.100778in}}%
\pgfpathlineto{\pgfqpoint{1.126653in}{5.015825in}}%
\pgfpathlineto{\pgfqpoint{1.142086in}{4.979467in}}%
\pgfpathlineto{\pgfqpoint{1.157520in}{4.948084in}}%
\pgfpathlineto{\pgfqpoint{1.172954in}{4.922234in}}%
\pgfpathlineto{\pgfqpoint{1.188387in}{4.902370in}}%
\pgfpathlineto{\pgfqpoint{1.203821in}{4.888831in}}%
\pgfpathlineto{\pgfqpoint{1.219255in}{4.881838in}}%
\pgfpathlineto{\pgfqpoint{1.234688in}{4.881493in}}%
\pgfpathlineto{\pgfqpoint{1.250122in}{4.887775in}}%
\pgfpathlineto{\pgfqpoint{1.265555in}{4.900543in}}%
\pgfpathlineto{\pgfqpoint{1.280989in}{4.919543in}}%
\pgfpathlineto{\pgfqpoint{1.296423in}{4.944415in}}%
\pgfpathlineto{\pgfqpoint{1.311856in}{4.974699in}}%
\pgfpathlineto{\pgfqpoint{1.327290in}{5.009849in}}%
\pgfpathlineto{\pgfqpoint{1.358157in}{5.092207in}}%
\pgfpathlineto{\pgfqpoint{1.389025in}{5.185880in}}%
\pgfpathlineto{\pgfqpoint{1.466193in}{5.429482in}}%
\pgfpathlineto{\pgfqpoint{1.497060in}{5.515451in}}%
\pgfpathlineto{\pgfqpoint{1.527927in}{5.587627in}}%
\pgfpathlineto{\pgfqpoint{1.543361in}{5.617444in}}%
\pgfpathlineto{\pgfqpoint{1.558795in}{5.642648in}}%
\pgfpathlineto{\pgfqpoint{1.574228in}{5.663013in}}%
\pgfpathlineto{\pgfqpoint{1.589662in}{5.678394in}}%
\pgfpathlineto{\pgfqpoint{1.605095in}{5.688722in}}%
\pgfpathlineto{\pgfqpoint{1.620529in}{5.693999in}}%
\pgfpathlineto{\pgfqpoint{1.635963in}{5.694301in}}%
\pgfpathlineto{\pgfqpoint{1.651396in}{5.689764in}}%
\pgfpathlineto{\pgfqpoint{1.666830in}{5.680587in}}%
\pgfpathlineto{\pgfqpoint{1.682264in}{5.667016in}}%
\pgfpathlineto{\pgfqpoint{1.697697in}{5.649347in}}%
\pgfpathlineto{\pgfqpoint{1.713131in}{5.627912in}}%
\pgfpathlineto{\pgfqpoint{1.728564in}{5.603077in}}%
\pgfpathlineto{\pgfqpoint{1.759432in}{5.544784in}}%
\pgfpathlineto{\pgfqpoint{1.790299in}{5.477791in}}%
\pgfpathlineto{\pgfqpoint{1.852034in}{5.331493in}}%
\pgfpathlineto{\pgfqpoint{1.898334in}{5.224118in}}%
\pgfpathlineto{\pgfqpoint{1.929202in}{5.159314in}}%
\pgfpathlineto{\pgfqpoint{1.960069in}{5.102901in}}%
\pgfpathlineto{\pgfqpoint{1.990936in}{5.056910in}}%
\pgfpathlineto{\pgfqpoint{2.006370in}{5.038314in}}%
\pgfpathlineto{\pgfqpoint{2.021804in}{5.022836in}}%
\pgfpathlineto{\pgfqpoint{2.037237in}{5.010567in}}%
\pgfpathlineto{\pgfqpoint{2.052671in}{5.001551in}}%
\pgfpathlineto{\pgfqpoint{2.068104in}{4.995790in}}%
\pgfpathlineto{\pgfqpoint{2.083538in}{4.993237in}}%
\pgfpathlineto{\pgfqpoint{2.098972in}{4.993798in}}%
\pgfpathlineto{\pgfqpoint{2.114405in}{4.997329in}}%
\pgfpathlineto{\pgfqpoint{2.129839in}{5.003641in}}%
\pgfpathlineto{\pgfqpoint{2.145273in}{5.012493in}}%
\pgfpathlineto{\pgfqpoint{2.160706in}{5.023602in}}%
\pgfpathlineto{\pgfqpoint{2.191574in}{5.051242in}}%
\pgfpathlineto{\pgfqpoint{2.237874in}{5.100283in}}%
\pgfpathlineto{\pgfqpoint{2.268742in}{5.132815in}}%
\pgfpathlineto{\pgfqpoint{2.299609in}{5.160852in}}%
\pgfpathlineto{\pgfqpoint{2.315043in}{5.172057in}}%
\pgfpathlineto{\pgfqpoint{2.330476in}{5.180834in}}%
\pgfpathlineto{\pgfqpoint{2.345910in}{5.186820in}}%
\pgfpathlineto{\pgfqpoint{2.361344in}{5.189697in}}%
\pgfpathlineto{\pgfqpoint{2.376777in}{5.189207in}}%
\pgfpathlineto{\pgfqpoint{2.392211in}{5.185156in}}%
\pgfpathlineto{\pgfqpoint{2.407644in}{5.177418in}}%
\pgfpathlineto{\pgfqpoint{2.423078in}{5.165945in}}%
\pgfpathlineto{\pgfqpoint{2.438512in}{5.150767in}}%
\pgfpathlineto{\pgfqpoint{2.453945in}{5.131991in}}%
\pgfpathlineto{\pgfqpoint{2.469379in}{5.109806in}}%
\pgfpathlineto{\pgfqpoint{2.500246in}{5.056339in}}%
\pgfpathlineto{\pgfqpoint{2.531114in}{4.993316in}}%
\pgfpathlineto{\pgfqpoint{2.639149in}{4.758186in}}%
\pgfpathlineto{\pgfqpoint{2.670016in}{4.705708in}}%
\pgfpathlineto{\pgfqpoint{2.685450in}{4.684456in}}%
\pgfpathlineto{\pgfqpoint{2.700884in}{4.667009in}}%
\pgfpathlineto{\pgfqpoint{2.716317in}{4.653634in}}%
\pgfpathlineto{\pgfqpoint{2.731751in}{4.644511in}}%
\pgfpathlineto{\pgfqpoint{2.747184in}{4.639732in}}%
\pgfpathlineto{\pgfqpoint{2.762618in}{4.639298in}}%
\pgfpathlineto{\pgfqpoint{2.778052in}{4.643120in}}%
\pgfpathlineto{\pgfqpoint{2.793485in}{4.651018in}}%
\pgfpathlineto{\pgfqpoint{2.808919in}{4.662733in}}%
\pgfpathlineto{\pgfqpoint{2.824353in}{4.677923in}}%
\pgfpathlineto{\pgfqpoint{2.839786in}{4.696180in}}%
\pgfpathlineto{\pgfqpoint{2.870654in}{4.739979in}}%
\pgfpathlineto{\pgfqpoint{2.916954in}{4.815707in}}%
\pgfpathlineto{\pgfqpoint{2.963255in}{4.889763in}}%
\pgfpathlineto{\pgfqpoint{2.994123in}{4.931175in}}%
\pgfpathlineto{\pgfqpoint{3.009556in}{4.948204in}}%
\pgfpathlineto{\pgfqpoint{3.024990in}{4.962355in}}%
\pgfpathlineto{\pgfqpoint{3.040424in}{4.973397in}}%
\pgfpathlineto{\pgfqpoint{3.055857in}{4.981176in}}%
\pgfpathlineto{\pgfqpoint{3.071291in}{4.985617in}}%
\pgfpathlineto{\pgfqpoint{3.086724in}{4.986721in}}%
\pgfpathlineto{\pgfqpoint{3.102158in}{4.984561in}}%
\pgfpathlineto{\pgfqpoint{3.117592in}{4.979283in}}%
\pgfpathlineto{\pgfqpoint{3.133025in}{4.971093in}}%
\pgfpathlineto{\pgfqpoint{3.148459in}{4.960253in}}%
\pgfpathlineto{\pgfqpoint{3.179326in}{4.931895in}}%
\pgfpathlineto{\pgfqpoint{3.210193in}{4.897059in}}%
\pgfpathlineto{\pgfqpoint{3.302795in}{4.784172in}}%
\pgfpathlineto{\pgfqpoint{3.333663in}{4.752545in}}%
\pgfpathlineto{\pgfqpoint{3.364530in}{4.726914in}}%
\pgfpathlineto{\pgfqpoint{3.395397in}{4.708049in}}%
\pgfpathlineto{\pgfqpoint{3.410831in}{4.701218in}}%
\pgfpathlineto{\pgfqpoint{3.426264in}{4.696086in}}%
\pgfpathlineto{\pgfqpoint{3.457132in}{4.690659in}}%
\pgfpathlineto{\pgfqpoint{3.487999in}{4.691084in}}%
\pgfpathlineto{\pgfqpoint{3.518866in}{4.696557in}}%
\pgfpathlineto{\pgfqpoint{3.549733in}{4.706338in}}%
\pgfpathlineto{\pgfqpoint{3.580601in}{4.719883in}}%
\pgfpathlineto{\pgfqpoint{3.611468in}{4.736923in}}%
\pgfpathlineto{\pgfqpoint{3.642335in}{4.757461in}}%
\pgfpathlineto{\pgfqpoint{3.673203in}{4.781709in}}%
\pgfpathlineto{\pgfqpoint{3.704070in}{4.809957in}}%
\pgfpathlineto{\pgfqpoint{3.734937in}{4.842414in}}%
\pgfpathlineto{\pgfqpoint{3.765804in}{4.879039in}}%
\pgfpathlineto{\pgfqpoint{3.812105in}{4.940704in}}%
\pgfpathlineto{\pgfqpoint{3.920141in}{5.092289in}}%
\pgfpathlineto{\pgfqpoint{3.951008in}{5.128413in}}%
\pgfpathlineto{\pgfqpoint{3.981875in}{5.157237in}}%
\pgfpathlineto{\pgfqpoint{3.997309in}{5.168328in}}%
\pgfpathlineto{\pgfqpoint{4.012743in}{5.176965in}}%
\pgfpathlineto{\pgfqpoint{4.028176in}{5.183028in}}%
\pgfpathlineto{\pgfqpoint{4.043610in}{5.186449in}}%
\pgfpathlineto{\pgfqpoint{4.059043in}{5.187218in}}%
\pgfpathlineto{\pgfqpoint{4.074477in}{5.185387in}}%
\pgfpathlineto{\pgfqpoint{4.089911in}{5.181067in}}%
\pgfpathlineto{\pgfqpoint{4.105344in}{5.174431in}}%
\pgfpathlineto{\pgfqpoint{4.120778in}{5.165712in}}%
\pgfpathlineto{\pgfqpoint{4.151645in}{5.143220in}}%
\pgfpathlineto{\pgfqpoint{4.197946in}{5.102463in}}%
\pgfpathlineto{\pgfqpoint{4.228813in}{5.075611in}}%
\pgfpathlineto{\pgfqpoint{4.259681in}{5.053139in}}%
\pgfpathlineto{\pgfqpoint{4.275114in}{5.044568in}}%
\pgfpathlineto{\pgfqpoint{4.290548in}{5.038247in}}%
\pgfpathlineto{\pgfqpoint{4.305982in}{5.034468in}}%
\pgfpathlineto{\pgfqpoint{4.321415in}{5.033464in}}%
\pgfpathlineto{\pgfqpoint{4.336849in}{5.035400in}}%
\pgfpathlineto{\pgfqpoint{4.352283in}{5.040371in}}%
\pgfpathlineto{\pgfqpoint{4.367716in}{5.048398in}}%
\pgfpathlineto{\pgfqpoint{4.383150in}{5.059424in}}%
\pgfpathlineto{\pgfqpoint{4.398583in}{5.073321in}}%
\pgfpathlineto{\pgfqpoint{4.414017in}{5.089889in}}%
\pgfpathlineto{\pgfqpoint{4.444884in}{5.129906in}}%
\pgfpathlineto{\pgfqpoint{4.475752in}{5.176675in}}%
\pgfpathlineto{\pgfqpoint{4.552920in}{5.299767in}}%
\pgfpathlineto{\pgfqpoint{4.583787in}{5.341840in}}%
\pgfpathlineto{\pgfqpoint{4.599221in}{5.359735in}}%
\pgfpathlineto{\pgfqpoint{4.614654in}{5.375135in}}%
\pgfpathlineto{\pgfqpoint{4.630088in}{5.387817in}}%
\pgfpathlineto{\pgfqpoint{4.645522in}{5.397623in}}%
\pgfpathlineto{\pgfqpoint{4.660955in}{5.404466in}}%
\pgfpathlineto{\pgfqpoint{4.660955in}{5.404466in}}%
\pgfusepath{stroke}%
\end{pgfscope}%
\begin{pgfscope}%
\pgfpathrectangle{\pgfqpoint{0.625831in}{3.799602in}}{\pgfqpoint{4.227273in}{2.745455in}} %
\pgfusepath{clip}%
\pgfsetrectcap%
\pgfsetroundjoin%
\pgfsetlinewidth{0.501875pt}%
\definecolor{currentstroke}{rgb}{0.456863,0.997705,0.730653}%
\pgfsetstrokecolor{currentstroke}%
\pgfsetdash{}{0pt}%
\pgfpathmoveto{\pgfqpoint{0.817980in}{5.613079in}}%
\pgfpathlineto{\pgfqpoint{0.833414in}{5.618107in}}%
\pgfpathlineto{\pgfqpoint{0.848847in}{5.618099in}}%
\pgfpathlineto{\pgfqpoint{0.864281in}{5.612857in}}%
\pgfpathlineto{\pgfqpoint{0.879715in}{5.602279in}}%
\pgfpathlineto{\pgfqpoint{0.895148in}{5.586358in}}%
\pgfpathlineto{\pgfqpoint{0.910582in}{5.565188in}}%
\pgfpathlineto{\pgfqpoint{0.926015in}{5.538967in}}%
\pgfpathlineto{\pgfqpoint{0.941449in}{5.507989in}}%
\pgfpathlineto{\pgfqpoint{0.956883in}{5.472645in}}%
\pgfpathlineto{\pgfqpoint{0.987750in}{5.390866in}}%
\pgfpathlineto{\pgfqpoint{1.018617in}{5.298397in}}%
\pgfpathlineto{\pgfqpoint{1.095785in}{5.059166in}}%
\pgfpathlineto{\pgfqpoint{1.126653in}{4.976594in}}%
\pgfpathlineto{\pgfqpoint{1.142086in}{4.941024in}}%
\pgfpathlineto{\pgfqpoint{1.157520in}{4.910076in}}%
\pgfpathlineto{\pgfqpoint{1.172954in}{4.884262in}}%
\pgfpathlineto{\pgfqpoint{1.188387in}{4.864009in}}%
\pgfpathlineto{\pgfqpoint{1.203821in}{4.849645in}}%
\pgfpathlineto{\pgfqpoint{1.219255in}{4.841398in}}%
\pgfpathlineto{\pgfqpoint{1.234688in}{4.839393in}}%
\pgfpathlineto{\pgfqpoint{1.250122in}{4.843649in}}%
\pgfpathlineto{\pgfqpoint{1.265555in}{4.854080in}}%
\pgfpathlineto{\pgfqpoint{1.280989in}{4.870499in}}%
\pgfpathlineto{\pgfqpoint{1.296423in}{4.892623in}}%
\pgfpathlineto{\pgfqpoint{1.311856in}{4.920079in}}%
\pgfpathlineto{\pgfqpoint{1.327290in}{4.952412in}}%
\pgfpathlineto{\pgfqpoint{1.358157in}{5.029539in}}%
\pgfpathlineto{\pgfqpoint{1.389025in}{5.119128in}}%
\pgfpathlineto{\pgfqpoint{1.497060in}{5.451077in}}%
\pgfpathlineto{\pgfqpoint{1.527927in}{5.529528in}}%
\pgfpathlineto{\pgfqpoint{1.543361in}{5.563327in}}%
\pgfpathlineto{\pgfqpoint{1.558795in}{5.593004in}}%
\pgfpathlineto{\pgfqpoint{1.574228in}{5.618268in}}%
\pgfpathlineto{\pgfqpoint{1.589662in}{5.638898in}}%
\pgfpathlineto{\pgfqpoint{1.605095in}{5.654748in}}%
\pgfpathlineto{\pgfqpoint{1.620529in}{5.665737in}}%
\pgfpathlineto{\pgfqpoint{1.635963in}{5.671857in}}%
\pgfpathlineto{\pgfqpoint{1.651396in}{5.673163in}}%
\pgfpathlineto{\pgfqpoint{1.666830in}{5.669771in}}%
\pgfpathlineto{\pgfqpoint{1.682264in}{5.661855in}}%
\pgfpathlineto{\pgfqpoint{1.697697in}{5.649640in}}%
\pgfpathlineto{\pgfqpoint{1.713131in}{5.633397in}}%
\pgfpathlineto{\pgfqpoint{1.728564in}{5.613438in}}%
\pgfpathlineto{\pgfqpoint{1.743998in}{5.590107in}}%
\pgfpathlineto{\pgfqpoint{1.774865in}{5.534851in}}%
\pgfpathlineto{\pgfqpoint{1.805733in}{5.470850in}}%
\pgfpathlineto{\pgfqpoint{1.852034in}{5.365892in}}%
\pgfpathlineto{\pgfqpoint{1.913768in}{5.226662in}}%
\pgfpathlineto{\pgfqpoint{1.944635in}{5.164082in}}%
\pgfpathlineto{\pgfqpoint{1.975503in}{5.109454in}}%
\pgfpathlineto{\pgfqpoint{2.006370in}{5.064572in}}%
\pgfpathlineto{\pgfqpoint{2.021804in}{5.046181in}}%
\pgfpathlineto{\pgfqpoint{2.037237in}{5.030618in}}%
\pgfpathlineto{\pgfqpoint{2.052671in}{5.017928in}}%
\pgfpathlineto{\pgfqpoint{2.068104in}{5.008109in}}%
\pgfpathlineto{\pgfqpoint{2.083538in}{5.001118in}}%
\pgfpathlineto{\pgfqpoint{2.098972in}{4.996868in}}%
\pgfpathlineto{\pgfqpoint{2.114405in}{4.995227in}}%
\pgfpathlineto{\pgfqpoint{2.129839in}{4.996025in}}%
\pgfpathlineto{\pgfqpoint{2.145273in}{4.999051in}}%
\pgfpathlineto{\pgfqpoint{2.160706in}{5.004060in}}%
\pgfpathlineto{\pgfqpoint{2.191574in}{5.018878in}}%
\pgfpathlineto{\pgfqpoint{2.237874in}{5.048168in}}%
\pgfpathlineto{\pgfqpoint{2.284175in}{5.077312in}}%
\pgfpathlineto{\pgfqpoint{2.315043in}{5.091901in}}%
\pgfpathlineto{\pgfqpoint{2.330476in}{5.096782in}}%
\pgfpathlineto{\pgfqpoint{2.345910in}{5.099698in}}%
\pgfpathlineto{\pgfqpoint{2.361344in}{5.100430in}}%
\pgfpathlineto{\pgfqpoint{2.376777in}{5.098808in}}%
\pgfpathlineto{\pgfqpoint{2.392211in}{5.094711in}}%
\pgfpathlineto{\pgfqpoint{2.407644in}{5.088074in}}%
\pgfpathlineto{\pgfqpoint{2.423078in}{5.078887in}}%
\pgfpathlineto{\pgfqpoint{2.438512in}{5.067201in}}%
\pgfpathlineto{\pgfqpoint{2.453945in}{5.053122in}}%
\pgfpathlineto{\pgfqpoint{2.484813in}{5.018497in}}%
\pgfpathlineto{\pgfqpoint{2.515680in}{4.976945in}}%
\pgfpathlineto{\pgfqpoint{2.577414in}{4.884104in}}%
\pgfpathlineto{\pgfqpoint{2.608282in}{4.839288in}}%
\pgfpathlineto{\pgfqpoint{2.639149in}{4.799920in}}%
\pgfpathlineto{\pgfqpoint{2.654583in}{4.783200in}}%
\pgfpathlineto{\pgfqpoint{2.670016in}{4.768877in}}%
\pgfpathlineto{\pgfqpoint{2.685450in}{4.757204in}}%
\pgfpathlineto{\pgfqpoint{2.700884in}{4.748383in}}%
\pgfpathlineto{\pgfqpoint{2.716317in}{4.742553in}}%
\pgfpathlineto{\pgfqpoint{2.731751in}{4.739791in}}%
\pgfpathlineto{\pgfqpoint{2.747184in}{4.740106in}}%
\pgfpathlineto{\pgfqpoint{2.762618in}{4.743445in}}%
\pgfpathlineto{\pgfqpoint{2.778052in}{4.749686in}}%
\pgfpathlineto{\pgfqpoint{2.793485in}{4.758646in}}%
\pgfpathlineto{\pgfqpoint{2.808919in}{4.770082in}}%
\pgfpathlineto{\pgfqpoint{2.839786in}{4.799165in}}%
\pgfpathlineto{\pgfqpoint{2.870654in}{4.834074in}}%
\pgfpathlineto{\pgfqpoint{2.947822in}{4.924740in}}%
\pgfpathlineto{\pgfqpoint{2.978689in}{4.953635in}}%
\pgfpathlineto{\pgfqpoint{2.994123in}{4.965086in}}%
\pgfpathlineto{\pgfqpoint{3.009556in}{4.974193in}}%
\pgfpathlineto{\pgfqpoint{3.024990in}{4.980764in}}%
\pgfpathlineto{\pgfqpoint{3.040424in}{4.984664in}}%
\pgfpathlineto{\pgfqpoint{3.055857in}{4.985818in}}%
\pgfpathlineto{\pgfqpoint{3.071291in}{4.984214in}}%
\pgfpathlineto{\pgfqpoint{3.086724in}{4.979894in}}%
\pgfpathlineto{\pgfqpoint{3.102158in}{4.972960in}}%
\pgfpathlineto{\pgfqpoint{3.117592in}{4.963565in}}%
\pgfpathlineto{\pgfqpoint{3.133025in}{4.951909in}}%
\pgfpathlineto{\pgfqpoint{3.163893in}{4.922814in}}%
\pgfpathlineto{\pgfqpoint{3.194760in}{4.887966in}}%
\pgfpathlineto{\pgfqpoint{3.302795in}{4.757955in}}%
\pgfpathlineto{\pgfqpoint{3.333663in}{4.728026in}}%
\pgfpathlineto{\pgfqpoint{3.364530in}{4.704277in}}%
\pgfpathlineto{\pgfqpoint{3.395397in}{4.687446in}}%
\pgfpathlineto{\pgfqpoint{3.410831in}{4.681712in}}%
\pgfpathlineto{\pgfqpoint{3.426264in}{4.677752in}}%
\pgfpathlineto{\pgfqpoint{3.441698in}{4.675529in}}%
\pgfpathlineto{\pgfqpoint{3.457132in}{4.674987in}}%
\pgfpathlineto{\pgfqpoint{3.487999in}{4.678649in}}%
\pgfpathlineto{\pgfqpoint{3.518866in}{4.688087in}}%
\pgfpathlineto{\pgfqpoint{3.549733in}{4.702631in}}%
\pgfpathlineto{\pgfqpoint{3.580601in}{4.721695in}}%
\pgfpathlineto{\pgfqpoint{3.611468in}{4.744836in}}%
\pgfpathlineto{\pgfqpoint{3.642335in}{4.771762in}}%
\pgfpathlineto{\pgfqpoint{3.673203in}{4.802291in}}%
\pgfpathlineto{\pgfqpoint{3.704070in}{4.836269in}}%
\pgfpathlineto{\pgfqpoint{3.750371in}{4.893148in}}%
\pgfpathlineto{\pgfqpoint{3.796672in}{4.955551in}}%
\pgfpathlineto{\pgfqpoint{3.889273in}{5.082369in}}%
\pgfpathlineto{\pgfqpoint{3.920141in}{5.119346in}}%
\pgfpathlineto{\pgfqpoint{3.951008in}{5.150581in}}%
\pgfpathlineto{\pgfqpoint{3.966442in}{5.163496in}}%
\pgfpathlineto{\pgfqpoint{3.981875in}{5.174354in}}%
\pgfpathlineto{\pgfqpoint{3.997309in}{5.182999in}}%
\pgfpathlineto{\pgfqpoint{4.012743in}{5.189309in}}%
\pgfpathlineto{\pgfqpoint{4.028176in}{5.193201in}}%
\pgfpathlineto{\pgfqpoint{4.043610in}{5.194636in}}%
\pgfpathlineto{\pgfqpoint{4.059043in}{5.193624in}}%
\pgfpathlineto{\pgfqpoint{4.074477in}{5.190221in}}%
\pgfpathlineto{\pgfqpoint{4.089911in}{5.184534in}}%
\pgfpathlineto{\pgfqpoint{4.105344in}{5.176721in}}%
\pgfpathlineto{\pgfqpoint{4.136212in}{5.155578in}}%
\pgfpathlineto{\pgfqpoint{4.167079in}{5.128938in}}%
\pgfpathlineto{\pgfqpoint{4.244247in}{5.056738in}}%
\pgfpathlineto{\pgfqpoint{4.275114in}{5.033615in}}%
\pgfpathlineto{\pgfqpoint{4.290548in}{5.024711in}}%
\pgfpathlineto{\pgfqpoint{4.305982in}{5.017967in}}%
\pgfpathlineto{\pgfqpoint{4.321415in}{5.013619in}}%
\pgfpathlineto{\pgfqpoint{4.336849in}{5.011854in}}%
\pgfpathlineto{\pgfqpoint{4.352283in}{5.012800in}}%
\pgfpathlineto{\pgfqpoint{4.367716in}{5.016529in}}%
\pgfpathlineto{\pgfqpoint{4.383150in}{5.023053in}}%
\pgfpathlineto{\pgfqpoint{4.398583in}{5.032322in}}%
\pgfpathlineto{\pgfqpoint{4.414017in}{5.044226in}}%
\pgfpathlineto{\pgfqpoint{4.429451in}{5.058599in}}%
\pgfpathlineto{\pgfqpoint{4.460318in}{5.093834in}}%
\pgfpathlineto{\pgfqpoint{4.491185in}{5.135749in}}%
\pgfpathlineto{\pgfqpoint{4.599221in}{5.292893in}}%
\pgfpathlineto{\pgfqpoint{4.630088in}{5.328806in}}%
\pgfpathlineto{\pgfqpoint{4.645522in}{5.343798in}}%
\pgfpathlineto{\pgfqpoint{4.660955in}{5.356569in}}%
\pgfpathlineto{\pgfqpoint{4.660955in}{5.356569in}}%
\pgfusepath{stroke}%
\end{pgfscope}%
\begin{pgfscope}%
\pgfpathrectangle{\pgfqpoint{0.625831in}{3.799602in}}{\pgfqpoint{4.227273in}{2.745455in}} %
\pgfusepath{clip}%
\pgfsetrectcap%
\pgfsetroundjoin%
\pgfsetlinewidth{0.501875pt}%
\definecolor{currentstroke}{rgb}{0.543137,0.997705,0.682749}%
\pgfsetstrokecolor{currentstroke}%
\pgfsetdash{}{0pt}%
\pgfpathmoveto{\pgfqpoint{0.817980in}{5.574445in}}%
\pgfpathlineto{\pgfqpoint{0.833414in}{5.579755in}}%
\pgfpathlineto{\pgfqpoint{0.848847in}{5.580677in}}%
\pgfpathlineto{\pgfqpoint{0.864281in}{5.576947in}}%
\pgfpathlineto{\pgfqpoint{0.879715in}{5.568380in}}%
\pgfpathlineto{\pgfqpoint{0.895148in}{5.554880in}}%
\pgfpathlineto{\pgfqpoint{0.910582in}{5.536444in}}%
\pgfpathlineto{\pgfqpoint{0.926015in}{5.513169in}}%
\pgfpathlineto{\pgfqpoint{0.941449in}{5.485248in}}%
\pgfpathlineto{\pgfqpoint{0.956883in}{5.452975in}}%
\pgfpathlineto{\pgfqpoint{0.987750in}{5.377014in}}%
\pgfpathlineto{\pgfqpoint{1.018617in}{5.289417in}}%
\pgfpathlineto{\pgfqpoint{1.111219in}{5.012819in}}%
\pgfpathlineto{\pgfqpoint{1.142086in}{4.936518in}}%
\pgfpathlineto{\pgfqpoint{1.157520in}{4.904594in}}%
\pgfpathlineto{\pgfqpoint{1.172954in}{4.877605in}}%
\pgfpathlineto{\pgfqpoint{1.188387in}{4.856032in}}%
\pgfpathlineto{\pgfqpoint{1.203821in}{4.840267in}}%
\pgfpathlineto{\pgfqpoint{1.219255in}{4.830594in}}%
\pgfpathlineto{\pgfqpoint{1.234688in}{4.827192in}}%
\pgfpathlineto{\pgfqpoint{1.250122in}{4.830127in}}%
\pgfpathlineto{\pgfqpoint{1.265555in}{4.839356in}}%
\pgfpathlineto{\pgfqpoint{1.280989in}{4.854723in}}%
\pgfpathlineto{\pgfqpoint{1.296423in}{4.875965in}}%
\pgfpathlineto{\pgfqpoint{1.311856in}{4.902722in}}%
\pgfpathlineto{\pgfqpoint{1.327290in}{4.934539in}}%
\pgfpathlineto{\pgfqpoint{1.358157in}{5.011139in}}%
\pgfpathlineto{\pgfqpoint{1.389025in}{5.100710in}}%
\pgfpathlineto{\pgfqpoint{1.497060in}{5.432180in}}%
\pgfpathlineto{\pgfqpoint{1.527927in}{5.509203in}}%
\pgfpathlineto{\pgfqpoint{1.543361in}{5.541986in}}%
\pgfpathlineto{\pgfqpoint{1.558795in}{5.570457in}}%
\pgfpathlineto{\pgfqpoint{1.574228in}{5.594349in}}%
\pgfpathlineto{\pgfqpoint{1.589662in}{5.613475in}}%
\pgfpathlineto{\pgfqpoint{1.605095in}{5.627729in}}%
\pgfpathlineto{\pgfqpoint{1.620529in}{5.637087in}}%
\pgfpathlineto{\pgfqpoint{1.635963in}{5.641598in}}%
\pgfpathlineto{\pgfqpoint{1.651396in}{5.641384in}}%
\pgfpathlineto{\pgfqpoint{1.666830in}{5.636629in}}%
\pgfpathlineto{\pgfqpoint{1.682264in}{5.627580in}}%
\pgfpathlineto{\pgfqpoint{1.697697in}{5.614530in}}%
\pgfpathlineto{\pgfqpoint{1.713131in}{5.597817in}}%
\pgfpathlineto{\pgfqpoint{1.728564in}{5.577814in}}%
\pgfpathlineto{\pgfqpoint{1.759432in}{5.529555in}}%
\pgfpathlineto{\pgfqpoint{1.790299in}{5.473125in}}%
\pgfpathlineto{\pgfqpoint{1.852034in}{5.349340in}}%
\pgfpathlineto{\pgfqpoint{1.898334in}{5.259111in}}%
\pgfpathlineto{\pgfqpoint{1.929202in}{5.204973in}}%
\pgfpathlineto{\pgfqpoint{1.960069in}{5.157808in}}%
\pgfpathlineto{\pgfqpoint{1.990936in}{5.118853in}}%
\pgfpathlineto{\pgfqpoint{2.006370in}{5.102690in}}%
\pgfpathlineto{\pgfqpoint{2.021804in}{5.088793in}}%
\pgfpathlineto{\pgfqpoint{2.037237in}{5.077163in}}%
\pgfpathlineto{\pgfqpoint{2.052671in}{5.067770in}}%
\pgfpathlineto{\pgfqpoint{2.068104in}{5.060553in}}%
\pgfpathlineto{\pgfqpoint{2.083538in}{5.055420in}}%
\pgfpathlineto{\pgfqpoint{2.098972in}{5.052252in}}%
\pgfpathlineto{\pgfqpoint{2.114405in}{5.050903in}}%
\pgfpathlineto{\pgfqpoint{2.145273in}{5.052958in}}%
\pgfpathlineto{\pgfqpoint{2.176140in}{5.059963in}}%
\pgfpathlineto{\pgfqpoint{2.222441in}{5.075559in}}%
\pgfpathlineto{\pgfqpoint{2.268742in}{5.091005in}}%
\pgfpathlineto{\pgfqpoint{2.299609in}{5.097798in}}%
\pgfpathlineto{\pgfqpoint{2.330476in}{5.099693in}}%
\pgfpathlineto{\pgfqpoint{2.345910in}{5.098358in}}%
\pgfpathlineto{\pgfqpoint{2.361344in}{5.095327in}}%
\pgfpathlineto{\pgfqpoint{2.376777in}{5.090515in}}%
\pgfpathlineto{\pgfqpoint{2.392211in}{5.083874in}}%
\pgfpathlineto{\pgfqpoint{2.407644in}{5.075403in}}%
\pgfpathlineto{\pgfqpoint{2.438512in}{5.053171in}}%
\pgfpathlineto{\pgfqpoint{2.469379in}{5.024657in}}%
\pgfpathlineto{\pgfqpoint{2.500246in}{4.991352in}}%
\pgfpathlineto{\pgfqpoint{2.592848in}{4.884728in}}%
\pgfpathlineto{\pgfqpoint{2.623715in}{4.855304in}}%
\pgfpathlineto{\pgfqpoint{2.654583in}{4.832787in}}%
\pgfpathlineto{\pgfqpoint{2.670016in}{4.824650in}}%
\pgfpathlineto{\pgfqpoint{2.685450in}{4.818791in}}%
\pgfpathlineto{\pgfqpoint{2.700884in}{4.815296in}}%
\pgfpathlineto{\pgfqpoint{2.716317in}{4.814197in}}%
\pgfpathlineto{\pgfqpoint{2.731751in}{4.815472in}}%
\pgfpathlineto{\pgfqpoint{2.747184in}{4.819044in}}%
\pgfpathlineto{\pgfqpoint{2.762618in}{4.824783in}}%
\pgfpathlineto{\pgfqpoint{2.778052in}{4.832507in}}%
\pgfpathlineto{\pgfqpoint{2.808919in}{4.852961in}}%
\pgfpathlineto{\pgfqpoint{2.839786in}{4.878136in}}%
\pgfpathlineto{\pgfqpoint{2.916954in}{4.943733in}}%
\pgfpathlineto{\pgfqpoint{2.947822in}{4.963893in}}%
\pgfpathlineto{\pgfqpoint{2.963255in}{4.971526in}}%
\pgfpathlineto{\pgfqpoint{2.978689in}{4.977258in}}%
\pgfpathlineto{\pgfqpoint{2.994123in}{4.980951in}}%
\pgfpathlineto{\pgfqpoint{3.009556in}{4.982515in}}%
\pgfpathlineto{\pgfqpoint{3.024990in}{4.981916in}}%
\pgfpathlineto{\pgfqpoint{3.040424in}{4.979174in}}%
\pgfpathlineto{\pgfqpoint{3.055857in}{4.974360in}}%
\pgfpathlineto{\pgfqpoint{3.071291in}{4.967594in}}%
\pgfpathlineto{\pgfqpoint{3.102158in}{4.948902in}}%
\pgfpathlineto{\pgfqpoint{3.133025in}{4.924842in}}%
\pgfpathlineto{\pgfqpoint{3.194760in}{4.869294in}}%
\pgfpathlineto{\pgfqpoint{3.241061in}{4.829748in}}%
\pgfpathlineto{\pgfqpoint{3.271928in}{4.807736in}}%
\pgfpathlineto{\pgfqpoint{3.302795in}{4.790458in}}%
\pgfpathlineto{\pgfqpoint{3.333663in}{4.778288in}}%
\pgfpathlineto{\pgfqpoint{3.364530in}{4.771011in}}%
\pgfpathlineto{\pgfqpoint{3.395397in}{4.767934in}}%
\pgfpathlineto{\pgfqpoint{3.426264in}{4.768061in}}%
\pgfpathlineto{\pgfqpoint{3.472565in}{4.771896in}}%
\pgfpathlineto{\pgfqpoint{3.565167in}{4.783767in}}%
\pgfpathlineto{\pgfqpoint{3.611468in}{4.791924in}}%
\pgfpathlineto{\pgfqpoint{3.642335in}{4.800029in}}%
\pgfpathlineto{\pgfqpoint{3.673203in}{4.811761in}}%
\pgfpathlineto{\pgfqpoint{3.704070in}{4.828407in}}%
\pgfpathlineto{\pgfqpoint{3.734937in}{4.851005in}}%
\pgfpathlineto{\pgfqpoint{3.765804in}{4.880137in}}%
\pgfpathlineto{\pgfqpoint{3.796672in}{4.915754in}}%
\pgfpathlineto{\pgfqpoint{3.827539in}{4.957088in}}%
\pgfpathlineto{\pgfqpoint{3.873840in}{5.026359in}}%
\pgfpathlineto{\pgfqpoint{3.935574in}{5.119909in}}%
\pgfpathlineto{\pgfqpoint{3.966442in}{5.160986in}}%
\pgfpathlineto{\pgfqpoint{3.997309in}{5.194705in}}%
\pgfpathlineto{\pgfqpoint{4.012743in}{5.208117in}}%
\pgfpathlineto{\pgfqpoint{4.028176in}{5.218948in}}%
\pgfpathlineto{\pgfqpoint{4.043610in}{5.227049in}}%
\pgfpathlineto{\pgfqpoint{4.059043in}{5.232331in}}%
\pgfpathlineto{\pgfqpoint{4.074477in}{5.234765in}}%
\pgfpathlineto{\pgfqpoint{4.089911in}{5.234385in}}%
\pgfpathlineto{\pgfqpoint{4.105344in}{5.231286in}}%
\pgfpathlineto{\pgfqpoint{4.120778in}{5.225626in}}%
\pgfpathlineto{\pgfqpoint{4.136212in}{5.217620in}}%
\pgfpathlineto{\pgfqpoint{4.167079in}{5.195690in}}%
\pgfpathlineto{\pgfqpoint{4.197946in}{5.168159in}}%
\pgfpathlineto{\pgfqpoint{4.259681in}{5.109105in}}%
\pgfpathlineto{\pgfqpoint{4.290548in}{5.084146in}}%
\pgfpathlineto{\pgfqpoint{4.305982in}{5.074106in}}%
\pgfpathlineto{\pgfqpoint{4.321415in}{5.066094in}}%
\pgfpathlineto{\pgfqpoint{4.336849in}{5.060348in}}%
\pgfpathlineto{\pgfqpoint{4.352283in}{5.057055in}}%
\pgfpathlineto{\pgfqpoint{4.367716in}{5.056341in}}%
\pgfpathlineto{\pgfqpoint{4.383150in}{5.058271in}}%
\pgfpathlineto{\pgfqpoint{4.398583in}{5.062853in}}%
\pgfpathlineto{\pgfqpoint{4.414017in}{5.070030in}}%
\pgfpathlineto{\pgfqpoint{4.429451in}{5.079691in}}%
\pgfpathlineto{\pgfqpoint{4.444884in}{5.091670in}}%
\pgfpathlineto{\pgfqpoint{4.475752in}{5.121677in}}%
\pgfpathlineto{\pgfqpoint{4.506619in}{5.157861in}}%
\pgfpathlineto{\pgfqpoint{4.614654in}{5.293995in}}%
\pgfpathlineto{\pgfqpoint{4.645522in}{5.324788in}}%
\pgfpathlineto{\pgfqpoint{4.660955in}{5.337533in}}%
\pgfpathlineto{\pgfqpoint{4.660955in}{5.337533in}}%
\pgfusepath{stroke}%
\end{pgfscope}%
\begin{pgfscope}%
\pgfpathrectangle{\pgfqpoint{0.625831in}{3.799602in}}{\pgfqpoint{4.227273in}{2.745455in}} %
\pgfusepath{clip}%
\pgfsetrectcap%
\pgfsetroundjoin%
\pgfsetlinewidth{0.501875pt}%
\definecolor{currentstroke}{rgb}{0.621569,0.981823,0.636474}%
\pgfsetstrokecolor{currentstroke}%
\pgfsetdash{}{0pt}%
\pgfpathmoveto{\pgfqpoint{0.817980in}{5.574324in}}%
\pgfpathlineto{\pgfqpoint{0.833414in}{5.575333in}}%
\pgfpathlineto{\pgfqpoint{0.848847in}{5.571818in}}%
\pgfpathlineto{\pgfqpoint{0.864281in}{5.563604in}}%
\pgfpathlineto{\pgfqpoint{0.879715in}{5.550598in}}%
\pgfpathlineto{\pgfqpoint{0.895148in}{5.532792in}}%
\pgfpathlineto{\pgfqpoint{0.910582in}{5.510266in}}%
\pgfpathlineto{\pgfqpoint{0.926015in}{5.483195in}}%
\pgfpathlineto{\pgfqpoint{0.941449in}{5.451840in}}%
\pgfpathlineto{\pgfqpoint{0.972316in}{5.377766in}}%
\pgfpathlineto{\pgfqpoint{1.003184in}{5.291799in}}%
\pgfpathlineto{\pgfqpoint{1.111219in}{4.972410in}}%
\pgfpathlineto{\pgfqpoint{1.142086in}{4.899382in}}%
\pgfpathlineto{\pgfqpoint{1.157520in}{4.869181in}}%
\pgfpathlineto{\pgfqpoint{1.172954in}{4.843867in}}%
\pgfpathlineto{\pgfqpoint{1.188387in}{4.823849in}}%
\pgfpathlineto{\pgfqpoint{1.203821in}{4.809448in}}%
\pgfpathlineto{\pgfqpoint{1.219255in}{4.800889in}}%
\pgfpathlineto{\pgfqpoint{1.234688in}{4.798300in}}%
\pgfpathlineto{\pgfqpoint{1.250122in}{4.801710in}}%
\pgfpathlineto{\pgfqpoint{1.265555in}{4.811047in}}%
\pgfpathlineto{\pgfqpoint{1.280989in}{4.826144in}}%
\pgfpathlineto{\pgfqpoint{1.296423in}{4.846740in}}%
\pgfpathlineto{\pgfqpoint{1.311856in}{4.872488in}}%
\pgfpathlineto{\pgfqpoint{1.327290in}{4.902964in}}%
\pgfpathlineto{\pgfqpoint{1.358157in}{4.976063in}}%
\pgfpathlineto{\pgfqpoint{1.389025in}{5.061439in}}%
\pgfpathlineto{\pgfqpoint{1.497060in}{5.381598in}}%
\pgfpathlineto{\pgfqpoint{1.527927in}{5.458778in}}%
\pgfpathlineto{\pgfqpoint{1.543361in}{5.492466in}}%
\pgfpathlineto{\pgfqpoint{1.558795in}{5.522421in}}%
\pgfpathlineto{\pgfqpoint{1.574228in}{5.548380in}}%
\pgfpathlineto{\pgfqpoint{1.589662in}{5.570143in}}%
\pgfpathlineto{\pgfqpoint{1.605095in}{5.587582in}}%
\pgfpathlineto{\pgfqpoint{1.620529in}{5.600630in}}%
\pgfpathlineto{\pgfqpoint{1.635963in}{5.609287in}}%
\pgfpathlineto{\pgfqpoint{1.651396in}{5.613609in}}%
\pgfpathlineto{\pgfqpoint{1.666830in}{5.613709in}}%
\pgfpathlineto{\pgfqpoint{1.682264in}{5.609750in}}%
\pgfpathlineto{\pgfqpoint{1.697697in}{5.601938in}}%
\pgfpathlineto{\pgfqpoint{1.713131in}{5.590519in}}%
\pgfpathlineto{\pgfqpoint{1.728564in}{5.575770in}}%
\pgfpathlineto{\pgfqpoint{1.743998in}{5.557997in}}%
\pgfpathlineto{\pgfqpoint{1.774865in}{5.514696in}}%
\pgfpathlineto{\pgfqpoint{1.805733in}{5.463372in}}%
\pgfpathlineto{\pgfqpoint{1.852034in}{5.377543in}}%
\pgfpathlineto{\pgfqpoint{1.913768in}{5.260733in}}%
\pgfpathlineto{\pgfqpoint{1.944635in}{5.206655in}}%
\pgfpathlineto{\pgfqpoint{1.975503in}{5.158032in}}%
\pgfpathlineto{\pgfqpoint{2.006370in}{5.116302in}}%
\pgfpathlineto{\pgfqpoint{2.037237in}{5.082465in}}%
\pgfpathlineto{\pgfqpoint{2.052671in}{5.068691in}}%
\pgfpathlineto{\pgfqpoint{2.068104in}{5.057051in}}%
\pgfpathlineto{\pgfqpoint{2.083538in}{5.047537in}}%
\pgfpathlineto{\pgfqpoint{2.098972in}{5.040107in}}%
\pgfpathlineto{\pgfqpoint{2.114405in}{5.034690in}}%
\pgfpathlineto{\pgfqpoint{2.129839in}{5.031185in}}%
\pgfpathlineto{\pgfqpoint{2.145273in}{5.029461in}}%
\pgfpathlineto{\pgfqpoint{2.176140in}{5.030704in}}%
\pgfpathlineto{\pgfqpoint{2.207007in}{5.036867in}}%
\pgfpathlineto{\pgfqpoint{2.253308in}{5.051201in}}%
\pgfpathlineto{\pgfqpoint{2.299609in}{5.065411in}}%
\pgfpathlineto{\pgfqpoint{2.330476in}{5.071436in}}%
\pgfpathlineto{\pgfqpoint{2.361344in}{5.072695in}}%
\pgfpathlineto{\pgfqpoint{2.376777in}{5.071133in}}%
\pgfpathlineto{\pgfqpoint{2.392211in}{5.067963in}}%
\pgfpathlineto{\pgfqpoint{2.407644in}{5.063127in}}%
\pgfpathlineto{\pgfqpoint{2.438512in}{5.048426in}}%
\pgfpathlineto{\pgfqpoint{2.469379in}{5.027419in}}%
\pgfpathlineto{\pgfqpoint{2.500246in}{5.001180in}}%
\pgfpathlineto{\pgfqpoint{2.546547in}{4.955830in}}%
\pgfpathlineto{\pgfqpoint{2.592848in}{4.910041in}}%
\pgfpathlineto{\pgfqpoint{2.623715in}{4.883334in}}%
\pgfpathlineto{\pgfqpoint{2.654583in}{4.862319in}}%
\pgfpathlineto{\pgfqpoint{2.670016in}{4.854524in}}%
\pgfpathlineto{\pgfqpoint{2.685450in}{4.848773in}}%
\pgfpathlineto{\pgfqpoint{2.700884in}{4.845183in}}%
\pgfpathlineto{\pgfqpoint{2.716317in}{4.843824in}}%
\pgfpathlineto{\pgfqpoint{2.731751in}{4.844715in}}%
\pgfpathlineto{\pgfqpoint{2.747184in}{4.847819in}}%
\pgfpathlineto{\pgfqpoint{2.762618in}{4.853048in}}%
\pgfpathlineto{\pgfqpoint{2.778052in}{4.860263in}}%
\pgfpathlineto{\pgfqpoint{2.808919in}{4.879853in}}%
\pgfpathlineto{\pgfqpoint{2.839786in}{4.904594in}}%
\pgfpathlineto{\pgfqpoint{2.916954in}{4.971993in}}%
\pgfpathlineto{\pgfqpoint{2.947822in}{4.994023in}}%
\pgfpathlineto{\pgfqpoint{2.963255in}{5.002760in}}%
\pgfpathlineto{\pgfqpoint{2.978689in}{5.009661in}}%
\pgfpathlineto{\pgfqpoint{2.994123in}{5.014547in}}%
\pgfpathlineto{\pgfqpoint{3.009556in}{5.017281in}}%
\pgfpathlineto{\pgfqpoint{3.024990in}{5.017780in}}%
\pgfpathlineto{\pgfqpoint{3.040424in}{5.016014in}}%
\pgfpathlineto{\pgfqpoint{3.055857in}{5.012001in}}%
\pgfpathlineto{\pgfqpoint{3.071291in}{5.005814in}}%
\pgfpathlineto{\pgfqpoint{3.086724in}{4.997569in}}%
\pgfpathlineto{\pgfqpoint{3.117592in}{4.975593in}}%
\pgfpathlineto{\pgfqpoint{3.148459in}{4.947792in}}%
\pgfpathlineto{\pgfqpoint{3.194760in}{4.899869in}}%
\pgfpathlineto{\pgfqpoint{3.241061in}{4.851313in}}%
\pgfpathlineto{\pgfqpoint{3.271928in}{4.822038in}}%
\pgfpathlineto{\pgfqpoint{3.302795in}{4.797091in}}%
\pgfpathlineto{\pgfqpoint{3.333663in}{4.777465in}}%
\pgfpathlineto{\pgfqpoint{3.364530in}{4.763602in}}%
\pgfpathlineto{\pgfqpoint{3.395397in}{4.755429in}}%
\pgfpathlineto{\pgfqpoint{3.426264in}{4.752477in}}%
\pgfpathlineto{\pgfqpoint{3.457132in}{4.754020in}}%
\pgfpathlineto{\pgfqpoint{3.487999in}{4.759254in}}%
\pgfpathlineto{\pgfqpoint{3.518866in}{4.767449in}}%
\pgfpathlineto{\pgfqpoint{3.549733in}{4.778086in}}%
\pgfpathlineto{\pgfqpoint{3.580601in}{4.790932in}}%
\pgfpathlineto{\pgfqpoint{3.611468in}{4.806064in}}%
\pgfpathlineto{\pgfqpoint{3.642335in}{4.823821in}}%
\pgfpathlineto{\pgfqpoint{3.673203in}{4.844710in}}%
\pgfpathlineto{\pgfqpoint{3.704070in}{4.869256in}}%
\pgfpathlineto{\pgfqpoint{3.734937in}{4.897850in}}%
\pgfpathlineto{\pgfqpoint{3.765804in}{4.930587in}}%
\pgfpathlineto{\pgfqpoint{3.796672in}{4.967142in}}%
\pgfpathlineto{\pgfqpoint{3.842973in}{5.027200in}}%
\pgfpathlineto{\pgfqpoint{3.904707in}{5.108949in}}%
\pgfpathlineto{\pgfqpoint{3.935574in}{5.146012in}}%
\pgfpathlineto{\pgfqpoint{3.966442in}{5.177639in}}%
\pgfpathlineto{\pgfqpoint{3.981875in}{5.190788in}}%
\pgfpathlineto{\pgfqpoint{3.997309in}{5.201876in}}%
\pgfpathlineto{\pgfqpoint{4.012743in}{5.210730in}}%
\pgfpathlineto{\pgfqpoint{4.028176in}{5.217217in}}%
\pgfpathlineto{\pgfqpoint{4.043610in}{5.221252in}}%
\pgfpathlineto{\pgfqpoint{4.059043in}{5.222797in}}%
\pgfpathlineto{\pgfqpoint{4.074477in}{5.221867in}}%
\pgfpathlineto{\pgfqpoint{4.089911in}{5.218529in}}%
\pgfpathlineto{\pgfqpoint{4.105344in}{5.212905in}}%
\pgfpathlineto{\pgfqpoint{4.120778in}{5.205167in}}%
\pgfpathlineto{\pgfqpoint{4.151645in}{5.184275in}}%
\pgfpathlineto{\pgfqpoint{4.182513in}{5.158120in}}%
\pgfpathlineto{\pgfqpoint{4.259681in}{5.088495in}}%
\pgfpathlineto{\pgfqpoint{4.290548in}{5.066797in}}%
\pgfpathlineto{\pgfqpoint{4.305982in}{5.058608in}}%
\pgfpathlineto{\pgfqpoint{4.321415in}{5.052539in}}%
\pgfpathlineto{\pgfqpoint{4.336849in}{5.048791in}}%
\pgfpathlineto{\pgfqpoint{4.352283in}{5.047513in}}%
\pgfpathlineto{\pgfqpoint{4.367716in}{5.048798in}}%
\pgfpathlineto{\pgfqpoint{4.383150in}{5.052677in}}%
\pgfpathlineto{\pgfqpoint{4.398583in}{5.059128in}}%
\pgfpathlineto{\pgfqpoint{4.414017in}{5.068067in}}%
\pgfpathlineto{\pgfqpoint{4.429451in}{5.079355in}}%
\pgfpathlineto{\pgfqpoint{4.460318in}{5.108174in}}%
\pgfpathlineto{\pgfqpoint{4.491185in}{5.143537in}}%
\pgfpathlineto{\pgfqpoint{4.552920in}{5.223249in}}%
\pgfpathlineto{\pgfqpoint{4.583787in}{5.261811in}}%
\pgfpathlineto{\pgfqpoint{4.614654in}{5.295867in}}%
\pgfpathlineto{\pgfqpoint{4.645522in}{5.323196in}}%
\pgfpathlineto{\pgfqpoint{4.660955in}{5.333812in}}%
\pgfpathlineto{\pgfqpoint{4.660955in}{5.333812in}}%
\pgfusepath{stroke}%
\end{pgfscope}%
\begin{pgfscope}%
\pgfpathrectangle{\pgfqpoint{0.625831in}{3.799602in}}{\pgfqpoint{4.227273in}{2.745455in}} %
\pgfusepath{clip}%
\pgfsetrectcap%
\pgfsetroundjoin%
\pgfsetlinewidth{0.501875pt}%
\definecolor{currentstroke}{rgb}{0.700000,0.951057,0.587785}%
\pgfsetstrokecolor{currentstroke}%
\pgfsetdash{}{0pt}%
\pgfpathmoveto{\pgfqpoint{0.817980in}{5.557174in}}%
\pgfpathlineto{\pgfqpoint{0.833414in}{5.558296in}}%
\pgfpathlineto{\pgfqpoint{0.848847in}{5.555148in}}%
\pgfpathlineto{\pgfqpoint{0.864281in}{5.547558in}}%
\pgfpathlineto{\pgfqpoint{0.879715in}{5.535432in}}%
\pgfpathlineto{\pgfqpoint{0.895148in}{5.518755in}}%
\pgfpathlineto{\pgfqpoint{0.910582in}{5.497597in}}%
\pgfpathlineto{\pgfqpoint{0.926015in}{5.472114in}}%
\pgfpathlineto{\pgfqpoint{0.941449in}{5.442549in}}%
\pgfpathlineto{\pgfqpoint{0.972316in}{5.372559in}}%
\pgfpathlineto{\pgfqpoint{1.003184in}{5.291164in}}%
\pgfpathlineto{\pgfqpoint{1.111219in}{4.988070in}}%
\pgfpathlineto{\pgfqpoint{1.142086in}{4.918815in}}%
\pgfpathlineto{\pgfqpoint{1.157520in}{4.890225in}}%
\pgfpathlineto{\pgfqpoint{1.172954in}{4.866310in}}%
\pgfpathlineto{\pgfqpoint{1.188387in}{4.847461in}}%
\pgfpathlineto{\pgfqpoint{1.203821in}{4.833981in}}%
\pgfpathlineto{\pgfqpoint{1.219255in}{4.826081in}}%
\pgfpathlineto{\pgfqpoint{1.234688in}{4.823875in}}%
\pgfpathlineto{\pgfqpoint{1.250122in}{4.827382in}}%
\pgfpathlineto{\pgfqpoint{1.265555in}{4.836522in}}%
\pgfpathlineto{\pgfqpoint{1.280989in}{4.851121in}}%
\pgfpathlineto{\pgfqpoint{1.296423in}{4.870917in}}%
\pgfpathlineto{\pgfqpoint{1.311856in}{4.895563in}}%
\pgfpathlineto{\pgfqpoint{1.327290in}{4.924640in}}%
\pgfpathlineto{\pgfqpoint{1.358157in}{4.994083in}}%
\pgfpathlineto{\pgfqpoint{1.389025in}{5.074765in}}%
\pgfpathlineto{\pgfqpoint{1.497060in}{5.373582in}}%
\pgfpathlineto{\pgfqpoint{1.527927in}{5.444565in}}%
\pgfpathlineto{\pgfqpoint{1.543361in}{5.475364in}}%
\pgfpathlineto{\pgfqpoint{1.558795in}{5.502629in}}%
\pgfpathlineto{\pgfqpoint{1.574228in}{5.526134in}}%
\pgfpathlineto{\pgfqpoint{1.589662in}{5.545720in}}%
\pgfpathlineto{\pgfqpoint{1.605095in}{5.561286in}}%
\pgfpathlineto{\pgfqpoint{1.620529in}{5.572794in}}%
\pgfpathlineto{\pgfqpoint{1.635963in}{5.580259in}}%
\pgfpathlineto{\pgfqpoint{1.651396in}{5.583752in}}%
\pgfpathlineto{\pgfqpoint{1.666830in}{5.583389in}}%
\pgfpathlineto{\pgfqpoint{1.682264in}{5.579329in}}%
\pgfpathlineto{\pgfqpoint{1.697697in}{5.571768in}}%
\pgfpathlineto{\pgfqpoint{1.713131in}{5.560932in}}%
\pgfpathlineto{\pgfqpoint{1.728564in}{5.547076in}}%
\pgfpathlineto{\pgfqpoint{1.743998in}{5.530473in}}%
\pgfpathlineto{\pgfqpoint{1.774865in}{5.490196in}}%
\pgfpathlineto{\pgfqpoint{1.805733in}{5.442531in}}%
\pgfpathlineto{\pgfqpoint{1.852034in}{5.362603in}}%
\pgfpathlineto{\pgfqpoint{1.929202in}{5.226562in}}%
\pgfpathlineto{\pgfqpoint{1.960069in}{5.177288in}}%
\pgfpathlineto{\pgfqpoint{1.990936in}{5.133561in}}%
\pgfpathlineto{\pgfqpoint{2.021804in}{5.096693in}}%
\pgfpathlineto{\pgfqpoint{2.052671in}{5.067616in}}%
\pgfpathlineto{\pgfqpoint{2.068104in}{5.056176in}}%
\pgfpathlineto{\pgfqpoint{2.083538in}{5.046835in}}%
\pgfpathlineto{\pgfqpoint{2.098972in}{5.039583in}}%
\pgfpathlineto{\pgfqpoint{2.114405in}{5.034377in}}%
\pgfpathlineto{\pgfqpoint{2.129839in}{5.031140in}}%
\pgfpathlineto{\pgfqpoint{2.145273in}{5.029768in}}%
\pgfpathlineto{\pgfqpoint{2.160706in}{5.030121in}}%
\pgfpathlineto{\pgfqpoint{2.191574in}{5.035312in}}%
\pgfpathlineto{\pgfqpoint{2.222441in}{5.045071in}}%
\pgfpathlineto{\pgfqpoint{2.330476in}{5.085479in}}%
\pgfpathlineto{\pgfqpoint{2.361344in}{5.090215in}}%
\pgfpathlineto{\pgfqpoint{2.376777in}{5.090369in}}%
\pgfpathlineto{\pgfqpoint{2.392211in}{5.088866in}}%
\pgfpathlineto{\pgfqpoint{2.407644in}{5.085618in}}%
\pgfpathlineto{\pgfqpoint{2.423078in}{5.080582in}}%
\pgfpathlineto{\pgfqpoint{2.438512in}{5.073758in}}%
\pgfpathlineto{\pgfqpoint{2.469379in}{5.054976in}}%
\pgfpathlineto{\pgfqpoint{2.500246in}{5.030161in}}%
\pgfpathlineto{\pgfqpoint{2.531114in}{5.000837in}}%
\pgfpathlineto{\pgfqpoint{2.623715in}{4.907330in}}%
\pgfpathlineto{\pgfqpoint{2.654583in}{4.882108in}}%
\pgfpathlineto{\pgfqpoint{2.685450in}{4.863249in}}%
\pgfpathlineto{\pgfqpoint{2.700884in}{4.856633in}}%
\pgfpathlineto{\pgfqpoint{2.716317in}{4.852036in}}%
\pgfpathlineto{\pgfqpoint{2.731751in}{4.849505in}}%
\pgfpathlineto{\pgfqpoint{2.747184in}{4.849039in}}%
\pgfpathlineto{\pgfqpoint{2.762618in}{4.850586in}}%
\pgfpathlineto{\pgfqpoint{2.778052in}{4.854049in}}%
\pgfpathlineto{\pgfqpoint{2.808919in}{4.866096in}}%
\pgfpathlineto{\pgfqpoint{2.839786in}{4.883543in}}%
\pgfpathlineto{\pgfqpoint{2.886087in}{4.915060in}}%
\pgfpathlineto{\pgfqpoint{2.932388in}{4.945612in}}%
\pgfpathlineto{\pgfqpoint{2.963255in}{4.961486in}}%
\pgfpathlineto{\pgfqpoint{2.978689in}{4.967323in}}%
\pgfpathlineto{\pgfqpoint{2.994123in}{4.971501in}}%
\pgfpathlineto{\pgfqpoint{3.009556in}{4.973877in}}%
\pgfpathlineto{\pgfqpoint{3.024990in}{4.974352in}}%
\pgfpathlineto{\pgfqpoint{3.040424in}{4.972873in}}%
\pgfpathlineto{\pgfqpoint{3.055857in}{4.969433in}}%
\pgfpathlineto{\pgfqpoint{3.071291in}{4.964070in}}%
\pgfpathlineto{\pgfqpoint{3.086724in}{4.956867in}}%
\pgfpathlineto{\pgfqpoint{3.117592in}{4.937465in}}%
\pgfpathlineto{\pgfqpoint{3.148459in}{4.912619in}}%
\pgfpathlineto{\pgfqpoint{3.194760in}{4.869150in}}%
\pgfpathlineto{\pgfqpoint{3.256494in}{4.810178in}}%
\pgfpathlineto{\pgfqpoint{3.287362in}{4.784382in}}%
\pgfpathlineto{\pgfqpoint{3.318229in}{4.762870in}}%
\pgfpathlineto{\pgfqpoint{3.349096in}{4.746432in}}%
\pgfpathlineto{\pgfqpoint{3.379963in}{4.735378in}}%
\pgfpathlineto{\pgfqpoint{3.410831in}{4.729594in}}%
\pgfpathlineto{\pgfqpoint{3.441698in}{4.728654in}}%
\pgfpathlineto{\pgfqpoint{3.472565in}{4.731956in}}%
\pgfpathlineto{\pgfqpoint{3.503433in}{4.738881in}}%
\pgfpathlineto{\pgfqpoint{3.534300in}{4.748922in}}%
\pgfpathlineto{\pgfqpoint{3.565167in}{4.761789in}}%
\pgfpathlineto{\pgfqpoint{3.596034in}{4.777459in}}%
\pgfpathlineto{\pgfqpoint{3.626902in}{4.796159in}}%
\pgfpathlineto{\pgfqpoint{3.657769in}{4.818303in}}%
\pgfpathlineto{\pgfqpoint{3.688636in}{4.844364in}}%
\pgfpathlineto{\pgfqpoint{3.719503in}{4.874725in}}%
\pgfpathlineto{\pgfqpoint{3.750371in}{4.909521in}}%
\pgfpathlineto{\pgfqpoint{3.781238in}{4.948490in}}%
\pgfpathlineto{\pgfqpoint{3.827539in}{5.012977in}}%
\pgfpathlineto{\pgfqpoint{3.904707in}{5.123423in}}%
\pgfpathlineto{\pgfqpoint{3.935574in}{5.162415in}}%
\pgfpathlineto{\pgfqpoint{3.966442in}{5.194977in}}%
\pgfpathlineto{\pgfqpoint{3.981875in}{5.208197in}}%
\pgfpathlineto{\pgfqpoint{3.997309in}{5.219089in}}%
\pgfpathlineto{\pgfqpoint{4.012743in}{5.227485in}}%
\pgfpathlineto{\pgfqpoint{4.028176in}{5.233263in}}%
\pgfpathlineto{\pgfqpoint{4.043610in}{5.236355in}}%
\pgfpathlineto{\pgfqpoint{4.059043in}{5.236744in}}%
\pgfpathlineto{\pgfqpoint{4.074477in}{5.234474in}}%
\pgfpathlineto{\pgfqpoint{4.089911in}{5.229643in}}%
\pgfpathlineto{\pgfqpoint{4.105344in}{5.222411in}}%
\pgfpathlineto{\pgfqpoint{4.120778in}{5.212986in}}%
\pgfpathlineto{\pgfqpoint{4.151645in}{5.188658in}}%
\pgfpathlineto{\pgfqpoint{4.182513in}{5.159255in}}%
\pgfpathlineto{\pgfqpoint{4.244247in}{5.097743in}}%
\pgfpathlineto{\pgfqpoint{4.275114in}{5.072065in}}%
\pgfpathlineto{\pgfqpoint{4.290548in}{5.061769in}}%
\pgfpathlineto{\pgfqpoint{4.305982in}{5.053557in}}%
\pgfpathlineto{\pgfqpoint{4.321415in}{5.047664in}}%
\pgfpathlineto{\pgfqpoint{4.336849in}{5.044264in}}%
\pgfpathlineto{\pgfqpoint{4.352283in}{5.043475in}}%
\pgfpathlineto{\pgfqpoint{4.367716in}{5.045351in}}%
\pgfpathlineto{\pgfqpoint{4.383150in}{5.049884in}}%
\pgfpathlineto{\pgfqpoint{4.398583in}{5.057005in}}%
\pgfpathlineto{\pgfqpoint{4.414017in}{5.066583in}}%
\pgfpathlineto{\pgfqpoint{4.429451in}{5.078434in}}%
\pgfpathlineto{\pgfqpoint{4.460318in}{5.107967in}}%
\pgfpathlineto{\pgfqpoint{4.491185in}{5.143240in}}%
\pgfpathlineto{\pgfqpoint{4.583787in}{5.254866in}}%
\pgfpathlineto{\pgfqpoint{4.614654in}{5.284765in}}%
\pgfpathlineto{\pgfqpoint{4.630088in}{5.297086in}}%
\pgfpathlineto{\pgfqpoint{4.645522in}{5.307413in}}%
\pgfpathlineto{\pgfqpoint{4.660955in}{5.315622in}}%
\pgfpathlineto{\pgfqpoint{4.660955in}{5.315622in}}%
\pgfusepath{stroke}%
\end{pgfscope}%
\begin{pgfscope}%
\pgfpathrectangle{\pgfqpoint{0.625831in}{3.799602in}}{\pgfqpoint{4.227273in}{2.745455in}} %
\pgfusepath{clip}%
\pgfsetrectcap%
\pgfsetroundjoin%
\pgfsetlinewidth{0.501875pt}%
\definecolor{currentstroke}{rgb}{0.778431,0.905873,0.536867}%
\pgfsetstrokecolor{currentstroke}%
\pgfsetdash{}{0pt}%
\pgfpathmoveto{\pgfqpoint{0.817980in}{5.553562in}}%
\pgfpathlineto{\pgfqpoint{0.833414in}{5.555050in}}%
\pgfpathlineto{\pgfqpoint{0.848847in}{5.552447in}}%
\pgfpathlineto{\pgfqpoint{0.864281in}{5.545580in}}%
\pgfpathlineto{\pgfqpoint{0.879715in}{5.534348in}}%
\pgfpathlineto{\pgfqpoint{0.895148in}{5.518730in}}%
\pgfpathlineto{\pgfqpoint{0.910582in}{5.498789in}}%
\pgfpathlineto{\pgfqpoint{0.926015in}{5.474672in}}%
\pgfpathlineto{\pgfqpoint{0.941449in}{5.446613in}}%
\pgfpathlineto{\pgfqpoint{0.972316in}{5.380008in}}%
\pgfpathlineto{\pgfqpoint{1.003184in}{5.302401in}}%
\pgfpathlineto{\pgfqpoint{1.111219in}{5.013522in}}%
\pgfpathlineto{\pgfqpoint{1.142086in}{4.947776in}}%
\pgfpathlineto{\pgfqpoint{1.157520in}{4.920704in}}%
\pgfpathlineto{\pgfqpoint{1.172954in}{4.898110in}}%
\pgfpathlineto{\pgfqpoint{1.188387in}{4.880356in}}%
\pgfpathlineto{\pgfqpoint{1.203821in}{4.867718in}}%
\pgfpathlineto{\pgfqpoint{1.219255in}{4.860384in}}%
\pgfpathlineto{\pgfqpoint{1.234688in}{4.858449in}}%
\pgfpathlineto{\pgfqpoint{1.250122in}{4.861914in}}%
\pgfpathlineto{\pgfqpoint{1.265555in}{4.870687in}}%
\pgfpathlineto{\pgfqpoint{1.280989in}{4.884591in}}%
\pgfpathlineto{\pgfqpoint{1.296423in}{4.903362in}}%
\pgfpathlineto{\pgfqpoint{1.311856in}{4.926661in}}%
\pgfpathlineto{\pgfqpoint{1.327290in}{4.954082in}}%
\pgfpathlineto{\pgfqpoint{1.358157in}{5.019371in}}%
\pgfpathlineto{\pgfqpoint{1.389025in}{5.094986in}}%
\pgfpathlineto{\pgfqpoint{1.497060in}{5.373990in}}%
\pgfpathlineto{\pgfqpoint{1.527927in}{5.440347in}}%
\pgfpathlineto{\pgfqpoint{1.558795in}{5.494860in}}%
\pgfpathlineto{\pgfqpoint{1.574228in}{5.517056in}}%
\pgfpathlineto{\pgfqpoint{1.589662in}{5.535661in}}%
\pgfpathlineto{\pgfqpoint{1.605095in}{5.550577in}}%
\pgfpathlineto{\pgfqpoint{1.620529in}{5.561760in}}%
\pgfpathlineto{\pgfqpoint{1.635963in}{5.569213in}}%
\pgfpathlineto{\pgfqpoint{1.651396in}{5.572983in}}%
\pgfpathlineto{\pgfqpoint{1.666830in}{5.573157in}}%
\pgfpathlineto{\pgfqpoint{1.682264in}{5.569861in}}%
\pgfpathlineto{\pgfqpoint{1.697697in}{5.563250in}}%
\pgfpathlineto{\pgfqpoint{1.713131in}{5.553508in}}%
\pgfpathlineto{\pgfqpoint{1.728564in}{5.540843in}}%
\pgfpathlineto{\pgfqpoint{1.743998in}{5.525484in}}%
\pgfpathlineto{\pgfqpoint{1.774865in}{5.487676in}}%
\pgfpathlineto{\pgfqpoint{1.805733in}{5.442187in}}%
\pgfpathlineto{\pgfqpoint{1.836600in}{5.391253in}}%
\pgfpathlineto{\pgfqpoint{1.960069in}{5.178528in}}%
\pgfpathlineto{\pgfqpoint{1.990936in}{5.133720in}}%
\pgfpathlineto{\pgfqpoint{2.021804in}{5.095790in}}%
\pgfpathlineto{\pgfqpoint{2.052671in}{5.065894in}}%
\pgfpathlineto{\pgfqpoint{2.068104in}{5.054189in}}%
\pgfpathlineto{\pgfqpoint{2.083538in}{5.044700in}}%
\pgfpathlineto{\pgfqpoint{2.098972in}{5.037423in}}%
\pgfpathlineto{\pgfqpoint{2.114405in}{5.032314in}}%
\pgfpathlineto{\pgfqpoint{2.129839in}{5.029294in}}%
\pgfpathlineto{\pgfqpoint{2.145273in}{5.028243in}}%
\pgfpathlineto{\pgfqpoint{2.160706in}{5.029007in}}%
\pgfpathlineto{\pgfqpoint{2.191574in}{5.035193in}}%
\pgfpathlineto{\pgfqpoint{2.222441in}{5.045988in}}%
\pgfpathlineto{\pgfqpoint{2.315043in}{5.083007in}}%
\pgfpathlineto{\pgfqpoint{2.345910in}{5.089095in}}%
\pgfpathlineto{\pgfqpoint{2.361344in}{5.089843in}}%
\pgfpathlineto{\pgfqpoint{2.376777in}{5.088826in}}%
\pgfpathlineto{\pgfqpoint{2.392211in}{5.085923in}}%
\pgfpathlineto{\pgfqpoint{2.407644in}{5.081056in}}%
\pgfpathlineto{\pgfqpoint{2.423078in}{5.074195in}}%
\pgfpathlineto{\pgfqpoint{2.438512in}{5.065361in}}%
\pgfpathlineto{\pgfqpoint{2.469379in}{5.042093in}}%
\pgfpathlineto{\pgfqpoint{2.500246in}{5.012357in}}%
\pgfpathlineto{\pgfqpoint{2.531114in}{4.977927in}}%
\pgfpathlineto{\pgfqpoint{2.623715in}{4.870517in}}%
\pgfpathlineto{\pgfqpoint{2.654583in}{4.841898in}}%
\pgfpathlineto{\pgfqpoint{2.670016in}{4.830220in}}%
\pgfpathlineto{\pgfqpoint{2.685450in}{4.820578in}}%
\pgfpathlineto{\pgfqpoint{2.700884in}{4.813128in}}%
\pgfpathlineto{\pgfqpoint{2.716317in}{4.807983in}}%
\pgfpathlineto{\pgfqpoint{2.731751in}{4.805202in}}%
\pgfpathlineto{\pgfqpoint{2.747184in}{4.804793in}}%
\pgfpathlineto{\pgfqpoint{2.762618in}{4.806714in}}%
\pgfpathlineto{\pgfqpoint{2.778052in}{4.810870in}}%
\pgfpathlineto{\pgfqpoint{2.793485in}{4.817117in}}%
\pgfpathlineto{\pgfqpoint{2.824353in}{4.835087in}}%
\pgfpathlineto{\pgfqpoint{2.855220in}{4.858642in}}%
\pgfpathlineto{\pgfqpoint{2.947822in}{4.937175in}}%
\pgfpathlineto{\pgfqpoint{2.978689in}{4.957124in}}%
\pgfpathlineto{\pgfqpoint{2.994123in}{4.964631in}}%
\pgfpathlineto{\pgfqpoint{3.009556in}{4.970214in}}%
\pgfpathlineto{\pgfqpoint{3.024990in}{4.973716in}}%
\pgfpathlineto{\pgfqpoint{3.040424in}{4.975029in}}%
\pgfpathlineto{\pgfqpoint{3.055857in}{4.974092in}}%
\pgfpathlineto{\pgfqpoint{3.071291in}{4.970893in}}%
\pgfpathlineto{\pgfqpoint{3.086724in}{4.965472in}}%
\pgfpathlineto{\pgfqpoint{3.102158in}{4.957914in}}%
\pgfpathlineto{\pgfqpoint{3.117592in}{4.948347in}}%
\pgfpathlineto{\pgfqpoint{3.148459in}{4.923909in}}%
\pgfpathlineto{\pgfqpoint{3.179326in}{4.893921in}}%
\pgfpathlineto{\pgfqpoint{3.241061in}{4.826061in}}%
\pgfpathlineto{\pgfqpoint{3.287362in}{4.777257in}}%
\pgfpathlineto{\pgfqpoint{3.318229in}{4.749484in}}%
\pgfpathlineto{\pgfqpoint{3.349096in}{4.727218in}}%
\pgfpathlineto{\pgfqpoint{3.379963in}{4.711321in}}%
\pgfpathlineto{\pgfqpoint{3.395397in}{4.705884in}}%
\pgfpathlineto{\pgfqpoint{3.410831in}{4.702123in}}%
\pgfpathlineto{\pgfqpoint{3.426264in}{4.700006in}}%
\pgfpathlineto{\pgfqpoint{3.457132in}{4.700463in}}%
\pgfpathlineto{\pgfqpoint{3.487999in}{4.706609in}}%
\pgfpathlineto{\pgfqpoint{3.518866in}{4.717650in}}%
\pgfpathlineto{\pgfqpoint{3.549733in}{4.732787in}}%
\pgfpathlineto{\pgfqpoint{3.580601in}{4.751343in}}%
\pgfpathlineto{\pgfqpoint{3.611468in}{4.772838in}}%
\pgfpathlineto{\pgfqpoint{3.642335in}{4.797017in}}%
\pgfpathlineto{\pgfqpoint{3.688636in}{4.838222in}}%
\pgfpathlineto{\pgfqpoint{3.734937in}{4.885497in}}%
\pgfpathlineto{\pgfqpoint{3.781238in}{4.938616in}}%
\pgfpathlineto{\pgfqpoint{3.842973in}{5.015641in}}%
\pgfpathlineto{\pgfqpoint{3.904707in}{5.091462in}}%
\pgfpathlineto{\pgfqpoint{3.935574in}{5.124847in}}%
\pgfpathlineto{\pgfqpoint{3.966442in}{5.152819in}}%
\pgfpathlineto{\pgfqpoint{3.981875in}{5.164260in}}%
\pgfpathlineto{\pgfqpoint{3.997309in}{5.173774in}}%
\pgfpathlineto{\pgfqpoint{4.012743in}{5.181228in}}%
\pgfpathlineto{\pgfqpoint{4.028176in}{5.186524in}}%
\pgfpathlineto{\pgfqpoint{4.043610in}{5.189609in}}%
\pgfpathlineto{\pgfqpoint{4.059043in}{5.190474in}}%
\pgfpathlineto{\pgfqpoint{4.074477in}{5.189159in}}%
\pgfpathlineto{\pgfqpoint{4.089911in}{5.185754in}}%
\pgfpathlineto{\pgfqpoint{4.105344in}{5.180395in}}%
\pgfpathlineto{\pgfqpoint{4.136212in}{5.164602in}}%
\pgfpathlineto{\pgfqpoint{4.167079in}{5.143749in}}%
\pgfpathlineto{\pgfqpoint{4.244247in}{5.086451in}}%
\pgfpathlineto{\pgfqpoint{4.275114in}{5.068672in}}%
\pgfpathlineto{\pgfqpoint{4.290548in}{5.062120in}}%
\pgfpathlineto{\pgfqpoint{4.305982in}{5.057443in}}%
\pgfpathlineto{\pgfqpoint{4.321415in}{5.054820in}}%
\pgfpathlineto{\pgfqpoint{4.336849in}{5.054380in}}%
\pgfpathlineto{\pgfqpoint{4.352283in}{5.056196in}}%
\pgfpathlineto{\pgfqpoint{4.367716in}{5.060286in}}%
\pgfpathlineto{\pgfqpoint{4.383150in}{5.066609in}}%
\pgfpathlineto{\pgfqpoint{4.398583in}{5.075073in}}%
\pgfpathlineto{\pgfqpoint{4.429451in}{5.097787in}}%
\pgfpathlineto{\pgfqpoint{4.460318in}{5.126717in}}%
\pgfpathlineto{\pgfqpoint{4.506619in}{5.176704in}}%
\pgfpathlineto{\pgfqpoint{4.552920in}{5.226785in}}%
\pgfpathlineto{\pgfqpoint{4.583787in}{5.256071in}}%
\pgfpathlineto{\pgfqpoint{4.614654in}{5.279723in}}%
\pgfpathlineto{\pgfqpoint{4.630088in}{5.288991in}}%
\pgfpathlineto{\pgfqpoint{4.645522in}{5.296401in}}%
\pgfpathlineto{\pgfqpoint{4.660955in}{5.301898in}}%
\pgfpathlineto{\pgfqpoint{4.660955in}{5.301898in}}%
\pgfusepath{stroke}%
\end{pgfscope}%
\begin{pgfscope}%
\pgfpathrectangle{\pgfqpoint{0.625831in}{3.799602in}}{\pgfqpoint{4.227273in}{2.745455in}} %
\pgfusepath{clip}%
\pgfsetrectcap%
\pgfsetroundjoin%
\pgfsetlinewidth{0.501875pt}%
\definecolor{currentstroke}{rgb}{0.864706,0.840344,0.478512}%
\pgfsetstrokecolor{currentstroke}%
\pgfsetdash{}{0pt}%
\pgfpathmoveto{\pgfqpoint{0.817980in}{5.550183in}}%
\pgfpathlineto{\pgfqpoint{0.833414in}{5.552638in}}%
\pgfpathlineto{\pgfqpoint{0.848847in}{5.551217in}}%
\pgfpathlineto{\pgfqpoint{0.864281in}{5.545700in}}%
\pgfpathlineto{\pgfqpoint{0.879715in}{5.535938in}}%
\pgfpathlineto{\pgfqpoint{0.895148in}{5.521860in}}%
\pgfpathlineto{\pgfqpoint{0.910582in}{5.503480in}}%
\pgfpathlineto{\pgfqpoint{0.926015in}{5.480899in}}%
\pgfpathlineto{\pgfqpoint{0.941449in}{5.454305in}}%
\pgfpathlineto{\pgfqpoint{0.972316in}{5.390274in}}%
\pgfpathlineto{\pgfqpoint{1.003184in}{5.314576in}}%
\pgfpathlineto{\pgfqpoint{1.049485in}{5.188834in}}%
\pgfpathlineto{\pgfqpoint{1.095785in}{5.064304in}}%
\pgfpathlineto{\pgfqpoint{1.126653in}{4.991117in}}%
\pgfpathlineto{\pgfqpoint{1.142086in}{4.959392in}}%
\pgfpathlineto{\pgfqpoint{1.157520in}{4.931654in}}%
\pgfpathlineto{\pgfqpoint{1.172954in}{4.908373in}}%
\pgfpathlineto{\pgfqpoint{1.188387in}{4.889938in}}%
\pgfpathlineto{\pgfqpoint{1.203821in}{4.876653in}}%
\pgfpathlineto{\pgfqpoint{1.219255in}{4.868728in}}%
\pgfpathlineto{\pgfqpoint{1.234688in}{4.866278in}}%
\pgfpathlineto{\pgfqpoint{1.250122in}{4.869318in}}%
\pgfpathlineto{\pgfqpoint{1.265555in}{4.877769in}}%
\pgfpathlineto{\pgfqpoint{1.280989in}{4.891455in}}%
\pgfpathlineto{\pgfqpoint{1.296423in}{4.910116in}}%
\pgfpathlineto{\pgfqpoint{1.311856in}{4.933406in}}%
\pgfpathlineto{\pgfqpoint{1.327290in}{4.960910in}}%
\pgfpathlineto{\pgfqpoint{1.358157in}{5.026588in}}%
\pgfpathlineto{\pgfqpoint{1.389025in}{5.102765in}}%
\pgfpathlineto{\pgfqpoint{1.497060in}{5.382853in}}%
\pgfpathlineto{\pgfqpoint{1.527927in}{5.448835in}}%
\pgfpathlineto{\pgfqpoint{1.543361in}{5.477400in}}%
\pgfpathlineto{\pgfqpoint{1.558795in}{5.502662in}}%
\pgfpathlineto{\pgfqpoint{1.574228in}{5.524433in}}%
\pgfpathlineto{\pgfqpoint{1.589662in}{5.542582in}}%
\pgfpathlineto{\pgfqpoint{1.605095in}{5.557035in}}%
\pgfpathlineto{\pgfqpoint{1.620529in}{5.567774in}}%
\pgfpathlineto{\pgfqpoint{1.635963in}{5.574825in}}%
\pgfpathlineto{\pgfqpoint{1.651396in}{5.578261in}}%
\pgfpathlineto{\pgfqpoint{1.666830in}{5.578194in}}%
\pgfpathlineto{\pgfqpoint{1.682264in}{5.574769in}}%
\pgfpathlineto{\pgfqpoint{1.697697in}{5.568160in}}%
\pgfpathlineto{\pgfqpoint{1.713131in}{5.558564in}}%
\pgfpathlineto{\pgfqpoint{1.728564in}{5.546197in}}%
\pgfpathlineto{\pgfqpoint{1.743998in}{5.531288in}}%
\pgfpathlineto{\pgfqpoint{1.774865in}{5.494812in}}%
\pgfpathlineto{\pgfqpoint{1.805733in}{5.451126in}}%
\pgfpathlineto{\pgfqpoint{1.836600in}{5.402258in}}%
\pgfpathlineto{\pgfqpoint{1.913768in}{5.270521in}}%
\pgfpathlineto{\pgfqpoint{1.960069in}{5.194486in}}%
\pgfpathlineto{\pgfqpoint{1.990936in}{5.148736in}}%
\pgfpathlineto{\pgfqpoint{2.021804in}{5.108716in}}%
\pgfpathlineto{\pgfqpoint{2.052671in}{5.075651in}}%
\pgfpathlineto{\pgfqpoint{2.083538in}{5.050396in}}%
\pgfpathlineto{\pgfqpoint{2.098972in}{5.040841in}}%
\pgfpathlineto{\pgfqpoint{2.114405in}{5.033346in}}%
\pgfpathlineto{\pgfqpoint{2.129839in}{5.027876in}}%
\pgfpathlineto{\pgfqpoint{2.145273in}{5.024361in}}%
\pgfpathlineto{\pgfqpoint{2.160706in}{5.022693in}}%
\pgfpathlineto{\pgfqpoint{2.191574in}{5.024295in}}%
\pgfpathlineto{\pgfqpoint{2.222441in}{5.031152in}}%
\pgfpathlineto{\pgfqpoint{2.268742in}{5.046899in}}%
\pgfpathlineto{\pgfqpoint{2.315043in}{5.062331in}}%
\pgfpathlineto{\pgfqpoint{2.345910in}{5.068665in}}%
\pgfpathlineto{\pgfqpoint{2.376777in}{5.069603in}}%
\pgfpathlineto{\pgfqpoint{2.392211in}{5.067607in}}%
\pgfpathlineto{\pgfqpoint{2.407644in}{5.063815in}}%
\pgfpathlineto{\pgfqpoint{2.423078in}{5.058170in}}%
\pgfpathlineto{\pgfqpoint{2.438512in}{5.050663in}}%
\pgfpathlineto{\pgfqpoint{2.469379in}{5.030276in}}%
\pgfpathlineto{\pgfqpoint{2.500246in}{5.003555in}}%
\pgfpathlineto{\pgfqpoint{2.531114in}{4.972098in}}%
\pgfpathlineto{\pgfqpoint{2.623715in}{4.872173in}}%
\pgfpathlineto{\pgfqpoint{2.654583in}{4.845337in}}%
\pgfpathlineto{\pgfqpoint{2.670016in}{4.834395in}}%
\pgfpathlineto{\pgfqpoint{2.685450in}{4.825380in}}%
\pgfpathlineto{\pgfqpoint{2.700884in}{4.818449in}}%
\pgfpathlineto{\pgfqpoint{2.716317in}{4.813709in}}%
\pgfpathlineto{\pgfqpoint{2.731751in}{4.811217in}}%
\pgfpathlineto{\pgfqpoint{2.747184in}{4.810980in}}%
\pgfpathlineto{\pgfqpoint{2.762618in}{4.812953in}}%
\pgfpathlineto{\pgfqpoint{2.778052in}{4.817039in}}%
\pgfpathlineto{\pgfqpoint{2.793485in}{4.823092in}}%
\pgfpathlineto{\pgfqpoint{2.824353in}{4.840298in}}%
\pgfpathlineto{\pgfqpoint{2.855220in}{4.862588in}}%
\pgfpathlineto{\pgfqpoint{2.947822in}{4.934621in}}%
\pgfpathlineto{\pgfqpoint{2.978689in}{4.951744in}}%
\pgfpathlineto{\pgfqpoint{2.994123in}{4.957779in}}%
\pgfpathlineto{\pgfqpoint{3.009556in}{4.961878in}}%
\pgfpathlineto{\pgfqpoint{3.024990in}{4.963905in}}%
\pgfpathlineto{\pgfqpoint{3.040424in}{4.963774in}}%
\pgfpathlineto{\pgfqpoint{3.055857in}{4.961449in}}%
\pgfpathlineto{\pgfqpoint{3.071291in}{4.956943in}}%
\pgfpathlineto{\pgfqpoint{3.086724in}{4.950320in}}%
\pgfpathlineto{\pgfqpoint{3.102158in}{4.941689in}}%
\pgfpathlineto{\pgfqpoint{3.133025in}{4.919057in}}%
\pgfpathlineto{\pgfqpoint{3.163893in}{4.890714in}}%
\pgfpathlineto{\pgfqpoint{3.210193in}{4.842167in}}%
\pgfpathlineto{\pgfqpoint{3.256494in}{4.793356in}}%
\pgfpathlineto{\pgfqpoint{3.287362in}{4.764285in}}%
\pgfpathlineto{\pgfqpoint{3.318229in}{4.739992in}}%
\pgfpathlineto{\pgfqpoint{3.349096in}{4.721601in}}%
\pgfpathlineto{\pgfqpoint{3.379963in}{4.709645in}}%
\pgfpathlineto{\pgfqpoint{3.410831in}{4.704090in}}%
\pgfpathlineto{\pgfqpoint{3.441698in}{4.704435in}}%
\pgfpathlineto{\pgfqpoint{3.472565in}{4.709857in}}%
\pgfpathlineto{\pgfqpoint{3.503433in}{4.719391in}}%
\pgfpathlineto{\pgfqpoint{3.534300in}{4.732108in}}%
\pgfpathlineto{\pgfqpoint{3.580601in}{4.755627in}}%
\pgfpathlineto{\pgfqpoint{3.626902in}{4.783539in}}%
\pgfpathlineto{\pgfqpoint{3.673203in}{4.816233in}}%
\pgfpathlineto{\pgfqpoint{3.704070in}{4.841344in}}%
\pgfpathlineto{\pgfqpoint{3.734937in}{4.869586in}}%
\pgfpathlineto{\pgfqpoint{3.765804in}{4.901151in}}%
\pgfpathlineto{\pgfqpoint{3.812105in}{4.954227in}}%
\pgfpathlineto{\pgfqpoint{3.935574in}{5.103009in}}%
\pgfpathlineto{\pgfqpoint{3.966442in}{5.132312in}}%
\pgfpathlineto{\pgfqpoint{3.997309in}{5.154749in}}%
\pgfpathlineto{\pgfqpoint{4.012743in}{5.162976in}}%
\pgfpathlineto{\pgfqpoint{4.028176in}{5.169067in}}%
\pgfpathlineto{\pgfqpoint{4.043610in}{5.172974in}}%
\pgfpathlineto{\pgfqpoint{4.059043in}{5.174699in}}%
\pgfpathlineto{\pgfqpoint{4.074477in}{5.174299in}}%
\pgfpathlineto{\pgfqpoint{4.089911in}{5.171881in}}%
\pgfpathlineto{\pgfqpoint{4.105344in}{5.167606in}}%
\pgfpathlineto{\pgfqpoint{4.136212in}{5.154360in}}%
\pgfpathlineto{\pgfqpoint{4.167079in}{5.136701in}}%
\pgfpathlineto{\pgfqpoint{4.228813in}{5.098784in}}%
\pgfpathlineto{\pgfqpoint{4.259681in}{5.084016in}}%
\pgfpathlineto{\pgfqpoint{4.275114in}{5.078750in}}%
\pgfpathlineto{\pgfqpoint{4.290548in}{5.075215in}}%
\pgfpathlineto{\pgfqpoint{4.305982in}{5.073590in}}%
\pgfpathlineto{\pgfqpoint{4.321415in}{5.074001in}}%
\pgfpathlineto{\pgfqpoint{4.336849in}{5.076517in}}%
\pgfpathlineto{\pgfqpoint{4.352283in}{5.081153in}}%
\pgfpathlineto{\pgfqpoint{4.367716in}{5.087861in}}%
\pgfpathlineto{\pgfqpoint{4.383150in}{5.096540in}}%
\pgfpathlineto{\pgfqpoint{4.414017in}{5.119148in}}%
\pgfpathlineto{\pgfqpoint{4.444884in}{5.147203in}}%
\pgfpathlineto{\pgfqpoint{4.552920in}{5.253143in}}%
\pgfpathlineto{\pgfqpoint{4.583787in}{5.275621in}}%
\pgfpathlineto{\pgfqpoint{4.599221in}{5.284393in}}%
\pgfpathlineto{\pgfqpoint{4.614654in}{5.291354in}}%
\pgfpathlineto{\pgfqpoint{4.630088in}{5.296437in}}%
\pgfpathlineto{\pgfqpoint{4.645522in}{5.299624in}}%
\pgfpathlineto{\pgfqpoint{4.660955in}{5.300944in}}%
\pgfpathlineto{\pgfqpoint{4.660955in}{5.300944in}}%
\pgfusepath{stroke}%
\end{pgfscope}%
\begin{pgfscope}%
\pgfpathrectangle{\pgfqpoint{0.625831in}{3.799602in}}{\pgfqpoint{4.227273in}{2.745455in}} %
\pgfusepath{clip}%
\pgfsetrectcap%
\pgfsetroundjoin%
\pgfsetlinewidth{0.501875pt}%
\definecolor{currentstroke}{rgb}{0.943137,0.767363,0.423549}%
\pgfsetstrokecolor{currentstroke}%
\pgfsetdash{}{0pt}%
\pgfpathmoveto{\pgfqpoint{0.817980in}{5.534574in}}%
\pgfpathlineto{\pgfqpoint{0.833414in}{5.535864in}}%
\pgfpathlineto{\pgfqpoint{0.848847in}{5.533466in}}%
\pgfpathlineto{\pgfqpoint{0.864281in}{5.527188in}}%
\pgfpathlineto{\pgfqpoint{0.879715in}{5.516905in}}%
\pgfpathlineto{\pgfqpoint{0.895148in}{5.502564in}}%
\pgfpathlineto{\pgfqpoint{0.910582in}{5.484188in}}%
\pgfpathlineto{\pgfqpoint{0.926015in}{5.461883in}}%
\pgfpathlineto{\pgfqpoint{0.941449in}{5.435835in}}%
\pgfpathlineto{\pgfqpoint{0.972316in}{5.373658in}}%
\pgfpathlineto{\pgfqpoint{1.003184in}{5.300696in}}%
\pgfpathlineto{\pgfqpoint{1.064918in}{5.139494in}}%
\pgfpathlineto{\pgfqpoint{1.095785in}{5.061295in}}%
\pgfpathlineto{\pgfqpoint{1.126653in}{4.991588in}}%
\pgfpathlineto{\pgfqpoint{1.142086in}{4.961411in}}%
\pgfpathlineto{\pgfqpoint{1.157520in}{4.935052in}}%
\pgfpathlineto{\pgfqpoint{1.172954in}{4.912958in}}%
\pgfpathlineto{\pgfqpoint{1.188387in}{4.895502in}}%
\pgfpathlineto{\pgfqpoint{1.203821in}{4.882973in}}%
\pgfpathlineto{\pgfqpoint{1.219255in}{4.875574in}}%
\pgfpathlineto{\pgfqpoint{1.234688in}{4.873417in}}%
\pgfpathlineto{\pgfqpoint{1.250122in}{4.876523in}}%
\pgfpathlineto{\pgfqpoint{1.265555in}{4.884820in}}%
\pgfpathlineto{\pgfqpoint{1.280989in}{4.898148in}}%
\pgfpathlineto{\pgfqpoint{1.296423in}{4.916261in}}%
\pgfpathlineto{\pgfqpoint{1.311856in}{4.938838in}}%
\pgfpathlineto{\pgfqpoint{1.327290in}{4.965485in}}%
\pgfpathlineto{\pgfqpoint{1.358157in}{5.029128in}}%
\pgfpathlineto{\pgfqpoint{1.389025in}{5.103027in}}%
\pgfpathlineto{\pgfqpoint{1.497060in}{5.376164in}}%
\pgfpathlineto{\pgfqpoint{1.527927in}{5.441051in}}%
\pgfpathlineto{\pgfqpoint{1.558795in}{5.494300in}}%
\pgfpathlineto{\pgfqpoint{1.574228in}{5.515971in}}%
\pgfpathlineto{\pgfqpoint{1.589662in}{5.534139in}}%
\pgfpathlineto{\pgfqpoint{1.605095in}{5.548723in}}%
\pgfpathlineto{\pgfqpoint{1.620529in}{5.559691in}}%
\pgfpathlineto{\pgfqpoint{1.635963in}{5.567060in}}%
\pgfpathlineto{\pgfqpoint{1.651396in}{5.570888in}}%
\pgfpathlineto{\pgfqpoint{1.666830in}{5.571275in}}%
\pgfpathlineto{\pgfqpoint{1.682264in}{5.568350in}}%
\pgfpathlineto{\pgfqpoint{1.697697in}{5.562273in}}%
\pgfpathlineto{\pgfqpoint{1.713131in}{5.553226in}}%
\pgfpathlineto{\pgfqpoint{1.728564in}{5.541413in}}%
\pgfpathlineto{\pgfqpoint{1.743998in}{5.527050in}}%
\pgfpathlineto{\pgfqpoint{1.774865in}{5.491595in}}%
\pgfpathlineto{\pgfqpoint{1.805733in}{5.448775in}}%
\pgfpathlineto{\pgfqpoint{1.836600in}{5.400560in}}%
\pgfpathlineto{\pgfqpoint{1.898334in}{5.295853in}}%
\pgfpathlineto{\pgfqpoint{1.960069in}{5.192955in}}%
\pgfpathlineto{\pgfqpoint{1.990936in}{5.146718in}}%
\pgfpathlineto{\pgfqpoint{2.021804in}{5.106082in}}%
\pgfpathlineto{\pgfqpoint{2.052671in}{5.072330in}}%
\pgfpathlineto{\pgfqpoint{2.083538in}{5.046377in}}%
\pgfpathlineto{\pgfqpoint{2.098972in}{5.036487in}}%
\pgfpathlineto{\pgfqpoint{2.114405in}{5.028678in}}%
\pgfpathlineto{\pgfqpoint{2.129839in}{5.022920in}}%
\pgfpathlineto{\pgfqpoint{2.145273in}{5.019151in}}%
\pgfpathlineto{\pgfqpoint{2.160706in}{5.017268in}}%
\pgfpathlineto{\pgfqpoint{2.176140in}{5.017135in}}%
\pgfpathlineto{\pgfqpoint{2.207007in}{5.021396in}}%
\pgfpathlineto{\pgfqpoint{2.237874in}{5.030191in}}%
\pgfpathlineto{\pgfqpoint{2.330476in}{5.061588in}}%
\pgfpathlineto{\pgfqpoint{2.361344in}{5.066107in}}%
\pgfpathlineto{\pgfqpoint{2.376777in}{5.066149in}}%
\pgfpathlineto{\pgfqpoint{2.392211in}{5.064488in}}%
\pgfpathlineto{\pgfqpoint{2.407644in}{5.061008in}}%
\pgfpathlineto{\pgfqpoint{2.423078in}{5.055641in}}%
\pgfpathlineto{\pgfqpoint{2.438512in}{5.048365in}}%
\pgfpathlineto{\pgfqpoint{2.453945in}{5.039211in}}%
\pgfpathlineto{\pgfqpoint{2.484813in}{5.015636in}}%
\pgfpathlineto{\pgfqpoint{2.515680in}{4.986113in}}%
\pgfpathlineto{\pgfqpoint{2.561981in}{4.934878in}}%
\pgfpathlineto{\pgfqpoint{2.608282in}{4.882612in}}%
\pgfpathlineto{\pgfqpoint{2.639149in}{4.851525in}}%
\pgfpathlineto{\pgfqpoint{2.670016in}{4.826214in}}%
\pgfpathlineto{\pgfqpoint{2.685450in}{4.816327in}}%
\pgfpathlineto{\pgfqpoint{2.700884in}{4.808530in}}%
\pgfpathlineto{\pgfqpoint{2.716317in}{4.802947in}}%
\pgfpathlineto{\pgfqpoint{2.731751in}{4.799653in}}%
\pgfpathlineto{\pgfqpoint{2.747184in}{4.798671in}}%
\pgfpathlineto{\pgfqpoint{2.762618in}{4.799971in}}%
\pgfpathlineto{\pgfqpoint{2.778052in}{4.803472in}}%
\pgfpathlineto{\pgfqpoint{2.793485in}{4.809042in}}%
\pgfpathlineto{\pgfqpoint{2.824353in}{4.825630in}}%
\pgfpathlineto{\pgfqpoint{2.855220in}{4.847829in}}%
\pgfpathlineto{\pgfqpoint{2.963255in}{4.933305in}}%
\pgfpathlineto{\pgfqpoint{2.994123in}{4.949437in}}%
\pgfpathlineto{\pgfqpoint{3.009556in}{4.954826in}}%
\pgfpathlineto{\pgfqpoint{3.024990in}{4.958217in}}%
\pgfpathlineto{\pgfqpoint{3.040424in}{4.959510in}}%
\pgfpathlineto{\pgfqpoint{3.055857in}{4.958651in}}%
\pgfpathlineto{\pgfqpoint{3.071291in}{4.955637in}}%
\pgfpathlineto{\pgfqpoint{3.086724in}{4.950515in}}%
\pgfpathlineto{\pgfqpoint{3.102158in}{4.943377in}}%
\pgfpathlineto{\pgfqpoint{3.133025in}{4.923640in}}%
\pgfpathlineto{\pgfqpoint{3.163893in}{4.897960in}}%
\pgfpathlineto{\pgfqpoint{3.210193in}{4.852727in}}%
\pgfpathlineto{\pgfqpoint{3.271928in}{4.791609in}}%
\pgfpathlineto{\pgfqpoint{3.302795in}{4.765322in}}%
\pgfpathlineto{\pgfqpoint{3.333663in}{4.743940in}}%
\pgfpathlineto{\pgfqpoint{3.364530in}{4.728354in}}%
\pgfpathlineto{\pgfqpoint{3.395397in}{4.718891in}}%
\pgfpathlineto{\pgfqpoint{3.426264in}{4.715369in}}%
\pgfpathlineto{\pgfqpoint{3.457132in}{4.717205in}}%
\pgfpathlineto{\pgfqpoint{3.487999in}{4.723567in}}%
\pgfpathlineto{\pgfqpoint{3.518866in}{4.733544in}}%
\pgfpathlineto{\pgfqpoint{3.549733in}{4.746308in}}%
\pgfpathlineto{\pgfqpoint{3.596034in}{4.769404in}}%
\pgfpathlineto{\pgfqpoint{3.642335in}{4.796577in}}%
\pgfpathlineto{\pgfqpoint{3.688636in}{4.828300in}}%
\pgfpathlineto{\pgfqpoint{3.719503in}{4.852541in}}%
\pgfpathlineto{\pgfqpoint{3.750371in}{4.879592in}}%
\pgfpathlineto{\pgfqpoint{3.796672in}{4.925437in}}%
\pgfpathlineto{\pgfqpoint{3.858406in}{4.993666in}}%
\pgfpathlineto{\pgfqpoint{3.920141in}{5.061436in}}%
\pgfpathlineto{\pgfqpoint{3.951008in}{5.091121in}}%
\pgfpathlineto{\pgfqpoint{3.981875in}{5.115817in}}%
\pgfpathlineto{\pgfqpoint{4.012743in}{5.134225in}}%
\pgfpathlineto{\pgfqpoint{4.028176in}{5.140799in}}%
\pgfpathlineto{\pgfqpoint{4.043610in}{5.145553in}}%
\pgfpathlineto{\pgfqpoint{4.059043in}{5.148490in}}%
\pgfpathlineto{\pgfqpoint{4.074477in}{5.149658in}}%
\pgfpathlineto{\pgfqpoint{4.105344in}{5.147109in}}%
\pgfpathlineto{\pgfqpoint{4.136212in}{5.139162in}}%
\pgfpathlineto{\pgfqpoint{4.182513in}{5.121232in}}%
\pgfpathlineto{\pgfqpoint{4.228813in}{5.102897in}}%
\pgfpathlineto{\pgfqpoint{4.259681in}{5.094242in}}%
\pgfpathlineto{\pgfqpoint{4.290548in}{5.090650in}}%
\pgfpathlineto{\pgfqpoint{4.305982in}{5.091169in}}%
\pgfpathlineto{\pgfqpoint{4.321415in}{5.093364in}}%
\pgfpathlineto{\pgfqpoint{4.336849in}{5.097274in}}%
\pgfpathlineto{\pgfqpoint{4.352283in}{5.102889in}}%
\pgfpathlineto{\pgfqpoint{4.383150in}{5.118940in}}%
\pgfpathlineto{\pgfqpoint{4.414017in}{5.140481in}}%
\pgfpathlineto{\pgfqpoint{4.460318in}{5.179344in}}%
\pgfpathlineto{\pgfqpoint{4.522052in}{5.232106in}}%
\pgfpathlineto{\pgfqpoint{4.552920in}{5.254071in}}%
\pgfpathlineto{\pgfqpoint{4.583787in}{5.270816in}}%
\pgfpathlineto{\pgfqpoint{4.614654in}{5.281324in}}%
\pgfpathlineto{\pgfqpoint{4.630088in}{5.284105in}}%
\pgfpathlineto{\pgfqpoint{4.645522in}{5.285249in}}%
\pgfpathlineto{\pgfqpoint{4.660955in}{5.284817in}}%
\pgfpathlineto{\pgfqpoint{4.660955in}{5.284817in}}%
\pgfusepath{stroke}%
\end{pgfscope}%
\begin{pgfscope}%
\pgfpathrectangle{\pgfqpoint{0.625831in}{3.799602in}}{\pgfqpoint{4.227273in}{2.745455in}} %
\pgfusepath{clip}%
\pgfsetrectcap%
\pgfsetroundjoin%
\pgfsetlinewidth{0.501875pt}%
\definecolor{currentstroke}{rgb}{1.000000,0.682749,0.366979}%
\pgfsetstrokecolor{currentstroke}%
\pgfsetdash{}{0pt}%
\pgfpathmoveto{\pgfqpoint{0.817980in}{5.536734in}}%
\pgfpathlineto{\pgfqpoint{0.833414in}{5.536977in}}%
\pgfpathlineto{\pgfqpoint{0.848847in}{5.533539in}}%
\pgfpathlineto{\pgfqpoint{0.864281in}{5.526254in}}%
\pgfpathlineto{\pgfqpoint{0.879715in}{5.515022in}}%
\pgfpathlineto{\pgfqpoint{0.895148in}{5.499812in}}%
\pgfpathlineto{\pgfqpoint{0.910582in}{5.480670in}}%
\pgfpathlineto{\pgfqpoint{0.926015in}{5.457718in}}%
\pgfpathlineto{\pgfqpoint{0.941449in}{5.431155in}}%
\pgfpathlineto{\pgfqpoint{0.972316in}{5.368369in}}%
\pgfpathlineto{\pgfqpoint{1.003184in}{5.295349in}}%
\pgfpathlineto{\pgfqpoint{1.111219in}{5.022013in}}%
\pgfpathlineto{\pgfqpoint{1.142086in}{4.958882in}}%
\pgfpathlineto{\pgfqpoint{1.157520in}{4.932644in}}%
\pgfpathlineto{\pgfqpoint{1.172954in}{4.910556in}}%
\pgfpathlineto{\pgfqpoint{1.188387in}{4.892980in}}%
\pgfpathlineto{\pgfqpoint{1.203821in}{4.880207in}}%
\pgfpathlineto{\pgfqpoint{1.219255in}{4.872444in}}%
\pgfpathlineto{\pgfqpoint{1.234688in}{4.869816in}}%
\pgfpathlineto{\pgfqpoint{1.250122in}{4.872362in}}%
\pgfpathlineto{\pgfqpoint{1.265555in}{4.880035in}}%
\pgfpathlineto{\pgfqpoint{1.280989in}{4.892703in}}%
\pgfpathlineto{\pgfqpoint{1.296423in}{4.910155in}}%
\pgfpathlineto{\pgfqpoint{1.311856in}{4.932102in}}%
\pgfpathlineto{\pgfqpoint{1.327290in}{4.958187in}}%
\pgfpathlineto{\pgfqpoint{1.358157in}{5.021043in}}%
\pgfpathlineto{\pgfqpoint{1.389025in}{5.094792in}}%
\pgfpathlineto{\pgfqpoint{1.450759in}{5.256931in}}%
\pgfpathlineto{\pgfqpoint{1.497060in}{5.373440in}}%
\pgfpathlineto{\pgfqpoint{1.527927in}{5.441108in}}%
\pgfpathlineto{\pgfqpoint{1.558795in}{5.497151in}}%
\pgfpathlineto{\pgfqpoint{1.574228in}{5.520114in}}%
\pgfpathlineto{\pgfqpoint{1.589662in}{5.539449in}}%
\pgfpathlineto{\pgfqpoint{1.605095in}{5.555037in}}%
\pgfpathlineto{\pgfqpoint{1.620529in}{5.566815in}}%
\pgfpathlineto{\pgfqpoint{1.635963in}{5.574772in}}%
\pgfpathlineto{\pgfqpoint{1.651396in}{5.578952in}}%
\pgfpathlineto{\pgfqpoint{1.666830in}{5.579440in}}%
\pgfpathlineto{\pgfqpoint{1.682264in}{5.576366in}}%
\pgfpathlineto{\pgfqpoint{1.697697in}{5.569896in}}%
\pgfpathlineto{\pgfqpoint{1.713131in}{5.560226in}}%
\pgfpathlineto{\pgfqpoint{1.728564in}{5.547580in}}%
\pgfpathlineto{\pgfqpoint{1.743998in}{5.532202in}}%
\pgfpathlineto{\pgfqpoint{1.774865in}{5.494310in}}%
\pgfpathlineto{\pgfqpoint{1.805733in}{5.448764in}}%
\pgfpathlineto{\pgfqpoint{1.852034in}{5.371119in}}%
\pgfpathlineto{\pgfqpoint{1.944635in}{5.209831in}}%
\pgfpathlineto{\pgfqpoint{1.975503in}{5.161747in}}%
\pgfpathlineto{\pgfqpoint{2.006370in}{5.119380in}}%
\pgfpathlineto{\pgfqpoint{2.037237in}{5.084046in}}%
\pgfpathlineto{\pgfqpoint{2.068104in}{5.056683in}}%
\pgfpathlineto{\pgfqpoint{2.083538in}{5.046160in}}%
\pgfpathlineto{\pgfqpoint{2.098972in}{5.037772in}}%
\pgfpathlineto{\pgfqpoint{2.114405in}{5.031494in}}%
\pgfpathlineto{\pgfqpoint{2.129839in}{5.027268in}}%
\pgfpathlineto{\pgfqpoint{2.145273in}{5.024999in}}%
\pgfpathlineto{\pgfqpoint{2.160706in}{5.024556in}}%
\pgfpathlineto{\pgfqpoint{2.191574in}{5.028450in}}%
\pgfpathlineto{\pgfqpoint{2.222441in}{5.037251in}}%
\pgfpathlineto{\pgfqpoint{2.315043in}{5.071025in}}%
\pgfpathlineto{\pgfqpoint{2.345910in}{5.077014in}}%
\pgfpathlineto{\pgfqpoint{2.361344in}{5.077876in}}%
\pgfpathlineto{\pgfqpoint{2.376777in}{5.077073in}}%
\pgfpathlineto{\pgfqpoint{2.392211in}{5.074474in}}%
\pgfpathlineto{\pgfqpoint{2.407644in}{5.069991in}}%
\pgfpathlineto{\pgfqpoint{2.423078in}{5.063587in}}%
\pgfpathlineto{\pgfqpoint{2.438512in}{5.055271in}}%
\pgfpathlineto{\pgfqpoint{2.469379in}{5.033194in}}%
\pgfpathlineto{\pgfqpoint{2.500246in}{5.004794in}}%
\pgfpathlineto{\pgfqpoint{2.531114in}{4.971762in}}%
\pgfpathlineto{\pgfqpoint{2.623715in}{4.867937in}}%
\pgfpathlineto{\pgfqpoint{2.654583in}{4.839914in}}%
\pgfpathlineto{\pgfqpoint{2.685450in}{4.818696in}}%
\pgfpathlineto{\pgfqpoint{2.700884in}{4.811087in}}%
\pgfpathlineto{\pgfqpoint{2.716317in}{4.805633in}}%
\pgfpathlineto{\pgfqpoint{2.731751in}{4.802394in}}%
\pgfpathlineto{\pgfqpoint{2.747184in}{4.801383in}}%
\pgfpathlineto{\pgfqpoint{2.762618in}{4.802563in}}%
\pgfpathlineto{\pgfqpoint{2.778052in}{4.805855in}}%
\pgfpathlineto{\pgfqpoint{2.793485in}{4.811130in}}%
\pgfpathlineto{\pgfqpoint{2.824353in}{4.826929in}}%
\pgfpathlineto{\pgfqpoint{2.855220in}{4.848193in}}%
\pgfpathlineto{\pgfqpoint{2.916954in}{4.898033in}}%
\pgfpathlineto{\pgfqpoint{2.947822in}{4.921649in}}%
\pgfpathlineto{\pgfqpoint{2.978689in}{4.941238in}}%
\pgfpathlineto{\pgfqpoint{3.009556in}{4.954848in}}%
\pgfpathlineto{\pgfqpoint{3.024990in}{4.958946in}}%
\pgfpathlineto{\pgfqpoint{3.040424in}{4.961070in}}%
\pgfpathlineto{\pgfqpoint{3.055857in}{4.961146in}}%
\pgfpathlineto{\pgfqpoint{3.071291in}{4.959148in}}%
\pgfpathlineto{\pgfqpoint{3.086724in}{4.955091in}}%
\pgfpathlineto{\pgfqpoint{3.102158in}{4.949038in}}%
\pgfpathlineto{\pgfqpoint{3.117592in}{4.941091in}}%
\pgfpathlineto{\pgfqpoint{3.148459in}{4.920123in}}%
\pgfpathlineto{\pgfqpoint{3.179326in}{4.893719in}}%
\pgfpathlineto{\pgfqpoint{3.225627in}{4.848188in}}%
\pgfpathlineto{\pgfqpoint{3.287362in}{4.787465in}}%
\pgfpathlineto{\pgfqpoint{3.318229in}{4.761379in}}%
\pgfpathlineto{\pgfqpoint{3.349096in}{4.740027in}}%
\pgfpathlineto{\pgfqpoint{3.379963in}{4.724224in}}%
\pgfpathlineto{\pgfqpoint{3.410831in}{4.714288in}}%
\pgfpathlineto{\pgfqpoint{3.441698in}{4.710101in}}%
\pgfpathlineto{\pgfqpoint{3.472565in}{4.711222in}}%
\pgfpathlineto{\pgfqpoint{3.503433in}{4.717025in}}%
\pgfpathlineto{\pgfqpoint{3.534300in}{4.726842in}}%
\pgfpathlineto{\pgfqpoint{3.565167in}{4.740092in}}%
\pgfpathlineto{\pgfqpoint{3.596034in}{4.756365in}}%
\pgfpathlineto{\pgfqpoint{3.626902in}{4.775449in}}%
\pgfpathlineto{\pgfqpoint{3.657769in}{4.797316in}}%
\pgfpathlineto{\pgfqpoint{3.688636in}{4.822040in}}%
\pgfpathlineto{\pgfqpoint{3.719503in}{4.849692in}}%
\pgfpathlineto{\pgfqpoint{3.765804in}{4.896489in}}%
\pgfpathlineto{\pgfqpoint{3.812105in}{4.948458in}}%
\pgfpathlineto{\pgfqpoint{3.904707in}{5.054888in}}%
\pgfpathlineto{\pgfqpoint{3.935574in}{5.085946in}}%
\pgfpathlineto{\pgfqpoint{3.966442in}{5.112291in}}%
\pgfpathlineto{\pgfqpoint{3.997309in}{5.132703in}}%
\pgfpathlineto{\pgfqpoint{4.012743in}{5.140415in}}%
\pgfpathlineto{\pgfqpoint{4.028176in}{5.146385in}}%
\pgfpathlineto{\pgfqpoint{4.043610in}{5.150600in}}%
\pgfpathlineto{\pgfqpoint{4.059043in}{5.153085in}}%
\pgfpathlineto{\pgfqpoint{4.074477in}{5.153904in}}%
\pgfpathlineto{\pgfqpoint{4.105344in}{5.150992in}}%
\pgfpathlineto{\pgfqpoint{4.136212in}{5.143110in}}%
\pgfpathlineto{\pgfqpoint{4.182513in}{5.125808in}}%
\pgfpathlineto{\pgfqpoint{4.228813in}{5.108110in}}%
\pgfpathlineto{\pgfqpoint{4.259681in}{5.099498in}}%
\pgfpathlineto{\pgfqpoint{4.290548in}{5.095420in}}%
\pgfpathlineto{\pgfqpoint{4.321415in}{5.097021in}}%
\pgfpathlineto{\pgfqpoint{4.352283in}{5.104806in}}%
\pgfpathlineto{\pgfqpoint{4.383150in}{5.118596in}}%
\pgfpathlineto{\pgfqpoint{4.414017in}{5.137550in}}%
\pgfpathlineto{\pgfqpoint{4.444884in}{5.160270in}}%
\pgfpathlineto{\pgfqpoint{4.537486in}{5.232356in}}%
\pgfpathlineto{\pgfqpoint{4.568353in}{5.251407in}}%
\pgfpathlineto{\pgfqpoint{4.599221in}{5.265513in}}%
\pgfpathlineto{\pgfqpoint{4.630088in}{5.273940in}}%
\pgfpathlineto{\pgfqpoint{4.660955in}{5.276547in}}%
\pgfpathlineto{\pgfqpoint{4.660955in}{5.276547in}}%
\pgfusepath{stroke}%
\end{pgfscope}%
\begin{pgfscope}%
\pgfpathrectangle{\pgfqpoint{0.625831in}{3.799602in}}{\pgfqpoint{4.227273in}{2.745455in}} %
\pgfusepath{clip}%
\pgfsetrectcap%
\pgfsetroundjoin%
\pgfsetlinewidth{0.501875pt}%
\definecolor{currentstroke}{rgb}{1.000000,0.587785,0.309017}%
\pgfsetstrokecolor{currentstroke}%
\pgfsetdash{}{0pt}%
\pgfpathmoveto{\pgfqpoint{0.817980in}{5.540228in}}%
\pgfpathlineto{\pgfqpoint{0.833414in}{5.539991in}}%
\pgfpathlineto{\pgfqpoint{0.848847in}{5.535897in}}%
\pgfpathlineto{\pgfqpoint{0.864281in}{5.527803in}}%
\pgfpathlineto{\pgfqpoint{0.879715in}{5.515636in}}%
\pgfpathlineto{\pgfqpoint{0.895148in}{5.499391in}}%
\pgfpathlineto{\pgfqpoint{0.910582in}{5.479141in}}%
\pgfpathlineto{\pgfqpoint{0.926015in}{5.455033in}}%
\pgfpathlineto{\pgfqpoint{0.941449in}{5.427291in}}%
\pgfpathlineto{\pgfqpoint{0.972316in}{5.362160in}}%
\pgfpathlineto{\pgfqpoint{1.003184in}{5.286919in}}%
\pgfpathlineto{\pgfqpoint{1.111219in}{5.007434in}}%
\pgfpathlineto{\pgfqpoint{1.142086in}{4.942925in}}%
\pgfpathlineto{\pgfqpoint{1.157520in}{4.916036in}}%
\pgfpathlineto{\pgfqpoint{1.172954in}{4.893317in}}%
\pgfpathlineto{\pgfqpoint{1.188387in}{4.875130in}}%
\pgfpathlineto{\pgfqpoint{1.203821in}{4.861767in}}%
\pgfpathlineto{\pgfqpoint{1.219255in}{4.853442in}}%
\pgfpathlineto{\pgfqpoint{1.234688in}{4.850285in}}%
\pgfpathlineto{\pgfqpoint{1.250122in}{4.852347in}}%
\pgfpathlineto{\pgfqpoint{1.265555in}{4.859590in}}%
\pgfpathlineto{\pgfqpoint{1.280989in}{4.871898in}}%
\pgfpathlineto{\pgfqpoint{1.296423in}{4.889073in}}%
\pgfpathlineto{\pgfqpoint{1.311856in}{4.910842in}}%
\pgfpathlineto{\pgfqpoint{1.327290in}{4.936863in}}%
\pgfpathlineto{\pgfqpoint{1.342724in}{4.966730in}}%
\pgfpathlineto{\pgfqpoint{1.373591in}{5.036114in}}%
\pgfpathlineto{\pgfqpoint{1.404458in}{5.114814in}}%
\pgfpathlineto{\pgfqpoint{1.497060in}{5.360554in}}%
\pgfpathlineto{\pgfqpoint{1.527927in}{5.430990in}}%
\pgfpathlineto{\pgfqpoint{1.558795in}{5.489662in}}%
\pgfpathlineto{\pgfqpoint{1.574228in}{5.513797in}}%
\pgfpathlineto{\pgfqpoint{1.589662in}{5.534160in}}%
\pgfpathlineto{\pgfqpoint{1.605095in}{5.550599in}}%
\pgfpathlineto{\pgfqpoint{1.620529in}{5.563020in}}%
\pgfpathlineto{\pgfqpoint{1.635963in}{5.571389in}}%
\pgfpathlineto{\pgfqpoint{1.651396in}{5.575724in}}%
\pgfpathlineto{\pgfqpoint{1.666830in}{5.576098in}}%
\pgfpathlineto{\pgfqpoint{1.682264in}{5.572631in}}%
\pgfpathlineto{\pgfqpoint{1.697697in}{5.565486in}}%
\pgfpathlineto{\pgfqpoint{1.713131in}{5.554866in}}%
\pgfpathlineto{\pgfqpoint{1.728564in}{5.541005in}}%
\pgfpathlineto{\pgfqpoint{1.743998in}{5.524169in}}%
\pgfpathlineto{\pgfqpoint{1.774865in}{5.482733in}}%
\pgfpathlineto{\pgfqpoint{1.805733in}{5.433048in}}%
\pgfpathlineto{\pgfqpoint{1.852034in}{5.348840in}}%
\pgfpathlineto{\pgfqpoint{1.929202in}{5.204506in}}%
\pgfpathlineto{\pgfqpoint{1.960069in}{5.152386in}}%
\pgfpathlineto{\pgfqpoint{1.990936in}{5.106518in}}%
\pgfpathlineto{\pgfqpoint{2.021804in}{5.068447in}}%
\pgfpathlineto{\pgfqpoint{2.037237in}{5.052697in}}%
\pgfpathlineto{\pgfqpoint{2.052671in}{5.039269in}}%
\pgfpathlineto{\pgfqpoint{2.068104in}{5.028216in}}%
\pgfpathlineto{\pgfqpoint{2.083538in}{5.019556in}}%
\pgfpathlineto{\pgfqpoint{2.098972in}{5.013270in}}%
\pgfpathlineto{\pgfqpoint{2.114405in}{5.009299in}}%
\pgfpathlineto{\pgfqpoint{2.129839in}{5.007545in}}%
\pgfpathlineto{\pgfqpoint{2.145273in}{5.007871in}}%
\pgfpathlineto{\pgfqpoint{2.160706in}{5.010103in}}%
\pgfpathlineto{\pgfqpoint{2.191574in}{5.019415in}}%
\pgfpathlineto{\pgfqpoint{2.222441in}{5.033445in}}%
\pgfpathlineto{\pgfqpoint{2.299609in}{5.072916in}}%
\pgfpathlineto{\pgfqpoint{2.330476in}{5.083759in}}%
\pgfpathlineto{\pgfqpoint{2.345910in}{5.086975in}}%
\pgfpathlineto{\pgfqpoint{2.361344in}{5.088425in}}%
\pgfpathlineto{\pgfqpoint{2.376777in}{5.087936in}}%
\pgfpathlineto{\pgfqpoint{2.392211in}{5.085382in}}%
\pgfpathlineto{\pgfqpoint{2.407644in}{5.080687in}}%
\pgfpathlineto{\pgfqpoint{2.423078in}{5.073825in}}%
\pgfpathlineto{\pgfqpoint{2.438512in}{5.064822in}}%
\pgfpathlineto{\pgfqpoint{2.453945in}{5.053756in}}%
\pgfpathlineto{\pgfqpoint{2.484813in}{5.025997in}}%
\pgfpathlineto{\pgfqpoint{2.515680in}{4.992116in}}%
\pgfpathlineto{\pgfqpoint{2.561981in}{4.934678in}}%
\pgfpathlineto{\pgfqpoint{2.608282in}{4.877201in}}%
\pgfpathlineto{\pgfqpoint{2.639149in}{4.843335in}}%
\pgfpathlineto{\pgfqpoint{2.670016in}{4.815798in}}%
\pgfpathlineto{\pgfqpoint{2.685450in}{4.804992in}}%
\pgfpathlineto{\pgfqpoint{2.700884in}{4.796397in}}%
\pgfpathlineto{\pgfqpoint{2.716317in}{4.790130in}}%
\pgfpathlineto{\pgfqpoint{2.731751in}{4.786265in}}%
\pgfpathlineto{\pgfqpoint{2.747184in}{4.784822in}}%
\pgfpathlineto{\pgfqpoint{2.762618in}{4.785774in}}%
\pgfpathlineto{\pgfqpoint{2.778052in}{4.789044in}}%
\pgfpathlineto{\pgfqpoint{2.793485in}{4.794509in}}%
\pgfpathlineto{\pgfqpoint{2.808919in}{4.802002in}}%
\pgfpathlineto{\pgfqpoint{2.839786in}{4.822203in}}%
\pgfpathlineto{\pgfqpoint{2.870654in}{4.847587in}}%
\pgfpathlineto{\pgfqpoint{2.963255in}{4.929609in}}%
\pgfpathlineto{\pgfqpoint{2.994123in}{4.950433in}}%
\pgfpathlineto{\pgfqpoint{3.009556in}{4.958397in}}%
\pgfpathlineto{\pgfqpoint{3.024990in}{4.964465in}}%
\pgfpathlineto{\pgfqpoint{3.040424in}{4.968492in}}%
\pgfpathlineto{\pgfqpoint{3.055857in}{4.970376in}}%
\pgfpathlineto{\pgfqpoint{3.071291in}{4.970060in}}%
\pgfpathlineto{\pgfqpoint{3.086724in}{4.967532in}}%
\pgfpathlineto{\pgfqpoint{3.102158in}{4.962830in}}%
\pgfpathlineto{\pgfqpoint{3.117592in}{4.956030in}}%
\pgfpathlineto{\pgfqpoint{3.133025in}{4.947256in}}%
\pgfpathlineto{\pgfqpoint{3.163893in}{4.924454in}}%
\pgfpathlineto{\pgfqpoint{3.194760in}{4.896077in}}%
\pgfpathlineto{\pgfqpoint{3.241061in}{4.847532in}}%
\pgfpathlineto{\pgfqpoint{3.302795in}{4.783033in}}%
\pgfpathlineto{\pgfqpoint{3.333663in}{4.755268in}}%
\pgfpathlineto{\pgfqpoint{3.364530in}{4.732444in}}%
\pgfpathlineto{\pgfqpoint{3.395397in}{4.715444in}}%
\pgfpathlineto{\pgfqpoint{3.426264in}{4.704671in}}%
\pgfpathlineto{\pgfqpoint{3.457132in}{4.700113in}}%
\pgfpathlineto{\pgfqpoint{3.487999in}{4.701433in}}%
\pgfpathlineto{\pgfqpoint{3.518866in}{4.708097in}}%
\pgfpathlineto{\pgfqpoint{3.549733in}{4.719497in}}%
\pgfpathlineto{\pgfqpoint{3.580601in}{4.735048in}}%
\pgfpathlineto{\pgfqpoint{3.611468in}{4.754259in}}%
\pgfpathlineto{\pgfqpoint{3.642335in}{4.776758in}}%
\pgfpathlineto{\pgfqpoint{3.673203in}{4.802264in}}%
\pgfpathlineto{\pgfqpoint{3.719503in}{4.845628in}}%
\pgfpathlineto{\pgfqpoint{3.765804in}{4.894125in}}%
\pgfpathlineto{\pgfqpoint{3.904707in}{5.046238in}}%
\pgfpathlineto{\pgfqpoint{3.935574in}{5.074276in}}%
\pgfpathlineto{\pgfqpoint{3.966442in}{5.097586in}}%
\pgfpathlineto{\pgfqpoint{3.997309in}{5.115264in}}%
\pgfpathlineto{\pgfqpoint{4.028176in}{5.126783in}}%
\pgfpathlineto{\pgfqpoint{4.059043in}{5.132093in}}%
\pgfpathlineto{\pgfqpoint{4.089911in}{5.131655in}}%
\pgfpathlineto{\pgfqpoint{4.120778in}{5.126432in}}%
\pgfpathlineto{\pgfqpoint{4.151645in}{5.117812in}}%
\pgfpathlineto{\pgfqpoint{4.228813in}{5.092864in}}%
\pgfpathlineto{\pgfqpoint{4.259681in}{5.086236in}}%
\pgfpathlineto{\pgfqpoint{4.290548in}{5.083805in}}%
\pgfpathlineto{\pgfqpoint{4.321415in}{5.086530in}}%
\pgfpathlineto{\pgfqpoint{4.352283in}{5.094815in}}%
\pgfpathlineto{\pgfqpoint{4.383150in}{5.108469in}}%
\pgfpathlineto{\pgfqpoint{4.414017in}{5.126724in}}%
\pgfpathlineto{\pgfqpoint{4.460318in}{5.159879in}}%
\pgfpathlineto{\pgfqpoint{4.522052in}{5.205963in}}%
\pgfpathlineto{\pgfqpoint{4.552920in}{5.225854in}}%
\pgfpathlineto{\pgfqpoint{4.583787in}{5.241655in}}%
\pgfpathlineto{\pgfqpoint{4.614654in}{5.252371in}}%
\pgfpathlineto{\pgfqpoint{4.645522in}{5.257511in}}%
\pgfpathlineto{\pgfqpoint{4.660955in}{5.257986in}}%
\pgfpathlineto{\pgfqpoint{4.660955in}{5.257986in}}%
\pgfusepath{stroke}%
\end{pgfscope}%
\begin{pgfscope}%
\pgfpathrectangle{\pgfqpoint{0.625831in}{3.799602in}}{\pgfqpoint{4.227273in}{2.745455in}} %
\pgfusepath{clip}%
\pgfsetrectcap%
\pgfsetroundjoin%
\pgfsetlinewidth{0.501875pt}%
\definecolor{currentstroke}{rgb}{1.000000,0.473094,0.243914}%
\pgfsetstrokecolor{currentstroke}%
\pgfsetdash{}{0pt}%
\pgfpathmoveto{\pgfqpoint{0.817980in}{5.541071in}}%
\pgfpathlineto{\pgfqpoint{0.833414in}{5.539837in}}%
\pgfpathlineto{\pgfqpoint{0.848847in}{5.534796in}}%
\pgfpathlineto{\pgfqpoint{0.864281in}{5.525822in}}%
\pgfpathlineto{\pgfqpoint{0.879715in}{5.512857in}}%
\pgfpathlineto{\pgfqpoint{0.895148in}{5.495912in}}%
\pgfpathlineto{\pgfqpoint{0.910582in}{5.475073in}}%
\pgfpathlineto{\pgfqpoint{0.926015in}{5.450497in}}%
\pgfpathlineto{\pgfqpoint{0.941449in}{5.422414in}}%
\pgfpathlineto{\pgfqpoint{0.972316in}{5.356987in}}%
\pgfpathlineto{\pgfqpoint{1.003184in}{5.281932in}}%
\pgfpathlineto{\pgfqpoint{1.111219in}{5.005384in}}%
\pgfpathlineto{\pgfqpoint{1.142086in}{4.941890in}}%
\pgfpathlineto{\pgfqpoint{1.157520in}{4.915486in}}%
\pgfpathlineto{\pgfqpoint{1.172954in}{4.893230in}}%
\pgfpathlineto{\pgfqpoint{1.188387in}{4.875481in}}%
\pgfpathlineto{\pgfqpoint{1.203821in}{4.862529in}}%
\pgfpathlineto{\pgfqpoint{1.219255in}{4.854581in}}%
\pgfpathlineto{\pgfqpoint{1.234688in}{4.851765in}}%
\pgfpathlineto{\pgfqpoint{1.250122in}{4.854123in}}%
\pgfpathlineto{\pgfqpoint{1.265555in}{4.861613in}}%
\pgfpathlineto{\pgfqpoint{1.280989in}{4.874107in}}%
\pgfpathlineto{\pgfqpoint{1.296423in}{4.891399in}}%
\pgfpathlineto{\pgfqpoint{1.311856in}{4.913205in}}%
\pgfpathlineto{\pgfqpoint{1.327290in}{4.939172in}}%
\pgfpathlineto{\pgfqpoint{1.358157in}{5.001884in}}%
\pgfpathlineto{\pgfqpoint{1.389025in}{5.075657in}}%
\pgfpathlineto{\pgfqpoint{1.435325in}{5.197389in}}%
\pgfpathlineto{\pgfqpoint{1.481626in}{5.318841in}}%
\pgfpathlineto{\pgfqpoint{1.512494in}{5.392677in}}%
\pgfpathlineto{\pgfqpoint{1.543361in}{5.456723in}}%
\pgfpathlineto{\pgfqpoint{1.558795in}{5.484208in}}%
\pgfpathlineto{\pgfqpoint{1.574228in}{5.508307in}}%
\pgfpathlineto{\pgfqpoint{1.589662in}{5.528815in}}%
\pgfpathlineto{\pgfqpoint{1.605095in}{5.545583in}}%
\pgfpathlineto{\pgfqpoint{1.620529in}{5.558513in}}%
\pgfpathlineto{\pgfqpoint{1.635963in}{5.567558in}}%
\pgfpathlineto{\pgfqpoint{1.651396in}{5.572725in}}%
\pgfpathlineto{\pgfqpoint{1.666830in}{5.574065in}}%
\pgfpathlineto{\pgfqpoint{1.682264in}{5.571679in}}%
\pgfpathlineto{\pgfqpoint{1.697697in}{5.565707in}}%
\pgfpathlineto{\pgfqpoint{1.713131in}{5.556330in}}%
\pgfpathlineto{\pgfqpoint{1.728564in}{5.543764in}}%
\pgfpathlineto{\pgfqpoint{1.743998in}{5.528254in}}%
\pgfpathlineto{\pgfqpoint{1.774865in}{5.489513in}}%
\pgfpathlineto{\pgfqpoint{1.805733in}{5.442507in}}%
\pgfpathlineto{\pgfqpoint{1.852034in}{5.362154in}}%
\pgfpathlineto{\pgfqpoint{1.929202in}{5.223100in}}%
\pgfpathlineto{\pgfqpoint{1.960069in}{5.172261in}}%
\pgfpathlineto{\pgfqpoint{1.990936in}{5.126903in}}%
\pgfpathlineto{\pgfqpoint{2.021804in}{5.088402in}}%
\pgfpathlineto{\pgfqpoint{2.052671in}{5.057759in}}%
\pgfpathlineto{\pgfqpoint{2.068104in}{5.045589in}}%
\pgfpathlineto{\pgfqpoint{2.083538in}{5.035573in}}%
\pgfpathlineto{\pgfqpoint{2.098972in}{5.027714in}}%
\pgfpathlineto{\pgfqpoint{2.114405in}{5.021983in}}%
\pgfpathlineto{\pgfqpoint{2.129839in}{5.018319in}}%
\pgfpathlineto{\pgfqpoint{2.145273in}{5.016625in}}%
\pgfpathlineto{\pgfqpoint{2.160706in}{5.016770in}}%
\pgfpathlineto{\pgfqpoint{2.191574in}{5.021878in}}%
\pgfpathlineto{\pgfqpoint{2.222441in}{5.031928in}}%
\pgfpathlineto{\pgfqpoint{2.315043in}{5.069089in}}%
\pgfpathlineto{\pgfqpoint{2.345910in}{5.075801in}}%
\pgfpathlineto{\pgfqpoint{2.361344in}{5.076912in}}%
\pgfpathlineto{\pgfqpoint{2.376777in}{5.076284in}}%
\pgfpathlineto{\pgfqpoint{2.392211in}{5.073789in}}%
\pgfpathlineto{\pgfqpoint{2.407644in}{5.069352in}}%
\pgfpathlineto{\pgfqpoint{2.423078in}{5.062947in}}%
\pgfpathlineto{\pgfqpoint{2.438512in}{5.054600in}}%
\pgfpathlineto{\pgfqpoint{2.469379in}{5.032439in}}%
\pgfpathlineto{\pgfqpoint{2.500246in}{5.004042in}}%
\pgfpathlineto{\pgfqpoint{2.531114in}{4.971202in}}%
\pgfpathlineto{\pgfqpoint{2.623715in}{4.869060in}}%
\pgfpathlineto{\pgfqpoint{2.654583in}{4.841658in}}%
\pgfpathlineto{\pgfqpoint{2.685450in}{4.820899in}}%
\pgfpathlineto{\pgfqpoint{2.700884in}{4.813440in}}%
\pgfpathlineto{\pgfqpoint{2.716317in}{4.808083in}}%
\pgfpathlineto{\pgfqpoint{2.731751in}{4.804893in}}%
\pgfpathlineto{\pgfqpoint{2.747184in}{4.803892in}}%
\pgfpathlineto{\pgfqpoint{2.762618in}{4.805059in}}%
\pgfpathlineto{\pgfqpoint{2.778052in}{4.808329in}}%
\pgfpathlineto{\pgfqpoint{2.793485in}{4.813595in}}%
\pgfpathlineto{\pgfqpoint{2.824353in}{4.829473in}}%
\pgfpathlineto{\pgfqpoint{2.855220in}{4.851043in}}%
\pgfpathlineto{\pgfqpoint{2.901521in}{4.889269in}}%
\pgfpathlineto{\pgfqpoint{2.947822in}{4.926980in}}%
\pgfpathlineto{\pgfqpoint{2.978689in}{4.947667in}}%
\pgfpathlineto{\pgfqpoint{3.009556in}{4.962315in}}%
\pgfpathlineto{\pgfqpoint{3.024990in}{4.966878in}}%
\pgfpathlineto{\pgfqpoint{3.040424in}{4.969420in}}%
\pgfpathlineto{\pgfqpoint{3.055857in}{4.969864in}}%
\pgfpathlineto{\pgfqpoint{3.071291in}{4.968180in}}%
\pgfpathlineto{\pgfqpoint{3.086724in}{4.964386in}}%
\pgfpathlineto{\pgfqpoint{3.102158in}{4.958544in}}%
\pgfpathlineto{\pgfqpoint{3.117592in}{4.950760in}}%
\pgfpathlineto{\pgfqpoint{3.148459in}{4.929976in}}%
\pgfpathlineto{\pgfqpoint{3.179326in}{4.903553in}}%
\pgfpathlineto{\pgfqpoint{3.225627in}{4.857491in}}%
\pgfpathlineto{\pgfqpoint{3.287362in}{4.794656in}}%
\pgfpathlineto{\pgfqpoint{3.318229in}{4.766757in}}%
\pgfpathlineto{\pgfqpoint{3.349096in}{4.743111in}}%
\pgfpathlineto{\pgfqpoint{3.379963in}{4.724667in}}%
\pgfpathlineto{\pgfqpoint{3.410831in}{4.711957in}}%
\pgfpathlineto{\pgfqpoint{3.441698in}{4.705133in}}%
\pgfpathlineto{\pgfqpoint{3.472565in}{4.704018in}}%
\pgfpathlineto{\pgfqpoint{3.503433in}{4.708201in}}%
\pgfpathlineto{\pgfqpoint{3.534300in}{4.717140in}}%
\pgfpathlineto{\pgfqpoint{3.565167in}{4.730268in}}%
\pgfpathlineto{\pgfqpoint{3.596034in}{4.747087in}}%
\pgfpathlineto{\pgfqpoint{3.626902in}{4.767227in}}%
\pgfpathlineto{\pgfqpoint{3.657769in}{4.790456in}}%
\pgfpathlineto{\pgfqpoint{3.688636in}{4.816646in}}%
\pgfpathlineto{\pgfqpoint{3.734937in}{4.861218in}}%
\pgfpathlineto{\pgfqpoint{3.781238in}{4.911318in}}%
\pgfpathlineto{\pgfqpoint{3.904707in}{5.051629in}}%
\pgfpathlineto{\pgfqpoint{3.935574in}{5.081323in}}%
\pgfpathlineto{\pgfqpoint{3.966442in}{5.106096in}}%
\pgfpathlineto{\pgfqpoint{3.997309in}{5.124923in}}%
\pgfpathlineto{\pgfqpoint{4.028176in}{5.137221in}}%
\pgfpathlineto{\pgfqpoint{4.043610in}{5.140878in}}%
\pgfpathlineto{\pgfqpoint{4.074477in}{5.143423in}}%
\pgfpathlineto{\pgfqpoint{4.105344in}{5.140268in}}%
\pgfpathlineto{\pgfqpoint{4.136212in}{5.132607in}}%
\pgfpathlineto{\pgfqpoint{4.182513in}{5.116129in}}%
\pgfpathlineto{\pgfqpoint{4.228813in}{5.099113in}}%
\pgfpathlineto{\pgfqpoint{4.259681in}{5.090496in}}%
\pgfpathlineto{\pgfqpoint{4.290548in}{5.085906in}}%
\pgfpathlineto{\pgfqpoint{4.321415in}{5.086507in}}%
\pgfpathlineto{\pgfqpoint{4.352283in}{5.092957in}}%
\pgfpathlineto{\pgfqpoint{4.383150in}{5.105315in}}%
\pgfpathlineto{\pgfqpoint{4.414017in}{5.123013in}}%
\pgfpathlineto{\pgfqpoint{4.444884in}{5.144889in}}%
\pgfpathlineto{\pgfqpoint{4.552920in}{5.228563in}}%
\pgfpathlineto{\pgfqpoint{4.583787in}{5.246649in}}%
\pgfpathlineto{\pgfqpoint{4.614654in}{5.259444in}}%
\pgfpathlineto{\pgfqpoint{4.645522in}{5.266315in}}%
\pgfpathlineto{\pgfqpoint{4.660955in}{5.267504in}}%
\pgfpathlineto{\pgfqpoint{4.660955in}{5.267504in}}%
\pgfusepath{stroke}%
\end{pgfscope}%
\begin{pgfscope}%
\pgfpathrectangle{\pgfqpoint{0.625831in}{3.799602in}}{\pgfqpoint{4.227273in}{2.745455in}} %
\pgfusepath{clip}%
\pgfsetrectcap%
\pgfsetroundjoin%
\pgfsetlinewidth{0.501875pt}%
\definecolor{currentstroke}{rgb}{1.000000,0.361242,0.183750}%
\pgfsetstrokecolor{currentstroke}%
\pgfsetdash{}{0pt}%
\pgfpathmoveto{\pgfqpoint{0.817980in}{5.547863in}}%
\pgfpathlineto{\pgfqpoint{0.833414in}{5.546495in}}%
\pgfpathlineto{\pgfqpoint{0.848847in}{5.541251in}}%
\pgfpathlineto{\pgfqpoint{0.864281in}{5.532006in}}%
\pgfpathlineto{\pgfqpoint{0.879715in}{5.518701in}}%
\pgfpathlineto{\pgfqpoint{0.895148in}{5.501344in}}%
\pgfpathlineto{\pgfqpoint{0.910582in}{5.480020in}}%
\pgfpathlineto{\pgfqpoint{0.926015in}{5.454883in}}%
\pgfpathlineto{\pgfqpoint{0.941449in}{5.426162in}}%
\pgfpathlineto{\pgfqpoint{0.972316in}{5.359223in}}%
\pgfpathlineto{\pgfqpoint{1.003184in}{5.282355in}}%
\pgfpathlineto{\pgfqpoint{1.111219in}{4.998237in}}%
\pgfpathlineto{\pgfqpoint{1.142086in}{4.932795in}}%
\pgfpathlineto{\pgfqpoint{1.157520in}{4.905548in}}%
\pgfpathlineto{\pgfqpoint{1.172954in}{4.882556in}}%
\pgfpathlineto{\pgfqpoint{1.188387in}{4.864190in}}%
\pgfpathlineto{\pgfqpoint{1.203821in}{4.850747in}}%
\pgfpathlineto{\pgfqpoint{1.219255in}{4.842438in}}%
\pgfpathlineto{\pgfqpoint{1.234688in}{4.839391in}}%
\pgfpathlineto{\pgfqpoint{1.250122in}{4.841646in}}%
\pgfpathlineto{\pgfqpoint{1.265555in}{4.849152in}}%
\pgfpathlineto{\pgfqpoint{1.280989in}{4.861774in}}%
\pgfpathlineto{\pgfqpoint{1.296423in}{4.879294in}}%
\pgfpathlineto{\pgfqpoint{1.311856in}{4.901418in}}%
\pgfpathlineto{\pgfqpoint{1.327290in}{4.927780in}}%
\pgfpathlineto{\pgfqpoint{1.358157in}{4.991470in}}%
\pgfpathlineto{\pgfqpoint{1.389025in}{5.066405in}}%
\pgfpathlineto{\pgfqpoint{1.435325in}{5.190105in}}%
\pgfpathlineto{\pgfqpoint{1.481626in}{5.313678in}}%
\pgfpathlineto{\pgfqpoint{1.512494in}{5.388949in}}%
\pgfpathlineto{\pgfqpoint{1.543361in}{5.454375in}}%
\pgfpathlineto{\pgfqpoint{1.558795in}{5.482502in}}%
\pgfpathlineto{\pgfqpoint{1.574228in}{5.507192in}}%
\pgfpathlineto{\pgfqpoint{1.589662in}{5.528228in}}%
\pgfpathlineto{\pgfqpoint{1.605095in}{5.545444in}}%
\pgfpathlineto{\pgfqpoint{1.620529in}{5.558728in}}%
\pgfpathlineto{\pgfqpoint{1.635963in}{5.568022in}}%
\pgfpathlineto{\pgfqpoint{1.651396in}{5.573318in}}%
\pgfpathlineto{\pgfqpoint{1.666830in}{5.574664in}}%
\pgfpathlineto{\pgfqpoint{1.682264in}{5.572152in}}%
\pgfpathlineto{\pgfqpoint{1.697697in}{5.565922in}}%
\pgfpathlineto{\pgfqpoint{1.713131in}{5.556159in}}%
\pgfpathlineto{\pgfqpoint{1.728564in}{5.543083in}}%
\pgfpathlineto{\pgfqpoint{1.743998in}{5.526953in}}%
\pgfpathlineto{\pgfqpoint{1.759432in}{5.508054in}}%
\pgfpathlineto{\pgfqpoint{1.790299in}{5.463210in}}%
\pgfpathlineto{\pgfqpoint{1.821166in}{5.411220in}}%
\pgfpathlineto{\pgfqpoint{1.882901in}{5.296882in}}%
\pgfpathlineto{\pgfqpoint{1.929202in}{5.212318in}}%
\pgfpathlineto{\pgfqpoint{1.960069in}{5.160720in}}%
\pgfpathlineto{\pgfqpoint{1.990936in}{5.115024in}}%
\pgfpathlineto{\pgfqpoint{2.021804in}{5.076598in}}%
\pgfpathlineto{\pgfqpoint{2.052671in}{5.046427in}}%
\pgfpathlineto{\pgfqpoint{2.068104in}{5.034633in}}%
\pgfpathlineto{\pgfqpoint{2.083538in}{5.025082in}}%
\pgfpathlineto{\pgfqpoint{2.098972in}{5.017774in}}%
\pgfpathlineto{\pgfqpoint{2.114405in}{5.012677in}}%
\pgfpathlineto{\pgfqpoint{2.129839in}{5.009723in}}%
\pgfpathlineto{\pgfqpoint{2.145273in}{5.008811in}}%
\pgfpathlineto{\pgfqpoint{2.160706in}{5.009803in}}%
\pgfpathlineto{\pgfqpoint{2.191574in}{5.016769in}}%
\pgfpathlineto{\pgfqpoint{2.222441in}{5.028825in}}%
\pgfpathlineto{\pgfqpoint{2.315043in}{5.071760in}}%
\pgfpathlineto{\pgfqpoint{2.345910in}{5.079903in}}%
\pgfpathlineto{\pgfqpoint{2.361344in}{5.081579in}}%
\pgfpathlineto{\pgfqpoint{2.376777in}{5.081402in}}%
\pgfpathlineto{\pgfqpoint{2.392211in}{5.079241in}}%
\pgfpathlineto{\pgfqpoint{2.407644in}{5.075019in}}%
\pgfpathlineto{\pgfqpoint{2.423078in}{5.068710in}}%
\pgfpathlineto{\pgfqpoint{2.438512in}{5.060345in}}%
\pgfpathlineto{\pgfqpoint{2.469379in}{5.037832in}}%
\pgfpathlineto{\pgfqpoint{2.500246in}{5.008728in}}%
\pgfpathlineto{\pgfqpoint{2.531114in}{4.974930in}}%
\pgfpathlineto{\pgfqpoint{2.623715in}{4.869498in}}%
\pgfpathlineto{\pgfqpoint{2.654583in}{4.841201in}}%
\pgfpathlineto{\pgfqpoint{2.685450in}{4.819791in}}%
\pgfpathlineto{\pgfqpoint{2.700884in}{4.812118in}}%
\pgfpathlineto{\pgfqpoint{2.716317in}{4.806630in}}%
\pgfpathlineto{\pgfqpoint{2.731751in}{4.803397in}}%
\pgfpathlineto{\pgfqpoint{2.747184in}{4.802446in}}%
\pgfpathlineto{\pgfqpoint{2.762618in}{4.803759in}}%
\pgfpathlineto{\pgfqpoint{2.778052in}{4.807271in}}%
\pgfpathlineto{\pgfqpoint{2.793485in}{4.812874in}}%
\pgfpathlineto{\pgfqpoint{2.808919in}{4.820415in}}%
\pgfpathlineto{\pgfqpoint{2.839786in}{4.840501in}}%
\pgfpathlineto{\pgfqpoint{2.870654in}{4.865544in}}%
\pgfpathlineto{\pgfqpoint{2.947822in}{4.933322in}}%
\pgfpathlineto{\pgfqpoint{2.978689in}{4.955544in}}%
\pgfpathlineto{\pgfqpoint{2.994123in}{4.964411in}}%
\pgfpathlineto{\pgfqpoint{3.009556in}{4.971472in}}%
\pgfpathlineto{\pgfqpoint{3.024990in}{4.976546in}}%
\pgfpathlineto{\pgfqpoint{3.040424in}{4.979499in}}%
\pgfpathlineto{\pgfqpoint{3.055857in}{4.980247in}}%
\pgfpathlineto{\pgfqpoint{3.071291in}{4.978753in}}%
\pgfpathlineto{\pgfqpoint{3.086724in}{4.975033in}}%
\pgfpathlineto{\pgfqpoint{3.102158in}{4.969150in}}%
\pgfpathlineto{\pgfqpoint{3.117592in}{4.961209in}}%
\pgfpathlineto{\pgfqpoint{3.148459in}{4.939782in}}%
\pgfpathlineto{\pgfqpoint{3.179326in}{4.912317in}}%
\pgfpathlineto{\pgfqpoint{3.225627in}{4.864086in}}%
\pgfpathlineto{\pgfqpoint{3.287362in}{4.797627in}}%
\pgfpathlineto{\pgfqpoint{3.318229in}{4.767792in}}%
\pgfpathlineto{\pgfqpoint{3.349096in}{4.742251in}}%
\pgfpathlineto{\pgfqpoint{3.379963in}{4.722054in}}%
\pgfpathlineto{\pgfqpoint{3.410831in}{4.707841in}}%
\pgfpathlineto{\pgfqpoint{3.441698in}{4.699867in}}%
\pgfpathlineto{\pgfqpoint{3.472565in}{4.698046in}}%
\pgfpathlineto{\pgfqpoint{3.503433in}{4.702028in}}%
\pgfpathlineto{\pgfqpoint{3.534300in}{4.711295in}}%
\pgfpathlineto{\pgfqpoint{3.565167in}{4.725257in}}%
\pgfpathlineto{\pgfqpoint{3.596034in}{4.743343in}}%
\pgfpathlineto{\pgfqpoint{3.626902in}{4.765073in}}%
\pgfpathlineto{\pgfqpoint{3.657769in}{4.790075in}}%
\pgfpathlineto{\pgfqpoint{3.688636in}{4.818071in}}%
\pgfpathlineto{\pgfqpoint{3.734937in}{4.865118in}}%
\pgfpathlineto{\pgfqpoint{3.781238in}{4.917096in}}%
\pgfpathlineto{\pgfqpoint{3.889273in}{5.042385in}}%
\pgfpathlineto{\pgfqpoint{3.920141in}{5.073637in}}%
\pgfpathlineto{\pgfqpoint{3.951008in}{5.100301in}}%
\pgfpathlineto{\pgfqpoint{3.981875in}{5.121208in}}%
\pgfpathlineto{\pgfqpoint{4.012743in}{5.135571in}}%
\pgfpathlineto{\pgfqpoint{4.028176in}{5.140182in}}%
\pgfpathlineto{\pgfqpoint{4.043610in}{5.143086in}}%
\pgfpathlineto{\pgfqpoint{4.059043in}{5.144331in}}%
\pgfpathlineto{\pgfqpoint{4.089911in}{5.142200in}}%
\pgfpathlineto{\pgfqpoint{4.120778in}{5.134815in}}%
\pgfpathlineto{\pgfqpoint{4.151645in}{5.123628in}}%
\pgfpathlineto{\pgfqpoint{4.244247in}{5.085420in}}%
\pgfpathlineto{\pgfqpoint{4.275114in}{5.077309in}}%
\pgfpathlineto{\pgfqpoint{4.305982in}{5.074093in}}%
\pgfpathlineto{\pgfqpoint{4.336849in}{5.076704in}}%
\pgfpathlineto{\pgfqpoint{4.367716in}{5.085498in}}%
\pgfpathlineto{\pgfqpoint{4.398583in}{5.100195in}}%
\pgfpathlineto{\pgfqpoint{4.429451in}{5.119892in}}%
\pgfpathlineto{\pgfqpoint{4.475752in}{5.155528in}}%
\pgfpathlineto{\pgfqpoint{4.537486in}{5.204321in}}%
\pgfpathlineto{\pgfqpoint{4.568353in}{5.224900in}}%
\pgfpathlineto{\pgfqpoint{4.599221in}{5.240870in}}%
\pgfpathlineto{\pgfqpoint{4.630088in}{5.251326in}}%
\pgfpathlineto{\pgfqpoint{4.660955in}{5.255980in}}%
\pgfpathlineto{\pgfqpoint{4.660955in}{5.255980in}}%
\pgfusepath{stroke}%
\end{pgfscope}%
\begin{pgfscope}%
\pgfpathrectangle{\pgfqpoint{0.625831in}{3.799602in}}{\pgfqpoint{4.227273in}{2.745455in}} %
\pgfusepath{clip}%
\pgfsetrectcap%
\pgfsetroundjoin%
\pgfsetlinewidth{0.501875pt}%
\definecolor{currentstroke}{rgb}{1.000000,0.243914,0.122888}%
\pgfsetstrokecolor{currentstroke}%
\pgfsetdash{}{0pt}%
\pgfpathmoveto{\pgfqpoint{0.817980in}{5.549360in}}%
\pgfpathlineto{\pgfqpoint{0.833414in}{5.547857in}}%
\pgfpathlineto{\pgfqpoint{0.848847in}{5.542545in}}%
\pgfpathlineto{\pgfqpoint{0.864281in}{5.533295in}}%
\pgfpathlineto{\pgfqpoint{0.879715in}{5.520041in}}%
\pgfpathlineto{\pgfqpoint{0.895148in}{5.502786in}}%
\pgfpathlineto{\pgfqpoint{0.910582in}{5.481607in}}%
\pgfpathlineto{\pgfqpoint{0.926015in}{5.456652in}}%
\pgfpathlineto{\pgfqpoint{0.941449in}{5.428143in}}%
\pgfpathlineto{\pgfqpoint{0.972316in}{5.361699in}}%
\pgfpathlineto{\pgfqpoint{1.003184in}{5.285395in}}%
\pgfpathlineto{\pgfqpoint{1.111219in}{5.003496in}}%
\pgfpathlineto{\pgfqpoint{1.142086in}{4.938656in}}%
\pgfpathlineto{\pgfqpoint{1.157520in}{4.911680in}}%
\pgfpathlineto{\pgfqpoint{1.172954in}{4.888931in}}%
\pgfpathlineto{\pgfqpoint{1.188387in}{4.870773in}}%
\pgfpathlineto{\pgfqpoint{1.203821in}{4.857496in}}%
\pgfpathlineto{\pgfqpoint{1.219255in}{4.849305in}}%
\pgfpathlineto{\pgfqpoint{1.234688in}{4.846324in}}%
\pgfpathlineto{\pgfqpoint{1.250122in}{4.848585in}}%
\pgfpathlineto{\pgfqpoint{1.265555in}{4.856037in}}%
\pgfpathlineto{\pgfqpoint{1.280989in}{4.868542in}}%
\pgfpathlineto{\pgfqpoint{1.296423in}{4.885879in}}%
\pgfpathlineto{\pgfqpoint{1.311856in}{4.907754in}}%
\pgfpathlineto{\pgfqpoint{1.327290in}{4.933802in}}%
\pgfpathlineto{\pgfqpoint{1.358157in}{4.996670in}}%
\pgfpathlineto{\pgfqpoint{1.389025in}{5.070536in}}%
\pgfpathlineto{\pgfqpoint{1.450759in}{5.233390in}}%
\pgfpathlineto{\pgfqpoint{1.497060in}{5.351167in}}%
\pgfpathlineto{\pgfqpoint{1.527927in}{5.420155in}}%
\pgfpathlineto{\pgfqpoint{1.558795in}{5.477884in}}%
\pgfpathlineto{\pgfqpoint{1.574228in}{5.501798in}}%
\pgfpathlineto{\pgfqpoint{1.589662in}{5.522127in}}%
\pgfpathlineto{\pgfqpoint{1.605095in}{5.538726in}}%
\pgfpathlineto{\pgfqpoint{1.620529in}{5.551501in}}%
\pgfpathlineto{\pgfqpoint{1.635963in}{5.560411in}}%
\pgfpathlineto{\pgfqpoint{1.651396in}{5.565465in}}%
\pgfpathlineto{\pgfqpoint{1.666830in}{5.566720in}}%
\pgfpathlineto{\pgfqpoint{1.682264in}{5.564278in}}%
\pgfpathlineto{\pgfqpoint{1.697697in}{5.558282in}}%
\pgfpathlineto{\pgfqpoint{1.713131in}{5.548915in}}%
\pgfpathlineto{\pgfqpoint{1.728564in}{5.536390in}}%
\pgfpathlineto{\pgfqpoint{1.743998in}{5.520953in}}%
\pgfpathlineto{\pgfqpoint{1.774865in}{5.482437in}}%
\pgfpathlineto{\pgfqpoint{1.805733in}{5.435726in}}%
\pgfpathlineto{\pgfqpoint{1.852034in}{5.355818in}}%
\pgfpathlineto{\pgfqpoint{1.944635in}{5.190870in}}%
\pgfpathlineto{\pgfqpoint{1.975503in}{5.142246in}}%
\pgfpathlineto{\pgfqpoint{2.006370in}{5.099666in}}%
\pgfpathlineto{\pgfqpoint{2.037237in}{5.064373in}}%
\pgfpathlineto{\pgfqpoint{2.068104in}{5.037250in}}%
\pgfpathlineto{\pgfqpoint{2.083538in}{5.026916in}}%
\pgfpathlineto{\pgfqpoint{2.098972in}{5.018764in}}%
\pgfpathlineto{\pgfqpoint{2.114405in}{5.012776in}}%
\pgfpathlineto{\pgfqpoint{2.129839in}{5.008903in}}%
\pgfpathlineto{\pgfqpoint{2.145273in}{5.007059in}}%
\pgfpathlineto{\pgfqpoint{2.160706in}{5.007125in}}%
\pgfpathlineto{\pgfqpoint{2.176140in}{5.008942in}}%
\pgfpathlineto{\pgfqpoint{2.207007in}{5.017040in}}%
\pgfpathlineto{\pgfqpoint{2.237874in}{5.029457in}}%
\pgfpathlineto{\pgfqpoint{2.315043in}{5.063933in}}%
\pgfpathlineto{\pgfqpoint{2.345910in}{5.072681in}}%
\pgfpathlineto{\pgfqpoint{2.361344in}{5.074897in}}%
\pgfpathlineto{\pgfqpoint{2.376777in}{5.075406in}}%
\pgfpathlineto{\pgfqpoint{2.392211in}{5.074064in}}%
\pgfpathlineto{\pgfqpoint{2.407644in}{5.070773in}}%
\pgfpathlineto{\pgfqpoint{2.423078in}{5.065486in}}%
\pgfpathlineto{\pgfqpoint{2.438512in}{5.058211in}}%
\pgfpathlineto{\pgfqpoint{2.453945in}{5.049003in}}%
\pgfpathlineto{\pgfqpoint{2.484813in}{5.025269in}}%
\pgfpathlineto{\pgfqpoint{2.515680in}{4.995682in}}%
\pgfpathlineto{\pgfqpoint{2.561981in}{4.944779in}}%
\pgfpathlineto{\pgfqpoint{2.608282in}{4.893343in}}%
\pgfpathlineto{\pgfqpoint{2.639149in}{4.862895in}}%
\pgfpathlineto{\pgfqpoint{2.670016in}{4.838098in}}%
\pgfpathlineto{\pgfqpoint{2.685450in}{4.828373in}}%
\pgfpathlineto{\pgfqpoint{2.700884in}{4.820656in}}%
\pgfpathlineto{\pgfqpoint{2.716317in}{4.815064in}}%
\pgfpathlineto{\pgfqpoint{2.731751in}{4.811669in}}%
\pgfpathlineto{\pgfqpoint{2.747184in}{4.810499in}}%
\pgfpathlineto{\pgfqpoint{2.762618in}{4.811533in}}%
\pgfpathlineto{\pgfqpoint{2.778052in}{4.814705in}}%
\pgfpathlineto{\pgfqpoint{2.793485in}{4.819900in}}%
\pgfpathlineto{\pgfqpoint{2.824353in}{4.835690in}}%
\pgfpathlineto{\pgfqpoint{2.855220in}{4.857172in}}%
\pgfpathlineto{\pgfqpoint{2.901521in}{4.895064in}}%
\pgfpathlineto{\pgfqpoint{2.947822in}{4.931980in}}%
\pgfpathlineto{\pgfqpoint{2.978689in}{4.951825in}}%
\pgfpathlineto{\pgfqpoint{2.994123in}{4.959502in}}%
\pgfpathlineto{\pgfqpoint{3.009556in}{4.965396in}}%
\pgfpathlineto{\pgfqpoint{3.024990in}{4.969347in}}%
\pgfpathlineto{\pgfqpoint{3.040424in}{4.971241in}}%
\pgfpathlineto{\pgfqpoint{3.055857in}{4.971012in}}%
\pgfpathlineto{\pgfqpoint{3.071291in}{4.968642in}}%
\pgfpathlineto{\pgfqpoint{3.086724in}{4.964162in}}%
\pgfpathlineto{\pgfqpoint{3.102158in}{4.957649in}}%
\pgfpathlineto{\pgfqpoint{3.133025in}{4.939042in}}%
\pgfpathlineto{\pgfqpoint{3.163893in}{4.914209in}}%
\pgfpathlineto{\pgfqpoint{3.194760in}{4.884976in}}%
\pgfpathlineto{\pgfqpoint{3.302795in}{4.777528in}}%
\pgfpathlineto{\pgfqpoint{3.333663in}{4.752906in}}%
\pgfpathlineto{\pgfqpoint{3.364530in}{4.733361in}}%
\pgfpathlineto{\pgfqpoint{3.395397in}{4.719504in}}%
\pgfpathlineto{\pgfqpoint{3.426264in}{4.711526in}}%
\pgfpathlineto{\pgfqpoint{3.457132in}{4.709264in}}%
\pgfpathlineto{\pgfqpoint{3.487999in}{4.712287in}}%
\pgfpathlineto{\pgfqpoint{3.518866in}{4.720008in}}%
\pgfpathlineto{\pgfqpoint{3.549733in}{4.731796in}}%
\pgfpathlineto{\pgfqpoint{3.580601in}{4.747079in}}%
\pgfpathlineto{\pgfqpoint{3.611468in}{4.765420in}}%
\pgfpathlineto{\pgfqpoint{3.642335in}{4.786545in}}%
\pgfpathlineto{\pgfqpoint{3.673203in}{4.810331in}}%
\pgfpathlineto{\pgfqpoint{3.719503in}{4.850912in}}%
\pgfpathlineto{\pgfqpoint{3.765804in}{4.897075in}}%
\pgfpathlineto{\pgfqpoint{3.827539in}{4.964982in}}%
\pgfpathlineto{\pgfqpoint{3.889273in}{5.033463in}}%
\pgfpathlineto{\pgfqpoint{3.920141in}{5.064605in}}%
\pgfpathlineto{\pgfqpoint{3.951008in}{5.091736in}}%
\pgfpathlineto{\pgfqpoint{3.981875in}{5.113574in}}%
\pgfpathlineto{\pgfqpoint{4.012743in}{5.129177in}}%
\pgfpathlineto{\pgfqpoint{4.028176in}{5.134473in}}%
\pgfpathlineto{\pgfqpoint{4.043610in}{5.138076in}}%
\pgfpathlineto{\pgfqpoint{4.059043in}{5.140016in}}%
\pgfpathlineto{\pgfqpoint{4.089911in}{5.139196in}}%
\pgfpathlineto{\pgfqpoint{4.120778in}{5.132936in}}%
\pgfpathlineto{\pgfqpoint{4.151645in}{5.122639in}}%
\pgfpathlineto{\pgfqpoint{4.259681in}{5.081346in}}%
\pgfpathlineto{\pgfqpoint{4.290548in}{5.075689in}}%
\pgfpathlineto{\pgfqpoint{4.321415in}{5.075384in}}%
\pgfpathlineto{\pgfqpoint{4.352283in}{5.081069in}}%
\pgfpathlineto{\pgfqpoint{4.383150in}{5.092755in}}%
\pgfpathlineto{\pgfqpoint{4.414017in}{5.109812in}}%
\pgfpathlineto{\pgfqpoint{4.444884in}{5.131037in}}%
\pgfpathlineto{\pgfqpoint{4.552920in}{5.212644in}}%
\pgfpathlineto{\pgfqpoint{4.583787in}{5.230544in}}%
\pgfpathlineto{\pgfqpoint{4.614654in}{5.243532in}}%
\pgfpathlineto{\pgfqpoint{4.645522in}{5.251059in}}%
\pgfpathlineto{\pgfqpoint{4.660955in}{5.252762in}}%
\pgfpathlineto{\pgfqpoint{4.660955in}{5.252762in}}%
\pgfusepath{stroke}%
\end{pgfscope}%
\begin{pgfscope}%
\pgfpathrectangle{\pgfqpoint{0.625831in}{3.799602in}}{\pgfqpoint{4.227273in}{2.745455in}} %
\pgfusepath{clip}%
\pgfsetrectcap%
\pgfsetroundjoin%
\pgfsetlinewidth{0.501875pt}%
\definecolor{currentstroke}{rgb}{1.000000,0.122888,0.061561}%
\pgfsetstrokecolor{currentstroke}%
\pgfsetdash{}{0pt}%
\pgfpathmoveto{\pgfqpoint{0.817980in}{5.551541in}}%
\pgfpathlineto{\pgfqpoint{0.833414in}{5.550025in}}%
\pgfpathlineto{\pgfqpoint{0.848847in}{5.544726in}}%
\pgfpathlineto{\pgfqpoint{0.864281in}{5.535517in}}%
\pgfpathlineto{\pgfqpoint{0.879715in}{5.522334in}}%
\pgfpathlineto{\pgfqpoint{0.895148in}{5.505185in}}%
\pgfpathlineto{\pgfqpoint{0.910582in}{5.484144in}}%
\pgfpathlineto{\pgfqpoint{0.926015in}{5.459362in}}%
\pgfpathlineto{\pgfqpoint{0.941449in}{5.431059in}}%
\pgfpathlineto{\pgfqpoint{0.972316in}{5.365122in}}%
\pgfpathlineto{\pgfqpoint{1.003184in}{5.289426in}}%
\pgfpathlineto{\pgfqpoint{1.111219in}{5.009614in}}%
\pgfpathlineto{\pgfqpoint{1.142086in}{4.945000in}}%
\pgfpathlineto{\pgfqpoint{1.157520in}{4.918018in}}%
\pgfpathlineto{\pgfqpoint{1.172954in}{4.895173in}}%
\pgfpathlineto{\pgfqpoint{1.188387in}{4.876825in}}%
\pgfpathlineto{\pgfqpoint{1.203821in}{4.863260in}}%
\pgfpathlineto{\pgfqpoint{1.219255in}{4.854688in}}%
\pgfpathlineto{\pgfqpoint{1.234688in}{4.851236in}}%
\pgfpathlineto{\pgfqpoint{1.250122in}{4.852948in}}%
\pgfpathlineto{\pgfqpoint{1.265555in}{4.859785in}}%
\pgfpathlineto{\pgfqpoint{1.280989in}{4.871624in}}%
\pgfpathlineto{\pgfqpoint{1.296423in}{4.888264in}}%
\pgfpathlineto{\pgfqpoint{1.311856in}{4.909430in}}%
\pgfpathlineto{\pgfqpoint{1.327290in}{4.934780in}}%
\pgfpathlineto{\pgfqpoint{1.342724in}{4.963910in}}%
\pgfpathlineto{\pgfqpoint{1.373591in}{5.031644in}}%
\pgfpathlineto{\pgfqpoint{1.404458in}{5.108532in}}%
\pgfpathlineto{\pgfqpoint{1.497060in}{5.349378in}}%
\pgfpathlineto{\pgfqpoint{1.527927in}{5.418950in}}%
\pgfpathlineto{\pgfqpoint{1.558795in}{5.477369in}}%
\pgfpathlineto{\pgfqpoint{1.574228in}{5.501624in}}%
\pgfpathlineto{\pgfqpoint{1.589662in}{5.522270in}}%
\pgfpathlineto{\pgfqpoint{1.605095in}{5.539148in}}%
\pgfpathlineto{\pgfqpoint{1.620529in}{5.552153in}}%
\pgfpathlineto{\pgfqpoint{1.635963in}{5.561237in}}%
\pgfpathlineto{\pgfqpoint{1.651396in}{5.566401in}}%
\pgfpathlineto{\pgfqpoint{1.666830in}{5.567701in}}%
\pgfpathlineto{\pgfqpoint{1.682264in}{5.565237in}}%
\pgfpathlineto{\pgfqpoint{1.697697in}{5.559155in}}%
\pgfpathlineto{\pgfqpoint{1.713131in}{5.549641in}}%
\pgfpathlineto{\pgfqpoint{1.728564in}{5.536913in}}%
\pgfpathlineto{\pgfqpoint{1.743998in}{5.521223in}}%
\pgfpathlineto{\pgfqpoint{1.774865in}{5.482082in}}%
\pgfpathlineto{\pgfqpoint{1.805733in}{5.434637in}}%
\pgfpathlineto{\pgfqpoint{1.852034in}{5.353558in}}%
\pgfpathlineto{\pgfqpoint{1.929202in}{5.213064in}}%
\pgfpathlineto{\pgfqpoint{1.960069in}{5.161594in}}%
\pgfpathlineto{\pgfqpoint{1.990936in}{5.115631in}}%
\pgfpathlineto{\pgfqpoint{2.021804in}{5.076603in}}%
\pgfpathlineto{\pgfqpoint{2.052671in}{5.045574in}}%
\pgfpathlineto{\pgfqpoint{2.068104in}{5.033279in}}%
\pgfpathlineto{\pgfqpoint{2.083538in}{5.023192in}}%
\pgfpathlineto{\pgfqpoint{2.098972in}{5.015319in}}%
\pgfpathlineto{\pgfqpoint{2.114405in}{5.009635in}}%
\pgfpathlineto{\pgfqpoint{2.129839in}{5.006081in}}%
\pgfpathlineto{\pgfqpoint{2.145273in}{5.004561in}}%
\pgfpathlineto{\pgfqpoint{2.160706in}{5.004945in}}%
\pgfpathlineto{\pgfqpoint{2.191574in}{5.010734in}}%
\pgfpathlineto{\pgfqpoint{2.222441in}{5.021742in}}%
\pgfpathlineto{\pgfqpoint{2.330476in}{5.068561in}}%
\pgfpathlineto{\pgfqpoint{2.361344in}{5.074535in}}%
\pgfpathlineto{\pgfqpoint{2.376777in}{5.075055in}}%
\pgfpathlineto{\pgfqpoint{2.392211in}{5.073727in}}%
\pgfpathlineto{\pgfqpoint{2.407644in}{5.070457in}}%
\pgfpathlineto{\pgfqpoint{2.423078in}{5.065202in}}%
\pgfpathlineto{\pgfqpoint{2.438512in}{5.057970in}}%
\pgfpathlineto{\pgfqpoint{2.453945in}{5.048817in}}%
\pgfpathlineto{\pgfqpoint{2.484813in}{5.025228in}}%
\pgfpathlineto{\pgfqpoint{2.515680in}{4.995821in}}%
\pgfpathlineto{\pgfqpoint{2.561981in}{4.945211in}}%
\pgfpathlineto{\pgfqpoint{2.608282in}{4.894016in}}%
\pgfpathlineto{\pgfqpoint{2.639149in}{4.863650in}}%
\pgfpathlineto{\pgfqpoint{2.670016in}{4.838824in}}%
\pgfpathlineto{\pgfqpoint{2.685450in}{4.829030in}}%
\pgfpathlineto{\pgfqpoint{2.700884in}{4.821200in}}%
\pgfpathlineto{\pgfqpoint{2.716317in}{4.815445in}}%
\pgfpathlineto{\pgfqpoint{2.731751in}{4.811833in}}%
\pgfpathlineto{\pgfqpoint{2.747184in}{4.810386in}}%
\pgfpathlineto{\pgfqpoint{2.762618in}{4.811083in}}%
\pgfpathlineto{\pgfqpoint{2.778052in}{4.813854in}}%
\pgfpathlineto{\pgfqpoint{2.793485in}{4.818586in}}%
\pgfpathlineto{\pgfqpoint{2.824353in}{4.833278in}}%
\pgfpathlineto{\pgfqpoint{2.855220in}{4.853477in}}%
\pgfpathlineto{\pgfqpoint{2.901521in}{4.889297in}}%
\pgfpathlineto{\pgfqpoint{2.947822in}{4.924308in}}%
\pgfpathlineto{\pgfqpoint{2.978689in}{4.943165in}}%
\pgfpathlineto{\pgfqpoint{3.009556in}{4.956068in}}%
\pgfpathlineto{\pgfqpoint{3.024990in}{4.959818in}}%
\pgfpathlineto{\pgfqpoint{3.040424in}{4.961599in}}%
\pgfpathlineto{\pgfqpoint{3.055857in}{4.961341in}}%
\pgfpathlineto{\pgfqpoint{3.071291in}{4.959021in}}%
\pgfpathlineto{\pgfqpoint{3.086724in}{4.954660in}}%
\pgfpathlineto{\pgfqpoint{3.102158in}{4.948324in}}%
\pgfpathlineto{\pgfqpoint{3.133025in}{4.930201in}}%
\pgfpathlineto{\pgfqpoint{3.163893in}{4.905942in}}%
\pgfpathlineto{\pgfqpoint{3.194760in}{4.877287in}}%
\pgfpathlineto{\pgfqpoint{3.302795in}{4.771323in}}%
\pgfpathlineto{\pgfqpoint{3.333663in}{4.746991in}}%
\pgfpathlineto{\pgfqpoint{3.364530in}{4.727740in}}%
\pgfpathlineto{\pgfqpoint{3.395397in}{4.714218in}}%
\pgfpathlineto{\pgfqpoint{3.426264in}{4.706650in}}%
\pgfpathlineto{\pgfqpoint{3.457132in}{4.704898in}}%
\pgfpathlineto{\pgfqpoint{3.487999in}{4.708551in}}%
\pgfpathlineto{\pgfqpoint{3.518866in}{4.717035in}}%
\pgfpathlineto{\pgfqpoint{3.549733in}{4.729723in}}%
\pgfpathlineto{\pgfqpoint{3.580601in}{4.746031in}}%
\pgfpathlineto{\pgfqpoint{3.611468in}{4.765485in}}%
\pgfpathlineto{\pgfqpoint{3.642335in}{4.787749in}}%
\pgfpathlineto{\pgfqpoint{3.688636in}{4.825953in}}%
\pgfpathlineto{\pgfqpoint{3.734937in}{4.869431in}}%
\pgfpathlineto{\pgfqpoint{3.781238in}{4.917242in}}%
\pgfpathlineto{\pgfqpoint{3.889273in}{5.032474in}}%
\pgfpathlineto{\pgfqpoint{3.920141in}{5.061357in}}%
\pgfpathlineto{\pgfqpoint{3.951008in}{5.086114in}}%
\pgfpathlineto{\pgfqpoint{3.981875in}{5.105660in}}%
\pgfpathlineto{\pgfqpoint{4.012743in}{5.119245in}}%
\pgfpathlineto{\pgfqpoint{4.043610in}{5.126561in}}%
\pgfpathlineto{\pgfqpoint{4.074477in}{5.127807in}}%
\pgfpathlineto{\pgfqpoint{4.105344in}{5.123701in}}%
\pgfpathlineto{\pgfqpoint{4.136212in}{5.115426in}}%
\pgfpathlineto{\pgfqpoint{4.182513in}{5.098668in}}%
\pgfpathlineto{\pgfqpoint{4.228813in}{5.082138in}}%
\pgfpathlineto{\pgfqpoint{4.259681in}{5.074225in}}%
\pgfpathlineto{\pgfqpoint{4.290548in}{5.070550in}}%
\pgfpathlineto{\pgfqpoint{4.321415in}{5.072152in}}%
\pgfpathlineto{\pgfqpoint{4.352283in}{5.079529in}}%
\pgfpathlineto{\pgfqpoint{4.383150in}{5.092582in}}%
\pgfpathlineto{\pgfqpoint{4.414017in}{5.110618in}}%
\pgfpathlineto{\pgfqpoint{4.444884in}{5.132417in}}%
\pgfpathlineto{\pgfqpoint{4.552920in}{5.213746in}}%
\pgfpathlineto{\pgfqpoint{4.583787in}{5.231343in}}%
\pgfpathlineto{\pgfqpoint{4.614654in}{5.244112in}}%
\pgfpathlineto{\pgfqpoint{4.645522in}{5.251561in}}%
\pgfpathlineto{\pgfqpoint{4.660955in}{5.253283in}}%
\pgfpathlineto{\pgfqpoint{4.660955in}{5.253283in}}%
\pgfusepath{stroke}%
\end{pgfscope}%
\begin{pgfscope}%
\pgfpathrectangle{\pgfqpoint{0.625831in}{3.799602in}}{\pgfqpoint{4.227273in}{2.745455in}} %
\pgfusepath{clip}%
\pgfsetrectcap%
\pgfsetroundjoin%
\pgfsetlinewidth{0.501875pt}%
\definecolor{currentstroke}{rgb}{0.000000,0.000000,0.000000}%
\pgfsetstrokecolor{currentstroke}%
\pgfsetdash{}{0pt}%
\pgfpathmoveto{\pgfqpoint{0.817980in}{5.552557in}}%
\pgfpathlineto{\pgfqpoint{0.833414in}{5.551201in}}%
\pgfpathlineto{\pgfqpoint{0.848847in}{5.546088in}}%
\pgfpathlineto{\pgfqpoint{0.864281in}{5.537075in}}%
\pgfpathlineto{\pgfqpoint{0.879715in}{5.524083in}}%
\pgfpathlineto{\pgfqpoint{0.895148in}{5.507104in}}%
\pgfpathlineto{\pgfqpoint{0.910582in}{5.486200in}}%
\pgfpathlineto{\pgfqpoint{0.926015in}{5.461511in}}%
\pgfpathlineto{\pgfqpoint{0.941449in}{5.433250in}}%
\pgfpathlineto{\pgfqpoint{0.972316in}{5.367232in}}%
\pgfpathlineto{\pgfqpoint{1.003184in}{5.291249in}}%
\pgfpathlineto{\pgfqpoint{1.111219in}{5.009702in}}%
\pgfpathlineto{\pgfqpoint{1.142086in}{4.944658in}}%
\pgfpathlineto{\pgfqpoint{1.157520in}{4.917498in}}%
\pgfpathlineto{\pgfqpoint{1.172954in}{4.894499in}}%
\pgfpathlineto{\pgfqpoint{1.188387in}{4.876021in}}%
\pgfpathlineto{\pgfqpoint{1.203821in}{4.862350in}}%
\pgfpathlineto{\pgfqpoint{1.219255in}{4.853693in}}%
\pgfpathlineto{\pgfqpoint{1.234688in}{4.850176in}}%
\pgfpathlineto{\pgfqpoint{1.250122in}{4.851842in}}%
\pgfpathlineto{\pgfqpoint{1.265555in}{4.858650in}}%
\pgfpathlineto{\pgfqpoint{1.280989in}{4.870479in}}%
\pgfpathlineto{\pgfqpoint{1.296423in}{4.887129in}}%
\pgfpathlineto{\pgfqpoint{1.311856in}{4.908326in}}%
\pgfpathlineto{\pgfqpoint{1.327290in}{4.933728in}}%
\pgfpathlineto{\pgfqpoint{1.342724in}{4.962935in}}%
\pgfpathlineto{\pgfqpoint{1.373591in}{5.030900in}}%
\pgfpathlineto{\pgfqpoint{1.404458in}{5.108121in}}%
\pgfpathlineto{\pgfqpoint{1.497060in}{5.350286in}}%
\pgfpathlineto{\pgfqpoint{1.527927in}{5.420241in}}%
\pgfpathlineto{\pgfqpoint{1.558795in}{5.478943in}}%
\pgfpathlineto{\pgfqpoint{1.574228in}{5.503299in}}%
\pgfpathlineto{\pgfqpoint{1.589662in}{5.524022in}}%
\pgfpathlineto{\pgfqpoint{1.605095in}{5.540960in}}%
\pgfpathlineto{\pgfqpoint{1.620529in}{5.554016in}}%
\pgfpathlineto{\pgfqpoint{1.635963in}{5.563151in}}%
\pgfpathlineto{\pgfqpoint{1.651396in}{5.568381in}}%
\pgfpathlineto{\pgfqpoint{1.666830in}{5.569772in}}%
\pgfpathlineto{\pgfqpoint{1.682264in}{5.567440in}}%
\pgfpathlineto{\pgfqpoint{1.697697in}{5.561542in}}%
\pgfpathlineto{\pgfqpoint{1.713131in}{5.552276in}}%
\pgfpathlineto{\pgfqpoint{1.728564in}{5.539870in}}%
\pgfpathlineto{\pgfqpoint{1.743998in}{5.524583in}}%
\pgfpathlineto{\pgfqpoint{1.774865in}{5.486499in}}%
\pgfpathlineto{\pgfqpoint{1.805733in}{5.440422in}}%
\pgfpathlineto{\pgfqpoint{1.852034in}{5.361781in}}%
\pgfpathlineto{\pgfqpoint{1.944635in}{5.199189in}}%
\pgfpathlineto{\pgfqpoint{1.975503in}{5.150825in}}%
\pgfpathlineto{\pgfqpoint{2.006370in}{5.108083in}}%
\pgfpathlineto{\pgfqpoint{2.037237in}{5.072182in}}%
\pgfpathlineto{\pgfqpoint{2.068104in}{5.044027in}}%
\pgfpathlineto{\pgfqpoint{2.083538in}{5.033040in}}%
\pgfpathlineto{\pgfqpoint{2.098972in}{5.024159in}}%
\pgfpathlineto{\pgfqpoint{2.114405in}{5.017382in}}%
\pgfpathlineto{\pgfqpoint{2.129839in}{5.012676in}}%
\pgfpathlineto{\pgfqpoint{2.145273in}{5.009971in}}%
\pgfpathlineto{\pgfqpoint{2.160706in}{5.009165in}}%
\pgfpathlineto{\pgfqpoint{2.176140in}{5.010119in}}%
\pgfpathlineto{\pgfqpoint{2.207007in}{5.016579in}}%
\pgfpathlineto{\pgfqpoint{2.237874in}{5.027573in}}%
\pgfpathlineto{\pgfqpoint{2.315043in}{5.059975in}}%
\pgfpathlineto{\pgfqpoint{2.345910in}{5.068594in}}%
\pgfpathlineto{\pgfqpoint{2.361344in}{5.070900in}}%
\pgfpathlineto{\pgfqpoint{2.376777in}{5.071602in}}%
\pgfpathlineto{\pgfqpoint{2.392211in}{5.070554in}}%
\pgfpathlineto{\pgfqpoint{2.407644in}{5.067657in}}%
\pgfpathlineto{\pgfqpoint{2.423078in}{5.062863in}}%
\pgfpathlineto{\pgfqpoint{2.438512in}{5.056177in}}%
\pgfpathlineto{\pgfqpoint{2.469379in}{5.037404in}}%
\pgfpathlineto{\pgfqpoint{2.500246in}{5.012364in}}%
\pgfpathlineto{\pgfqpoint{2.531114in}{4.982738in}}%
\pgfpathlineto{\pgfqpoint{2.623715in}{4.888684in}}%
\pgfpathlineto{\pgfqpoint{2.654583in}{4.863290in}}%
\pgfpathlineto{\pgfqpoint{2.685450in}{4.844096in}}%
\pgfpathlineto{\pgfqpoint{2.700884in}{4.837228in}}%
\pgfpathlineto{\pgfqpoint{2.716317in}{4.832317in}}%
\pgfpathlineto{\pgfqpoint{2.731751in}{4.829418in}}%
\pgfpathlineto{\pgfqpoint{2.747184in}{4.828541in}}%
\pgfpathlineto{\pgfqpoint{2.762618in}{4.829655in}}%
\pgfpathlineto{\pgfqpoint{2.778052in}{4.832686in}}%
\pgfpathlineto{\pgfqpoint{2.793485in}{4.837519in}}%
\pgfpathlineto{\pgfqpoint{2.824353in}{4.851939in}}%
\pgfpathlineto{\pgfqpoint{2.855220in}{4.871269in}}%
\pgfpathlineto{\pgfqpoint{2.947822in}{4.936576in}}%
\pgfpathlineto{\pgfqpoint{2.978689in}{4.952878in}}%
\pgfpathlineto{\pgfqpoint{2.994123in}{4.958812in}}%
\pgfpathlineto{\pgfqpoint{3.009556in}{4.963008in}}%
\pgfpathlineto{\pgfqpoint{3.024990in}{4.965320in}}%
\pgfpathlineto{\pgfqpoint{3.040424in}{4.965648in}}%
\pgfpathlineto{\pgfqpoint{3.055857in}{4.963937in}}%
\pgfpathlineto{\pgfqpoint{3.071291in}{4.960183in}}%
\pgfpathlineto{\pgfqpoint{3.086724in}{4.954425in}}%
\pgfpathlineto{\pgfqpoint{3.102158in}{4.946752in}}%
\pgfpathlineto{\pgfqpoint{3.133025in}{4.926216in}}%
\pgfpathlineto{\pgfqpoint{3.163893in}{4.900038in}}%
\pgfpathlineto{\pgfqpoint{3.210193in}{4.854411in}}%
\pgfpathlineto{\pgfqpoint{3.271928in}{4.792783in}}%
\pgfpathlineto{\pgfqpoint{3.302795in}{4.765972in}}%
\pgfpathlineto{\pgfqpoint{3.333663in}{4.743773in}}%
\pgfpathlineto{\pgfqpoint{3.364530in}{4.727067in}}%
\pgfpathlineto{\pgfqpoint{3.395397in}{4.716250in}}%
\pgfpathlineto{\pgfqpoint{3.426264in}{4.711283in}}%
\pgfpathlineto{\pgfqpoint{3.457132in}{4.711790in}}%
\pgfpathlineto{\pgfqpoint{3.487999in}{4.717175in}}%
\pgfpathlineto{\pgfqpoint{3.518866in}{4.726759in}}%
\pgfpathlineto{\pgfqpoint{3.549733in}{4.739903in}}%
\pgfpathlineto{\pgfqpoint{3.580601in}{4.756101in}}%
\pgfpathlineto{\pgfqpoint{3.611468in}{4.775030in}}%
\pgfpathlineto{\pgfqpoint{3.642335in}{4.796552in}}%
\pgfpathlineto{\pgfqpoint{3.673203in}{4.820661in}}%
\pgfpathlineto{\pgfqpoint{3.719503in}{4.861736in}}%
\pgfpathlineto{\pgfqpoint{3.765804in}{4.908379in}}%
\pgfpathlineto{\pgfqpoint{3.827539in}{4.976470in}}%
\pgfpathlineto{\pgfqpoint{3.889273in}{5.044012in}}%
\pgfpathlineto{\pgfqpoint{3.920141in}{5.074201in}}%
\pgfpathlineto{\pgfqpoint{3.951008in}{5.100128in}}%
\pgfpathlineto{\pgfqpoint{3.981875in}{5.120616in}}%
\pgfpathlineto{\pgfqpoint{4.012743in}{5.134844in}}%
\pgfpathlineto{\pgfqpoint{4.028176in}{5.139479in}}%
\pgfpathlineto{\pgfqpoint{4.043610in}{5.142455in}}%
\pgfpathlineto{\pgfqpoint{4.074477in}{5.143621in}}%
\pgfpathlineto{\pgfqpoint{4.105344in}{5.139043in}}%
\pgfpathlineto{\pgfqpoint{4.136212in}{5.129910in}}%
\pgfpathlineto{\pgfqpoint{4.182513in}{5.111199in}}%
\pgfpathlineto{\pgfqpoint{4.228813in}{5.092110in}}%
\pgfpathlineto{\pgfqpoint{4.259681in}{5.082327in}}%
\pgfpathlineto{\pgfqpoint{4.290548in}{5.076809in}}%
\pgfpathlineto{\pgfqpoint{4.321415in}{5.076735in}}%
\pgfpathlineto{\pgfqpoint{4.352283in}{5.082733in}}%
\pgfpathlineto{\pgfqpoint{4.383150in}{5.094800in}}%
\pgfpathlineto{\pgfqpoint{4.414017in}{5.112296in}}%
\pgfpathlineto{\pgfqpoint{4.444884in}{5.134001in}}%
\pgfpathlineto{\pgfqpoint{4.568353in}{5.227527in}}%
\pgfpathlineto{\pgfqpoint{4.599221in}{5.243898in}}%
\pgfpathlineto{\pgfqpoint{4.630088in}{5.255134in}}%
\pgfpathlineto{\pgfqpoint{4.660955in}{5.260968in}}%
\pgfpathlineto{\pgfqpoint{4.660955in}{5.260968in}}%
\pgfusepath{stroke}%
\end{pgfscope}%
\begin{pgfscope}%
\pgfsetrectcap%
\pgfsetmiterjoin%
\pgfsetlinewidth{0.803000pt}%
\definecolor{currentstroke}{rgb}{0.000000,0.000000,0.000000}%
\pgfsetstrokecolor{currentstroke}%
\pgfsetdash{}{0pt}%
\pgfpathmoveto{\pgfqpoint{0.625831in}{3.799602in}}%
\pgfpathlineto{\pgfqpoint{0.625831in}{6.545056in}}%
\pgfusepath{stroke}%
\end{pgfscope}%
\begin{pgfscope}%
\pgfsetrectcap%
\pgfsetmiterjoin%
\pgfsetlinewidth{0.803000pt}%
\definecolor{currentstroke}{rgb}{0.000000,0.000000,0.000000}%
\pgfsetstrokecolor{currentstroke}%
\pgfsetdash{}{0pt}%
\pgfpathmoveto{\pgfqpoint{4.853104in}{3.799602in}}%
\pgfpathlineto{\pgfqpoint{4.853104in}{6.545056in}}%
\pgfusepath{stroke}%
\end{pgfscope}%
\begin{pgfscope}%
\pgfsetrectcap%
\pgfsetmiterjoin%
\pgfsetlinewidth{0.803000pt}%
\definecolor{currentstroke}{rgb}{0.000000,0.000000,0.000000}%
\pgfsetstrokecolor{currentstroke}%
\pgfsetdash{}{0pt}%
\pgfpathmoveto{\pgfqpoint{0.625831in}{3.799602in}}%
\pgfpathlineto{\pgfqpoint{4.853104in}{3.799602in}}%
\pgfusepath{stroke}%
\end{pgfscope}%
\begin{pgfscope}%
\pgfsetrectcap%
\pgfsetmiterjoin%
\pgfsetlinewidth{0.803000pt}%
\definecolor{currentstroke}{rgb}{0.000000,0.000000,0.000000}%
\pgfsetstrokecolor{currentstroke}%
\pgfsetdash{}{0pt}%
\pgfpathmoveto{\pgfqpoint{0.625831in}{6.545056in}}%
\pgfpathlineto{\pgfqpoint{4.853104in}{6.545056in}}%
\pgfusepath{stroke}%
\end{pgfscope}%
\begin{pgfscope}%
\pgftext[x=5.275831in,y=6.628389in,,base]{\rmfamily\fontsize{12.000000}{14.400000}\selectfont \(\displaystyle \widetilde{K}u \approx Ku\), realization 1}%
\end{pgfscope}%
\begin{pgfscope}%
\pgfsetbuttcap%
\pgfsetmiterjoin%
\definecolor{currentfill}{rgb}{1.000000,1.000000,1.000000}%
\pgfsetfillcolor{currentfill}%
\pgfsetfillopacity{0.800000}%
\pgfsetlinewidth{1.003750pt}%
\definecolor{currentstroke}{rgb}{0.800000,0.800000,0.800000}%
\pgfsetstrokecolor{currentstroke}%
\pgfsetstrokeopacity{0.800000}%
\pgfsetdash{}{0pt}%
\pgfpathmoveto{\pgfqpoint{4.211136in}{3.869046in}}%
\pgfpathlineto{\pgfqpoint{4.755882in}{3.869046in}}%
\pgfpathquadraticcurveto{\pgfqpoint{4.783660in}{3.869046in}}{\pgfqpoint{4.783660in}{3.896824in}}%
\pgfpathlineto{\pgfqpoint{4.783660in}{4.079305in}}%
\pgfpathquadraticcurveto{\pgfqpoint{4.783660in}{4.107083in}}{\pgfqpoint{4.755882in}{4.107083in}}%
\pgfpathlineto{\pgfqpoint{4.211136in}{4.107083in}}%
\pgfpathquadraticcurveto{\pgfqpoint{4.183358in}{4.107083in}}{\pgfqpoint{4.183358in}{4.079305in}}%
\pgfpathlineto{\pgfqpoint{4.183358in}{3.896824in}}%
\pgfpathquadraticcurveto{\pgfqpoint{4.183358in}{3.869046in}}{\pgfqpoint{4.211136in}{3.869046in}}%
\pgfpathclose%
\pgfusepath{stroke,fill}%
\end{pgfscope}%
\begin{pgfscope}%
\pgfsetrectcap%
\pgfsetroundjoin%
\pgfsetlinewidth{0.501875pt}%
\definecolor{currentstroke}{rgb}{0.000000,0.000000,0.000000}%
\pgfsetstrokecolor{currentstroke}%
\pgfsetdash{}{0pt}%
\pgfpathmoveto{\pgfqpoint{4.238914in}{4.002916in}}%
\pgfpathlineto{\pgfqpoint{4.516692in}{4.002916in}}%
\pgfusepath{stroke}%
\end{pgfscope}%
\begin{pgfscope}%
\pgftext[x=4.627803in,y=3.954305in,left,base]{\rmfamily\fontsize{10.000000}{12.000000}\selectfont K}%
\end{pgfscope}%
\begin{pgfscope}%
\pgfsetbuttcap%
\pgfsetmiterjoin%
\definecolor{currentfill}{rgb}{1.000000,1.000000,1.000000}%
\pgfsetfillcolor{currentfill}%
\pgfsetlinewidth{0.000000pt}%
\definecolor{currentstroke}{rgb}{0.000000,0.000000,0.000000}%
\pgfsetstrokecolor{currentstroke}%
\pgfsetstrokeopacity{0.000000}%
\pgfsetdash{}{0pt}%
\pgfpathmoveto{\pgfqpoint{5.698559in}{3.799602in}}%
\pgfpathlineto{\pgfqpoint{9.925831in}{3.799602in}}%
\pgfpathlineto{\pgfqpoint{9.925831in}{6.545056in}}%
\pgfpathlineto{\pgfqpoint{5.698559in}{6.545056in}}%
\pgfpathclose%
\pgfusepath{fill}%
\end{pgfscope}%
\begin{pgfscope}%
\pgfsetbuttcap%
\pgfsetroundjoin%
\definecolor{currentfill}{rgb}{0.000000,0.000000,0.000000}%
\pgfsetfillcolor{currentfill}%
\pgfsetlinewidth{0.803000pt}%
\definecolor{currentstroke}{rgb}{0.000000,0.000000,0.000000}%
\pgfsetstrokecolor{currentstroke}%
\pgfsetdash{}{0pt}%
\pgfsys@defobject{currentmarker}{\pgfqpoint{0.000000in}{-0.048611in}}{\pgfqpoint{0.000000in}{0.000000in}}{%
\pgfpathmoveto{\pgfqpoint{0.000000in}{0.000000in}}%
\pgfpathlineto{\pgfqpoint{0.000000in}{-0.048611in}}%
\pgfusepath{stroke,fill}%
}%
\begin{pgfscope}%
\pgfsys@transformshift{5.890707in}{3.799602in}%
\pgfsys@useobject{currentmarker}{}%
\end{pgfscope}%
\end{pgfscope}%
\begin{pgfscope}%
\pgfsetbuttcap%
\pgfsetroundjoin%
\definecolor{currentfill}{rgb}{0.000000,0.000000,0.000000}%
\pgfsetfillcolor{currentfill}%
\pgfsetlinewidth{0.803000pt}%
\definecolor{currentstroke}{rgb}{0.000000,0.000000,0.000000}%
\pgfsetstrokecolor{currentstroke}%
\pgfsetdash{}{0pt}%
\pgfsys@defobject{currentmarker}{\pgfqpoint{0.000000in}{-0.048611in}}{\pgfqpoint{0.000000in}{0.000000in}}{%
\pgfpathmoveto{\pgfqpoint{0.000000in}{0.000000in}}%
\pgfpathlineto{\pgfqpoint{0.000000in}{-0.048611in}}%
\pgfusepath{stroke,fill}%
}%
\begin{pgfscope}%
\pgfsys@transformshift{6.371079in}{3.799602in}%
\pgfsys@useobject{currentmarker}{}%
\end{pgfscope}%
\end{pgfscope}%
\begin{pgfscope}%
\pgfsetbuttcap%
\pgfsetroundjoin%
\definecolor{currentfill}{rgb}{0.000000,0.000000,0.000000}%
\pgfsetfillcolor{currentfill}%
\pgfsetlinewidth{0.803000pt}%
\definecolor{currentstroke}{rgb}{0.000000,0.000000,0.000000}%
\pgfsetstrokecolor{currentstroke}%
\pgfsetdash{}{0pt}%
\pgfsys@defobject{currentmarker}{\pgfqpoint{0.000000in}{-0.048611in}}{\pgfqpoint{0.000000in}{0.000000in}}{%
\pgfpathmoveto{\pgfqpoint{0.000000in}{0.000000in}}%
\pgfpathlineto{\pgfqpoint{0.000000in}{-0.048611in}}%
\pgfusepath{stroke,fill}%
}%
\begin{pgfscope}%
\pgfsys@transformshift{6.851451in}{3.799602in}%
\pgfsys@useobject{currentmarker}{}%
\end{pgfscope}%
\end{pgfscope}%
\begin{pgfscope}%
\pgfsetbuttcap%
\pgfsetroundjoin%
\definecolor{currentfill}{rgb}{0.000000,0.000000,0.000000}%
\pgfsetfillcolor{currentfill}%
\pgfsetlinewidth{0.803000pt}%
\definecolor{currentstroke}{rgb}{0.000000,0.000000,0.000000}%
\pgfsetstrokecolor{currentstroke}%
\pgfsetdash{}{0pt}%
\pgfsys@defobject{currentmarker}{\pgfqpoint{0.000000in}{-0.048611in}}{\pgfqpoint{0.000000in}{0.000000in}}{%
\pgfpathmoveto{\pgfqpoint{0.000000in}{0.000000in}}%
\pgfpathlineto{\pgfqpoint{0.000000in}{-0.048611in}}%
\pgfusepath{stroke,fill}%
}%
\begin{pgfscope}%
\pgfsys@transformshift{7.331823in}{3.799602in}%
\pgfsys@useobject{currentmarker}{}%
\end{pgfscope}%
\end{pgfscope}%
\begin{pgfscope}%
\pgfsetbuttcap%
\pgfsetroundjoin%
\definecolor{currentfill}{rgb}{0.000000,0.000000,0.000000}%
\pgfsetfillcolor{currentfill}%
\pgfsetlinewidth{0.803000pt}%
\definecolor{currentstroke}{rgb}{0.000000,0.000000,0.000000}%
\pgfsetstrokecolor{currentstroke}%
\pgfsetdash{}{0pt}%
\pgfsys@defobject{currentmarker}{\pgfqpoint{0.000000in}{-0.048611in}}{\pgfqpoint{0.000000in}{0.000000in}}{%
\pgfpathmoveto{\pgfqpoint{0.000000in}{0.000000in}}%
\pgfpathlineto{\pgfqpoint{0.000000in}{-0.048611in}}%
\pgfusepath{stroke,fill}%
}%
\begin{pgfscope}%
\pgfsys@transformshift{7.812195in}{3.799602in}%
\pgfsys@useobject{currentmarker}{}%
\end{pgfscope}%
\end{pgfscope}%
\begin{pgfscope}%
\pgfsetbuttcap%
\pgfsetroundjoin%
\definecolor{currentfill}{rgb}{0.000000,0.000000,0.000000}%
\pgfsetfillcolor{currentfill}%
\pgfsetlinewidth{0.803000pt}%
\definecolor{currentstroke}{rgb}{0.000000,0.000000,0.000000}%
\pgfsetstrokecolor{currentstroke}%
\pgfsetdash{}{0pt}%
\pgfsys@defobject{currentmarker}{\pgfqpoint{0.000000in}{-0.048611in}}{\pgfqpoint{0.000000in}{0.000000in}}{%
\pgfpathmoveto{\pgfqpoint{0.000000in}{0.000000in}}%
\pgfpathlineto{\pgfqpoint{0.000000in}{-0.048611in}}%
\pgfusepath{stroke,fill}%
}%
\begin{pgfscope}%
\pgfsys@transformshift{8.292567in}{3.799602in}%
\pgfsys@useobject{currentmarker}{}%
\end{pgfscope}%
\end{pgfscope}%
\begin{pgfscope}%
\pgfsetbuttcap%
\pgfsetroundjoin%
\definecolor{currentfill}{rgb}{0.000000,0.000000,0.000000}%
\pgfsetfillcolor{currentfill}%
\pgfsetlinewidth{0.803000pt}%
\definecolor{currentstroke}{rgb}{0.000000,0.000000,0.000000}%
\pgfsetstrokecolor{currentstroke}%
\pgfsetdash{}{0pt}%
\pgfsys@defobject{currentmarker}{\pgfqpoint{0.000000in}{-0.048611in}}{\pgfqpoint{0.000000in}{0.000000in}}{%
\pgfpathmoveto{\pgfqpoint{0.000000in}{0.000000in}}%
\pgfpathlineto{\pgfqpoint{0.000000in}{-0.048611in}}%
\pgfusepath{stroke,fill}%
}%
\begin{pgfscope}%
\pgfsys@transformshift{8.772939in}{3.799602in}%
\pgfsys@useobject{currentmarker}{}%
\end{pgfscope}%
\end{pgfscope}%
\begin{pgfscope}%
\pgfsetbuttcap%
\pgfsetroundjoin%
\definecolor{currentfill}{rgb}{0.000000,0.000000,0.000000}%
\pgfsetfillcolor{currentfill}%
\pgfsetlinewidth{0.803000pt}%
\definecolor{currentstroke}{rgb}{0.000000,0.000000,0.000000}%
\pgfsetstrokecolor{currentstroke}%
\pgfsetdash{}{0pt}%
\pgfsys@defobject{currentmarker}{\pgfqpoint{0.000000in}{-0.048611in}}{\pgfqpoint{0.000000in}{0.000000in}}{%
\pgfpathmoveto{\pgfqpoint{0.000000in}{0.000000in}}%
\pgfpathlineto{\pgfqpoint{0.000000in}{-0.048611in}}%
\pgfusepath{stroke,fill}%
}%
\begin{pgfscope}%
\pgfsys@transformshift{9.253311in}{3.799602in}%
\pgfsys@useobject{currentmarker}{}%
\end{pgfscope}%
\end{pgfscope}%
\begin{pgfscope}%
\pgfsetbuttcap%
\pgfsetroundjoin%
\definecolor{currentfill}{rgb}{0.000000,0.000000,0.000000}%
\pgfsetfillcolor{currentfill}%
\pgfsetlinewidth{0.803000pt}%
\definecolor{currentstroke}{rgb}{0.000000,0.000000,0.000000}%
\pgfsetstrokecolor{currentstroke}%
\pgfsetdash{}{0pt}%
\pgfsys@defobject{currentmarker}{\pgfqpoint{0.000000in}{-0.048611in}}{\pgfqpoint{0.000000in}{0.000000in}}{%
\pgfpathmoveto{\pgfqpoint{0.000000in}{0.000000in}}%
\pgfpathlineto{\pgfqpoint{0.000000in}{-0.048611in}}%
\pgfusepath{stroke,fill}%
}%
\begin{pgfscope}%
\pgfsys@transformshift{9.733682in}{3.799602in}%
\pgfsys@useobject{currentmarker}{}%
\end{pgfscope}%
\end{pgfscope}%
\begin{pgfscope}%
\pgfsetbuttcap%
\pgfsetroundjoin%
\definecolor{currentfill}{rgb}{0.000000,0.000000,0.000000}%
\pgfsetfillcolor{currentfill}%
\pgfsetlinewidth{0.803000pt}%
\definecolor{currentstroke}{rgb}{0.000000,0.000000,0.000000}%
\pgfsetstrokecolor{currentstroke}%
\pgfsetdash{}{0pt}%
\pgfsys@defobject{currentmarker}{\pgfqpoint{-0.048611in}{0.000000in}}{\pgfqpoint{0.000000in}{0.000000in}}{%
\pgfpathmoveto{\pgfqpoint{0.000000in}{0.000000in}}%
\pgfpathlineto{\pgfqpoint{-0.048611in}{0.000000in}}%
\pgfusepath{stroke,fill}%
}%
\begin{pgfscope}%
\pgfsys@transformshift{5.698559in}{4.245102in}%
\pgfsys@useobject{currentmarker}{}%
\end{pgfscope}%
\end{pgfscope}%
\begin{pgfscope}%
\pgfsetbuttcap%
\pgfsetroundjoin%
\definecolor{currentfill}{rgb}{0.000000,0.000000,0.000000}%
\pgfsetfillcolor{currentfill}%
\pgfsetlinewidth{0.803000pt}%
\definecolor{currentstroke}{rgb}{0.000000,0.000000,0.000000}%
\pgfsetstrokecolor{currentstroke}%
\pgfsetdash{}{0pt}%
\pgfsys@defobject{currentmarker}{\pgfqpoint{-0.048611in}{0.000000in}}{\pgfqpoint{0.000000in}{0.000000in}}{%
\pgfpathmoveto{\pgfqpoint{0.000000in}{0.000000in}}%
\pgfpathlineto{\pgfqpoint{-0.048611in}{0.000000in}}%
\pgfusepath{stroke,fill}%
}%
\begin{pgfscope}%
\pgfsys@transformshift{5.698559in}{4.705406in}%
\pgfsys@useobject{currentmarker}{}%
\end{pgfscope}%
\end{pgfscope}%
\begin{pgfscope}%
\pgfsetbuttcap%
\pgfsetroundjoin%
\definecolor{currentfill}{rgb}{0.000000,0.000000,0.000000}%
\pgfsetfillcolor{currentfill}%
\pgfsetlinewidth{0.803000pt}%
\definecolor{currentstroke}{rgb}{0.000000,0.000000,0.000000}%
\pgfsetstrokecolor{currentstroke}%
\pgfsetdash{}{0pt}%
\pgfsys@defobject{currentmarker}{\pgfqpoint{-0.048611in}{0.000000in}}{\pgfqpoint{0.000000in}{0.000000in}}{%
\pgfpathmoveto{\pgfqpoint{0.000000in}{0.000000in}}%
\pgfpathlineto{\pgfqpoint{-0.048611in}{0.000000in}}%
\pgfusepath{stroke,fill}%
}%
\begin{pgfscope}%
\pgfsys@transformshift{5.698559in}{5.165710in}%
\pgfsys@useobject{currentmarker}{}%
\end{pgfscope}%
\end{pgfscope}%
\begin{pgfscope}%
\pgfsetbuttcap%
\pgfsetroundjoin%
\definecolor{currentfill}{rgb}{0.000000,0.000000,0.000000}%
\pgfsetfillcolor{currentfill}%
\pgfsetlinewidth{0.803000pt}%
\definecolor{currentstroke}{rgb}{0.000000,0.000000,0.000000}%
\pgfsetstrokecolor{currentstroke}%
\pgfsetdash{}{0pt}%
\pgfsys@defobject{currentmarker}{\pgfqpoint{-0.048611in}{0.000000in}}{\pgfqpoint{0.000000in}{0.000000in}}{%
\pgfpathmoveto{\pgfqpoint{0.000000in}{0.000000in}}%
\pgfpathlineto{\pgfqpoint{-0.048611in}{0.000000in}}%
\pgfusepath{stroke,fill}%
}%
\begin{pgfscope}%
\pgfsys@transformshift{5.698559in}{5.626014in}%
\pgfsys@useobject{currentmarker}{}%
\end{pgfscope}%
\end{pgfscope}%
\begin{pgfscope}%
\pgfsetbuttcap%
\pgfsetroundjoin%
\definecolor{currentfill}{rgb}{0.000000,0.000000,0.000000}%
\pgfsetfillcolor{currentfill}%
\pgfsetlinewidth{0.803000pt}%
\definecolor{currentstroke}{rgb}{0.000000,0.000000,0.000000}%
\pgfsetstrokecolor{currentstroke}%
\pgfsetdash{}{0pt}%
\pgfsys@defobject{currentmarker}{\pgfqpoint{-0.048611in}{0.000000in}}{\pgfqpoint{0.000000in}{0.000000in}}{%
\pgfpathmoveto{\pgfqpoint{0.000000in}{0.000000in}}%
\pgfpathlineto{\pgfqpoint{-0.048611in}{0.000000in}}%
\pgfusepath{stroke,fill}%
}%
\begin{pgfscope}%
\pgfsys@transformshift{5.698559in}{6.086318in}%
\pgfsys@useobject{currentmarker}{}%
\end{pgfscope}%
\end{pgfscope}%
\begin{pgfscope}%
\pgfsetbuttcap%
\pgfsetroundjoin%
\definecolor{currentfill}{rgb}{0.000000,0.000000,0.000000}%
\pgfsetfillcolor{currentfill}%
\pgfsetlinewidth{0.803000pt}%
\definecolor{currentstroke}{rgb}{0.000000,0.000000,0.000000}%
\pgfsetstrokecolor{currentstroke}%
\pgfsetdash{}{0pt}%
\pgfsys@defobject{currentmarker}{\pgfqpoint{-0.048611in}{0.000000in}}{\pgfqpoint{0.000000in}{0.000000in}}{%
\pgfpathmoveto{\pgfqpoint{0.000000in}{0.000000in}}%
\pgfpathlineto{\pgfqpoint{-0.048611in}{0.000000in}}%
\pgfusepath{stroke,fill}%
}%
\begin{pgfscope}%
\pgfsys@transformshift{5.698559in}{6.546622in}%
\pgfsys@useobject{currentmarker}{}%
\end{pgfscope}%
\end{pgfscope}%
\begin{pgfscope}%
\pgftext[x=5.643003in,y=5.172329in,,bottom,rotate=90.000000]{\rmfamily\fontsize{10.000000}{12.000000}\selectfont \(\displaystyle y_2\)}%
\end{pgfscope}%
\begin{pgfscope}%
\pgfpathrectangle{\pgfqpoint{5.698559in}{3.799602in}}{\pgfqpoint{4.227273in}{2.745455in}} %
\pgfusepath{clip}%
\pgfsetrectcap%
\pgfsetroundjoin%
\pgfsetlinewidth{0.501875pt}%
\definecolor{currentstroke}{rgb}{0.500000,0.000000,1.000000}%
\pgfsetstrokecolor{currentstroke}%
\pgfsetdash{}{0pt}%
\pgfpathmoveto{\pgfqpoint{5.890707in}{5.298468in}}%
\pgfpathlineto{\pgfqpoint{5.906141in}{5.451527in}}%
\pgfpathlineto{\pgfqpoint{5.921575in}{5.598859in}}%
\pgfpathlineto{\pgfqpoint{5.952442in}{5.864698in}}%
\pgfpathlineto{\pgfqpoint{5.967875in}{5.977877in}}%
\pgfpathlineto{\pgfqpoint{5.983309in}{6.074779in}}%
\pgfpathlineto{\pgfqpoint{5.998743in}{6.153460in}}%
\pgfpathlineto{\pgfqpoint{6.014176in}{6.212344in}}%
\pgfpathlineto{\pgfqpoint{6.029610in}{6.250251in}}%
\pgfpathlineto{\pgfqpoint{6.045044in}{6.266420in}}%
\pgfpathlineto{\pgfqpoint{6.060477in}{6.260529in}}%
\pgfpathlineto{\pgfqpoint{6.075911in}{6.232694in}}%
\pgfpathlineto{\pgfqpoint{6.091345in}{6.183473in}}%
\pgfpathlineto{\pgfqpoint{6.106778in}{6.113854in}}%
\pgfpathlineto{\pgfqpoint{6.122212in}{6.025232in}}%
\pgfpathlineto{\pgfqpoint{6.137645in}{5.919382in}}%
\pgfpathlineto{\pgfqpoint{6.153079in}{5.798426in}}%
\pgfpathlineto{\pgfqpoint{6.183946in}{5.521150in}}%
\pgfpathlineto{\pgfqpoint{6.276548in}{4.617038in}}%
\pgfpathlineto{\pgfqpoint{6.291982in}{4.487687in}}%
\pgfpathlineto{\pgfqpoint{6.307415in}{4.371925in}}%
\pgfpathlineto{\pgfqpoint{6.322849in}{4.272073in}}%
\pgfpathlineto{\pgfqpoint{6.338283in}{4.190133in}}%
\pgfpathlineto{\pgfqpoint{6.353716in}{4.127745in}}%
\pgfpathlineto{\pgfqpoint{6.369150in}{4.086161in}}%
\pgfpathlineto{\pgfqpoint{6.384584in}{4.066214in}}%
\pgfpathlineto{\pgfqpoint{6.400017in}{4.068304in}}%
\pgfpathlineto{\pgfqpoint{6.415451in}{4.092389in}}%
\pgfpathlineto{\pgfqpoint{6.430885in}{4.137987in}}%
\pgfpathlineto{\pgfqpoint{6.446318in}{4.204183in}}%
\pgfpathlineto{\pgfqpoint{6.461752in}{4.289650in}}%
\pgfpathlineto{\pgfqpoint{6.477185in}{4.392677in}}%
\pgfpathlineto{\pgfqpoint{6.492619in}{4.511196in}}%
\pgfpathlineto{\pgfqpoint{6.523486in}{4.784953in}}%
\pgfpathlineto{\pgfqpoint{6.569787in}{5.244997in}}%
\pgfpathlineto{\pgfqpoint{6.600655in}{5.548967in}}%
\pgfpathlineto{\pgfqpoint{6.631522in}{5.822365in}}%
\pgfpathlineto{\pgfqpoint{6.646955in}{5.940639in}}%
\pgfpathlineto{\pgfqpoint{6.662389in}{6.043382in}}%
\pgfpathlineto{\pgfqpoint{6.677823in}{6.128534in}}%
\pgfpathlineto{\pgfqpoint{6.693256in}{6.194388in}}%
\pgfpathlineto{\pgfqpoint{6.708690in}{6.239624in}}%
\pgfpathlineto{\pgfqpoint{6.724124in}{6.263336in}}%
\pgfpathlineto{\pgfqpoint{6.739557in}{6.265049in}}%
\pgfpathlineto{\pgfqpoint{6.754991in}{6.244728in}}%
\pgfpathlineto{\pgfqpoint{6.770424in}{6.202781in}}%
\pgfpathlineto{\pgfqpoint{6.785858in}{6.140048in}}%
\pgfpathlineto{\pgfqpoint{6.801292in}{6.057786in}}%
\pgfpathlineto{\pgfqpoint{6.816725in}{5.957645in}}%
\pgfpathlineto{\pgfqpoint{6.832159in}{5.841631in}}%
\pgfpathlineto{\pgfqpoint{6.863026in}{5.571558in}}%
\pgfpathlineto{\pgfqpoint{6.909327in}{5.113237in}}%
\pgfpathlineto{\pgfqpoint{6.955628in}{4.664255in}}%
\pgfpathlineto{\pgfqpoint{6.986495in}{4.410097in}}%
\pgfpathlineto{\pgfqpoint{7.001929in}{4.304524in}}%
\pgfpathlineto{\pgfqpoint{7.017363in}{4.216212in}}%
\pgfpathlineto{\pgfqpoint{7.032796in}{4.146931in}}%
\pgfpathlineto{\pgfqpoint{7.048230in}{4.098069in}}%
\pgfpathlineto{\pgfqpoint{7.063664in}{4.070605in}}%
\pgfpathlineto{\pgfqpoint{7.079097in}{4.065091in}}%
\pgfpathlineto{\pgfqpoint{7.094531in}{4.081635in}}%
\pgfpathlineto{\pgfqpoint{7.109964in}{4.119908in}}%
\pgfpathlineto{\pgfqpoint{7.125398in}{4.179141in}}%
\pgfpathlineto{\pgfqpoint{7.140832in}{4.258148in}}%
\pgfpathlineto{\pgfqpoint{7.156265in}{4.355345in}}%
\pgfpathlineto{\pgfqpoint{7.171699in}{4.468784in}}%
\pgfpathlineto{\pgfqpoint{7.202566in}{4.735013in}}%
\pgfpathlineto{\pgfqpoint{7.248867in}{5.191337in}}%
\pgfpathlineto{\pgfqpoint{7.295168in}{5.643101in}}%
\pgfpathlineto{\pgfqpoint{7.326035in}{5.901557in}}%
\pgfpathlineto{\pgfqpoint{7.341469in}{6.009897in}}%
\pgfpathlineto{\pgfqpoint{7.356903in}{6.101316in}}%
\pgfpathlineto{\pgfqpoint{7.372336in}{6.173983in}}%
\pgfpathlineto{\pgfqpoint{7.387770in}{6.226442in}}%
\pgfpathlineto{\pgfqpoint{7.403204in}{6.257641in}}%
\pgfpathlineto{\pgfqpoint{7.418637in}{6.266954in}}%
\pgfpathlineto{\pgfqpoint{7.434071in}{6.254196in}}%
\pgfpathlineto{\pgfqpoint{7.449504in}{6.219621in}}%
\pgfpathlineto{\pgfqpoint{7.464938in}{6.163923in}}%
\pgfpathlineto{\pgfqpoint{7.480372in}{6.088218in}}%
\pgfpathlineto{\pgfqpoint{7.495805in}{5.994023in}}%
\pgfpathlineto{\pgfqpoint{7.511239in}{5.883227in}}%
\pgfpathlineto{\pgfqpoint{7.542106in}{5.621000in}}%
\pgfpathlineto{\pgfqpoint{7.572974in}{5.322455in}}%
\pgfpathlineto{\pgfqpoint{7.634708in}{4.712667in}}%
\pgfpathlineto{\pgfqpoint{7.665575in}{4.450066in}}%
\pgfpathlineto{\pgfqpoint{7.681009in}{4.339024in}}%
\pgfpathlineto{\pgfqpoint{7.696443in}{4.244551in}}%
\pgfpathlineto{\pgfqpoint{7.711876in}{4.168541in}}%
\pgfpathlineto{\pgfqpoint{7.727310in}{4.112517in}}%
\pgfpathlineto{\pgfqpoint{7.742744in}{4.077602in}}%
\pgfpathlineto{\pgfqpoint{7.758177in}{4.064496in}}%
\pgfpathlineto{\pgfqpoint{7.773611in}{4.073460in}}%
\pgfpathlineto{\pgfqpoint{7.789044in}{4.104317in}}%
\pgfpathlineto{\pgfqpoint{7.804478in}{4.156447in}}%
\pgfpathlineto{\pgfqpoint{7.819912in}{4.228805in}}%
\pgfpathlineto{\pgfqpoint{7.835345in}{4.319941in}}%
\pgfpathlineto{\pgfqpoint{7.850779in}{4.428029in}}%
\pgfpathlineto{\pgfqpoint{7.881646in}{4.686098in}}%
\pgfpathlineto{\pgfqpoint{7.912514in}{4.982425in}}%
\pgfpathlineto{\pgfqpoint{7.974248in}{5.594136in}}%
\pgfpathlineto{\pgfqpoint{8.005115in}{5.860725in}}%
\pgfpathlineto{\pgfqpoint{8.020549in}{5.974403in}}%
\pgfpathlineto{\pgfqpoint{8.035983in}{6.071872in}}%
\pgfpathlineto{\pgfqpoint{8.051416in}{6.151180in}}%
\pgfpathlineto{\pgfqpoint{8.066850in}{6.210736in}}%
\pgfpathlineto{\pgfqpoint{8.082284in}{6.249347in}}%
\pgfpathlineto{\pgfqpoint{8.097717in}{6.266239in}}%
\pgfpathlineto{\pgfqpoint{8.113151in}{6.261073in}}%
\pgfpathlineto{\pgfqpoint{8.128584in}{6.233953in}}%
\pgfpathlineto{\pgfqpoint{8.144018in}{6.185423in}}%
\pgfpathlineto{\pgfqpoint{8.159452in}{6.116454in}}%
\pgfpathlineto{\pgfqpoint{8.174885in}{6.028430in}}%
\pgfpathlineto{\pgfqpoint{8.190319in}{5.923115in}}%
\pgfpathlineto{\pgfqpoint{8.205753in}{5.802620in}}%
\pgfpathlineto{\pgfqpoint{8.236620in}{5.526003in}}%
\pgfpathlineto{\pgfqpoint{8.329222in}{4.621493in}}%
\pgfpathlineto{\pgfqpoint{8.344655in}{4.491738in}}%
\pgfpathlineto{\pgfqpoint{8.360089in}{4.375491in}}%
\pgfpathlineto{\pgfqpoint{8.375523in}{4.275082in}}%
\pgfpathlineto{\pgfqpoint{8.390956in}{4.192524in}}%
\pgfpathlineto{\pgfqpoint{8.406390in}{4.129471in}}%
\pgfpathlineto{\pgfqpoint{8.421823in}{4.087188in}}%
\pgfpathlineto{\pgfqpoint{8.437257in}{4.066521in}}%
\pgfpathlineto{\pgfqpoint{8.452691in}{4.067884in}}%
\pgfpathlineto{\pgfqpoint{8.468124in}{4.091252in}}%
\pgfpathlineto{\pgfqpoint{8.483558in}{4.136154in}}%
\pgfpathlineto{\pgfqpoint{8.498992in}{4.201691in}}%
\pgfpathlineto{\pgfqpoint{8.514425in}{4.286550in}}%
\pgfpathlineto{\pgfqpoint{8.529859in}{4.389030in}}%
\pgfpathlineto{\pgfqpoint{8.545293in}{4.507077in}}%
\pgfpathlineto{\pgfqpoint{8.576160in}{4.780142in}}%
\pgfpathlineto{\pgfqpoint{8.622461in}{5.239878in}}%
\pgfpathlineto{\pgfqpoint{8.653328in}{5.544152in}}%
\pgfpathlineto{\pgfqpoint{8.684195in}{5.818238in}}%
\pgfpathlineto{\pgfqpoint{8.699629in}{5.936985in}}%
\pgfpathlineto{\pgfqpoint{8.715063in}{6.040273in}}%
\pgfpathlineto{\pgfqpoint{8.730496in}{6.126032in}}%
\pgfpathlineto{\pgfqpoint{8.745930in}{6.192544in}}%
\pgfpathlineto{\pgfqpoint{8.761363in}{6.238475in}}%
\pgfpathlineto{\pgfqpoint{8.776797in}{6.262905in}}%
\pgfpathlineto{\pgfqpoint{8.792231in}{6.265344in}}%
\pgfpathlineto{\pgfqpoint{8.807664in}{6.245744in}}%
\pgfpathlineto{\pgfqpoint{8.823098in}{6.204496in}}%
\pgfpathlineto{\pgfqpoint{8.838532in}{6.142429in}}%
\pgfpathlineto{\pgfqpoint{8.853965in}{6.060785in}}%
\pgfpathlineto{\pgfqpoint{8.869399in}{5.961202in}}%
\pgfpathlineto{\pgfqpoint{8.884833in}{5.845675in}}%
\pgfpathlineto{\pgfqpoint{8.915700in}{5.576324in}}%
\pgfpathlineto{\pgfqpoint{8.962001in}{5.118363in}}%
\pgfpathlineto{\pgfqpoint{9.008302in}{4.668830in}}%
\pgfpathlineto{\pgfqpoint{9.039169in}{4.413838in}}%
\pgfpathlineto{\pgfqpoint{9.054603in}{4.307732in}}%
\pgfpathlineto{\pgfqpoint{9.070036in}{4.218822in}}%
\pgfpathlineto{\pgfqpoint{9.085470in}{4.148891in}}%
\pgfpathlineto{\pgfqpoint{9.100903in}{4.099340in}}%
\pgfpathlineto{\pgfqpoint{9.116337in}{4.071161in}}%
\pgfpathlineto{\pgfqpoint{9.131771in}{4.064921in}}%
\pgfpathlineto{\pgfqpoint{9.147204in}{4.080743in}}%
\pgfpathlineto{\pgfqpoint{9.162638in}{4.118311in}}%
\pgfpathlineto{\pgfqpoint{9.178072in}{4.176871in}}%
\pgfpathlineto{\pgfqpoint{9.193505in}{4.255251in}}%
\pgfpathlineto{\pgfqpoint{9.208939in}{4.351879in}}%
\pgfpathlineto{\pgfqpoint{9.224373in}{4.464818in}}%
\pgfpathlineto{\pgfqpoint{9.255240in}{4.730295in}}%
\pgfpathlineto{\pgfqpoint{9.301541in}{5.186207in}}%
\pgfpathlineto{\pgfqpoint{9.347842in}{5.638471in}}%
\pgfpathlineto{\pgfqpoint{9.378709in}{5.897732in}}%
\pgfpathlineto{\pgfqpoint{9.394143in}{6.006592in}}%
\pgfpathlineto{\pgfqpoint{9.409576in}{6.098599in}}%
\pgfpathlineto{\pgfqpoint{9.425010in}{6.171908in}}%
\pgfpathlineto{\pgfqpoint{9.440443in}{6.225050in}}%
\pgfpathlineto{\pgfqpoint{9.455877in}{6.256961in}}%
\pgfpathlineto{\pgfqpoint{9.471311in}{6.266999in}}%
\pgfpathlineto{\pgfqpoint{9.486744in}{6.254965in}}%
\pgfpathlineto{\pgfqpoint{9.502178in}{6.221099in}}%
\pgfpathlineto{\pgfqpoint{9.517612in}{6.166080in}}%
\pgfpathlineto{\pgfqpoint{9.533045in}{6.091010in}}%
\pgfpathlineto{\pgfqpoint{9.548479in}{5.997396in}}%
\pgfpathlineto{\pgfqpoint{9.563913in}{5.887112in}}%
\pgfpathlineto{\pgfqpoint{9.594780in}{5.625667in}}%
\pgfpathlineto{\pgfqpoint{9.625647in}{5.327532in}}%
\pgfpathlineto{\pgfqpoint{9.687382in}{4.717349in}}%
\pgfpathlineto{\pgfqpoint{9.718249in}{4.453974in}}%
\pgfpathlineto{\pgfqpoint{9.733682in}{4.342423in}}%
\pgfpathlineto{\pgfqpoint{9.733682in}{4.342423in}}%
\pgfusepath{stroke}%
\end{pgfscope}%
\begin{pgfscope}%
\pgfpathrectangle{\pgfqpoint{5.698559in}{3.799602in}}{\pgfqpoint{4.227273in}{2.745455in}} %
\pgfusepath{clip}%
\pgfsetrectcap%
\pgfsetroundjoin%
\pgfsetlinewidth{0.501875pt}%
\definecolor{currentstroke}{rgb}{0.421569,0.122888,0.998103}%
\pgfsetstrokecolor{currentstroke}%
\pgfsetdash{}{0pt}%
\pgfpathmoveto{\pgfqpoint{5.890707in}{5.298468in}}%
\pgfpathlineto{\pgfqpoint{5.906141in}{5.451527in}}%
\pgfpathlineto{\pgfqpoint{5.921575in}{5.598859in}}%
\pgfpathlineto{\pgfqpoint{5.952442in}{5.864698in}}%
\pgfpathlineto{\pgfqpoint{5.967875in}{5.977877in}}%
\pgfpathlineto{\pgfqpoint{5.983309in}{6.074779in}}%
\pgfpathlineto{\pgfqpoint{5.998743in}{6.153460in}}%
\pgfpathlineto{\pgfqpoint{6.014176in}{6.212344in}}%
\pgfpathlineto{\pgfqpoint{6.029610in}{6.250251in}}%
\pgfpathlineto{\pgfqpoint{6.045044in}{6.266420in}}%
\pgfpathlineto{\pgfqpoint{6.060477in}{6.260529in}}%
\pgfpathlineto{\pgfqpoint{6.075911in}{6.232694in}}%
\pgfpathlineto{\pgfqpoint{6.091345in}{6.183473in}}%
\pgfpathlineto{\pgfqpoint{6.106778in}{6.113854in}}%
\pgfpathlineto{\pgfqpoint{6.122212in}{6.025232in}}%
\pgfpathlineto{\pgfqpoint{6.137645in}{5.919382in}}%
\pgfpathlineto{\pgfqpoint{6.153079in}{5.798426in}}%
\pgfpathlineto{\pgfqpoint{6.183946in}{5.521150in}}%
\pgfpathlineto{\pgfqpoint{6.276548in}{4.617038in}}%
\pgfpathlineto{\pgfqpoint{6.291982in}{4.487687in}}%
\pgfpathlineto{\pgfqpoint{6.307415in}{4.371925in}}%
\pgfpathlineto{\pgfqpoint{6.322849in}{4.272073in}}%
\pgfpathlineto{\pgfqpoint{6.338283in}{4.190133in}}%
\pgfpathlineto{\pgfqpoint{6.353716in}{4.127745in}}%
\pgfpathlineto{\pgfqpoint{6.369150in}{4.086161in}}%
\pgfpathlineto{\pgfqpoint{6.384584in}{4.066214in}}%
\pgfpathlineto{\pgfqpoint{6.400017in}{4.068304in}}%
\pgfpathlineto{\pgfqpoint{6.415451in}{4.092389in}}%
\pgfpathlineto{\pgfqpoint{6.430885in}{4.137987in}}%
\pgfpathlineto{\pgfqpoint{6.446318in}{4.204183in}}%
\pgfpathlineto{\pgfqpoint{6.461752in}{4.289650in}}%
\pgfpathlineto{\pgfqpoint{6.477185in}{4.392677in}}%
\pgfpathlineto{\pgfqpoint{6.492619in}{4.511196in}}%
\pgfpathlineto{\pgfqpoint{6.523486in}{4.784953in}}%
\pgfpathlineto{\pgfqpoint{6.569787in}{5.244997in}}%
\pgfpathlineto{\pgfqpoint{6.600655in}{5.548967in}}%
\pgfpathlineto{\pgfqpoint{6.631522in}{5.822365in}}%
\pgfpathlineto{\pgfqpoint{6.646955in}{5.940639in}}%
\pgfpathlineto{\pgfqpoint{6.662389in}{6.043382in}}%
\pgfpathlineto{\pgfqpoint{6.677823in}{6.128534in}}%
\pgfpathlineto{\pgfqpoint{6.693256in}{6.194388in}}%
\pgfpathlineto{\pgfqpoint{6.708690in}{6.239624in}}%
\pgfpathlineto{\pgfqpoint{6.724124in}{6.263336in}}%
\pgfpathlineto{\pgfqpoint{6.739557in}{6.265049in}}%
\pgfpathlineto{\pgfqpoint{6.754991in}{6.244728in}}%
\pgfpathlineto{\pgfqpoint{6.770424in}{6.202781in}}%
\pgfpathlineto{\pgfqpoint{6.785858in}{6.140048in}}%
\pgfpathlineto{\pgfqpoint{6.801292in}{6.057786in}}%
\pgfpathlineto{\pgfqpoint{6.816725in}{5.957645in}}%
\pgfpathlineto{\pgfqpoint{6.832159in}{5.841631in}}%
\pgfpathlineto{\pgfqpoint{6.863026in}{5.571558in}}%
\pgfpathlineto{\pgfqpoint{6.909327in}{5.113237in}}%
\pgfpathlineto{\pgfqpoint{6.955628in}{4.664255in}}%
\pgfpathlineto{\pgfqpoint{6.986495in}{4.410097in}}%
\pgfpathlineto{\pgfqpoint{7.001929in}{4.304524in}}%
\pgfpathlineto{\pgfqpoint{7.017363in}{4.216212in}}%
\pgfpathlineto{\pgfqpoint{7.032796in}{4.146931in}}%
\pgfpathlineto{\pgfqpoint{7.048230in}{4.098069in}}%
\pgfpathlineto{\pgfqpoint{7.063664in}{4.070605in}}%
\pgfpathlineto{\pgfqpoint{7.079097in}{4.065091in}}%
\pgfpathlineto{\pgfqpoint{7.094531in}{4.081635in}}%
\pgfpathlineto{\pgfqpoint{7.109964in}{4.119908in}}%
\pgfpathlineto{\pgfqpoint{7.125398in}{4.179141in}}%
\pgfpathlineto{\pgfqpoint{7.140832in}{4.258148in}}%
\pgfpathlineto{\pgfqpoint{7.156265in}{4.355345in}}%
\pgfpathlineto{\pgfqpoint{7.171699in}{4.468784in}}%
\pgfpathlineto{\pgfqpoint{7.202566in}{4.735013in}}%
\pgfpathlineto{\pgfqpoint{7.248867in}{5.191337in}}%
\pgfpathlineto{\pgfqpoint{7.295168in}{5.643101in}}%
\pgfpathlineto{\pgfqpoint{7.326035in}{5.901557in}}%
\pgfpathlineto{\pgfqpoint{7.341469in}{6.009897in}}%
\pgfpathlineto{\pgfqpoint{7.356903in}{6.101316in}}%
\pgfpathlineto{\pgfqpoint{7.372336in}{6.173983in}}%
\pgfpathlineto{\pgfqpoint{7.387770in}{6.226442in}}%
\pgfpathlineto{\pgfqpoint{7.403204in}{6.257641in}}%
\pgfpathlineto{\pgfqpoint{7.418637in}{6.266954in}}%
\pgfpathlineto{\pgfqpoint{7.434071in}{6.254196in}}%
\pgfpathlineto{\pgfqpoint{7.449504in}{6.219621in}}%
\pgfpathlineto{\pgfqpoint{7.464938in}{6.163923in}}%
\pgfpathlineto{\pgfqpoint{7.480372in}{6.088218in}}%
\pgfpathlineto{\pgfqpoint{7.495805in}{5.994023in}}%
\pgfpathlineto{\pgfqpoint{7.511239in}{5.883227in}}%
\pgfpathlineto{\pgfqpoint{7.542106in}{5.621000in}}%
\pgfpathlineto{\pgfqpoint{7.572974in}{5.322455in}}%
\pgfpathlineto{\pgfqpoint{7.634708in}{4.712667in}}%
\pgfpathlineto{\pgfqpoint{7.665575in}{4.450066in}}%
\pgfpathlineto{\pgfqpoint{7.681009in}{4.339024in}}%
\pgfpathlineto{\pgfqpoint{7.696443in}{4.244551in}}%
\pgfpathlineto{\pgfqpoint{7.711876in}{4.168541in}}%
\pgfpathlineto{\pgfqpoint{7.727310in}{4.112517in}}%
\pgfpathlineto{\pgfqpoint{7.742744in}{4.077602in}}%
\pgfpathlineto{\pgfqpoint{7.758177in}{4.064496in}}%
\pgfpathlineto{\pgfqpoint{7.773611in}{4.073460in}}%
\pgfpathlineto{\pgfqpoint{7.789044in}{4.104317in}}%
\pgfpathlineto{\pgfqpoint{7.804478in}{4.156447in}}%
\pgfpathlineto{\pgfqpoint{7.819912in}{4.228805in}}%
\pgfpathlineto{\pgfqpoint{7.835345in}{4.319941in}}%
\pgfpathlineto{\pgfqpoint{7.850779in}{4.428029in}}%
\pgfpathlineto{\pgfqpoint{7.881646in}{4.686098in}}%
\pgfpathlineto{\pgfqpoint{7.912514in}{4.982425in}}%
\pgfpathlineto{\pgfqpoint{7.974248in}{5.594136in}}%
\pgfpathlineto{\pgfqpoint{8.005115in}{5.860725in}}%
\pgfpathlineto{\pgfqpoint{8.020549in}{5.974403in}}%
\pgfpathlineto{\pgfqpoint{8.035983in}{6.071872in}}%
\pgfpathlineto{\pgfqpoint{8.051416in}{6.151180in}}%
\pgfpathlineto{\pgfqpoint{8.066850in}{6.210736in}}%
\pgfpathlineto{\pgfqpoint{8.082284in}{6.249347in}}%
\pgfpathlineto{\pgfqpoint{8.097717in}{6.266239in}}%
\pgfpathlineto{\pgfqpoint{8.113151in}{6.261073in}}%
\pgfpathlineto{\pgfqpoint{8.128584in}{6.233953in}}%
\pgfpathlineto{\pgfqpoint{8.144018in}{6.185423in}}%
\pgfpathlineto{\pgfqpoint{8.159452in}{6.116454in}}%
\pgfpathlineto{\pgfqpoint{8.174885in}{6.028430in}}%
\pgfpathlineto{\pgfqpoint{8.190319in}{5.923115in}}%
\pgfpathlineto{\pgfqpoint{8.205753in}{5.802620in}}%
\pgfpathlineto{\pgfqpoint{8.236620in}{5.526003in}}%
\pgfpathlineto{\pgfqpoint{8.329222in}{4.621493in}}%
\pgfpathlineto{\pgfqpoint{8.344655in}{4.491738in}}%
\pgfpathlineto{\pgfqpoint{8.360089in}{4.375491in}}%
\pgfpathlineto{\pgfqpoint{8.375523in}{4.275082in}}%
\pgfpathlineto{\pgfqpoint{8.390956in}{4.192524in}}%
\pgfpathlineto{\pgfqpoint{8.406390in}{4.129471in}}%
\pgfpathlineto{\pgfqpoint{8.421823in}{4.087188in}}%
\pgfpathlineto{\pgfqpoint{8.437257in}{4.066521in}}%
\pgfpathlineto{\pgfqpoint{8.452691in}{4.067884in}}%
\pgfpathlineto{\pgfqpoint{8.468124in}{4.091252in}}%
\pgfpathlineto{\pgfqpoint{8.483558in}{4.136154in}}%
\pgfpathlineto{\pgfqpoint{8.498992in}{4.201691in}}%
\pgfpathlineto{\pgfqpoint{8.514425in}{4.286550in}}%
\pgfpathlineto{\pgfqpoint{8.529859in}{4.389030in}}%
\pgfpathlineto{\pgfqpoint{8.545293in}{4.507077in}}%
\pgfpathlineto{\pgfqpoint{8.576160in}{4.780142in}}%
\pgfpathlineto{\pgfqpoint{8.622461in}{5.239878in}}%
\pgfpathlineto{\pgfqpoint{8.653328in}{5.544152in}}%
\pgfpathlineto{\pgfqpoint{8.684195in}{5.818238in}}%
\pgfpathlineto{\pgfqpoint{8.699629in}{5.936985in}}%
\pgfpathlineto{\pgfqpoint{8.715063in}{6.040273in}}%
\pgfpathlineto{\pgfqpoint{8.730496in}{6.126032in}}%
\pgfpathlineto{\pgfqpoint{8.745930in}{6.192544in}}%
\pgfpathlineto{\pgfqpoint{8.761363in}{6.238475in}}%
\pgfpathlineto{\pgfqpoint{8.776797in}{6.262905in}}%
\pgfpathlineto{\pgfqpoint{8.792231in}{6.265344in}}%
\pgfpathlineto{\pgfqpoint{8.807664in}{6.245744in}}%
\pgfpathlineto{\pgfqpoint{8.823098in}{6.204496in}}%
\pgfpathlineto{\pgfqpoint{8.838532in}{6.142429in}}%
\pgfpathlineto{\pgfqpoint{8.853965in}{6.060785in}}%
\pgfpathlineto{\pgfqpoint{8.869399in}{5.961202in}}%
\pgfpathlineto{\pgfqpoint{8.884833in}{5.845675in}}%
\pgfpathlineto{\pgfqpoint{8.915700in}{5.576324in}}%
\pgfpathlineto{\pgfqpoint{8.962001in}{5.118363in}}%
\pgfpathlineto{\pgfqpoint{9.008302in}{4.668830in}}%
\pgfpathlineto{\pgfqpoint{9.039169in}{4.413838in}}%
\pgfpathlineto{\pgfqpoint{9.054603in}{4.307732in}}%
\pgfpathlineto{\pgfqpoint{9.070036in}{4.218822in}}%
\pgfpathlineto{\pgfqpoint{9.085470in}{4.148891in}}%
\pgfpathlineto{\pgfqpoint{9.100903in}{4.099340in}}%
\pgfpathlineto{\pgfqpoint{9.116337in}{4.071161in}}%
\pgfpathlineto{\pgfqpoint{9.131771in}{4.064921in}}%
\pgfpathlineto{\pgfqpoint{9.147204in}{4.080743in}}%
\pgfpathlineto{\pgfqpoint{9.162638in}{4.118311in}}%
\pgfpathlineto{\pgfqpoint{9.178072in}{4.176871in}}%
\pgfpathlineto{\pgfqpoint{9.193505in}{4.255251in}}%
\pgfpathlineto{\pgfqpoint{9.208939in}{4.351879in}}%
\pgfpathlineto{\pgfqpoint{9.224373in}{4.464818in}}%
\pgfpathlineto{\pgfqpoint{9.255240in}{4.730295in}}%
\pgfpathlineto{\pgfqpoint{9.301541in}{5.186207in}}%
\pgfpathlineto{\pgfqpoint{9.347842in}{5.638471in}}%
\pgfpathlineto{\pgfqpoint{9.378709in}{5.897732in}}%
\pgfpathlineto{\pgfqpoint{9.394143in}{6.006592in}}%
\pgfpathlineto{\pgfqpoint{9.409576in}{6.098599in}}%
\pgfpathlineto{\pgfqpoint{9.425010in}{6.171908in}}%
\pgfpathlineto{\pgfqpoint{9.440443in}{6.225050in}}%
\pgfpathlineto{\pgfqpoint{9.455877in}{6.256961in}}%
\pgfpathlineto{\pgfqpoint{9.471311in}{6.266999in}}%
\pgfpathlineto{\pgfqpoint{9.486744in}{6.254965in}}%
\pgfpathlineto{\pgfqpoint{9.502178in}{6.221099in}}%
\pgfpathlineto{\pgfqpoint{9.517612in}{6.166080in}}%
\pgfpathlineto{\pgfqpoint{9.533045in}{6.091010in}}%
\pgfpathlineto{\pgfqpoint{9.548479in}{5.997396in}}%
\pgfpathlineto{\pgfqpoint{9.563913in}{5.887112in}}%
\pgfpathlineto{\pgfqpoint{9.594780in}{5.625667in}}%
\pgfpathlineto{\pgfqpoint{9.625647in}{5.327532in}}%
\pgfpathlineto{\pgfqpoint{9.687382in}{4.717349in}}%
\pgfpathlineto{\pgfqpoint{9.718249in}{4.453974in}}%
\pgfpathlineto{\pgfqpoint{9.733682in}{4.342423in}}%
\pgfpathlineto{\pgfqpoint{9.733682in}{4.342423in}}%
\pgfusepath{stroke}%
\end{pgfscope}%
\begin{pgfscope}%
\pgfpathrectangle{\pgfqpoint{5.698559in}{3.799602in}}{\pgfqpoint{4.227273in}{2.745455in}} %
\pgfusepath{clip}%
\pgfsetrectcap%
\pgfsetroundjoin%
\pgfsetlinewidth{0.501875pt}%
\definecolor{currentstroke}{rgb}{0.343137,0.243914,0.992421}%
\pgfsetstrokecolor{currentstroke}%
\pgfsetdash{}{0pt}%
\pgfpathmoveto{\pgfqpoint{5.890707in}{5.133561in}}%
\pgfpathlineto{\pgfqpoint{5.906141in}{5.223554in}}%
\pgfpathlineto{\pgfqpoint{5.921575in}{5.310770in}}%
\pgfpathlineto{\pgfqpoint{5.952442in}{5.470999in}}%
\pgfpathlineto{\pgfqpoint{5.967875in}{5.541318in}}%
\pgfpathlineto{\pgfqpoint{5.983309in}{5.603531in}}%
\pgfpathlineto{\pgfqpoint{5.998743in}{5.656650in}}%
\pgfpathlineto{\pgfqpoint{6.014176in}{5.699874in}}%
\pgfpathlineto{\pgfqpoint{6.029610in}{5.732598in}}%
\pgfpathlineto{\pgfqpoint{6.045044in}{5.754428in}}%
\pgfpathlineto{\pgfqpoint{6.060477in}{5.765187in}}%
\pgfpathlineto{\pgfqpoint{6.075911in}{5.764921in}}%
\pgfpathlineto{\pgfqpoint{6.091345in}{5.753895in}}%
\pgfpathlineto{\pgfqpoint{6.106778in}{5.732587in}}%
\pgfpathlineto{\pgfqpoint{6.122212in}{5.701682in}}%
\pgfpathlineto{\pgfqpoint{6.137645in}{5.662054in}}%
\pgfpathlineto{\pgfqpoint{6.153079in}{5.614749in}}%
\pgfpathlineto{\pgfqpoint{6.183946in}{5.502032in}}%
\pgfpathlineto{\pgfqpoint{6.230247in}{5.308905in}}%
\pgfpathlineto{\pgfqpoint{6.261115in}{5.181859in}}%
\pgfpathlineto{\pgfqpoint{6.291982in}{5.070188in}}%
\pgfpathlineto{\pgfqpoint{6.307415in}{5.023525in}}%
\pgfpathlineto{\pgfqpoint{6.322849in}{4.984554in}}%
\pgfpathlineto{\pgfqpoint{6.338283in}{4.954264in}}%
\pgfpathlineto{\pgfqpoint{6.353716in}{4.933465in}}%
\pgfpathlineto{\pgfqpoint{6.369150in}{4.922772in}}%
\pgfpathlineto{\pgfqpoint{6.384584in}{4.922590in}}%
\pgfpathlineto{\pgfqpoint{6.400017in}{4.933111in}}%
\pgfpathlineto{\pgfqpoint{6.415451in}{4.954302in}}%
\pgfpathlineto{\pgfqpoint{6.430885in}{4.985914in}}%
\pgfpathlineto{\pgfqpoint{6.446318in}{5.027480in}}%
\pgfpathlineto{\pgfqpoint{6.461752in}{5.078327in}}%
\pgfpathlineto{\pgfqpoint{6.477185in}{5.137592in}}%
\pgfpathlineto{\pgfqpoint{6.508053in}{5.277056in}}%
\pgfpathlineto{\pgfqpoint{6.538920in}{5.435836in}}%
\pgfpathlineto{\pgfqpoint{6.600655in}{5.763833in}}%
\pgfpathlineto{\pgfqpoint{6.631522in}{5.908466in}}%
\pgfpathlineto{\pgfqpoint{6.646955in}{5.970930in}}%
\pgfpathlineto{\pgfqpoint{6.662389in}{6.025197in}}%
\pgfpathlineto{\pgfqpoint{6.677823in}{6.070233in}}%
\pgfpathlineto{\pgfqpoint{6.693256in}{6.105184in}}%
\pgfpathlineto{\pgfqpoint{6.708690in}{6.129387in}}%
\pgfpathlineto{\pgfqpoint{6.724124in}{6.142386in}}%
\pgfpathlineto{\pgfqpoint{6.739557in}{6.143944in}}%
\pgfpathlineto{\pgfqpoint{6.754991in}{6.134043in}}%
\pgfpathlineto{\pgfqpoint{6.770424in}{6.112885in}}%
\pgfpathlineto{\pgfqpoint{6.785858in}{6.080892in}}%
\pgfpathlineto{\pgfqpoint{6.801292in}{6.038692in}}%
\pgfpathlineto{\pgfqpoint{6.816725in}{5.987112in}}%
\pgfpathlineto{\pgfqpoint{6.832159in}{5.927156in}}%
\pgfpathlineto{\pgfqpoint{6.863026in}{5.786911in}}%
\pgfpathlineto{\pgfqpoint{6.909327in}{5.546707in}}%
\pgfpathlineto{\pgfqpoint{6.955628in}{5.307389in}}%
\pgfpathlineto{\pgfqpoint{6.986495in}{5.168389in}}%
\pgfpathlineto{\pgfqpoint{7.001929in}{5.109051in}}%
\pgfpathlineto{\pgfqpoint{7.017363in}{5.057956in}}%
\pgfpathlineto{\pgfqpoint{7.032796in}{5.015997in}}%
\pgfpathlineto{\pgfqpoint{7.048230in}{4.983877in}}%
\pgfpathlineto{\pgfqpoint{7.063664in}{4.962092in}}%
\pgfpathlineto{\pgfqpoint{7.079097in}{4.950926in}}%
\pgfpathlineto{\pgfqpoint{7.094531in}{4.950443in}}%
\pgfpathlineto{\pgfqpoint{7.109964in}{4.960487in}}%
\pgfpathlineto{\pgfqpoint{7.125398in}{4.980683in}}%
\pgfpathlineto{\pgfqpoint{7.140832in}{5.010447in}}%
\pgfpathlineto{\pgfqpoint{7.156265in}{5.048998in}}%
\pgfpathlineto{\pgfqpoint{7.171699in}{5.095372in}}%
\pgfpathlineto{\pgfqpoint{7.202566in}{5.206945in}}%
\pgfpathlineto{\pgfqpoint{7.248867in}{5.400836in}}%
\pgfpathlineto{\pgfqpoint{7.279734in}{5.530234in}}%
\pgfpathlineto{\pgfqpoint{7.310602in}{5.645553in}}%
\pgfpathlineto{\pgfqpoint{7.326035in}{5.694404in}}%
\pgfpathlineto{\pgfqpoint{7.341469in}{5.735715in}}%
\pgfpathlineto{\pgfqpoint{7.356903in}{5.768414in}}%
\pgfpathlineto{\pgfqpoint{7.372336in}{5.791597in}}%
\pgfpathlineto{\pgfqpoint{7.387770in}{5.804550in}}%
\pgfpathlineto{\pgfqpoint{7.403204in}{5.806761in}}%
\pgfpathlineto{\pgfqpoint{7.418637in}{5.797931in}}%
\pgfpathlineto{\pgfqpoint{7.434071in}{5.777981in}}%
\pgfpathlineto{\pgfqpoint{7.449504in}{5.747052in}}%
\pgfpathlineto{\pgfqpoint{7.464938in}{5.705506in}}%
\pgfpathlineto{\pgfqpoint{7.480372in}{5.653914in}}%
\pgfpathlineto{\pgfqpoint{7.495805in}{5.593049in}}%
\pgfpathlineto{\pgfqpoint{7.511239in}{5.523870in}}%
\pgfpathlineto{\pgfqpoint{7.542106in}{5.365211in}}%
\pgfpathlineto{\pgfqpoint{7.588407in}{5.097117in}}%
\pgfpathlineto{\pgfqpoint{7.634708in}{4.829557in}}%
\pgfpathlineto{\pgfqpoint{7.665575in}{4.671852in}}%
\pgfpathlineto{\pgfqpoint{7.681009in}{4.603332in}}%
\pgfpathlineto{\pgfqpoint{7.696443in}{4.543251in}}%
\pgfpathlineto{\pgfqpoint{7.711876in}{4.492569in}}%
\pgfpathlineto{\pgfqpoint{7.727310in}{4.452060in}}%
\pgfpathlineto{\pgfqpoint{7.742744in}{4.422299in}}%
\pgfpathlineto{\pgfqpoint{7.758177in}{4.403647in}}%
\pgfpathlineto{\pgfqpoint{7.773611in}{4.396250in}}%
\pgfpathlineto{\pgfqpoint{7.789044in}{4.400030in}}%
\pgfpathlineto{\pgfqpoint{7.804478in}{4.414689in}}%
\pgfpathlineto{\pgfqpoint{7.819912in}{4.439716in}}%
\pgfpathlineto{\pgfqpoint{7.835345in}{4.474399in}}%
\pgfpathlineto{\pgfqpoint{7.850779in}{4.517835in}}%
\pgfpathlineto{\pgfqpoint{7.866213in}{4.568951in}}%
\pgfpathlineto{\pgfqpoint{7.897080in}{4.689220in}}%
\pgfpathlineto{\pgfqpoint{7.989682in}{5.089086in}}%
\pgfpathlineto{\pgfqpoint{8.005115in}{5.145605in}}%
\pgfpathlineto{\pgfqpoint{8.020549in}{5.195538in}}%
\pgfpathlineto{\pgfqpoint{8.035983in}{5.237754in}}%
\pgfpathlineto{\pgfqpoint{8.051416in}{5.271283in}}%
\pgfpathlineto{\pgfqpoint{8.066850in}{5.295336in}}%
\pgfpathlineto{\pgfqpoint{8.082284in}{5.309322in}}%
\pgfpathlineto{\pgfqpoint{8.097717in}{5.312860in}}%
\pgfpathlineto{\pgfqpoint{8.113151in}{5.305785in}}%
\pgfpathlineto{\pgfqpoint{8.128584in}{5.288154in}}%
\pgfpathlineto{\pgfqpoint{8.144018in}{5.260244in}}%
\pgfpathlineto{\pgfqpoint{8.159452in}{5.222543in}}%
\pgfpathlineto{\pgfqpoint{8.174885in}{5.175747in}}%
\pgfpathlineto{\pgfqpoint{8.190319in}{5.120741in}}%
\pgfpathlineto{\pgfqpoint{8.221186in}{4.990479in}}%
\pgfpathlineto{\pgfqpoint{8.252053in}{4.841894in}}%
\pgfpathlineto{\pgfqpoint{8.313788in}{4.537333in}}%
\pgfpathlineto{\pgfqpoint{8.344655in}{4.405809in}}%
\pgfpathlineto{\pgfqpoint{8.360089in}{4.350184in}}%
\pgfpathlineto{\pgfqpoint{8.375523in}{4.302912in}}%
\pgfpathlineto{\pgfqpoint{8.390956in}{4.264997in}}%
\pgfpathlineto{\pgfqpoint{8.406390in}{4.237263in}}%
\pgfpathlineto{\pgfqpoint{8.421823in}{4.220336in}}%
\pgfpathlineto{\pgfqpoint{8.437257in}{4.214638in}}%
\pgfpathlineto{\pgfqpoint{8.452691in}{4.220370in}}%
\pgfpathlineto{\pgfqpoint{8.468124in}{4.237513in}}%
\pgfpathlineto{\pgfqpoint{8.483558in}{4.265827in}}%
\pgfpathlineto{\pgfqpoint{8.498992in}{4.304857in}}%
\pgfpathlineto{\pgfqpoint{8.514425in}{4.353939in}}%
\pgfpathlineto{\pgfqpoint{8.529859in}{4.412215in}}%
\pgfpathlineto{\pgfqpoint{8.545293in}{4.478653in}}%
\pgfpathlineto{\pgfqpoint{8.576160in}{4.631114in}}%
\pgfpathlineto{\pgfqpoint{8.622461in}{4.887500in}}%
\pgfpathlineto{\pgfqpoint{8.668762in}{5.140215in}}%
\pgfpathlineto{\pgfqpoint{8.699629in}{5.286625in}}%
\pgfpathlineto{\pgfqpoint{8.715063in}{5.349247in}}%
\pgfpathlineto{\pgfqpoint{8.730496in}{5.403367in}}%
\pgfpathlineto{\pgfqpoint{8.745930in}{5.448114in}}%
\pgfpathlineto{\pgfqpoint{8.761363in}{5.482809in}}%
\pgfpathlineto{\pgfqpoint{8.776797in}{5.506980in}}%
\pgfpathlineto{\pgfqpoint{8.792231in}{5.520369in}}%
\pgfpathlineto{\pgfqpoint{8.807664in}{5.522940in}}%
\pgfpathlineto{\pgfqpoint{8.823098in}{5.514877in}}%
\pgfpathlineto{\pgfqpoint{8.838532in}{5.496580in}}%
\pgfpathlineto{\pgfqpoint{8.853965in}{5.468657in}}%
\pgfpathlineto{\pgfqpoint{8.869399in}{5.431913in}}%
\pgfpathlineto{\pgfqpoint{8.884833in}{5.387334in}}%
\pgfpathlineto{\pgfqpoint{8.900266in}{5.336062in}}%
\pgfpathlineto{\pgfqpoint{8.931133in}{5.218677in}}%
\pgfpathlineto{\pgfqpoint{9.008302in}{4.907619in}}%
\pgfpathlineto{\pgfqpoint{9.023735in}{4.854493in}}%
\pgfpathlineto{\pgfqpoint{9.039169in}{4.807683in}}%
\pgfpathlineto{\pgfqpoint{9.054603in}{4.768389in}}%
\pgfpathlineto{\pgfqpoint{9.070036in}{4.737661in}}%
\pgfpathlineto{\pgfqpoint{9.085470in}{4.716375in}}%
\pgfpathlineto{\pgfqpoint{9.100903in}{4.705218in}}%
\pgfpathlineto{\pgfqpoint{9.116337in}{4.704672in}}%
\pgfpathlineto{\pgfqpoint{9.131771in}{4.715007in}}%
\pgfpathlineto{\pgfqpoint{9.147204in}{4.736269in}}%
\pgfpathlineto{\pgfqpoint{9.162638in}{4.768288in}}%
\pgfpathlineto{\pgfqpoint{9.178072in}{4.810673in}}%
\pgfpathlineto{\pgfqpoint{9.193505in}{4.862823in}}%
\pgfpathlineto{\pgfqpoint{9.208939in}{4.923940in}}%
\pgfpathlineto{\pgfqpoint{9.239806in}{5.068984in}}%
\pgfpathlineto{\pgfqpoint{9.270673in}{5.236132in}}%
\pgfpathlineto{\pgfqpoint{9.347842in}{5.673263in}}%
\pgfpathlineto{\pgfqpoint{9.378709in}{5.823673in}}%
\pgfpathlineto{\pgfqpoint{9.394143in}{5.888046in}}%
\pgfpathlineto{\pgfqpoint{9.409576in}{5.943681in}}%
\pgfpathlineto{\pgfqpoint{9.425010in}{5.989645in}}%
\pgfpathlineto{\pgfqpoint{9.440443in}{6.025194in}}%
\pgfpathlineto{\pgfqpoint{9.455877in}{6.049786in}}%
\pgfpathlineto{\pgfqpoint{9.471311in}{6.063092in}}%
\pgfpathlineto{\pgfqpoint{9.486744in}{6.065004in}}%
\pgfpathlineto{\pgfqpoint{9.502178in}{6.055634in}}%
\pgfpathlineto{\pgfqpoint{9.517612in}{6.035314in}}%
\pgfpathlineto{\pgfqpoint{9.533045in}{6.004586in}}%
\pgfpathlineto{\pgfqpoint{9.548479in}{5.964197in}}%
\pgfpathlineto{\pgfqpoint{9.563913in}{5.915079in}}%
\pgfpathlineto{\pgfqpoint{9.579346in}{5.858329in}}%
\pgfpathlineto{\pgfqpoint{9.610213in}{5.727034in}}%
\pgfpathlineto{\pgfqpoint{9.702815in}{5.296953in}}%
\pgfpathlineto{\pgfqpoint{9.718249in}{5.235844in}}%
\pgfpathlineto{\pgfqpoint{9.733682in}{5.181428in}}%
\pgfpathlineto{\pgfqpoint{9.733682in}{5.181428in}}%
\pgfusepath{stroke}%
\end{pgfscope}%
\begin{pgfscope}%
\pgfpathrectangle{\pgfqpoint{5.698559in}{3.799602in}}{\pgfqpoint{4.227273in}{2.745455in}} %
\pgfusepath{clip}%
\pgfsetrectcap%
\pgfsetroundjoin%
\pgfsetlinewidth{0.501875pt}%
\definecolor{currentstroke}{rgb}{0.264706,0.361242,0.982973}%
\pgfsetstrokecolor{currentstroke}%
\pgfsetdash{}{0pt}%
\pgfpathmoveto{\pgfqpoint{5.890707in}{5.266738in}}%
\pgfpathlineto{\pgfqpoint{5.906141in}{5.289487in}}%
\pgfpathlineto{\pgfqpoint{5.921575in}{5.309859in}}%
\pgfpathlineto{\pgfqpoint{5.937008in}{5.327093in}}%
\pgfpathlineto{\pgfqpoint{5.952442in}{5.340489in}}%
\pgfpathlineto{\pgfqpoint{5.967875in}{5.349423in}}%
\pgfpathlineto{\pgfqpoint{5.983309in}{5.353364in}}%
\pgfpathlineto{\pgfqpoint{5.998743in}{5.351887in}}%
\pgfpathlineto{\pgfqpoint{6.014176in}{5.344684in}}%
\pgfpathlineto{\pgfqpoint{6.029610in}{5.331575in}}%
\pgfpathlineto{\pgfqpoint{6.045044in}{5.312514in}}%
\pgfpathlineto{\pgfqpoint{6.060477in}{5.287590in}}%
\pgfpathlineto{\pgfqpoint{6.075911in}{5.257032in}}%
\pgfpathlineto{\pgfqpoint{6.091345in}{5.221203in}}%
\pgfpathlineto{\pgfqpoint{6.122212in}{5.135840in}}%
\pgfpathlineto{\pgfqpoint{6.153079in}{5.036882in}}%
\pgfpathlineto{\pgfqpoint{6.230247in}{4.778259in}}%
\pgfpathlineto{\pgfqpoint{6.261115in}{4.692201in}}%
\pgfpathlineto{\pgfqpoint{6.276548in}{4.657034in}}%
\pgfpathlineto{\pgfqpoint{6.291982in}{4.628340in}}%
\pgfpathlineto{\pgfqpoint{6.307415in}{4.606935in}}%
\pgfpathlineto{\pgfqpoint{6.322849in}{4.593523in}}%
\pgfpathlineto{\pgfqpoint{6.338283in}{4.588687in}}%
\pgfpathlineto{\pgfqpoint{6.353716in}{4.592872in}}%
\pgfpathlineto{\pgfqpoint{6.369150in}{4.606376in}}%
\pgfpathlineto{\pgfqpoint{6.384584in}{4.629342in}}%
\pgfpathlineto{\pgfqpoint{6.400017in}{4.661755in}}%
\pgfpathlineto{\pgfqpoint{6.415451in}{4.703437in}}%
\pgfpathlineto{\pgfqpoint{6.430885in}{4.754051in}}%
\pgfpathlineto{\pgfqpoint{6.446318in}{4.813103in}}%
\pgfpathlineto{\pgfqpoint{6.461752in}{4.879952in}}%
\pgfpathlineto{\pgfqpoint{6.492619in}{5.033793in}}%
\pgfpathlineto{\pgfqpoint{6.523486in}{5.207919in}}%
\pgfpathlineto{\pgfqpoint{6.616088in}{5.757050in}}%
\pgfpathlineto{\pgfqpoint{6.646955in}{5.915868in}}%
\pgfpathlineto{\pgfqpoint{6.662389in}{5.985607in}}%
\pgfpathlineto{\pgfqpoint{6.677823in}{6.047668in}}%
\pgfpathlineto{\pgfqpoint{6.693256in}{6.101281in}}%
\pgfpathlineto{\pgfqpoint{6.708690in}{6.145819in}}%
\pgfpathlineto{\pgfqpoint{6.724124in}{6.180803in}}%
\pgfpathlineto{\pgfqpoint{6.739557in}{6.205911in}}%
\pgfpathlineto{\pgfqpoint{6.754991in}{6.220982in}}%
\pgfpathlineto{\pgfqpoint{6.770424in}{6.226018in}}%
\pgfpathlineto{\pgfqpoint{6.785858in}{6.221181in}}%
\pgfpathlineto{\pgfqpoint{6.801292in}{6.206790in}}%
\pgfpathlineto{\pgfqpoint{6.816725in}{6.183308in}}%
\pgfpathlineto{\pgfqpoint{6.832159in}{6.151340in}}%
\pgfpathlineto{\pgfqpoint{6.847593in}{6.111612in}}%
\pgfpathlineto{\pgfqpoint{6.863026in}{6.064960in}}%
\pgfpathlineto{\pgfqpoint{6.893894in}{5.954668in}}%
\pgfpathlineto{\pgfqpoint{6.924761in}{5.828647in}}%
\pgfpathlineto{\pgfqpoint{7.001929in}{5.501314in}}%
\pgfpathlineto{\pgfqpoint{7.032796in}{5.386163in}}%
\pgfpathlineto{\pgfqpoint{7.063664in}{5.289210in}}%
\pgfpathlineto{\pgfqpoint{7.079097in}{5.248760in}}%
\pgfpathlineto{\pgfqpoint{7.094531in}{5.214034in}}%
\pgfpathlineto{\pgfqpoint{7.109964in}{5.185140in}}%
\pgfpathlineto{\pgfqpoint{7.125398in}{5.162050in}}%
\pgfpathlineto{\pgfqpoint{7.140832in}{5.144599in}}%
\pgfpathlineto{\pgfqpoint{7.156265in}{5.132497in}}%
\pgfpathlineto{\pgfqpoint{7.171699in}{5.125335in}}%
\pgfpathlineto{\pgfqpoint{7.187133in}{5.122598in}}%
\pgfpathlineto{\pgfqpoint{7.202566in}{5.123676in}}%
\pgfpathlineto{\pgfqpoint{7.218000in}{5.127884in}}%
\pgfpathlineto{\pgfqpoint{7.248867in}{5.142666in}}%
\pgfpathlineto{\pgfqpoint{7.295168in}{5.168713in}}%
\pgfpathlineto{\pgfqpoint{7.310602in}{5.175249in}}%
\pgfpathlineto{\pgfqpoint{7.326035in}{5.179489in}}%
\pgfpathlineto{\pgfqpoint{7.341469in}{5.180794in}}%
\pgfpathlineto{\pgfqpoint{7.356903in}{5.178611in}}%
\pgfpathlineto{\pgfqpoint{7.372336in}{5.172487in}}%
\pgfpathlineto{\pgfqpoint{7.387770in}{5.162079in}}%
\pgfpathlineto{\pgfqpoint{7.403204in}{5.147162in}}%
\pgfpathlineto{\pgfqpoint{7.418637in}{5.127639in}}%
\pgfpathlineto{\pgfqpoint{7.434071in}{5.103540in}}%
\pgfpathlineto{\pgfqpoint{7.449504in}{5.075029in}}%
\pgfpathlineto{\pgfqpoint{7.464938in}{5.042393in}}%
\pgfpathlineto{\pgfqpoint{7.495805in}{4.966510in}}%
\pgfpathlineto{\pgfqpoint{7.526673in}{4.880509in}}%
\pgfpathlineto{\pgfqpoint{7.588407in}{4.703432in}}%
\pgfpathlineto{\pgfqpoint{7.619274in}{4.626831in}}%
\pgfpathlineto{\pgfqpoint{7.634708in}{4.594713in}}%
\pgfpathlineto{\pgfqpoint{7.650142in}{4.567839in}}%
\pgfpathlineto{\pgfqpoint{7.665575in}{4.546966in}}%
\pgfpathlineto{\pgfqpoint{7.681009in}{4.532755in}}%
\pgfpathlineto{\pgfqpoint{7.696443in}{4.525764in}}%
\pgfpathlineto{\pgfqpoint{7.711876in}{4.526428in}}%
\pgfpathlineto{\pgfqpoint{7.727310in}{4.535052in}}%
\pgfpathlineto{\pgfqpoint{7.742744in}{4.551797in}}%
\pgfpathlineto{\pgfqpoint{7.758177in}{4.576680in}}%
\pgfpathlineto{\pgfqpoint{7.773611in}{4.609565in}}%
\pgfpathlineto{\pgfqpoint{7.789044in}{4.650169in}}%
\pgfpathlineto{\pgfqpoint{7.804478in}{4.698057in}}%
\pgfpathlineto{\pgfqpoint{7.819912in}{4.752653in}}%
\pgfpathlineto{\pgfqpoint{7.850779in}{4.879008in}}%
\pgfpathlineto{\pgfqpoint{7.881646in}{5.022165in}}%
\pgfpathlineto{\pgfqpoint{7.958814in}{5.395425in}}%
\pgfpathlineto{\pgfqpoint{7.989682in}{5.525869in}}%
\pgfpathlineto{\pgfqpoint{8.005115in}{5.582494in}}%
\pgfpathlineto{\pgfqpoint{8.020549in}{5.632073in}}%
\pgfpathlineto{\pgfqpoint{8.035983in}{5.673748in}}%
\pgfpathlineto{\pgfqpoint{8.051416in}{5.706791in}}%
\pgfpathlineto{\pgfqpoint{8.066850in}{5.730608in}}%
\pgfpathlineto{\pgfqpoint{8.082284in}{5.744759in}}%
\pgfpathlineto{\pgfqpoint{8.097717in}{5.748960in}}%
\pgfpathlineto{\pgfqpoint{8.113151in}{5.743088in}}%
\pgfpathlineto{\pgfqpoint{8.128584in}{5.727190in}}%
\pgfpathlineto{\pgfqpoint{8.144018in}{5.701474in}}%
\pgfpathlineto{\pgfqpoint{8.159452in}{5.666312in}}%
\pgfpathlineto{\pgfqpoint{8.174885in}{5.622228in}}%
\pgfpathlineto{\pgfqpoint{8.190319in}{5.569892in}}%
\pgfpathlineto{\pgfqpoint{8.205753in}{5.510105in}}%
\pgfpathlineto{\pgfqpoint{8.236620in}{5.371961in}}%
\pgfpathlineto{\pgfqpoint{8.267487in}{5.216251in}}%
\pgfpathlineto{\pgfqpoint{8.344655in}{4.812645in}}%
\pgfpathlineto{\pgfqpoint{8.375523in}{4.670090in}}%
\pgfpathlineto{\pgfqpoint{8.390956in}{4.606982in}}%
\pgfpathlineto{\pgfqpoint{8.406390in}{4.550403in}}%
\pgfpathlineto{\pgfqpoint{8.421823in}{4.501007in}}%
\pgfpathlineto{\pgfqpoint{8.437257in}{4.459309in}}%
\pgfpathlineto{\pgfqpoint{8.452691in}{4.425678in}}%
\pgfpathlineto{\pgfqpoint{8.468124in}{4.400333in}}%
\pgfpathlineto{\pgfqpoint{8.483558in}{4.383341in}}%
\pgfpathlineto{\pgfqpoint{8.498992in}{4.374619in}}%
\pgfpathlineto{\pgfqpoint{8.514425in}{4.373940in}}%
\pgfpathlineto{\pgfqpoint{8.529859in}{4.380939in}}%
\pgfpathlineto{\pgfqpoint{8.545293in}{4.395123in}}%
\pgfpathlineto{\pgfqpoint{8.560726in}{4.415885in}}%
\pgfpathlineto{\pgfqpoint{8.576160in}{4.442515in}}%
\pgfpathlineto{\pgfqpoint{8.591593in}{4.474222in}}%
\pgfpathlineto{\pgfqpoint{8.622461in}{4.549390in}}%
\pgfpathlineto{\pgfqpoint{8.668762in}{4.677670in}}%
\pgfpathlineto{\pgfqpoint{8.715063in}{4.802907in}}%
\pgfpathlineto{\pgfqpoint{8.745930in}{4.874342in}}%
\pgfpathlineto{\pgfqpoint{8.761363in}{4.904691in}}%
\pgfpathlineto{\pgfqpoint{8.776797in}{4.930942in}}%
\pgfpathlineto{\pgfqpoint{8.792231in}{4.952860in}}%
\pgfpathlineto{\pgfqpoint{8.807664in}{4.970342in}}%
\pgfpathlineto{\pgfqpoint{8.823098in}{4.983412in}}%
\pgfpathlineto{\pgfqpoint{8.838532in}{4.992223in}}%
\pgfpathlineto{\pgfqpoint{8.853965in}{4.997051in}}%
\pgfpathlineto{\pgfqpoint{8.869399in}{4.998289in}}%
\pgfpathlineto{\pgfqpoint{8.884833in}{4.996432in}}%
\pgfpathlineto{\pgfqpoint{8.900266in}{4.992072in}}%
\pgfpathlineto{\pgfqpoint{8.931133in}{4.978585in}}%
\pgfpathlineto{\pgfqpoint{8.962001in}{4.963828in}}%
\pgfpathlineto{\pgfqpoint{8.977434in}{4.957982in}}%
\pgfpathlineto{\pgfqpoint{8.992868in}{4.954225in}}%
\pgfpathlineto{\pgfqpoint{9.008302in}{4.953327in}}%
\pgfpathlineto{\pgfqpoint{9.023735in}{4.956004in}}%
\pgfpathlineto{\pgfqpoint{9.039169in}{4.962905in}}%
\pgfpathlineto{\pgfqpoint{9.054603in}{4.974596in}}%
\pgfpathlineto{\pgfqpoint{9.070036in}{4.991541in}}%
\pgfpathlineto{\pgfqpoint{9.085470in}{5.014095in}}%
\pgfpathlineto{\pgfqpoint{9.100903in}{5.042487in}}%
\pgfpathlineto{\pgfqpoint{9.116337in}{5.076818in}}%
\pgfpathlineto{\pgfqpoint{9.131771in}{5.117053in}}%
\pgfpathlineto{\pgfqpoint{9.147204in}{5.163018in}}%
\pgfpathlineto{\pgfqpoint{9.178072in}{5.270760in}}%
\pgfpathlineto{\pgfqpoint{9.208939in}{5.395980in}}%
\pgfpathlineto{\pgfqpoint{9.255240in}{5.603146in}}%
\pgfpathlineto{\pgfqpoint{9.301541in}{5.810109in}}%
\pgfpathlineto{\pgfqpoint{9.332408in}{5.933635in}}%
\pgfpathlineto{\pgfqpoint{9.347842in}{5.987867in}}%
\pgfpathlineto{\pgfqpoint{9.363275in}{6.035765in}}%
\pgfpathlineto{\pgfqpoint{9.378709in}{6.076443in}}%
\pgfpathlineto{\pgfqpoint{9.394143in}{6.109113in}}%
\pgfpathlineto{\pgfqpoint{9.409576in}{6.133105in}}%
\pgfpathlineto{\pgfqpoint{9.425010in}{6.147877in}}%
\pgfpathlineto{\pgfqpoint{9.440443in}{6.153033in}}%
\pgfpathlineto{\pgfqpoint{9.455877in}{6.148324in}}%
\pgfpathlineto{\pgfqpoint{9.471311in}{6.133664in}}%
\pgfpathlineto{\pgfqpoint{9.486744in}{6.109123in}}%
\pgfpathlineto{\pgfqpoint{9.502178in}{6.074936in}}%
\pgfpathlineto{\pgfqpoint{9.517612in}{6.031498in}}%
\pgfpathlineto{\pgfqpoint{9.533045in}{5.979355in}}%
\pgfpathlineto{\pgfqpoint{9.548479in}{5.919199in}}%
\pgfpathlineto{\pgfqpoint{9.579346in}{5.778277in}}%
\pgfpathlineto{\pgfqpoint{9.610213in}{5.616682in}}%
\pgfpathlineto{\pgfqpoint{9.702815in}{5.103903in}}%
\pgfpathlineto{\pgfqpoint{9.733682in}{4.956446in}}%
\pgfpathlineto{\pgfqpoint{9.733682in}{4.956446in}}%
\pgfusepath{stroke}%
\end{pgfscope}%
\begin{pgfscope}%
\pgfpathrectangle{\pgfqpoint{5.698559in}{3.799602in}}{\pgfqpoint{4.227273in}{2.745455in}} %
\pgfusepath{clip}%
\pgfsetrectcap%
\pgfsetroundjoin%
\pgfsetlinewidth{0.501875pt}%
\definecolor{currentstroke}{rgb}{0.186275,0.473094,0.969797}%
\pgfsetstrokecolor{currentstroke}%
\pgfsetdash{}{0pt}%
\pgfpathmoveto{\pgfqpoint{5.890707in}{5.474521in}}%
\pgfpathlineto{\pgfqpoint{5.906141in}{5.510407in}}%
\pgfpathlineto{\pgfqpoint{5.921575in}{5.536373in}}%
\pgfpathlineto{\pgfqpoint{5.937008in}{5.551503in}}%
\pgfpathlineto{\pgfqpoint{5.952442in}{5.555200in}}%
\pgfpathlineto{\pgfqpoint{5.967875in}{5.547201in}}%
\pgfpathlineto{\pgfqpoint{5.983309in}{5.527584in}}%
\pgfpathlineto{\pgfqpoint{5.998743in}{5.496758in}}%
\pgfpathlineto{\pgfqpoint{6.014176in}{5.455454in}}%
\pgfpathlineto{\pgfqpoint{6.029610in}{5.404696in}}%
\pgfpathlineto{\pgfqpoint{6.045044in}{5.345771in}}%
\pgfpathlineto{\pgfqpoint{6.075911in}{5.209635in}}%
\pgfpathlineto{\pgfqpoint{6.153079in}{4.847130in}}%
\pgfpathlineto{\pgfqpoint{6.168513in}{4.785459in}}%
\pgfpathlineto{\pgfqpoint{6.183946in}{4.731165in}}%
\pgfpathlineto{\pgfqpoint{6.199380in}{4.685493in}}%
\pgfpathlineto{\pgfqpoint{6.214814in}{4.649439in}}%
\pgfpathlineto{\pgfqpoint{6.230247in}{4.623730in}}%
\pgfpathlineto{\pgfqpoint{6.245681in}{4.608810in}}%
\pgfpathlineto{\pgfqpoint{6.261115in}{4.604838in}}%
\pgfpathlineto{\pgfqpoint{6.276548in}{4.611688in}}%
\pgfpathlineto{\pgfqpoint{6.291982in}{4.628968in}}%
\pgfpathlineto{\pgfqpoint{6.307415in}{4.656035in}}%
\pgfpathlineto{\pgfqpoint{6.322849in}{4.692023in}}%
\pgfpathlineto{\pgfqpoint{6.338283in}{4.735883in}}%
\pgfpathlineto{\pgfqpoint{6.353716in}{4.786413in}}%
\pgfpathlineto{\pgfqpoint{6.384584in}{4.902192in}}%
\pgfpathlineto{\pgfqpoint{6.477185in}{5.271319in}}%
\pgfpathlineto{\pgfqpoint{6.508053in}{5.372530in}}%
\pgfpathlineto{\pgfqpoint{6.523486in}{5.416059in}}%
\pgfpathlineto{\pgfqpoint{6.538920in}{5.454622in}}%
\pgfpathlineto{\pgfqpoint{6.554354in}{5.488270in}}%
\pgfpathlineto{\pgfqpoint{6.569787in}{5.517228in}}%
\pgfpathlineto{\pgfqpoint{6.585221in}{5.541870in}}%
\pgfpathlineto{\pgfqpoint{6.600655in}{5.562694in}}%
\pgfpathlineto{\pgfqpoint{6.616088in}{5.580294in}}%
\pgfpathlineto{\pgfqpoint{6.646955in}{5.608468in}}%
\pgfpathlineto{\pgfqpoint{6.739557in}{5.680913in}}%
\pgfpathlineto{\pgfqpoint{6.770424in}{5.710967in}}%
\pgfpathlineto{\pgfqpoint{6.832159in}{5.775015in}}%
\pgfpathlineto{\pgfqpoint{6.847593in}{5.788667in}}%
\pgfpathlineto{\pgfqpoint{6.863026in}{5.799921in}}%
\pgfpathlineto{\pgfqpoint{6.878460in}{5.807977in}}%
\pgfpathlineto{\pgfqpoint{6.893894in}{5.812051in}}%
\pgfpathlineto{\pgfqpoint{6.909327in}{5.811404in}}%
\pgfpathlineto{\pgfqpoint{6.924761in}{5.805381in}}%
\pgfpathlineto{\pgfqpoint{6.940194in}{5.793437in}}%
\pgfpathlineto{\pgfqpoint{6.955628in}{5.775172in}}%
\pgfpathlineto{\pgfqpoint{6.971062in}{5.750349in}}%
\pgfpathlineto{\pgfqpoint{6.986495in}{5.718914in}}%
\pgfpathlineto{\pgfqpoint{7.001929in}{5.681010in}}%
\pgfpathlineto{\pgfqpoint{7.017363in}{5.636982in}}%
\pgfpathlineto{\pgfqpoint{7.048230in}{5.532918in}}%
\pgfpathlineto{\pgfqpoint{7.079097in}{5.413257in}}%
\pgfpathlineto{\pgfqpoint{7.140832in}{5.164215in}}%
\pgfpathlineto{\pgfqpoint{7.171699in}{5.055633in}}%
\pgfpathlineto{\pgfqpoint{7.187133in}{5.009628in}}%
\pgfpathlineto{\pgfqpoint{7.202566in}{4.970484in}}%
\pgfpathlineto{\pgfqpoint{7.218000in}{4.938976in}}%
\pgfpathlineto{\pgfqpoint{7.233434in}{4.915666in}}%
\pgfpathlineto{\pgfqpoint{7.248867in}{4.900875in}}%
\pgfpathlineto{\pgfqpoint{7.264301in}{4.894672in}}%
\pgfpathlineto{\pgfqpoint{7.279734in}{4.896865in}}%
\pgfpathlineto{\pgfqpoint{7.295168in}{4.907001in}}%
\pgfpathlineto{\pgfqpoint{7.310602in}{4.924378in}}%
\pgfpathlineto{\pgfqpoint{7.326035in}{4.948062in}}%
\pgfpathlineto{\pgfqpoint{7.341469in}{4.976911in}}%
\pgfpathlineto{\pgfqpoint{7.372336in}{5.044707in}}%
\pgfpathlineto{\pgfqpoint{7.418637in}{5.148846in}}%
\pgfpathlineto{\pgfqpoint{7.434071in}{5.177979in}}%
\pgfpathlineto{\pgfqpoint{7.449504in}{5.201900in}}%
\pgfpathlineto{\pgfqpoint{7.464938in}{5.219360in}}%
\pgfpathlineto{\pgfqpoint{7.480372in}{5.229315in}}%
\pgfpathlineto{\pgfqpoint{7.495805in}{5.230956in}}%
\pgfpathlineto{\pgfqpoint{7.511239in}{5.223748in}}%
\pgfpathlineto{\pgfqpoint{7.526673in}{5.207447in}}%
\pgfpathlineto{\pgfqpoint{7.542106in}{5.182117in}}%
\pgfpathlineto{\pgfqpoint{7.557540in}{5.148136in}}%
\pgfpathlineto{\pgfqpoint{7.572974in}{5.106188in}}%
\pgfpathlineto{\pgfqpoint{7.588407in}{5.057254in}}%
\pgfpathlineto{\pgfqpoint{7.619274in}{4.943655in}}%
\pgfpathlineto{\pgfqpoint{7.681009in}{4.700959in}}%
\pgfpathlineto{\pgfqpoint{7.696443in}{4.648147in}}%
\pgfpathlineto{\pgfqpoint{7.711876in}{4.602290in}}%
\pgfpathlineto{\pgfqpoint{7.727310in}{4.565116in}}%
\pgfpathlineto{\pgfqpoint{7.742744in}{4.538155in}}%
\pgfpathlineto{\pgfqpoint{7.758177in}{4.522698in}}%
\pgfpathlineto{\pgfqpoint{7.773611in}{4.519744in}}%
\pgfpathlineto{\pgfqpoint{7.789044in}{4.529973in}}%
\pgfpathlineto{\pgfqpoint{7.804478in}{4.553713in}}%
\pgfpathlineto{\pgfqpoint{7.819912in}{4.590926in}}%
\pgfpathlineto{\pgfqpoint{7.835345in}{4.641199in}}%
\pgfpathlineto{\pgfqpoint{7.850779in}{4.703752in}}%
\pgfpathlineto{\pgfqpoint{7.866213in}{4.777452in}}%
\pgfpathlineto{\pgfqpoint{7.897080in}{4.952147in}}%
\pgfpathlineto{\pgfqpoint{7.943381in}{5.252716in}}%
\pgfpathlineto{\pgfqpoint{7.974248in}{5.451953in}}%
\pgfpathlineto{\pgfqpoint{8.005115in}{5.628067in}}%
\pgfpathlineto{\pgfqpoint{8.020549in}{5.701971in}}%
\pgfpathlineto{\pgfqpoint{8.035983in}{5.763922in}}%
\pgfpathlineto{\pgfqpoint{8.051416in}{5.812381in}}%
\pgfpathlineto{\pgfqpoint{8.066850in}{5.846140in}}%
\pgfpathlineto{\pgfqpoint{8.082284in}{5.864353in}}%
\pgfpathlineto{\pgfqpoint{8.097717in}{5.866561in}}%
\pgfpathlineto{\pgfqpoint{8.113151in}{5.852705in}}%
\pgfpathlineto{\pgfqpoint{8.128584in}{5.823124in}}%
\pgfpathlineto{\pgfqpoint{8.144018in}{5.778544in}}%
\pgfpathlineto{\pgfqpoint{8.159452in}{5.720062in}}%
\pgfpathlineto{\pgfqpoint{8.174885in}{5.649109in}}%
\pgfpathlineto{\pgfqpoint{8.190319in}{5.567410in}}%
\pgfpathlineto{\pgfqpoint{8.221186in}{5.379860in}}%
\pgfpathlineto{\pgfqpoint{8.298354in}{4.876677in}}%
\pgfpathlineto{\pgfqpoint{8.329222in}{4.708530in}}%
\pgfpathlineto{\pgfqpoint{8.344655in}{4.638957in}}%
\pgfpathlineto{\pgfqpoint{8.360089in}{4.580731in}}%
\pgfpathlineto{\pgfqpoint{8.375523in}{4.534699in}}%
\pgfpathlineto{\pgfqpoint{8.390956in}{4.501355in}}%
\pgfpathlineto{\pgfqpoint{8.406390in}{4.480833in}}%
\pgfpathlineto{\pgfqpoint{8.421823in}{4.472916in}}%
\pgfpathlineto{\pgfqpoint{8.437257in}{4.477050in}}%
\pgfpathlineto{\pgfqpoint{8.452691in}{4.492366in}}%
\pgfpathlineto{\pgfqpoint{8.468124in}{4.517720in}}%
\pgfpathlineto{\pgfqpoint{8.483558in}{4.551728in}}%
\pgfpathlineto{\pgfqpoint{8.498992in}{4.592818in}}%
\pgfpathlineto{\pgfqpoint{8.529859in}{4.689323in}}%
\pgfpathlineto{\pgfqpoint{8.576160in}{4.842975in}}%
\pgfpathlineto{\pgfqpoint{8.607027in}{4.931844in}}%
\pgfpathlineto{\pgfqpoint{8.622461in}{4.968096in}}%
\pgfpathlineto{\pgfqpoint{8.637894in}{4.997586in}}%
\pgfpathlineto{\pgfqpoint{8.653328in}{5.019665in}}%
\pgfpathlineto{\pgfqpoint{8.668762in}{5.033970in}}%
\pgfpathlineto{\pgfqpoint{8.684195in}{5.040426in}}%
\pgfpathlineto{\pgfqpoint{8.699629in}{5.039242in}}%
\pgfpathlineto{\pgfqpoint{8.715063in}{5.030893in}}%
\pgfpathlineto{\pgfqpoint{8.730496in}{5.016106in}}%
\pgfpathlineto{\pgfqpoint{8.745930in}{4.995821in}}%
\pgfpathlineto{\pgfqpoint{8.761363in}{4.971161in}}%
\pgfpathlineto{\pgfqpoint{8.792231in}{4.913859in}}%
\pgfpathlineto{\pgfqpoint{8.823098in}{4.855166in}}%
\pgfpathlineto{\pgfqpoint{8.838532in}{4.828771in}}%
\pgfpathlineto{\pgfqpoint{8.853965in}{4.806071in}}%
\pgfpathlineto{\pgfqpoint{8.869399in}{4.788211in}}%
\pgfpathlineto{\pgfqpoint{8.884833in}{4.776170in}}%
\pgfpathlineto{\pgfqpoint{8.900266in}{4.770736in}}%
\pgfpathlineto{\pgfqpoint{8.915700in}{4.772480in}}%
\pgfpathlineto{\pgfqpoint{8.931133in}{4.781746in}}%
\pgfpathlineto{\pgfqpoint{8.946567in}{4.798641in}}%
\pgfpathlineto{\pgfqpoint{8.962001in}{4.823036in}}%
\pgfpathlineto{\pgfqpoint{8.977434in}{4.854579in}}%
\pgfpathlineto{\pgfqpoint{8.992868in}{4.892709in}}%
\pgfpathlineto{\pgfqpoint{9.008302in}{4.936678in}}%
\pgfpathlineto{\pgfqpoint{9.039169in}{5.038393in}}%
\pgfpathlineto{\pgfqpoint{9.147204in}{5.422474in}}%
\pgfpathlineto{\pgfqpoint{9.162638in}{5.467055in}}%
\pgfpathlineto{\pgfqpoint{9.178072in}{5.506809in}}%
\pgfpathlineto{\pgfqpoint{9.193505in}{5.541388in}}%
\pgfpathlineto{\pgfqpoint{9.208939in}{5.570626in}}%
\pgfpathlineto{\pgfqpoint{9.224373in}{5.594541in}}%
\pgfpathlineto{\pgfqpoint{9.239806in}{5.613320in}}%
\pgfpathlineto{\pgfqpoint{9.255240in}{5.627310in}}%
\pgfpathlineto{\pgfqpoint{9.270673in}{5.636989in}}%
\pgfpathlineto{\pgfqpoint{9.286107in}{5.642946in}}%
\pgfpathlineto{\pgfqpoint{9.301541in}{5.645848in}}%
\pgfpathlineto{\pgfqpoint{9.316974in}{5.646407in}}%
\pgfpathlineto{\pgfqpoint{9.347842in}{5.643381in}}%
\pgfpathlineto{\pgfqpoint{9.394143in}{5.638069in}}%
\pgfpathlineto{\pgfqpoint{9.425010in}{5.639199in}}%
\pgfpathlineto{\pgfqpoint{9.455877in}{5.645324in}}%
\pgfpathlineto{\pgfqpoint{9.517612in}{5.663283in}}%
\pgfpathlineto{\pgfqpoint{9.533045in}{5.665519in}}%
\pgfpathlineto{\pgfqpoint{9.548479in}{5.665314in}}%
\pgfpathlineto{\pgfqpoint{9.563913in}{5.661836in}}%
\pgfpathlineto{\pgfqpoint{9.579346in}{5.654275in}}%
\pgfpathlineto{\pgfqpoint{9.594780in}{5.641881in}}%
\pgfpathlineto{\pgfqpoint{9.610213in}{5.623997in}}%
\pgfpathlineto{\pgfqpoint{9.625647in}{5.600096in}}%
\pgfpathlineto{\pgfqpoint{9.641081in}{5.569811in}}%
\pgfpathlineto{\pgfqpoint{9.656514in}{5.532957in}}%
\pgfpathlineto{\pgfqpoint{9.671948in}{5.489557in}}%
\pgfpathlineto{\pgfqpoint{9.687382in}{5.439852in}}%
\pgfpathlineto{\pgfqpoint{9.718249in}{5.323610in}}%
\pgfpathlineto{\pgfqpoint{9.733682in}{5.258662in}}%
\pgfpathlineto{\pgfqpoint{9.733682in}{5.258662in}}%
\pgfusepath{stroke}%
\end{pgfscope}%
\begin{pgfscope}%
\pgfpathrectangle{\pgfqpoint{5.698559in}{3.799602in}}{\pgfqpoint{4.227273in}{2.745455in}} %
\pgfusepath{clip}%
\pgfsetrectcap%
\pgfsetroundjoin%
\pgfsetlinewidth{0.501875pt}%
\definecolor{currentstroke}{rgb}{0.100000,0.587785,0.951057}%
\pgfsetstrokecolor{currentstroke}%
\pgfsetdash{}{0pt}%
\pgfpathmoveto{\pgfqpoint{5.890707in}{5.424415in}}%
\pgfpathlineto{\pgfqpoint{5.906141in}{5.448680in}}%
\pgfpathlineto{\pgfqpoint{5.921575in}{5.465794in}}%
\pgfpathlineto{\pgfqpoint{5.937008in}{5.474842in}}%
\pgfpathlineto{\pgfqpoint{5.952442in}{5.475141in}}%
\pgfpathlineto{\pgfqpoint{5.967875in}{5.466253in}}%
\pgfpathlineto{\pgfqpoint{5.983309in}{5.448002in}}%
\pgfpathlineto{\pgfqpoint{5.998743in}{5.420473in}}%
\pgfpathlineto{\pgfqpoint{6.014176in}{5.384009in}}%
\pgfpathlineto{\pgfqpoint{6.029610in}{5.339203in}}%
\pgfpathlineto{\pgfqpoint{6.045044in}{5.286875in}}%
\pgfpathlineto{\pgfqpoint{6.075911in}{5.163942in}}%
\pgfpathlineto{\pgfqpoint{6.122212in}{4.953815in}}%
\pgfpathlineto{\pgfqpoint{6.153079in}{4.814003in}}%
\pgfpathlineto{\pgfqpoint{6.183946in}{4.688386in}}%
\pgfpathlineto{\pgfqpoint{6.199380in}{4.634272in}}%
\pgfpathlineto{\pgfqpoint{6.214814in}{4.587483in}}%
\pgfpathlineto{\pgfqpoint{6.230247in}{4.548952in}}%
\pgfpathlineto{\pgfqpoint{6.245681in}{4.519415in}}%
\pgfpathlineto{\pgfqpoint{6.261115in}{4.499403in}}%
\pgfpathlineto{\pgfqpoint{6.276548in}{4.489230in}}%
\pgfpathlineto{\pgfqpoint{6.291982in}{4.489000in}}%
\pgfpathlineto{\pgfqpoint{6.307415in}{4.498603in}}%
\pgfpathlineto{\pgfqpoint{6.322849in}{4.517736in}}%
\pgfpathlineto{\pgfqpoint{6.338283in}{4.545910in}}%
\pgfpathlineto{\pgfqpoint{6.353716in}{4.582476in}}%
\pgfpathlineto{\pgfqpoint{6.369150in}{4.626645in}}%
\pgfpathlineto{\pgfqpoint{6.384584in}{4.677519in}}%
\pgfpathlineto{\pgfqpoint{6.415451in}{4.795390in}}%
\pgfpathlineto{\pgfqpoint{6.461752in}{4.996698in}}%
\pgfpathlineto{\pgfqpoint{6.523486in}{5.269220in}}%
\pgfpathlineto{\pgfqpoint{6.554354in}{5.393035in}}%
\pgfpathlineto{\pgfqpoint{6.585221in}{5.503163in}}%
\pgfpathlineto{\pgfqpoint{6.616088in}{5.597706in}}%
\pgfpathlineto{\pgfqpoint{6.646955in}{5.676189in}}%
\pgfpathlineto{\pgfqpoint{6.662389in}{5.709544in}}%
\pgfpathlineto{\pgfqpoint{6.677823in}{5.739130in}}%
\pgfpathlineto{\pgfqpoint{6.693256in}{5.765086in}}%
\pgfpathlineto{\pgfqpoint{6.708690in}{5.787552in}}%
\pgfpathlineto{\pgfqpoint{6.724124in}{5.806656in}}%
\pgfpathlineto{\pgfqpoint{6.739557in}{5.822505in}}%
\pgfpathlineto{\pgfqpoint{6.754991in}{5.835173in}}%
\pgfpathlineto{\pgfqpoint{6.770424in}{5.844702in}}%
\pgfpathlineto{\pgfqpoint{6.785858in}{5.851091in}}%
\pgfpathlineto{\pgfqpoint{6.801292in}{5.854305in}}%
\pgfpathlineto{\pgfqpoint{6.816725in}{5.854275in}}%
\pgfpathlineto{\pgfqpoint{6.832159in}{5.850906in}}%
\pgfpathlineto{\pgfqpoint{6.847593in}{5.844084in}}%
\pgfpathlineto{\pgfqpoint{6.863026in}{5.833689in}}%
\pgfpathlineto{\pgfqpoint{6.878460in}{5.819607in}}%
\pgfpathlineto{\pgfqpoint{6.893894in}{5.801743in}}%
\pgfpathlineto{\pgfqpoint{6.909327in}{5.780035in}}%
\pgfpathlineto{\pgfqpoint{6.924761in}{5.754466in}}%
\pgfpathlineto{\pgfqpoint{6.940194in}{5.725078in}}%
\pgfpathlineto{\pgfqpoint{6.971062in}{5.655360in}}%
\pgfpathlineto{\pgfqpoint{7.001929in}{5.572701in}}%
\pgfpathlineto{\pgfqpoint{7.032796in}{5.480235in}}%
\pgfpathlineto{\pgfqpoint{7.125398in}{5.191658in}}%
\pgfpathlineto{\pgfqpoint{7.156265in}{5.110090in}}%
\pgfpathlineto{\pgfqpoint{7.171699in}{5.074924in}}%
\pgfpathlineto{\pgfqpoint{7.187133in}{5.044133in}}%
\pgfpathlineto{\pgfqpoint{7.202566in}{5.018051in}}%
\pgfpathlineto{\pgfqpoint{7.218000in}{4.996888in}}%
\pgfpathlineto{\pgfqpoint{7.233434in}{4.980715in}}%
\pgfpathlineto{\pgfqpoint{7.248867in}{4.969463in}}%
\pgfpathlineto{\pgfqpoint{7.264301in}{4.962915in}}%
\pgfpathlineto{\pgfqpoint{7.279734in}{4.960715in}}%
\pgfpathlineto{\pgfqpoint{7.295168in}{4.962370in}}%
\pgfpathlineto{\pgfqpoint{7.310602in}{4.967268in}}%
\pgfpathlineto{\pgfqpoint{7.326035in}{4.974688in}}%
\pgfpathlineto{\pgfqpoint{7.403204in}{5.019988in}}%
\pgfpathlineto{\pgfqpoint{7.418637in}{5.024554in}}%
\pgfpathlineto{\pgfqpoint{7.434071in}{5.025758in}}%
\pgfpathlineto{\pgfqpoint{7.449504in}{5.022984in}}%
\pgfpathlineto{\pgfqpoint{7.464938in}{5.015758in}}%
\pgfpathlineto{\pgfqpoint{7.480372in}{5.003765in}}%
\pgfpathlineto{\pgfqpoint{7.495805in}{4.986868in}}%
\pgfpathlineto{\pgfqpoint{7.511239in}{4.965123in}}%
\pgfpathlineto{\pgfqpoint{7.526673in}{4.938778in}}%
\pgfpathlineto{\pgfqpoint{7.542106in}{4.908281in}}%
\pgfpathlineto{\pgfqpoint{7.572974in}{4.837557in}}%
\pgfpathlineto{\pgfqpoint{7.634708in}{4.685063in}}%
\pgfpathlineto{\pgfqpoint{7.650142in}{4.651728in}}%
\pgfpathlineto{\pgfqpoint{7.665575in}{4.622939in}}%
\pgfpathlineto{\pgfqpoint{7.681009in}{4.599979in}}%
\pgfpathlineto{\pgfqpoint{7.696443in}{4.584047in}}%
\pgfpathlineto{\pgfqpoint{7.711876in}{4.576222in}}%
\pgfpathlineto{\pgfqpoint{7.727310in}{4.577422in}}%
\pgfpathlineto{\pgfqpoint{7.742744in}{4.588376in}}%
\pgfpathlineto{\pgfqpoint{7.758177in}{4.609588in}}%
\pgfpathlineto{\pgfqpoint{7.773611in}{4.641323in}}%
\pgfpathlineto{\pgfqpoint{7.789044in}{4.683582in}}%
\pgfpathlineto{\pgfqpoint{7.804478in}{4.736099in}}%
\pgfpathlineto{\pgfqpoint{7.819912in}{4.798334in}}%
\pgfpathlineto{\pgfqpoint{7.835345in}{4.869480in}}%
\pgfpathlineto{\pgfqpoint{7.866213in}{5.034029in}}%
\pgfpathlineto{\pgfqpoint{7.912514in}{5.314202in}}%
\pgfpathlineto{\pgfqpoint{7.958814in}{5.591258in}}%
\pgfpathlineto{\pgfqpoint{7.989682in}{5.748891in}}%
\pgfpathlineto{\pgfqpoint{8.005115in}{5.814316in}}%
\pgfpathlineto{\pgfqpoint{8.020549in}{5.868840in}}%
\pgfpathlineto{\pgfqpoint{8.035983in}{5.911259in}}%
\pgfpathlineto{\pgfqpoint{8.051416in}{5.940634in}}%
\pgfpathlineto{\pgfqpoint{8.066850in}{5.956314in}}%
\pgfpathlineto{\pgfqpoint{8.082284in}{5.957954in}}%
\pgfpathlineto{\pgfqpoint{8.097717in}{5.945526in}}%
\pgfpathlineto{\pgfqpoint{8.113151in}{5.919318in}}%
\pgfpathlineto{\pgfqpoint{8.128584in}{5.879931in}}%
\pgfpathlineto{\pgfqpoint{8.144018in}{5.828260in}}%
\pgfpathlineto{\pgfqpoint{8.159452in}{5.765472in}}%
\pgfpathlineto{\pgfqpoint{8.174885in}{5.692976in}}%
\pgfpathlineto{\pgfqpoint{8.205753in}{5.525482in}}%
\pgfpathlineto{\pgfqpoint{8.298354in}{4.979232in}}%
\pgfpathlineto{\pgfqpoint{8.329222in}{4.831327in}}%
\pgfpathlineto{\pgfqpoint{8.344655in}{4.770347in}}%
\pgfpathlineto{\pgfqpoint{8.360089in}{4.719209in}}%
\pgfpathlineto{\pgfqpoint{8.375523in}{4.678454in}}%
\pgfpathlineto{\pgfqpoint{8.390956in}{4.648337in}}%
\pgfpathlineto{\pgfqpoint{8.406390in}{4.628816in}}%
\pgfpathlineto{\pgfqpoint{8.421823in}{4.619569in}}%
\pgfpathlineto{\pgfqpoint{8.437257in}{4.620004in}}%
\pgfpathlineto{\pgfqpoint{8.452691in}{4.629287in}}%
\pgfpathlineto{\pgfqpoint{8.468124in}{4.646370in}}%
\pgfpathlineto{\pgfqpoint{8.483558in}{4.670026in}}%
\pgfpathlineto{\pgfqpoint{8.498992in}{4.698894in}}%
\pgfpathlineto{\pgfqpoint{8.529859in}{4.766388in}}%
\pgfpathlineto{\pgfqpoint{8.576160in}{4.869711in}}%
\pgfpathlineto{\pgfqpoint{8.591593in}{4.899181in}}%
\pgfpathlineto{\pgfqpoint{8.607027in}{4.924232in}}%
\pgfpathlineto{\pgfqpoint{8.622461in}{4.943976in}}%
\pgfpathlineto{\pgfqpoint{8.637894in}{4.957744in}}%
\pgfpathlineto{\pgfqpoint{8.653328in}{4.965102in}}%
\pgfpathlineto{\pgfqpoint{8.668762in}{4.965860in}}%
\pgfpathlineto{\pgfqpoint{8.684195in}{4.960075in}}%
\pgfpathlineto{\pgfqpoint{8.699629in}{4.948049in}}%
\pgfpathlineto{\pgfqpoint{8.715063in}{4.930308in}}%
\pgfpathlineto{\pgfqpoint{8.730496in}{4.907593in}}%
\pgfpathlineto{\pgfqpoint{8.761363in}{4.851083in}}%
\pgfpathlineto{\pgfqpoint{8.823098in}{4.727243in}}%
\pgfpathlineto{\pgfqpoint{8.838532in}{4.701568in}}%
\pgfpathlineto{\pgfqpoint{8.853965in}{4.680499in}}%
\pgfpathlineto{\pgfqpoint{8.869399in}{4.665118in}}%
\pgfpathlineto{\pgfqpoint{8.884833in}{4.656364in}}%
\pgfpathlineto{\pgfqpoint{8.900266in}{4.655001in}}%
\pgfpathlineto{\pgfqpoint{8.915700in}{4.661601in}}%
\pgfpathlineto{\pgfqpoint{8.931133in}{4.676523in}}%
\pgfpathlineto{\pgfqpoint{8.946567in}{4.699911in}}%
\pgfpathlineto{\pgfqpoint{8.962001in}{4.731686in}}%
\pgfpathlineto{\pgfqpoint{8.977434in}{4.771554in}}%
\pgfpathlineto{\pgfqpoint{8.992868in}{4.819015in}}%
\pgfpathlineto{\pgfqpoint{9.008302in}{4.873377in}}%
\pgfpathlineto{\pgfqpoint{9.039169in}{4.999233in}}%
\pgfpathlineto{\pgfqpoint{9.085470in}{5.214343in}}%
\pgfpathlineto{\pgfqpoint{9.131771in}{5.431781in}}%
\pgfpathlineto{\pgfqpoint{9.162638in}{5.562424in}}%
\pgfpathlineto{\pgfqpoint{9.193505in}{5.672788in}}%
\pgfpathlineto{\pgfqpoint{9.208939in}{5.718693in}}%
\pgfpathlineto{\pgfqpoint{9.224373in}{5.757843in}}%
\pgfpathlineto{\pgfqpoint{9.239806in}{5.790022in}}%
\pgfpathlineto{\pgfqpoint{9.255240in}{5.815173in}}%
\pgfpathlineto{\pgfqpoint{9.270673in}{5.833396in}}%
\pgfpathlineto{\pgfqpoint{9.286107in}{5.844932in}}%
\pgfpathlineto{\pgfqpoint{9.301541in}{5.850146in}}%
\pgfpathlineto{\pgfqpoint{9.316974in}{5.849505in}}%
\pgfpathlineto{\pgfqpoint{9.332408in}{5.843558in}}%
\pgfpathlineto{\pgfqpoint{9.347842in}{5.832912in}}%
\pgfpathlineto{\pgfqpoint{9.363275in}{5.818203in}}%
\pgfpathlineto{\pgfqpoint{9.378709in}{5.800077in}}%
\pgfpathlineto{\pgfqpoint{9.409576in}{5.756049in}}%
\pgfpathlineto{\pgfqpoint{9.440443in}{5.705301in}}%
\pgfpathlineto{\pgfqpoint{9.502178in}{5.595668in}}%
\pgfpathlineto{\pgfqpoint{9.563913in}{5.481116in}}%
\pgfpathlineto{\pgfqpoint{9.610213in}{5.387437in}}%
\pgfpathlineto{\pgfqpoint{9.641081in}{5.318207in}}%
\pgfpathlineto{\pgfqpoint{9.671948in}{5.242360in}}%
\pgfpathlineto{\pgfqpoint{9.718249in}{5.117137in}}%
\pgfpathlineto{\pgfqpoint{9.733682in}{5.073312in}}%
\pgfpathlineto{\pgfqpoint{9.733682in}{5.073312in}}%
\pgfusepath{stroke}%
\end{pgfscope}%
\begin{pgfscope}%
\pgfpathrectangle{\pgfqpoint{5.698559in}{3.799602in}}{\pgfqpoint{4.227273in}{2.745455in}} %
\pgfusepath{clip}%
\pgfsetrectcap%
\pgfsetroundjoin%
\pgfsetlinewidth{0.501875pt}%
\definecolor{currentstroke}{rgb}{0.021569,0.682749,0.930229}%
\pgfsetstrokecolor{currentstroke}%
\pgfsetdash{}{0pt}%
\pgfpathmoveto{\pgfqpoint{5.890707in}{5.435668in}}%
\pgfpathlineto{\pgfqpoint{5.906141in}{5.440742in}}%
\pgfpathlineto{\pgfqpoint{5.921575in}{5.441551in}}%
\pgfpathlineto{\pgfqpoint{5.937008in}{5.437607in}}%
\pgfpathlineto{\pgfqpoint{5.952442in}{5.428558in}}%
\pgfpathlineto{\pgfqpoint{5.967875in}{5.414202in}}%
\pgfpathlineto{\pgfqpoint{5.983309in}{5.394491in}}%
\pgfpathlineto{\pgfqpoint{5.998743in}{5.369532in}}%
\pgfpathlineto{\pgfqpoint{6.014176in}{5.339586in}}%
\pgfpathlineto{\pgfqpoint{6.029610in}{5.305060in}}%
\pgfpathlineto{\pgfqpoint{6.060477in}{5.224559in}}%
\pgfpathlineto{\pgfqpoint{6.106778in}{5.086551in}}%
\pgfpathlineto{\pgfqpoint{6.153079in}{4.949429in}}%
\pgfpathlineto{\pgfqpoint{6.183946in}{4.870883in}}%
\pgfpathlineto{\pgfqpoint{6.199380in}{4.837985in}}%
\pgfpathlineto{\pgfqpoint{6.214814in}{4.810264in}}%
\pgfpathlineto{\pgfqpoint{6.230247in}{4.788280in}}%
\pgfpathlineto{\pgfqpoint{6.245681in}{4.772473in}}%
\pgfpathlineto{\pgfqpoint{6.261115in}{4.763153in}}%
\pgfpathlineto{\pgfqpoint{6.276548in}{4.760495in}}%
\pgfpathlineto{\pgfqpoint{6.291982in}{4.764544in}}%
\pgfpathlineto{\pgfqpoint{6.307415in}{4.775213in}}%
\pgfpathlineto{\pgfqpoint{6.322849in}{4.792292in}}%
\pgfpathlineto{\pgfqpoint{6.338283in}{4.815457in}}%
\pgfpathlineto{\pgfqpoint{6.353716in}{4.844284in}}%
\pgfpathlineto{\pgfqpoint{6.369150in}{4.878260in}}%
\pgfpathlineto{\pgfqpoint{6.400017in}{4.959279in}}%
\pgfpathlineto{\pgfqpoint{6.430885in}{5.053311in}}%
\pgfpathlineto{\pgfqpoint{6.492619in}{5.258615in}}%
\pgfpathlineto{\pgfqpoint{6.538920in}{5.408489in}}%
\pgfpathlineto{\pgfqpoint{6.569787in}{5.499506in}}%
\pgfpathlineto{\pgfqpoint{6.600655in}{5.580743in}}%
\pgfpathlineto{\pgfqpoint{6.631522in}{5.651016in}}%
\pgfpathlineto{\pgfqpoint{6.662389in}{5.709902in}}%
\pgfpathlineto{\pgfqpoint{6.693256in}{5.757462in}}%
\pgfpathlineto{\pgfqpoint{6.708690in}{5.777077in}}%
\pgfpathlineto{\pgfqpoint{6.724124in}{5.793961in}}%
\pgfpathlineto{\pgfqpoint{6.739557in}{5.808137in}}%
\pgfpathlineto{\pgfqpoint{6.754991in}{5.819613in}}%
\pgfpathlineto{\pgfqpoint{6.770424in}{5.828383in}}%
\pgfpathlineto{\pgfqpoint{6.785858in}{5.834417in}}%
\pgfpathlineto{\pgfqpoint{6.801292in}{5.837673in}}%
\pgfpathlineto{\pgfqpoint{6.816725in}{5.838089in}}%
\pgfpathlineto{\pgfqpoint{6.832159in}{5.835595in}}%
\pgfpathlineto{\pgfqpoint{6.847593in}{5.830113in}}%
\pgfpathlineto{\pgfqpoint{6.863026in}{5.821568in}}%
\pgfpathlineto{\pgfqpoint{6.878460in}{5.809891in}}%
\pgfpathlineto{\pgfqpoint{6.893894in}{5.795031in}}%
\pgfpathlineto{\pgfqpoint{6.909327in}{5.776962in}}%
\pgfpathlineto{\pgfqpoint{6.924761in}{5.755689in}}%
\pgfpathlineto{\pgfqpoint{6.940194in}{5.731256in}}%
\pgfpathlineto{\pgfqpoint{6.971062in}{5.673326in}}%
\pgfpathlineto{\pgfqpoint{7.001929in}{5.604515in}}%
\pgfpathlineto{\pgfqpoint{7.032796in}{5.527020in}}%
\pgfpathlineto{\pgfqpoint{7.094531in}{5.358543in}}%
\pgfpathlineto{\pgfqpoint{7.140832in}{5.235145in}}%
\pgfpathlineto{\pgfqpoint{7.171699in}{5.161268in}}%
\pgfpathlineto{\pgfqpoint{7.202566in}{5.097567in}}%
\pgfpathlineto{\pgfqpoint{7.233434in}{5.045856in}}%
\pgfpathlineto{\pgfqpoint{7.248867in}{5.024711in}}%
\pgfpathlineto{\pgfqpoint{7.264301in}{5.006647in}}%
\pgfpathlineto{\pgfqpoint{7.279734in}{4.991511in}}%
\pgfpathlineto{\pgfqpoint{7.295168in}{4.979064in}}%
\pgfpathlineto{\pgfqpoint{7.310602in}{4.968996in}}%
\pgfpathlineto{\pgfqpoint{7.341469in}{4.954430in}}%
\pgfpathlineto{\pgfqpoint{7.434071in}{4.921352in}}%
\pgfpathlineto{\pgfqpoint{7.449504in}{4.912572in}}%
\pgfpathlineto{\pgfqpoint{7.464938in}{4.901924in}}%
\pgfpathlineto{\pgfqpoint{7.480372in}{4.889234in}}%
\pgfpathlineto{\pgfqpoint{7.511239in}{4.857541in}}%
\pgfpathlineto{\pgfqpoint{7.542106in}{4.818239in}}%
\pgfpathlineto{\pgfqpoint{7.634708in}{4.687570in}}%
\pgfpathlineto{\pgfqpoint{7.650142in}{4.670599in}}%
\pgfpathlineto{\pgfqpoint{7.665575in}{4.656972in}}%
\pgfpathlineto{\pgfqpoint{7.681009in}{4.647416in}}%
\pgfpathlineto{\pgfqpoint{7.696443in}{4.642602in}}%
\pgfpathlineto{\pgfqpoint{7.711876in}{4.643129in}}%
\pgfpathlineto{\pgfqpoint{7.727310in}{4.649496in}}%
\pgfpathlineto{\pgfqpoint{7.742744in}{4.662084in}}%
\pgfpathlineto{\pgfqpoint{7.758177in}{4.681141in}}%
\pgfpathlineto{\pgfqpoint{7.773611in}{4.706768in}}%
\pgfpathlineto{\pgfqpoint{7.789044in}{4.738906in}}%
\pgfpathlineto{\pgfqpoint{7.804478in}{4.777336in}}%
\pgfpathlineto{\pgfqpoint{7.819912in}{4.821672in}}%
\pgfpathlineto{\pgfqpoint{7.850779in}{4.925735in}}%
\pgfpathlineto{\pgfqpoint{7.881646in}{5.044991in}}%
\pgfpathlineto{\pgfqpoint{7.958814in}{5.354042in}}%
\pgfpathlineto{\pgfqpoint{7.989682in}{5.457249in}}%
\pgfpathlineto{\pgfqpoint{8.005115in}{5.499996in}}%
\pgfpathlineto{\pgfqpoint{8.020549in}{5.535606in}}%
\pgfpathlineto{\pgfqpoint{8.035983in}{5.563324in}}%
\pgfpathlineto{\pgfqpoint{8.051416in}{5.582559in}}%
\pgfpathlineto{\pgfqpoint{8.066850in}{5.592897in}}%
\pgfpathlineto{\pgfqpoint{8.082284in}{5.594112in}}%
\pgfpathlineto{\pgfqpoint{8.097717in}{5.586172in}}%
\pgfpathlineto{\pgfqpoint{8.113151in}{5.569241in}}%
\pgfpathlineto{\pgfqpoint{8.128584in}{5.543672in}}%
\pgfpathlineto{\pgfqpoint{8.144018in}{5.510003in}}%
\pgfpathlineto{\pgfqpoint{8.159452in}{5.468938in}}%
\pgfpathlineto{\pgfqpoint{8.174885in}{5.421332in}}%
\pgfpathlineto{\pgfqpoint{8.205753in}{5.310524in}}%
\pgfpathlineto{\pgfqpoint{8.252053in}{5.122521in}}%
\pgfpathlineto{\pgfqpoint{8.298354in}{4.936944in}}%
\pgfpathlineto{\pgfqpoint{8.329222in}{4.828939in}}%
\pgfpathlineto{\pgfqpoint{8.344655in}{4.782180in}}%
\pgfpathlineto{\pgfqpoint{8.360089in}{4.741039in}}%
\pgfpathlineto{\pgfqpoint{8.375523in}{4.705897in}}%
\pgfpathlineto{\pgfqpoint{8.390956in}{4.676963in}}%
\pgfpathlineto{\pgfqpoint{8.406390in}{4.654275in}}%
\pgfpathlineto{\pgfqpoint{8.421823in}{4.637698in}}%
\pgfpathlineto{\pgfqpoint{8.437257in}{4.626939in}}%
\pgfpathlineto{\pgfqpoint{8.452691in}{4.621560in}}%
\pgfpathlineto{\pgfqpoint{8.468124in}{4.620994in}}%
\pgfpathlineto{\pgfqpoint{8.483558in}{4.624568in}}%
\pgfpathlineto{\pgfqpoint{8.498992in}{4.631525in}}%
\pgfpathlineto{\pgfqpoint{8.529859in}{4.652303in}}%
\pgfpathlineto{\pgfqpoint{8.576160in}{4.688092in}}%
\pgfpathlineto{\pgfqpoint{8.591593in}{4.698151in}}%
\pgfpathlineto{\pgfqpoint{8.607027in}{4.706195in}}%
\pgfpathlineto{\pgfqpoint{8.622461in}{4.711736in}}%
\pgfpathlineto{\pgfqpoint{8.637894in}{4.714410in}}%
\pgfpathlineto{\pgfqpoint{8.653328in}{4.713994in}}%
\pgfpathlineto{\pgfqpoint{8.668762in}{4.710407in}}%
\pgfpathlineto{\pgfqpoint{8.684195in}{4.703710in}}%
\pgfpathlineto{\pgfqpoint{8.699629in}{4.694108in}}%
\pgfpathlineto{\pgfqpoint{8.715063in}{4.681936in}}%
\pgfpathlineto{\pgfqpoint{8.745930in}{4.651820in}}%
\pgfpathlineto{\pgfqpoint{8.807664in}{4.586881in}}%
\pgfpathlineto{\pgfqpoint{8.823098in}{4.574049in}}%
\pgfpathlineto{\pgfqpoint{8.838532in}{4.564099in}}%
\pgfpathlineto{\pgfqpoint{8.853965in}{4.557727in}}%
\pgfpathlineto{\pgfqpoint{8.869399in}{4.555554in}}%
\pgfpathlineto{\pgfqpoint{8.884833in}{4.558111in}}%
\pgfpathlineto{\pgfqpoint{8.900266in}{4.565819in}}%
\pgfpathlineto{\pgfqpoint{8.915700in}{4.578980in}}%
\pgfpathlineto{\pgfqpoint{8.931133in}{4.597765in}}%
\pgfpathlineto{\pgfqpoint{8.946567in}{4.622213in}}%
\pgfpathlineto{\pgfqpoint{8.962001in}{4.652223in}}%
\pgfpathlineto{\pgfqpoint{8.977434in}{4.687566in}}%
\pgfpathlineto{\pgfqpoint{8.992868in}{4.727884in}}%
\pgfpathlineto{\pgfqpoint{9.023735in}{4.821448in}}%
\pgfpathlineto{\pgfqpoint{9.054603in}{4.927989in}}%
\pgfpathlineto{\pgfqpoint{9.147204in}{5.263734in}}%
\pgfpathlineto{\pgfqpoint{9.178072in}{5.360712in}}%
\pgfpathlineto{\pgfqpoint{9.208939in}{5.442206in}}%
\pgfpathlineto{\pgfqpoint{9.224373in}{5.476242in}}%
\pgfpathlineto{\pgfqpoint{9.239806in}{5.505525in}}%
\pgfpathlineto{\pgfqpoint{9.255240in}{5.529975in}}%
\pgfpathlineto{\pgfqpoint{9.270673in}{5.549606in}}%
\pgfpathlineto{\pgfqpoint{9.286107in}{5.564522in}}%
\pgfpathlineto{\pgfqpoint{9.301541in}{5.574903in}}%
\pgfpathlineto{\pgfqpoint{9.316974in}{5.580998in}}%
\pgfpathlineto{\pgfqpoint{9.332408in}{5.583108in}}%
\pgfpathlineto{\pgfqpoint{9.347842in}{5.581571in}}%
\pgfpathlineto{\pgfqpoint{9.363275in}{5.576750in}}%
\pgfpathlineto{\pgfqpoint{9.378709in}{5.569019in}}%
\pgfpathlineto{\pgfqpoint{9.394143in}{5.558743in}}%
\pgfpathlineto{\pgfqpoint{9.425010in}{5.531929in}}%
\pgfpathlineto{\pgfqpoint{9.455877in}{5.498708in}}%
\pgfpathlineto{\pgfqpoint{9.486744in}{5.460779in}}%
\pgfpathlineto{\pgfqpoint{9.517612in}{5.419037in}}%
\pgfpathlineto{\pgfqpoint{9.548479in}{5.373624in}}%
\pgfpathlineto{\pgfqpoint{9.579346in}{5.324130in}}%
\pgfpathlineto{\pgfqpoint{9.610213in}{5.269890in}}%
\pgfpathlineto{\pgfqpoint{9.641081in}{5.210345in}}%
\pgfpathlineto{\pgfqpoint{9.671948in}{5.145388in}}%
\pgfpathlineto{\pgfqpoint{9.718249in}{5.039437in}}%
\pgfpathlineto{\pgfqpoint{9.733682in}{5.002679in}}%
\pgfpathlineto{\pgfqpoint{9.733682in}{5.002679in}}%
\pgfusepath{stroke}%
\end{pgfscope}%
\begin{pgfscope}%
\pgfpathrectangle{\pgfqpoint{5.698559in}{3.799602in}}{\pgfqpoint{4.227273in}{2.745455in}} %
\pgfusepath{clip}%
\pgfsetrectcap%
\pgfsetroundjoin%
\pgfsetlinewidth{0.501875pt}%
\definecolor{currentstroke}{rgb}{0.056863,0.767363,0.905873}%
\pgfsetstrokecolor{currentstroke}%
\pgfsetdash{}{0pt}%
\pgfpathmoveto{\pgfqpoint{5.890707in}{5.504363in}}%
\pgfpathlineto{\pgfqpoint{5.906141in}{5.502347in}}%
\pgfpathlineto{\pgfqpoint{5.921575in}{5.496182in}}%
\pgfpathlineto{\pgfqpoint{5.937008in}{5.485598in}}%
\pgfpathlineto{\pgfqpoint{5.952442in}{5.470441in}}%
\pgfpathlineto{\pgfqpoint{5.967875in}{5.450676in}}%
\pgfpathlineto{\pgfqpoint{5.983309in}{5.426392in}}%
\pgfpathlineto{\pgfqpoint{5.998743in}{5.397798in}}%
\pgfpathlineto{\pgfqpoint{6.029610in}{5.329118in}}%
\pgfpathlineto{\pgfqpoint{6.060477in}{5.248564in}}%
\pgfpathlineto{\pgfqpoint{6.153079in}{4.992977in}}%
\pgfpathlineto{\pgfqpoint{6.183946in}{4.924391in}}%
\pgfpathlineto{\pgfqpoint{6.199380in}{4.896526in}}%
\pgfpathlineto{\pgfqpoint{6.214814in}{4.873714in}}%
\pgfpathlineto{\pgfqpoint{6.230247in}{4.856416in}}%
\pgfpathlineto{\pgfqpoint{6.245681in}{4.844985in}}%
\pgfpathlineto{\pgfqpoint{6.261115in}{4.839662in}}%
\pgfpathlineto{\pgfqpoint{6.276548in}{4.840575in}}%
\pgfpathlineto{\pgfqpoint{6.291982in}{4.847735in}}%
\pgfpathlineto{\pgfqpoint{6.307415in}{4.861042in}}%
\pgfpathlineto{\pgfqpoint{6.322849in}{4.880288in}}%
\pgfpathlineto{\pgfqpoint{6.338283in}{4.905167in}}%
\pgfpathlineto{\pgfqpoint{6.353716in}{4.935279in}}%
\pgfpathlineto{\pgfqpoint{6.369150in}{4.970149in}}%
\pgfpathlineto{\pgfqpoint{6.400017in}{5.051942in}}%
\pgfpathlineto{\pgfqpoint{6.430885in}{5.145670in}}%
\pgfpathlineto{\pgfqpoint{6.538920in}{5.493491in}}%
\pgfpathlineto{\pgfqpoint{6.569787in}{5.580008in}}%
\pgfpathlineto{\pgfqpoint{6.600655in}{5.654879in}}%
\pgfpathlineto{\pgfqpoint{6.631522in}{5.716333in}}%
\pgfpathlineto{\pgfqpoint{6.646955in}{5.741701in}}%
\pgfpathlineto{\pgfqpoint{6.662389in}{5.763412in}}%
\pgfpathlineto{\pgfqpoint{6.677823in}{5.781444in}}%
\pgfpathlineto{\pgfqpoint{6.693256in}{5.795809in}}%
\pgfpathlineto{\pgfqpoint{6.708690in}{5.806540in}}%
\pgfpathlineto{\pgfqpoint{6.724124in}{5.813686in}}%
\pgfpathlineto{\pgfqpoint{6.739557in}{5.817312in}}%
\pgfpathlineto{\pgfqpoint{6.754991in}{5.817491in}}%
\pgfpathlineto{\pgfqpoint{6.770424in}{5.814302in}}%
\pgfpathlineto{\pgfqpoint{6.785858in}{5.807828in}}%
\pgfpathlineto{\pgfqpoint{6.801292in}{5.798156in}}%
\pgfpathlineto{\pgfqpoint{6.816725in}{5.785374in}}%
\pgfpathlineto{\pgfqpoint{6.832159in}{5.769577in}}%
\pgfpathlineto{\pgfqpoint{6.847593in}{5.750867in}}%
\pgfpathlineto{\pgfqpoint{6.863026in}{5.729354in}}%
\pgfpathlineto{\pgfqpoint{6.893894in}{5.678425in}}%
\pgfpathlineto{\pgfqpoint{6.924761in}{5.617983in}}%
\pgfpathlineto{\pgfqpoint{6.955628in}{5.549583in}}%
\pgfpathlineto{\pgfqpoint{7.001929in}{5.436528in}}%
\pgfpathlineto{\pgfqpoint{7.079097in}{5.242441in}}%
\pgfpathlineto{\pgfqpoint{7.109964in}{5.171690in}}%
\pgfpathlineto{\pgfqpoint{7.140832in}{5.109317in}}%
\pgfpathlineto{\pgfqpoint{7.171699in}{5.057683in}}%
\pgfpathlineto{\pgfqpoint{7.187133in}{5.036417in}}%
\pgfpathlineto{\pgfqpoint{7.202566in}{5.018344in}}%
\pgfpathlineto{\pgfqpoint{7.218000in}{5.003497in}}%
\pgfpathlineto{\pgfqpoint{7.233434in}{4.991837in}}%
\pgfpathlineto{\pgfqpoint{7.248867in}{4.983249in}}%
\pgfpathlineto{\pgfqpoint{7.264301in}{4.977547in}}%
\pgfpathlineto{\pgfqpoint{7.279734in}{4.974474in}}%
\pgfpathlineto{\pgfqpoint{7.295168in}{4.973707in}}%
\pgfpathlineto{\pgfqpoint{7.326035in}{4.977525in}}%
\pgfpathlineto{\pgfqpoint{7.403204in}{4.996373in}}%
\pgfpathlineto{\pgfqpoint{7.418637in}{4.997836in}}%
\pgfpathlineto{\pgfqpoint{7.434071in}{4.997502in}}%
\pgfpathlineto{\pgfqpoint{7.449504in}{4.995026in}}%
\pgfpathlineto{\pgfqpoint{7.464938in}{4.990141in}}%
\pgfpathlineto{\pgfqpoint{7.480372in}{4.982656in}}%
\pgfpathlineto{\pgfqpoint{7.495805in}{4.972477in}}%
\pgfpathlineto{\pgfqpoint{7.511239in}{4.959603in}}%
\pgfpathlineto{\pgfqpoint{7.526673in}{4.944132in}}%
\pgfpathlineto{\pgfqpoint{7.557540in}{4.906290in}}%
\pgfpathlineto{\pgfqpoint{7.588407in}{4.861667in}}%
\pgfpathlineto{\pgfqpoint{7.650142in}{4.769056in}}%
\pgfpathlineto{\pgfqpoint{7.665575in}{4.748923in}}%
\pgfpathlineto{\pgfqpoint{7.681009in}{4.731330in}}%
\pgfpathlineto{\pgfqpoint{7.696443in}{4.716908in}}%
\pgfpathlineto{\pgfqpoint{7.711876in}{4.706239in}}%
\pgfpathlineto{\pgfqpoint{7.727310in}{4.699838in}}%
\pgfpathlineto{\pgfqpoint{7.742744in}{4.698135in}}%
\pgfpathlineto{\pgfqpoint{7.758177in}{4.701465in}}%
\pgfpathlineto{\pgfqpoint{7.773611in}{4.710051in}}%
\pgfpathlineto{\pgfqpoint{7.789044in}{4.723996in}}%
\pgfpathlineto{\pgfqpoint{7.804478in}{4.743280in}}%
\pgfpathlineto{\pgfqpoint{7.819912in}{4.767756in}}%
\pgfpathlineto{\pgfqpoint{7.835345in}{4.797147in}}%
\pgfpathlineto{\pgfqpoint{7.850779in}{4.831058in}}%
\pgfpathlineto{\pgfqpoint{7.881646in}{4.910278in}}%
\pgfpathlineto{\pgfqpoint{7.912514in}{5.000144in}}%
\pgfpathlineto{\pgfqpoint{7.974248in}{5.185444in}}%
\pgfpathlineto{\pgfqpoint{8.005115in}{5.266933in}}%
\pgfpathlineto{\pgfqpoint{8.020549in}{5.302051in}}%
\pgfpathlineto{\pgfqpoint{8.035983in}{5.332517in}}%
\pgfpathlineto{\pgfqpoint{8.051416in}{5.357778in}}%
\pgfpathlineto{\pgfqpoint{8.066850in}{5.377395in}}%
\pgfpathlineto{\pgfqpoint{8.082284in}{5.391052in}}%
\pgfpathlineto{\pgfqpoint{8.097717in}{5.398565in}}%
\pgfpathlineto{\pgfqpoint{8.113151in}{5.399880in}}%
\pgfpathlineto{\pgfqpoint{8.128584in}{5.395076in}}%
\pgfpathlineto{\pgfqpoint{8.144018in}{5.384361in}}%
\pgfpathlineto{\pgfqpoint{8.159452in}{5.368063in}}%
\pgfpathlineto{\pgfqpoint{8.174885in}{5.346616in}}%
\pgfpathlineto{\pgfqpoint{8.190319in}{5.320555in}}%
\pgfpathlineto{\pgfqpoint{8.221186in}{5.257098in}}%
\pgfpathlineto{\pgfqpoint{8.252053in}{5.183222in}}%
\pgfpathlineto{\pgfqpoint{8.329222in}{4.990800in}}%
\pgfpathlineto{\pgfqpoint{8.360089in}{4.923703in}}%
\pgfpathlineto{\pgfqpoint{8.390956in}{4.867135in}}%
\pgfpathlineto{\pgfqpoint{8.406390in}{4.843207in}}%
\pgfpathlineto{\pgfqpoint{8.421823in}{4.822193in}}%
\pgfpathlineto{\pgfqpoint{8.437257in}{4.803982in}}%
\pgfpathlineto{\pgfqpoint{8.452691in}{4.788372in}}%
\pgfpathlineto{\pgfqpoint{8.483558in}{4.763786in}}%
\pgfpathlineto{\pgfqpoint{8.514425in}{4.745561in}}%
\pgfpathlineto{\pgfqpoint{8.591593in}{4.705991in}}%
\pgfpathlineto{\pgfqpoint{8.622461in}{4.685127in}}%
\pgfpathlineto{\pgfqpoint{8.653328in}{4.659086in}}%
\pgfpathlineto{\pgfqpoint{8.684195in}{4.628143in}}%
\pgfpathlineto{\pgfqpoint{8.776797in}{4.528433in}}%
\pgfpathlineto{\pgfqpoint{8.792231in}{4.515533in}}%
\pgfpathlineto{\pgfqpoint{8.807664in}{4.505090in}}%
\pgfpathlineto{\pgfqpoint{8.823098in}{4.497615in}}%
\pgfpathlineto{\pgfqpoint{8.838532in}{4.493578in}}%
\pgfpathlineto{\pgfqpoint{8.853965in}{4.493396in}}%
\pgfpathlineto{\pgfqpoint{8.869399in}{4.497415in}}%
\pgfpathlineto{\pgfqpoint{8.884833in}{4.505905in}}%
\pgfpathlineto{\pgfqpoint{8.900266in}{4.519040in}}%
\pgfpathlineto{\pgfqpoint{8.915700in}{4.536902in}}%
\pgfpathlineto{\pgfqpoint{8.931133in}{4.559467in}}%
\pgfpathlineto{\pgfqpoint{8.946567in}{4.586608in}}%
\pgfpathlineto{\pgfqpoint{8.962001in}{4.618099in}}%
\pgfpathlineto{\pgfqpoint{8.992868in}{4.692729in}}%
\pgfpathlineto{\pgfqpoint{9.023735in}{4.779715in}}%
\pgfpathlineto{\pgfqpoint{9.085470in}{4.971126in}}%
\pgfpathlineto{\pgfqpoint{9.116337in}{5.064625in}}%
\pgfpathlineto{\pgfqpoint{9.147204in}{5.149658in}}%
\pgfpathlineto{\pgfqpoint{9.178072in}{5.221907in}}%
\pgfpathlineto{\pgfqpoint{9.193505in}{5.252207in}}%
\pgfpathlineto{\pgfqpoint{9.208939in}{5.278245in}}%
\pgfpathlineto{\pgfqpoint{9.224373in}{5.299851in}}%
\pgfpathlineto{\pgfqpoint{9.239806in}{5.316946in}}%
\pgfpathlineto{\pgfqpoint{9.255240in}{5.329540in}}%
\pgfpathlineto{\pgfqpoint{9.270673in}{5.337730in}}%
\pgfpathlineto{\pgfqpoint{9.286107in}{5.341689in}}%
\pgfpathlineto{\pgfqpoint{9.301541in}{5.341661in}}%
\pgfpathlineto{\pgfqpoint{9.316974in}{5.337949in}}%
\pgfpathlineto{\pgfqpoint{9.332408in}{5.330904in}}%
\pgfpathlineto{\pgfqpoint{9.347842in}{5.320912in}}%
\pgfpathlineto{\pgfqpoint{9.363275in}{5.308383in}}%
\pgfpathlineto{\pgfqpoint{9.394143in}{5.277393in}}%
\pgfpathlineto{\pgfqpoint{9.425010in}{5.241205in}}%
\pgfpathlineto{\pgfqpoint{9.533045in}{5.109777in}}%
\pgfpathlineto{\pgfqpoint{9.579346in}{5.060706in}}%
\pgfpathlineto{\pgfqpoint{9.625647in}{5.016286in}}%
\pgfpathlineto{\pgfqpoint{9.702815in}{4.947585in}}%
\pgfpathlineto{\pgfqpoint{9.733682in}{4.921280in}}%
\pgfpathlineto{\pgfqpoint{9.733682in}{4.921280in}}%
\pgfusepath{stroke}%
\end{pgfscope}%
\begin{pgfscope}%
\pgfpathrectangle{\pgfqpoint{5.698559in}{3.799602in}}{\pgfqpoint{4.227273in}{2.745455in}} %
\pgfusepath{clip}%
\pgfsetrectcap%
\pgfsetroundjoin%
\pgfsetlinewidth{0.501875pt}%
\definecolor{currentstroke}{rgb}{0.135294,0.840344,0.878081}%
\pgfsetstrokecolor{currentstroke}%
\pgfsetdash{}{0pt}%
\pgfpathmoveto{\pgfqpoint{5.890707in}{5.495191in}}%
\pgfpathlineto{\pgfqpoint{5.906141in}{5.487954in}}%
\pgfpathlineto{\pgfqpoint{5.921575in}{5.478390in}}%
\pgfpathlineto{\pgfqpoint{5.937008in}{5.466515in}}%
\pgfpathlineto{\pgfqpoint{5.952442in}{5.452370in}}%
\pgfpathlineto{\pgfqpoint{5.983309in}{5.417556in}}%
\pgfpathlineto{\pgfqpoint{6.014176in}{5.374788in}}%
\pgfpathlineto{\pgfqpoint{6.045044in}{5.325350in}}%
\pgfpathlineto{\pgfqpoint{6.091345in}{5.242727in}}%
\pgfpathlineto{\pgfqpoint{6.153079in}{5.130357in}}%
\pgfpathlineto{\pgfqpoint{6.183946in}{5.080621in}}%
\pgfpathlineto{\pgfqpoint{6.199380in}{5.059004in}}%
\pgfpathlineto{\pgfqpoint{6.214814in}{5.040201in}}%
\pgfpathlineto{\pgfqpoint{6.230247in}{5.024688in}}%
\pgfpathlineto{\pgfqpoint{6.245681in}{5.012917in}}%
\pgfpathlineto{\pgfqpoint{6.261115in}{5.005309in}}%
\pgfpathlineto{\pgfqpoint{6.276548in}{5.002236in}}%
\pgfpathlineto{\pgfqpoint{6.291982in}{5.004016in}}%
\pgfpathlineto{\pgfqpoint{6.307415in}{5.010897in}}%
\pgfpathlineto{\pgfqpoint{6.322849in}{5.023045in}}%
\pgfpathlineto{\pgfqpoint{6.338283in}{5.040541in}}%
\pgfpathlineto{\pgfqpoint{6.353716in}{5.063370in}}%
\pgfpathlineto{\pgfqpoint{6.369150in}{5.091415in}}%
\pgfpathlineto{\pgfqpoint{6.384584in}{5.124459in}}%
\pgfpathlineto{\pgfqpoint{6.400017in}{5.162180in}}%
\pgfpathlineto{\pgfqpoint{6.430885in}{5.249884in}}%
\pgfpathlineto{\pgfqpoint{6.461752in}{5.350105in}}%
\pgfpathlineto{\pgfqpoint{6.554354in}{5.667296in}}%
\pgfpathlineto{\pgfqpoint{6.585221in}{5.757829in}}%
\pgfpathlineto{\pgfqpoint{6.600655in}{5.797138in}}%
\pgfpathlineto{\pgfqpoint{6.616088in}{5.831789in}}%
\pgfpathlineto{\pgfqpoint{6.631522in}{5.861387in}}%
\pgfpathlineto{\pgfqpoint{6.646955in}{5.885646in}}%
\pgfpathlineto{\pgfqpoint{6.662389in}{5.904386in}}%
\pgfpathlineto{\pgfqpoint{6.677823in}{5.917538in}}%
\pgfpathlineto{\pgfqpoint{6.693256in}{5.925140in}}%
\pgfpathlineto{\pgfqpoint{6.708690in}{5.927331in}}%
\pgfpathlineto{\pgfqpoint{6.724124in}{5.924338in}}%
\pgfpathlineto{\pgfqpoint{6.739557in}{5.916474in}}%
\pgfpathlineto{\pgfqpoint{6.754991in}{5.904113in}}%
\pgfpathlineto{\pgfqpoint{6.770424in}{5.887687in}}%
\pgfpathlineto{\pgfqpoint{6.785858in}{5.867661in}}%
\pgfpathlineto{\pgfqpoint{6.816725in}{5.818766in}}%
\pgfpathlineto{\pgfqpoint{6.847593in}{5.761314in}}%
\pgfpathlineto{\pgfqpoint{6.893894in}{5.666679in}}%
\pgfpathlineto{\pgfqpoint{6.986495in}{5.475122in}}%
\pgfpathlineto{\pgfqpoint{7.048230in}{5.355581in}}%
\pgfpathlineto{\pgfqpoint{7.094531in}{5.271279in}}%
\pgfpathlineto{\pgfqpoint{7.140832in}{5.193357in}}%
\pgfpathlineto{\pgfqpoint{7.171699in}{5.146822in}}%
\pgfpathlineto{\pgfqpoint{7.202566in}{5.106437in}}%
\pgfpathlineto{\pgfqpoint{7.233434in}{5.073827in}}%
\pgfpathlineto{\pgfqpoint{7.248867in}{5.060869in}}%
\pgfpathlineto{\pgfqpoint{7.264301in}{5.050301in}}%
\pgfpathlineto{\pgfqpoint{7.279734in}{5.042175in}}%
\pgfpathlineto{\pgfqpoint{7.295168in}{5.036484in}}%
\pgfpathlineto{\pgfqpoint{7.310602in}{5.033148in}}%
\pgfpathlineto{\pgfqpoint{7.326035in}{5.032017in}}%
\pgfpathlineto{\pgfqpoint{7.341469in}{5.032863in}}%
\pgfpathlineto{\pgfqpoint{7.372336in}{5.039215in}}%
\pgfpathlineto{\pgfqpoint{7.449504in}{5.061528in}}%
\pgfpathlineto{\pgfqpoint{7.464938in}{5.062995in}}%
\pgfpathlineto{\pgfqpoint{7.480372in}{5.062289in}}%
\pgfpathlineto{\pgfqpoint{7.495805in}{5.058991in}}%
\pgfpathlineto{\pgfqpoint{7.511239in}{5.052757in}}%
\pgfpathlineto{\pgfqpoint{7.526673in}{5.043334in}}%
\pgfpathlineto{\pgfqpoint{7.542106in}{5.030579in}}%
\pgfpathlineto{\pgfqpoint{7.557540in}{5.014460in}}%
\pgfpathlineto{\pgfqpoint{7.572974in}{4.995073in}}%
\pgfpathlineto{\pgfqpoint{7.603841in}{4.947499in}}%
\pgfpathlineto{\pgfqpoint{7.634708in}{4.891078in}}%
\pgfpathlineto{\pgfqpoint{7.696443in}{4.772536in}}%
\pgfpathlineto{\pgfqpoint{7.727310in}{4.723059in}}%
\pgfpathlineto{\pgfqpoint{7.742744in}{4.703555in}}%
\pgfpathlineto{\pgfqpoint{7.758177in}{4.688470in}}%
\pgfpathlineto{\pgfqpoint{7.773611in}{4.678383in}}%
\pgfpathlineto{\pgfqpoint{7.789044in}{4.673752in}}%
\pgfpathlineto{\pgfqpoint{7.804478in}{4.674895in}}%
\pgfpathlineto{\pgfqpoint{7.819912in}{4.681982in}}%
\pgfpathlineto{\pgfqpoint{7.835345in}{4.695021in}}%
\pgfpathlineto{\pgfqpoint{7.850779in}{4.713859in}}%
\pgfpathlineto{\pgfqpoint{7.866213in}{4.738178in}}%
\pgfpathlineto{\pgfqpoint{7.881646in}{4.767509in}}%
\pgfpathlineto{\pgfqpoint{7.912514in}{4.838633in}}%
\pgfpathlineto{\pgfqpoint{7.943381in}{4.920959in}}%
\pgfpathlineto{\pgfqpoint{8.005115in}{5.088726in}}%
\pgfpathlineto{\pgfqpoint{8.035983in}{5.158892in}}%
\pgfpathlineto{\pgfqpoint{8.051416in}{5.187681in}}%
\pgfpathlineto{\pgfqpoint{8.066850in}{5.211472in}}%
\pgfpathlineto{\pgfqpoint{8.082284in}{5.229838in}}%
\pgfpathlineto{\pgfqpoint{8.097717in}{5.242508in}}%
\pgfpathlineto{\pgfqpoint{8.113151in}{5.249364in}}%
\pgfpathlineto{\pgfqpoint{8.128584in}{5.250443in}}%
\pgfpathlineto{\pgfqpoint{8.144018in}{5.245935in}}%
\pgfpathlineto{\pgfqpoint{8.159452in}{5.236164in}}%
\pgfpathlineto{\pgfqpoint{8.174885in}{5.221579in}}%
\pgfpathlineto{\pgfqpoint{8.190319in}{5.202731in}}%
\pgfpathlineto{\pgfqpoint{8.205753in}{5.180252in}}%
\pgfpathlineto{\pgfqpoint{8.236620in}{5.127184in}}%
\pgfpathlineto{\pgfqpoint{8.329222in}{4.952528in}}%
\pgfpathlineto{\pgfqpoint{8.360089in}{4.903319in}}%
\pgfpathlineto{\pgfqpoint{8.390956in}{4.861911in}}%
\pgfpathlineto{\pgfqpoint{8.421823in}{4.827711in}}%
\pgfpathlineto{\pgfqpoint{8.452691in}{4.798731in}}%
\pgfpathlineto{\pgfqpoint{8.529859in}{4.729946in}}%
\pgfpathlineto{\pgfqpoint{8.560726in}{4.696817in}}%
\pgfpathlineto{\pgfqpoint{8.591593in}{4.658333in}}%
\pgfpathlineto{\pgfqpoint{8.622461in}{4.615010in}}%
\pgfpathlineto{\pgfqpoint{8.699629in}{4.502129in}}%
\pgfpathlineto{\pgfqpoint{8.730496in}{4.465538in}}%
\pgfpathlineto{\pgfqpoint{8.745930in}{4.451311in}}%
\pgfpathlineto{\pgfqpoint{8.761363in}{4.440488in}}%
\pgfpathlineto{\pgfqpoint{8.776797in}{4.433521in}}%
\pgfpathlineto{\pgfqpoint{8.792231in}{4.430784in}}%
\pgfpathlineto{\pgfqpoint{8.807664in}{4.432565in}}%
\pgfpathlineto{\pgfqpoint{8.823098in}{4.439052in}}%
\pgfpathlineto{\pgfqpoint{8.838532in}{4.450326in}}%
\pgfpathlineto{\pgfqpoint{8.853965in}{4.466360in}}%
\pgfpathlineto{\pgfqpoint{8.869399in}{4.487020in}}%
\pgfpathlineto{\pgfqpoint{8.884833in}{4.512065in}}%
\pgfpathlineto{\pgfqpoint{8.900266in}{4.541159in}}%
\pgfpathlineto{\pgfqpoint{8.931133in}{4.609728in}}%
\pgfpathlineto{\pgfqpoint{8.962001in}{4.688557in}}%
\pgfpathlineto{\pgfqpoint{9.054603in}{4.938366in}}%
\pgfpathlineto{\pgfqpoint{9.085470in}{5.011491in}}%
\pgfpathlineto{\pgfqpoint{9.116337in}{5.074428in}}%
\pgfpathlineto{\pgfqpoint{9.147204in}{5.125863in}}%
\pgfpathlineto{\pgfqpoint{9.162638in}{5.147172in}}%
\pgfpathlineto{\pgfqpoint{9.178072in}{5.165620in}}%
\pgfpathlineto{\pgfqpoint{9.193505in}{5.181328in}}%
\pgfpathlineto{\pgfqpoint{9.208939in}{5.194456in}}%
\pgfpathlineto{\pgfqpoint{9.224373in}{5.205193in}}%
\pgfpathlineto{\pgfqpoint{9.239806in}{5.213745in}}%
\pgfpathlineto{\pgfqpoint{9.255240in}{5.220326in}}%
\pgfpathlineto{\pgfqpoint{9.286107in}{5.228405in}}%
\pgfpathlineto{\pgfqpoint{9.316974in}{5.230956in}}%
\pgfpathlineto{\pgfqpoint{9.347842in}{5.229167in}}%
\pgfpathlineto{\pgfqpoint{9.378709in}{5.223832in}}%
\pgfpathlineto{\pgfqpoint{9.409576in}{5.215389in}}%
\pgfpathlineto{\pgfqpoint{9.440443in}{5.204065in}}%
\pgfpathlineto{\pgfqpoint{9.471311in}{5.190054in}}%
\pgfpathlineto{\pgfqpoint{9.517612in}{5.164840in}}%
\pgfpathlineto{\pgfqpoint{9.610213in}{5.110443in}}%
\pgfpathlineto{\pgfqpoint{9.641081in}{5.096107in}}%
\pgfpathlineto{\pgfqpoint{9.671948in}{5.086094in}}%
\pgfpathlineto{\pgfqpoint{9.702815in}{5.081456in}}%
\pgfpathlineto{\pgfqpoint{9.733682in}{5.082670in}}%
\pgfpathlineto{\pgfqpoint{9.733682in}{5.082670in}}%
\pgfusepath{stroke}%
\end{pgfscope}%
\begin{pgfscope}%
\pgfpathrectangle{\pgfqpoint{5.698559in}{3.799602in}}{\pgfqpoint{4.227273in}{2.745455in}} %
\pgfusepath{clip}%
\pgfsetrectcap%
\pgfsetroundjoin%
\pgfsetlinewidth{0.501875pt}%
\definecolor{currentstroke}{rgb}{0.221569,0.905873,0.843667}%
\pgfsetstrokecolor{currentstroke}%
\pgfsetdash{}{0pt}%
\pgfpathmoveto{\pgfqpoint{5.890707in}{5.576767in}}%
\pgfpathlineto{\pgfqpoint{5.906141in}{5.571359in}}%
\pgfpathlineto{\pgfqpoint{5.921575in}{5.561412in}}%
\pgfpathlineto{\pgfqpoint{5.937008in}{5.546904in}}%
\pgfpathlineto{\pgfqpoint{5.952442in}{5.527895in}}%
\pgfpathlineto{\pgfqpoint{5.967875in}{5.504528in}}%
\pgfpathlineto{\pgfqpoint{5.983309in}{5.477035in}}%
\pgfpathlineto{\pgfqpoint{6.014176in}{5.410992in}}%
\pgfpathlineto{\pgfqpoint{6.045044in}{5.333184in}}%
\pgfpathlineto{\pgfqpoint{6.106778in}{5.161106in}}%
\pgfpathlineto{\pgfqpoint{6.137645in}{5.077980in}}%
\pgfpathlineto{\pgfqpoint{6.168513in}{5.004650in}}%
\pgfpathlineto{\pgfqpoint{6.183946in}{4.973429in}}%
\pgfpathlineto{\pgfqpoint{6.199380in}{4.946701in}}%
\pgfpathlineto{\pgfqpoint{6.214814in}{4.925042in}}%
\pgfpathlineto{\pgfqpoint{6.230247in}{4.908955in}}%
\pgfpathlineto{\pgfqpoint{6.245681in}{4.898856in}}%
\pgfpathlineto{\pgfqpoint{6.261115in}{4.895070in}}%
\pgfpathlineto{\pgfqpoint{6.276548in}{4.897816in}}%
\pgfpathlineto{\pgfqpoint{6.291982in}{4.907207in}}%
\pgfpathlineto{\pgfqpoint{6.307415in}{4.923242in}}%
\pgfpathlineto{\pgfqpoint{6.322849in}{4.945803in}}%
\pgfpathlineto{\pgfqpoint{6.338283in}{4.974660in}}%
\pgfpathlineto{\pgfqpoint{6.353716in}{5.009470in}}%
\pgfpathlineto{\pgfqpoint{6.369150in}{5.049782in}}%
\pgfpathlineto{\pgfqpoint{6.400017in}{5.144634in}}%
\pgfpathlineto{\pgfqpoint{6.430885in}{5.253844in}}%
\pgfpathlineto{\pgfqpoint{6.538920in}{5.654462in}}%
\pgfpathlineto{\pgfqpoint{6.569787in}{5.747538in}}%
\pgfpathlineto{\pgfqpoint{6.585221in}{5.787203in}}%
\pgfpathlineto{\pgfqpoint{6.600655in}{5.821703in}}%
\pgfpathlineto{\pgfqpoint{6.616088in}{5.850728in}}%
\pgfpathlineto{\pgfqpoint{6.631522in}{5.874073in}}%
\pgfpathlineto{\pgfqpoint{6.646955in}{5.891634in}}%
\pgfpathlineto{\pgfqpoint{6.662389in}{5.903408in}}%
\pgfpathlineto{\pgfqpoint{6.677823in}{5.909488in}}%
\pgfpathlineto{\pgfqpoint{6.693256in}{5.910049in}}%
\pgfpathlineto{\pgfqpoint{6.708690in}{5.905344in}}%
\pgfpathlineto{\pgfqpoint{6.724124in}{5.895693in}}%
\pgfpathlineto{\pgfqpoint{6.739557in}{5.881463in}}%
\pgfpathlineto{\pgfqpoint{6.754991in}{5.863063in}}%
\pgfpathlineto{\pgfqpoint{6.770424in}{5.840926in}}%
\pgfpathlineto{\pgfqpoint{6.801292in}{5.787224in}}%
\pgfpathlineto{\pgfqpoint{6.832159in}{5.723859in}}%
\pgfpathlineto{\pgfqpoint{6.878460in}{5.617541in}}%
\pgfpathlineto{\pgfqpoint{7.032796in}{5.247716in}}%
\pgfpathlineto{\pgfqpoint{7.079097in}{5.149083in}}%
\pgfpathlineto{\pgfqpoint{7.109964in}{5.090911in}}%
\pgfpathlineto{\pgfqpoint{7.140832in}{5.041171in}}%
\pgfpathlineto{\pgfqpoint{7.156265in}{5.020153in}}%
\pgfpathlineto{\pgfqpoint{7.171699in}{5.002065in}}%
\pgfpathlineto{\pgfqpoint{7.187133in}{4.987152in}}%
\pgfpathlineto{\pgfqpoint{7.202566in}{4.975631in}}%
\pgfpathlineto{\pgfqpoint{7.218000in}{4.967671in}}%
\pgfpathlineto{\pgfqpoint{7.233434in}{4.963383in}}%
\pgfpathlineto{\pgfqpoint{7.248867in}{4.962814in}}%
\pgfpathlineto{\pgfqpoint{7.264301in}{4.965927in}}%
\pgfpathlineto{\pgfqpoint{7.279734in}{4.972605in}}%
\pgfpathlineto{\pgfqpoint{7.295168in}{4.982637in}}%
\pgfpathlineto{\pgfqpoint{7.310602in}{4.995719in}}%
\pgfpathlineto{\pgfqpoint{7.341469in}{5.029364in}}%
\pgfpathlineto{\pgfqpoint{7.387770in}{5.090131in}}%
\pgfpathlineto{\pgfqpoint{7.418637in}{5.129575in}}%
\pgfpathlineto{\pgfqpoint{7.434071in}{5.146754in}}%
\pgfpathlineto{\pgfqpoint{7.449504in}{5.161265in}}%
\pgfpathlineto{\pgfqpoint{7.464938in}{5.172429in}}%
\pgfpathlineto{\pgfqpoint{7.480372in}{5.179637in}}%
\pgfpathlineto{\pgfqpoint{7.495805in}{5.182368in}}%
\pgfpathlineto{\pgfqpoint{7.511239in}{5.180211in}}%
\pgfpathlineto{\pgfqpoint{7.526673in}{5.172878in}}%
\pgfpathlineto{\pgfqpoint{7.542106in}{5.160223in}}%
\pgfpathlineto{\pgfqpoint{7.557540in}{5.142247in}}%
\pgfpathlineto{\pgfqpoint{7.572974in}{5.119105in}}%
\pgfpathlineto{\pgfqpoint{7.588407in}{5.091106in}}%
\pgfpathlineto{\pgfqpoint{7.603841in}{5.058711in}}%
\pgfpathlineto{\pgfqpoint{7.634708in}{4.983266in}}%
\pgfpathlineto{\pgfqpoint{7.727310in}{4.735295in}}%
\pgfpathlineto{\pgfqpoint{7.742744in}{4.701302in}}%
\pgfpathlineto{\pgfqpoint{7.758177in}{4.671969in}}%
\pgfpathlineto{\pgfqpoint{7.773611in}{4.648086in}}%
\pgfpathlineto{\pgfqpoint{7.789044in}{4.630312in}}%
\pgfpathlineto{\pgfqpoint{7.804478in}{4.619162in}}%
\pgfpathlineto{\pgfqpoint{7.819912in}{4.614987in}}%
\pgfpathlineto{\pgfqpoint{7.835345in}{4.617965in}}%
\pgfpathlineto{\pgfqpoint{7.850779in}{4.628096in}}%
\pgfpathlineto{\pgfqpoint{7.866213in}{4.645203in}}%
\pgfpathlineto{\pgfqpoint{7.881646in}{4.668933in}}%
\pgfpathlineto{\pgfqpoint{7.897080in}{4.698775in}}%
\pgfpathlineto{\pgfqpoint{7.912514in}{4.734065in}}%
\pgfpathlineto{\pgfqpoint{7.943381in}{4.817734in}}%
\pgfpathlineto{\pgfqpoint{7.989682in}{4.961546in}}%
\pgfpathlineto{\pgfqpoint{8.035983in}{5.102793in}}%
\pgfpathlineto{\pgfqpoint{8.066850in}{5.183043in}}%
\pgfpathlineto{\pgfqpoint{8.082284in}{5.216781in}}%
\pgfpathlineto{\pgfqpoint{8.097717in}{5.245568in}}%
\pgfpathlineto{\pgfqpoint{8.113151in}{5.269064in}}%
\pgfpathlineto{\pgfqpoint{8.128584in}{5.287072in}}%
\pgfpathlineto{\pgfqpoint{8.144018in}{5.299543in}}%
\pgfpathlineto{\pgfqpoint{8.159452in}{5.306560in}}%
\pgfpathlineto{\pgfqpoint{8.174885in}{5.308338in}}%
\pgfpathlineto{\pgfqpoint{8.190319in}{5.305197in}}%
\pgfpathlineto{\pgfqpoint{8.205753in}{5.297556in}}%
\pgfpathlineto{\pgfqpoint{8.221186in}{5.285902in}}%
\pgfpathlineto{\pgfqpoint{8.236620in}{5.270776in}}%
\pgfpathlineto{\pgfqpoint{8.267487in}{5.232385in}}%
\pgfpathlineto{\pgfqpoint{8.298354in}{5.186867in}}%
\pgfpathlineto{\pgfqpoint{8.421823in}{4.995169in}}%
\pgfpathlineto{\pgfqpoint{8.498992in}{4.878566in}}%
\pgfpathlineto{\pgfqpoint{8.529859in}{4.827108in}}%
\pgfpathlineto{\pgfqpoint{8.560726in}{4.771044in}}%
\pgfpathlineto{\pgfqpoint{8.607027in}{4.678855in}}%
\pgfpathlineto{\pgfqpoint{8.668762in}{4.551893in}}%
\pgfpathlineto{\pgfqpoint{8.699629in}{4.495170in}}%
\pgfpathlineto{\pgfqpoint{8.715063in}{4.470574in}}%
\pgfpathlineto{\pgfqpoint{8.730496in}{4.449310in}}%
\pgfpathlineto{\pgfqpoint{8.745930in}{4.431964in}}%
\pgfpathlineto{\pgfqpoint{8.761363in}{4.419073in}}%
\pgfpathlineto{\pgfqpoint{8.776797in}{4.411104in}}%
\pgfpathlineto{\pgfqpoint{8.792231in}{4.408443in}}%
\pgfpathlineto{\pgfqpoint{8.807664in}{4.411373in}}%
\pgfpathlineto{\pgfqpoint{8.823098in}{4.420064in}}%
\pgfpathlineto{\pgfqpoint{8.838532in}{4.434569in}}%
\pgfpathlineto{\pgfqpoint{8.853965in}{4.454810in}}%
\pgfpathlineto{\pgfqpoint{8.869399in}{4.480586in}}%
\pgfpathlineto{\pgfqpoint{8.884833in}{4.511569in}}%
\pgfpathlineto{\pgfqpoint{8.900266in}{4.547314in}}%
\pgfpathlineto{\pgfqpoint{8.931133in}{4.630778in}}%
\pgfpathlineto{\pgfqpoint{8.962001in}{4.725510in}}%
\pgfpathlineto{\pgfqpoint{9.039169in}{4.968894in}}%
\pgfpathlineto{\pgfqpoint{9.070036in}{5.052098in}}%
\pgfpathlineto{\pgfqpoint{9.085470in}{5.087998in}}%
\pgfpathlineto{\pgfqpoint{9.100903in}{5.119471in}}%
\pgfpathlineto{\pgfqpoint{9.116337in}{5.146184in}}%
\pgfpathlineto{\pgfqpoint{9.131771in}{5.167918in}}%
\pgfpathlineto{\pgfqpoint{9.147204in}{5.184573in}}%
\pgfpathlineto{\pgfqpoint{9.162638in}{5.196163in}}%
\pgfpathlineto{\pgfqpoint{9.178072in}{5.202811in}}%
\pgfpathlineto{\pgfqpoint{9.193505in}{5.204743in}}%
\pgfpathlineto{\pgfqpoint{9.208939in}{5.202275in}}%
\pgfpathlineto{\pgfqpoint{9.224373in}{5.195803in}}%
\pgfpathlineto{\pgfqpoint{9.239806in}{5.185790in}}%
\pgfpathlineto{\pgfqpoint{9.255240in}{5.172747in}}%
\pgfpathlineto{\pgfqpoint{9.286107in}{5.139782in}}%
\pgfpathlineto{\pgfqpoint{9.332408in}{5.081631in}}%
\pgfpathlineto{\pgfqpoint{9.363275in}{5.043282in}}%
\pgfpathlineto{\pgfqpoint{9.394143in}{5.009441in}}%
\pgfpathlineto{\pgfqpoint{9.425010in}{4.982703in}}%
\pgfpathlineto{\pgfqpoint{9.440443in}{4.972543in}}%
\pgfpathlineto{\pgfqpoint{9.455877in}{4.964691in}}%
\pgfpathlineto{\pgfqpoint{9.471311in}{4.959204in}}%
\pgfpathlineto{\pgfqpoint{9.486744in}{4.956084in}}%
\pgfpathlineto{\pgfqpoint{9.502178in}{4.955288in}}%
\pgfpathlineto{\pgfqpoint{9.517612in}{4.956735in}}%
\pgfpathlineto{\pgfqpoint{9.533045in}{4.960306in}}%
\pgfpathlineto{\pgfqpoint{9.548479in}{4.965857in}}%
\pgfpathlineto{\pgfqpoint{9.579346in}{4.982217in}}%
\pgfpathlineto{\pgfqpoint{9.610213in}{5.004333in}}%
\pgfpathlineto{\pgfqpoint{9.641081in}{5.030626in}}%
\pgfpathlineto{\pgfqpoint{9.702815in}{5.089594in}}%
\pgfpathlineto{\pgfqpoint{9.733682in}{5.119424in}}%
\pgfpathlineto{\pgfqpoint{9.733682in}{5.119424in}}%
\pgfusepath{stroke}%
\end{pgfscope}%
\begin{pgfscope}%
\pgfpathrectangle{\pgfqpoint{5.698559in}{3.799602in}}{\pgfqpoint{4.227273in}{2.745455in}} %
\pgfusepath{clip}%
\pgfsetrectcap%
\pgfsetroundjoin%
\pgfsetlinewidth{0.501875pt}%
\definecolor{currentstroke}{rgb}{0.300000,0.951057,0.809017}%
\pgfsetstrokecolor{currentstroke}%
\pgfsetdash{}{0pt}%
\pgfpathmoveto{\pgfqpoint{5.890707in}{5.537929in}}%
\pgfpathlineto{\pgfqpoint{5.906141in}{5.538775in}}%
\pgfpathlineto{\pgfqpoint{5.921575in}{5.535782in}}%
\pgfpathlineto{\pgfqpoint{5.937008in}{5.528819in}}%
\pgfpathlineto{\pgfqpoint{5.952442in}{5.517827in}}%
\pgfpathlineto{\pgfqpoint{5.967875in}{5.502813in}}%
\pgfpathlineto{\pgfqpoint{5.983309in}{5.483861in}}%
\pgfpathlineto{\pgfqpoint{5.998743in}{5.461129in}}%
\pgfpathlineto{\pgfqpoint{6.014176in}{5.434851in}}%
\pgfpathlineto{\pgfqpoint{6.045044in}{5.372953in}}%
\pgfpathlineto{\pgfqpoint{6.075911in}{5.301451in}}%
\pgfpathlineto{\pgfqpoint{6.168513in}{5.075103in}}%
\pgfpathlineto{\pgfqpoint{6.199380in}{5.013251in}}%
\pgfpathlineto{\pgfqpoint{6.214814in}{4.987819in}}%
\pgfpathlineto{\pgfqpoint{6.230247in}{4.966818in}}%
\pgfpathlineto{\pgfqpoint{6.245681in}{4.950750in}}%
\pgfpathlineto{\pgfqpoint{6.261115in}{4.940045in}}%
\pgfpathlineto{\pgfqpoint{6.276548in}{4.935051in}}%
\pgfpathlineto{\pgfqpoint{6.291982in}{4.936026in}}%
\pgfpathlineto{\pgfqpoint{6.307415in}{4.943128in}}%
\pgfpathlineto{\pgfqpoint{6.322849in}{4.956415in}}%
\pgfpathlineto{\pgfqpoint{6.338283in}{4.975839in}}%
\pgfpathlineto{\pgfqpoint{6.353716in}{5.001246in}}%
\pgfpathlineto{\pgfqpoint{6.369150in}{5.032377in}}%
\pgfpathlineto{\pgfqpoint{6.384584in}{5.068873in}}%
\pgfpathlineto{\pgfqpoint{6.415451in}{5.156059in}}%
\pgfpathlineto{\pgfqpoint{6.446318in}{5.258196in}}%
\pgfpathlineto{\pgfqpoint{6.492619in}{5.426927in}}%
\pgfpathlineto{\pgfqpoint{6.538920in}{5.595435in}}%
\pgfpathlineto{\pgfqpoint{6.569787in}{5.697428in}}%
\pgfpathlineto{\pgfqpoint{6.600655in}{5.784954in}}%
\pgfpathlineto{\pgfqpoint{6.616088in}{5.821969in}}%
\pgfpathlineto{\pgfqpoint{6.631522in}{5.853946in}}%
\pgfpathlineto{\pgfqpoint{6.646955in}{5.880590in}}%
\pgfpathlineto{\pgfqpoint{6.662389in}{5.901697in}}%
\pgfpathlineto{\pgfqpoint{6.677823in}{5.917156in}}%
\pgfpathlineto{\pgfqpoint{6.693256in}{5.926942in}}%
\pgfpathlineto{\pgfqpoint{6.708690in}{5.931117in}}%
\pgfpathlineto{\pgfqpoint{6.724124in}{5.929819in}}%
\pgfpathlineto{\pgfqpoint{6.739557in}{5.923254in}}%
\pgfpathlineto{\pgfqpoint{6.754991in}{5.911694in}}%
\pgfpathlineto{\pgfqpoint{6.770424in}{5.895461in}}%
\pgfpathlineto{\pgfqpoint{6.785858in}{5.874918in}}%
\pgfpathlineto{\pgfqpoint{6.801292in}{5.850462in}}%
\pgfpathlineto{\pgfqpoint{6.832159in}{5.791504in}}%
\pgfpathlineto{\pgfqpoint{6.863026in}{5.722047in}}%
\pgfpathlineto{\pgfqpoint{6.909327in}{5.605611in}}%
\pgfpathlineto{\pgfqpoint{7.001929in}{5.365171in}}%
\pgfpathlineto{\pgfqpoint{7.032796in}{5.291590in}}%
\pgfpathlineto{\pgfqpoint{7.063664in}{5.224496in}}%
\pgfpathlineto{\pgfqpoint{7.094531in}{5.165485in}}%
\pgfpathlineto{\pgfqpoint{7.125398in}{5.115953in}}%
\pgfpathlineto{\pgfqpoint{7.140832in}{5.095116in}}%
\pgfpathlineto{\pgfqpoint{7.156265in}{5.077059in}}%
\pgfpathlineto{\pgfqpoint{7.171699in}{5.061873in}}%
\pgfpathlineto{\pgfqpoint{7.187133in}{5.049618in}}%
\pgfpathlineto{\pgfqpoint{7.202566in}{5.040320in}}%
\pgfpathlineto{\pgfqpoint{7.218000in}{5.033965in}}%
\pgfpathlineto{\pgfqpoint{7.233434in}{5.030495in}}%
\pgfpathlineto{\pgfqpoint{7.248867in}{5.029804in}}%
\pgfpathlineto{\pgfqpoint{7.264301in}{5.031735in}}%
\pgfpathlineto{\pgfqpoint{7.279734in}{5.036077in}}%
\pgfpathlineto{\pgfqpoint{7.295168in}{5.042569in}}%
\pgfpathlineto{\pgfqpoint{7.326035in}{5.060691in}}%
\pgfpathlineto{\pgfqpoint{7.418637in}{5.125491in}}%
\pgfpathlineto{\pgfqpoint{7.434071in}{5.132797in}}%
\pgfpathlineto{\pgfqpoint{7.449504in}{5.137860in}}%
\pgfpathlineto{\pgfqpoint{7.464938in}{5.140289in}}%
\pgfpathlineto{\pgfqpoint{7.480372in}{5.139751in}}%
\pgfpathlineto{\pgfqpoint{7.495805in}{5.135985in}}%
\pgfpathlineto{\pgfqpoint{7.511239in}{5.128814in}}%
\pgfpathlineto{\pgfqpoint{7.526673in}{5.118150in}}%
\pgfpathlineto{\pgfqpoint{7.542106in}{5.104000in}}%
\pgfpathlineto{\pgfqpoint{7.557540in}{5.086475in}}%
\pgfpathlineto{\pgfqpoint{7.572974in}{5.065784in}}%
\pgfpathlineto{\pgfqpoint{7.603841in}{5.016235in}}%
\pgfpathlineto{\pgfqpoint{7.634708in}{4.958890in}}%
\pgfpathlineto{\pgfqpoint{7.696443in}{4.840313in}}%
\pgfpathlineto{\pgfqpoint{7.727310in}{4.789997in}}%
\pgfpathlineto{\pgfqpoint{7.742744in}{4.769377in}}%
\pgfpathlineto{\pgfqpoint{7.758177in}{4.752526in}}%
\pgfpathlineto{\pgfqpoint{7.773611in}{4.739898in}}%
\pgfpathlineto{\pgfqpoint{7.789044in}{4.731850in}}%
\pgfpathlineto{\pgfqpoint{7.804478in}{4.728629in}}%
\pgfpathlineto{\pgfqpoint{7.819912in}{4.730366in}}%
\pgfpathlineto{\pgfqpoint{7.835345in}{4.737068in}}%
\pgfpathlineto{\pgfqpoint{7.850779in}{4.748621in}}%
\pgfpathlineto{\pgfqpoint{7.866213in}{4.764789in}}%
\pgfpathlineto{\pgfqpoint{7.881646in}{4.785221in}}%
\pgfpathlineto{\pgfqpoint{7.897080in}{4.809463in}}%
\pgfpathlineto{\pgfqpoint{7.927947in}{4.867106in}}%
\pgfpathlineto{\pgfqpoint{7.974248in}{4.966321in}}%
\pgfpathlineto{\pgfqpoint{8.020549in}{5.063289in}}%
\pgfpathlineto{\pgfqpoint{8.051416in}{5.117591in}}%
\pgfpathlineto{\pgfqpoint{8.066850in}{5.139992in}}%
\pgfpathlineto{\pgfqpoint{8.082284in}{5.158699in}}%
\pgfpathlineto{\pgfqpoint{8.097717in}{5.173444in}}%
\pgfpathlineto{\pgfqpoint{8.113151in}{5.184066in}}%
\pgfpathlineto{\pgfqpoint{8.128584in}{5.190511in}}%
\pgfpathlineto{\pgfqpoint{8.144018in}{5.192826in}}%
\pgfpathlineto{\pgfqpoint{8.159452in}{5.191159in}}%
\pgfpathlineto{\pgfqpoint{8.174885in}{5.185742in}}%
\pgfpathlineto{\pgfqpoint{8.190319in}{5.176883in}}%
\pgfpathlineto{\pgfqpoint{8.205753in}{5.164953in}}%
\pgfpathlineto{\pgfqpoint{8.221186in}{5.150368in}}%
\pgfpathlineto{\pgfqpoint{8.252053in}{5.115020in}}%
\pgfpathlineto{\pgfqpoint{8.298354in}{5.053270in}}%
\pgfpathlineto{\pgfqpoint{8.360089in}{4.970134in}}%
\pgfpathlineto{\pgfqpoint{8.406390in}{4.914239in}}%
\pgfpathlineto{\pgfqpoint{8.452691in}{4.863851in}}%
\pgfpathlineto{\pgfqpoint{8.529859in}{4.781136in}}%
\pgfpathlineto{\pgfqpoint{8.576160in}{4.726029in}}%
\pgfpathlineto{\pgfqpoint{8.684195in}{4.591926in}}%
\pgfpathlineto{\pgfqpoint{8.715063in}{4.562913in}}%
\pgfpathlineto{\pgfqpoint{8.730496in}{4.552022in}}%
\pgfpathlineto{\pgfqpoint{8.745930in}{4.544099in}}%
\pgfpathlineto{\pgfqpoint{8.761363in}{4.539520in}}%
\pgfpathlineto{\pgfqpoint{8.776797in}{4.538611in}}%
\pgfpathlineto{\pgfqpoint{8.792231in}{4.541636in}}%
\pgfpathlineto{\pgfqpoint{8.807664in}{4.548784in}}%
\pgfpathlineto{\pgfqpoint{8.823098in}{4.560164in}}%
\pgfpathlineto{\pgfqpoint{8.838532in}{4.575793in}}%
\pgfpathlineto{\pgfqpoint{8.853965in}{4.595598in}}%
\pgfpathlineto{\pgfqpoint{8.869399in}{4.619412in}}%
\pgfpathlineto{\pgfqpoint{8.884833in}{4.646975in}}%
\pgfpathlineto{\pgfqpoint{8.915700in}{4.711890in}}%
\pgfpathlineto{\pgfqpoint{8.946567in}{4.786685in}}%
\pgfpathlineto{\pgfqpoint{9.039169in}{5.022901in}}%
\pgfpathlineto{\pgfqpoint{9.070036in}{5.089440in}}%
\pgfpathlineto{\pgfqpoint{9.085470in}{5.118126in}}%
\pgfpathlineto{\pgfqpoint{9.100903in}{5.143291in}}%
\pgfpathlineto{\pgfqpoint{9.116337in}{5.164693in}}%
\pgfpathlineto{\pgfqpoint{9.131771in}{5.182176in}}%
\pgfpathlineto{\pgfqpoint{9.147204in}{5.195675in}}%
\pgfpathlineto{\pgfqpoint{9.162638in}{5.205214in}}%
\pgfpathlineto{\pgfqpoint{9.178072in}{5.210902in}}%
\pgfpathlineto{\pgfqpoint{9.193505in}{5.212928in}}%
\pgfpathlineto{\pgfqpoint{9.208939in}{5.211551in}}%
\pgfpathlineto{\pgfqpoint{9.224373in}{5.207094in}}%
\pgfpathlineto{\pgfqpoint{9.239806in}{5.199932in}}%
\pgfpathlineto{\pgfqpoint{9.255240in}{5.190480in}}%
\pgfpathlineto{\pgfqpoint{9.286107in}{5.166495in}}%
\pgfpathlineto{\pgfqpoint{9.378709in}{5.086188in}}%
\pgfpathlineto{\pgfqpoint{9.409576in}{5.066863in}}%
\pgfpathlineto{\pgfqpoint{9.425010in}{5.059812in}}%
\pgfpathlineto{\pgfqpoint{9.440443in}{5.054691in}}%
\pgfpathlineto{\pgfqpoint{9.455877in}{5.051577in}}%
\pgfpathlineto{\pgfqpoint{9.471311in}{5.050495in}}%
\pgfpathlineto{\pgfqpoint{9.486744in}{5.051422in}}%
\pgfpathlineto{\pgfqpoint{9.502178in}{5.054293in}}%
\pgfpathlineto{\pgfqpoint{9.517612in}{5.059006in}}%
\pgfpathlineto{\pgfqpoint{9.548479in}{5.073381in}}%
\pgfpathlineto{\pgfqpoint{9.579346in}{5.093171in}}%
\pgfpathlineto{\pgfqpoint{9.610213in}{5.116747in}}%
\pgfpathlineto{\pgfqpoint{9.733682in}{5.216417in}}%
\pgfpathlineto{\pgfqpoint{9.733682in}{5.216417in}}%
\pgfusepath{stroke}%
\end{pgfscope}%
\begin{pgfscope}%
\pgfpathrectangle{\pgfqpoint{5.698559in}{3.799602in}}{\pgfqpoint{4.227273in}{2.745455in}} %
\pgfusepath{clip}%
\pgfsetrectcap%
\pgfsetroundjoin%
\pgfsetlinewidth{0.501875pt}%
\definecolor{currentstroke}{rgb}{0.378431,0.981823,0.771298}%
\pgfsetstrokecolor{currentstroke}%
\pgfsetdash{}{0pt}%
\pgfpathmoveto{\pgfqpoint{5.890707in}{5.473543in}}%
\pgfpathlineto{\pgfqpoint{5.906141in}{5.481402in}}%
\pgfpathlineto{\pgfqpoint{5.921575in}{5.485709in}}%
\pgfpathlineto{\pgfqpoint{5.937008in}{5.486286in}}%
\pgfpathlineto{\pgfqpoint{5.952442in}{5.483016in}}%
\pgfpathlineto{\pgfqpoint{5.967875in}{5.475852in}}%
\pgfpathlineto{\pgfqpoint{5.983309in}{5.464818in}}%
\pgfpathlineto{\pgfqpoint{5.998743in}{5.450010in}}%
\pgfpathlineto{\pgfqpoint{6.014176in}{5.431600in}}%
\pgfpathlineto{\pgfqpoint{6.029610in}{5.409827in}}%
\pgfpathlineto{\pgfqpoint{6.060477in}{5.357512in}}%
\pgfpathlineto{\pgfqpoint{6.091345in}{5.296325in}}%
\pgfpathlineto{\pgfqpoint{6.183946in}{5.103566in}}%
\pgfpathlineto{\pgfqpoint{6.214814in}{5.052678in}}%
\pgfpathlineto{\pgfqpoint{6.230247in}{5.032431in}}%
\pgfpathlineto{\pgfqpoint{6.245681in}{5.016331in}}%
\pgfpathlineto{\pgfqpoint{6.261115in}{5.004817in}}%
\pgfpathlineto{\pgfqpoint{6.276548in}{4.998258in}}%
\pgfpathlineto{\pgfqpoint{6.291982in}{4.996945in}}%
\pgfpathlineto{\pgfqpoint{6.307415in}{5.001084in}}%
\pgfpathlineto{\pgfqpoint{6.322849in}{5.010790in}}%
\pgfpathlineto{\pgfqpoint{6.338283in}{5.026087in}}%
\pgfpathlineto{\pgfqpoint{6.353716in}{5.046901in}}%
\pgfpathlineto{\pgfqpoint{6.369150in}{5.073063in}}%
\pgfpathlineto{\pgfqpoint{6.384584in}{5.104315in}}%
\pgfpathlineto{\pgfqpoint{6.400017in}{5.140307in}}%
\pgfpathlineto{\pgfqpoint{6.430885in}{5.224707in}}%
\pgfpathlineto{\pgfqpoint{6.461752in}{5.321966in}}%
\pgfpathlineto{\pgfqpoint{6.569787in}{5.685548in}}%
\pgfpathlineto{\pgfqpoint{6.600655in}{5.773147in}}%
\pgfpathlineto{\pgfqpoint{6.616088in}{5.811195in}}%
\pgfpathlineto{\pgfqpoint{6.631522in}{5.844803in}}%
\pgfpathlineto{\pgfqpoint{6.646955in}{5.873608in}}%
\pgfpathlineto{\pgfqpoint{6.662389in}{5.897326in}}%
\pgfpathlineto{\pgfqpoint{6.677823in}{5.915755in}}%
\pgfpathlineto{\pgfqpoint{6.693256in}{5.928770in}}%
\pgfpathlineto{\pgfqpoint{6.708690in}{5.936328in}}%
\pgfpathlineto{\pgfqpoint{6.724124in}{5.938462in}}%
\pgfpathlineto{\pgfqpoint{6.739557in}{5.935275in}}%
\pgfpathlineto{\pgfqpoint{6.754991in}{5.926938in}}%
\pgfpathlineto{\pgfqpoint{6.770424in}{5.913683in}}%
\pgfpathlineto{\pgfqpoint{6.785858in}{5.895794in}}%
\pgfpathlineto{\pgfqpoint{6.801292in}{5.873601in}}%
\pgfpathlineto{\pgfqpoint{6.816725in}{5.847471in}}%
\pgfpathlineto{\pgfqpoint{6.847593in}{5.785019in}}%
\pgfpathlineto{\pgfqpoint{6.878460in}{5.711846in}}%
\pgfpathlineto{\pgfqpoint{6.924761in}{5.589735in}}%
\pgfpathlineto{\pgfqpoint{7.001929in}{5.381357in}}%
\pgfpathlineto{\pgfqpoint{7.032796in}{5.304960in}}%
\pgfpathlineto{\pgfqpoint{7.063664in}{5.236184in}}%
\pgfpathlineto{\pgfqpoint{7.094531in}{5.176900in}}%
\pgfpathlineto{\pgfqpoint{7.125398in}{5.128516in}}%
\pgfpathlineto{\pgfqpoint{7.140832in}{5.108714in}}%
\pgfpathlineto{\pgfqpoint{7.156265in}{5.091932in}}%
\pgfpathlineto{\pgfqpoint{7.171699in}{5.078194in}}%
\pgfpathlineto{\pgfqpoint{7.187133in}{5.067481in}}%
\pgfpathlineto{\pgfqpoint{7.202566in}{5.059736in}}%
\pgfpathlineto{\pgfqpoint{7.218000in}{5.054858in}}%
\pgfpathlineto{\pgfqpoint{7.233434in}{5.052700in}}%
\pgfpathlineto{\pgfqpoint{7.248867in}{5.053073in}}%
\pgfpathlineto{\pgfqpoint{7.264301in}{5.055742in}}%
\pgfpathlineto{\pgfqpoint{7.279734in}{5.060432in}}%
\pgfpathlineto{\pgfqpoint{7.310602in}{5.074570in}}%
\pgfpathlineto{\pgfqpoint{7.403204in}{5.126534in}}%
\pgfpathlineto{\pgfqpoint{7.418637in}{5.132108in}}%
\pgfpathlineto{\pgfqpoint{7.434071in}{5.135714in}}%
\pgfpathlineto{\pgfqpoint{7.449504in}{5.137012in}}%
\pgfpathlineto{\pgfqpoint{7.464938in}{5.135713in}}%
\pgfpathlineto{\pgfqpoint{7.480372in}{5.131585in}}%
\pgfpathlineto{\pgfqpoint{7.495805in}{5.124468in}}%
\pgfpathlineto{\pgfqpoint{7.511239in}{5.114272in}}%
\pgfpathlineto{\pgfqpoint{7.526673in}{5.100988in}}%
\pgfpathlineto{\pgfqpoint{7.542106in}{5.084692in}}%
\pgfpathlineto{\pgfqpoint{7.557540in}{5.065541in}}%
\pgfpathlineto{\pgfqpoint{7.588407in}{5.019712in}}%
\pgfpathlineto{\pgfqpoint{7.619274in}{4.966335in}}%
\pgfpathlineto{\pgfqpoint{7.696443in}{4.826789in}}%
\pgfpathlineto{\pgfqpoint{7.727310in}{4.780791in}}%
\pgfpathlineto{\pgfqpoint{7.742744in}{4.762138in}}%
\pgfpathlineto{\pgfqpoint{7.758177in}{4.747001in}}%
\pgfpathlineto{\pgfqpoint{7.773611in}{4.735755in}}%
\pgfpathlineto{\pgfqpoint{7.789044in}{4.728692in}}%
\pgfpathlineto{\pgfqpoint{7.804478in}{4.726011in}}%
\pgfpathlineto{\pgfqpoint{7.819912in}{4.727812in}}%
\pgfpathlineto{\pgfqpoint{7.835345in}{4.734094in}}%
\pgfpathlineto{\pgfqpoint{7.850779in}{4.744752in}}%
\pgfpathlineto{\pgfqpoint{7.866213in}{4.759578in}}%
\pgfpathlineto{\pgfqpoint{7.881646in}{4.778271in}}%
\pgfpathlineto{\pgfqpoint{7.897080in}{4.800440in}}%
\pgfpathlineto{\pgfqpoint{7.927947in}{4.853253in}}%
\pgfpathlineto{\pgfqpoint{7.974248in}{4.944895in}}%
\pgfpathlineto{\pgfqpoint{8.020549in}{5.036062in}}%
\pgfpathlineto{\pgfqpoint{8.051416in}{5.088366in}}%
\pgfpathlineto{\pgfqpoint{8.066850in}{5.110418in}}%
\pgfpathlineto{\pgfqpoint{8.082284in}{5.129203in}}%
\pgfpathlineto{\pgfqpoint{8.097717in}{5.144420in}}%
\pgfpathlineto{\pgfqpoint{8.113151in}{5.155856in}}%
\pgfpathlineto{\pgfqpoint{8.128584in}{5.163387in}}%
\pgfpathlineto{\pgfqpoint{8.144018in}{5.166976in}}%
\pgfpathlineto{\pgfqpoint{8.159452in}{5.166673in}}%
\pgfpathlineto{\pgfqpoint{8.174885in}{5.162604in}}%
\pgfpathlineto{\pgfqpoint{8.190319in}{5.154971in}}%
\pgfpathlineto{\pgfqpoint{8.205753in}{5.144035in}}%
\pgfpathlineto{\pgfqpoint{8.221186in}{5.130110in}}%
\pgfpathlineto{\pgfqpoint{8.252053in}{5.094744in}}%
\pgfpathlineto{\pgfqpoint{8.282921in}{5.051972in}}%
\pgfpathlineto{\pgfqpoint{8.344655in}{4.956607in}}%
\pgfpathlineto{\pgfqpoint{8.390956in}{4.886540in}}%
\pgfpathlineto{\pgfqpoint{8.421823in}{4.843681in}}%
\pgfpathlineto{\pgfqpoint{8.452691in}{4.804786in}}%
\pgfpathlineto{\pgfqpoint{8.483558in}{4.769995in}}%
\pgfpathlineto{\pgfqpoint{8.514425in}{4.739125in}}%
\pgfpathlineto{\pgfqpoint{8.545293in}{4.711894in}}%
\pgfpathlineto{\pgfqpoint{8.576160in}{4.688146in}}%
\pgfpathlineto{\pgfqpoint{8.607027in}{4.668005in}}%
\pgfpathlineto{\pgfqpoint{8.637894in}{4.651967in}}%
\pgfpathlineto{\pgfqpoint{8.668762in}{4.640882in}}%
\pgfpathlineto{\pgfqpoint{8.684195in}{4.637537in}}%
\pgfpathlineto{\pgfqpoint{8.699629in}{4.635853in}}%
\pgfpathlineto{\pgfqpoint{8.715063in}{4.635980in}}%
\pgfpathlineto{\pgfqpoint{8.730496in}{4.638056in}}%
\pgfpathlineto{\pgfqpoint{8.745930in}{4.642204in}}%
\pgfpathlineto{\pgfqpoint{8.761363in}{4.648520in}}%
\pgfpathlineto{\pgfqpoint{8.776797in}{4.657074in}}%
\pgfpathlineto{\pgfqpoint{8.792231in}{4.667895in}}%
\pgfpathlineto{\pgfqpoint{8.807664in}{4.680975in}}%
\pgfpathlineto{\pgfqpoint{8.838532in}{4.713639in}}%
\pgfpathlineto{\pgfqpoint{8.869399in}{4.754062in}}%
\pgfpathlineto{\pgfqpoint{8.900266in}{4.800509in}}%
\pgfpathlineto{\pgfqpoint{9.008302in}{4.973540in}}%
\pgfpathlineto{\pgfqpoint{9.039169in}{5.014479in}}%
\pgfpathlineto{\pgfqpoint{9.070036in}{5.047402in}}%
\pgfpathlineto{\pgfqpoint{9.085470in}{5.060442in}}%
\pgfpathlineto{\pgfqpoint{9.100903in}{5.071090in}}%
\pgfpathlineto{\pgfqpoint{9.116337in}{5.079336in}}%
\pgfpathlineto{\pgfqpoint{9.131771in}{5.085230in}}%
\pgfpathlineto{\pgfqpoint{9.147204in}{5.088878in}}%
\pgfpathlineto{\pgfqpoint{9.162638in}{5.090442in}}%
\pgfpathlineto{\pgfqpoint{9.178072in}{5.090129in}}%
\pgfpathlineto{\pgfqpoint{9.208939in}{5.084918in}}%
\pgfpathlineto{\pgfqpoint{9.239806in}{5.075629in}}%
\pgfpathlineto{\pgfqpoint{9.301541in}{5.055522in}}%
\pgfpathlineto{\pgfqpoint{9.332408in}{5.049782in}}%
\pgfpathlineto{\pgfqpoint{9.347842in}{5.048909in}}%
\pgfpathlineto{\pgfqpoint{9.363275in}{5.049616in}}%
\pgfpathlineto{\pgfqpoint{9.378709in}{5.052036in}}%
\pgfpathlineto{\pgfqpoint{9.394143in}{5.056250in}}%
\pgfpathlineto{\pgfqpoint{9.409576in}{5.062294in}}%
\pgfpathlineto{\pgfqpoint{9.425010in}{5.070155in}}%
\pgfpathlineto{\pgfqpoint{9.455877in}{5.091045in}}%
\pgfpathlineto{\pgfqpoint{9.486744in}{5.117953in}}%
\pgfpathlineto{\pgfqpoint{9.517612in}{5.149366in}}%
\pgfpathlineto{\pgfqpoint{9.625647in}{5.266634in}}%
\pgfpathlineto{\pgfqpoint{9.656514in}{5.294265in}}%
\pgfpathlineto{\pgfqpoint{9.687382in}{5.316483in}}%
\pgfpathlineto{\pgfqpoint{9.718249in}{5.332408in}}%
\pgfpathlineto{\pgfqpoint{9.733682in}{5.337864in}}%
\pgfpathlineto{\pgfqpoint{9.733682in}{5.337864in}}%
\pgfusepath{stroke}%
\end{pgfscope}%
\begin{pgfscope}%
\pgfpathrectangle{\pgfqpoint{5.698559in}{3.799602in}}{\pgfqpoint{4.227273in}{2.745455in}} %
\pgfusepath{clip}%
\pgfsetrectcap%
\pgfsetroundjoin%
\pgfsetlinewidth{0.501875pt}%
\definecolor{currentstroke}{rgb}{0.456863,0.997705,0.730653}%
\pgfsetstrokecolor{currentstroke}%
\pgfsetdash{}{0pt}%
\pgfpathmoveto{\pgfqpoint{5.890707in}{5.481827in}}%
\pgfpathlineto{\pgfqpoint{5.906141in}{5.484577in}}%
\pgfpathlineto{\pgfqpoint{5.921575in}{5.483904in}}%
\pgfpathlineto{\pgfqpoint{5.937008in}{5.479686in}}%
\pgfpathlineto{\pgfqpoint{5.952442in}{5.471860in}}%
\pgfpathlineto{\pgfqpoint{5.967875in}{5.460424in}}%
\pgfpathlineto{\pgfqpoint{5.983309in}{5.445437in}}%
\pgfpathlineto{\pgfqpoint{5.998743in}{5.427023in}}%
\pgfpathlineto{\pgfqpoint{6.014176in}{5.405370in}}%
\pgfpathlineto{\pgfqpoint{6.045044in}{5.353413in}}%
\pgfpathlineto{\pgfqpoint{6.075911in}{5.292284in}}%
\pgfpathlineto{\pgfqpoint{6.183946in}{5.062489in}}%
\pgfpathlineto{\pgfqpoint{6.214814in}{5.010888in}}%
\pgfpathlineto{\pgfqpoint{6.230247in}{4.990199in}}%
\pgfpathlineto{\pgfqpoint{6.245681in}{4.973550in}}%
\pgfpathlineto{\pgfqpoint{6.261115in}{4.961347in}}%
\pgfpathlineto{\pgfqpoint{6.276548in}{4.953938in}}%
\pgfpathlineto{\pgfqpoint{6.291982in}{4.951598in}}%
\pgfpathlineto{\pgfqpoint{6.307415in}{4.954528in}}%
\pgfpathlineto{\pgfqpoint{6.322849in}{4.962847in}}%
\pgfpathlineto{\pgfqpoint{6.338283in}{4.976591in}}%
\pgfpathlineto{\pgfqpoint{6.353716in}{4.995711in}}%
\pgfpathlineto{\pgfqpoint{6.369150in}{5.020073in}}%
\pgfpathlineto{\pgfqpoint{6.384584in}{5.049456in}}%
\pgfpathlineto{\pgfqpoint{6.400017in}{5.083561in}}%
\pgfpathlineto{\pgfqpoint{6.430885in}{5.164359in}}%
\pgfpathlineto{\pgfqpoint{6.461752in}{5.258658in}}%
\pgfpathlineto{\pgfqpoint{6.508053in}{5.415052in}}%
\pgfpathlineto{\pgfqpoint{6.569787in}{5.623727in}}%
\pgfpathlineto{\pgfqpoint{6.600655in}{5.716433in}}%
\pgfpathlineto{\pgfqpoint{6.631522in}{5.795370in}}%
\pgfpathlineto{\pgfqpoint{6.646955in}{5.828557in}}%
\pgfpathlineto{\pgfqpoint{6.662389in}{5.857090in}}%
\pgfpathlineto{\pgfqpoint{6.677823in}{5.880705in}}%
\pgfpathlineto{\pgfqpoint{6.693256in}{5.899211in}}%
\pgfpathlineto{\pgfqpoint{6.708690in}{5.912490in}}%
\pgfpathlineto{\pgfqpoint{6.724124in}{5.920495in}}%
\pgfpathlineto{\pgfqpoint{6.739557in}{5.923247in}}%
\pgfpathlineto{\pgfqpoint{6.754991in}{5.920834in}}%
\pgfpathlineto{\pgfqpoint{6.770424in}{5.913405in}}%
\pgfpathlineto{\pgfqpoint{6.785858in}{5.901165in}}%
\pgfpathlineto{\pgfqpoint{6.801292in}{5.884370in}}%
\pgfpathlineto{\pgfqpoint{6.816725in}{5.863319in}}%
\pgfpathlineto{\pgfqpoint{6.832159in}{5.838349in}}%
\pgfpathlineto{\pgfqpoint{6.863026in}{5.778153in}}%
\pgfpathlineto{\pgfqpoint{6.893894in}{5.706983in}}%
\pgfpathlineto{\pgfqpoint{6.940194in}{5.587080in}}%
\pgfpathlineto{\pgfqpoint{7.032796in}{5.340491in}}%
\pgfpathlineto{\pgfqpoint{7.063664in}{5.266768in}}%
\pgfpathlineto{\pgfqpoint{7.094531in}{5.201001in}}%
\pgfpathlineto{\pgfqpoint{7.125398in}{5.144705in}}%
\pgfpathlineto{\pgfqpoint{7.156265in}{5.098850in}}%
\pgfpathlineto{\pgfqpoint{7.171699in}{5.079984in}}%
\pgfpathlineto{\pgfqpoint{7.187133in}{5.063829in}}%
\pgfpathlineto{\pgfqpoint{7.202566in}{5.050340in}}%
\pgfpathlineto{\pgfqpoint{7.218000in}{5.039434in}}%
\pgfpathlineto{\pgfqpoint{7.233434in}{5.030989in}}%
\pgfpathlineto{\pgfqpoint{7.248867in}{5.024848in}}%
\pgfpathlineto{\pgfqpoint{7.264301in}{5.020820in}}%
\pgfpathlineto{\pgfqpoint{7.279734in}{5.018679in}}%
\pgfpathlineto{\pgfqpoint{7.310602in}{5.019018in}}%
\pgfpathlineto{\pgfqpoint{7.341469in}{5.023555in}}%
\pgfpathlineto{\pgfqpoint{7.403204in}{5.034965in}}%
\pgfpathlineto{\pgfqpoint{7.434071in}{5.036826in}}%
\pgfpathlineto{\pgfqpoint{7.464938in}{5.033324in}}%
\pgfpathlineto{\pgfqpoint{7.480372in}{5.029108in}}%
\pgfpathlineto{\pgfqpoint{7.495805in}{5.023101in}}%
\pgfpathlineto{\pgfqpoint{7.511239in}{5.015263in}}%
\pgfpathlineto{\pgfqpoint{7.542106in}{4.994217in}}%
\pgfpathlineto{\pgfqpoint{7.572974in}{4.966808in}}%
\pgfpathlineto{\pgfqpoint{7.603841in}{4.934777in}}%
\pgfpathlineto{\pgfqpoint{7.665575in}{4.867416in}}%
\pgfpathlineto{\pgfqpoint{7.696443in}{4.838419in}}%
\pgfpathlineto{\pgfqpoint{7.711876in}{4.826509in}}%
\pgfpathlineto{\pgfqpoint{7.727310in}{4.816817in}}%
\pgfpathlineto{\pgfqpoint{7.742744in}{4.809665in}}%
\pgfpathlineto{\pgfqpoint{7.758177in}{4.805326in}}%
\pgfpathlineto{\pgfqpoint{7.773611in}{4.804017in}}%
\pgfpathlineto{\pgfqpoint{7.789044in}{4.805888in}}%
\pgfpathlineto{\pgfqpoint{7.804478in}{4.811019in}}%
\pgfpathlineto{\pgfqpoint{7.819912in}{4.819417in}}%
\pgfpathlineto{\pgfqpoint{7.835345in}{4.831012in}}%
\pgfpathlineto{\pgfqpoint{7.850779in}{4.845661in}}%
\pgfpathlineto{\pgfqpoint{7.866213in}{4.863146in}}%
\pgfpathlineto{\pgfqpoint{7.897080in}{4.905417in}}%
\pgfpathlineto{\pgfqpoint{7.927947in}{4.954838in}}%
\pgfpathlineto{\pgfqpoint{8.005115in}{5.084589in}}%
\pgfpathlineto{\pgfqpoint{8.035983in}{5.128459in}}%
\pgfpathlineto{\pgfqpoint{8.051416in}{5.146871in}}%
\pgfpathlineto{\pgfqpoint{8.066850in}{5.162463in}}%
\pgfpathlineto{\pgfqpoint{8.082284in}{5.174956in}}%
\pgfpathlineto{\pgfqpoint{8.097717in}{5.184139in}}%
\pgfpathlineto{\pgfqpoint{8.113151in}{5.189868in}}%
\pgfpathlineto{\pgfqpoint{8.128584in}{5.192070in}}%
\pgfpathlineto{\pgfqpoint{8.144018in}{5.190743in}}%
\pgfpathlineto{\pgfqpoint{8.159452in}{5.185951in}}%
\pgfpathlineto{\pgfqpoint{8.174885in}{5.177822in}}%
\pgfpathlineto{\pgfqpoint{8.190319in}{5.166541in}}%
\pgfpathlineto{\pgfqpoint{8.205753in}{5.152344in}}%
\pgfpathlineto{\pgfqpoint{8.221186in}{5.135507in}}%
\pgfpathlineto{\pgfqpoint{8.252053in}{5.095187in}}%
\pgfpathlineto{\pgfqpoint{8.282921in}{5.048304in}}%
\pgfpathlineto{\pgfqpoint{8.421823in}{4.825506in}}%
\pgfpathlineto{\pgfqpoint{8.452691in}{4.784455in}}%
\pgfpathlineto{\pgfqpoint{8.483558in}{4.748506in}}%
\pgfpathlineto{\pgfqpoint{8.514425in}{4.717847in}}%
\pgfpathlineto{\pgfqpoint{8.545293in}{4.692493in}}%
\pgfpathlineto{\pgfqpoint{8.576160in}{4.672441in}}%
\pgfpathlineto{\pgfqpoint{8.607027in}{4.657782in}}%
\pgfpathlineto{\pgfqpoint{8.637894in}{4.648767in}}%
\pgfpathlineto{\pgfqpoint{8.653328in}{4.646502in}}%
\pgfpathlineto{\pgfqpoint{8.668762in}{4.645814in}}%
\pgfpathlineto{\pgfqpoint{8.684195in}{4.646771in}}%
\pgfpathlineto{\pgfqpoint{8.699629in}{4.649442in}}%
\pgfpathlineto{\pgfqpoint{8.715063in}{4.653888in}}%
\pgfpathlineto{\pgfqpoint{8.730496in}{4.660164in}}%
\pgfpathlineto{\pgfqpoint{8.745930in}{4.668311in}}%
\pgfpathlineto{\pgfqpoint{8.761363in}{4.678349in}}%
\pgfpathlineto{\pgfqpoint{8.792231in}{4.704071in}}%
\pgfpathlineto{\pgfqpoint{8.823098in}{4.736985in}}%
\pgfpathlineto{\pgfqpoint{8.853965in}{4.776248in}}%
\pgfpathlineto{\pgfqpoint{8.884833in}{4.820496in}}%
\pgfpathlineto{\pgfqpoint{9.008302in}{5.006513in}}%
\pgfpathlineto{\pgfqpoint{9.039169in}{5.043968in}}%
\pgfpathlineto{\pgfqpoint{9.070036in}{5.073982in}}%
\pgfpathlineto{\pgfqpoint{9.085470in}{5.085850in}}%
\pgfpathlineto{\pgfqpoint{9.100903in}{5.095526in}}%
\pgfpathlineto{\pgfqpoint{9.116337in}{5.102991in}}%
\pgfpathlineto{\pgfqpoint{9.131771in}{5.108276in}}%
\pgfpathlineto{\pgfqpoint{9.147204in}{5.111453in}}%
\pgfpathlineto{\pgfqpoint{9.162638in}{5.112639in}}%
\pgfpathlineto{\pgfqpoint{9.178072in}{5.111992in}}%
\pgfpathlineto{\pgfqpoint{9.208939in}{5.105998in}}%
\pgfpathlineto{\pgfqpoint{9.239806in}{5.095346in}}%
\pgfpathlineto{\pgfqpoint{9.332408in}{5.057440in}}%
\pgfpathlineto{\pgfqpoint{9.363275in}{5.049803in}}%
\pgfpathlineto{\pgfqpoint{9.394143in}{5.047360in}}%
\pgfpathlineto{\pgfqpoint{9.409576in}{5.048383in}}%
\pgfpathlineto{\pgfqpoint{9.425010in}{5.050991in}}%
\pgfpathlineto{\pgfqpoint{9.455877in}{5.061011in}}%
\pgfpathlineto{\pgfqpoint{9.486744in}{5.077186in}}%
\pgfpathlineto{\pgfqpoint{9.517612in}{5.098798in}}%
\pgfpathlineto{\pgfqpoint{9.548479in}{5.124739in}}%
\pgfpathlineto{\pgfqpoint{9.594780in}{5.168713in}}%
\pgfpathlineto{\pgfqpoint{9.656514in}{5.228734in}}%
\pgfpathlineto{\pgfqpoint{9.687382in}{5.256014in}}%
\pgfpathlineto{\pgfqpoint{9.718249in}{5.279790in}}%
\pgfpathlineto{\pgfqpoint{9.733682in}{5.290049in}}%
\pgfpathlineto{\pgfqpoint{9.733682in}{5.290049in}}%
\pgfusepath{stroke}%
\end{pgfscope}%
\begin{pgfscope}%
\pgfpathrectangle{\pgfqpoint{5.698559in}{3.799602in}}{\pgfqpoint{4.227273in}{2.745455in}} %
\pgfusepath{clip}%
\pgfsetrectcap%
\pgfsetroundjoin%
\pgfsetlinewidth{0.501875pt}%
\definecolor{currentstroke}{rgb}{0.543137,0.997705,0.682749}%
\pgfsetstrokecolor{currentstroke}%
\pgfsetdash{}{0pt}%
\pgfpathmoveto{\pgfqpoint{5.890707in}{5.421220in}}%
\pgfpathlineto{\pgfqpoint{5.906141in}{5.424569in}}%
\pgfpathlineto{\pgfqpoint{5.921575in}{5.425169in}}%
\pgfpathlineto{\pgfqpoint{5.937008in}{5.422821in}}%
\pgfpathlineto{\pgfqpoint{5.952442in}{5.417377in}}%
\pgfpathlineto{\pgfqpoint{5.967875in}{5.408740in}}%
\pgfpathlineto{\pgfqpoint{5.983309in}{5.396874in}}%
\pgfpathlineto{\pgfqpoint{5.998743in}{5.381807in}}%
\pgfpathlineto{\pgfqpoint{6.014176in}{5.363634in}}%
\pgfpathlineto{\pgfqpoint{6.029610in}{5.342517in}}%
\pgfpathlineto{\pgfqpoint{6.060477in}{5.292442in}}%
\pgfpathlineto{\pgfqpoint{6.091345in}{5.234213in}}%
\pgfpathlineto{\pgfqpoint{6.183946in}{5.049560in}}%
\pgfpathlineto{\pgfqpoint{6.214814in}{5.000009in}}%
\pgfpathlineto{\pgfqpoint{6.230247in}{4.980084in}}%
\pgfpathlineto{\pgfqpoint{6.245681in}{4.964075in}}%
\pgfpathlineto{\pgfqpoint{6.261115in}{4.952425in}}%
\pgfpathlineto{\pgfqpoint{6.276548in}{4.945515in}}%
\pgfpathlineto{\pgfqpoint{6.291982in}{4.943653in}}%
\pgfpathlineto{\pgfqpoint{6.307415in}{4.947064in}}%
\pgfpathlineto{\pgfqpoint{6.322849in}{4.955888in}}%
\pgfpathlineto{\pgfqpoint{6.338283in}{4.970175in}}%
\pgfpathlineto{\pgfqpoint{6.353716in}{4.989883in}}%
\pgfpathlineto{\pgfqpoint{6.369150in}{5.014875in}}%
\pgfpathlineto{\pgfqpoint{6.384584in}{5.044925in}}%
\pgfpathlineto{\pgfqpoint{6.400017in}{5.079719in}}%
\pgfpathlineto{\pgfqpoint{6.430885in}{5.161884in}}%
\pgfpathlineto{\pgfqpoint{6.461752in}{5.257366in}}%
\pgfpathlineto{\pgfqpoint{6.508053in}{5.414708in}}%
\pgfpathlineto{\pgfqpoint{6.554354in}{5.572603in}}%
\pgfpathlineto{\pgfqpoint{6.585221in}{5.669208in}}%
\pgfpathlineto{\pgfqpoint{6.616088in}{5.753350in}}%
\pgfpathlineto{\pgfqpoint{6.631522in}{5.789530in}}%
\pgfpathlineto{\pgfqpoint{6.646955in}{5.821264in}}%
\pgfpathlineto{\pgfqpoint{6.662389in}{5.848251in}}%
\pgfpathlineto{\pgfqpoint{6.677823in}{5.870269in}}%
\pgfpathlineto{\pgfqpoint{6.693256in}{5.887166in}}%
\pgfpathlineto{\pgfqpoint{6.708690in}{5.898869in}}%
\pgfpathlineto{\pgfqpoint{6.724124in}{5.905373in}}%
\pgfpathlineto{\pgfqpoint{6.739557in}{5.906745in}}%
\pgfpathlineto{\pgfqpoint{6.754991in}{5.903113in}}%
\pgfpathlineto{\pgfqpoint{6.770424in}{5.894665in}}%
\pgfpathlineto{\pgfqpoint{6.785858in}{5.881640in}}%
\pgfpathlineto{\pgfqpoint{6.801292in}{5.864322in}}%
\pgfpathlineto{\pgfqpoint{6.816725in}{5.843034in}}%
\pgfpathlineto{\pgfqpoint{6.832159in}{5.818129in}}%
\pgfpathlineto{\pgfqpoint{6.863026in}{5.758993in}}%
\pgfpathlineto{\pgfqpoint{6.893894in}{5.690090in}}%
\pgfpathlineto{\pgfqpoint{6.940194in}{5.575510in}}%
\pgfpathlineto{\pgfqpoint{7.017363in}{5.379623in}}%
\pgfpathlineto{\pgfqpoint{7.048230in}{5.307001in}}%
\pgfpathlineto{\pgfqpoint{7.079097in}{5.240633in}}%
\pgfpathlineto{\pgfqpoint{7.109964in}{5.181923in}}%
\pgfpathlineto{\pgfqpoint{7.140832in}{5.131822in}}%
\pgfpathlineto{\pgfqpoint{7.171699in}{5.090818in}}%
\pgfpathlineto{\pgfqpoint{7.202566in}{5.058918in}}%
\pgfpathlineto{\pgfqpoint{7.218000in}{5.046252in}}%
\pgfpathlineto{\pgfqpoint{7.233434in}{5.035646in}}%
\pgfpathlineto{\pgfqpoint{7.264301in}{5.020053in}}%
\pgfpathlineto{\pgfqpoint{7.295168in}{5.010763in}}%
\pgfpathlineto{\pgfqpoint{7.326035in}{5.006042in}}%
\pgfpathlineto{\pgfqpoint{7.434071in}{4.996718in}}%
\pgfpathlineto{\pgfqpoint{7.464938in}{4.988562in}}%
\pgfpathlineto{\pgfqpoint{7.495805in}{4.975752in}}%
\pgfpathlineto{\pgfqpoint{7.526673in}{4.958106in}}%
\pgfpathlineto{\pgfqpoint{7.557540in}{4.936210in}}%
\pgfpathlineto{\pgfqpoint{7.619274in}{4.885724in}}%
\pgfpathlineto{\pgfqpoint{7.650142in}{4.861625in}}%
\pgfpathlineto{\pgfqpoint{7.681009in}{4.841821in}}%
\pgfpathlineto{\pgfqpoint{7.696443in}{4.834362in}}%
\pgfpathlineto{\pgfqpoint{7.711876in}{4.828938in}}%
\pgfpathlineto{\pgfqpoint{7.727310in}{4.825814in}}%
\pgfpathlineto{\pgfqpoint{7.742744in}{4.825213in}}%
\pgfpathlineto{\pgfqpoint{7.758177in}{4.827306in}}%
\pgfpathlineto{\pgfqpoint{7.773611in}{4.832209in}}%
\pgfpathlineto{\pgfqpoint{7.789044in}{4.839974in}}%
\pgfpathlineto{\pgfqpoint{7.804478in}{4.850589in}}%
\pgfpathlineto{\pgfqpoint{7.819912in}{4.863976in}}%
\pgfpathlineto{\pgfqpoint{7.835345in}{4.879987in}}%
\pgfpathlineto{\pgfqpoint{7.866213in}{4.918976in}}%
\pgfpathlineto{\pgfqpoint{7.897080in}{4.965172in}}%
\pgfpathlineto{\pgfqpoint{7.989682in}{5.113610in}}%
\pgfpathlineto{\pgfqpoint{8.020549in}{5.154324in}}%
\pgfpathlineto{\pgfqpoint{8.035983in}{5.171216in}}%
\pgfpathlineto{\pgfqpoint{8.051416in}{5.185397in}}%
\pgfpathlineto{\pgfqpoint{8.066850in}{5.196637in}}%
\pgfpathlineto{\pgfqpoint{8.082284in}{5.204774in}}%
\pgfpathlineto{\pgfqpoint{8.097717in}{5.209710in}}%
\pgfpathlineto{\pgfqpoint{8.113151in}{5.211416in}}%
\pgfpathlineto{\pgfqpoint{8.128584in}{5.209929in}}%
\pgfpathlineto{\pgfqpoint{8.144018in}{5.205349in}}%
\pgfpathlineto{\pgfqpoint{8.159452in}{5.197836in}}%
\pgfpathlineto{\pgfqpoint{8.174885in}{5.187599in}}%
\pgfpathlineto{\pgfqpoint{8.190319in}{5.174894in}}%
\pgfpathlineto{\pgfqpoint{8.221186in}{5.143273in}}%
\pgfpathlineto{\pgfqpoint{8.252053in}{5.105558in}}%
\pgfpathlineto{\pgfqpoint{8.375523in}{4.943390in}}%
\pgfpathlineto{\pgfqpoint{8.406390in}{4.908637in}}%
\pgfpathlineto{\pgfqpoint{8.437257in}{4.877509in}}%
\pgfpathlineto{\pgfqpoint{8.468124in}{4.849687in}}%
\pgfpathlineto{\pgfqpoint{8.514425in}{4.812751in}}%
\pgfpathlineto{\pgfqpoint{8.560726in}{4.779601in}}%
\pgfpathlineto{\pgfqpoint{8.607027in}{4.749194in}}%
\pgfpathlineto{\pgfqpoint{8.653328in}{4.722679in}}%
\pgfpathlineto{\pgfqpoint{8.684195in}{4.708782in}}%
\pgfpathlineto{\pgfqpoint{8.715063in}{4.699535in}}%
\pgfpathlineto{\pgfqpoint{8.730496in}{4.697132in}}%
\pgfpathlineto{\pgfqpoint{8.745930in}{4.696457in}}%
\pgfpathlineto{\pgfqpoint{8.761363in}{4.697678in}}%
\pgfpathlineto{\pgfqpoint{8.776797in}{4.700937in}}%
\pgfpathlineto{\pgfqpoint{8.792231in}{4.706349in}}%
\pgfpathlineto{\pgfqpoint{8.807664in}{4.713992in}}%
\pgfpathlineto{\pgfqpoint{8.823098in}{4.723904in}}%
\pgfpathlineto{\pgfqpoint{8.838532in}{4.736079in}}%
\pgfpathlineto{\pgfqpoint{8.869399in}{4.766956in}}%
\pgfpathlineto{\pgfqpoint{8.900266in}{4.805633in}}%
\pgfpathlineto{\pgfqpoint{8.931133in}{4.850417in}}%
\pgfpathlineto{\pgfqpoint{8.992868in}{4.948847in}}%
\pgfpathlineto{\pgfqpoint{9.039169in}{5.019770in}}%
\pgfpathlineto{\pgfqpoint{9.070036in}{5.060827in}}%
\pgfpathlineto{\pgfqpoint{9.100903in}{5.094550in}}%
\pgfpathlineto{\pgfqpoint{9.116337in}{5.108235in}}%
\pgfpathlineto{\pgfqpoint{9.131771in}{5.119671in}}%
\pgfpathlineto{\pgfqpoint{9.147204in}{5.128821in}}%
\pgfpathlineto{\pgfqpoint{9.162638in}{5.135702in}}%
\pgfpathlineto{\pgfqpoint{9.178072in}{5.140376in}}%
\pgfpathlineto{\pgfqpoint{9.193505in}{5.142955in}}%
\pgfpathlineto{\pgfqpoint{9.208939in}{5.143590in}}%
\pgfpathlineto{\pgfqpoint{9.239806in}{5.139822in}}%
\pgfpathlineto{\pgfqpoint{9.270673in}{5.130933in}}%
\pgfpathlineto{\pgfqpoint{9.316974in}{5.112745in}}%
\pgfpathlineto{\pgfqpoint{9.363275in}{5.095282in}}%
\pgfpathlineto{\pgfqpoint{9.394143in}{5.087214in}}%
\pgfpathlineto{\pgfqpoint{9.425010in}{5.083635in}}%
\pgfpathlineto{\pgfqpoint{9.455877in}{5.085367in}}%
\pgfpathlineto{\pgfqpoint{9.486744in}{5.092698in}}%
\pgfpathlineto{\pgfqpoint{9.517612in}{5.105416in}}%
\pgfpathlineto{\pgfqpoint{9.548479in}{5.122883in}}%
\pgfpathlineto{\pgfqpoint{9.579346in}{5.144136in}}%
\pgfpathlineto{\pgfqpoint{9.625647in}{5.180504in}}%
\pgfpathlineto{\pgfqpoint{9.702815in}{5.242381in}}%
\pgfpathlineto{\pgfqpoint{9.733682in}{5.264052in}}%
\pgfpathlineto{\pgfqpoint{9.733682in}{5.264052in}}%
\pgfusepath{stroke}%
\end{pgfscope}%
\begin{pgfscope}%
\pgfpathrectangle{\pgfqpoint{5.698559in}{3.799602in}}{\pgfqpoint{4.227273in}{2.745455in}} %
\pgfusepath{clip}%
\pgfsetrectcap%
\pgfsetroundjoin%
\pgfsetlinewidth{0.501875pt}%
\definecolor{currentstroke}{rgb}{0.621569,0.981823,0.636474}%
\pgfsetstrokecolor{currentstroke}%
\pgfsetdash{}{0pt}%
\pgfpathmoveto{\pgfqpoint{5.890707in}{5.410306in}}%
\pgfpathlineto{\pgfqpoint{5.906141in}{5.409794in}}%
\pgfpathlineto{\pgfqpoint{5.921575in}{5.406328in}}%
\pgfpathlineto{\pgfqpoint{5.937008in}{5.399790in}}%
\pgfpathlineto{\pgfqpoint{5.952442in}{5.390113in}}%
\pgfpathlineto{\pgfqpoint{5.967875in}{5.377288in}}%
\pgfpathlineto{\pgfqpoint{5.983309in}{5.361365in}}%
\pgfpathlineto{\pgfqpoint{5.998743in}{5.342453in}}%
\pgfpathlineto{\pgfqpoint{6.029610in}{5.296418in}}%
\pgfpathlineto{\pgfqpoint{6.060477in}{5.241288in}}%
\pgfpathlineto{\pgfqpoint{6.106778in}{5.148290in}}%
\pgfpathlineto{\pgfqpoint{6.153079in}{5.054641in}}%
\pgfpathlineto{\pgfqpoint{6.183946in}{4.999076in}}%
\pgfpathlineto{\pgfqpoint{6.199380in}{4.974985in}}%
\pgfpathlineto{\pgfqpoint{6.214814in}{4.954036in}}%
\pgfpathlineto{\pgfqpoint{6.230247in}{4.936692in}}%
\pgfpathlineto{\pgfqpoint{6.245681in}{4.923372in}}%
\pgfpathlineto{\pgfqpoint{6.261115in}{4.914440in}}%
\pgfpathlineto{\pgfqpoint{6.276548in}{4.910197in}}%
\pgfpathlineto{\pgfqpoint{6.291982in}{4.910873in}}%
\pgfpathlineto{\pgfqpoint{6.307415in}{4.916628in}}%
\pgfpathlineto{\pgfqpoint{6.322849in}{4.927540in}}%
\pgfpathlineto{\pgfqpoint{6.338283in}{4.943610in}}%
\pgfpathlineto{\pgfqpoint{6.353716in}{4.964758in}}%
\pgfpathlineto{\pgfqpoint{6.369150in}{4.990824in}}%
\pgfpathlineto{\pgfqpoint{6.384584in}{5.021572in}}%
\pgfpathlineto{\pgfqpoint{6.400017in}{5.056693in}}%
\pgfpathlineto{\pgfqpoint{6.430885in}{5.138481in}}%
\pgfpathlineto{\pgfqpoint{6.461752in}{5.232480in}}%
\pgfpathlineto{\pgfqpoint{6.508053in}{5.386576in}}%
\pgfpathlineto{\pgfqpoint{6.569787in}{5.590889in}}%
\pgfpathlineto{\pgfqpoint{6.600655in}{5.681955in}}%
\pgfpathlineto{\pgfqpoint{6.631522in}{5.760232in}}%
\pgfpathlineto{\pgfqpoint{6.646955in}{5.793576in}}%
\pgfpathlineto{\pgfqpoint{6.662389in}{5.822636in}}%
\pgfpathlineto{\pgfqpoint{6.677823in}{5.847172in}}%
\pgfpathlineto{\pgfqpoint{6.693256in}{5.867006in}}%
\pgfpathlineto{\pgfqpoint{6.708690in}{5.882019in}}%
\pgfpathlineto{\pgfqpoint{6.724124in}{5.892155in}}%
\pgfpathlineto{\pgfqpoint{6.739557in}{5.897414in}}%
\pgfpathlineto{\pgfqpoint{6.754991in}{5.897852in}}%
\pgfpathlineto{\pgfqpoint{6.770424in}{5.893576in}}%
\pgfpathlineto{\pgfqpoint{6.785858in}{5.884741in}}%
\pgfpathlineto{\pgfqpoint{6.801292in}{5.871544in}}%
\pgfpathlineto{\pgfqpoint{6.816725in}{5.854222in}}%
\pgfpathlineto{\pgfqpoint{6.832159in}{5.833044in}}%
\pgfpathlineto{\pgfqpoint{6.847593in}{5.808306in}}%
\pgfpathlineto{\pgfqpoint{6.878460in}{5.749453in}}%
\pgfpathlineto{\pgfqpoint{6.909327in}{5.680424in}}%
\pgfpathlineto{\pgfqpoint{6.955628in}{5.564179in}}%
\pgfpathlineto{\pgfqpoint{7.063664in}{5.283358in}}%
\pgfpathlineto{\pgfqpoint{7.094531in}{5.211902in}}%
\pgfpathlineto{\pgfqpoint{7.125398in}{5.147906in}}%
\pgfpathlineto{\pgfqpoint{7.156265in}{5.092615in}}%
\pgfpathlineto{\pgfqpoint{7.187133in}{5.046768in}}%
\pgfpathlineto{\pgfqpoint{7.218000in}{5.010555in}}%
\pgfpathlineto{\pgfqpoint{7.233434in}{4.995968in}}%
\pgfpathlineto{\pgfqpoint{7.248867in}{4.983610in}}%
\pgfpathlineto{\pgfqpoint{7.264301in}{4.973350in}}%
\pgfpathlineto{\pgfqpoint{7.279734in}{4.965030in}}%
\pgfpathlineto{\pgfqpoint{7.310602in}{4.953441in}}%
\pgfpathlineto{\pgfqpoint{7.341469in}{4.947094in}}%
\pgfpathlineto{\pgfqpoint{7.387770in}{4.943074in}}%
\pgfpathlineto{\pgfqpoint{7.434071in}{4.939757in}}%
\pgfpathlineto{\pgfqpoint{7.464938in}{4.935174in}}%
\pgfpathlineto{\pgfqpoint{7.495805in}{4.927441in}}%
\pgfpathlineto{\pgfqpoint{7.526673in}{4.916200in}}%
\pgfpathlineto{\pgfqpoint{7.557540in}{4.901809in}}%
\pgfpathlineto{\pgfqpoint{7.650142in}{4.853271in}}%
\pgfpathlineto{\pgfqpoint{7.681009in}{4.842159in}}%
\pgfpathlineto{\pgfqpoint{7.696443in}{4.838850in}}%
\pgfpathlineto{\pgfqpoint{7.711876in}{4.837396in}}%
\pgfpathlineto{\pgfqpoint{7.727310in}{4.838029in}}%
\pgfpathlineto{\pgfqpoint{7.742744in}{4.840944in}}%
\pgfpathlineto{\pgfqpoint{7.758177in}{4.846291in}}%
\pgfpathlineto{\pgfqpoint{7.773611in}{4.854172in}}%
\pgfpathlineto{\pgfqpoint{7.789044in}{4.864633in}}%
\pgfpathlineto{\pgfqpoint{7.804478in}{4.877660in}}%
\pgfpathlineto{\pgfqpoint{7.819912in}{4.893183in}}%
\pgfpathlineto{\pgfqpoint{7.850779in}{4.931127in}}%
\pgfpathlineto{\pgfqpoint{7.881646in}{4.976727in}}%
\pgfpathlineto{\pgfqpoint{7.927947in}{5.053765in}}%
\pgfpathlineto{\pgfqpoint{7.974248in}{5.131443in}}%
\pgfpathlineto{\pgfqpoint{8.005115in}{5.177867in}}%
\pgfpathlineto{\pgfqpoint{8.035983in}{5.216332in}}%
\pgfpathlineto{\pgfqpoint{8.051416in}{5.231768in}}%
\pgfpathlineto{\pgfqpoint{8.066850in}{5.244334in}}%
\pgfpathlineto{\pgfqpoint{8.082284in}{5.253847in}}%
\pgfpathlineto{\pgfqpoint{8.097717in}{5.260180in}}%
\pgfpathlineto{\pgfqpoint{8.113151in}{5.263269in}}%
\pgfpathlineto{\pgfqpoint{8.128584in}{5.263109in}}%
\pgfpathlineto{\pgfqpoint{8.144018in}{5.259755in}}%
\pgfpathlineto{\pgfqpoint{8.159452in}{5.253316in}}%
\pgfpathlineto{\pgfqpoint{8.174885in}{5.243954in}}%
\pgfpathlineto{\pgfqpoint{8.190319in}{5.231874in}}%
\pgfpathlineto{\pgfqpoint{8.205753in}{5.217322in}}%
\pgfpathlineto{\pgfqpoint{8.236620in}{5.181921in}}%
\pgfpathlineto{\pgfqpoint{8.267487in}{5.140180in}}%
\pgfpathlineto{\pgfqpoint{8.329222in}{5.047639in}}%
\pgfpathlineto{\pgfqpoint{8.375523in}{4.978825in}}%
\pgfpathlineto{\pgfqpoint{8.421823in}{4.916130in}}%
\pgfpathlineto{\pgfqpoint{8.452691in}{4.878957in}}%
\pgfpathlineto{\pgfqpoint{8.483558in}{4.845707in}}%
\pgfpathlineto{\pgfqpoint{8.514425in}{4.816261in}}%
\pgfpathlineto{\pgfqpoint{8.545293in}{4.790399in}}%
\pgfpathlineto{\pgfqpoint{8.576160in}{4.767956in}}%
\pgfpathlineto{\pgfqpoint{8.607027in}{4.748951in}}%
\pgfpathlineto{\pgfqpoint{8.637894in}{4.733658in}}%
\pgfpathlineto{\pgfqpoint{8.668762in}{4.722618in}}%
\pgfpathlineto{\pgfqpoint{8.699629in}{4.716586in}}%
\pgfpathlineto{\pgfqpoint{8.715063in}{4.715716in}}%
\pgfpathlineto{\pgfqpoint{8.730496in}{4.716418in}}%
\pgfpathlineto{\pgfqpoint{8.745930in}{4.718793in}}%
\pgfpathlineto{\pgfqpoint{8.761363in}{4.722926in}}%
\pgfpathlineto{\pgfqpoint{8.776797in}{4.728884in}}%
\pgfpathlineto{\pgfqpoint{8.792231in}{4.736711in}}%
\pgfpathlineto{\pgfqpoint{8.807664in}{4.746424in}}%
\pgfpathlineto{\pgfqpoint{8.838532in}{4.771419in}}%
\pgfpathlineto{\pgfqpoint{8.869399in}{4.803349in}}%
\pgfpathlineto{\pgfqpoint{8.900266in}{4.841135in}}%
\pgfpathlineto{\pgfqpoint{8.946567in}{4.905165in}}%
\pgfpathlineto{\pgfqpoint{9.008302in}{4.993015in}}%
\pgfpathlineto{\pgfqpoint{9.039169in}{5.032830in}}%
\pgfpathlineto{\pgfqpoint{9.070036in}{5.067137in}}%
\pgfpathlineto{\pgfqpoint{9.100903in}{5.094392in}}%
\pgfpathlineto{\pgfqpoint{9.116337in}{5.105056in}}%
\pgfpathlineto{\pgfqpoint{9.131771in}{5.113660in}}%
\pgfpathlineto{\pgfqpoint{9.147204in}{5.120197in}}%
\pgfpathlineto{\pgfqpoint{9.162638in}{5.124705in}}%
\pgfpathlineto{\pgfqpoint{9.178072in}{5.127265in}}%
\pgfpathlineto{\pgfqpoint{9.193505in}{5.128000in}}%
\pgfpathlineto{\pgfqpoint{9.224373in}{5.124672in}}%
\pgfpathlineto{\pgfqpoint{9.255240in}{5.116379in}}%
\pgfpathlineto{\pgfqpoint{9.301541in}{5.099096in}}%
\pgfpathlineto{\pgfqpoint{9.347842in}{5.082476in}}%
\pgfpathlineto{\pgfqpoint{9.378709in}{5.075059in}}%
\pgfpathlineto{\pgfqpoint{9.409576in}{5.072356in}}%
\pgfpathlineto{\pgfqpoint{9.440443in}{5.075327in}}%
\pgfpathlineto{\pgfqpoint{9.471311in}{5.084358in}}%
\pgfpathlineto{\pgfqpoint{9.502178in}{5.099262in}}%
\pgfpathlineto{\pgfqpoint{9.533045in}{5.119330in}}%
\pgfpathlineto{\pgfqpoint{9.563913in}{5.143426in}}%
\pgfpathlineto{\pgfqpoint{9.625647in}{5.197789in}}%
\pgfpathlineto{\pgfqpoint{9.671948in}{5.237635in}}%
\pgfpathlineto{\pgfqpoint{9.702815in}{5.260939in}}%
\pgfpathlineto{\pgfqpoint{9.733682in}{5.280221in}}%
\pgfpathlineto{\pgfqpoint{9.733682in}{5.280221in}}%
\pgfusepath{stroke}%
\end{pgfscope}%
\begin{pgfscope}%
\pgfpathrectangle{\pgfqpoint{5.698559in}{3.799602in}}{\pgfqpoint{4.227273in}{2.745455in}} %
\pgfusepath{clip}%
\pgfsetrectcap%
\pgfsetroundjoin%
\pgfsetlinewidth{0.501875pt}%
\definecolor{currentstroke}{rgb}{0.700000,0.951057,0.587785}%
\pgfsetstrokecolor{currentstroke}%
\pgfsetdash{}{0pt}%
\pgfpathmoveto{\pgfqpoint{5.890707in}{5.420589in}}%
\pgfpathlineto{\pgfqpoint{5.906141in}{5.421925in}}%
\pgfpathlineto{\pgfqpoint{5.921575in}{5.420256in}}%
\pgfpathlineto{\pgfqpoint{5.937008in}{5.415474in}}%
\pgfpathlineto{\pgfqpoint{5.952442in}{5.407524in}}%
\pgfpathlineto{\pgfqpoint{5.967875in}{5.396409in}}%
\pgfpathlineto{\pgfqpoint{5.983309in}{5.382194in}}%
\pgfpathlineto{\pgfqpoint{5.998743in}{5.365002in}}%
\pgfpathlineto{\pgfqpoint{6.014176in}{5.345021in}}%
\pgfpathlineto{\pgfqpoint{6.045044in}{5.297729in}}%
\pgfpathlineto{\pgfqpoint{6.075911in}{5.242940in}}%
\pgfpathlineto{\pgfqpoint{6.168513in}{5.069719in}}%
\pgfpathlineto{\pgfqpoint{6.199380in}{5.022619in}}%
\pgfpathlineto{\pgfqpoint{6.214814in}{5.003306in}}%
\pgfpathlineto{\pgfqpoint{6.230247in}{4.987392in}}%
\pgfpathlineto{\pgfqpoint{6.245681in}{4.975251in}}%
\pgfpathlineto{\pgfqpoint{6.261115in}{4.967204in}}%
\pgfpathlineto{\pgfqpoint{6.276548in}{4.963510in}}%
\pgfpathlineto{\pgfqpoint{6.291982in}{4.964362in}}%
\pgfpathlineto{\pgfqpoint{6.307415in}{4.969886in}}%
\pgfpathlineto{\pgfqpoint{6.322849in}{4.980135in}}%
\pgfpathlineto{\pgfqpoint{6.338283in}{4.995090in}}%
\pgfpathlineto{\pgfqpoint{6.353716in}{5.014658in}}%
\pgfpathlineto{\pgfqpoint{6.369150in}{5.038677in}}%
\pgfpathlineto{\pgfqpoint{6.384584in}{5.066915in}}%
\pgfpathlineto{\pgfqpoint{6.400017in}{5.099079in}}%
\pgfpathlineto{\pgfqpoint{6.430885in}{5.173712in}}%
\pgfpathlineto{\pgfqpoint{6.461752in}{5.259156in}}%
\pgfpathlineto{\pgfqpoint{6.523486in}{5.446029in}}%
\pgfpathlineto{\pgfqpoint{6.569787in}{5.582892in}}%
\pgfpathlineto{\pgfqpoint{6.600655in}{5.664717in}}%
\pgfpathlineto{\pgfqpoint{6.631522in}{5.734854in}}%
\pgfpathlineto{\pgfqpoint{6.646955in}{5.764641in}}%
\pgfpathlineto{\pgfqpoint{6.662389in}{5.790524in}}%
\pgfpathlineto{\pgfqpoint{6.677823in}{5.812285in}}%
\pgfpathlineto{\pgfqpoint{6.693256in}{5.829758in}}%
\pgfpathlineto{\pgfqpoint{6.708690in}{5.842832in}}%
\pgfpathlineto{\pgfqpoint{6.724124in}{5.851450in}}%
\pgfpathlineto{\pgfqpoint{6.739557in}{5.855605in}}%
\pgfpathlineto{\pgfqpoint{6.754991in}{5.855343in}}%
\pgfpathlineto{\pgfqpoint{6.770424in}{5.850753in}}%
\pgfpathlineto{\pgfqpoint{6.785858in}{5.841970in}}%
\pgfpathlineto{\pgfqpoint{6.801292in}{5.829169in}}%
\pgfpathlineto{\pgfqpoint{6.816725in}{5.812558in}}%
\pgfpathlineto{\pgfqpoint{6.832159in}{5.792380in}}%
\pgfpathlineto{\pgfqpoint{6.847593in}{5.768904in}}%
\pgfpathlineto{\pgfqpoint{6.878460in}{5.713243in}}%
\pgfpathlineto{\pgfqpoint{6.909327in}{5.648110in}}%
\pgfpathlineto{\pgfqpoint{6.955628in}{5.538594in}}%
\pgfpathlineto{\pgfqpoint{7.048230in}{5.310468in}}%
\pgfpathlineto{\pgfqpoint{7.079097in}{5.240747in}}%
\pgfpathlineto{\pgfqpoint{7.109964in}{5.177545in}}%
\pgfpathlineto{\pgfqpoint{7.140832in}{5.122304in}}%
\pgfpathlineto{\pgfqpoint{7.171699in}{5.075997in}}%
\pgfpathlineto{\pgfqpoint{7.202566in}{5.039082in}}%
\pgfpathlineto{\pgfqpoint{7.218000in}{5.024141in}}%
\pgfpathlineto{\pgfqpoint{7.233434in}{5.011473in}}%
\pgfpathlineto{\pgfqpoint{7.248867in}{5.000982in}}%
\pgfpathlineto{\pgfqpoint{7.264301in}{4.992540in}}%
\pgfpathlineto{\pgfqpoint{7.279734in}{4.985990in}}%
\pgfpathlineto{\pgfqpoint{7.310602in}{4.977799in}}%
\pgfpathlineto{\pgfqpoint{7.341469in}{4.974664in}}%
\pgfpathlineto{\pgfqpoint{7.387770in}{4.975067in}}%
\pgfpathlineto{\pgfqpoint{7.434071in}{4.975543in}}%
\pgfpathlineto{\pgfqpoint{7.464938in}{4.972988in}}%
\pgfpathlineto{\pgfqpoint{7.495805in}{4.966735in}}%
\pgfpathlineto{\pgfqpoint{7.526673in}{4.956272in}}%
\pgfpathlineto{\pgfqpoint{7.557540in}{4.941798in}}%
\pgfpathlineto{\pgfqpoint{7.603841in}{4.914730in}}%
\pgfpathlineto{\pgfqpoint{7.650142in}{4.886427in}}%
\pgfpathlineto{\pgfqpoint{7.681009in}{4.870549in}}%
\pgfpathlineto{\pgfqpoint{7.711876in}{4.859770in}}%
\pgfpathlineto{\pgfqpoint{7.727310in}{4.856952in}}%
\pgfpathlineto{\pgfqpoint{7.742744in}{4.856144in}}%
\pgfpathlineto{\pgfqpoint{7.758177in}{4.857521in}}%
\pgfpathlineto{\pgfqpoint{7.773611in}{4.861209in}}%
\pgfpathlineto{\pgfqpoint{7.789044in}{4.867286in}}%
\pgfpathlineto{\pgfqpoint{7.804478in}{4.875773in}}%
\pgfpathlineto{\pgfqpoint{7.819912in}{4.886634in}}%
\pgfpathlineto{\pgfqpoint{7.835345in}{4.899778in}}%
\pgfpathlineto{\pgfqpoint{7.866213in}{4.932261in}}%
\pgfpathlineto{\pgfqpoint{7.897080in}{4.971404in}}%
\pgfpathlineto{\pgfqpoint{7.958814in}{5.059145in}}%
\pgfpathlineto{\pgfqpoint{7.989682in}{5.101589in}}%
\pgfpathlineto{\pgfqpoint{8.020549in}{5.138973in}}%
\pgfpathlineto{\pgfqpoint{8.051416in}{5.168651in}}%
\pgfpathlineto{\pgfqpoint{8.066850in}{5.179943in}}%
\pgfpathlineto{\pgfqpoint{8.082284in}{5.188618in}}%
\pgfpathlineto{\pgfqpoint{8.097717in}{5.194550in}}%
\pgfpathlineto{\pgfqpoint{8.113151in}{5.197674in}}%
\pgfpathlineto{\pgfqpoint{8.128584in}{5.197975in}}%
\pgfpathlineto{\pgfqpoint{8.144018in}{5.195496in}}%
\pgfpathlineto{\pgfqpoint{8.159452in}{5.190329in}}%
\pgfpathlineto{\pgfqpoint{8.174885in}{5.182612in}}%
\pgfpathlineto{\pgfqpoint{8.190319in}{5.172524in}}%
\pgfpathlineto{\pgfqpoint{8.221186in}{5.146122in}}%
\pgfpathlineto{\pgfqpoint{8.252053in}{5.113123in}}%
\pgfpathlineto{\pgfqpoint{8.298354in}{5.056059in}}%
\pgfpathlineto{\pgfqpoint{8.375523in}{4.957429in}}%
\pgfpathlineto{\pgfqpoint{8.421823in}{4.903933in}}%
\pgfpathlineto{\pgfqpoint{8.452691in}{4.872106in}}%
\pgfpathlineto{\pgfqpoint{8.483558in}{4.843524in}}%
\pgfpathlineto{\pgfqpoint{8.514425in}{4.818105in}}%
\pgfpathlineto{\pgfqpoint{8.545293in}{4.795728in}}%
\pgfpathlineto{\pgfqpoint{8.576160in}{4.776372in}}%
\pgfpathlineto{\pgfqpoint{8.607027in}{4.760209in}}%
\pgfpathlineto{\pgfqpoint{8.637894in}{4.747665in}}%
\pgfpathlineto{\pgfqpoint{8.668762in}{4.739397in}}%
\pgfpathlineto{\pgfqpoint{8.699629in}{4.736228in}}%
\pgfpathlineto{\pgfqpoint{8.715063in}{4.736829in}}%
\pgfpathlineto{\pgfqpoint{8.730496in}{4.739025in}}%
\pgfpathlineto{\pgfqpoint{8.745930in}{4.742904in}}%
\pgfpathlineto{\pgfqpoint{8.761363in}{4.748542in}}%
\pgfpathlineto{\pgfqpoint{8.776797in}{4.755988in}}%
\pgfpathlineto{\pgfqpoint{8.792231in}{4.765269in}}%
\pgfpathlineto{\pgfqpoint{8.823098in}{4.789284in}}%
\pgfpathlineto{\pgfqpoint{8.853965in}{4.820151in}}%
\pgfpathlineto{\pgfqpoint{8.884833in}{4.856879in}}%
\pgfpathlineto{\pgfqpoint{8.931133in}{4.919480in}}%
\pgfpathlineto{\pgfqpoint{8.992868in}{5.005817in}}%
\pgfpathlineto{\pgfqpoint{9.023735in}{5.044932in}}%
\pgfpathlineto{\pgfqpoint{9.054603in}{5.078412in}}%
\pgfpathlineto{\pgfqpoint{9.085470in}{5.104534in}}%
\pgfpathlineto{\pgfqpoint{9.100903in}{5.114456in}}%
\pgfpathlineto{\pgfqpoint{9.116337in}{5.122165in}}%
\pgfpathlineto{\pgfqpoint{9.131771in}{5.127632in}}%
\pgfpathlineto{\pgfqpoint{9.147204in}{5.130873in}}%
\pgfpathlineto{\pgfqpoint{9.162638in}{5.131949in}}%
\pgfpathlineto{\pgfqpoint{9.178072in}{5.130970in}}%
\pgfpathlineto{\pgfqpoint{9.193505in}{5.128086in}}%
\pgfpathlineto{\pgfqpoint{9.224373in}{5.117400in}}%
\pgfpathlineto{\pgfqpoint{9.255240in}{5.101792in}}%
\pgfpathlineto{\pgfqpoint{9.347842in}{5.048675in}}%
\pgfpathlineto{\pgfqpoint{9.378709in}{5.036515in}}%
\pgfpathlineto{\pgfqpoint{9.394143in}{5.032541in}}%
\pgfpathlineto{\pgfqpoint{9.409576in}{5.030185in}}%
\pgfpathlineto{\pgfqpoint{9.425010in}{5.029558in}}%
\pgfpathlineto{\pgfqpoint{9.440443in}{5.030729in}}%
\pgfpathlineto{\pgfqpoint{9.455877in}{5.033727in}}%
\pgfpathlineto{\pgfqpoint{9.471311in}{5.038539in}}%
\pgfpathlineto{\pgfqpoint{9.502178in}{5.053349in}}%
\pgfpathlineto{\pgfqpoint{9.533045in}{5.074296in}}%
\pgfpathlineto{\pgfqpoint{9.563913in}{5.100018in}}%
\pgfpathlineto{\pgfqpoint{9.610213in}{5.143778in}}%
\pgfpathlineto{\pgfqpoint{9.671948in}{5.201949in}}%
\pgfpathlineto{\pgfqpoint{9.702815in}{5.227092in}}%
\pgfpathlineto{\pgfqpoint{9.733682in}{5.247852in}}%
\pgfpathlineto{\pgfqpoint{9.733682in}{5.247852in}}%
\pgfusepath{stroke}%
\end{pgfscope}%
\begin{pgfscope}%
\pgfpathrectangle{\pgfqpoint{5.698559in}{3.799602in}}{\pgfqpoint{4.227273in}{2.745455in}} %
\pgfusepath{clip}%
\pgfsetrectcap%
\pgfsetroundjoin%
\pgfsetlinewidth{0.501875pt}%
\definecolor{currentstroke}{rgb}{0.778431,0.905873,0.536867}%
\pgfsetstrokecolor{currentstroke}%
\pgfsetdash{}{0pt}%
\pgfpathmoveto{\pgfqpoint{5.890707in}{5.417266in}}%
\pgfpathlineto{\pgfqpoint{5.906141in}{5.418024in}}%
\pgfpathlineto{\pgfqpoint{5.921575in}{5.415872in}}%
\pgfpathlineto{\pgfqpoint{5.937008in}{5.410722in}}%
\pgfpathlineto{\pgfqpoint{5.952442in}{5.402546in}}%
\pgfpathlineto{\pgfqpoint{5.967875in}{5.391369in}}%
\pgfpathlineto{\pgfqpoint{5.983309in}{5.377281in}}%
\pgfpathlineto{\pgfqpoint{5.998743in}{5.360427in}}%
\pgfpathlineto{\pgfqpoint{6.029610in}{5.319295in}}%
\pgfpathlineto{\pgfqpoint{6.060477in}{5.270240in}}%
\pgfpathlineto{\pgfqpoint{6.122212in}{5.160904in}}%
\pgfpathlineto{\pgfqpoint{6.153079in}{5.107985in}}%
\pgfpathlineto{\pgfqpoint{6.183946in}{5.061363in}}%
\pgfpathlineto{\pgfqpoint{6.199380in}{5.041558in}}%
\pgfpathlineto{\pgfqpoint{6.214814in}{5.024650in}}%
\pgfpathlineto{\pgfqpoint{6.230247in}{5.011010in}}%
\pgfpathlineto{\pgfqpoint{6.245681in}{5.000966in}}%
\pgfpathlineto{\pgfqpoint{6.261115in}{4.994795in}}%
\pgfpathlineto{\pgfqpoint{6.276548in}{4.992721in}}%
\pgfpathlineto{\pgfqpoint{6.291982in}{4.994907in}}%
\pgfpathlineto{\pgfqpoint{6.307415in}{5.001454in}}%
\pgfpathlineto{\pgfqpoint{6.322849in}{5.012400in}}%
\pgfpathlineto{\pgfqpoint{6.338283in}{5.027720in}}%
\pgfpathlineto{\pgfqpoint{6.353716in}{5.047323in}}%
\pgfpathlineto{\pgfqpoint{6.369150in}{5.071057in}}%
\pgfpathlineto{\pgfqpoint{6.384584in}{5.098709in}}%
\pgfpathlineto{\pgfqpoint{6.415451in}{5.164637in}}%
\pgfpathlineto{\pgfqpoint{6.446318in}{5.242352in}}%
\pgfpathlineto{\pgfqpoint{6.492619in}{5.373410in}}%
\pgfpathlineto{\pgfqpoint{6.569787in}{5.597199in}}%
\pgfpathlineto{\pgfqpoint{6.600655in}{5.676883in}}%
\pgfpathlineto{\pgfqpoint{6.631522in}{5.745598in}}%
\pgfpathlineto{\pgfqpoint{6.646955in}{5.774941in}}%
\pgfpathlineto{\pgfqpoint{6.662389in}{5.800549in}}%
\pgfpathlineto{\pgfqpoint{6.677823in}{5.822188in}}%
\pgfpathlineto{\pgfqpoint{6.693256in}{5.839674in}}%
\pgfpathlineto{\pgfqpoint{6.708690in}{5.852876in}}%
\pgfpathlineto{\pgfqpoint{6.724124in}{5.861715in}}%
\pgfpathlineto{\pgfqpoint{6.739557in}{5.866161in}}%
\pgfpathlineto{\pgfqpoint{6.754991in}{5.866237in}}%
\pgfpathlineto{\pgfqpoint{6.770424in}{5.862011in}}%
\pgfpathlineto{\pgfqpoint{6.785858in}{5.853597in}}%
\pgfpathlineto{\pgfqpoint{6.801292in}{5.841151in}}%
\pgfpathlineto{\pgfqpoint{6.816725in}{5.824869in}}%
\pgfpathlineto{\pgfqpoint{6.832159in}{5.804980in}}%
\pgfpathlineto{\pgfqpoint{6.847593in}{5.781743in}}%
\pgfpathlineto{\pgfqpoint{6.878460in}{5.726389in}}%
\pgfpathlineto{\pgfqpoint{6.909327in}{5.661328in}}%
\pgfpathlineto{\pgfqpoint{6.955628in}{5.551525in}}%
\pgfpathlineto{\pgfqpoint{7.063664in}{5.286074in}}%
\pgfpathlineto{\pgfqpoint{7.094531in}{5.218754in}}%
\pgfpathlineto{\pgfqpoint{7.125398in}{5.158662in}}%
\pgfpathlineto{\pgfqpoint{7.156265in}{5.106990in}}%
\pgfpathlineto{\pgfqpoint{7.187133in}{5.064431in}}%
\pgfpathlineto{\pgfqpoint{7.218000in}{5.031152in}}%
\pgfpathlineto{\pgfqpoint{7.233434in}{5.017898in}}%
\pgfpathlineto{\pgfqpoint{7.248867in}{5.006785in}}%
\pgfpathlineto{\pgfqpoint{7.264301in}{4.997684in}}%
\pgfpathlineto{\pgfqpoint{7.295168in}{4.984872in}}%
\pgfpathlineto{\pgfqpoint{7.326035in}{4.977913in}}%
\pgfpathlineto{\pgfqpoint{7.356903in}{4.974941in}}%
\pgfpathlineto{\pgfqpoint{7.449504in}{4.969922in}}%
\pgfpathlineto{\pgfqpoint{7.480372in}{4.963685in}}%
\pgfpathlineto{\pgfqpoint{7.511239in}{4.953284in}}%
\pgfpathlineto{\pgfqpoint{7.542106in}{4.938530in}}%
\pgfpathlineto{\pgfqpoint{7.572974in}{4.919969in}}%
\pgfpathlineto{\pgfqpoint{7.681009in}{4.848206in}}%
\pgfpathlineto{\pgfqpoint{7.711876in}{4.835132in}}%
\pgfpathlineto{\pgfqpoint{7.727310in}{4.831320in}}%
\pgfpathlineto{\pgfqpoint{7.742744in}{4.829655in}}%
\pgfpathlineto{\pgfqpoint{7.758177in}{4.830337in}}%
\pgfpathlineto{\pgfqpoint{7.773611in}{4.833519in}}%
\pgfpathlineto{\pgfqpoint{7.789044in}{4.839299in}}%
\pgfpathlineto{\pgfqpoint{7.804478in}{4.847720in}}%
\pgfpathlineto{\pgfqpoint{7.819912in}{4.858764in}}%
\pgfpathlineto{\pgfqpoint{7.835345in}{4.872352in}}%
\pgfpathlineto{\pgfqpoint{7.866213in}{4.906535in}}%
\pgfpathlineto{\pgfqpoint{7.897080in}{4.948457in}}%
\pgfpathlineto{\pgfqpoint{7.943381in}{5.019994in}}%
\pgfpathlineto{\pgfqpoint{7.989682in}{5.092118in}}%
\pgfpathlineto{\pgfqpoint{8.020549in}{5.134864in}}%
\pgfpathlineto{\pgfqpoint{8.051416in}{5.169780in}}%
\pgfpathlineto{\pgfqpoint{8.066850in}{5.183532in}}%
\pgfpathlineto{\pgfqpoint{8.082284in}{5.194501in}}%
\pgfpathlineto{\pgfqpoint{8.097717in}{5.202525in}}%
\pgfpathlineto{\pgfqpoint{8.113151in}{5.207498in}}%
\pgfpathlineto{\pgfqpoint{8.128584in}{5.209374in}}%
\pgfpathlineto{\pgfqpoint{8.144018in}{5.208167in}}%
\pgfpathlineto{\pgfqpoint{8.159452in}{5.203946in}}%
\pgfpathlineto{\pgfqpoint{8.174885in}{5.196834in}}%
\pgfpathlineto{\pgfqpoint{8.190319in}{5.186998in}}%
\pgfpathlineto{\pgfqpoint{8.205753in}{5.174650in}}%
\pgfpathlineto{\pgfqpoint{8.236620in}{5.143415in}}%
\pgfpathlineto{\pgfqpoint{8.267487in}{5.105345in}}%
\pgfpathlineto{\pgfqpoint{8.313788in}{5.040620in}}%
\pgfpathlineto{\pgfqpoint{8.390956in}{4.930198in}}%
\pgfpathlineto{\pgfqpoint{8.437257in}{4.870820in}}%
\pgfpathlineto{\pgfqpoint{8.468124in}{4.835812in}}%
\pgfpathlineto{\pgfqpoint{8.498992in}{4.804801in}}%
\pgfpathlineto{\pgfqpoint{8.529859in}{4.777855in}}%
\pgfpathlineto{\pgfqpoint{8.560726in}{4.754990in}}%
\pgfpathlineto{\pgfqpoint{8.591593in}{4.736285in}}%
\pgfpathlineto{\pgfqpoint{8.622461in}{4.721952in}}%
\pgfpathlineto{\pgfqpoint{8.653328in}{4.712365in}}%
\pgfpathlineto{\pgfqpoint{8.684195in}{4.708022in}}%
\pgfpathlineto{\pgfqpoint{8.715063in}{4.709472in}}%
\pgfpathlineto{\pgfqpoint{8.730496in}{4.712527in}}%
\pgfpathlineto{\pgfqpoint{8.745930in}{4.717201in}}%
\pgfpathlineto{\pgfqpoint{8.776797in}{4.731508in}}%
\pgfpathlineto{\pgfqpoint{8.807664in}{4.752388in}}%
\pgfpathlineto{\pgfqpoint{8.838532in}{4.779443in}}%
\pgfpathlineto{\pgfqpoint{8.869399in}{4.811837in}}%
\pgfpathlineto{\pgfqpoint{8.915700in}{4.867576in}}%
\pgfpathlineto{\pgfqpoint{8.992868in}{4.964831in}}%
\pgfpathlineto{\pgfqpoint{9.023735in}{4.999650in}}%
\pgfpathlineto{\pgfqpoint{9.054603in}{5.029542in}}%
\pgfpathlineto{\pgfqpoint{9.085470in}{5.053266in}}%
\pgfpathlineto{\pgfqpoint{9.116337in}{5.070094in}}%
\pgfpathlineto{\pgfqpoint{9.131771in}{5.075854in}}%
\pgfpathlineto{\pgfqpoint{9.147204in}{5.079882in}}%
\pgfpathlineto{\pgfqpoint{9.178072in}{5.083075in}}%
\pgfpathlineto{\pgfqpoint{9.208939in}{5.080658in}}%
\pgfpathlineto{\pgfqpoint{9.239806in}{5.074061in}}%
\pgfpathlineto{\pgfqpoint{9.347842in}{5.043944in}}%
\pgfpathlineto{\pgfqpoint{9.378709in}{5.040554in}}%
\pgfpathlineto{\pgfqpoint{9.409576in}{5.041925in}}%
\pgfpathlineto{\pgfqpoint{9.440443in}{5.048687in}}%
\pgfpathlineto{\pgfqpoint{9.471311in}{5.060948in}}%
\pgfpathlineto{\pgfqpoint{9.502178in}{5.078304in}}%
\pgfpathlineto{\pgfqpoint{9.533045in}{5.099889in}}%
\pgfpathlineto{\pgfqpoint{9.579346in}{5.137426in}}%
\pgfpathlineto{\pgfqpoint{9.641081in}{5.189091in}}%
\pgfpathlineto{\pgfqpoint{9.671948in}{5.212170in}}%
\pgfpathlineto{\pgfqpoint{9.702815in}{5.231689in}}%
\pgfpathlineto{\pgfqpoint{9.733682in}{5.246755in}}%
\pgfpathlineto{\pgfqpoint{9.733682in}{5.246755in}}%
\pgfusepath{stroke}%
\end{pgfscope}%
\begin{pgfscope}%
\pgfpathrectangle{\pgfqpoint{5.698559in}{3.799602in}}{\pgfqpoint{4.227273in}{2.745455in}} %
\pgfusepath{clip}%
\pgfsetrectcap%
\pgfsetroundjoin%
\pgfsetlinewidth{0.501875pt}%
\definecolor{currentstroke}{rgb}{0.864706,0.840344,0.478512}%
\pgfsetstrokecolor{currentstroke}%
\pgfsetdash{}{0pt}%
\pgfpathmoveto{\pgfqpoint{5.890707in}{5.430441in}}%
\pgfpathlineto{\pgfqpoint{5.906141in}{5.430894in}}%
\pgfpathlineto{\pgfqpoint{5.921575in}{5.428577in}}%
\pgfpathlineto{\pgfqpoint{5.937008in}{5.423388in}}%
\pgfpathlineto{\pgfqpoint{5.952442in}{5.415279in}}%
\pgfpathlineto{\pgfqpoint{5.967875in}{5.404256in}}%
\pgfpathlineto{\pgfqpoint{5.983309in}{5.390383in}}%
\pgfpathlineto{\pgfqpoint{5.998743in}{5.373781in}}%
\pgfpathlineto{\pgfqpoint{6.029610in}{5.333161in}}%
\pgfpathlineto{\pgfqpoint{6.060477in}{5.284458in}}%
\pgfpathlineto{\pgfqpoint{6.106778in}{5.202571in}}%
\pgfpathlineto{\pgfqpoint{6.153079in}{5.120670in}}%
\pgfpathlineto{\pgfqpoint{6.183946in}{5.072424in}}%
\pgfpathlineto{\pgfqpoint{6.199380in}{5.051615in}}%
\pgfpathlineto{\pgfqpoint{6.214814in}{5.033597in}}%
\pgfpathlineto{\pgfqpoint{6.230247in}{5.018763in}}%
\pgfpathlineto{\pgfqpoint{6.245681in}{5.007469in}}%
\pgfpathlineto{\pgfqpoint{6.261115in}{5.000020in}}%
\pgfpathlineto{\pgfqpoint{6.276548in}{4.996671in}}%
\pgfpathlineto{\pgfqpoint{6.291982in}{4.997616in}}%
\pgfpathlineto{\pgfqpoint{6.307415in}{5.002987in}}%
\pgfpathlineto{\pgfqpoint{6.322849in}{5.012853in}}%
\pgfpathlineto{\pgfqpoint{6.338283in}{5.027214in}}%
\pgfpathlineto{\pgfqpoint{6.353716in}{5.046001in}}%
\pgfpathlineto{\pgfqpoint{6.369150in}{5.069080in}}%
\pgfpathlineto{\pgfqpoint{6.384584in}{5.096247in}}%
\pgfpathlineto{\pgfqpoint{6.400017in}{5.127238in}}%
\pgfpathlineto{\pgfqpoint{6.430885in}{5.199335in}}%
\pgfpathlineto{\pgfqpoint{6.461752in}{5.282154in}}%
\pgfpathlineto{\pgfqpoint{6.508053in}{5.417891in}}%
\pgfpathlineto{\pgfqpoint{6.554354in}{5.554451in}}%
\pgfpathlineto{\pgfqpoint{6.585221in}{5.638783in}}%
\pgfpathlineto{\pgfqpoint{6.616088in}{5.713249in}}%
\pgfpathlineto{\pgfqpoint{6.646955in}{5.774708in}}%
\pgfpathlineto{\pgfqpoint{6.662389in}{5.799796in}}%
\pgfpathlineto{\pgfqpoint{6.677823in}{5.820829in}}%
\pgfpathlineto{\pgfqpoint{6.693256in}{5.837655in}}%
\pgfpathlineto{\pgfqpoint{6.708690in}{5.850183in}}%
\pgfpathlineto{\pgfqpoint{6.724124in}{5.858375in}}%
\pgfpathlineto{\pgfqpoint{6.739557in}{5.862246in}}%
\pgfpathlineto{\pgfqpoint{6.754991in}{5.861864in}}%
\pgfpathlineto{\pgfqpoint{6.770424in}{5.857339in}}%
\pgfpathlineto{\pgfqpoint{6.785858in}{5.848825in}}%
\pgfpathlineto{\pgfqpoint{6.801292in}{5.836509in}}%
\pgfpathlineto{\pgfqpoint{6.816725in}{5.820610in}}%
\pgfpathlineto{\pgfqpoint{6.832159in}{5.801373in}}%
\pgfpathlineto{\pgfqpoint{6.847593in}{5.779063in}}%
\pgfpathlineto{\pgfqpoint{6.878460in}{5.726349in}}%
\pgfpathlineto{\pgfqpoint{6.909327in}{5.664810in}}%
\pgfpathlineto{\pgfqpoint{6.955628in}{5.561180in}}%
\pgfpathlineto{\pgfqpoint{7.079097in}{5.272065in}}%
\pgfpathlineto{\pgfqpoint{7.109964in}{5.207722in}}%
\pgfpathlineto{\pgfqpoint{7.140832in}{5.149863in}}%
\pgfpathlineto{\pgfqpoint{7.171699in}{5.099684in}}%
\pgfpathlineto{\pgfqpoint{7.202566in}{5.058002in}}%
\pgfpathlineto{\pgfqpoint{7.233434in}{5.025183in}}%
\pgfpathlineto{\pgfqpoint{7.248867in}{5.012070in}}%
\pgfpathlineto{\pgfqpoint{7.264301in}{5.001073in}}%
\pgfpathlineto{\pgfqpoint{7.279734in}{4.992086in}}%
\pgfpathlineto{\pgfqpoint{7.295168in}{4.984972in}}%
\pgfpathlineto{\pgfqpoint{7.326035in}{4.975651in}}%
\pgfpathlineto{\pgfqpoint{7.356903in}{4.971428in}}%
\pgfpathlineto{\pgfqpoint{7.403204in}{4.970242in}}%
\pgfpathlineto{\pgfqpoint{7.449504in}{4.968944in}}%
\pgfpathlineto{\pgfqpoint{7.480372in}{4.964955in}}%
\pgfpathlineto{\pgfqpoint{7.511239in}{4.957010in}}%
\pgfpathlineto{\pgfqpoint{7.542106in}{4.944627in}}%
\pgfpathlineto{\pgfqpoint{7.572974in}{4.928112in}}%
\pgfpathlineto{\pgfqpoint{7.619274in}{4.898173in}}%
\pgfpathlineto{\pgfqpoint{7.665575in}{4.867967in}}%
\pgfpathlineto{\pgfqpoint{7.696443in}{4.851765in}}%
\pgfpathlineto{\pgfqpoint{7.711876in}{4.845794in}}%
\pgfpathlineto{\pgfqpoint{7.727310in}{4.841629in}}%
\pgfpathlineto{\pgfqpoint{7.742744in}{4.839525in}}%
\pgfpathlineto{\pgfqpoint{7.758177in}{4.839692in}}%
\pgfpathlineto{\pgfqpoint{7.773611in}{4.842296in}}%
\pgfpathlineto{\pgfqpoint{7.789044in}{4.847447in}}%
\pgfpathlineto{\pgfqpoint{7.804478in}{4.855196in}}%
\pgfpathlineto{\pgfqpoint{7.819912in}{4.865531in}}%
\pgfpathlineto{\pgfqpoint{7.835345in}{4.878378in}}%
\pgfpathlineto{\pgfqpoint{7.850779in}{4.893598in}}%
\pgfpathlineto{\pgfqpoint{7.881646in}{4.930298in}}%
\pgfpathlineto{\pgfqpoint{7.912514in}{4.973373in}}%
\pgfpathlineto{\pgfqpoint{8.005115in}{5.109306in}}%
\pgfpathlineto{\pgfqpoint{8.035983in}{5.145510in}}%
\pgfpathlineto{\pgfqpoint{8.051416in}{5.160195in}}%
\pgfpathlineto{\pgfqpoint{8.066850in}{5.172229in}}%
\pgfpathlineto{\pgfqpoint{8.082284in}{5.181399in}}%
\pgfpathlineto{\pgfqpoint{8.097717in}{5.187555in}}%
\pgfpathlineto{\pgfqpoint{8.113151in}{5.190610in}}%
\pgfpathlineto{\pgfqpoint{8.128584in}{5.190542in}}%
\pgfpathlineto{\pgfqpoint{8.144018in}{5.187390in}}%
\pgfpathlineto{\pgfqpoint{8.159452in}{5.181253in}}%
\pgfpathlineto{\pgfqpoint{8.174885in}{5.172289in}}%
\pgfpathlineto{\pgfqpoint{8.190319in}{5.160701in}}%
\pgfpathlineto{\pgfqpoint{8.205753in}{5.146737in}}%
\pgfpathlineto{\pgfqpoint{8.236620in}{5.112830in}}%
\pgfpathlineto{\pgfqpoint{8.267487in}{5.073068in}}%
\pgfpathlineto{\pgfqpoint{8.390956in}{4.903812in}}%
\pgfpathlineto{\pgfqpoint{8.421823in}{4.867609in}}%
\pgfpathlineto{\pgfqpoint{8.452691in}{4.835393in}}%
\pgfpathlineto{\pgfqpoint{8.483558in}{4.807075in}}%
\pgfpathlineto{\pgfqpoint{8.514425in}{4.782304in}}%
\pgfpathlineto{\pgfqpoint{8.545293in}{4.760673in}}%
\pgfpathlineto{\pgfqpoint{8.576160in}{4.741903in}}%
\pgfpathlineto{\pgfqpoint{8.607027in}{4.725988in}}%
\pgfpathlineto{\pgfqpoint{8.637894in}{4.713264in}}%
\pgfpathlineto{\pgfqpoint{8.668762in}{4.704403in}}%
\pgfpathlineto{\pgfqpoint{8.699629in}{4.700318in}}%
\pgfpathlineto{\pgfqpoint{8.715063in}{4.700381in}}%
\pgfpathlineto{\pgfqpoint{8.730496in}{4.702011in}}%
\pgfpathlineto{\pgfqpoint{8.745930in}{4.705315in}}%
\pgfpathlineto{\pgfqpoint{8.761363in}{4.710379in}}%
\pgfpathlineto{\pgfqpoint{8.776797in}{4.717268in}}%
\pgfpathlineto{\pgfqpoint{8.792231in}{4.726013in}}%
\pgfpathlineto{\pgfqpoint{8.823098in}{4.749028in}}%
\pgfpathlineto{\pgfqpoint{8.853965in}{4.778956in}}%
\pgfpathlineto{\pgfqpoint{8.884833in}{4.814710in}}%
\pgfpathlineto{\pgfqpoint{8.931133in}{4.875537in}}%
\pgfpathlineto{\pgfqpoint{8.992868in}{4.958640in}}%
\pgfpathlineto{\pgfqpoint{9.023735in}{4.995963in}}%
\pgfpathlineto{\pgfqpoint{9.054603in}{5.027902in}}%
\pgfpathlineto{\pgfqpoint{9.085470in}{5.053168in}}%
\pgfpathlineto{\pgfqpoint{9.100903in}{5.063075in}}%
\pgfpathlineto{\pgfqpoint{9.116337in}{5.071133in}}%
\pgfpathlineto{\pgfqpoint{9.131771in}{5.077376in}}%
\pgfpathlineto{\pgfqpoint{9.147204in}{5.081875in}}%
\pgfpathlineto{\pgfqpoint{9.178072in}{5.086131in}}%
\pgfpathlineto{\pgfqpoint{9.208939in}{5.085186in}}%
\pgfpathlineto{\pgfqpoint{9.239806in}{5.080711in}}%
\pgfpathlineto{\pgfqpoint{9.316974in}{5.066187in}}%
\pgfpathlineto{\pgfqpoint{9.347842in}{5.063453in}}%
\pgfpathlineto{\pgfqpoint{9.378709in}{5.064333in}}%
\pgfpathlineto{\pgfqpoint{9.409576in}{5.069449in}}%
\pgfpathlineto{\pgfqpoint{9.440443in}{5.078930in}}%
\pgfpathlineto{\pgfqpoint{9.471311in}{5.092438in}}%
\pgfpathlineto{\pgfqpoint{9.502178in}{5.109240in}}%
\pgfpathlineto{\pgfqpoint{9.548479in}{5.138297in}}%
\pgfpathlineto{\pgfqpoint{9.610213in}{5.177665in}}%
\pgfpathlineto{\pgfqpoint{9.641081in}{5.194851in}}%
\pgfpathlineto{\pgfqpoint{9.671948in}{5.209020in}}%
\pgfpathlineto{\pgfqpoint{9.702815in}{5.219495in}}%
\pgfpathlineto{\pgfqpoint{9.733682in}{5.225930in}}%
\pgfpathlineto{\pgfqpoint{9.733682in}{5.225930in}}%
\pgfusepath{stroke}%
\end{pgfscope}%
\begin{pgfscope}%
\pgfpathrectangle{\pgfqpoint{5.698559in}{3.799602in}}{\pgfqpoint{4.227273in}{2.745455in}} %
\pgfusepath{clip}%
\pgfsetrectcap%
\pgfsetroundjoin%
\pgfsetlinewidth{0.501875pt}%
\definecolor{currentstroke}{rgb}{0.943137,0.767363,0.423549}%
\pgfsetstrokecolor{currentstroke}%
\pgfsetdash{}{0pt}%
\pgfpathmoveto{\pgfqpoint{5.890707in}{5.411377in}}%
\pgfpathlineto{\pgfqpoint{5.906141in}{5.410286in}}%
\pgfpathlineto{\pgfqpoint{5.921575in}{5.406643in}}%
\pgfpathlineto{\pgfqpoint{5.937008in}{5.400366in}}%
\pgfpathlineto{\pgfqpoint{5.952442in}{5.391419in}}%
\pgfpathlineto{\pgfqpoint{5.967875in}{5.379816in}}%
\pgfpathlineto{\pgfqpoint{5.983309in}{5.365625in}}%
\pgfpathlineto{\pgfqpoint{5.998743in}{5.348963in}}%
\pgfpathlineto{\pgfqpoint{6.029610in}{5.308959in}}%
\pgfpathlineto{\pgfqpoint{6.060477in}{5.261749in}}%
\pgfpathlineto{\pgfqpoint{6.122212in}{5.156834in}}%
\pgfpathlineto{\pgfqpoint{6.153079in}{5.105857in}}%
\pgfpathlineto{\pgfqpoint{6.183946in}{5.060701in}}%
\pgfpathlineto{\pgfqpoint{6.199380in}{5.041418in}}%
\pgfpathlineto{\pgfqpoint{6.214814in}{5.024886in}}%
\pgfpathlineto{\pgfqpoint{6.230247in}{5.011486in}}%
\pgfpathlineto{\pgfqpoint{6.245681in}{5.001556in}}%
\pgfpathlineto{\pgfqpoint{6.261115in}{4.995393in}}%
\pgfpathlineto{\pgfqpoint{6.276548in}{4.993240in}}%
\pgfpathlineto{\pgfqpoint{6.291982in}{4.995287in}}%
\pgfpathlineto{\pgfqpoint{6.307415in}{5.001661in}}%
\pgfpathlineto{\pgfqpoint{6.322849in}{5.012428in}}%
\pgfpathlineto{\pgfqpoint{6.338283in}{5.027587in}}%
\pgfpathlineto{\pgfqpoint{6.353716in}{5.047071in}}%
\pgfpathlineto{\pgfqpoint{6.369150in}{5.070745in}}%
\pgfpathlineto{\pgfqpoint{6.384584in}{5.098413in}}%
\pgfpathlineto{\pgfqpoint{6.415451in}{5.164622in}}%
\pgfpathlineto{\pgfqpoint{6.446318in}{5.242935in}}%
\pgfpathlineto{\pgfqpoint{6.492619in}{5.375241in}}%
\pgfpathlineto{\pgfqpoint{6.554354in}{5.557418in}}%
\pgfpathlineto{\pgfqpoint{6.585221in}{5.641385in}}%
\pgfpathlineto{\pgfqpoint{6.616088in}{5.715460in}}%
\pgfpathlineto{\pgfqpoint{6.646955in}{5.776536in}}%
\pgfpathlineto{\pgfqpoint{6.662389in}{5.801446in}}%
\pgfpathlineto{\pgfqpoint{6.677823in}{5.822313in}}%
\pgfpathlineto{\pgfqpoint{6.693256in}{5.838987in}}%
\pgfpathlineto{\pgfqpoint{6.708690in}{5.851378in}}%
\pgfpathlineto{\pgfqpoint{6.724124in}{5.859449in}}%
\pgfpathlineto{\pgfqpoint{6.739557in}{5.863217in}}%
\pgfpathlineto{\pgfqpoint{6.754991in}{5.862747in}}%
\pgfpathlineto{\pgfqpoint{6.770424in}{5.858150in}}%
\pgfpathlineto{\pgfqpoint{6.785858in}{5.849578in}}%
\pgfpathlineto{\pgfqpoint{6.801292in}{5.837218in}}%
\pgfpathlineto{\pgfqpoint{6.816725in}{5.821288in}}%
\pgfpathlineto{\pgfqpoint{6.832159in}{5.802031in}}%
\pgfpathlineto{\pgfqpoint{6.847593in}{5.779708in}}%
\pgfpathlineto{\pgfqpoint{6.878460in}{5.726988in}}%
\pgfpathlineto{\pgfqpoint{6.909327in}{5.665449in}}%
\pgfpathlineto{\pgfqpoint{6.955628in}{5.561774in}}%
\pgfpathlineto{\pgfqpoint{7.079097in}{5.271568in}}%
\pgfpathlineto{\pgfqpoint{7.109964in}{5.206581in}}%
\pgfpathlineto{\pgfqpoint{7.140832in}{5.147912in}}%
\pgfpathlineto{\pgfqpoint{7.171699in}{5.096780in}}%
\pgfpathlineto{\pgfqpoint{7.202566in}{5.054052in}}%
\pgfpathlineto{\pgfqpoint{7.233434in}{5.020166in}}%
\pgfpathlineto{\pgfqpoint{7.248867in}{5.006539in}}%
\pgfpathlineto{\pgfqpoint{7.264301in}{4.995057in}}%
\pgfpathlineto{\pgfqpoint{7.279734in}{4.985627in}}%
\pgfpathlineto{\pgfqpoint{7.295168in}{4.978122in}}%
\pgfpathlineto{\pgfqpoint{7.326035in}{4.968221in}}%
\pgfpathlineto{\pgfqpoint{7.356903in}{4.963743in}}%
\pgfpathlineto{\pgfqpoint{7.403204in}{4.962853in}}%
\pgfpathlineto{\pgfqpoint{7.449504in}{4.962599in}}%
\pgfpathlineto{\pgfqpoint{7.480372in}{4.959594in}}%
\pgfpathlineto{\pgfqpoint{7.511239in}{4.952725in}}%
\pgfpathlineto{\pgfqpoint{7.542106in}{4.941370in}}%
\pgfpathlineto{\pgfqpoint{7.572974in}{4.925697in}}%
\pgfpathlineto{\pgfqpoint{7.619274in}{4.896436in}}%
\pgfpathlineto{\pgfqpoint{7.665575in}{4.866037in}}%
\pgfpathlineto{\pgfqpoint{7.696443in}{4.849237in}}%
\pgfpathlineto{\pgfqpoint{7.711876in}{4.842852in}}%
\pgfpathlineto{\pgfqpoint{7.727310in}{4.838214in}}%
\pgfpathlineto{\pgfqpoint{7.742744in}{4.835593in}}%
\pgfpathlineto{\pgfqpoint{7.758177in}{4.835219in}}%
\pgfpathlineto{\pgfqpoint{7.773611in}{4.837273in}}%
\pgfpathlineto{\pgfqpoint{7.789044in}{4.841886in}}%
\pgfpathlineto{\pgfqpoint{7.804478in}{4.849126in}}%
\pgfpathlineto{\pgfqpoint{7.819912in}{4.859000in}}%
\pgfpathlineto{\pgfqpoint{7.835345in}{4.871449in}}%
\pgfpathlineto{\pgfqpoint{7.850779in}{4.886349in}}%
\pgfpathlineto{\pgfqpoint{7.881646in}{4.922691in}}%
\pgfpathlineto{\pgfqpoint{7.912514in}{4.965840in}}%
\pgfpathlineto{\pgfqpoint{8.005115in}{5.104739in}}%
\pgfpathlineto{\pgfqpoint{8.035983in}{5.142735in}}%
\pgfpathlineto{\pgfqpoint{8.051416in}{5.158432in}}%
\pgfpathlineto{\pgfqpoint{8.066850in}{5.171539in}}%
\pgfpathlineto{\pgfqpoint{8.082284in}{5.181836in}}%
\pgfpathlineto{\pgfqpoint{8.097717in}{5.189160in}}%
\pgfpathlineto{\pgfqpoint{8.113151in}{5.193413in}}%
\pgfpathlineto{\pgfqpoint{8.128584in}{5.194562in}}%
\pgfpathlineto{\pgfqpoint{8.144018in}{5.192635in}}%
\pgfpathlineto{\pgfqpoint{8.159452in}{5.187721in}}%
\pgfpathlineto{\pgfqpoint{8.174885in}{5.179962in}}%
\pgfpathlineto{\pgfqpoint{8.190319in}{5.169553in}}%
\pgfpathlineto{\pgfqpoint{8.205753in}{5.156729in}}%
\pgfpathlineto{\pgfqpoint{8.236620in}{5.124935in}}%
\pgfpathlineto{\pgfqpoint{8.267487in}{5.086982in}}%
\pgfpathlineto{\pgfqpoint{8.329222in}{5.002496in}}%
\pgfpathlineto{\pgfqpoint{8.375523in}{4.939939in}}%
\pgfpathlineto{\pgfqpoint{8.421823in}{4.883271in}}%
\pgfpathlineto{\pgfqpoint{8.452691in}{4.849720in}}%
\pgfpathlineto{\pgfqpoint{8.483558in}{4.819581in}}%
\pgfpathlineto{\pgfqpoint{8.514425in}{4.792602in}}%
\pgfpathlineto{\pgfqpoint{8.545293in}{4.768489in}}%
\pgfpathlineto{\pgfqpoint{8.576160in}{4.747080in}}%
\pgfpathlineto{\pgfqpoint{8.607027in}{4.728472in}}%
\pgfpathlineto{\pgfqpoint{8.637894in}{4.713082in}}%
\pgfpathlineto{\pgfqpoint{8.668762in}{4.701627in}}%
\pgfpathlineto{\pgfqpoint{8.699629in}{4.695036in}}%
\pgfpathlineto{\pgfqpoint{8.715063in}{4.693875in}}%
\pgfpathlineto{\pgfqpoint{8.730496in}{4.694294in}}%
\pgfpathlineto{\pgfqpoint{8.745930in}{4.696396in}}%
\pgfpathlineto{\pgfqpoint{8.761363in}{4.700264in}}%
\pgfpathlineto{\pgfqpoint{8.776797in}{4.705956in}}%
\pgfpathlineto{\pgfqpoint{8.792231in}{4.713501in}}%
\pgfpathlineto{\pgfqpoint{8.807664in}{4.722896in}}%
\pgfpathlineto{\pgfqpoint{8.838532in}{4.747040in}}%
\pgfpathlineto{\pgfqpoint{8.869399in}{4.777637in}}%
\pgfpathlineto{\pgfqpoint{8.900266in}{4.813376in}}%
\pgfpathlineto{\pgfqpoint{8.946567in}{4.872672in}}%
\pgfpathlineto{\pgfqpoint{9.008302in}{4.951336in}}%
\pgfpathlineto{\pgfqpoint{9.039169in}{4.985993in}}%
\pgfpathlineto{\pgfqpoint{9.070036in}{5.015482in}}%
\pgfpathlineto{\pgfqpoint{9.100903in}{5.038943in}}%
\pgfpathlineto{\pgfqpoint{9.131771in}{5.056150in}}%
\pgfpathlineto{\pgfqpoint{9.162638in}{5.067496in}}%
\pgfpathlineto{\pgfqpoint{9.193505in}{5.073911in}}%
\pgfpathlineto{\pgfqpoint{9.224373in}{5.076726in}}%
\pgfpathlineto{\pgfqpoint{9.347842in}{5.082409in}}%
\pgfpathlineto{\pgfqpoint{9.378709in}{5.088642in}}%
\pgfpathlineto{\pgfqpoint{9.409576in}{5.098049in}}%
\pgfpathlineto{\pgfqpoint{9.440443in}{5.110501in}}%
\pgfpathlineto{\pgfqpoint{9.486744in}{5.133673in}}%
\pgfpathlineto{\pgfqpoint{9.594780in}{5.192056in}}%
\pgfpathlineto{\pgfqpoint{9.625647in}{5.205213in}}%
\pgfpathlineto{\pgfqpoint{9.656514in}{5.215310in}}%
\pgfpathlineto{\pgfqpoint{9.687382in}{5.221932in}}%
\pgfpathlineto{\pgfqpoint{9.718249in}{5.224966in}}%
\pgfpathlineto{\pgfqpoint{9.733682in}{5.225186in}}%
\pgfpathlineto{\pgfqpoint{9.733682in}{5.225186in}}%
\pgfusepath{stroke}%
\end{pgfscope}%
\begin{pgfscope}%
\pgfpathrectangle{\pgfqpoint{5.698559in}{3.799602in}}{\pgfqpoint{4.227273in}{2.745455in}} %
\pgfusepath{clip}%
\pgfsetrectcap%
\pgfsetroundjoin%
\pgfsetlinewidth{0.501875pt}%
\definecolor{currentstroke}{rgb}{1.000000,0.682749,0.366979}%
\pgfsetstrokecolor{currentstroke}%
\pgfsetdash{}{0pt}%
\pgfpathmoveto{\pgfqpoint{5.890707in}{5.415720in}}%
\pgfpathlineto{\pgfqpoint{5.906141in}{5.414112in}}%
\pgfpathlineto{\pgfqpoint{5.921575in}{5.409951in}}%
\pgfpathlineto{\pgfqpoint{5.937008in}{5.403170in}}%
\pgfpathlineto{\pgfqpoint{5.952442in}{5.393747in}}%
\pgfpathlineto{\pgfqpoint{5.967875in}{5.381709in}}%
\pgfpathlineto{\pgfqpoint{5.983309in}{5.367134in}}%
\pgfpathlineto{\pgfqpoint{6.014176in}{5.330925in}}%
\pgfpathlineto{\pgfqpoint{6.045044in}{5.286706in}}%
\pgfpathlineto{\pgfqpoint{6.075911in}{5.236764in}}%
\pgfpathlineto{\pgfqpoint{6.153079in}{5.106657in}}%
\pgfpathlineto{\pgfqpoint{6.183946in}{5.061489in}}%
\pgfpathlineto{\pgfqpoint{6.199380in}{5.042115in}}%
\pgfpathlineto{\pgfqpoint{6.214814in}{5.025421in}}%
\pgfpathlineto{\pgfqpoint{6.230247in}{5.011781in}}%
\pgfpathlineto{\pgfqpoint{6.245681in}{5.001533in}}%
\pgfpathlineto{\pgfqpoint{6.261115in}{4.994975in}}%
\pgfpathlineto{\pgfqpoint{6.276548in}{4.992357in}}%
\pgfpathlineto{\pgfqpoint{6.291982in}{4.993881in}}%
\pgfpathlineto{\pgfqpoint{6.307415in}{4.999689in}}%
\pgfpathlineto{\pgfqpoint{6.322849in}{5.009866in}}%
\pgfpathlineto{\pgfqpoint{6.338283in}{5.024433in}}%
\pgfpathlineto{\pgfqpoint{6.353716in}{5.043346in}}%
\pgfpathlineto{\pgfqpoint{6.369150in}{5.066496in}}%
\pgfpathlineto{\pgfqpoint{6.384584in}{5.093711in}}%
\pgfpathlineto{\pgfqpoint{6.400017in}{5.124751in}}%
\pgfpathlineto{\pgfqpoint{6.430885in}{5.197059in}}%
\pgfpathlineto{\pgfqpoint{6.461752in}{5.280375in}}%
\pgfpathlineto{\pgfqpoint{6.508053in}{5.417593in}}%
\pgfpathlineto{\pgfqpoint{6.569787in}{5.600512in}}%
\pgfpathlineto{\pgfqpoint{6.600655in}{5.682146in}}%
\pgfpathlineto{\pgfqpoint{6.631522in}{5.752112in}}%
\pgfpathlineto{\pgfqpoint{6.646955in}{5.781759in}}%
\pgfpathlineto{\pgfqpoint{6.662389in}{5.807444in}}%
\pgfpathlineto{\pgfqpoint{6.677823in}{5.828940in}}%
\pgfpathlineto{\pgfqpoint{6.693256in}{5.846080in}}%
\pgfpathlineto{\pgfqpoint{6.708690in}{5.858757in}}%
\pgfpathlineto{\pgfqpoint{6.724124in}{5.866929in}}%
\pgfpathlineto{\pgfqpoint{6.739557in}{5.870607in}}%
\pgfpathlineto{\pgfqpoint{6.754991in}{5.869861in}}%
\pgfpathlineto{\pgfqpoint{6.770424in}{5.864811in}}%
\pgfpathlineto{\pgfqpoint{6.785858in}{5.855619in}}%
\pgfpathlineto{\pgfqpoint{6.801292in}{5.842492in}}%
\pgfpathlineto{\pgfqpoint{6.816725in}{5.825667in}}%
\pgfpathlineto{\pgfqpoint{6.832159in}{5.805412in}}%
\pgfpathlineto{\pgfqpoint{6.847593in}{5.782013in}}%
\pgfpathlineto{\pgfqpoint{6.878460in}{5.727017in}}%
\pgfpathlineto{\pgfqpoint{6.909327in}{5.663218in}}%
\pgfpathlineto{\pgfqpoint{6.955628in}{5.556625in}}%
\pgfpathlineto{\pgfqpoint{7.063664in}{5.298561in}}%
\pgfpathlineto{\pgfqpoint{7.094531in}{5.231715in}}%
\pgfpathlineto{\pgfqpoint{7.125398in}{5.171028in}}%
\pgfpathlineto{\pgfqpoint{7.156265in}{5.117754in}}%
\pgfpathlineto{\pgfqpoint{7.187133in}{5.072794in}}%
\pgfpathlineto{\pgfqpoint{7.218000in}{5.036631in}}%
\pgfpathlineto{\pgfqpoint{7.233434in}{5.021867in}}%
\pgfpathlineto{\pgfqpoint{7.248867in}{5.009259in}}%
\pgfpathlineto{\pgfqpoint{7.264301in}{4.998725in}}%
\pgfpathlineto{\pgfqpoint{7.279734in}{4.990150in}}%
\pgfpathlineto{\pgfqpoint{7.310602in}{4.978254in}}%
\pgfpathlineto{\pgfqpoint{7.341469in}{4.972055in}}%
\pgfpathlineto{\pgfqpoint{7.372336in}{4.969681in}}%
\pgfpathlineto{\pgfqpoint{7.449504in}{4.967047in}}%
\pgfpathlineto{\pgfqpoint{7.480372in}{4.962461in}}%
\pgfpathlineto{\pgfqpoint{7.511239in}{4.953943in}}%
\pgfpathlineto{\pgfqpoint{7.542106in}{4.941093in}}%
\pgfpathlineto{\pgfqpoint{7.572974in}{4.924265in}}%
\pgfpathlineto{\pgfqpoint{7.619274in}{4.894148in}}%
\pgfpathlineto{\pgfqpoint{7.665575in}{4.863942in}}%
\pgfpathlineto{\pgfqpoint{7.696443in}{4.847680in}}%
\pgfpathlineto{\pgfqpoint{7.711876in}{4.841620in}}%
\pgfpathlineto{\pgfqpoint{7.727310in}{4.837312in}}%
\pgfpathlineto{\pgfqpoint{7.742744in}{4.835001in}}%
\pgfpathlineto{\pgfqpoint{7.758177in}{4.834892in}}%
\pgfpathlineto{\pgfqpoint{7.773611in}{4.837147in}}%
\pgfpathlineto{\pgfqpoint{7.789044in}{4.841880in}}%
\pgfpathlineto{\pgfqpoint{7.804478in}{4.849147in}}%
\pgfpathlineto{\pgfqpoint{7.819912in}{4.858949in}}%
\pgfpathlineto{\pgfqpoint{7.835345in}{4.871226in}}%
\pgfpathlineto{\pgfqpoint{7.850779in}{4.885861in}}%
\pgfpathlineto{\pgfqpoint{7.881646in}{4.921441in}}%
\pgfpathlineto{\pgfqpoint{7.912514in}{4.963655in}}%
\pgfpathlineto{\pgfqpoint{8.020549in}{5.120989in}}%
\pgfpathlineto{\pgfqpoint{8.051416in}{5.155488in}}%
\pgfpathlineto{\pgfqpoint{8.066850in}{5.169249in}}%
\pgfpathlineto{\pgfqpoint{8.082284in}{5.180360in}}%
\pgfpathlineto{\pgfqpoint{8.097717in}{5.188644in}}%
\pgfpathlineto{\pgfqpoint{8.113151in}{5.193983in}}%
\pgfpathlineto{\pgfqpoint{8.128584in}{5.196318in}}%
\pgfpathlineto{\pgfqpoint{8.144018in}{5.195646in}}%
\pgfpathlineto{\pgfqpoint{8.159452in}{5.192023in}}%
\pgfpathlineto{\pgfqpoint{8.174885in}{5.185558in}}%
\pgfpathlineto{\pgfqpoint{8.190319in}{5.176408in}}%
\pgfpathlineto{\pgfqpoint{8.205753in}{5.164772in}}%
\pgfpathlineto{\pgfqpoint{8.236620in}{5.135014in}}%
\pgfpathlineto{\pgfqpoint{8.267487in}{5.098454in}}%
\pgfpathlineto{\pgfqpoint{8.313788in}{5.035990in}}%
\pgfpathlineto{\pgfqpoint{8.390956in}{4.928966in}}%
\pgfpathlineto{\pgfqpoint{8.437257in}{4.870903in}}%
\pgfpathlineto{\pgfqpoint{8.468124in}{4.836179in}}%
\pgfpathlineto{\pgfqpoint{8.498992in}{4.804804in}}%
\pgfpathlineto{\pgfqpoint{8.529859in}{4.776735in}}%
\pgfpathlineto{\pgfqpoint{8.560726in}{4.751947in}}%
\pgfpathlineto{\pgfqpoint{8.591593in}{4.730570in}}%
\pgfpathlineto{\pgfqpoint{8.622461in}{4.712970in}}%
\pgfpathlineto{\pgfqpoint{8.653328in}{4.699757in}}%
\pgfpathlineto{\pgfqpoint{8.684195in}{4.691730in}}%
\pgfpathlineto{\pgfqpoint{8.699629in}{4.689934in}}%
\pgfpathlineto{\pgfqpoint{8.715063in}{4.689755in}}%
\pgfpathlineto{\pgfqpoint{8.730496in}{4.691285in}}%
\pgfpathlineto{\pgfqpoint{8.745930in}{4.694599in}}%
\pgfpathlineto{\pgfqpoint{8.761363in}{4.699751in}}%
\pgfpathlineto{\pgfqpoint{8.776797in}{4.706768in}}%
\pgfpathlineto{\pgfqpoint{8.792231in}{4.715648in}}%
\pgfpathlineto{\pgfqpoint{8.823098in}{4.738819in}}%
\pgfpathlineto{\pgfqpoint{8.853965in}{4.768570in}}%
\pgfpathlineto{\pgfqpoint{8.884833in}{4.803675in}}%
\pgfpathlineto{\pgfqpoint{8.931133in}{4.862621in}}%
\pgfpathlineto{\pgfqpoint{8.992868in}{4.942319in}}%
\pgfpathlineto{\pgfqpoint{9.023735in}{4.978190in}}%
\pgfpathlineto{\pgfqpoint{9.054603in}{5.009304in}}%
\pgfpathlineto{\pgfqpoint{9.085470in}{5.034688in}}%
\pgfpathlineto{\pgfqpoint{9.116337in}{5.053956in}}%
\pgfpathlineto{\pgfqpoint{9.147204in}{5.067301in}}%
\pgfpathlineto{\pgfqpoint{9.178072in}{5.075434in}}%
\pgfpathlineto{\pgfqpoint{9.208939in}{5.079466in}}%
\pgfpathlineto{\pgfqpoint{9.255240in}{5.080826in}}%
\pgfpathlineto{\pgfqpoint{9.316974in}{5.081260in}}%
\pgfpathlineto{\pgfqpoint{9.347842in}{5.083545in}}%
\pgfpathlineto{\pgfqpoint{9.378709in}{5.088243in}}%
\pgfpathlineto{\pgfqpoint{9.409576in}{5.095676in}}%
\pgfpathlineto{\pgfqpoint{9.440443in}{5.105799in}}%
\pgfpathlineto{\pgfqpoint{9.486744in}{5.125145in}}%
\pgfpathlineto{\pgfqpoint{9.625647in}{5.188862in}}%
\pgfpathlineto{\pgfqpoint{9.656514in}{5.198909in}}%
\pgfpathlineto{\pgfqpoint{9.687382in}{5.206189in}}%
\pgfpathlineto{\pgfqpoint{9.718249in}{5.210518in}}%
\pgfpathlineto{\pgfqpoint{9.733682in}{5.211588in}}%
\pgfpathlineto{\pgfqpoint{9.733682in}{5.211588in}}%
\pgfusepath{stroke}%
\end{pgfscope}%
\begin{pgfscope}%
\pgfpathrectangle{\pgfqpoint{5.698559in}{3.799602in}}{\pgfqpoint{4.227273in}{2.745455in}} %
\pgfusepath{clip}%
\pgfsetrectcap%
\pgfsetroundjoin%
\pgfsetlinewidth{0.501875pt}%
\definecolor{currentstroke}{rgb}{1.000000,0.587785,0.309017}%
\pgfsetstrokecolor{currentstroke}%
\pgfsetdash{}{0pt}%
\pgfpathmoveto{\pgfqpoint{5.890707in}{5.425893in}}%
\pgfpathlineto{\pgfqpoint{5.906141in}{5.423909in}}%
\pgfpathlineto{\pgfqpoint{5.921575in}{5.419251in}}%
\pgfpathlineto{\pgfqpoint{5.937008in}{5.411866in}}%
\pgfpathlineto{\pgfqpoint{5.952442in}{5.401743in}}%
\pgfpathlineto{\pgfqpoint{5.967875in}{5.388922in}}%
\pgfpathlineto{\pgfqpoint{5.983309in}{5.373491in}}%
\pgfpathlineto{\pgfqpoint{6.014176in}{5.335398in}}%
\pgfpathlineto{\pgfqpoint{6.045044in}{5.289127in}}%
\pgfpathlineto{\pgfqpoint{6.091345in}{5.209683in}}%
\pgfpathlineto{\pgfqpoint{6.153079in}{5.101325in}}%
\pgfpathlineto{\pgfqpoint{6.183946in}{5.053921in}}%
\pgfpathlineto{\pgfqpoint{6.199380in}{5.033447in}}%
\pgfpathlineto{\pgfqpoint{6.214814in}{5.015674in}}%
\pgfpathlineto{\pgfqpoint{6.230247in}{5.000982in}}%
\pgfpathlineto{\pgfqpoint{6.245681in}{4.989717in}}%
\pgfpathlineto{\pgfqpoint{6.261115in}{4.982187in}}%
\pgfpathlineto{\pgfqpoint{6.276548in}{4.978652in}}%
\pgfpathlineto{\pgfqpoint{6.291982in}{4.979323in}}%
\pgfpathlineto{\pgfqpoint{6.307415in}{4.984356in}}%
\pgfpathlineto{\pgfqpoint{6.322849in}{4.993845in}}%
\pgfpathlineto{\pgfqpoint{6.338283in}{5.007823in}}%
\pgfpathlineto{\pgfqpoint{6.353716in}{5.026258in}}%
\pgfpathlineto{\pgfqpoint{6.369150in}{5.049050in}}%
\pgfpathlineto{\pgfqpoint{6.384584in}{5.076034in}}%
\pgfpathlineto{\pgfqpoint{6.400017in}{5.106979in}}%
\pgfpathlineto{\pgfqpoint{6.430885in}{5.179516in}}%
\pgfpathlineto{\pgfqpoint{6.461752in}{5.263624in}}%
\pgfpathlineto{\pgfqpoint{6.508053in}{5.402980in}}%
\pgfpathlineto{\pgfqpoint{6.569787in}{5.589934in}}%
\pgfpathlineto{\pgfqpoint{6.600655in}{5.673711in}}%
\pgfpathlineto{\pgfqpoint{6.631522in}{5.745630in}}%
\pgfpathlineto{\pgfqpoint{6.646955in}{5.776116in}}%
\pgfpathlineto{\pgfqpoint{6.662389in}{5.802517in}}%
\pgfpathlineto{\pgfqpoint{6.677823in}{5.824580in}}%
\pgfpathlineto{\pgfqpoint{6.693256in}{5.842120in}}%
\pgfpathlineto{\pgfqpoint{6.708690in}{5.855013in}}%
\pgfpathlineto{\pgfqpoint{6.724124in}{5.863201in}}%
\pgfpathlineto{\pgfqpoint{6.739557in}{5.866688in}}%
\pgfpathlineto{\pgfqpoint{6.754991in}{5.865538in}}%
\pgfpathlineto{\pgfqpoint{6.770424in}{5.859869in}}%
\pgfpathlineto{\pgfqpoint{6.785858in}{5.849849in}}%
\pgfpathlineto{\pgfqpoint{6.801292in}{5.835693in}}%
\pgfpathlineto{\pgfqpoint{6.816725in}{5.817652in}}%
\pgfpathlineto{\pgfqpoint{6.832159in}{5.796012in}}%
\pgfpathlineto{\pgfqpoint{6.847593in}{5.771083in}}%
\pgfpathlineto{\pgfqpoint{6.878460in}{5.712703in}}%
\pgfpathlineto{\pgfqpoint{6.909327in}{5.645304in}}%
\pgfpathlineto{\pgfqpoint{6.955628in}{5.533548in}}%
\pgfpathlineto{\pgfqpoint{7.048230in}{5.304834in}}%
\pgfpathlineto{\pgfqpoint{7.079097in}{5.235621in}}%
\pgfpathlineto{\pgfqpoint{7.109964in}{5.173058in}}%
\pgfpathlineto{\pgfqpoint{7.140832in}{5.118514in}}%
\pgfpathlineto{\pgfqpoint{7.171699in}{5.072947in}}%
\pgfpathlineto{\pgfqpoint{7.202566in}{5.036841in}}%
\pgfpathlineto{\pgfqpoint{7.218000in}{5.022346in}}%
\pgfpathlineto{\pgfqpoint{7.233434in}{5.010162in}}%
\pgfpathlineto{\pgfqpoint{7.248867in}{5.000196in}}%
\pgfpathlineto{\pgfqpoint{7.264301in}{4.992324in}}%
\pgfpathlineto{\pgfqpoint{7.279734in}{4.986388in}}%
\pgfpathlineto{\pgfqpoint{7.310602in}{4.979546in}}%
\pgfpathlineto{\pgfqpoint{7.341469in}{4.977854in}}%
\pgfpathlineto{\pgfqpoint{7.387770in}{4.980307in}}%
\pgfpathlineto{\pgfqpoint{7.434071in}{4.982153in}}%
\pgfpathlineto{\pgfqpoint{7.464938in}{4.979808in}}%
\pgfpathlineto{\pgfqpoint{7.495805in}{4.973002in}}%
\pgfpathlineto{\pgfqpoint{7.526673in}{4.961109in}}%
\pgfpathlineto{\pgfqpoint{7.557540in}{4.944270in}}%
\pgfpathlineto{\pgfqpoint{7.588407in}{4.923406in}}%
\pgfpathlineto{\pgfqpoint{7.681009in}{4.855497in}}%
\pgfpathlineto{\pgfqpoint{7.711876in}{4.839193in}}%
\pgfpathlineto{\pgfqpoint{7.727310in}{4.833621in}}%
\pgfpathlineto{\pgfqpoint{7.742744in}{4.830119in}}%
\pgfpathlineto{\pgfqpoint{7.758177in}{4.828904in}}%
\pgfpathlineto{\pgfqpoint{7.773611in}{4.830148in}}%
\pgfpathlineto{\pgfqpoint{7.789044in}{4.833972in}}%
\pgfpathlineto{\pgfqpoint{7.804478in}{4.840442in}}%
\pgfpathlineto{\pgfqpoint{7.819912in}{4.849568in}}%
\pgfpathlineto{\pgfqpoint{7.835345in}{4.861299in}}%
\pgfpathlineto{\pgfqpoint{7.850779in}{4.875524in}}%
\pgfpathlineto{\pgfqpoint{7.881646in}{4.910731in}}%
\pgfpathlineto{\pgfqpoint{7.912514in}{4.953215in}}%
\pgfpathlineto{\pgfqpoint{7.958814in}{5.024577in}}%
\pgfpathlineto{\pgfqpoint{8.005115in}{5.095391in}}%
\pgfpathlineto{\pgfqpoint{8.035983in}{5.136806in}}%
\pgfpathlineto{\pgfqpoint{8.066850in}{5.170187in}}%
\pgfpathlineto{\pgfqpoint{8.082284in}{5.183140in}}%
\pgfpathlineto{\pgfqpoint{8.097717in}{5.193311in}}%
\pgfpathlineto{\pgfqpoint{8.113151in}{5.200550in}}%
\pgfpathlineto{\pgfqpoint{8.128584in}{5.204763in}}%
\pgfpathlineto{\pgfqpoint{8.144018in}{5.205915in}}%
\pgfpathlineto{\pgfqpoint{8.159452in}{5.204026in}}%
\pgfpathlineto{\pgfqpoint{8.174885in}{5.199170in}}%
\pgfpathlineto{\pgfqpoint{8.190319in}{5.191471in}}%
\pgfpathlineto{\pgfqpoint{8.205753in}{5.181097in}}%
\pgfpathlineto{\pgfqpoint{8.221186in}{5.168257in}}%
\pgfpathlineto{\pgfqpoint{8.252053in}{5.136160in}}%
\pgfpathlineto{\pgfqpoint{8.282921in}{5.097345in}}%
\pgfpathlineto{\pgfqpoint{8.329222in}{5.031573in}}%
\pgfpathlineto{\pgfqpoint{8.406390in}{4.918935in}}%
\pgfpathlineto{\pgfqpoint{8.452691in}{4.857590in}}%
\pgfpathlineto{\pgfqpoint{8.483558in}{4.820968in}}%
\pgfpathlineto{\pgfqpoint{8.514425in}{4.788163in}}%
\pgfpathlineto{\pgfqpoint{8.545293in}{4.759364in}}%
\pgfpathlineto{\pgfqpoint{8.576160in}{4.734764in}}%
\pgfpathlineto{\pgfqpoint{8.607027in}{4.714660in}}%
\pgfpathlineto{\pgfqpoint{8.637894in}{4.699494in}}%
\pgfpathlineto{\pgfqpoint{8.668762in}{4.689844in}}%
\pgfpathlineto{\pgfqpoint{8.684195in}{4.687286in}}%
\pgfpathlineto{\pgfqpoint{8.699629in}{4.686345in}}%
\pgfpathlineto{\pgfqpoint{8.715063in}{4.687090in}}%
\pgfpathlineto{\pgfqpoint{8.730496in}{4.689577in}}%
\pgfpathlineto{\pgfqpoint{8.745930in}{4.693848in}}%
\pgfpathlineto{\pgfqpoint{8.761363in}{4.699923in}}%
\pgfpathlineto{\pgfqpoint{8.776797in}{4.707798in}}%
\pgfpathlineto{\pgfqpoint{8.807664in}{4.728792in}}%
\pgfpathlineto{\pgfqpoint{8.838532in}{4.756226in}}%
\pgfpathlineto{\pgfqpoint{8.869399in}{4.789025in}}%
\pgfpathlineto{\pgfqpoint{8.915700in}{4.844924in}}%
\pgfpathlineto{\pgfqpoint{8.992868in}{4.940416in}}%
\pgfpathlineto{\pgfqpoint{9.023735in}{4.974098in}}%
\pgfpathlineto{\pgfqpoint{9.054603in}{5.003071in}}%
\pgfpathlineto{\pgfqpoint{9.085470in}{5.026552in}}%
\pgfpathlineto{\pgfqpoint{9.116337in}{5.044300in}}%
\pgfpathlineto{\pgfqpoint{9.147204in}{5.056606in}}%
\pgfpathlineto{\pgfqpoint{9.178072in}{5.064215in}}%
\pgfpathlineto{\pgfqpoint{9.208939in}{5.068213in}}%
\pgfpathlineto{\pgfqpoint{9.255240in}{5.070256in}}%
\pgfpathlineto{\pgfqpoint{9.316974in}{5.072275in}}%
\pgfpathlineto{\pgfqpoint{9.347842in}{5.075309in}}%
\pgfpathlineto{\pgfqpoint{9.378709in}{5.080552in}}%
\pgfpathlineto{\pgfqpoint{9.409576in}{5.088232in}}%
\pgfpathlineto{\pgfqpoint{9.455877in}{5.104027in}}%
\pgfpathlineto{\pgfqpoint{9.502178in}{5.123561in}}%
\pgfpathlineto{\pgfqpoint{9.594780in}{5.163978in}}%
\pgfpathlineto{\pgfqpoint{9.641081in}{5.179659in}}%
\pgfpathlineto{\pgfqpoint{9.671948in}{5.187060in}}%
\pgfpathlineto{\pgfqpoint{9.702815in}{5.191661in}}%
\pgfpathlineto{\pgfqpoint{9.733682in}{5.193401in}}%
\pgfpathlineto{\pgfqpoint{9.733682in}{5.193401in}}%
\pgfusepath{stroke}%
\end{pgfscope}%
\begin{pgfscope}%
\pgfpathrectangle{\pgfqpoint{5.698559in}{3.799602in}}{\pgfqpoint{4.227273in}{2.745455in}} %
\pgfusepath{clip}%
\pgfsetrectcap%
\pgfsetroundjoin%
\pgfsetlinewidth{0.501875pt}%
\definecolor{currentstroke}{rgb}{1.000000,0.473094,0.243914}%
\pgfsetstrokecolor{currentstroke}%
\pgfsetdash{}{0pt}%
\pgfpathmoveto{\pgfqpoint{5.890707in}{5.425631in}}%
\pgfpathlineto{\pgfqpoint{5.906141in}{5.422892in}}%
\pgfpathlineto{\pgfqpoint{5.921575in}{5.417488in}}%
\pgfpathlineto{\pgfqpoint{5.937008in}{5.409375in}}%
\pgfpathlineto{\pgfqpoint{5.952442in}{5.398554in}}%
\pgfpathlineto{\pgfqpoint{5.967875in}{5.385078in}}%
\pgfpathlineto{\pgfqpoint{5.983309in}{5.369050in}}%
\pgfpathlineto{\pgfqpoint{6.014176in}{5.329990in}}%
\pgfpathlineto{\pgfqpoint{6.045044in}{5.283128in}}%
\pgfpathlineto{\pgfqpoint{6.091345in}{5.203585in}}%
\pgfpathlineto{\pgfqpoint{6.153079in}{5.096405in}}%
\pgfpathlineto{\pgfqpoint{6.183946in}{5.049976in}}%
\pgfpathlineto{\pgfqpoint{6.199380in}{5.030046in}}%
\pgfpathlineto{\pgfqpoint{6.214814in}{5.012843in}}%
\pgfpathlineto{\pgfqpoint{6.230247in}{4.998737in}}%
\pgfpathlineto{\pgfqpoint{6.245681in}{4.988067in}}%
\pgfpathlineto{\pgfqpoint{6.261115in}{4.981135in}}%
\pgfpathlineto{\pgfqpoint{6.276548in}{4.978195in}}%
\pgfpathlineto{\pgfqpoint{6.291982in}{4.979451in}}%
\pgfpathlineto{\pgfqpoint{6.307415in}{4.985050in}}%
\pgfpathlineto{\pgfqpoint{6.322849in}{4.995079in}}%
\pgfpathlineto{\pgfqpoint{6.338283in}{5.009559in}}%
\pgfpathlineto{\pgfqpoint{6.353716in}{5.028446in}}%
\pgfpathlineto{\pgfqpoint{6.369150in}{5.051626in}}%
\pgfpathlineto{\pgfqpoint{6.384584in}{5.078921in}}%
\pgfpathlineto{\pgfqpoint{6.400017in}{5.110084in}}%
\pgfpathlineto{\pgfqpoint{6.430885in}{5.182729in}}%
\pgfpathlineto{\pgfqpoint{6.461752in}{5.266448in}}%
\pgfpathlineto{\pgfqpoint{6.508053in}{5.404301in}}%
\pgfpathlineto{\pgfqpoint{6.569787in}{5.588264in}}%
\pgfpathlineto{\pgfqpoint{6.600655in}{5.670692in}}%
\pgfpathlineto{\pgfqpoint{6.631522in}{5.741749in}}%
\pgfpathlineto{\pgfqpoint{6.646955in}{5.772066in}}%
\pgfpathlineto{\pgfqpoint{6.662389in}{5.798500in}}%
\pgfpathlineto{\pgfqpoint{6.677823in}{5.820813in}}%
\pgfpathlineto{\pgfqpoint{6.693256in}{5.838821in}}%
\pgfpathlineto{\pgfqpoint{6.708690in}{5.852397in}}%
\pgfpathlineto{\pgfqpoint{6.724124in}{5.861470in}}%
\pgfpathlineto{\pgfqpoint{6.739557in}{5.866021in}}%
\pgfpathlineto{\pgfqpoint{6.754991in}{5.866088in}}%
\pgfpathlineto{\pgfqpoint{6.770424in}{5.861757in}}%
\pgfpathlineto{\pgfqpoint{6.785858in}{5.853162in}}%
\pgfpathlineto{\pgfqpoint{6.801292in}{5.840482in}}%
\pgfpathlineto{\pgfqpoint{6.816725in}{5.823932in}}%
\pgfpathlineto{\pgfqpoint{6.832159in}{5.803765in}}%
\pgfpathlineto{\pgfqpoint{6.847593in}{5.780261in}}%
\pgfpathlineto{\pgfqpoint{6.878460in}{5.724484in}}%
\pgfpathlineto{\pgfqpoint{6.909327in}{5.659243in}}%
\pgfpathlineto{\pgfqpoint{6.955628in}{5.549713in}}%
\pgfpathlineto{\pgfqpoint{7.063664in}{5.285730in}}%
\pgfpathlineto{\pgfqpoint{7.094531in}{5.218263in}}%
\pgfpathlineto{\pgfqpoint{7.125398in}{5.157531in}}%
\pgfpathlineto{\pgfqpoint{7.156265in}{5.104739in}}%
\pgfpathlineto{\pgfqpoint{7.187133in}{5.060704in}}%
\pgfpathlineto{\pgfqpoint{7.218000in}{5.025799in}}%
\pgfpathlineto{\pgfqpoint{7.233434in}{5.011753in}}%
\pgfpathlineto{\pgfqpoint{7.248867in}{4.999901in}}%
\pgfpathlineto{\pgfqpoint{7.264301in}{4.990143in}}%
\pgfpathlineto{\pgfqpoint{7.279734in}{4.982346in}}%
\pgfpathlineto{\pgfqpoint{7.310602in}{4.971942in}}%
\pgfpathlineto{\pgfqpoint{7.341469in}{4.967024in}}%
\pgfpathlineto{\pgfqpoint{7.387770in}{4.965509in}}%
\pgfpathlineto{\pgfqpoint{7.434071in}{4.964665in}}%
\pgfpathlineto{\pgfqpoint{7.464938in}{4.961361in}}%
\pgfpathlineto{\pgfqpoint{7.495805in}{4.954365in}}%
\pgfpathlineto{\pgfqpoint{7.526673in}{4.943143in}}%
\pgfpathlineto{\pgfqpoint{7.557540in}{4.927906in}}%
\pgfpathlineto{\pgfqpoint{7.603841in}{4.899745in}}%
\pgfpathlineto{\pgfqpoint{7.665575in}{4.861590in}}%
\pgfpathlineto{\pgfqpoint{7.696443in}{4.847148in}}%
\pgfpathlineto{\pgfqpoint{7.711876in}{4.842048in}}%
\pgfpathlineto{\pgfqpoint{7.727310in}{4.838691in}}%
\pgfpathlineto{\pgfqpoint{7.742744in}{4.837289in}}%
\pgfpathlineto{\pgfqpoint{7.758177in}{4.838019in}}%
\pgfpathlineto{\pgfqpoint{7.773611in}{4.841021in}}%
\pgfpathlineto{\pgfqpoint{7.789044in}{4.846391in}}%
\pgfpathlineto{\pgfqpoint{7.804478in}{4.854175in}}%
\pgfpathlineto{\pgfqpoint{7.819912in}{4.864372in}}%
\pgfpathlineto{\pgfqpoint{7.835345in}{4.876926in}}%
\pgfpathlineto{\pgfqpoint{7.866213in}{4.908630in}}%
\pgfpathlineto{\pgfqpoint{7.897080in}{4.947814in}}%
\pgfpathlineto{\pgfqpoint{7.943381in}{5.015606in}}%
\pgfpathlineto{\pgfqpoint{8.005115in}{5.107238in}}%
\pgfpathlineto{\pgfqpoint{8.035983in}{5.146434in}}%
\pgfpathlineto{\pgfqpoint{8.051416in}{5.163099in}}%
\pgfpathlineto{\pgfqpoint{8.066850in}{5.177411in}}%
\pgfpathlineto{\pgfqpoint{8.082284in}{5.189127in}}%
\pgfpathlineto{\pgfqpoint{8.097717in}{5.198060in}}%
\pgfpathlineto{\pgfqpoint{8.113151in}{5.204078in}}%
\pgfpathlineto{\pgfqpoint{8.128584in}{5.207110in}}%
\pgfpathlineto{\pgfqpoint{8.144018in}{5.207140in}}%
\pgfpathlineto{\pgfqpoint{8.159452in}{5.204210in}}%
\pgfpathlineto{\pgfqpoint{8.174885in}{5.198417in}}%
\pgfpathlineto{\pgfqpoint{8.190319in}{5.189902in}}%
\pgfpathlineto{\pgfqpoint{8.205753in}{5.178852in}}%
\pgfpathlineto{\pgfqpoint{8.221186in}{5.165485in}}%
\pgfpathlineto{\pgfqpoint{8.252053in}{5.132806in}}%
\pgfpathlineto{\pgfqpoint{8.282921in}{5.094014in}}%
\pgfpathlineto{\pgfqpoint{8.329222in}{5.029146in}}%
\pgfpathlineto{\pgfqpoint{8.406390in}{4.918718in}}%
\pgfpathlineto{\pgfqpoint{8.452691in}{4.858229in}}%
\pgfpathlineto{\pgfqpoint{8.483558in}{4.821884in}}%
\pgfpathlineto{\pgfqpoint{8.514425in}{4.789202in}}%
\pgfpathlineto{\pgfqpoint{8.545293in}{4.760481in}}%
\pgfpathlineto{\pgfqpoint{8.576160in}{4.736010in}}%
\pgfpathlineto{\pgfqpoint{8.607027in}{4.716127in}}%
\pgfpathlineto{\pgfqpoint{8.637894in}{4.701251in}}%
\pgfpathlineto{\pgfqpoint{8.668762in}{4.691873in}}%
\pgfpathlineto{\pgfqpoint{8.684195in}{4.689411in}}%
\pgfpathlineto{\pgfqpoint{8.699629in}{4.688519in}}%
\pgfpathlineto{\pgfqpoint{8.715063in}{4.689257in}}%
\pgfpathlineto{\pgfqpoint{8.730496in}{4.691674in}}%
\pgfpathlineto{\pgfqpoint{8.745930in}{4.695810in}}%
\pgfpathlineto{\pgfqpoint{8.761363in}{4.701688in}}%
\pgfpathlineto{\pgfqpoint{8.776797in}{4.709312in}}%
\pgfpathlineto{\pgfqpoint{8.807664in}{4.729702in}}%
\pgfpathlineto{\pgfqpoint{8.838532in}{4.756534in}}%
\pgfpathlineto{\pgfqpoint{8.869399in}{4.788922in}}%
\pgfpathlineto{\pgfqpoint{8.900266in}{4.825541in}}%
\pgfpathlineto{\pgfqpoint{9.023735in}{4.978275in}}%
\pgfpathlineto{\pgfqpoint{9.054603in}{5.008967in}}%
\pgfpathlineto{\pgfqpoint{9.085470in}{5.034052in}}%
\pgfpathlineto{\pgfqpoint{9.116337in}{5.053110in}}%
\pgfpathlineto{\pgfqpoint{9.147204in}{5.066291in}}%
\pgfpathlineto{\pgfqpoint{9.178072in}{5.074248in}}%
\pgfpathlineto{\pgfqpoint{9.208939in}{5.078015in}}%
\pgfpathlineto{\pgfqpoint{9.255240in}{5.078617in}}%
\pgfpathlineto{\pgfqpoint{9.332408in}{5.077044in}}%
\pgfpathlineto{\pgfqpoint{9.363275in}{5.078700in}}%
\pgfpathlineto{\pgfqpoint{9.394143in}{5.082742in}}%
\pgfpathlineto{\pgfqpoint{9.425010in}{5.089476in}}%
\pgfpathlineto{\pgfqpoint{9.455877in}{5.098892in}}%
\pgfpathlineto{\pgfqpoint{9.502178in}{5.117274in}}%
\pgfpathlineto{\pgfqpoint{9.641081in}{5.178575in}}%
\pgfpathlineto{\pgfqpoint{9.671948in}{5.187682in}}%
\pgfpathlineto{\pgfqpoint{9.702815in}{5.193720in}}%
\pgfpathlineto{\pgfqpoint{9.733682in}{5.196529in}}%
\pgfpathlineto{\pgfqpoint{9.733682in}{5.196529in}}%
\pgfusepath{stroke}%
\end{pgfscope}%
\begin{pgfscope}%
\pgfpathrectangle{\pgfqpoint{5.698559in}{3.799602in}}{\pgfqpoint{4.227273in}{2.745455in}} %
\pgfusepath{clip}%
\pgfsetrectcap%
\pgfsetroundjoin%
\pgfsetlinewidth{0.501875pt}%
\definecolor{currentstroke}{rgb}{1.000000,0.361242,0.183750}%
\pgfsetstrokecolor{currentstroke}%
\pgfsetdash{}{0pt}%
\pgfpathmoveto{\pgfqpoint{5.890707in}{5.434278in}}%
\pgfpathlineto{\pgfqpoint{5.906141in}{5.431511in}}%
\pgfpathlineto{\pgfqpoint{5.921575in}{5.426000in}}%
\pgfpathlineto{\pgfqpoint{5.937008in}{5.417701in}}%
\pgfpathlineto{\pgfqpoint{5.952442in}{5.406620in}}%
\pgfpathlineto{\pgfqpoint{5.967875in}{5.392815in}}%
\pgfpathlineto{\pgfqpoint{5.983309in}{5.376391in}}%
\pgfpathlineto{\pgfqpoint{6.014176in}{5.336351in}}%
\pgfpathlineto{\pgfqpoint{6.045044in}{5.288272in}}%
\pgfpathlineto{\pgfqpoint{6.091345in}{5.206499in}}%
\pgfpathlineto{\pgfqpoint{6.153079in}{5.095722in}}%
\pgfpathlineto{\pgfqpoint{6.183946in}{5.047334in}}%
\pgfpathlineto{\pgfqpoint{6.199380in}{5.026417in}}%
\pgfpathlineto{\pgfqpoint{6.214814in}{5.008236in}}%
\pgfpathlineto{\pgfqpoint{6.230247in}{4.993176in}}%
\pgfpathlineto{\pgfqpoint{6.245681in}{4.981590in}}%
\pgfpathlineto{\pgfqpoint{6.261115in}{4.973790in}}%
\pgfpathlineto{\pgfqpoint{6.276548in}{4.970044in}}%
\pgfpathlineto{\pgfqpoint{6.291982in}{4.970570in}}%
\pgfpathlineto{\pgfqpoint{6.307415in}{4.975525in}}%
\pgfpathlineto{\pgfqpoint{6.322849in}{4.985006in}}%
\pgfpathlineto{\pgfqpoint{6.338283in}{4.999044in}}%
\pgfpathlineto{\pgfqpoint{6.353716in}{5.017600in}}%
\pgfpathlineto{\pgfqpoint{6.369150in}{5.040566in}}%
\pgfpathlineto{\pgfqpoint{6.384584in}{5.067765in}}%
\pgfpathlineto{\pgfqpoint{6.400017in}{5.098952in}}%
\pgfpathlineto{\pgfqpoint{6.430885in}{5.171990in}}%
\pgfpathlineto{\pgfqpoint{6.461752in}{5.256525in}}%
\pgfpathlineto{\pgfqpoint{6.508053in}{5.396223in}}%
\pgfpathlineto{\pgfqpoint{6.569787in}{5.583215in}}%
\pgfpathlineto{\pgfqpoint{6.600655in}{5.667109in}}%
\pgfpathlineto{\pgfqpoint{6.631522in}{5.739425in}}%
\pgfpathlineto{\pgfqpoint{6.646955in}{5.770259in}}%
\pgfpathlineto{\pgfqpoint{6.662389in}{5.797119in}}%
\pgfpathlineto{\pgfqpoint{6.677823in}{5.819760in}}%
\pgfpathlineto{\pgfqpoint{6.693256in}{5.837990in}}%
\pgfpathlineto{\pgfqpoint{6.708690in}{5.851678in}}%
\pgfpathlineto{\pgfqpoint{6.724124in}{5.860753in}}%
\pgfpathlineto{\pgfqpoint{6.739557in}{5.865199in}}%
\pgfpathlineto{\pgfqpoint{6.754991in}{5.865056in}}%
\pgfpathlineto{\pgfqpoint{6.770424in}{5.860420in}}%
\pgfpathlineto{\pgfqpoint{6.785858in}{5.851432in}}%
\pgfpathlineto{\pgfqpoint{6.801292in}{5.838281in}}%
\pgfpathlineto{\pgfqpoint{6.816725in}{5.821197in}}%
\pgfpathlineto{\pgfqpoint{6.832159in}{5.800443in}}%
\pgfpathlineto{\pgfqpoint{6.847593in}{5.776315in}}%
\pgfpathlineto{\pgfqpoint{6.878460in}{5.719229in}}%
\pgfpathlineto{\pgfqpoint{6.909327in}{5.652677in}}%
\pgfpathlineto{\pgfqpoint{6.955628in}{5.541359in}}%
\pgfpathlineto{\pgfqpoint{7.048230in}{5.310860in}}%
\pgfpathlineto{\pgfqpoint{7.079097in}{5.240318in}}%
\pgfpathlineto{\pgfqpoint{7.109964in}{5.176073in}}%
\pgfpathlineto{\pgfqpoint{7.140832in}{5.119534in}}%
\pgfpathlineto{\pgfqpoint{7.171699in}{5.071736in}}%
\pgfpathlineto{\pgfqpoint{7.202566in}{5.033283in}}%
\pgfpathlineto{\pgfqpoint{7.218000in}{5.017615in}}%
\pgfpathlineto{\pgfqpoint{7.233434in}{5.004281in}}%
\pgfpathlineto{\pgfqpoint{7.248867in}{4.993207in}}%
\pgfpathlineto{\pgfqpoint{7.264301in}{4.984283in}}%
\pgfpathlineto{\pgfqpoint{7.279734in}{4.977364in}}%
\pgfpathlineto{\pgfqpoint{7.310602in}{4.968794in}}%
\pgfpathlineto{\pgfqpoint{7.341469in}{4.965719in}}%
\pgfpathlineto{\pgfqpoint{7.387770in}{4.966728in}}%
\pgfpathlineto{\pgfqpoint{7.434071in}{4.967827in}}%
\pgfpathlineto{\pgfqpoint{7.464938in}{4.965391in}}%
\pgfpathlineto{\pgfqpoint{7.495805in}{4.958898in}}%
\pgfpathlineto{\pgfqpoint{7.526673in}{4.947821in}}%
\pgfpathlineto{\pgfqpoint{7.557540in}{4.932403in}}%
\pgfpathlineto{\pgfqpoint{7.603841in}{4.903463in}}%
\pgfpathlineto{\pgfqpoint{7.665575in}{4.863634in}}%
\pgfpathlineto{\pgfqpoint{7.696443in}{4.848247in}}%
\pgfpathlineto{\pgfqpoint{7.711876in}{4.842686in}}%
\pgfpathlineto{\pgfqpoint{7.727310in}{4.838892in}}%
\pgfpathlineto{\pgfqpoint{7.742744in}{4.837089in}}%
\pgfpathlineto{\pgfqpoint{7.758177in}{4.837470in}}%
\pgfpathlineto{\pgfqpoint{7.773611in}{4.840185in}}%
\pgfpathlineto{\pgfqpoint{7.789044in}{4.845343in}}%
\pgfpathlineto{\pgfqpoint{7.804478in}{4.853002in}}%
\pgfpathlineto{\pgfqpoint{7.819912in}{4.863170in}}%
\pgfpathlineto{\pgfqpoint{7.835345in}{4.875799in}}%
\pgfpathlineto{\pgfqpoint{7.850779in}{4.890788in}}%
\pgfpathlineto{\pgfqpoint{7.881646in}{4.927159in}}%
\pgfpathlineto{\pgfqpoint{7.912514in}{4.970399in}}%
\pgfpathlineto{\pgfqpoint{8.020549in}{5.134158in}}%
\pgfpathlineto{\pgfqpoint{8.051416in}{5.171052in}}%
\pgfpathlineto{\pgfqpoint{8.066850in}{5.186008in}}%
\pgfpathlineto{\pgfqpoint{8.082284in}{5.198284in}}%
\pgfpathlineto{\pgfqpoint{8.097717in}{5.207683in}}%
\pgfpathlineto{\pgfqpoint{8.113151in}{5.214069in}}%
\pgfpathlineto{\pgfqpoint{8.128584in}{5.217368in}}%
\pgfpathlineto{\pgfqpoint{8.144018in}{5.217564in}}%
\pgfpathlineto{\pgfqpoint{8.159452in}{5.214703in}}%
\pgfpathlineto{\pgfqpoint{8.174885in}{5.208885in}}%
\pgfpathlineto{\pgfqpoint{8.190319in}{5.200259in}}%
\pgfpathlineto{\pgfqpoint{8.205753in}{5.189017in}}%
\pgfpathlineto{\pgfqpoint{8.221186in}{5.175384in}}%
\pgfpathlineto{\pgfqpoint{8.252053in}{5.141981in}}%
\pgfpathlineto{\pgfqpoint{8.282921in}{5.102245in}}%
\pgfpathlineto{\pgfqpoint{8.329222in}{5.035632in}}%
\pgfpathlineto{\pgfqpoint{8.421823in}{4.900112in}}%
\pgfpathlineto{\pgfqpoint{8.468124in}{4.839813in}}%
\pgfpathlineto{\pgfqpoint{8.498992in}{4.804114in}}%
\pgfpathlineto{\pgfqpoint{8.529859in}{4.772544in}}%
\pgfpathlineto{\pgfqpoint{8.560726in}{4.745469in}}%
\pgfpathlineto{\pgfqpoint{8.591593in}{4.723244in}}%
\pgfpathlineto{\pgfqpoint{8.622461in}{4.706240in}}%
\pgfpathlineto{\pgfqpoint{8.653328in}{4.694859in}}%
\pgfpathlineto{\pgfqpoint{8.684195in}{4.689521in}}%
\pgfpathlineto{\pgfqpoint{8.699629in}{4.689245in}}%
\pgfpathlineto{\pgfqpoint{8.715063in}{4.690625in}}%
\pgfpathlineto{\pgfqpoint{8.730496in}{4.693696in}}%
\pgfpathlineto{\pgfqpoint{8.745930in}{4.698485in}}%
\pgfpathlineto{\pgfqpoint{8.761363in}{4.705002in}}%
\pgfpathlineto{\pgfqpoint{8.792231in}{4.723185in}}%
\pgfpathlineto{\pgfqpoint{8.823098in}{4.747950in}}%
\pgfpathlineto{\pgfqpoint{8.853965in}{4.778611in}}%
\pgfpathlineto{\pgfqpoint{8.884833in}{4.814047in}}%
\pgfpathlineto{\pgfqpoint{8.931133in}{4.872715in}}%
\pgfpathlineto{\pgfqpoint{8.992868in}{4.951102in}}%
\pgfpathlineto{\pgfqpoint{9.023735in}{4.985949in}}%
\pgfpathlineto{\pgfqpoint{9.054603in}{5.015731in}}%
\pgfpathlineto{\pgfqpoint{9.085470in}{5.039432in}}%
\pgfpathlineto{\pgfqpoint{9.116337in}{5.056651in}}%
\pgfpathlineto{\pgfqpoint{9.147204in}{5.067620in}}%
\pgfpathlineto{\pgfqpoint{9.178072in}{5.073128in}}%
\pgfpathlineto{\pgfqpoint{9.208939in}{5.074384in}}%
\pgfpathlineto{\pgfqpoint{9.255240in}{5.071484in}}%
\pgfpathlineto{\pgfqpoint{9.316974in}{5.066295in}}%
\pgfpathlineto{\pgfqpoint{9.347842in}{5.065814in}}%
\pgfpathlineto{\pgfqpoint{9.378709in}{5.067790in}}%
\pgfpathlineto{\pgfqpoint{9.409576in}{5.072592in}}%
\pgfpathlineto{\pgfqpoint{9.440443in}{5.080265in}}%
\pgfpathlineto{\pgfqpoint{9.471311in}{5.090563in}}%
\pgfpathlineto{\pgfqpoint{9.517612in}{5.109786in}}%
\pgfpathlineto{\pgfqpoint{9.625647in}{5.158389in}}%
\pgfpathlineto{\pgfqpoint{9.656514in}{5.169391in}}%
\pgfpathlineto{\pgfqpoint{9.687382in}{5.177858in}}%
\pgfpathlineto{\pgfqpoint{9.718249in}{5.183460in}}%
\pgfpathlineto{\pgfqpoint{9.733682in}{5.185156in}}%
\pgfpathlineto{\pgfqpoint{9.733682in}{5.185156in}}%
\pgfusepath{stroke}%
\end{pgfscope}%
\begin{pgfscope}%
\pgfpathrectangle{\pgfqpoint{5.698559in}{3.799602in}}{\pgfqpoint{4.227273in}{2.745455in}} %
\pgfusepath{clip}%
\pgfsetrectcap%
\pgfsetroundjoin%
\pgfsetlinewidth{0.501875pt}%
\definecolor{currentstroke}{rgb}{1.000000,0.243914,0.122888}%
\pgfsetstrokecolor{currentstroke}%
\pgfsetdash{}{0pt}%
\pgfpathmoveto{\pgfqpoint{5.890707in}{5.432949in}}%
\pgfpathlineto{\pgfqpoint{5.906141in}{5.430041in}}%
\pgfpathlineto{\pgfqpoint{5.921575in}{5.424506in}}%
\pgfpathlineto{\pgfqpoint{5.937008in}{5.416299in}}%
\pgfpathlineto{\pgfqpoint{5.952442in}{5.405424in}}%
\pgfpathlineto{\pgfqpoint{5.967875in}{5.391928in}}%
\pgfpathlineto{\pgfqpoint{5.983309in}{5.375906in}}%
\pgfpathlineto{\pgfqpoint{6.014176in}{5.336896in}}%
\pgfpathlineto{\pgfqpoint{6.045044in}{5.290049in}}%
\pgfpathlineto{\pgfqpoint{6.091345in}{5.210246in}}%
\pgfpathlineto{\pgfqpoint{6.153079in}{5.101852in}}%
\pgfpathlineto{\pgfqpoint{6.183946in}{5.054422in}}%
\pgfpathlineto{\pgfqpoint{6.199380in}{5.033912in}}%
\pgfpathlineto{\pgfqpoint{6.214814in}{5.016086in}}%
\pgfpathlineto{\pgfqpoint{6.230247in}{5.001327in}}%
\pgfpathlineto{\pgfqpoint{6.245681in}{4.989985in}}%
\pgfpathlineto{\pgfqpoint{6.261115in}{4.982372in}}%
\pgfpathlineto{\pgfqpoint{6.276548in}{4.978755in}}%
\pgfpathlineto{\pgfqpoint{6.291982in}{4.979350in}}%
\pgfpathlineto{\pgfqpoint{6.307415in}{4.984312in}}%
\pgfpathlineto{\pgfqpoint{6.322849in}{4.993740in}}%
\pgfpathlineto{\pgfqpoint{6.338283in}{5.007662in}}%
\pgfpathlineto{\pgfqpoint{6.353716in}{5.026041in}}%
\pgfpathlineto{\pgfqpoint{6.369150in}{5.048770in}}%
\pgfpathlineto{\pgfqpoint{6.384584in}{5.075673in}}%
\pgfpathlineto{\pgfqpoint{6.400017in}{5.106507in}}%
\pgfpathlineto{\pgfqpoint{6.430885in}{5.178678in}}%
\pgfpathlineto{\pgfqpoint{6.461752in}{5.262148in}}%
\pgfpathlineto{\pgfqpoint{6.508053in}{5.399925in}}%
\pgfpathlineto{\pgfqpoint{6.569787in}{5.583880in}}%
\pgfpathlineto{\pgfqpoint{6.600655in}{5.666125in}}%
\pgfpathlineto{\pgfqpoint{6.631522in}{5.736779in}}%
\pgfpathlineto{\pgfqpoint{6.646955in}{5.766803in}}%
\pgfpathlineto{\pgfqpoint{6.662389in}{5.792883in}}%
\pgfpathlineto{\pgfqpoint{6.677823in}{5.814788in}}%
\pgfpathlineto{\pgfqpoint{6.693256in}{5.832342in}}%
\pgfpathlineto{\pgfqpoint{6.708690in}{5.845429in}}%
\pgfpathlineto{\pgfqpoint{6.724124in}{5.853992in}}%
\pgfpathlineto{\pgfqpoint{6.739557in}{5.858031in}}%
\pgfpathlineto{\pgfqpoint{6.754991in}{5.857599in}}%
\pgfpathlineto{\pgfqpoint{6.770424in}{5.852800in}}%
\pgfpathlineto{\pgfqpoint{6.785858in}{5.843785in}}%
\pgfpathlineto{\pgfqpoint{6.801292in}{5.830746in}}%
\pgfpathlineto{\pgfqpoint{6.816725in}{5.813913in}}%
\pgfpathlineto{\pgfqpoint{6.832159in}{5.793546in}}%
\pgfpathlineto{\pgfqpoint{6.847593in}{5.769932in}}%
\pgfpathlineto{\pgfqpoint{6.878460in}{5.714205in}}%
\pgfpathlineto{\pgfqpoint{6.909327in}{5.649332in}}%
\pgfpathlineto{\pgfqpoint{6.955628in}{5.540730in}}%
\pgfpathlineto{\pgfqpoint{7.063664in}{5.278404in}}%
\pgfpathlineto{\pgfqpoint{7.094531in}{5.210900in}}%
\pgfpathlineto{\pgfqpoint{7.125398in}{5.149915in}}%
\pgfpathlineto{\pgfqpoint{7.156265in}{5.096746in}}%
\pgfpathlineto{\pgfqpoint{7.187133in}{5.052330in}}%
\pgfpathlineto{\pgfqpoint{7.218000in}{5.017176in}}%
\pgfpathlineto{\pgfqpoint{7.233434in}{5.003092in}}%
\pgfpathlineto{\pgfqpoint{7.248867in}{4.991282in}}%
\pgfpathlineto{\pgfqpoint{7.264301in}{4.981655in}}%
\pgfpathlineto{\pgfqpoint{7.279734in}{4.974088in}}%
\pgfpathlineto{\pgfqpoint{7.295168in}{4.968423in}}%
\pgfpathlineto{\pgfqpoint{7.326035in}{4.962004in}}%
\pgfpathlineto{\pgfqpoint{7.356903in}{4.960545in}}%
\pgfpathlineto{\pgfqpoint{7.464938in}{4.963375in}}%
\pgfpathlineto{\pgfqpoint{7.495805in}{4.958610in}}%
\pgfpathlineto{\pgfqpoint{7.526673in}{4.949330in}}%
\pgfpathlineto{\pgfqpoint{7.557540in}{4.935583in}}%
\pgfpathlineto{\pgfqpoint{7.588407in}{4.918185in}}%
\pgfpathlineto{\pgfqpoint{7.681009in}{4.861580in}}%
\pgfpathlineto{\pgfqpoint{7.711876in}{4.848816in}}%
\pgfpathlineto{\pgfqpoint{7.727310in}{4.844882in}}%
\pgfpathlineto{\pgfqpoint{7.742744in}{4.842908in}}%
\pgfpathlineto{\pgfqpoint{7.758177in}{4.843100in}}%
\pgfpathlineto{\pgfqpoint{7.773611in}{4.845618in}}%
\pgfpathlineto{\pgfqpoint{7.789044in}{4.850578in}}%
\pgfpathlineto{\pgfqpoint{7.804478in}{4.858041in}}%
\pgfpathlineto{\pgfqpoint{7.819912in}{4.868016in}}%
\pgfpathlineto{\pgfqpoint{7.835345in}{4.880452in}}%
\pgfpathlineto{\pgfqpoint{7.850779in}{4.895239in}}%
\pgfpathlineto{\pgfqpoint{7.881646in}{4.931151in}}%
\pgfpathlineto{\pgfqpoint{7.912514in}{4.973782in}}%
\pgfpathlineto{\pgfqpoint{8.020549in}{5.133285in}}%
\pgfpathlineto{\pgfqpoint{8.051416in}{5.168355in}}%
\pgfpathlineto{\pgfqpoint{8.066850in}{5.182343in}}%
\pgfpathlineto{\pgfqpoint{8.082284in}{5.193635in}}%
\pgfpathlineto{\pgfqpoint{8.097717in}{5.202053in}}%
\pgfpathlineto{\pgfqpoint{8.113151in}{5.207478in}}%
\pgfpathlineto{\pgfqpoint{8.128584in}{5.209856in}}%
\pgfpathlineto{\pgfqpoint{8.144018in}{5.209191in}}%
\pgfpathlineto{\pgfqpoint{8.159452in}{5.205548in}}%
\pgfpathlineto{\pgfqpoint{8.174885in}{5.199045in}}%
\pgfpathlineto{\pgfqpoint{8.190319in}{5.189849in}}%
\pgfpathlineto{\pgfqpoint{8.205753in}{5.178168in}}%
\pgfpathlineto{\pgfqpoint{8.236620in}{5.148344in}}%
\pgfpathlineto{\pgfqpoint{8.267487in}{5.111758in}}%
\pgfpathlineto{\pgfqpoint{8.313788in}{5.049239in}}%
\pgfpathlineto{\pgfqpoint{8.406390in}{4.920937in}}%
\pgfpathlineto{\pgfqpoint{8.452691in}{4.863830in}}%
\pgfpathlineto{\pgfqpoint{8.483558in}{4.829776in}}%
\pgfpathlineto{\pgfqpoint{8.514425in}{4.799168in}}%
\pgfpathlineto{\pgfqpoint{8.545293in}{4.772113in}}%
\pgfpathlineto{\pgfqpoint{8.576160in}{4.748753in}}%
\pgfpathlineto{\pgfqpoint{8.607027in}{4.729336in}}%
\pgfpathlineto{\pgfqpoint{8.637894in}{4.714260in}}%
\pgfpathlineto{\pgfqpoint{8.668762in}{4.704065in}}%
\pgfpathlineto{\pgfqpoint{8.699629in}{4.699381in}}%
\pgfpathlineto{\pgfqpoint{8.715063in}{4.699307in}}%
\pgfpathlineto{\pgfqpoint{8.730496in}{4.700844in}}%
\pgfpathlineto{\pgfqpoint{8.745930in}{4.704051in}}%
\pgfpathlineto{\pgfqpoint{8.761363in}{4.708976in}}%
\pgfpathlineto{\pgfqpoint{8.776797in}{4.715643in}}%
\pgfpathlineto{\pgfqpoint{8.807664in}{4.734190in}}%
\pgfpathlineto{\pgfqpoint{8.838532in}{4.759385in}}%
\pgfpathlineto{\pgfqpoint{8.869399in}{4.790430in}}%
\pgfpathlineto{\pgfqpoint{8.900266in}{4.826039in}}%
\pgfpathlineto{\pgfqpoint{9.023735in}{4.976695in}}%
\pgfpathlineto{\pgfqpoint{9.054603in}{5.006707in}}%
\pgfpathlineto{\pgfqpoint{9.085470in}{5.030790in}}%
\pgfpathlineto{\pgfqpoint{9.116337in}{5.048472in}}%
\pgfpathlineto{\pgfqpoint{9.147204in}{5.059922in}}%
\pgfpathlineto{\pgfqpoint{9.178072in}{5.065890in}}%
\pgfpathlineto{\pgfqpoint{9.208939in}{5.067569in}}%
\pgfpathlineto{\pgfqpoint{9.255240in}{5.065261in}}%
\pgfpathlineto{\pgfqpoint{9.316974in}{5.060953in}}%
\pgfpathlineto{\pgfqpoint{9.347842in}{5.061014in}}%
\pgfpathlineto{\pgfqpoint{9.378709in}{5.063599in}}%
\pgfpathlineto{\pgfqpoint{9.409576in}{5.069033in}}%
\pgfpathlineto{\pgfqpoint{9.440443in}{5.077294in}}%
\pgfpathlineto{\pgfqpoint{9.486744in}{5.094193in}}%
\pgfpathlineto{\pgfqpoint{9.548479in}{5.121711in}}%
\pgfpathlineto{\pgfqpoint{9.610213in}{5.148954in}}%
\pgfpathlineto{\pgfqpoint{9.656514in}{5.165522in}}%
\pgfpathlineto{\pgfqpoint{9.687382in}{5.173717in}}%
\pgfpathlineto{\pgfqpoint{9.718249in}{5.179344in}}%
\pgfpathlineto{\pgfqpoint{9.733682in}{5.181189in}}%
\pgfpathlineto{\pgfqpoint{9.733682in}{5.181189in}}%
\pgfusepath{stroke}%
\end{pgfscope}%
\begin{pgfscope}%
\pgfpathrectangle{\pgfqpoint{5.698559in}{3.799602in}}{\pgfqpoint{4.227273in}{2.745455in}} %
\pgfusepath{clip}%
\pgfsetrectcap%
\pgfsetroundjoin%
\pgfsetlinewidth{0.501875pt}%
\definecolor{currentstroke}{rgb}{1.000000,0.122888,0.061561}%
\pgfsetstrokecolor{currentstroke}%
\pgfsetdash{}{0pt}%
\pgfpathmoveto{\pgfqpoint{5.890707in}{5.436237in}}%
\pgfpathlineto{\pgfqpoint{5.906141in}{5.433575in}}%
\pgfpathlineto{\pgfqpoint{5.921575in}{5.428322in}}%
\pgfpathlineto{\pgfqpoint{5.937008in}{5.420435in}}%
\pgfpathlineto{\pgfqpoint{5.952442in}{5.409914in}}%
\pgfpathlineto{\pgfqpoint{5.967875in}{5.396807in}}%
\pgfpathlineto{\pgfqpoint{5.983309in}{5.381205in}}%
\pgfpathlineto{\pgfqpoint{6.014176in}{5.343106in}}%
\pgfpathlineto{\pgfqpoint{6.045044in}{5.297219in}}%
\pgfpathlineto{\pgfqpoint{6.091345in}{5.218806in}}%
\pgfpathlineto{\pgfqpoint{6.153079in}{5.111780in}}%
\pgfpathlineto{\pgfqpoint{6.183946in}{5.064670in}}%
\pgfpathlineto{\pgfqpoint{6.199380in}{5.044204in}}%
\pgfpathlineto{\pgfqpoint{6.214814in}{5.026339in}}%
\pgfpathlineto{\pgfqpoint{6.230247in}{5.011453in}}%
\pgfpathlineto{\pgfqpoint{6.245681in}{4.999899in}}%
\pgfpathlineto{\pgfqpoint{6.261115in}{4.991987in}}%
\pgfpathlineto{\pgfqpoint{6.276548in}{4.987989in}}%
\pgfpathlineto{\pgfqpoint{6.291982in}{4.988125in}}%
\pgfpathlineto{\pgfqpoint{6.307415in}{4.992562in}}%
\pgfpathlineto{\pgfqpoint{6.322849in}{5.001404in}}%
\pgfpathlineto{\pgfqpoint{6.338283in}{5.014696in}}%
\pgfpathlineto{\pgfqpoint{6.353716in}{5.032412in}}%
\pgfpathlineto{\pgfqpoint{6.369150in}{5.054461in}}%
\pgfpathlineto{\pgfqpoint{6.384584in}{5.080682in}}%
\pgfpathlineto{\pgfqpoint{6.400017in}{5.110847in}}%
\pgfpathlineto{\pgfqpoint{6.430885in}{5.181775in}}%
\pgfpathlineto{\pgfqpoint{6.461752in}{5.264206in}}%
\pgfpathlineto{\pgfqpoint{6.508053in}{5.400922in}}%
\pgfpathlineto{\pgfqpoint{6.569787in}{5.584310in}}%
\pgfpathlineto{\pgfqpoint{6.600655in}{5.666473in}}%
\pgfpathlineto{\pgfqpoint{6.631522in}{5.737053in}}%
\pgfpathlineto{\pgfqpoint{6.646955in}{5.767012in}}%
\pgfpathlineto{\pgfqpoint{6.662389in}{5.793000in}}%
\pgfpathlineto{\pgfqpoint{6.677823in}{5.814778in}}%
\pgfpathlineto{\pgfqpoint{6.693256in}{5.832167in}}%
\pgfpathlineto{\pgfqpoint{6.708690in}{5.845051in}}%
\pgfpathlineto{\pgfqpoint{6.724124in}{5.853376in}}%
\pgfpathlineto{\pgfqpoint{6.739557in}{5.857144in}}%
\pgfpathlineto{\pgfqpoint{6.754991in}{5.856417in}}%
\pgfpathlineto{\pgfqpoint{6.770424in}{5.851306in}}%
\pgfpathlineto{\pgfqpoint{6.785858in}{5.841971in}}%
\pgfpathlineto{\pgfqpoint{6.801292in}{5.828613in}}%
\pgfpathlineto{\pgfqpoint{6.816725in}{5.811471in}}%
\pgfpathlineto{\pgfqpoint{6.832159in}{5.790815in}}%
\pgfpathlineto{\pgfqpoint{6.847593in}{5.766937in}}%
\pgfpathlineto{\pgfqpoint{6.878460in}{5.710785in}}%
\pgfpathlineto{\pgfqpoint{6.909327in}{5.645648in}}%
\pgfpathlineto{\pgfqpoint{6.955628in}{5.536972in}}%
\pgfpathlineto{\pgfqpoint{7.063664in}{5.275781in}}%
\pgfpathlineto{\pgfqpoint{7.094531in}{5.208878in}}%
\pgfpathlineto{\pgfqpoint{7.125398in}{5.148595in}}%
\pgfpathlineto{\pgfqpoint{7.156265in}{5.096208in}}%
\pgfpathlineto{\pgfqpoint{7.187133in}{5.052626in}}%
\pgfpathlineto{\pgfqpoint{7.218000in}{5.018317in}}%
\pgfpathlineto{\pgfqpoint{7.233434in}{5.004646in}}%
\pgfpathlineto{\pgfqpoint{7.248867in}{4.993233in}}%
\pgfpathlineto{\pgfqpoint{7.264301in}{4.983984in}}%
\pgfpathlineto{\pgfqpoint{7.279734in}{4.976769in}}%
\pgfpathlineto{\pgfqpoint{7.295168in}{4.971425in}}%
\pgfpathlineto{\pgfqpoint{7.326035in}{4.965546in}}%
\pgfpathlineto{\pgfqpoint{7.356903in}{4.964481in}}%
\pgfpathlineto{\pgfqpoint{7.464938in}{4.967727in}}%
\pgfpathlineto{\pgfqpoint{7.495805in}{4.962916in}}%
\pgfpathlineto{\pgfqpoint{7.526673in}{4.953560in}}%
\pgfpathlineto{\pgfqpoint{7.557540in}{4.939715in}}%
\pgfpathlineto{\pgfqpoint{7.588407in}{4.922194in}}%
\pgfpathlineto{\pgfqpoint{7.681009in}{4.864907in}}%
\pgfpathlineto{\pgfqpoint{7.711876in}{4.851722in}}%
\pgfpathlineto{\pgfqpoint{7.727310in}{4.847523in}}%
\pgfpathlineto{\pgfqpoint{7.742744in}{4.845243in}}%
\pgfpathlineto{\pgfqpoint{7.758177in}{4.845080in}}%
\pgfpathlineto{\pgfqpoint{7.773611in}{4.847194in}}%
\pgfpathlineto{\pgfqpoint{7.789044in}{4.851696in}}%
\pgfpathlineto{\pgfqpoint{7.804478in}{4.858647in}}%
\pgfpathlineto{\pgfqpoint{7.819912in}{4.868052in}}%
\pgfpathlineto{\pgfqpoint{7.835345in}{4.879861in}}%
\pgfpathlineto{\pgfqpoint{7.850779in}{4.893969in}}%
\pgfpathlineto{\pgfqpoint{7.881646in}{4.928380in}}%
\pgfpathlineto{\pgfqpoint{7.912514in}{4.969375in}}%
\pgfpathlineto{\pgfqpoint{8.020549in}{5.123603in}}%
\pgfpathlineto{\pgfqpoint{8.051416in}{5.157773in}}%
\pgfpathlineto{\pgfqpoint{8.066850in}{5.171466in}}%
\pgfpathlineto{\pgfqpoint{8.082284in}{5.182571in}}%
\pgfpathlineto{\pgfqpoint{8.097717in}{5.190907in}}%
\pgfpathlineto{\pgfqpoint{8.113151in}{5.196354in}}%
\pgfpathlineto{\pgfqpoint{8.128584in}{5.198848in}}%
\pgfpathlineto{\pgfqpoint{8.144018in}{5.198384in}}%
\pgfpathlineto{\pgfqpoint{8.159452in}{5.195013in}}%
\pgfpathlineto{\pgfqpoint{8.174885in}{5.188839in}}%
\pgfpathlineto{\pgfqpoint{8.190319in}{5.180015in}}%
\pgfpathlineto{\pgfqpoint{8.205753in}{5.168735in}}%
\pgfpathlineto{\pgfqpoint{8.236620in}{5.139742in}}%
\pgfpathlineto{\pgfqpoint{8.267487in}{5.103954in}}%
\pgfpathlineto{\pgfqpoint{8.313788in}{5.042467in}}%
\pgfpathlineto{\pgfqpoint{8.406390in}{4.915796in}}%
\pgfpathlineto{\pgfqpoint{8.452691in}{4.859603in}}%
\pgfpathlineto{\pgfqpoint{8.483558in}{4.826278in}}%
\pgfpathlineto{\pgfqpoint{8.514425in}{4.796500in}}%
\pgfpathlineto{\pgfqpoint{8.545293in}{4.770360in}}%
\pgfpathlineto{\pgfqpoint{8.576160in}{4.747973in}}%
\pgfpathlineto{\pgfqpoint{8.607027in}{4.729559in}}%
\pgfpathlineto{\pgfqpoint{8.637894in}{4.715488in}}%
\pgfpathlineto{\pgfqpoint{8.668762in}{4.706274in}}%
\pgfpathlineto{\pgfqpoint{8.699629in}{4.702513in}}%
\pgfpathlineto{\pgfqpoint{8.715063in}{4.702867in}}%
\pgfpathlineto{\pgfqpoint{8.730496in}{4.704798in}}%
\pgfpathlineto{\pgfqpoint{8.745930in}{4.708361in}}%
\pgfpathlineto{\pgfqpoint{8.761363in}{4.713590in}}%
\pgfpathlineto{\pgfqpoint{8.792231in}{4.729087in}}%
\pgfpathlineto{\pgfqpoint{8.823098in}{4.751115in}}%
\pgfpathlineto{\pgfqpoint{8.853965in}{4.779044in}}%
\pgfpathlineto{\pgfqpoint{8.884833in}{4.811769in}}%
\pgfpathlineto{\pgfqpoint{8.931133in}{4.866410in}}%
\pgfpathlineto{\pgfqpoint{8.992868in}{4.939616in}}%
\pgfpathlineto{\pgfqpoint{9.023735in}{4.972087in}}%
\pgfpathlineto{\pgfqpoint{9.054603in}{4.999752in}}%
\pgfpathlineto{\pgfqpoint{9.085470in}{5.021686in}}%
\pgfpathlineto{\pgfqpoint{9.116337in}{5.037561in}}%
\pgfpathlineto{\pgfqpoint{9.147204in}{5.047660in}}%
\pgfpathlineto{\pgfqpoint{9.178072in}{5.052799in}}%
\pgfpathlineto{\pgfqpoint{9.208939in}{5.054190in}}%
\pgfpathlineto{\pgfqpoint{9.255240in}{5.052414in}}%
\pgfpathlineto{\pgfqpoint{9.316974in}{5.050260in}}%
\pgfpathlineto{\pgfqpoint{9.347842in}{5.051797in}}%
\pgfpathlineto{\pgfqpoint{9.378709in}{5.055966in}}%
\pgfpathlineto{\pgfqpoint{9.409576in}{5.062973in}}%
\pgfpathlineto{\pgfqpoint{9.440443in}{5.072680in}}%
\pgfpathlineto{\pgfqpoint{9.486744in}{5.091315in}}%
\pgfpathlineto{\pgfqpoint{9.625647in}{5.153600in}}%
\pgfpathlineto{\pgfqpoint{9.656514in}{5.164076in}}%
\pgfpathlineto{\pgfqpoint{9.687382in}{5.172252in}}%
\pgfpathlineto{\pgfqpoint{9.718249in}{5.177973in}}%
\pgfpathlineto{\pgfqpoint{9.733682in}{5.179918in}}%
\pgfpathlineto{\pgfqpoint{9.733682in}{5.179918in}}%
\pgfusepath{stroke}%
\end{pgfscope}%
\begin{pgfscope}%
\pgfpathrectangle{\pgfqpoint{5.698559in}{3.799602in}}{\pgfqpoint{4.227273in}{2.745455in}} %
\pgfusepath{clip}%
\pgfsetrectcap%
\pgfsetroundjoin%
\pgfsetlinewidth{0.501875pt}%
\definecolor{currentstroke}{rgb}{0.000000,0.000000,0.000000}%
\pgfsetstrokecolor{currentstroke}%
\pgfsetdash{}{0pt}%
\pgfpathmoveto{\pgfqpoint{5.890707in}{5.434951in}}%
\pgfpathlineto{\pgfqpoint{5.906141in}{5.432764in}}%
\pgfpathlineto{\pgfqpoint{5.921575in}{5.427982in}}%
\pgfpathlineto{\pgfqpoint{5.937008in}{5.420544in}}%
\pgfpathlineto{\pgfqpoint{5.952442in}{5.410433in}}%
\pgfpathlineto{\pgfqpoint{5.967875in}{5.397679in}}%
\pgfpathlineto{\pgfqpoint{5.983309in}{5.382362in}}%
\pgfpathlineto{\pgfqpoint{6.014176in}{5.344598in}}%
\pgfpathlineto{\pgfqpoint{6.045044in}{5.298728in}}%
\pgfpathlineto{\pgfqpoint{6.075911in}{5.247026in}}%
\pgfpathlineto{\pgfqpoint{6.168513in}{5.087135in}}%
\pgfpathlineto{\pgfqpoint{6.199380in}{5.043573in}}%
\pgfpathlineto{\pgfqpoint{6.214814in}{5.025524in}}%
\pgfpathlineto{\pgfqpoint{6.230247in}{5.010477in}}%
\pgfpathlineto{\pgfqpoint{6.245681in}{4.998777in}}%
\pgfpathlineto{\pgfqpoint{6.261115in}{4.990732in}}%
\pgfpathlineto{\pgfqpoint{6.276548in}{4.986608in}}%
\pgfpathlineto{\pgfqpoint{6.291982in}{4.986620in}}%
\pgfpathlineto{\pgfqpoint{6.307415in}{4.990930in}}%
\pgfpathlineto{\pgfqpoint{6.322849in}{4.999639in}}%
\pgfpathlineto{\pgfqpoint{6.338283in}{5.012788in}}%
\pgfpathlineto{\pgfqpoint{6.353716in}{5.030353in}}%
\pgfpathlineto{\pgfqpoint{6.369150in}{5.052243in}}%
\pgfpathlineto{\pgfqpoint{6.384584in}{5.078298in}}%
\pgfpathlineto{\pgfqpoint{6.400017in}{5.108295in}}%
\pgfpathlineto{\pgfqpoint{6.430885in}{5.178900in}}%
\pgfpathlineto{\pgfqpoint{6.461752in}{5.261052in}}%
\pgfpathlineto{\pgfqpoint{6.508053in}{5.397488in}}%
\pgfpathlineto{\pgfqpoint{6.569787in}{5.580766in}}%
\pgfpathlineto{\pgfqpoint{6.600655in}{5.662954in}}%
\pgfpathlineto{\pgfqpoint{6.631522in}{5.733596in}}%
\pgfpathlineto{\pgfqpoint{6.646955in}{5.763604in}}%
\pgfpathlineto{\pgfqpoint{6.662389in}{5.789657in}}%
\pgfpathlineto{\pgfqpoint{6.677823in}{5.811522in}}%
\pgfpathlineto{\pgfqpoint{6.693256in}{5.829028in}}%
\pgfpathlineto{\pgfqpoint{6.708690in}{5.842067in}}%
\pgfpathlineto{\pgfqpoint{6.724124in}{5.850594in}}%
\pgfpathlineto{\pgfqpoint{6.739557in}{5.854620in}}%
\pgfpathlineto{\pgfqpoint{6.754991in}{5.854213in}}%
\pgfpathlineto{\pgfqpoint{6.770424in}{5.849492in}}%
\pgfpathlineto{\pgfqpoint{6.785858in}{5.840620in}}%
\pgfpathlineto{\pgfqpoint{6.801292in}{5.827802in}}%
\pgfpathlineto{\pgfqpoint{6.816725in}{5.811273in}}%
\pgfpathlineto{\pgfqpoint{6.832159in}{5.791297in}}%
\pgfpathlineto{\pgfqpoint{6.847593in}{5.768161in}}%
\pgfpathlineto{\pgfqpoint{6.878460in}{5.713618in}}%
\pgfpathlineto{\pgfqpoint{6.909327in}{5.650148in}}%
\pgfpathlineto{\pgfqpoint{6.955628in}{5.543722in}}%
\pgfpathlineto{\pgfqpoint{7.079097in}{5.249683in}}%
\pgfpathlineto{\pgfqpoint{7.109964in}{5.184959in}}%
\pgfpathlineto{\pgfqpoint{7.140832in}{5.127156in}}%
\pgfpathlineto{\pgfqpoint{7.171699in}{5.077509in}}%
\pgfpathlineto{\pgfqpoint{7.202566in}{5.036856in}}%
\pgfpathlineto{\pgfqpoint{7.218000in}{5.020030in}}%
\pgfpathlineto{\pgfqpoint{7.233434in}{5.005537in}}%
\pgfpathlineto{\pgfqpoint{7.248867in}{4.993330in}}%
\pgfpathlineto{\pgfqpoint{7.264301in}{4.983322in}}%
\pgfpathlineto{\pgfqpoint{7.279734in}{4.975391in}}%
\pgfpathlineto{\pgfqpoint{7.295168in}{4.969379in}}%
\pgfpathlineto{\pgfqpoint{7.326035in}{4.962311in}}%
\pgfpathlineto{\pgfqpoint{7.356903in}{4.960263in}}%
\pgfpathlineto{\pgfqpoint{7.403204in}{4.961909in}}%
\pgfpathlineto{\pgfqpoint{7.449504in}{4.962824in}}%
\pgfpathlineto{\pgfqpoint{7.480372in}{4.960099in}}%
\pgfpathlineto{\pgfqpoint{7.511239in}{4.953445in}}%
\pgfpathlineto{\pgfqpoint{7.542106in}{4.942569in}}%
\pgfpathlineto{\pgfqpoint{7.572974in}{4.927921in}}%
\pgfpathlineto{\pgfqpoint{7.619274in}{4.901537in}}%
\pgfpathlineto{\pgfqpoint{7.665575in}{4.875434in}}%
\pgfpathlineto{\pgfqpoint{7.696443in}{4.861856in}}%
\pgfpathlineto{\pgfqpoint{7.711876in}{4.857047in}}%
\pgfpathlineto{\pgfqpoint{7.727310in}{4.853895in}}%
\pgfpathlineto{\pgfqpoint{7.742744in}{4.852620in}}%
\pgfpathlineto{\pgfqpoint{7.758177in}{4.853409in}}%
\pgfpathlineto{\pgfqpoint{7.773611in}{4.856409in}}%
\pgfpathlineto{\pgfqpoint{7.789044in}{4.861720in}}%
\pgfpathlineto{\pgfqpoint{7.804478in}{4.869396in}}%
\pgfpathlineto{\pgfqpoint{7.819912in}{4.879435in}}%
\pgfpathlineto{\pgfqpoint{7.835345in}{4.891781in}}%
\pgfpathlineto{\pgfqpoint{7.866213in}{4.922899in}}%
\pgfpathlineto{\pgfqpoint{7.897080in}{4.961218in}}%
\pgfpathlineto{\pgfqpoint{7.943381in}{5.026994in}}%
\pgfpathlineto{\pgfqpoint{7.989682in}{5.093666in}}%
\pgfpathlineto{\pgfqpoint{8.020549in}{5.133212in}}%
\pgfpathlineto{\pgfqpoint{8.051416in}{5.165355in}}%
\pgfpathlineto{\pgfqpoint{8.066850in}{5.177888in}}%
\pgfpathlineto{\pgfqpoint{8.082284in}{5.187753in}}%
\pgfpathlineto{\pgfqpoint{8.097717in}{5.194786in}}%
\pgfpathlineto{\pgfqpoint{8.113151in}{5.198886in}}%
\pgfpathlineto{\pgfqpoint{8.128584in}{5.200011in}}%
\pgfpathlineto{\pgfqpoint{8.144018in}{5.198181in}}%
\pgfpathlineto{\pgfqpoint{8.159452in}{5.193472in}}%
\pgfpathlineto{\pgfqpoint{8.174885in}{5.186015in}}%
\pgfpathlineto{\pgfqpoint{8.190319in}{5.175988in}}%
\pgfpathlineto{\pgfqpoint{8.205753in}{5.163610in}}%
\pgfpathlineto{\pgfqpoint{8.236620in}{5.132825in}}%
\pgfpathlineto{\pgfqpoint{8.267487in}{5.095905in}}%
\pgfpathlineto{\pgfqpoint{8.313788in}{5.034121in}}%
\pgfpathlineto{\pgfqpoint{8.390956in}{4.930640in}}%
\pgfpathlineto{\pgfqpoint{8.437257in}{4.875465in}}%
\pgfpathlineto{\pgfqpoint{8.468124in}{4.842824in}}%
\pgfpathlineto{\pgfqpoint{8.498992in}{4.813642in}}%
\pgfpathlineto{\pgfqpoint{8.529859in}{4.787884in}}%
\pgfpathlineto{\pgfqpoint{8.560726in}{4.765516in}}%
\pgfpathlineto{\pgfqpoint{8.591593in}{4.746612in}}%
\pgfpathlineto{\pgfqpoint{8.622461in}{4.731421in}}%
\pgfpathlineto{\pgfqpoint{8.653328in}{4.720387in}}%
\pgfpathlineto{\pgfqpoint{8.684195in}{4.714117in}}%
\pgfpathlineto{\pgfqpoint{8.715063in}{4.713294in}}%
\pgfpathlineto{\pgfqpoint{8.730496in}{4.715134in}}%
\pgfpathlineto{\pgfqpoint{8.745930in}{4.718567in}}%
\pgfpathlineto{\pgfqpoint{8.761363in}{4.723646in}}%
\pgfpathlineto{\pgfqpoint{8.792231in}{4.738837in}}%
\pgfpathlineto{\pgfqpoint{8.823098in}{4.760643in}}%
\pgfpathlineto{\pgfqpoint{8.853965in}{4.788520in}}%
\pgfpathlineto{\pgfqpoint{8.884833in}{4.821415in}}%
\pgfpathlineto{\pgfqpoint{8.931133in}{4.876733in}}%
\pgfpathlineto{\pgfqpoint{8.992868in}{4.951383in}}%
\pgfpathlineto{\pgfqpoint{9.023735in}{4.984620in}}%
\pgfpathlineto{\pgfqpoint{9.054603in}{5.012951in}}%
\pgfpathlineto{\pgfqpoint{9.085470in}{5.035350in}}%
\pgfpathlineto{\pgfqpoint{9.116337in}{5.051406in}}%
\pgfpathlineto{\pgfqpoint{9.147204in}{5.061331in}}%
\pgfpathlineto{\pgfqpoint{9.178072in}{5.065891in}}%
\pgfpathlineto{\pgfqpoint{9.208939in}{5.066281in}}%
\pgfpathlineto{\pgfqpoint{9.255240in}{5.062246in}}%
\pgfpathlineto{\pgfqpoint{9.316974in}{5.056030in}}%
\pgfpathlineto{\pgfqpoint{9.347842in}{5.055356in}}%
\pgfpathlineto{\pgfqpoint{9.378709in}{5.057399in}}%
\pgfpathlineto{\pgfqpoint{9.409576in}{5.062523in}}%
\pgfpathlineto{\pgfqpoint{9.440443in}{5.070728in}}%
\pgfpathlineto{\pgfqpoint{9.471311in}{5.081690in}}%
\pgfpathlineto{\pgfqpoint{9.517612in}{5.101968in}}%
\pgfpathlineto{\pgfqpoint{9.625647in}{5.152744in}}%
\pgfpathlineto{\pgfqpoint{9.671948in}{5.169646in}}%
\pgfpathlineto{\pgfqpoint{9.702815in}{5.178016in}}%
\pgfpathlineto{\pgfqpoint{9.733682in}{5.183905in}}%
\pgfpathlineto{\pgfqpoint{9.733682in}{5.183905in}}%
\pgfusepath{stroke}%
\end{pgfscope}%
\begin{pgfscope}%
\pgfsetrectcap%
\pgfsetmiterjoin%
\pgfsetlinewidth{0.803000pt}%
\definecolor{currentstroke}{rgb}{0.000000,0.000000,0.000000}%
\pgfsetstrokecolor{currentstroke}%
\pgfsetdash{}{0pt}%
\pgfpathmoveto{\pgfqpoint{5.698559in}{3.799602in}}%
\pgfpathlineto{\pgfqpoint{5.698559in}{6.545056in}}%
\pgfusepath{stroke}%
\end{pgfscope}%
\begin{pgfscope}%
\pgfsetrectcap%
\pgfsetmiterjoin%
\pgfsetlinewidth{0.803000pt}%
\definecolor{currentstroke}{rgb}{0.000000,0.000000,0.000000}%
\pgfsetstrokecolor{currentstroke}%
\pgfsetdash{}{0pt}%
\pgfpathmoveto{\pgfqpoint{9.925831in}{3.799602in}}%
\pgfpathlineto{\pgfqpoint{9.925831in}{6.545056in}}%
\pgfusepath{stroke}%
\end{pgfscope}%
\begin{pgfscope}%
\pgfsetrectcap%
\pgfsetmiterjoin%
\pgfsetlinewidth{0.803000pt}%
\definecolor{currentstroke}{rgb}{0.000000,0.000000,0.000000}%
\pgfsetstrokecolor{currentstroke}%
\pgfsetdash{}{0pt}%
\pgfpathmoveto{\pgfqpoint{5.698559in}{3.799602in}}%
\pgfpathlineto{\pgfqpoint{9.925831in}{3.799602in}}%
\pgfusepath{stroke}%
\end{pgfscope}%
\begin{pgfscope}%
\pgfsetrectcap%
\pgfsetmiterjoin%
\pgfsetlinewidth{0.803000pt}%
\definecolor{currentstroke}{rgb}{0.000000,0.000000,0.000000}%
\pgfsetstrokecolor{currentstroke}%
\pgfsetdash{}{0pt}%
\pgfpathmoveto{\pgfqpoint{5.698559in}{6.545056in}}%
\pgfpathlineto{\pgfqpoint{9.925831in}{6.545056in}}%
\pgfusepath{stroke}%
\end{pgfscope}%
\begin{pgfscope}%
\pgfsetbuttcap%
\pgfsetmiterjoin%
\definecolor{currentfill}{rgb}{1.000000,1.000000,1.000000}%
\pgfsetfillcolor{currentfill}%
\pgfsetfillopacity{0.800000}%
\pgfsetlinewidth{1.003750pt}%
\definecolor{currentstroke}{rgb}{0.800000,0.800000,0.800000}%
\pgfsetstrokecolor{currentstroke}%
\pgfsetstrokeopacity{0.800000}%
\pgfsetdash{}{0pt}%
\pgfpathmoveto{\pgfqpoint{9.283863in}{3.869046in}}%
\pgfpathlineto{\pgfqpoint{9.828609in}{3.869046in}}%
\pgfpathquadraticcurveto{\pgfqpoint{9.856387in}{3.869046in}}{\pgfqpoint{9.856387in}{3.896824in}}%
\pgfpathlineto{\pgfqpoint{9.856387in}{4.079305in}}%
\pgfpathquadraticcurveto{\pgfqpoint{9.856387in}{4.107083in}}{\pgfqpoint{9.828609in}{4.107083in}}%
\pgfpathlineto{\pgfqpoint{9.283863in}{4.107083in}}%
\pgfpathquadraticcurveto{\pgfqpoint{9.256086in}{4.107083in}}{\pgfqpoint{9.256086in}{4.079305in}}%
\pgfpathlineto{\pgfqpoint{9.256086in}{3.896824in}}%
\pgfpathquadraticcurveto{\pgfqpoint{9.256086in}{3.869046in}}{\pgfqpoint{9.283863in}{3.869046in}}%
\pgfpathclose%
\pgfusepath{stroke,fill}%
\end{pgfscope}%
\begin{pgfscope}%
\pgfsetrectcap%
\pgfsetroundjoin%
\pgfsetlinewidth{0.501875pt}%
\definecolor{currentstroke}{rgb}{0.000000,0.000000,0.000000}%
\pgfsetstrokecolor{currentstroke}%
\pgfsetdash{}{0pt}%
\pgfpathmoveto{\pgfqpoint{9.311641in}{4.002916in}}%
\pgfpathlineto{\pgfqpoint{9.589419in}{4.002916in}}%
\pgfusepath{stroke}%
\end{pgfscope}%
\begin{pgfscope}%
\pgftext[x=9.700530in,y=3.954305in,left,base]{\rmfamily\fontsize{10.000000}{12.000000}\selectfont K}%
\end{pgfscope}%
\begin{pgfscope}%
\pgfsetbuttcap%
\pgfsetmiterjoin%
\definecolor{currentfill}{rgb}{1.000000,1.000000,1.000000}%
\pgfsetfillcolor{currentfill}%
\pgfsetlinewidth{0.000000pt}%
\definecolor{currentstroke}{rgb}{0.000000,0.000000,0.000000}%
\pgfsetstrokecolor{currentstroke}%
\pgfsetstrokeopacity{0.000000}%
\pgfsetdash{}{0pt}%
\pgfpathmoveto{\pgfqpoint{0.625831in}{0.505056in}}%
\pgfpathlineto{\pgfqpoint{4.853104in}{0.505056in}}%
\pgfpathlineto{\pgfqpoint{4.853104in}{3.250511in}}%
\pgfpathlineto{\pgfqpoint{0.625831in}{3.250511in}}%
\pgfpathclose%
\pgfusepath{fill}%
\end{pgfscope}%
\begin{pgfscope}%
\pgfsetbuttcap%
\pgfsetroundjoin%
\definecolor{currentfill}{rgb}{0.000000,0.000000,0.000000}%
\pgfsetfillcolor{currentfill}%
\pgfsetlinewidth{0.803000pt}%
\definecolor{currentstroke}{rgb}{0.000000,0.000000,0.000000}%
\pgfsetstrokecolor{currentstroke}%
\pgfsetdash{}{0pt}%
\pgfsys@defobject{currentmarker}{\pgfqpoint{0.000000in}{-0.048611in}}{\pgfqpoint{0.000000in}{0.000000in}}{%
\pgfpathmoveto{\pgfqpoint{0.000000in}{0.000000in}}%
\pgfpathlineto{\pgfqpoint{0.000000in}{-0.048611in}}%
\pgfusepath{stroke,fill}%
}%
\begin{pgfscope}%
\pgfsys@transformshift{0.817980in}{0.505056in}%
\pgfsys@useobject{currentmarker}{}%
\end{pgfscope}%
\end{pgfscope}%
\begin{pgfscope}%
\pgftext[x=0.817980in,y=0.407834in,,top]{\rmfamily\fontsize{10.000000}{12.000000}\selectfont \(\displaystyle -1.00\)}%
\end{pgfscope}%
\begin{pgfscope}%
\pgfsetbuttcap%
\pgfsetroundjoin%
\definecolor{currentfill}{rgb}{0.000000,0.000000,0.000000}%
\pgfsetfillcolor{currentfill}%
\pgfsetlinewidth{0.803000pt}%
\definecolor{currentstroke}{rgb}{0.000000,0.000000,0.000000}%
\pgfsetstrokecolor{currentstroke}%
\pgfsetdash{}{0pt}%
\pgfsys@defobject{currentmarker}{\pgfqpoint{0.000000in}{-0.048611in}}{\pgfqpoint{0.000000in}{0.000000in}}{%
\pgfpathmoveto{\pgfqpoint{0.000000in}{0.000000in}}%
\pgfpathlineto{\pgfqpoint{0.000000in}{-0.048611in}}%
\pgfusepath{stroke,fill}%
}%
\begin{pgfscope}%
\pgfsys@transformshift{1.298352in}{0.505056in}%
\pgfsys@useobject{currentmarker}{}%
\end{pgfscope}%
\end{pgfscope}%
\begin{pgfscope}%
\pgftext[x=1.298352in,y=0.407834in,,top]{\rmfamily\fontsize{10.000000}{12.000000}\selectfont \(\displaystyle -0.75\)}%
\end{pgfscope}%
\begin{pgfscope}%
\pgfsetbuttcap%
\pgfsetroundjoin%
\definecolor{currentfill}{rgb}{0.000000,0.000000,0.000000}%
\pgfsetfillcolor{currentfill}%
\pgfsetlinewidth{0.803000pt}%
\definecolor{currentstroke}{rgb}{0.000000,0.000000,0.000000}%
\pgfsetstrokecolor{currentstroke}%
\pgfsetdash{}{0pt}%
\pgfsys@defobject{currentmarker}{\pgfqpoint{0.000000in}{-0.048611in}}{\pgfqpoint{0.000000in}{0.000000in}}{%
\pgfpathmoveto{\pgfqpoint{0.000000in}{0.000000in}}%
\pgfpathlineto{\pgfqpoint{0.000000in}{-0.048611in}}%
\pgfusepath{stroke,fill}%
}%
\begin{pgfscope}%
\pgfsys@transformshift{1.778724in}{0.505056in}%
\pgfsys@useobject{currentmarker}{}%
\end{pgfscope}%
\end{pgfscope}%
\begin{pgfscope}%
\pgftext[x=1.778724in,y=0.407834in,,top]{\rmfamily\fontsize{10.000000}{12.000000}\selectfont \(\displaystyle -0.50\)}%
\end{pgfscope}%
\begin{pgfscope}%
\pgfsetbuttcap%
\pgfsetroundjoin%
\definecolor{currentfill}{rgb}{0.000000,0.000000,0.000000}%
\pgfsetfillcolor{currentfill}%
\pgfsetlinewidth{0.803000pt}%
\definecolor{currentstroke}{rgb}{0.000000,0.000000,0.000000}%
\pgfsetstrokecolor{currentstroke}%
\pgfsetdash{}{0pt}%
\pgfsys@defobject{currentmarker}{\pgfqpoint{0.000000in}{-0.048611in}}{\pgfqpoint{0.000000in}{0.000000in}}{%
\pgfpathmoveto{\pgfqpoint{0.000000in}{0.000000in}}%
\pgfpathlineto{\pgfqpoint{0.000000in}{-0.048611in}}%
\pgfusepath{stroke,fill}%
}%
\begin{pgfscope}%
\pgfsys@transformshift{2.259096in}{0.505056in}%
\pgfsys@useobject{currentmarker}{}%
\end{pgfscope}%
\end{pgfscope}%
\begin{pgfscope}%
\pgftext[x=2.259096in,y=0.407834in,,top]{\rmfamily\fontsize{10.000000}{12.000000}\selectfont \(\displaystyle -0.25\)}%
\end{pgfscope}%
\begin{pgfscope}%
\pgfsetbuttcap%
\pgfsetroundjoin%
\definecolor{currentfill}{rgb}{0.000000,0.000000,0.000000}%
\pgfsetfillcolor{currentfill}%
\pgfsetlinewidth{0.803000pt}%
\definecolor{currentstroke}{rgb}{0.000000,0.000000,0.000000}%
\pgfsetstrokecolor{currentstroke}%
\pgfsetdash{}{0pt}%
\pgfsys@defobject{currentmarker}{\pgfqpoint{0.000000in}{-0.048611in}}{\pgfqpoint{0.000000in}{0.000000in}}{%
\pgfpathmoveto{\pgfqpoint{0.000000in}{0.000000in}}%
\pgfpathlineto{\pgfqpoint{0.000000in}{-0.048611in}}%
\pgfusepath{stroke,fill}%
}%
\begin{pgfscope}%
\pgfsys@transformshift{2.739468in}{0.505056in}%
\pgfsys@useobject{currentmarker}{}%
\end{pgfscope}%
\end{pgfscope}%
\begin{pgfscope}%
\pgftext[x=2.739468in,y=0.407834in,,top]{\rmfamily\fontsize{10.000000}{12.000000}\selectfont \(\displaystyle 0.00\)}%
\end{pgfscope}%
\begin{pgfscope}%
\pgfsetbuttcap%
\pgfsetroundjoin%
\definecolor{currentfill}{rgb}{0.000000,0.000000,0.000000}%
\pgfsetfillcolor{currentfill}%
\pgfsetlinewidth{0.803000pt}%
\definecolor{currentstroke}{rgb}{0.000000,0.000000,0.000000}%
\pgfsetstrokecolor{currentstroke}%
\pgfsetdash{}{0pt}%
\pgfsys@defobject{currentmarker}{\pgfqpoint{0.000000in}{-0.048611in}}{\pgfqpoint{0.000000in}{0.000000in}}{%
\pgfpathmoveto{\pgfqpoint{0.000000in}{0.000000in}}%
\pgfpathlineto{\pgfqpoint{0.000000in}{-0.048611in}}%
\pgfusepath{stroke,fill}%
}%
\begin{pgfscope}%
\pgfsys@transformshift{3.219840in}{0.505056in}%
\pgfsys@useobject{currentmarker}{}%
\end{pgfscope}%
\end{pgfscope}%
\begin{pgfscope}%
\pgftext[x=3.219840in,y=0.407834in,,top]{\rmfamily\fontsize{10.000000}{12.000000}\selectfont \(\displaystyle 0.25\)}%
\end{pgfscope}%
\begin{pgfscope}%
\pgfsetbuttcap%
\pgfsetroundjoin%
\definecolor{currentfill}{rgb}{0.000000,0.000000,0.000000}%
\pgfsetfillcolor{currentfill}%
\pgfsetlinewidth{0.803000pt}%
\definecolor{currentstroke}{rgb}{0.000000,0.000000,0.000000}%
\pgfsetstrokecolor{currentstroke}%
\pgfsetdash{}{0pt}%
\pgfsys@defobject{currentmarker}{\pgfqpoint{0.000000in}{-0.048611in}}{\pgfqpoint{0.000000in}{0.000000in}}{%
\pgfpathmoveto{\pgfqpoint{0.000000in}{0.000000in}}%
\pgfpathlineto{\pgfqpoint{0.000000in}{-0.048611in}}%
\pgfusepath{stroke,fill}%
}%
\begin{pgfscope}%
\pgfsys@transformshift{3.700211in}{0.505056in}%
\pgfsys@useobject{currentmarker}{}%
\end{pgfscope}%
\end{pgfscope}%
\begin{pgfscope}%
\pgftext[x=3.700211in,y=0.407834in,,top]{\rmfamily\fontsize{10.000000}{12.000000}\selectfont \(\displaystyle 0.50\)}%
\end{pgfscope}%
\begin{pgfscope}%
\pgfsetbuttcap%
\pgfsetroundjoin%
\definecolor{currentfill}{rgb}{0.000000,0.000000,0.000000}%
\pgfsetfillcolor{currentfill}%
\pgfsetlinewidth{0.803000pt}%
\definecolor{currentstroke}{rgb}{0.000000,0.000000,0.000000}%
\pgfsetstrokecolor{currentstroke}%
\pgfsetdash{}{0pt}%
\pgfsys@defobject{currentmarker}{\pgfqpoint{0.000000in}{-0.048611in}}{\pgfqpoint{0.000000in}{0.000000in}}{%
\pgfpathmoveto{\pgfqpoint{0.000000in}{0.000000in}}%
\pgfpathlineto{\pgfqpoint{0.000000in}{-0.048611in}}%
\pgfusepath{stroke,fill}%
}%
\begin{pgfscope}%
\pgfsys@transformshift{4.180583in}{0.505056in}%
\pgfsys@useobject{currentmarker}{}%
\end{pgfscope}%
\end{pgfscope}%
\begin{pgfscope}%
\pgftext[x=4.180583in,y=0.407834in,,top]{\rmfamily\fontsize{10.000000}{12.000000}\selectfont \(\displaystyle 0.75\)}%
\end{pgfscope}%
\begin{pgfscope}%
\pgfsetbuttcap%
\pgfsetroundjoin%
\definecolor{currentfill}{rgb}{0.000000,0.000000,0.000000}%
\pgfsetfillcolor{currentfill}%
\pgfsetlinewidth{0.803000pt}%
\definecolor{currentstroke}{rgb}{0.000000,0.000000,0.000000}%
\pgfsetstrokecolor{currentstroke}%
\pgfsetdash{}{0pt}%
\pgfsys@defobject{currentmarker}{\pgfqpoint{0.000000in}{-0.048611in}}{\pgfqpoint{0.000000in}{0.000000in}}{%
\pgfpathmoveto{\pgfqpoint{0.000000in}{0.000000in}}%
\pgfpathlineto{\pgfqpoint{0.000000in}{-0.048611in}}%
\pgfusepath{stroke,fill}%
}%
\begin{pgfscope}%
\pgfsys@transformshift{4.660955in}{0.505056in}%
\pgfsys@useobject{currentmarker}{}%
\end{pgfscope}%
\end{pgfscope}%
\begin{pgfscope}%
\pgftext[x=4.660955in,y=0.407834in,,top]{\rmfamily\fontsize{10.000000}{12.000000}\selectfont \(\displaystyle 1.00\)}%
\end{pgfscope}%
\begin{pgfscope}%
\pgftext[x=2.739468in,y=0.226139in,,top]{\rmfamily\fontsize{10.000000}{12.000000}\selectfont \(\displaystyle x\)}%
\end{pgfscope}%
\begin{pgfscope}%
\pgfsetbuttcap%
\pgfsetroundjoin%
\definecolor{currentfill}{rgb}{0.000000,0.000000,0.000000}%
\pgfsetfillcolor{currentfill}%
\pgfsetlinewidth{0.803000pt}%
\definecolor{currentstroke}{rgb}{0.000000,0.000000,0.000000}%
\pgfsetstrokecolor{currentstroke}%
\pgfsetdash{}{0pt}%
\pgfsys@defobject{currentmarker}{\pgfqpoint{-0.048611in}{0.000000in}}{\pgfqpoint{0.000000in}{0.000000in}}{%
\pgfpathmoveto{\pgfqpoint{0.000000in}{0.000000in}}%
\pgfpathlineto{\pgfqpoint{-0.048611in}{0.000000in}}%
\pgfusepath{stroke,fill}%
}%
\begin{pgfscope}%
\pgfsys@transformshift{0.625831in}{0.950556in}%
\pgfsys@useobject{currentmarker}{}%
\end{pgfscope}%
\end{pgfscope}%
\begin{pgfscope}%
\pgftext[x=0.281695in,y=0.902338in,left,base]{\rmfamily\fontsize{10.000000}{12.000000}\selectfont \(\displaystyle -20\)}%
\end{pgfscope}%
\begin{pgfscope}%
\pgfsetbuttcap%
\pgfsetroundjoin%
\definecolor{currentfill}{rgb}{0.000000,0.000000,0.000000}%
\pgfsetfillcolor{currentfill}%
\pgfsetlinewidth{0.803000pt}%
\definecolor{currentstroke}{rgb}{0.000000,0.000000,0.000000}%
\pgfsetstrokecolor{currentstroke}%
\pgfsetdash{}{0pt}%
\pgfsys@defobject{currentmarker}{\pgfqpoint{-0.048611in}{0.000000in}}{\pgfqpoint{0.000000in}{0.000000in}}{%
\pgfpathmoveto{\pgfqpoint{0.000000in}{0.000000in}}%
\pgfpathlineto{\pgfqpoint{-0.048611in}{0.000000in}}%
\pgfusepath{stroke,fill}%
}%
\begin{pgfscope}%
\pgfsys@transformshift{0.625831in}{1.410860in}%
\pgfsys@useobject{currentmarker}{}%
\end{pgfscope}%
\end{pgfscope}%
\begin{pgfscope}%
\pgftext[x=0.281695in,y=1.362642in,left,base]{\rmfamily\fontsize{10.000000}{12.000000}\selectfont \(\displaystyle -10\)}%
\end{pgfscope}%
\begin{pgfscope}%
\pgfsetbuttcap%
\pgfsetroundjoin%
\definecolor{currentfill}{rgb}{0.000000,0.000000,0.000000}%
\pgfsetfillcolor{currentfill}%
\pgfsetlinewidth{0.803000pt}%
\definecolor{currentstroke}{rgb}{0.000000,0.000000,0.000000}%
\pgfsetstrokecolor{currentstroke}%
\pgfsetdash{}{0pt}%
\pgfsys@defobject{currentmarker}{\pgfqpoint{-0.048611in}{0.000000in}}{\pgfqpoint{0.000000in}{0.000000in}}{%
\pgfpathmoveto{\pgfqpoint{0.000000in}{0.000000in}}%
\pgfpathlineto{\pgfqpoint{-0.048611in}{0.000000in}}%
\pgfusepath{stroke,fill}%
}%
\begin{pgfscope}%
\pgfsys@transformshift{0.625831in}{1.871164in}%
\pgfsys@useobject{currentmarker}{}%
\end{pgfscope}%
\end{pgfscope}%
\begin{pgfscope}%
\pgftext[x=0.459164in,y=1.822947in,left,base]{\rmfamily\fontsize{10.000000}{12.000000}\selectfont \(\displaystyle 0\)}%
\end{pgfscope}%
\begin{pgfscope}%
\pgfsetbuttcap%
\pgfsetroundjoin%
\definecolor{currentfill}{rgb}{0.000000,0.000000,0.000000}%
\pgfsetfillcolor{currentfill}%
\pgfsetlinewidth{0.803000pt}%
\definecolor{currentstroke}{rgb}{0.000000,0.000000,0.000000}%
\pgfsetstrokecolor{currentstroke}%
\pgfsetdash{}{0pt}%
\pgfsys@defobject{currentmarker}{\pgfqpoint{-0.048611in}{0.000000in}}{\pgfqpoint{0.000000in}{0.000000in}}{%
\pgfpathmoveto{\pgfqpoint{0.000000in}{0.000000in}}%
\pgfpathlineto{\pgfqpoint{-0.048611in}{0.000000in}}%
\pgfusepath{stroke,fill}%
}%
\begin{pgfscope}%
\pgfsys@transformshift{0.625831in}{2.331468in}%
\pgfsys@useobject{currentmarker}{}%
\end{pgfscope}%
\end{pgfscope}%
\begin{pgfscope}%
\pgftext[x=0.389720in,y=2.283251in,left,base]{\rmfamily\fontsize{10.000000}{12.000000}\selectfont \(\displaystyle 10\)}%
\end{pgfscope}%
\begin{pgfscope}%
\pgfsetbuttcap%
\pgfsetroundjoin%
\definecolor{currentfill}{rgb}{0.000000,0.000000,0.000000}%
\pgfsetfillcolor{currentfill}%
\pgfsetlinewidth{0.803000pt}%
\definecolor{currentstroke}{rgb}{0.000000,0.000000,0.000000}%
\pgfsetstrokecolor{currentstroke}%
\pgfsetdash{}{0pt}%
\pgfsys@defobject{currentmarker}{\pgfqpoint{-0.048611in}{0.000000in}}{\pgfqpoint{0.000000in}{0.000000in}}{%
\pgfpathmoveto{\pgfqpoint{0.000000in}{0.000000in}}%
\pgfpathlineto{\pgfqpoint{-0.048611in}{0.000000in}}%
\pgfusepath{stroke,fill}%
}%
\begin{pgfscope}%
\pgfsys@transformshift{0.625831in}{2.791773in}%
\pgfsys@useobject{currentmarker}{}%
\end{pgfscope}%
\end{pgfscope}%
\begin{pgfscope}%
\pgftext[x=0.389720in,y=2.743555in,left,base]{\rmfamily\fontsize{10.000000}{12.000000}\selectfont \(\displaystyle 20\)}%
\end{pgfscope}%
\begin{pgfscope}%
\pgfsetbuttcap%
\pgfsetroundjoin%
\definecolor{currentfill}{rgb}{0.000000,0.000000,0.000000}%
\pgfsetfillcolor{currentfill}%
\pgfsetlinewidth{0.803000pt}%
\definecolor{currentstroke}{rgb}{0.000000,0.000000,0.000000}%
\pgfsetstrokecolor{currentstroke}%
\pgfsetdash{}{0pt}%
\pgfsys@defobject{currentmarker}{\pgfqpoint{-0.048611in}{0.000000in}}{\pgfqpoint{0.000000in}{0.000000in}}{%
\pgfpathmoveto{\pgfqpoint{0.000000in}{0.000000in}}%
\pgfpathlineto{\pgfqpoint{-0.048611in}{0.000000in}}%
\pgfusepath{stroke,fill}%
}%
\begin{pgfscope}%
\pgfsys@transformshift{0.625831in}{3.252077in}%
\pgfsys@useobject{currentmarker}{}%
\end{pgfscope}%
\end{pgfscope}%
\begin{pgfscope}%
\pgftext[x=0.389720in,y=3.203859in,left,base]{\rmfamily\fontsize{10.000000}{12.000000}\selectfont \(\displaystyle 30\)}%
\end{pgfscope}%
\begin{pgfscope}%
\pgftext[x=0.226139in,y=1.877783in,,bottom,rotate=90.000000]{\rmfamily\fontsize{10.000000}{12.000000}\selectfont \(\displaystyle y_1\)}%
\end{pgfscope}%
\begin{pgfscope}%
\pgfpathrectangle{\pgfqpoint{0.625831in}{0.505056in}}{\pgfqpoint{4.227273in}{2.745455in}} %
\pgfusepath{clip}%
\pgfsetrectcap%
\pgfsetroundjoin%
\pgfsetlinewidth{0.501875pt}%
\definecolor{currentstroke}{rgb}{0.500000,0.000000,1.000000}%
\pgfsetstrokecolor{currentstroke}%
\pgfsetdash{}{0pt}%
\pgfpathmoveto{\pgfqpoint{0.817980in}{1.843684in}}%
\pgfpathlineto{\pgfqpoint{0.956883in}{2.065785in}}%
\pgfpathlineto{\pgfqpoint{1.034051in}{2.183331in}}%
\pgfpathlineto{\pgfqpoint{1.095785in}{2.271714in}}%
\pgfpathlineto{\pgfqpoint{1.142086in}{2.333739in}}%
\pgfpathlineto{\pgfqpoint{1.188387in}{2.391466in}}%
\pgfpathlineto{\pgfqpoint{1.234688in}{2.444359in}}%
\pgfpathlineto{\pgfqpoint{1.280989in}{2.491928in}}%
\pgfpathlineto{\pgfqpoint{1.327290in}{2.533729in}}%
\pgfpathlineto{\pgfqpoint{1.358157in}{2.558198in}}%
\pgfpathlineto{\pgfqpoint{1.389025in}{2.579830in}}%
\pgfpathlineto{\pgfqpoint{1.419892in}{2.598534in}}%
\pgfpathlineto{\pgfqpoint{1.450759in}{2.614233in}}%
\pgfpathlineto{\pgfqpoint{1.481626in}{2.626863in}}%
\pgfpathlineto{\pgfqpoint{1.512494in}{2.636372in}}%
\pgfpathlineto{\pgfqpoint{1.543361in}{2.642719in}}%
\pgfpathlineto{\pgfqpoint{1.574228in}{2.645880in}}%
\pgfpathlineto{\pgfqpoint{1.605095in}{2.645840in}}%
\pgfpathlineto{\pgfqpoint{1.635963in}{2.642600in}}%
\pgfpathlineto{\pgfqpoint{1.666830in}{2.636174in}}%
\pgfpathlineto{\pgfqpoint{1.697697in}{2.626587in}}%
\pgfpathlineto{\pgfqpoint{1.728564in}{2.613880in}}%
\pgfpathlineto{\pgfqpoint{1.759432in}{2.598105in}}%
\pgfpathlineto{\pgfqpoint{1.790299in}{2.579327in}}%
\pgfpathlineto{\pgfqpoint{1.821166in}{2.557624in}}%
\pgfpathlineto{\pgfqpoint{1.852034in}{2.533085in}}%
\pgfpathlineto{\pgfqpoint{1.898334in}{2.491185in}}%
\pgfpathlineto{\pgfqpoint{1.944635in}{2.443525in}}%
\pgfpathlineto{\pgfqpoint{1.990936in}{2.390547in}}%
\pgfpathlineto{\pgfqpoint{2.037237in}{2.332744in}}%
\pgfpathlineto{\pgfqpoint{2.083538in}{2.270653in}}%
\pgfpathlineto{\pgfqpoint{2.145273in}{2.182197in}}%
\pgfpathlineto{\pgfqpoint{2.222441in}{2.064586in}}%
\pgfpathlineto{\pgfqpoint{2.330476in}{1.892262in}}%
\pgfpathlineto{\pgfqpoint{2.469379in}{1.670369in}}%
\pgfpathlineto{\pgfqpoint{2.546547in}{1.553163in}}%
\pgfpathlineto{\pgfqpoint{2.608282in}{1.465161in}}%
\pgfpathlineto{\pgfqpoint{2.654583in}{1.403482in}}%
\pgfpathlineto{\pgfqpoint{2.700884in}{1.346147in}}%
\pgfpathlineto{\pgfqpoint{2.747184in}{1.293690in}}%
\pgfpathlineto{\pgfqpoint{2.793485in}{1.246598in}}%
\pgfpathlineto{\pgfqpoint{2.839786in}{1.205309in}}%
\pgfpathlineto{\pgfqpoint{2.870654in}{1.181197in}}%
\pgfpathlineto{\pgfqpoint{2.901521in}{1.159936in}}%
\pgfpathlineto{\pgfqpoint{2.932388in}{1.141613in}}%
\pgfpathlineto{\pgfqpoint{2.963255in}{1.126304in}}%
\pgfpathlineto{\pgfqpoint{2.994123in}{1.114071in}}%
\pgfpathlineto{\pgfqpoint{3.024990in}{1.104966in}}%
\pgfpathlineto{\pgfqpoint{3.055857in}{1.099026in}}%
\pgfpathlineto{\pgfqpoint{3.086724in}{1.096275in}}%
\pgfpathlineto{\pgfqpoint{3.117592in}{1.096725in}}%
\pgfpathlineto{\pgfqpoint{3.148459in}{1.100375in}}%
\pgfpathlineto{\pgfqpoint{3.179326in}{1.107208in}}%
\pgfpathlineto{\pgfqpoint{3.210193in}{1.117197in}}%
\pgfpathlineto{\pgfqpoint{3.241061in}{1.130300in}}%
\pgfpathlineto{\pgfqpoint{3.271928in}{1.146464in}}%
\pgfpathlineto{\pgfqpoint{3.302795in}{1.165622in}}%
\pgfpathlineto{\pgfqpoint{3.333663in}{1.187694in}}%
\pgfpathlineto{\pgfqpoint{3.364530in}{1.212589in}}%
\pgfpathlineto{\pgfqpoint{3.410831in}{1.254997in}}%
\pgfpathlineto{\pgfqpoint{3.457132in}{1.303130in}}%
\pgfpathlineto{\pgfqpoint{3.503433in}{1.356539in}}%
\pgfpathlineto{\pgfqpoint{3.549733in}{1.414730in}}%
\pgfpathlineto{\pgfqpoint{3.596034in}{1.477162in}}%
\pgfpathlineto{\pgfqpoint{3.657769in}{1.565992in}}%
\pgfpathlineto{\pgfqpoint{3.734937in}{1.683933in}}%
\pgfpathlineto{\pgfqpoint{3.842973in}{1.856451in}}%
\pgfpathlineto{\pgfqpoint{3.981875in}{2.078121in}}%
\pgfpathlineto{\pgfqpoint{4.059043in}{2.194978in}}%
\pgfpathlineto{\pgfqpoint{4.120778in}{2.282594in}}%
\pgfpathlineto{\pgfqpoint{4.167079in}{2.343924in}}%
\pgfpathlineto{\pgfqpoint{4.213380in}{2.400862in}}%
\pgfpathlineto{\pgfqpoint{4.259681in}{2.452879in}}%
\pgfpathlineto{\pgfqpoint{4.305982in}{2.499491in}}%
\pgfpathlineto{\pgfqpoint{4.352283in}{2.540266in}}%
\pgfpathlineto{\pgfqpoint{4.383150in}{2.564018in}}%
\pgfpathlineto{\pgfqpoint{4.414017in}{2.584907in}}%
\pgfpathlineto{\pgfqpoint{4.444884in}{2.602848in}}%
\pgfpathlineto{\pgfqpoint{4.475752in}{2.617766in}}%
\pgfpathlineto{\pgfqpoint{4.506619in}{2.629600in}}%
\pgfpathlineto{\pgfqpoint{4.537486in}{2.638302in}}%
\pgfpathlineto{\pgfqpoint{4.568353in}{2.643834in}}%
\pgfpathlineto{\pgfqpoint{4.599221in}{2.646175in}}%
\pgfpathlineto{\pgfqpoint{4.630088in}{2.645314in}}%
\pgfpathlineto{\pgfqpoint{4.660955in}{2.641256in}}%
\pgfpathlineto{\pgfqpoint{4.660955in}{2.641256in}}%
\pgfusepath{stroke}%
\end{pgfscope}%
\begin{pgfscope}%
\pgfpathrectangle{\pgfqpoint{0.625831in}{0.505056in}}{\pgfqpoint{4.227273in}{2.745455in}} %
\pgfusepath{clip}%
\pgfsetrectcap%
\pgfsetroundjoin%
\pgfsetlinewidth{0.501875pt}%
\definecolor{currentstroke}{rgb}{0.421569,0.122888,0.998103}%
\pgfsetstrokecolor{currentstroke}%
\pgfsetdash{}{0pt}%
\pgfpathmoveto{\pgfqpoint{0.817980in}{1.843684in}}%
\pgfpathlineto{\pgfqpoint{0.956883in}{2.065785in}}%
\pgfpathlineto{\pgfqpoint{1.034051in}{2.183331in}}%
\pgfpathlineto{\pgfqpoint{1.095785in}{2.271714in}}%
\pgfpathlineto{\pgfqpoint{1.142086in}{2.333739in}}%
\pgfpathlineto{\pgfqpoint{1.188387in}{2.391466in}}%
\pgfpathlineto{\pgfqpoint{1.234688in}{2.444359in}}%
\pgfpathlineto{\pgfqpoint{1.280989in}{2.491928in}}%
\pgfpathlineto{\pgfqpoint{1.327290in}{2.533729in}}%
\pgfpathlineto{\pgfqpoint{1.358157in}{2.558198in}}%
\pgfpathlineto{\pgfqpoint{1.389025in}{2.579830in}}%
\pgfpathlineto{\pgfqpoint{1.419892in}{2.598534in}}%
\pgfpathlineto{\pgfqpoint{1.450759in}{2.614233in}}%
\pgfpathlineto{\pgfqpoint{1.481626in}{2.626863in}}%
\pgfpathlineto{\pgfqpoint{1.512494in}{2.636372in}}%
\pgfpathlineto{\pgfqpoint{1.543361in}{2.642719in}}%
\pgfpathlineto{\pgfqpoint{1.574228in}{2.645880in}}%
\pgfpathlineto{\pgfqpoint{1.605095in}{2.645840in}}%
\pgfpathlineto{\pgfqpoint{1.635963in}{2.642600in}}%
\pgfpathlineto{\pgfqpoint{1.666830in}{2.636174in}}%
\pgfpathlineto{\pgfqpoint{1.697697in}{2.626587in}}%
\pgfpathlineto{\pgfqpoint{1.728564in}{2.613880in}}%
\pgfpathlineto{\pgfqpoint{1.759432in}{2.598105in}}%
\pgfpathlineto{\pgfqpoint{1.790299in}{2.579327in}}%
\pgfpathlineto{\pgfqpoint{1.821166in}{2.557624in}}%
\pgfpathlineto{\pgfqpoint{1.852034in}{2.533085in}}%
\pgfpathlineto{\pgfqpoint{1.898334in}{2.491185in}}%
\pgfpathlineto{\pgfqpoint{1.944635in}{2.443525in}}%
\pgfpathlineto{\pgfqpoint{1.990936in}{2.390547in}}%
\pgfpathlineto{\pgfqpoint{2.037237in}{2.332744in}}%
\pgfpathlineto{\pgfqpoint{2.083538in}{2.270653in}}%
\pgfpathlineto{\pgfqpoint{2.145273in}{2.182197in}}%
\pgfpathlineto{\pgfqpoint{2.222441in}{2.064586in}}%
\pgfpathlineto{\pgfqpoint{2.330476in}{1.892262in}}%
\pgfpathlineto{\pgfqpoint{2.469379in}{1.670369in}}%
\pgfpathlineto{\pgfqpoint{2.546547in}{1.553163in}}%
\pgfpathlineto{\pgfqpoint{2.608282in}{1.465161in}}%
\pgfpathlineto{\pgfqpoint{2.654583in}{1.403482in}}%
\pgfpathlineto{\pgfqpoint{2.700884in}{1.346147in}}%
\pgfpathlineto{\pgfqpoint{2.747184in}{1.293690in}}%
\pgfpathlineto{\pgfqpoint{2.793485in}{1.246598in}}%
\pgfpathlineto{\pgfqpoint{2.839786in}{1.205309in}}%
\pgfpathlineto{\pgfqpoint{2.870654in}{1.181197in}}%
\pgfpathlineto{\pgfqpoint{2.901521in}{1.159936in}}%
\pgfpathlineto{\pgfqpoint{2.932388in}{1.141613in}}%
\pgfpathlineto{\pgfqpoint{2.963255in}{1.126304in}}%
\pgfpathlineto{\pgfqpoint{2.994123in}{1.114071in}}%
\pgfpathlineto{\pgfqpoint{3.024990in}{1.104966in}}%
\pgfpathlineto{\pgfqpoint{3.055857in}{1.099026in}}%
\pgfpathlineto{\pgfqpoint{3.086724in}{1.096275in}}%
\pgfpathlineto{\pgfqpoint{3.117592in}{1.096725in}}%
\pgfpathlineto{\pgfqpoint{3.148459in}{1.100375in}}%
\pgfpathlineto{\pgfqpoint{3.179326in}{1.107208in}}%
\pgfpathlineto{\pgfqpoint{3.210193in}{1.117197in}}%
\pgfpathlineto{\pgfqpoint{3.241061in}{1.130300in}}%
\pgfpathlineto{\pgfqpoint{3.271928in}{1.146464in}}%
\pgfpathlineto{\pgfqpoint{3.302795in}{1.165622in}}%
\pgfpathlineto{\pgfqpoint{3.333663in}{1.187694in}}%
\pgfpathlineto{\pgfqpoint{3.364530in}{1.212589in}}%
\pgfpathlineto{\pgfqpoint{3.410831in}{1.254997in}}%
\pgfpathlineto{\pgfqpoint{3.457132in}{1.303130in}}%
\pgfpathlineto{\pgfqpoint{3.503433in}{1.356539in}}%
\pgfpathlineto{\pgfqpoint{3.549733in}{1.414730in}}%
\pgfpathlineto{\pgfqpoint{3.596034in}{1.477162in}}%
\pgfpathlineto{\pgfqpoint{3.657769in}{1.565992in}}%
\pgfpathlineto{\pgfqpoint{3.734937in}{1.683933in}}%
\pgfpathlineto{\pgfqpoint{3.842973in}{1.856451in}}%
\pgfpathlineto{\pgfqpoint{3.981875in}{2.078121in}}%
\pgfpathlineto{\pgfqpoint{4.059043in}{2.194978in}}%
\pgfpathlineto{\pgfqpoint{4.120778in}{2.282594in}}%
\pgfpathlineto{\pgfqpoint{4.167079in}{2.343924in}}%
\pgfpathlineto{\pgfqpoint{4.213380in}{2.400862in}}%
\pgfpathlineto{\pgfqpoint{4.259681in}{2.452879in}}%
\pgfpathlineto{\pgfqpoint{4.305982in}{2.499491in}}%
\pgfpathlineto{\pgfqpoint{4.352283in}{2.540266in}}%
\pgfpathlineto{\pgfqpoint{4.383150in}{2.564018in}}%
\pgfpathlineto{\pgfqpoint{4.414017in}{2.584907in}}%
\pgfpathlineto{\pgfqpoint{4.444884in}{2.602848in}}%
\pgfpathlineto{\pgfqpoint{4.475752in}{2.617766in}}%
\pgfpathlineto{\pgfqpoint{4.506619in}{2.629600in}}%
\pgfpathlineto{\pgfqpoint{4.537486in}{2.638302in}}%
\pgfpathlineto{\pgfqpoint{4.568353in}{2.643834in}}%
\pgfpathlineto{\pgfqpoint{4.599221in}{2.646175in}}%
\pgfpathlineto{\pgfqpoint{4.630088in}{2.645314in}}%
\pgfpathlineto{\pgfqpoint{4.660955in}{2.641256in}}%
\pgfpathlineto{\pgfqpoint{4.660955in}{2.641256in}}%
\pgfusepath{stroke}%
\end{pgfscope}%
\begin{pgfscope}%
\pgfpathrectangle{\pgfqpoint{0.625831in}{0.505056in}}{\pgfqpoint{4.227273in}{2.745455in}} %
\pgfusepath{clip}%
\pgfsetrectcap%
\pgfsetroundjoin%
\pgfsetlinewidth{0.501875pt}%
\definecolor{currentstroke}{rgb}{0.343137,0.243914,0.992421}%
\pgfsetstrokecolor{currentstroke}%
\pgfsetdash{}{0pt}%
\pgfpathmoveto{\pgfqpoint{0.817980in}{2.588088in}}%
\pgfpathlineto{\pgfqpoint{0.833414in}{2.622201in}}%
\pgfpathlineto{\pgfqpoint{0.848847in}{2.635404in}}%
\pgfpathlineto{\pgfqpoint{0.864281in}{2.626397in}}%
\pgfpathlineto{\pgfqpoint{0.879715in}{2.594582in}}%
\pgfpathlineto{\pgfqpoint{0.895148in}{2.540089in}}%
\pgfpathlineto{\pgfqpoint{0.910582in}{2.463776in}}%
\pgfpathlineto{\pgfqpoint{0.926015in}{2.367207in}}%
\pgfpathlineto{\pgfqpoint{0.941449in}{2.252608in}}%
\pgfpathlineto{\pgfqpoint{0.972316in}{1.981110in}}%
\pgfpathlineto{\pgfqpoint{1.049485in}{1.232793in}}%
\pgfpathlineto{\pgfqpoint{1.064918in}{1.104031in}}%
\pgfpathlineto{\pgfqpoint{1.080352in}{0.991180in}}%
\pgfpathlineto{\pgfqpoint{1.095785in}{0.897341in}}%
\pgfpathlineto{\pgfqpoint{1.111219in}{0.825059in}}%
\pgfpathlineto{\pgfqpoint{1.126653in}{0.776243in}}%
\pgfpathlineto{\pgfqpoint{1.142086in}{0.752109in}}%
\pgfpathlineto{\pgfqpoint{1.157520in}{0.753139in}}%
\pgfpathlineto{\pgfqpoint{1.172954in}{0.779072in}}%
\pgfpathlineto{\pgfqpoint{1.188387in}{0.828912in}}%
\pgfpathlineto{\pgfqpoint{1.203821in}{0.900959in}}%
\pgfpathlineto{\pgfqpoint{1.219255in}{0.992868in}}%
\pgfpathlineto{\pgfqpoint{1.234688in}{1.101723in}}%
\pgfpathlineto{\pgfqpoint{1.265555in}{1.356325in}}%
\pgfpathlineto{\pgfqpoint{1.327290in}{1.901829in}}%
\pgfpathlineto{\pgfqpoint{1.342724in}{2.022411in}}%
\pgfpathlineto{\pgfqpoint{1.358157in}{2.129522in}}%
\pgfpathlineto{\pgfqpoint{1.373591in}{2.220330in}}%
\pgfpathlineto{\pgfqpoint{1.389025in}{2.292575in}}%
\pgfpathlineto{\pgfqpoint{1.404458in}{2.344644in}}%
\pgfpathlineto{\pgfqpoint{1.419892in}{2.375615in}}%
\pgfpathlineto{\pgfqpoint{1.435325in}{2.385283in}}%
\pgfpathlineto{\pgfqpoint{1.450759in}{2.374165in}}%
\pgfpathlineto{\pgfqpoint{1.466193in}{2.343472in}}%
\pgfpathlineto{\pgfqpoint{1.481626in}{2.295070in}}%
\pgfpathlineto{\pgfqpoint{1.497060in}{2.231412in}}%
\pgfpathlineto{\pgfqpoint{1.512494in}{2.155453in}}%
\pgfpathlineto{\pgfqpoint{1.543361in}{1.980348in}}%
\pgfpathlineto{\pgfqpoint{1.574228in}{1.799314in}}%
\pgfpathlineto{\pgfqpoint{1.589662in}{1.716076in}}%
\pgfpathlineto{\pgfqpoint{1.605095in}{1.642472in}}%
\pgfpathlineto{\pgfqpoint{1.620529in}{1.581690in}}%
\pgfpathlineto{\pgfqpoint{1.635963in}{1.536472in}}%
\pgfpathlineto{\pgfqpoint{1.651396in}{1.509019in}}%
\pgfpathlineto{\pgfqpoint{1.666830in}{1.500915in}}%
\pgfpathlineto{\pgfqpoint{1.682264in}{1.513071in}}%
\pgfpathlineto{\pgfqpoint{1.697697in}{1.545692in}}%
\pgfpathlineto{\pgfqpoint{1.713131in}{1.598269in}}%
\pgfpathlineto{\pgfqpoint{1.728564in}{1.669586in}}%
\pgfpathlineto{\pgfqpoint{1.743998in}{1.757764in}}%
\pgfpathlineto{\pgfqpoint{1.759432in}{1.860314in}}%
\pgfpathlineto{\pgfqpoint{1.790299in}{2.096014in}}%
\pgfpathlineto{\pgfqpoint{1.836600in}{2.470117in}}%
\pgfpathlineto{\pgfqpoint{1.852034in}{2.584325in}}%
\pgfpathlineto{\pgfqpoint{1.867467in}{2.686795in}}%
\pgfpathlineto{\pgfqpoint{1.882901in}{2.774017in}}%
\pgfpathlineto{\pgfqpoint{1.898334in}{2.842911in}}%
\pgfpathlineto{\pgfqpoint{1.913768in}{2.890917in}}%
\pgfpathlineto{\pgfqpoint{1.929202in}{2.916089in}}%
\pgfpathlineto{\pgfqpoint{1.944635in}{2.917154in}}%
\pgfpathlineto{\pgfqpoint{1.960069in}{2.893558in}}%
\pgfpathlineto{\pgfqpoint{1.975503in}{2.845490in}}%
\pgfpathlineto{\pgfqpoint{1.990936in}{2.773874in}}%
\pgfpathlineto{\pgfqpoint{2.006370in}{2.680346in}}%
\pgfpathlineto{\pgfqpoint{2.021804in}{2.567207in}}%
\pgfpathlineto{\pgfqpoint{2.037237in}{2.437351in}}%
\pgfpathlineto{\pgfqpoint{2.068104in}{2.141474in}}%
\pgfpathlineto{\pgfqpoint{2.129839in}{1.518583in}}%
\pgfpathlineto{\pgfqpoint{2.145273in}{1.380601in}}%
\pgfpathlineto{\pgfqpoint{2.160706in}{1.257293in}}%
\pgfpathlineto{\pgfqpoint{2.176140in}{1.151750in}}%
\pgfpathlineto{\pgfqpoint{2.191574in}{1.066499in}}%
\pgfpathlineto{\pgfqpoint{2.207007in}{1.003433in}}%
\pgfpathlineto{\pgfqpoint{2.222441in}{0.963750in}}%
\pgfpathlineto{\pgfqpoint{2.237874in}{0.947922in}}%
\pgfpathlineto{\pgfqpoint{2.253308in}{0.955684in}}%
\pgfpathlineto{\pgfqpoint{2.268742in}{0.986046in}}%
\pgfpathlineto{\pgfqpoint{2.284175in}{1.037333in}}%
\pgfpathlineto{\pgfqpoint{2.299609in}{1.107240in}}%
\pgfpathlineto{\pgfqpoint{2.315043in}{1.192911in}}%
\pgfpathlineto{\pgfqpoint{2.345910in}{1.397969in}}%
\pgfpathlineto{\pgfqpoint{2.407644in}{1.835544in}}%
\pgfpathlineto{\pgfqpoint{2.423078in}{1.928384in}}%
\pgfpathlineto{\pgfqpoint{2.438512in}{2.007905in}}%
\pgfpathlineto{\pgfqpoint{2.453945in}{2.071524in}}%
\pgfpathlineto{\pgfqpoint{2.469379in}{2.117245in}}%
\pgfpathlineto{\pgfqpoint{2.484813in}{2.143718in}}%
\pgfpathlineto{\pgfqpoint{2.500246in}{2.150288in}}%
\pgfpathlineto{\pgfqpoint{2.515680in}{2.137009in}}%
\pgfpathlineto{\pgfqpoint{2.531114in}{2.104644in}}%
\pgfpathlineto{\pgfqpoint{2.546547in}{2.054640in}}%
\pgfpathlineto{\pgfqpoint{2.561981in}{1.989071in}}%
\pgfpathlineto{\pgfqpoint{2.577414in}{1.910579in}}%
\pgfpathlineto{\pgfqpoint{2.608282in}{1.727644in}}%
\pgfpathlineto{\pgfqpoint{2.654583in}{1.443658in}}%
\pgfpathlineto{\pgfqpoint{2.670016in}{1.361714in}}%
\pgfpathlineto{\pgfqpoint{2.685450in}{1.292118in}}%
\pgfpathlineto{\pgfqpoint{2.700884in}{1.237973in}}%
\pgfpathlineto{\pgfqpoint{2.716317in}{1.201903in}}%
\pgfpathlineto{\pgfqpoint{2.731751in}{1.185964in}}%
\pgfpathlineto{\pgfqpoint{2.747184in}{1.191570in}}%
\pgfpathlineto{\pgfqpoint{2.762618in}{1.219442in}}%
\pgfpathlineto{\pgfqpoint{2.778052in}{1.269580in}}%
\pgfpathlineto{\pgfqpoint{2.793485in}{1.341259in}}%
\pgfpathlineto{\pgfqpoint{2.808919in}{1.433043in}}%
\pgfpathlineto{\pgfqpoint{2.824353in}{1.542827in}}%
\pgfpathlineto{\pgfqpoint{2.855220in}{1.805040in}}%
\pgfpathlineto{\pgfqpoint{2.932388in}{2.535463in}}%
\pgfpathlineto{\pgfqpoint{2.947822in}{2.661765in}}%
\pgfpathlineto{\pgfqpoint{2.963255in}{2.772481in}}%
\pgfpathlineto{\pgfqpoint{2.978689in}{2.864453in}}%
\pgfpathlineto{\pgfqpoint{2.994123in}{2.935064in}}%
\pgfpathlineto{\pgfqpoint{3.009556in}{2.982322in}}%
\pgfpathlineto{\pgfqpoint{3.024990in}{3.004920in}}%
\pgfpathlineto{\pgfqpoint{3.040424in}{3.002278in}}%
\pgfpathlineto{\pgfqpoint{3.055857in}{2.974564in}}%
\pgfpathlineto{\pgfqpoint{3.071291in}{2.922682in}}%
\pgfpathlineto{\pgfqpoint{3.086724in}{2.848248in}}%
\pgfpathlineto{\pgfqpoint{3.102158in}{2.753534in}}%
\pgfpathlineto{\pgfqpoint{3.117592in}{2.641401in}}%
\pgfpathlineto{\pgfqpoint{3.148459in}{2.378673in}}%
\pgfpathlineto{\pgfqpoint{3.210193in}{1.810836in}}%
\pgfpathlineto{\pgfqpoint{3.225627in}{1.683695in}}%
\pgfpathlineto{\pgfqpoint{3.241061in}{1.569813in}}%
\pgfpathlineto{\pgfqpoint{3.256494in}{1.472145in}}%
\pgfpathlineto{\pgfqpoint{3.271928in}{1.393082in}}%
\pgfpathlineto{\pgfqpoint{3.287362in}{1.334377in}}%
\pgfpathlineto{\pgfqpoint{3.302795in}{1.297097in}}%
\pgfpathlineto{\pgfqpoint{3.318229in}{1.281590in}}%
\pgfpathlineto{\pgfqpoint{3.333663in}{1.287482in}}%
\pgfpathlineto{\pgfqpoint{3.349096in}{1.313691in}}%
\pgfpathlineto{\pgfqpoint{3.364530in}{1.358472in}}%
\pgfpathlineto{\pgfqpoint{3.379963in}{1.419475in}}%
\pgfpathlineto{\pgfqpoint{3.395397in}{1.493828in}}%
\pgfpathlineto{\pgfqpoint{3.426264in}{1.669094in}}%
\pgfpathlineto{\pgfqpoint{3.472565in}{1.942206in}}%
\pgfpathlineto{\pgfqpoint{3.487999in}{2.020918in}}%
\pgfpathlineto{\pgfqpoint{3.503433in}{2.087759in}}%
\pgfpathlineto{\pgfqpoint{3.518866in}{2.139879in}}%
\pgfpathlineto{\pgfqpoint{3.534300in}{2.174949in}}%
\pgfpathlineto{\pgfqpoint{3.549733in}{2.191245in}}%
\pgfpathlineto{\pgfqpoint{3.565167in}{2.187702in}}%
\pgfpathlineto{\pgfqpoint{3.580601in}{2.163958in}}%
\pgfpathlineto{\pgfqpoint{3.596034in}{2.120361in}}%
\pgfpathlineto{\pgfqpoint{3.611468in}{2.057970in}}%
\pgfpathlineto{\pgfqpoint{3.626902in}{1.978513in}}%
\pgfpathlineto{\pgfqpoint{3.642335in}{1.884340in}}%
\pgfpathlineto{\pgfqpoint{3.673203in}{1.663873in}}%
\pgfpathlineto{\pgfqpoint{3.734937in}{1.197677in}}%
\pgfpathlineto{\pgfqpoint{3.750371in}{1.098786in}}%
\pgfpathlineto{\pgfqpoint{3.765804in}{1.014382in}}%
\pgfpathlineto{\pgfqpoint{3.781238in}{0.947596in}}%
\pgfpathlineto{\pgfqpoint{3.796672in}{0.901051in}}%
\pgfpathlineto{\pgfqpoint{3.812105in}{0.876777in}}%
\pgfpathlineto{\pgfqpoint{3.827539in}{0.876139in}}%
\pgfpathlineto{\pgfqpoint{3.842973in}{0.899792in}}%
\pgfpathlineto{\pgfqpoint{3.858406in}{0.947653in}}%
\pgfpathlineto{\pgfqpoint{3.873840in}{1.018903in}}%
\pgfpathlineto{\pgfqpoint{3.889273in}{1.112006in}}%
\pgfpathlineto{\pgfqpoint{3.904707in}{1.224756in}}%
\pgfpathlineto{\pgfqpoint{3.920141in}{1.354342in}}%
\pgfpathlineto{\pgfqpoint{3.951008in}{1.650283in}}%
\pgfpathlineto{\pgfqpoint{4.012743in}{2.276464in}}%
\pgfpathlineto{\pgfqpoint{4.028176in}{2.415844in}}%
\pgfpathlineto{\pgfqpoint{4.043610in}{2.540675in}}%
\pgfpathlineto{\pgfqpoint{4.059043in}{2.647783in}}%
\pgfpathlineto{\pgfqpoint{4.074477in}{2.734545in}}%
\pgfpathlineto{\pgfqpoint{4.089911in}{2.798964in}}%
\pgfpathlineto{\pgfqpoint{4.105344in}{2.839731in}}%
\pgfpathlineto{\pgfqpoint{4.120778in}{2.856263in}}%
\pgfpathlineto{\pgfqpoint{4.136212in}{2.848714in}}%
\pgfpathlineto{\pgfqpoint{4.151645in}{2.817973in}}%
\pgfpathlineto{\pgfqpoint{4.167079in}{2.765621in}}%
\pgfpathlineto{\pgfqpoint{4.182513in}{2.693885in}}%
\pgfpathlineto{\pgfqpoint{4.197946in}{2.605559in}}%
\pgfpathlineto{\pgfqpoint{4.228813in}{2.392575in}}%
\pgfpathlineto{\pgfqpoint{4.290548in}{1.928976in}}%
\pgfpathlineto{\pgfqpoint{4.305982in}{1.827851in}}%
\pgfpathlineto{\pgfqpoint{4.321415in}{1.739540in}}%
\pgfpathlineto{\pgfqpoint{4.336849in}{1.666759in}}%
\pgfpathlineto{\pgfqpoint{4.352283in}{1.611653in}}%
\pgfpathlineto{\pgfqpoint{4.367716in}{1.575727in}}%
\pgfpathlineto{\pgfqpoint{4.383150in}{1.559802in}}%
\pgfpathlineto{\pgfqpoint{4.398583in}{1.563984in}}%
\pgfpathlineto{\pgfqpoint{4.414017in}{1.587672in}}%
\pgfpathlineto{\pgfqpoint{4.429451in}{1.629571in}}%
\pgfpathlineto{\pgfqpoint{4.444884in}{1.687745in}}%
\pgfpathlineto{\pgfqpoint{4.460318in}{1.759679in}}%
\pgfpathlineto{\pgfqpoint{4.491185in}{1.932402in}}%
\pgfpathlineto{\pgfqpoint{4.537486in}{2.209216in}}%
\pgfpathlineto{\pgfqpoint{4.552920in}{2.290999in}}%
\pgfpathlineto{\pgfqpoint{4.568353in}{2.361501in}}%
\pgfpathlineto{\pgfqpoint{4.583787in}{2.417553in}}%
\pgfpathlineto{\pgfqpoint{4.599221in}{2.456443in}}%
\pgfpathlineto{\pgfqpoint{4.614654in}{2.476013in}}%
\pgfpathlineto{\pgfqpoint{4.630088in}{2.474729in}}%
\pgfpathlineto{\pgfqpoint{4.645522in}{2.451742in}}%
\pgfpathlineto{\pgfqpoint{4.660955in}{2.406915in}}%
\pgfpathlineto{\pgfqpoint{4.660955in}{2.406915in}}%
\pgfusepath{stroke}%
\end{pgfscope}%
\begin{pgfscope}%
\pgfpathrectangle{\pgfqpoint{0.625831in}{0.505056in}}{\pgfqpoint{4.227273in}{2.745455in}} %
\pgfusepath{clip}%
\pgfsetrectcap%
\pgfsetroundjoin%
\pgfsetlinewidth{0.501875pt}%
\definecolor{currentstroke}{rgb}{0.264706,0.361242,0.982973}%
\pgfsetstrokecolor{currentstroke}%
\pgfsetdash{}{0pt}%
\pgfpathmoveto{\pgfqpoint{0.817980in}{2.442017in}}%
\pgfpathlineto{\pgfqpoint{0.833414in}{2.507179in}}%
\pgfpathlineto{\pgfqpoint{0.848847in}{2.555333in}}%
\pgfpathlineto{\pgfqpoint{0.864281in}{2.584729in}}%
\pgfpathlineto{\pgfqpoint{0.879715in}{2.594176in}}%
\pgfpathlineto{\pgfqpoint{0.895148in}{2.583072in}}%
\pgfpathlineto{\pgfqpoint{0.910582in}{2.551426in}}%
\pgfpathlineto{\pgfqpoint{0.926015in}{2.499849in}}%
\pgfpathlineto{\pgfqpoint{0.941449in}{2.429541in}}%
\pgfpathlineto{\pgfqpoint{0.956883in}{2.342247in}}%
\pgfpathlineto{\pgfqpoint{0.972316in}{2.240206in}}%
\pgfpathlineto{\pgfqpoint{1.003184in}{2.002882in}}%
\pgfpathlineto{\pgfqpoint{1.080352in}{1.368183in}}%
\pgfpathlineto{\pgfqpoint{1.095785in}{1.260902in}}%
\pgfpathlineto{\pgfqpoint{1.111219in}{1.167124in}}%
\pgfpathlineto{\pgfqpoint{1.126653in}{1.089146in}}%
\pgfpathlineto{\pgfqpoint{1.142086in}{1.028802in}}%
\pgfpathlineto{\pgfqpoint{1.157520in}{0.987409in}}%
\pgfpathlineto{\pgfqpoint{1.172954in}{0.965740in}}%
\pgfpathlineto{\pgfqpoint{1.188387in}{0.964006in}}%
\pgfpathlineto{\pgfqpoint{1.203821in}{0.981861in}}%
\pgfpathlineto{\pgfqpoint{1.219255in}{1.018420in}}%
\pgfpathlineto{\pgfqpoint{1.234688in}{1.072294in}}%
\pgfpathlineto{\pgfqpoint{1.250122in}{1.141643in}}%
\pgfpathlineto{\pgfqpoint{1.265555in}{1.224233in}}%
\pgfpathlineto{\pgfqpoint{1.296423in}{1.418715in}}%
\pgfpathlineto{\pgfqpoint{1.373591in}{1.940929in}}%
\pgfpathlineto{\pgfqpoint{1.389025in}{2.029345in}}%
\pgfpathlineto{\pgfqpoint{1.404458in}{2.107081in}}%
\pgfpathlineto{\pgfqpoint{1.419892in}{2.172554in}}%
\pgfpathlineto{\pgfqpoint{1.435325in}{2.224622in}}%
\pgfpathlineto{\pgfqpoint{1.450759in}{2.262600in}}%
\pgfpathlineto{\pgfqpoint{1.466193in}{2.286276in}}%
\pgfpathlineto{\pgfqpoint{1.481626in}{2.295900in}}%
\pgfpathlineto{\pgfqpoint{1.497060in}{2.292166in}}%
\pgfpathlineto{\pgfqpoint{1.512494in}{2.276173in}}%
\pgfpathlineto{\pgfqpoint{1.527927in}{2.249377in}}%
\pgfpathlineto{\pgfqpoint{1.543361in}{2.213539in}}%
\pgfpathlineto{\pgfqpoint{1.558795in}{2.170648in}}%
\pgfpathlineto{\pgfqpoint{1.589662in}{2.072400in}}%
\pgfpathlineto{\pgfqpoint{1.620529in}{1.972367in}}%
\pgfpathlineto{\pgfqpoint{1.635963in}{1.926976in}}%
\pgfpathlineto{\pgfqpoint{1.651396in}{1.887152in}}%
\pgfpathlineto{\pgfqpoint{1.666830in}{1.854432in}}%
\pgfpathlineto{\pgfqpoint{1.682264in}{1.830035in}}%
\pgfpathlineto{\pgfqpoint{1.697697in}{1.814828in}}%
\pgfpathlineto{\pgfqpoint{1.713131in}{1.809300in}}%
\pgfpathlineto{\pgfqpoint{1.728564in}{1.813552in}}%
\pgfpathlineto{\pgfqpoint{1.743998in}{1.827303in}}%
\pgfpathlineto{\pgfqpoint{1.759432in}{1.849904in}}%
\pgfpathlineto{\pgfqpoint{1.774865in}{1.880370in}}%
\pgfpathlineto{\pgfqpoint{1.790299in}{1.917419in}}%
\pgfpathlineto{\pgfqpoint{1.821166in}{2.004971in}}%
\pgfpathlineto{\pgfqpoint{1.867467in}{2.142778in}}%
\pgfpathlineto{\pgfqpoint{1.882901in}{2.183007in}}%
\pgfpathlineto{\pgfqpoint{1.898334in}{2.217520in}}%
\pgfpathlineto{\pgfqpoint{1.913768in}{2.244870in}}%
\pgfpathlineto{\pgfqpoint{1.929202in}{2.263864in}}%
\pgfpathlineto{\pgfqpoint{1.944635in}{2.273606in}}%
\pgfpathlineto{\pgfqpoint{1.960069in}{2.273527in}}%
\pgfpathlineto{\pgfqpoint{1.975503in}{2.263403in}}%
\pgfpathlineto{\pgfqpoint{1.990936in}{2.243358in}}%
\pgfpathlineto{\pgfqpoint{2.006370in}{2.213866in}}%
\pgfpathlineto{\pgfqpoint{2.021804in}{2.175723in}}%
\pgfpathlineto{\pgfqpoint{2.037237in}{2.130023in}}%
\pgfpathlineto{\pgfqpoint{2.068104in}{2.021558in}}%
\pgfpathlineto{\pgfqpoint{2.129839in}{1.783864in}}%
\pgfpathlineto{\pgfqpoint{2.145273in}{1.730432in}}%
\pgfpathlineto{\pgfqpoint{2.160706in}{1.682622in}}%
\pgfpathlineto{\pgfqpoint{2.176140in}{1.641723in}}%
\pgfpathlineto{\pgfqpoint{2.191574in}{1.608760in}}%
\pgfpathlineto{\pgfqpoint{2.207007in}{1.584461in}}%
\pgfpathlineto{\pgfqpoint{2.222441in}{1.569225in}}%
\pgfpathlineto{\pgfqpoint{2.237874in}{1.563117in}}%
\pgfpathlineto{\pgfqpoint{2.253308in}{1.565860in}}%
\pgfpathlineto{\pgfqpoint{2.268742in}{1.576851in}}%
\pgfpathlineto{\pgfqpoint{2.284175in}{1.595179in}}%
\pgfpathlineto{\pgfqpoint{2.299609in}{1.619664in}}%
\pgfpathlineto{\pgfqpoint{2.330476in}{1.681305in}}%
\pgfpathlineto{\pgfqpoint{2.376777in}{1.780406in}}%
\pgfpathlineto{\pgfqpoint{2.392211in}{1.808371in}}%
\pgfpathlineto{\pgfqpoint{2.407644in}{1.831190in}}%
\pgfpathlineto{\pgfqpoint{2.423078in}{1.847569in}}%
\pgfpathlineto{\pgfqpoint{2.438512in}{1.856473in}}%
\pgfpathlineto{\pgfqpoint{2.453945in}{1.857169in}}%
\pgfpathlineto{\pgfqpoint{2.469379in}{1.849259in}}%
\pgfpathlineto{\pgfqpoint{2.484813in}{1.832705in}}%
\pgfpathlineto{\pgfqpoint{2.500246in}{1.807833in}}%
\pgfpathlineto{\pgfqpoint{2.515680in}{1.775334in}}%
\pgfpathlineto{\pgfqpoint{2.531114in}{1.736245in}}%
\pgfpathlineto{\pgfqpoint{2.561981in}{1.643984in}}%
\pgfpathlineto{\pgfqpoint{2.608282in}{1.497776in}}%
\pgfpathlineto{\pgfqpoint{2.623715in}{1.455190in}}%
\pgfpathlineto{\pgfqpoint{2.639149in}{1.419166in}}%
\pgfpathlineto{\pgfqpoint{2.654583in}{1.391652in}}%
\pgfpathlineto{\pgfqpoint{2.670016in}{1.374390in}}%
\pgfpathlineto{\pgfqpoint{2.685450in}{1.368856in}}%
\pgfpathlineto{\pgfqpoint{2.700884in}{1.376199in}}%
\pgfpathlineto{\pgfqpoint{2.716317in}{1.397190in}}%
\pgfpathlineto{\pgfqpoint{2.731751in}{1.432187in}}%
\pgfpathlineto{\pgfqpoint{2.747184in}{1.481104in}}%
\pgfpathlineto{\pgfqpoint{2.762618in}{1.543406in}}%
\pgfpathlineto{\pgfqpoint{2.778052in}{1.618110in}}%
\pgfpathlineto{\pgfqpoint{2.793485in}{1.703802in}}%
\pgfpathlineto{\pgfqpoint{2.824353in}{1.900567in}}%
\pgfpathlineto{\pgfqpoint{2.901521in}{2.423455in}}%
\pgfpathlineto{\pgfqpoint{2.916954in}{2.511067in}}%
\pgfpathlineto{\pgfqpoint{2.932388in}{2.586768in}}%
\pgfpathlineto{\pgfqpoint{2.947822in}{2.648310in}}%
\pgfpathlineto{\pgfqpoint{2.963255in}{2.693806in}}%
\pgfpathlineto{\pgfqpoint{2.978689in}{2.721788in}}%
\pgfpathlineto{\pgfqpoint{2.994123in}{2.731263in}}%
\pgfpathlineto{\pgfqpoint{3.009556in}{2.721751in}}%
\pgfpathlineto{\pgfqpoint{3.024990in}{2.693305in}}%
\pgfpathlineto{\pgfqpoint{3.040424in}{2.646521in}}%
\pgfpathlineto{\pgfqpoint{3.055857in}{2.582525in}}%
\pgfpathlineto{\pgfqpoint{3.071291in}{2.502946in}}%
\pgfpathlineto{\pgfqpoint{3.086724in}{2.409876in}}%
\pgfpathlineto{\pgfqpoint{3.117592in}{2.193583in}}%
\pgfpathlineto{\pgfqpoint{3.179326in}{1.726504in}}%
\pgfpathlineto{\pgfqpoint{3.194760in}{1.621463in}}%
\pgfpathlineto{\pgfqpoint{3.210193in}{1.527230in}}%
\pgfpathlineto{\pgfqpoint{3.225627in}{1.446407in}}%
\pgfpathlineto{\pgfqpoint{3.241061in}{1.381193in}}%
\pgfpathlineto{\pgfqpoint{3.256494in}{1.333322in}}%
\pgfpathlineto{\pgfqpoint{3.271928in}{1.304004in}}%
\pgfpathlineto{\pgfqpoint{3.287362in}{1.293888in}}%
\pgfpathlineto{\pgfqpoint{3.302795in}{1.303036in}}%
\pgfpathlineto{\pgfqpoint{3.318229in}{1.330921in}}%
\pgfpathlineto{\pgfqpoint{3.333663in}{1.376437in}}%
\pgfpathlineto{\pgfqpoint{3.349096in}{1.437932in}}%
\pgfpathlineto{\pgfqpoint{3.364530in}{1.513256in}}%
\pgfpathlineto{\pgfqpoint{3.395397in}{1.694689in}}%
\pgfpathlineto{\pgfqpoint{3.457132in}{2.090928in}}%
\pgfpathlineto{\pgfqpoint{3.472565in}{2.177219in}}%
\pgfpathlineto{\pgfqpoint{3.487999in}{2.252038in}}%
\pgfpathlineto{\pgfqpoint{3.503433in}{2.312699in}}%
\pgfpathlineto{\pgfqpoint{3.518866in}{2.356936in}}%
\pgfpathlineto{\pgfqpoint{3.534300in}{2.382981in}}%
\pgfpathlineto{\pgfqpoint{3.549733in}{2.389621in}}%
\pgfpathlineto{\pgfqpoint{3.565167in}{2.376239in}}%
\pgfpathlineto{\pgfqpoint{3.580601in}{2.342835in}}%
\pgfpathlineto{\pgfqpoint{3.596034in}{2.290035in}}%
\pgfpathlineto{\pgfqpoint{3.611468in}{2.219070in}}%
\pgfpathlineto{\pgfqpoint{3.626902in}{2.131749in}}%
\pgfpathlineto{\pgfqpoint{3.642335in}{2.030403in}}%
\pgfpathlineto{\pgfqpoint{3.673203in}{1.797175in}}%
\pgfpathlineto{\pgfqpoint{3.734937in}{1.305006in}}%
\pgfpathlineto{\pgfqpoint{3.750371in}{1.197797in}}%
\pgfpathlineto{\pgfqpoint{3.765804in}{1.103788in}}%
\pgfpathlineto{\pgfqpoint{3.781238in}{1.025906in}}%
\pgfpathlineto{\pgfqpoint{3.796672in}{0.966644in}}%
\pgfpathlineto{\pgfqpoint{3.812105in}{0.927981in}}%
\pgfpathlineto{\pgfqpoint{3.827539in}{0.911328in}}%
\pgfpathlineto{\pgfqpoint{3.842973in}{0.917478in}}%
\pgfpathlineto{\pgfqpoint{3.858406in}{0.946588in}}%
\pgfpathlineto{\pgfqpoint{3.873840in}{0.998168in}}%
\pgfpathlineto{\pgfqpoint{3.889273in}{1.071098in}}%
\pgfpathlineto{\pgfqpoint{3.904707in}{1.163659in}}%
\pgfpathlineto{\pgfqpoint{3.920141in}{1.273581in}}%
\pgfpathlineto{\pgfqpoint{3.951008in}{1.534094in}}%
\pgfpathlineto{\pgfqpoint{4.043610in}{2.387640in}}%
\pgfpathlineto{\pgfqpoint{4.059043in}{2.502451in}}%
\pgfpathlineto{\pgfqpoint{4.074477in}{2.601057in}}%
\pgfpathlineto{\pgfqpoint{4.089911in}{2.681214in}}%
\pgfpathlineto{\pgfqpoint{4.105344in}{2.741215in}}%
\pgfpathlineto{\pgfqpoint{4.120778in}{2.779937in}}%
\pgfpathlineto{\pgfqpoint{4.136212in}{2.796865in}}%
\pgfpathlineto{\pgfqpoint{4.151645in}{2.792100in}}%
\pgfpathlineto{\pgfqpoint{4.167079in}{2.766349in}}%
\pgfpathlineto{\pgfqpoint{4.182513in}{2.720895in}}%
\pgfpathlineto{\pgfqpoint{4.197946in}{2.657553in}}%
\pgfpathlineto{\pgfqpoint{4.213380in}{2.578608in}}%
\pgfpathlineto{\pgfqpoint{4.228813in}{2.486739in}}%
\pgfpathlineto{\pgfqpoint{4.259681in}{2.276408in}}%
\pgfpathlineto{\pgfqpoint{4.305982in}{1.943666in}}%
\pgfpathlineto{\pgfqpoint{4.336849in}{1.747625in}}%
\pgfpathlineto{\pgfqpoint{4.352283in}{1.665642in}}%
\pgfpathlineto{\pgfqpoint{4.367716in}{1.597194in}}%
\pgfpathlineto{\pgfqpoint{4.383150in}{1.543836in}}%
\pgfpathlineto{\pgfqpoint{4.398583in}{1.506608in}}%
\pgfpathlineto{\pgfqpoint{4.414017in}{1.486014in}}%
\pgfpathlineto{\pgfqpoint{4.429451in}{1.482017in}}%
\pgfpathlineto{\pgfqpoint{4.444884in}{1.494050in}}%
\pgfpathlineto{\pgfqpoint{4.460318in}{1.521044in}}%
\pgfpathlineto{\pgfqpoint{4.475752in}{1.561473in}}%
\pgfpathlineto{\pgfqpoint{4.491185in}{1.613405in}}%
\pgfpathlineto{\pgfqpoint{4.506619in}{1.674577in}}%
\pgfpathlineto{\pgfqpoint{4.537486in}{1.814396in}}%
\pgfpathlineto{\pgfqpoint{4.583787in}{2.026793in}}%
\pgfpathlineto{\pgfqpoint{4.599221in}{2.087669in}}%
\pgfpathlineto{\pgfqpoint{4.614654in}{2.139677in}}%
\pgfpathlineto{\pgfqpoint{4.630088in}{2.180921in}}%
\pgfpathlineto{\pgfqpoint{4.645522in}{2.209890in}}%
\pgfpathlineto{\pgfqpoint{4.660955in}{2.225496in}}%
\pgfpathlineto{\pgfqpoint{4.660955in}{2.225496in}}%
\pgfusepath{stroke}%
\end{pgfscope}%
\begin{pgfscope}%
\pgfpathrectangle{\pgfqpoint{0.625831in}{0.505056in}}{\pgfqpoint{4.227273in}{2.745455in}} %
\pgfusepath{clip}%
\pgfsetrectcap%
\pgfsetroundjoin%
\pgfsetlinewidth{0.501875pt}%
\definecolor{currentstroke}{rgb}{0.186275,0.473094,0.969797}%
\pgfsetstrokecolor{currentstroke}%
\pgfsetdash{}{0pt}%
\pgfpathmoveto{\pgfqpoint{0.817980in}{2.219557in}}%
\pgfpathlineto{\pgfqpoint{0.833414in}{2.217647in}}%
\pgfpathlineto{\pgfqpoint{0.848847in}{2.212798in}}%
\pgfpathlineto{\pgfqpoint{0.864281in}{2.204540in}}%
\pgfpathlineto{\pgfqpoint{0.879715in}{2.192471in}}%
\pgfpathlineto{\pgfqpoint{0.895148in}{2.176272in}}%
\pgfpathlineto{\pgfqpoint{0.910582in}{2.155714in}}%
\pgfpathlineto{\pgfqpoint{0.926015in}{2.130669in}}%
\pgfpathlineto{\pgfqpoint{0.941449in}{2.101116in}}%
\pgfpathlineto{\pgfqpoint{0.956883in}{2.067144in}}%
\pgfpathlineto{\pgfqpoint{0.987750in}{1.986845in}}%
\pgfpathlineto{\pgfqpoint{1.018617in}{1.892656in}}%
\pgfpathlineto{\pgfqpoint{1.064918in}{1.735412in}}%
\pgfpathlineto{\pgfqpoint{1.111219in}{1.577054in}}%
\pgfpathlineto{\pgfqpoint{1.142086in}{1.482046in}}%
\pgfpathlineto{\pgfqpoint{1.157520in}{1.440211in}}%
\pgfpathlineto{\pgfqpoint{1.172954in}{1.403179in}}%
\pgfpathlineto{\pgfqpoint{1.188387in}{1.371636in}}%
\pgfpathlineto{\pgfqpoint{1.203821in}{1.346188in}}%
\pgfpathlineto{\pgfqpoint{1.219255in}{1.327343in}}%
\pgfpathlineto{\pgfqpoint{1.234688in}{1.315502in}}%
\pgfpathlineto{\pgfqpoint{1.250122in}{1.310951in}}%
\pgfpathlineto{\pgfqpoint{1.265555in}{1.313850in}}%
\pgfpathlineto{\pgfqpoint{1.280989in}{1.324234in}}%
\pgfpathlineto{\pgfqpoint{1.296423in}{1.342005in}}%
\pgfpathlineto{\pgfqpoint{1.311856in}{1.366941in}}%
\pgfpathlineto{\pgfqpoint{1.327290in}{1.398691in}}%
\pgfpathlineto{\pgfqpoint{1.342724in}{1.436786in}}%
\pgfpathlineto{\pgfqpoint{1.358157in}{1.480648in}}%
\pgfpathlineto{\pgfqpoint{1.389025in}{1.582875in}}%
\pgfpathlineto{\pgfqpoint{1.419892in}{1.699003in}}%
\pgfpathlineto{\pgfqpoint{1.497060in}{2.001966in}}%
\pgfpathlineto{\pgfqpoint{1.527927in}{2.109048in}}%
\pgfpathlineto{\pgfqpoint{1.543361in}{2.156355in}}%
\pgfpathlineto{\pgfqpoint{1.558795in}{2.198671in}}%
\pgfpathlineto{\pgfqpoint{1.574228in}{2.235495in}}%
\pgfpathlineto{\pgfqpoint{1.589662in}{2.266443in}}%
\pgfpathlineto{\pgfqpoint{1.605095in}{2.291247in}}%
\pgfpathlineto{\pgfqpoint{1.620529in}{2.309760in}}%
\pgfpathlineto{\pgfqpoint{1.635963in}{2.321958in}}%
\pgfpathlineto{\pgfqpoint{1.651396in}{2.327937in}}%
\pgfpathlineto{\pgfqpoint{1.666830in}{2.327907in}}%
\pgfpathlineto{\pgfqpoint{1.682264in}{2.322187in}}%
\pgfpathlineto{\pgfqpoint{1.697697in}{2.311193in}}%
\pgfpathlineto{\pgfqpoint{1.713131in}{2.295428in}}%
\pgfpathlineto{\pgfqpoint{1.728564in}{2.275469in}}%
\pgfpathlineto{\pgfqpoint{1.743998in}{2.251952in}}%
\pgfpathlineto{\pgfqpoint{1.774865in}{2.196984in}}%
\pgfpathlineto{\pgfqpoint{1.867467in}{2.018516in}}%
\pgfpathlineto{\pgfqpoint{1.898334in}{1.970431in}}%
\pgfpathlineto{\pgfqpoint{1.913768in}{1.950385in}}%
\pgfpathlineto{\pgfqpoint{1.929202in}{1.933252in}}%
\pgfpathlineto{\pgfqpoint{1.944635in}{1.919092in}}%
\pgfpathlineto{\pgfqpoint{1.960069in}{1.907862in}}%
\pgfpathlineto{\pgfqpoint{1.975503in}{1.899420in}}%
\pgfpathlineto{\pgfqpoint{1.990936in}{1.893532in}}%
\pgfpathlineto{\pgfqpoint{2.006370in}{1.889876in}}%
\pgfpathlineto{\pgfqpoint{2.037237in}{1.887597in}}%
\pgfpathlineto{\pgfqpoint{2.098972in}{1.888643in}}%
\pgfpathlineto{\pgfqpoint{2.114405in}{1.886749in}}%
\pgfpathlineto{\pgfqpoint{2.129839in}{1.882866in}}%
\pgfpathlineto{\pgfqpoint{2.145273in}{1.876485in}}%
\pgfpathlineto{\pgfqpoint{2.160706in}{1.867154in}}%
\pgfpathlineto{\pgfqpoint{2.176140in}{1.854489in}}%
\pgfpathlineto{\pgfqpoint{2.191574in}{1.838188in}}%
\pgfpathlineto{\pgfqpoint{2.207007in}{1.818039in}}%
\pgfpathlineto{\pgfqpoint{2.222441in}{1.793928in}}%
\pgfpathlineto{\pgfqpoint{2.237874in}{1.765846in}}%
\pgfpathlineto{\pgfqpoint{2.253308in}{1.733892in}}%
\pgfpathlineto{\pgfqpoint{2.284175in}{1.659297in}}%
\pgfpathlineto{\pgfqpoint{2.315043in}{1.573039in}}%
\pgfpathlineto{\pgfqpoint{2.423078in}{1.252062in}}%
\pgfpathlineto{\pgfqpoint{2.438512in}{1.214077in}}%
\pgfpathlineto{\pgfqpoint{2.453945in}{1.180424in}}%
\pgfpathlineto{\pgfqpoint{2.469379in}{1.151802in}}%
\pgfpathlineto{\pgfqpoint{2.484813in}{1.128840in}}%
\pgfpathlineto{\pgfqpoint{2.500246in}{1.112083in}}%
\pgfpathlineto{\pgfqpoint{2.515680in}{1.101974in}}%
\pgfpathlineto{\pgfqpoint{2.531114in}{1.098851in}}%
\pgfpathlineto{\pgfqpoint{2.546547in}{1.102931in}}%
\pgfpathlineto{\pgfqpoint{2.561981in}{1.114310in}}%
\pgfpathlineto{\pgfqpoint{2.577414in}{1.132954in}}%
\pgfpathlineto{\pgfqpoint{2.592848in}{1.158704in}}%
\pgfpathlineto{\pgfqpoint{2.608282in}{1.191272in}}%
\pgfpathlineto{\pgfqpoint{2.623715in}{1.230250in}}%
\pgfpathlineto{\pgfqpoint{2.639149in}{1.275113in}}%
\pgfpathlineto{\pgfqpoint{2.670016in}{1.379886in}}%
\pgfpathlineto{\pgfqpoint{2.700884in}{1.499508in}}%
\pgfpathlineto{\pgfqpoint{2.793485in}{1.874463in}}%
\pgfpathlineto{\pgfqpoint{2.824353in}{1.980328in}}%
\pgfpathlineto{\pgfqpoint{2.839786in}{2.026087in}}%
\pgfpathlineto{\pgfqpoint{2.855220in}{2.066271in}}%
\pgfpathlineto{\pgfqpoint{2.870654in}{2.100419in}}%
\pgfpathlineto{\pgfqpoint{2.886087in}{2.128189in}}%
\pgfpathlineto{\pgfqpoint{2.901521in}{2.149360in}}%
\pgfpathlineto{\pgfqpoint{2.916954in}{2.163837in}}%
\pgfpathlineto{\pgfqpoint{2.932388in}{2.171648in}}%
\pgfpathlineto{\pgfqpoint{2.947822in}{2.172942in}}%
\pgfpathlineto{\pgfqpoint{2.963255in}{2.167985in}}%
\pgfpathlineto{\pgfqpoint{2.978689in}{2.157150in}}%
\pgfpathlineto{\pgfqpoint{2.994123in}{2.140906in}}%
\pgfpathlineto{\pgfqpoint{3.009556in}{2.119811in}}%
\pgfpathlineto{\pgfqpoint{3.024990in}{2.094493in}}%
\pgfpathlineto{\pgfqpoint{3.055857in}{2.033965in}}%
\pgfpathlineto{\pgfqpoint{3.102158in}{1.929566in}}%
\pgfpathlineto{\pgfqpoint{3.148459in}{1.826250in}}%
\pgfpathlineto{\pgfqpoint{3.179326in}{1.766248in}}%
\pgfpathlineto{\pgfqpoint{3.194760in}{1.740287in}}%
\pgfpathlineto{\pgfqpoint{3.210193in}{1.717418in}}%
\pgfpathlineto{\pgfqpoint{3.225627in}{1.697812in}}%
\pgfpathlineto{\pgfqpoint{3.241061in}{1.681539in}}%
\pgfpathlineto{\pgfqpoint{3.256494in}{1.668562in}}%
\pgfpathlineto{\pgfqpoint{3.271928in}{1.658745in}}%
\pgfpathlineto{\pgfqpoint{3.287362in}{1.651859in}}%
\pgfpathlineto{\pgfqpoint{3.302795in}{1.647584in}}%
\pgfpathlineto{\pgfqpoint{3.318229in}{1.645527in}}%
\pgfpathlineto{\pgfqpoint{3.349096in}{1.646170in}}%
\pgfpathlineto{\pgfqpoint{3.395397in}{1.650867in}}%
\pgfpathlineto{\pgfqpoint{3.410831in}{1.651142in}}%
\pgfpathlineto{\pgfqpoint{3.426264in}{1.649849in}}%
\pgfpathlineto{\pgfqpoint{3.441698in}{1.646486in}}%
\pgfpathlineto{\pgfqpoint{3.457132in}{1.640607in}}%
\pgfpathlineto{\pgfqpoint{3.472565in}{1.631835in}}%
\pgfpathlineto{\pgfqpoint{3.487999in}{1.619875in}}%
\pgfpathlineto{\pgfqpoint{3.503433in}{1.604518in}}%
\pgfpathlineto{\pgfqpoint{3.518866in}{1.585656in}}%
\pgfpathlineto{\pgfqpoint{3.534300in}{1.563281in}}%
\pgfpathlineto{\pgfqpoint{3.549733in}{1.537491in}}%
\pgfpathlineto{\pgfqpoint{3.580601in}{1.476583in}}%
\pgfpathlineto{\pgfqpoint{3.611468in}{1.405765in}}%
\pgfpathlineto{\pgfqpoint{3.704070in}{1.181616in}}%
\pgfpathlineto{\pgfqpoint{3.719503in}{1.150238in}}%
\pgfpathlineto{\pgfqpoint{3.734937in}{1.122598in}}%
\pgfpathlineto{\pgfqpoint{3.750371in}{1.099401in}}%
\pgfpathlineto{\pgfqpoint{3.765804in}{1.081292in}}%
\pgfpathlineto{\pgfqpoint{3.781238in}{1.068842in}}%
\pgfpathlineto{\pgfqpoint{3.796672in}{1.062534in}}%
\pgfpathlineto{\pgfqpoint{3.812105in}{1.062753in}}%
\pgfpathlineto{\pgfqpoint{3.827539in}{1.069771in}}%
\pgfpathlineto{\pgfqpoint{3.842973in}{1.083746in}}%
\pgfpathlineto{\pgfqpoint{3.858406in}{1.104710in}}%
\pgfpathlineto{\pgfqpoint{3.873840in}{1.132571in}}%
\pgfpathlineto{\pgfqpoint{3.889273in}{1.167111in}}%
\pgfpathlineto{\pgfqpoint{3.904707in}{1.207988in}}%
\pgfpathlineto{\pgfqpoint{3.920141in}{1.254743in}}%
\pgfpathlineto{\pgfqpoint{3.951008in}{1.363507in}}%
\pgfpathlineto{\pgfqpoint{3.981875in}{1.487711in}}%
\pgfpathlineto{\pgfqpoint{4.074477in}{1.881602in}}%
\pgfpathlineto{\pgfqpoint{4.105344in}{1.995205in}}%
\pgfpathlineto{\pgfqpoint{4.120778in}{2.044925in}}%
\pgfpathlineto{\pgfqpoint{4.136212in}{2.089052in}}%
\pgfpathlineto{\pgfqpoint{4.151645in}{2.127061in}}%
\pgfpathlineto{\pgfqpoint{4.167079in}{2.158542in}}%
\pgfpathlineto{\pgfqpoint{4.182513in}{2.183204in}}%
\pgfpathlineto{\pgfqpoint{4.197946in}{2.200884in}}%
\pgfpathlineto{\pgfqpoint{4.213380in}{2.211544in}}%
\pgfpathlineto{\pgfqpoint{4.228813in}{2.215272in}}%
\pgfpathlineto{\pgfqpoint{4.244247in}{2.212278in}}%
\pgfpathlineto{\pgfqpoint{4.259681in}{2.202888in}}%
\pgfpathlineto{\pgfqpoint{4.275114in}{2.187534in}}%
\pgfpathlineto{\pgfqpoint{4.290548in}{2.166746in}}%
\pgfpathlineto{\pgfqpoint{4.305982in}{2.141136in}}%
\pgfpathlineto{\pgfqpoint{4.321415in}{2.111386in}}%
\pgfpathlineto{\pgfqpoint{4.352283in}{2.042441in}}%
\pgfpathlineto{\pgfqpoint{4.460318in}{1.782738in}}%
\pgfpathlineto{\pgfqpoint{4.491185in}{1.725382in}}%
\pgfpathlineto{\pgfqpoint{4.506619in}{1.701821in}}%
\pgfpathlineto{\pgfqpoint{4.522052in}{1.681940in}}%
\pgfpathlineto{\pgfqpoint{4.537486in}{1.665807in}}%
\pgfpathlineto{\pgfqpoint{4.552920in}{1.653384in}}%
\pgfpathlineto{\pgfqpoint{4.568353in}{1.644532in}}%
\pgfpathlineto{\pgfqpoint{4.583787in}{1.639015in}}%
\pgfpathlineto{\pgfqpoint{4.599221in}{1.636510in}}%
\pgfpathlineto{\pgfqpoint{4.614654in}{1.636615in}}%
\pgfpathlineto{\pgfqpoint{4.630088in}{1.638862in}}%
\pgfpathlineto{\pgfqpoint{4.660955in}{1.647675in}}%
\pgfpathlineto{\pgfqpoint{4.660955in}{1.647675in}}%
\pgfusepath{stroke}%
\end{pgfscope}%
\begin{pgfscope}%
\pgfpathrectangle{\pgfqpoint{0.625831in}{0.505056in}}{\pgfqpoint{4.227273in}{2.745455in}} %
\pgfusepath{clip}%
\pgfsetrectcap%
\pgfsetroundjoin%
\pgfsetlinewidth{0.501875pt}%
\definecolor{currentstroke}{rgb}{0.100000,0.587785,0.951057}%
\pgfsetstrokecolor{currentstroke}%
\pgfsetdash{}{0pt}%
\pgfpathmoveto{\pgfqpoint{0.817980in}{2.312834in}}%
\pgfpathlineto{\pgfqpoint{0.833414in}{2.289284in}}%
\pgfpathlineto{\pgfqpoint{0.864281in}{2.232798in}}%
\pgfpathlineto{\pgfqpoint{0.895148in}{2.165815in}}%
\pgfpathlineto{\pgfqpoint{0.926015in}{2.091245in}}%
\pgfpathlineto{\pgfqpoint{1.034051in}{1.820273in}}%
\pgfpathlineto{\pgfqpoint{1.064918in}{1.754678in}}%
\pgfpathlineto{\pgfqpoint{1.095785in}{1.700245in}}%
\pgfpathlineto{\pgfqpoint{1.111219in}{1.678013in}}%
\pgfpathlineto{\pgfqpoint{1.126653in}{1.659448in}}%
\pgfpathlineto{\pgfqpoint{1.142086in}{1.644756in}}%
\pgfpathlineto{\pgfqpoint{1.157520in}{1.634091in}}%
\pgfpathlineto{\pgfqpoint{1.172954in}{1.627560in}}%
\pgfpathlineto{\pgfqpoint{1.188387in}{1.625215in}}%
\pgfpathlineto{\pgfqpoint{1.203821in}{1.627061in}}%
\pgfpathlineto{\pgfqpoint{1.219255in}{1.633045in}}%
\pgfpathlineto{\pgfqpoint{1.234688in}{1.643067in}}%
\pgfpathlineto{\pgfqpoint{1.250122in}{1.656974in}}%
\pgfpathlineto{\pgfqpoint{1.265555in}{1.674567in}}%
\pgfpathlineto{\pgfqpoint{1.280989in}{1.695599in}}%
\pgfpathlineto{\pgfqpoint{1.311856in}{1.746790in}}%
\pgfpathlineto{\pgfqpoint{1.342724in}{1.807797in}}%
\pgfpathlineto{\pgfqpoint{1.389025in}{1.910573in}}%
\pgfpathlineto{\pgfqpoint{1.450759in}{2.049768in}}%
\pgfpathlineto{\pgfqpoint{1.481626in}{2.112205in}}%
\pgfpathlineto{\pgfqpoint{1.512494in}{2.165508in}}%
\pgfpathlineto{\pgfqpoint{1.527927in}{2.187867in}}%
\pgfpathlineto{\pgfqpoint{1.543361in}{2.206973in}}%
\pgfpathlineto{\pgfqpoint{1.558795in}{2.222586in}}%
\pgfpathlineto{\pgfqpoint{1.574228in}{2.234510in}}%
\pgfpathlineto{\pgfqpoint{1.589662in}{2.242596in}}%
\pgfpathlineto{\pgfqpoint{1.605095in}{2.246743in}}%
\pgfpathlineto{\pgfqpoint{1.620529in}{2.246902in}}%
\pgfpathlineto{\pgfqpoint{1.635963in}{2.243071in}}%
\pgfpathlineto{\pgfqpoint{1.651396in}{2.235301in}}%
\pgfpathlineto{\pgfqpoint{1.666830in}{2.223689in}}%
\pgfpathlineto{\pgfqpoint{1.682264in}{2.208378in}}%
\pgfpathlineto{\pgfqpoint{1.697697in}{2.189559in}}%
\pgfpathlineto{\pgfqpoint{1.713131in}{2.167462in}}%
\pgfpathlineto{\pgfqpoint{1.743998in}{2.114544in}}%
\pgfpathlineto{\pgfqpoint{1.774865in}{2.052162in}}%
\pgfpathlineto{\pgfqpoint{1.821166in}{1.947346in}}%
\pgfpathlineto{\pgfqpoint{1.898334in}{1.769328in}}%
\pgfpathlineto{\pgfqpoint{1.929202in}{1.705886in}}%
\pgfpathlineto{\pgfqpoint{1.960069in}{1.650752in}}%
\pgfpathlineto{\pgfqpoint{1.990936in}{1.605799in}}%
\pgfpathlineto{\pgfqpoint{2.006370in}{1.587551in}}%
\pgfpathlineto{\pgfqpoint{2.021804in}{1.572255in}}%
\pgfpathlineto{\pgfqpoint{2.037237in}{1.559954in}}%
\pgfpathlineto{\pgfqpoint{2.052671in}{1.550648in}}%
\pgfpathlineto{\pgfqpoint{2.068104in}{1.544290in}}%
\pgfpathlineto{\pgfqpoint{2.083538in}{1.540792in}}%
\pgfpathlineto{\pgfqpoint{2.098972in}{1.540021in}}%
\pgfpathlineto{\pgfqpoint{2.114405in}{1.541807in}}%
\pgfpathlineto{\pgfqpoint{2.129839in}{1.545943in}}%
\pgfpathlineto{\pgfqpoint{2.145273in}{1.552188in}}%
\pgfpathlineto{\pgfqpoint{2.176140in}{1.569897in}}%
\pgfpathlineto{\pgfqpoint{2.207007in}{1.592496in}}%
\pgfpathlineto{\pgfqpoint{2.284175in}{1.652519in}}%
\pgfpathlineto{\pgfqpoint{2.315043in}{1.670823in}}%
\pgfpathlineto{\pgfqpoint{2.330476in}{1.677524in}}%
\pgfpathlineto{\pgfqpoint{2.345910in}{1.682251in}}%
\pgfpathlineto{\pgfqpoint{2.361344in}{1.684788in}}%
\pgfpathlineto{\pgfqpoint{2.376777in}{1.684955in}}%
\pgfpathlineto{\pgfqpoint{2.392211in}{1.682609in}}%
\pgfpathlineto{\pgfqpoint{2.407644in}{1.677648in}}%
\pgfpathlineto{\pgfqpoint{2.423078in}{1.670012in}}%
\pgfpathlineto{\pgfqpoint{2.438512in}{1.659684in}}%
\pgfpathlineto{\pgfqpoint{2.453945in}{1.646689in}}%
\pgfpathlineto{\pgfqpoint{2.469379in}{1.631097in}}%
\pgfpathlineto{\pgfqpoint{2.500246in}{1.592608in}}%
\pgfpathlineto{\pgfqpoint{2.531114in}{1.545586in}}%
\pgfpathlineto{\pgfqpoint{2.561981in}{1.491971in}}%
\pgfpathlineto{\pgfqpoint{2.623715in}{1.374933in}}%
\pgfpathlineto{\pgfqpoint{2.670016in}{1.289750in}}%
\pgfpathlineto{\pgfqpoint{2.700884in}{1.239631in}}%
\pgfpathlineto{\pgfqpoint{2.731751in}{1.197948in}}%
\pgfpathlineto{\pgfqpoint{2.747184in}{1.180986in}}%
\pgfpathlineto{\pgfqpoint{2.762618in}{1.166930in}}%
\pgfpathlineto{\pgfqpoint{2.778052in}{1.155977in}}%
\pgfpathlineto{\pgfqpoint{2.793485in}{1.148279in}}%
\pgfpathlineto{\pgfqpoint{2.808919in}{1.143951in}}%
\pgfpathlineto{\pgfqpoint{2.824353in}{1.143060in}}%
\pgfpathlineto{\pgfqpoint{2.839786in}{1.145628in}}%
\pgfpathlineto{\pgfqpoint{2.855220in}{1.151632in}}%
\pgfpathlineto{\pgfqpoint{2.870654in}{1.161001in}}%
\pgfpathlineto{\pgfqpoint{2.886087in}{1.173617in}}%
\pgfpathlineto{\pgfqpoint{2.901521in}{1.189320in}}%
\pgfpathlineto{\pgfqpoint{2.916954in}{1.207904in}}%
\pgfpathlineto{\pgfqpoint{2.947822in}{1.252700in}}%
\pgfpathlineto{\pgfqpoint{2.978689in}{1.305615in}}%
\pgfpathlineto{\pgfqpoint{3.024990in}{1.393843in}}%
\pgfpathlineto{\pgfqpoint{3.071291in}{1.482921in}}%
\pgfpathlineto{\pgfqpoint{3.102158in}{1.537192in}}%
\pgfpathlineto{\pgfqpoint{3.133025in}{1.583670in}}%
\pgfpathlineto{\pgfqpoint{3.148459in}{1.603097in}}%
\pgfpathlineto{\pgfqpoint{3.163893in}{1.619570in}}%
\pgfpathlineto{\pgfqpoint{3.179326in}{1.632822in}}%
\pgfpathlineto{\pgfqpoint{3.194760in}{1.642627in}}%
\pgfpathlineto{\pgfqpoint{3.210193in}{1.648806in}}%
\pgfpathlineto{\pgfqpoint{3.225627in}{1.651222in}}%
\pgfpathlineto{\pgfqpoint{3.241061in}{1.649790in}}%
\pgfpathlineto{\pgfqpoint{3.256494in}{1.644473in}}%
\pgfpathlineto{\pgfqpoint{3.271928in}{1.635286in}}%
\pgfpathlineto{\pgfqpoint{3.287362in}{1.622293in}}%
\pgfpathlineto{\pgfqpoint{3.302795in}{1.605609in}}%
\pgfpathlineto{\pgfqpoint{3.318229in}{1.585401in}}%
\pgfpathlineto{\pgfqpoint{3.333663in}{1.561878in}}%
\pgfpathlineto{\pgfqpoint{3.364530in}{1.505956in}}%
\pgfpathlineto{\pgfqpoint{3.395397in}{1.440372in}}%
\pgfpathlineto{\pgfqpoint{3.441698in}{1.330688in}}%
\pgfpathlineto{\pgfqpoint{3.503433in}{1.181945in}}%
\pgfpathlineto{\pgfqpoint{3.534300in}{1.114726in}}%
\pgfpathlineto{\pgfqpoint{3.565167in}{1.056854in}}%
\pgfpathlineto{\pgfqpoint{3.580601in}{1.032363in}}%
\pgfpathlineto{\pgfqpoint{3.596034in}{1.011272in}}%
\pgfpathlineto{\pgfqpoint{3.611468in}{0.993856in}}%
\pgfpathlineto{\pgfqpoint{3.626902in}{0.980349in}}%
\pgfpathlineto{\pgfqpoint{3.642335in}{0.970938in}}%
\pgfpathlineto{\pgfqpoint{3.657769in}{0.965759in}}%
\pgfpathlineto{\pgfqpoint{3.673203in}{0.964900in}}%
\pgfpathlineto{\pgfqpoint{3.688636in}{0.968396in}}%
\pgfpathlineto{\pgfqpoint{3.704070in}{0.976229in}}%
\pgfpathlineto{\pgfqpoint{3.719503in}{0.988329in}}%
\pgfpathlineto{\pgfqpoint{3.734937in}{1.004577in}}%
\pgfpathlineto{\pgfqpoint{3.750371in}{1.024804in}}%
\pgfpathlineto{\pgfqpoint{3.765804in}{1.048792in}}%
\pgfpathlineto{\pgfqpoint{3.781238in}{1.076280in}}%
\pgfpathlineto{\pgfqpoint{3.812105in}{1.140518in}}%
\pgfpathlineto{\pgfqpoint{3.842973in}{1.214691in}}%
\pgfpathlineto{\pgfqpoint{3.889273in}{1.337362in}}%
\pgfpathlineto{\pgfqpoint{3.951008in}{1.503485in}}%
\pgfpathlineto{\pgfqpoint{3.981875in}{1.579743in}}%
\pgfpathlineto{\pgfqpoint{4.012743in}{1.647281in}}%
\pgfpathlineto{\pgfqpoint{4.043610in}{1.703559in}}%
\pgfpathlineto{\pgfqpoint{4.059043in}{1.726853in}}%
\pgfpathlineto{\pgfqpoint{4.074477in}{1.746670in}}%
\pgfpathlineto{\pgfqpoint{4.089911in}{1.762886in}}%
\pgfpathlineto{\pgfqpoint{4.105344in}{1.775422in}}%
\pgfpathlineto{\pgfqpoint{4.120778in}{1.784251in}}%
\pgfpathlineto{\pgfqpoint{4.136212in}{1.789393in}}%
\pgfpathlineto{\pgfqpoint{4.151645in}{1.790914in}}%
\pgfpathlineto{\pgfqpoint{4.167079in}{1.788931in}}%
\pgfpathlineto{\pgfqpoint{4.182513in}{1.783601in}}%
\pgfpathlineto{\pgfqpoint{4.197946in}{1.775126in}}%
\pgfpathlineto{\pgfqpoint{4.213380in}{1.763745in}}%
\pgfpathlineto{\pgfqpoint{4.228813in}{1.749732in}}%
\pgfpathlineto{\pgfqpoint{4.259681in}{1.715064in}}%
\pgfpathlineto{\pgfqpoint{4.290548in}{1.673854in}}%
\pgfpathlineto{\pgfqpoint{4.383150in}{1.541214in}}%
\pgfpathlineto{\pgfqpoint{4.414017in}{1.503855in}}%
\pgfpathlineto{\pgfqpoint{4.444884in}{1.474165in}}%
\pgfpathlineto{\pgfqpoint{4.460318in}{1.462792in}}%
\pgfpathlineto{\pgfqpoint{4.475752in}{1.453979in}}%
\pgfpathlineto{\pgfqpoint{4.491185in}{1.447857in}}%
\pgfpathlineto{\pgfqpoint{4.506619in}{1.444514in}}%
\pgfpathlineto{\pgfqpoint{4.522052in}{1.443997in}}%
\pgfpathlineto{\pgfqpoint{4.537486in}{1.446305in}}%
\pgfpathlineto{\pgfqpoint{4.552920in}{1.451395in}}%
\pgfpathlineto{\pgfqpoint{4.568353in}{1.459183in}}%
\pgfpathlineto{\pgfqpoint{4.583787in}{1.469540in}}%
\pgfpathlineto{\pgfqpoint{4.599221in}{1.482299in}}%
\pgfpathlineto{\pgfqpoint{4.630088in}{1.514176in}}%
\pgfpathlineto{\pgfqpoint{4.660955in}{1.552808in}}%
\pgfpathlineto{\pgfqpoint{4.660955in}{1.552808in}}%
\pgfusepath{stroke}%
\end{pgfscope}%
\begin{pgfscope}%
\pgfpathrectangle{\pgfqpoint{0.625831in}{0.505056in}}{\pgfqpoint{4.227273in}{2.745455in}} %
\pgfusepath{clip}%
\pgfsetrectcap%
\pgfsetroundjoin%
\pgfsetlinewidth{0.501875pt}%
\definecolor{currentstroke}{rgb}{0.021569,0.682749,0.930229}%
\pgfsetstrokecolor{currentstroke}%
\pgfsetdash{}{0pt}%
\pgfpathmoveto{\pgfqpoint{0.817980in}{2.492762in}}%
\pgfpathlineto{\pgfqpoint{0.833414in}{2.477873in}}%
\pgfpathlineto{\pgfqpoint{0.848847in}{2.458545in}}%
\pgfpathlineto{\pgfqpoint{0.864281in}{2.435040in}}%
\pgfpathlineto{\pgfqpoint{0.879715in}{2.407679in}}%
\pgfpathlineto{\pgfqpoint{0.910582in}{2.342931in}}%
\pgfpathlineto{\pgfqpoint{0.941449in}{2.267836in}}%
\pgfpathlineto{\pgfqpoint{1.003184in}{2.103386in}}%
\pgfpathlineto{\pgfqpoint{1.049485in}{1.985306in}}%
\pgfpathlineto{\pgfqpoint{1.080352in}{1.917274in}}%
\pgfpathlineto{\pgfqpoint{1.095785in}{1.887916in}}%
\pgfpathlineto{\pgfqpoint{1.111219in}{1.862216in}}%
\pgfpathlineto{\pgfqpoint{1.126653in}{1.840525in}}%
\pgfpathlineto{\pgfqpoint{1.142086in}{1.823139in}}%
\pgfpathlineto{\pgfqpoint{1.157520in}{1.810293in}}%
\pgfpathlineto{\pgfqpoint{1.172954in}{1.802159in}}%
\pgfpathlineto{\pgfqpoint{1.188387in}{1.798845in}}%
\pgfpathlineto{\pgfqpoint{1.203821in}{1.800388in}}%
\pgfpathlineto{\pgfqpoint{1.219255in}{1.806760in}}%
\pgfpathlineto{\pgfqpoint{1.234688in}{1.817865in}}%
\pgfpathlineto{\pgfqpoint{1.250122in}{1.833539in}}%
\pgfpathlineto{\pgfqpoint{1.265555in}{1.853556in}}%
\pgfpathlineto{\pgfqpoint{1.280989in}{1.877627in}}%
\pgfpathlineto{\pgfqpoint{1.296423in}{1.905407in}}%
\pgfpathlineto{\pgfqpoint{1.327290in}{1.970456in}}%
\pgfpathlineto{\pgfqpoint{1.358157in}{2.044997in}}%
\pgfpathlineto{\pgfqpoint{1.450759in}{2.281802in}}%
\pgfpathlineto{\pgfqpoint{1.481626in}{2.350006in}}%
\pgfpathlineto{\pgfqpoint{1.497060in}{2.379711in}}%
\pgfpathlineto{\pgfqpoint{1.512494in}{2.405884in}}%
\pgfpathlineto{\pgfqpoint{1.527927in}{2.428132in}}%
\pgfpathlineto{\pgfqpoint{1.543361in}{2.446114in}}%
\pgfpathlineto{\pgfqpoint{1.558795in}{2.459547in}}%
\pgfpathlineto{\pgfqpoint{1.574228in}{2.468207in}}%
\pgfpathlineto{\pgfqpoint{1.589662in}{2.471938in}}%
\pgfpathlineto{\pgfqpoint{1.605095in}{2.470646in}}%
\pgfpathlineto{\pgfqpoint{1.620529in}{2.464308in}}%
\pgfpathlineto{\pgfqpoint{1.635963in}{2.452965in}}%
\pgfpathlineto{\pgfqpoint{1.651396in}{2.436727in}}%
\pgfpathlineto{\pgfqpoint{1.666830in}{2.415768in}}%
\pgfpathlineto{\pgfqpoint{1.682264in}{2.390327in}}%
\pgfpathlineto{\pgfqpoint{1.697697in}{2.360697in}}%
\pgfpathlineto{\pgfqpoint{1.728564in}{2.290329in}}%
\pgfpathlineto{\pgfqpoint{1.759432in}{2.208032in}}%
\pgfpathlineto{\pgfqpoint{1.805733in}{2.071009in}}%
\pgfpathlineto{\pgfqpoint{1.867467in}{1.886053in}}%
\pgfpathlineto{\pgfqpoint{1.898334in}{1.802364in}}%
\pgfpathlineto{\pgfqpoint{1.929202in}{1.729655in}}%
\pgfpathlineto{\pgfqpoint{1.944635in}{1.698398in}}%
\pgfpathlineto{\pgfqpoint{1.960069in}{1.670961in}}%
\pgfpathlineto{\pgfqpoint{1.975503in}{1.647590in}}%
\pgfpathlineto{\pgfqpoint{1.990936in}{1.628467in}}%
\pgfpathlineto{\pgfqpoint{2.006370in}{1.613713in}}%
\pgfpathlineto{\pgfqpoint{2.021804in}{1.603383in}}%
\pgfpathlineto{\pgfqpoint{2.037237in}{1.597467in}}%
\pgfpathlineto{\pgfqpoint{2.052671in}{1.595888in}}%
\pgfpathlineto{\pgfqpoint{2.068104in}{1.598505in}}%
\pgfpathlineto{\pgfqpoint{2.083538in}{1.605118in}}%
\pgfpathlineto{\pgfqpoint{2.098972in}{1.615464in}}%
\pgfpathlineto{\pgfqpoint{2.114405in}{1.629228in}}%
\pgfpathlineto{\pgfqpoint{2.129839in}{1.646042in}}%
\pgfpathlineto{\pgfqpoint{2.160706in}{1.687135in}}%
\pgfpathlineto{\pgfqpoint{2.191574in}{1.735023in}}%
\pgfpathlineto{\pgfqpoint{2.253308in}{1.834516in}}%
\pgfpathlineto{\pgfqpoint{2.284175in}{1.877627in}}%
\pgfpathlineto{\pgfqpoint{2.299609in}{1.895767in}}%
\pgfpathlineto{\pgfqpoint{2.315043in}{1.911036in}}%
\pgfpathlineto{\pgfqpoint{2.330476in}{1.923037in}}%
\pgfpathlineto{\pgfqpoint{2.345910in}{1.931421in}}%
\pgfpathlineto{\pgfqpoint{2.361344in}{1.935890in}}%
\pgfpathlineto{\pgfqpoint{2.376777in}{1.936205in}}%
\pgfpathlineto{\pgfqpoint{2.392211in}{1.932186in}}%
\pgfpathlineto{\pgfqpoint{2.407644in}{1.923716in}}%
\pgfpathlineto{\pgfqpoint{2.423078in}{1.910742in}}%
\pgfpathlineto{\pgfqpoint{2.438512in}{1.893278in}}%
\pgfpathlineto{\pgfqpoint{2.453945in}{1.871403in}}%
\pgfpathlineto{\pgfqpoint{2.469379in}{1.845263in}}%
\pgfpathlineto{\pgfqpoint{2.484813in}{1.815065in}}%
\pgfpathlineto{\pgfqpoint{2.515680in}{1.743620in}}%
\pgfpathlineto{\pgfqpoint{2.546547in}{1.659866in}}%
\pgfpathlineto{\pgfqpoint{2.592848in}{1.519070in}}%
\pgfpathlineto{\pgfqpoint{2.670016in}{1.279624in}}%
\pgfpathlineto{\pgfqpoint{2.700884in}{1.195118in}}%
\pgfpathlineto{\pgfqpoint{2.731751in}{1.123176in}}%
\pgfpathlineto{\pgfqpoint{2.747184in}{1.092979in}}%
\pgfpathlineto{\pgfqpoint{2.762618in}{1.067104in}}%
\pgfpathlineto{\pgfqpoint{2.778052in}{1.045836in}}%
\pgfpathlineto{\pgfqpoint{2.793485in}{1.029398in}}%
\pgfpathlineto{\pgfqpoint{2.808919in}{1.017953in}}%
\pgfpathlineto{\pgfqpoint{2.824353in}{1.011597in}}%
\pgfpathlineto{\pgfqpoint{2.839786in}{1.010358in}}%
\pgfpathlineto{\pgfqpoint{2.855220in}{1.014198in}}%
\pgfpathlineto{\pgfqpoint{2.870654in}{1.023013in}}%
\pgfpathlineto{\pgfqpoint{2.886087in}{1.036631in}}%
\pgfpathlineto{\pgfqpoint{2.901521in}{1.054818in}}%
\pgfpathlineto{\pgfqpoint{2.916954in}{1.077281in}}%
\pgfpathlineto{\pgfqpoint{2.932388in}{1.103669in}}%
\pgfpathlineto{\pgfqpoint{2.963255in}{1.166574in}}%
\pgfpathlineto{\pgfqpoint{2.994123in}{1.239814in}}%
\pgfpathlineto{\pgfqpoint{3.102158in}{1.514477in}}%
\pgfpathlineto{\pgfqpoint{3.133025in}{1.579919in}}%
\pgfpathlineto{\pgfqpoint{3.148459in}{1.607989in}}%
\pgfpathlineto{\pgfqpoint{3.163893in}{1.632433in}}%
\pgfpathlineto{\pgfqpoint{3.179326in}{1.652924in}}%
\pgfpathlineto{\pgfqpoint{3.194760in}{1.669193in}}%
\pgfpathlineto{\pgfqpoint{3.210193in}{1.681031in}}%
\pgfpathlineto{\pgfqpoint{3.225627in}{1.688293in}}%
\pgfpathlineto{\pgfqpoint{3.241061in}{1.690899in}}%
\pgfpathlineto{\pgfqpoint{3.256494in}{1.688838in}}%
\pgfpathlineto{\pgfqpoint{3.271928in}{1.682165in}}%
\pgfpathlineto{\pgfqpoint{3.287362in}{1.671002in}}%
\pgfpathlineto{\pgfqpoint{3.302795in}{1.655532in}}%
\pgfpathlineto{\pgfqpoint{3.318229in}{1.636004in}}%
\pgfpathlineto{\pgfqpoint{3.333663in}{1.612723in}}%
\pgfpathlineto{\pgfqpoint{3.364530in}{1.556383in}}%
\pgfpathlineto{\pgfqpoint{3.395397in}{1.489931in}}%
\pgfpathlineto{\pgfqpoint{3.457132in}{1.343034in}}%
\pgfpathlineto{\pgfqpoint{3.487999in}{1.271406in}}%
\pgfpathlineto{\pgfqpoint{3.518866in}{1.206805in}}%
\pgfpathlineto{\pgfqpoint{3.549733in}{1.153178in}}%
\pgfpathlineto{\pgfqpoint{3.565167in}{1.131554in}}%
\pgfpathlineto{\pgfqpoint{3.580601in}{1.113872in}}%
\pgfpathlineto{\pgfqpoint{3.596034in}{1.100426in}}%
\pgfpathlineto{\pgfqpoint{3.611468in}{1.091454in}}%
\pgfpathlineto{\pgfqpoint{3.626902in}{1.087130in}}%
\pgfpathlineto{\pgfqpoint{3.642335in}{1.087566in}}%
\pgfpathlineto{\pgfqpoint{3.657769in}{1.092808in}}%
\pgfpathlineto{\pgfqpoint{3.673203in}{1.102837in}}%
\pgfpathlineto{\pgfqpoint{3.688636in}{1.117569in}}%
\pgfpathlineto{\pgfqpoint{3.704070in}{1.136855in}}%
\pgfpathlineto{\pgfqpoint{3.719503in}{1.160484in}}%
\pgfpathlineto{\pgfqpoint{3.734937in}{1.188188in}}%
\pgfpathlineto{\pgfqpoint{3.765804in}{1.254472in}}%
\pgfpathlineto{\pgfqpoint{3.796672in}{1.332547in}}%
\pgfpathlineto{\pgfqpoint{3.842973in}{1.463383in}}%
\pgfpathlineto{\pgfqpoint{3.904707in}{1.641099in}}%
\pgfpathlineto{\pgfqpoint{3.935574in}{1.721791in}}%
\pgfpathlineto{\pgfqpoint{3.966442in}{1.791944in}}%
\pgfpathlineto{\pgfqpoint{3.981875in}{1.822077in}}%
\pgfpathlineto{\pgfqpoint{3.997309in}{1.848483in}}%
\pgfpathlineto{\pgfqpoint{4.012743in}{1.870902in}}%
\pgfpathlineto{\pgfqpoint{4.028176in}{1.889138in}}%
\pgfpathlineto{\pgfqpoint{4.043610in}{1.903054in}}%
\pgfpathlineto{\pgfqpoint{4.059043in}{1.912574in}}%
\pgfpathlineto{\pgfqpoint{4.074477in}{1.917691in}}%
\pgfpathlineto{\pgfqpoint{4.089911in}{1.918457in}}%
\pgfpathlineto{\pgfqpoint{4.105344in}{1.914991in}}%
\pgfpathlineto{\pgfqpoint{4.120778in}{1.907470in}}%
\pgfpathlineto{\pgfqpoint{4.136212in}{1.896133in}}%
\pgfpathlineto{\pgfqpoint{4.151645in}{1.881271in}}%
\pgfpathlineto{\pgfqpoint{4.167079in}{1.863228in}}%
\pgfpathlineto{\pgfqpoint{4.197946in}{1.819197in}}%
\pgfpathlineto{\pgfqpoint{4.228813in}{1.767596in}}%
\pgfpathlineto{\pgfqpoint{4.305982in}{1.632109in}}%
\pgfpathlineto{\pgfqpoint{4.336849in}{1.586186in}}%
\pgfpathlineto{\pgfqpoint{4.352283in}{1.566924in}}%
\pgfpathlineto{\pgfqpoint{4.367716in}{1.550682in}}%
\pgfpathlineto{\pgfqpoint{4.383150in}{1.537822in}}%
\pgfpathlineto{\pgfqpoint{4.398583in}{1.528656in}}%
\pgfpathlineto{\pgfqpoint{4.414017in}{1.523443in}}%
\pgfpathlineto{\pgfqpoint{4.429451in}{1.522386in}}%
\pgfpathlineto{\pgfqpoint{4.444884in}{1.525627in}}%
\pgfpathlineto{\pgfqpoint{4.460318in}{1.533244in}}%
\pgfpathlineto{\pgfqpoint{4.475752in}{1.545253in}}%
\pgfpathlineto{\pgfqpoint{4.491185in}{1.561608in}}%
\pgfpathlineto{\pgfqpoint{4.506619in}{1.582196in}}%
\pgfpathlineto{\pgfqpoint{4.522052in}{1.606843in}}%
\pgfpathlineto{\pgfqpoint{4.537486in}{1.635317in}}%
\pgfpathlineto{\pgfqpoint{4.568353in}{1.702536in}}%
\pgfpathlineto{\pgfqpoint{4.599221in}{1.780936in}}%
\pgfpathlineto{\pgfqpoint{4.645522in}{1.911515in}}%
\pgfpathlineto{\pgfqpoint{4.660955in}{1.956451in}}%
\pgfpathlineto{\pgfqpoint{4.660955in}{1.956451in}}%
\pgfusepath{stroke}%
\end{pgfscope}%
\begin{pgfscope}%
\pgfpathrectangle{\pgfqpoint{0.625831in}{0.505056in}}{\pgfqpoint{4.227273in}{2.745455in}} %
\pgfusepath{clip}%
\pgfsetrectcap%
\pgfsetroundjoin%
\pgfsetlinewidth{0.501875pt}%
\definecolor{currentstroke}{rgb}{0.056863,0.767363,0.905873}%
\pgfsetstrokecolor{currentstroke}%
\pgfsetdash{}{0pt}%
\pgfpathmoveto{\pgfqpoint{0.817980in}{2.372354in}}%
\pgfpathlineto{\pgfqpoint{0.833414in}{2.373076in}}%
\pgfpathlineto{\pgfqpoint{0.848847in}{2.368004in}}%
\pgfpathlineto{\pgfqpoint{0.864281in}{2.356956in}}%
\pgfpathlineto{\pgfqpoint{0.879715in}{2.339885in}}%
\pgfpathlineto{\pgfqpoint{0.895148in}{2.316891in}}%
\pgfpathlineto{\pgfqpoint{0.910582in}{2.288225in}}%
\pgfpathlineto{\pgfqpoint{0.926015in}{2.254280in}}%
\pgfpathlineto{\pgfqpoint{0.956883in}{2.172825in}}%
\pgfpathlineto{\pgfqpoint{0.987750in}{2.078271in}}%
\pgfpathlineto{\pgfqpoint{1.049485in}{1.879828in}}%
\pgfpathlineto{\pgfqpoint{1.080352in}{1.792213in}}%
\pgfpathlineto{\pgfqpoint{1.095785in}{1.754677in}}%
\pgfpathlineto{\pgfqpoint{1.111219in}{1.722406in}}%
\pgfpathlineto{\pgfqpoint{1.126653in}{1.696072in}}%
\pgfpathlineto{\pgfqpoint{1.142086in}{1.676207in}}%
\pgfpathlineto{\pgfqpoint{1.157520in}{1.663196in}}%
\pgfpathlineto{\pgfqpoint{1.172954in}{1.657267in}}%
\pgfpathlineto{\pgfqpoint{1.188387in}{1.658479in}}%
\pgfpathlineto{\pgfqpoint{1.203821in}{1.666730in}}%
\pgfpathlineto{\pgfqpoint{1.219255in}{1.681756in}}%
\pgfpathlineto{\pgfqpoint{1.234688in}{1.703141in}}%
\pgfpathlineto{\pgfqpoint{1.250122in}{1.730329in}}%
\pgfpathlineto{\pgfqpoint{1.265555in}{1.762639in}}%
\pgfpathlineto{\pgfqpoint{1.296423in}{1.839399in}}%
\pgfpathlineto{\pgfqpoint{1.342724in}{1.971121in}}%
\pgfpathlineto{\pgfqpoint{1.389025in}{2.099956in}}%
\pgfpathlineto{\pgfqpoint{1.419892in}{2.172885in}}%
\pgfpathlineto{\pgfqpoint{1.435325in}{2.203480in}}%
\pgfpathlineto{\pgfqpoint{1.450759in}{2.229549in}}%
\pgfpathlineto{\pgfqpoint{1.466193in}{2.250797in}}%
\pgfpathlineto{\pgfqpoint{1.481626in}{2.267060in}}%
\pgfpathlineto{\pgfqpoint{1.497060in}{2.278304in}}%
\pgfpathlineto{\pgfqpoint{1.512494in}{2.284621in}}%
\pgfpathlineto{\pgfqpoint{1.527927in}{2.286220in}}%
\pgfpathlineto{\pgfqpoint{1.543361in}{2.283410in}}%
\pgfpathlineto{\pgfqpoint{1.558795in}{2.276591in}}%
\pgfpathlineto{\pgfqpoint{1.574228in}{2.266232in}}%
\pgfpathlineto{\pgfqpoint{1.589662in}{2.252851in}}%
\pgfpathlineto{\pgfqpoint{1.620529in}{2.219247in}}%
\pgfpathlineto{\pgfqpoint{1.666830in}{2.159953in}}%
\pgfpathlineto{\pgfqpoint{1.713131in}{2.101100in}}%
\pgfpathlineto{\pgfqpoint{1.743998in}{2.066096in}}%
\pgfpathlineto{\pgfqpoint{1.774865in}{2.035400in}}%
\pgfpathlineto{\pgfqpoint{1.805733in}{2.008431in}}%
\pgfpathlineto{\pgfqpoint{1.898334in}{1.932947in}}%
\pgfpathlineto{\pgfqpoint{1.929202in}{1.903473in}}%
\pgfpathlineto{\pgfqpoint{1.960069in}{1.870168in}}%
\pgfpathlineto{\pgfqpoint{2.006370in}{1.814527in}}%
\pgfpathlineto{\pgfqpoint{2.052671in}{1.758425in}}%
\pgfpathlineto{\pgfqpoint{2.083538in}{1.726107in}}%
\pgfpathlineto{\pgfqpoint{2.098972in}{1.712845in}}%
\pgfpathlineto{\pgfqpoint{2.114405in}{1.702104in}}%
\pgfpathlineto{\pgfqpoint{2.129839in}{1.694281in}}%
\pgfpathlineto{\pgfqpoint{2.145273in}{1.689709in}}%
\pgfpathlineto{\pgfqpoint{2.160706in}{1.688652in}}%
\pgfpathlineto{\pgfqpoint{2.176140in}{1.691277in}}%
\pgfpathlineto{\pgfqpoint{2.191574in}{1.697655in}}%
\pgfpathlineto{\pgfqpoint{2.207007in}{1.707741in}}%
\pgfpathlineto{\pgfqpoint{2.222441in}{1.721377in}}%
\pgfpathlineto{\pgfqpoint{2.237874in}{1.738285in}}%
\pgfpathlineto{\pgfqpoint{2.268742in}{1.780235in}}%
\pgfpathlineto{\pgfqpoint{2.315043in}{1.854535in}}%
\pgfpathlineto{\pgfqpoint{2.345910in}{1.902935in}}%
\pgfpathlineto{\pgfqpoint{2.361344in}{1.924313in}}%
\pgfpathlineto{\pgfqpoint{2.376777in}{1.942706in}}%
\pgfpathlineto{\pgfqpoint{2.392211in}{1.957339in}}%
\pgfpathlineto{\pgfqpoint{2.407644in}{1.967500in}}%
\pgfpathlineto{\pgfqpoint{2.423078in}{1.972571in}}%
\pgfpathlineto{\pgfqpoint{2.438512in}{1.972042in}}%
\pgfpathlineto{\pgfqpoint{2.453945in}{1.965533in}}%
\pgfpathlineto{\pgfqpoint{2.469379in}{1.952806in}}%
\pgfpathlineto{\pgfqpoint{2.484813in}{1.933775in}}%
\pgfpathlineto{\pgfqpoint{2.500246in}{1.908512in}}%
\pgfpathlineto{\pgfqpoint{2.515680in}{1.877251in}}%
\pgfpathlineto{\pgfqpoint{2.531114in}{1.840381in}}%
\pgfpathlineto{\pgfqpoint{2.561981in}{1.752114in}}%
\pgfpathlineto{\pgfqpoint{2.592848in}{1.649584in}}%
\pgfpathlineto{\pgfqpoint{2.670016in}{1.381885in}}%
\pgfpathlineto{\pgfqpoint{2.700884in}{1.291920in}}%
\pgfpathlineto{\pgfqpoint{2.716317in}{1.254307in}}%
\pgfpathlineto{\pgfqpoint{2.731751in}{1.222552in}}%
\pgfpathlineto{\pgfqpoint{2.747184in}{1.197191in}}%
\pgfpathlineto{\pgfqpoint{2.762618in}{1.178611in}}%
\pgfpathlineto{\pgfqpoint{2.778052in}{1.167035in}}%
\pgfpathlineto{\pgfqpoint{2.793485in}{1.162525in}}%
\pgfpathlineto{\pgfqpoint{2.808919in}{1.164976in}}%
\pgfpathlineto{\pgfqpoint{2.824353in}{1.174126in}}%
\pgfpathlineto{\pgfqpoint{2.839786in}{1.189567in}}%
\pgfpathlineto{\pgfqpoint{2.855220in}{1.210753in}}%
\pgfpathlineto{\pgfqpoint{2.870654in}{1.237022in}}%
\pgfpathlineto{\pgfqpoint{2.886087in}{1.267616in}}%
\pgfpathlineto{\pgfqpoint{2.916954in}{1.338399in}}%
\pgfpathlineto{\pgfqpoint{2.994123in}{1.529774in}}%
\pgfpathlineto{\pgfqpoint{3.024990in}{1.595080in}}%
\pgfpathlineto{\pgfqpoint{3.040424in}{1.622847in}}%
\pgfpathlineto{\pgfqpoint{3.055857in}{1.646858in}}%
\pgfpathlineto{\pgfqpoint{3.071291in}{1.666889in}}%
\pgfpathlineto{\pgfqpoint{3.086724in}{1.682841in}}%
\pgfpathlineto{\pgfqpoint{3.102158in}{1.694734in}}%
\pgfpathlineto{\pgfqpoint{3.117592in}{1.702696in}}%
\pgfpathlineto{\pgfqpoint{3.133025in}{1.706953in}}%
\pgfpathlineto{\pgfqpoint{3.148459in}{1.707818in}}%
\pgfpathlineto{\pgfqpoint{3.163893in}{1.705668in}}%
\pgfpathlineto{\pgfqpoint{3.179326in}{1.700932in}}%
\pgfpathlineto{\pgfqpoint{3.210193in}{1.685550in}}%
\pgfpathlineto{\pgfqpoint{3.241061in}{1.665372in}}%
\pgfpathlineto{\pgfqpoint{3.302795in}{1.623063in}}%
\pgfpathlineto{\pgfqpoint{3.333663in}{1.604765in}}%
\pgfpathlineto{\pgfqpoint{3.379963in}{1.581738in}}%
\pgfpathlineto{\pgfqpoint{3.441698in}{1.552426in}}%
\pgfpathlineto{\pgfqpoint{3.472565in}{1.534039in}}%
\pgfpathlineto{\pgfqpoint{3.503433in}{1.510949in}}%
\pgfpathlineto{\pgfqpoint{3.534300in}{1.482501in}}%
\pgfpathlineto{\pgfqpoint{3.565167in}{1.449256in}}%
\pgfpathlineto{\pgfqpoint{3.642335in}{1.360711in}}%
\pgfpathlineto{\pgfqpoint{3.657769in}{1.345829in}}%
\pgfpathlineto{\pgfqpoint{3.673203in}{1.333209in}}%
\pgfpathlineto{\pgfqpoint{3.688636in}{1.323436in}}%
\pgfpathlineto{\pgfqpoint{3.704070in}{1.317067in}}%
\pgfpathlineto{\pgfqpoint{3.719503in}{1.314611in}}%
\pgfpathlineto{\pgfqpoint{3.734937in}{1.316512in}}%
\pgfpathlineto{\pgfqpoint{3.750371in}{1.323128in}}%
\pgfpathlineto{\pgfqpoint{3.765804in}{1.334719in}}%
\pgfpathlineto{\pgfqpoint{3.781238in}{1.351428in}}%
\pgfpathlineto{\pgfqpoint{3.796672in}{1.373274in}}%
\pgfpathlineto{\pgfqpoint{3.812105in}{1.400146in}}%
\pgfpathlineto{\pgfqpoint{3.827539in}{1.431798in}}%
\pgfpathlineto{\pgfqpoint{3.858406in}{1.507805in}}%
\pgfpathlineto{\pgfqpoint{3.889273in}{1.596804in}}%
\pgfpathlineto{\pgfqpoint{3.966442in}{1.833462in}}%
\pgfpathlineto{\pgfqpoint{3.997309in}{1.914488in}}%
\pgfpathlineto{\pgfqpoint{4.012743in}{1.948596in}}%
\pgfpathlineto{\pgfqpoint{4.028176in}{1.977492in}}%
\pgfpathlineto{\pgfqpoint{4.043610in}{2.000617in}}%
\pgfpathlineto{\pgfqpoint{4.059043in}{2.017547in}}%
\pgfpathlineto{\pgfqpoint{4.074477in}{2.028008in}}%
\pgfpathlineto{\pgfqpoint{4.089911in}{2.031885in}}%
\pgfpathlineto{\pgfqpoint{4.105344in}{2.029224in}}%
\pgfpathlineto{\pgfqpoint{4.120778in}{2.020232in}}%
\pgfpathlineto{\pgfqpoint{4.136212in}{2.005274in}}%
\pgfpathlineto{\pgfqpoint{4.151645in}{1.984863in}}%
\pgfpathlineto{\pgfqpoint{4.167079in}{1.959644in}}%
\pgfpathlineto{\pgfqpoint{4.197946in}{1.897929in}}%
\pgfpathlineto{\pgfqpoint{4.290548in}{1.690370in}}%
\pgfpathlineto{\pgfqpoint{4.305982in}{1.662555in}}%
\pgfpathlineto{\pgfqpoint{4.321415in}{1.638897in}}%
\pgfpathlineto{\pgfqpoint{4.336849in}{1.620008in}}%
\pgfpathlineto{\pgfqpoint{4.352283in}{1.606368in}}%
\pgfpathlineto{\pgfqpoint{4.367716in}{1.598312in}}%
\pgfpathlineto{\pgfqpoint{4.383150in}{1.596031in}}%
\pgfpathlineto{\pgfqpoint{4.398583in}{1.599562in}}%
\pgfpathlineto{\pgfqpoint{4.414017in}{1.608798in}}%
\pgfpathlineto{\pgfqpoint{4.429451in}{1.623490in}}%
\pgfpathlineto{\pgfqpoint{4.444884in}{1.643260in}}%
\pgfpathlineto{\pgfqpoint{4.460318in}{1.667616in}}%
\pgfpathlineto{\pgfqpoint{4.475752in}{1.695965in}}%
\pgfpathlineto{\pgfqpoint{4.506619in}{1.761907in}}%
\pgfpathlineto{\pgfqpoint{4.614654in}{2.011590in}}%
\pgfpathlineto{\pgfqpoint{4.645522in}{2.066892in}}%
\pgfpathlineto{\pgfqpoint{4.660955in}{2.089735in}}%
\pgfpathlineto{\pgfqpoint{4.660955in}{2.089735in}}%
\pgfusepath{stroke}%
\end{pgfscope}%
\begin{pgfscope}%
\pgfpathrectangle{\pgfqpoint{0.625831in}{0.505056in}}{\pgfqpoint{4.227273in}{2.745455in}} %
\pgfusepath{clip}%
\pgfsetrectcap%
\pgfsetroundjoin%
\pgfsetlinewidth{0.501875pt}%
\definecolor{currentstroke}{rgb}{0.135294,0.840344,0.878081}%
\pgfsetstrokecolor{currentstroke}%
\pgfsetdash{}{0pt}%
\pgfpathmoveto{\pgfqpoint{0.817980in}{2.512776in}}%
\pgfpathlineto{\pgfqpoint{0.833414in}{2.508565in}}%
\pgfpathlineto{\pgfqpoint{0.848847in}{2.497386in}}%
\pgfpathlineto{\pgfqpoint{0.864281in}{2.479115in}}%
\pgfpathlineto{\pgfqpoint{0.879715in}{2.453763in}}%
\pgfpathlineto{\pgfqpoint{0.895148in}{2.421481in}}%
\pgfpathlineto{\pgfqpoint{0.910582in}{2.382561in}}%
\pgfpathlineto{\pgfqpoint{0.926015in}{2.337437in}}%
\pgfpathlineto{\pgfqpoint{0.956883in}{2.230981in}}%
\pgfpathlineto{\pgfqpoint{0.987750in}{2.108132in}}%
\pgfpathlineto{\pgfqpoint{1.080352in}{1.721636in}}%
\pgfpathlineto{\pgfqpoint{1.111219in}{1.615724in}}%
\pgfpathlineto{\pgfqpoint{1.126653in}{1.571525in}}%
\pgfpathlineto{\pgfqpoint{1.142086in}{1.534194in}}%
\pgfpathlineto{\pgfqpoint{1.157520in}{1.504347in}}%
\pgfpathlineto{\pgfqpoint{1.172954in}{1.482453in}}%
\pgfpathlineto{\pgfqpoint{1.188387in}{1.468833in}}%
\pgfpathlineto{\pgfqpoint{1.203821in}{1.463646in}}%
\pgfpathlineto{\pgfqpoint{1.219255in}{1.466895in}}%
\pgfpathlineto{\pgfqpoint{1.234688in}{1.478424in}}%
\pgfpathlineto{\pgfqpoint{1.250122in}{1.497924in}}%
\pgfpathlineto{\pgfqpoint{1.265555in}{1.524945in}}%
\pgfpathlineto{\pgfqpoint{1.280989in}{1.558905in}}%
\pgfpathlineto{\pgfqpoint{1.296423in}{1.599108in}}%
\pgfpathlineto{\pgfqpoint{1.327290in}{1.694980in}}%
\pgfpathlineto{\pgfqpoint{1.358157in}{1.805374in}}%
\pgfpathlineto{\pgfqpoint{1.435325in}{2.094108in}}%
\pgfpathlineto{\pgfqpoint{1.466193in}{2.195929in}}%
\pgfpathlineto{\pgfqpoint{1.497060in}{2.281476in}}%
\pgfpathlineto{\pgfqpoint{1.512494in}{2.316977in}}%
\pgfpathlineto{\pgfqpoint{1.527927in}{2.347217in}}%
\pgfpathlineto{\pgfqpoint{1.543361in}{2.372016in}}%
\pgfpathlineto{\pgfqpoint{1.558795in}{2.391297in}}%
\pgfpathlineto{\pgfqpoint{1.574228in}{2.405074in}}%
\pgfpathlineto{\pgfqpoint{1.589662in}{2.413446in}}%
\pgfpathlineto{\pgfqpoint{1.605095in}{2.416587in}}%
\pgfpathlineto{\pgfqpoint{1.620529in}{2.414736in}}%
\pgfpathlineto{\pgfqpoint{1.635963in}{2.408184in}}%
\pgfpathlineto{\pgfqpoint{1.651396in}{2.397262in}}%
\pgfpathlineto{\pgfqpoint{1.666830in}{2.382330in}}%
\pgfpathlineto{\pgfqpoint{1.682264in}{2.363767in}}%
\pgfpathlineto{\pgfqpoint{1.697697in}{2.341958in}}%
\pgfpathlineto{\pgfqpoint{1.728564in}{2.290130in}}%
\pgfpathlineto{\pgfqpoint{1.759432in}{2.229778in}}%
\pgfpathlineto{\pgfqpoint{1.805733in}{2.128928in}}%
\pgfpathlineto{\pgfqpoint{1.944635in}{1.814093in}}%
\pgfpathlineto{\pgfqpoint{1.975503in}{1.752508in}}%
\pgfpathlineto{\pgfqpoint{2.006370in}{1.697875in}}%
\pgfpathlineto{\pgfqpoint{2.037237in}{1.652204in}}%
\pgfpathlineto{\pgfqpoint{2.052671in}{1.633349in}}%
\pgfpathlineto{\pgfqpoint{2.068104in}{1.617457in}}%
\pgfpathlineto{\pgfqpoint{2.083538in}{1.604731in}}%
\pgfpathlineto{\pgfqpoint{2.098972in}{1.595344in}}%
\pgfpathlineto{\pgfqpoint{2.114405in}{1.589429in}}%
\pgfpathlineto{\pgfqpoint{2.129839in}{1.587071in}}%
\pgfpathlineto{\pgfqpoint{2.145273in}{1.588304in}}%
\pgfpathlineto{\pgfqpoint{2.160706in}{1.593097in}}%
\pgfpathlineto{\pgfqpoint{2.176140in}{1.601358in}}%
\pgfpathlineto{\pgfqpoint{2.191574in}{1.612922in}}%
\pgfpathlineto{\pgfqpoint{2.207007in}{1.627559in}}%
\pgfpathlineto{\pgfqpoint{2.222441in}{1.644963in}}%
\pgfpathlineto{\pgfqpoint{2.253308in}{1.686537in}}%
\pgfpathlineto{\pgfqpoint{2.299609in}{1.758525in}}%
\pgfpathlineto{\pgfqpoint{2.330476in}{1.806326in}}%
\pgfpathlineto{\pgfqpoint{2.361344in}{1.848337in}}%
\pgfpathlineto{\pgfqpoint{2.376777in}{1.865718in}}%
\pgfpathlineto{\pgfqpoint{2.392211in}{1.879981in}}%
\pgfpathlineto{\pgfqpoint{2.407644in}{1.890671in}}%
\pgfpathlineto{\pgfqpoint{2.423078in}{1.897408in}}%
\pgfpathlineto{\pgfqpoint{2.438512in}{1.899891in}}%
\pgfpathlineto{\pgfqpoint{2.453945in}{1.897915in}}%
\pgfpathlineto{\pgfqpoint{2.469379in}{1.891371in}}%
\pgfpathlineto{\pgfqpoint{2.484813in}{1.880256in}}%
\pgfpathlineto{\pgfqpoint{2.500246in}{1.864670in}}%
\pgfpathlineto{\pgfqpoint{2.515680in}{1.844818in}}%
\pgfpathlineto{\pgfqpoint{2.531114in}{1.821006in}}%
\pgfpathlineto{\pgfqpoint{2.561981in}{1.763175in}}%
\pgfpathlineto{\pgfqpoint{2.592848in}{1.695313in}}%
\pgfpathlineto{\pgfqpoint{2.670016in}{1.516099in}}%
\pgfpathlineto{\pgfqpoint{2.700884in}{1.454799in}}%
\pgfpathlineto{\pgfqpoint{2.716317in}{1.428768in}}%
\pgfpathlineto{\pgfqpoint{2.731751in}{1.406421in}}%
\pgfpathlineto{\pgfqpoint{2.747184in}{1.388103in}}%
\pgfpathlineto{\pgfqpoint{2.762618in}{1.374063in}}%
\pgfpathlineto{\pgfqpoint{2.778052in}{1.364450in}}%
\pgfpathlineto{\pgfqpoint{2.793485in}{1.359310in}}%
\pgfpathlineto{\pgfqpoint{2.808919in}{1.358587in}}%
\pgfpathlineto{\pgfqpoint{2.824353in}{1.362131in}}%
\pgfpathlineto{\pgfqpoint{2.839786in}{1.369697in}}%
\pgfpathlineto{\pgfqpoint{2.855220in}{1.380962in}}%
\pgfpathlineto{\pgfqpoint{2.870654in}{1.395530in}}%
\pgfpathlineto{\pgfqpoint{2.886087in}{1.412946in}}%
\pgfpathlineto{\pgfqpoint{2.916954in}{1.454292in}}%
\pgfpathlineto{\pgfqpoint{3.024990in}{1.611901in}}%
\pgfpathlineto{\pgfqpoint{3.055857in}{1.645759in}}%
\pgfpathlineto{\pgfqpoint{3.071291in}{1.659346in}}%
\pgfpathlineto{\pgfqpoint{3.086724in}{1.670549in}}%
\pgfpathlineto{\pgfqpoint{3.102158in}{1.679331in}}%
\pgfpathlineto{\pgfqpoint{3.117592in}{1.685717in}}%
\pgfpathlineto{\pgfqpoint{3.133025in}{1.689785in}}%
\pgfpathlineto{\pgfqpoint{3.148459in}{1.691657in}}%
\pgfpathlineto{\pgfqpoint{3.163893in}{1.691493in}}%
\pgfpathlineto{\pgfqpoint{3.179326in}{1.689479in}}%
\pgfpathlineto{\pgfqpoint{3.210193in}{1.680718in}}%
\pgfpathlineto{\pgfqpoint{3.241061in}{1.667020in}}%
\pgfpathlineto{\pgfqpoint{3.271928in}{1.649863in}}%
\pgfpathlineto{\pgfqpoint{3.318229in}{1.619911in}}%
\pgfpathlineto{\pgfqpoint{3.364530in}{1.586194in}}%
\pgfpathlineto{\pgfqpoint{3.410831in}{1.548327in}}%
\pgfpathlineto{\pgfqpoint{3.457132in}{1.505277in}}%
\pgfpathlineto{\pgfqpoint{3.503433in}{1.457324in}}%
\pgfpathlineto{\pgfqpoint{3.565167in}{1.392108in}}%
\pgfpathlineto{\pgfqpoint{3.596034in}{1.363957in}}%
\pgfpathlineto{\pgfqpoint{3.611468in}{1.352328in}}%
\pgfpathlineto{\pgfqpoint{3.626902in}{1.342906in}}%
\pgfpathlineto{\pgfqpoint{3.642335in}{1.336128in}}%
\pgfpathlineto{\pgfqpoint{3.657769in}{1.332420in}}%
\pgfpathlineto{\pgfqpoint{3.673203in}{1.332184in}}%
\pgfpathlineto{\pgfqpoint{3.688636in}{1.335782in}}%
\pgfpathlineto{\pgfqpoint{3.704070in}{1.343522in}}%
\pgfpathlineto{\pgfqpoint{3.719503in}{1.355644in}}%
\pgfpathlineto{\pgfqpoint{3.734937in}{1.372308in}}%
\pgfpathlineto{\pgfqpoint{3.750371in}{1.393586in}}%
\pgfpathlineto{\pgfqpoint{3.765804in}{1.419447in}}%
\pgfpathlineto{\pgfqpoint{3.781238in}{1.449758in}}%
\pgfpathlineto{\pgfqpoint{3.796672in}{1.484275in}}%
\pgfpathlineto{\pgfqpoint{3.827539in}{1.564414in}}%
\pgfpathlineto{\pgfqpoint{3.858406in}{1.655812in}}%
\pgfpathlineto{\pgfqpoint{3.935574in}{1.895161in}}%
\pgfpathlineto{\pgfqpoint{3.966442in}{1.977460in}}%
\pgfpathlineto{\pgfqpoint{3.981875in}{2.012392in}}%
\pgfpathlineto{\pgfqpoint{3.997309in}{2.042221in}}%
\pgfpathlineto{\pgfqpoint{4.012743in}{2.066341in}}%
\pgfpathlineto{\pgfqpoint{4.028176in}{2.084259in}}%
\pgfpathlineto{\pgfqpoint{4.043610in}{2.095610in}}%
\pgfpathlineto{\pgfqpoint{4.059043in}{2.100164in}}%
\pgfpathlineto{\pgfqpoint{4.074477in}{2.097840in}}%
\pgfpathlineto{\pgfqpoint{4.089911in}{2.088704in}}%
\pgfpathlineto{\pgfqpoint{4.105344in}{2.072976in}}%
\pgfpathlineto{\pgfqpoint{4.120778in}{2.051021in}}%
\pgfpathlineto{\pgfqpoint{4.136212in}{2.023346in}}%
\pgfpathlineto{\pgfqpoint{4.151645in}{1.990587in}}%
\pgfpathlineto{\pgfqpoint{4.182513in}{1.912923in}}%
\pgfpathlineto{\pgfqpoint{4.228813in}{1.779856in}}%
\pgfpathlineto{\pgfqpoint{4.259681in}{1.691812in}}%
\pgfpathlineto{\pgfqpoint{4.290548in}{1.613563in}}%
\pgfpathlineto{\pgfqpoint{4.305982in}{1.580331in}}%
\pgfpathlineto{\pgfqpoint{4.321415in}{1.552029in}}%
\pgfpathlineto{\pgfqpoint{4.336849in}{1.529269in}}%
\pgfpathlineto{\pgfqpoint{4.352283in}{1.512526in}}%
\pgfpathlineto{\pgfqpoint{4.367716in}{1.502141in}}%
\pgfpathlineto{\pgfqpoint{4.383150in}{1.498305in}}%
\pgfpathlineto{\pgfqpoint{4.398583in}{1.501063in}}%
\pgfpathlineto{\pgfqpoint{4.414017in}{1.510315in}}%
\pgfpathlineto{\pgfqpoint{4.429451in}{1.525820in}}%
\pgfpathlineto{\pgfqpoint{4.444884in}{1.547206in}}%
\pgfpathlineto{\pgfqpoint{4.460318in}{1.573983in}}%
\pgfpathlineto{\pgfqpoint{4.475752in}{1.605560in}}%
\pgfpathlineto{\pgfqpoint{4.506619in}{1.680323in}}%
\pgfpathlineto{\pgfqpoint{4.537486in}{1.765383in}}%
\pgfpathlineto{\pgfqpoint{4.599221in}{1.940538in}}%
\pgfpathlineto{\pgfqpoint{4.630088in}{2.018873in}}%
\pgfpathlineto{\pgfqpoint{4.660955in}{2.085001in}}%
\pgfpathlineto{\pgfqpoint{4.660955in}{2.085001in}}%
\pgfusepath{stroke}%
\end{pgfscope}%
\begin{pgfscope}%
\pgfpathrectangle{\pgfqpoint{0.625831in}{0.505056in}}{\pgfqpoint{4.227273in}{2.745455in}} %
\pgfusepath{clip}%
\pgfsetrectcap%
\pgfsetroundjoin%
\pgfsetlinewidth{0.501875pt}%
\definecolor{currentstroke}{rgb}{0.221569,0.905873,0.843667}%
\pgfsetstrokecolor{currentstroke}%
\pgfsetdash{}{0pt}%
\pgfpathmoveto{\pgfqpoint{0.817980in}{2.254288in}}%
\pgfpathlineto{\pgfqpoint{0.833414in}{2.266329in}}%
\pgfpathlineto{\pgfqpoint{0.848847in}{2.274769in}}%
\pgfpathlineto{\pgfqpoint{0.864281in}{2.279356in}}%
\pgfpathlineto{\pgfqpoint{0.879715in}{2.279896in}}%
\pgfpathlineto{\pgfqpoint{0.895148in}{2.276257in}}%
\pgfpathlineto{\pgfqpoint{0.910582in}{2.268372in}}%
\pgfpathlineto{\pgfqpoint{0.926015in}{2.256249in}}%
\pgfpathlineto{\pgfqpoint{0.941449in}{2.239964in}}%
\pgfpathlineto{\pgfqpoint{0.956883in}{2.219672in}}%
\pgfpathlineto{\pgfqpoint{0.972316in}{2.195598in}}%
\pgfpathlineto{\pgfqpoint{1.003184in}{2.137369in}}%
\pgfpathlineto{\pgfqpoint{1.034051in}{2.068459in}}%
\pgfpathlineto{\pgfqpoint{1.095785in}{1.915612in}}%
\pgfpathlineto{\pgfqpoint{1.126653in}{1.841512in}}%
\pgfpathlineto{\pgfqpoint{1.157520in}{1.775732in}}%
\pgfpathlineto{\pgfqpoint{1.172954in}{1.747452in}}%
\pgfpathlineto{\pgfqpoint{1.188387in}{1.722964in}}%
\pgfpathlineto{\pgfqpoint{1.203821in}{1.702737in}}%
\pgfpathlineto{\pgfqpoint{1.219255in}{1.687175in}}%
\pgfpathlineto{\pgfqpoint{1.234688in}{1.676612in}}%
\pgfpathlineto{\pgfqpoint{1.250122in}{1.671299in}}%
\pgfpathlineto{\pgfqpoint{1.265555in}{1.671405in}}%
\pgfpathlineto{\pgfqpoint{1.280989in}{1.677006in}}%
\pgfpathlineto{\pgfqpoint{1.296423in}{1.688093in}}%
\pgfpathlineto{\pgfqpoint{1.311856in}{1.704560in}}%
\pgfpathlineto{\pgfqpoint{1.327290in}{1.726215in}}%
\pgfpathlineto{\pgfqpoint{1.342724in}{1.752777in}}%
\pgfpathlineto{\pgfqpoint{1.358157in}{1.783884in}}%
\pgfpathlineto{\pgfqpoint{1.389025in}{1.857906in}}%
\pgfpathlineto{\pgfqpoint{1.419892in}{1.943997in}}%
\pgfpathlineto{\pgfqpoint{1.527927in}{2.264109in}}%
\pgfpathlineto{\pgfqpoint{1.558795in}{2.338781in}}%
\pgfpathlineto{\pgfqpoint{1.574228in}{2.370397in}}%
\pgfpathlineto{\pgfqpoint{1.589662in}{2.397625in}}%
\pgfpathlineto{\pgfqpoint{1.605095in}{2.420127in}}%
\pgfpathlineto{\pgfqpoint{1.620529in}{2.437644in}}%
\pgfpathlineto{\pgfqpoint{1.635963in}{2.450004in}}%
\pgfpathlineto{\pgfqpoint{1.651396in}{2.457119in}}%
\pgfpathlineto{\pgfqpoint{1.666830in}{2.458984in}}%
\pgfpathlineto{\pgfqpoint{1.682264in}{2.455680in}}%
\pgfpathlineto{\pgfqpoint{1.697697in}{2.447368in}}%
\pgfpathlineto{\pgfqpoint{1.713131in}{2.434284in}}%
\pgfpathlineto{\pgfqpoint{1.728564in}{2.416733in}}%
\pgfpathlineto{\pgfqpoint{1.743998in}{2.395083in}}%
\pgfpathlineto{\pgfqpoint{1.759432in}{2.369755in}}%
\pgfpathlineto{\pgfqpoint{1.790299in}{2.309970in}}%
\pgfpathlineto{\pgfqpoint{1.821166in}{2.241469in}}%
\pgfpathlineto{\pgfqpoint{1.913768in}{2.026078in}}%
\pgfpathlineto{\pgfqpoint{1.944635in}{1.963464in}}%
\pgfpathlineto{\pgfqpoint{1.975503in}{1.909982in}}%
\pgfpathlineto{\pgfqpoint{2.006370in}{1.867002in}}%
\pgfpathlineto{\pgfqpoint{2.021804in}{1.849611in}}%
\pgfpathlineto{\pgfqpoint{2.037237in}{1.834919in}}%
\pgfpathlineto{\pgfqpoint{2.052671in}{1.822833in}}%
\pgfpathlineto{\pgfqpoint{2.068104in}{1.813210in}}%
\pgfpathlineto{\pgfqpoint{2.083538in}{1.805862in}}%
\pgfpathlineto{\pgfqpoint{2.098972in}{1.800563in}}%
\pgfpathlineto{\pgfqpoint{2.129839in}{1.795064in}}%
\pgfpathlineto{\pgfqpoint{2.160706in}{1.794418in}}%
\pgfpathlineto{\pgfqpoint{2.253308in}{1.797925in}}%
\pgfpathlineto{\pgfqpoint{2.284175in}{1.794268in}}%
\pgfpathlineto{\pgfqpoint{2.315043in}{1.786069in}}%
\pgfpathlineto{\pgfqpoint{2.345910in}{1.772946in}}%
\pgfpathlineto{\pgfqpoint{2.376777in}{1.755130in}}%
\pgfpathlineto{\pgfqpoint{2.407644in}{1.733387in}}%
\pgfpathlineto{\pgfqpoint{2.453945in}{1.696092in}}%
\pgfpathlineto{\pgfqpoint{2.515680in}{1.645540in}}%
\pgfpathlineto{\pgfqpoint{2.546547in}{1.623276in}}%
\pgfpathlineto{\pgfqpoint{2.577414in}{1.604636in}}%
\pgfpathlineto{\pgfqpoint{2.608282in}{1.590386in}}%
\pgfpathlineto{\pgfqpoint{2.639149in}{1.580856in}}%
\pgfpathlineto{\pgfqpoint{2.670016in}{1.575922in}}%
\pgfpathlineto{\pgfqpoint{2.700884in}{1.575038in}}%
\pgfpathlineto{\pgfqpoint{2.731751in}{1.577305in}}%
\pgfpathlineto{\pgfqpoint{2.793485in}{1.586523in}}%
\pgfpathlineto{\pgfqpoint{2.839786in}{1.592421in}}%
\pgfpathlineto{\pgfqpoint{2.870654in}{1.593627in}}%
\pgfpathlineto{\pgfqpoint{2.901521in}{1.591592in}}%
\pgfpathlineto{\pgfqpoint{2.932388in}{1.585739in}}%
\pgfpathlineto{\pgfqpoint{2.963255in}{1.575833in}}%
\pgfpathlineto{\pgfqpoint{2.994123in}{1.561969in}}%
\pgfpathlineto{\pgfqpoint{3.024990in}{1.544554in}}%
\pgfpathlineto{\pgfqpoint{3.071291in}{1.513271in}}%
\pgfpathlineto{\pgfqpoint{3.210193in}{1.411707in}}%
\pgfpathlineto{\pgfqpoint{3.241061in}{1.393647in}}%
\pgfpathlineto{\pgfqpoint{3.271928in}{1.378939in}}%
\pgfpathlineto{\pgfqpoint{3.302795in}{1.368155in}}%
\pgfpathlineto{\pgfqpoint{3.333663in}{1.361753in}}%
\pgfpathlineto{\pgfqpoint{3.364530in}{1.360092in}}%
\pgfpathlineto{\pgfqpoint{3.395397in}{1.363452in}}%
\pgfpathlineto{\pgfqpoint{3.426264in}{1.372038in}}%
\pgfpathlineto{\pgfqpoint{3.457132in}{1.385989in}}%
\pgfpathlineto{\pgfqpoint{3.487999in}{1.405366in}}%
\pgfpathlineto{\pgfqpoint{3.518866in}{1.430128in}}%
\pgfpathlineto{\pgfqpoint{3.549733in}{1.460105in}}%
\pgfpathlineto{\pgfqpoint{3.580601in}{1.494959in}}%
\pgfpathlineto{\pgfqpoint{3.611468in}{1.534151in}}%
\pgfpathlineto{\pgfqpoint{3.657769in}{1.599321in}}%
\pgfpathlineto{\pgfqpoint{3.781238in}{1.781123in}}%
\pgfpathlineto{\pgfqpoint{3.812105in}{1.820033in}}%
\pgfpathlineto{\pgfqpoint{3.842973in}{1.853119in}}%
\pgfpathlineto{\pgfqpoint{3.873840in}{1.879044in}}%
\pgfpathlineto{\pgfqpoint{3.889273in}{1.888994in}}%
\pgfpathlineto{\pgfqpoint{3.904707in}{1.896816in}}%
\pgfpathlineto{\pgfqpoint{3.920141in}{1.902457in}}%
\pgfpathlineto{\pgfqpoint{3.935574in}{1.905901in}}%
\pgfpathlineto{\pgfqpoint{3.951008in}{1.907165in}}%
\pgfpathlineto{\pgfqpoint{3.966442in}{1.906305in}}%
\pgfpathlineto{\pgfqpoint{3.981875in}{1.903413in}}%
\pgfpathlineto{\pgfqpoint{3.997309in}{1.898621in}}%
\pgfpathlineto{\pgfqpoint{4.028176in}{1.884038in}}%
\pgfpathlineto{\pgfqpoint{4.059043in}{1.864294in}}%
\pgfpathlineto{\pgfqpoint{4.151645in}{1.797516in}}%
\pgfpathlineto{\pgfqpoint{4.182513in}{1.781451in}}%
\pgfpathlineto{\pgfqpoint{4.197946in}{1.775997in}}%
\pgfpathlineto{\pgfqpoint{4.213380in}{1.772619in}}%
\pgfpathlineto{\pgfqpoint{4.228813in}{1.771542in}}%
\pgfpathlineto{\pgfqpoint{4.244247in}{1.772954in}}%
\pgfpathlineto{\pgfqpoint{4.259681in}{1.776999in}}%
\pgfpathlineto{\pgfqpoint{4.275114in}{1.783775in}}%
\pgfpathlineto{\pgfqpoint{4.290548in}{1.793330in}}%
\pgfpathlineto{\pgfqpoint{4.305982in}{1.805660in}}%
\pgfpathlineto{\pgfqpoint{4.321415in}{1.820710in}}%
\pgfpathlineto{\pgfqpoint{4.336849in}{1.838370in}}%
\pgfpathlineto{\pgfqpoint{4.367716in}{1.880839in}}%
\pgfpathlineto{\pgfqpoint{4.398583in}{1.931228in}}%
\pgfpathlineto{\pgfqpoint{4.444884in}{2.016071in}}%
\pgfpathlineto{\pgfqpoint{4.506619in}{2.130223in}}%
\pgfpathlineto{\pgfqpoint{4.537486in}{2.180481in}}%
\pgfpathlineto{\pgfqpoint{4.568353in}{2.222302in}}%
\pgfpathlineto{\pgfqpoint{4.583787in}{2.239310in}}%
\pgfpathlineto{\pgfqpoint{4.599221in}{2.253412in}}%
\pgfpathlineto{\pgfqpoint{4.614654in}{2.264437in}}%
\pgfpathlineto{\pgfqpoint{4.630088in}{2.272272in}}%
\pgfpathlineto{\pgfqpoint{4.645522in}{2.276858in}}%
\pgfpathlineto{\pgfqpoint{4.660955in}{2.278192in}}%
\pgfpathlineto{\pgfqpoint{4.660955in}{2.278192in}}%
\pgfusepath{stroke}%
\end{pgfscope}%
\begin{pgfscope}%
\pgfpathrectangle{\pgfqpoint{0.625831in}{0.505056in}}{\pgfqpoint{4.227273in}{2.745455in}} %
\pgfusepath{clip}%
\pgfsetrectcap%
\pgfsetroundjoin%
\pgfsetlinewidth{0.501875pt}%
\definecolor{currentstroke}{rgb}{0.300000,0.951057,0.809017}%
\pgfsetstrokecolor{currentstroke}%
\pgfsetdash{}{0pt}%
\pgfpathmoveto{\pgfqpoint{0.817980in}{2.307411in}}%
\pgfpathlineto{\pgfqpoint{0.833414in}{2.307919in}}%
\pgfpathlineto{\pgfqpoint{0.848847in}{2.303667in}}%
\pgfpathlineto{\pgfqpoint{0.864281in}{2.294587in}}%
\pgfpathlineto{\pgfqpoint{0.879715in}{2.280681in}}%
\pgfpathlineto{\pgfqpoint{0.895148in}{2.262022in}}%
\pgfpathlineto{\pgfqpoint{0.910582in}{2.238755in}}%
\pgfpathlineto{\pgfqpoint{0.926015in}{2.211093in}}%
\pgfpathlineto{\pgfqpoint{0.941449in}{2.179323in}}%
\pgfpathlineto{\pgfqpoint{0.972316in}{2.104923in}}%
\pgfpathlineto{\pgfqpoint{1.003184in}{2.019078in}}%
\pgfpathlineto{\pgfqpoint{1.064918in}{1.830972in}}%
\pgfpathlineto{\pgfqpoint{1.111219in}{1.695488in}}%
\pgfpathlineto{\pgfqpoint{1.142086in}{1.617442in}}%
\pgfpathlineto{\pgfqpoint{1.157520in}{1.583868in}}%
\pgfpathlineto{\pgfqpoint{1.172954in}{1.554614in}}%
\pgfpathlineto{\pgfqpoint{1.188387in}{1.530127in}}%
\pgfpathlineto{\pgfqpoint{1.203821in}{1.510788in}}%
\pgfpathlineto{\pgfqpoint{1.219255in}{1.496910in}}%
\pgfpathlineto{\pgfqpoint{1.234688in}{1.488730in}}%
\pgfpathlineto{\pgfqpoint{1.250122in}{1.486404in}}%
\pgfpathlineto{\pgfqpoint{1.265555in}{1.490006in}}%
\pgfpathlineto{\pgfqpoint{1.280989in}{1.499527in}}%
\pgfpathlineto{\pgfqpoint{1.296423in}{1.514873in}}%
\pgfpathlineto{\pgfqpoint{1.311856in}{1.535868in}}%
\pgfpathlineto{\pgfqpoint{1.327290in}{1.562254in}}%
\pgfpathlineto{\pgfqpoint{1.342724in}{1.593697in}}%
\pgfpathlineto{\pgfqpoint{1.358157in}{1.629793in}}%
\pgfpathlineto{\pgfqpoint{1.389025in}{1.713994in}}%
\pgfpathlineto{\pgfqpoint{1.419892in}{1.810430in}}%
\pgfpathlineto{\pgfqpoint{1.527927in}{2.168485in}}%
\pgfpathlineto{\pgfqpoint{1.558795in}{2.254785in}}%
\pgfpathlineto{\pgfqpoint{1.574228in}{2.292342in}}%
\pgfpathlineto{\pgfqpoint{1.589662in}{2.325567in}}%
\pgfpathlineto{\pgfqpoint{1.605095in}{2.354092in}}%
\pgfpathlineto{\pgfqpoint{1.620529in}{2.377620in}}%
\pgfpathlineto{\pgfqpoint{1.635963in}{2.395926in}}%
\pgfpathlineto{\pgfqpoint{1.651396in}{2.408865in}}%
\pgfpathlineto{\pgfqpoint{1.666830in}{2.416370in}}%
\pgfpathlineto{\pgfqpoint{1.682264in}{2.418450in}}%
\pgfpathlineto{\pgfqpoint{1.697697in}{2.415193in}}%
\pgfpathlineto{\pgfqpoint{1.713131in}{2.406757in}}%
\pgfpathlineto{\pgfqpoint{1.728564in}{2.393373in}}%
\pgfpathlineto{\pgfqpoint{1.743998in}{2.375333in}}%
\pgfpathlineto{\pgfqpoint{1.759432in}{2.352990in}}%
\pgfpathlineto{\pgfqpoint{1.774865in}{2.326749in}}%
\pgfpathlineto{\pgfqpoint{1.805733in}{2.264401in}}%
\pgfpathlineto{\pgfqpoint{1.836600in}{2.192255in}}%
\pgfpathlineto{\pgfqpoint{1.944635in}{1.923559in}}%
\pgfpathlineto{\pgfqpoint{1.975503in}{1.857924in}}%
\pgfpathlineto{\pgfqpoint{2.006370in}{1.802489in}}%
\pgfpathlineto{\pgfqpoint{2.021804in}{1.779108in}}%
\pgfpathlineto{\pgfqpoint{2.037237in}{1.758775in}}%
\pgfpathlineto{\pgfqpoint{2.052671in}{1.741536in}}%
\pgfpathlineto{\pgfqpoint{2.068104in}{1.727378in}}%
\pgfpathlineto{\pgfqpoint{2.083538in}{1.716235in}}%
\pgfpathlineto{\pgfqpoint{2.098972in}{1.707989in}}%
\pgfpathlineto{\pgfqpoint{2.114405in}{1.702473in}}%
\pgfpathlineto{\pgfqpoint{2.129839in}{1.699479in}}%
\pgfpathlineto{\pgfqpoint{2.145273in}{1.698762in}}%
\pgfpathlineto{\pgfqpoint{2.160706in}{1.700047in}}%
\pgfpathlineto{\pgfqpoint{2.191574in}{1.707395in}}%
\pgfpathlineto{\pgfqpoint{2.222441in}{1.718938in}}%
\pgfpathlineto{\pgfqpoint{2.284175in}{1.744141in}}%
\pgfpathlineto{\pgfqpoint{2.315043in}{1.753108in}}%
\pgfpathlineto{\pgfqpoint{2.345910in}{1.757229in}}%
\pgfpathlineto{\pgfqpoint{2.361344in}{1.757104in}}%
\pgfpathlineto{\pgfqpoint{2.376777in}{1.755405in}}%
\pgfpathlineto{\pgfqpoint{2.392211in}{1.752097in}}%
\pgfpathlineto{\pgfqpoint{2.423078in}{1.740721in}}%
\pgfpathlineto{\pgfqpoint{2.453945in}{1.723500in}}%
\pgfpathlineto{\pgfqpoint{2.484813in}{1.701523in}}%
\pgfpathlineto{\pgfqpoint{2.531114in}{1.663114in}}%
\pgfpathlineto{\pgfqpoint{2.592848in}{1.611801in}}%
\pgfpathlineto{\pgfqpoint{2.623715in}{1.590479in}}%
\pgfpathlineto{\pgfqpoint{2.654583in}{1.574281in}}%
\pgfpathlineto{\pgfqpoint{2.670016in}{1.568496in}}%
\pgfpathlineto{\pgfqpoint{2.685450in}{1.564403in}}%
\pgfpathlineto{\pgfqpoint{2.700884in}{1.562071in}}%
\pgfpathlineto{\pgfqpoint{2.716317in}{1.561533in}}%
\pgfpathlineto{\pgfqpoint{2.731751in}{1.562785in}}%
\pgfpathlineto{\pgfqpoint{2.747184in}{1.565787in}}%
\pgfpathlineto{\pgfqpoint{2.778052in}{1.576697in}}%
\pgfpathlineto{\pgfqpoint{2.808919in}{1.593241in}}%
\pgfpathlineto{\pgfqpoint{2.839786in}{1.613912in}}%
\pgfpathlineto{\pgfqpoint{2.932388in}{1.680820in}}%
\pgfpathlineto{\pgfqpoint{2.963255in}{1.697548in}}%
\pgfpathlineto{\pgfqpoint{2.978689in}{1.703718in}}%
\pgfpathlineto{\pgfqpoint{2.994123in}{1.708142in}}%
\pgfpathlineto{\pgfqpoint{3.009556in}{1.710639in}}%
\pgfpathlineto{\pgfqpoint{3.024990in}{1.711060in}}%
\pgfpathlineto{\pgfqpoint{3.040424in}{1.709288in}}%
\pgfpathlineto{\pgfqpoint{3.055857in}{1.705245in}}%
\pgfpathlineto{\pgfqpoint{3.071291in}{1.698887in}}%
\pgfpathlineto{\pgfqpoint{3.086724in}{1.690210in}}%
\pgfpathlineto{\pgfqpoint{3.102158in}{1.679250in}}%
\pgfpathlineto{\pgfqpoint{3.117592in}{1.666082in}}%
\pgfpathlineto{\pgfqpoint{3.148459in}{1.633612in}}%
\pgfpathlineto{\pgfqpoint{3.179326in}{1.594140in}}%
\pgfpathlineto{\pgfqpoint{3.210193in}{1.549528in}}%
\pgfpathlineto{\pgfqpoint{3.302795in}{1.408875in}}%
\pgfpathlineto{\pgfqpoint{3.333663in}{1.368557in}}%
\pgfpathlineto{\pgfqpoint{3.364530in}{1.335811in}}%
\pgfpathlineto{\pgfqpoint{3.379963in}{1.322975in}}%
\pgfpathlineto{\pgfqpoint{3.395397in}{1.312816in}}%
\pgfpathlineto{\pgfqpoint{3.410831in}{1.305533in}}%
\pgfpathlineto{\pgfqpoint{3.426264in}{1.301291in}}%
\pgfpathlineto{\pgfqpoint{3.441698in}{1.300218in}}%
\pgfpathlineto{\pgfqpoint{3.457132in}{1.302400in}}%
\pgfpathlineto{\pgfqpoint{3.472565in}{1.307885in}}%
\pgfpathlineto{\pgfqpoint{3.487999in}{1.316675in}}%
\pgfpathlineto{\pgfqpoint{3.503433in}{1.328733in}}%
\pgfpathlineto{\pgfqpoint{3.518866in}{1.343977in}}%
\pgfpathlineto{\pgfqpoint{3.534300in}{1.362285in}}%
\pgfpathlineto{\pgfqpoint{3.549733in}{1.383495in}}%
\pgfpathlineto{\pgfqpoint{3.580601in}{1.433787in}}%
\pgfpathlineto{\pgfqpoint{3.611468in}{1.492845in}}%
\pgfpathlineto{\pgfqpoint{3.657769in}{1.592376in}}%
\pgfpathlineto{\pgfqpoint{3.734937in}{1.763285in}}%
\pgfpathlineto{\pgfqpoint{3.765804in}{1.824761in}}%
\pgfpathlineto{\pgfqpoint{3.796672in}{1.878252in}}%
\pgfpathlineto{\pgfqpoint{3.827539in}{1.921581in}}%
\pgfpathlineto{\pgfqpoint{3.842973in}{1.938901in}}%
\pgfpathlineto{\pgfqpoint{3.858406in}{1.953116in}}%
\pgfpathlineto{\pgfqpoint{3.873840in}{1.964121in}}%
\pgfpathlineto{\pgfqpoint{3.889273in}{1.971854in}}%
\pgfpathlineto{\pgfqpoint{3.904707in}{1.976296in}}%
\pgfpathlineto{\pgfqpoint{3.920141in}{1.977475in}}%
\pgfpathlineto{\pgfqpoint{3.935574in}{1.975460in}}%
\pgfpathlineto{\pgfqpoint{3.951008in}{1.970365in}}%
\pgfpathlineto{\pgfqpoint{3.966442in}{1.962345in}}%
\pgfpathlineto{\pgfqpoint{3.981875in}{1.951593in}}%
\pgfpathlineto{\pgfqpoint{3.997309in}{1.938341in}}%
\pgfpathlineto{\pgfqpoint{4.028176in}{1.905425in}}%
\pgfpathlineto{\pgfqpoint{4.059043in}{1.866050in}}%
\pgfpathlineto{\pgfqpoint{4.167079in}{1.718992in}}%
\pgfpathlineto{\pgfqpoint{4.197946in}{1.686368in}}%
\pgfpathlineto{\pgfqpoint{4.213380in}{1.673249in}}%
\pgfpathlineto{\pgfqpoint{4.228813in}{1.662600in}}%
\pgfpathlineto{\pgfqpoint{4.244247in}{1.654633in}}%
\pgfpathlineto{\pgfqpoint{4.259681in}{1.649519in}}%
\pgfpathlineto{\pgfqpoint{4.275114in}{1.647391in}}%
\pgfpathlineto{\pgfqpoint{4.290548in}{1.648336in}}%
\pgfpathlineto{\pgfqpoint{4.305982in}{1.652398in}}%
\pgfpathlineto{\pgfqpoint{4.321415in}{1.659575in}}%
\pgfpathlineto{\pgfqpoint{4.336849in}{1.669821in}}%
\pgfpathlineto{\pgfqpoint{4.352283in}{1.683044in}}%
\pgfpathlineto{\pgfqpoint{4.367716in}{1.699110in}}%
\pgfpathlineto{\pgfqpoint{4.383150in}{1.717844in}}%
\pgfpathlineto{\pgfqpoint{4.414017in}{1.762428in}}%
\pgfpathlineto{\pgfqpoint{4.444884in}{1.814683in}}%
\pgfpathlineto{\pgfqpoint{4.491185in}{1.901792in}}%
\pgfpathlineto{\pgfqpoint{4.552920in}{2.019048in}}%
\pgfpathlineto{\pgfqpoint{4.583787in}{2.071637in}}%
\pgfpathlineto{\pgfqpoint{4.614654in}{2.116728in}}%
\pgfpathlineto{\pgfqpoint{4.630088in}{2.135789in}}%
\pgfpathlineto{\pgfqpoint{4.645522in}{2.152231in}}%
\pgfpathlineto{\pgfqpoint{4.660955in}{2.165880in}}%
\pgfpathlineto{\pgfqpoint{4.660955in}{2.165880in}}%
\pgfusepath{stroke}%
\end{pgfscope}%
\begin{pgfscope}%
\pgfpathrectangle{\pgfqpoint{0.625831in}{0.505056in}}{\pgfqpoint{4.227273in}{2.745455in}} %
\pgfusepath{clip}%
\pgfsetrectcap%
\pgfsetroundjoin%
\pgfsetlinewidth{0.501875pt}%
\definecolor{currentstroke}{rgb}{0.378431,0.981823,0.771298}%
\pgfsetstrokecolor{currentstroke}%
\pgfsetdash{}{0pt}%
\pgfpathmoveto{\pgfqpoint{0.817980in}{2.237916in}}%
\pgfpathlineto{\pgfqpoint{0.833414in}{2.236848in}}%
\pgfpathlineto{\pgfqpoint{0.848847in}{2.232110in}}%
\pgfpathlineto{\pgfqpoint{0.864281in}{2.223533in}}%
\pgfpathlineto{\pgfqpoint{0.879715in}{2.211008in}}%
\pgfpathlineto{\pgfqpoint{0.895148in}{2.194486in}}%
\pgfpathlineto{\pgfqpoint{0.910582in}{2.173989in}}%
\pgfpathlineto{\pgfqpoint{0.926015in}{2.149608in}}%
\pgfpathlineto{\pgfqpoint{0.941449in}{2.121506in}}%
\pgfpathlineto{\pgfqpoint{0.972316in}{2.055144in}}%
\pgfpathlineto{\pgfqpoint{1.003184in}{1.977594in}}%
\pgfpathlineto{\pgfqpoint{1.049485in}{1.848552in}}%
\pgfpathlineto{\pgfqpoint{1.095785in}{1.718176in}}%
\pgfpathlineto{\pgfqpoint{1.126653in}{1.638940in}}%
\pgfpathlineto{\pgfqpoint{1.157520in}{1.571614in}}%
\pgfpathlineto{\pgfqpoint{1.172954in}{1.543809in}}%
\pgfpathlineto{\pgfqpoint{1.188387in}{1.520553in}}%
\pgfpathlineto{\pgfqpoint{1.203821in}{1.502249in}}%
\pgfpathlineto{\pgfqpoint{1.219255in}{1.489227in}}%
\pgfpathlineto{\pgfqpoint{1.234688in}{1.481738in}}%
\pgfpathlineto{\pgfqpoint{1.250122in}{1.479950in}}%
\pgfpathlineto{\pgfqpoint{1.265555in}{1.483938in}}%
\pgfpathlineto{\pgfqpoint{1.280989in}{1.493690in}}%
\pgfpathlineto{\pgfqpoint{1.296423in}{1.509102in}}%
\pgfpathlineto{\pgfqpoint{1.311856in}{1.529978in}}%
\pgfpathlineto{\pgfqpoint{1.327290in}{1.556038in}}%
\pgfpathlineto{\pgfqpoint{1.342724in}{1.586917in}}%
\pgfpathlineto{\pgfqpoint{1.358157in}{1.622175in}}%
\pgfpathlineto{\pgfqpoint{1.389025in}{1.703731in}}%
\pgfpathlineto{\pgfqpoint{1.419892in}{1.795973in}}%
\pgfpathlineto{\pgfqpoint{1.512494in}{2.082402in}}%
\pgfpathlineto{\pgfqpoint{1.543361in}{2.163272in}}%
\pgfpathlineto{\pgfqpoint{1.558795in}{2.198373in}}%
\pgfpathlineto{\pgfqpoint{1.574228in}{2.229343in}}%
\pgfpathlineto{\pgfqpoint{1.589662in}{2.255840in}}%
\pgfpathlineto{\pgfqpoint{1.605095in}{2.277605in}}%
\pgfpathlineto{\pgfqpoint{1.620529in}{2.294463in}}%
\pgfpathlineto{\pgfqpoint{1.635963in}{2.306321in}}%
\pgfpathlineto{\pgfqpoint{1.651396in}{2.313171in}}%
\pgfpathlineto{\pgfqpoint{1.666830in}{2.315088in}}%
\pgfpathlineto{\pgfqpoint{1.682264in}{2.312226in}}%
\pgfpathlineto{\pgfqpoint{1.697697in}{2.304812in}}%
\pgfpathlineto{\pgfqpoint{1.713131in}{2.293141in}}%
\pgfpathlineto{\pgfqpoint{1.728564in}{2.277567in}}%
\pgfpathlineto{\pgfqpoint{1.743998in}{2.258498in}}%
\pgfpathlineto{\pgfqpoint{1.774865in}{2.211699in}}%
\pgfpathlineto{\pgfqpoint{1.805733in}{2.156657in}}%
\pgfpathlineto{\pgfqpoint{1.898334in}{1.981932in}}%
\pgfpathlineto{\pgfqpoint{1.929202in}{1.931882in}}%
\pgfpathlineto{\pgfqpoint{1.960069in}{1.889835in}}%
\pgfpathlineto{\pgfqpoint{1.990936in}{1.856761in}}%
\pgfpathlineto{\pgfqpoint{2.006370in}{1.843628in}}%
\pgfpathlineto{\pgfqpoint{2.021804in}{1.832674in}}%
\pgfpathlineto{\pgfqpoint{2.037237in}{1.823764in}}%
\pgfpathlineto{\pgfqpoint{2.068104in}{1.811326in}}%
\pgfpathlineto{\pgfqpoint{2.098972in}{1.804474in}}%
\pgfpathlineto{\pgfqpoint{2.145273in}{1.799793in}}%
\pgfpathlineto{\pgfqpoint{2.191574in}{1.795012in}}%
\pgfpathlineto{\pgfqpoint{2.222441in}{1.788520in}}%
\pgfpathlineto{\pgfqpoint{2.253308in}{1.777846in}}%
\pgfpathlineto{\pgfqpoint{2.284175in}{1.762414in}}%
\pgfpathlineto{\pgfqpoint{2.315043in}{1.742373in}}%
\pgfpathlineto{\pgfqpoint{2.345910in}{1.718565in}}%
\pgfpathlineto{\pgfqpoint{2.453945in}{1.629641in}}%
\pgfpathlineto{\pgfqpoint{2.484813in}{1.611018in}}%
\pgfpathlineto{\pgfqpoint{2.500246in}{1.604031in}}%
\pgfpathlineto{\pgfqpoint{2.515680in}{1.598829in}}%
\pgfpathlineto{\pgfqpoint{2.531114in}{1.595546in}}%
\pgfpathlineto{\pgfqpoint{2.546547in}{1.594271in}}%
\pgfpathlineto{\pgfqpoint{2.561981in}{1.595054in}}%
\pgfpathlineto{\pgfqpoint{2.577414in}{1.597897in}}%
\pgfpathlineto{\pgfqpoint{2.592848in}{1.602757in}}%
\pgfpathlineto{\pgfqpoint{2.608282in}{1.609545in}}%
\pgfpathlineto{\pgfqpoint{2.639149in}{1.628331in}}%
\pgfpathlineto{\pgfqpoint{2.670016in}{1.652717in}}%
\pgfpathlineto{\pgfqpoint{2.716317in}{1.695173in}}%
\pgfpathlineto{\pgfqpoint{2.762618in}{1.737266in}}%
\pgfpathlineto{\pgfqpoint{2.793485in}{1.760947in}}%
\pgfpathlineto{\pgfqpoint{2.824353in}{1.778543in}}%
\pgfpathlineto{\pgfqpoint{2.839786in}{1.784510in}}%
\pgfpathlineto{\pgfqpoint{2.855220in}{1.788371in}}%
\pgfpathlineto{\pgfqpoint{2.870654in}{1.790018in}}%
\pgfpathlineto{\pgfqpoint{2.886087in}{1.789386in}}%
\pgfpathlineto{\pgfqpoint{2.901521in}{1.786456in}}%
\pgfpathlineto{\pgfqpoint{2.916954in}{1.781257in}}%
\pgfpathlineto{\pgfqpoint{2.932388in}{1.773865in}}%
\pgfpathlineto{\pgfqpoint{2.947822in}{1.764398in}}%
\pgfpathlineto{\pgfqpoint{2.978689in}{1.739915in}}%
\pgfpathlineto{\pgfqpoint{3.009556in}{1.709500in}}%
\pgfpathlineto{\pgfqpoint{3.055857in}{1.657470in}}%
\pgfpathlineto{\pgfqpoint{3.102158in}{1.604858in}}%
\pgfpathlineto{\pgfqpoint{3.133025in}{1.573245in}}%
\pgfpathlineto{\pgfqpoint{3.163893in}{1.546596in}}%
\pgfpathlineto{\pgfqpoint{3.194760in}{1.526234in}}%
\pgfpathlineto{\pgfqpoint{3.210193in}{1.518659in}}%
\pgfpathlineto{\pgfqpoint{3.225627in}{1.512875in}}%
\pgfpathlineto{\pgfqpoint{3.241061in}{1.508871in}}%
\pgfpathlineto{\pgfqpoint{3.256494in}{1.506599in}}%
\pgfpathlineto{\pgfqpoint{3.287362in}{1.506879in}}%
\pgfpathlineto{\pgfqpoint{3.318229in}{1.512665in}}%
\pgfpathlineto{\pgfqpoint{3.349096in}{1.522524in}}%
\pgfpathlineto{\pgfqpoint{3.410831in}{1.547849in}}%
\pgfpathlineto{\pgfqpoint{3.457132in}{1.565589in}}%
\pgfpathlineto{\pgfqpoint{3.487999in}{1.574623in}}%
\pgfpathlineto{\pgfqpoint{3.518866in}{1.580786in}}%
\pgfpathlineto{\pgfqpoint{3.549733in}{1.584137in}}%
\pgfpathlineto{\pgfqpoint{3.596034in}{1.585329in}}%
\pgfpathlineto{\pgfqpoint{3.657769in}{1.586226in}}%
\pgfpathlineto{\pgfqpoint{3.688636in}{1.589867in}}%
\pgfpathlineto{\pgfqpoint{3.719503in}{1.597708in}}%
\pgfpathlineto{\pgfqpoint{3.750371in}{1.610999in}}%
\pgfpathlineto{\pgfqpoint{3.781238in}{1.630553in}}%
\pgfpathlineto{\pgfqpoint{3.812105in}{1.656616in}}%
\pgfpathlineto{\pgfqpoint{3.842973in}{1.688801in}}%
\pgfpathlineto{\pgfqpoint{3.873840in}{1.726081in}}%
\pgfpathlineto{\pgfqpoint{3.920141in}{1.787886in}}%
\pgfpathlineto{\pgfqpoint{3.966442in}{1.850141in}}%
\pgfpathlineto{\pgfqpoint{3.997309in}{1.887794in}}%
\pgfpathlineto{\pgfqpoint{4.028176in}{1.919571in}}%
\pgfpathlineto{\pgfqpoint{4.043610in}{1.932600in}}%
\pgfpathlineto{\pgfqpoint{4.059043in}{1.943431in}}%
\pgfpathlineto{\pgfqpoint{4.074477in}{1.951895in}}%
\pgfpathlineto{\pgfqpoint{4.089911in}{1.957865in}}%
\pgfpathlineto{\pgfqpoint{4.105344in}{1.961260in}}%
\pgfpathlineto{\pgfqpoint{4.120778in}{1.962050in}}%
\pgfpathlineto{\pgfqpoint{4.136212in}{1.960258in}}%
\pgfpathlineto{\pgfqpoint{4.151645in}{1.955956in}}%
\pgfpathlineto{\pgfqpoint{4.167079in}{1.949268in}}%
\pgfpathlineto{\pgfqpoint{4.182513in}{1.940368in}}%
\pgfpathlineto{\pgfqpoint{4.213380in}{1.916855in}}%
\pgfpathlineto{\pgfqpoint{4.244247in}{1.887659in}}%
\pgfpathlineto{\pgfqpoint{4.321415in}{1.808461in}}%
\pgfpathlineto{\pgfqpoint{4.352283in}{1.782261in}}%
\pgfpathlineto{\pgfqpoint{4.367716in}{1.771749in}}%
\pgfpathlineto{\pgfqpoint{4.383150in}{1.763352in}}%
\pgfpathlineto{\pgfqpoint{4.398583in}{1.757312in}}%
\pgfpathlineto{\pgfqpoint{4.414017in}{1.753824in}}%
\pgfpathlineto{\pgfqpoint{4.429451in}{1.753031in}}%
\pgfpathlineto{\pgfqpoint{4.444884in}{1.755020in}}%
\pgfpathlineto{\pgfqpoint{4.460318in}{1.759821in}}%
\pgfpathlineto{\pgfqpoint{4.475752in}{1.767408in}}%
\pgfpathlineto{\pgfqpoint{4.491185in}{1.777693in}}%
\pgfpathlineto{\pgfqpoint{4.506619in}{1.790535in}}%
\pgfpathlineto{\pgfqpoint{4.537486in}{1.823049in}}%
\pgfpathlineto{\pgfqpoint{4.568353in}{1.862800in}}%
\pgfpathlineto{\pgfqpoint{4.614654in}{1.929786in}}%
\pgfpathlineto{\pgfqpoint{4.660955in}{1.995815in}}%
\pgfpathlineto{\pgfqpoint{4.660955in}{1.995815in}}%
\pgfusepath{stroke}%
\end{pgfscope}%
\begin{pgfscope}%
\pgfpathrectangle{\pgfqpoint{0.625831in}{0.505056in}}{\pgfqpoint{4.227273in}{2.745455in}} %
\pgfusepath{clip}%
\pgfsetrectcap%
\pgfsetroundjoin%
\pgfsetlinewidth{0.501875pt}%
\definecolor{currentstroke}{rgb}{0.456863,0.997705,0.730653}%
\pgfsetstrokecolor{currentstroke}%
\pgfsetdash{}{0pt}%
\pgfpathmoveto{\pgfqpoint{0.817980in}{2.245366in}}%
\pgfpathlineto{\pgfqpoint{0.833414in}{2.242914in}}%
\pgfpathlineto{\pgfqpoint{0.848847in}{2.236189in}}%
\pgfpathlineto{\pgfqpoint{0.864281in}{2.225015in}}%
\pgfpathlineto{\pgfqpoint{0.879715in}{2.209293in}}%
\pgfpathlineto{\pgfqpoint{0.895148in}{2.189003in}}%
\pgfpathlineto{\pgfqpoint{0.910582in}{2.164213in}}%
\pgfpathlineto{\pgfqpoint{0.926015in}{2.135078in}}%
\pgfpathlineto{\pgfqpoint{0.941449in}{2.101840in}}%
\pgfpathlineto{\pgfqpoint{0.972316in}{2.024460in}}%
\pgfpathlineto{\pgfqpoint{1.003184in}{1.935674in}}%
\pgfpathlineto{\pgfqpoint{1.111219in}{1.608803in}}%
\pgfpathlineto{\pgfqpoint{1.142086in}{1.534223in}}%
\pgfpathlineto{\pgfqpoint{1.157520in}{1.503366in}}%
\pgfpathlineto{\pgfqpoint{1.172954in}{1.477486in}}%
\pgfpathlineto{\pgfqpoint{1.188387in}{1.457000in}}%
\pgfpathlineto{\pgfqpoint{1.203821in}{1.442236in}}%
\pgfpathlineto{\pgfqpoint{1.219255in}{1.433423in}}%
\pgfpathlineto{\pgfqpoint{1.234688in}{1.430693in}}%
\pgfpathlineto{\pgfqpoint{1.250122in}{1.434073in}}%
\pgfpathlineto{\pgfqpoint{1.265555in}{1.443491in}}%
\pgfpathlineto{\pgfqpoint{1.280989in}{1.458774in}}%
\pgfpathlineto{\pgfqpoint{1.296423in}{1.479653in}}%
\pgfpathlineto{\pgfqpoint{1.311856in}{1.505774in}}%
\pgfpathlineto{\pgfqpoint{1.327290in}{1.536700in}}%
\pgfpathlineto{\pgfqpoint{1.358157in}{1.610878in}}%
\pgfpathlineto{\pgfqpoint{1.389025in}{1.697488in}}%
\pgfpathlineto{\pgfqpoint{1.497060in}{2.022331in}}%
\pgfpathlineto{\pgfqpoint{1.527927in}{2.101088in}}%
\pgfpathlineto{\pgfqpoint{1.558795in}{2.166683in}}%
\pgfpathlineto{\pgfqpoint{1.574228in}{2.193831in}}%
\pgfpathlineto{\pgfqpoint{1.589662in}{2.216970in}}%
\pgfpathlineto{\pgfqpoint{1.605095in}{2.236003in}}%
\pgfpathlineto{\pgfqpoint{1.620529in}{2.250897in}}%
\pgfpathlineto{\pgfqpoint{1.635963in}{2.261679in}}%
\pgfpathlineto{\pgfqpoint{1.651396in}{2.268434in}}%
\pgfpathlineto{\pgfqpoint{1.666830in}{2.271295in}}%
\pgfpathlineto{\pgfqpoint{1.682264in}{2.270439in}}%
\pgfpathlineto{\pgfqpoint{1.697697in}{2.266081in}}%
\pgfpathlineto{\pgfqpoint{1.713131in}{2.258466in}}%
\pgfpathlineto{\pgfqpoint{1.728564in}{2.247862in}}%
\pgfpathlineto{\pgfqpoint{1.743998in}{2.234556in}}%
\pgfpathlineto{\pgfqpoint{1.774865in}{2.201033in}}%
\pgfpathlineto{\pgfqpoint{1.805733in}{2.160316in}}%
\pgfpathlineto{\pgfqpoint{1.852034in}{2.090889in}}%
\pgfpathlineto{\pgfqpoint{1.944635in}{1.947101in}}%
\pgfpathlineto{\pgfqpoint{1.975503in}{1.903660in}}%
\pgfpathlineto{\pgfqpoint{2.006370in}{1.864464in}}%
\pgfpathlineto{\pgfqpoint{2.037237in}{1.830337in}}%
\pgfpathlineto{\pgfqpoint{2.068104in}{1.801847in}}%
\pgfpathlineto{\pgfqpoint{2.098972in}{1.779298in}}%
\pgfpathlineto{\pgfqpoint{2.129839in}{1.762709in}}%
\pgfpathlineto{\pgfqpoint{2.160706in}{1.751793in}}%
\pgfpathlineto{\pgfqpoint{2.191574in}{1.745956in}}%
\pgfpathlineto{\pgfqpoint{2.222441in}{1.744307in}}%
\pgfpathlineto{\pgfqpoint{2.253308in}{1.745699in}}%
\pgfpathlineto{\pgfqpoint{2.361344in}{1.755052in}}%
\pgfpathlineto{\pgfqpoint{2.392211in}{1.753806in}}%
\pgfpathlineto{\pgfqpoint{2.423078in}{1.749260in}}%
\pgfpathlineto{\pgfqpoint{2.453945in}{1.741246in}}%
\pgfpathlineto{\pgfqpoint{2.484813in}{1.730105in}}%
\pgfpathlineto{\pgfqpoint{2.531114in}{1.709483in}}%
\pgfpathlineto{\pgfqpoint{2.592848in}{1.681973in}}%
\pgfpathlineto{\pgfqpoint{2.623715in}{1.671816in}}%
\pgfpathlineto{\pgfqpoint{2.654583in}{1.666006in}}%
\pgfpathlineto{\pgfqpoint{2.685450in}{1.665557in}}%
\pgfpathlineto{\pgfqpoint{2.716317in}{1.670923in}}%
\pgfpathlineto{\pgfqpoint{2.747184in}{1.681908in}}%
\pgfpathlineto{\pgfqpoint{2.778052in}{1.697637in}}%
\pgfpathlineto{\pgfqpoint{2.824353in}{1.726723in}}%
\pgfpathlineto{\pgfqpoint{2.870654in}{1.756158in}}%
\pgfpathlineto{\pgfqpoint{2.901521in}{1.772177in}}%
\pgfpathlineto{\pgfqpoint{2.932388in}{1.782862in}}%
\pgfpathlineto{\pgfqpoint{2.947822in}{1.785676in}}%
\pgfpathlineto{\pgfqpoint{2.963255in}{1.786600in}}%
\pgfpathlineto{\pgfqpoint{2.978689in}{1.785534in}}%
\pgfpathlineto{\pgfqpoint{2.994123in}{1.782428in}}%
\pgfpathlineto{\pgfqpoint{3.009556in}{1.777285in}}%
\pgfpathlineto{\pgfqpoint{3.024990in}{1.770155in}}%
\pgfpathlineto{\pgfqpoint{3.055857in}{1.750386in}}%
\pgfpathlineto{\pgfqpoint{3.086724in}{1.724471in}}%
\pgfpathlineto{\pgfqpoint{3.133025in}{1.678430in}}%
\pgfpathlineto{\pgfqpoint{3.194760in}{1.616095in}}%
\pgfpathlineto{\pgfqpoint{3.225627in}{1.589641in}}%
\pgfpathlineto{\pgfqpoint{3.256494in}{1.568650in}}%
\pgfpathlineto{\pgfqpoint{3.287362in}{1.554078in}}%
\pgfpathlineto{\pgfqpoint{3.302795in}{1.549298in}}%
\pgfpathlineto{\pgfqpoint{3.333663in}{1.544553in}}%
\pgfpathlineto{\pgfqpoint{3.364530in}{1.545496in}}%
\pgfpathlineto{\pgfqpoint{3.395397in}{1.550816in}}%
\pgfpathlineto{\pgfqpoint{3.441698in}{1.563473in}}%
\pgfpathlineto{\pgfqpoint{3.503433in}{1.580991in}}%
\pgfpathlineto{\pgfqpoint{3.549733in}{1.590273in}}%
\pgfpathlineto{\pgfqpoint{3.611468in}{1.597682in}}%
\pgfpathlineto{\pgfqpoint{3.657769in}{1.604084in}}%
\pgfpathlineto{\pgfqpoint{3.688636in}{1.611731in}}%
\pgfpathlineto{\pgfqpoint{3.719503in}{1.623992in}}%
\pgfpathlineto{\pgfqpoint{3.750371in}{1.642167in}}%
\pgfpathlineto{\pgfqpoint{3.781238in}{1.666994in}}%
\pgfpathlineto{\pgfqpoint{3.812105in}{1.698481in}}%
\pgfpathlineto{\pgfqpoint{3.842973in}{1.735828in}}%
\pgfpathlineto{\pgfqpoint{3.889273in}{1.799158in}}%
\pgfpathlineto{\pgfqpoint{3.951008in}{1.884179in}}%
\pgfpathlineto{\pgfqpoint{3.981875in}{1.920502in}}%
\pgfpathlineto{\pgfqpoint{3.997309in}{1.935913in}}%
\pgfpathlineto{\pgfqpoint{4.012743in}{1.949103in}}%
\pgfpathlineto{\pgfqpoint{4.028176in}{1.959833in}}%
\pgfpathlineto{\pgfqpoint{4.043610in}{1.967917in}}%
\pgfpathlineto{\pgfqpoint{4.059043in}{1.973225in}}%
\pgfpathlineto{\pgfqpoint{4.074477in}{1.975688in}}%
\pgfpathlineto{\pgfqpoint{4.089911in}{1.975300in}}%
\pgfpathlineto{\pgfqpoint{4.105344in}{1.972121in}}%
\pgfpathlineto{\pgfqpoint{4.120778in}{1.966274in}}%
\pgfpathlineto{\pgfqpoint{4.136212in}{1.957943in}}%
\pgfpathlineto{\pgfqpoint{4.151645in}{1.947370in}}%
\pgfpathlineto{\pgfqpoint{4.182513in}{1.920713in}}%
\pgfpathlineto{\pgfqpoint{4.228813in}{1.872516in}}%
\pgfpathlineto{\pgfqpoint{4.275114in}{1.824401in}}%
\pgfpathlineto{\pgfqpoint{4.305982in}{1.797797in}}%
\pgfpathlineto{\pgfqpoint{4.321415in}{1.787194in}}%
\pgfpathlineto{\pgfqpoint{4.336849in}{1.778762in}}%
\pgfpathlineto{\pgfqpoint{4.352283in}{1.772715in}}%
\pgfpathlineto{\pgfqpoint{4.367716in}{1.769210in}}%
\pgfpathlineto{\pgfqpoint{4.383150in}{1.768341in}}%
\pgfpathlineto{\pgfqpoint{4.398583in}{1.770141in}}%
\pgfpathlineto{\pgfqpoint{4.414017in}{1.774580in}}%
\pgfpathlineto{\pgfqpoint{4.429451in}{1.781567in}}%
\pgfpathlineto{\pgfqpoint{4.444884in}{1.790954in}}%
\pgfpathlineto{\pgfqpoint{4.460318in}{1.802540in}}%
\pgfpathlineto{\pgfqpoint{4.491185in}{1.831271in}}%
\pgfpathlineto{\pgfqpoint{4.522052in}{1.865323in}}%
\pgfpathlineto{\pgfqpoint{4.599221in}{1.954820in}}%
\pgfpathlineto{\pgfqpoint{4.630088in}{1.984898in}}%
\pgfpathlineto{\pgfqpoint{4.660955in}{2.008146in}}%
\pgfpathlineto{\pgfqpoint{4.660955in}{2.008146in}}%
\pgfusepath{stroke}%
\end{pgfscope}%
\begin{pgfscope}%
\pgfpathrectangle{\pgfqpoint{0.625831in}{0.505056in}}{\pgfqpoint{4.227273in}{2.745455in}} %
\pgfusepath{clip}%
\pgfsetrectcap%
\pgfsetroundjoin%
\pgfsetlinewidth{0.501875pt}%
\definecolor{currentstroke}{rgb}{0.543137,0.997705,0.682749}%
\pgfsetstrokecolor{currentstroke}%
\pgfsetdash{}{0pt}%
\pgfpathmoveto{\pgfqpoint{0.817980in}{2.242735in}}%
\pgfpathlineto{\pgfqpoint{0.833414in}{2.239414in}}%
\pgfpathlineto{\pgfqpoint{0.848847in}{2.231957in}}%
\pgfpathlineto{\pgfqpoint{0.864281in}{2.220309in}}%
\pgfpathlineto{\pgfqpoint{0.879715in}{2.204482in}}%
\pgfpathlineto{\pgfqpoint{0.895148in}{2.184559in}}%
\pgfpathlineto{\pgfqpoint{0.910582in}{2.160696in}}%
\pgfpathlineto{\pgfqpoint{0.926015in}{2.133118in}}%
\pgfpathlineto{\pgfqpoint{0.956883in}{2.068058in}}%
\pgfpathlineto{\pgfqpoint{0.987750in}{1.992492in}}%
\pgfpathlineto{\pgfqpoint{1.049485in}{1.826331in}}%
\pgfpathlineto{\pgfqpoint{1.080352in}{1.745181in}}%
\pgfpathlineto{\pgfqpoint{1.111219in}{1.671751in}}%
\pgfpathlineto{\pgfqpoint{1.142086in}{1.610442in}}%
\pgfpathlineto{\pgfqpoint{1.157520in}{1.585516in}}%
\pgfpathlineto{\pgfqpoint{1.172954in}{1.564924in}}%
\pgfpathlineto{\pgfqpoint{1.188387in}{1.548966in}}%
\pgfpathlineto{\pgfqpoint{1.203821in}{1.537867in}}%
\pgfpathlineto{\pgfqpoint{1.219255in}{1.531776in}}%
\pgfpathlineto{\pgfqpoint{1.234688in}{1.530760in}}%
\pgfpathlineto{\pgfqpoint{1.250122in}{1.534805in}}%
\pgfpathlineto{\pgfqpoint{1.265555in}{1.543820in}}%
\pgfpathlineto{\pgfqpoint{1.280989in}{1.557635in}}%
\pgfpathlineto{\pgfqpoint{1.296423in}{1.576010in}}%
\pgfpathlineto{\pgfqpoint{1.311856in}{1.598639in}}%
\pgfpathlineto{\pgfqpoint{1.327290in}{1.625154in}}%
\pgfpathlineto{\pgfqpoint{1.358157in}{1.688130in}}%
\pgfpathlineto{\pgfqpoint{1.389025in}{1.761140in}}%
\pgfpathlineto{\pgfqpoint{1.512494in}{2.070075in}}%
\pgfpathlineto{\pgfqpoint{1.543361in}{2.132603in}}%
\pgfpathlineto{\pgfqpoint{1.558795in}{2.159651in}}%
\pgfpathlineto{\pgfqpoint{1.574228in}{2.183577in}}%
\pgfpathlineto{\pgfqpoint{1.589662in}{2.204207in}}%
\pgfpathlineto{\pgfqpoint{1.605095in}{2.221416in}}%
\pgfpathlineto{\pgfqpoint{1.620529in}{2.235124in}}%
\pgfpathlineto{\pgfqpoint{1.635963in}{2.245292in}}%
\pgfpathlineto{\pgfqpoint{1.651396in}{2.251925in}}%
\pgfpathlineto{\pgfqpoint{1.666830in}{2.255067in}}%
\pgfpathlineto{\pgfqpoint{1.682264in}{2.254793in}}%
\pgfpathlineto{\pgfqpoint{1.697697in}{2.251216in}}%
\pgfpathlineto{\pgfqpoint{1.713131in}{2.244474in}}%
\pgfpathlineto{\pgfqpoint{1.728564in}{2.234733in}}%
\pgfpathlineto{\pgfqpoint{1.743998in}{2.222182in}}%
\pgfpathlineto{\pgfqpoint{1.759432in}{2.207031in}}%
\pgfpathlineto{\pgfqpoint{1.790299in}{2.169855in}}%
\pgfpathlineto{\pgfqpoint{1.821166in}{2.125203in}}%
\pgfpathlineto{\pgfqpoint{1.867467in}{2.049004in}}%
\pgfpathlineto{\pgfqpoint{1.944635in}{1.916630in}}%
\pgfpathlineto{\pgfqpoint{1.975503in}{1.868367in}}%
\pgfpathlineto{\pgfqpoint{2.006370in}{1.825784in}}%
\pgfpathlineto{\pgfqpoint{2.037237in}{1.790455in}}%
\pgfpathlineto{\pgfqpoint{2.052671in}{1.775880in}}%
\pgfpathlineto{\pgfqpoint{2.068104in}{1.763498in}}%
\pgfpathlineto{\pgfqpoint{2.083538in}{1.753363in}}%
\pgfpathlineto{\pgfqpoint{2.098972in}{1.745493in}}%
\pgfpathlineto{\pgfqpoint{2.114405in}{1.739866in}}%
\pgfpathlineto{\pgfqpoint{2.129839in}{1.736421in}}%
\pgfpathlineto{\pgfqpoint{2.145273in}{1.735059in}}%
\pgfpathlineto{\pgfqpoint{2.160706in}{1.735643in}}%
\pgfpathlineto{\pgfqpoint{2.176140in}{1.738000in}}%
\pgfpathlineto{\pgfqpoint{2.207007in}{1.747193in}}%
\pgfpathlineto{\pgfqpoint{2.237874in}{1.760692in}}%
\pgfpathlineto{\pgfqpoint{2.315043in}{1.798336in}}%
\pgfpathlineto{\pgfqpoint{2.345910in}{1.809000in}}%
\pgfpathlineto{\pgfqpoint{2.376777in}{1.814293in}}%
\pgfpathlineto{\pgfqpoint{2.392211in}{1.814512in}}%
\pgfpathlineto{\pgfqpoint{2.407644in}{1.812969in}}%
\pgfpathlineto{\pgfqpoint{2.423078in}{1.809610in}}%
\pgfpathlineto{\pgfqpoint{2.438512in}{1.804423in}}%
\pgfpathlineto{\pgfqpoint{2.453945in}{1.797439in}}%
\pgfpathlineto{\pgfqpoint{2.484813in}{1.778417in}}%
\pgfpathlineto{\pgfqpoint{2.515680in}{1.753601in}}%
\pgfpathlineto{\pgfqpoint{2.546547in}{1.724581in}}%
\pgfpathlineto{\pgfqpoint{2.639149in}{1.632793in}}%
\pgfpathlineto{\pgfqpoint{2.670016in}{1.607649in}}%
\pgfpathlineto{\pgfqpoint{2.700884in}{1.588182in}}%
\pgfpathlineto{\pgfqpoint{2.716317in}{1.580945in}}%
\pgfpathlineto{\pgfqpoint{2.731751in}{1.575496in}}%
\pgfpathlineto{\pgfqpoint{2.747184in}{1.571879in}}%
\pgfpathlineto{\pgfqpoint{2.762618in}{1.570099in}}%
\pgfpathlineto{\pgfqpoint{2.778052in}{1.570122in}}%
\pgfpathlineto{\pgfqpoint{2.793485in}{1.571875in}}%
\pgfpathlineto{\pgfqpoint{2.824353in}{1.580093in}}%
\pgfpathlineto{\pgfqpoint{2.855220in}{1.593485in}}%
\pgfpathlineto{\pgfqpoint{2.901521in}{1.619484in}}%
\pgfpathlineto{\pgfqpoint{2.947822in}{1.646518in}}%
\pgfpathlineto{\pgfqpoint{2.978689in}{1.661703in}}%
\pgfpathlineto{\pgfqpoint{3.009556in}{1.672492in}}%
\pgfpathlineto{\pgfqpoint{3.024990in}{1.675779in}}%
\pgfpathlineto{\pgfqpoint{3.040424in}{1.677472in}}%
\pgfpathlineto{\pgfqpoint{3.055857in}{1.677472in}}%
\pgfpathlineto{\pgfqpoint{3.071291in}{1.675715in}}%
\pgfpathlineto{\pgfqpoint{3.086724in}{1.672171in}}%
\pgfpathlineto{\pgfqpoint{3.102158in}{1.666848in}}%
\pgfpathlineto{\pgfqpoint{3.133025in}{1.651072in}}%
\pgfpathlineto{\pgfqpoint{3.163893in}{1.629130in}}%
\pgfpathlineto{\pgfqpoint{3.194760in}{1.602233in}}%
\pgfpathlineto{\pgfqpoint{3.241061in}{1.556080in}}%
\pgfpathlineto{\pgfqpoint{3.302795in}{1.493208in}}%
\pgfpathlineto{\pgfqpoint{3.333663in}{1.465181in}}%
\pgfpathlineto{\pgfqpoint{3.364530in}{1.441437in}}%
\pgfpathlineto{\pgfqpoint{3.395397in}{1.423104in}}%
\pgfpathlineto{\pgfqpoint{3.426264in}{1.410965in}}%
\pgfpathlineto{\pgfqpoint{3.441698in}{1.407369in}}%
\pgfpathlineto{\pgfqpoint{3.457132in}{1.405462in}}%
\pgfpathlineto{\pgfqpoint{3.472565in}{1.405255in}}%
\pgfpathlineto{\pgfqpoint{3.487999in}{1.406742in}}%
\pgfpathlineto{\pgfqpoint{3.503433in}{1.409905in}}%
\pgfpathlineto{\pgfqpoint{3.534300in}{1.421144in}}%
\pgfpathlineto{\pgfqpoint{3.565167in}{1.438673in}}%
\pgfpathlineto{\pgfqpoint{3.596034in}{1.462124in}}%
\pgfpathlineto{\pgfqpoint{3.626902in}{1.491093in}}%
\pgfpathlineto{\pgfqpoint{3.657769in}{1.525144in}}%
\pgfpathlineto{\pgfqpoint{3.688636in}{1.563779in}}%
\pgfpathlineto{\pgfqpoint{3.719503in}{1.606384in}}%
\pgfpathlineto{\pgfqpoint{3.765804in}{1.675940in}}%
\pgfpathlineto{\pgfqpoint{3.873840in}{1.842264in}}%
\pgfpathlineto{\pgfqpoint{3.904707in}{1.883097in}}%
\pgfpathlineto{\pgfqpoint{3.935574in}{1.917384in}}%
\pgfpathlineto{\pgfqpoint{3.951008in}{1.931501in}}%
\pgfpathlineto{\pgfqpoint{3.966442in}{1.943332in}}%
\pgfpathlineto{\pgfqpoint{3.981875in}{1.952715in}}%
\pgfpathlineto{\pgfqpoint{3.997309in}{1.959520in}}%
\pgfpathlineto{\pgfqpoint{4.012743in}{1.963659in}}%
\pgfpathlineto{\pgfqpoint{4.028176in}{1.965086in}}%
\pgfpathlineto{\pgfqpoint{4.043610in}{1.963807in}}%
\pgfpathlineto{\pgfqpoint{4.059043in}{1.959876in}}%
\pgfpathlineto{\pgfqpoint{4.074477in}{1.953400in}}%
\pgfpathlineto{\pgfqpoint{4.089911in}{1.944537in}}%
\pgfpathlineto{\pgfqpoint{4.105344in}{1.933494in}}%
\pgfpathlineto{\pgfqpoint{4.136212in}{1.905940in}}%
\pgfpathlineto{\pgfqpoint{4.167079in}{1.873270in}}%
\pgfpathlineto{\pgfqpoint{4.228813in}{1.804910in}}%
\pgfpathlineto{\pgfqpoint{4.259681in}{1.775697in}}%
\pgfpathlineto{\pgfqpoint{4.275114in}{1.763652in}}%
\pgfpathlineto{\pgfqpoint{4.290548in}{1.753734in}}%
\pgfpathlineto{\pgfqpoint{4.305982in}{1.746206in}}%
\pgfpathlineto{\pgfqpoint{4.321415in}{1.741275in}}%
\pgfpathlineto{\pgfqpoint{4.336849in}{1.739091in}}%
\pgfpathlineto{\pgfqpoint{4.352283in}{1.739744in}}%
\pgfpathlineto{\pgfqpoint{4.367716in}{1.743260in}}%
\pgfpathlineto{\pgfqpoint{4.383150in}{1.749602in}}%
\pgfpathlineto{\pgfqpoint{4.398583in}{1.758671in}}%
\pgfpathlineto{\pgfqpoint{4.414017in}{1.770306in}}%
\pgfpathlineto{\pgfqpoint{4.429451in}{1.784291in}}%
\pgfpathlineto{\pgfqpoint{4.460318in}{1.818204in}}%
\pgfpathlineto{\pgfqpoint{4.491185in}{1.857795in}}%
\pgfpathlineto{\pgfqpoint{4.568353in}{1.961183in}}%
\pgfpathlineto{\pgfqpoint{4.599221in}{1.996178in}}%
\pgfpathlineto{\pgfqpoint{4.630088in}{2.023654in}}%
\pgfpathlineto{\pgfqpoint{4.645522in}{2.034011in}}%
\pgfpathlineto{\pgfqpoint{4.660955in}{2.041915in}}%
\pgfpathlineto{\pgfqpoint{4.660955in}{2.041915in}}%
\pgfusepath{stroke}%
\end{pgfscope}%
\begin{pgfscope}%
\pgfpathrectangle{\pgfqpoint{0.625831in}{0.505056in}}{\pgfqpoint{4.227273in}{2.745455in}} %
\pgfusepath{clip}%
\pgfsetrectcap%
\pgfsetroundjoin%
\pgfsetlinewidth{0.501875pt}%
\definecolor{currentstroke}{rgb}{0.621569,0.981823,0.636474}%
\pgfsetstrokecolor{currentstroke}%
\pgfsetdash{}{0pt}%
\pgfpathmoveto{\pgfqpoint{0.817980in}{2.265081in}}%
\pgfpathlineto{\pgfqpoint{0.833414in}{2.268901in}}%
\pgfpathlineto{\pgfqpoint{0.848847in}{2.268689in}}%
\pgfpathlineto{\pgfqpoint{0.864281in}{2.264236in}}%
\pgfpathlineto{\pgfqpoint{0.879715in}{2.255405in}}%
\pgfpathlineto{\pgfqpoint{0.895148in}{2.242138in}}%
\pgfpathlineto{\pgfqpoint{0.910582in}{2.224464in}}%
\pgfpathlineto{\pgfqpoint{0.926015in}{2.202499in}}%
\pgfpathlineto{\pgfqpoint{0.941449in}{2.176445in}}%
\pgfpathlineto{\pgfqpoint{0.956883in}{2.146594in}}%
\pgfpathlineto{\pgfqpoint{0.987750in}{2.077072in}}%
\pgfpathlineto{\pgfqpoint{1.018617in}{1.997805in}}%
\pgfpathlineto{\pgfqpoint{1.095785in}{1.789987in}}%
\pgfpathlineto{\pgfqpoint{1.126653in}{1.717299in}}%
\pgfpathlineto{\pgfqpoint{1.142086in}{1.685767in}}%
\pgfpathlineto{\pgfqpoint{1.157520in}{1.658163in}}%
\pgfpathlineto{\pgfqpoint{1.172954in}{1.634942in}}%
\pgfpathlineto{\pgfqpoint{1.188387in}{1.616477in}}%
\pgfpathlineto{\pgfqpoint{1.203821in}{1.603051in}}%
\pgfpathlineto{\pgfqpoint{1.219255in}{1.594855in}}%
\pgfpathlineto{\pgfqpoint{1.234688in}{1.591986in}}%
\pgfpathlineto{\pgfqpoint{1.250122in}{1.594440in}}%
\pgfpathlineto{\pgfqpoint{1.265555in}{1.602124in}}%
\pgfpathlineto{\pgfqpoint{1.280989in}{1.614849in}}%
\pgfpathlineto{\pgfqpoint{1.296423in}{1.632345in}}%
\pgfpathlineto{\pgfqpoint{1.311856in}{1.654263in}}%
\pgfpathlineto{\pgfqpoint{1.327290in}{1.680186in}}%
\pgfpathlineto{\pgfqpoint{1.358157in}{1.742109in}}%
\pgfpathlineto{\pgfqpoint{1.389025in}{1.813840in}}%
\pgfpathlineto{\pgfqpoint{1.481626in}{2.041996in}}%
\pgfpathlineto{\pgfqpoint{1.512494in}{2.108379in}}%
\pgfpathlineto{\pgfqpoint{1.543361in}{2.164464in}}%
\pgfpathlineto{\pgfqpoint{1.558795in}{2.187964in}}%
\pgfpathlineto{\pgfqpoint{1.574228in}{2.208176in}}%
\pgfpathlineto{\pgfqpoint{1.589662in}{2.224974in}}%
\pgfpathlineto{\pgfqpoint{1.605095in}{2.238275in}}%
\pgfpathlineto{\pgfqpoint{1.620529in}{2.248045in}}%
\pgfpathlineto{\pgfqpoint{1.635963in}{2.254288in}}%
\pgfpathlineto{\pgfqpoint{1.651396in}{2.257050in}}%
\pgfpathlineto{\pgfqpoint{1.666830in}{2.256406in}}%
\pgfpathlineto{\pgfqpoint{1.682264in}{2.252465in}}%
\pgfpathlineto{\pgfqpoint{1.697697in}{2.245360in}}%
\pgfpathlineto{\pgfqpoint{1.713131in}{2.235249in}}%
\pgfpathlineto{\pgfqpoint{1.728564in}{2.222310in}}%
\pgfpathlineto{\pgfqpoint{1.743998in}{2.206739in}}%
\pgfpathlineto{\pgfqpoint{1.774865in}{2.168570in}}%
\pgfpathlineto{\pgfqpoint{1.805733in}{2.122618in}}%
\pgfpathlineto{\pgfqpoint{1.836600in}{2.070963in}}%
\pgfpathlineto{\pgfqpoint{1.960069in}{1.853875in}}%
\pgfpathlineto{\pgfqpoint{1.990936in}{1.808948in}}%
\pgfpathlineto{\pgfqpoint{2.021804in}{1.772069in}}%
\pgfpathlineto{\pgfqpoint{2.037237in}{1.757142in}}%
\pgfpathlineto{\pgfqpoint{2.052671in}{1.744750in}}%
\pgfpathlineto{\pgfqpoint{2.068104in}{1.734986in}}%
\pgfpathlineto{\pgfqpoint{2.083538in}{1.727897in}}%
\pgfpathlineto{\pgfqpoint{2.098972in}{1.723481in}}%
\pgfpathlineto{\pgfqpoint{2.114405in}{1.721685in}}%
\pgfpathlineto{\pgfqpoint{2.129839in}{1.722409in}}%
\pgfpathlineto{\pgfqpoint{2.145273in}{1.725498in}}%
\pgfpathlineto{\pgfqpoint{2.160706in}{1.730751in}}%
\pgfpathlineto{\pgfqpoint{2.191574in}{1.746726in}}%
\pgfpathlineto{\pgfqpoint{2.222441in}{1.767904in}}%
\pgfpathlineto{\pgfqpoint{2.284175in}{1.814094in}}%
\pgfpathlineto{\pgfqpoint{2.315043in}{1.832941in}}%
\pgfpathlineto{\pgfqpoint{2.330476in}{1.840047in}}%
\pgfpathlineto{\pgfqpoint{2.345910in}{1.845207in}}%
\pgfpathlineto{\pgfqpoint{2.361344in}{1.848172in}}%
\pgfpathlineto{\pgfqpoint{2.376777in}{1.848742in}}%
\pgfpathlineto{\pgfqpoint{2.392211in}{1.846772in}}%
\pgfpathlineto{\pgfqpoint{2.407644in}{1.842172in}}%
\pgfpathlineto{\pgfqpoint{2.423078in}{1.834912in}}%
\pgfpathlineto{\pgfqpoint{2.438512in}{1.825023in}}%
\pgfpathlineto{\pgfqpoint{2.453945in}{1.812592in}}%
\pgfpathlineto{\pgfqpoint{2.469379in}{1.797766in}}%
\pgfpathlineto{\pgfqpoint{2.500246in}{1.761773in}}%
\pgfpathlineto{\pgfqpoint{2.531114in}{1.719187in}}%
\pgfpathlineto{\pgfqpoint{2.639149in}{1.559867in}}%
\pgfpathlineto{\pgfqpoint{2.670016in}{1.523804in}}%
\pgfpathlineto{\pgfqpoint{2.685450in}{1.508960in}}%
\pgfpathlineto{\pgfqpoint{2.700884in}{1.496537in}}%
\pgfpathlineto{\pgfqpoint{2.716317in}{1.486698in}}%
\pgfpathlineto{\pgfqpoint{2.731751in}{1.479553in}}%
\pgfpathlineto{\pgfqpoint{2.747184in}{1.475155in}}%
\pgfpathlineto{\pgfqpoint{2.762618in}{1.473505in}}%
\pgfpathlineto{\pgfqpoint{2.778052in}{1.474547in}}%
\pgfpathlineto{\pgfqpoint{2.793485in}{1.478170in}}%
\pgfpathlineto{\pgfqpoint{2.808919in}{1.484215in}}%
\pgfpathlineto{\pgfqpoint{2.824353in}{1.492471in}}%
\pgfpathlineto{\pgfqpoint{2.855220in}{1.514571in}}%
\pgfpathlineto{\pgfqpoint{2.886087in}{1.542023in}}%
\pgfpathlineto{\pgfqpoint{2.963255in}{1.614966in}}%
\pgfpathlineto{\pgfqpoint{2.994123in}{1.638125in}}%
\pgfpathlineto{\pgfqpoint{3.009556in}{1.647105in}}%
\pgfpathlineto{\pgfqpoint{3.024990in}{1.654011in}}%
\pgfpathlineto{\pgfqpoint{3.040424in}{1.658642in}}%
\pgfpathlineto{\pgfqpoint{3.055857in}{1.660846in}}%
\pgfpathlineto{\pgfqpoint{3.071291in}{1.660523in}}%
\pgfpathlineto{\pgfqpoint{3.086724in}{1.657627in}}%
\pgfpathlineto{\pgfqpoint{3.102158in}{1.652166in}}%
\pgfpathlineto{\pgfqpoint{3.117592in}{1.644204in}}%
\pgfpathlineto{\pgfqpoint{3.133025in}{1.633857in}}%
\pgfpathlineto{\pgfqpoint{3.148459in}{1.621290in}}%
\pgfpathlineto{\pgfqpoint{3.179326in}{1.590388in}}%
\pgfpathlineto{\pgfqpoint{3.210193in}{1.553648in}}%
\pgfpathlineto{\pgfqpoint{3.318229in}{1.417835in}}%
\pgfpathlineto{\pgfqpoint{3.349096in}{1.387814in}}%
\pgfpathlineto{\pgfqpoint{3.364530in}{1.375587in}}%
\pgfpathlineto{\pgfqpoint{3.379963in}{1.365438in}}%
\pgfpathlineto{\pgfqpoint{3.395397in}{1.357482in}}%
\pgfpathlineto{\pgfqpoint{3.410831in}{1.351789in}}%
\pgfpathlineto{\pgfqpoint{3.426264in}{1.348390in}}%
\pgfpathlineto{\pgfqpoint{3.441698in}{1.347276in}}%
\pgfpathlineto{\pgfqpoint{3.457132in}{1.348401in}}%
\pgfpathlineto{\pgfqpoint{3.472565in}{1.351688in}}%
\pgfpathlineto{\pgfqpoint{3.487999in}{1.357032in}}%
\pgfpathlineto{\pgfqpoint{3.503433in}{1.364307in}}%
\pgfpathlineto{\pgfqpoint{3.534300in}{1.384065in}}%
\pgfpathlineto{\pgfqpoint{3.565167in}{1.409722in}}%
\pgfpathlineto{\pgfqpoint{3.596034in}{1.440022in}}%
\pgfpathlineto{\pgfqpoint{3.642335in}{1.491750in}}%
\pgfpathlineto{\pgfqpoint{3.688636in}{1.548426in}}%
\pgfpathlineto{\pgfqpoint{3.765804in}{1.648504in}}%
\pgfpathlineto{\pgfqpoint{3.842973in}{1.748016in}}%
\pgfpathlineto{\pgfqpoint{3.889273in}{1.801733in}}%
\pgfpathlineto{\pgfqpoint{3.920141in}{1.832384in}}%
\pgfpathlineto{\pgfqpoint{3.951008in}{1.857242in}}%
\pgfpathlineto{\pgfqpoint{3.966442in}{1.867087in}}%
\pgfpathlineto{\pgfqpoint{3.981875in}{1.875023in}}%
\pgfpathlineto{\pgfqpoint{3.997309in}{1.880945in}}%
\pgfpathlineto{\pgfqpoint{4.012743in}{1.884778in}}%
\pgfpathlineto{\pgfqpoint{4.028176in}{1.886488in}}%
\pgfpathlineto{\pgfqpoint{4.043610in}{1.886081in}}%
\pgfpathlineto{\pgfqpoint{4.059043in}{1.883612in}}%
\pgfpathlineto{\pgfqpoint{4.074477in}{1.879178in}}%
\pgfpathlineto{\pgfqpoint{4.105344in}{1.865047in}}%
\pgfpathlineto{\pgfqpoint{4.136212in}{1.845386in}}%
\pgfpathlineto{\pgfqpoint{4.228813in}{1.777948in}}%
\pgfpathlineto{\pgfqpoint{4.259681in}{1.761771in}}%
\pgfpathlineto{\pgfqpoint{4.275114in}{1.756246in}}%
\pgfpathlineto{\pgfqpoint{4.290548in}{1.752726in}}%
\pgfpathlineto{\pgfqpoint{4.305982in}{1.751370in}}%
\pgfpathlineto{\pgfqpoint{4.321415in}{1.752278in}}%
\pgfpathlineto{\pgfqpoint{4.336849in}{1.755492in}}%
\pgfpathlineto{\pgfqpoint{4.352283in}{1.760994in}}%
\pgfpathlineto{\pgfqpoint{4.367716in}{1.768704in}}%
\pgfpathlineto{\pgfqpoint{4.383150in}{1.778485in}}%
\pgfpathlineto{\pgfqpoint{4.414017in}{1.803440in}}%
\pgfpathlineto{\pgfqpoint{4.444884in}{1.833789in}}%
\pgfpathlineto{\pgfqpoint{4.537486in}{1.930469in}}%
\pgfpathlineto{\pgfqpoint{4.568353in}{1.955805in}}%
\pgfpathlineto{\pgfqpoint{4.599221in}{1.974386in}}%
\pgfpathlineto{\pgfqpoint{4.614654in}{1.980827in}}%
\pgfpathlineto{\pgfqpoint{4.630088in}{1.985305in}}%
\pgfpathlineto{\pgfqpoint{4.645522in}{1.987840in}}%
\pgfpathlineto{\pgfqpoint{4.660955in}{1.988501in}}%
\pgfpathlineto{\pgfqpoint{4.660955in}{1.988501in}}%
\pgfusepath{stroke}%
\end{pgfscope}%
\begin{pgfscope}%
\pgfpathrectangle{\pgfqpoint{0.625831in}{0.505056in}}{\pgfqpoint{4.227273in}{2.745455in}} %
\pgfusepath{clip}%
\pgfsetrectcap%
\pgfsetroundjoin%
\pgfsetlinewidth{0.501875pt}%
\definecolor{currentstroke}{rgb}{0.700000,0.951057,0.587785}%
\pgfsetstrokecolor{currentstroke}%
\pgfsetdash{}{0pt}%
\pgfpathmoveto{\pgfqpoint{0.817980in}{2.261494in}}%
\pgfpathlineto{\pgfqpoint{0.833414in}{2.265816in}}%
\pgfpathlineto{\pgfqpoint{0.848847in}{2.266500in}}%
\pgfpathlineto{\pgfqpoint{0.864281in}{2.263274in}}%
\pgfpathlineto{\pgfqpoint{0.879715in}{2.255936in}}%
\pgfpathlineto{\pgfqpoint{0.895148in}{2.244363in}}%
\pgfpathlineto{\pgfqpoint{0.910582in}{2.228517in}}%
\pgfpathlineto{\pgfqpoint{0.926015in}{2.208451in}}%
\pgfpathlineto{\pgfqpoint{0.941449in}{2.184311in}}%
\pgfpathlineto{\pgfqpoint{0.956883in}{2.156333in}}%
\pgfpathlineto{\pgfqpoint{0.987750in}{2.090261in}}%
\pgfpathlineto{\pgfqpoint{1.018617in}{2.013822in}}%
\pgfpathlineto{\pgfqpoint{1.111219in}{1.771902in}}%
\pgfpathlineto{\pgfqpoint{1.142086in}{1.705173in}}%
\pgfpathlineto{\pgfqpoint{1.157520in}{1.677240in}}%
\pgfpathlineto{\pgfqpoint{1.172954in}{1.653587in}}%
\pgfpathlineto{\pgfqpoint{1.188387in}{1.634611in}}%
\pgfpathlineto{\pgfqpoint{1.203821in}{1.620620in}}%
\pgfpathlineto{\pgfqpoint{1.219255in}{1.611827in}}%
\pgfpathlineto{\pgfqpoint{1.234688in}{1.608349in}}%
\pgfpathlineto{\pgfqpoint{1.250122in}{1.610203in}}%
\pgfpathlineto{\pgfqpoint{1.265555in}{1.617309in}}%
\pgfpathlineto{\pgfqpoint{1.280989in}{1.629494in}}%
\pgfpathlineto{\pgfqpoint{1.296423in}{1.646495in}}%
\pgfpathlineto{\pgfqpoint{1.311856in}{1.667971in}}%
\pgfpathlineto{\pgfqpoint{1.327290in}{1.693509in}}%
\pgfpathlineto{\pgfqpoint{1.358157in}{1.754829in}}%
\pgfpathlineto{\pgfqpoint{1.389025in}{1.826154in}}%
\pgfpathlineto{\pgfqpoint{1.497060in}{2.087899in}}%
\pgfpathlineto{\pgfqpoint{1.527927in}{2.149288in}}%
\pgfpathlineto{\pgfqpoint{1.543361in}{2.175839in}}%
\pgfpathlineto{\pgfqpoint{1.558795in}{2.199318in}}%
\pgfpathlineto{\pgfqpoint{1.574228in}{2.219562in}}%
\pgfpathlineto{\pgfqpoint{1.589662in}{2.236463in}}%
\pgfpathlineto{\pgfqpoint{1.605095in}{2.249962in}}%
\pgfpathlineto{\pgfqpoint{1.620529in}{2.260050in}}%
\pgfpathlineto{\pgfqpoint{1.635963in}{2.266755in}}%
\pgfpathlineto{\pgfqpoint{1.651396in}{2.270144in}}%
\pgfpathlineto{\pgfqpoint{1.666830in}{2.270315in}}%
\pgfpathlineto{\pgfqpoint{1.682264in}{2.267392in}}%
\pgfpathlineto{\pgfqpoint{1.697697in}{2.261522in}}%
\pgfpathlineto{\pgfqpoint{1.713131in}{2.252865in}}%
\pgfpathlineto{\pgfqpoint{1.728564in}{2.241599in}}%
\pgfpathlineto{\pgfqpoint{1.743998in}{2.227907in}}%
\pgfpathlineto{\pgfqpoint{1.774865in}{2.194025in}}%
\pgfpathlineto{\pgfqpoint{1.805733in}{2.152815in}}%
\pgfpathlineto{\pgfqpoint{1.836600in}{2.105936in}}%
\pgfpathlineto{\pgfqpoint{1.882901in}{2.028760in}}%
\pgfpathlineto{\pgfqpoint{1.960069in}{1.897120in}}%
\pgfpathlineto{\pgfqpoint{1.990936in}{1.848973in}}%
\pgfpathlineto{\pgfqpoint{2.021804in}{1.806191in}}%
\pgfpathlineto{\pgfqpoint{2.052671in}{1.770377in}}%
\pgfpathlineto{\pgfqpoint{2.068104in}{1.755487in}}%
\pgfpathlineto{\pgfqpoint{2.083538in}{1.742765in}}%
\pgfpathlineto{\pgfqpoint{2.098972in}{1.732281in}}%
\pgfpathlineto{\pgfqpoint{2.114405in}{1.724067in}}%
\pgfpathlineto{\pgfqpoint{2.129839in}{1.718111in}}%
\pgfpathlineto{\pgfqpoint{2.145273in}{1.714356in}}%
\pgfpathlineto{\pgfqpoint{2.160706in}{1.712701in}}%
\pgfpathlineto{\pgfqpoint{2.176140in}{1.712996in}}%
\pgfpathlineto{\pgfqpoint{2.207007in}{1.718636in}}%
\pgfpathlineto{\pgfqpoint{2.237874in}{1.729292in}}%
\pgfpathlineto{\pgfqpoint{2.315043in}{1.761118in}}%
\pgfpathlineto{\pgfqpoint{2.345910in}{1.768852in}}%
\pgfpathlineto{\pgfqpoint{2.361344in}{1.770439in}}%
\pgfpathlineto{\pgfqpoint{2.376777in}{1.770200in}}%
\pgfpathlineto{\pgfqpoint{2.392211in}{1.767962in}}%
\pgfpathlineto{\pgfqpoint{2.407644in}{1.763602in}}%
\pgfpathlineto{\pgfqpoint{2.423078in}{1.757056in}}%
\pgfpathlineto{\pgfqpoint{2.438512in}{1.748315in}}%
\pgfpathlineto{\pgfqpoint{2.453945in}{1.737430in}}%
\pgfpathlineto{\pgfqpoint{2.484813in}{1.709714in}}%
\pgfpathlineto{\pgfqpoint{2.515680in}{1.675404in}}%
\pgfpathlineto{\pgfqpoint{2.561981in}{1.616586in}}%
\pgfpathlineto{\pgfqpoint{2.608282in}{1.557433in}}%
\pgfpathlineto{\pgfqpoint{2.639149in}{1.522740in}}%
\pgfpathlineto{\pgfqpoint{2.670016in}{1.494941in}}%
\pgfpathlineto{\pgfqpoint{2.685450in}{1.484293in}}%
\pgfpathlineto{\pgfqpoint{2.700884in}{1.476080in}}%
\pgfpathlineto{\pgfqpoint{2.716317in}{1.470436in}}%
\pgfpathlineto{\pgfqpoint{2.731751in}{1.467440in}}%
\pgfpathlineto{\pgfqpoint{2.747184in}{1.467114in}}%
\pgfpathlineto{\pgfqpoint{2.762618in}{1.469421in}}%
\pgfpathlineto{\pgfqpoint{2.778052in}{1.474268in}}%
\pgfpathlineto{\pgfqpoint{2.793485in}{1.481505in}}%
\pgfpathlineto{\pgfqpoint{2.808919in}{1.490930in}}%
\pgfpathlineto{\pgfqpoint{2.839786in}{1.515305in}}%
\pgfpathlineto{\pgfqpoint{2.870654in}{1.544925in}}%
\pgfpathlineto{\pgfqpoint{2.947822in}{1.622275in}}%
\pgfpathlineto{\pgfqpoint{2.978689in}{1.646517in}}%
\pgfpathlineto{\pgfqpoint{2.994123in}{1.655850in}}%
\pgfpathlineto{\pgfqpoint{3.009556in}{1.662978in}}%
\pgfpathlineto{\pgfqpoint{3.024990in}{1.667698in}}%
\pgfpathlineto{\pgfqpoint{3.040424in}{1.669863in}}%
\pgfpathlineto{\pgfqpoint{3.055857in}{1.669381in}}%
\pgfpathlineto{\pgfqpoint{3.071291in}{1.666219in}}%
\pgfpathlineto{\pgfqpoint{3.086724in}{1.660404in}}%
\pgfpathlineto{\pgfqpoint{3.102158in}{1.652020in}}%
\pgfpathlineto{\pgfqpoint{3.117592in}{1.641208in}}%
\pgfpathlineto{\pgfqpoint{3.148459in}{1.613124in}}%
\pgfpathlineto{\pgfqpoint{3.179326in}{1.578241in}}%
\pgfpathlineto{\pgfqpoint{3.225627in}{1.519038in}}%
\pgfpathlineto{\pgfqpoint{3.271928in}{1.460408in}}%
\pgfpathlineto{\pgfqpoint{3.302795in}{1.426269in}}%
\pgfpathlineto{\pgfqpoint{3.333663in}{1.398683in}}%
\pgfpathlineto{\pgfqpoint{3.349096in}{1.387838in}}%
\pgfpathlineto{\pgfqpoint{3.364530in}{1.379123in}}%
\pgfpathlineto{\pgfqpoint{3.379963in}{1.372601in}}%
\pgfpathlineto{\pgfqpoint{3.395397in}{1.368284in}}%
\pgfpathlineto{\pgfqpoint{3.410831in}{1.366141in}}%
\pgfpathlineto{\pgfqpoint{3.426264in}{1.366096in}}%
\pgfpathlineto{\pgfqpoint{3.441698in}{1.368036in}}%
\pgfpathlineto{\pgfqpoint{3.457132in}{1.371819in}}%
\pgfpathlineto{\pgfqpoint{3.487999in}{1.384226in}}%
\pgfpathlineto{\pgfqpoint{3.518866in}{1.401812in}}%
\pgfpathlineto{\pgfqpoint{3.549733in}{1.423037in}}%
\pgfpathlineto{\pgfqpoint{3.596034in}{1.458792in}}%
\pgfpathlineto{\pgfqpoint{3.688636in}{1.535242in}}%
\pgfpathlineto{\pgfqpoint{3.750371in}{1.588915in}}%
\pgfpathlineto{\pgfqpoint{3.812105in}{1.646297in}}%
\pgfpathlineto{\pgfqpoint{3.889273in}{1.719195in}}%
\pgfpathlineto{\pgfqpoint{3.920141in}{1.745528in}}%
\pgfpathlineto{\pgfqpoint{3.951008in}{1.768198in}}%
\pgfpathlineto{\pgfqpoint{3.981875in}{1.785895in}}%
\pgfpathlineto{\pgfqpoint{4.012743in}{1.797639in}}%
\pgfpathlineto{\pgfqpoint{4.028176in}{1.801113in}}%
\pgfpathlineto{\pgfqpoint{4.043610in}{1.802981in}}%
\pgfpathlineto{\pgfqpoint{4.074477in}{1.802139in}}%
\pgfpathlineto{\pgfqpoint{4.105344in}{1.796047in}}%
\pgfpathlineto{\pgfqpoint{4.136212in}{1.786302in}}%
\pgfpathlineto{\pgfqpoint{4.197946in}{1.764574in}}%
\pgfpathlineto{\pgfqpoint{4.228813in}{1.757359in}}%
\pgfpathlineto{\pgfqpoint{4.259681in}{1.755446in}}%
\pgfpathlineto{\pgfqpoint{4.275114in}{1.756975in}}%
\pgfpathlineto{\pgfqpoint{4.290548in}{1.760335in}}%
\pgfpathlineto{\pgfqpoint{4.305982in}{1.765590in}}%
\pgfpathlineto{\pgfqpoint{4.321415in}{1.772748in}}%
\pgfpathlineto{\pgfqpoint{4.352283in}{1.792518in}}%
\pgfpathlineto{\pgfqpoint{4.383150in}{1.818594in}}%
\pgfpathlineto{\pgfqpoint{4.414017in}{1.849152in}}%
\pgfpathlineto{\pgfqpoint{4.491185in}{1.928917in}}%
\pgfpathlineto{\pgfqpoint{4.522052in}{1.955352in}}%
\pgfpathlineto{\pgfqpoint{4.552920in}{1.975397in}}%
\pgfpathlineto{\pgfqpoint{4.568353in}{1.982586in}}%
\pgfpathlineto{\pgfqpoint{4.583787in}{1.987754in}}%
\pgfpathlineto{\pgfqpoint{4.599221in}{1.990874in}}%
\pgfpathlineto{\pgfqpoint{4.614654in}{1.991969in}}%
\pgfpathlineto{\pgfqpoint{4.630088in}{1.991120in}}%
\pgfpathlineto{\pgfqpoint{4.645522in}{1.988457in}}%
\pgfpathlineto{\pgfqpoint{4.660955in}{1.984154in}}%
\pgfpathlineto{\pgfqpoint{4.660955in}{1.984154in}}%
\pgfusepath{stroke}%
\end{pgfscope}%
\begin{pgfscope}%
\pgfpathrectangle{\pgfqpoint{0.625831in}{0.505056in}}{\pgfqpoint{4.227273in}{2.745455in}} %
\pgfusepath{clip}%
\pgfsetrectcap%
\pgfsetroundjoin%
\pgfsetlinewidth{0.501875pt}%
\definecolor{currentstroke}{rgb}{0.778431,0.905873,0.536867}%
\pgfsetstrokecolor{currentstroke}%
\pgfsetdash{}{0pt}%
\pgfpathmoveto{\pgfqpoint{0.817980in}{2.212040in}}%
\pgfpathlineto{\pgfqpoint{0.833414in}{2.213368in}}%
\pgfpathlineto{\pgfqpoint{0.848847in}{2.211262in}}%
\pgfpathlineto{\pgfqpoint{0.864281in}{2.205502in}}%
\pgfpathlineto{\pgfqpoint{0.879715in}{2.195929in}}%
\pgfpathlineto{\pgfqpoint{0.895148in}{2.182458in}}%
\pgfpathlineto{\pgfqpoint{0.910582in}{2.165074in}}%
\pgfpathlineto{\pgfqpoint{0.926015in}{2.143847in}}%
\pgfpathlineto{\pgfqpoint{0.941449in}{2.118926in}}%
\pgfpathlineto{\pgfqpoint{0.972316in}{2.059006in}}%
\pgfpathlineto{\pgfqpoint{1.003184in}{1.988094in}}%
\pgfpathlineto{\pgfqpoint{1.049485in}{1.869819in}}%
\pgfpathlineto{\pgfqpoint{1.095785in}{1.751830in}}%
\pgfpathlineto{\pgfqpoint{1.126653in}{1.681911in}}%
\pgfpathlineto{\pgfqpoint{1.142086in}{1.651404in}}%
\pgfpathlineto{\pgfqpoint{1.157520in}{1.624582in}}%
\pgfpathlineto{\pgfqpoint{1.172954in}{1.601906in}}%
\pgfpathlineto{\pgfqpoint{1.188387in}{1.583766in}}%
\pgfpathlineto{\pgfqpoint{1.203821in}{1.570474in}}%
\pgfpathlineto{\pgfqpoint{1.219255in}{1.562251in}}%
\pgfpathlineto{\pgfqpoint{1.234688in}{1.559233in}}%
\pgfpathlineto{\pgfqpoint{1.250122in}{1.561459in}}%
\pgfpathlineto{\pgfqpoint{1.265555in}{1.568880in}}%
\pgfpathlineto{\pgfqpoint{1.280989in}{1.581353in}}%
\pgfpathlineto{\pgfqpoint{1.296423in}{1.598653in}}%
\pgfpathlineto{\pgfqpoint{1.311856in}{1.620472in}}%
\pgfpathlineto{\pgfqpoint{1.327290in}{1.646431in}}%
\pgfpathlineto{\pgfqpoint{1.358157in}{1.708947in}}%
\pgfpathlineto{\pgfqpoint{1.389025in}{1.782094in}}%
\pgfpathlineto{\pgfqpoint{1.497060in}{2.055567in}}%
\pgfpathlineto{\pgfqpoint{1.527927in}{2.121284in}}%
\pgfpathlineto{\pgfqpoint{1.558795in}{2.175648in}}%
\pgfpathlineto{\pgfqpoint{1.574228in}{2.197984in}}%
\pgfpathlineto{\pgfqpoint{1.589662in}{2.216892in}}%
\pgfpathlineto{\pgfqpoint{1.605095in}{2.232295in}}%
\pgfpathlineto{\pgfqpoint{1.620529in}{2.244165in}}%
\pgfpathlineto{\pgfqpoint{1.635963in}{2.252525in}}%
\pgfpathlineto{\pgfqpoint{1.651396in}{2.257438in}}%
\pgfpathlineto{\pgfqpoint{1.666830in}{2.259005in}}%
\pgfpathlineto{\pgfqpoint{1.682264in}{2.257361in}}%
\pgfpathlineto{\pgfqpoint{1.697697in}{2.252663in}}%
\pgfpathlineto{\pgfqpoint{1.713131in}{2.245094in}}%
\pgfpathlineto{\pgfqpoint{1.728564in}{2.234848in}}%
\pgfpathlineto{\pgfqpoint{1.743998in}{2.222133in}}%
\pgfpathlineto{\pgfqpoint{1.774865in}{2.190160in}}%
\pgfpathlineto{\pgfqpoint{1.805733in}{2.150925in}}%
\pgfpathlineto{\pgfqpoint{1.836600in}{2.106177in}}%
\pgfpathlineto{\pgfqpoint{1.882901in}{2.032455in}}%
\pgfpathlineto{\pgfqpoint{1.975503in}{1.881801in}}%
\pgfpathlineto{\pgfqpoint{2.006370in}{1.836575in}}%
\pgfpathlineto{\pgfqpoint{2.037237in}{1.796390in}}%
\pgfpathlineto{\pgfqpoint{2.068104in}{1.762536in}}%
\pgfpathlineto{\pgfqpoint{2.098972in}{1.736005in}}%
\pgfpathlineto{\pgfqpoint{2.114405in}{1.725684in}}%
\pgfpathlineto{\pgfqpoint{2.129839in}{1.717372in}}%
\pgfpathlineto{\pgfqpoint{2.145273in}{1.711057in}}%
\pgfpathlineto{\pgfqpoint{2.160706in}{1.706692in}}%
\pgfpathlineto{\pgfqpoint{2.176140in}{1.704193in}}%
\pgfpathlineto{\pgfqpoint{2.191574in}{1.703437in}}%
\pgfpathlineto{\pgfqpoint{2.222441in}{1.706482in}}%
\pgfpathlineto{\pgfqpoint{2.253308in}{1.714158in}}%
\pgfpathlineto{\pgfqpoint{2.345910in}{1.743101in}}%
\pgfpathlineto{\pgfqpoint{2.376777in}{1.747144in}}%
\pgfpathlineto{\pgfqpoint{2.392211in}{1.747027in}}%
\pgfpathlineto{\pgfqpoint{2.407644in}{1.745271in}}%
\pgfpathlineto{\pgfqpoint{2.423078in}{1.741770in}}%
\pgfpathlineto{\pgfqpoint{2.438512in}{1.736469in}}%
\pgfpathlineto{\pgfqpoint{2.453945in}{1.729364in}}%
\pgfpathlineto{\pgfqpoint{2.484813in}{1.709983in}}%
\pgfpathlineto{\pgfqpoint{2.515680in}{1.684613in}}%
\pgfpathlineto{\pgfqpoint{2.546547in}{1.654958in}}%
\pgfpathlineto{\pgfqpoint{2.623715in}{1.577574in}}%
\pgfpathlineto{\pgfqpoint{2.654583in}{1.552070in}}%
\pgfpathlineto{\pgfqpoint{2.670016in}{1.541659in}}%
\pgfpathlineto{\pgfqpoint{2.685450in}{1.533114in}}%
\pgfpathlineto{\pgfqpoint{2.700884in}{1.526608in}}%
\pgfpathlineto{\pgfqpoint{2.716317in}{1.522267in}}%
\pgfpathlineto{\pgfqpoint{2.731751in}{1.520164in}}%
\pgfpathlineto{\pgfqpoint{2.747184in}{1.520319in}}%
\pgfpathlineto{\pgfqpoint{2.762618in}{1.522693in}}%
\pgfpathlineto{\pgfqpoint{2.778052in}{1.527197in}}%
\pgfpathlineto{\pgfqpoint{2.793485in}{1.533686in}}%
\pgfpathlineto{\pgfqpoint{2.824353in}{1.551808in}}%
\pgfpathlineto{\pgfqpoint{2.855220in}{1.575024in}}%
\pgfpathlineto{\pgfqpoint{2.932388in}{1.638201in}}%
\pgfpathlineto{\pgfqpoint{2.963255in}{1.658372in}}%
\pgfpathlineto{\pgfqpoint{2.978689in}{1.666152in}}%
\pgfpathlineto{\pgfqpoint{2.994123in}{1.672087in}}%
\pgfpathlineto{\pgfqpoint{3.009556in}{1.676000in}}%
\pgfpathlineto{\pgfqpoint{3.024990in}{1.677766in}}%
\pgfpathlineto{\pgfqpoint{3.040424in}{1.677310in}}%
\pgfpathlineto{\pgfqpoint{3.055857in}{1.674610in}}%
\pgfpathlineto{\pgfqpoint{3.071291in}{1.669698in}}%
\pgfpathlineto{\pgfqpoint{3.086724in}{1.662658in}}%
\pgfpathlineto{\pgfqpoint{3.102158in}{1.653624in}}%
\pgfpathlineto{\pgfqpoint{3.133025in}{1.630321in}}%
\pgfpathlineto{\pgfqpoint{3.163893in}{1.601655in}}%
\pgfpathlineto{\pgfqpoint{3.271928in}{1.493058in}}%
\pgfpathlineto{\pgfqpoint{3.302795in}{1.468913in}}%
\pgfpathlineto{\pgfqpoint{3.333663in}{1.450761in}}%
\pgfpathlineto{\pgfqpoint{3.349096in}{1.444152in}}%
\pgfpathlineto{\pgfqpoint{3.364530in}{1.439223in}}%
\pgfpathlineto{\pgfqpoint{3.379963in}{1.435948in}}%
\pgfpathlineto{\pgfqpoint{3.410831in}{1.434090in}}%
\pgfpathlineto{\pgfqpoint{3.441698in}{1.437751in}}%
\pgfpathlineto{\pgfqpoint{3.472565in}{1.445778in}}%
\pgfpathlineto{\pgfqpoint{3.503433in}{1.456899in}}%
\pgfpathlineto{\pgfqpoint{3.565167in}{1.483922in}}%
\pgfpathlineto{\pgfqpoint{3.657769in}{1.528220in}}%
\pgfpathlineto{\pgfqpoint{3.704070in}{1.553173in}}%
\pgfpathlineto{\pgfqpoint{3.750371in}{1.582598in}}%
\pgfpathlineto{\pgfqpoint{3.781238in}{1.605453in}}%
\pgfpathlineto{\pgfqpoint{3.827539in}{1.644560in}}%
\pgfpathlineto{\pgfqpoint{3.951008in}{1.756853in}}%
\pgfpathlineto{\pgfqpoint{3.981875in}{1.779406in}}%
\pgfpathlineto{\pgfqpoint{4.012743in}{1.797150in}}%
\pgfpathlineto{\pgfqpoint{4.043610in}{1.809424in}}%
\pgfpathlineto{\pgfqpoint{4.074477in}{1.816176in}}%
\pgfpathlineto{\pgfqpoint{4.105344in}{1.817994in}}%
\pgfpathlineto{\pgfqpoint{4.136212in}{1.816068in}}%
\pgfpathlineto{\pgfqpoint{4.228813in}{1.805420in}}%
\pgfpathlineto{\pgfqpoint{4.259681in}{1.806431in}}%
\pgfpathlineto{\pgfqpoint{4.290548in}{1.812120in}}%
\pgfpathlineto{\pgfqpoint{4.321415in}{1.823083in}}%
\pgfpathlineto{\pgfqpoint{4.352283in}{1.839193in}}%
\pgfpathlineto{\pgfqpoint{4.383150in}{1.859605in}}%
\pgfpathlineto{\pgfqpoint{4.444884in}{1.907030in}}%
\pgfpathlineto{\pgfqpoint{4.475752in}{1.930017in}}%
\pgfpathlineto{\pgfqpoint{4.506619in}{1.949750in}}%
\pgfpathlineto{\pgfqpoint{4.537486in}{1.964467in}}%
\pgfpathlineto{\pgfqpoint{4.552920in}{1.969530in}}%
\pgfpathlineto{\pgfqpoint{4.568353in}{1.972921in}}%
\pgfpathlineto{\pgfqpoint{4.583787in}{1.974588in}}%
\pgfpathlineto{\pgfqpoint{4.599221in}{1.974526in}}%
\pgfpathlineto{\pgfqpoint{4.614654in}{1.972775in}}%
\pgfpathlineto{\pgfqpoint{4.645522in}{1.964583in}}%
\pgfpathlineto{\pgfqpoint{4.660955in}{1.958431in}}%
\pgfpathlineto{\pgfqpoint{4.660955in}{1.958431in}}%
\pgfusepath{stroke}%
\end{pgfscope}%
\begin{pgfscope}%
\pgfpathrectangle{\pgfqpoint{0.625831in}{0.505056in}}{\pgfqpoint{4.227273in}{2.745455in}} %
\pgfusepath{clip}%
\pgfsetrectcap%
\pgfsetroundjoin%
\pgfsetlinewidth{0.501875pt}%
\definecolor{currentstroke}{rgb}{0.864706,0.840344,0.478512}%
\pgfsetstrokecolor{currentstroke}%
\pgfsetdash{}{0pt}%
\pgfpathmoveto{\pgfqpoint{0.817980in}{2.248307in}}%
\pgfpathlineto{\pgfqpoint{0.833414in}{2.247215in}}%
\pgfpathlineto{\pgfqpoint{0.848847in}{2.242439in}}%
\pgfpathlineto{\pgfqpoint{0.864281in}{2.233854in}}%
\pgfpathlineto{\pgfqpoint{0.879715in}{2.221403in}}%
\pgfpathlineto{\pgfqpoint{0.895148in}{2.205093in}}%
\pgfpathlineto{\pgfqpoint{0.910582in}{2.185007in}}%
\pgfpathlineto{\pgfqpoint{0.926015in}{2.161297in}}%
\pgfpathlineto{\pgfqpoint{0.956883in}{2.103975in}}%
\pgfpathlineto{\pgfqpoint{0.987750in}{2.035727in}}%
\pgfpathlineto{\pgfqpoint{1.034051in}{1.920806in}}%
\pgfpathlineto{\pgfqpoint{1.080352in}{1.803993in}}%
\pgfpathlineto{\pgfqpoint{1.111219in}{1.732854in}}%
\pgfpathlineto{\pgfqpoint{1.142086in}{1.672285in}}%
\pgfpathlineto{\pgfqpoint{1.157520in}{1.647181in}}%
\pgfpathlineto{\pgfqpoint{1.172954in}{1.626079in}}%
\pgfpathlineto{\pgfqpoint{1.188387in}{1.609312in}}%
\pgfpathlineto{\pgfqpoint{1.203821in}{1.597149in}}%
\pgfpathlineto{\pgfqpoint{1.219255in}{1.589782in}}%
\pgfpathlineto{\pgfqpoint{1.234688in}{1.587329in}}%
\pgfpathlineto{\pgfqpoint{1.250122in}{1.589831in}}%
\pgfpathlineto{\pgfqpoint{1.265555in}{1.597250in}}%
\pgfpathlineto{\pgfqpoint{1.280989in}{1.609473in}}%
\pgfpathlineto{\pgfqpoint{1.296423in}{1.626312in}}%
\pgfpathlineto{\pgfqpoint{1.311856in}{1.647508in}}%
\pgfpathlineto{\pgfqpoint{1.327290in}{1.672738in}}%
\pgfpathlineto{\pgfqpoint{1.358157in}{1.733718in}}%
\pgfpathlineto{\pgfqpoint{1.389025in}{1.805612in}}%
\pgfpathlineto{\pgfqpoint{1.435325in}{1.924646in}}%
\pgfpathlineto{\pgfqpoint{1.481626in}{2.043727in}}%
\pgfpathlineto{\pgfqpoint{1.512494in}{2.116078in}}%
\pgfpathlineto{\pgfqpoint{1.543361in}{2.178581in}}%
\pgfpathlineto{\pgfqpoint{1.558795in}{2.205237in}}%
\pgfpathlineto{\pgfqpoint{1.574228in}{2.228455in}}%
\pgfpathlineto{\pgfqpoint{1.589662in}{2.248028in}}%
\pgfpathlineto{\pgfqpoint{1.605095in}{2.263805in}}%
\pgfpathlineto{\pgfqpoint{1.620529in}{2.275693in}}%
\pgfpathlineto{\pgfqpoint{1.635963in}{2.283657in}}%
\pgfpathlineto{\pgfqpoint{1.651396in}{2.287715in}}%
\pgfpathlineto{\pgfqpoint{1.666830in}{2.287937in}}%
\pgfpathlineto{\pgfqpoint{1.682264in}{2.284439in}}%
\pgfpathlineto{\pgfqpoint{1.697697in}{2.277378in}}%
\pgfpathlineto{\pgfqpoint{1.713131in}{2.266950in}}%
\pgfpathlineto{\pgfqpoint{1.728564in}{2.253382in}}%
\pgfpathlineto{\pgfqpoint{1.743998in}{2.236927in}}%
\pgfpathlineto{\pgfqpoint{1.774865in}{2.196477in}}%
\pgfpathlineto{\pgfqpoint{1.805733in}{2.147980in}}%
\pgfpathlineto{\pgfqpoint{1.852034in}{2.065644in}}%
\pgfpathlineto{\pgfqpoint{1.929202in}{1.923501in}}%
\pgfpathlineto{\pgfqpoint{1.960069in}{1.871600in}}%
\pgfpathlineto{\pgfqpoint{1.990936in}{1.825486in}}%
\pgfpathlineto{\pgfqpoint{2.021804in}{1.786718in}}%
\pgfpathlineto{\pgfqpoint{2.052671in}{1.756451in}}%
\pgfpathlineto{\pgfqpoint{2.068104in}{1.744736in}}%
\pgfpathlineto{\pgfqpoint{2.083538in}{1.735357in}}%
\pgfpathlineto{\pgfqpoint{2.098972in}{1.728309in}}%
\pgfpathlineto{\pgfqpoint{2.114405in}{1.723549in}}%
\pgfpathlineto{\pgfqpoint{2.129839in}{1.720996in}}%
\pgfpathlineto{\pgfqpoint{2.145273in}{1.720526in}}%
\pgfpathlineto{\pgfqpoint{2.160706in}{1.721980in}}%
\pgfpathlineto{\pgfqpoint{2.191574in}{1.729832in}}%
\pgfpathlineto{\pgfqpoint{2.222441in}{1.742577in}}%
\pgfpathlineto{\pgfqpoint{2.299609in}{1.779733in}}%
\pgfpathlineto{\pgfqpoint{2.330476in}{1.789961in}}%
\pgfpathlineto{\pgfqpoint{2.345910in}{1.792921in}}%
\pgfpathlineto{\pgfqpoint{2.361344in}{1.794142in}}%
\pgfpathlineto{\pgfqpoint{2.376777in}{1.793447in}}%
\pgfpathlineto{\pgfqpoint{2.392211in}{1.790707in}}%
\pgfpathlineto{\pgfqpoint{2.407644in}{1.785841in}}%
\pgfpathlineto{\pgfqpoint{2.423078in}{1.778820in}}%
\pgfpathlineto{\pgfqpoint{2.438512in}{1.769667in}}%
\pgfpathlineto{\pgfqpoint{2.453945in}{1.758456in}}%
\pgfpathlineto{\pgfqpoint{2.484813in}{1.730402in}}%
\pgfpathlineto{\pgfqpoint{2.515680in}{1.696179in}}%
\pgfpathlineto{\pgfqpoint{2.561981in}{1.637996in}}%
\pgfpathlineto{\pgfqpoint{2.608282in}{1.579238in}}%
\pgfpathlineto{\pgfqpoint{2.639149in}{1.544072in}}%
\pgfpathlineto{\pgfqpoint{2.670016in}{1.514803in}}%
\pgfpathlineto{\pgfqpoint{2.685450in}{1.502962in}}%
\pgfpathlineto{\pgfqpoint{2.700884in}{1.493220in}}%
\pgfpathlineto{\pgfqpoint{2.716317in}{1.485706in}}%
\pgfpathlineto{\pgfqpoint{2.731751in}{1.480503in}}%
\pgfpathlineto{\pgfqpoint{2.747184in}{1.477648in}}%
\pgfpathlineto{\pgfqpoint{2.762618in}{1.477135in}}%
\pgfpathlineto{\pgfqpoint{2.778052in}{1.478912in}}%
\pgfpathlineto{\pgfqpoint{2.793485in}{1.482883in}}%
\pgfpathlineto{\pgfqpoint{2.808919in}{1.488911in}}%
\pgfpathlineto{\pgfqpoint{2.839786in}{1.506395in}}%
\pgfpathlineto{\pgfqpoint{2.870654in}{1.529558in}}%
\pgfpathlineto{\pgfqpoint{2.916954in}{1.570005in}}%
\pgfpathlineto{\pgfqpoint{2.963255in}{1.609828in}}%
\pgfpathlineto{\pgfqpoint{2.994123in}{1.631992in}}%
\pgfpathlineto{\pgfqpoint{3.024990in}{1.648248in}}%
\pgfpathlineto{\pgfqpoint{3.040424in}{1.653671in}}%
\pgfpathlineto{\pgfqpoint{3.055857in}{1.657106in}}%
\pgfpathlineto{\pgfqpoint{3.071291in}{1.658468in}}%
\pgfpathlineto{\pgfqpoint{3.086724in}{1.657718in}}%
\pgfpathlineto{\pgfqpoint{3.102158in}{1.654860in}}%
\pgfpathlineto{\pgfqpoint{3.117592in}{1.649945in}}%
\pgfpathlineto{\pgfqpoint{3.133025in}{1.643066in}}%
\pgfpathlineto{\pgfqpoint{3.163893in}{1.623980in}}%
\pgfpathlineto{\pgfqpoint{3.194760in}{1.599070in}}%
\pgfpathlineto{\pgfqpoint{3.241061in}{1.554949in}}%
\pgfpathlineto{\pgfqpoint{3.302795in}{1.494437in}}%
\pgfpathlineto{\pgfqpoint{3.333663in}{1.467743in}}%
\pgfpathlineto{\pgfqpoint{3.364530in}{1.445356in}}%
\pgfpathlineto{\pgfqpoint{3.395397in}{1.428165in}}%
\pgfpathlineto{\pgfqpoint{3.426264in}{1.416609in}}%
\pgfpathlineto{\pgfqpoint{3.457132in}{1.410734in}}%
\pgfpathlineto{\pgfqpoint{3.487999in}{1.410296in}}%
\pgfpathlineto{\pgfqpoint{3.518866in}{1.414881in}}%
\pgfpathlineto{\pgfqpoint{3.549733in}{1.424026in}}%
\pgfpathlineto{\pgfqpoint{3.580601in}{1.437301in}}%
\pgfpathlineto{\pgfqpoint{3.611468in}{1.454362in}}%
\pgfpathlineto{\pgfqpoint{3.642335in}{1.474954in}}%
\pgfpathlineto{\pgfqpoint{3.673203in}{1.498874in}}%
\pgfpathlineto{\pgfqpoint{3.704070in}{1.525903in}}%
\pgfpathlineto{\pgfqpoint{3.750371in}{1.571546in}}%
\pgfpathlineto{\pgfqpoint{3.812105in}{1.638861in}}%
\pgfpathlineto{\pgfqpoint{3.873840in}{1.706777in}}%
\pgfpathlineto{\pgfqpoint{3.904707in}{1.738026in}}%
\pgfpathlineto{\pgfqpoint{3.935574in}{1.765854in}}%
\pgfpathlineto{\pgfqpoint{3.966442in}{1.789225in}}%
\pgfpathlineto{\pgfqpoint{3.997309in}{1.807376in}}%
\pgfpathlineto{\pgfqpoint{4.028176in}{1.819914in}}%
\pgfpathlineto{\pgfqpoint{4.059043in}{1.826874in}}%
\pgfpathlineto{\pgfqpoint{4.089911in}{1.828735in}}%
\pgfpathlineto{\pgfqpoint{4.120778in}{1.826398in}}%
\pgfpathlineto{\pgfqpoint{4.167079in}{1.817833in}}%
\pgfpathlineto{\pgfqpoint{4.228813in}{1.805104in}}%
\pgfpathlineto{\pgfqpoint{4.259681in}{1.801685in}}%
\pgfpathlineto{\pgfqpoint{4.290548in}{1.801962in}}%
\pgfpathlineto{\pgfqpoint{4.321415in}{1.806775in}}%
\pgfpathlineto{\pgfqpoint{4.352283in}{1.816478in}}%
\pgfpathlineto{\pgfqpoint{4.383150in}{1.830889in}}%
\pgfpathlineto{\pgfqpoint{4.414017in}{1.849305in}}%
\pgfpathlineto{\pgfqpoint{4.460318in}{1.881792in}}%
\pgfpathlineto{\pgfqpoint{4.522052in}{1.925870in}}%
\pgfpathlineto{\pgfqpoint{4.552920in}{1.944533in}}%
\pgfpathlineto{\pgfqpoint{4.583787in}{1.959105in}}%
\pgfpathlineto{\pgfqpoint{4.614654in}{1.968709in}}%
\pgfpathlineto{\pgfqpoint{4.645522in}{1.972987in}}%
\pgfpathlineto{\pgfqpoint{4.660955in}{1.973166in}}%
\pgfpathlineto{\pgfqpoint{4.660955in}{1.973166in}}%
\pgfusepath{stroke}%
\end{pgfscope}%
\begin{pgfscope}%
\pgfpathrectangle{\pgfqpoint{0.625831in}{0.505056in}}{\pgfqpoint{4.227273in}{2.745455in}} %
\pgfusepath{clip}%
\pgfsetrectcap%
\pgfsetroundjoin%
\pgfsetlinewidth{0.501875pt}%
\definecolor{currentstroke}{rgb}{0.943137,0.767363,0.423549}%
\pgfsetstrokecolor{currentstroke}%
\pgfsetdash{}{0pt}%
\pgfpathmoveto{\pgfqpoint{0.817980in}{2.243563in}}%
\pgfpathlineto{\pgfqpoint{0.833414in}{2.241490in}}%
\pgfpathlineto{\pgfqpoint{0.848847in}{2.235544in}}%
\pgfpathlineto{\pgfqpoint{0.864281in}{2.225628in}}%
\pgfpathlineto{\pgfqpoint{0.879715in}{2.211715in}}%
\pgfpathlineto{\pgfqpoint{0.895148in}{2.193842in}}%
\pgfpathlineto{\pgfqpoint{0.910582in}{2.172119in}}%
\pgfpathlineto{\pgfqpoint{0.926015in}{2.146725in}}%
\pgfpathlineto{\pgfqpoint{0.956883in}{2.085983in}}%
\pgfpathlineto{\pgfqpoint{0.987750in}{2.014359in}}%
\pgfpathlineto{\pgfqpoint{1.034051in}{1.894634in}}%
\pgfpathlineto{\pgfqpoint{1.080352in}{1.773383in}}%
\pgfpathlineto{\pgfqpoint{1.111219in}{1.699463in}}%
\pgfpathlineto{\pgfqpoint{1.142086in}{1.636234in}}%
\pgfpathlineto{\pgfqpoint{1.157520in}{1.609848in}}%
\pgfpathlineto{\pgfqpoint{1.172954in}{1.587500in}}%
\pgfpathlineto{\pgfqpoint{1.188387in}{1.569534in}}%
\pgfpathlineto{\pgfqpoint{1.203821in}{1.556225in}}%
\pgfpathlineto{\pgfqpoint{1.219255in}{1.547782in}}%
\pgfpathlineto{\pgfqpoint{1.234688in}{1.544338in}}%
\pgfpathlineto{\pgfqpoint{1.250122in}{1.545953in}}%
\pgfpathlineto{\pgfqpoint{1.265555in}{1.552610in}}%
\pgfpathlineto{\pgfqpoint{1.280989in}{1.564218in}}%
\pgfpathlineto{\pgfqpoint{1.296423in}{1.580610in}}%
\pgfpathlineto{\pgfqpoint{1.311856in}{1.601550in}}%
\pgfpathlineto{\pgfqpoint{1.327290in}{1.626737in}}%
\pgfpathlineto{\pgfqpoint{1.342724in}{1.655805in}}%
\pgfpathlineto{\pgfqpoint{1.373591in}{1.723867in}}%
\pgfpathlineto{\pgfqpoint{1.404458in}{1.801856in}}%
\pgfpathlineto{\pgfqpoint{1.512494in}{2.088523in}}%
\pgfpathlineto{\pgfqpoint{1.543361in}{2.156378in}}%
\pgfpathlineto{\pgfqpoint{1.558795in}{2.185657in}}%
\pgfpathlineto{\pgfqpoint{1.574228in}{2.211389in}}%
\pgfpathlineto{\pgfqpoint{1.589662in}{2.233310in}}%
\pgfpathlineto{\pgfqpoint{1.605095in}{2.251215in}}%
\pgfpathlineto{\pgfqpoint{1.620529in}{2.264963in}}%
\pgfpathlineto{\pgfqpoint{1.635963in}{2.274473in}}%
\pgfpathlineto{\pgfqpoint{1.651396in}{2.279724in}}%
\pgfpathlineto{\pgfqpoint{1.666830in}{2.280755in}}%
\pgfpathlineto{\pgfqpoint{1.682264in}{2.277661in}}%
\pgfpathlineto{\pgfqpoint{1.697697in}{2.270590in}}%
\pgfpathlineto{\pgfqpoint{1.713131in}{2.259736in}}%
\pgfpathlineto{\pgfqpoint{1.728564in}{2.245339in}}%
\pgfpathlineto{\pgfqpoint{1.743998in}{2.227674in}}%
\pgfpathlineto{\pgfqpoint{1.759432in}{2.207051in}}%
\pgfpathlineto{\pgfqpoint{1.790299in}{2.158300in}}%
\pgfpathlineto{\pgfqpoint{1.821166in}{2.102001in}}%
\pgfpathlineto{\pgfqpoint{1.882901in}{1.979105in}}%
\pgfpathlineto{\pgfqpoint{1.929202in}{1.889721in}}%
\pgfpathlineto{\pgfqpoint{1.960069in}{1.836463in}}%
\pgfpathlineto{\pgfqpoint{1.990936in}{1.790764in}}%
\pgfpathlineto{\pgfqpoint{2.021804in}{1.754192in}}%
\pgfpathlineto{\pgfqpoint{2.037237in}{1.739657in}}%
\pgfpathlineto{\pgfqpoint{2.052671in}{1.727720in}}%
\pgfpathlineto{\pgfqpoint{2.068104in}{1.718400in}}%
\pgfpathlineto{\pgfqpoint{2.083538in}{1.711675in}}%
\pgfpathlineto{\pgfqpoint{2.098972in}{1.707480in}}%
\pgfpathlineto{\pgfqpoint{2.114405in}{1.705706in}}%
\pgfpathlineto{\pgfqpoint{2.129839in}{1.706206in}}%
\pgfpathlineto{\pgfqpoint{2.145273in}{1.708792in}}%
\pgfpathlineto{\pgfqpoint{2.176140in}{1.719290in}}%
\pgfpathlineto{\pgfqpoint{2.207007in}{1.735035in}}%
\pgfpathlineto{\pgfqpoint{2.299609in}{1.787624in}}%
\pgfpathlineto{\pgfqpoint{2.330476in}{1.798180in}}%
\pgfpathlineto{\pgfqpoint{2.345910in}{1.800889in}}%
\pgfpathlineto{\pgfqpoint{2.361344in}{1.801620in}}%
\pgfpathlineto{\pgfqpoint{2.376777in}{1.800231in}}%
\pgfpathlineto{\pgfqpoint{2.392211in}{1.796629in}}%
\pgfpathlineto{\pgfqpoint{2.407644in}{1.790771in}}%
\pgfpathlineto{\pgfqpoint{2.423078in}{1.782667in}}%
\pgfpathlineto{\pgfqpoint{2.438512in}{1.772377in}}%
\pgfpathlineto{\pgfqpoint{2.469379in}{1.745733in}}%
\pgfpathlineto{\pgfqpoint{2.500246in}{1.712281in}}%
\pgfpathlineto{\pgfqpoint{2.546547in}{1.653964in}}%
\pgfpathlineto{\pgfqpoint{2.608282in}{1.574619in}}%
\pgfpathlineto{\pgfqpoint{2.639149in}{1.540245in}}%
\pgfpathlineto{\pgfqpoint{2.670016in}{1.512503in}}%
\pgfpathlineto{\pgfqpoint{2.685450in}{1.501679in}}%
\pgfpathlineto{\pgfqpoint{2.700884in}{1.493106in}}%
\pgfpathlineto{\pgfqpoint{2.716317in}{1.486897in}}%
\pgfpathlineto{\pgfqpoint{2.731751in}{1.483117in}}%
\pgfpathlineto{\pgfqpoint{2.747184in}{1.481786in}}%
\pgfpathlineto{\pgfqpoint{2.762618in}{1.482875in}}%
\pgfpathlineto{\pgfqpoint{2.778052in}{1.486312in}}%
\pgfpathlineto{\pgfqpoint{2.793485in}{1.491980in}}%
\pgfpathlineto{\pgfqpoint{2.808919in}{1.499719in}}%
\pgfpathlineto{\pgfqpoint{2.839786in}{1.520593in}}%
\pgfpathlineto{\pgfqpoint{2.870654in}{1.546971in}}%
\pgfpathlineto{\pgfqpoint{2.978689in}{1.647268in}}%
\pgfpathlineto{\pgfqpoint{3.009556in}{1.667972in}}%
\pgfpathlineto{\pgfqpoint{3.024990in}{1.675684in}}%
\pgfpathlineto{\pgfqpoint{3.040424in}{1.681394in}}%
\pgfpathlineto{\pgfqpoint{3.055857in}{1.684969in}}%
\pgfpathlineto{\pgfqpoint{3.071291in}{1.686321in}}%
\pgfpathlineto{\pgfqpoint{3.086724in}{1.685404in}}%
\pgfpathlineto{\pgfqpoint{3.102158in}{1.682218in}}%
\pgfpathlineto{\pgfqpoint{3.117592in}{1.676805in}}%
\pgfpathlineto{\pgfqpoint{3.133025in}{1.669251in}}%
\pgfpathlineto{\pgfqpoint{3.148459in}{1.659683in}}%
\pgfpathlineto{\pgfqpoint{3.179326in}{1.635186in}}%
\pgfpathlineto{\pgfqpoint{3.210193in}{1.604980in}}%
\pgfpathlineto{\pgfqpoint{3.256494in}{1.553439in}}%
\pgfpathlineto{\pgfqpoint{3.318229in}{1.484508in}}%
\pgfpathlineto{\pgfqpoint{3.349096in}{1.454433in}}%
\pgfpathlineto{\pgfqpoint{3.379963in}{1.429360in}}%
\pgfpathlineto{\pgfqpoint{3.410831in}{1.410317in}}%
\pgfpathlineto{\pgfqpoint{3.441698in}{1.397891in}}%
\pgfpathlineto{\pgfqpoint{3.457132in}{1.394231in}}%
\pgfpathlineto{\pgfqpoint{3.472565in}{1.392268in}}%
\pgfpathlineto{\pgfqpoint{3.487999in}{1.391973in}}%
\pgfpathlineto{\pgfqpoint{3.503433in}{1.393305in}}%
\pgfpathlineto{\pgfqpoint{3.534300in}{1.400618in}}%
\pgfpathlineto{\pgfqpoint{3.565167in}{1.413672in}}%
\pgfpathlineto{\pgfqpoint{3.596034in}{1.431855in}}%
\pgfpathlineto{\pgfqpoint{3.626902in}{1.454529in}}%
\pgfpathlineto{\pgfqpoint{3.657769in}{1.481050in}}%
\pgfpathlineto{\pgfqpoint{3.704070in}{1.526610in}}%
\pgfpathlineto{\pgfqpoint{3.750371in}{1.576997in}}%
\pgfpathlineto{\pgfqpoint{3.858406in}{1.697438in}}%
\pgfpathlineto{\pgfqpoint{3.889273in}{1.728243in}}%
\pgfpathlineto{\pgfqpoint{3.920141in}{1.755656in}}%
\pgfpathlineto{\pgfqpoint{3.951008in}{1.778855in}}%
\pgfpathlineto{\pgfqpoint{3.981875in}{1.797232in}}%
\pgfpathlineto{\pgfqpoint{4.012743in}{1.810454in}}%
\pgfpathlineto{\pgfqpoint{4.043610in}{1.818507in}}%
\pgfpathlineto{\pgfqpoint{4.074477in}{1.821726in}}%
\pgfpathlineto{\pgfqpoint{4.105344in}{1.820775in}}%
\pgfpathlineto{\pgfqpoint{4.136212in}{1.816615in}}%
\pgfpathlineto{\pgfqpoint{4.197946in}{1.803558in}}%
\pgfpathlineto{\pgfqpoint{4.244247in}{1.794868in}}%
\pgfpathlineto{\pgfqpoint{4.275114in}{1.791933in}}%
\pgfpathlineto{\pgfqpoint{4.305982in}{1.792405in}}%
\pgfpathlineto{\pgfqpoint{4.336849in}{1.796869in}}%
\pgfpathlineto{\pgfqpoint{4.367716in}{1.805515in}}%
\pgfpathlineto{\pgfqpoint{4.398583in}{1.818110in}}%
\pgfpathlineto{\pgfqpoint{4.429451in}{1.834017in}}%
\pgfpathlineto{\pgfqpoint{4.475752in}{1.861866in}}%
\pgfpathlineto{\pgfqpoint{4.537486in}{1.899658in}}%
\pgfpathlineto{\pgfqpoint{4.568353in}{1.915827in}}%
\pgfpathlineto{\pgfqpoint{4.599221in}{1.928651in}}%
\pgfpathlineto{\pgfqpoint{4.630088in}{1.937346in}}%
\pgfpathlineto{\pgfqpoint{4.660955in}{1.941507in}}%
\pgfpathlineto{\pgfqpoint{4.660955in}{1.941507in}}%
\pgfusepath{stroke}%
\end{pgfscope}%
\begin{pgfscope}%
\pgfpathrectangle{\pgfqpoint{0.625831in}{0.505056in}}{\pgfqpoint{4.227273in}{2.745455in}} %
\pgfusepath{clip}%
\pgfsetrectcap%
\pgfsetroundjoin%
\pgfsetlinewidth{0.501875pt}%
\definecolor{currentstroke}{rgb}{1.000000,0.682749,0.366979}%
\pgfsetstrokecolor{currentstroke}%
\pgfsetdash{}{0pt}%
\pgfpathmoveto{\pgfqpoint{0.817980in}{2.244977in}}%
\pgfpathlineto{\pgfqpoint{0.833414in}{2.242178in}}%
\pgfpathlineto{\pgfqpoint{0.848847in}{2.235447in}}%
\pgfpathlineto{\pgfqpoint{0.864281in}{2.224683in}}%
\pgfpathlineto{\pgfqpoint{0.879715in}{2.209853in}}%
\pgfpathlineto{\pgfqpoint{0.895148in}{2.190995in}}%
\pgfpathlineto{\pgfqpoint{0.910582in}{2.168221in}}%
\pgfpathlineto{\pgfqpoint{0.926015in}{2.141713in}}%
\pgfpathlineto{\pgfqpoint{0.956883in}{2.078590in}}%
\pgfpathlineto{\pgfqpoint{0.987750in}{2.004464in}}%
\pgfpathlineto{\pgfqpoint{1.034051in}{1.881075in}}%
\pgfpathlineto{\pgfqpoint{1.080352in}{1.756800in}}%
\pgfpathlineto{\pgfqpoint{1.111219in}{1.681516in}}%
\pgfpathlineto{\pgfqpoint{1.142086in}{1.617611in}}%
\pgfpathlineto{\pgfqpoint{1.157520in}{1.591170in}}%
\pgfpathlineto{\pgfqpoint{1.172954in}{1.568960in}}%
\pgfpathlineto{\pgfqpoint{1.188387in}{1.551320in}}%
\pgfpathlineto{\pgfqpoint{1.203821in}{1.538518in}}%
\pgfpathlineto{\pgfqpoint{1.219255in}{1.530744in}}%
\pgfpathlineto{\pgfqpoint{1.234688in}{1.528112in}}%
\pgfpathlineto{\pgfqpoint{1.250122in}{1.530656in}}%
\pgfpathlineto{\pgfqpoint{1.265555in}{1.538328in}}%
\pgfpathlineto{\pgfqpoint{1.280989in}{1.551005in}}%
\pgfpathlineto{\pgfqpoint{1.296423in}{1.568485in}}%
\pgfpathlineto{\pgfqpoint{1.311856in}{1.590497in}}%
\pgfpathlineto{\pgfqpoint{1.327290in}{1.616703in}}%
\pgfpathlineto{\pgfqpoint{1.358157in}{1.680057in}}%
\pgfpathlineto{\pgfqpoint{1.389025in}{1.754806in}}%
\pgfpathlineto{\pgfqpoint{1.435325in}{1.878824in}}%
\pgfpathlineto{\pgfqpoint{1.481626in}{2.003534in}}%
\pgfpathlineto{\pgfqpoint{1.512494in}{2.079902in}}%
\pgfpathlineto{\pgfqpoint{1.543361in}{2.146556in}}%
\pgfpathlineto{\pgfqpoint{1.558795in}{2.175311in}}%
\pgfpathlineto{\pgfqpoint{1.574228in}{2.200627in}}%
\pgfpathlineto{\pgfqpoint{1.589662in}{2.222279in}}%
\pgfpathlineto{\pgfqpoint{1.605095in}{2.240098in}}%
\pgfpathlineto{\pgfqpoint{1.620529in}{2.253974in}}%
\pgfpathlineto{\pgfqpoint{1.635963in}{2.263849in}}%
\pgfpathlineto{\pgfqpoint{1.651396in}{2.269722in}}%
\pgfpathlineto{\pgfqpoint{1.666830in}{2.271642in}}%
\pgfpathlineto{\pgfqpoint{1.682264in}{2.269706in}}%
\pgfpathlineto{\pgfqpoint{1.697697in}{2.264054in}}%
\pgfpathlineto{\pgfqpoint{1.713131in}{2.254868in}}%
\pgfpathlineto{\pgfqpoint{1.728564in}{2.242366in}}%
\pgfpathlineto{\pgfqpoint{1.743998in}{2.226794in}}%
\pgfpathlineto{\pgfqpoint{1.759432in}{2.208429in}}%
\pgfpathlineto{\pgfqpoint{1.790299in}{2.164521in}}%
\pgfpathlineto{\pgfqpoint{1.821166in}{2.113198in}}%
\pgfpathlineto{\pgfqpoint{1.867467in}{2.028220in}}%
\pgfpathlineto{\pgfqpoint{1.929202in}{1.913982in}}%
\pgfpathlineto{\pgfqpoint{1.960069in}{1.861945in}}%
\pgfpathlineto{\pgfqpoint{1.990936in}{1.816021in}}%
\pgfpathlineto{\pgfqpoint{2.021804in}{1.777760in}}%
\pgfpathlineto{\pgfqpoint{2.037237in}{1.761857in}}%
\pgfpathlineto{\pgfqpoint{2.052671in}{1.748228in}}%
\pgfpathlineto{\pgfqpoint{2.068104in}{1.736920in}}%
\pgfpathlineto{\pgfqpoint{2.083538in}{1.727946in}}%
\pgfpathlineto{\pgfqpoint{2.098972in}{1.721279in}}%
\pgfpathlineto{\pgfqpoint{2.114405in}{1.716856in}}%
\pgfpathlineto{\pgfqpoint{2.129839in}{1.714575in}}%
\pgfpathlineto{\pgfqpoint{2.145273in}{1.714301in}}%
\pgfpathlineto{\pgfqpoint{2.160706in}{1.715862in}}%
\pgfpathlineto{\pgfqpoint{2.191574in}{1.723652in}}%
\pgfpathlineto{\pgfqpoint{2.222441in}{1.736002in}}%
\pgfpathlineto{\pgfqpoint{2.299609in}{1.771623in}}%
\pgfpathlineto{\pgfqpoint{2.330476in}{1.781551in}}%
\pgfpathlineto{\pgfqpoint{2.345910in}{1.784528in}}%
\pgfpathlineto{\pgfqpoint{2.361344in}{1.785906in}}%
\pgfpathlineto{\pgfqpoint{2.376777in}{1.785522in}}%
\pgfpathlineto{\pgfqpoint{2.392211in}{1.783255in}}%
\pgfpathlineto{\pgfqpoint{2.407644in}{1.779031in}}%
\pgfpathlineto{\pgfqpoint{2.423078in}{1.772819in}}%
\pgfpathlineto{\pgfqpoint{2.438512in}{1.764638in}}%
\pgfpathlineto{\pgfqpoint{2.469379in}{1.742678in}}%
\pgfpathlineto{\pgfqpoint{2.500246in}{1.714221in}}%
\pgfpathlineto{\pgfqpoint{2.531114in}{1.680982in}}%
\pgfpathlineto{\pgfqpoint{2.623715in}{1.575934in}}%
\pgfpathlineto{\pgfqpoint{2.654583in}{1.547505in}}%
\pgfpathlineto{\pgfqpoint{2.670016in}{1.535795in}}%
\pgfpathlineto{\pgfqpoint{2.685450in}{1.526039in}}%
\pgfpathlineto{\pgfqpoint{2.700884in}{1.518402in}}%
\pgfpathlineto{\pgfqpoint{2.716317in}{1.513003in}}%
\pgfpathlineto{\pgfqpoint{2.731751in}{1.509916in}}%
\pgfpathlineto{\pgfqpoint{2.747184in}{1.509165in}}%
\pgfpathlineto{\pgfqpoint{2.762618in}{1.510725in}}%
\pgfpathlineto{\pgfqpoint{2.778052in}{1.514523in}}%
\pgfpathlineto{\pgfqpoint{2.793485in}{1.520440in}}%
\pgfpathlineto{\pgfqpoint{2.808919in}{1.528312in}}%
\pgfpathlineto{\pgfqpoint{2.839786in}{1.549065in}}%
\pgfpathlineto{\pgfqpoint{2.870654in}{1.574750in}}%
\pgfpathlineto{\pgfqpoint{2.963255in}{1.656450in}}%
\pgfpathlineto{\pgfqpoint{2.994123in}{1.676900in}}%
\pgfpathlineto{\pgfqpoint{3.009556in}{1.684656in}}%
\pgfpathlineto{\pgfqpoint{3.024990in}{1.690514in}}%
\pgfpathlineto{\pgfqpoint{3.040424in}{1.694336in}}%
\pgfpathlineto{\pgfqpoint{3.055857in}{1.696023in}}%
\pgfpathlineto{\pgfqpoint{3.071291in}{1.695525in}}%
\pgfpathlineto{\pgfqpoint{3.086724in}{1.692833in}}%
\pgfpathlineto{\pgfqpoint{3.102158in}{1.687985in}}%
\pgfpathlineto{\pgfqpoint{3.117592in}{1.681058in}}%
\pgfpathlineto{\pgfqpoint{3.133025in}{1.672169in}}%
\pgfpathlineto{\pgfqpoint{3.163893in}{1.649153in}}%
\pgfpathlineto{\pgfqpoint{3.194760in}{1.620512in}}%
\pgfpathlineto{\pgfqpoint{3.241061in}{1.571242in}}%
\pgfpathlineto{\pgfqpoint{3.302795in}{1.504559in}}%
\pgfpathlineto{\pgfqpoint{3.333663in}{1.475000in}}%
\pgfpathlineto{\pgfqpoint{3.364530in}{1.449901in}}%
\pgfpathlineto{\pgfqpoint{3.395397in}{1.430213in}}%
\pgfpathlineto{\pgfqpoint{3.426264in}{1.416475in}}%
\pgfpathlineto{\pgfqpoint{3.457132in}{1.408855in}}%
\pgfpathlineto{\pgfqpoint{3.487999in}{1.407226in}}%
\pgfpathlineto{\pgfqpoint{3.518866in}{1.411260in}}%
\pgfpathlineto{\pgfqpoint{3.549733in}{1.420526in}}%
\pgfpathlineto{\pgfqpoint{3.580601in}{1.434565in}}%
\pgfpathlineto{\pgfqpoint{3.611468in}{1.452948in}}%
\pgfpathlineto{\pgfqpoint{3.642335in}{1.475291in}}%
\pgfpathlineto{\pgfqpoint{3.673203in}{1.501245in}}%
\pgfpathlineto{\pgfqpoint{3.704070in}{1.530455in}}%
\pgfpathlineto{\pgfqpoint{3.750371in}{1.579414in}}%
\pgfpathlineto{\pgfqpoint{3.812105in}{1.650946in}}%
\pgfpathlineto{\pgfqpoint{3.873840in}{1.722716in}}%
\pgfpathlineto{\pgfqpoint{3.904707in}{1.755686in}}%
\pgfpathlineto{\pgfqpoint{3.935574in}{1.784996in}}%
\pgfpathlineto{\pgfqpoint{3.966442in}{1.809478in}}%
\pgfpathlineto{\pgfqpoint{3.997309in}{1.828207in}}%
\pgfpathlineto{\pgfqpoint{4.028176in}{1.840614in}}%
\pgfpathlineto{\pgfqpoint{4.043610in}{1.844389in}}%
\pgfpathlineto{\pgfqpoint{4.074477in}{1.847234in}}%
\pgfpathlineto{\pgfqpoint{4.105344in}{1.844370in}}%
\pgfpathlineto{\pgfqpoint{4.136212in}{1.836879in}}%
\pgfpathlineto{\pgfqpoint{4.182513in}{1.820260in}}%
\pgfpathlineto{\pgfqpoint{4.244247in}{1.797417in}}%
\pgfpathlineto{\pgfqpoint{4.275114in}{1.789671in}}%
\pgfpathlineto{\pgfqpoint{4.305982in}{1.786299in}}%
\pgfpathlineto{\pgfqpoint{4.336849in}{1.788211in}}%
\pgfpathlineto{\pgfqpoint{4.367716in}{1.795775in}}%
\pgfpathlineto{\pgfqpoint{4.398583in}{1.808781in}}%
\pgfpathlineto{\pgfqpoint{4.429451in}{1.826459in}}%
\pgfpathlineto{\pgfqpoint{4.475752in}{1.858911in}}%
\pgfpathlineto{\pgfqpoint{4.537486in}{1.904421in}}%
\pgfpathlineto{\pgfqpoint{4.568353in}{1.924160in}}%
\pgfpathlineto{\pgfqpoint{4.599221in}{1.939837in}}%
\pgfpathlineto{\pgfqpoint{4.630088in}{1.950366in}}%
\pgfpathlineto{\pgfqpoint{4.660955in}{1.955146in}}%
\pgfpathlineto{\pgfqpoint{4.660955in}{1.955146in}}%
\pgfusepath{stroke}%
\end{pgfscope}%
\begin{pgfscope}%
\pgfpathrectangle{\pgfqpoint{0.625831in}{0.505056in}}{\pgfqpoint{4.227273in}{2.745455in}} %
\pgfusepath{clip}%
\pgfsetrectcap%
\pgfsetroundjoin%
\pgfsetlinewidth{0.501875pt}%
\definecolor{currentstroke}{rgb}{1.000000,0.587785,0.309017}%
\pgfsetstrokecolor{currentstroke}%
\pgfsetdash{}{0pt}%
\pgfpathmoveto{\pgfqpoint{0.817980in}{2.252294in}}%
\pgfpathlineto{\pgfqpoint{0.833414in}{2.249599in}}%
\pgfpathlineto{\pgfqpoint{0.848847in}{2.243083in}}%
\pgfpathlineto{\pgfqpoint{0.864281in}{2.232644in}}%
\pgfpathlineto{\pgfqpoint{0.879715in}{2.218242in}}%
\pgfpathlineto{\pgfqpoint{0.895148in}{2.199902in}}%
\pgfpathlineto{\pgfqpoint{0.910582in}{2.177716in}}%
\pgfpathlineto{\pgfqpoint{0.926015in}{2.151841in}}%
\pgfpathlineto{\pgfqpoint{0.956883in}{2.089985in}}%
\pgfpathlineto{\pgfqpoint{0.987750in}{2.016877in}}%
\pgfpathlineto{\pgfqpoint{1.034051in}{1.893920in}}%
\pgfpathlineto{\pgfqpoint{1.095785in}{1.728643in}}%
\pgfpathlineto{\pgfqpoint{1.126653in}{1.656432in}}%
\pgfpathlineto{\pgfqpoint{1.142086in}{1.625007in}}%
\pgfpathlineto{\pgfqpoint{1.157520in}{1.597404in}}%
\pgfpathlineto{\pgfqpoint{1.172954in}{1.574078in}}%
\pgfpathlineto{\pgfqpoint{1.188387in}{1.555417in}}%
\pgfpathlineto{\pgfqpoint{1.203821in}{1.541730in}}%
\pgfpathlineto{\pgfqpoint{1.219255in}{1.533242in}}%
\pgfpathlineto{\pgfqpoint{1.234688in}{1.530090in}}%
\pgfpathlineto{\pgfqpoint{1.250122in}{1.532315in}}%
\pgfpathlineto{\pgfqpoint{1.265555in}{1.539869in}}%
\pgfpathlineto{\pgfqpoint{1.280989in}{1.552609in}}%
\pgfpathlineto{\pgfqpoint{1.296423in}{1.570309in}}%
\pgfpathlineto{\pgfqpoint{1.311856in}{1.592663in}}%
\pgfpathlineto{\pgfqpoint{1.327290in}{1.619293in}}%
\pgfpathlineto{\pgfqpoint{1.358157in}{1.683585in}}%
\pgfpathlineto{\pgfqpoint{1.389025in}{1.759151in}}%
\pgfpathlineto{\pgfqpoint{1.435325in}{1.883847in}}%
\pgfpathlineto{\pgfqpoint{1.481626in}{2.008711in}}%
\pgfpathlineto{\pgfqpoint{1.512494in}{2.085199in}}%
\pgfpathlineto{\pgfqpoint{1.543361in}{2.152220in}}%
\pgfpathlineto{\pgfqpoint{1.558795in}{2.181284in}}%
\pgfpathlineto{\pgfqpoint{1.574228in}{2.206991in}}%
\pgfpathlineto{\pgfqpoint{1.589662in}{2.229104in}}%
\pgfpathlineto{\pgfqpoint{1.605095in}{2.247430in}}%
\pgfpathlineto{\pgfqpoint{1.620529in}{2.261830in}}%
\pgfpathlineto{\pgfqpoint{1.635963in}{2.272211in}}%
\pgfpathlineto{\pgfqpoint{1.651396in}{2.278536in}}%
\pgfpathlineto{\pgfqpoint{1.666830in}{2.280818in}}%
\pgfpathlineto{\pgfqpoint{1.682264in}{2.279124in}}%
\pgfpathlineto{\pgfqpoint{1.697697in}{2.273576in}}%
\pgfpathlineto{\pgfqpoint{1.713131in}{2.264343in}}%
\pgfpathlineto{\pgfqpoint{1.728564in}{2.251646in}}%
\pgfpathlineto{\pgfqpoint{1.743998in}{2.235747in}}%
\pgfpathlineto{\pgfqpoint{1.759432in}{2.216947in}}%
\pgfpathlineto{\pgfqpoint{1.790299in}{2.172010in}}%
\pgfpathlineto{\pgfqpoint{1.821166in}{2.119755in}}%
\pgfpathlineto{\pgfqpoint{1.882901in}{2.005447in}}%
\pgfpathlineto{\pgfqpoint{1.929202in}{1.921997in}}%
\pgfpathlineto{\pgfqpoint{1.960069in}{1.871593in}}%
\pgfpathlineto{\pgfqpoint{1.990936in}{1.827211in}}%
\pgfpathlineto{\pgfqpoint{2.021804in}{1.789949in}}%
\pgfpathlineto{\pgfqpoint{2.052671in}{1.760553in}}%
\pgfpathlineto{\pgfqpoint{2.068104in}{1.748949in}}%
\pgfpathlineto{\pgfqpoint{2.083538in}{1.739445in}}%
\pgfpathlineto{\pgfqpoint{2.098972in}{1.732040in}}%
\pgfpathlineto{\pgfqpoint{2.114405in}{1.726710in}}%
\pgfpathlineto{\pgfqpoint{2.129839in}{1.723403in}}%
\pgfpathlineto{\pgfqpoint{2.145273in}{1.722035in}}%
\pgfpathlineto{\pgfqpoint{2.160706in}{1.722490in}}%
\pgfpathlineto{\pgfqpoint{2.191574in}{1.728232in}}%
\pgfpathlineto{\pgfqpoint{2.222441in}{1.739000in}}%
\pgfpathlineto{\pgfqpoint{2.315043in}{1.778549in}}%
\pgfpathlineto{\pgfqpoint{2.345910in}{1.785791in}}%
\pgfpathlineto{\pgfqpoint{2.361344in}{1.787069in}}%
\pgfpathlineto{\pgfqpoint{2.376777in}{1.786542in}}%
\pgfpathlineto{\pgfqpoint{2.392211in}{1.784094in}}%
\pgfpathlineto{\pgfqpoint{2.407644in}{1.779663in}}%
\pgfpathlineto{\pgfqpoint{2.423078in}{1.773248in}}%
\pgfpathlineto{\pgfqpoint{2.438512in}{1.764900in}}%
\pgfpathlineto{\pgfqpoint{2.469379in}{1.742887in}}%
\pgfpathlineto{\pgfqpoint{2.500246in}{1.715023in}}%
\pgfpathlineto{\pgfqpoint{2.546547in}{1.666625in}}%
\pgfpathlineto{\pgfqpoint{2.608282in}{1.601803in}}%
\pgfpathlineto{\pgfqpoint{2.639149in}{1.574217in}}%
\pgfpathlineto{\pgfqpoint{2.670016in}{1.552296in}}%
\pgfpathlineto{\pgfqpoint{2.685450in}{1.543898in}}%
\pgfpathlineto{\pgfqpoint{2.700884in}{1.537385in}}%
\pgfpathlineto{\pgfqpoint{2.716317in}{1.532853in}}%
\pgfpathlineto{\pgfqpoint{2.731751in}{1.530361in}}%
\pgfpathlineto{\pgfqpoint{2.747184in}{1.529937in}}%
\pgfpathlineto{\pgfqpoint{2.762618in}{1.531569in}}%
\pgfpathlineto{\pgfqpoint{2.778052in}{1.535208in}}%
\pgfpathlineto{\pgfqpoint{2.793485in}{1.540763in}}%
\pgfpathlineto{\pgfqpoint{2.824353in}{1.557062in}}%
\pgfpathlineto{\pgfqpoint{2.855220in}{1.578951in}}%
\pgfpathlineto{\pgfqpoint{2.901521in}{1.617592in}}%
\pgfpathlineto{\pgfqpoint{2.947822in}{1.655507in}}%
\pgfpathlineto{\pgfqpoint{2.978689in}{1.676030in}}%
\pgfpathlineto{\pgfqpoint{2.994123in}{1.684006in}}%
\pgfpathlineto{\pgfqpoint{3.009556in}{1.690158in}}%
\pgfpathlineto{\pgfqpoint{3.024990in}{1.694317in}}%
\pgfpathlineto{\pgfqpoint{3.040424in}{1.696366in}}%
\pgfpathlineto{\pgfqpoint{3.055857in}{1.696239in}}%
\pgfpathlineto{\pgfqpoint{3.071291in}{1.693926in}}%
\pgfpathlineto{\pgfqpoint{3.086724in}{1.689466in}}%
\pgfpathlineto{\pgfqpoint{3.102158in}{1.682947in}}%
\pgfpathlineto{\pgfqpoint{3.117592in}{1.674499in}}%
\pgfpathlineto{\pgfqpoint{3.148459in}{1.652516in}}%
\pgfpathlineto{\pgfqpoint{3.179326in}{1.625177in}}%
\pgfpathlineto{\pgfqpoint{3.225627in}{1.578322in}}%
\pgfpathlineto{\pgfqpoint{3.287362in}{1.514852in}}%
\pgfpathlineto{\pgfqpoint{3.318229in}{1.486360in}}%
\pgfpathlineto{\pgfqpoint{3.349096in}{1.461709in}}%
\pgfpathlineto{\pgfqpoint{3.379963in}{1.441800in}}%
\pgfpathlineto{\pgfqpoint{3.410831in}{1.427245in}}%
\pgfpathlineto{\pgfqpoint{3.441698in}{1.418384in}}%
\pgfpathlineto{\pgfqpoint{3.472565in}{1.415294in}}%
\pgfpathlineto{\pgfqpoint{3.503433in}{1.417817in}}%
\pgfpathlineto{\pgfqpoint{3.534300in}{1.425614in}}%
\pgfpathlineto{\pgfqpoint{3.565167in}{1.438228in}}%
\pgfpathlineto{\pgfqpoint{3.596034in}{1.455177in}}%
\pgfpathlineto{\pgfqpoint{3.626902in}{1.476024in}}%
\pgfpathlineto{\pgfqpoint{3.657769in}{1.500422in}}%
\pgfpathlineto{\pgfqpoint{3.688636in}{1.528115in}}%
\pgfpathlineto{\pgfqpoint{3.719503in}{1.558876in}}%
\pgfpathlineto{\pgfqpoint{3.765804in}{1.610065in}}%
\pgfpathlineto{\pgfqpoint{3.827539in}{1.684418in}}%
\pgfpathlineto{\pgfqpoint{3.889273in}{1.757491in}}%
\pgfpathlineto{\pgfqpoint{3.920141in}{1.789792in}}%
\pgfpathlineto{\pgfqpoint{3.951008in}{1.817228in}}%
\pgfpathlineto{\pgfqpoint{3.981875in}{1.838572in}}%
\pgfpathlineto{\pgfqpoint{3.997309in}{1.846694in}}%
\pgfpathlineto{\pgfqpoint{4.012743in}{1.853039in}}%
\pgfpathlineto{\pgfqpoint{4.028176in}{1.857593in}}%
\pgfpathlineto{\pgfqpoint{4.043610in}{1.860379in}}%
\pgfpathlineto{\pgfqpoint{4.059043in}{1.861452in}}%
\pgfpathlineto{\pgfqpoint{4.089911in}{1.858848in}}%
\pgfpathlineto{\pgfqpoint{4.120778in}{1.850833in}}%
\pgfpathlineto{\pgfqpoint{4.151645in}{1.838834in}}%
\pgfpathlineto{\pgfqpoint{4.259681in}{1.790559in}}%
\pgfpathlineto{\pgfqpoint{4.290548in}{1.782258in}}%
\pgfpathlineto{\pgfqpoint{4.321415in}{1.779138in}}%
\pgfpathlineto{\pgfqpoint{4.336849in}{1.779821in}}%
\pgfpathlineto{\pgfqpoint{4.352283in}{1.782094in}}%
\pgfpathlineto{\pgfqpoint{4.367716in}{1.785979in}}%
\pgfpathlineto{\pgfqpoint{4.398583in}{1.798473in}}%
\pgfpathlineto{\pgfqpoint{4.429451in}{1.816648in}}%
\pgfpathlineto{\pgfqpoint{4.460318in}{1.839213in}}%
\pgfpathlineto{\pgfqpoint{4.568353in}{1.924792in}}%
\pgfpathlineto{\pgfqpoint{4.599221in}{1.942989in}}%
\pgfpathlineto{\pgfqpoint{4.630088in}{1.955856in}}%
\pgfpathlineto{\pgfqpoint{4.660955in}{1.962984in}}%
\pgfpathlineto{\pgfqpoint{4.660955in}{1.962984in}}%
\pgfusepath{stroke}%
\end{pgfscope}%
\begin{pgfscope}%
\pgfpathrectangle{\pgfqpoint{0.625831in}{0.505056in}}{\pgfqpoint{4.227273in}{2.745455in}} %
\pgfusepath{clip}%
\pgfsetrectcap%
\pgfsetroundjoin%
\pgfsetlinewidth{0.501875pt}%
\definecolor{currentstroke}{rgb}{1.000000,0.473094,0.243914}%
\pgfsetstrokecolor{currentstroke}%
\pgfsetdash{}{0pt}%
\pgfpathmoveto{\pgfqpoint{0.817980in}{2.265086in}}%
\pgfpathlineto{\pgfqpoint{0.833414in}{2.263258in}}%
\pgfpathlineto{\pgfqpoint{0.848847in}{2.257583in}}%
\pgfpathlineto{\pgfqpoint{0.864281in}{2.247935in}}%
\pgfpathlineto{\pgfqpoint{0.879715in}{2.234252in}}%
\pgfpathlineto{\pgfqpoint{0.895148in}{2.216536in}}%
\pgfpathlineto{\pgfqpoint{0.910582in}{2.194862in}}%
\pgfpathlineto{\pgfqpoint{0.926015in}{2.169375in}}%
\pgfpathlineto{\pgfqpoint{0.941449in}{2.140296in}}%
\pgfpathlineto{\pgfqpoint{0.972316in}{2.072582in}}%
\pgfpathlineto{\pgfqpoint{1.003184in}{1.994834in}}%
\pgfpathlineto{\pgfqpoint{1.111219in}{1.707135in}}%
\pgfpathlineto{\pgfqpoint{1.142086in}{1.640774in}}%
\pgfpathlineto{\pgfqpoint{1.157520in}{1.613116in}}%
\pgfpathlineto{\pgfqpoint{1.172954in}{1.589747in}}%
\pgfpathlineto{\pgfqpoint{1.188387in}{1.571038in}}%
\pgfpathlineto{\pgfqpoint{1.203821in}{1.557277in}}%
\pgfpathlineto{\pgfqpoint{1.219255in}{1.548672in}}%
\pgfpathlineto{\pgfqpoint{1.234688in}{1.545342in}}%
\pgfpathlineto{\pgfqpoint{1.250122in}{1.547317in}}%
\pgfpathlineto{\pgfqpoint{1.265555in}{1.554537in}}%
\pgfpathlineto{\pgfqpoint{1.280989in}{1.566858in}}%
\pgfpathlineto{\pgfqpoint{1.296423in}{1.584053in}}%
\pgfpathlineto{\pgfqpoint{1.311856in}{1.605821in}}%
\pgfpathlineto{\pgfqpoint{1.327290in}{1.631791in}}%
\pgfpathlineto{\pgfqpoint{1.358157in}{1.694575in}}%
\pgfpathlineto{\pgfqpoint{1.389025in}{1.768440in}}%
\pgfpathlineto{\pgfqpoint{1.435325in}{1.890285in}}%
\pgfpathlineto{\pgfqpoint{1.481626in}{2.011858in}}%
\pgfpathlineto{\pgfqpoint{1.512494in}{2.085822in}}%
\pgfpathlineto{\pgfqpoint{1.543361in}{2.150048in}}%
\pgfpathlineto{\pgfqpoint{1.558795in}{2.177637in}}%
\pgfpathlineto{\pgfqpoint{1.574228in}{2.201845in}}%
\pgfpathlineto{\pgfqpoint{1.589662in}{2.222461in}}%
\pgfpathlineto{\pgfqpoint{1.605095in}{2.239328in}}%
\pgfpathlineto{\pgfqpoint{1.620529in}{2.252342in}}%
\pgfpathlineto{\pgfqpoint{1.635963in}{2.261450in}}%
\pgfpathlineto{\pgfqpoint{1.651396in}{2.266651in}}%
\pgfpathlineto{\pgfqpoint{1.666830in}{2.267992in}}%
\pgfpathlineto{\pgfqpoint{1.682264in}{2.265567in}}%
\pgfpathlineto{\pgfqpoint{1.697697in}{2.259514in}}%
\pgfpathlineto{\pgfqpoint{1.713131in}{2.250011in}}%
\pgfpathlineto{\pgfqpoint{1.728564in}{2.237270in}}%
\pgfpathlineto{\pgfqpoint{1.743998in}{2.221539in}}%
\pgfpathlineto{\pgfqpoint{1.759432in}{2.203089in}}%
\pgfpathlineto{\pgfqpoint{1.790299in}{2.159225in}}%
\pgfpathlineto{\pgfqpoint{1.821166in}{2.108205in}}%
\pgfpathlineto{\pgfqpoint{1.867467in}{2.024018in}}%
\pgfpathlineto{\pgfqpoint{1.929202in}{1.910766in}}%
\pgfpathlineto{\pgfqpoint{1.960069in}{1.858787in}}%
\pgfpathlineto{\pgfqpoint{1.990936in}{1.812406in}}%
\pgfpathlineto{\pgfqpoint{2.021804in}{1.773089in}}%
\pgfpathlineto{\pgfqpoint{2.052671in}{1.741934in}}%
\pgfpathlineto{\pgfqpoint{2.068104in}{1.729649in}}%
\pgfpathlineto{\pgfqpoint{2.083538in}{1.719625in}}%
\pgfpathlineto{\pgfqpoint{2.098972in}{1.711876in}}%
\pgfpathlineto{\pgfqpoint{2.114405in}{1.706379in}}%
\pgfpathlineto{\pgfqpoint{2.129839in}{1.703079in}}%
\pgfpathlineto{\pgfqpoint{2.145273in}{1.701884in}}%
\pgfpathlineto{\pgfqpoint{2.160706in}{1.702666in}}%
\pgfpathlineto{\pgfqpoint{2.176140in}{1.705262in}}%
\pgfpathlineto{\pgfqpoint{2.207007in}{1.715066in}}%
\pgfpathlineto{\pgfqpoint{2.237874in}{1.729338in}}%
\pgfpathlineto{\pgfqpoint{2.315043in}{1.768698in}}%
\pgfpathlineto{\pgfqpoint{2.345910in}{1.779349in}}%
\pgfpathlineto{\pgfqpoint{2.361344in}{1.782476in}}%
\pgfpathlineto{\pgfqpoint{2.376777in}{1.783862in}}%
\pgfpathlineto{\pgfqpoint{2.392211in}{1.783353in}}%
\pgfpathlineto{\pgfqpoint{2.407644in}{1.780844in}}%
\pgfpathlineto{\pgfqpoint{2.423078in}{1.776283in}}%
\pgfpathlineto{\pgfqpoint{2.438512in}{1.769668in}}%
\pgfpathlineto{\pgfqpoint{2.453945in}{1.761049in}}%
\pgfpathlineto{\pgfqpoint{2.484813in}{1.738259in}}%
\pgfpathlineto{\pgfqpoint{2.515680in}{1.709275in}}%
\pgfpathlineto{\pgfqpoint{2.561981in}{1.658618in}}%
\pgfpathlineto{\pgfqpoint{2.608282in}{1.606706in}}%
\pgfpathlineto{\pgfqpoint{2.639149in}{1.575630in}}%
\pgfpathlineto{\pgfqpoint{2.670016in}{1.550040in}}%
\pgfpathlineto{\pgfqpoint{2.685450in}{1.539881in}}%
\pgfpathlineto{\pgfqpoint{2.700884in}{1.531716in}}%
\pgfpathlineto{\pgfqpoint{2.716317in}{1.525671in}}%
\pgfpathlineto{\pgfqpoint{2.731751in}{1.521826in}}%
\pgfpathlineto{\pgfqpoint{2.747184in}{1.520213in}}%
\pgfpathlineto{\pgfqpoint{2.762618in}{1.520818in}}%
\pgfpathlineto{\pgfqpoint{2.778052in}{1.523577in}}%
\pgfpathlineto{\pgfqpoint{2.793485in}{1.528377in}}%
\pgfpathlineto{\pgfqpoint{2.808919in}{1.535062in}}%
\pgfpathlineto{\pgfqpoint{2.839786in}{1.553253in}}%
\pgfpathlineto{\pgfqpoint{2.870654in}{1.576131in}}%
\pgfpathlineto{\pgfqpoint{2.947822in}{1.637321in}}%
\pgfpathlineto{\pgfqpoint{2.978689in}{1.656611in}}%
\pgfpathlineto{\pgfqpoint{2.994123in}{1.664008in}}%
\pgfpathlineto{\pgfqpoint{3.009556in}{1.669621in}}%
\pgfpathlineto{\pgfqpoint{3.024990in}{1.673289in}}%
\pgfpathlineto{\pgfqpoint{3.040424in}{1.674897in}}%
\pgfpathlineto{\pgfqpoint{3.055857in}{1.674382in}}%
\pgfpathlineto{\pgfqpoint{3.071291in}{1.671725in}}%
\pgfpathlineto{\pgfqpoint{3.086724in}{1.666959in}}%
\pgfpathlineto{\pgfqpoint{3.102158in}{1.660163in}}%
\pgfpathlineto{\pgfqpoint{3.133025in}{1.641002in}}%
\pgfpathlineto{\pgfqpoint{3.163893in}{1.615643in}}%
\pgfpathlineto{\pgfqpoint{3.194760in}{1.585924in}}%
\pgfpathlineto{\pgfqpoint{3.287362in}{1.491312in}}%
\pgfpathlineto{\pgfqpoint{3.318229in}{1.464337in}}%
\pgfpathlineto{\pgfqpoint{3.349096in}{1.442083in}}%
\pgfpathlineto{\pgfqpoint{3.379963in}{1.425402in}}%
\pgfpathlineto{\pgfqpoint{3.410831in}{1.414715in}}%
\pgfpathlineto{\pgfqpoint{3.441698in}{1.410052in}}%
\pgfpathlineto{\pgfqpoint{3.472565in}{1.411129in}}%
\pgfpathlineto{\pgfqpoint{3.503433in}{1.417436in}}%
\pgfpathlineto{\pgfqpoint{3.534300in}{1.428346in}}%
\pgfpathlineto{\pgfqpoint{3.565167in}{1.443214in}}%
\pgfpathlineto{\pgfqpoint{3.596034in}{1.461464in}}%
\pgfpathlineto{\pgfqpoint{3.626902in}{1.482646in}}%
\pgfpathlineto{\pgfqpoint{3.673203in}{1.519270in}}%
\pgfpathlineto{\pgfqpoint{3.719503in}{1.561200in}}%
\pgfpathlineto{\pgfqpoint{3.765804in}{1.607833in}}%
\pgfpathlineto{\pgfqpoint{3.842973in}{1.691912in}}%
\pgfpathlineto{\pgfqpoint{3.889273in}{1.741297in}}%
\pgfpathlineto{\pgfqpoint{3.920141in}{1.771164in}}%
\pgfpathlineto{\pgfqpoint{3.951008in}{1.796976in}}%
\pgfpathlineto{\pgfqpoint{3.981875in}{1.817502in}}%
\pgfpathlineto{\pgfqpoint{4.012743in}{1.831820in}}%
\pgfpathlineto{\pgfqpoint{4.028176in}{1.836483in}}%
\pgfpathlineto{\pgfqpoint{4.043610in}{1.839457in}}%
\pgfpathlineto{\pgfqpoint{4.059043in}{1.840769in}}%
\pgfpathlineto{\pgfqpoint{4.089911in}{1.838691in}}%
\pgfpathlineto{\pgfqpoint{4.120778in}{1.831165in}}%
\pgfpathlineto{\pgfqpoint{4.151645in}{1.819604in}}%
\pgfpathlineto{\pgfqpoint{4.244247in}{1.779343in}}%
\pgfpathlineto{\pgfqpoint{4.275114in}{1.770510in}}%
\pgfpathlineto{\pgfqpoint{4.305982in}{1.766637in}}%
\pgfpathlineto{\pgfqpoint{4.336849in}{1.768637in}}%
\pgfpathlineto{\pgfqpoint{4.367716in}{1.776807in}}%
\pgfpathlineto{\pgfqpoint{4.398583in}{1.790795in}}%
\pgfpathlineto{\pgfqpoint{4.429451in}{1.809646in}}%
\pgfpathlineto{\pgfqpoint{4.475752in}{1.843766in}}%
\pgfpathlineto{\pgfqpoint{4.537486in}{1.890554in}}%
\pgfpathlineto{\pgfqpoint{4.568353in}{1.910485in}}%
\pgfpathlineto{\pgfqpoint{4.599221in}{1.926264in}}%
\pgfpathlineto{\pgfqpoint{4.630088in}{1.937106in}}%
\pgfpathlineto{\pgfqpoint{4.660955in}{1.942775in}}%
\pgfpathlineto{\pgfqpoint{4.660955in}{1.942775in}}%
\pgfusepath{stroke}%
\end{pgfscope}%
\begin{pgfscope}%
\pgfpathrectangle{\pgfqpoint{0.625831in}{0.505056in}}{\pgfqpoint{4.227273in}{2.745455in}} %
\pgfusepath{clip}%
\pgfsetrectcap%
\pgfsetroundjoin%
\pgfsetlinewidth{0.501875pt}%
\definecolor{currentstroke}{rgb}{1.000000,0.361242,0.183750}%
\pgfsetstrokecolor{currentstroke}%
\pgfsetdash{}{0pt}%
\pgfpathmoveto{\pgfqpoint{0.817980in}{2.258473in}}%
\pgfpathlineto{\pgfqpoint{0.833414in}{2.256466in}}%
\pgfpathlineto{\pgfqpoint{0.848847in}{2.250818in}}%
\pgfpathlineto{\pgfqpoint{0.864281in}{2.241405in}}%
\pgfpathlineto{\pgfqpoint{0.879715in}{2.228166in}}%
\pgfpathlineto{\pgfqpoint{0.895148in}{2.211105in}}%
\pgfpathlineto{\pgfqpoint{0.910582in}{2.190297in}}%
\pgfpathlineto{\pgfqpoint{0.926015in}{2.165887in}}%
\pgfpathlineto{\pgfqpoint{0.956883in}{2.107189in}}%
\pgfpathlineto{\pgfqpoint{0.987750in}{2.037537in}}%
\pgfpathlineto{\pgfqpoint{1.034051in}{1.920318in}}%
\pgfpathlineto{\pgfqpoint{1.080352in}{1.800872in}}%
\pgfpathlineto{\pgfqpoint{1.111219in}{1.727784in}}%
\pgfpathlineto{\pgfqpoint{1.142086in}{1.665134in}}%
\pgfpathlineto{\pgfqpoint{1.157520in}{1.638950in}}%
\pgfpathlineto{\pgfqpoint{1.172954in}{1.616747in}}%
\pgfpathlineto{\pgfqpoint{1.188387in}{1.598862in}}%
\pgfpathlineto{\pgfqpoint{1.203821in}{1.585568in}}%
\pgfpathlineto{\pgfqpoint{1.219255in}{1.577060in}}%
\pgfpathlineto{\pgfqpoint{1.234688in}{1.573459in}}%
\pgfpathlineto{\pgfqpoint{1.250122in}{1.574810in}}%
\pgfpathlineto{\pgfqpoint{1.265555in}{1.581076in}}%
\pgfpathlineto{\pgfqpoint{1.280989in}{1.592148in}}%
\pgfpathlineto{\pgfqpoint{1.296423in}{1.607840in}}%
\pgfpathlineto{\pgfqpoint{1.311856in}{1.627898in}}%
\pgfpathlineto{\pgfqpoint{1.327290in}{1.652002in}}%
\pgfpathlineto{\pgfqpoint{1.342724in}{1.679775in}}%
\pgfpathlineto{\pgfqpoint{1.373591in}{1.744570in}}%
\pgfpathlineto{\pgfqpoint{1.404458in}{1.818391in}}%
\pgfpathlineto{\pgfqpoint{1.497060in}{2.050838in}}%
\pgfpathlineto{\pgfqpoint{1.527927in}{2.118195in}}%
\pgfpathlineto{\pgfqpoint{1.558795in}{2.174790in}}%
\pgfpathlineto{\pgfqpoint{1.574228in}{2.198296in}}%
\pgfpathlineto{\pgfqpoint{1.589662in}{2.218314in}}%
\pgfpathlineto{\pgfqpoint{1.605095in}{2.234693in}}%
\pgfpathlineto{\pgfqpoint{1.620529in}{2.247338in}}%
\pgfpathlineto{\pgfqpoint{1.635963in}{2.256210in}}%
\pgfpathlineto{\pgfqpoint{1.651396in}{2.261321in}}%
\pgfpathlineto{\pgfqpoint{1.666830in}{2.262731in}}%
\pgfpathlineto{\pgfqpoint{1.682264in}{2.260546in}}%
\pgfpathlineto{\pgfqpoint{1.697697in}{2.254914in}}%
\pgfpathlineto{\pgfqpoint{1.713131in}{2.246014in}}%
\pgfpathlineto{\pgfqpoint{1.728564in}{2.234061in}}%
\pgfpathlineto{\pgfqpoint{1.743998in}{2.219291in}}%
\pgfpathlineto{\pgfqpoint{1.774865in}{2.182351in}}%
\pgfpathlineto{\pgfqpoint{1.805733in}{2.137417in}}%
\pgfpathlineto{\pgfqpoint{1.836600in}{2.086826in}}%
\pgfpathlineto{\pgfqpoint{1.975503in}{1.849226in}}%
\pgfpathlineto{\pgfqpoint{2.006370in}{1.805893in}}%
\pgfpathlineto{\pgfqpoint{2.037237in}{1.769342in}}%
\pgfpathlineto{\pgfqpoint{2.068104in}{1.740545in}}%
\pgfpathlineto{\pgfqpoint{2.083538in}{1.729246in}}%
\pgfpathlineto{\pgfqpoint{2.098972in}{1.720061in}}%
\pgfpathlineto{\pgfqpoint{2.114405in}{1.712983in}}%
\pgfpathlineto{\pgfqpoint{2.129839in}{1.707975in}}%
\pgfpathlineto{\pgfqpoint{2.145273in}{1.704964in}}%
\pgfpathlineto{\pgfqpoint{2.160706in}{1.703846in}}%
\pgfpathlineto{\pgfqpoint{2.176140in}{1.704479in}}%
\pgfpathlineto{\pgfqpoint{2.207007in}{1.710279in}}%
\pgfpathlineto{\pgfqpoint{2.237874in}{1.720642in}}%
\pgfpathlineto{\pgfqpoint{2.330476in}{1.757380in}}%
\pgfpathlineto{\pgfqpoint{2.361344in}{1.764073in}}%
\pgfpathlineto{\pgfqpoint{2.376777in}{1.765304in}}%
\pgfpathlineto{\pgfqpoint{2.392211in}{1.764908in}}%
\pgfpathlineto{\pgfqpoint{2.407644in}{1.762772in}}%
\pgfpathlineto{\pgfqpoint{2.423078in}{1.758827in}}%
\pgfpathlineto{\pgfqpoint{2.438512in}{1.753055in}}%
\pgfpathlineto{\pgfqpoint{2.453945in}{1.745482in}}%
\pgfpathlineto{\pgfqpoint{2.484813in}{1.725280in}}%
\pgfpathlineto{\pgfqpoint{2.515680in}{1.699346in}}%
\pgfpathlineto{\pgfqpoint{2.561981in}{1.653591in}}%
\pgfpathlineto{\pgfqpoint{2.623715in}{1.591587in}}%
\pgfpathlineto{\pgfqpoint{2.654583in}{1.565339in}}%
\pgfpathlineto{\pgfqpoint{2.685450in}{1.544868in}}%
\pgfpathlineto{\pgfqpoint{2.700884in}{1.537238in}}%
\pgfpathlineto{\pgfqpoint{2.716317in}{1.531500in}}%
\pgfpathlineto{\pgfqpoint{2.731751in}{1.527717in}}%
\pgfpathlineto{\pgfqpoint{2.747184in}{1.525907in}}%
\pgfpathlineto{\pgfqpoint{2.762618in}{1.526042in}}%
\pgfpathlineto{\pgfqpoint{2.778052in}{1.528051in}}%
\pgfpathlineto{\pgfqpoint{2.793485in}{1.531820in}}%
\pgfpathlineto{\pgfqpoint{2.824353in}{1.543992in}}%
\pgfpathlineto{\pgfqpoint{2.855220in}{1.560941in}}%
\pgfpathlineto{\pgfqpoint{2.963255in}{1.627167in}}%
\pgfpathlineto{\pgfqpoint{2.994123in}{1.639172in}}%
\pgfpathlineto{\pgfqpoint{3.009556in}{1.642916in}}%
\pgfpathlineto{\pgfqpoint{3.024990in}{1.644966in}}%
\pgfpathlineto{\pgfqpoint{3.040424in}{1.645232in}}%
\pgfpathlineto{\pgfqpoint{3.055857in}{1.643663in}}%
\pgfpathlineto{\pgfqpoint{3.071291in}{1.640251in}}%
\pgfpathlineto{\pgfqpoint{3.086724in}{1.635028in}}%
\pgfpathlineto{\pgfqpoint{3.117592in}{1.619481in}}%
\pgfpathlineto{\pgfqpoint{3.148459in}{1.598043in}}%
\pgfpathlineto{\pgfqpoint{3.179326in}{1.572221in}}%
\pgfpathlineto{\pgfqpoint{3.287362in}{1.474519in}}%
\pgfpathlineto{\pgfqpoint{3.318229in}{1.451972in}}%
\pgfpathlineto{\pgfqpoint{3.349096in}{1.434271in}}%
\pgfpathlineto{\pgfqpoint{3.379963in}{1.422053in}}%
\pgfpathlineto{\pgfqpoint{3.410831in}{1.415493in}}%
\pgfpathlineto{\pgfqpoint{3.441698in}{1.414370in}}%
\pgfpathlineto{\pgfqpoint{3.472565in}{1.418176in}}%
\pgfpathlineto{\pgfqpoint{3.503433in}{1.426243in}}%
\pgfpathlineto{\pgfqpoint{3.534300in}{1.437879in}}%
\pgfpathlineto{\pgfqpoint{3.565167in}{1.452471in}}%
\pgfpathlineto{\pgfqpoint{3.611468in}{1.478961in}}%
\pgfpathlineto{\pgfqpoint{3.657769in}{1.510347in}}%
\pgfpathlineto{\pgfqpoint{3.704070in}{1.546493in}}%
\pgfpathlineto{\pgfqpoint{3.750371in}{1.587299in}}%
\pgfpathlineto{\pgfqpoint{3.812105in}{1.647438in}}%
\pgfpathlineto{\pgfqpoint{3.889273in}{1.723870in}}%
\pgfpathlineto{\pgfqpoint{3.920141in}{1.751365in}}%
\pgfpathlineto{\pgfqpoint{3.951008in}{1.775279in}}%
\pgfpathlineto{\pgfqpoint{3.981875in}{1.794550in}}%
\pgfpathlineto{\pgfqpoint{4.012743in}{1.808404in}}%
\pgfpathlineto{\pgfqpoint{4.043610in}{1.816473in}}%
\pgfpathlineto{\pgfqpoint{4.074477in}{1.818868in}}%
\pgfpathlineto{\pgfqpoint{4.105344in}{1.816206in}}%
\pgfpathlineto{\pgfqpoint{4.136212in}{1.809577in}}%
\pgfpathlineto{\pgfqpoint{4.182513in}{1.795487in}}%
\pgfpathlineto{\pgfqpoint{4.228813in}{1.781659in}}%
\pgfpathlineto{\pgfqpoint{4.259681in}{1.775454in}}%
\pgfpathlineto{\pgfqpoint{4.290548in}{1.773318in}}%
\pgfpathlineto{\pgfqpoint{4.321415in}{1.776192in}}%
\pgfpathlineto{\pgfqpoint{4.352283in}{1.784473in}}%
\pgfpathlineto{\pgfqpoint{4.383150in}{1.797976in}}%
\pgfpathlineto{\pgfqpoint{4.414017in}{1.815955in}}%
\pgfpathlineto{\pgfqpoint{4.460318in}{1.848577in}}%
\pgfpathlineto{\pgfqpoint{4.522052in}{1.894216in}}%
\pgfpathlineto{\pgfqpoint{4.552920in}{1.914299in}}%
\pgfpathlineto{\pgfqpoint{4.583787in}{1.930778in}}%
\pgfpathlineto{\pgfqpoint{4.614654in}{1.942798in}}%
\pgfpathlineto{\pgfqpoint{4.645522in}{1.949976in}}%
\pgfpathlineto{\pgfqpoint{4.660955in}{1.951759in}}%
\pgfpathlineto{\pgfqpoint{4.660955in}{1.951759in}}%
\pgfusepath{stroke}%
\end{pgfscope}%
\begin{pgfscope}%
\pgfpathrectangle{\pgfqpoint{0.625831in}{0.505056in}}{\pgfqpoint{4.227273in}{2.745455in}} %
\pgfusepath{clip}%
\pgfsetrectcap%
\pgfsetroundjoin%
\pgfsetlinewidth{0.501875pt}%
\definecolor{currentstroke}{rgb}{1.000000,0.243914,0.122888}%
\pgfsetstrokecolor{currentstroke}%
\pgfsetdash{}{0pt}%
\pgfpathmoveto{\pgfqpoint{0.817980in}{2.262091in}}%
\pgfpathlineto{\pgfqpoint{0.833414in}{2.261252in}}%
\pgfpathlineto{\pgfqpoint{0.848847in}{2.256668in}}%
\pgfpathlineto{\pgfqpoint{0.864281in}{2.248193in}}%
\pgfpathlineto{\pgfqpoint{0.879715in}{2.235743in}}%
\pgfpathlineto{\pgfqpoint{0.895148in}{2.219304in}}%
\pgfpathlineto{\pgfqpoint{0.910582in}{2.198937in}}%
\pgfpathlineto{\pgfqpoint{0.926015in}{2.174774in}}%
\pgfpathlineto{\pgfqpoint{0.941449in}{2.147022in}}%
\pgfpathlineto{\pgfqpoint{0.972316in}{2.081950in}}%
\pgfpathlineto{\pgfqpoint{1.003184in}{2.006779in}}%
\pgfpathlineto{\pgfqpoint{1.111219in}{1.726487in}}%
\pgfpathlineto{\pgfqpoint{1.142086in}{1.661233in}}%
\pgfpathlineto{\pgfqpoint{1.157520in}{1.633857in}}%
\pgfpathlineto{\pgfqpoint{1.172954in}{1.610568in}}%
\pgfpathlineto{\pgfqpoint{1.188387in}{1.591723in}}%
\pgfpathlineto{\pgfqpoint{1.203821in}{1.577609in}}%
\pgfpathlineto{\pgfqpoint{1.219255in}{1.568435in}}%
\pgfpathlineto{\pgfqpoint{1.234688in}{1.564329in}}%
\pgfpathlineto{\pgfqpoint{1.250122in}{1.565339in}}%
\pgfpathlineto{\pgfqpoint{1.265555in}{1.571432in}}%
\pgfpathlineto{\pgfqpoint{1.280989in}{1.582493in}}%
\pgfpathlineto{\pgfqpoint{1.296423in}{1.598333in}}%
\pgfpathlineto{\pgfqpoint{1.311856in}{1.618690in}}%
\pgfpathlineto{\pgfqpoint{1.327290in}{1.643234in}}%
\pgfpathlineto{\pgfqpoint{1.342724in}{1.671578in}}%
\pgfpathlineto{\pgfqpoint{1.373591in}{1.737856in}}%
\pgfpathlineto{\pgfqpoint{1.404458in}{1.813526in}}%
\pgfpathlineto{\pgfqpoint{1.497060in}{2.052515in}}%
\pgfpathlineto{\pgfqpoint{1.527927in}{2.121958in}}%
\pgfpathlineto{\pgfqpoint{1.558795in}{2.180375in}}%
\pgfpathlineto{\pgfqpoint{1.574228in}{2.204655in}}%
\pgfpathlineto{\pgfqpoint{1.589662in}{2.225339in}}%
\pgfpathlineto{\pgfqpoint{1.605095in}{2.242264in}}%
\pgfpathlineto{\pgfqpoint{1.620529in}{2.255329in}}%
\pgfpathlineto{\pgfqpoint{1.635963in}{2.264486in}}%
\pgfpathlineto{\pgfqpoint{1.651396in}{2.269746in}}%
\pgfpathlineto{\pgfqpoint{1.666830in}{2.271168in}}%
\pgfpathlineto{\pgfqpoint{1.682264in}{2.268863in}}%
\pgfpathlineto{\pgfqpoint{1.697697in}{2.262982in}}%
\pgfpathlineto{\pgfqpoint{1.713131in}{2.253717in}}%
\pgfpathlineto{\pgfqpoint{1.728564in}{2.241292in}}%
\pgfpathlineto{\pgfqpoint{1.743998in}{2.225958in}}%
\pgfpathlineto{\pgfqpoint{1.774865in}{2.187679in}}%
\pgfpathlineto{\pgfqpoint{1.805733in}{2.141240in}}%
\pgfpathlineto{\pgfqpoint{1.836600in}{2.089116in}}%
\pgfpathlineto{\pgfqpoint{1.960069in}{1.871092in}}%
\pgfpathlineto{\pgfqpoint{1.990936in}{1.824582in}}%
\pgfpathlineto{\pgfqpoint{2.021804in}{1.784614in}}%
\pgfpathlineto{\pgfqpoint{2.052671in}{1.752344in}}%
\pgfpathlineto{\pgfqpoint{2.068104in}{1.739352in}}%
\pgfpathlineto{\pgfqpoint{2.083538in}{1.728538in}}%
\pgfpathlineto{\pgfqpoint{2.098972in}{1.719924in}}%
\pgfpathlineto{\pgfqpoint{2.114405in}{1.713499in}}%
\pgfpathlineto{\pgfqpoint{2.129839in}{1.709219in}}%
\pgfpathlineto{\pgfqpoint{2.145273in}{1.707000in}}%
\pgfpathlineto{\pgfqpoint{2.160706in}{1.706725in}}%
\pgfpathlineto{\pgfqpoint{2.176140in}{1.708240in}}%
\pgfpathlineto{\pgfqpoint{2.207007in}{1.715849in}}%
\pgfpathlineto{\pgfqpoint{2.237874in}{1.727948in}}%
\pgfpathlineto{\pgfqpoint{2.315043in}{1.762560in}}%
\pgfpathlineto{\pgfqpoint{2.345910in}{1.771813in}}%
\pgfpathlineto{\pgfqpoint{2.361344in}{1.774385in}}%
\pgfpathlineto{\pgfqpoint{2.376777in}{1.775317in}}%
\pgfpathlineto{\pgfqpoint{2.392211in}{1.774461in}}%
\pgfpathlineto{\pgfqpoint{2.407644in}{1.771711in}}%
\pgfpathlineto{\pgfqpoint{2.423078in}{1.767015in}}%
\pgfpathlineto{\pgfqpoint{2.438512in}{1.760369in}}%
\pgfpathlineto{\pgfqpoint{2.453945in}{1.751821in}}%
\pgfpathlineto{\pgfqpoint{2.484813in}{1.729455in}}%
\pgfpathlineto{\pgfqpoint{2.515680in}{1.701239in}}%
\pgfpathlineto{\pgfqpoint{2.561981in}{1.652271in}}%
\pgfpathlineto{\pgfqpoint{2.608282in}{1.602520in}}%
\pgfpathlineto{\pgfqpoint{2.639149in}{1.573011in}}%
\pgfpathlineto{\pgfqpoint{2.670016in}{1.548956in}}%
\pgfpathlineto{\pgfqpoint{2.685450in}{1.539508in}}%
\pgfpathlineto{\pgfqpoint{2.700884in}{1.531990in}}%
\pgfpathlineto{\pgfqpoint{2.716317in}{1.526507in}}%
\pgfpathlineto{\pgfqpoint{2.731751in}{1.523118in}}%
\pgfpathlineto{\pgfqpoint{2.747184in}{1.521837in}}%
\pgfpathlineto{\pgfqpoint{2.762618in}{1.522631in}}%
\pgfpathlineto{\pgfqpoint{2.778052in}{1.525423in}}%
\pgfpathlineto{\pgfqpoint{2.793485in}{1.530091in}}%
\pgfpathlineto{\pgfqpoint{2.824353in}{1.544371in}}%
\pgfpathlineto{\pgfqpoint{2.855220in}{1.563754in}}%
\pgfpathlineto{\pgfqpoint{2.963255in}{1.638881in}}%
\pgfpathlineto{\pgfqpoint{2.994123in}{1.652719in}}%
\pgfpathlineto{\pgfqpoint{3.009556in}{1.657128in}}%
\pgfpathlineto{\pgfqpoint{3.024990in}{1.659649in}}%
\pgfpathlineto{\pgfqpoint{3.040424in}{1.660170in}}%
\pgfpathlineto{\pgfqpoint{3.055857in}{1.658629in}}%
\pgfpathlineto{\pgfqpoint{3.071291in}{1.655005in}}%
\pgfpathlineto{\pgfqpoint{3.086724in}{1.649330in}}%
\pgfpathlineto{\pgfqpoint{3.102158in}{1.641676in}}%
\pgfpathlineto{\pgfqpoint{3.133025in}{1.620947in}}%
\pgfpathlineto{\pgfqpoint{3.163893in}{1.594220in}}%
\pgfpathlineto{\pgfqpoint{3.210193in}{1.547095in}}%
\pgfpathlineto{\pgfqpoint{3.271928in}{1.482681in}}%
\pgfpathlineto{\pgfqpoint{3.302795in}{1.454503in}}%
\pgfpathlineto{\pgfqpoint{3.333663in}{1.431211in}}%
\pgfpathlineto{\pgfqpoint{3.364530in}{1.413859in}}%
\pgfpathlineto{\pgfqpoint{3.395397in}{1.402972in}}%
\pgfpathlineto{\pgfqpoint{3.410831in}{1.399982in}}%
\pgfpathlineto{\pgfqpoint{3.441698in}{1.398734in}}%
\pgfpathlineto{\pgfqpoint{3.472565in}{1.403339in}}%
\pgfpathlineto{\pgfqpoint{3.503433in}{1.413082in}}%
\pgfpathlineto{\pgfqpoint{3.534300in}{1.427183in}}%
\pgfpathlineto{\pgfqpoint{3.565167in}{1.444908in}}%
\pgfpathlineto{\pgfqpoint{3.596034in}{1.465647in}}%
\pgfpathlineto{\pgfqpoint{3.642335in}{1.501449in}}%
\pgfpathlineto{\pgfqpoint{3.688636in}{1.542030in}}%
\pgfpathlineto{\pgfqpoint{3.734937in}{1.586611in}}%
\pgfpathlineto{\pgfqpoint{3.812105in}{1.666312in}}%
\pgfpathlineto{\pgfqpoint{3.873840in}{1.728464in}}%
\pgfpathlineto{\pgfqpoint{3.904707in}{1.756197in}}%
\pgfpathlineto{\pgfqpoint{3.935574in}{1.780194in}}%
\pgfpathlineto{\pgfqpoint{3.966442in}{1.799446in}}%
\pgfpathlineto{\pgfqpoint{3.997309in}{1.813194in}}%
\pgfpathlineto{\pgfqpoint{4.028176in}{1.821045in}}%
\pgfpathlineto{\pgfqpoint{4.059043in}{1.823053in}}%
\pgfpathlineto{\pgfqpoint{4.089911in}{1.819763in}}%
\pgfpathlineto{\pgfqpoint{4.120778in}{1.812197in}}%
\pgfpathlineto{\pgfqpoint{4.167079in}{1.796047in}}%
\pgfpathlineto{\pgfqpoint{4.213380in}{1.779459in}}%
\pgfpathlineto{\pgfqpoint{4.244247in}{1.771194in}}%
\pgfpathlineto{\pgfqpoint{4.275114in}{1.767023in}}%
\pgfpathlineto{\pgfqpoint{4.305982in}{1.768097in}}%
\pgfpathlineto{\pgfqpoint{4.336849in}{1.775031in}}%
\pgfpathlineto{\pgfqpoint{4.367716in}{1.787831in}}%
\pgfpathlineto{\pgfqpoint{4.398583in}{1.805893in}}%
\pgfpathlineto{\pgfqpoint{4.429451in}{1.828070in}}%
\pgfpathlineto{\pgfqpoint{4.552920in}{1.924221in}}%
\pgfpathlineto{\pgfqpoint{4.583787in}{1.941656in}}%
\pgfpathlineto{\pgfqpoint{4.614654in}{1.954036in}}%
\pgfpathlineto{\pgfqpoint{4.645522in}{1.960975in}}%
\pgfpathlineto{\pgfqpoint{4.660955in}{1.962424in}}%
\pgfpathlineto{\pgfqpoint{4.660955in}{1.962424in}}%
\pgfusepath{stroke}%
\end{pgfscope}%
\begin{pgfscope}%
\pgfpathrectangle{\pgfqpoint{0.625831in}{0.505056in}}{\pgfqpoint{4.227273in}{2.745455in}} %
\pgfusepath{clip}%
\pgfsetrectcap%
\pgfsetroundjoin%
\pgfsetlinewidth{0.501875pt}%
\definecolor{currentstroke}{rgb}{1.000000,0.122888,0.061561}%
\pgfsetstrokecolor{currentstroke}%
\pgfsetdash{}{0pt}%
\pgfpathmoveto{\pgfqpoint{0.817980in}{2.258962in}}%
\pgfpathlineto{\pgfqpoint{0.833414in}{2.257819in}}%
\pgfpathlineto{\pgfqpoint{0.848847in}{2.252982in}}%
\pgfpathlineto{\pgfqpoint{0.864281in}{2.244303in}}%
\pgfpathlineto{\pgfqpoint{0.879715in}{2.231696in}}%
\pgfpathlineto{\pgfqpoint{0.895148in}{2.215144in}}%
\pgfpathlineto{\pgfqpoint{0.910582in}{2.194699in}}%
\pgfpathlineto{\pgfqpoint{0.926015in}{2.170490in}}%
\pgfpathlineto{\pgfqpoint{0.941449in}{2.142719in}}%
\pgfpathlineto{\pgfqpoint{0.972316in}{2.077669in}}%
\pgfpathlineto{\pgfqpoint{1.003184in}{2.002583in}}%
\pgfpathlineto{\pgfqpoint{1.111219in}{1.723468in}}%
\pgfpathlineto{\pgfqpoint{1.142086in}{1.659058in}}%
\pgfpathlineto{\pgfqpoint{1.157520in}{1.632225in}}%
\pgfpathlineto{\pgfqpoint{1.172954in}{1.609559in}}%
\pgfpathlineto{\pgfqpoint{1.188387in}{1.591414in}}%
\pgfpathlineto{\pgfqpoint{1.203821in}{1.578068in}}%
\pgfpathlineto{\pgfqpoint{1.219255in}{1.569718in}}%
\pgfpathlineto{\pgfqpoint{1.234688in}{1.566478in}}%
\pgfpathlineto{\pgfqpoint{1.250122in}{1.568377in}}%
\pgfpathlineto{\pgfqpoint{1.265555in}{1.575362in}}%
\pgfpathlineto{\pgfqpoint{1.280989in}{1.587297in}}%
\pgfpathlineto{\pgfqpoint{1.296423in}{1.603972in}}%
\pgfpathlineto{\pgfqpoint{1.311856in}{1.625105in}}%
\pgfpathlineto{\pgfqpoint{1.327290in}{1.650349in}}%
\pgfpathlineto{\pgfqpoint{1.358157in}{1.711510in}}%
\pgfpathlineto{\pgfqpoint{1.389025in}{1.783697in}}%
\pgfpathlineto{\pgfqpoint{1.435325in}{1.903266in}}%
\pgfpathlineto{\pgfqpoint{1.481626in}{2.023045in}}%
\pgfpathlineto{\pgfqpoint{1.512494in}{2.096029in}}%
\pgfpathlineto{\pgfqpoint{1.543361in}{2.159350in}}%
\pgfpathlineto{\pgfqpoint{1.558795in}{2.186492in}}%
\pgfpathlineto{\pgfqpoint{1.574228in}{2.210252in}}%
\pgfpathlineto{\pgfqpoint{1.589662in}{2.230422in}}%
\pgfpathlineto{\pgfqpoint{1.605095in}{2.246853in}}%
\pgfpathlineto{\pgfqpoint{1.620529in}{2.259455in}}%
\pgfpathlineto{\pgfqpoint{1.635963in}{2.268193in}}%
\pgfpathlineto{\pgfqpoint{1.651396in}{2.273089in}}%
\pgfpathlineto{\pgfqpoint{1.666830in}{2.274217in}}%
\pgfpathlineto{\pgfqpoint{1.682264in}{2.271695in}}%
\pgfpathlineto{\pgfqpoint{1.697697in}{2.265687in}}%
\pgfpathlineto{\pgfqpoint{1.713131in}{2.256392in}}%
\pgfpathlineto{\pgfqpoint{1.728564in}{2.244040in}}%
\pgfpathlineto{\pgfqpoint{1.743998in}{2.228888in}}%
\pgfpathlineto{\pgfqpoint{1.774865in}{2.191294in}}%
\pgfpathlineto{\pgfqpoint{1.805733in}{2.145941in}}%
\pgfpathlineto{\pgfqpoint{1.852034in}{2.068596in}}%
\pgfpathlineto{\pgfqpoint{1.944635in}{1.907991in}}%
\pgfpathlineto{\pgfqpoint{1.975503in}{1.859866in}}%
\pgfpathlineto{\pgfqpoint{2.006370in}{1.817148in}}%
\pgfpathlineto{\pgfqpoint{2.037237in}{1.781103in}}%
\pgfpathlineto{\pgfqpoint{2.068104in}{1.752700in}}%
\pgfpathlineto{\pgfqpoint{2.083538in}{1.741568in}}%
\pgfpathlineto{\pgfqpoint{2.098972in}{1.732538in}}%
\pgfpathlineto{\pgfqpoint{2.114405in}{1.725613in}}%
\pgfpathlineto{\pgfqpoint{2.129839in}{1.720764in}}%
\pgfpathlineto{\pgfqpoint{2.145273in}{1.717927in}}%
\pgfpathlineto{\pgfqpoint{2.160706in}{1.717003in}}%
\pgfpathlineto{\pgfqpoint{2.176140in}{1.717854in}}%
\pgfpathlineto{\pgfqpoint{2.207007in}{1.724161in}}%
\pgfpathlineto{\pgfqpoint{2.237874in}{1.735075in}}%
\pgfpathlineto{\pgfqpoint{2.315043in}{1.767531in}}%
\pgfpathlineto{\pgfqpoint{2.345910in}{1.776202in}}%
\pgfpathlineto{\pgfqpoint{2.361344in}{1.778518in}}%
\pgfpathlineto{\pgfqpoint{2.376777in}{1.779210in}}%
\pgfpathlineto{\pgfqpoint{2.392211in}{1.778125in}}%
\pgfpathlineto{\pgfqpoint{2.407644in}{1.775156in}}%
\pgfpathlineto{\pgfqpoint{2.423078in}{1.770248in}}%
\pgfpathlineto{\pgfqpoint{2.438512in}{1.763398in}}%
\pgfpathlineto{\pgfqpoint{2.453945in}{1.754656in}}%
\pgfpathlineto{\pgfqpoint{2.484813in}{1.731949in}}%
\pgfpathlineto{\pgfqpoint{2.515680in}{1.703476in}}%
\pgfpathlineto{\pgfqpoint{2.561981in}{1.654321in}}%
\pgfpathlineto{\pgfqpoint{2.608282in}{1.604584in}}%
\pgfpathlineto{\pgfqpoint{2.639149in}{1.575138in}}%
\pgfpathlineto{\pgfqpoint{2.670016in}{1.551145in}}%
\pgfpathlineto{\pgfqpoint{2.685450in}{1.541721in}}%
\pgfpathlineto{\pgfqpoint{2.700884in}{1.534223in}}%
\pgfpathlineto{\pgfqpoint{2.716317in}{1.528756in}}%
\pgfpathlineto{\pgfqpoint{2.731751in}{1.525386in}}%
\pgfpathlineto{\pgfqpoint{2.747184in}{1.524132in}}%
\pgfpathlineto{\pgfqpoint{2.762618in}{1.524968in}}%
\pgfpathlineto{\pgfqpoint{2.778052in}{1.527823in}}%
\pgfpathlineto{\pgfqpoint{2.793485in}{1.532582in}}%
\pgfpathlineto{\pgfqpoint{2.824353in}{1.547150in}}%
\pgfpathlineto{\pgfqpoint{2.855220in}{1.566977in}}%
\pgfpathlineto{\pgfqpoint{2.963255in}{1.644405in}}%
\pgfpathlineto{\pgfqpoint{2.994123in}{1.658845in}}%
\pgfpathlineto{\pgfqpoint{3.009556in}{1.663512in}}%
\pgfpathlineto{\pgfqpoint{3.024990in}{1.666255in}}%
\pgfpathlineto{\pgfqpoint{3.040424in}{1.666967in}}%
\pgfpathlineto{\pgfqpoint{3.055857in}{1.665584in}}%
\pgfpathlineto{\pgfqpoint{3.071291in}{1.662094in}}%
\pgfpathlineto{\pgfqpoint{3.086724in}{1.656531in}}%
\pgfpathlineto{\pgfqpoint{3.102158in}{1.648977in}}%
\pgfpathlineto{\pgfqpoint{3.133025in}{1.628444in}}%
\pgfpathlineto{\pgfqpoint{3.163893in}{1.601965in}}%
\pgfpathlineto{\pgfqpoint{3.210193in}{1.555429in}}%
\pgfpathlineto{\pgfqpoint{3.271928in}{1.492298in}}%
\pgfpathlineto{\pgfqpoint{3.302795in}{1.464891in}}%
\pgfpathlineto{\pgfqpoint{3.333663in}{1.442327in}}%
\pgfpathlineto{\pgfqpoint{3.364530in}{1.425518in}}%
\pgfpathlineto{\pgfqpoint{3.395397in}{1.414833in}}%
\pgfpathlineto{\pgfqpoint{3.426264in}{1.410161in}}%
\pgfpathlineto{\pgfqpoint{3.457132in}{1.411033in}}%
\pgfpathlineto{\pgfqpoint{3.487999in}{1.416773in}}%
\pgfpathlineto{\pgfqpoint{3.518866in}{1.426658in}}%
\pgfpathlineto{\pgfqpoint{3.549733in}{1.440042in}}%
\pgfpathlineto{\pgfqpoint{3.580601in}{1.456444in}}%
\pgfpathlineto{\pgfqpoint{3.611468in}{1.475577in}}%
\pgfpathlineto{\pgfqpoint{3.642335in}{1.497323in}}%
\pgfpathlineto{\pgfqpoint{3.673203in}{1.521667in}}%
\pgfpathlineto{\pgfqpoint{3.719503in}{1.563020in}}%
\pgfpathlineto{\pgfqpoint{3.765804in}{1.609700in}}%
\pgfpathlineto{\pgfqpoint{3.827539in}{1.677366in}}%
\pgfpathlineto{\pgfqpoint{3.889273in}{1.744312in}}%
\pgfpathlineto{\pgfqpoint{3.920141in}{1.774352in}}%
\pgfpathlineto{\pgfqpoint{3.951008in}{1.800314in}}%
\pgfpathlineto{\pgfqpoint{3.981875in}{1.821041in}}%
\pgfpathlineto{\pgfqpoint{4.012743in}{1.835693in}}%
\pgfpathlineto{\pgfqpoint{4.028176in}{1.840597in}}%
\pgfpathlineto{\pgfqpoint{4.043610in}{1.843874in}}%
\pgfpathlineto{\pgfqpoint{4.074477in}{1.845712in}}%
\pgfpathlineto{\pgfqpoint{4.105344in}{1.841881in}}%
\pgfpathlineto{\pgfqpoint{4.136212in}{1.833561in}}%
\pgfpathlineto{\pgfqpoint{4.182513in}{1.816213in}}%
\pgfpathlineto{\pgfqpoint{4.228813in}{1.798667in}}%
\pgfpathlineto{\pgfqpoint{4.259681in}{1.789974in}}%
\pgfpathlineto{\pgfqpoint{4.290548in}{1.785527in}}%
\pgfpathlineto{\pgfqpoint{4.321415in}{1.786423in}}%
\pgfpathlineto{\pgfqpoint{4.352283in}{1.793205in}}%
\pgfpathlineto{\pgfqpoint{4.383150in}{1.805796in}}%
\pgfpathlineto{\pgfqpoint{4.414017in}{1.823510in}}%
\pgfpathlineto{\pgfqpoint{4.444884in}{1.845117in}}%
\pgfpathlineto{\pgfqpoint{4.552920in}{1.926495in}}%
\pgfpathlineto{\pgfqpoint{4.583787in}{1.944247in}}%
\pgfpathlineto{\pgfqpoint{4.614654in}{1.957241in}}%
\pgfpathlineto{\pgfqpoint{4.645522in}{1.965011in}}%
\pgfpathlineto{\pgfqpoint{4.660955in}{1.966941in}}%
\pgfpathlineto{\pgfqpoint{4.660955in}{1.966941in}}%
\pgfusepath{stroke}%
\end{pgfscope}%
\begin{pgfscope}%
\pgfpathrectangle{\pgfqpoint{0.625831in}{0.505056in}}{\pgfqpoint{4.227273in}{2.745455in}} %
\pgfusepath{clip}%
\pgfsetrectcap%
\pgfsetroundjoin%
\pgfsetlinewidth{0.501875pt}%
\definecolor{currentstroke}{rgb}{0.000000,0.000000,0.000000}%
\pgfsetstrokecolor{currentstroke}%
\pgfsetdash{}{0pt}%
\pgfpathmoveto{\pgfqpoint{0.817980in}{2.258012in}}%
\pgfpathlineto{\pgfqpoint{0.833414in}{2.256655in}}%
\pgfpathlineto{\pgfqpoint{0.848847in}{2.251542in}}%
\pgfpathlineto{\pgfqpoint{0.864281in}{2.242529in}}%
\pgfpathlineto{\pgfqpoint{0.879715in}{2.229538in}}%
\pgfpathlineto{\pgfqpoint{0.895148in}{2.212558in}}%
\pgfpathlineto{\pgfqpoint{0.910582in}{2.191654in}}%
\pgfpathlineto{\pgfqpoint{0.926015in}{2.166965in}}%
\pgfpathlineto{\pgfqpoint{0.941449in}{2.138704in}}%
\pgfpathlineto{\pgfqpoint{0.972316in}{2.072686in}}%
\pgfpathlineto{\pgfqpoint{1.003184in}{1.996704in}}%
\pgfpathlineto{\pgfqpoint{1.111219in}{1.715156in}}%
\pgfpathlineto{\pgfqpoint{1.142086in}{1.650113in}}%
\pgfpathlineto{\pgfqpoint{1.157520in}{1.622952in}}%
\pgfpathlineto{\pgfqpoint{1.172954in}{1.599954in}}%
\pgfpathlineto{\pgfqpoint{1.188387in}{1.581476in}}%
\pgfpathlineto{\pgfqpoint{1.203821in}{1.567805in}}%
\pgfpathlineto{\pgfqpoint{1.219255in}{1.559148in}}%
\pgfpathlineto{\pgfqpoint{1.234688in}{1.555631in}}%
\pgfpathlineto{\pgfqpoint{1.250122in}{1.557297in}}%
\pgfpathlineto{\pgfqpoint{1.265555in}{1.564105in}}%
\pgfpathlineto{\pgfqpoint{1.280989in}{1.575934in}}%
\pgfpathlineto{\pgfqpoint{1.296423in}{1.592584in}}%
\pgfpathlineto{\pgfqpoint{1.311856in}{1.613780in}}%
\pgfpathlineto{\pgfqpoint{1.327290in}{1.639183in}}%
\pgfpathlineto{\pgfqpoint{1.342724in}{1.668389in}}%
\pgfpathlineto{\pgfqpoint{1.373591in}{1.736354in}}%
\pgfpathlineto{\pgfqpoint{1.404458in}{1.813575in}}%
\pgfpathlineto{\pgfqpoint{1.497060in}{2.055740in}}%
\pgfpathlineto{\pgfqpoint{1.527927in}{2.125695in}}%
\pgfpathlineto{\pgfqpoint{1.558795in}{2.184398in}}%
\pgfpathlineto{\pgfqpoint{1.574228in}{2.208753in}}%
\pgfpathlineto{\pgfqpoint{1.589662in}{2.229477in}}%
\pgfpathlineto{\pgfqpoint{1.605095in}{2.246415in}}%
\pgfpathlineto{\pgfqpoint{1.620529in}{2.259471in}}%
\pgfpathlineto{\pgfqpoint{1.635963in}{2.268606in}}%
\pgfpathlineto{\pgfqpoint{1.651396in}{2.273836in}}%
\pgfpathlineto{\pgfqpoint{1.666830in}{2.275227in}}%
\pgfpathlineto{\pgfqpoint{1.682264in}{2.272894in}}%
\pgfpathlineto{\pgfqpoint{1.697697in}{2.266997in}}%
\pgfpathlineto{\pgfqpoint{1.713131in}{2.257730in}}%
\pgfpathlineto{\pgfqpoint{1.728564in}{2.245325in}}%
\pgfpathlineto{\pgfqpoint{1.743998in}{2.230038in}}%
\pgfpathlineto{\pgfqpoint{1.774865in}{2.191954in}}%
\pgfpathlineto{\pgfqpoint{1.805733in}{2.145877in}}%
\pgfpathlineto{\pgfqpoint{1.852034in}{2.067235in}}%
\pgfpathlineto{\pgfqpoint{1.944635in}{1.904643in}}%
\pgfpathlineto{\pgfqpoint{1.975503in}{1.856280in}}%
\pgfpathlineto{\pgfqpoint{2.006370in}{1.813538in}}%
\pgfpathlineto{\pgfqpoint{2.037237in}{1.777636in}}%
\pgfpathlineto{\pgfqpoint{2.068104in}{1.749482in}}%
\pgfpathlineto{\pgfqpoint{2.083538in}{1.738495in}}%
\pgfpathlineto{\pgfqpoint{2.098972in}{1.729614in}}%
\pgfpathlineto{\pgfqpoint{2.114405in}{1.722837in}}%
\pgfpathlineto{\pgfqpoint{2.129839in}{1.718130in}}%
\pgfpathlineto{\pgfqpoint{2.145273in}{1.715425in}}%
\pgfpathlineto{\pgfqpoint{2.160706in}{1.714620in}}%
\pgfpathlineto{\pgfqpoint{2.176140in}{1.715574in}}%
\pgfpathlineto{\pgfqpoint{2.207007in}{1.722033in}}%
\pgfpathlineto{\pgfqpoint{2.237874in}{1.733027in}}%
\pgfpathlineto{\pgfqpoint{2.315043in}{1.765430in}}%
\pgfpathlineto{\pgfqpoint{2.345910in}{1.774049in}}%
\pgfpathlineto{\pgfqpoint{2.361344in}{1.776355in}}%
\pgfpathlineto{\pgfqpoint{2.376777in}{1.777057in}}%
\pgfpathlineto{\pgfqpoint{2.392211in}{1.776009in}}%
\pgfpathlineto{\pgfqpoint{2.407644in}{1.773111in}}%
\pgfpathlineto{\pgfqpoint{2.423078in}{1.768317in}}%
\pgfpathlineto{\pgfqpoint{2.438512in}{1.761631in}}%
\pgfpathlineto{\pgfqpoint{2.469379in}{1.742859in}}%
\pgfpathlineto{\pgfqpoint{2.500246in}{1.717819in}}%
\pgfpathlineto{\pgfqpoint{2.531114in}{1.688192in}}%
\pgfpathlineto{\pgfqpoint{2.623715in}{1.594138in}}%
\pgfpathlineto{\pgfqpoint{2.654583in}{1.568745in}}%
\pgfpathlineto{\pgfqpoint{2.685450in}{1.549551in}}%
\pgfpathlineto{\pgfqpoint{2.700884in}{1.542682in}}%
\pgfpathlineto{\pgfqpoint{2.716317in}{1.537772in}}%
\pgfpathlineto{\pgfqpoint{2.731751in}{1.534872in}}%
\pgfpathlineto{\pgfqpoint{2.747184in}{1.533995in}}%
\pgfpathlineto{\pgfqpoint{2.762618in}{1.535109in}}%
\pgfpathlineto{\pgfqpoint{2.778052in}{1.538140in}}%
\pgfpathlineto{\pgfqpoint{2.793485in}{1.542973in}}%
\pgfpathlineto{\pgfqpoint{2.824353in}{1.557393in}}%
\pgfpathlineto{\pgfqpoint{2.855220in}{1.576724in}}%
\pgfpathlineto{\pgfqpoint{2.947822in}{1.642031in}}%
\pgfpathlineto{\pgfqpoint{2.978689in}{1.658333in}}%
\pgfpathlineto{\pgfqpoint{2.994123in}{1.664267in}}%
\pgfpathlineto{\pgfqpoint{3.009556in}{1.668462in}}%
\pgfpathlineto{\pgfqpoint{3.024990in}{1.670774in}}%
\pgfpathlineto{\pgfqpoint{3.040424in}{1.671102in}}%
\pgfpathlineto{\pgfqpoint{3.055857in}{1.669392in}}%
\pgfpathlineto{\pgfqpoint{3.071291in}{1.665637in}}%
\pgfpathlineto{\pgfqpoint{3.086724in}{1.659880in}}%
\pgfpathlineto{\pgfqpoint{3.102158in}{1.652207in}}%
\pgfpathlineto{\pgfqpoint{3.133025in}{1.631670in}}%
\pgfpathlineto{\pgfqpoint{3.163893in}{1.605493in}}%
\pgfpathlineto{\pgfqpoint{3.210193in}{1.559865in}}%
\pgfpathlineto{\pgfqpoint{3.271928in}{1.498238in}}%
\pgfpathlineto{\pgfqpoint{3.302795in}{1.471426in}}%
\pgfpathlineto{\pgfqpoint{3.333663in}{1.449227in}}%
\pgfpathlineto{\pgfqpoint{3.364530in}{1.432521in}}%
\pgfpathlineto{\pgfqpoint{3.395397in}{1.421704in}}%
\pgfpathlineto{\pgfqpoint{3.426264in}{1.416738in}}%
\pgfpathlineto{\pgfqpoint{3.457132in}{1.417245in}}%
\pgfpathlineto{\pgfqpoint{3.487999in}{1.422629in}}%
\pgfpathlineto{\pgfqpoint{3.518866in}{1.432213in}}%
\pgfpathlineto{\pgfqpoint{3.549733in}{1.445358in}}%
\pgfpathlineto{\pgfqpoint{3.580601in}{1.461555in}}%
\pgfpathlineto{\pgfqpoint{3.611468in}{1.480484in}}%
\pgfpathlineto{\pgfqpoint{3.642335in}{1.502006in}}%
\pgfpathlineto{\pgfqpoint{3.673203in}{1.526116in}}%
\pgfpathlineto{\pgfqpoint{3.719503in}{1.567190in}}%
\pgfpathlineto{\pgfqpoint{3.765804in}{1.613834in}}%
\pgfpathlineto{\pgfqpoint{3.827539in}{1.681925in}}%
\pgfpathlineto{\pgfqpoint{3.889273in}{1.749466in}}%
\pgfpathlineto{\pgfqpoint{3.920141in}{1.779655in}}%
\pgfpathlineto{\pgfqpoint{3.951008in}{1.805583in}}%
\pgfpathlineto{\pgfqpoint{3.981875in}{1.826071in}}%
\pgfpathlineto{\pgfqpoint{4.012743in}{1.840298in}}%
\pgfpathlineto{\pgfqpoint{4.028176in}{1.844933in}}%
\pgfpathlineto{\pgfqpoint{4.043610in}{1.847910in}}%
\pgfpathlineto{\pgfqpoint{4.074477in}{1.849076in}}%
\pgfpathlineto{\pgfqpoint{4.105344in}{1.844498in}}%
\pgfpathlineto{\pgfqpoint{4.136212in}{1.835365in}}%
\pgfpathlineto{\pgfqpoint{4.182513in}{1.816653in}}%
\pgfpathlineto{\pgfqpoint{4.228813in}{1.797564in}}%
\pgfpathlineto{\pgfqpoint{4.259681in}{1.787782in}}%
\pgfpathlineto{\pgfqpoint{4.290548in}{1.782263in}}%
\pgfpathlineto{\pgfqpoint{4.321415in}{1.782189in}}%
\pgfpathlineto{\pgfqpoint{4.352283in}{1.788187in}}%
\pgfpathlineto{\pgfqpoint{4.383150in}{1.800255in}}%
\pgfpathlineto{\pgfqpoint{4.414017in}{1.817750in}}%
\pgfpathlineto{\pgfqpoint{4.444884in}{1.839456in}}%
\pgfpathlineto{\pgfqpoint{4.568353in}{1.932981in}}%
\pgfpathlineto{\pgfqpoint{4.599221in}{1.949352in}}%
\pgfpathlineto{\pgfqpoint{4.630088in}{1.960588in}}%
\pgfpathlineto{\pgfqpoint{4.660955in}{1.966422in}}%
\pgfpathlineto{\pgfqpoint{4.660955in}{1.966422in}}%
\pgfusepath{stroke}%
\end{pgfscope}%
\begin{pgfscope}%
\pgfsetrectcap%
\pgfsetmiterjoin%
\pgfsetlinewidth{0.803000pt}%
\definecolor{currentstroke}{rgb}{0.000000,0.000000,0.000000}%
\pgfsetstrokecolor{currentstroke}%
\pgfsetdash{}{0pt}%
\pgfpathmoveto{\pgfqpoint{0.625831in}{0.505056in}}%
\pgfpathlineto{\pgfqpoint{0.625831in}{3.250511in}}%
\pgfusepath{stroke}%
\end{pgfscope}%
\begin{pgfscope}%
\pgfsetrectcap%
\pgfsetmiterjoin%
\pgfsetlinewidth{0.803000pt}%
\definecolor{currentstroke}{rgb}{0.000000,0.000000,0.000000}%
\pgfsetstrokecolor{currentstroke}%
\pgfsetdash{}{0pt}%
\pgfpathmoveto{\pgfqpoint{4.853104in}{0.505056in}}%
\pgfpathlineto{\pgfqpoint{4.853104in}{3.250511in}}%
\pgfusepath{stroke}%
\end{pgfscope}%
\begin{pgfscope}%
\pgfsetrectcap%
\pgfsetmiterjoin%
\pgfsetlinewidth{0.803000pt}%
\definecolor{currentstroke}{rgb}{0.000000,0.000000,0.000000}%
\pgfsetstrokecolor{currentstroke}%
\pgfsetdash{}{0pt}%
\pgfpathmoveto{\pgfqpoint{0.625831in}{0.505056in}}%
\pgfpathlineto{\pgfqpoint{4.853104in}{0.505056in}}%
\pgfusepath{stroke}%
\end{pgfscope}%
\begin{pgfscope}%
\pgfsetrectcap%
\pgfsetmiterjoin%
\pgfsetlinewidth{0.803000pt}%
\definecolor{currentstroke}{rgb}{0.000000,0.000000,0.000000}%
\pgfsetstrokecolor{currentstroke}%
\pgfsetdash{}{0pt}%
\pgfpathmoveto{\pgfqpoint{0.625831in}{3.250511in}}%
\pgfpathlineto{\pgfqpoint{4.853104in}{3.250511in}}%
\pgfusepath{stroke}%
\end{pgfscope}%
\begin{pgfscope}%
\pgftext[x=5.275831in,y=3.333844in,,base]{\rmfamily\fontsize{12.000000}{14.400000}\selectfont \(\displaystyle \widetilde{K}u \approx Ku\), realization 2}%
\end{pgfscope}%
\begin{pgfscope}%
\pgfsetbuttcap%
\pgfsetmiterjoin%
\definecolor{currentfill}{rgb}{1.000000,1.000000,1.000000}%
\pgfsetfillcolor{currentfill}%
\pgfsetfillopacity{0.800000}%
\pgfsetlinewidth{1.003750pt}%
\definecolor{currentstroke}{rgb}{0.800000,0.800000,0.800000}%
\pgfsetstrokecolor{currentstroke}%
\pgfsetstrokeopacity{0.800000}%
\pgfsetdash{}{0pt}%
\pgfpathmoveto{\pgfqpoint{4.211136in}{0.574501in}}%
\pgfpathlineto{\pgfqpoint{4.755882in}{0.574501in}}%
\pgfpathquadraticcurveto{\pgfqpoint{4.783660in}{0.574501in}}{\pgfqpoint{4.783660in}{0.602278in}}%
\pgfpathlineto{\pgfqpoint{4.783660in}{0.784760in}}%
\pgfpathquadraticcurveto{\pgfqpoint{4.783660in}{0.812538in}}{\pgfqpoint{4.755882in}{0.812538in}}%
\pgfpathlineto{\pgfqpoint{4.211136in}{0.812538in}}%
\pgfpathquadraticcurveto{\pgfqpoint{4.183358in}{0.812538in}}{\pgfqpoint{4.183358in}{0.784760in}}%
\pgfpathlineto{\pgfqpoint{4.183358in}{0.602278in}}%
\pgfpathquadraticcurveto{\pgfqpoint{4.183358in}{0.574501in}}{\pgfqpoint{4.211136in}{0.574501in}}%
\pgfpathclose%
\pgfusepath{stroke,fill}%
\end{pgfscope}%
\begin{pgfscope}%
\pgfsetrectcap%
\pgfsetroundjoin%
\pgfsetlinewidth{0.501875pt}%
\definecolor{currentstroke}{rgb}{0.000000,0.000000,0.000000}%
\pgfsetstrokecolor{currentstroke}%
\pgfsetdash{}{0pt}%
\pgfpathmoveto{\pgfqpoint{4.238914in}{0.708371in}}%
\pgfpathlineto{\pgfqpoint{4.516692in}{0.708371in}}%
\pgfusepath{stroke}%
\end{pgfscope}%
\begin{pgfscope}%
\pgftext[x=4.627803in,y=0.659760in,left,base]{\rmfamily\fontsize{10.000000}{12.000000}\selectfont K}%
\end{pgfscope}%
\begin{pgfscope}%
\pgfsetbuttcap%
\pgfsetmiterjoin%
\definecolor{currentfill}{rgb}{1.000000,1.000000,1.000000}%
\pgfsetfillcolor{currentfill}%
\pgfsetlinewidth{0.000000pt}%
\definecolor{currentstroke}{rgb}{0.000000,0.000000,0.000000}%
\pgfsetstrokecolor{currentstroke}%
\pgfsetstrokeopacity{0.000000}%
\pgfsetdash{}{0pt}%
\pgfpathmoveto{\pgfqpoint{5.698559in}{0.505056in}}%
\pgfpathlineto{\pgfqpoint{9.925831in}{0.505056in}}%
\pgfpathlineto{\pgfqpoint{9.925831in}{3.250511in}}%
\pgfpathlineto{\pgfqpoint{5.698559in}{3.250511in}}%
\pgfpathclose%
\pgfusepath{fill}%
\end{pgfscope}%
\begin{pgfscope}%
\pgfsetbuttcap%
\pgfsetroundjoin%
\definecolor{currentfill}{rgb}{0.000000,0.000000,0.000000}%
\pgfsetfillcolor{currentfill}%
\pgfsetlinewidth{0.803000pt}%
\definecolor{currentstroke}{rgb}{0.000000,0.000000,0.000000}%
\pgfsetstrokecolor{currentstroke}%
\pgfsetdash{}{0pt}%
\pgfsys@defobject{currentmarker}{\pgfqpoint{0.000000in}{-0.048611in}}{\pgfqpoint{0.000000in}{0.000000in}}{%
\pgfpathmoveto{\pgfqpoint{0.000000in}{0.000000in}}%
\pgfpathlineto{\pgfqpoint{0.000000in}{-0.048611in}}%
\pgfusepath{stroke,fill}%
}%
\begin{pgfscope}%
\pgfsys@transformshift{5.890707in}{0.505056in}%
\pgfsys@useobject{currentmarker}{}%
\end{pgfscope}%
\end{pgfscope}%
\begin{pgfscope}%
\pgftext[x=5.890707in,y=0.407834in,,top]{\rmfamily\fontsize{10.000000}{12.000000}\selectfont \(\displaystyle -1.00\)}%
\end{pgfscope}%
\begin{pgfscope}%
\pgfsetbuttcap%
\pgfsetroundjoin%
\definecolor{currentfill}{rgb}{0.000000,0.000000,0.000000}%
\pgfsetfillcolor{currentfill}%
\pgfsetlinewidth{0.803000pt}%
\definecolor{currentstroke}{rgb}{0.000000,0.000000,0.000000}%
\pgfsetstrokecolor{currentstroke}%
\pgfsetdash{}{0pt}%
\pgfsys@defobject{currentmarker}{\pgfqpoint{0.000000in}{-0.048611in}}{\pgfqpoint{0.000000in}{0.000000in}}{%
\pgfpathmoveto{\pgfqpoint{0.000000in}{0.000000in}}%
\pgfpathlineto{\pgfqpoint{0.000000in}{-0.048611in}}%
\pgfusepath{stroke,fill}%
}%
\begin{pgfscope}%
\pgfsys@transformshift{6.371079in}{0.505056in}%
\pgfsys@useobject{currentmarker}{}%
\end{pgfscope}%
\end{pgfscope}%
\begin{pgfscope}%
\pgftext[x=6.371079in,y=0.407834in,,top]{\rmfamily\fontsize{10.000000}{12.000000}\selectfont \(\displaystyle -0.75\)}%
\end{pgfscope}%
\begin{pgfscope}%
\pgfsetbuttcap%
\pgfsetroundjoin%
\definecolor{currentfill}{rgb}{0.000000,0.000000,0.000000}%
\pgfsetfillcolor{currentfill}%
\pgfsetlinewidth{0.803000pt}%
\definecolor{currentstroke}{rgb}{0.000000,0.000000,0.000000}%
\pgfsetstrokecolor{currentstroke}%
\pgfsetdash{}{0pt}%
\pgfsys@defobject{currentmarker}{\pgfqpoint{0.000000in}{-0.048611in}}{\pgfqpoint{0.000000in}{0.000000in}}{%
\pgfpathmoveto{\pgfqpoint{0.000000in}{0.000000in}}%
\pgfpathlineto{\pgfqpoint{0.000000in}{-0.048611in}}%
\pgfusepath{stroke,fill}%
}%
\begin{pgfscope}%
\pgfsys@transformshift{6.851451in}{0.505056in}%
\pgfsys@useobject{currentmarker}{}%
\end{pgfscope}%
\end{pgfscope}%
\begin{pgfscope}%
\pgftext[x=6.851451in,y=0.407834in,,top]{\rmfamily\fontsize{10.000000}{12.000000}\selectfont \(\displaystyle -0.50\)}%
\end{pgfscope}%
\begin{pgfscope}%
\pgfsetbuttcap%
\pgfsetroundjoin%
\definecolor{currentfill}{rgb}{0.000000,0.000000,0.000000}%
\pgfsetfillcolor{currentfill}%
\pgfsetlinewidth{0.803000pt}%
\definecolor{currentstroke}{rgb}{0.000000,0.000000,0.000000}%
\pgfsetstrokecolor{currentstroke}%
\pgfsetdash{}{0pt}%
\pgfsys@defobject{currentmarker}{\pgfqpoint{0.000000in}{-0.048611in}}{\pgfqpoint{0.000000in}{0.000000in}}{%
\pgfpathmoveto{\pgfqpoint{0.000000in}{0.000000in}}%
\pgfpathlineto{\pgfqpoint{0.000000in}{-0.048611in}}%
\pgfusepath{stroke,fill}%
}%
\begin{pgfscope}%
\pgfsys@transformshift{7.331823in}{0.505056in}%
\pgfsys@useobject{currentmarker}{}%
\end{pgfscope}%
\end{pgfscope}%
\begin{pgfscope}%
\pgftext[x=7.331823in,y=0.407834in,,top]{\rmfamily\fontsize{10.000000}{12.000000}\selectfont \(\displaystyle -0.25\)}%
\end{pgfscope}%
\begin{pgfscope}%
\pgfsetbuttcap%
\pgfsetroundjoin%
\definecolor{currentfill}{rgb}{0.000000,0.000000,0.000000}%
\pgfsetfillcolor{currentfill}%
\pgfsetlinewidth{0.803000pt}%
\definecolor{currentstroke}{rgb}{0.000000,0.000000,0.000000}%
\pgfsetstrokecolor{currentstroke}%
\pgfsetdash{}{0pt}%
\pgfsys@defobject{currentmarker}{\pgfqpoint{0.000000in}{-0.048611in}}{\pgfqpoint{0.000000in}{0.000000in}}{%
\pgfpathmoveto{\pgfqpoint{0.000000in}{0.000000in}}%
\pgfpathlineto{\pgfqpoint{0.000000in}{-0.048611in}}%
\pgfusepath{stroke,fill}%
}%
\begin{pgfscope}%
\pgfsys@transformshift{7.812195in}{0.505056in}%
\pgfsys@useobject{currentmarker}{}%
\end{pgfscope}%
\end{pgfscope}%
\begin{pgfscope}%
\pgftext[x=7.812195in,y=0.407834in,,top]{\rmfamily\fontsize{10.000000}{12.000000}\selectfont \(\displaystyle 0.00\)}%
\end{pgfscope}%
\begin{pgfscope}%
\pgfsetbuttcap%
\pgfsetroundjoin%
\definecolor{currentfill}{rgb}{0.000000,0.000000,0.000000}%
\pgfsetfillcolor{currentfill}%
\pgfsetlinewidth{0.803000pt}%
\definecolor{currentstroke}{rgb}{0.000000,0.000000,0.000000}%
\pgfsetstrokecolor{currentstroke}%
\pgfsetdash{}{0pt}%
\pgfsys@defobject{currentmarker}{\pgfqpoint{0.000000in}{-0.048611in}}{\pgfqpoint{0.000000in}{0.000000in}}{%
\pgfpathmoveto{\pgfqpoint{0.000000in}{0.000000in}}%
\pgfpathlineto{\pgfqpoint{0.000000in}{-0.048611in}}%
\pgfusepath{stroke,fill}%
}%
\begin{pgfscope}%
\pgfsys@transformshift{8.292567in}{0.505056in}%
\pgfsys@useobject{currentmarker}{}%
\end{pgfscope}%
\end{pgfscope}%
\begin{pgfscope}%
\pgftext[x=8.292567in,y=0.407834in,,top]{\rmfamily\fontsize{10.000000}{12.000000}\selectfont \(\displaystyle 0.25\)}%
\end{pgfscope}%
\begin{pgfscope}%
\pgfsetbuttcap%
\pgfsetroundjoin%
\definecolor{currentfill}{rgb}{0.000000,0.000000,0.000000}%
\pgfsetfillcolor{currentfill}%
\pgfsetlinewidth{0.803000pt}%
\definecolor{currentstroke}{rgb}{0.000000,0.000000,0.000000}%
\pgfsetstrokecolor{currentstroke}%
\pgfsetdash{}{0pt}%
\pgfsys@defobject{currentmarker}{\pgfqpoint{0.000000in}{-0.048611in}}{\pgfqpoint{0.000000in}{0.000000in}}{%
\pgfpathmoveto{\pgfqpoint{0.000000in}{0.000000in}}%
\pgfpathlineto{\pgfqpoint{0.000000in}{-0.048611in}}%
\pgfusepath{stroke,fill}%
}%
\begin{pgfscope}%
\pgfsys@transformshift{8.772939in}{0.505056in}%
\pgfsys@useobject{currentmarker}{}%
\end{pgfscope}%
\end{pgfscope}%
\begin{pgfscope}%
\pgftext[x=8.772939in,y=0.407834in,,top]{\rmfamily\fontsize{10.000000}{12.000000}\selectfont \(\displaystyle 0.50\)}%
\end{pgfscope}%
\begin{pgfscope}%
\pgfsetbuttcap%
\pgfsetroundjoin%
\definecolor{currentfill}{rgb}{0.000000,0.000000,0.000000}%
\pgfsetfillcolor{currentfill}%
\pgfsetlinewidth{0.803000pt}%
\definecolor{currentstroke}{rgb}{0.000000,0.000000,0.000000}%
\pgfsetstrokecolor{currentstroke}%
\pgfsetdash{}{0pt}%
\pgfsys@defobject{currentmarker}{\pgfqpoint{0.000000in}{-0.048611in}}{\pgfqpoint{0.000000in}{0.000000in}}{%
\pgfpathmoveto{\pgfqpoint{0.000000in}{0.000000in}}%
\pgfpathlineto{\pgfqpoint{0.000000in}{-0.048611in}}%
\pgfusepath{stroke,fill}%
}%
\begin{pgfscope}%
\pgfsys@transformshift{9.253311in}{0.505056in}%
\pgfsys@useobject{currentmarker}{}%
\end{pgfscope}%
\end{pgfscope}%
\begin{pgfscope}%
\pgftext[x=9.253311in,y=0.407834in,,top]{\rmfamily\fontsize{10.000000}{12.000000}\selectfont \(\displaystyle 0.75\)}%
\end{pgfscope}%
\begin{pgfscope}%
\pgfsetbuttcap%
\pgfsetroundjoin%
\definecolor{currentfill}{rgb}{0.000000,0.000000,0.000000}%
\pgfsetfillcolor{currentfill}%
\pgfsetlinewidth{0.803000pt}%
\definecolor{currentstroke}{rgb}{0.000000,0.000000,0.000000}%
\pgfsetstrokecolor{currentstroke}%
\pgfsetdash{}{0pt}%
\pgfsys@defobject{currentmarker}{\pgfqpoint{0.000000in}{-0.048611in}}{\pgfqpoint{0.000000in}{0.000000in}}{%
\pgfpathmoveto{\pgfqpoint{0.000000in}{0.000000in}}%
\pgfpathlineto{\pgfqpoint{0.000000in}{-0.048611in}}%
\pgfusepath{stroke,fill}%
}%
\begin{pgfscope}%
\pgfsys@transformshift{9.733682in}{0.505056in}%
\pgfsys@useobject{currentmarker}{}%
\end{pgfscope}%
\end{pgfscope}%
\begin{pgfscope}%
\pgftext[x=9.733682in,y=0.407834in,,top]{\rmfamily\fontsize{10.000000}{12.000000}\selectfont \(\displaystyle 1.00\)}%
\end{pgfscope}%
\begin{pgfscope}%
\pgftext[x=7.812195in,y=0.226139in,,top]{\rmfamily\fontsize{10.000000}{12.000000}\selectfont \(\displaystyle x\)}%
\end{pgfscope}%
\begin{pgfscope}%
\pgfsetbuttcap%
\pgfsetroundjoin%
\definecolor{currentfill}{rgb}{0.000000,0.000000,0.000000}%
\pgfsetfillcolor{currentfill}%
\pgfsetlinewidth{0.803000pt}%
\definecolor{currentstroke}{rgb}{0.000000,0.000000,0.000000}%
\pgfsetstrokecolor{currentstroke}%
\pgfsetdash{}{0pt}%
\pgfsys@defobject{currentmarker}{\pgfqpoint{-0.048611in}{0.000000in}}{\pgfqpoint{0.000000in}{0.000000in}}{%
\pgfpathmoveto{\pgfqpoint{0.000000in}{0.000000in}}%
\pgfpathlineto{\pgfqpoint{-0.048611in}{0.000000in}}%
\pgfusepath{stroke,fill}%
}%
\begin{pgfscope}%
\pgfsys@transformshift{5.698559in}{0.950556in}%
\pgfsys@useobject{currentmarker}{}%
\end{pgfscope}%
\end{pgfscope}%
\begin{pgfscope}%
\pgfsetbuttcap%
\pgfsetroundjoin%
\definecolor{currentfill}{rgb}{0.000000,0.000000,0.000000}%
\pgfsetfillcolor{currentfill}%
\pgfsetlinewidth{0.803000pt}%
\definecolor{currentstroke}{rgb}{0.000000,0.000000,0.000000}%
\pgfsetstrokecolor{currentstroke}%
\pgfsetdash{}{0pt}%
\pgfsys@defobject{currentmarker}{\pgfqpoint{-0.048611in}{0.000000in}}{\pgfqpoint{0.000000in}{0.000000in}}{%
\pgfpathmoveto{\pgfqpoint{0.000000in}{0.000000in}}%
\pgfpathlineto{\pgfqpoint{-0.048611in}{0.000000in}}%
\pgfusepath{stroke,fill}%
}%
\begin{pgfscope}%
\pgfsys@transformshift{5.698559in}{1.410860in}%
\pgfsys@useobject{currentmarker}{}%
\end{pgfscope}%
\end{pgfscope}%
\begin{pgfscope}%
\pgfsetbuttcap%
\pgfsetroundjoin%
\definecolor{currentfill}{rgb}{0.000000,0.000000,0.000000}%
\pgfsetfillcolor{currentfill}%
\pgfsetlinewidth{0.803000pt}%
\definecolor{currentstroke}{rgb}{0.000000,0.000000,0.000000}%
\pgfsetstrokecolor{currentstroke}%
\pgfsetdash{}{0pt}%
\pgfsys@defobject{currentmarker}{\pgfqpoint{-0.048611in}{0.000000in}}{\pgfqpoint{0.000000in}{0.000000in}}{%
\pgfpathmoveto{\pgfqpoint{0.000000in}{0.000000in}}%
\pgfpathlineto{\pgfqpoint{-0.048611in}{0.000000in}}%
\pgfusepath{stroke,fill}%
}%
\begin{pgfscope}%
\pgfsys@transformshift{5.698559in}{1.871164in}%
\pgfsys@useobject{currentmarker}{}%
\end{pgfscope}%
\end{pgfscope}%
\begin{pgfscope}%
\pgfsetbuttcap%
\pgfsetroundjoin%
\definecolor{currentfill}{rgb}{0.000000,0.000000,0.000000}%
\pgfsetfillcolor{currentfill}%
\pgfsetlinewidth{0.803000pt}%
\definecolor{currentstroke}{rgb}{0.000000,0.000000,0.000000}%
\pgfsetstrokecolor{currentstroke}%
\pgfsetdash{}{0pt}%
\pgfsys@defobject{currentmarker}{\pgfqpoint{-0.048611in}{0.000000in}}{\pgfqpoint{0.000000in}{0.000000in}}{%
\pgfpathmoveto{\pgfqpoint{0.000000in}{0.000000in}}%
\pgfpathlineto{\pgfqpoint{-0.048611in}{0.000000in}}%
\pgfusepath{stroke,fill}%
}%
\begin{pgfscope}%
\pgfsys@transformshift{5.698559in}{2.331468in}%
\pgfsys@useobject{currentmarker}{}%
\end{pgfscope}%
\end{pgfscope}%
\begin{pgfscope}%
\pgfsetbuttcap%
\pgfsetroundjoin%
\definecolor{currentfill}{rgb}{0.000000,0.000000,0.000000}%
\pgfsetfillcolor{currentfill}%
\pgfsetlinewidth{0.803000pt}%
\definecolor{currentstroke}{rgb}{0.000000,0.000000,0.000000}%
\pgfsetstrokecolor{currentstroke}%
\pgfsetdash{}{0pt}%
\pgfsys@defobject{currentmarker}{\pgfqpoint{-0.048611in}{0.000000in}}{\pgfqpoint{0.000000in}{0.000000in}}{%
\pgfpathmoveto{\pgfqpoint{0.000000in}{0.000000in}}%
\pgfpathlineto{\pgfqpoint{-0.048611in}{0.000000in}}%
\pgfusepath{stroke,fill}%
}%
\begin{pgfscope}%
\pgfsys@transformshift{5.698559in}{2.791773in}%
\pgfsys@useobject{currentmarker}{}%
\end{pgfscope}%
\end{pgfscope}%
\begin{pgfscope}%
\pgfsetbuttcap%
\pgfsetroundjoin%
\definecolor{currentfill}{rgb}{0.000000,0.000000,0.000000}%
\pgfsetfillcolor{currentfill}%
\pgfsetlinewidth{0.803000pt}%
\definecolor{currentstroke}{rgb}{0.000000,0.000000,0.000000}%
\pgfsetstrokecolor{currentstroke}%
\pgfsetdash{}{0pt}%
\pgfsys@defobject{currentmarker}{\pgfqpoint{-0.048611in}{0.000000in}}{\pgfqpoint{0.000000in}{0.000000in}}{%
\pgfpathmoveto{\pgfqpoint{0.000000in}{0.000000in}}%
\pgfpathlineto{\pgfqpoint{-0.048611in}{0.000000in}}%
\pgfusepath{stroke,fill}%
}%
\begin{pgfscope}%
\pgfsys@transformshift{5.698559in}{3.252077in}%
\pgfsys@useobject{currentmarker}{}%
\end{pgfscope}%
\end{pgfscope}%
\begin{pgfscope}%
\pgftext[x=5.643003in,y=1.877783in,,bottom,rotate=90.000000]{\rmfamily\fontsize{10.000000}{12.000000}\selectfont \(\displaystyle y_2\)}%
\end{pgfscope}%
\begin{pgfscope}%
\pgfpathrectangle{\pgfqpoint{5.698559in}{0.505056in}}{\pgfqpoint{4.227273in}{2.745455in}} %
\pgfusepath{clip}%
\pgfsetrectcap%
\pgfsetroundjoin%
\pgfsetlinewidth{0.501875pt}%
\definecolor{currentstroke}{rgb}{0.500000,0.000000,1.000000}%
\pgfsetstrokecolor{currentstroke}%
\pgfsetdash{}{0pt}%
\pgfpathmoveto{\pgfqpoint{5.890707in}{1.674110in}}%
\pgfpathlineto{\pgfqpoint{5.998743in}{1.865294in}}%
\pgfpathlineto{\pgfqpoint{6.122212in}{2.083587in}}%
\pgfpathlineto{\pgfqpoint{6.199380in}{2.213811in}}%
\pgfpathlineto{\pgfqpoint{6.261115in}{2.311786in}}%
\pgfpathlineto{\pgfqpoint{6.307415in}{2.380579in}}%
\pgfpathlineto{\pgfqpoint{6.353716in}{2.444639in}}%
\pgfpathlineto{\pgfqpoint{6.400017in}{2.503372in}}%
\pgfpathlineto{\pgfqpoint{6.446318in}{2.556231in}}%
\pgfpathlineto{\pgfqpoint{6.492619in}{2.602725in}}%
\pgfpathlineto{\pgfqpoint{6.523486in}{2.629970in}}%
\pgfpathlineto{\pgfqpoint{6.554354in}{2.654080in}}%
\pgfpathlineto{\pgfqpoint{6.585221in}{2.674957in}}%
\pgfpathlineto{\pgfqpoint{6.616088in}{2.692512in}}%
\pgfpathlineto{\pgfqpoint{6.646955in}{2.706675in}}%
\pgfpathlineto{\pgfqpoint{6.677823in}{2.717387in}}%
\pgfpathlineto{\pgfqpoint{6.708690in}{2.724603in}}%
\pgfpathlineto{\pgfqpoint{6.739557in}{2.728294in}}%
\pgfpathlineto{\pgfqpoint{6.770424in}{2.728444in}}%
\pgfpathlineto{\pgfqpoint{6.801292in}{2.725053in}}%
\pgfpathlineto{\pgfqpoint{6.832159in}{2.718135in}}%
\pgfpathlineto{\pgfqpoint{6.863026in}{2.707718in}}%
\pgfpathlineto{\pgfqpoint{6.893894in}{2.693846in}}%
\pgfpathlineto{\pgfqpoint{6.924761in}{2.676575in}}%
\pgfpathlineto{\pgfqpoint{6.955628in}{2.655977in}}%
\pgfpathlineto{\pgfqpoint{6.986495in}{2.632137in}}%
\pgfpathlineto{\pgfqpoint{7.017363in}{2.605154in}}%
\pgfpathlineto{\pgfqpoint{7.048230in}{2.575139in}}%
\pgfpathlineto{\pgfqpoint{7.094531in}{2.524706in}}%
\pgfpathlineto{\pgfqpoint{7.140832in}{2.468201in}}%
\pgfpathlineto{\pgfqpoint{7.187133in}{2.406150in}}%
\pgfpathlineto{\pgfqpoint{7.233434in}{2.339128in}}%
\pgfpathlineto{\pgfqpoint{7.295168in}{2.243117in}}%
\pgfpathlineto{\pgfqpoint{7.356903in}{2.140966in}}%
\pgfpathlineto{\pgfqpoint{7.434071in}{2.007220in}}%
\pgfpathlineto{\pgfqpoint{7.681009in}{1.573095in}}%
\pgfpathlineto{\pgfqpoint{7.742744in}{1.472450in}}%
\pgfpathlineto{\pgfqpoint{7.804478in}{1.378385in}}%
\pgfpathlineto{\pgfqpoint{7.850779in}{1.313095in}}%
\pgfpathlineto{\pgfqpoint{7.897080in}{1.252989in}}%
\pgfpathlineto{\pgfqpoint{7.943381in}{1.198626in}}%
\pgfpathlineto{\pgfqpoint{7.989682in}{1.150511in}}%
\pgfpathlineto{\pgfqpoint{8.020549in}{1.122129in}}%
\pgfpathlineto{\pgfqpoint{8.051416in}{1.096841in}}%
\pgfpathlineto{\pgfqpoint{8.082284in}{1.074751in}}%
\pgfpathlineto{\pgfqpoint{8.113151in}{1.055951in}}%
\pgfpathlineto{\pgfqpoint{8.144018in}{1.040519in}}%
\pgfpathlineto{\pgfqpoint{8.174885in}{1.028518in}}%
\pgfpathlineto{\pgfqpoint{8.205753in}{1.019998in}}%
\pgfpathlineto{\pgfqpoint{8.236620in}{1.014994in}}%
\pgfpathlineto{\pgfqpoint{8.267487in}{1.013526in}}%
\pgfpathlineto{\pgfqpoint{8.298354in}{1.015601in}}%
\pgfpathlineto{\pgfqpoint{8.329222in}{1.021211in}}%
\pgfpathlineto{\pgfqpoint{8.360089in}{1.030331in}}%
\pgfpathlineto{\pgfqpoint{8.390956in}{1.042925in}}%
\pgfpathlineto{\pgfqpoint{8.421823in}{1.058940in}}%
\pgfpathlineto{\pgfqpoint{8.452691in}{1.078310in}}%
\pgfpathlineto{\pgfqpoint{8.483558in}{1.100955in}}%
\pgfpathlineto{\pgfqpoint{8.514425in}{1.126782in}}%
\pgfpathlineto{\pgfqpoint{8.545293in}{1.155683in}}%
\pgfpathlineto{\pgfqpoint{8.591593in}{1.204536in}}%
\pgfpathlineto{\pgfqpoint{8.637894in}{1.259583in}}%
\pgfpathlineto{\pgfqpoint{8.684195in}{1.320311in}}%
\pgfpathlineto{\pgfqpoint{8.730496in}{1.386157in}}%
\pgfpathlineto{\pgfqpoint{8.792231in}{1.480849in}}%
\pgfpathlineto{\pgfqpoint{8.853965in}{1.581983in}}%
\pgfpathlineto{\pgfqpoint{8.931133in}{1.714909in}}%
\pgfpathlineto{\pgfqpoint{9.208939in}{2.201455in}}%
\pgfpathlineto{\pgfqpoint{9.270673in}{2.300208in}}%
\pgfpathlineto{\pgfqpoint{9.316974in}{2.369710in}}%
\pgfpathlineto{\pgfqpoint{9.363275in}{2.434581in}}%
\pgfpathlineto{\pgfqpoint{9.409576in}{2.494217in}}%
\pgfpathlineto{\pgfqpoint{9.455877in}{2.548065in}}%
\pgfpathlineto{\pgfqpoint{9.502178in}{2.595625in}}%
\pgfpathlineto{\pgfqpoint{9.533045in}{2.623616in}}%
\pgfpathlineto{\pgfqpoint{9.563913in}{2.648500in}}%
\pgfpathlineto{\pgfqpoint{9.594780in}{2.670173in}}%
\pgfpathlineto{\pgfqpoint{9.625647in}{2.688545in}}%
\pgfpathlineto{\pgfqpoint{9.656514in}{2.703541in}}%
\pgfpathlineto{\pgfqpoint{9.687382in}{2.715098in}}%
\pgfpathlineto{\pgfqpoint{9.718249in}{2.723169in}}%
\pgfpathlineto{\pgfqpoint{9.733682in}{2.725886in}}%
\pgfpathlineto{\pgfqpoint{9.733682in}{2.725886in}}%
\pgfusepath{stroke}%
\end{pgfscope}%
\begin{pgfscope}%
\pgfpathrectangle{\pgfqpoint{5.698559in}{0.505056in}}{\pgfqpoint{4.227273in}{2.745455in}} %
\pgfusepath{clip}%
\pgfsetrectcap%
\pgfsetroundjoin%
\pgfsetlinewidth{0.501875pt}%
\definecolor{currentstroke}{rgb}{0.421569,0.122888,0.998103}%
\pgfsetstrokecolor{currentstroke}%
\pgfsetdash{}{0pt}%
\pgfpathmoveto{\pgfqpoint{5.890707in}{1.674110in}}%
\pgfpathlineto{\pgfqpoint{5.998743in}{1.865294in}}%
\pgfpathlineto{\pgfqpoint{6.122212in}{2.083587in}}%
\pgfpathlineto{\pgfqpoint{6.199380in}{2.213811in}}%
\pgfpathlineto{\pgfqpoint{6.261115in}{2.311786in}}%
\pgfpathlineto{\pgfqpoint{6.307415in}{2.380579in}}%
\pgfpathlineto{\pgfqpoint{6.353716in}{2.444639in}}%
\pgfpathlineto{\pgfqpoint{6.400017in}{2.503372in}}%
\pgfpathlineto{\pgfqpoint{6.446318in}{2.556231in}}%
\pgfpathlineto{\pgfqpoint{6.492619in}{2.602725in}}%
\pgfpathlineto{\pgfqpoint{6.523486in}{2.629970in}}%
\pgfpathlineto{\pgfqpoint{6.554354in}{2.654080in}}%
\pgfpathlineto{\pgfqpoint{6.585221in}{2.674957in}}%
\pgfpathlineto{\pgfqpoint{6.616088in}{2.692512in}}%
\pgfpathlineto{\pgfqpoint{6.646955in}{2.706675in}}%
\pgfpathlineto{\pgfqpoint{6.677823in}{2.717387in}}%
\pgfpathlineto{\pgfqpoint{6.708690in}{2.724603in}}%
\pgfpathlineto{\pgfqpoint{6.739557in}{2.728294in}}%
\pgfpathlineto{\pgfqpoint{6.770424in}{2.728444in}}%
\pgfpathlineto{\pgfqpoint{6.801292in}{2.725053in}}%
\pgfpathlineto{\pgfqpoint{6.832159in}{2.718135in}}%
\pgfpathlineto{\pgfqpoint{6.863026in}{2.707718in}}%
\pgfpathlineto{\pgfqpoint{6.893894in}{2.693846in}}%
\pgfpathlineto{\pgfqpoint{6.924761in}{2.676575in}}%
\pgfpathlineto{\pgfqpoint{6.955628in}{2.655977in}}%
\pgfpathlineto{\pgfqpoint{6.986495in}{2.632137in}}%
\pgfpathlineto{\pgfqpoint{7.017363in}{2.605154in}}%
\pgfpathlineto{\pgfqpoint{7.048230in}{2.575139in}}%
\pgfpathlineto{\pgfqpoint{7.094531in}{2.524706in}}%
\pgfpathlineto{\pgfqpoint{7.140832in}{2.468201in}}%
\pgfpathlineto{\pgfqpoint{7.187133in}{2.406150in}}%
\pgfpathlineto{\pgfqpoint{7.233434in}{2.339128in}}%
\pgfpathlineto{\pgfqpoint{7.295168in}{2.243117in}}%
\pgfpathlineto{\pgfqpoint{7.356903in}{2.140966in}}%
\pgfpathlineto{\pgfqpoint{7.434071in}{2.007220in}}%
\pgfpathlineto{\pgfqpoint{7.681009in}{1.573095in}}%
\pgfpathlineto{\pgfqpoint{7.742744in}{1.472450in}}%
\pgfpathlineto{\pgfqpoint{7.804478in}{1.378385in}}%
\pgfpathlineto{\pgfqpoint{7.850779in}{1.313095in}}%
\pgfpathlineto{\pgfqpoint{7.897080in}{1.252989in}}%
\pgfpathlineto{\pgfqpoint{7.943381in}{1.198626in}}%
\pgfpathlineto{\pgfqpoint{7.989682in}{1.150511in}}%
\pgfpathlineto{\pgfqpoint{8.020549in}{1.122129in}}%
\pgfpathlineto{\pgfqpoint{8.051416in}{1.096841in}}%
\pgfpathlineto{\pgfqpoint{8.082284in}{1.074751in}}%
\pgfpathlineto{\pgfqpoint{8.113151in}{1.055951in}}%
\pgfpathlineto{\pgfqpoint{8.144018in}{1.040519in}}%
\pgfpathlineto{\pgfqpoint{8.174885in}{1.028518in}}%
\pgfpathlineto{\pgfqpoint{8.205753in}{1.019998in}}%
\pgfpathlineto{\pgfqpoint{8.236620in}{1.014994in}}%
\pgfpathlineto{\pgfqpoint{8.267487in}{1.013526in}}%
\pgfpathlineto{\pgfqpoint{8.298354in}{1.015601in}}%
\pgfpathlineto{\pgfqpoint{8.329222in}{1.021211in}}%
\pgfpathlineto{\pgfqpoint{8.360089in}{1.030331in}}%
\pgfpathlineto{\pgfqpoint{8.390956in}{1.042925in}}%
\pgfpathlineto{\pgfqpoint{8.421823in}{1.058940in}}%
\pgfpathlineto{\pgfqpoint{8.452691in}{1.078310in}}%
\pgfpathlineto{\pgfqpoint{8.483558in}{1.100955in}}%
\pgfpathlineto{\pgfqpoint{8.514425in}{1.126782in}}%
\pgfpathlineto{\pgfqpoint{8.545293in}{1.155683in}}%
\pgfpathlineto{\pgfqpoint{8.591593in}{1.204536in}}%
\pgfpathlineto{\pgfqpoint{8.637894in}{1.259583in}}%
\pgfpathlineto{\pgfqpoint{8.684195in}{1.320311in}}%
\pgfpathlineto{\pgfqpoint{8.730496in}{1.386157in}}%
\pgfpathlineto{\pgfqpoint{8.792231in}{1.480849in}}%
\pgfpathlineto{\pgfqpoint{8.853965in}{1.581983in}}%
\pgfpathlineto{\pgfqpoint{8.931133in}{1.714909in}}%
\pgfpathlineto{\pgfqpoint{9.208939in}{2.201455in}}%
\pgfpathlineto{\pgfqpoint{9.270673in}{2.300208in}}%
\pgfpathlineto{\pgfqpoint{9.316974in}{2.369710in}}%
\pgfpathlineto{\pgfqpoint{9.363275in}{2.434581in}}%
\pgfpathlineto{\pgfqpoint{9.409576in}{2.494217in}}%
\pgfpathlineto{\pgfqpoint{9.455877in}{2.548065in}}%
\pgfpathlineto{\pgfqpoint{9.502178in}{2.595625in}}%
\pgfpathlineto{\pgfqpoint{9.533045in}{2.623616in}}%
\pgfpathlineto{\pgfqpoint{9.563913in}{2.648500in}}%
\pgfpathlineto{\pgfqpoint{9.594780in}{2.670173in}}%
\pgfpathlineto{\pgfqpoint{9.625647in}{2.688545in}}%
\pgfpathlineto{\pgfqpoint{9.656514in}{2.703541in}}%
\pgfpathlineto{\pgfqpoint{9.687382in}{2.715098in}}%
\pgfpathlineto{\pgfqpoint{9.718249in}{2.723169in}}%
\pgfpathlineto{\pgfqpoint{9.733682in}{2.725886in}}%
\pgfpathlineto{\pgfqpoint{9.733682in}{2.725886in}}%
\pgfusepath{stroke}%
\end{pgfscope}%
\begin{pgfscope}%
\pgfpathrectangle{\pgfqpoint{5.698559in}{0.505056in}}{\pgfqpoint{4.227273in}{2.745455in}} %
\pgfusepath{clip}%
\pgfsetrectcap%
\pgfsetroundjoin%
\pgfsetlinewidth{0.501875pt}%
\definecolor{currentstroke}{rgb}{0.343137,0.243914,0.992421}%
\pgfsetstrokecolor{currentstroke}%
\pgfsetdash{}{0pt}%
\pgfpathmoveto{\pgfqpoint{5.890707in}{2.520935in}}%
\pgfpathlineto{\pgfqpoint{5.906141in}{2.502715in}}%
\pgfpathlineto{\pgfqpoint{5.921575in}{2.467431in}}%
\pgfpathlineto{\pgfqpoint{5.937008in}{2.414741in}}%
\pgfpathlineto{\pgfqpoint{5.952442in}{2.344855in}}%
\pgfpathlineto{\pgfqpoint{5.967875in}{2.258539in}}%
\pgfpathlineto{\pgfqpoint{5.983309in}{2.157092in}}%
\pgfpathlineto{\pgfqpoint{6.014176in}{1.916489in}}%
\pgfpathlineto{\pgfqpoint{6.060477in}{1.500644in}}%
\pgfpathlineto{\pgfqpoint{6.091345in}{1.223404in}}%
\pgfpathlineto{\pgfqpoint{6.122212in}{0.976929in}}%
\pgfpathlineto{\pgfqpoint{6.137645in}{0.872834in}}%
\pgfpathlineto{\pgfqpoint{6.153079in}{0.784965in}}%
\pgfpathlineto{\pgfqpoint{6.168513in}{0.715406in}}%
\pgfpathlineto{\pgfqpoint{6.183946in}{0.665753in}}%
\pgfpathlineto{\pgfqpoint{6.199380in}{0.637070in}}%
\pgfpathlineto{\pgfqpoint{6.214814in}{0.629849in}}%
\pgfpathlineto{\pgfqpoint{6.230247in}{0.644008in}}%
\pgfpathlineto{\pgfqpoint{6.245681in}{0.678886in}}%
\pgfpathlineto{\pgfqpoint{6.261115in}{0.733271in}}%
\pgfpathlineto{\pgfqpoint{6.276548in}{0.805438in}}%
\pgfpathlineto{\pgfqpoint{6.291982in}{0.893203in}}%
\pgfpathlineto{\pgfqpoint{6.322849in}{1.104947in}}%
\pgfpathlineto{\pgfqpoint{6.415451in}{1.807956in}}%
\pgfpathlineto{\pgfqpoint{6.430885in}{1.903655in}}%
\pgfpathlineto{\pgfqpoint{6.446318in}{1.986866in}}%
\pgfpathlineto{\pgfqpoint{6.461752in}{2.056130in}}%
\pgfpathlineto{\pgfqpoint{6.477185in}{2.110500in}}%
\pgfpathlineto{\pgfqpoint{6.492619in}{2.149565in}}%
\pgfpathlineto{\pgfqpoint{6.508053in}{2.173458in}}%
\pgfpathlineto{\pgfqpoint{6.523486in}{2.182838in}}%
\pgfpathlineto{\pgfqpoint{6.538920in}{2.178863in}}%
\pgfpathlineto{\pgfqpoint{6.554354in}{2.163142in}}%
\pgfpathlineto{\pgfqpoint{6.569787in}{2.137676in}}%
\pgfpathlineto{\pgfqpoint{6.585221in}{2.104782in}}%
\pgfpathlineto{\pgfqpoint{6.646955in}{1.951460in}}%
\pgfpathlineto{\pgfqpoint{6.662389in}{1.921050in}}%
\pgfpathlineto{\pgfqpoint{6.677823in}{1.898706in}}%
\pgfpathlineto{\pgfqpoint{6.693256in}{1.886407in}}%
\pgfpathlineto{\pgfqpoint{6.708690in}{1.885739in}}%
\pgfpathlineto{\pgfqpoint{6.724124in}{1.897834in}}%
\pgfpathlineto{\pgfqpoint{6.739557in}{1.923336in}}%
\pgfpathlineto{\pgfqpoint{6.754991in}{1.962363in}}%
\pgfpathlineto{\pgfqpoint{6.770424in}{2.014504in}}%
\pgfpathlineto{\pgfqpoint{6.785858in}{2.078825in}}%
\pgfpathlineto{\pgfqpoint{6.801292in}{2.153893in}}%
\pgfpathlineto{\pgfqpoint{6.832159in}{2.328329in}}%
\pgfpathlineto{\pgfqpoint{6.893894in}{2.700805in}}%
\pgfpathlineto{\pgfqpoint{6.909327in}{2.781556in}}%
\pgfpathlineto{\pgfqpoint{6.924761in}{2.851448in}}%
\pgfpathlineto{\pgfqpoint{6.940194in}{2.907839in}}%
\pgfpathlineto{\pgfqpoint{6.955628in}{2.948410in}}%
\pgfpathlineto{\pgfqpoint{6.971062in}{2.971244in}}%
\pgfpathlineto{\pgfqpoint{6.986495in}{2.974892in}}%
\pgfpathlineto{\pgfqpoint{7.001929in}{2.958424in}}%
\pgfpathlineto{\pgfqpoint{7.017363in}{2.921463in}}%
\pgfpathlineto{\pgfqpoint{7.032796in}{2.864204in}}%
\pgfpathlineto{\pgfqpoint{7.048230in}{2.787414in}}%
\pgfpathlineto{\pgfqpoint{7.063664in}{2.692414in}}%
\pgfpathlineto{\pgfqpoint{7.079097in}{2.581042in}}%
\pgfpathlineto{\pgfqpoint{7.109964in}{2.318800in}}%
\pgfpathlineto{\pgfqpoint{7.156265in}{1.871524in}}%
\pgfpathlineto{\pgfqpoint{7.187133in}{1.576268in}}%
\pgfpathlineto{\pgfqpoint{7.218000in}{1.313909in}}%
\pgfpathlineto{\pgfqpoint{7.233434in}{1.202150in}}%
\pgfpathlineto{\pgfqpoint{7.248867in}{1.106368in}}%
\pgfpathlineto{\pgfqpoint{7.264301in}{1.028248in}}%
\pgfpathlineto{\pgfqpoint{7.279734in}{0.968950in}}%
\pgfpathlineto{\pgfqpoint{7.295168in}{0.929075in}}%
\pgfpathlineto{\pgfqpoint{7.310602in}{0.908658in}}%
\pgfpathlineto{\pgfqpoint{7.326035in}{0.907174in}}%
\pgfpathlineto{\pgfqpoint{7.341469in}{0.923560in}}%
\pgfpathlineto{\pgfqpoint{7.356903in}{0.956258in}}%
\pgfpathlineto{\pgfqpoint{7.372336in}{1.003272in}}%
\pgfpathlineto{\pgfqpoint{7.387770in}{1.062237in}}%
\pgfpathlineto{\pgfqpoint{7.418637in}{1.205215in}}%
\pgfpathlineto{\pgfqpoint{7.480372in}{1.510170in}}%
\pgfpathlineto{\pgfqpoint{7.495805in}{1.574318in}}%
\pgfpathlineto{\pgfqpoint{7.511239in}{1.629080in}}%
\pgfpathlineto{\pgfqpoint{7.526673in}{1.672847in}}%
\pgfpathlineto{\pgfqpoint{7.542106in}{1.704477in}}%
\pgfpathlineto{\pgfqpoint{7.557540in}{1.723338in}}%
\pgfpathlineto{\pgfqpoint{7.572974in}{1.729322in}}%
\pgfpathlineto{\pgfqpoint{7.588407in}{1.722845in}}%
\pgfpathlineto{\pgfqpoint{7.603841in}{1.704832in}}%
\pgfpathlineto{\pgfqpoint{7.619274in}{1.676681in}}%
\pgfpathlineto{\pgfqpoint{7.634708in}{1.640213in}}%
\pgfpathlineto{\pgfqpoint{7.665575in}{1.551333in}}%
\pgfpathlineto{\pgfqpoint{7.696443in}{1.458496in}}%
\pgfpathlineto{\pgfqpoint{7.711876in}{1.417466in}}%
\pgfpathlineto{\pgfqpoint{7.727310in}{1.383626in}}%
\pgfpathlineto{\pgfqpoint{7.742744in}{1.359467in}}%
\pgfpathlineto{\pgfqpoint{7.758177in}{1.347203in}}%
\pgfpathlineto{\pgfqpoint{7.773611in}{1.348693in}}%
\pgfpathlineto{\pgfqpoint{7.789044in}{1.365372in}}%
\pgfpathlineto{\pgfqpoint{7.804478in}{1.398193in}}%
\pgfpathlineto{\pgfqpoint{7.819912in}{1.447590in}}%
\pgfpathlineto{\pgfqpoint{7.835345in}{1.513454in}}%
\pgfpathlineto{\pgfqpoint{7.850779in}{1.595125in}}%
\pgfpathlineto{\pgfqpoint{7.866213in}{1.691406in}}%
\pgfpathlineto{\pgfqpoint{7.897080in}{1.920529in}}%
\pgfpathlineto{\pgfqpoint{7.943381in}{2.317614in}}%
\pgfpathlineto{\pgfqpoint{7.974248in}{2.582107in}}%
\pgfpathlineto{\pgfqpoint{8.005115in}{2.815984in}}%
\pgfpathlineto{\pgfqpoint{8.020549in}{2.913858in}}%
\pgfpathlineto{\pgfqpoint{8.035983in}{2.995544in}}%
\pgfpathlineto{\pgfqpoint{8.051416in}{3.058934in}}%
\pgfpathlineto{\pgfqpoint{8.066850in}{3.102398in}}%
\pgfpathlineto{\pgfqpoint{8.082284in}{3.124838in}}%
\pgfpathlineto{\pgfqpoint{8.097717in}{3.125717in}}%
\pgfpathlineto{\pgfqpoint{8.113151in}{3.105080in}}%
\pgfpathlineto{\pgfqpoint{8.128584in}{3.063547in}}%
\pgfpathlineto{\pgfqpoint{8.144018in}{3.002299in}}%
\pgfpathlineto{\pgfqpoint{8.159452in}{2.923036in}}%
\pgfpathlineto{\pgfqpoint{8.174885in}{2.827924in}}%
\pgfpathlineto{\pgfqpoint{8.205753in}{2.600751in}}%
\pgfpathlineto{\pgfqpoint{8.298354in}{1.848282in}}%
\pgfpathlineto{\pgfqpoint{8.313788in}{1.744658in}}%
\pgfpathlineto{\pgfqpoint{8.329222in}{1.653890in}}%
\pgfpathlineto{\pgfqpoint{8.344655in}{1.577559in}}%
\pgfpathlineto{\pgfqpoint{8.360089in}{1.516732in}}%
\pgfpathlineto{\pgfqpoint{8.375523in}{1.471941in}}%
\pgfpathlineto{\pgfqpoint{8.390956in}{1.443167in}}%
\pgfpathlineto{\pgfqpoint{8.406390in}{1.429858in}}%
\pgfpathlineto{\pgfqpoint{8.421823in}{1.430951in}}%
\pgfpathlineto{\pgfqpoint{8.437257in}{1.444917in}}%
\pgfpathlineto{\pgfqpoint{8.452691in}{1.469818in}}%
\pgfpathlineto{\pgfqpoint{8.468124in}{1.503382in}}%
\pgfpathlineto{\pgfqpoint{8.498992in}{1.586233in}}%
\pgfpathlineto{\pgfqpoint{8.529859in}{1.671884in}}%
\pgfpathlineto{\pgfqpoint{8.545293in}{1.709060in}}%
\pgfpathlineto{\pgfqpoint{8.560726in}{1.739222in}}%
\pgfpathlineto{\pgfqpoint{8.576160in}{1.760291in}}%
\pgfpathlineto{\pgfqpoint{8.591593in}{1.770565in}}%
\pgfpathlineto{\pgfqpoint{8.607027in}{1.768777in}}%
\pgfpathlineto{\pgfqpoint{8.622461in}{1.754147in}}%
\pgfpathlineto{\pgfqpoint{8.637894in}{1.726410in}}%
\pgfpathlineto{\pgfqpoint{8.653328in}{1.685826in}}%
\pgfpathlineto{\pgfqpoint{8.668762in}{1.633181in}}%
\pgfpathlineto{\pgfqpoint{8.684195in}{1.569760in}}%
\pgfpathlineto{\pgfqpoint{8.715063in}{1.417994in}}%
\pgfpathlineto{\pgfqpoint{8.776797in}{1.085250in}}%
\pgfpathlineto{\pgfqpoint{8.792231in}{1.012804in}}%
\pgfpathlineto{\pgfqpoint{8.807664in}{0.950480in}}%
\pgfpathlineto{\pgfqpoint{8.823098in}{0.900910in}}%
\pgfpathlineto{\pgfqpoint{8.838532in}{0.866417in}}%
\pgfpathlineto{\pgfqpoint{8.853965in}{0.848940in}}%
\pgfpathlineto{\pgfqpoint{8.869399in}{0.849963in}}%
\pgfpathlineto{\pgfqpoint{8.884833in}{0.870461in}}%
\pgfpathlineto{\pgfqpoint{8.900266in}{0.910864in}}%
\pgfpathlineto{\pgfqpoint{8.915700in}{0.971036in}}%
\pgfpathlineto{\pgfqpoint{8.931133in}{1.050271in}}%
\pgfpathlineto{\pgfqpoint{8.946567in}{1.147309in}}%
\pgfpathlineto{\pgfqpoint{8.962001in}{1.260370in}}%
\pgfpathlineto{\pgfqpoint{8.992868in}{1.525138in}}%
\pgfpathlineto{\pgfqpoint{9.039169in}{1.974557in}}%
\pgfpathlineto{\pgfqpoint{9.070036in}{2.270240in}}%
\pgfpathlineto{\pgfqpoint{9.100903in}{2.532032in}}%
\pgfpathlineto{\pgfqpoint{9.116337in}{2.643024in}}%
\pgfpathlineto{\pgfqpoint{9.131771in}{2.737641in}}%
\pgfpathlineto{\pgfqpoint{9.147204in}{2.814133in}}%
\pgfpathlineto{\pgfqpoint{9.162638in}{2.871276in}}%
\pgfpathlineto{\pgfqpoint{9.178072in}{2.908400in}}%
\pgfpathlineto{\pgfqpoint{9.193505in}{2.925407in}}%
\pgfpathlineto{\pgfqpoint{9.208939in}{2.922767in}}%
\pgfpathlineto{\pgfqpoint{9.224373in}{2.901494in}}%
\pgfpathlineto{\pgfqpoint{9.239806in}{2.863111in}}%
\pgfpathlineto{\pgfqpoint{9.255240in}{2.809594in}}%
\pgfpathlineto{\pgfqpoint{9.270673in}{2.743301in}}%
\pgfpathlineto{\pgfqpoint{9.301541in}{2.583261in}}%
\pgfpathlineto{\pgfqpoint{9.363275in}{2.236094in}}%
\pgfpathlineto{\pgfqpoint{9.378709in}{2.160237in}}%
\pgfpathlineto{\pgfqpoint{9.394143in}{2.093545in}}%
\pgfpathlineto{\pgfqpoint{9.409576in}{2.037773in}}%
\pgfpathlineto{\pgfqpoint{9.425010in}{1.994207in}}%
\pgfpathlineto{\pgfqpoint{9.440443in}{1.963635in}}%
\pgfpathlineto{\pgfqpoint{9.455877in}{1.946316in}}%
\pgfpathlineto{\pgfqpoint{9.471311in}{1.941984in}}%
\pgfpathlineto{\pgfqpoint{9.486744in}{1.949855in}}%
\pgfpathlineto{\pgfqpoint{9.502178in}{1.968664in}}%
\pgfpathlineto{\pgfqpoint{9.517612in}{1.996707in}}%
\pgfpathlineto{\pgfqpoint{9.533045in}{2.031907in}}%
\pgfpathlineto{\pgfqpoint{9.594780in}{2.194024in}}%
\pgfpathlineto{\pgfqpoint{9.610213in}{2.226414in}}%
\pgfpathlineto{\pgfqpoint{9.625647in}{2.250336in}}%
\pgfpathlineto{\pgfqpoint{9.641081in}{2.263531in}}%
\pgfpathlineto{\pgfqpoint{9.656514in}{2.264081in}}%
\pgfpathlineto{\pgfqpoint{9.671948in}{2.250473in}}%
\pgfpathlineto{\pgfqpoint{9.687382in}{2.221663in}}%
\pgfpathlineto{\pgfqpoint{9.702815in}{2.177115in}}%
\pgfpathlineto{\pgfqpoint{9.718249in}{2.116829in}}%
\pgfpathlineto{\pgfqpoint{9.733682in}{2.041351in}}%
\pgfpathlineto{\pgfqpoint{9.733682in}{2.041351in}}%
\pgfusepath{stroke}%
\end{pgfscope}%
\begin{pgfscope}%
\pgfpathrectangle{\pgfqpoint{5.698559in}{0.505056in}}{\pgfqpoint{4.227273in}{2.745455in}} %
\pgfusepath{clip}%
\pgfsetrectcap%
\pgfsetroundjoin%
\pgfsetlinewidth{0.501875pt}%
\definecolor{currentstroke}{rgb}{0.264706,0.361242,0.982973}%
\pgfsetstrokecolor{currentstroke}%
\pgfsetdash{}{0pt}%
\pgfpathmoveto{\pgfqpoint{5.890707in}{2.287545in}}%
\pgfpathlineto{\pgfqpoint{5.906141in}{2.313328in}}%
\pgfpathlineto{\pgfqpoint{5.921575in}{2.327184in}}%
\pgfpathlineto{\pgfqpoint{5.937008in}{2.328046in}}%
\pgfpathlineto{\pgfqpoint{5.952442in}{2.315247in}}%
\pgfpathlineto{\pgfqpoint{5.967875in}{2.288538in}}%
\pgfpathlineto{\pgfqpoint{5.983309in}{2.248095in}}%
\pgfpathlineto{\pgfqpoint{5.998743in}{2.194514in}}%
\pgfpathlineto{\pgfqpoint{6.014176in}{2.128789in}}%
\pgfpathlineto{\pgfqpoint{6.029610in}{2.052285in}}%
\pgfpathlineto{\pgfqpoint{6.060477in}{1.873972in}}%
\pgfpathlineto{\pgfqpoint{6.153079in}{1.296579in}}%
\pgfpathlineto{\pgfqpoint{6.168513in}{1.218454in}}%
\pgfpathlineto{\pgfqpoint{6.183946in}{1.150929in}}%
\pgfpathlineto{\pgfqpoint{6.199380in}{1.095488in}}%
\pgfpathlineto{\pgfqpoint{6.214814in}{1.053282in}}%
\pgfpathlineto{\pgfqpoint{6.230247in}{1.025099in}}%
\pgfpathlineto{\pgfqpoint{6.245681in}{1.011349in}}%
\pgfpathlineto{\pgfqpoint{6.261115in}{1.012059in}}%
\pgfpathlineto{\pgfqpoint{6.276548in}{1.026881in}}%
\pgfpathlineto{\pgfqpoint{6.291982in}{1.055114in}}%
\pgfpathlineto{\pgfqpoint{6.307415in}{1.095727in}}%
\pgfpathlineto{\pgfqpoint{6.322849in}{1.147404in}}%
\pgfpathlineto{\pgfqpoint{6.338283in}{1.208588in}}%
\pgfpathlineto{\pgfqpoint{6.369150in}{1.352369in}}%
\pgfpathlineto{\pgfqpoint{6.461752in}{1.818899in}}%
\pgfpathlineto{\pgfqpoint{6.477185in}{1.884035in}}%
\pgfpathlineto{\pgfqpoint{6.492619in}{1.942301in}}%
\pgfpathlineto{\pgfqpoint{6.508053in}{1.993062in}}%
\pgfpathlineto{\pgfqpoint{6.523486in}{2.035976in}}%
\pgfpathlineto{\pgfqpoint{6.538920in}{2.070990in}}%
\pgfpathlineto{\pgfqpoint{6.554354in}{2.098332in}}%
\pgfpathlineto{\pgfqpoint{6.569787in}{2.118486in}}%
\pgfpathlineto{\pgfqpoint{6.585221in}{2.132160in}}%
\pgfpathlineto{\pgfqpoint{6.600655in}{2.140256in}}%
\pgfpathlineto{\pgfqpoint{6.616088in}{2.143817in}}%
\pgfpathlineto{\pgfqpoint{6.631522in}{2.143988in}}%
\pgfpathlineto{\pgfqpoint{6.662389in}{2.138935in}}%
\pgfpathlineto{\pgfqpoint{6.693256in}{2.134345in}}%
\pgfpathlineto{\pgfqpoint{6.708690in}{2.134739in}}%
\pgfpathlineto{\pgfqpoint{6.724124in}{2.137963in}}%
\pgfpathlineto{\pgfqpoint{6.739557in}{2.144553in}}%
\pgfpathlineto{\pgfqpoint{6.754991in}{2.154820in}}%
\pgfpathlineto{\pgfqpoint{6.770424in}{2.168840in}}%
\pgfpathlineto{\pgfqpoint{6.785858in}{2.186454in}}%
\pgfpathlineto{\pgfqpoint{6.801292in}{2.207274in}}%
\pgfpathlineto{\pgfqpoint{6.832159in}{2.255927in}}%
\pgfpathlineto{\pgfqpoint{6.878460in}{2.332357in}}%
\pgfpathlineto{\pgfqpoint{6.893894in}{2.354272in}}%
\pgfpathlineto{\pgfqpoint{6.909327in}{2.372440in}}%
\pgfpathlineto{\pgfqpoint{6.924761in}{2.385740in}}%
\pgfpathlineto{\pgfqpoint{6.940194in}{2.393158in}}%
\pgfpathlineto{\pgfqpoint{6.955628in}{2.393824in}}%
\pgfpathlineto{\pgfqpoint{6.971062in}{2.387052in}}%
\pgfpathlineto{\pgfqpoint{6.986495in}{2.372371in}}%
\pgfpathlineto{\pgfqpoint{7.001929in}{2.349542in}}%
\pgfpathlineto{\pgfqpoint{7.017363in}{2.318582in}}%
\pgfpathlineto{\pgfqpoint{7.032796in}{2.279761in}}%
\pgfpathlineto{\pgfqpoint{7.048230in}{2.233607in}}%
\pgfpathlineto{\pgfqpoint{7.063664in}{2.180885in}}%
\pgfpathlineto{\pgfqpoint{7.094531in}{2.059882in}}%
\pgfpathlineto{\pgfqpoint{7.187133in}{1.671123in}}%
\pgfpathlineto{\pgfqpoint{7.202566in}{1.617881in}}%
\pgfpathlineto{\pgfqpoint{7.218000in}{1.571346in}}%
\pgfpathlineto{\pgfqpoint{7.233434in}{1.532408in}}%
\pgfpathlineto{\pgfqpoint{7.248867in}{1.501713in}}%
\pgfpathlineto{\pgfqpoint{7.264301in}{1.479643in}}%
\pgfpathlineto{\pgfqpoint{7.279734in}{1.466294in}}%
\pgfpathlineto{\pgfqpoint{7.295168in}{1.461482in}}%
\pgfpathlineto{\pgfqpoint{7.310602in}{1.464741in}}%
\pgfpathlineto{\pgfqpoint{7.326035in}{1.475342in}}%
\pgfpathlineto{\pgfqpoint{7.341469in}{1.492316in}}%
\pgfpathlineto{\pgfqpoint{7.356903in}{1.514490in}}%
\pgfpathlineto{\pgfqpoint{7.387770in}{1.568957in}}%
\pgfpathlineto{\pgfqpoint{7.418637in}{1.626924in}}%
\pgfpathlineto{\pgfqpoint{7.434071in}{1.653429in}}%
\pgfpathlineto{\pgfqpoint{7.449504in}{1.676394in}}%
\pgfpathlineto{\pgfqpoint{7.464938in}{1.694578in}}%
\pgfpathlineto{\pgfqpoint{7.480372in}{1.706941in}}%
\pgfpathlineto{\pgfqpoint{7.495805in}{1.712681in}}%
\pgfpathlineto{\pgfqpoint{7.511239in}{1.711276in}}%
\pgfpathlineto{\pgfqpoint{7.526673in}{1.702501in}}%
\pgfpathlineto{\pgfqpoint{7.542106in}{1.686447in}}%
\pgfpathlineto{\pgfqpoint{7.557540in}{1.663525in}}%
\pgfpathlineto{\pgfqpoint{7.572974in}{1.634457in}}%
\pgfpathlineto{\pgfqpoint{7.588407in}{1.600262in}}%
\pgfpathlineto{\pgfqpoint{7.619274in}{1.521861in}}%
\pgfpathlineto{\pgfqpoint{7.650142in}{1.441071in}}%
\pgfpathlineto{\pgfqpoint{7.665575in}{1.404367in}}%
\pgfpathlineto{\pgfqpoint{7.681009in}{1.372657in}}%
\pgfpathlineto{\pgfqpoint{7.696443in}{1.347785in}}%
\pgfpathlineto{\pgfqpoint{7.711876in}{1.331470in}}%
\pgfpathlineto{\pgfqpoint{7.727310in}{1.325248in}}%
\pgfpathlineto{\pgfqpoint{7.742744in}{1.330409in}}%
\pgfpathlineto{\pgfqpoint{7.758177in}{1.347951in}}%
\pgfpathlineto{\pgfqpoint{7.773611in}{1.378534in}}%
\pgfpathlineto{\pgfqpoint{7.789044in}{1.422450in}}%
\pgfpathlineto{\pgfqpoint{7.804478in}{1.479596in}}%
\pgfpathlineto{\pgfqpoint{7.819912in}{1.549469in}}%
\pgfpathlineto{\pgfqpoint{7.835345in}{1.631166in}}%
\pgfpathlineto{\pgfqpoint{7.866213in}{1.824520in}}%
\pgfpathlineto{\pgfqpoint{7.897080in}{2.045303in}}%
\pgfpathlineto{\pgfqpoint{7.943381in}{2.386567in}}%
\pgfpathlineto{\pgfqpoint{7.974248in}{2.590287in}}%
\pgfpathlineto{\pgfqpoint{7.989682in}{2.677367in}}%
\pgfpathlineto{\pgfqpoint{8.005115in}{2.751593in}}%
\pgfpathlineto{\pgfqpoint{8.020549in}{2.811061in}}%
\pgfpathlineto{\pgfqpoint{8.035983in}{2.854224in}}%
\pgfpathlineto{\pgfqpoint{8.051416in}{2.879932in}}%
\pgfpathlineto{\pgfqpoint{8.066850in}{2.887474in}}%
\pgfpathlineto{\pgfqpoint{8.082284in}{2.876603in}}%
\pgfpathlineto{\pgfqpoint{8.097717in}{2.847547in}}%
\pgfpathlineto{\pgfqpoint{8.113151in}{2.801003in}}%
\pgfpathlineto{\pgfqpoint{8.128584in}{2.738127in}}%
\pgfpathlineto{\pgfqpoint{8.144018in}{2.660499in}}%
\pgfpathlineto{\pgfqpoint{8.159452in}{2.570086in}}%
\pgfpathlineto{\pgfqpoint{8.190319in}{2.360338in}}%
\pgfpathlineto{\pgfqpoint{8.267487in}{1.795699in}}%
\pgfpathlineto{\pgfqpoint{8.282921in}{1.698166in}}%
\pgfpathlineto{\pgfqpoint{8.298354in}{1.611506in}}%
\pgfpathlineto{\pgfqpoint{8.313788in}{1.537597in}}%
\pgfpathlineto{\pgfqpoint{8.329222in}{1.477917in}}%
\pgfpathlineto{\pgfqpoint{8.344655in}{1.433504in}}%
\pgfpathlineto{\pgfqpoint{8.360089in}{1.404928in}}%
\pgfpathlineto{\pgfqpoint{8.375523in}{1.392274in}}%
\pgfpathlineto{\pgfqpoint{8.390956in}{1.395147in}}%
\pgfpathlineto{\pgfqpoint{8.406390in}{1.412684in}}%
\pgfpathlineto{\pgfqpoint{8.421823in}{1.443585in}}%
\pgfpathlineto{\pgfqpoint{8.437257in}{1.486153in}}%
\pgfpathlineto{\pgfqpoint{8.452691in}{1.538350in}}%
\pgfpathlineto{\pgfqpoint{8.483558in}{1.662167in}}%
\pgfpathlineto{\pgfqpoint{8.529859in}{1.857292in}}%
\pgfpathlineto{\pgfqpoint{8.545293in}{1.914340in}}%
\pgfpathlineto{\pgfqpoint{8.560726in}{1.963376in}}%
\pgfpathlineto{\pgfqpoint{8.576160in}{2.002361in}}%
\pgfpathlineto{\pgfqpoint{8.591593in}{2.029593in}}%
\pgfpathlineto{\pgfqpoint{8.607027in}{2.043769in}}%
\pgfpathlineto{\pgfqpoint{8.622461in}{2.044017in}}%
\pgfpathlineto{\pgfqpoint{8.637894in}{2.029930in}}%
\pgfpathlineto{\pgfqpoint{8.653328in}{2.001577in}}%
\pgfpathlineto{\pgfqpoint{8.668762in}{1.959504in}}%
\pgfpathlineto{\pgfqpoint{8.684195in}{1.904720in}}%
\pgfpathlineto{\pgfqpoint{8.699629in}{1.838663in}}%
\pgfpathlineto{\pgfqpoint{8.730496in}{1.680386in}}%
\pgfpathlineto{\pgfqpoint{8.807664in}{1.247760in}}%
\pgfpathlineto{\pgfqpoint{8.823098in}{1.176527in}}%
\pgfpathlineto{\pgfqpoint{8.838532in}{1.116368in}}%
\pgfpathlineto{\pgfqpoint{8.853965in}{1.069401in}}%
\pgfpathlineto{\pgfqpoint{8.869399in}{1.037402in}}%
\pgfpathlineto{\pgfqpoint{8.884833in}{1.021751in}}%
\pgfpathlineto{\pgfqpoint{8.900266in}{1.023385in}}%
\pgfpathlineto{\pgfqpoint{8.915700in}{1.042777in}}%
\pgfpathlineto{\pgfqpoint{8.931133in}{1.079916in}}%
\pgfpathlineto{\pgfqpoint{8.946567in}{1.134310in}}%
\pgfpathlineto{\pgfqpoint{8.962001in}{1.204998in}}%
\pgfpathlineto{\pgfqpoint{8.977434in}{1.290582in}}%
\pgfpathlineto{\pgfqpoint{8.992868in}{1.389259in}}%
\pgfpathlineto{\pgfqpoint{9.023735in}{1.617017in}}%
\pgfpathlineto{\pgfqpoint{9.116337in}{2.350783in}}%
\pgfpathlineto{\pgfqpoint{9.131771in}{2.451963in}}%
\pgfpathlineto{\pgfqpoint{9.147204in}{2.540889in}}%
\pgfpathlineto{\pgfqpoint{9.162638in}{2.615900in}}%
\pgfpathlineto{\pgfqpoint{9.178072in}{2.675723in}}%
\pgfpathlineto{\pgfqpoint{9.193505in}{2.719513in}}%
\pgfpathlineto{\pgfqpoint{9.208939in}{2.746860in}}%
\pgfpathlineto{\pgfqpoint{9.224373in}{2.757792in}}%
\pgfpathlineto{\pgfqpoint{9.239806in}{2.752770in}}%
\pgfpathlineto{\pgfqpoint{9.255240in}{2.732655in}}%
\pgfpathlineto{\pgfqpoint{9.270673in}{2.698679in}}%
\pgfpathlineto{\pgfqpoint{9.286107in}{2.652398in}}%
\pgfpathlineto{\pgfqpoint{9.301541in}{2.595637in}}%
\pgfpathlineto{\pgfqpoint{9.332408in}{2.458936in}}%
\pgfpathlineto{\pgfqpoint{9.409576in}{2.084448in}}%
\pgfpathlineto{\pgfqpoint{9.425010in}{2.020126in}}%
\pgfpathlineto{\pgfqpoint{9.440443in}{1.963024in}}%
\pgfpathlineto{\pgfqpoint{9.455877in}{1.914151in}}%
\pgfpathlineto{\pgfqpoint{9.471311in}{1.874177in}}%
\pgfpathlineto{\pgfqpoint{9.486744in}{1.843420in}}%
\pgfpathlineto{\pgfqpoint{9.502178in}{1.821854in}}%
\pgfpathlineto{\pgfqpoint{9.517612in}{1.809119in}}%
\pgfpathlineto{\pgfqpoint{9.533045in}{1.804541in}}%
\pgfpathlineto{\pgfqpoint{9.548479in}{1.807172in}}%
\pgfpathlineto{\pgfqpoint{9.563913in}{1.815828in}}%
\pgfpathlineto{\pgfqpoint{9.579346in}{1.829140in}}%
\pgfpathlineto{\pgfqpoint{9.610213in}{1.863647in}}%
\pgfpathlineto{\pgfqpoint{9.641081in}{1.898137in}}%
\pgfpathlineto{\pgfqpoint{9.656514in}{1.911586in}}%
\pgfpathlineto{\pgfqpoint{9.671948in}{1.920719in}}%
\pgfpathlineto{\pgfqpoint{9.687382in}{1.924431in}}%
\pgfpathlineto{\pgfqpoint{9.702815in}{1.921847in}}%
\pgfpathlineto{\pgfqpoint{9.718249in}{1.912357in}}%
\pgfpathlineto{\pgfqpoint{9.733682in}{1.895632in}}%
\pgfpathlineto{\pgfqpoint{9.733682in}{1.895632in}}%
\pgfusepath{stroke}%
\end{pgfscope}%
\begin{pgfscope}%
\pgfpathrectangle{\pgfqpoint{5.698559in}{0.505056in}}{\pgfqpoint{4.227273in}{2.745455in}} %
\pgfusepath{clip}%
\pgfsetrectcap%
\pgfsetroundjoin%
\pgfsetlinewidth{0.501875pt}%
\definecolor{currentstroke}{rgb}{0.186275,0.473094,0.969797}%
\pgfsetstrokecolor{currentstroke}%
\pgfsetdash{}{0pt}%
\pgfpathmoveto{\pgfqpoint{5.890707in}{2.197256in}}%
\pgfpathlineto{\pgfqpoint{5.937008in}{2.110051in}}%
\pgfpathlineto{\pgfqpoint{5.967875in}{2.045573in}}%
\pgfpathlineto{\pgfqpoint{5.998743in}{1.973960in}}%
\pgfpathlineto{\pgfqpoint{6.029610in}{1.894406in}}%
\pgfpathlineto{\pgfqpoint{6.060477in}{1.807229in}}%
\pgfpathlineto{\pgfqpoint{6.106778in}{1.665939in}}%
\pgfpathlineto{\pgfqpoint{6.168513in}{1.475602in}}%
\pgfpathlineto{\pgfqpoint{6.199380in}{1.390094in}}%
\pgfpathlineto{\pgfqpoint{6.230247in}{1.318189in}}%
\pgfpathlineto{\pgfqpoint{6.245681in}{1.289021in}}%
\pgfpathlineto{\pgfqpoint{6.261115in}{1.265220in}}%
\pgfpathlineto{\pgfqpoint{6.276548in}{1.247358in}}%
\pgfpathlineto{\pgfqpoint{6.291982in}{1.235943in}}%
\pgfpathlineto{\pgfqpoint{6.307415in}{1.231408in}}%
\pgfpathlineto{\pgfqpoint{6.322849in}{1.234099in}}%
\pgfpathlineto{\pgfqpoint{6.338283in}{1.244264in}}%
\pgfpathlineto{\pgfqpoint{6.353716in}{1.262046in}}%
\pgfpathlineto{\pgfqpoint{6.369150in}{1.287481in}}%
\pgfpathlineto{\pgfqpoint{6.384584in}{1.320488in}}%
\pgfpathlineto{\pgfqpoint{6.400017in}{1.360871in}}%
\pgfpathlineto{\pgfqpoint{6.415451in}{1.408321in}}%
\pgfpathlineto{\pgfqpoint{6.430885in}{1.462417in}}%
\pgfpathlineto{\pgfqpoint{6.461752in}{1.588335in}}%
\pgfpathlineto{\pgfqpoint{6.492619in}{1.733277in}}%
\pgfpathlineto{\pgfqpoint{6.538920in}{1.971720in}}%
\pgfpathlineto{\pgfqpoint{6.600655in}{2.289303in}}%
\pgfpathlineto{\pgfqpoint{6.631522in}{2.431098in}}%
\pgfpathlineto{\pgfqpoint{6.662389in}{2.552833in}}%
\pgfpathlineto{\pgfqpoint{6.677823in}{2.604557in}}%
\pgfpathlineto{\pgfqpoint{6.693256in}{2.649516in}}%
\pgfpathlineto{\pgfqpoint{6.708690in}{2.687341in}}%
\pgfpathlineto{\pgfqpoint{6.724124in}{2.717777in}}%
\pgfpathlineto{\pgfqpoint{6.739557in}{2.740680in}}%
\pgfpathlineto{\pgfqpoint{6.754991in}{2.756025in}}%
\pgfpathlineto{\pgfqpoint{6.770424in}{2.763893in}}%
\pgfpathlineto{\pgfqpoint{6.785858in}{2.764474in}}%
\pgfpathlineto{\pgfqpoint{6.801292in}{2.758059in}}%
\pgfpathlineto{\pgfqpoint{6.816725in}{2.745026in}}%
\pgfpathlineto{\pgfqpoint{6.832159in}{2.725838in}}%
\pgfpathlineto{\pgfqpoint{6.847593in}{2.701022in}}%
\pgfpathlineto{\pgfqpoint{6.863026in}{2.671164in}}%
\pgfpathlineto{\pgfqpoint{6.878460in}{2.636892in}}%
\pgfpathlineto{\pgfqpoint{6.909327in}{2.557741in}}%
\pgfpathlineto{\pgfqpoint{6.940194in}{2.468887in}}%
\pgfpathlineto{\pgfqpoint{7.048230in}{2.148499in}}%
\pgfpathlineto{\pgfqpoint{7.079097in}{2.067345in}}%
\pgfpathlineto{\pgfqpoint{7.109964in}{1.993132in}}%
\pgfpathlineto{\pgfqpoint{7.140832in}{1.925224in}}%
\pgfpathlineto{\pgfqpoint{7.187133in}{1.831606in}}%
\pgfpathlineto{\pgfqpoint{7.264301in}{1.678623in}}%
\pgfpathlineto{\pgfqpoint{7.310602in}{1.577414in}}%
\pgfpathlineto{\pgfqpoint{7.356903in}{1.465690in}}%
\pgfpathlineto{\pgfqpoint{7.464938in}{1.194908in}}%
\pgfpathlineto{\pgfqpoint{7.495805in}{1.131050in}}%
\pgfpathlineto{\pgfqpoint{7.511239in}{1.104292in}}%
\pgfpathlineto{\pgfqpoint{7.526673in}{1.081797in}}%
\pgfpathlineto{\pgfqpoint{7.542106in}{1.064149in}}%
\pgfpathlineto{\pgfqpoint{7.557540in}{1.051893in}}%
\pgfpathlineto{\pgfqpoint{7.572974in}{1.045521in}}%
\pgfpathlineto{\pgfqpoint{7.588407in}{1.045460in}}%
\pgfpathlineto{\pgfqpoint{7.603841in}{1.052060in}}%
\pgfpathlineto{\pgfqpoint{7.619274in}{1.065582in}}%
\pgfpathlineto{\pgfqpoint{7.634708in}{1.086193in}}%
\pgfpathlineto{\pgfqpoint{7.650142in}{1.113955in}}%
\pgfpathlineto{\pgfqpoint{7.665575in}{1.148824in}}%
\pgfpathlineto{\pgfqpoint{7.681009in}{1.190643in}}%
\pgfpathlineto{\pgfqpoint{7.696443in}{1.239143in}}%
\pgfpathlineto{\pgfqpoint{7.711876in}{1.293948in}}%
\pgfpathlineto{\pgfqpoint{7.742744in}{1.420436in}}%
\pgfpathlineto{\pgfqpoint{7.773611in}{1.565082in}}%
\pgfpathlineto{\pgfqpoint{7.819912in}{1.802010in}}%
\pgfpathlineto{\pgfqpoint{7.866213in}{2.040703in}}%
\pgfpathlineto{\pgfqpoint{7.897080in}{2.188118in}}%
\pgfpathlineto{\pgfqpoint{7.927947in}{2.317911in}}%
\pgfpathlineto{\pgfqpoint{7.943381in}{2.374291in}}%
\pgfpathlineto{\pgfqpoint{7.958814in}{2.424144in}}%
\pgfpathlineto{\pgfqpoint{7.974248in}{2.466954in}}%
\pgfpathlineto{\pgfqpoint{7.989682in}{2.502315in}}%
\pgfpathlineto{\pgfqpoint{8.005115in}{2.529933in}}%
\pgfpathlineto{\pgfqpoint{8.020549in}{2.549633in}}%
\pgfpathlineto{\pgfqpoint{8.035983in}{2.561353in}}%
\pgfpathlineto{\pgfqpoint{8.051416in}{2.565153in}}%
\pgfpathlineto{\pgfqpoint{8.066850in}{2.561201in}}%
\pgfpathlineto{\pgfqpoint{8.082284in}{2.549774in}}%
\pgfpathlineto{\pgfqpoint{8.097717in}{2.531249in}}%
\pgfpathlineto{\pgfqpoint{8.113151in}{2.506092in}}%
\pgfpathlineto{\pgfqpoint{8.128584in}{2.474849in}}%
\pgfpathlineto{\pgfqpoint{8.144018in}{2.438131in}}%
\pgfpathlineto{\pgfqpoint{8.174885in}{2.350967in}}%
\pgfpathlineto{\pgfqpoint{8.205753in}{2.250291in}}%
\pgfpathlineto{\pgfqpoint{8.329222in}{1.822278in}}%
\pgfpathlineto{\pgfqpoint{8.360089in}{1.729936in}}%
\pgfpathlineto{\pgfqpoint{8.390956in}{1.647838in}}%
\pgfpathlineto{\pgfqpoint{8.421823in}{1.575946in}}%
\pgfpathlineto{\pgfqpoint{8.452691in}{1.513127in}}%
\pgfpathlineto{\pgfqpoint{8.483558in}{1.457423in}}%
\pgfpathlineto{\pgfqpoint{8.529859in}{1.381856in}}%
\pgfpathlineto{\pgfqpoint{8.607027in}{1.258109in}}%
\pgfpathlineto{\pgfqpoint{8.668762in}{1.150675in}}%
\pgfpathlineto{\pgfqpoint{8.730496in}{1.043456in}}%
\pgfpathlineto{\pgfqpoint{8.761363in}{0.997129in}}%
\pgfpathlineto{\pgfqpoint{8.776797in}{0.977522in}}%
\pgfpathlineto{\pgfqpoint{8.792231in}{0.961015in}}%
\pgfpathlineto{\pgfqpoint{8.807664in}{0.948161in}}%
\pgfpathlineto{\pgfqpoint{8.823098in}{0.939496in}}%
\pgfpathlineto{\pgfqpoint{8.838532in}{0.935531in}}%
\pgfpathlineto{\pgfqpoint{8.853965in}{0.936737in}}%
\pgfpathlineto{\pgfqpoint{8.869399in}{0.943532in}}%
\pgfpathlineto{\pgfqpoint{8.884833in}{0.956265in}}%
\pgfpathlineto{\pgfqpoint{8.900266in}{0.975212in}}%
\pgfpathlineto{\pgfqpoint{8.915700in}{1.000563in}}%
\pgfpathlineto{\pgfqpoint{8.931133in}{1.032409in}}%
\pgfpathlineto{\pgfqpoint{8.946567in}{1.070746in}}%
\pgfpathlineto{\pgfqpoint{8.962001in}{1.115463in}}%
\pgfpathlineto{\pgfqpoint{8.977434in}{1.166340in}}%
\pgfpathlineto{\pgfqpoint{9.008302in}{1.285174in}}%
\pgfpathlineto{\pgfqpoint{9.039169in}{1.423424in}}%
\pgfpathlineto{\pgfqpoint{9.085470in}{1.655280in}}%
\pgfpathlineto{\pgfqpoint{9.147204in}{1.974186in}}%
\pgfpathlineto{\pgfqpoint{9.178072in}{2.121090in}}%
\pgfpathlineto{\pgfqpoint{9.208939in}{2.250107in}}%
\pgfpathlineto{\pgfqpoint{9.224373in}{2.305963in}}%
\pgfpathlineto{\pgfqpoint{9.239806in}{2.355170in}}%
\pgfpathlineto{\pgfqpoint{9.255240in}{2.397185in}}%
\pgfpathlineto{\pgfqpoint{9.270673in}{2.431572in}}%
\pgfpathlineto{\pgfqpoint{9.286107in}{2.458007in}}%
\pgfpathlineto{\pgfqpoint{9.301541in}{2.476283in}}%
\pgfpathlineto{\pgfqpoint{9.316974in}{2.486311in}}%
\pgfpathlineto{\pgfqpoint{9.332408in}{2.488123in}}%
\pgfpathlineto{\pgfqpoint{9.347842in}{2.481867in}}%
\pgfpathlineto{\pgfqpoint{9.363275in}{2.467806in}}%
\pgfpathlineto{\pgfqpoint{9.378709in}{2.446307in}}%
\pgfpathlineto{\pgfqpoint{9.394143in}{2.417836in}}%
\pgfpathlineto{\pgfqpoint{9.409576in}{2.382945in}}%
\pgfpathlineto{\pgfqpoint{9.425010in}{2.342263in}}%
\pgfpathlineto{\pgfqpoint{9.455877in}{2.246341in}}%
\pgfpathlineto{\pgfqpoint{9.486744in}{2.136090in}}%
\pgfpathlineto{\pgfqpoint{9.610213in}{1.672839in}}%
\pgfpathlineto{\pgfqpoint{9.641081in}{1.575949in}}%
\pgfpathlineto{\pgfqpoint{9.671948in}{1.492413in}}%
\pgfpathlineto{\pgfqpoint{9.702815in}{1.422748in}}%
\pgfpathlineto{\pgfqpoint{9.733682in}{1.366212in}}%
\pgfpathlineto{\pgfqpoint{9.733682in}{1.366212in}}%
\pgfusepath{stroke}%
\end{pgfscope}%
\begin{pgfscope}%
\pgfpathrectangle{\pgfqpoint{5.698559in}{0.505056in}}{\pgfqpoint{4.227273in}{2.745455in}} %
\pgfusepath{clip}%
\pgfsetrectcap%
\pgfsetroundjoin%
\pgfsetlinewidth{0.501875pt}%
\definecolor{currentstroke}{rgb}{0.100000,0.587785,0.951057}%
\pgfsetstrokecolor{currentstroke}%
\pgfsetdash{}{0pt}%
\pgfpathmoveto{\pgfqpoint{5.890707in}{2.437351in}}%
\pgfpathlineto{\pgfqpoint{5.921575in}{2.379598in}}%
\pgfpathlineto{\pgfqpoint{5.952442in}{2.310886in}}%
\pgfpathlineto{\pgfqpoint{5.983309in}{2.233939in}}%
\pgfpathlineto{\pgfqpoint{6.106778in}{1.911908in}}%
\pgfpathlineto{\pgfqpoint{6.137645in}{1.845953in}}%
\pgfpathlineto{\pgfqpoint{6.153079in}{1.817487in}}%
\pgfpathlineto{\pgfqpoint{6.168513in}{1.792494in}}%
\pgfpathlineto{\pgfqpoint{6.183946in}{1.771279in}}%
\pgfpathlineto{\pgfqpoint{6.199380in}{1.754102in}}%
\pgfpathlineto{\pgfqpoint{6.214814in}{1.741180in}}%
\pgfpathlineto{\pgfqpoint{6.230247in}{1.732679in}}%
\pgfpathlineto{\pgfqpoint{6.245681in}{1.728718in}}%
\pgfpathlineto{\pgfqpoint{6.261115in}{1.729359in}}%
\pgfpathlineto{\pgfqpoint{6.276548in}{1.734615in}}%
\pgfpathlineto{\pgfqpoint{6.291982in}{1.744443in}}%
\pgfpathlineto{\pgfqpoint{6.307415in}{1.758747in}}%
\pgfpathlineto{\pgfqpoint{6.322849in}{1.777379in}}%
\pgfpathlineto{\pgfqpoint{6.338283in}{1.800141in}}%
\pgfpathlineto{\pgfqpoint{6.353716in}{1.826785in}}%
\pgfpathlineto{\pgfqpoint{6.384584in}{1.890509in}}%
\pgfpathlineto{\pgfqpoint{6.415451in}{1.965729in}}%
\pgfpathlineto{\pgfqpoint{6.461752in}{2.092676in}}%
\pgfpathlineto{\pgfqpoint{6.538920in}{2.310009in}}%
\pgfpathlineto{\pgfqpoint{6.569787in}{2.387687in}}%
\pgfpathlineto{\pgfqpoint{6.600655in}{2.454801in}}%
\pgfpathlineto{\pgfqpoint{6.616088in}{2.483475in}}%
\pgfpathlineto{\pgfqpoint{6.631522in}{2.508479in}}%
\pgfpathlineto{\pgfqpoint{6.646955in}{2.529554in}}%
\pgfpathlineto{\pgfqpoint{6.662389in}{2.546491in}}%
\pgfpathlineto{\pgfqpoint{6.677823in}{2.559127in}}%
\pgfpathlineto{\pgfqpoint{6.693256in}{2.567350in}}%
\pgfpathlineto{\pgfqpoint{6.708690in}{2.571101in}}%
\pgfpathlineto{\pgfqpoint{6.724124in}{2.570369in}}%
\pgfpathlineto{\pgfqpoint{6.739557in}{2.565197in}}%
\pgfpathlineto{\pgfqpoint{6.754991in}{2.555676in}}%
\pgfpathlineto{\pgfqpoint{6.770424in}{2.541945in}}%
\pgfpathlineto{\pgfqpoint{6.785858in}{2.524192in}}%
\pgfpathlineto{\pgfqpoint{6.801292in}{2.502643in}}%
\pgfpathlineto{\pgfqpoint{6.816725in}{2.477567in}}%
\pgfpathlineto{\pgfqpoint{6.847593in}{2.418078in}}%
\pgfpathlineto{\pgfqpoint{6.878460in}{2.348501in}}%
\pgfpathlineto{\pgfqpoint{6.924761in}{2.232099in}}%
\pgfpathlineto{\pgfqpoint{7.001929in}{2.033331in}}%
\pgfpathlineto{\pgfqpoint{7.032796in}{1.961083in}}%
\pgfpathlineto{\pgfqpoint{7.063664in}{1.896761in}}%
\pgfpathlineto{\pgfqpoint{7.094531in}{1.842152in}}%
\pgfpathlineto{\pgfqpoint{7.109964in}{1.818874in}}%
\pgfpathlineto{\pgfqpoint{7.125398in}{1.798400in}}%
\pgfpathlineto{\pgfqpoint{7.140832in}{1.780767in}}%
\pgfpathlineto{\pgfqpoint{7.156265in}{1.765970in}}%
\pgfpathlineto{\pgfqpoint{7.171699in}{1.753961in}}%
\pgfpathlineto{\pgfqpoint{7.187133in}{1.744652in}}%
\pgfpathlineto{\pgfqpoint{7.202566in}{1.737917in}}%
\pgfpathlineto{\pgfqpoint{7.218000in}{1.733595in}}%
\pgfpathlineto{\pgfqpoint{7.233434in}{1.731493in}}%
\pgfpathlineto{\pgfqpoint{7.248867in}{1.731387in}}%
\pgfpathlineto{\pgfqpoint{7.279734in}{1.736153in}}%
\pgfpathlineto{\pgfqpoint{7.310602in}{1.745688in}}%
\pgfpathlineto{\pgfqpoint{7.403204in}{1.778989in}}%
\pgfpathlineto{\pgfqpoint{7.434071in}{1.784120in}}%
\pgfpathlineto{\pgfqpoint{7.449504in}{1.784506in}}%
\pgfpathlineto{\pgfqpoint{7.464938in}{1.783213in}}%
\pgfpathlineto{\pgfqpoint{7.480372in}{1.780110in}}%
\pgfpathlineto{\pgfqpoint{7.495805in}{1.775106in}}%
\pgfpathlineto{\pgfqpoint{7.511239in}{1.768142in}}%
\pgfpathlineto{\pgfqpoint{7.526673in}{1.759200in}}%
\pgfpathlineto{\pgfqpoint{7.542106in}{1.748298in}}%
\pgfpathlineto{\pgfqpoint{7.572974in}{1.720875in}}%
\pgfpathlineto{\pgfqpoint{7.603841in}{1.686753in}}%
\pgfpathlineto{\pgfqpoint{7.634708in}{1.647342in}}%
\pgfpathlineto{\pgfqpoint{7.696443in}{1.560380in}}%
\pgfpathlineto{\pgfqpoint{7.742744in}{1.497054in}}%
\pgfpathlineto{\pgfqpoint{7.773611in}{1.460225in}}%
\pgfpathlineto{\pgfqpoint{7.804478in}{1.430304in}}%
\pgfpathlineto{\pgfqpoint{7.819912in}{1.418535in}}%
\pgfpathlineto{\pgfqpoint{7.835345in}{1.409163in}}%
\pgfpathlineto{\pgfqpoint{7.850779in}{1.402352in}}%
\pgfpathlineto{\pgfqpoint{7.866213in}{1.398228in}}%
\pgfpathlineto{\pgfqpoint{7.881646in}{1.396883in}}%
\pgfpathlineto{\pgfqpoint{7.897080in}{1.398369in}}%
\pgfpathlineto{\pgfqpoint{7.912514in}{1.402695in}}%
\pgfpathlineto{\pgfqpoint{7.927947in}{1.409831in}}%
\pgfpathlineto{\pgfqpoint{7.943381in}{1.419704in}}%
\pgfpathlineto{\pgfqpoint{7.958814in}{1.432201in}}%
\pgfpathlineto{\pgfqpoint{7.974248in}{1.447165in}}%
\pgfpathlineto{\pgfqpoint{8.005115in}{1.483688in}}%
\pgfpathlineto{\pgfqpoint{8.035983in}{1.527295in}}%
\pgfpathlineto{\pgfqpoint{8.082284in}{1.600476in}}%
\pgfpathlineto{\pgfqpoint{8.128584in}{1.674195in}}%
\pgfpathlineto{\pgfqpoint{8.159452in}{1.718454in}}%
\pgfpathlineto{\pgfqpoint{8.190319in}{1.755254in}}%
\pgfpathlineto{\pgfqpoint{8.205753in}{1.769982in}}%
\pgfpathlineto{\pgfqpoint{8.221186in}{1.781842in}}%
\pgfpathlineto{\pgfqpoint{8.236620in}{1.790559in}}%
\pgfpathlineto{\pgfqpoint{8.252053in}{1.795892in}}%
\pgfpathlineto{\pgfqpoint{8.267487in}{1.797641in}}%
\pgfpathlineto{\pgfqpoint{8.282921in}{1.795647in}}%
\pgfpathlineto{\pgfqpoint{8.298354in}{1.789796in}}%
\pgfpathlineto{\pgfqpoint{8.313788in}{1.780022in}}%
\pgfpathlineto{\pgfqpoint{8.329222in}{1.766308in}}%
\pgfpathlineto{\pgfqpoint{8.344655in}{1.748686in}}%
\pgfpathlineto{\pgfqpoint{8.360089in}{1.727238in}}%
\pgfpathlineto{\pgfqpoint{8.375523in}{1.702096in}}%
\pgfpathlineto{\pgfqpoint{8.390956in}{1.673440in}}%
\pgfpathlineto{\pgfqpoint{8.421823in}{1.606535in}}%
\pgfpathlineto{\pgfqpoint{8.452691in}{1.528851in}}%
\pgfpathlineto{\pgfqpoint{8.498992in}{1.398661in}}%
\pgfpathlineto{\pgfqpoint{8.576160in}{1.175040in}}%
\pgfpathlineto{\pgfqpoint{8.607027in}{1.094352in}}%
\pgfpathlineto{\pgfqpoint{8.637894in}{1.024127in}}%
\pgfpathlineto{\pgfqpoint{8.653328in}{0.993940in}}%
\pgfpathlineto{\pgfqpoint{8.668762in}{0.967507in}}%
\pgfpathlineto{\pgfqpoint{8.684195in}{0.945129in}}%
\pgfpathlineto{\pgfqpoint{8.699629in}{0.927066in}}%
\pgfpathlineto{\pgfqpoint{8.715063in}{0.913530in}}%
\pgfpathlineto{\pgfqpoint{8.730496in}{0.904683in}}%
\pgfpathlineto{\pgfqpoint{8.745930in}{0.900635in}}%
\pgfpathlineto{\pgfqpoint{8.761363in}{0.901445in}}%
\pgfpathlineto{\pgfqpoint{8.776797in}{0.907115in}}%
\pgfpathlineto{\pgfqpoint{8.792231in}{0.917596in}}%
\pgfpathlineto{\pgfqpoint{8.807664in}{0.932783in}}%
\pgfpathlineto{\pgfqpoint{8.823098in}{0.952521in}}%
\pgfpathlineto{\pgfqpoint{8.838532in}{0.976605in}}%
\pgfpathlineto{\pgfqpoint{8.853965in}{1.004780in}}%
\pgfpathlineto{\pgfqpoint{8.884833in}{1.072173in}}%
\pgfpathlineto{\pgfqpoint{8.915700in}{1.151853in}}%
\pgfpathlineto{\pgfqpoint{8.946567in}{1.240461in}}%
\pgfpathlineto{\pgfqpoint{9.070036in}{1.608761in}}%
\pgfpathlineto{\pgfqpoint{9.100903in}{1.685929in}}%
\pgfpathlineto{\pgfqpoint{9.131771in}{1.750953in}}%
\pgfpathlineto{\pgfqpoint{9.147204in}{1.778228in}}%
\pgfpathlineto{\pgfqpoint{9.162638in}{1.801735in}}%
\pgfpathlineto{\pgfqpoint{9.178072in}{1.821328in}}%
\pgfpathlineto{\pgfqpoint{9.193505in}{1.836910in}}%
\pgfpathlineto{\pgfqpoint{9.208939in}{1.848431in}}%
\pgfpathlineto{\pgfqpoint{9.224373in}{1.855892in}}%
\pgfpathlineto{\pgfqpoint{9.239806in}{1.859345in}}%
\pgfpathlineto{\pgfqpoint{9.255240in}{1.858885in}}%
\pgfpathlineto{\pgfqpoint{9.270673in}{1.854656in}}%
\pgfpathlineto{\pgfqpoint{9.286107in}{1.846843in}}%
\pgfpathlineto{\pgfqpoint{9.301541in}{1.835672in}}%
\pgfpathlineto{\pgfqpoint{9.316974in}{1.821404in}}%
\pgfpathlineto{\pgfqpoint{9.332408in}{1.804333in}}%
\pgfpathlineto{\pgfqpoint{9.363275in}{1.763092in}}%
\pgfpathlineto{\pgfqpoint{9.394143in}{1.714768in}}%
\pgfpathlineto{\pgfqpoint{9.517612in}{1.510761in}}%
\pgfpathlineto{\pgfqpoint{9.548479in}{1.470770in}}%
\pgfpathlineto{\pgfqpoint{9.563913in}{1.453895in}}%
\pgfpathlineto{\pgfqpoint{9.579346in}{1.439324in}}%
\pgfpathlineto{\pgfqpoint{9.594780in}{1.427175in}}%
\pgfpathlineto{\pgfqpoint{9.610213in}{1.417525in}}%
\pgfpathlineto{\pgfqpoint{9.625647in}{1.410412in}}%
\pgfpathlineto{\pgfqpoint{9.641081in}{1.405831in}}%
\pgfpathlineto{\pgfqpoint{9.656514in}{1.403738in}}%
\pgfpathlineto{\pgfqpoint{9.671948in}{1.404049in}}%
\pgfpathlineto{\pgfqpoint{9.687382in}{1.406643in}}%
\pgfpathlineto{\pgfqpoint{9.702815in}{1.411368in}}%
\pgfpathlineto{\pgfqpoint{9.718249in}{1.418035in}}%
\pgfpathlineto{\pgfqpoint{9.733682in}{1.426431in}}%
\pgfpathlineto{\pgfqpoint{9.733682in}{1.426431in}}%
\pgfusepath{stroke}%
\end{pgfscope}%
\begin{pgfscope}%
\pgfpathrectangle{\pgfqpoint{5.698559in}{0.505056in}}{\pgfqpoint{4.227273in}{2.745455in}} %
\pgfusepath{clip}%
\pgfsetrectcap%
\pgfsetroundjoin%
\pgfsetlinewidth{0.501875pt}%
\definecolor{currentstroke}{rgb}{0.021569,0.682749,0.930229}%
\pgfsetstrokecolor{currentstroke}%
\pgfsetdash{}{0pt}%
\pgfpathmoveto{\pgfqpoint{5.890707in}{2.353871in}}%
\pgfpathlineto{\pgfqpoint{5.906141in}{2.343997in}}%
\pgfpathlineto{\pgfqpoint{5.921575in}{2.330521in}}%
\pgfpathlineto{\pgfqpoint{5.937008in}{2.313652in}}%
\pgfpathlineto{\pgfqpoint{5.952442in}{2.293654in}}%
\pgfpathlineto{\pgfqpoint{5.983309in}{2.245570in}}%
\pgfpathlineto{\pgfqpoint{6.014176in}{2.189297in}}%
\pgfpathlineto{\pgfqpoint{6.106778in}{2.008775in}}%
\pgfpathlineto{\pgfqpoint{6.137645in}{1.958074in}}%
\pgfpathlineto{\pgfqpoint{6.153079in}{1.936665in}}%
\pgfpathlineto{\pgfqpoint{6.168513in}{1.918444in}}%
\pgfpathlineto{\pgfqpoint{6.183946in}{1.903782in}}%
\pgfpathlineto{\pgfqpoint{6.199380in}{1.893005in}}%
\pgfpathlineto{\pgfqpoint{6.214814in}{1.886389in}}%
\pgfpathlineto{\pgfqpoint{6.230247in}{1.884155in}}%
\pgfpathlineto{\pgfqpoint{6.245681in}{1.886465in}}%
\pgfpathlineto{\pgfqpoint{6.261115in}{1.893421in}}%
\pgfpathlineto{\pgfqpoint{6.276548in}{1.905060in}}%
\pgfpathlineto{\pgfqpoint{6.291982in}{1.921358in}}%
\pgfpathlineto{\pgfqpoint{6.307415in}{1.942225in}}%
\pgfpathlineto{\pgfqpoint{6.322849in}{1.967507in}}%
\pgfpathlineto{\pgfqpoint{6.338283in}{1.996989in}}%
\pgfpathlineto{\pgfqpoint{6.369150in}{2.067403in}}%
\pgfpathlineto{\pgfqpoint{6.400017in}{2.150624in}}%
\pgfpathlineto{\pgfqpoint{6.430885in}{2.243061in}}%
\pgfpathlineto{\pgfqpoint{6.523486in}{2.532542in}}%
\pgfpathlineto{\pgfqpoint{6.554354in}{2.617903in}}%
\pgfpathlineto{\pgfqpoint{6.585221in}{2.690528in}}%
\pgfpathlineto{\pgfqpoint{6.600655in}{2.720923in}}%
\pgfpathlineto{\pgfqpoint{6.616088in}{2.746851in}}%
\pgfpathlineto{\pgfqpoint{6.631522in}{2.767990in}}%
\pgfpathlineto{\pgfqpoint{6.646955in}{2.784070in}}%
\pgfpathlineto{\pgfqpoint{6.662389in}{2.794885in}}%
\pgfpathlineto{\pgfqpoint{6.677823in}{2.800291in}}%
\pgfpathlineto{\pgfqpoint{6.693256in}{2.800211in}}%
\pgfpathlineto{\pgfqpoint{6.708690in}{2.794630in}}%
\pgfpathlineto{\pgfqpoint{6.724124in}{2.783604in}}%
\pgfpathlineto{\pgfqpoint{6.739557in}{2.767250in}}%
\pgfpathlineto{\pgfqpoint{6.754991in}{2.745750in}}%
\pgfpathlineto{\pgfqpoint{6.770424in}{2.719347in}}%
\pgfpathlineto{\pgfqpoint{6.785858in}{2.688342in}}%
\pgfpathlineto{\pgfqpoint{6.801292in}{2.653086in}}%
\pgfpathlineto{\pgfqpoint{6.832159in}{2.571473in}}%
\pgfpathlineto{\pgfqpoint{6.863026in}{2.478191in}}%
\pgfpathlineto{\pgfqpoint{6.909327in}{2.325561in}}%
\pgfpathlineto{\pgfqpoint{6.971062in}{2.121467in}}%
\pgfpathlineto{\pgfqpoint{7.001929in}{2.028609in}}%
\pgfpathlineto{\pgfqpoint{7.032796in}{1.946661in}}%
\pgfpathlineto{\pgfqpoint{7.063664in}{1.878352in}}%
\pgfpathlineto{\pgfqpoint{7.079097in}{1.849930in}}%
\pgfpathlineto{\pgfqpoint{7.094531in}{1.825546in}}%
\pgfpathlineto{\pgfqpoint{7.109964in}{1.805282in}}%
\pgfpathlineto{\pgfqpoint{7.125398in}{1.789162in}}%
\pgfpathlineto{\pgfqpoint{7.140832in}{1.777149in}}%
\pgfpathlineto{\pgfqpoint{7.156265in}{1.769143in}}%
\pgfpathlineto{\pgfqpoint{7.171699in}{1.764988in}}%
\pgfpathlineto{\pgfqpoint{7.187133in}{1.764472in}}%
\pgfpathlineto{\pgfqpoint{7.202566in}{1.767332in}}%
\pgfpathlineto{\pgfqpoint{7.218000in}{1.773256in}}%
\pgfpathlineto{\pgfqpoint{7.233434in}{1.781893in}}%
\pgfpathlineto{\pgfqpoint{7.248867in}{1.792850in}}%
\pgfpathlineto{\pgfqpoint{7.279734in}{1.820022in}}%
\pgfpathlineto{\pgfqpoint{7.372336in}{1.910261in}}%
\pgfpathlineto{\pgfqpoint{7.387770in}{1.921740in}}%
\pgfpathlineto{\pgfqpoint{7.403204in}{1.931083in}}%
\pgfpathlineto{\pgfqpoint{7.418637in}{1.937942in}}%
\pgfpathlineto{\pgfqpoint{7.434071in}{1.942008in}}%
\pgfpathlineto{\pgfqpoint{7.449504in}{1.943019in}}%
\pgfpathlineto{\pgfqpoint{7.464938in}{1.940765in}}%
\pgfpathlineto{\pgfqpoint{7.480372in}{1.935090in}}%
\pgfpathlineto{\pgfqpoint{7.495805in}{1.925892in}}%
\pgfpathlineto{\pgfqpoint{7.511239in}{1.913129in}}%
\pgfpathlineto{\pgfqpoint{7.526673in}{1.896818in}}%
\pgfpathlineto{\pgfqpoint{7.542106in}{1.877033in}}%
\pgfpathlineto{\pgfqpoint{7.557540in}{1.853904in}}%
\pgfpathlineto{\pgfqpoint{7.588407in}{1.798422in}}%
\pgfpathlineto{\pgfqpoint{7.619274in}{1.732487in}}%
\pgfpathlineto{\pgfqpoint{7.650142in}{1.658931in}}%
\pgfpathlineto{\pgfqpoint{7.742744in}{1.428062in}}%
\pgfpathlineto{\pgfqpoint{7.773611in}{1.360466in}}%
\pgfpathlineto{\pgfqpoint{7.804478in}{1.303607in}}%
\pgfpathlineto{\pgfqpoint{7.819912in}{1.280162in}}%
\pgfpathlineto{\pgfqpoint{7.835345in}{1.260470in}}%
\pgfpathlineto{\pgfqpoint{7.850779in}{1.244792in}}%
\pgfpathlineto{\pgfqpoint{7.866213in}{1.233339in}}%
\pgfpathlineto{\pgfqpoint{7.881646in}{1.226264in}}%
\pgfpathlineto{\pgfqpoint{7.897080in}{1.223661in}}%
\pgfpathlineto{\pgfqpoint{7.912514in}{1.225563in}}%
\pgfpathlineto{\pgfqpoint{7.927947in}{1.231943in}}%
\pgfpathlineto{\pgfqpoint{7.943381in}{1.242711in}}%
\pgfpathlineto{\pgfqpoint{7.958814in}{1.257717in}}%
\pgfpathlineto{\pgfqpoint{7.974248in}{1.276750in}}%
\pgfpathlineto{\pgfqpoint{7.989682in}{1.299544in}}%
\pgfpathlineto{\pgfqpoint{8.005115in}{1.325780in}}%
\pgfpathlineto{\pgfqpoint{8.035983in}{1.387057in}}%
\pgfpathlineto{\pgfqpoint{8.066850in}{1.457127in}}%
\pgfpathlineto{\pgfqpoint{8.159452in}{1.678986in}}%
\pgfpathlineto{\pgfqpoint{8.190319in}{1.742484in}}%
\pgfpathlineto{\pgfqpoint{8.205753in}{1.769990in}}%
\pgfpathlineto{\pgfqpoint{8.221186in}{1.794085in}}%
\pgfpathlineto{\pgfqpoint{8.236620in}{1.814384in}}%
\pgfpathlineto{\pgfqpoint{8.252053in}{1.830552in}}%
\pgfpathlineto{\pgfqpoint{8.267487in}{1.842308in}}%
\pgfpathlineto{\pgfqpoint{8.282921in}{1.849425in}}%
\pgfpathlineto{\pgfqpoint{8.298354in}{1.851742in}}%
\pgfpathlineto{\pgfqpoint{8.313788in}{1.849155in}}%
\pgfpathlineto{\pgfqpoint{8.329222in}{1.841630in}}%
\pgfpathlineto{\pgfqpoint{8.344655in}{1.829194in}}%
\pgfpathlineto{\pgfqpoint{8.360089in}{1.811941in}}%
\pgfpathlineto{\pgfqpoint{8.375523in}{1.790029in}}%
\pgfpathlineto{\pgfqpoint{8.390956in}{1.763676in}}%
\pgfpathlineto{\pgfqpoint{8.406390in}{1.733161in}}%
\pgfpathlineto{\pgfqpoint{8.437257in}{1.661028in}}%
\pgfpathlineto{\pgfqpoint{8.468124in}{1.576879in}}%
\pgfpathlineto{\pgfqpoint{8.514425in}{1.436752in}}%
\pgfpathlineto{\pgfqpoint{8.576160in}{1.247141in}}%
\pgfpathlineto{\pgfqpoint{8.607027in}{1.161199in}}%
\pgfpathlineto{\pgfqpoint{8.637894in}{1.086635in}}%
\pgfpathlineto{\pgfqpoint{8.653328in}{1.054715in}}%
\pgfpathlineto{\pgfqpoint{8.668762in}{1.026865in}}%
\pgfpathlineto{\pgfqpoint{8.684195in}{1.003389in}}%
\pgfpathlineto{\pgfqpoint{8.699629in}{0.984538in}}%
\pgfpathlineto{\pgfqpoint{8.715063in}{0.970502in}}%
\pgfpathlineto{\pgfqpoint{8.730496in}{0.961409in}}%
\pgfpathlineto{\pgfqpoint{8.745930in}{0.957324in}}%
\pgfpathlineto{\pgfqpoint{8.761363in}{0.958248in}}%
\pgfpathlineto{\pgfqpoint{8.776797in}{0.964122in}}%
\pgfpathlineto{\pgfqpoint{8.792231in}{0.974820in}}%
\pgfpathlineto{\pgfqpoint{8.807664in}{0.990161in}}%
\pgfpathlineto{\pgfqpoint{8.823098in}{1.009904in}}%
\pgfpathlineto{\pgfqpoint{8.838532in}{1.033756in}}%
\pgfpathlineto{\pgfqpoint{8.853965in}{1.061376in}}%
\pgfpathlineto{\pgfqpoint{8.884833in}{1.126339in}}%
\pgfpathlineto{\pgfqpoint{8.915700in}{1.201292in}}%
\pgfpathlineto{\pgfqpoint{8.977434in}{1.365361in}}%
\pgfpathlineto{\pgfqpoint{9.023735in}{1.484892in}}%
\pgfpathlineto{\pgfqpoint{9.054603in}{1.555904in}}%
\pgfpathlineto{\pgfqpoint{9.085470in}{1.616415in}}%
\pgfpathlineto{\pgfqpoint{9.100903in}{1.641997in}}%
\pgfpathlineto{\pgfqpoint{9.116337in}{1.664175in}}%
\pgfpathlineto{\pgfqpoint{9.131771in}{1.682799in}}%
\pgfpathlineto{\pgfqpoint{9.147204in}{1.697780in}}%
\pgfpathlineto{\pgfqpoint{9.162638in}{1.709085in}}%
\pgfpathlineto{\pgfqpoint{9.178072in}{1.716740in}}%
\pgfpathlineto{\pgfqpoint{9.193505in}{1.720827in}}%
\pgfpathlineto{\pgfqpoint{9.208939in}{1.721486in}}%
\pgfpathlineto{\pgfqpoint{9.224373in}{1.718909in}}%
\pgfpathlineto{\pgfqpoint{9.239806in}{1.713339in}}%
\pgfpathlineto{\pgfqpoint{9.255240in}{1.705062in}}%
\pgfpathlineto{\pgfqpoint{9.270673in}{1.694409in}}%
\pgfpathlineto{\pgfqpoint{9.301541in}{1.667468in}}%
\pgfpathlineto{\pgfqpoint{9.347842in}{1.619238in}}%
\pgfpathlineto{\pgfqpoint{9.394143in}{1.572319in}}%
\pgfpathlineto{\pgfqpoint{9.425010in}{1.547618in}}%
\pgfpathlineto{\pgfqpoint{9.440443in}{1.538439in}}%
\pgfpathlineto{\pgfqpoint{9.455877in}{1.531836in}}%
\pgfpathlineto{\pgfqpoint{9.471311in}{1.528107in}}%
\pgfpathlineto{\pgfqpoint{9.486744in}{1.527504in}}%
\pgfpathlineto{\pgfqpoint{9.502178in}{1.530231in}}%
\pgfpathlineto{\pgfqpoint{9.517612in}{1.536440in}}%
\pgfpathlineto{\pgfqpoint{9.533045in}{1.546228in}}%
\pgfpathlineto{\pgfqpoint{9.548479in}{1.559636in}}%
\pgfpathlineto{\pgfqpoint{9.563913in}{1.576647in}}%
\pgfpathlineto{\pgfqpoint{9.579346in}{1.597188in}}%
\pgfpathlineto{\pgfqpoint{9.594780in}{1.621129in}}%
\pgfpathlineto{\pgfqpoint{9.625647in}{1.678411in}}%
\pgfpathlineto{\pgfqpoint{9.656514in}{1.746395in}}%
\pgfpathlineto{\pgfqpoint{9.687382in}{1.822256in}}%
\pgfpathlineto{\pgfqpoint{9.733682in}{1.943274in}}%
\pgfpathlineto{\pgfqpoint{9.733682in}{1.943274in}}%
\pgfusepath{stroke}%
\end{pgfscope}%
\begin{pgfscope}%
\pgfpathrectangle{\pgfqpoint{5.698559in}{0.505056in}}{\pgfqpoint{4.227273in}{2.745455in}} %
\pgfusepath{clip}%
\pgfsetrectcap%
\pgfsetroundjoin%
\pgfsetlinewidth{0.501875pt}%
\definecolor{currentstroke}{rgb}{0.056863,0.767363,0.905873}%
\pgfsetstrokecolor{currentstroke}%
\pgfsetdash{}{0pt}%
\pgfpathmoveto{\pgfqpoint{5.890707in}{2.138130in}}%
\pgfpathlineto{\pgfqpoint{5.906141in}{2.130990in}}%
\pgfpathlineto{\pgfqpoint{5.921575in}{2.121394in}}%
\pgfpathlineto{\pgfqpoint{5.937008in}{2.109475in}}%
\pgfpathlineto{\pgfqpoint{5.967875in}{2.079320in}}%
\pgfpathlineto{\pgfqpoint{5.998743in}{2.042024in}}%
\pgfpathlineto{\pgfqpoint{6.029610in}{1.999387in}}%
\pgfpathlineto{\pgfqpoint{6.137645in}{1.842026in}}%
\pgfpathlineto{\pgfqpoint{6.168513in}{1.806878in}}%
\pgfpathlineto{\pgfqpoint{6.183946in}{1.793060in}}%
\pgfpathlineto{\pgfqpoint{6.199380in}{1.782335in}}%
\pgfpathlineto{\pgfqpoint{6.214814in}{1.775128in}}%
\pgfpathlineto{\pgfqpoint{6.230247in}{1.771838in}}%
\pgfpathlineto{\pgfqpoint{6.245681in}{1.772824in}}%
\pgfpathlineto{\pgfqpoint{6.261115in}{1.778393in}}%
\pgfpathlineto{\pgfqpoint{6.276548in}{1.788787in}}%
\pgfpathlineto{\pgfqpoint{6.291982in}{1.804174in}}%
\pgfpathlineto{\pgfqpoint{6.307415in}{1.824633in}}%
\pgfpathlineto{\pgfqpoint{6.322849in}{1.850145in}}%
\pgfpathlineto{\pgfqpoint{6.338283in}{1.880591in}}%
\pgfpathlineto{\pgfqpoint{6.353716in}{1.915740in}}%
\pgfpathlineto{\pgfqpoint{6.384584in}{1.998689in}}%
\pgfpathlineto{\pgfqpoint{6.415451in}{2.095027in}}%
\pgfpathlineto{\pgfqpoint{6.523486in}{2.451349in}}%
\pgfpathlineto{\pgfqpoint{6.554354in}{2.531206in}}%
\pgfpathlineto{\pgfqpoint{6.569787in}{2.563819in}}%
\pgfpathlineto{\pgfqpoint{6.585221in}{2.590900in}}%
\pgfpathlineto{\pgfqpoint{6.600655in}{2.612113in}}%
\pgfpathlineto{\pgfqpoint{6.616088in}{2.627257in}}%
\pgfpathlineto{\pgfqpoint{6.631522in}{2.636274in}}%
\pgfpathlineto{\pgfqpoint{6.646955in}{2.639243in}}%
\pgfpathlineto{\pgfqpoint{6.662389in}{2.636380in}}%
\pgfpathlineto{\pgfqpoint{6.677823in}{2.628025in}}%
\pgfpathlineto{\pgfqpoint{6.693256in}{2.614627in}}%
\pgfpathlineto{\pgfqpoint{6.708690in}{2.596731in}}%
\pgfpathlineto{\pgfqpoint{6.724124in}{2.574955in}}%
\pgfpathlineto{\pgfqpoint{6.754991in}{2.522468in}}%
\pgfpathlineto{\pgfqpoint{6.801292in}{2.431719in}}%
\pgfpathlineto{\pgfqpoint{6.847593in}{2.340982in}}%
\pgfpathlineto{\pgfqpoint{6.878460in}{2.285873in}}%
\pgfpathlineto{\pgfqpoint{6.909327in}{2.236394in}}%
\pgfpathlineto{\pgfqpoint{6.955628in}{2.170717in}}%
\pgfpathlineto{\pgfqpoint{7.017363in}{2.086541in}}%
\pgfpathlineto{\pgfqpoint{7.048230in}{2.039844in}}%
\pgfpathlineto{\pgfqpoint{7.079097in}{1.987743in}}%
\pgfpathlineto{\pgfqpoint{7.109964in}{1.930264in}}%
\pgfpathlineto{\pgfqpoint{7.202566in}{1.750918in}}%
\pgfpathlineto{\pgfqpoint{7.218000in}{1.725695in}}%
\pgfpathlineto{\pgfqpoint{7.233434in}{1.703481in}}%
\pgfpathlineto{\pgfqpoint{7.248867in}{1.684857in}}%
\pgfpathlineto{\pgfqpoint{7.264301in}{1.670327in}}%
\pgfpathlineto{\pgfqpoint{7.279734in}{1.660294in}}%
\pgfpathlineto{\pgfqpoint{7.295168in}{1.655033in}}%
\pgfpathlineto{\pgfqpoint{7.310602in}{1.654679in}}%
\pgfpathlineto{\pgfqpoint{7.326035in}{1.659209in}}%
\pgfpathlineto{\pgfqpoint{7.341469in}{1.668433in}}%
\pgfpathlineto{\pgfqpoint{7.356903in}{1.681996in}}%
\pgfpathlineto{\pgfqpoint{7.372336in}{1.699381in}}%
\pgfpathlineto{\pgfqpoint{7.403204in}{1.742800in}}%
\pgfpathlineto{\pgfqpoint{7.464938in}{1.838455in}}%
\pgfpathlineto{\pgfqpoint{7.480372in}{1.858178in}}%
\pgfpathlineto{\pgfqpoint{7.495805in}{1.874184in}}%
\pgfpathlineto{\pgfqpoint{7.511239in}{1.885575in}}%
\pgfpathlineto{\pgfqpoint{7.526673in}{1.891583in}}%
\pgfpathlineto{\pgfqpoint{7.542106in}{1.891596in}}%
\pgfpathlineto{\pgfqpoint{7.557540in}{1.885185in}}%
\pgfpathlineto{\pgfqpoint{7.572974in}{1.872122in}}%
\pgfpathlineto{\pgfqpoint{7.588407in}{1.852395in}}%
\pgfpathlineto{\pgfqpoint{7.603841in}{1.826210in}}%
\pgfpathlineto{\pgfqpoint{7.619274in}{1.793995in}}%
\pgfpathlineto{\pgfqpoint{7.634708in}{1.756384in}}%
\pgfpathlineto{\pgfqpoint{7.665575in}{1.668447in}}%
\pgfpathlineto{\pgfqpoint{7.742744in}{1.428467in}}%
\pgfpathlineto{\pgfqpoint{7.758177in}{1.387259in}}%
\pgfpathlineto{\pgfqpoint{7.773611in}{1.351155in}}%
\pgfpathlineto{\pgfqpoint{7.789044in}{1.321159in}}%
\pgfpathlineto{\pgfqpoint{7.804478in}{1.298102in}}%
\pgfpathlineto{\pgfqpoint{7.819912in}{1.282619in}}%
\pgfpathlineto{\pgfqpoint{7.835345in}{1.275128in}}%
\pgfpathlineto{\pgfqpoint{7.850779in}{1.275817in}}%
\pgfpathlineto{\pgfqpoint{7.866213in}{1.284642in}}%
\pgfpathlineto{\pgfqpoint{7.881646in}{1.301327in}}%
\pgfpathlineto{\pgfqpoint{7.897080in}{1.325378in}}%
\pgfpathlineto{\pgfqpoint{7.912514in}{1.356099in}}%
\pgfpathlineto{\pgfqpoint{7.927947in}{1.392620in}}%
\pgfpathlineto{\pgfqpoint{7.958814in}{1.478888in}}%
\pgfpathlineto{\pgfqpoint{8.051416in}{1.758502in}}%
\pgfpathlineto{\pgfqpoint{8.066850in}{1.796539in}}%
\pgfpathlineto{\pgfqpoint{8.082284in}{1.829890in}}%
\pgfpathlineto{\pgfqpoint{8.097717in}{1.858082in}}%
\pgfpathlineto{\pgfqpoint{8.113151in}{1.880832in}}%
\pgfpathlineto{\pgfqpoint{8.128584in}{1.898046in}}%
\pgfpathlineto{\pgfqpoint{8.144018in}{1.909813in}}%
\pgfpathlineto{\pgfqpoint{8.159452in}{1.916394in}}%
\pgfpathlineto{\pgfqpoint{8.174885in}{1.918198in}}%
\pgfpathlineto{\pgfqpoint{8.190319in}{1.915760in}}%
\pgfpathlineto{\pgfqpoint{8.205753in}{1.909712in}}%
\pgfpathlineto{\pgfqpoint{8.221186in}{1.900749in}}%
\pgfpathlineto{\pgfqpoint{8.252053in}{1.876980in}}%
\pgfpathlineto{\pgfqpoint{8.313788in}{1.824485in}}%
\pgfpathlineto{\pgfqpoint{8.344655in}{1.803110in}}%
\pgfpathlineto{\pgfqpoint{8.375523in}{1.786518in}}%
\pgfpathlineto{\pgfqpoint{8.452691in}{1.751283in}}%
\pgfpathlineto{\pgfqpoint{8.468124in}{1.741281in}}%
\pgfpathlineto{\pgfqpoint{8.483558in}{1.728948in}}%
\pgfpathlineto{\pgfqpoint{8.498992in}{1.713763in}}%
\pgfpathlineto{\pgfqpoint{8.514425in}{1.695288in}}%
\pgfpathlineto{\pgfqpoint{8.529859in}{1.673191in}}%
\pgfpathlineto{\pgfqpoint{8.545293in}{1.647272in}}%
\pgfpathlineto{\pgfqpoint{8.560726in}{1.617479in}}%
\pgfpathlineto{\pgfqpoint{8.591593in}{1.546855in}}%
\pgfpathlineto{\pgfqpoint{8.622461in}{1.464117in}}%
\pgfpathlineto{\pgfqpoint{8.699629in}{1.242876in}}%
\pgfpathlineto{\pgfqpoint{8.730496in}{1.168907in}}%
\pgfpathlineto{\pgfqpoint{8.745930in}{1.139005in}}%
\pgfpathlineto{\pgfqpoint{8.761363in}{1.114957in}}%
\pgfpathlineto{\pgfqpoint{8.776797in}{1.097454in}}%
\pgfpathlineto{\pgfqpoint{8.792231in}{1.087031in}}%
\pgfpathlineto{\pgfqpoint{8.807664in}{1.084047in}}%
\pgfpathlineto{\pgfqpoint{8.823098in}{1.088669in}}%
\pgfpathlineto{\pgfqpoint{8.838532in}{1.100867in}}%
\pgfpathlineto{\pgfqpoint{8.853965in}{1.120415in}}%
\pgfpathlineto{\pgfqpoint{8.869399in}{1.146893in}}%
\pgfpathlineto{\pgfqpoint{8.884833in}{1.179702in}}%
\pgfpathlineto{\pgfqpoint{8.900266in}{1.218081in}}%
\pgfpathlineto{\pgfqpoint{8.931133in}{1.307845in}}%
\pgfpathlineto{\pgfqpoint{9.023735in}{1.602894in}}%
\pgfpathlineto{\pgfqpoint{9.039169in}{1.644633in}}%
\pgfpathlineto{\pgfqpoint{9.054603in}{1.681922in}}%
\pgfpathlineto{\pgfqpoint{9.070036in}{1.714189in}}%
\pgfpathlineto{\pgfqpoint{9.085470in}{1.741027in}}%
\pgfpathlineto{\pgfqpoint{9.100903in}{1.762203in}}%
\pgfpathlineto{\pgfqpoint{9.116337in}{1.777657in}}%
\pgfpathlineto{\pgfqpoint{9.131771in}{1.787498in}}%
\pgfpathlineto{\pgfqpoint{9.147204in}{1.791989in}}%
\pgfpathlineto{\pgfqpoint{9.162638in}{1.791538in}}%
\pgfpathlineto{\pgfqpoint{9.178072in}{1.786676in}}%
\pgfpathlineto{\pgfqpoint{9.193505in}{1.778032in}}%
\pgfpathlineto{\pgfqpoint{9.208939in}{1.766313in}}%
\pgfpathlineto{\pgfqpoint{9.239806in}{1.736690in}}%
\pgfpathlineto{\pgfqpoint{9.286107in}{1.688265in}}%
\pgfpathlineto{\pgfqpoint{9.316974in}{1.661323in}}%
\pgfpathlineto{\pgfqpoint{9.332408in}{1.651075in}}%
\pgfpathlineto{\pgfqpoint{9.347842in}{1.643472in}}%
\pgfpathlineto{\pgfqpoint{9.363275in}{1.638757in}}%
\pgfpathlineto{\pgfqpoint{9.378709in}{1.637066in}}%
\pgfpathlineto{\pgfqpoint{9.394143in}{1.638438in}}%
\pgfpathlineto{\pgfqpoint{9.409576in}{1.642818in}}%
\pgfpathlineto{\pgfqpoint{9.425010in}{1.650074in}}%
\pgfpathlineto{\pgfqpoint{9.440443in}{1.660003in}}%
\pgfpathlineto{\pgfqpoint{9.455877in}{1.672352in}}%
\pgfpathlineto{\pgfqpoint{9.486744in}{1.703127in}}%
\pgfpathlineto{\pgfqpoint{9.517612in}{1.739900in}}%
\pgfpathlineto{\pgfqpoint{9.563913in}{1.801151in}}%
\pgfpathlineto{\pgfqpoint{9.687382in}{1.967294in}}%
\pgfpathlineto{\pgfqpoint{9.733682in}{2.026602in}}%
\pgfpathlineto{\pgfqpoint{9.733682in}{2.026602in}}%
\pgfusepath{stroke}%
\end{pgfscope}%
\begin{pgfscope}%
\pgfpathrectangle{\pgfqpoint{5.698559in}{0.505056in}}{\pgfqpoint{4.227273in}{2.745455in}} %
\pgfusepath{clip}%
\pgfsetrectcap%
\pgfsetroundjoin%
\pgfsetlinewidth{0.501875pt}%
\definecolor{currentstroke}{rgb}{0.135294,0.840344,0.878081}%
\pgfsetstrokecolor{currentstroke}%
\pgfsetdash{}{0pt}%
\pgfpathmoveto{\pgfqpoint{5.890707in}{2.376995in}}%
\pgfpathlineto{\pgfqpoint{5.906141in}{2.371587in}}%
\pgfpathlineto{\pgfqpoint{5.921575in}{2.360484in}}%
\pgfpathlineto{\pgfqpoint{5.937008in}{2.343655in}}%
\pgfpathlineto{\pgfqpoint{5.952442in}{2.321177in}}%
\pgfpathlineto{\pgfqpoint{5.967875in}{2.293244in}}%
\pgfpathlineto{\pgfqpoint{5.983309in}{2.260158in}}%
\pgfpathlineto{\pgfqpoint{5.998743in}{2.222336in}}%
\pgfpathlineto{\pgfqpoint{6.029610in}{2.134652in}}%
\pgfpathlineto{\pgfqpoint{6.060477in}{2.035448in}}%
\pgfpathlineto{\pgfqpoint{6.137645in}{1.780696in}}%
\pgfpathlineto{\pgfqpoint{6.168513in}{1.694758in}}%
\pgfpathlineto{\pgfqpoint{6.183946in}{1.658537in}}%
\pgfpathlineto{\pgfqpoint{6.199380in}{1.627741in}}%
\pgfpathlineto{\pgfqpoint{6.214814in}{1.602948in}}%
\pgfpathlineto{\pgfqpoint{6.230247in}{1.584634in}}%
\pgfpathlineto{\pgfqpoint{6.245681in}{1.573157in}}%
\pgfpathlineto{\pgfqpoint{6.261115in}{1.568757in}}%
\pgfpathlineto{\pgfqpoint{6.276548in}{1.571547in}}%
\pgfpathlineto{\pgfqpoint{6.291982in}{1.581516in}}%
\pgfpathlineto{\pgfqpoint{6.307415in}{1.598531in}}%
\pgfpathlineto{\pgfqpoint{6.322849in}{1.622339in}}%
\pgfpathlineto{\pgfqpoint{6.338283in}{1.652575in}}%
\pgfpathlineto{\pgfqpoint{6.353716in}{1.688774in}}%
\pgfpathlineto{\pgfqpoint{6.369150in}{1.730379in}}%
\pgfpathlineto{\pgfqpoint{6.400017in}{1.827212in}}%
\pgfpathlineto{\pgfqpoint{6.430885in}{1.937336in}}%
\pgfpathlineto{\pgfqpoint{6.523486in}{2.285150in}}%
\pgfpathlineto{\pgfqpoint{6.554354in}{2.387640in}}%
\pgfpathlineto{\pgfqpoint{6.585221in}{2.475755in}}%
\pgfpathlineto{\pgfqpoint{6.600655in}{2.513447in}}%
\pgfpathlineto{\pgfqpoint{6.616088in}{2.546522in}}%
\pgfpathlineto{\pgfqpoint{6.631522in}{2.574789in}}%
\pgfpathlineto{\pgfqpoint{6.646955in}{2.598121in}}%
\pgfpathlineto{\pgfqpoint{6.662389in}{2.616458in}}%
\pgfpathlineto{\pgfqpoint{6.677823in}{2.629792in}}%
\pgfpathlineto{\pgfqpoint{6.693256in}{2.638169in}}%
\pgfpathlineto{\pgfqpoint{6.708690in}{2.641678in}}%
\pgfpathlineto{\pgfqpoint{6.724124in}{2.640446in}}%
\pgfpathlineto{\pgfqpoint{6.739557in}{2.634632in}}%
\pgfpathlineto{\pgfqpoint{6.754991in}{2.624419in}}%
\pgfpathlineto{\pgfqpoint{6.770424in}{2.610013in}}%
\pgfpathlineto{\pgfqpoint{6.785858in}{2.591633in}}%
\pgfpathlineto{\pgfqpoint{6.801292in}{2.569514in}}%
\pgfpathlineto{\pgfqpoint{6.816725in}{2.543897in}}%
\pgfpathlineto{\pgfqpoint{6.847593in}{2.483181in}}%
\pgfpathlineto{\pgfqpoint{6.878460in}{2.411568in}}%
\pgfpathlineto{\pgfqpoint{6.909327in}{2.331265in}}%
\pgfpathlineto{\pgfqpoint{6.955628in}{2.199723in}}%
\pgfpathlineto{\pgfqpoint{7.048230in}{1.930311in}}%
\pgfpathlineto{\pgfqpoint{7.079097in}{1.849657in}}%
\pgfpathlineto{\pgfqpoint{7.109964in}{1.778701in}}%
\pgfpathlineto{\pgfqpoint{7.140832in}{1.720189in}}%
\pgfpathlineto{\pgfqpoint{7.156265in}{1.696315in}}%
\pgfpathlineto{\pgfqpoint{7.171699in}{1.676322in}}%
\pgfpathlineto{\pgfqpoint{7.187133in}{1.660360in}}%
\pgfpathlineto{\pgfqpoint{7.202566in}{1.648517in}}%
\pgfpathlineto{\pgfqpoint{7.218000in}{1.640816in}}%
\pgfpathlineto{\pgfqpoint{7.233434in}{1.637205in}}%
\pgfpathlineto{\pgfqpoint{7.248867in}{1.637562in}}%
\pgfpathlineto{\pgfqpoint{7.264301in}{1.641689in}}%
\pgfpathlineto{\pgfqpoint{7.279734in}{1.649318in}}%
\pgfpathlineto{\pgfqpoint{7.295168in}{1.660110in}}%
\pgfpathlineto{\pgfqpoint{7.310602in}{1.673662in}}%
\pgfpathlineto{\pgfqpoint{7.341469in}{1.707161in}}%
\pgfpathlineto{\pgfqpoint{7.434071in}{1.818909in}}%
\pgfpathlineto{\pgfqpoint{7.449504in}{1.833206in}}%
\pgfpathlineto{\pgfqpoint{7.464938in}{1.844905in}}%
\pgfpathlineto{\pgfqpoint{7.480372in}{1.853615in}}%
\pgfpathlineto{\pgfqpoint{7.495805in}{1.859015in}}%
\pgfpathlineto{\pgfqpoint{7.511239in}{1.860869in}}%
\pgfpathlineto{\pgfqpoint{7.526673in}{1.859029in}}%
\pgfpathlineto{\pgfqpoint{7.542106in}{1.853440in}}%
\pgfpathlineto{\pgfqpoint{7.557540in}{1.844140in}}%
\pgfpathlineto{\pgfqpoint{7.572974in}{1.831266in}}%
\pgfpathlineto{\pgfqpoint{7.588407in}{1.815041in}}%
\pgfpathlineto{\pgfqpoint{7.603841in}{1.795777in}}%
\pgfpathlineto{\pgfqpoint{7.634708in}{1.749759in}}%
\pgfpathlineto{\pgfqpoint{7.681009in}{1.669654in}}%
\pgfpathlineto{\pgfqpoint{7.727310in}{1.590233in}}%
\pgfpathlineto{\pgfqpoint{7.758177in}{1.545433in}}%
\pgfpathlineto{\pgfqpoint{7.773611in}{1.527009in}}%
\pgfpathlineto{\pgfqpoint{7.789044in}{1.511781in}}%
\pgfpathlineto{\pgfqpoint{7.804478in}{1.500066in}}%
\pgfpathlineto{\pgfqpoint{7.819912in}{1.492098in}}%
\pgfpathlineto{\pgfqpoint{7.835345in}{1.488024in}}%
\pgfpathlineto{\pgfqpoint{7.850779in}{1.487900in}}%
\pgfpathlineto{\pgfqpoint{7.866213in}{1.491694in}}%
\pgfpathlineto{\pgfqpoint{7.881646in}{1.499287in}}%
\pgfpathlineto{\pgfqpoint{7.897080in}{1.510477in}}%
\pgfpathlineto{\pgfqpoint{7.912514in}{1.524987in}}%
\pgfpathlineto{\pgfqpoint{7.927947in}{1.542473in}}%
\pgfpathlineto{\pgfqpoint{7.958814in}{1.584733in}}%
\pgfpathlineto{\pgfqpoint{7.989682in}{1.633587in}}%
\pgfpathlineto{\pgfqpoint{8.051416in}{1.735205in}}%
\pgfpathlineto{\pgfqpoint{8.082284in}{1.780571in}}%
\pgfpathlineto{\pgfqpoint{8.113151in}{1.818370in}}%
\pgfpathlineto{\pgfqpoint{8.128584in}{1.833802in}}%
\pgfpathlineto{\pgfqpoint{8.144018in}{1.846701in}}%
\pgfpathlineto{\pgfqpoint{8.159452in}{1.856973in}}%
\pgfpathlineto{\pgfqpoint{8.174885in}{1.864579in}}%
\pgfpathlineto{\pgfqpoint{8.190319in}{1.869530in}}%
\pgfpathlineto{\pgfqpoint{8.205753in}{1.871878in}}%
\pgfpathlineto{\pgfqpoint{8.221186in}{1.871716in}}%
\pgfpathlineto{\pgfqpoint{8.236620in}{1.869162in}}%
\pgfpathlineto{\pgfqpoint{8.252053in}{1.864358in}}%
\pgfpathlineto{\pgfqpoint{8.267487in}{1.857459in}}%
\pgfpathlineto{\pgfqpoint{8.298354in}{1.838014in}}%
\pgfpathlineto{\pgfqpoint{8.329222in}{1.812052in}}%
\pgfpathlineto{\pgfqpoint{8.360089in}{1.780595in}}%
\pgfpathlineto{\pgfqpoint{8.390956in}{1.744363in}}%
\pgfpathlineto{\pgfqpoint{8.421823in}{1.703764in}}%
\pgfpathlineto{\pgfqpoint{8.452691in}{1.658978in}}%
\pgfpathlineto{\pgfqpoint{8.483558in}{1.610123in}}%
\pgfpathlineto{\pgfqpoint{8.529859in}{1.529883in}}%
\pgfpathlineto{\pgfqpoint{8.653328in}{1.304367in}}%
\pgfpathlineto{\pgfqpoint{8.684195in}{1.258466in}}%
\pgfpathlineto{\pgfqpoint{8.699629in}{1.239381in}}%
\pgfpathlineto{\pgfqpoint{8.715063in}{1.223475in}}%
\pgfpathlineto{\pgfqpoint{8.730496in}{1.211178in}}%
\pgfpathlineto{\pgfqpoint{8.745930in}{1.202890in}}%
\pgfpathlineto{\pgfqpoint{8.761363in}{1.198965in}}%
\pgfpathlineto{\pgfqpoint{8.776797in}{1.199698in}}%
\pgfpathlineto{\pgfqpoint{8.792231in}{1.205316in}}%
\pgfpathlineto{\pgfqpoint{8.807664in}{1.215968in}}%
\pgfpathlineto{\pgfqpoint{8.823098in}{1.231713in}}%
\pgfpathlineto{\pgfqpoint{8.838532in}{1.252515in}}%
\pgfpathlineto{\pgfqpoint{8.853965in}{1.278238in}}%
\pgfpathlineto{\pgfqpoint{8.869399in}{1.308646in}}%
\pgfpathlineto{\pgfqpoint{8.884833in}{1.343398in}}%
\pgfpathlineto{\pgfqpoint{8.915700in}{1.424099in}}%
\pgfpathlineto{\pgfqpoint{8.946567in}{1.515795in}}%
\pgfpathlineto{\pgfqpoint{9.023735in}{1.754224in}}%
\pgfpathlineto{\pgfqpoint{9.054603in}{1.836418in}}%
\pgfpathlineto{\pgfqpoint{9.070036in}{1.871689in}}%
\pgfpathlineto{\pgfqpoint{9.085470in}{1.902255in}}%
\pgfpathlineto{\pgfqpoint{9.100903in}{1.927619in}}%
\pgfpathlineto{\pgfqpoint{9.116337in}{1.947389in}}%
\pgfpathlineto{\pgfqpoint{9.131771in}{1.961292in}}%
\pgfpathlineto{\pgfqpoint{9.147204in}{1.969181in}}%
\pgfpathlineto{\pgfqpoint{9.162638in}{1.971037in}}%
\pgfpathlineto{\pgfqpoint{9.178072in}{1.966971in}}%
\pgfpathlineto{\pgfqpoint{9.193505in}{1.957222in}}%
\pgfpathlineto{\pgfqpoint{9.208939in}{1.942151in}}%
\pgfpathlineto{\pgfqpoint{9.224373in}{1.922232in}}%
\pgfpathlineto{\pgfqpoint{9.239806in}{1.898044in}}%
\pgfpathlineto{\pgfqpoint{9.270673in}{1.839593in}}%
\pgfpathlineto{\pgfqpoint{9.316974in}{1.738516in}}%
\pgfpathlineto{\pgfqpoint{9.347842in}{1.671932in}}%
\pgfpathlineto{\pgfqpoint{9.378709in}{1.613468in}}%
\pgfpathlineto{\pgfqpoint{9.394143in}{1.589029in}}%
\pgfpathlineto{\pgfqpoint{9.409576in}{1.568559in}}%
\pgfpathlineto{\pgfqpoint{9.425010in}{1.552517in}}%
\pgfpathlineto{\pgfqpoint{9.440443in}{1.541255in}}%
\pgfpathlineto{\pgfqpoint{9.455877in}{1.535013in}}%
\pgfpathlineto{\pgfqpoint{9.471311in}{1.533916in}}%
\pgfpathlineto{\pgfqpoint{9.486744in}{1.537973in}}%
\pgfpathlineto{\pgfqpoint{9.502178in}{1.547084in}}%
\pgfpathlineto{\pgfqpoint{9.517612in}{1.561043in}}%
\pgfpathlineto{\pgfqpoint{9.533045in}{1.579547in}}%
\pgfpathlineto{\pgfqpoint{9.548479in}{1.602207in}}%
\pgfpathlineto{\pgfqpoint{9.563913in}{1.628562in}}%
\pgfpathlineto{\pgfqpoint{9.594780in}{1.690223in}}%
\pgfpathlineto{\pgfqpoint{9.641081in}{1.796244in}}%
\pgfpathlineto{\pgfqpoint{9.687382in}{1.904199in}}%
\pgfpathlineto{\pgfqpoint{9.718249in}{1.970186in}}%
\pgfpathlineto{\pgfqpoint{9.733682in}{2.000130in}}%
\pgfpathlineto{\pgfqpoint{9.733682in}{2.000130in}}%
\pgfusepath{stroke}%
\end{pgfscope}%
\begin{pgfscope}%
\pgfpathrectangle{\pgfqpoint{5.698559in}{0.505056in}}{\pgfqpoint{4.227273in}{2.745455in}} %
\pgfusepath{clip}%
\pgfsetrectcap%
\pgfsetroundjoin%
\pgfsetlinewidth{0.501875pt}%
\definecolor{currentstroke}{rgb}{0.221569,0.905873,0.843667}%
\pgfsetstrokecolor{currentstroke}%
\pgfsetdash{}{0pt}%
\pgfpathmoveto{\pgfqpoint{5.890707in}{2.095747in}}%
\pgfpathlineto{\pgfqpoint{5.906141in}{2.107803in}}%
\pgfpathlineto{\pgfqpoint{5.921575in}{2.117566in}}%
\pgfpathlineto{\pgfqpoint{5.937008in}{2.124833in}}%
\pgfpathlineto{\pgfqpoint{5.952442in}{2.129438in}}%
\pgfpathlineto{\pgfqpoint{5.967875in}{2.131252in}}%
\pgfpathlineto{\pgfqpoint{5.983309in}{2.130194in}}%
\pgfpathlineto{\pgfqpoint{5.998743in}{2.126229in}}%
\pgfpathlineto{\pgfqpoint{6.014176in}{2.119373in}}%
\pgfpathlineto{\pgfqpoint{6.029610in}{2.109698in}}%
\pgfpathlineto{\pgfqpoint{6.045044in}{2.097327in}}%
\pgfpathlineto{\pgfqpoint{6.060477in}{2.082438in}}%
\pgfpathlineto{\pgfqpoint{6.091345in}{2.046080in}}%
\pgfpathlineto{\pgfqpoint{6.122212in}{2.003057in}}%
\pgfpathlineto{\pgfqpoint{6.214814in}{1.867299in}}%
\pgfpathlineto{\pgfqpoint{6.245681in}{1.832410in}}%
\pgfpathlineto{\pgfqpoint{6.261115in}{1.819002in}}%
\pgfpathlineto{\pgfqpoint{6.276548in}{1.808826in}}%
\pgfpathlineto{\pgfqpoint{6.291982in}{1.802239in}}%
\pgfpathlineto{\pgfqpoint{6.307415in}{1.799542in}}%
\pgfpathlineto{\pgfqpoint{6.322849in}{1.800982in}}%
\pgfpathlineto{\pgfqpoint{6.338283in}{1.806743in}}%
\pgfpathlineto{\pgfqpoint{6.353716in}{1.816941in}}%
\pgfpathlineto{\pgfqpoint{6.369150in}{1.831621in}}%
\pgfpathlineto{\pgfqpoint{6.384584in}{1.850754in}}%
\pgfpathlineto{\pgfqpoint{6.400017in}{1.874242in}}%
\pgfpathlineto{\pgfqpoint{6.415451in}{1.901907in}}%
\pgfpathlineto{\pgfqpoint{6.430885in}{1.933506in}}%
\pgfpathlineto{\pgfqpoint{6.461752in}{2.007183in}}%
\pgfpathlineto{\pgfqpoint{6.492619in}{2.092034in}}%
\pgfpathlineto{\pgfqpoint{6.538920in}{2.231241in}}%
\pgfpathlineto{\pgfqpoint{6.585221in}{2.370647in}}%
\pgfpathlineto{\pgfqpoint{6.616088in}{2.455655in}}%
\pgfpathlineto{\pgfqpoint{6.646955in}{2.529109in}}%
\pgfpathlineto{\pgfqpoint{6.662389in}{2.560300in}}%
\pgfpathlineto{\pgfqpoint{6.677823in}{2.587263in}}%
\pgfpathlineto{\pgfqpoint{6.693256in}{2.609674in}}%
\pgfpathlineto{\pgfqpoint{6.708690in}{2.627276in}}%
\pgfpathlineto{\pgfqpoint{6.724124in}{2.639883in}}%
\pgfpathlineto{\pgfqpoint{6.739557in}{2.647383in}}%
\pgfpathlineto{\pgfqpoint{6.754991in}{2.649739in}}%
\pgfpathlineto{\pgfqpoint{6.770424in}{2.646988in}}%
\pgfpathlineto{\pgfqpoint{6.785858in}{2.639237in}}%
\pgfpathlineto{\pgfqpoint{6.801292in}{2.626667in}}%
\pgfpathlineto{\pgfqpoint{6.816725in}{2.609518in}}%
\pgfpathlineto{\pgfqpoint{6.832159in}{2.588092in}}%
\pgfpathlineto{\pgfqpoint{6.847593in}{2.562746in}}%
\pgfpathlineto{\pgfqpoint{6.878460in}{2.501928in}}%
\pgfpathlineto{\pgfqpoint{6.909327in}{2.430669in}}%
\pgfpathlineto{\pgfqpoint{6.955628in}{2.312814in}}%
\pgfpathlineto{\pgfqpoint{7.017363in}{2.155346in}}%
\pgfpathlineto{\pgfqpoint{7.048230in}{2.083856in}}%
\pgfpathlineto{\pgfqpoint{7.079097in}{2.020359in}}%
\pgfpathlineto{\pgfqpoint{7.109964in}{1.966200in}}%
\pgfpathlineto{\pgfqpoint{7.140832in}{1.921866in}}%
\pgfpathlineto{\pgfqpoint{7.171699in}{1.887029in}}%
\pgfpathlineto{\pgfqpoint{7.202566in}{1.860661in}}%
\pgfpathlineto{\pgfqpoint{7.233434in}{1.841195in}}%
\pgfpathlineto{\pgfqpoint{7.264301in}{1.826719in}}%
\pgfpathlineto{\pgfqpoint{7.310602in}{1.809899in}}%
\pgfpathlineto{\pgfqpoint{7.356903in}{1.793320in}}%
\pgfpathlineto{\pgfqpoint{7.387770in}{1.779964in}}%
\pgfpathlineto{\pgfqpoint{7.418637in}{1.763761in}}%
\pgfpathlineto{\pgfqpoint{7.449504in}{1.744467in}}%
\pgfpathlineto{\pgfqpoint{7.495805in}{1.710533in}}%
\pgfpathlineto{\pgfqpoint{7.603841in}{1.626908in}}%
\pgfpathlineto{\pgfqpoint{7.634708in}{1.608246in}}%
\pgfpathlineto{\pgfqpoint{7.665575in}{1.594348in}}%
\pgfpathlineto{\pgfqpoint{7.696443in}{1.586100in}}%
\pgfpathlineto{\pgfqpoint{7.727310in}{1.583964in}}%
\pgfpathlineto{\pgfqpoint{7.758177in}{1.587930in}}%
\pgfpathlineto{\pgfqpoint{7.789044in}{1.597511in}}%
\pgfpathlineto{\pgfqpoint{7.819912in}{1.611786in}}%
\pgfpathlineto{\pgfqpoint{7.850779in}{1.629468in}}%
\pgfpathlineto{\pgfqpoint{7.943381in}{1.686979in}}%
\pgfpathlineto{\pgfqpoint{7.974248in}{1.702135in}}%
\pgfpathlineto{\pgfqpoint{8.005115in}{1.712871in}}%
\pgfpathlineto{\pgfqpoint{8.035983in}{1.718154in}}%
\pgfpathlineto{\pgfqpoint{8.051416in}{1.718528in}}%
\pgfpathlineto{\pgfqpoint{8.066850in}{1.717321in}}%
\pgfpathlineto{\pgfqpoint{8.082284in}{1.714514in}}%
\pgfpathlineto{\pgfqpoint{8.113151in}{1.704138in}}%
\pgfpathlineto{\pgfqpoint{8.144018in}{1.687720in}}%
\pgfpathlineto{\pgfqpoint{8.174885in}{1.665918in}}%
\pgfpathlineto{\pgfqpoint{8.205753in}{1.639668in}}%
\pgfpathlineto{\pgfqpoint{8.252053in}{1.594484in}}%
\pgfpathlineto{\pgfqpoint{8.360089in}{1.484333in}}%
\pgfpathlineto{\pgfqpoint{8.390956in}{1.457114in}}%
\pgfpathlineto{\pgfqpoint{8.421823in}{1.433643in}}%
\pgfpathlineto{\pgfqpoint{8.452691in}{1.414672in}}%
\pgfpathlineto{\pgfqpoint{8.483558in}{1.400779in}}%
\pgfpathlineto{\pgfqpoint{8.514425in}{1.392370in}}%
\pgfpathlineto{\pgfqpoint{8.545293in}{1.389686in}}%
\pgfpathlineto{\pgfqpoint{8.576160in}{1.392809in}}%
\pgfpathlineto{\pgfqpoint{8.607027in}{1.401669in}}%
\pgfpathlineto{\pgfqpoint{8.637894in}{1.416044in}}%
\pgfpathlineto{\pgfqpoint{8.668762in}{1.435563in}}%
\pgfpathlineto{\pgfqpoint{8.699629in}{1.459701in}}%
\pgfpathlineto{\pgfqpoint{8.730496in}{1.487780in}}%
\pgfpathlineto{\pgfqpoint{8.776797in}{1.535445in}}%
\pgfpathlineto{\pgfqpoint{8.900266in}{1.669935in}}%
\pgfpathlineto{\pgfqpoint{8.931133in}{1.699044in}}%
\pgfpathlineto{\pgfqpoint{8.962001in}{1.724256in}}%
\pgfpathlineto{\pgfqpoint{8.992868in}{1.744822in}}%
\pgfpathlineto{\pgfqpoint{9.023735in}{1.760279in}}%
\pgfpathlineto{\pgfqpoint{9.054603in}{1.770499in}}%
\pgfpathlineto{\pgfqpoint{9.085470in}{1.775729in}}%
\pgfpathlineto{\pgfqpoint{9.116337in}{1.776601in}}%
\pgfpathlineto{\pgfqpoint{9.147204in}{1.774110in}}%
\pgfpathlineto{\pgfqpoint{9.255240in}{1.759707in}}%
\pgfpathlineto{\pgfqpoint{9.286107in}{1.760741in}}%
\pgfpathlineto{\pgfqpoint{9.316974in}{1.766896in}}%
\pgfpathlineto{\pgfqpoint{9.332408in}{1.772262in}}%
\pgfpathlineto{\pgfqpoint{9.347842in}{1.779301in}}%
\pgfpathlineto{\pgfqpoint{9.363275in}{1.788086in}}%
\pgfpathlineto{\pgfqpoint{9.394143in}{1.811028in}}%
\pgfpathlineto{\pgfqpoint{9.425010in}{1.841008in}}%
\pgfpathlineto{\pgfqpoint{9.455877in}{1.877394in}}%
\pgfpathlineto{\pgfqpoint{9.486744in}{1.919013in}}%
\pgfpathlineto{\pgfqpoint{9.533045in}{1.987569in}}%
\pgfpathlineto{\pgfqpoint{9.594780in}{2.079347in}}%
\pgfpathlineto{\pgfqpoint{9.625647in}{2.120316in}}%
\pgfpathlineto{\pgfqpoint{9.656514in}{2.155197in}}%
\pgfpathlineto{\pgfqpoint{9.671948in}{2.169804in}}%
\pgfpathlineto{\pgfqpoint{9.687382in}{2.182286in}}%
\pgfpathlineto{\pgfqpoint{9.702815in}{2.192506in}}%
\pgfpathlineto{\pgfqpoint{9.718249in}{2.200365in}}%
\pgfpathlineto{\pgfqpoint{9.733682in}{2.205804in}}%
\pgfpathlineto{\pgfqpoint{9.733682in}{2.205804in}}%
\pgfusepath{stroke}%
\end{pgfscope}%
\begin{pgfscope}%
\pgfpathrectangle{\pgfqpoint{5.698559in}{0.505056in}}{\pgfqpoint{4.227273in}{2.745455in}} %
\pgfusepath{clip}%
\pgfsetrectcap%
\pgfsetroundjoin%
\pgfsetlinewidth{0.501875pt}%
\definecolor{currentstroke}{rgb}{0.300000,0.951057,0.809017}%
\pgfsetstrokecolor{currentstroke}%
\pgfsetdash{}{0pt}%
\pgfpathmoveto{\pgfqpoint{5.890707in}{2.155326in}}%
\pgfpathlineto{\pgfqpoint{5.906141in}{2.158426in}}%
\pgfpathlineto{\pgfqpoint{5.921575in}{2.158017in}}%
\pgfpathlineto{\pgfqpoint{5.937008in}{2.154012in}}%
\pgfpathlineto{\pgfqpoint{5.952442in}{2.146381in}}%
\pgfpathlineto{\pgfqpoint{5.967875in}{2.135149in}}%
\pgfpathlineto{\pgfqpoint{5.983309in}{2.120399in}}%
\pgfpathlineto{\pgfqpoint{5.998743in}{2.102271in}}%
\pgfpathlineto{\pgfqpoint{6.014176in}{2.080963in}}%
\pgfpathlineto{\pgfqpoint{6.045044in}{2.029880in}}%
\pgfpathlineto{\pgfqpoint{6.075911in}{1.969795in}}%
\pgfpathlineto{\pgfqpoint{6.122212in}{1.870310in}}%
\pgfpathlineto{\pgfqpoint{6.168513in}{1.771688in}}%
\pgfpathlineto{\pgfqpoint{6.199380in}{1.713547in}}%
\pgfpathlineto{\pgfqpoint{6.214814in}{1.688326in}}%
\pgfpathlineto{\pgfqpoint{6.230247in}{1.666316in}}%
\pgfpathlineto{\pgfqpoint{6.245681in}{1.647958in}}%
\pgfpathlineto{\pgfqpoint{6.261115in}{1.633648in}}%
\pgfpathlineto{\pgfqpoint{6.276548in}{1.623734in}}%
\pgfpathlineto{\pgfqpoint{6.291982in}{1.618505in}}%
\pgfpathlineto{\pgfqpoint{6.307415in}{1.618191in}}%
\pgfpathlineto{\pgfqpoint{6.322849in}{1.622954in}}%
\pgfpathlineto{\pgfqpoint{6.338283in}{1.632888in}}%
\pgfpathlineto{\pgfqpoint{6.353716in}{1.648014in}}%
\pgfpathlineto{\pgfqpoint{6.369150in}{1.668283in}}%
\pgfpathlineto{\pgfqpoint{6.384584in}{1.693571in}}%
\pgfpathlineto{\pgfqpoint{6.400017in}{1.723683in}}%
\pgfpathlineto{\pgfqpoint{6.415451in}{1.758355in}}%
\pgfpathlineto{\pgfqpoint{6.446318in}{1.839990in}}%
\pgfpathlineto{\pgfqpoint{6.477185in}{1.935121in}}%
\pgfpathlineto{\pgfqpoint{6.523486in}{2.093858in}}%
\pgfpathlineto{\pgfqpoint{6.585221in}{2.309897in}}%
\pgfpathlineto{\pgfqpoint{6.616088in}{2.408537in}}%
\pgfpathlineto{\pgfqpoint{6.646955in}{2.494746in}}%
\pgfpathlineto{\pgfqpoint{6.662389in}{2.531947in}}%
\pgfpathlineto{\pgfqpoint{6.677823in}{2.564653in}}%
\pgfpathlineto{\pgfqpoint{6.693256in}{2.592518in}}%
\pgfpathlineto{\pgfqpoint{6.708690in}{2.615261in}}%
\pgfpathlineto{\pgfqpoint{6.724124in}{2.632667in}}%
\pgfpathlineto{\pgfqpoint{6.739557in}{2.644593in}}%
\pgfpathlineto{\pgfqpoint{6.754991in}{2.650966in}}%
\pgfpathlineto{\pgfqpoint{6.770424in}{2.651784in}}%
\pgfpathlineto{\pgfqpoint{6.785858in}{2.647116in}}%
\pgfpathlineto{\pgfqpoint{6.801292in}{2.637096in}}%
\pgfpathlineto{\pgfqpoint{6.816725in}{2.621925in}}%
\pgfpathlineto{\pgfqpoint{6.832159in}{2.601865in}}%
\pgfpathlineto{\pgfqpoint{6.847593in}{2.577229in}}%
\pgfpathlineto{\pgfqpoint{6.863026in}{2.548384in}}%
\pgfpathlineto{\pgfqpoint{6.893894in}{2.479733in}}%
\pgfpathlineto{\pgfqpoint{6.924761in}{2.399559in}}%
\pgfpathlineto{\pgfqpoint{6.971062in}{2.266419in}}%
\pgfpathlineto{\pgfqpoint{7.032796in}{2.085637in}}%
\pgfpathlineto{\pgfqpoint{7.063664in}{2.001758in}}%
\pgfpathlineto{\pgfqpoint{7.094531in}{1.925870in}}%
\pgfpathlineto{\pgfqpoint{7.125398in}{1.859789in}}%
\pgfpathlineto{\pgfqpoint{7.156265in}{1.804525in}}%
\pgfpathlineto{\pgfqpoint{7.187133in}{1.760291in}}%
\pgfpathlineto{\pgfqpoint{7.202566in}{1.742173in}}%
\pgfpathlineto{\pgfqpoint{7.218000in}{1.726565in}}%
\pgfpathlineto{\pgfqpoint{7.233434in}{1.713307in}}%
\pgfpathlineto{\pgfqpoint{7.248867in}{1.702208in}}%
\pgfpathlineto{\pgfqpoint{7.279734in}{1.685611in}}%
\pgfpathlineto{\pgfqpoint{7.310602in}{1.674870in}}%
\pgfpathlineto{\pgfqpoint{7.341469in}{1.667977in}}%
\pgfpathlineto{\pgfqpoint{7.449504in}{1.648782in}}%
\pgfpathlineto{\pgfqpoint{7.480372in}{1.639881in}}%
\pgfpathlineto{\pgfqpoint{7.526673in}{1.622504in}}%
\pgfpathlineto{\pgfqpoint{7.634708in}{1.577461in}}%
\pgfpathlineto{\pgfqpoint{7.665575in}{1.569462in}}%
\pgfpathlineto{\pgfqpoint{7.696443in}{1.566336in}}%
\pgfpathlineto{\pgfqpoint{7.711876in}{1.566995in}}%
\pgfpathlineto{\pgfqpoint{7.727310in}{1.569304in}}%
\pgfpathlineto{\pgfqpoint{7.742744in}{1.573358in}}%
\pgfpathlineto{\pgfqpoint{7.758177in}{1.579222in}}%
\pgfpathlineto{\pgfqpoint{7.773611in}{1.586930in}}%
\pgfpathlineto{\pgfqpoint{7.804478in}{1.607844in}}%
\pgfpathlineto{\pgfqpoint{7.835345in}{1.635687in}}%
\pgfpathlineto{\pgfqpoint{7.866213in}{1.669496in}}%
\pgfpathlineto{\pgfqpoint{7.897080in}{1.707798in}}%
\pgfpathlineto{\pgfqpoint{8.005115in}{1.847201in}}%
\pgfpathlineto{\pgfqpoint{8.035983in}{1.878789in}}%
\pgfpathlineto{\pgfqpoint{8.051416in}{1.891763in}}%
\pgfpathlineto{\pgfqpoint{8.066850in}{1.902525in}}%
\pgfpathlineto{\pgfqpoint{8.082284in}{1.910858in}}%
\pgfpathlineto{\pgfqpoint{8.097717in}{1.916576in}}%
\pgfpathlineto{\pgfqpoint{8.113151in}{1.919526in}}%
\pgfpathlineto{\pgfqpoint{8.128584in}{1.919591in}}%
\pgfpathlineto{\pgfqpoint{8.144018in}{1.916693in}}%
\pgfpathlineto{\pgfqpoint{8.159452in}{1.910793in}}%
\pgfpathlineto{\pgfqpoint{8.174885in}{1.901894in}}%
\pgfpathlineto{\pgfqpoint{8.190319in}{1.890041in}}%
\pgfpathlineto{\pgfqpoint{8.205753in}{1.875318in}}%
\pgfpathlineto{\pgfqpoint{8.221186in}{1.857852in}}%
\pgfpathlineto{\pgfqpoint{8.252053in}{1.815383in}}%
\pgfpathlineto{\pgfqpoint{8.282921in}{1.764372in}}%
\pgfpathlineto{\pgfqpoint{8.313788in}{1.707046in}}%
\pgfpathlineto{\pgfqpoint{8.421823in}{1.495753in}}%
\pgfpathlineto{\pgfqpoint{8.452691in}{1.443966in}}%
\pgfpathlineto{\pgfqpoint{8.483558in}{1.400680in}}%
\pgfpathlineto{\pgfqpoint{8.498992in}{1.382853in}}%
\pgfpathlineto{\pgfqpoint{8.514425in}{1.367843in}}%
\pgfpathlineto{\pgfqpoint{8.529859in}{1.355809in}}%
\pgfpathlineto{\pgfqpoint{8.545293in}{1.346870in}}%
\pgfpathlineto{\pgfqpoint{8.560726in}{1.341110in}}%
\pgfpathlineto{\pgfqpoint{8.576160in}{1.338567in}}%
\pgfpathlineto{\pgfqpoint{8.591593in}{1.339243in}}%
\pgfpathlineto{\pgfqpoint{8.607027in}{1.343097in}}%
\pgfpathlineto{\pgfqpoint{8.622461in}{1.350049in}}%
\pgfpathlineto{\pgfqpoint{8.637894in}{1.359982in}}%
\pgfpathlineto{\pgfqpoint{8.653328in}{1.372742in}}%
\pgfpathlineto{\pgfqpoint{8.668762in}{1.388141in}}%
\pgfpathlineto{\pgfqpoint{8.699629in}{1.425956in}}%
\pgfpathlineto{\pgfqpoint{8.730496in}{1.471368in}}%
\pgfpathlineto{\pgfqpoint{8.776797in}{1.548471in}}%
\pgfpathlineto{\pgfqpoint{8.853965in}{1.679436in}}%
\pgfpathlineto{\pgfqpoint{8.884833in}{1.725508in}}%
\pgfpathlineto{\pgfqpoint{8.915700in}{1.764873in}}%
\pgfpathlineto{\pgfqpoint{8.946567in}{1.796040in}}%
\pgfpathlineto{\pgfqpoint{8.962001in}{1.808224in}}%
\pgfpathlineto{\pgfqpoint{8.977434in}{1.818037in}}%
\pgfpathlineto{\pgfqpoint{8.992868in}{1.825445in}}%
\pgfpathlineto{\pgfqpoint{9.008302in}{1.830452in}}%
\pgfpathlineto{\pgfqpoint{9.023735in}{1.833096in}}%
\pgfpathlineto{\pgfqpoint{9.039169in}{1.833451in}}%
\pgfpathlineto{\pgfqpoint{9.054603in}{1.831623in}}%
\pgfpathlineto{\pgfqpoint{9.070036in}{1.827752in}}%
\pgfpathlineto{\pgfqpoint{9.085470in}{1.822005in}}%
\pgfpathlineto{\pgfqpoint{9.116337in}{1.805689in}}%
\pgfpathlineto{\pgfqpoint{9.147204in}{1.784502in}}%
\pgfpathlineto{\pgfqpoint{9.255240in}{1.703215in}}%
\pgfpathlineto{\pgfqpoint{9.286107in}{1.686975in}}%
\pgfpathlineto{\pgfqpoint{9.301541in}{1.681233in}}%
\pgfpathlineto{\pgfqpoint{9.316974in}{1.677325in}}%
\pgfpathlineto{\pgfqpoint{9.332408in}{1.675408in}}%
\pgfpathlineto{\pgfqpoint{9.347842in}{1.675605in}}%
\pgfpathlineto{\pgfqpoint{9.363275in}{1.678008in}}%
\pgfpathlineto{\pgfqpoint{9.378709in}{1.682678in}}%
\pgfpathlineto{\pgfqpoint{9.394143in}{1.689640in}}%
\pgfpathlineto{\pgfqpoint{9.409576in}{1.698885in}}%
\pgfpathlineto{\pgfqpoint{9.425010in}{1.710370in}}%
\pgfpathlineto{\pgfqpoint{9.455877in}{1.739722in}}%
\pgfpathlineto{\pgfqpoint{9.486744in}{1.776705in}}%
\pgfpathlineto{\pgfqpoint{9.517612in}{1.819867in}}%
\pgfpathlineto{\pgfqpoint{9.563913in}{1.892121in}}%
\pgfpathlineto{\pgfqpoint{9.641081in}{2.014424in}}%
\pgfpathlineto{\pgfqpoint{9.671948in}{2.057497in}}%
\pgfpathlineto{\pgfqpoint{9.702815in}{2.094201in}}%
\pgfpathlineto{\pgfqpoint{9.733682in}{2.122936in}}%
\pgfpathlineto{\pgfqpoint{9.733682in}{2.122936in}}%
\pgfusepath{stroke}%
\end{pgfscope}%
\begin{pgfscope}%
\pgfpathrectangle{\pgfqpoint{5.698559in}{0.505056in}}{\pgfqpoint{4.227273in}{2.745455in}} %
\pgfusepath{clip}%
\pgfsetrectcap%
\pgfsetroundjoin%
\pgfsetlinewidth{0.501875pt}%
\definecolor{currentstroke}{rgb}{0.378431,0.981823,0.771298}%
\pgfsetstrokecolor{currentstroke}%
\pgfsetdash{}{0pt}%
\pgfpathmoveto{\pgfqpoint{5.890707in}{2.058276in}}%
\pgfpathlineto{\pgfqpoint{5.906141in}{2.056282in}}%
\pgfpathlineto{\pgfqpoint{5.921575in}{2.052235in}}%
\pgfpathlineto{\pgfqpoint{5.937008in}{2.045996in}}%
\pgfpathlineto{\pgfqpoint{5.952442in}{2.037458in}}%
\pgfpathlineto{\pgfqpoint{5.967875in}{2.026555in}}%
\pgfpathlineto{\pgfqpoint{5.983309in}{2.013262in}}%
\pgfpathlineto{\pgfqpoint{5.998743in}{1.997599in}}%
\pgfpathlineto{\pgfqpoint{6.029610in}{1.959500in}}%
\pgfpathlineto{\pgfqpoint{6.060477in}{1.913429in}}%
\pgfpathlineto{\pgfqpoint{6.091345in}{1.861362in}}%
\pgfpathlineto{\pgfqpoint{6.183946in}{1.699063in}}%
\pgfpathlineto{\pgfqpoint{6.214814in}{1.655032in}}%
\pgfpathlineto{\pgfqpoint{6.230247in}{1.637037in}}%
\pgfpathlineto{\pgfqpoint{6.245681in}{1.622312in}}%
\pgfpathlineto{\pgfqpoint{6.261115in}{1.611258in}}%
\pgfpathlineto{\pgfqpoint{6.276548in}{1.604232in}}%
\pgfpathlineto{\pgfqpoint{6.291982in}{1.601542in}}%
\pgfpathlineto{\pgfqpoint{6.307415in}{1.603441in}}%
\pgfpathlineto{\pgfqpoint{6.322849in}{1.610115in}}%
\pgfpathlineto{\pgfqpoint{6.338283in}{1.621684in}}%
\pgfpathlineto{\pgfqpoint{6.353716in}{1.638197in}}%
\pgfpathlineto{\pgfqpoint{6.369150in}{1.659626in}}%
\pgfpathlineto{\pgfqpoint{6.384584in}{1.685869in}}%
\pgfpathlineto{\pgfqpoint{6.400017in}{1.716748in}}%
\pgfpathlineto{\pgfqpoint{6.415451in}{1.752012in}}%
\pgfpathlineto{\pgfqpoint{6.446318in}{1.834329in}}%
\pgfpathlineto{\pgfqpoint{6.477185in}{1.929463in}}%
\pgfpathlineto{\pgfqpoint{6.523486in}{2.086769in}}%
\pgfpathlineto{\pgfqpoint{6.585221in}{2.297589in}}%
\pgfpathlineto{\pgfqpoint{6.616088in}{2.391994in}}%
\pgfpathlineto{\pgfqpoint{6.646955in}{2.472896in}}%
\pgfpathlineto{\pgfqpoint{6.662389in}{2.507084in}}%
\pgfpathlineto{\pgfqpoint{6.677823in}{2.536577in}}%
\pgfpathlineto{\pgfqpoint{6.693256in}{2.561069in}}%
\pgfpathlineto{\pgfqpoint{6.708690in}{2.580328in}}%
\pgfpathlineto{\pgfqpoint{6.724124in}{2.594197in}}%
\pgfpathlineto{\pgfqpoint{6.739557in}{2.602595in}}%
\pgfpathlineto{\pgfqpoint{6.754991in}{2.605519in}}%
\pgfpathlineto{\pgfqpoint{6.770424in}{2.603041in}}%
\pgfpathlineto{\pgfqpoint{6.785858in}{2.595305in}}%
\pgfpathlineto{\pgfqpoint{6.801292in}{2.582524in}}%
\pgfpathlineto{\pgfqpoint{6.816725in}{2.564973in}}%
\pgfpathlineto{\pgfqpoint{6.832159in}{2.542985in}}%
\pgfpathlineto{\pgfqpoint{6.847593in}{2.516943in}}%
\pgfpathlineto{\pgfqpoint{6.878460in}{2.454423in}}%
\pgfpathlineto{\pgfqpoint{6.909327in}{2.381144in}}%
\pgfpathlineto{\pgfqpoint{6.955628in}{2.259734in}}%
\pgfpathlineto{\pgfqpoint{7.017363in}{2.096090in}}%
\pgfpathlineto{\pgfqpoint{7.048230in}{2.020489in}}%
\pgfpathlineto{\pgfqpoint{7.079097in}{1.951884in}}%
\pgfpathlineto{\pgfqpoint{7.109964in}{1.891393in}}%
\pgfpathlineto{\pgfqpoint{7.140832in}{1.839316in}}%
\pgfpathlineto{\pgfqpoint{7.171699in}{1.795226in}}%
\pgfpathlineto{\pgfqpoint{7.202566in}{1.758112in}}%
\pgfpathlineto{\pgfqpoint{7.233434in}{1.726565in}}%
\pgfpathlineto{\pgfqpoint{7.264301in}{1.698990in}}%
\pgfpathlineto{\pgfqpoint{7.326035in}{1.649712in}}%
\pgfpathlineto{\pgfqpoint{7.495805in}{1.519663in}}%
\pgfpathlineto{\pgfqpoint{7.526673in}{1.502144in}}%
\pgfpathlineto{\pgfqpoint{7.557540in}{1.490151in}}%
\pgfpathlineto{\pgfqpoint{7.572974in}{1.486738in}}%
\pgfpathlineto{\pgfqpoint{7.588407in}{1.485293in}}%
\pgfpathlineto{\pgfqpoint{7.603841in}{1.485973in}}%
\pgfpathlineto{\pgfqpoint{7.619274in}{1.488905in}}%
\pgfpathlineto{\pgfqpoint{7.634708in}{1.494189in}}%
\pgfpathlineto{\pgfqpoint{7.650142in}{1.501884in}}%
\pgfpathlineto{\pgfqpoint{7.665575in}{1.512013in}}%
\pgfpathlineto{\pgfqpoint{7.681009in}{1.524560in}}%
\pgfpathlineto{\pgfqpoint{7.696443in}{1.539464in}}%
\pgfpathlineto{\pgfqpoint{7.727310in}{1.575899in}}%
\pgfpathlineto{\pgfqpoint{7.758177in}{1.620032in}}%
\pgfpathlineto{\pgfqpoint{7.789044in}{1.669994in}}%
\pgfpathlineto{\pgfqpoint{7.897080in}{1.855028in}}%
\pgfpathlineto{\pgfqpoint{7.927947in}{1.899877in}}%
\pgfpathlineto{\pgfqpoint{7.958814in}{1.936844in}}%
\pgfpathlineto{\pgfqpoint{7.974248in}{1.951833in}}%
\pgfpathlineto{\pgfqpoint{7.989682in}{1.964281in}}%
\pgfpathlineto{\pgfqpoint{8.005115in}{1.974077in}}%
\pgfpathlineto{\pgfqpoint{8.020549in}{1.981157in}}%
\pgfpathlineto{\pgfqpoint{8.035983in}{1.985495in}}%
\pgfpathlineto{\pgfqpoint{8.051416in}{1.987109in}}%
\pgfpathlineto{\pgfqpoint{8.066850in}{1.986059in}}%
\pgfpathlineto{\pgfqpoint{8.082284in}{1.982440in}}%
\pgfpathlineto{\pgfqpoint{8.097717in}{1.976386in}}%
\pgfpathlineto{\pgfqpoint{8.113151in}{1.968060in}}%
\pgfpathlineto{\pgfqpoint{8.128584in}{1.957653in}}%
\pgfpathlineto{\pgfqpoint{8.159452in}{1.931472in}}%
\pgfpathlineto{\pgfqpoint{8.190319in}{1.899736in}}%
\pgfpathlineto{\pgfqpoint{8.236620in}{1.846096in}}%
\pgfpathlineto{\pgfqpoint{8.313788in}{1.755696in}}%
\pgfpathlineto{\pgfqpoint{8.344655in}{1.723134in}}%
\pgfpathlineto{\pgfqpoint{8.375523in}{1.693736in}}%
\pgfpathlineto{\pgfqpoint{8.406390in}{1.667646in}}%
\pgfpathlineto{\pgfqpoint{8.437257in}{1.644639in}}%
\pgfpathlineto{\pgfqpoint{8.483558in}{1.614758in}}%
\pgfpathlineto{\pgfqpoint{8.545293in}{1.580023in}}%
\pgfpathlineto{\pgfqpoint{8.653328in}{1.523238in}}%
\pgfpathlineto{\pgfqpoint{8.699629in}{1.501654in}}%
\pgfpathlineto{\pgfqpoint{8.730496in}{1.490104in}}%
\pgfpathlineto{\pgfqpoint{8.761363in}{1.482181in}}%
\pgfpathlineto{\pgfqpoint{8.792231in}{1.479106in}}%
\pgfpathlineto{\pgfqpoint{8.807664in}{1.479744in}}%
\pgfpathlineto{\pgfqpoint{8.823098in}{1.481997in}}%
\pgfpathlineto{\pgfqpoint{8.838532in}{1.485968in}}%
\pgfpathlineto{\pgfqpoint{8.853965in}{1.491733in}}%
\pgfpathlineto{\pgfqpoint{8.869399in}{1.499343in}}%
\pgfpathlineto{\pgfqpoint{8.884833in}{1.508821in}}%
\pgfpathlineto{\pgfqpoint{8.915700in}{1.533309in}}%
\pgfpathlineto{\pgfqpoint{8.946567in}{1.564723in}}%
\pgfpathlineto{\pgfqpoint{8.977434in}{1.602058in}}%
\pgfpathlineto{\pgfqpoint{9.023735in}{1.665783in}}%
\pgfpathlineto{\pgfqpoint{9.100903in}{1.775514in}}%
\pgfpathlineto{\pgfqpoint{9.131771in}{1.814132in}}%
\pgfpathlineto{\pgfqpoint{9.162638in}{1.846666in}}%
\pgfpathlineto{\pgfqpoint{9.178072in}{1.860149in}}%
\pgfpathlineto{\pgfqpoint{9.193505in}{1.871573in}}%
\pgfpathlineto{\pgfqpoint{9.208939in}{1.880832in}}%
\pgfpathlineto{\pgfqpoint{9.224373in}{1.887863in}}%
\pgfpathlineto{\pgfqpoint{9.239806in}{1.892642in}}%
\pgfpathlineto{\pgfqpoint{9.255240in}{1.895192in}}%
\pgfpathlineto{\pgfqpoint{9.270673in}{1.895577in}}%
\pgfpathlineto{\pgfqpoint{9.286107in}{1.893904in}}%
\pgfpathlineto{\pgfqpoint{9.301541in}{1.890321in}}%
\pgfpathlineto{\pgfqpoint{9.332408in}{1.878202in}}%
\pgfpathlineto{\pgfqpoint{9.363275in}{1.861088in}}%
\pgfpathlineto{\pgfqpoint{9.455877in}{1.803242in}}%
\pgfpathlineto{\pgfqpoint{9.486744in}{1.789845in}}%
\pgfpathlineto{\pgfqpoint{9.502178in}{1.785441in}}%
\pgfpathlineto{\pgfqpoint{9.517612in}{1.782825in}}%
\pgfpathlineto{\pgfqpoint{9.533045in}{1.782147in}}%
\pgfpathlineto{\pgfqpoint{9.548479in}{1.783510in}}%
\pgfpathlineto{\pgfqpoint{9.563913in}{1.786975in}}%
\pgfpathlineto{\pgfqpoint{9.579346in}{1.792554in}}%
\pgfpathlineto{\pgfqpoint{9.594780in}{1.800215in}}%
\pgfpathlineto{\pgfqpoint{9.610213in}{1.809879in}}%
\pgfpathlineto{\pgfqpoint{9.641081in}{1.834672in}}%
\pgfpathlineto{\pgfqpoint{9.671948in}{1.865447in}}%
\pgfpathlineto{\pgfqpoint{9.718249in}{1.918250in}}%
\pgfpathlineto{\pgfqpoint{9.733682in}{1.936390in}}%
\pgfpathlineto{\pgfqpoint{9.733682in}{1.936390in}}%
\pgfusepath{stroke}%
\end{pgfscope}%
\begin{pgfscope}%
\pgfpathrectangle{\pgfqpoint{5.698559in}{0.505056in}}{\pgfqpoint{4.227273in}{2.745455in}} %
\pgfusepath{clip}%
\pgfsetrectcap%
\pgfsetroundjoin%
\pgfsetlinewidth{0.501875pt}%
\definecolor{currentstroke}{rgb}{0.456863,0.997705,0.730653}%
\pgfsetstrokecolor{currentstroke}%
\pgfsetdash{}{0pt}%
\pgfpathmoveto{\pgfqpoint{5.890707in}{2.055107in}}%
\pgfpathlineto{\pgfqpoint{5.906141in}{2.051494in}}%
\pgfpathlineto{\pgfqpoint{5.921575in}{2.045203in}}%
\pgfpathlineto{\pgfqpoint{5.937008in}{2.036105in}}%
\pgfpathlineto{\pgfqpoint{5.952442in}{2.024117in}}%
\pgfpathlineto{\pgfqpoint{5.967875in}{2.009215in}}%
\pgfpathlineto{\pgfqpoint{5.983309in}{1.991434in}}%
\pgfpathlineto{\pgfqpoint{5.998743in}{1.970873in}}%
\pgfpathlineto{\pgfqpoint{6.029610in}{1.922119in}}%
\pgfpathlineto{\pgfqpoint{6.060477in}{1.865010in}}%
\pgfpathlineto{\pgfqpoint{6.122212in}{1.738453in}}%
\pgfpathlineto{\pgfqpoint{6.153079in}{1.677025in}}%
\pgfpathlineto{\pgfqpoint{6.183946in}{1.622705in}}%
\pgfpathlineto{\pgfqpoint{6.199380in}{1.599570in}}%
\pgfpathlineto{\pgfqpoint{6.214814in}{1.579791in}}%
\pgfpathlineto{\pgfqpoint{6.230247in}{1.563825in}}%
\pgfpathlineto{\pgfqpoint{6.245681in}{1.552074in}}%
\pgfpathlineto{\pgfqpoint{6.261115in}{1.544884in}}%
\pgfpathlineto{\pgfqpoint{6.276548in}{1.542531in}}%
\pgfpathlineto{\pgfqpoint{6.291982in}{1.545218in}}%
\pgfpathlineto{\pgfqpoint{6.307415in}{1.553071in}}%
\pgfpathlineto{\pgfqpoint{6.322849in}{1.566136in}}%
\pgfpathlineto{\pgfqpoint{6.338283in}{1.584377in}}%
\pgfpathlineto{\pgfqpoint{6.353716in}{1.607678in}}%
\pgfpathlineto{\pgfqpoint{6.369150in}{1.635844in}}%
\pgfpathlineto{\pgfqpoint{6.384584in}{1.668604in}}%
\pgfpathlineto{\pgfqpoint{6.415451in}{1.746485in}}%
\pgfpathlineto{\pgfqpoint{6.446318in}{1.837890in}}%
\pgfpathlineto{\pgfqpoint{6.492619in}{1.991077in}}%
\pgfpathlineto{\pgfqpoint{6.554354in}{2.200692in}}%
\pgfpathlineto{\pgfqpoint{6.585221in}{2.297445in}}%
\pgfpathlineto{\pgfqpoint{6.616088in}{2.383455in}}%
\pgfpathlineto{\pgfqpoint{6.646955in}{2.455539in}}%
\pgfpathlineto{\pgfqpoint{6.662389in}{2.485588in}}%
\pgfpathlineto{\pgfqpoint{6.677823in}{2.511338in}}%
\pgfpathlineto{\pgfqpoint{6.693256in}{2.532630in}}%
\pgfpathlineto{\pgfqpoint{6.708690in}{2.549357in}}%
\pgfpathlineto{\pgfqpoint{6.724124in}{2.561465in}}%
\pgfpathlineto{\pgfqpoint{6.739557in}{2.568948in}}%
\pgfpathlineto{\pgfqpoint{6.754991in}{2.571847in}}%
\pgfpathlineto{\pgfqpoint{6.770424in}{2.570246in}}%
\pgfpathlineto{\pgfqpoint{6.785858in}{2.564266in}}%
\pgfpathlineto{\pgfqpoint{6.801292in}{2.554065in}}%
\pgfpathlineto{\pgfqpoint{6.816725in}{2.539835in}}%
\pgfpathlineto{\pgfqpoint{6.832159in}{2.521792in}}%
\pgfpathlineto{\pgfqpoint{6.847593in}{2.500178in}}%
\pgfpathlineto{\pgfqpoint{6.863026in}{2.475256in}}%
\pgfpathlineto{\pgfqpoint{6.893894in}{2.416624in}}%
\pgfpathlineto{\pgfqpoint{6.924761in}{2.348289in}}%
\pgfpathlineto{\pgfqpoint{6.955628in}{2.272788in}}%
\pgfpathlineto{\pgfqpoint{7.017363in}{2.110648in}}%
\pgfpathlineto{\pgfqpoint{7.079097in}{1.950329in}}%
\pgfpathlineto{\pgfqpoint{7.109964in}{1.876477in}}%
\pgfpathlineto{\pgfqpoint{7.140832in}{1.809269in}}%
\pgfpathlineto{\pgfqpoint{7.171699in}{1.750042in}}%
\pgfpathlineto{\pgfqpoint{7.202566in}{1.699657in}}%
\pgfpathlineto{\pgfqpoint{7.233434in}{1.658453in}}%
\pgfpathlineto{\pgfqpoint{7.264301in}{1.626239in}}%
\pgfpathlineto{\pgfqpoint{7.295168in}{1.602319in}}%
\pgfpathlineto{\pgfqpoint{7.326035in}{1.585569in}}%
\pgfpathlineto{\pgfqpoint{7.356903in}{1.574553in}}%
\pgfpathlineto{\pgfqpoint{7.387770in}{1.567675in}}%
\pgfpathlineto{\pgfqpoint{7.434071in}{1.561711in}}%
\pgfpathlineto{\pgfqpoint{7.557540in}{1.548270in}}%
\pgfpathlineto{\pgfqpoint{7.603841in}{1.544449in}}%
\pgfpathlineto{\pgfqpoint{7.634708in}{1.545090in}}%
\pgfpathlineto{\pgfqpoint{7.665575in}{1.550183in}}%
\pgfpathlineto{\pgfqpoint{7.681009in}{1.554907in}}%
\pgfpathlineto{\pgfqpoint{7.696443in}{1.561329in}}%
\pgfpathlineto{\pgfqpoint{7.711876in}{1.569602in}}%
\pgfpathlineto{\pgfqpoint{7.727310in}{1.579851in}}%
\pgfpathlineto{\pgfqpoint{7.742744in}{1.592160in}}%
\pgfpathlineto{\pgfqpoint{7.773611in}{1.623086in}}%
\pgfpathlineto{\pgfqpoint{7.804478in}{1.662159in}}%
\pgfpathlineto{\pgfqpoint{7.835345in}{1.708362in}}%
\pgfpathlineto{\pgfqpoint{7.881646in}{1.786952in}}%
\pgfpathlineto{\pgfqpoint{7.943381in}{1.895143in}}%
\pgfpathlineto{\pgfqpoint{7.974248in}{1.943863in}}%
\pgfpathlineto{\pgfqpoint{8.005115in}{1.985091in}}%
\pgfpathlineto{\pgfqpoint{8.020549in}{2.002124in}}%
\pgfpathlineto{\pgfqpoint{8.035983in}{2.016439in}}%
\pgfpathlineto{\pgfqpoint{8.051416in}{2.027849in}}%
\pgfpathlineto{\pgfqpoint{8.066850in}{2.036223in}}%
\pgfpathlineto{\pgfqpoint{8.082284in}{2.041485in}}%
\pgfpathlineto{\pgfqpoint{8.097717in}{2.043616in}}%
\pgfpathlineto{\pgfqpoint{8.113151in}{2.042656in}}%
\pgfpathlineto{\pgfqpoint{8.128584in}{2.038698in}}%
\pgfpathlineto{\pgfqpoint{8.144018in}{2.031888in}}%
\pgfpathlineto{\pgfqpoint{8.159452in}{2.022417in}}%
\pgfpathlineto{\pgfqpoint{8.174885in}{2.010518in}}%
\pgfpathlineto{\pgfqpoint{8.205753in}{1.980524in}}%
\pgfpathlineto{\pgfqpoint{8.236620in}{1.944296in}}%
\pgfpathlineto{\pgfqpoint{8.282921in}{1.883783in}}%
\pgfpathlineto{\pgfqpoint{8.344655in}{1.803305in}}%
\pgfpathlineto{\pgfqpoint{8.375523in}{1.766737in}}%
\pgfpathlineto{\pgfqpoint{8.406390in}{1.733632in}}%
\pgfpathlineto{\pgfqpoint{8.437257in}{1.703999in}}%
\pgfpathlineto{\pgfqpoint{8.483558in}{1.664937in}}%
\pgfpathlineto{\pgfqpoint{8.545293in}{1.618650in}}%
\pgfpathlineto{\pgfqpoint{8.653328in}{1.541894in}}%
\pgfpathlineto{\pgfqpoint{8.684195in}{1.522340in}}%
\pgfpathlineto{\pgfqpoint{8.715063in}{1.505938in}}%
\pgfpathlineto{\pgfqpoint{8.745930in}{1.494308in}}%
\pgfpathlineto{\pgfqpoint{8.761363in}{1.490797in}}%
\pgfpathlineto{\pgfqpoint{8.776797in}{1.489077in}}%
\pgfpathlineto{\pgfqpoint{8.792231in}{1.489314in}}%
\pgfpathlineto{\pgfqpoint{8.807664in}{1.491645in}}%
\pgfpathlineto{\pgfqpoint{8.823098in}{1.496174in}}%
\pgfpathlineto{\pgfqpoint{8.838532in}{1.502964in}}%
\pgfpathlineto{\pgfqpoint{8.853965in}{1.512037in}}%
\pgfpathlineto{\pgfqpoint{8.869399in}{1.523366in}}%
\pgfpathlineto{\pgfqpoint{8.884833in}{1.536878in}}%
\pgfpathlineto{\pgfqpoint{8.915700in}{1.569921in}}%
\pgfpathlineto{\pgfqpoint{8.946567in}{1.609651in}}%
\pgfpathlineto{\pgfqpoint{8.992868in}{1.676968in}}%
\pgfpathlineto{\pgfqpoint{9.054603in}{1.767033in}}%
\pgfpathlineto{\pgfqpoint{9.085470in}{1.806081in}}%
\pgfpathlineto{\pgfqpoint{9.116337in}{1.838092in}}%
\pgfpathlineto{\pgfqpoint{9.131771in}{1.850954in}}%
\pgfpathlineto{\pgfqpoint{9.147204in}{1.861546in}}%
\pgfpathlineto{\pgfqpoint{9.162638in}{1.869801in}}%
\pgfpathlineto{\pgfqpoint{9.178072in}{1.875707in}}%
\pgfpathlineto{\pgfqpoint{9.193505in}{1.879306in}}%
\pgfpathlineto{\pgfqpoint{9.208939in}{1.880688in}}%
\pgfpathlineto{\pgfqpoint{9.224373in}{1.879996in}}%
\pgfpathlineto{\pgfqpoint{9.239806in}{1.877413in}}%
\pgfpathlineto{\pgfqpoint{9.270673in}{1.867505in}}%
\pgfpathlineto{\pgfqpoint{9.301541in}{1.853105in}}%
\pgfpathlineto{\pgfqpoint{9.378709in}{1.813436in}}%
\pgfpathlineto{\pgfqpoint{9.409576in}{1.802139in}}%
\pgfpathlineto{\pgfqpoint{9.440443in}{1.796125in}}%
\pgfpathlineto{\pgfqpoint{9.455877in}{1.795436in}}%
\pgfpathlineto{\pgfqpoint{9.471311in}{1.796387in}}%
\pgfpathlineto{\pgfqpoint{9.486744in}{1.798999in}}%
\pgfpathlineto{\pgfqpoint{9.517612in}{1.809073in}}%
\pgfpathlineto{\pgfqpoint{9.548479in}{1.825027in}}%
\pgfpathlineto{\pgfqpoint{9.579346in}{1.845710in}}%
\pgfpathlineto{\pgfqpoint{9.625647in}{1.882203in}}%
\pgfpathlineto{\pgfqpoint{9.687382in}{1.931894in}}%
\pgfpathlineto{\pgfqpoint{9.718249in}{1.953458in}}%
\pgfpathlineto{\pgfqpoint{9.733682in}{1.962819in}}%
\pgfpathlineto{\pgfqpoint{9.733682in}{1.962819in}}%
\pgfusepath{stroke}%
\end{pgfscope}%
\begin{pgfscope}%
\pgfpathrectangle{\pgfqpoint{5.698559in}{0.505056in}}{\pgfqpoint{4.227273in}{2.745455in}} %
\pgfusepath{clip}%
\pgfsetrectcap%
\pgfsetroundjoin%
\pgfsetlinewidth{0.501875pt}%
\definecolor{currentstroke}{rgb}{0.543137,0.997705,0.682749}%
\pgfsetstrokecolor{currentstroke}%
\pgfsetdash{}{0pt}%
\pgfpathmoveto{\pgfqpoint{5.890707in}{2.106816in}}%
\pgfpathlineto{\pgfqpoint{5.906141in}{2.105586in}}%
\pgfpathlineto{\pgfqpoint{5.921575in}{2.101229in}}%
\pgfpathlineto{\pgfqpoint{5.937008in}{2.093742in}}%
\pgfpathlineto{\pgfqpoint{5.952442in}{2.083181in}}%
\pgfpathlineto{\pgfqpoint{5.967875in}{2.069656in}}%
\pgfpathlineto{\pgfqpoint{5.983309in}{2.053333in}}%
\pgfpathlineto{\pgfqpoint{6.014176in}{2.013231in}}%
\pgfpathlineto{\pgfqpoint{6.045044in}{1.965214in}}%
\pgfpathlineto{\pgfqpoint{6.091345in}{1.884996in}}%
\pgfpathlineto{\pgfqpoint{6.137645in}{1.805438in}}%
\pgfpathlineto{\pgfqpoint{6.168513in}{1.758761in}}%
\pgfpathlineto{\pgfqpoint{6.183946in}{1.738601in}}%
\pgfpathlineto{\pgfqpoint{6.199380in}{1.721067in}}%
\pgfpathlineto{\pgfqpoint{6.214814in}{1.706493in}}%
\pgfpathlineto{\pgfqpoint{6.230247in}{1.695176in}}%
\pgfpathlineto{\pgfqpoint{6.245681in}{1.687364in}}%
\pgfpathlineto{\pgfqpoint{6.261115in}{1.683260in}}%
\pgfpathlineto{\pgfqpoint{6.276548in}{1.683014in}}%
\pgfpathlineto{\pgfqpoint{6.291982in}{1.686728in}}%
\pgfpathlineto{\pgfqpoint{6.307415in}{1.694446in}}%
\pgfpathlineto{\pgfqpoint{6.322849in}{1.706164in}}%
\pgfpathlineto{\pgfqpoint{6.338283in}{1.721822in}}%
\pgfpathlineto{\pgfqpoint{6.353716in}{1.741311in}}%
\pgfpathlineto{\pgfqpoint{6.369150in}{1.764475in}}%
\pgfpathlineto{\pgfqpoint{6.384584in}{1.791110in}}%
\pgfpathlineto{\pgfqpoint{6.415451in}{1.853765in}}%
\pgfpathlineto{\pgfqpoint{6.446318in}{1.926856in}}%
\pgfpathlineto{\pgfqpoint{6.492619in}{2.049572in}}%
\pgfpathlineto{\pgfqpoint{6.569787in}{2.260813in}}%
\pgfpathlineto{\pgfqpoint{6.600655in}{2.337560in}}%
\pgfpathlineto{\pgfqpoint{6.631522in}{2.405053in}}%
\pgfpathlineto{\pgfqpoint{6.662389in}{2.460575in}}%
\pgfpathlineto{\pgfqpoint{6.677823in}{2.483141in}}%
\pgfpathlineto{\pgfqpoint{6.693256in}{2.501939in}}%
\pgfpathlineto{\pgfqpoint{6.708690in}{2.516796in}}%
\pgfpathlineto{\pgfqpoint{6.724124in}{2.527578in}}%
\pgfpathlineto{\pgfqpoint{6.739557in}{2.534198in}}%
\pgfpathlineto{\pgfqpoint{6.754991in}{2.536612in}}%
\pgfpathlineto{\pgfqpoint{6.770424in}{2.534825in}}%
\pgfpathlineto{\pgfqpoint{6.785858in}{2.528883in}}%
\pgfpathlineto{\pgfqpoint{6.801292in}{2.518879in}}%
\pgfpathlineto{\pgfqpoint{6.816725in}{2.504948in}}%
\pgfpathlineto{\pgfqpoint{6.832159in}{2.487264in}}%
\pgfpathlineto{\pgfqpoint{6.847593in}{2.466042in}}%
\pgfpathlineto{\pgfqpoint{6.863026in}{2.441528in}}%
\pgfpathlineto{\pgfqpoint{6.893894in}{2.383770in}}%
\pgfpathlineto{\pgfqpoint{6.924761in}{2.316528in}}%
\pgfpathlineto{\pgfqpoint{6.971062in}{2.204126in}}%
\pgfpathlineto{\pgfqpoint{7.048230in}{2.010686in}}%
\pgfpathlineto{\pgfqpoint{7.079097in}{1.939190in}}%
\pgfpathlineto{\pgfqpoint{7.109964in}{1.874435in}}%
\pgfpathlineto{\pgfqpoint{7.140832in}{1.818014in}}%
\pgfpathlineto{\pgfqpoint{7.171699in}{1.770944in}}%
\pgfpathlineto{\pgfqpoint{7.202566in}{1.733637in}}%
\pgfpathlineto{\pgfqpoint{7.218000in}{1.718607in}}%
\pgfpathlineto{\pgfqpoint{7.233434in}{1.705900in}}%
\pgfpathlineto{\pgfqpoint{7.248867in}{1.695405in}}%
\pgfpathlineto{\pgfqpoint{7.264301in}{1.686978in}}%
\pgfpathlineto{\pgfqpoint{7.279734in}{1.680450in}}%
\pgfpathlineto{\pgfqpoint{7.310602in}{1.672293in}}%
\pgfpathlineto{\pgfqpoint{7.341469in}{1.669165in}}%
\pgfpathlineto{\pgfqpoint{7.387770in}{1.669664in}}%
\pgfpathlineto{\pgfqpoint{7.434071in}{1.670744in}}%
\pgfpathlineto{\pgfqpoint{7.464938in}{1.669154in}}%
\pgfpathlineto{\pgfqpoint{7.495805in}{1.664479in}}%
\pgfpathlineto{\pgfqpoint{7.526673in}{1.656275in}}%
\pgfpathlineto{\pgfqpoint{7.557540in}{1.644699in}}%
\pgfpathlineto{\pgfqpoint{7.603841in}{1.622810in}}%
\pgfpathlineto{\pgfqpoint{7.665575in}{1.592922in}}%
\pgfpathlineto{\pgfqpoint{7.696443in}{1.581849in}}%
\pgfpathlineto{\pgfqpoint{7.727310in}{1.575864in}}%
\pgfpathlineto{\pgfqpoint{7.742744in}{1.575285in}}%
\pgfpathlineto{\pgfqpoint{7.758177in}{1.576530in}}%
\pgfpathlineto{\pgfqpoint{7.773611in}{1.579715in}}%
\pgfpathlineto{\pgfqpoint{7.789044in}{1.584920in}}%
\pgfpathlineto{\pgfqpoint{7.804478in}{1.592179in}}%
\pgfpathlineto{\pgfqpoint{7.819912in}{1.601481in}}%
\pgfpathlineto{\pgfqpoint{7.850779in}{1.625946in}}%
\pgfpathlineto{\pgfqpoint{7.881646in}{1.657332in}}%
\pgfpathlineto{\pgfqpoint{7.912514in}{1.694003in}}%
\pgfpathlineto{\pgfqpoint{8.020549in}{1.830484in}}%
\pgfpathlineto{\pgfqpoint{8.051416in}{1.861492in}}%
\pgfpathlineto{\pgfqpoint{8.066850in}{1.874283in}}%
\pgfpathlineto{\pgfqpoint{8.082284in}{1.884996in}}%
\pgfpathlineto{\pgfqpoint{8.097717in}{1.893476in}}%
\pgfpathlineto{\pgfqpoint{8.113151in}{1.899610in}}%
\pgfpathlineto{\pgfqpoint{8.128584in}{1.903324in}}%
\pgfpathlineto{\pgfqpoint{8.144018in}{1.904585in}}%
\pgfpathlineto{\pgfqpoint{8.159452in}{1.903400in}}%
\pgfpathlineto{\pgfqpoint{8.174885in}{1.899815in}}%
\pgfpathlineto{\pgfqpoint{8.190319in}{1.893909in}}%
\pgfpathlineto{\pgfqpoint{8.205753in}{1.885794in}}%
\pgfpathlineto{\pgfqpoint{8.221186in}{1.875609in}}%
\pgfpathlineto{\pgfqpoint{8.252053in}{1.849694in}}%
\pgfpathlineto{\pgfqpoint{8.282921in}{1.817652in}}%
\pgfpathlineto{\pgfqpoint{8.313788in}{1.781116in}}%
\pgfpathlineto{\pgfqpoint{8.375523in}{1.701054in}}%
\pgfpathlineto{\pgfqpoint{8.437257in}{1.621196in}}%
\pgfpathlineto{\pgfqpoint{8.483558in}{1.566913in}}%
\pgfpathlineto{\pgfqpoint{8.514425in}{1.535040in}}%
\pgfpathlineto{\pgfqpoint{8.545293in}{1.507478in}}%
\pgfpathlineto{\pgfqpoint{8.576160in}{1.484904in}}%
\pgfpathlineto{\pgfqpoint{8.607027in}{1.467967in}}%
\pgfpathlineto{\pgfqpoint{8.637894in}{1.457284in}}%
\pgfpathlineto{\pgfqpoint{8.653328in}{1.454463in}}%
\pgfpathlineto{\pgfqpoint{8.668762in}{1.453404in}}%
\pgfpathlineto{\pgfqpoint{8.684195in}{1.454154in}}%
\pgfpathlineto{\pgfqpoint{8.699629in}{1.456753in}}%
\pgfpathlineto{\pgfqpoint{8.715063in}{1.461221in}}%
\pgfpathlineto{\pgfqpoint{8.730496in}{1.467566in}}%
\pgfpathlineto{\pgfqpoint{8.745930in}{1.475775in}}%
\pgfpathlineto{\pgfqpoint{8.776797in}{1.497623in}}%
\pgfpathlineto{\pgfqpoint{8.807664in}{1.526216in}}%
\pgfpathlineto{\pgfqpoint{8.838532in}{1.560611in}}%
\pgfpathlineto{\pgfqpoint{8.869399in}{1.599482in}}%
\pgfpathlineto{\pgfqpoint{8.992868in}{1.762974in}}%
\pgfpathlineto{\pgfqpoint{9.023735in}{1.795671in}}%
\pgfpathlineto{\pgfqpoint{9.054603in}{1.821426in}}%
\pgfpathlineto{\pgfqpoint{9.070036in}{1.831298in}}%
\pgfpathlineto{\pgfqpoint{9.085470in}{1.839017in}}%
\pgfpathlineto{\pgfqpoint{9.100903in}{1.844517in}}%
\pgfpathlineto{\pgfqpoint{9.116337in}{1.847772in}}%
\pgfpathlineto{\pgfqpoint{9.131771in}{1.848795in}}%
\pgfpathlineto{\pgfqpoint{9.147204in}{1.847644in}}%
\pgfpathlineto{\pgfqpoint{9.162638in}{1.844413in}}%
\pgfpathlineto{\pgfqpoint{9.178072in}{1.839240in}}%
\pgfpathlineto{\pgfqpoint{9.208939in}{1.823798in}}%
\pgfpathlineto{\pgfqpoint{9.239806in}{1.803109in}}%
\pgfpathlineto{\pgfqpoint{9.347842in}{1.723283in}}%
\pgfpathlineto{\pgfqpoint{9.378709in}{1.708492in}}%
\pgfpathlineto{\pgfqpoint{9.394143in}{1.703765in}}%
\pgfpathlineto{\pgfqpoint{9.409576in}{1.701066in}}%
\pgfpathlineto{\pgfqpoint{9.425010in}{1.700531in}}%
\pgfpathlineto{\pgfqpoint{9.440443in}{1.702250in}}%
\pgfpathlineto{\pgfqpoint{9.455877in}{1.706267in}}%
\pgfpathlineto{\pgfqpoint{9.471311in}{1.712577in}}%
\pgfpathlineto{\pgfqpoint{9.486744in}{1.721125in}}%
\pgfpathlineto{\pgfqpoint{9.502178in}{1.731812in}}%
\pgfpathlineto{\pgfqpoint{9.533045in}{1.758971in}}%
\pgfpathlineto{\pgfqpoint{9.563913in}{1.792396in}}%
\pgfpathlineto{\pgfqpoint{9.610213in}{1.849419in}}%
\pgfpathlineto{\pgfqpoint{9.656514in}{1.907062in}}%
\pgfpathlineto{\pgfqpoint{9.687382in}{1.941635in}}%
\pgfpathlineto{\pgfqpoint{9.718249in}{1.970617in}}%
\pgfpathlineto{\pgfqpoint{9.733682in}{1.982502in}}%
\pgfpathlineto{\pgfqpoint{9.733682in}{1.982502in}}%
\pgfusepath{stroke}%
\end{pgfscope}%
\begin{pgfscope}%
\pgfpathrectangle{\pgfqpoint{5.698559in}{0.505056in}}{\pgfqpoint{4.227273in}{2.745455in}} %
\pgfusepath{clip}%
\pgfsetrectcap%
\pgfsetroundjoin%
\pgfsetlinewidth{0.501875pt}%
\definecolor{currentstroke}{rgb}{0.621569,0.981823,0.636474}%
\pgfsetstrokecolor{currentstroke}%
\pgfsetdash{}{0pt}%
\pgfpathmoveto{\pgfqpoint{5.890707in}{2.147103in}}%
\pgfpathlineto{\pgfqpoint{5.906141in}{2.151977in}}%
\pgfpathlineto{\pgfqpoint{5.921575in}{2.153647in}}%
\pgfpathlineto{\pgfqpoint{5.937008in}{2.152025in}}%
\pgfpathlineto{\pgfqpoint{5.952442in}{2.147086in}}%
\pgfpathlineto{\pgfqpoint{5.967875in}{2.138870in}}%
\pgfpathlineto{\pgfqpoint{5.983309in}{2.127486in}}%
\pgfpathlineto{\pgfqpoint{5.998743in}{2.113102in}}%
\pgfpathlineto{\pgfqpoint{6.014176in}{2.095952in}}%
\pgfpathlineto{\pgfqpoint{6.045044in}{2.054547in}}%
\pgfpathlineto{\pgfqpoint{6.075911in}{2.006132in}}%
\pgfpathlineto{\pgfqpoint{6.153079in}{1.877694in}}%
\pgfpathlineto{\pgfqpoint{6.183946in}{1.833231in}}%
\pgfpathlineto{\pgfqpoint{6.199380in}{1.814263in}}%
\pgfpathlineto{\pgfqpoint{6.214814in}{1.797985in}}%
\pgfpathlineto{\pgfqpoint{6.230247in}{1.784735in}}%
\pgfpathlineto{\pgfqpoint{6.245681in}{1.774808in}}%
\pgfpathlineto{\pgfqpoint{6.261115in}{1.768447in}}%
\pgfpathlineto{\pgfqpoint{6.276548in}{1.765845in}}%
\pgfpathlineto{\pgfqpoint{6.291982in}{1.767142in}}%
\pgfpathlineto{\pgfqpoint{6.307415in}{1.772422in}}%
\pgfpathlineto{\pgfqpoint{6.322849in}{1.781711in}}%
\pgfpathlineto{\pgfqpoint{6.338283in}{1.794982in}}%
\pgfpathlineto{\pgfqpoint{6.353716in}{1.812151in}}%
\pgfpathlineto{\pgfqpoint{6.369150in}{1.833080in}}%
\pgfpathlineto{\pgfqpoint{6.384584in}{1.857580in}}%
\pgfpathlineto{\pgfqpoint{6.415451in}{1.916300in}}%
\pgfpathlineto{\pgfqpoint{6.446318in}{1.985883in}}%
\pgfpathlineto{\pgfqpoint{6.477185in}{2.063310in}}%
\pgfpathlineto{\pgfqpoint{6.585221in}{2.344330in}}%
\pgfpathlineto{\pgfqpoint{6.616088in}{2.412271in}}%
\pgfpathlineto{\pgfqpoint{6.646955in}{2.468641in}}%
\pgfpathlineto{\pgfqpoint{6.662389in}{2.491674in}}%
\pgfpathlineto{\pgfqpoint{6.677823in}{2.510927in}}%
\pgfpathlineto{\pgfqpoint{6.693256in}{2.526196in}}%
\pgfpathlineto{\pgfqpoint{6.708690in}{2.537330in}}%
\pgfpathlineto{\pgfqpoint{6.724124in}{2.544230in}}%
\pgfpathlineto{\pgfqpoint{6.739557in}{2.546848in}}%
\pgfpathlineto{\pgfqpoint{6.754991in}{2.545189in}}%
\pgfpathlineto{\pgfqpoint{6.770424in}{2.539310in}}%
\pgfpathlineto{\pgfqpoint{6.785858in}{2.529316in}}%
\pgfpathlineto{\pgfqpoint{6.801292in}{2.515360in}}%
\pgfpathlineto{\pgfqpoint{6.816725in}{2.497638in}}%
\pgfpathlineto{\pgfqpoint{6.832159in}{2.476386in}}%
\pgfpathlineto{\pgfqpoint{6.847593in}{2.451877in}}%
\pgfpathlineto{\pgfqpoint{6.878460in}{2.394321in}}%
\pgfpathlineto{\pgfqpoint{6.909327in}{2.327668in}}%
\pgfpathlineto{\pgfqpoint{6.955628in}{2.217091in}}%
\pgfpathlineto{\pgfqpoint{7.032796in}{2.029137in}}%
\pgfpathlineto{\pgfqpoint{7.063664in}{1.960433in}}%
\pgfpathlineto{\pgfqpoint{7.094531in}{1.898637in}}%
\pgfpathlineto{\pgfqpoint{7.125398in}{1.845264in}}%
\pgfpathlineto{\pgfqpoint{7.156265in}{1.801305in}}%
\pgfpathlineto{\pgfqpoint{7.171699in}{1.783018in}}%
\pgfpathlineto{\pgfqpoint{7.187133in}{1.767210in}}%
\pgfpathlineto{\pgfqpoint{7.202566in}{1.753851in}}%
\pgfpathlineto{\pgfqpoint{7.218000in}{1.742874in}}%
\pgfpathlineto{\pgfqpoint{7.233434in}{1.734181in}}%
\pgfpathlineto{\pgfqpoint{7.248867in}{1.727641in}}%
\pgfpathlineto{\pgfqpoint{7.264301in}{1.723091in}}%
\pgfpathlineto{\pgfqpoint{7.279734in}{1.720341in}}%
\pgfpathlineto{\pgfqpoint{7.310602in}{1.719345in}}%
\pgfpathlineto{\pgfqpoint{7.341469in}{1.722650in}}%
\pgfpathlineto{\pgfqpoint{7.418637in}{1.734842in}}%
\pgfpathlineto{\pgfqpoint{7.449504in}{1.735438in}}%
\pgfpathlineto{\pgfqpoint{7.480372in}{1.730831in}}%
\pgfpathlineto{\pgfqpoint{7.495805in}{1.726207in}}%
\pgfpathlineto{\pgfqpoint{7.511239in}{1.719925in}}%
\pgfpathlineto{\pgfqpoint{7.542106in}{1.702365in}}%
\pgfpathlineto{\pgfqpoint{7.572974in}{1.678626in}}%
\pgfpathlineto{\pgfqpoint{7.603841in}{1.650023in}}%
\pgfpathlineto{\pgfqpoint{7.696443in}{1.558352in}}%
\pgfpathlineto{\pgfqpoint{7.727310in}{1.535460in}}%
\pgfpathlineto{\pgfqpoint{7.742744in}{1.527062in}}%
\pgfpathlineto{\pgfqpoint{7.758177in}{1.521089in}}%
\pgfpathlineto{\pgfqpoint{7.773611in}{1.517779in}}%
\pgfpathlineto{\pgfqpoint{7.789044in}{1.517313in}}%
\pgfpathlineto{\pgfqpoint{7.804478in}{1.519811in}}%
\pgfpathlineto{\pgfqpoint{7.819912in}{1.525327in}}%
\pgfpathlineto{\pgfqpoint{7.835345in}{1.533845in}}%
\pgfpathlineto{\pgfqpoint{7.850779in}{1.545279in}}%
\pgfpathlineto{\pgfqpoint{7.866213in}{1.559474in}}%
\pgfpathlineto{\pgfqpoint{7.897080in}{1.595198in}}%
\pgfpathlineto{\pgfqpoint{7.927947in}{1.638545in}}%
\pgfpathlineto{\pgfqpoint{8.035983in}{1.800945in}}%
\pgfpathlineto{\pgfqpoint{8.066850in}{1.835772in}}%
\pgfpathlineto{\pgfqpoint{8.082284in}{1.849411in}}%
\pgfpathlineto{\pgfqpoint{8.097717in}{1.860214in}}%
\pgfpathlineto{\pgfqpoint{8.113151in}{1.868020in}}%
\pgfpathlineto{\pgfqpoint{8.128584in}{1.872729in}}%
\pgfpathlineto{\pgfqpoint{8.144018in}{1.874305in}}%
\pgfpathlineto{\pgfqpoint{8.159452in}{1.872774in}}%
\pgfpathlineto{\pgfqpoint{8.174885in}{1.868218in}}%
\pgfpathlineto{\pgfqpoint{8.190319in}{1.860776in}}%
\pgfpathlineto{\pgfqpoint{8.205753in}{1.850631in}}%
\pgfpathlineto{\pgfqpoint{8.221186in}{1.838007in}}%
\pgfpathlineto{\pgfqpoint{8.252053in}{1.806376in}}%
\pgfpathlineto{\pgfqpoint{8.282921in}{1.768183in}}%
\pgfpathlineto{\pgfqpoint{8.329222in}{1.703892in}}%
\pgfpathlineto{\pgfqpoint{8.390956in}{1.616636in}}%
\pgfpathlineto{\pgfqpoint{8.437257in}{1.557358in}}%
\pgfpathlineto{\pgfqpoint{8.468124in}{1.522775in}}%
\pgfpathlineto{\pgfqpoint{8.498992in}{1.492833in}}%
\pgfpathlineto{\pgfqpoint{8.529859in}{1.467878in}}%
\pgfpathlineto{\pgfqpoint{8.560726in}{1.448150in}}%
\pgfpathlineto{\pgfqpoint{8.591593in}{1.433863in}}%
\pgfpathlineto{\pgfqpoint{8.622461in}{1.425253in}}%
\pgfpathlineto{\pgfqpoint{8.653328in}{1.422576in}}%
\pgfpathlineto{\pgfqpoint{8.668762in}{1.423539in}}%
\pgfpathlineto{\pgfqpoint{8.684195in}{1.426068in}}%
\pgfpathlineto{\pgfqpoint{8.715063in}{1.435870in}}%
\pgfpathlineto{\pgfqpoint{8.745930in}{1.451934in}}%
\pgfpathlineto{\pgfqpoint{8.776797in}{1.473942in}}%
\pgfpathlineto{\pgfqpoint{8.807664in}{1.501236in}}%
\pgfpathlineto{\pgfqpoint{8.838532in}{1.532798in}}%
\pgfpathlineto{\pgfqpoint{8.884833in}{1.585088in}}%
\pgfpathlineto{\pgfqpoint{8.946567in}{1.655081in}}%
\pgfpathlineto{\pgfqpoint{8.977434in}{1.686173in}}%
\pgfpathlineto{\pgfqpoint{9.008302in}{1.712530in}}%
\pgfpathlineto{\pgfqpoint{9.039169in}{1.732961in}}%
\pgfpathlineto{\pgfqpoint{9.070036in}{1.746738in}}%
\pgfpathlineto{\pgfqpoint{9.085470in}{1.751048in}}%
\pgfpathlineto{\pgfqpoint{9.100903in}{1.753662in}}%
\pgfpathlineto{\pgfqpoint{9.131771in}{1.754075in}}%
\pgfpathlineto{\pgfqpoint{9.162638in}{1.748836in}}%
\pgfpathlineto{\pgfqpoint{9.193505in}{1.739234in}}%
\pgfpathlineto{\pgfqpoint{9.239806in}{1.720227in}}%
\pgfpathlineto{\pgfqpoint{9.286107in}{1.701122in}}%
\pgfpathlineto{\pgfqpoint{9.316974in}{1.691266in}}%
\pgfpathlineto{\pgfqpoint{9.347842in}{1.685472in}}%
\pgfpathlineto{\pgfqpoint{9.378709in}{1.684856in}}%
\pgfpathlineto{\pgfqpoint{9.409576in}{1.690099in}}%
\pgfpathlineto{\pgfqpoint{9.440443in}{1.701397in}}%
\pgfpathlineto{\pgfqpoint{9.471311in}{1.718453in}}%
\pgfpathlineto{\pgfqpoint{9.502178in}{1.740504in}}%
\pgfpathlineto{\pgfqpoint{9.533045in}{1.766387in}}%
\pgfpathlineto{\pgfqpoint{9.656514in}{1.877220in}}%
\pgfpathlineto{\pgfqpoint{9.687382in}{1.898993in}}%
\pgfpathlineto{\pgfqpoint{9.718249in}{1.916078in}}%
\pgfpathlineto{\pgfqpoint{9.733682in}{1.922705in}}%
\pgfpathlineto{\pgfqpoint{9.733682in}{1.922705in}}%
\pgfusepath{stroke}%
\end{pgfscope}%
\begin{pgfscope}%
\pgfpathrectangle{\pgfqpoint{5.698559in}{0.505056in}}{\pgfqpoint{4.227273in}{2.745455in}} %
\pgfusepath{clip}%
\pgfsetrectcap%
\pgfsetroundjoin%
\pgfsetlinewidth{0.501875pt}%
\definecolor{currentstroke}{rgb}{0.700000,0.951057,0.587785}%
\pgfsetstrokecolor{currentstroke}%
\pgfsetdash{}{0pt}%
\pgfpathmoveto{\pgfqpoint{5.890707in}{2.150274in}}%
\pgfpathlineto{\pgfqpoint{5.906141in}{2.151128in}}%
\pgfpathlineto{\pgfqpoint{5.921575in}{2.149338in}}%
\pgfpathlineto{\pgfqpoint{5.937008in}{2.144809in}}%
\pgfpathlineto{\pgfqpoint{5.952442in}{2.137498in}}%
\pgfpathlineto{\pgfqpoint{5.967875in}{2.127417in}}%
\pgfpathlineto{\pgfqpoint{5.983309in}{2.114636in}}%
\pgfpathlineto{\pgfqpoint{5.998743in}{2.099276in}}%
\pgfpathlineto{\pgfqpoint{6.029610in}{2.061572in}}%
\pgfpathlineto{\pgfqpoint{6.060477in}{2.016278in}}%
\pgfpathlineto{\pgfqpoint{6.106778in}{1.940004in}}%
\pgfpathlineto{\pgfqpoint{6.153079in}{1.863441in}}%
\pgfpathlineto{\pgfqpoint{6.183946in}{1.818035in}}%
\pgfpathlineto{\pgfqpoint{6.199380in}{1.798312in}}%
\pgfpathlineto{\pgfqpoint{6.214814in}{1.781114in}}%
\pgfpathlineto{\pgfqpoint{6.230247in}{1.766812in}}%
\pgfpathlineto{\pgfqpoint{6.245681in}{1.755743in}}%
\pgfpathlineto{\pgfqpoint{6.261115in}{1.748208in}}%
\pgfpathlineto{\pgfqpoint{6.276548in}{1.744461in}}%
\pgfpathlineto{\pgfqpoint{6.291982in}{1.744709in}}%
\pgfpathlineto{\pgfqpoint{6.307415in}{1.749103in}}%
\pgfpathlineto{\pgfqpoint{6.322849in}{1.757739in}}%
\pgfpathlineto{\pgfqpoint{6.338283in}{1.770650in}}%
\pgfpathlineto{\pgfqpoint{6.353716in}{1.787806in}}%
\pgfpathlineto{\pgfqpoint{6.369150in}{1.809114in}}%
\pgfpathlineto{\pgfqpoint{6.384584in}{1.834414in}}%
\pgfpathlineto{\pgfqpoint{6.400017in}{1.863485in}}%
\pgfpathlineto{\pgfqpoint{6.430885in}{1.931744in}}%
\pgfpathlineto{\pgfqpoint{6.461752in}{2.010948in}}%
\pgfpathlineto{\pgfqpoint{6.508053in}{2.142046in}}%
\pgfpathlineto{\pgfqpoint{6.569787in}{2.317150in}}%
\pgfpathlineto{\pgfqpoint{6.600655in}{2.395119in}}%
\pgfpathlineto{\pgfqpoint{6.631522in}{2.461665in}}%
\pgfpathlineto{\pgfqpoint{6.646955in}{2.489726in}}%
\pgfpathlineto{\pgfqpoint{6.662389in}{2.513931in}}%
\pgfpathlineto{\pgfqpoint{6.677823in}{2.534070in}}%
\pgfpathlineto{\pgfqpoint{6.693256in}{2.549998in}}%
\pgfpathlineto{\pgfqpoint{6.708690in}{2.561629in}}%
\pgfpathlineto{\pgfqpoint{6.724124in}{2.568943in}}%
\pgfpathlineto{\pgfqpoint{6.739557in}{2.571975in}}%
\pgfpathlineto{\pgfqpoint{6.754991in}{2.570812in}}%
\pgfpathlineto{\pgfqpoint{6.770424in}{2.565592in}}%
\pgfpathlineto{\pgfqpoint{6.785858in}{2.556491in}}%
\pgfpathlineto{\pgfqpoint{6.801292in}{2.543722in}}%
\pgfpathlineto{\pgfqpoint{6.816725in}{2.527525in}}%
\pgfpathlineto{\pgfqpoint{6.832159in}{2.508162in}}%
\pgfpathlineto{\pgfqpoint{6.863026in}{2.461053in}}%
\pgfpathlineto{\pgfqpoint{6.893894in}{2.404676in}}%
\pgfpathlineto{\pgfqpoint{6.924761in}{2.341300in}}%
\pgfpathlineto{\pgfqpoint{6.971062in}{2.237797in}}%
\pgfpathlineto{\pgfqpoint{7.079097in}{1.989989in}}%
\pgfpathlineto{\pgfqpoint{7.109964in}{1.925250in}}%
\pgfpathlineto{\pgfqpoint{7.140832in}{1.866085in}}%
\pgfpathlineto{\pgfqpoint{7.171699in}{1.813865in}}%
\pgfpathlineto{\pgfqpoint{7.202566in}{1.769702in}}%
\pgfpathlineto{\pgfqpoint{7.233434in}{1.734324in}}%
\pgfpathlineto{\pgfqpoint{7.248867in}{1.720019in}}%
\pgfpathlineto{\pgfqpoint{7.264301in}{1.707947in}}%
\pgfpathlineto{\pgfqpoint{7.279734in}{1.698038in}}%
\pgfpathlineto{\pgfqpoint{7.295168in}{1.690185in}}%
\pgfpathlineto{\pgfqpoint{7.310602in}{1.684238in}}%
\pgfpathlineto{\pgfqpoint{7.341469in}{1.677281in}}%
\pgfpathlineto{\pgfqpoint{7.372336in}{1.675279in}}%
\pgfpathlineto{\pgfqpoint{7.464938in}{1.675796in}}%
\pgfpathlineto{\pgfqpoint{7.495805in}{1.670759in}}%
\pgfpathlineto{\pgfqpoint{7.526673in}{1.660564in}}%
\pgfpathlineto{\pgfqpoint{7.557540in}{1.644840in}}%
\pgfpathlineto{\pgfqpoint{7.588407in}{1.624160in}}%
\pgfpathlineto{\pgfqpoint{7.634708in}{1.587343in}}%
\pgfpathlineto{\pgfqpoint{7.681009in}{1.551032in}}%
\pgfpathlineto{\pgfqpoint{7.711876in}{1.532042in}}%
\pgfpathlineto{\pgfqpoint{7.727310in}{1.525223in}}%
\pgfpathlineto{\pgfqpoint{7.742744in}{1.520627in}}%
\pgfpathlineto{\pgfqpoint{7.758177in}{1.518533in}}%
\pgfpathlineto{\pgfqpoint{7.773611in}{1.519168in}}%
\pgfpathlineto{\pgfqpoint{7.789044in}{1.522699in}}%
\pgfpathlineto{\pgfqpoint{7.804478in}{1.529223in}}%
\pgfpathlineto{\pgfqpoint{7.819912in}{1.538769in}}%
\pgfpathlineto{\pgfqpoint{7.835345in}{1.551288in}}%
\pgfpathlineto{\pgfqpoint{7.850779in}{1.566658in}}%
\pgfpathlineto{\pgfqpoint{7.866213in}{1.584681in}}%
\pgfpathlineto{\pgfqpoint{7.897080in}{1.627553in}}%
\pgfpathlineto{\pgfqpoint{7.927947in}{1.677029in}}%
\pgfpathlineto{\pgfqpoint{8.005115in}{1.804818in}}%
\pgfpathlineto{\pgfqpoint{8.035983in}{1.847027in}}%
\pgfpathlineto{\pgfqpoint{8.051416in}{1.864403in}}%
\pgfpathlineto{\pgfqpoint{8.066850in}{1.878822in}}%
\pgfpathlineto{\pgfqpoint{8.082284in}{1.890008in}}%
\pgfpathlineto{\pgfqpoint{8.097717in}{1.897757in}}%
\pgfpathlineto{\pgfqpoint{8.113151in}{1.901939in}}%
\pgfpathlineto{\pgfqpoint{8.128584in}{1.902501in}}%
\pgfpathlineto{\pgfqpoint{8.144018in}{1.899467in}}%
\pgfpathlineto{\pgfqpoint{8.159452in}{1.892935in}}%
\pgfpathlineto{\pgfqpoint{8.174885in}{1.883071in}}%
\pgfpathlineto{\pgfqpoint{8.190319in}{1.870106in}}%
\pgfpathlineto{\pgfqpoint{8.205753in}{1.854324in}}%
\pgfpathlineto{\pgfqpoint{8.236620in}{1.815664in}}%
\pgfpathlineto{\pgfqpoint{8.267487in}{1.770084in}}%
\pgfpathlineto{\pgfqpoint{8.375523in}{1.599850in}}%
\pgfpathlineto{\pgfqpoint{8.406390in}{1.557854in}}%
\pgfpathlineto{\pgfqpoint{8.437257in}{1.521100in}}%
\pgfpathlineto{\pgfqpoint{8.468124in}{1.489697in}}%
\pgfpathlineto{\pgfqpoint{8.498992in}{1.463263in}}%
\pgfpathlineto{\pgfqpoint{8.529859in}{1.441158in}}%
\pgfpathlineto{\pgfqpoint{8.560726in}{1.422729in}}%
\pgfpathlineto{\pgfqpoint{8.591593in}{1.407536in}}%
\pgfpathlineto{\pgfqpoint{8.622461in}{1.395489in}}%
\pgfpathlineto{\pgfqpoint{8.653328in}{1.386903in}}%
\pgfpathlineto{\pgfqpoint{8.684195in}{1.382444in}}%
\pgfpathlineto{\pgfqpoint{8.715063in}{1.382985in}}%
\pgfpathlineto{\pgfqpoint{8.730496in}{1.385408in}}%
\pgfpathlineto{\pgfqpoint{8.745930in}{1.389396in}}%
\pgfpathlineto{\pgfqpoint{8.761363in}{1.395017in}}%
\pgfpathlineto{\pgfqpoint{8.792231in}{1.411303in}}%
\pgfpathlineto{\pgfqpoint{8.823098in}{1.434186in}}%
\pgfpathlineto{\pgfqpoint{8.853965in}{1.462969in}}%
\pgfpathlineto{\pgfqpoint{8.884833in}{1.496327in}}%
\pgfpathlineto{\pgfqpoint{8.992868in}{1.620423in}}%
\pgfpathlineto{\pgfqpoint{9.023735in}{1.649411in}}%
\pgfpathlineto{\pgfqpoint{9.054603in}{1.672708in}}%
\pgfpathlineto{\pgfqpoint{9.085470in}{1.689808in}}%
\pgfpathlineto{\pgfqpoint{9.116337in}{1.700961in}}%
\pgfpathlineto{\pgfqpoint{9.147204in}{1.707109in}}%
\pgfpathlineto{\pgfqpoint{9.178072in}{1.709731in}}%
\pgfpathlineto{\pgfqpoint{9.270673in}{1.714453in}}%
\pgfpathlineto{\pgfqpoint{9.301541in}{1.720376in}}%
\pgfpathlineto{\pgfqpoint{9.332408in}{1.730194in}}%
\pgfpathlineto{\pgfqpoint{9.363275in}{1.744128in}}%
\pgfpathlineto{\pgfqpoint{9.394143in}{1.761850in}}%
\pgfpathlineto{\pgfqpoint{9.440443in}{1.793720in}}%
\pgfpathlineto{\pgfqpoint{9.533045in}{1.861273in}}%
\pgfpathlineto{\pgfqpoint{9.563913in}{1.880233in}}%
\pgfpathlineto{\pgfqpoint{9.594780in}{1.895683in}}%
\pgfpathlineto{\pgfqpoint{9.625647in}{1.906980in}}%
\pgfpathlineto{\pgfqpoint{9.656514in}{1.913794in}}%
\pgfpathlineto{\pgfqpoint{9.687382in}{1.916118in}}%
\pgfpathlineto{\pgfqpoint{9.718249in}{1.914249in}}%
\pgfpathlineto{\pgfqpoint{9.733682in}{1.911914in}}%
\pgfpathlineto{\pgfqpoint{9.733682in}{1.911914in}}%
\pgfusepath{stroke}%
\end{pgfscope}%
\begin{pgfscope}%
\pgfpathrectangle{\pgfqpoint{5.698559in}{0.505056in}}{\pgfqpoint{4.227273in}{2.745455in}} %
\pgfusepath{clip}%
\pgfsetrectcap%
\pgfsetroundjoin%
\pgfsetlinewidth{0.501875pt}%
\definecolor{currentstroke}{rgb}{0.778431,0.905873,0.536867}%
\pgfsetstrokecolor{currentstroke}%
\pgfsetdash{}{0pt}%
\pgfpathmoveto{\pgfqpoint{5.890707in}{2.104165in}}%
\pgfpathlineto{\pgfqpoint{5.906141in}{2.101455in}}%
\pgfpathlineto{\pgfqpoint{5.921575in}{2.096507in}}%
\pgfpathlineto{\pgfqpoint{5.937008in}{2.089232in}}%
\pgfpathlineto{\pgfqpoint{5.952442in}{2.079583in}}%
\pgfpathlineto{\pgfqpoint{5.967875in}{2.067553in}}%
\pgfpathlineto{\pgfqpoint{5.983309in}{2.053179in}}%
\pgfpathlineto{\pgfqpoint{6.014176in}{2.017776in}}%
\pgfpathlineto{\pgfqpoint{6.045044in}{1.974591in}}%
\pgfpathlineto{\pgfqpoint{6.075911in}{1.925557in}}%
\pgfpathlineto{\pgfqpoint{6.168513in}{1.771496in}}%
\pgfpathlineto{\pgfqpoint{6.199380in}{1.729047in}}%
\pgfpathlineto{\pgfqpoint{6.214814in}{1.711462in}}%
\pgfpathlineto{\pgfqpoint{6.230247in}{1.696850in}}%
\pgfpathlineto{\pgfqpoint{6.245681in}{1.685580in}}%
\pgfpathlineto{\pgfqpoint{6.261115in}{1.677986in}}%
\pgfpathlineto{\pgfqpoint{6.276548in}{1.674354in}}%
\pgfpathlineto{\pgfqpoint{6.291982in}{1.674917in}}%
\pgfpathlineto{\pgfqpoint{6.307415in}{1.679849in}}%
\pgfpathlineto{\pgfqpoint{6.322849in}{1.689261in}}%
\pgfpathlineto{\pgfqpoint{6.338283in}{1.703193in}}%
\pgfpathlineto{\pgfqpoint{6.353716in}{1.721616in}}%
\pgfpathlineto{\pgfqpoint{6.369150in}{1.744428in}}%
\pgfpathlineto{\pgfqpoint{6.384584in}{1.771453in}}%
\pgfpathlineto{\pgfqpoint{6.400017in}{1.802447in}}%
\pgfpathlineto{\pgfqpoint{6.430885in}{1.875025in}}%
\pgfpathlineto{\pgfqpoint{6.461752in}{1.958934in}}%
\pgfpathlineto{\pgfqpoint{6.508053in}{2.097126in}}%
\pgfpathlineto{\pgfqpoint{6.554354in}{2.236400in}}%
\pgfpathlineto{\pgfqpoint{6.585221in}{2.322260in}}%
\pgfpathlineto{\pgfqpoint{6.616088in}{2.397815in}}%
\pgfpathlineto{\pgfqpoint{6.631522in}{2.430692in}}%
\pgfpathlineto{\pgfqpoint{6.646955in}{2.459865in}}%
\pgfpathlineto{\pgfqpoint{6.662389in}{2.485083in}}%
\pgfpathlineto{\pgfqpoint{6.677823in}{2.506161in}}%
\pgfpathlineto{\pgfqpoint{6.693256in}{2.522978in}}%
\pgfpathlineto{\pgfqpoint{6.708690in}{2.535478in}}%
\pgfpathlineto{\pgfqpoint{6.724124in}{2.543661in}}%
\pgfpathlineto{\pgfqpoint{6.739557in}{2.547587in}}%
\pgfpathlineto{\pgfqpoint{6.754991in}{2.547364in}}%
\pgfpathlineto{\pgfqpoint{6.770424in}{2.543141in}}%
\pgfpathlineto{\pgfqpoint{6.785858in}{2.535108in}}%
\pgfpathlineto{\pgfqpoint{6.801292in}{2.523482in}}%
\pgfpathlineto{\pgfqpoint{6.816725in}{2.508504in}}%
\pgfpathlineto{\pgfqpoint{6.832159in}{2.490430in}}%
\pgfpathlineto{\pgfqpoint{6.863026in}{2.446055in}}%
\pgfpathlineto{\pgfqpoint{6.893894in}{2.392499in}}%
\pgfpathlineto{\pgfqpoint{6.924761in}{2.331830in}}%
\pgfpathlineto{\pgfqpoint{6.971062in}{2.231676in}}%
\pgfpathlineto{\pgfqpoint{7.109964in}{1.918481in}}%
\pgfpathlineto{\pgfqpoint{7.140832in}{1.856678in}}%
\pgfpathlineto{\pgfqpoint{7.171699in}{1.801149in}}%
\pgfpathlineto{\pgfqpoint{7.202566in}{1.753251in}}%
\pgfpathlineto{\pgfqpoint{7.233434in}{1.714000in}}%
\pgfpathlineto{\pgfqpoint{7.248867in}{1.697805in}}%
\pgfpathlineto{\pgfqpoint{7.264301in}{1.683925in}}%
\pgfpathlineto{\pgfqpoint{7.279734in}{1.672331in}}%
\pgfpathlineto{\pgfqpoint{7.295168in}{1.662951in}}%
\pgfpathlineto{\pgfqpoint{7.310602in}{1.655672in}}%
\pgfpathlineto{\pgfqpoint{7.326035in}{1.650340in}}%
\pgfpathlineto{\pgfqpoint{7.356903in}{1.644707in}}%
\pgfpathlineto{\pgfqpoint{7.387770in}{1.644131in}}%
\pgfpathlineto{\pgfqpoint{7.495805in}{1.649071in}}%
\pgfpathlineto{\pgfqpoint{7.526673in}{1.644332in}}%
\pgfpathlineto{\pgfqpoint{7.557540in}{1.634955in}}%
\pgfpathlineto{\pgfqpoint{7.588407in}{1.621263in}}%
\pgfpathlineto{\pgfqpoint{7.634708in}{1.595512in}}%
\pgfpathlineto{\pgfqpoint{7.681009in}{1.570077in}}%
\pgfpathlineto{\pgfqpoint{7.711876in}{1.557743in}}%
\pgfpathlineto{\pgfqpoint{7.727310in}{1.553968in}}%
\pgfpathlineto{\pgfqpoint{7.742744in}{1.552185in}}%
\pgfpathlineto{\pgfqpoint{7.758177in}{1.552643in}}%
\pgfpathlineto{\pgfqpoint{7.773611in}{1.555542in}}%
\pgfpathlineto{\pgfqpoint{7.789044in}{1.561023in}}%
\pgfpathlineto{\pgfqpoint{7.804478in}{1.569165in}}%
\pgfpathlineto{\pgfqpoint{7.819912in}{1.579978in}}%
\pgfpathlineto{\pgfqpoint{7.835345in}{1.593402in}}%
\pgfpathlineto{\pgfqpoint{7.850779in}{1.609304in}}%
\pgfpathlineto{\pgfqpoint{7.881646in}{1.647681in}}%
\pgfpathlineto{\pgfqpoint{7.912514in}{1.692766in}}%
\pgfpathlineto{\pgfqpoint{8.005115in}{1.834613in}}%
\pgfpathlineto{\pgfqpoint{8.035983in}{1.871885in}}%
\pgfpathlineto{\pgfqpoint{8.051416in}{1.886811in}}%
\pgfpathlineto{\pgfqpoint{8.066850in}{1.898870in}}%
\pgfpathlineto{\pgfqpoint{8.082284in}{1.907844in}}%
\pgfpathlineto{\pgfqpoint{8.097717in}{1.913584in}}%
\pgfpathlineto{\pgfqpoint{8.113151in}{1.916013in}}%
\pgfpathlineto{\pgfqpoint{8.128584in}{1.915127in}}%
\pgfpathlineto{\pgfqpoint{8.144018in}{1.910989in}}%
\pgfpathlineto{\pgfqpoint{8.159452in}{1.903733in}}%
\pgfpathlineto{\pgfqpoint{8.174885in}{1.893551in}}%
\pgfpathlineto{\pgfqpoint{8.190319in}{1.880692in}}%
\pgfpathlineto{\pgfqpoint{8.221186in}{1.848147in}}%
\pgfpathlineto{\pgfqpoint{8.252053in}{1.808805in}}%
\pgfpathlineto{\pgfqpoint{8.313788in}{1.721334in}}%
\pgfpathlineto{\pgfqpoint{8.360089in}{1.658041in}}%
\pgfpathlineto{\pgfqpoint{8.390956in}{1.620217in}}%
\pgfpathlineto{\pgfqpoint{8.421823in}{1.586760in}}%
\pgfpathlineto{\pgfqpoint{8.452691in}{1.557639in}}%
\pgfpathlineto{\pgfqpoint{8.483558in}{1.532314in}}%
\pgfpathlineto{\pgfqpoint{8.529859in}{1.499644in}}%
\pgfpathlineto{\pgfqpoint{8.576160in}{1.471112in}}%
\pgfpathlineto{\pgfqpoint{8.622461in}{1.445755in}}%
\pgfpathlineto{\pgfqpoint{8.653328in}{1.431247in}}%
\pgfpathlineto{\pgfqpoint{8.684195in}{1.419778in}}%
\pgfpathlineto{\pgfqpoint{8.715063in}{1.412649in}}%
\pgfpathlineto{\pgfqpoint{8.730496in}{1.411141in}}%
\pgfpathlineto{\pgfqpoint{8.745930in}{1.411216in}}%
\pgfpathlineto{\pgfqpoint{8.761363in}{1.413012in}}%
\pgfpathlineto{\pgfqpoint{8.776797in}{1.416638in}}%
\pgfpathlineto{\pgfqpoint{8.792231in}{1.422167in}}%
\pgfpathlineto{\pgfqpoint{8.807664in}{1.429636in}}%
\pgfpathlineto{\pgfqpoint{8.823098in}{1.439037in}}%
\pgfpathlineto{\pgfqpoint{8.853965in}{1.463385in}}%
\pgfpathlineto{\pgfqpoint{8.884833in}{1.494262in}}%
\pgfpathlineto{\pgfqpoint{8.915700in}{1.530086in}}%
\pgfpathlineto{\pgfqpoint{9.023735in}{1.663492in}}%
\pgfpathlineto{\pgfqpoint{9.054603in}{1.695371in}}%
\pgfpathlineto{\pgfqpoint{9.085470in}{1.721916in}}%
\pgfpathlineto{\pgfqpoint{9.116337in}{1.742798in}}%
\pgfpathlineto{\pgfqpoint{9.147204in}{1.758385in}}%
\pgfpathlineto{\pgfqpoint{9.178072in}{1.769643in}}%
\pgfpathlineto{\pgfqpoint{9.224373in}{1.781486in}}%
\pgfpathlineto{\pgfqpoint{9.286107in}{1.795865in}}%
\pgfpathlineto{\pgfqpoint{9.316974in}{1.805253in}}%
\pgfpathlineto{\pgfqpoint{9.347842in}{1.816997in}}%
\pgfpathlineto{\pgfqpoint{9.394143in}{1.838913in}}%
\pgfpathlineto{\pgfqpoint{9.517612in}{1.902743in}}%
\pgfpathlineto{\pgfqpoint{9.548479in}{1.913629in}}%
\pgfpathlineto{\pgfqpoint{9.579346in}{1.920508in}}%
\pgfpathlineto{\pgfqpoint{9.610213in}{1.922923in}}%
\pgfpathlineto{\pgfqpoint{9.641081in}{1.920823in}}%
\pgfpathlineto{\pgfqpoint{9.671948in}{1.914551in}}%
\pgfpathlineto{\pgfqpoint{9.702815in}{1.904793in}}%
\pgfpathlineto{\pgfqpoint{9.733682in}{1.892502in}}%
\pgfpathlineto{\pgfqpoint{9.733682in}{1.892502in}}%
\pgfusepath{stroke}%
\end{pgfscope}%
\begin{pgfscope}%
\pgfpathrectangle{\pgfqpoint{5.698559in}{0.505056in}}{\pgfqpoint{4.227273in}{2.745455in}} %
\pgfusepath{clip}%
\pgfsetrectcap%
\pgfsetroundjoin%
\pgfsetlinewidth{0.501875pt}%
\definecolor{currentstroke}{rgb}{0.864706,0.840344,0.478512}%
\pgfsetstrokecolor{currentstroke}%
\pgfsetdash{}{0pt}%
\pgfpathmoveto{\pgfqpoint{5.890707in}{2.125697in}}%
\pgfpathlineto{\pgfqpoint{5.906141in}{2.122893in}}%
\pgfpathlineto{\pgfqpoint{5.921575in}{2.117565in}}%
\pgfpathlineto{\pgfqpoint{5.937008in}{2.109669in}}%
\pgfpathlineto{\pgfqpoint{5.952442in}{2.099208in}}%
\pgfpathlineto{\pgfqpoint{5.967875in}{2.086229in}}%
\pgfpathlineto{\pgfqpoint{5.983309in}{2.070830in}}%
\pgfpathlineto{\pgfqpoint{6.014176in}{2.033389in}}%
\pgfpathlineto{\pgfqpoint{6.045044in}{1.988574in}}%
\pgfpathlineto{\pgfqpoint{6.091345in}{1.912781in}}%
\pgfpathlineto{\pgfqpoint{6.137645in}{1.835675in}}%
\pgfpathlineto{\pgfqpoint{6.168513in}{1.789064in}}%
\pgfpathlineto{\pgfqpoint{6.199380in}{1.750186in}}%
\pgfpathlineto{\pgfqpoint{6.214814in}{1.734627in}}%
\pgfpathlineto{\pgfqpoint{6.230247in}{1.722129in}}%
\pgfpathlineto{\pgfqpoint{6.245681in}{1.713009in}}%
\pgfpathlineto{\pgfqpoint{6.261115in}{1.707541in}}%
\pgfpathlineto{\pgfqpoint{6.276548in}{1.705959in}}%
\pgfpathlineto{\pgfqpoint{6.291982in}{1.708444in}}%
\pgfpathlineto{\pgfqpoint{6.307415in}{1.715129in}}%
\pgfpathlineto{\pgfqpoint{6.322849in}{1.726085in}}%
\pgfpathlineto{\pgfqpoint{6.338283in}{1.741328in}}%
\pgfpathlineto{\pgfqpoint{6.353716in}{1.760811in}}%
\pgfpathlineto{\pgfqpoint{6.369150in}{1.784428in}}%
\pgfpathlineto{\pgfqpoint{6.384584in}{1.812007in}}%
\pgfpathlineto{\pgfqpoint{6.415451in}{1.878075in}}%
\pgfpathlineto{\pgfqpoint{6.446318in}{1.956491in}}%
\pgfpathlineto{\pgfqpoint{6.477185in}{2.043921in}}%
\pgfpathlineto{\pgfqpoint{6.585221in}{2.361184in}}%
\pgfpathlineto{\pgfqpoint{6.616088in}{2.437338in}}%
\pgfpathlineto{\pgfqpoint{6.631522in}{2.470598in}}%
\pgfpathlineto{\pgfqpoint{6.646955in}{2.500151in}}%
\pgfpathlineto{\pgfqpoint{6.662389in}{2.525695in}}%
\pgfpathlineto{\pgfqpoint{6.677823in}{2.546985in}}%
\pgfpathlineto{\pgfqpoint{6.693256in}{2.563840in}}%
\pgfpathlineto{\pgfqpoint{6.708690in}{2.576140in}}%
\pgfpathlineto{\pgfqpoint{6.724124in}{2.583826in}}%
\pgfpathlineto{\pgfqpoint{6.739557in}{2.586903in}}%
\pgfpathlineto{\pgfqpoint{6.754991in}{2.585428in}}%
\pgfpathlineto{\pgfqpoint{6.770424in}{2.579518in}}%
\pgfpathlineto{\pgfqpoint{6.785858in}{2.569334in}}%
\pgfpathlineto{\pgfqpoint{6.801292in}{2.555083in}}%
\pgfpathlineto{\pgfqpoint{6.816725in}{2.537010in}}%
\pgfpathlineto{\pgfqpoint{6.832159in}{2.515391in}}%
\pgfpathlineto{\pgfqpoint{6.863026in}{2.462746in}}%
\pgfpathlineto{\pgfqpoint{6.893894in}{2.399775in}}%
\pgfpathlineto{\pgfqpoint{6.924761in}{2.329250in}}%
\pgfpathlineto{\pgfqpoint{6.986495in}{2.176549in}}%
\pgfpathlineto{\pgfqpoint{7.048230in}{2.025209in}}%
\pgfpathlineto{\pgfqpoint{7.079097in}{1.955547in}}%
\pgfpathlineto{\pgfqpoint{7.109964in}{1.892268in}}%
\pgfpathlineto{\pgfqpoint{7.140832in}{1.836734in}}%
\pgfpathlineto{\pgfqpoint{7.171699in}{1.789914in}}%
\pgfpathlineto{\pgfqpoint{7.202566in}{1.752324in}}%
\pgfpathlineto{\pgfqpoint{7.218000in}{1.737014in}}%
\pgfpathlineto{\pgfqpoint{7.233434in}{1.723976in}}%
\pgfpathlineto{\pgfqpoint{7.248867in}{1.713128in}}%
\pgfpathlineto{\pgfqpoint{7.264301in}{1.704354in}}%
\pgfpathlineto{\pgfqpoint{7.279734in}{1.697507in}}%
\pgfpathlineto{\pgfqpoint{7.310602in}{1.688858in}}%
\pgfpathlineto{\pgfqpoint{7.341469in}{1.685459in}}%
\pgfpathlineto{\pgfqpoint{7.387770in}{1.685736in}}%
\pgfpathlineto{\pgfqpoint{7.434071in}{1.686077in}}%
\pgfpathlineto{\pgfqpoint{7.464938in}{1.683167in}}%
\pgfpathlineto{\pgfqpoint{7.495805in}{1.676136in}}%
\pgfpathlineto{\pgfqpoint{7.526673in}{1.664301in}}%
\pgfpathlineto{\pgfqpoint{7.557540in}{1.647706in}}%
\pgfpathlineto{\pgfqpoint{7.588407in}{1.627142in}}%
\pgfpathlineto{\pgfqpoint{7.696443in}{1.549687in}}%
\pgfpathlineto{\pgfqpoint{7.727310in}{1.535594in}}%
\pgfpathlineto{\pgfqpoint{7.742744in}{1.531335in}}%
\pgfpathlineto{\pgfqpoint{7.758177in}{1.529243in}}%
\pgfpathlineto{\pgfqpoint{7.773611in}{1.529498in}}%
\pgfpathlineto{\pgfqpoint{7.789044in}{1.532237in}}%
\pgfpathlineto{\pgfqpoint{7.804478in}{1.537540in}}%
\pgfpathlineto{\pgfqpoint{7.819912in}{1.545438in}}%
\pgfpathlineto{\pgfqpoint{7.835345in}{1.555902in}}%
\pgfpathlineto{\pgfqpoint{7.850779in}{1.568847in}}%
\pgfpathlineto{\pgfqpoint{7.881646in}{1.601550in}}%
\pgfpathlineto{\pgfqpoint{7.912514in}{1.641774in}}%
\pgfpathlineto{\pgfqpoint{7.958814in}{1.710603in}}%
\pgfpathlineto{\pgfqpoint{8.005115in}{1.780307in}}%
\pgfpathlineto{\pgfqpoint{8.035983in}{1.821884in}}%
\pgfpathlineto{\pgfqpoint{8.066850in}{1.856151in}}%
\pgfpathlineto{\pgfqpoint{8.082284in}{1.869803in}}%
\pgfpathlineto{\pgfqpoint{8.097717in}{1.880831in}}%
\pgfpathlineto{\pgfqpoint{8.113151in}{1.889073in}}%
\pgfpathlineto{\pgfqpoint{8.128584in}{1.894420in}}%
\pgfpathlineto{\pgfqpoint{8.144018in}{1.896823in}}%
\pgfpathlineto{\pgfqpoint{8.159452in}{1.896283in}}%
\pgfpathlineto{\pgfqpoint{8.174885in}{1.892857in}}%
\pgfpathlineto{\pgfqpoint{8.190319in}{1.886650in}}%
\pgfpathlineto{\pgfqpoint{8.205753in}{1.877813in}}%
\pgfpathlineto{\pgfqpoint{8.221186in}{1.866531in}}%
\pgfpathlineto{\pgfqpoint{8.252053in}{1.837538in}}%
\pgfpathlineto{\pgfqpoint{8.282921in}{1.801666in}}%
\pgfpathlineto{\pgfqpoint{8.329222in}{1.739636in}}%
\pgfpathlineto{\pgfqpoint{8.437257in}{1.588295in}}%
\pgfpathlineto{\pgfqpoint{8.483558in}{1.530181in}}%
\pgfpathlineto{\pgfqpoint{8.514425in}{1.495178in}}%
\pgfpathlineto{\pgfqpoint{8.545293in}{1.463566in}}%
\pgfpathlineto{\pgfqpoint{8.576160in}{1.435738in}}%
\pgfpathlineto{\pgfqpoint{8.607027in}{1.412215in}}%
\pgfpathlineto{\pgfqpoint{8.637894in}{1.393665in}}%
\pgfpathlineto{\pgfqpoint{8.668762in}{1.380856in}}%
\pgfpathlineto{\pgfqpoint{8.684195in}{1.376851in}}%
\pgfpathlineto{\pgfqpoint{8.699629in}{1.374566in}}%
\pgfpathlineto{\pgfqpoint{8.715063in}{1.374077in}}%
\pgfpathlineto{\pgfqpoint{8.730496in}{1.375444in}}%
\pgfpathlineto{\pgfqpoint{8.745930in}{1.378707in}}%
\pgfpathlineto{\pgfqpoint{8.761363in}{1.383880in}}%
\pgfpathlineto{\pgfqpoint{8.776797in}{1.390951in}}%
\pgfpathlineto{\pgfqpoint{8.792231in}{1.399878in}}%
\pgfpathlineto{\pgfqpoint{8.823098in}{1.422985in}}%
\pgfpathlineto{\pgfqpoint{8.853965in}{1.452271in}}%
\pgfpathlineto{\pgfqpoint{8.884833in}{1.486379in}}%
\pgfpathlineto{\pgfqpoint{8.931133in}{1.542858in}}%
\pgfpathlineto{\pgfqpoint{8.992868in}{1.618417in}}%
\pgfpathlineto{\pgfqpoint{9.023735in}{1.652534in}}%
\pgfpathlineto{\pgfqpoint{9.054603in}{1.682551in}}%
\pgfpathlineto{\pgfqpoint{9.085470in}{1.707774in}}%
\pgfpathlineto{\pgfqpoint{9.116337in}{1.727991in}}%
\pgfpathlineto{\pgfqpoint{9.147204in}{1.743447in}}%
\pgfpathlineto{\pgfqpoint{9.178072in}{1.754768in}}%
\pgfpathlineto{\pgfqpoint{9.208939in}{1.762861in}}%
\pgfpathlineto{\pgfqpoint{9.255240in}{1.771310in}}%
\pgfpathlineto{\pgfqpoint{9.332408in}{1.784388in}}%
\pgfpathlineto{\pgfqpoint{9.378709in}{1.796081in}}%
\pgfpathlineto{\pgfqpoint{9.425010in}{1.812163in}}%
\pgfpathlineto{\pgfqpoint{9.471311in}{1.832161in}}%
\pgfpathlineto{\pgfqpoint{9.594780in}{1.889019in}}%
\pgfpathlineto{\pgfqpoint{9.625647in}{1.899797in}}%
\pgfpathlineto{\pgfqpoint{9.656514in}{1.907880in}}%
\pgfpathlineto{\pgfqpoint{9.687382in}{1.912947in}}%
\pgfpathlineto{\pgfqpoint{9.718249in}{1.914943in}}%
\pgfpathlineto{\pgfqpoint{9.733682in}{1.914846in}}%
\pgfpathlineto{\pgfqpoint{9.733682in}{1.914846in}}%
\pgfusepath{stroke}%
\end{pgfscope}%
\begin{pgfscope}%
\pgfpathrectangle{\pgfqpoint{5.698559in}{0.505056in}}{\pgfqpoint{4.227273in}{2.745455in}} %
\pgfusepath{clip}%
\pgfsetrectcap%
\pgfsetroundjoin%
\pgfsetlinewidth{0.501875pt}%
\definecolor{currentstroke}{rgb}{0.943137,0.767363,0.423549}%
\pgfsetstrokecolor{currentstroke}%
\pgfsetdash{}{0pt}%
\pgfpathmoveto{\pgfqpoint{5.890707in}{2.136554in}}%
\pgfpathlineto{\pgfqpoint{5.906141in}{2.132858in}}%
\pgfpathlineto{\pgfqpoint{5.921575in}{2.126538in}}%
\pgfpathlineto{\pgfqpoint{5.937008in}{2.117569in}}%
\pgfpathlineto{\pgfqpoint{5.952442in}{2.105970in}}%
\pgfpathlineto{\pgfqpoint{5.967875in}{2.091804in}}%
\pgfpathlineto{\pgfqpoint{5.983309in}{2.075176in}}%
\pgfpathlineto{\pgfqpoint{6.014176in}{2.035172in}}%
\pgfpathlineto{\pgfqpoint{6.045044in}{1.987638in}}%
\pgfpathlineto{\pgfqpoint{6.091345in}{1.907337in}}%
\pgfpathlineto{\pgfqpoint{6.153079in}{1.798902in}}%
\pgfpathlineto{\pgfqpoint{6.183946in}{1.751476in}}%
\pgfpathlineto{\pgfqpoint{6.199380in}{1.730918in}}%
\pgfpathlineto{\pgfqpoint{6.214814in}{1.712991in}}%
\pgfpathlineto{\pgfqpoint{6.230247in}{1.698065in}}%
\pgfpathlineto{\pgfqpoint{6.245681in}{1.686480in}}%
\pgfpathlineto{\pgfqpoint{6.261115in}{1.678542in}}%
\pgfpathlineto{\pgfqpoint{6.276548in}{1.674517in}}%
\pgfpathlineto{\pgfqpoint{6.291982in}{1.674625in}}%
\pgfpathlineto{\pgfqpoint{6.307415in}{1.679035in}}%
\pgfpathlineto{\pgfqpoint{6.322849in}{1.687859in}}%
\pgfpathlineto{\pgfqpoint{6.338283in}{1.701153in}}%
\pgfpathlineto{\pgfqpoint{6.353716in}{1.718907in}}%
\pgfpathlineto{\pgfqpoint{6.369150in}{1.741048in}}%
\pgfpathlineto{\pgfqpoint{6.384584in}{1.767437in}}%
\pgfpathlineto{\pgfqpoint{6.400017in}{1.797870in}}%
\pgfpathlineto{\pgfqpoint{6.430885in}{1.869729in}}%
\pgfpathlineto{\pgfqpoint{6.461752in}{1.953756in}}%
\pgfpathlineto{\pgfqpoint{6.508053in}{2.094322in}}%
\pgfpathlineto{\pgfqpoint{6.569787in}{2.285258in}}%
\pgfpathlineto{\pgfqpoint{6.600655in}{2.371717in}}%
\pgfpathlineto{\pgfqpoint{6.631522in}{2.446452in}}%
\pgfpathlineto{\pgfqpoint{6.646955in}{2.478306in}}%
\pgfpathlineto{\pgfqpoint{6.662389in}{2.505997in}}%
\pgfpathlineto{\pgfqpoint{6.677823in}{2.529237in}}%
\pgfpathlineto{\pgfqpoint{6.693256in}{2.547805in}}%
\pgfpathlineto{\pgfqpoint{6.708690in}{2.561547in}}%
\pgfpathlineto{\pgfqpoint{6.724124in}{2.570380in}}%
\pgfpathlineto{\pgfqpoint{6.739557in}{2.574286in}}%
\pgfpathlineto{\pgfqpoint{6.754991in}{2.573312in}}%
\pgfpathlineto{\pgfqpoint{6.770424in}{2.567569in}}%
\pgfpathlineto{\pgfqpoint{6.785858in}{2.557224in}}%
\pgfpathlineto{\pgfqpoint{6.801292in}{2.542494in}}%
\pgfpathlineto{\pgfqpoint{6.816725in}{2.523644in}}%
\pgfpathlineto{\pgfqpoint{6.832159in}{2.500978in}}%
\pgfpathlineto{\pgfqpoint{6.847593in}{2.474832in}}%
\pgfpathlineto{\pgfqpoint{6.878460in}{2.413568in}}%
\pgfpathlineto{\pgfqpoint{6.909327in}{2.342938in}}%
\pgfpathlineto{\pgfqpoint{6.955628in}{2.226431in}}%
\pgfpathlineto{\pgfqpoint{7.032796in}{2.029297in}}%
\pgfpathlineto{\pgfqpoint{7.063664in}{1.957075in}}%
\pgfpathlineto{\pgfqpoint{7.094531in}{1.891763in}}%
\pgfpathlineto{\pgfqpoint{7.125398in}{1.834843in}}%
\pgfpathlineto{\pgfqpoint{7.156265in}{1.787303in}}%
\pgfpathlineto{\pgfqpoint{7.187133in}{1.749610in}}%
\pgfpathlineto{\pgfqpoint{7.202566in}{1.734452in}}%
\pgfpathlineto{\pgfqpoint{7.218000in}{1.721682in}}%
\pgfpathlineto{\pgfqpoint{7.233434in}{1.711202in}}%
\pgfpathlineto{\pgfqpoint{7.248867in}{1.702883in}}%
\pgfpathlineto{\pgfqpoint{7.264301in}{1.696564in}}%
\pgfpathlineto{\pgfqpoint{7.295168in}{1.689133in}}%
\pgfpathlineto{\pgfqpoint{7.326035in}{1.687094in}}%
\pgfpathlineto{\pgfqpoint{7.372336in}{1.689516in}}%
\pgfpathlineto{\pgfqpoint{7.418637in}{1.691952in}}%
\pgfpathlineto{\pgfqpoint{7.449504in}{1.690318in}}%
\pgfpathlineto{\pgfqpoint{7.480372in}{1.684392in}}%
\pgfpathlineto{\pgfqpoint{7.511239in}{1.673415in}}%
\pgfpathlineto{\pgfqpoint{7.542106in}{1.657347in}}%
\pgfpathlineto{\pgfqpoint{7.572974in}{1.636894in}}%
\pgfpathlineto{\pgfqpoint{7.619274in}{1.601220in}}%
\pgfpathlineto{\pgfqpoint{7.665575in}{1.565929in}}%
\pgfpathlineto{\pgfqpoint{7.696443in}{1.546718in}}%
\pgfpathlineto{\pgfqpoint{7.711876in}{1.539332in}}%
\pgfpathlineto{\pgfqpoint{7.727310in}{1.533798in}}%
\pgfpathlineto{\pgfqpoint{7.742744in}{1.530360in}}%
\pgfpathlineto{\pgfqpoint{7.758177in}{1.529225in}}%
\pgfpathlineto{\pgfqpoint{7.773611in}{1.530558in}}%
\pgfpathlineto{\pgfqpoint{7.789044in}{1.534474in}}%
\pgfpathlineto{\pgfqpoint{7.804478in}{1.541038in}}%
\pgfpathlineto{\pgfqpoint{7.819912in}{1.550258in}}%
\pgfpathlineto{\pgfqpoint{7.835345in}{1.562089in}}%
\pgfpathlineto{\pgfqpoint{7.850779in}{1.576425in}}%
\pgfpathlineto{\pgfqpoint{7.881646in}{1.611931in}}%
\pgfpathlineto{\pgfqpoint{7.912514in}{1.654898in}}%
\pgfpathlineto{\pgfqpoint{7.958814in}{1.727586in}}%
\pgfpathlineto{\pgfqpoint{8.005115in}{1.800742in}}%
\pgfpathlineto{\pgfqpoint{8.035983in}{1.844360in}}%
\pgfpathlineto{\pgfqpoint{8.066850in}{1.880419in}}%
\pgfpathlineto{\pgfqpoint{8.082284in}{1.894856in}}%
\pgfpathlineto{\pgfqpoint{8.097717in}{1.906582in}}%
\pgfpathlineto{\pgfqpoint{8.113151in}{1.915421in}}%
\pgfpathlineto{\pgfqpoint{8.128584in}{1.921250in}}%
\pgfpathlineto{\pgfqpoint{8.144018in}{1.924004in}}%
\pgfpathlineto{\pgfqpoint{8.159452in}{1.923667in}}%
\pgfpathlineto{\pgfqpoint{8.174885in}{1.920280in}}%
\pgfpathlineto{\pgfqpoint{8.190319in}{1.913931in}}%
\pgfpathlineto{\pgfqpoint{8.205753in}{1.904755in}}%
\pgfpathlineto{\pgfqpoint{8.221186in}{1.892926in}}%
\pgfpathlineto{\pgfqpoint{8.236620in}{1.878655in}}%
\pgfpathlineto{\pgfqpoint{8.267487in}{1.843759in}}%
\pgfpathlineto{\pgfqpoint{8.298354in}{1.802186in}}%
\pgfpathlineto{\pgfqpoint{8.344655in}{1.732197in}}%
\pgfpathlineto{\pgfqpoint{8.437257in}{1.589233in}}%
\pgfpathlineto{\pgfqpoint{8.468124in}{1.546121in}}%
\pgfpathlineto{\pgfqpoint{8.498992in}{1.506865in}}%
\pgfpathlineto{\pgfqpoint{8.529859in}{1.472062in}}%
\pgfpathlineto{\pgfqpoint{8.560726in}{1.442194in}}%
\pgfpathlineto{\pgfqpoint{8.591593in}{1.417715in}}%
\pgfpathlineto{\pgfqpoint{8.622461in}{1.399099in}}%
\pgfpathlineto{\pgfqpoint{8.653328in}{1.386841in}}%
\pgfpathlineto{\pgfqpoint{8.668762in}{1.383247in}}%
\pgfpathlineto{\pgfqpoint{8.684195in}{1.381415in}}%
\pgfpathlineto{\pgfqpoint{8.699629in}{1.381389in}}%
\pgfpathlineto{\pgfqpoint{8.715063in}{1.383199in}}%
\pgfpathlineto{\pgfqpoint{8.730496in}{1.386861in}}%
\pgfpathlineto{\pgfqpoint{8.745930in}{1.392372in}}%
\pgfpathlineto{\pgfqpoint{8.761363in}{1.399707in}}%
\pgfpathlineto{\pgfqpoint{8.792231in}{1.419634in}}%
\pgfpathlineto{\pgfqpoint{8.823098in}{1.445963in}}%
\pgfpathlineto{\pgfqpoint{8.853965in}{1.477618in}}%
\pgfpathlineto{\pgfqpoint{8.900266in}{1.531905in}}%
\pgfpathlineto{\pgfqpoint{8.977434in}{1.626185in}}%
\pgfpathlineto{\pgfqpoint{9.008302in}{1.660341in}}%
\pgfpathlineto{\pgfqpoint{9.039169in}{1.690487in}}%
\pgfpathlineto{\pgfqpoint{9.070036in}{1.715848in}}%
\pgfpathlineto{\pgfqpoint{9.100903in}{1.736114in}}%
\pgfpathlineto{\pgfqpoint{9.131771in}{1.751423in}}%
\pgfpathlineto{\pgfqpoint{9.162638in}{1.762304in}}%
\pgfpathlineto{\pgfqpoint{9.193505in}{1.769575in}}%
\pgfpathlineto{\pgfqpoint{9.239806in}{1.775905in}}%
\pgfpathlineto{\pgfqpoint{9.363275in}{1.788529in}}%
\pgfpathlineto{\pgfqpoint{9.409576in}{1.797534in}}%
\pgfpathlineto{\pgfqpoint{9.455877in}{1.809560in}}%
\pgfpathlineto{\pgfqpoint{9.548479in}{1.838299in}}%
\pgfpathlineto{\pgfqpoint{9.610213in}{1.855803in}}%
\pgfpathlineto{\pgfqpoint{9.656514in}{1.865504in}}%
\pgfpathlineto{\pgfqpoint{9.702815in}{1.871161in}}%
\pgfpathlineto{\pgfqpoint{9.733682in}{1.872478in}}%
\pgfpathlineto{\pgfqpoint{9.733682in}{1.872478in}}%
\pgfusepath{stroke}%
\end{pgfscope}%
\begin{pgfscope}%
\pgfpathrectangle{\pgfqpoint{5.698559in}{0.505056in}}{\pgfqpoint{4.227273in}{2.745455in}} %
\pgfusepath{clip}%
\pgfsetrectcap%
\pgfsetroundjoin%
\pgfsetlinewidth{0.501875pt}%
\definecolor{currentstroke}{rgb}{1.000000,0.682749,0.366979}%
\pgfsetstrokecolor{currentstroke}%
\pgfsetdash{}{0pt}%
\pgfpathmoveto{\pgfqpoint{5.890707in}{2.132051in}}%
\pgfpathlineto{\pgfqpoint{5.906141in}{2.127736in}}%
\pgfpathlineto{\pgfqpoint{5.921575in}{2.120711in}}%
\pgfpathlineto{\pgfqpoint{5.937008in}{2.110941in}}%
\pgfpathlineto{\pgfqpoint{5.952442in}{2.098438in}}%
\pgfpathlineto{\pgfqpoint{5.967875in}{2.083261in}}%
\pgfpathlineto{\pgfqpoint{5.983309in}{2.065517in}}%
\pgfpathlineto{\pgfqpoint{6.014176in}{2.023004in}}%
\pgfpathlineto{\pgfqpoint{6.045044in}{1.972707in}}%
\pgfpathlineto{\pgfqpoint{6.091345in}{1.888257in}}%
\pgfpathlineto{\pgfqpoint{6.137645in}{1.802534in}}%
\pgfpathlineto{\pgfqpoint{6.168513in}{1.750561in}}%
\pgfpathlineto{\pgfqpoint{6.199380in}{1.706883in}}%
\pgfpathlineto{\pgfqpoint{6.214814in}{1.689198in}}%
\pgfpathlineto{\pgfqpoint{6.230247in}{1.674785in}}%
\pgfpathlineto{\pgfqpoint{6.245681in}{1.663977in}}%
\pgfpathlineto{\pgfqpoint{6.261115in}{1.657063in}}%
\pgfpathlineto{\pgfqpoint{6.276548in}{1.654284in}}%
\pgfpathlineto{\pgfqpoint{6.291982in}{1.655826in}}%
\pgfpathlineto{\pgfqpoint{6.307415in}{1.661820in}}%
\pgfpathlineto{\pgfqpoint{6.322849in}{1.672332in}}%
\pgfpathlineto{\pgfqpoint{6.338283in}{1.687369in}}%
\pgfpathlineto{\pgfqpoint{6.353716in}{1.706871in}}%
\pgfpathlineto{\pgfqpoint{6.369150in}{1.730714in}}%
\pgfpathlineto{\pgfqpoint{6.384584in}{1.758711in}}%
\pgfpathlineto{\pgfqpoint{6.400017in}{1.790611in}}%
\pgfpathlineto{\pgfqpoint{6.430885in}{1.864836in}}%
\pgfpathlineto{\pgfqpoint{6.461752in}{1.950298in}}%
\pgfpathlineto{\pgfqpoint{6.508053in}{2.091158in}}%
\pgfpathlineto{\pgfqpoint{6.569787in}{2.279885in}}%
\pgfpathlineto{\pgfqpoint{6.600655in}{2.364875in}}%
\pgfpathlineto{\pgfqpoint{6.631522in}{2.438449in}}%
\pgfpathlineto{\pgfqpoint{6.646955in}{2.469958in}}%
\pgfpathlineto{\pgfqpoint{6.662389in}{2.497516in}}%
\pgfpathlineto{\pgfqpoint{6.677823in}{2.520867in}}%
\pgfpathlineto{\pgfqpoint{6.693256in}{2.539813in}}%
\pgfpathlineto{\pgfqpoint{6.708690in}{2.554213in}}%
\pgfpathlineto{\pgfqpoint{6.724124in}{2.563985in}}%
\pgfpathlineto{\pgfqpoint{6.739557in}{2.569105in}}%
\pgfpathlineto{\pgfqpoint{6.754991in}{2.569603in}}%
\pgfpathlineto{\pgfqpoint{6.770424in}{2.565561in}}%
\pgfpathlineto{\pgfqpoint{6.785858in}{2.557112in}}%
\pgfpathlineto{\pgfqpoint{6.801292in}{2.544432in}}%
\pgfpathlineto{\pgfqpoint{6.816725in}{2.527739in}}%
\pgfpathlineto{\pgfqpoint{6.832159in}{2.507284in}}%
\pgfpathlineto{\pgfqpoint{6.847593in}{2.483350in}}%
\pgfpathlineto{\pgfqpoint{6.878460in}{2.426299in}}%
\pgfpathlineto{\pgfqpoint{6.909327in}{2.359269in}}%
\pgfpathlineto{\pgfqpoint{6.955628in}{2.246288in}}%
\pgfpathlineto{\pgfqpoint{7.048230in}{2.010633in}}%
\pgfpathlineto{\pgfqpoint{7.079097in}{1.938589in}}%
\pgfpathlineto{\pgfqpoint{7.109964in}{1.873268in}}%
\pgfpathlineto{\pgfqpoint{7.140832in}{1.816154in}}%
\pgfpathlineto{\pgfqpoint{7.171699in}{1.768244in}}%
\pgfpathlineto{\pgfqpoint{7.202566in}{1.730000in}}%
\pgfpathlineto{\pgfqpoint{7.218000in}{1.714495in}}%
\pgfpathlineto{\pgfqpoint{7.233434in}{1.701323in}}%
\pgfpathlineto{\pgfqpoint{7.248867in}{1.690386in}}%
\pgfpathlineto{\pgfqpoint{7.264301in}{1.681549in}}%
\pgfpathlineto{\pgfqpoint{7.279734in}{1.674649in}}%
\pgfpathlineto{\pgfqpoint{7.310602in}{1.665865in}}%
\pgfpathlineto{\pgfqpoint{7.341469in}{1.662230in}}%
\pgfpathlineto{\pgfqpoint{7.387770in}{1.661955in}}%
\pgfpathlineto{\pgfqpoint{7.434071in}{1.661753in}}%
\pgfpathlineto{\pgfqpoint{7.464938in}{1.658743in}}%
\pgfpathlineto{\pgfqpoint{7.495805in}{1.652058in}}%
\pgfpathlineto{\pgfqpoint{7.526673in}{1.641220in}}%
\pgfpathlineto{\pgfqpoint{7.557540in}{1.626454in}}%
\pgfpathlineto{\pgfqpoint{7.603841in}{1.599136in}}%
\pgfpathlineto{\pgfqpoint{7.650142in}{1.570779in}}%
\pgfpathlineto{\pgfqpoint{7.681009in}{1.554929in}}%
\pgfpathlineto{\pgfqpoint{7.711876in}{1.544199in}}%
\pgfpathlineto{\pgfqpoint{7.727310in}{1.541415in}}%
\pgfpathlineto{\pgfqpoint{7.742744in}{1.540651in}}%
\pgfpathlineto{\pgfqpoint{7.758177in}{1.542091in}}%
\pgfpathlineto{\pgfqpoint{7.773611in}{1.545874in}}%
\pgfpathlineto{\pgfqpoint{7.789044in}{1.552093in}}%
\pgfpathlineto{\pgfqpoint{7.804478in}{1.560791in}}%
\pgfpathlineto{\pgfqpoint{7.819912in}{1.571956in}}%
\pgfpathlineto{\pgfqpoint{7.835345in}{1.585525in}}%
\pgfpathlineto{\pgfqpoint{7.866213in}{1.619344in}}%
\pgfpathlineto{\pgfqpoint{7.897080in}{1.660690in}}%
\pgfpathlineto{\pgfqpoint{7.943381in}{1.731705in}}%
\pgfpathlineto{\pgfqpoint{8.005115in}{1.827761in}}%
\pgfpathlineto{\pgfqpoint{8.035983in}{1.869427in}}%
\pgfpathlineto{\pgfqpoint{8.066850in}{1.903156in}}%
\pgfpathlineto{\pgfqpoint{8.082284in}{1.916359in}}%
\pgfpathlineto{\pgfqpoint{8.097717in}{1.926841in}}%
\pgfpathlineto{\pgfqpoint{8.113151in}{1.934454in}}%
\pgfpathlineto{\pgfqpoint{8.128584in}{1.939099in}}%
\pgfpathlineto{\pgfqpoint{8.144018in}{1.940731in}}%
\pgfpathlineto{\pgfqpoint{8.159452in}{1.939354in}}%
\pgfpathlineto{\pgfqpoint{8.174885in}{1.935021in}}%
\pgfpathlineto{\pgfqpoint{8.190319in}{1.927833in}}%
\pgfpathlineto{\pgfqpoint{8.205753in}{1.917929in}}%
\pgfpathlineto{\pgfqpoint{8.221186in}{1.905489in}}%
\pgfpathlineto{\pgfqpoint{8.252053in}{1.873861in}}%
\pgfpathlineto{\pgfqpoint{8.282921in}{1.834886in}}%
\pgfpathlineto{\pgfqpoint{8.313788in}{1.790700in}}%
\pgfpathlineto{\pgfqpoint{8.390956in}{1.671064in}}%
\pgfpathlineto{\pgfqpoint{8.437257in}{1.601626in}}%
\pgfpathlineto{\pgfqpoint{8.483558in}{1.538919in}}%
\pgfpathlineto{\pgfqpoint{8.514425in}{1.502063in}}%
\pgfpathlineto{\pgfqpoint{8.545293in}{1.469736in}}%
\pgfpathlineto{\pgfqpoint{8.576160in}{1.442335in}}%
\pgfpathlineto{\pgfqpoint{8.607027in}{1.420257in}}%
\pgfpathlineto{\pgfqpoint{8.637894in}{1.403938in}}%
\pgfpathlineto{\pgfqpoint{8.668762in}{1.393850in}}%
\pgfpathlineto{\pgfqpoint{8.684195in}{1.391290in}}%
\pgfpathlineto{\pgfqpoint{8.699629in}{1.390458in}}%
\pgfpathlineto{\pgfqpoint{8.715063in}{1.391396in}}%
\pgfpathlineto{\pgfqpoint{8.730496in}{1.394138in}}%
\pgfpathlineto{\pgfqpoint{8.745930in}{1.398702in}}%
\pgfpathlineto{\pgfqpoint{8.761363in}{1.405089in}}%
\pgfpathlineto{\pgfqpoint{8.776797in}{1.413278in}}%
\pgfpathlineto{\pgfqpoint{8.807664in}{1.434871in}}%
\pgfpathlineto{\pgfqpoint{8.838532in}{1.462853in}}%
\pgfpathlineto{\pgfqpoint{8.869399in}{1.496197in}}%
\pgfpathlineto{\pgfqpoint{8.915700in}{1.553152in}}%
\pgfpathlineto{\pgfqpoint{8.992868in}{1.651991in}}%
\pgfpathlineto{\pgfqpoint{9.023735in}{1.687699in}}%
\pgfpathlineto{\pgfqpoint{9.054603in}{1.718979in}}%
\pgfpathlineto{\pgfqpoint{9.085470in}{1.744849in}}%
\pgfpathlineto{\pgfqpoint{9.116337in}{1.764811in}}%
\pgfpathlineto{\pgfqpoint{9.147204in}{1.778867in}}%
\pgfpathlineto{\pgfqpoint{9.178072in}{1.787492in}}%
\pgfpathlineto{\pgfqpoint{9.208939in}{1.791556in}}%
\pgfpathlineto{\pgfqpoint{9.239806in}{1.792217in}}%
\pgfpathlineto{\pgfqpoint{9.301541in}{1.788600in}}%
\pgfpathlineto{\pgfqpoint{9.347842in}{1.786522in}}%
\pgfpathlineto{\pgfqpoint{9.378709in}{1.787311in}}%
\pgfpathlineto{\pgfqpoint{9.409576in}{1.790533in}}%
\pgfpathlineto{\pgfqpoint{9.440443in}{1.796401in}}%
\pgfpathlineto{\pgfqpoint{9.471311in}{1.804788in}}%
\pgfpathlineto{\pgfqpoint{9.517612in}{1.821098in}}%
\pgfpathlineto{\pgfqpoint{9.625647in}{1.863032in}}%
\pgfpathlineto{\pgfqpoint{9.656514in}{1.872173in}}%
\pgfpathlineto{\pgfqpoint{9.687382in}{1.878766in}}%
\pgfpathlineto{\pgfqpoint{9.718249in}{1.882419in}}%
\pgfpathlineto{\pgfqpoint{9.733682in}{1.883088in}}%
\pgfpathlineto{\pgfqpoint{9.733682in}{1.883088in}}%
\pgfusepath{stroke}%
\end{pgfscope}%
\begin{pgfscope}%
\pgfpathrectangle{\pgfqpoint{5.698559in}{0.505056in}}{\pgfqpoint{4.227273in}{2.745455in}} %
\pgfusepath{clip}%
\pgfsetrectcap%
\pgfsetroundjoin%
\pgfsetlinewidth{0.501875pt}%
\definecolor{currentstroke}{rgb}{1.000000,0.587785,0.309017}%
\pgfsetstrokecolor{currentstroke}%
\pgfsetdash{}{0pt}%
\pgfpathmoveto{\pgfqpoint{5.890707in}{2.133705in}}%
\pgfpathlineto{\pgfqpoint{5.906141in}{2.130003in}}%
\pgfpathlineto{\pgfqpoint{5.921575in}{2.123585in}}%
\pgfpathlineto{\pgfqpoint{5.937008in}{2.114418in}}%
\pgfpathlineto{\pgfqpoint{5.952442in}{2.102520in}}%
\pgfpathlineto{\pgfqpoint{5.967875in}{2.087959in}}%
\pgfpathlineto{\pgfqpoint{5.983309in}{2.070848in}}%
\pgfpathlineto{\pgfqpoint{6.014176in}{2.029673in}}%
\pgfpathlineto{\pgfqpoint{6.045044in}{1.980787in}}%
\pgfpathlineto{\pgfqpoint{6.091345in}{1.898328in}}%
\pgfpathlineto{\pgfqpoint{6.153079in}{1.787132in}}%
\pgfpathlineto{\pgfqpoint{6.183946in}{1.738546in}}%
\pgfpathlineto{\pgfqpoint{6.199380in}{1.717518in}}%
\pgfpathlineto{\pgfqpoint{6.214814in}{1.699221in}}%
\pgfpathlineto{\pgfqpoint{6.230247in}{1.684049in}}%
\pgfpathlineto{\pgfqpoint{6.245681in}{1.672364in}}%
\pgfpathlineto{\pgfqpoint{6.261115in}{1.664493in}}%
\pgfpathlineto{\pgfqpoint{6.276548in}{1.660718in}}%
\pgfpathlineto{\pgfqpoint{6.291982in}{1.661268in}}%
\pgfpathlineto{\pgfqpoint{6.307415in}{1.666314in}}%
\pgfpathlineto{\pgfqpoint{6.322849in}{1.675961in}}%
\pgfpathlineto{\pgfqpoint{6.338283in}{1.690245in}}%
\pgfpathlineto{\pgfqpoint{6.353716in}{1.709127in}}%
\pgfpathlineto{\pgfqpoint{6.369150in}{1.732497in}}%
\pgfpathlineto{\pgfqpoint{6.384584in}{1.760170in}}%
\pgfpathlineto{\pgfqpoint{6.400017in}{1.791888in}}%
\pgfpathlineto{\pgfqpoint{6.430885in}{1.866107in}}%
\pgfpathlineto{\pgfqpoint{6.461752in}{1.951877in}}%
\pgfpathlineto{\pgfqpoint{6.508053in}{2.093331in}}%
\pgfpathlineto{\pgfqpoint{6.569787in}{2.282424in}}%
\pgfpathlineto{\pgfqpoint{6.600655in}{2.367423in}}%
\pgfpathlineto{\pgfqpoint{6.631522in}{2.441014in}}%
\pgfpathlineto{\pgfqpoint{6.646955in}{2.472569in}}%
\pgfpathlineto{\pgfqpoint{6.662389in}{2.500209in}}%
\pgfpathlineto{\pgfqpoint{6.677823in}{2.523681in}}%
\pgfpathlineto{\pgfqpoint{6.693256in}{2.542785in}}%
\pgfpathlineto{\pgfqpoint{6.708690in}{2.557373in}}%
\pgfpathlineto{\pgfqpoint{6.724124in}{2.567352in}}%
\pgfpathlineto{\pgfqpoint{6.739557in}{2.572683in}}%
\pgfpathlineto{\pgfqpoint{6.754991in}{2.573381in}}%
\pgfpathlineto{\pgfqpoint{6.770424in}{2.569516in}}%
\pgfpathlineto{\pgfqpoint{6.785858in}{2.561205in}}%
\pgfpathlineto{\pgfqpoint{6.801292in}{2.548617in}}%
\pgfpathlineto{\pgfqpoint{6.816725in}{2.531964in}}%
\pgfpathlineto{\pgfqpoint{6.832159in}{2.511502in}}%
\pgfpathlineto{\pgfqpoint{6.847593in}{2.487520in}}%
\pgfpathlineto{\pgfqpoint{6.878460in}{2.430304in}}%
\pgfpathlineto{\pgfqpoint{6.909327in}{2.363141in}}%
\pgfpathlineto{\pgfqpoint{6.955628in}{2.250333in}}%
\pgfpathlineto{\pgfqpoint{7.048230in}{2.016603in}}%
\pgfpathlineto{\pgfqpoint{7.079097in}{1.945147in}}%
\pgfpathlineto{\pgfqpoint{7.109964in}{1.880027in}}%
\pgfpathlineto{\pgfqpoint{7.140832in}{1.822589in}}%
\pgfpathlineto{\pgfqpoint{7.171699in}{1.773820in}}%
\pgfpathlineto{\pgfqpoint{7.202566in}{1.734300in}}%
\pgfpathlineto{\pgfqpoint{7.218000in}{1.718065in}}%
\pgfpathlineto{\pgfqpoint{7.233434in}{1.704144in}}%
\pgfpathlineto{\pgfqpoint{7.248867in}{1.692467in}}%
\pgfpathlineto{\pgfqpoint{7.264301in}{1.682926in}}%
\pgfpathlineto{\pgfqpoint{7.279734in}{1.675381in}}%
\pgfpathlineto{\pgfqpoint{7.310602in}{1.665533in}}%
\pgfpathlineto{\pgfqpoint{7.341469in}{1.661151in}}%
\pgfpathlineto{\pgfqpoint{7.387770in}{1.660169in}}%
\pgfpathlineto{\pgfqpoint{7.434071in}{1.659384in}}%
\pgfpathlineto{\pgfqpoint{7.464938in}{1.655944in}}%
\pgfpathlineto{\pgfqpoint{7.495805in}{1.648863in}}%
\pgfpathlineto{\pgfqpoint{7.526673in}{1.637836in}}%
\pgfpathlineto{\pgfqpoint{7.557540in}{1.623302in}}%
\pgfpathlineto{\pgfqpoint{7.619274in}{1.588631in}}%
\pgfpathlineto{\pgfqpoint{7.650142in}{1.572025in}}%
\pgfpathlineto{\pgfqpoint{7.681009in}{1.558587in}}%
\pgfpathlineto{\pgfqpoint{7.711876in}{1.550287in}}%
\pgfpathlineto{\pgfqpoint{7.727310in}{1.548620in}}%
\pgfpathlineto{\pgfqpoint{7.742744in}{1.548863in}}%
\pgfpathlineto{\pgfqpoint{7.758177in}{1.551172in}}%
\pgfpathlineto{\pgfqpoint{7.773611in}{1.555671in}}%
\pgfpathlineto{\pgfqpoint{7.789044in}{1.562448in}}%
\pgfpathlineto{\pgfqpoint{7.804478in}{1.571550in}}%
\pgfpathlineto{\pgfqpoint{7.819912in}{1.582981in}}%
\pgfpathlineto{\pgfqpoint{7.835345in}{1.596696in}}%
\pgfpathlineto{\pgfqpoint{7.866213in}{1.630558in}}%
\pgfpathlineto{\pgfqpoint{7.897080in}{1.671783in}}%
\pgfpathlineto{\pgfqpoint{7.943381in}{1.742555in}}%
\pgfpathlineto{\pgfqpoint{7.989682in}{1.815115in}}%
\pgfpathlineto{\pgfqpoint{8.020549in}{1.858914in}}%
\pgfpathlineto{\pgfqpoint{8.051416in}{1.895337in}}%
\pgfpathlineto{\pgfqpoint{8.066850in}{1.909949in}}%
\pgfpathlineto{\pgfqpoint{8.082284in}{1.921815in}}%
\pgfpathlineto{\pgfqpoint{8.097717in}{1.930752in}}%
\pgfpathlineto{\pgfqpoint{8.113151in}{1.936640in}}%
\pgfpathlineto{\pgfqpoint{8.128584in}{1.939425in}}%
\pgfpathlineto{\pgfqpoint{8.144018in}{1.939117in}}%
\pgfpathlineto{\pgfqpoint{8.159452in}{1.935788in}}%
\pgfpathlineto{\pgfqpoint{8.174885in}{1.929563in}}%
\pgfpathlineto{\pgfqpoint{8.190319in}{1.920615in}}%
\pgfpathlineto{\pgfqpoint{8.205753in}{1.909158in}}%
\pgfpathlineto{\pgfqpoint{8.236620in}{1.879696in}}%
\pgfpathlineto{\pgfqpoint{8.267487in}{1.843282in}}%
\pgfpathlineto{\pgfqpoint{8.298354in}{1.802052in}}%
\pgfpathlineto{\pgfqpoint{8.360089in}{1.712661in}}%
\pgfpathlineto{\pgfqpoint{8.421823in}{1.623532in}}%
\pgfpathlineto{\pgfqpoint{8.468124in}{1.561763in}}%
\pgfpathlineto{\pgfqpoint{8.498992in}{1.524489in}}%
\pgfpathlineto{\pgfqpoint{8.529859in}{1.491168in}}%
\pgfpathlineto{\pgfqpoint{8.560726in}{1.462447in}}%
\pgfpathlineto{\pgfqpoint{8.591593in}{1.438896in}}%
\pgfpathlineto{\pgfqpoint{8.622461in}{1.420982in}}%
\pgfpathlineto{\pgfqpoint{8.653328in}{1.409064in}}%
\pgfpathlineto{\pgfqpoint{8.668762in}{1.405441in}}%
\pgfpathlineto{\pgfqpoint{8.684195in}{1.403411in}}%
\pgfpathlineto{\pgfqpoint{8.699629in}{1.402999in}}%
\pgfpathlineto{\pgfqpoint{8.715063in}{1.404222in}}%
\pgfpathlineto{\pgfqpoint{8.730496in}{1.407098in}}%
\pgfpathlineto{\pgfqpoint{8.745930in}{1.411636in}}%
\pgfpathlineto{\pgfqpoint{8.776797in}{1.425706in}}%
\pgfpathlineto{\pgfqpoint{8.807664in}{1.446341in}}%
\pgfpathlineto{\pgfqpoint{8.838532in}{1.473200in}}%
\pgfpathlineto{\pgfqpoint{8.869399in}{1.505592in}}%
\pgfpathlineto{\pgfqpoint{8.900266in}{1.542398in}}%
\pgfpathlineto{\pgfqpoint{9.023735in}{1.698299in}}%
\pgfpathlineto{\pgfqpoint{9.054603in}{1.729508in}}%
\pgfpathlineto{\pgfqpoint{9.085470in}{1.754477in}}%
\pgfpathlineto{\pgfqpoint{9.116337in}{1.772626in}}%
\pgfpathlineto{\pgfqpoint{9.147204in}{1.784074in}}%
\pgfpathlineto{\pgfqpoint{9.178072in}{1.789568in}}%
\pgfpathlineto{\pgfqpoint{9.208939in}{1.790329in}}%
\pgfpathlineto{\pgfqpoint{9.239806in}{1.787839in}}%
\pgfpathlineto{\pgfqpoint{9.347842in}{1.774559in}}%
\pgfpathlineto{\pgfqpoint{9.378709in}{1.774126in}}%
\pgfpathlineto{\pgfqpoint{9.409576in}{1.776442in}}%
\pgfpathlineto{\pgfqpoint{9.440443in}{1.781739in}}%
\pgfpathlineto{\pgfqpoint{9.471311in}{1.789985in}}%
\pgfpathlineto{\pgfqpoint{9.502178in}{1.800886in}}%
\pgfpathlineto{\pgfqpoint{9.548479in}{1.820971in}}%
\pgfpathlineto{\pgfqpoint{9.641081in}{1.863854in}}%
\pgfpathlineto{\pgfqpoint{9.671948in}{1.875474in}}%
\pgfpathlineto{\pgfqpoint{9.702815in}{1.884497in}}%
\pgfpathlineto{\pgfqpoint{9.733682in}{1.890565in}}%
\pgfpathlineto{\pgfqpoint{9.733682in}{1.890565in}}%
\pgfusepath{stroke}%
\end{pgfscope}%
\begin{pgfscope}%
\pgfpathrectangle{\pgfqpoint{5.698559in}{0.505056in}}{\pgfqpoint{4.227273in}{2.745455in}} %
\pgfusepath{clip}%
\pgfsetrectcap%
\pgfsetroundjoin%
\pgfsetlinewidth{0.501875pt}%
\definecolor{currentstroke}{rgb}{1.000000,0.473094,0.243914}%
\pgfsetstrokecolor{currentstroke}%
\pgfsetdash{}{0pt}%
\pgfpathmoveto{\pgfqpoint{5.890707in}{2.152416in}}%
\pgfpathlineto{\pgfqpoint{5.906141in}{2.149567in}}%
\pgfpathlineto{\pgfqpoint{5.921575in}{2.144059in}}%
\pgfpathlineto{\pgfqpoint{5.937008in}{2.135857in}}%
\pgfpathlineto{\pgfqpoint{5.952442in}{2.124964in}}%
\pgfpathlineto{\pgfqpoint{5.967875in}{2.111429in}}%
\pgfpathlineto{\pgfqpoint{5.983309in}{2.095347in}}%
\pgfpathlineto{\pgfqpoint{6.014176in}{2.056134in}}%
\pgfpathlineto{\pgfqpoint{6.045044in}{2.008936in}}%
\pgfpathlineto{\pgfqpoint{6.091345in}{1.928214in}}%
\pgfpathlineto{\pgfqpoint{6.153079in}{1.817682in}}%
\pgfpathlineto{\pgfqpoint{6.183946in}{1.768797in}}%
\pgfpathlineto{\pgfqpoint{6.199380in}{1.747479in}}%
\pgfpathlineto{\pgfqpoint{6.214814in}{1.728801in}}%
\pgfpathlineto{\pgfqpoint{6.230247in}{1.713156in}}%
\pgfpathlineto{\pgfqpoint{6.245681in}{1.700902in}}%
\pgfpathlineto{\pgfqpoint{6.261115in}{1.692362in}}%
\pgfpathlineto{\pgfqpoint{6.276548in}{1.687811in}}%
\pgfpathlineto{\pgfqpoint{6.291982in}{1.687474in}}%
\pgfpathlineto{\pgfqpoint{6.307415in}{1.691519in}}%
\pgfpathlineto{\pgfqpoint{6.322849in}{1.700050in}}%
\pgfpathlineto{\pgfqpoint{6.338283in}{1.713105in}}%
\pgfpathlineto{\pgfqpoint{6.353716in}{1.730656in}}%
\pgfpathlineto{\pgfqpoint{6.369150in}{1.752601in}}%
\pgfpathlineto{\pgfqpoint{6.384584in}{1.778773in}}%
\pgfpathlineto{\pgfqpoint{6.400017in}{1.808932in}}%
\pgfpathlineto{\pgfqpoint{6.430885in}{1.879943in}}%
\pgfpathlineto{\pgfqpoint{6.461752in}{1.962523in}}%
\pgfpathlineto{\pgfqpoint{6.508053in}{2.099474in}}%
\pgfpathlineto{\pgfqpoint{6.569787in}{2.283087in}}%
\pgfpathlineto{\pgfqpoint{6.600655in}{2.365331in}}%
\pgfpathlineto{\pgfqpoint{6.631522in}{2.435983in}}%
\pgfpathlineto{\pgfqpoint{6.646955in}{2.465975in}}%
\pgfpathlineto{\pgfqpoint{6.662389in}{2.491994in}}%
\pgfpathlineto{\pgfqpoint{6.677823in}{2.513798in}}%
\pgfpathlineto{\pgfqpoint{6.693256in}{2.531206in}}%
\pgfpathlineto{\pgfqpoint{6.708690in}{2.544100in}}%
\pgfpathlineto{\pgfqpoint{6.724124in}{2.552420in}}%
\pgfpathlineto{\pgfqpoint{6.739557in}{2.556168in}}%
\pgfpathlineto{\pgfqpoint{6.754991in}{2.555400in}}%
\pgfpathlineto{\pgfqpoint{6.770424in}{2.550225in}}%
\pgfpathlineto{\pgfqpoint{6.785858in}{2.540800in}}%
\pgfpathlineto{\pgfqpoint{6.801292in}{2.527328in}}%
\pgfpathlineto{\pgfqpoint{6.816725in}{2.510044in}}%
\pgfpathlineto{\pgfqpoint{6.832159in}{2.489221in}}%
\pgfpathlineto{\pgfqpoint{6.847593in}{2.465153in}}%
\pgfpathlineto{\pgfqpoint{6.878460in}{2.408560in}}%
\pgfpathlineto{\pgfqpoint{6.909327in}{2.342930in}}%
\pgfpathlineto{\pgfqpoint{6.955628in}{2.233507in}}%
\pgfpathlineto{\pgfqpoint{7.048230in}{2.006720in}}%
\pgfpathlineto{\pgfqpoint{7.079097in}{1.936977in}}%
\pgfpathlineto{\pgfqpoint{7.109964in}{1.873258in}}%
\pgfpathlineto{\pgfqpoint{7.140832in}{1.817014in}}%
\pgfpathlineto{\pgfqpoint{7.171699in}{1.769364in}}%
\pgfpathlineto{\pgfqpoint{7.202566in}{1.731019in}}%
\pgfpathlineto{\pgfqpoint{7.218000in}{1.715427in}}%
\pgfpathlineto{\pgfqpoint{7.233434in}{1.702205in}}%
\pgfpathlineto{\pgfqpoint{7.248867in}{1.691296in}}%
\pgfpathlineto{\pgfqpoint{7.264301in}{1.682604in}}%
\pgfpathlineto{\pgfqpoint{7.279734in}{1.676000in}}%
\pgfpathlineto{\pgfqpoint{7.295168in}{1.671318in}}%
\pgfpathlineto{\pgfqpoint{7.326035in}{1.666903in}}%
\pgfpathlineto{\pgfqpoint{7.356903in}{1.667458in}}%
\pgfpathlineto{\pgfqpoint{7.464938in}{1.676735in}}%
\pgfpathlineto{\pgfqpoint{7.495805in}{1.673402in}}%
\pgfpathlineto{\pgfqpoint{7.526673in}{1.665238in}}%
\pgfpathlineto{\pgfqpoint{7.557540in}{1.652208in}}%
\pgfpathlineto{\pgfqpoint{7.588407in}{1.635061in}}%
\pgfpathlineto{\pgfqpoint{7.696443in}{1.568419in}}%
\pgfpathlineto{\pgfqpoint{7.727310in}{1.557181in}}%
\pgfpathlineto{\pgfqpoint{7.742744in}{1.554330in}}%
\pgfpathlineto{\pgfqpoint{7.758177in}{1.553634in}}%
\pgfpathlineto{\pgfqpoint{7.773611in}{1.555275in}}%
\pgfpathlineto{\pgfqpoint{7.789044in}{1.559386in}}%
\pgfpathlineto{\pgfqpoint{7.804478in}{1.566046in}}%
\pgfpathlineto{\pgfqpoint{7.819912in}{1.575275in}}%
\pgfpathlineto{\pgfqpoint{7.835345in}{1.587032in}}%
\pgfpathlineto{\pgfqpoint{7.850779in}{1.601216in}}%
\pgfpathlineto{\pgfqpoint{7.881646in}{1.636150in}}%
\pgfpathlineto{\pgfqpoint{7.912514in}{1.678090in}}%
\pgfpathlineto{\pgfqpoint{8.020549in}{1.836324in}}%
\pgfpathlineto{\pgfqpoint{8.051416in}{1.871024in}}%
\pgfpathlineto{\pgfqpoint{8.066850in}{1.884798in}}%
\pgfpathlineto{\pgfqpoint{8.082284in}{1.895857in}}%
\pgfpathlineto{\pgfqpoint{8.097717in}{1.904028in}}%
\pgfpathlineto{\pgfqpoint{8.113151in}{1.909198in}}%
\pgfpathlineto{\pgfqpoint{8.128584in}{1.911320in}}%
\pgfpathlineto{\pgfqpoint{8.144018in}{1.910408in}}%
\pgfpathlineto{\pgfqpoint{8.159452in}{1.906536in}}%
\pgfpathlineto{\pgfqpoint{8.174885in}{1.899832in}}%
\pgfpathlineto{\pgfqpoint{8.190319in}{1.890472in}}%
\pgfpathlineto{\pgfqpoint{8.205753in}{1.878673in}}%
\pgfpathlineto{\pgfqpoint{8.236620in}{1.848775in}}%
\pgfpathlineto{\pgfqpoint{8.267487in}{1.812358in}}%
\pgfpathlineto{\pgfqpoint{8.313788in}{1.750551in}}%
\pgfpathlineto{\pgfqpoint{8.390956in}{1.644821in}}%
\pgfpathlineto{\pgfqpoint{8.437257in}{1.587126in}}%
\pgfpathlineto{\pgfqpoint{8.468124in}{1.552506in}}%
\pgfpathlineto{\pgfqpoint{8.498992in}{1.521274in}}%
\pgfpathlineto{\pgfqpoint{8.529859in}{1.493550in}}%
\pgfpathlineto{\pgfqpoint{8.560726in}{1.469443in}}%
\pgfpathlineto{\pgfqpoint{8.591593in}{1.449136in}}%
\pgfpathlineto{\pgfqpoint{8.622461in}{1.432934in}}%
\pgfpathlineto{\pgfqpoint{8.653328in}{1.421278in}}%
\pgfpathlineto{\pgfqpoint{8.684195in}{1.414715in}}%
\pgfpathlineto{\pgfqpoint{8.715063in}{1.413840in}}%
\pgfpathlineto{\pgfqpoint{8.730496in}{1.415709in}}%
\pgfpathlineto{\pgfqpoint{8.745930in}{1.419194in}}%
\pgfpathlineto{\pgfqpoint{8.761363in}{1.424334in}}%
\pgfpathlineto{\pgfqpoint{8.792231in}{1.439645in}}%
\pgfpathlineto{\pgfqpoint{8.823098in}{1.461537in}}%
\pgfpathlineto{\pgfqpoint{8.853965in}{1.489460in}}%
\pgfpathlineto{\pgfqpoint{8.884833in}{1.522382in}}%
\pgfpathlineto{\pgfqpoint{8.931133in}{1.577752in}}%
\pgfpathlineto{\pgfqpoint{8.992868in}{1.652445in}}%
\pgfpathlineto{\pgfqpoint{9.023735in}{1.685567in}}%
\pgfpathlineto{\pgfqpoint{9.054603in}{1.713573in}}%
\pgfpathlineto{\pgfqpoint{9.085470in}{1.735360in}}%
\pgfpathlineto{\pgfqpoint{9.116337in}{1.750476in}}%
\pgfpathlineto{\pgfqpoint{9.147204in}{1.759153in}}%
\pgfpathlineto{\pgfqpoint{9.178072in}{1.762240in}}%
\pgfpathlineto{\pgfqpoint{9.208939in}{1.761065in}}%
\pgfpathlineto{\pgfqpoint{9.255240in}{1.754845in}}%
\pgfpathlineto{\pgfqpoint{9.316974in}{1.746610in}}%
\pgfpathlineto{\pgfqpoint{9.347842in}{1.745365in}}%
\pgfpathlineto{\pgfqpoint{9.378709in}{1.747032in}}%
\pgfpathlineto{\pgfqpoint{9.409576in}{1.751842in}}%
\pgfpathlineto{\pgfqpoint{9.440443in}{1.759657in}}%
\pgfpathlineto{\pgfqpoint{9.471311in}{1.770047in}}%
\pgfpathlineto{\pgfqpoint{9.517612in}{1.789043in}}%
\pgfpathlineto{\pgfqpoint{9.625647in}{1.836177in}}%
\pgfpathlineto{\pgfqpoint{9.671948in}{1.852277in}}%
\pgfpathlineto{\pgfqpoint{9.718249in}{1.863976in}}%
\pgfpathlineto{\pgfqpoint{9.733682in}{1.866832in}}%
\pgfpathlineto{\pgfqpoint{9.733682in}{1.866832in}}%
\pgfusepath{stroke}%
\end{pgfscope}%
\begin{pgfscope}%
\pgfpathrectangle{\pgfqpoint{5.698559in}{0.505056in}}{\pgfqpoint{4.227273in}{2.745455in}} %
\pgfusepath{clip}%
\pgfsetrectcap%
\pgfsetroundjoin%
\pgfsetlinewidth{0.501875pt}%
\definecolor{currentstroke}{rgb}{1.000000,0.361242,0.183750}%
\pgfsetstrokecolor{currentstroke}%
\pgfsetdash{}{0pt}%
\pgfpathmoveto{\pgfqpoint{5.890707in}{2.145381in}}%
\pgfpathlineto{\pgfqpoint{5.906141in}{2.142353in}}%
\pgfpathlineto{\pgfqpoint{5.921575in}{2.136904in}}%
\pgfpathlineto{\pgfqpoint{5.937008in}{2.129000in}}%
\pgfpathlineto{\pgfqpoint{5.952442in}{2.118647in}}%
\pgfpathlineto{\pgfqpoint{5.967875in}{2.105888in}}%
\pgfpathlineto{\pgfqpoint{5.983309in}{2.090807in}}%
\pgfpathlineto{\pgfqpoint{6.014176in}{2.054223in}}%
\pgfpathlineto{\pgfqpoint{6.045044in}{2.010364in}}%
\pgfpathlineto{\pgfqpoint{6.091345in}{1.935570in}}%
\pgfpathlineto{\pgfqpoint{6.153079in}{1.833425in}}%
\pgfpathlineto{\pgfqpoint{6.183946in}{1.788350in}}%
\pgfpathlineto{\pgfqpoint{6.214814in}{1.751557in}}%
\pgfpathlineto{\pgfqpoint{6.230247in}{1.737214in}}%
\pgfpathlineto{\pgfqpoint{6.245681in}{1.726030in}}%
\pgfpathlineto{\pgfqpoint{6.261115in}{1.718312in}}%
\pgfpathlineto{\pgfqpoint{6.276548in}{1.714322in}}%
\pgfpathlineto{\pgfqpoint{6.291982in}{1.714280in}}%
\pgfpathlineto{\pgfqpoint{6.307415in}{1.718352in}}%
\pgfpathlineto{\pgfqpoint{6.322849in}{1.726652in}}%
\pgfpathlineto{\pgfqpoint{6.338283in}{1.739230in}}%
\pgfpathlineto{\pgfqpoint{6.353716in}{1.756074in}}%
\pgfpathlineto{\pgfqpoint{6.369150in}{1.777109in}}%
\pgfpathlineto{\pgfqpoint{6.384584in}{1.802192in}}%
\pgfpathlineto{\pgfqpoint{6.400017in}{1.831113in}}%
\pgfpathlineto{\pgfqpoint{6.430885in}{1.899321in}}%
\pgfpathlineto{\pgfqpoint{6.461752in}{1.978849in}}%
\pgfpathlineto{\pgfqpoint{6.508053in}{2.111151in}}%
\pgfpathlineto{\pgfqpoint{6.569787in}{2.288950in}}%
\pgfpathlineto{\pgfqpoint{6.600655in}{2.368532in}}%
\pgfpathlineto{\pgfqpoint{6.631522in}{2.436712in}}%
\pgfpathlineto{\pgfqpoint{6.646955in}{2.465558in}}%
\pgfpathlineto{\pgfqpoint{6.662389in}{2.490505in}}%
\pgfpathlineto{\pgfqpoint{6.677823in}{2.511326in}}%
\pgfpathlineto{\pgfqpoint{6.693256in}{2.527858in}}%
\pgfpathlineto{\pgfqpoint{6.708690in}{2.540005in}}%
\pgfpathlineto{\pgfqpoint{6.724124in}{2.547728in}}%
\pgfpathlineto{\pgfqpoint{6.739557in}{2.551050in}}%
\pgfpathlineto{\pgfqpoint{6.754991in}{2.550048in}}%
\pgfpathlineto{\pgfqpoint{6.770424in}{2.544849in}}%
\pgfpathlineto{\pgfqpoint{6.785858in}{2.535622in}}%
\pgfpathlineto{\pgfqpoint{6.801292in}{2.522577in}}%
\pgfpathlineto{\pgfqpoint{6.816725in}{2.505953in}}%
\pgfpathlineto{\pgfqpoint{6.832159in}{2.486014in}}%
\pgfpathlineto{\pgfqpoint{6.863026in}{2.437339in}}%
\pgfpathlineto{\pgfqpoint{6.893894in}{2.378935in}}%
\pgfpathlineto{\pgfqpoint{6.924761in}{2.313244in}}%
\pgfpathlineto{\pgfqpoint{6.971062in}{2.206232in}}%
\pgfpathlineto{\pgfqpoint{7.063664in}{1.988538in}}%
\pgfpathlineto{\pgfqpoint{7.094531in}{1.922015in}}%
\pgfpathlineto{\pgfqpoint{7.125398in}{1.861339in}}%
\pgfpathlineto{\pgfqpoint{7.156265in}{1.807891in}}%
\pgfpathlineto{\pgfqpoint{7.187133in}{1.762739in}}%
\pgfpathlineto{\pgfqpoint{7.218000in}{1.726539in}}%
\pgfpathlineto{\pgfqpoint{7.233434in}{1.711869in}}%
\pgfpathlineto{\pgfqpoint{7.248867in}{1.699453in}}%
\pgfpathlineto{\pgfqpoint{7.264301in}{1.689225in}}%
\pgfpathlineto{\pgfqpoint{7.279734in}{1.681079in}}%
\pgfpathlineto{\pgfqpoint{7.295168in}{1.674876in}}%
\pgfpathlineto{\pgfqpoint{7.326035in}{1.667576in}}%
\pgfpathlineto{\pgfqpoint{7.356903in}{1.665606in}}%
\pgfpathlineto{\pgfqpoint{7.403204in}{1.668125in}}%
\pgfpathlineto{\pgfqpoint{7.449504in}{1.670793in}}%
\pgfpathlineto{\pgfqpoint{7.480372in}{1.669543in}}%
\pgfpathlineto{\pgfqpoint{7.511239in}{1.664391in}}%
\pgfpathlineto{\pgfqpoint{7.542106in}{1.654825in}}%
\pgfpathlineto{\pgfqpoint{7.572974in}{1.641100in}}%
\pgfpathlineto{\pgfqpoint{7.619274in}{1.615083in}}%
\pgfpathlineto{\pgfqpoint{7.665575in}{1.588019in}}%
\pgfpathlineto{\pgfqpoint{7.696443in}{1.573148in}}%
\pgfpathlineto{\pgfqpoint{7.727310in}{1.563487in}}%
\pgfpathlineto{\pgfqpoint{7.742744in}{1.561242in}}%
\pgfpathlineto{\pgfqpoint{7.758177in}{1.560995in}}%
\pgfpathlineto{\pgfqpoint{7.773611in}{1.562901in}}%
\pgfpathlineto{\pgfqpoint{7.789044in}{1.567067in}}%
\pgfpathlineto{\pgfqpoint{7.804478in}{1.573548in}}%
\pgfpathlineto{\pgfqpoint{7.819912in}{1.582344in}}%
\pgfpathlineto{\pgfqpoint{7.835345in}{1.593400in}}%
\pgfpathlineto{\pgfqpoint{7.866213in}{1.621798in}}%
\pgfpathlineto{\pgfqpoint{7.897080in}{1.657227in}}%
\pgfpathlineto{\pgfqpoint{7.943381in}{1.718494in}}%
\pgfpathlineto{\pgfqpoint{7.989682in}{1.780857in}}%
\pgfpathlineto{\pgfqpoint{8.020549in}{1.817960in}}%
\pgfpathlineto{\pgfqpoint{8.051416in}{1.848236in}}%
\pgfpathlineto{\pgfqpoint{8.066850in}{1.860102in}}%
\pgfpathlineto{\pgfqpoint{8.082284in}{1.869494in}}%
\pgfpathlineto{\pgfqpoint{8.097717in}{1.876255in}}%
\pgfpathlineto{\pgfqpoint{8.113151in}{1.880282in}}%
\pgfpathlineto{\pgfqpoint{8.128584in}{1.881529in}}%
\pgfpathlineto{\pgfqpoint{8.144018in}{1.880003in}}%
\pgfpathlineto{\pgfqpoint{8.159452in}{1.875766in}}%
\pgfpathlineto{\pgfqpoint{8.174885in}{1.868928in}}%
\pgfpathlineto{\pgfqpoint{8.190319in}{1.859646in}}%
\pgfpathlineto{\pgfqpoint{8.205753in}{1.848116in}}%
\pgfpathlineto{\pgfqpoint{8.236620in}{1.819255in}}%
\pgfpathlineto{\pgfqpoint{8.267487in}{1.784444in}}%
\pgfpathlineto{\pgfqpoint{8.313788in}{1.725987in}}%
\pgfpathlineto{\pgfqpoint{8.375523in}{1.646991in}}%
\pgfpathlineto{\pgfqpoint{8.421823in}{1.593189in}}%
\pgfpathlineto{\pgfqpoint{8.452691in}{1.561225in}}%
\pgfpathlineto{\pgfqpoint{8.483558in}{1.532525in}}%
\pgfpathlineto{\pgfqpoint{8.514425in}{1.506922in}}%
\pgfpathlineto{\pgfqpoint{8.545293in}{1.484183in}}%
\pgfpathlineto{\pgfqpoint{8.576160in}{1.464167in}}%
\pgfpathlineto{\pgfqpoint{8.607027in}{1.446955in}}%
\pgfpathlineto{\pgfqpoint{8.637894in}{1.432901in}}%
\pgfpathlineto{\pgfqpoint{8.668762in}{1.422633in}}%
\pgfpathlineto{\pgfqpoint{8.699629in}{1.416962in}}%
\pgfpathlineto{\pgfqpoint{8.730496in}{1.416746in}}%
\pgfpathlineto{\pgfqpoint{8.745930in}{1.418922in}}%
\pgfpathlineto{\pgfqpoint{8.761363in}{1.422720in}}%
\pgfpathlineto{\pgfqpoint{8.776797in}{1.428181in}}%
\pgfpathlineto{\pgfqpoint{8.807664in}{1.444134in}}%
\pgfpathlineto{\pgfqpoint{8.838532in}{1.466528in}}%
\pgfpathlineto{\pgfqpoint{8.869399in}{1.494572in}}%
\pgfpathlineto{\pgfqpoint{8.900266in}{1.526966in}}%
\pgfpathlineto{\pgfqpoint{9.023735in}{1.663464in}}%
\pgfpathlineto{\pgfqpoint{9.054603in}{1.690242in}}%
\pgfpathlineto{\pgfqpoint{9.085470in}{1.711564in}}%
\pgfpathlineto{\pgfqpoint{9.116337in}{1.727107in}}%
\pgfpathlineto{\pgfqpoint{9.147204in}{1.737148in}}%
\pgfpathlineto{\pgfqpoint{9.178072in}{1.742491in}}%
\pgfpathlineto{\pgfqpoint{9.208939in}{1.744333in}}%
\pgfpathlineto{\pgfqpoint{9.270673in}{1.743251in}}%
\pgfpathlineto{\pgfqpoint{9.316974in}{1.743669in}}%
\pgfpathlineto{\pgfqpoint{9.347842in}{1.746484in}}%
\pgfpathlineto{\pgfqpoint{9.378709in}{1.751961in}}%
\pgfpathlineto{\pgfqpoint{9.409576in}{1.760189in}}%
\pgfpathlineto{\pgfqpoint{9.455877in}{1.777027in}}%
\pgfpathlineto{\pgfqpoint{9.517612in}{1.804544in}}%
\pgfpathlineto{\pgfqpoint{9.579346in}{1.832221in}}%
\pgfpathlineto{\pgfqpoint{9.625647in}{1.849917in}}%
\pgfpathlineto{\pgfqpoint{9.671948in}{1.863544in}}%
\pgfpathlineto{\pgfqpoint{9.718249in}{1.872752in}}%
\pgfpathlineto{\pgfqpoint{9.733682in}{1.874883in}}%
\pgfpathlineto{\pgfqpoint{9.733682in}{1.874883in}}%
\pgfusepath{stroke}%
\end{pgfscope}%
\begin{pgfscope}%
\pgfpathrectangle{\pgfqpoint{5.698559in}{0.505056in}}{\pgfqpoint{4.227273in}{2.745455in}} %
\pgfusepath{clip}%
\pgfsetrectcap%
\pgfsetroundjoin%
\pgfsetlinewidth{0.501875pt}%
\definecolor{currentstroke}{rgb}{1.000000,0.243914,0.122888}%
\pgfsetstrokecolor{currentstroke}%
\pgfsetdash{}{0pt}%
\pgfpathmoveto{\pgfqpoint{5.890707in}{2.149128in}}%
\pgfpathlineto{\pgfqpoint{5.906141in}{2.147569in}}%
\pgfpathlineto{\pgfqpoint{5.921575in}{2.143458in}}%
\pgfpathlineto{\pgfqpoint{5.937008in}{2.136734in}}%
\pgfpathlineto{\pgfqpoint{5.952442in}{2.127377in}}%
\pgfpathlineto{\pgfqpoint{5.967875in}{2.115414in}}%
\pgfpathlineto{\pgfqpoint{5.983309in}{2.100915in}}%
\pgfpathlineto{\pgfqpoint{6.014176in}{2.064825in}}%
\pgfpathlineto{\pgfqpoint{6.045044in}{2.020585in}}%
\pgfpathlineto{\pgfqpoint{6.075911in}{1.970341in}}%
\pgfpathlineto{\pgfqpoint{6.168513in}{1.812669in}}%
\pgfpathlineto{\pgfqpoint{6.199380in}{1.768889in}}%
\pgfpathlineto{\pgfqpoint{6.214814in}{1.750539in}}%
\pgfpathlineto{\pgfqpoint{6.230247in}{1.735064in}}%
\pgfpathlineto{\pgfqpoint{6.245681in}{1.722817in}}%
\pgfpathlineto{\pgfqpoint{6.261115in}{1.714112in}}%
\pgfpathlineto{\pgfqpoint{6.276548in}{1.709224in}}%
\pgfpathlineto{\pgfqpoint{6.291982in}{1.708379in}}%
\pgfpathlineto{\pgfqpoint{6.307415in}{1.711751in}}%
\pgfpathlineto{\pgfqpoint{6.322849in}{1.719457in}}%
\pgfpathlineto{\pgfqpoint{6.338283in}{1.731553in}}%
\pgfpathlineto{\pgfqpoint{6.353716in}{1.748030in}}%
\pgfpathlineto{\pgfqpoint{6.369150in}{1.768814in}}%
\pgfpathlineto{\pgfqpoint{6.384584in}{1.793765in}}%
\pgfpathlineto{\pgfqpoint{6.400017in}{1.822675in}}%
\pgfpathlineto{\pgfqpoint{6.430885in}{1.891220in}}%
\pgfpathlineto{\pgfqpoint{6.461752in}{1.971551in}}%
\pgfpathlineto{\pgfqpoint{6.508053in}{2.105826in}}%
\pgfpathlineto{\pgfqpoint{6.569787in}{2.287185in}}%
\pgfpathlineto{\pgfqpoint{6.600655in}{2.368634in}}%
\pgfpathlineto{\pgfqpoint{6.631522in}{2.438532in}}%
\pgfpathlineto{\pgfqpoint{6.646955in}{2.468135in}}%
\pgfpathlineto{\pgfqpoint{6.662389in}{2.493749in}}%
\pgfpathlineto{\pgfqpoint{6.677823in}{2.515138in}}%
\pgfpathlineto{\pgfqpoint{6.693256in}{2.532131in}}%
\pgfpathlineto{\pgfqpoint{6.708690in}{2.544623in}}%
\pgfpathlineto{\pgfqpoint{6.724124in}{2.552576in}}%
\pgfpathlineto{\pgfqpoint{6.739557in}{2.556014in}}%
\pgfpathlineto{\pgfqpoint{6.754991in}{2.555019in}}%
\pgfpathlineto{\pgfqpoint{6.770424in}{2.549724in}}%
\pgfpathlineto{\pgfqpoint{6.785858in}{2.540310in}}%
\pgfpathlineto{\pgfqpoint{6.801292in}{2.526997in}}%
\pgfpathlineto{\pgfqpoint{6.816725in}{2.510038in}}%
\pgfpathlineto{\pgfqpoint{6.832159in}{2.489709in}}%
\pgfpathlineto{\pgfqpoint{6.863026in}{2.440140in}}%
\pgfpathlineto{\pgfqpoint{6.893894in}{2.380775in}}%
\pgfpathlineto{\pgfqpoint{6.924761in}{2.314136in}}%
\pgfpathlineto{\pgfqpoint{6.971062in}{2.205862in}}%
\pgfpathlineto{\pgfqpoint{7.063664in}{1.986761in}}%
\pgfpathlineto{\pgfqpoint{7.094531in}{1.920231in}}%
\pgfpathlineto{\pgfqpoint{7.125398in}{1.859826in}}%
\pgfpathlineto{\pgfqpoint{7.156265in}{1.806930in}}%
\pgfpathlineto{\pgfqpoint{7.187133in}{1.762589in}}%
\pgfpathlineto{\pgfqpoint{7.218000in}{1.727402in}}%
\pgfpathlineto{\pgfqpoint{7.233434in}{1.713286in}}%
\pgfpathlineto{\pgfqpoint{7.248867in}{1.701439in}}%
\pgfpathlineto{\pgfqpoint{7.264301in}{1.691779in}}%
\pgfpathlineto{\pgfqpoint{7.279734in}{1.684186in}}%
\pgfpathlineto{\pgfqpoint{7.295168in}{1.678504in}}%
\pgfpathlineto{\pgfqpoint{7.326035in}{1.672097in}}%
\pgfpathlineto{\pgfqpoint{7.356903in}{1.670741in}}%
\pgfpathlineto{\pgfqpoint{7.403204in}{1.673532in}}%
\pgfpathlineto{\pgfqpoint{7.449504in}{1.675682in}}%
\pgfpathlineto{\pgfqpoint{7.480372in}{1.673714in}}%
\pgfpathlineto{\pgfqpoint{7.511239in}{1.667620in}}%
\pgfpathlineto{\pgfqpoint{7.542106in}{1.656955in}}%
\pgfpathlineto{\pgfqpoint{7.572974in}{1.642029in}}%
\pgfpathlineto{\pgfqpoint{7.619274in}{1.614118in}}%
\pgfpathlineto{\pgfqpoint{7.665575in}{1.585169in}}%
\pgfpathlineto{\pgfqpoint{7.696443in}{1.569138in}}%
\pgfpathlineto{\pgfqpoint{7.727310in}{1.558495in}}%
\pgfpathlineto{\pgfqpoint{7.742744in}{1.555859in}}%
\pgfpathlineto{\pgfqpoint{7.758177in}{1.555310in}}%
\pgfpathlineto{\pgfqpoint{7.773611in}{1.557017in}}%
\pgfpathlineto{\pgfqpoint{7.789044in}{1.561102in}}%
\pgfpathlineto{\pgfqpoint{7.804478in}{1.567633in}}%
\pgfpathlineto{\pgfqpoint{7.819912in}{1.576624in}}%
\pgfpathlineto{\pgfqpoint{7.835345in}{1.588027in}}%
\pgfpathlineto{\pgfqpoint{7.850779in}{1.601734in}}%
\pgfpathlineto{\pgfqpoint{7.881646in}{1.635349in}}%
\pgfpathlineto{\pgfqpoint{7.912514in}{1.675501in}}%
\pgfpathlineto{\pgfqpoint{8.020549in}{1.825238in}}%
\pgfpathlineto{\pgfqpoint{8.051416in}{1.857448in}}%
\pgfpathlineto{\pgfqpoint{8.066850in}{1.870047in}}%
\pgfpathlineto{\pgfqpoint{8.082284in}{1.879989in}}%
\pgfpathlineto{\pgfqpoint{8.097717in}{1.887105in}}%
\pgfpathlineto{\pgfqpoint{8.113151in}{1.891285in}}%
\pgfpathlineto{\pgfqpoint{8.128584in}{1.892480in}}%
\pgfpathlineto{\pgfqpoint{8.144018in}{1.890702in}}%
\pgfpathlineto{\pgfqpoint{8.159452in}{1.886017in}}%
\pgfpathlineto{\pgfqpoint{8.174885in}{1.878550in}}%
\pgfpathlineto{\pgfqpoint{8.190319in}{1.868472in}}%
\pgfpathlineto{\pgfqpoint{8.205753in}{1.855996in}}%
\pgfpathlineto{\pgfqpoint{8.236620in}{1.824872in}}%
\pgfpathlineto{\pgfqpoint{8.267487in}{1.787436in}}%
\pgfpathlineto{\pgfqpoint{8.313788in}{1.724711in}}%
\pgfpathlineto{\pgfqpoint{8.375523in}{1.640153in}}%
\pgfpathlineto{\pgfqpoint{8.421823in}{1.582861in}}%
\pgfpathlineto{\pgfqpoint{8.452691in}{1.549151in}}%
\pgfpathlineto{\pgfqpoint{8.483558in}{1.519333in}}%
\pgfpathlineto{\pgfqpoint{8.514425in}{1.493370in}}%
\pgfpathlineto{\pgfqpoint{8.545293in}{1.471128in}}%
\pgfpathlineto{\pgfqpoint{8.576160in}{1.452518in}}%
\pgfpathlineto{\pgfqpoint{8.607027in}{1.437601in}}%
\pgfpathlineto{\pgfqpoint{8.637894in}{1.426642in}}%
\pgfpathlineto{\pgfqpoint{8.668762in}{1.420099in}}%
\pgfpathlineto{\pgfqpoint{8.699629in}{1.418559in}}%
\pgfpathlineto{\pgfqpoint{8.730496in}{1.422615in}}%
\pgfpathlineto{\pgfqpoint{8.761363in}{1.432731in}}%
\pgfpathlineto{\pgfqpoint{8.792231in}{1.449087in}}%
\pgfpathlineto{\pgfqpoint{8.823098in}{1.471470in}}%
\pgfpathlineto{\pgfqpoint{8.853965in}{1.499194in}}%
\pgfpathlineto{\pgfqpoint{8.884833in}{1.531097in}}%
\pgfpathlineto{\pgfqpoint{9.008302in}{1.666064in}}%
\pgfpathlineto{\pgfqpoint{9.039169in}{1.692634in}}%
\pgfpathlineto{\pgfqpoint{9.070036in}{1.713661in}}%
\pgfpathlineto{\pgfqpoint{9.100903in}{1.728711in}}%
\pgfpathlineto{\pgfqpoint{9.131771in}{1.737970in}}%
\pgfpathlineto{\pgfqpoint{9.162638in}{1.742198in}}%
\pgfpathlineto{\pgfqpoint{9.193505in}{1.742609in}}%
\pgfpathlineto{\pgfqpoint{9.255240in}{1.738041in}}%
\pgfpathlineto{\pgfqpoint{9.301541in}{1.735837in}}%
\pgfpathlineto{\pgfqpoint{9.332408in}{1.737193in}}%
\pgfpathlineto{\pgfqpoint{9.363275in}{1.741604in}}%
\pgfpathlineto{\pgfqpoint{9.394143in}{1.749271in}}%
\pgfpathlineto{\pgfqpoint{9.425010in}{1.759991in}}%
\pgfpathlineto{\pgfqpoint{9.471311in}{1.780584in}}%
\pgfpathlineto{\pgfqpoint{9.610213in}{1.849056in}}%
\pgfpathlineto{\pgfqpoint{9.656514in}{1.865834in}}%
\pgfpathlineto{\pgfqpoint{9.687382in}{1.874097in}}%
\pgfpathlineto{\pgfqpoint{9.718249in}{1.879947in}}%
\pgfpathlineto{\pgfqpoint{9.733682in}{1.881998in}}%
\pgfpathlineto{\pgfqpoint{9.733682in}{1.881998in}}%
\pgfusepath{stroke}%
\end{pgfscope}%
\begin{pgfscope}%
\pgfpathrectangle{\pgfqpoint{5.698559in}{0.505056in}}{\pgfqpoint{4.227273in}{2.745455in}} %
\pgfusepath{clip}%
\pgfsetrectcap%
\pgfsetroundjoin%
\pgfsetlinewidth{0.501875pt}%
\definecolor{currentstroke}{rgb}{1.000000,0.122888,0.061561}%
\pgfsetstrokecolor{currentstroke}%
\pgfsetdash{}{0pt}%
\pgfpathmoveto{\pgfqpoint{5.890707in}{2.141809in}}%
\pgfpathlineto{\pgfqpoint{5.906141in}{2.139937in}}%
\pgfpathlineto{\pgfqpoint{5.921575in}{2.135525in}}%
\pgfpathlineto{\pgfqpoint{5.937008in}{2.128503in}}%
\pgfpathlineto{\pgfqpoint{5.952442in}{2.118849in}}%
\pgfpathlineto{\pgfqpoint{5.967875in}{2.106582in}}%
\pgfpathlineto{\pgfqpoint{5.983309in}{2.091772in}}%
\pgfpathlineto{\pgfqpoint{6.014176in}{2.055043in}}%
\pgfpathlineto{\pgfqpoint{6.045044in}{2.010171in}}%
\pgfpathlineto{\pgfqpoint{6.075911in}{1.959375in}}%
\pgfpathlineto{\pgfqpoint{6.168513in}{1.801595in}}%
\pgfpathlineto{\pgfqpoint{6.199380in}{1.758621in}}%
\pgfpathlineto{\pgfqpoint{6.214814in}{1.740860in}}%
\pgfpathlineto{\pgfqpoint{6.230247in}{1.726098in}}%
\pgfpathlineto{\pgfqpoint{6.245681in}{1.714677in}}%
\pgfpathlineto{\pgfqpoint{6.261115in}{1.706904in}}%
\pgfpathlineto{\pgfqpoint{6.276548in}{1.703039in}}%
\pgfpathlineto{\pgfqpoint{6.291982in}{1.703290in}}%
\pgfpathlineto{\pgfqpoint{6.307415in}{1.707812in}}%
\pgfpathlineto{\pgfqpoint{6.322849in}{1.716701in}}%
\pgfpathlineto{\pgfqpoint{6.338283in}{1.729990in}}%
\pgfpathlineto{\pgfqpoint{6.353716in}{1.747648in}}%
\pgfpathlineto{\pgfqpoint{6.369150in}{1.769578in}}%
\pgfpathlineto{\pgfqpoint{6.384584in}{1.795619in}}%
\pgfpathlineto{\pgfqpoint{6.400017in}{1.825544in}}%
\pgfpathlineto{\pgfqpoint{6.430885in}{1.895830in}}%
\pgfpathlineto{\pgfqpoint{6.461752in}{1.977446in}}%
\pgfpathlineto{\pgfqpoint{6.508053in}{2.112735in}}%
\pgfpathlineto{\pgfqpoint{6.569787in}{2.294022in}}%
\pgfpathlineto{\pgfqpoint{6.600655in}{2.375074in}}%
\pgfpathlineto{\pgfqpoint{6.631522in}{2.444518in}}%
\pgfpathlineto{\pgfqpoint{6.646955in}{2.473913in}}%
\pgfpathlineto{\pgfqpoint{6.662389in}{2.499352in}}%
\pgfpathlineto{\pgfqpoint{6.677823in}{2.520607in}}%
\pgfpathlineto{\pgfqpoint{6.693256in}{2.537518in}}%
\pgfpathlineto{\pgfqpoint{6.708690in}{2.549986in}}%
\pgfpathlineto{\pgfqpoint{6.724124in}{2.557976in}}%
\pgfpathlineto{\pgfqpoint{6.739557in}{2.561512in}}%
\pgfpathlineto{\pgfqpoint{6.754991in}{2.560674in}}%
\pgfpathlineto{\pgfqpoint{6.770424in}{2.555592in}}%
\pgfpathlineto{\pgfqpoint{6.785858in}{2.546441in}}%
\pgfpathlineto{\pgfqpoint{6.801292in}{2.533432in}}%
\pgfpathlineto{\pgfqpoint{6.816725in}{2.516808in}}%
\pgfpathlineto{\pgfqpoint{6.832159in}{2.496836in}}%
\pgfpathlineto{\pgfqpoint{6.863026in}{2.448005in}}%
\pgfpathlineto{\pgfqpoint{6.893894in}{2.389335in}}%
\pgfpathlineto{\pgfqpoint{6.924761in}{2.323264in}}%
\pgfpathlineto{\pgfqpoint{6.971062in}{2.215450in}}%
\pgfpathlineto{\pgfqpoint{7.063664in}{1.995387in}}%
\pgfpathlineto{\pgfqpoint{7.094531in}{1.927992in}}%
\pgfpathlineto{\pgfqpoint{7.125398in}{1.866523in}}%
\pgfpathlineto{\pgfqpoint{7.156265in}{1.812438in}}%
\pgfpathlineto{\pgfqpoint{7.187133in}{1.766854in}}%
\pgfpathlineto{\pgfqpoint{7.218000in}{1.730436in}}%
\pgfpathlineto{\pgfqpoint{7.233434in}{1.715721in}}%
\pgfpathlineto{\pgfqpoint{7.248867in}{1.703293in}}%
\pgfpathlineto{\pgfqpoint{7.264301in}{1.693069in}}%
\pgfpathlineto{\pgfqpoint{7.279734in}{1.684929in}}%
\pgfpathlineto{\pgfqpoint{7.295168in}{1.678717in}}%
\pgfpathlineto{\pgfqpoint{7.326035in}{1.671289in}}%
\pgfpathlineto{\pgfqpoint{7.356903in}{1.668950in}}%
\pgfpathlineto{\pgfqpoint{7.403204in}{1.670321in}}%
\pgfpathlineto{\pgfqpoint{7.449504in}{1.671174in}}%
\pgfpathlineto{\pgfqpoint{7.480372in}{1.668485in}}%
\pgfpathlineto{\pgfqpoint{7.511239in}{1.661854in}}%
\pgfpathlineto{\pgfqpoint{7.542106in}{1.650891in}}%
\pgfpathlineto{\pgfqpoint{7.572974in}{1.635937in}}%
\pgfpathlineto{\pgfqpoint{7.619274in}{1.608485in}}%
\pgfpathlineto{\pgfqpoint{7.665575in}{1.580485in}}%
\pgfpathlineto{\pgfqpoint{7.696443in}{1.565265in}}%
\pgfpathlineto{\pgfqpoint{7.727310in}{1.555508in}}%
\pgfpathlineto{\pgfqpoint{7.742744in}{1.553332in}}%
\pgfpathlineto{\pgfqpoint{7.758177in}{1.553248in}}%
\pgfpathlineto{\pgfqpoint{7.773611in}{1.555424in}}%
\pgfpathlineto{\pgfqpoint{7.789044in}{1.559983in}}%
\pgfpathlineto{\pgfqpoint{7.804478in}{1.566994in}}%
\pgfpathlineto{\pgfqpoint{7.819912in}{1.576469in}}%
\pgfpathlineto{\pgfqpoint{7.835345in}{1.588363in}}%
\pgfpathlineto{\pgfqpoint{7.850779in}{1.602571in}}%
\pgfpathlineto{\pgfqpoint{7.881646in}{1.637212in}}%
\pgfpathlineto{\pgfqpoint{7.912514in}{1.678421in}}%
\pgfpathlineto{\pgfqpoint{8.020549in}{1.831557in}}%
\pgfpathlineto{\pgfqpoint{8.051416in}{1.864439in}}%
\pgfpathlineto{\pgfqpoint{8.066850in}{1.877298in}}%
\pgfpathlineto{\pgfqpoint{8.082284in}{1.887449in}}%
\pgfpathlineto{\pgfqpoint{8.097717in}{1.894725in}}%
\pgfpathlineto{\pgfqpoint{8.113151in}{1.899020in}}%
\pgfpathlineto{\pgfqpoint{8.128584in}{1.900290in}}%
\pgfpathlineto{\pgfqpoint{8.144018in}{1.898555in}}%
\pgfpathlineto{\pgfqpoint{8.159452in}{1.893893in}}%
\pgfpathlineto{\pgfqpoint{8.174885in}{1.886436in}}%
\pgfpathlineto{\pgfqpoint{8.190319in}{1.876366in}}%
\pgfpathlineto{\pgfqpoint{8.205753in}{1.863906in}}%
\pgfpathlineto{\pgfqpoint{8.236620in}{1.832870in}}%
\pgfpathlineto{\pgfqpoint{8.267487in}{1.795629in}}%
\pgfpathlineto{\pgfqpoint{8.313788in}{1.733380in}}%
\pgfpathlineto{\pgfqpoint{8.375523in}{1.649441in}}%
\pgfpathlineto{\pgfqpoint{8.421823in}{1.592122in}}%
\pgfpathlineto{\pgfqpoint{8.452691in}{1.557966in}}%
\pgfpathlineto{\pgfqpoint{8.483558in}{1.527292in}}%
\pgfpathlineto{\pgfqpoint{8.514425in}{1.500059in}}%
\pgfpathlineto{\pgfqpoint{8.545293in}{1.476177in}}%
\pgfpathlineto{\pgfqpoint{8.576160in}{1.455635in}}%
\pgfpathlineto{\pgfqpoint{8.607027in}{1.438594in}}%
\pgfpathlineto{\pgfqpoint{8.637894in}{1.425426in}}%
\pgfpathlineto{\pgfqpoint{8.668762in}{1.416696in}}%
\pgfpathlineto{\pgfqpoint{8.699629in}{1.413092in}}%
\pgfpathlineto{\pgfqpoint{8.730496in}{1.415312in}}%
\pgfpathlineto{\pgfqpoint{8.745930in}{1.418793in}}%
\pgfpathlineto{\pgfqpoint{8.761363in}{1.423928in}}%
\pgfpathlineto{\pgfqpoint{8.792231in}{1.439239in}}%
\pgfpathlineto{\pgfqpoint{8.823098in}{1.461152in}}%
\pgfpathlineto{\pgfqpoint{8.853965in}{1.489092in}}%
\pgfpathlineto{\pgfqpoint{8.884833in}{1.521985in}}%
\pgfpathlineto{\pgfqpoint{8.931133in}{1.577167in}}%
\pgfpathlineto{\pgfqpoint{8.992868in}{1.651453in}}%
\pgfpathlineto{\pgfqpoint{9.023735in}{1.684475in}}%
\pgfpathlineto{\pgfqpoint{9.054603in}{1.712597in}}%
\pgfpathlineto{\pgfqpoint{9.085470in}{1.734813in}}%
\pgfpathlineto{\pgfqpoint{9.116337in}{1.750731in}}%
\pgfpathlineto{\pgfqpoint{9.147204in}{1.760591in}}%
\pgfpathlineto{\pgfqpoint{9.178072in}{1.765201in}}%
\pgfpathlineto{\pgfqpoint{9.208939in}{1.765802in}}%
\pgfpathlineto{\pgfqpoint{9.255240in}{1.762497in}}%
\pgfpathlineto{\pgfqpoint{9.316974in}{1.758151in}}%
\pgfpathlineto{\pgfqpoint{9.347842in}{1.758737in}}%
\pgfpathlineto{\pgfqpoint{9.378709in}{1.762139in}}%
\pgfpathlineto{\pgfqpoint{9.409576in}{1.768586in}}%
\pgfpathlineto{\pgfqpoint{9.440443in}{1.777931in}}%
\pgfpathlineto{\pgfqpoint{9.486744in}{1.796320in}}%
\pgfpathlineto{\pgfqpoint{9.641081in}{1.864704in}}%
\pgfpathlineto{\pgfqpoint{9.671948in}{1.874451in}}%
\pgfpathlineto{\pgfqpoint{9.702815in}{1.881938in}}%
\pgfpathlineto{\pgfqpoint{9.733682in}{1.887131in}}%
\pgfpathlineto{\pgfqpoint{9.733682in}{1.887131in}}%
\pgfusepath{stroke}%
\end{pgfscope}%
\begin{pgfscope}%
\pgfpathrectangle{\pgfqpoint{5.698559in}{0.505056in}}{\pgfqpoint{4.227273in}{2.745455in}} %
\pgfusepath{clip}%
\pgfsetrectcap%
\pgfsetroundjoin%
\pgfsetlinewidth{0.501875pt}%
\definecolor{currentstroke}{rgb}{0.000000,0.000000,0.000000}%
\pgfsetstrokecolor{currentstroke}%
\pgfsetdash{}{0pt}%
\pgfpathmoveto{\pgfqpoint{5.890707in}{2.140406in}}%
\pgfpathlineto{\pgfqpoint{5.906141in}{2.138218in}}%
\pgfpathlineto{\pgfqpoint{5.921575in}{2.133437in}}%
\pgfpathlineto{\pgfqpoint{5.937008in}{2.125999in}}%
\pgfpathlineto{\pgfqpoint{5.952442in}{2.115888in}}%
\pgfpathlineto{\pgfqpoint{5.967875in}{2.103134in}}%
\pgfpathlineto{\pgfqpoint{5.983309in}{2.087816in}}%
\pgfpathlineto{\pgfqpoint{6.014176in}{2.050052in}}%
\pgfpathlineto{\pgfqpoint{6.045044in}{2.004183in}}%
\pgfpathlineto{\pgfqpoint{6.075911in}{1.952481in}}%
\pgfpathlineto{\pgfqpoint{6.168513in}{1.792590in}}%
\pgfpathlineto{\pgfqpoint{6.199380in}{1.749027in}}%
\pgfpathlineto{\pgfqpoint{6.214814in}{1.730979in}}%
\pgfpathlineto{\pgfqpoint{6.230247in}{1.715932in}}%
\pgfpathlineto{\pgfqpoint{6.245681in}{1.704231in}}%
\pgfpathlineto{\pgfqpoint{6.261115in}{1.696187in}}%
\pgfpathlineto{\pgfqpoint{6.276548in}{1.692063in}}%
\pgfpathlineto{\pgfqpoint{6.291982in}{1.692075in}}%
\pgfpathlineto{\pgfqpoint{6.307415in}{1.696384in}}%
\pgfpathlineto{\pgfqpoint{6.322849in}{1.705093in}}%
\pgfpathlineto{\pgfqpoint{6.338283in}{1.718243in}}%
\pgfpathlineto{\pgfqpoint{6.353716in}{1.735808in}}%
\pgfpathlineto{\pgfqpoint{6.369150in}{1.757697in}}%
\pgfpathlineto{\pgfqpoint{6.384584in}{1.783753in}}%
\pgfpathlineto{\pgfqpoint{6.400017in}{1.813750in}}%
\pgfpathlineto{\pgfqpoint{6.430885in}{1.884354in}}%
\pgfpathlineto{\pgfqpoint{6.461752in}{1.966507in}}%
\pgfpathlineto{\pgfqpoint{6.508053in}{2.102943in}}%
\pgfpathlineto{\pgfqpoint{6.569787in}{2.286221in}}%
\pgfpathlineto{\pgfqpoint{6.600655in}{2.368409in}}%
\pgfpathlineto{\pgfqpoint{6.631522in}{2.439050in}}%
\pgfpathlineto{\pgfqpoint{6.646955in}{2.469058in}}%
\pgfpathlineto{\pgfqpoint{6.662389in}{2.495111in}}%
\pgfpathlineto{\pgfqpoint{6.677823in}{2.516976in}}%
\pgfpathlineto{\pgfqpoint{6.693256in}{2.534482in}}%
\pgfpathlineto{\pgfqpoint{6.708690in}{2.547522in}}%
\pgfpathlineto{\pgfqpoint{6.724124in}{2.556048in}}%
\pgfpathlineto{\pgfqpoint{6.739557in}{2.560074in}}%
\pgfpathlineto{\pgfqpoint{6.754991in}{2.559667in}}%
\pgfpathlineto{\pgfqpoint{6.770424in}{2.554946in}}%
\pgfpathlineto{\pgfqpoint{6.785858in}{2.546075in}}%
\pgfpathlineto{\pgfqpoint{6.801292in}{2.533256in}}%
\pgfpathlineto{\pgfqpoint{6.816725in}{2.516727in}}%
\pgfpathlineto{\pgfqpoint{6.832159in}{2.496752in}}%
\pgfpathlineto{\pgfqpoint{6.847593in}{2.473615in}}%
\pgfpathlineto{\pgfqpoint{6.878460in}{2.419073in}}%
\pgfpathlineto{\pgfqpoint{6.909327in}{2.355602in}}%
\pgfpathlineto{\pgfqpoint{6.955628in}{2.249176in}}%
\pgfpathlineto{\pgfqpoint{7.079097in}{1.955137in}}%
\pgfpathlineto{\pgfqpoint{7.109964in}{1.890414in}}%
\pgfpathlineto{\pgfqpoint{7.140832in}{1.832610in}}%
\pgfpathlineto{\pgfqpoint{7.171699in}{1.782964in}}%
\pgfpathlineto{\pgfqpoint{7.202566in}{1.742311in}}%
\pgfpathlineto{\pgfqpoint{7.218000in}{1.725485in}}%
\pgfpathlineto{\pgfqpoint{7.233434in}{1.710992in}}%
\pgfpathlineto{\pgfqpoint{7.248867in}{1.698784in}}%
\pgfpathlineto{\pgfqpoint{7.264301in}{1.688777in}}%
\pgfpathlineto{\pgfqpoint{7.279734in}{1.680846in}}%
\pgfpathlineto{\pgfqpoint{7.295168in}{1.674833in}}%
\pgfpathlineto{\pgfqpoint{7.326035in}{1.667765in}}%
\pgfpathlineto{\pgfqpoint{7.356903in}{1.665718in}}%
\pgfpathlineto{\pgfqpoint{7.403204in}{1.667363in}}%
\pgfpathlineto{\pgfqpoint{7.449504in}{1.668278in}}%
\pgfpathlineto{\pgfqpoint{7.480372in}{1.665554in}}%
\pgfpathlineto{\pgfqpoint{7.511239in}{1.658900in}}%
\pgfpathlineto{\pgfqpoint{7.542106in}{1.648024in}}%
\pgfpathlineto{\pgfqpoint{7.572974in}{1.633376in}}%
\pgfpathlineto{\pgfqpoint{7.619274in}{1.606991in}}%
\pgfpathlineto{\pgfqpoint{7.665575in}{1.580888in}}%
\pgfpathlineto{\pgfqpoint{7.696443in}{1.567311in}}%
\pgfpathlineto{\pgfqpoint{7.711876in}{1.562501in}}%
\pgfpathlineto{\pgfqpoint{7.727310in}{1.559349in}}%
\pgfpathlineto{\pgfqpoint{7.742744in}{1.558074in}}%
\pgfpathlineto{\pgfqpoint{7.758177in}{1.558863in}}%
\pgfpathlineto{\pgfqpoint{7.773611in}{1.561863in}}%
\pgfpathlineto{\pgfqpoint{7.789044in}{1.567175in}}%
\pgfpathlineto{\pgfqpoint{7.804478in}{1.574851in}}%
\pgfpathlineto{\pgfqpoint{7.819912in}{1.584889in}}%
\pgfpathlineto{\pgfqpoint{7.835345in}{1.597236in}}%
\pgfpathlineto{\pgfqpoint{7.866213in}{1.628353in}}%
\pgfpathlineto{\pgfqpoint{7.897080in}{1.666672in}}%
\pgfpathlineto{\pgfqpoint{7.943381in}{1.732449in}}%
\pgfpathlineto{\pgfqpoint{7.989682in}{1.799121in}}%
\pgfpathlineto{\pgfqpoint{8.020549in}{1.838666in}}%
\pgfpathlineto{\pgfqpoint{8.051416in}{1.870809in}}%
\pgfpathlineto{\pgfqpoint{8.066850in}{1.883343in}}%
\pgfpathlineto{\pgfqpoint{8.082284in}{1.893207in}}%
\pgfpathlineto{\pgfqpoint{8.097717in}{1.900241in}}%
\pgfpathlineto{\pgfqpoint{8.113151in}{1.904341in}}%
\pgfpathlineto{\pgfqpoint{8.128584in}{1.905466in}}%
\pgfpathlineto{\pgfqpoint{8.144018in}{1.903635in}}%
\pgfpathlineto{\pgfqpoint{8.159452in}{1.898927in}}%
\pgfpathlineto{\pgfqpoint{8.174885in}{1.891470in}}%
\pgfpathlineto{\pgfqpoint{8.190319in}{1.881443in}}%
\pgfpathlineto{\pgfqpoint{8.205753in}{1.869064in}}%
\pgfpathlineto{\pgfqpoint{8.236620in}{1.838280in}}%
\pgfpathlineto{\pgfqpoint{8.267487in}{1.801359in}}%
\pgfpathlineto{\pgfqpoint{8.313788in}{1.739576in}}%
\pgfpathlineto{\pgfqpoint{8.390956in}{1.636095in}}%
\pgfpathlineto{\pgfqpoint{8.437257in}{1.580920in}}%
\pgfpathlineto{\pgfqpoint{8.468124in}{1.548279in}}%
\pgfpathlineto{\pgfqpoint{8.498992in}{1.519097in}}%
\pgfpathlineto{\pgfqpoint{8.529859in}{1.493339in}}%
\pgfpathlineto{\pgfqpoint{8.560726in}{1.470971in}}%
\pgfpathlineto{\pgfqpoint{8.591593in}{1.452066in}}%
\pgfpathlineto{\pgfqpoint{8.622461in}{1.436876in}}%
\pgfpathlineto{\pgfqpoint{8.653328in}{1.425842in}}%
\pgfpathlineto{\pgfqpoint{8.684195in}{1.419571in}}%
\pgfpathlineto{\pgfqpoint{8.715063in}{1.418749in}}%
\pgfpathlineto{\pgfqpoint{8.730496in}{1.420588in}}%
\pgfpathlineto{\pgfqpoint{8.745930in}{1.424022in}}%
\pgfpathlineto{\pgfqpoint{8.761363in}{1.429100in}}%
\pgfpathlineto{\pgfqpoint{8.792231in}{1.444292in}}%
\pgfpathlineto{\pgfqpoint{8.823098in}{1.466097in}}%
\pgfpathlineto{\pgfqpoint{8.853965in}{1.493974in}}%
\pgfpathlineto{\pgfqpoint{8.884833in}{1.526869in}}%
\pgfpathlineto{\pgfqpoint{8.931133in}{1.582187in}}%
\pgfpathlineto{\pgfqpoint{8.992868in}{1.656838in}}%
\pgfpathlineto{\pgfqpoint{9.023735in}{1.690075in}}%
\pgfpathlineto{\pgfqpoint{9.054603in}{1.718405in}}%
\pgfpathlineto{\pgfqpoint{9.085470in}{1.740805in}}%
\pgfpathlineto{\pgfqpoint{9.116337in}{1.756861in}}%
\pgfpathlineto{\pgfqpoint{9.147204in}{1.766785in}}%
\pgfpathlineto{\pgfqpoint{9.178072in}{1.771346in}}%
\pgfpathlineto{\pgfqpoint{9.208939in}{1.771736in}}%
\pgfpathlineto{\pgfqpoint{9.255240in}{1.767701in}}%
\pgfpathlineto{\pgfqpoint{9.316974in}{1.761484in}}%
\pgfpathlineto{\pgfqpoint{9.347842in}{1.760810in}}%
\pgfpathlineto{\pgfqpoint{9.378709in}{1.762853in}}%
\pgfpathlineto{\pgfqpoint{9.409576in}{1.767977in}}%
\pgfpathlineto{\pgfqpoint{9.440443in}{1.776182in}}%
\pgfpathlineto{\pgfqpoint{9.471311in}{1.787144in}}%
\pgfpathlineto{\pgfqpoint{9.517612in}{1.807422in}}%
\pgfpathlineto{\pgfqpoint{9.625647in}{1.858199in}}%
\pgfpathlineto{\pgfqpoint{9.671948in}{1.875101in}}%
\pgfpathlineto{\pgfqpoint{9.702815in}{1.883471in}}%
\pgfpathlineto{\pgfqpoint{9.733682in}{1.889360in}}%
\pgfpathlineto{\pgfqpoint{9.733682in}{1.889360in}}%
\pgfusepath{stroke}%
\end{pgfscope}%
\begin{pgfscope}%
\pgfsetrectcap%
\pgfsetmiterjoin%
\pgfsetlinewidth{0.803000pt}%
\definecolor{currentstroke}{rgb}{0.000000,0.000000,0.000000}%
\pgfsetstrokecolor{currentstroke}%
\pgfsetdash{}{0pt}%
\pgfpathmoveto{\pgfqpoint{5.698559in}{0.505056in}}%
\pgfpathlineto{\pgfqpoint{5.698559in}{3.250511in}}%
\pgfusepath{stroke}%
\end{pgfscope}%
\begin{pgfscope}%
\pgfsetrectcap%
\pgfsetmiterjoin%
\pgfsetlinewidth{0.803000pt}%
\definecolor{currentstroke}{rgb}{0.000000,0.000000,0.000000}%
\pgfsetstrokecolor{currentstroke}%
\pgfsetdash{}{0pt}%
\pgfpathmoveto{\pgfqpoint{9.925831in}{0.505056in}}%
\pgfpathlineto{\pgfqpoint{9.925831in}{3.250511in}}%
\pgfusepath{stroke}%
\end{pgfscope}%
\begin{pgfscope}%
\pgfsetrectcap%
\pgfsetmiterjoin%
\pgfsetlinewidth{0.803000pt}%
\definecolor{currentstroke}{rgb}{0.000000,0.000000,0.000000}%
\pgfsetstrokecolor{currentstroke}%
\pgfsetdash{}{0pt}%
\pgfpathmoveto{\pgfqpoint{5.698559in}{0.505056in}}%
\pgfpathlineto{\pgfqpoint{9.925831in}{0.505056in}}%
\pgfusepath{stroke}%
\end{pgfscope}%
\begin{pgfscope}%
\pgfsetrectcap%
\pgfsetmiterjoin%
\pgfsetlinewidth{0.803000pt}%
\definecolor{currentstroke}{rgb}{0.000000,0.000000,0.000000}%
\pgfsetstrokecolor{currentstroke}%
\pgfsetdash{}{0pt}%
\pgfpathmoveto{\pgfqpoint{5.698559in}{3.250511in}}%
\pgfpathlineto{\pgfqpoint{9.925831in}{3.250511in}}%
\pgfusepath{stroke}%
\end{pgfscope}%
\begin{pgfscope}%
\pgfsetbuttcap%
\pgfsetmiterjoin%
\definecolor{currentfill}{rgb}{1.000000,1.000000,1.000000}%
\pgfsetfillcolor{currentfill}%
\pgfsetfillopacity{0.800000}%
\pgfsetlinewidth{1.003750pt}%
\definecolor{currentstroke}{rgb}{0.800000,0.800000,0.800000}%
\pgfsetstrokecolor{currentstroke}%
\pgfsetstrokeopacity{0.800000}%
\pgfsetdash{}{0pt}%
\pgfpathmoveto{\pgfqpoint{9.283863in}{0.574501in}}%
\pgfpathlineto{\pgfqpoint{9.828609in}{0.574501in}}%
\pgfpathquadraticcurveto{\pgfqpoint{9.856387in}{0.574501in}}{\pgfqpoint{9.856387in}{0.602278in}}%
\pgfpathlineto{\pgfqpoint{9.856387in}{0.784760in}}%
\pgfpathquadraticcurveto{\pgfqpoint{9.856387in}{0.812538in}}{\pgfqpoint{9.828609in}{0.812538in}}%
\pgfpathlineto{\pgfqpoint{9.283863in}{0.812538in}}%
\pgfpathquadraticcurveto{\pgfqpoint{9.256086in}{0.812538in}}{\pgfqpoint{9.256086in}{0.784760in}}%
\pgfpathlineto{\pgfqpoint{9.256086in}{0.602278in}}%
\pgfpathquadraticcurveto{\pgfqpoint{9.256086in}{0.574501in}}{\pgfqpoint{9.283863in}{0.574501in}}%
\pgfpathclose%
\pgfusepath{stroke,fill}%
\end{pgfscope}%
\begin{pgfscope}%
\pgfsetrectcap%
\pgfsetroundjoin%
\pgfsetlinewidth{0.501875pt}%
\definecolor{currentstroke}{rgb}{0.000000,0.000000,0.000000}%
\pgfsetstrokecolor{currentstroke}%
\pgfsetdash{}{0pt}%
\pgfpathmoveto{\pgfqpoint{9.311641in}{0.708371in}}%
\pgfpathlineto{\pgfqpoint{9.589419in}{0.708371in}}%
\pgfusepath{stroke}%
\end{pgfscope}%
\begin{pgfscope}%
\pgftext[x=9.700530in,y=0.659760in,left,base]{\rmfamily\fontsize{10.000000}{12.000000}\selectfont K}%
\end{pgfscope}%
\begin{pgfscope}%
\pgfpathrectangle{\pgfqpoint{10.165831in}{0.305056in}}{\pgfqpoint{0.120000in}{6.400000in}} %
\pgfusepath{clip}%
\pgfsetbuttcap%
\pgfsetmiterjoin%
\definecolor{currentfill}{rgb}{1.000000,1.000000,1.000000}%
\pgfsetfillcolor{currentfill}%
\pgfsetlinewidth{0.010037pt}%
\definecolor{currentstroke}{rgb}{1.000000,1.000000,1.000000}%
\pgfsetstrokecolor{currentstroke}%
\pgfsetdash{}{0pt}%
\pgfpathmoveto{\pgfqpoint{10.165831in}{0.305056in}}%
\pgfpathlineto{\pgfqpoint{10.165831in}{0.330056in}}%
\pgfpathlineto{\pgfqpoint{10.165831in}{6.680056in}}%
\pgfpathlineto{\pgfqpoint{10.165831in}{6.705056in}}%
\pgfpathlineto{\pgfqpoint{10.285831in}{6.705056in}}%
\pgfpathlineto{\pgfqpoint{10.285831in}{6.680056in}}%
\pgfpathlineto{\pgfqpoint{10.285831in}{0.330056in}}%
\pgfpathlineto{\pgfqpoint{10.285831in}{0.305056in}}%
\pgfpathclose%
\pgfusepath{stroke,fill}%
\end{pgfscope}%
\begin{pgfscope}%
\pgfsys@transformshift{10.166667in}{0.316557in}%
\pgftext[left,bottom]{\pgfimage[interpolate=true,width=0.125000in,height=6.402778in]{gfx/not_Mercer-img0.png}}%
\end{pgfscope}%
\begin{pgfscope}%
\pgfsetbuttcap%
\pgfsetroundjoin%
\definecolor{currentfill}{rgb}{0.000000,0.000000,0.000000}%
\pgfsetfillcolor{currentfill}%
\pgfsetlinewidth{0.803000pt}%
\definecolor{currentstroke}{rgb}{0.000000,0.000000,0.000000}%
\pgfsetstrokecolor{currentstroke}%
\pgfsetdash{}{0pt}%
\pgfsys@defobject{currentmarker}{\pgfqpoint{0.000000in}{0.000000in}}{\pgfqpoint{0.048611in}{0.000000in}}{%
\pgfpathmoveto{\pgfqpoint{0.000000in}{0.000000in}}%
\pgfpathlineto{\pgfqpoint{0.048611in}{0.000000in}}%
\pgfusepath{stroke,fill}%
}%
\begin{pgfscope}%
\pgfsys@transformshift{10.285831in}{0.305056in}%
\pgfsys@useobject{currentmarker}{}%
\end{pgfscope}%
\end{pgfscope}%
\begin{pgfscope}%
\pgftext[x=10.383053in,y=0.256838in,left,base]{\rmfamily\fontsize{10.000000}{12.000000}\selectfont \(\displaystyle 10^{0}\)}%
\end{pgfscope}%
\begin{pgfscope}%
\pgfsetbuttcap%
\pgfsetroundjoin%
\definecolor{currentfill}{rgb}{0.000000,0.000000,0.000000}%
\pgfsetfillcolor{currentfill}%
\pgfsetlinewidth{0.803000pt}%
\definecolor{currentstroke}{rgb}{0.000000,0.000000,0.000000}%
\pgfsetstrokecolor{currentstroke}%
\pgfsetdash{}{0pt}%
\pgfsys@defobject{currentmarker}{\pgfqpoint{0.000000in}{0.000000in}}{\pgfqpoint{0.048611in}{0.000000in}}{%
\pgfpathmoveto{\pgfqpoint{0.000000in}{0.000000in}}%
\pgfpathlineto{\pgfqpoint{0.048611in}{0.000000in}}%
\pgfusepath{stroke,fill}%
}%
\begin{pgfscope}%
\pgfsys@transformshift{10.285831in}{0.786704in}%
\pgfsys@useobject{currentmarker}{}%
\end{pgfscope}%
\end{pgfscope}%
\begin{pgfscope}%
\pgfsetbuttcap%
\pgfsetroundjoin%
\definecolor{currentfill}{rgb}{0.000000,0.000000,0.000000}%
\pgfsetfillcolor{currentfill}%
\pgfsetlinewidth{0.803000pt}%
\definecolor{currentstroke}{rgb}{0.000000,0.000000,0.000000}%
\pgfsetstrokecolor{currentstroke}%
\pgfsetdash{}{0pt}%
\pgfsys@defobject{currentmarker}{\pgfqpoint{0.000000in}{0.000000in}}{\pgfqpoint{0.048611in}{0.000000in}}{%
\pgfpathmoveto{\pgfqpoint{0.000000in}{0.000000in}}%
\pgfpathlineto{\pgfqpoint{0.048611in}{0.000000in}}%
\pgfusepath{stroke,fill}%
}%
\begin{pgfscope}%
\pgfsys@transformshift{10.285831in}{1.068450in}%
\pgfsys@useobject{currentmarker}{}%
\end{pgfscope}%
\end{pgfscope}%
\begin{pgfscope}%
\pgfsetbuttcap%
\pgfsetroundjoin%
\definecolor{currentfill}{rgb}{0.000000,0.000000,0.000000}%
\pgfsetfillcolor{currentfill}%
\pgfsetlinewidth{0.803000pt}%
\definecolor{currentstroke}{rgb}{0.000000,0.000000,0.000000}%
\pgfsetstrokecolor{currentstroke}%
\pgfsetdash{}{0pt}%
\pgfsys@defobject{currentmarker}{\pgfqpoint{0.000000in}{0.000000in}}{\pgfqpoint{0.048611in}{0.000000in}}{%
\pgfpathmoveto{\pgfqpoint{0.000000in}{0.000000in}}%
\pgfpathlineto{\pgfqpoint{0.048611in}{0.000000in}}%
\pgfusepath{stroke,fill}%
}%
\begin{pgfscope}%
\pgfsys@transformshift{10.285831in}{1.268352in}%
\pgfsys@useobject{currentmarker}{}%
\end{pgfscope}%
\end{pgfscope}%
\begin{pgfscope}%
\pgfsetbuttcap%
\pgfsetroundjoin%
\definecolor{currentfill}{rgb}{0.000000,0.000000,0.000000}%
\pgfsetfillcolor{currentfill}%
\pgfsetlinewidth{0.803000pt}%
\definecolor{currentstroke}{rgb}{0.000000,0.000000,0.000000}%
\pgfsetstrokecolor{currentstroke}%
\pgfsetdash{}{0pt}%
\pgfsys@defobject{currentmarker}{\pgfqpoint{0.000000in}{0.000000in}}{\pgfqpoint{0.048611in}{0.000000in}}{%
\pgfpathmoveto{\pgfqpoint{0.000000in}{0.000000in}}%
\pgfpathlineto{\pgfqpoint{0.048611in}{0.000000in}}%
\pgfusepath{stroke,fill}%
}%
\begin{pgfscope}%
\pgfsys@transformshift{10.285831in}{1.423408in}%
\pgfsys@useobject{currentmarker}{}%
\end{pgfscope}%
\end{pgfscope}%
\begin{pgfscope}%
\pgfsetbuttcap%
\pgfsetroundjoin%
\definecolor{currentfill}{rgb}{0.000000,0.000000,0.000000}%
\pgfsetfillcolor{currentfill}%
\pgfsetlinewidth{0.803000pt}%
\definecolor{currentstroke}{rgb}{0.000000,0.000000,0.000000}%
\pgfsetstrokecolor{currentstroke}%
\pgfsetdash{}{0pt}%
\pgfsys@defobject{currentmarker}{\pgfqpoint{0.000000in}{0.000000in}}{\pgfqpoint{0.048611in}{0.000000in}}{%
\pgfpathmoveto{\pgfqpoint{0.000000in}{0.000000in}}%
\pgfpathlineto{\pgfqpoint{0.048611in}{0.000000in}}%
\pgfusepath{stroke,fill}%
}%
\begin{pgfscope}%
\pgfsys@transformshift{10.285831in}{1.550098in}%
\pgfsys@useobject{currentmarker}{}%
\end{pgfscope}%
\end{pgfscope}%
\begin{pgfscope}%
\pgfsetbuttcap%
\pgfsetroundjoin%
\definecolor{currentfill}{rgb}{0.000000,0.000000,0.000000}%
\pgfsetfillcolor{currentfill}%
\pgfsetlinewidth{0.803000pt}%
\definecolor{currentstroke}{rgb}{0.000000,0.000000,0.000000}%
\pgfsetstrokecolor{currentstroke}%
\pgfsetdash{}{0pt}%
\pgfsys@defobject{currentmarker}{\pgfqpoint{0.000000in}{0.000000in}}{\pgfqpoint{0.048611in}{0.000000in}}{%
\pgfpathmoveto{\pgfqpoint{0.000000in}{0.000000in}}%
\pgfpathlineto{\pgfqpoint{0.048611in}{0.000000in}}%
\pgfusepath{stroke,fill}%
}%
\begin{pgfscope}%
\pgfsys@transformshift{10.285831in}{1.657213in}%
\pgfsys@useobject{currentmarker}{}%
\end{pgfscope}%
\end{pgfscope}%
\begin{pgfscope}%
\pgfsetbuttcap%
\pgfsetroundjoin%
\definecolor{currentfill}{rgb}{0.000000,0.000000,0.000000}%
\pgfsetfillcolor{currentfill}%
\pgfsetlinewidth{0.803000pt}%
\definecolor{currentstroke}{rgb}{0.000000,0.000000,0.000000}%
\pgfsetstrokecolor{currentstroke}%
\pgfsetdash{}{0pt}%
\pgfsys@defobject{currentmarker}{\pgfqpoint{0.000000in}{0.000000in}}{\pgfqpoint{0.048611in}{0.000000in}}{%
\pgfpathmoveto{\pgfqpoint{0.000000in}{0.000000in}}%
\pgfpathlineto{\pgfqpoint{0.048611in}{0.000000in}}%
\pgfusepath{stroke,fill}%
}%
\begin{pgfscope}%
\pgfsys@transformshift{10.285831in}{1.750000in}%
\pgfsys@useobject{currentmarker}{}%
\end{pgfscope}%
\end{pgfscope}%
\begin{pgfscope}%
\pgfsetbuttcap%
\pgfsetroundjoin%
\definecolor{currentfill}{rgb}{0.000000,0.000000,0.000000}%
\pgfsetfillcolor{currentfill}%
\pgfsetlinewidth{0.803000pt}%
\definecolor{currentstroke}{rgb}{0.000000,0.000000,0.000000}%
\pgfsetstrokecolor{currentstroke}%
\pgfsetdash{}{0pt}%
\pgfsys@defobject{currentmarker}{\pgfqpoint{0.000000in}{0.000000in}}{\pgfqpoint{0.048611in}{0.000000in}}{%
\pgfpathmoveto{\pgfqpoint{0.000000in}{0.000000in}}%
\pgfpathlineto{\pgfqpoint{0.048611in}{0.000000in}}%
\pgfusepath{stroke,fill}%
}%
\begin{pgfscope}%
\pgfsys@transformshift{10.285831in}{1.831844in}%
\pgfsys@useobject{currentmarker}{}%
\end{pgfscope}%
\end{pgfscope}%
\begin{pgfscope}%
\pgfsetbuttcap%
\pgfsetroundjoin%
\definecolor{currentfill}{rgb}{0.000000,0.000000,0.000000}%
\pgfsetfillcolor{currentfill}%
\pgfsetlinewidth{0.803000pt}%
\definecolor{currentstroke}{rgb}{0.000000,0.000000,0.000000}%
\pgfsetstrokecolor{currentstroke}%
\pgfsetdash{}{0pt}%
\pgfsys@defobject{currentmarker}{\pgfqpoint{0.000000in}{0.000000in}}{\pgfqpoint{0.048611in}{0.000000in}}{%
\pgfpathmoveto{\pgfqpoint{0.000000in}{0.000000in}}%
\pgfpathlineto{\pgfqpoint{0.048611in}{0.000000in}}%
\pgfusepath{stroke,fill}%
}%
\begin{pgfscope}%
\pgfsys@transformshift{10.285831in}{1.905056in}%
\pgfsys@useobject{currentmarker}{}%
\end{pgfscope}%
\end{pgfscope}%
\begin{pgfscope}%
\pgftext[x=10.383053in,y=1.856838in,left,base]{\rmfamily\fontsize{10.000000}{12.000000}\selectfont \(\displaystyle 10^{1}\)}%
\end{pgfscope}%
\begin{pgfscope}%
\pgfsetbuttcap%
\pgfsetroundjoin%
\definecolor{currentfill}{rgb}{0.000000,0.000000,0.000000}%
\pgfsetfillcolor{currentfill}%
\pgfsetlinewidth{0.803000pt}%
\definecolor{currentstroke}{rgb}{0.000000,0.000000,0.000000}%
\pgfsetstrokecolor{currentstroke}%
\pgfsetdash{}{0pt}%
\pgfsys@defobject{currentmarker}{\pgfqpoint{0.000000in}{0.000000in}}{\pgfqpoint{0.048611in}{0.000000in}}{%
\pgfpathmoveto{\pgfqpoint{0.000000in}{0.000000in}}%
\pgfpathlineto{\pgfqpoint{0.048611in}{0.000000in}}%
\pgfusepath{stroke,fill}%
}%
\begin{pgfscope}%
\pgfsys@transformshift{10.285831in}{2.386704in}%
\pgfsys@useobject{currentmarker}{}%
\end{pgfscope}%
\end{pgfscope}%
\begin{pgfscope}%
\pgfsetbuttcap%
\pgfsetroundjoin%
\definecolor{currentfill}{rgb}{0.000000,0.000000,0.000000}%
\pgfsetfillcolor{currentfill}%
\pgfsetlinewidth{0.803000pt}%
\definecolor{currentstroke}{rgb}{0.000000,0.000000,0.000000}%
\pgfsetstrokecolor{currentstroke}%
\pgfsetdash{}{0pt}%
\pgfsys@defobject{currentmarker}{\pgfqpoint{0.000000in}{0.000000in}}{\pgfqpoint{0.048611in}{0.000000in}}{%
\pgfpathmoveto{\pgfqpoint{0.000000in}{0.000000in}}%
\pgfpathlineto{\pgfqpoint{0.048611in}{0.000000in}}%
\pgfusepath{stroke,fill}%
}%
\begin{pgfscope}%
\pgfsys@transformshift{10.285831in}{2.668450in}%
\pgfsys@useobject{currentmarker}{}%
\end{pgfscope}%
\end{pgfscope}%
\begin{pgfscope}%
\pgfsetbuttcap%
\pgfsetroundjoin%
\definecolor{currentfill}{rgb}{0.000000,0.000000,0.000000}%
\pgfsetfillcolor{currentfill}%
\pgfsetlinewidth{0.803000pt}%
\definecolor{currentstroke}{rgb}{0.000000,0.000000,0.000000}%
\pgfsetstrokecolor{currentstroke}%
\pgfsetdash{}{0pt}%
\pgfsys@defobject{currentmarker}{\pgfqpoint{0.000000in}{0.000000in}}{\pgfqpoint{0.048611in}{0.000000in}}{%
\pgfpathmoveto{\pgfqpoint{0.000000in}{0.000000in}}%
\pgfpathlineto{\pgfqpoint{0.048611in}{0.000000in}}%
\pgfusepath{stroke,fill}%
}%
\begin{pgfscope}%
\pgfsys@transformshift{10.285831in}{2.868352in}%
\pgfsys@useobject{currentmarker}{}%
\end{pgfscope}%
\end{pgfscope}%
\begin{pgfscope}%
\pgfsetbuttcap%
\pgfsetroundjoin%
\definecolor{currentfill}{rgb}{0.000000,0.000000,0.000000}%
\pgfsetfillcolor{currentfill}%
\pgfsetlinewidth{0.803000pt}%
\definecolor{currentstroke}{rgb}{0.000000,0.000000,0.000000}%
\pgfsetstrokecolor{currentstroke}%
\pgfsetdash{}{0pt}%
\pgfsys@defobject{currentmarker}{\pgfqpoint{0.000000in}{0.000000in}}{\pgfqpoint{0.048611in}{0.000000in}}{%
\pgfpathmoveto{\pgfqpoint{0.000000in}{0.000000in}}%
\pgfpathlineto{\pgfqpoint{0.048611in}{0.000000in}}%
\pgfusepath{stroke,fill}%
}%
\begin{pgfscope}%
\pgfsys@transformshift{10.285831in}{3.023408in}%
\pgfsys@useobject{currentmarker}{}%
\end{pgfscope}%
\end{pgfscope}%
\begin{pgfscope}%
\pgfsetbuttcap%
\pgfsetroundjoin%
\definecolor{currentfill}{rgb}{0.000000,0.000000,0.000000}%
\pgfsetfillcolor{currentfill}%
\pgfsetlinewidth{0.803000pt}%
\definecolor{currentstroke}{rgb}{0.000000,0.000000,0.000000}%
\pgfsetstrokecolor{currentstroke}%
\pgfsetdash{}{0pt}%
\pgfsys@defobject{currentmarker}{\pgfqpoint{0.000000in}{0.000000in}}{\pgfqpoint{0.048611in}{0.000000in}}{%
\pgfpathmoveto{\pgfqpoint{0.000000in}{0.000000in}}%
\pgfpathlineto{\pgfqpoint{0.048611in}{0.000000in}}%
\pgfusepath{stroke,fill}%
}%
\begin{pgfscope}%
\pgfsys@transformshift{10.285831in}{3.150098in}%
\pgfsys@useobject{currentmarker}{}%
\end{pgfscope}%
\end{pgfscope}%
\begin{pgfscope}%
\pgfsetbuttcap%
\pgfsetroundjoin%
\definecolor{currentfill}{rgb}{0.000000,0.000000,0.000000}%
\pgfsetfillcolor{currentfill}%
\pgfsetlinewidth{0.803000pt}%
\definecolor{currentstroke}{rgb}{0.000000,0.000000,0.000000}%
\pgfsetstrokecolor{currentstroke}%
\pgfsetdash{}{0pt}%
\pgfsys@defobject{currentmarker}{\pgfqpoint{0.000000in}{0.000000in}}{\pgfqpoint{0.048611in}{0.000000in}}{%
\pgfpathmoveto{\pgfqpoint{0.000000in}{0.000000in}}%
\pgfpathlineto{\pgfqpoint{0.048611in}{0.000000in}}%
\pgfusepath{stroke,fill}%
}%
\begin{pgfscope}%
\pgfsys@transformshift{10.285831in}{3.257213in}%
\pgfsys@useobject{currentmarker}{}%
\end{pgfscope}%
\end{pgfscope}%
\begin{pgfscope}%
\pgfsetbuttcap%
\pgfsetroundjoin%
\definecolor{currentfill}{rgb}{0.000000,0.000000,0.000000}%
\pgfsetfillcolor{currentfill}%
\pgfsetlinewidth{0.803000pt}%
\definecolor{currentstroke}{rgb}{0.000000,0.000000,0.000000}%
\pgfsetstrokecolor{currentstroke}%
\pgfsetdash{}{0pt}%
\pgfsys@defobject{currentmarker}{\pgfqpoint{0.000000in}{0.000000in}}{\pgfqpoint{0.048611in}{0.000000in}}{%
\pgfpathmoveto{\pgfqpoint{0.000000in}{0.000000in}}%
\pgfpathlineto{\pgfqpoint{0.048611in}{0.000000in}}%
\pgfusepath{stroke,fill}%
}%
\begin{pgfscope}%
\pgfsys@transformshift{10.285831in}{3.350000in}%
\pgfsys@useobject{currentmarker}{}%
\end{pgfscope}%
\end{pgfscope}%
\begin{pgfscope}%
\pgfsetbuttcap%
\pgfsetroundjoin%
\definecolor{currentfill}{rgb}{0.000000,0.000000,0.000000}%
\pgfsetfillcolor{currentfill}%
\pgfsetlinewidth{0.803000pt}%
\definecolor{currentstroke}{rgb}{0.000000,0.000000,0.000000}%
\pgfsetstrokecolor{currentstroke}%
\pgfsetdash{}{0pt}%
\pgfsys@defobject{currentmarker}{\pgfqpoint{0.000000in}{0.000000in}}{\pgfqpoint{0.048611in}{0.000000in}}{%
\pgfpathmoveto{\pgfqpoint{0.000000in}{0.000000in}}%
\pgfpathlineto{\pgfqpoint{0.048611in}{0.000000in}}%
\pgfusepath{stroke,fill}%
}%
\begin{pgfscope}%
\pgfsys@transformshift{10.285831in}{3.431844in}%
\pgfsys@useobject{currentmarker}{}%
\end{pgfscope}%
\end{pgfscope}%
\begin{pgfscope}%
\pgfsetbuttcap%
\pgfsetroundjoin%
\definecolor{currentfill}{rgb}{0.000000,0.000000,0.000000}%
\pgfsetfillcolor{currentfill}%
\pgfsetlinewidth{0.803000pt}%
\definecolor{currentstroke}{rgb}{0.000000,0.000000,0.000000}%
\pgfsetstrokecolor{currentstroke}%
\pgfsetdash{}{0pt}%
\pgfsys@defobject{currentmarker}{\pgfqpoint{0.000000in}{0.000000in}}{\pgfqpoint{0.048611in}{0.000000in}}{%
\pgfpathmoveto{\pgfqpoint{0.000000in}{0.000000in}}%
\pgfpathlineto{\pgfqpoint{0.048611in}{0.000000in}}%
\pgfusepath{stroke,fill}%
}%
\begin{pgfscope}%
\pgfsys@transformshift{10.285831in}{3.505056in}%
\pgfsys@useobject{currentmarker}{}%
\end{pgfscope}%
\end{pgfscope}%
\begin{pgfscope}%
\pgftext[x=10.383053in,y=3.456838in,left,base]{\rmfamily\fontsize{10.000000}{12.000000}\selectfont \(\displaystyle 10^{2}\)}%
\end{pgfscope}%
\begin{pgfscope}%
\pgfsetbuttcap%
\pgfsetroundjoin%
\definecolor{currentfill}{rgb}{0.000000,0.000000,0.000000}%
\pgfsetfillcolor{currentfill}%
\pgfsetlinewidth{0.803000pt}%
\definecolor{currentstroke}{rgb}{0.000000,0.000000,0.000000}%
\pgfsetstrokecolor{currentstroke}%
\pgfsetdash{}{0pt}%
\pgfsys@defobject{currentmarker}{\pgfqpoint{0.000000in}{0.000000in}}{\pgfqpoint{0.048611in}{0.000000in}}{%
\pgfpathmoveto{\pgfqpoint{0.000000in}{0.000000in}}%
\pgfpathlineto{\pgfqpoint{0.048611in}{0.000000in}}%
\pgfusepath{stroke,fill}%
}%
\begin{pgfscope}%
\pgfsys@transformshift{10.285831in}{3.986704in}%
\pgfsys@useobject{currentmarker}{}%
\end{pgfscope}%
\end{pgfscope}%
\begin{pgfscope}%
\pgfsetbuttcap%
\pgfsetroundjoin%
\definecolor{currentfill}{rgb}{0.000000,0.000000,0.000000}%
\pgfsetfillcolor{currentfill}%
\pgfsetlinewidth{0.803000pt}%
\definecolor{currentstroke}{rgb}{0.000000,0.000000,0.000000}%
\pgfsetstrokecolor{currentstroke}%
\pgfsetdash{}{0pt}%
\pgfsys@defobject{currentmarker}{\pgfqpoint{0.000000in}{0.000000in}}{\pgfqpoint{0.048611in}{0.000000in}}{%
\pgfpathmoveto{\pgfqpoint{0.000000in}{0.000000in}}%
\pgfpathlineto{\pgfqpoint{0.048611in}{0.000000in}}%
\pgfusepath{stroke,fill}%
}%
\begin{pgfscope}%
\pgfsys@transformshift{10.285831in}{4.268450in}%
\pgfsys@useobject{currentmarker}{}%
\end{pgfscope}%
\end{pgfscope}%
\begin{pgfscope}%
\pgfsetbuttcap%
\pgfsetroundjoin%
\definecolor{currentfill}{rgb}{0.000000,0.000000,0.000000}%
\pgfsetfillcolor{currentfill}%
\pgfsetlinewidth{0.803000pt}%
\definecolor{currentstroke}{rgb}{0.000000,0.000000,0.000000}%
\pgfsetstrokecolor{currentstroke}%
\pgfsetdash{}{0pt}%
\pgfsys@defobject{currentmarker}{\pgfqpoint{0.000000in}{0.000000in}}{\pgfqpoint{0.048611in}{0.000000in}}{%
\pgfpathmoveto{\pgfqpoint{0.000000in}{0.000000in}}%
\pgfpathlineto{\pgfqpoint{0.048611in}{0.000000in}}%
\pgfusepath{stroke,fill}%
}%
\begin{pgfscope}%
\pgfsys@transformshift{10.285831in}{4.468352in}%
\pgfsys@useobject{currentmarker}{}%
\end{pgfscope}%
\end{pgfscope}%
\begin{pgfscope}%
\pgfsetbuttcap%
\pgfsetroundjoin%
\definecolor{currentfill}{rgb}{0.000000,0.000000,0.000000}%
\pgfsetfillcolor{currentfill}%
\pgfsetlinewidth{0.803000pt}%
\definecolor{currentstroke}{rgb}{0.000000,0.000000,0.000000}%
\pgfsetstrokecolor{currentstroke}%
\pgfsetdash{}{0pt}%
\pgfsys@defobject{currentmarker}{\pgfqpoint{0.000000in}{0.000000in}}{\pgfqpoint{0.048611in}{0.000000in}}{%
\pgfpathmoveto{\pgfqpoint{0.000000in}{0.000000in}}%
\pgfpathlineto{\pgfqpoint{0.048611in}{0.000000in}}%
\pgfusepath{stroke,fill}%
}%
\begin{pgfscope}%
\pgfsys@transformshift{10.285831in}{4.623408in}%
\pgfsys@useobject{currentmarker}{}%
\end{pgfscope}%
\end{pgfscope}%
\begin{pgfscope}%
\pgfsetbuttcap%
\pgfsetroundjoin%
\definecolor{currentfill}{rgb}{0.000000,0.000000,0.000000}%
\pgfsetfillcolor{currentfill}%
\pgfsetlinewidth{0.803000pt}%
\definecolor{currentstroke}{rgb}{0.000000,0.000000,0.000000}%
\pgfsetstrokecolor{currentstroke}%
\pgfsetdash{}{0pt}%
\pgfsys@defobject{currentmarker}{\pgfqpoint{0.000000in}{0.000000in}}{\pgfqpoint{0.048611in}{0.000000in}}{%
\pgfpathmoveto{\pgfqpoint{0.000000in}{0.000000in}}%
\pgfpathlineto{\pgfqpoint{0.048611in}{0.000000in}}%
\pgfusepath{stroke,fill}%
}%
\begin{pgfscope}%
\pgfsys@transformshift{10.285831in}{4.750098in}%
\pgfsys@useobject{currentmarker}{}%
\end{pgfscope}%
\end{pgfscope}%
\begin{pgfscope}%
\pgfsetbuttcap%
\pgfsetroundjoin%
\definecolor{currentfill}{rgb}{0.000000,0.000000,0.000000}%
\pgfsetfillcolor{currentfill}%
\pgfsetlinewidth{0.803000pt}%
\definecolor{currentstroke}{rgb}{0.000000,0.000000,0.000000}%
\pgfsetstrokecolor{currentstroke}%
\pgfsetdash{}{0pt}%
\pgfsys@defobject{currentmarker}{\pgfqpoint{0.000000in}{0.000000in}}{\pgfqpoint{0.048611in}{0.000000in}}{%
\pgfpathmoveto{\pgfqpoint{0.000000in}{0.000000in}}%
\pgfpathlineto{\pgfqpoint{0.048611in}{0.000000in}}%
\pgfusepath{stroke,fill}%
}%
\begin{pgfscope}%
\pgfsys@transformshift{10.285831in}{4.857213in}%
\pgfsys@useobject{currentmarker}{}%
\end{pgfscope}%
\end{pgfscope}%
\begin{pgfscope}%
\pgfsetbuttcap%
\pgfsetroundjoin%
\definecolor{currentfill}{rgb}{0.000000,0.000000,0.000000}%
\pgfsetfillcolor{currentfill}%
\pgfsetlinewidth{0.803000pt}%
\definecolor{currentstroke}{rgb}{0.000000,0.000000,0.000000}%
\pgfsetstrokecolor{currentstroke}%
\pgfsetdash{}{0pt}%
\pgfsys@defobject{currentmarker}{\pgfqpoint{0.000000in}{0.000000in}}{\pgfqpoint{0.048611in}{0.000000in}}{%
\pgfpathmoveto{\pgfqpoint{0.000000in}{0.000000in}}%
\pgfpathlineto{\pgfqpoint{0.048611in}{0.000000in}}%
\pgfusepath{stroke,fill}%
}%
\begin{pgfscope}%
\pgfsys@transformshift{10.285831in}{4.950000in}%
\pgfsys@useobject{currentmarker}{}%
\end{pgfscope}%
\end{pgfscope}%
\begin{pgfscope}%
\pgfsetbuttcap%
\pgfsetroundjoin%
\definecolor{currentfill}{rgb}{0.000000,0.000000,0.000000}%
\pgfsetfillcolor{currentfill}%
\pgfsetlinewidth{0.803000pt}%
\definecolor{currentstroke}{rgb}{0.000000,0.000000,0.000000}%
\pgfsetstrokecolor{currentstroke}%
\pgfsetdash{}{0pt}%
\pgfsys@defobject{currentmarker}{\pgfqpoint{0.000000in}{0.000000in}}{\pgfqpoint{0.048611in}{0.000000in}}{%
\pgfpathmoveto{\pgfqpoint{0.000000in}{0.000000in}}%
\pgfpathlineto{\pgfqpoint{0.048611in}{0.000000in}}%
\pgfusepath{stroke,fill}%
}%
\begin{pgfscope}%
\pgfsys@transformshift{10.285831in}{5.031844in}%
\pgfsys@useobject{currentmarker}{}%
\end{pgfscope}%
\end{pgfscope}%
\begin{pgfscope}%
\pgfsetbuttcap%
\pgfsetroundjoin%
\definecolor{currentfill}{rgb}{0.000000,0.000000,0.000000}%
\pgfsetfillcolor{currentfill}%
\pgfsetlinewidth{0.803000pt}%
\definecolor{currentstroke}{rgb}{0.000000,0.000000,0.000000}%
\pgfsetstrokecolor{currentstroke}%
\pgfsetdash{}{0pt}%
\pgfsys@defobject{currentmarker}{\pgfqpoint{0.000000in}{0.000000in}}{\pgfqpoint{0.048611in}{0.000000in}}{%
\pgfpathmoveto{\pgfqpoint{0.000000in}{0.000000in}}%
\pgfpathlineto{\pgfqpoint{0.048611in}{0.000000in}}%
\pgfusepath{stroke,fill}%
}%
\begin{pgfscope}%
\pgfsys@transformshift{10.285831in}{5.105056in}%
\pgfsys@useobject{currentmarker}{}%
\end{pgfscope}%
\end{pgfscope}%
\begin{pgfscope}%
\pgftext[x=10.383053in,y=5.056838in,left,base]{\rmfamily\fontsize{10.000000}{12.000000}\selectfont \(\displaystyle 10^{3}\)}%
\end{pgfscope}%
\begin{pgfscope}%
\pgfsetbuttcap%
\pgfsetroundjoin%
\definecolor{currentfill}{rgb}{0.000000,0.000000,0.000000}%
\pgfsetfillcolor{currentfill}%
\pgfsetlinewidth{0.803000pt}%
\definecolor{currentstroke}{rgb}{0.000000,0.000000,0.000000}%
\pgfsetstrokecolor{currentstroke}%
\pgfsetdash{}{0pt}%
\pgfsys@defobject{currentmarker}{\pgfqpoint{0.000000in}{0.000000in}}{\pgfqpoint{0.048611in}{0.000000in}}{%
\pgfpathmoveto{\pgfqpoint{0.000000in}{0.000000in}}%
\pgfpathlineto{\pgfqpoint{0.048611in}{0.000000in}}%
\pgfusepath{stroke,fill}%
}%
\begin{pgfscope}%
\pgfsys@transformshift{10.285831in}{5.586704in}%
\pgfsys@useobject{currentmarker}{}%
\end{pgfscope}%
\end{pgfscope}%
\begin{pgfscope}%
\pgfsetbuttcap%
\pgfsetroundjoin%
\definecolor{currentfill}{rgb}{0.000000,0.000000,0.000000}%
\pgfsetfillcolor{currentfill}%
\pgfsetlinewidth{0.803000pt}%
\definecolor{currentstroke}{rgb}{0.000000,0.000000,0.000000}%
\pgfsetstrokecolor{currentstroke}%
\pgfsetdash{}{0pt}%
\pgfsys@defobject{currentmarker}{\pgfqpoint{0.000000in}{0.000000in}}{\pgfqpoint{0.048611in}{0.000000in}}{%
\pgfpathmoveto{\pgfqpoint{0.000000in}{0.000000in}}%
\pgfpathlineto{\pgfqpoint{0.048611in}{0.000000in}}%
\pgfusepath{stroke,fill}%
}%
\begin{pgfscope}%
\pgfsys@transformshift{10.285831in}{5.868450in}%
\pgfsys@useobject{currentmarker}{}%
\end{pgfscope}%
\end{pgfscope}%
\begin{pgfscope}%
\pgfsetbuttcap%
\pgfsetroundjoin%
\definecolor{currentfill}{rgb}{0.000000,0.000000,0.000000}%
\pgfsetfillcolor{currentfill}%
\pgfsetlinewidth{0.803000pt}%
\definecolor{currentstroke}{rgb}{0.000000,0.000000,0.000000}%
\pgfsetstrokecolor{currentstroke}%
\pgfsetdash{}{0pt}%
\pgfsys@defobject{currentmarker}{\pgfqpoint{0.000000in}{0.000000in}}{\pgfqpoint{0.048611in}{0.000000in}}{%
\pgfpathmoveto{\pgfqpoint{0.000000in}{0.000000in}}%
\pgfpathlineto{\pgfqpoint{0.048611in}{0.000000in}}%
\pgfusepath{stroke,fill}%
}%
\begin{pgfscope}%
\pgfsys@transformshift{10.285831in}{6.068352in}%
\pgfsys@useobject{currentmarker}{}%
\end{pgfscope}%
\end{pgfscope}%
\begin{pgfscope}%
\pgfsetbuttcap%
\pgfsetroundjoin%
\definecolor{currentfill}{rgb}{0.000000,0.000000,0.000000}%
\pgfsetfillcolor{currentfill}%
\pgfsetlinewidth{0.803000pt}%
\definecolor{currentstroke}{rgb}{0.000000,0.000000,0.000000}%
\pgfsetstrokecolor{currentstroke}%
\pgfsetdash{}{0pt}%
\pgfsys@defobject{currentmarker}{\pgfqpoint{0.000000in}{0.000000in}}{\pgfqpoint{0.048611in}{0.000000in}}{%
\pgfpathmoveto{\pgfqpoint{0.000000in}{0.000000in}}%
\pgfpathlineto{\pgfqpoint{0.048611in}{0.000000in}}%
\pgfusepath{stroke,fill}%
}%
\begin{pgfscope}%
\pgfsys@transformshift{10.285831in}{6.223408in}%
\pgfsys@useobject{currentmarker}{}%
\end{pgfscope}%
\end{pgfscope}%
\begin{pgfscope}%
\pgfsetbuttcap%
\pgfsetroundjoin%
\definecolor{currentfill}{rgb}{0.000000,0.000000,0.000000}%
\pgfsetfillcolor{currentfill}%
\pgfsetlinewidth{0.803000pt}%
\definecolor{currentstroke}{rgb}{0.000000,0.000000,0.000000}%
\pgfsetstrokecolor{currentstroke}%
\pgfsetdash{}{0pt}%
\pgfsys@defobject{currentmarker}{\pgfqpoint{0.000000in}{0.000000in}}{\pgfqpoint{0.048611in}{0.000000in}}{%
\pgfpathmoveto{\pgfqpoint{0.000000in}{0.000000in}}%
\pgfpathlineto{\pgfqpoint{0.048611in}{0.000000in}}%
\pgfusepath{stroke,fill}%
}%
\begin{pgfscope}%
\pgfsys@transformshift{10.285831in}{6.350098in}%
\pgfsys@useobject{currentmarker}{}%
\end{pgfscope}%
\end{pgfscope}%
\begin{pgfscope}%
\pgfsetbuttcap%
\pgfsetroundjoin%
\definecolor{currentfill}{rgb}{0.000000,0.000000,0.000000}%
\pgfsetfillcolor{currentfill}%
\pgfsetlinewidth{0.803000pt}%
\definecolor{currentstroke}{rgb}{0.000000,0.000000,0.000000}%
\pgfsetstrokecolor{currentstroke}%
\pgfsetdash{}{0pt}%
\pgfsys@defobject{currentmarker}{\pgfqpoint{0.000000in}{0.000000in}}{\pgfqpoint{0.048611in}{0.000000in}}{%
\pgfpathmoveto{\pgfqpoint{0.000000in}{0.000000in}}%
\pgfpathlineto{\pgfqpoint{0.048611in}{0.000000in}}%
\pgfusepath{stroke,fill}%
}%
\begin{pgfscope}%
\pgfsys@transformshift{10.285831in}{6.457213in}%
\pgfsys@useobject{currentmarker}{}%
\end{pgfscope}%
\end{pgfscope}%
\begin{pgfscope}%
\pgfsetbuttcap%
\pgfsetroundjoin%
\definecolor{currentfill}{rgb}{0.000000,0.000000,0.000000}%
\pgfsetfillcolor{currentfill}%
\pgfsetlinewidth{0.803000pt}%
\definecolor{currentstroke}{rgb}{0.000000,0.000000,0.000000}%
\pgfsetstrokecolor{currentstroke}%
\pgfsetdash{}{0pt}%
\pgfsys@defobject{currentmarker}{\pgfqpoint{0.000000in}{0.000000in}}{\pgfqpoint{0.048611in}{0.000000in}}{%
\pgfpathmoveto{\pgfqpoint{0.000000in}{0.000000in}}%
\pgfpathlineto{\pgfqpoint{0.048611in}{0.000000in}}%
\pgfusepath{stroke,fill}%
}%
\begin{pgfscope}%
\pgfsys@transformshift{10.285831in}{6.550000in}%
\pgfsys@useobject{currentmarker}{}%
\end{pgfscope}%
\end{pgfscope}%
\begin{pgfscope}%
\pgfsetbuttcap%
\pgfsetroundjoin%
\definecolor{currentfill}{rgb}{0.000000,0.000000,0.000000}%
\pgfsetfillcolor{currentfill}%
\pgfsetlinewidth{0.803000pt}%
\definecolor{currentstroke}{rgb}{0.000000,0.000000,0.000000}%
\pgfsetstrokecolor{currentstroke}%
\pgfsetdash{}{0pt}%
\pgfsys@defobject{currentmarker}{\pgfqpoint{0.000000in}{0.000000in}}{\pgfqpoint{0.048611in}{0.000000in}}{%
\pgfpathmoveto{\pgfqpoint{0.000000in}{0.000000in}}%
\pgfpathlineto{\pgfqpoint{0.048611in}{0.000000in}}%
\pgfusepath{stroke,fill}%
}%
\begin{pgfscope}%
\pgfsys@transformshift{10.285831in}{6.631844in}%
\pgfsys@useobject{currentmarker}{}%
\end{pgfscope}%
\end{pgfscope}%
\begin{pgfscope}%
\pgfsetbuttcap%
\pgfsetroundjoin%
\definecolor{currentfill}{rgb}{0.000000,0.000000,0.000000}%
\pgfsetfillcolor{currentfill}%
\pgfsetlinewidth{0.803000pt}%
\definecolor{currentstroke}{rgb}{0.000000,0.000000,0.000000}%
\pgfsetstrokecolor{currentstroke}%
\pgfsetdash{}{0pt}%
\pgfsys@defobject{currentmarker}{\pgfqpoint{0.000000in}{0.000000in}}{\pgfqpoint{0.048611in}{0.000000in}}{%
\pgfpathmoveto{\pgfqpoint{0.000000in}{0.000000in}}%
\pgfpathlineto{\pgfqpoint{0.048611in}{0.000000in}}%
\pgfusepath{stroke,fill}%
}%
\begin{pgfscope}%
\pgfsys@transformshift{10.285831in}{6.705056in}%
\pgfsys@useobject{currentmarker}{}%
\end{pgfscope}%
\end{pgfscope}%
\begin{pgfscope}%
\pgftext[x=10.383053in,y=6.656838in,left,base]{\rmfamily\fontsize{10.000000}{12.000000}\selectfont \(\displaystyle 10^{4}\)}%
\end{pgfscope}%
\begin{pgfscope}%
\pgftext[x=10.639806in,y=3.505056in,,top,rotate=90.000000]{\rmfamily\fontsize{10.000000}{12.000000}\selectfont D=}%
\end{pgfscope}%
\begin{pgfscope}%
\pgfsetbuttcap%
\pgfsetmiterjoin%
\pgfsetlinewidth{0.803000pt}%
\definecolor{currentstroke}{rgb}{0.000000,0.000000,0.000000}%
\pgfsetstrokecolor{currentstroke}%
\pgfsetdash{}{0pt}%
\pgfpathmoveto{\pgfqpoint{10.165831in}{0.305056in}}%
\pgfpathlineto{\pgfqpoint{10.165831in}{0.330056in}}%
\pgfpathlineto{\pgfqpoint{10.165831in}{6.680056in}}%
\pgfpathlineto{\pgfqpoint{10.165831in}{6.705056in}}%
\pgfpathlineto{\pgfqpoint{10.285831in}{6.705056in}}%
\pgfpathlineto{\pgfqpoint{10.285831in}{6.680056in}}%
\pgfpathlineto{\pgfqpoint{10.285831in}{0.330056in}}%
\pgfpathlineto{\pgfqpoint{10.285831in}{0.305056in}}%
\pgfpathclose%
\pgfusepath{stroke}%
\end{pgfscope}%
\begin{pgfscope}%
\pgftext[x=10.225831in,y=6.788389in,,base]{\rmfamily\fontsize{12.000000}{14.400000}\selectfont \(\displaystyle \widetilde{K}\)}%
\end{pgfscope}%
\end{pgfpicture}%
\makeatother%
\endgroup%

}
\caption[Different outcomes of a Gaussian kernel approximation]{Different outcomes of a Gaussian kernel approximation. Top row and bottom row correspond to two different outcomes of $\tildeK{\omega}$, which are \emph{different} \acl{OVK}. However when $D$ tends to infinity, the different outcomes fo $\tildeK{\omega}$ yield the samel \acs{OVK}.}
\label{fig:not_Mercer}
\end{figure}
We computed the Gram Matrix of the Gaussian decomposable kernel
\begin{dmath*}
K(x,z)_{ij}=\exp\left(-\frac{1}{2(0.1)^2(x_i - x_j)^2}\right)\Gamma \condition{for $i$, $j\in\mathbb{N}_{250}$.}
\end{dmath*}
We computed a reference function (black line) defined as $(y_1, y_2)^T = f(x_i)=\sum_{j=1}^{250}K(x_i,x_j)u_j$ where $u_j\sim\mathcal{N}(0,1)$ \iid. We took $\Gamma=.5 I_2 + .5 1_2$ such that the outputs $y_1$ and $y_2$ share some similarities. Then we computed an approximate kernel matrix $\tildeK{\omega}\approx K$ for $25$ increasing values of $D$ ranging from $1$ to $10^4$. The two graphs on the top row shows that the more the number of features increase the closer the model $\widetilde{f}(x_i)=\sum_{j=1}^{250}\tildeK{\omega}(x_i,x_j)u_j$ is to $f$. The bottom row shows the same experiment but for a different realization of $\tildeK{\omega}$. When $D$ is small the curves of the bottom and top rows are very dissimilar --and sine wave like-- while they both converge to $f$ when $D$ increase.
\paragraph{}
In the same way we defined an \acs{ORFF}, we can define an approximate feature operator $\tildeW{\omega}$ which maps $\tildeH{\omega}$ onto $\mathcal{H}_{\tildeK{\omega}}$, where \begin{dmath*}
\tildeK{\omega}(x,z)=\tildePhi{\omega}(x)^\adjoint\tildePhi{\omega}(z)
\end{dmath*}
for all $x$, $z\in\mathcal{X}$.
\begin{definition}[Random Fourier feature operator] Let $\seq{\omega}=(\omega_j)_{j=1}^D\in\dual{\mathcal{X}}^D$ and let
\begin{dmath*}
\tildeK{\omega}_e=\frac{1}{D}\sum_{j=1}^D \conj{\pairing{\cdot,\omega_j}}B(\omega_j)B(\omega_j)^*.
\end{dmath*}
We call random Fourier feature operator the linear application $\tildeW{\omega}:\tildeH{\omega}\to \mathcal{H}_{\tildeK{\omega}}$ defined as
\begin{dmath*}
\left(\tildeW{\omega} \theta\right)(x) \colonequals \tildePhi{\omega}(x)^\adjoint \theta =\frac{1}{D}\sum_{j=1}^D \conj{\pairing{x,\omega_j}}B(\omega_j)g(\omega_j)
\end{dmath*}
where $\theta=\frac{1}{\sqrt{D}}\Vect_{j=1}^Dg(\omega_j)\in\tildeH{\omega}$. Then,
\begin{dmath*}
\left(\Ker \tildeW{\omega}\right)^\perp = \lspan\Set{\tildePhi{\omega}(x)y | \forall x\in\mathcal{X},\enskip \forall y\in\mathcal{Y}} \hiderel{\subseteq} \tildeH{\omega}.
\end{dmath*}
\end{definition}
The random Fourier feature operator is useful to show the relations between the random Fourier feature map with the functional feature map defined in \cref{pr:fourier_feature_map}. The relationship between the generic feature map (defined for all \acl{OVK}) the functional feature map (defining a shift-invariant $\mathcal{Y}$-Mercer \acl{OVK}) and the random Fourier feature map is presented in \cref{fig:rel_features}.
\begin{proposition}
\label{pr:phitilde_phi_rel}
For any $g\in \mathcal{H}=L^2(\mathcal{\dual{X}},\probability_{\dual{\Haar},\rho};\mathcal{Y}')$, let
\begin{dmath*}
\theta \colonequals \frac{1}{\sqrt{D}}\Vect_{j=1}^D g(\omega_j), \enskip \omega_j \sim \probability_{\dual{\Haar},\rho} \enskip\text{\iid}.
\end{dmath*}
Then
\begin{propenum}
\item \label{pr:cv_feature_map_1} $\left(\tildeW{\omega} \theta\right)(x)=\tildePhi{\omega}(x)^\adjoint \theta \converges{\asurely}{D\to\infty} \Phi_x^\adjoint g=(Wg)(x)$,
\item \label{pr:cv_feature_map_2} $\norm{\theta}_{\tildeH{\omega}}^2 \converges{\asurely}{D\to\infty} \norm{g}_{\mathcal{H}}^2$,
\end{propenum}
\end{proposition}
\begin{proof}[of \cref{pr:cv_feature_map_1}] since $(\omega_j)_{j=1}^D$ are \iid~random vectors, for all $y\in \mathcal{Y}$ and for all $y'\in\mathcal{Y}'$, $\inner{y, B(\cdot)y'}\in L^2(\dual{\mathcal{X}},\probability_{\dual{\Haar},\rho})$ and $g\in L^2(\dual{\mathcal{X}},\probability_{\dual{\Haar},\rho};\mathcal{Y}')$, from the strong law of large numbers
\begin{dmath*}
(\tildeW{\omega} \theta)(x)=\tildePhi{\omega}(x)^\adjoint \theta=\frac{1}{D}\sum_{j=1}^D \conj{\pairing{x,\omega_j}}B(\omega_j)g(\omega_j), \qquad \omega_j \hiderel{\sim} \probability_{\dual{\Haar},\rho} \enskip \text{\iid} \\
\converges{\asurely}{D\to\infty} \int_{\dual{\mathcal{X}}}\conj{\pairing{x,\omega}}B(\omega)g(\omega)d\probability_{\dual{\Haar},\rho}(\omega)
= (Wg)(x) \hiderel{\colonequals} \Phi_x^\adjoint g.\qquad\ensuremath{\Box}
\end{dmath*}
\end{proof}
\begin{proof}[of \cref{pr:cv_feature_map_2}] again, since $(\omega_j)_{j-1}^D$ are \iid~random vectors and $g\in L^2(\dual{\mathcal{X}},\probability_{\dual{\Haar},\rho};\mathcal{Y}')$, from the strong law of large numbers
\begin{dmath*}
\norm{\theta}^2_{\tildeH{\omega}}=\frac{1}{D}\sum_{j=1}^D\norm{g(\omega_j)}^2_{\mathcal{Y}'}, \qquad \omega_j \hiderel{\sim} \probability_{\dual{\Haar},\rho} \enskip \text{\iid} \\
\converges{\asurely}{D\to\infty} \int_{\dual{\mathcal{X}}} \norm{g(\omega)}_{\mathcal{Y}'}^2d\probability_{\dual{\Haar},\rho}(\omega) \\
= \norm{g}_{L^2\left(\dual{\mathcal{X}}, \probability_{\dual{\Haar},\rho}; \mathcal{Y}'\right)}^2.\qquad\ensuremath{\Box}
\end{dmath*}
\end{proof}
% Hence the sequence of function $\tildef{D}_j\colonequals (\tildePhi{\omega}_{1:j}(\cdot)^\adjoint \theta)_{j=1}^D$ converges almost surely to a function $f\in\mathcal{H}_{(A,\probability_{\dual{\Haar},\rho})}{\scriptstyle\implies} \mathcal{H}_K$.
% \paragraph{}
We write $\tildePhi{\omega}(x)^\adjoint \tildePhi{\omega}(x)\approx K(x,z)$ when $\tildePhi{\omega}(x)^\adjoint \tildePhi{\omega}(x)\converges{\asurely}{} K(x,z)$ in the weak operator topology when $D$ tends to infinity. With mild abuse of notation we say that $\tildePhi{\omega}(x)$ is an approximate feature map of $\Phi_x$ \ie~$\tildePhi{\omega}(x)\approx \Phi_x$, when for all $y'$, $y\in\mathcal{Y}$,
\begin{dmath*}
\inner{y, K(x,z)y'}_{\mathcal{Y}}=\inner{\Phi_x y, \Phi_z y'}_{\mathcal{L^2(\dual{\mathcal{X}},\probability_{\dual{\Haar},\rho};\mathcal{Y'})}}\approx \inner{\tildePhi{\omega}(x)y, \tildePhi{\omega}(x)y'}_{\tildeH{\omega}}\colonequals \inner{y, \tilde{K}(x,z)y'}_{\mathcal{Y}}
\end{dmath*}
where $\Phi_x$ is defined in the sense of \cref{pr:fourier_feature_map}. Then \cref{cr:ORFF-map-kernel} exhibit a construction of an \acs{ORFF} directly from an \acs{OVK}.
\begin{corollary}
\label{cr:ORFF-map-kernel}
If $K(x,z)$ is a shift-invariant $\mathcal{Y}$-Mercer kernel such that for all $y$, $y'\in\mathcal{Y}$, $\inner{y', K_e(\cdot)y}\in L^1(\mathcal{X},\Haar)$. Then
\begin{equation}
\tildePhi{\omega}(x)y= \frac{1}{\sqrt{D}}\Vect_{j=1}^D\pairing{x, \omega_j}B(\omega_j)^\adjoint y, \qquad \omega_j \sim \probability_{\dual{\Haar},\rho} \enskip\text{\iid},
\end{equation}
where $\inner{y, B(\omega)B(\omega)^\adjoint y'}\rho(\omega)=\FT{\inner{y', K_e(\cdot)y}}(\omega)$, is an approximated feature map of $K$.
\end{corollary}
\begin{proof}
Find $(A, \probability_{\dual{\Haar},\rho})$ from \cref{pr:spectral}, find a decomposition of $A(\omega)=B(\omega)B(\omega)^*$ for $\probability_{\dual{\Haar},\rho}$-almost all $\omega$ and apply \cref{pr:ORFF-map}.
\end{proof}
\begin{remark}
We find a decomposition such that for all $j=1, \ldots, D$, $A(\omega_j)=B(\omega_j)B(\omega_j)^\adjoint $ either by exhibiting an analytic closed-form or using a numerical decomposition.
\end{remark}
\Cref{cr:ORFF-map-kernel} allows us to define \cref{alg:ORFF_construction} for constructing \acs{ORFF} from an operator valued kernel.
\SetKwInOut{Input}{Input}
\SetKwInOut{Output}{Output}
\begin{center}
\begin{algorithm2e}[H]\label{alg:ORFF_construction}
	\SetAlgoLined
    \Input{$K(x, z)=K_e(\delta)$ a $\mathcal{Y}$-shift-invariant Mercer kernel such that $\forall y,y'\in\mathcal{Y},$ $\inner{y', K_e(\cdot)y}\in L^1(\mathbb{R}^d, \Haar)$ and $D$ the number of features.}
    \Output{A random feature $\tildePhi{\omega}(x)$ such that $\tildePhi{\omega}(x)^\adjoint \tildePhi{\omega}(z) \approx K(x,z)$}
    \BlankLine
	Define the pairing $\pairing{x, \omega}$ from the \acs{LCA} group $(\mathcal{X}, \groupop)$\;
	Find a decomposition $(B(\omega),\probability_{\dual{\Haar},\rho})$ such that $B(\omega)B(\omega)^\adjoint \rho(\omega)=\IFT{K_e}(\omega)$\;
	Draw $D$ random vectors $(\omega_j)_{j=1}^D$ \iid~from the probability law $\probability_{\dual{\Haar},\rho}$\;
   \Return $\tildePhi{\omega}(x)=\frac{1}{\sqrt{D}}\Vect_{j=1}^D\pairing{x, \omega_j}B(\omega_j)^\adjoint $\;
   \caption{Construction of \acs{ORFF} from \acs{OVK}}
   \label{al:ORFF_construction}
\end{algorithm2e}
\end{center}

\afterpage{
\begin{landscape}
\begin{figure}[htb]
\centering
\resizebox{\textheight}{!}{%
\begin{tikzpicture}
  \tikzstyle{every node}=[font=\Huge]
  \matrix (m) [matrix of math nodes, nodes in empty cells, ampersand replacement=\&, row sep=3em, column sep=4em, minimum width=2em]
  {
     \Phi_x\in\mathcal{L}(\mathcal{Y}\text{;} \mathcal{H}) \& \& \mathcal{Y} \& \Phi_x\in\mathcal{L}\left(\mathcal{Y}\text{;} L^2\left(\dual{\mathcal{X}}, \probability_{\dual{\Haar},\rho}\text{;} \mathcal{Y}'\right)\right) \& \& \mathcal{Y} \& \tildePhi{\omega}(x)\in\mathcal{L}\left(\mathcal{Y}\text{;} \tildeH{\omega}\right) \& \& \mathcal{Y}\\
     \& \& \& \& \& \& \& \& \\
     x\in\mathcal{X} \& \& \& x\in\mathcal{X} \& \& \& x\in\mathcal{X} \& \& \\
      \& \& \& \& \& \& \& \& \\
  };
  \path[-stealth, very thick]
    (m-1-1) edge node [above] {$\Phi_x^\adjoint g$} (m-1-3)
    (m-3-1) edge node [below] {$f$} (m-1-3)
    (m-3-1) edge node [left] {$\Phi$} (m-1-1)

    (m-1-4) edge node [above] {$\Phi_x^\adjoint g$} (m-1-6)
    (m-3-4) edge node [below] {$f$} (m-1-6)
    (m-3-4) edge node [left] {$\Phi$} (m-1-4)

    (m-1-7) edge node [above] {$\tildePhi{\omega}(x)^\adjoint \theta$} (m-1-9)
    (m-3-7) edge node [below] {$\tildef{\omega}$} (m-1-9)
    (m-3-7) edge node [left] {$\tildePhi{\omega}$} (m-1-7)
    ;

    \node[rectangle,above delimiter=\}] (del-top-1) at ($0.5*(m-4-1.south west) + 0.5*(m-4-3.south east)$) {\tikz{\path (m-4-1.north) rectangle (m-4-3.north);}};
    \node[rectangle,above delimiter=\}] (del-top-2) at ($0.5*(m-4-4.south west) + 0.5*(m-4-6.south east)$) {\tikz{\path (m-4-4.north) rectangle (m-4-6.north);}};
    \node[rectangle,above delimiter=\}] (del-top-3) at ($0.5*(m-4-7.south west) + 0.5*(m-4-9.south east)$) {\tikz{\path (m-4-7.north) rectangle (m-4-9.north);}};

   \node[rectangle,above] (ker-top-1) at (-17, 5) {$\Phi_x^\adjoint \Phi_z=K(x, z)$};
   \node[rectangle,above] (ker-top-1) at (-1.5, 5) {$K_e\left(x\groupop z^{-1}\right)$};
   \node[rectangle,above] (ker-top-1) at (14.5, 5) {$\tildeK{\omega}_e\left(x\groupop z^{-1}\right)=\tildePhi{\omega}(x)^\adjoint \tildePhi{\omega}(x)$};

   \node[rectangle,above] (sym-top-1) at (-9.75, 5.25) {$=$};
   \node[rectangle,above] (sym-top-1) at (8.5, 5.25) {$\approx$};

   \draw[very thick] (-20.5,4.5) -- (20.5,4.5);

   \path[-stealth, very thick]
    (del-top-1.south) edge [bend right] node [below, text width=10cm] {\centering\huge Fourier, \\ $\quad\Phi_x(\omega)y=\pairing{x,\omega}B(\omega)^\adjoint y$.} (del-top-2.south)
    (del-top-2.south) edge [bend right] node [below, text width=14cm] {\centering\huge Monte-Carlo, \\ $\tildePhi{\omega}(x)y=\frac{1}{\sqrt{D}}\Vect_{j=1}^D(\Phi_x y)(\omega_j)$, $\omega_j\sim \probability_{\dual{\Haar},\rho}$ \ac{iid}.} (del-top-3.south)
    ;

\end{tikzpicture}
}

\caption[Relationships between feature-maps.]{Relationships between feature-maps. $\tildeH{\omega} = \Vect_{j=1}^D \mathcal{Y}' }$.}
\label{fig:rel_features}
\end{figure}
\end{landscape}
}

\subsection{Examples of Operator Random Fourier Feature maps}
We now give two examples of operator-valued random Fourier feature map when. First we introduce the general form of an approximated feature map for a matrix-valued kernel on the additive group $(\mathbb{R}^d,+)$.
\begin{example}[Matrix-valued kernel on the additive group]\label{ex:additive_group}
In the following, $K(x,z)=K_0(x-z)$ is a $\mathcal{Y}$-Mercer matrix-valued kernel on $\mathcal{X}=\mathbb{R}^d$ invariant w.r.t. the group operation $+$. %
Then the function $\tildePhi{\omega}$ defined as follow is an \acl{ORFF} of $K_{0}$.
\begin{dmath*}
\tildePhi{\omega}(x)y=\frac{1}{\sqrt{D}}\Vect_{j=1}^D\begin{pmatrix}\cos{\inner{x,\omega_j}}B(\omega_j)^\adjoint y \\ \sin{\inner{x,\omega_j}}B(\omega_j)^\adjoint y\end{pmatrix}, \enskip \omega_j \hiderel{\sim} \probability_{\dual{\Haar},\rho} \enskip\text{\iid}.
\end{dmath*}
for all $y\in\mathcal{Y}$.
\end{example}
\begin{proof}
The (Pontryagin) dual of $\mathcal{X}=\mathbb{R}^d$
is $\dual{\mathcal{X}}\cong\mathbb{R}^d$, and the duality pairing is $\pairing{x-z,\omega}=\exp(i\inner{x-z, \omega})$. The kernel approximation yields:
\begin{dmath*}
\tilde{K}(x,z)=\tildePhi{\omega}(x)^\adjoint \tildePhi{\omega}(z)
= \frac{1}{D} \sum_{j=1}^D \begin{pmatrix} \cos{\inner{x,\omega_j}} & \sin{\inner{x,\omega_j}} \end{pmatrix}\begin{pmatrix}\cos{\inner{z,\omega_j}} \\ \sin{\inner{z,\omega_j}} \end{pmatrix} A(\omega_j)
= \frac{1}{D} \sum_{j=1}^D \cos{\inner{x-z,\omega_j}}A(\omega_j) \\
\converges{\asurely}{D\to\infty}\expectation_\rho\left[\cos{\inner{x-z,\omega}}A(\omega)\right]
\end{dmath*}
in the weak operator topology. Since for all $x\in\mathcal{X}$, $\sin\inner{x, \cdot}$ is an odd function and $A(\cdot)\rho(\cdot)$ is even,
\begin{dmath*}
\expectation_\rho\left[\cos{\inner{x-z,\omega}}A(\omega)\right]=\expectation_\rho\left[\exp(-i\inner{x-z,\omega})A(\omega)\right]\hiderel{=}K(x,z).
\end{dmath*}
Hence $\tilde{K}(x,z)\converges{\asurely}{D\to\infty}K(x,z)$.
\end{proof}
In particular we deduce the following features maps for the kernels proposed in \cref{subsec:dec_examples}.
\begin{itemize}
\item For the decomposable gaussian kernel $K_0^{dec,gauss}(\delta)=k_0^{gauss}(\delta)\Gamma$ for all $\delta\in\mathbb{R}^d$, let $BB^\adjoint=\Gamma$. A bounded --and unbounded-- \acs{ORFF} map is
\begin{dmath*}
\tildePhi{\omega}(x)y=\frac{1}{\sqrt{D}}\Vect_{j=1}^D\begin{pmatrix}\cos{\inner{x,\omega_j}}B^\adjoint y\\ \sin{\inner{x,\omega_j}}B^\adjoint y \end{pmatrix}, \enskip \omega_j \hiderel{\sim} \probability_{\mathcal{N}(0,\sigma^{-2}I_d)} \enskip\text{\iid}
=(\tildePhi{\omega}(x)\otimes B^\adjoint)y,
\end{dmath*}
where $\tildePhi{\omega}(x)=\frac{1}{\sqrt{D}}\Vect_{j=1}^D\begin{pmatrix}\cos{\inner{x,\omega_j}} \\ \sin{\inner{x,\omega_j}}\end{pmatrix}$ is a scalar \acs{RFF} map \cite{Rahimi2007}.
\item For the curl-free gaussian kernel, $K_0^{curl,gauss}=-\nabla\nabla^T k_0^{gauss}$ an unbounded \acs{ORFF} map is
\begin{dmath*}
\tildePhi{\omega}(x)y=\frac{1}{\sqrt{D}}\Vect_{j=1}^D\begin{pmatrix}\cos{\inner{x,\omega_j}}\omega_j^T y\\ \sin{\inner{x,\omega_j}}\omega_j^T y\end{pmatrix}, \enskip \omega_j \hiderel{\sim} \probability_{\mathcal{N}(0,\sigma^{-2}I_d)} \enskip\text{\iid}
\end{dmath*}
and a bounded \acs{ORFF} map is
\begin{dmath*}
\tildePhi{\omega}(x) y=\frac{1}{\sqrt{D}}\Vect_{j=1}^D\begin{pmatrix}\cos{\inner{x,\omega_j}}\frac{\omega_j^T}{\norm{\omega_j}} y \\ \sin{\inner{x,\omega_j}}\frac{\omega_j^T}{\norm{\omega_j}} y \end{pmatrix}, \enskip \omega_j \hiderel{\sim} \probability_{\rho} \enskip\text{\iid}.
\end{dmath*}
where $\rho(\omega)=\frac{\sigma^2\norm{\omega}^2}{d}\mathcal{N}(0,\sigma^{-2}I_d)(\omega)$ for all $\omega\in\mathbb{R}^d$.
\item For the divergence-free gaussian kernel $K_0^{div,gauss}(x,z)=(\nabla\nabla^T-\Delta I_d) k_0^{gauss}(x,z)$ an unbounded \acs{ORFF} map is
\begin{dmath*}
\tildePhi{\omega}(x) y=\frac{1}{\sqrt{D}}\Vect_{j=1}^D\begin{pmatrix}\cos{\inner{x,\omega_j}}B(\omega_j)^T y\\ \sin{\inner{x,\omega_j}}B(\omega_j)^T y\end{pmatrix}, \enskip \omega_j \hiderel{\sim} \probability_{\rho} \enskip\text{\iid}
\end{dmath*}
where $B(\omega)=\left(I_d-\frac{\omega^T\omega}{\norm{\omega}}\right)$ and $\rho=\mathcal{N}(0,\sigma^{-2}I_d)$ for all $\omega\in\mathbb{R}^d$. A bounded \acs{ORFF} map is
\begin{dmath*}
\tildePhi{\omega}(x) y=\frac{1}{\sqrt{D}}\Vect_{j=1}^D\begin{pmatrix}\cos{\inner{x,\omega_j}}B(\omega_j)^T y\\ \sin{\inner{x,\omega_j}}B(\omega_j)^T y\end{pmatrix}, \enskip \omega_j \hiderel{\sim} \probability_{\rho} \enskip\text{\iid}
\end{dmath*}
where $B(\omega)=\left(I_d-\frac{\omega^T\omega}{\norm{\omega}^2}\right)$ and $\rho(\omega)=\frac{\sigma^2\norm{\omega}^2}{d}\mathcal{N}(0,\sigma^{-2}I_d)$ for all $\omega\in\mathbb{R}^d$.
\end{itemize}
The second example extends scalar-valued Random Fourier Features on the skewed multiplicative group --described in \cref{subsec:characters} and \cref{subsubsec:skewedchi2}-- to the operator-valued case.
\begin{example}[Matrix-valued kernel on the skewed multiplicative group]
In the following, $K(x,z)=K_{1-c}(x\odot z^{-1})$ is a $\mathcal{Y}$-Mercer matrix-valued kernel on $\mathcal{X}=(-c;+\infty)^d$ invariant \wrt~the group operation\mpar{The group operation $\odot$ is defined in \cref{subsubsec:skewedchi2}.} $\odot$. Then the function $\tildePhi{\omega}$ defined as follow is an \acl{ORFF} of $K_{1-c}$.
\begin{dmath*}
\tildePhi{\omega}(x) y=\frac{1}{\sqrt{D}}\Vect_{j=1}^D\begin{pmatrix}\cos{\inner{\log(x+c),\omega_j}}B(\omega_j)^\adjoint y\\ \sin{\inner{\log(x+c),\omega_j}}B(\omega_j)^\adjoint y\end{pmatrix},
\end{dmath*}
$\omega_j \sim \probability_{\dual{\Haar},\rho}$ iid, for all $y\in\mathcal{Y}$.
\end{example}
\begin{proof}
The dual of $\mathcal{X}=(-c;+\infty)^d$
is $\dual{\mathcal{X}}\cong\mathbb{R}^d$, and the duality pairing is $\pairing{x \odot \myinv{z},\omega}=\exp(i\inner{\log(x\odot z^{-1}+c), \omega})$. Following the proof of \cref{ex:additive_group}, we have
\begin{equation*}
\begin{aligned}
\tilde{K}(x,z)&= \frac{1}{D} \sum_{j=1}^D \cos{\inner*{\log\left(\frac{x+c}{z+c}\right), \omega_j}}A(\omega_j).
\end{aligned}
\end{equation*}
which converges almost surely to
\begin{dmath*}
\expectation_{\rho}[\exp(-i\inner*{\log(x\odot z^{-1}+c)})A(\omega) ]\hiderel{=}\expectation_{\rho}[\conj{\pairing{x\odot z^{-1},\omega}}A(\omega)]\hiderel{=}K(x, z)
\end{dmath*}
when $D$ tends to infinity, in the weak operator topology.
\end{proof}
\begin{itemize}
\item For the skewed-$\chi^2$ decomposable kernel defined as $K_{1-c}^{dec,skewed}(\delta)=k_{1-c}^{skewed}(\delta)\Gamma$ for all $\delta\in\mathcal{X}$, let $BB^*=\Gamma$. A bounded --and unbounded-- \acs{ORFF} map is
\begin{dmath*}
\tildePhi{\omega}(x)y=\frac{1}{\sqrt{D}}\Vect_{j=1}^D\begin{pmatrix}\cos{\inner{\log(x+c),\omega_j}}B^\adjoint y\\ \sin{\inner{\log(x+c),\omega_j}}B^\adjoint y\end{pmatrix}, \enskip \omega_j \hiderel{\sim} \probability_{\rho} \enskip\text{\iid}
=(\tildePhi{\omega}(x) \otimes B^\adjoint)y,
\end{dmath*}
where $\rho=\mathcal{S}(0,2^{-1})$ and $\tildePhi{\omega}(x)=\frac{1}{\sqrt{D}}\Vect_{j=1}^D\begin{pmatrix}\cos{\inner{\log(x+c),\omega_j}} \\ \sin{\inner{\log(x+c),\omega_j}}\end{pmatrix}$ is a scalar \acs{RFF} map \cite{li2010random}.
\end{itemize}
\subsection{Regularization property}
We have shown so far that it is always possible to construct a feature map that allows to approximate a shift-invariant $\mathcal{Y}$-Mercer kernel. However we could also propose a construction of such map by studying the regularization induced with respect to the \acl{FT} of a target function $f\in \mathcal{H}_K$. In other words, what is the norm in $L^2(\dual{\mathcal{X}}, \dual{\Haar}; \mathcal{Y}')$ induced by $\norm{\cdot}_K$?
\begin{proposition}
Let $K$ be a shift-invariant $\mathcal{Y}$-Mercer Kernel such that for all $y$, $y'$ in $\mathcal{Y}$, $\inner{y', K_e(\cdot)y}\in L^1(\mathcal{X}, \Haar)$. Then for all $f\in\mathcal{H}_K$
\begin{dmath}
\norm{f}^2_K=\displaystyle\int_{\dual{\mathcal{X}}}\frac{\inner*{\FT{f}(\omega), A\left(\omega\right)^\dagger\FT{f}(\omega)}_{\mathcal{Y}}}{\rho(\omega)}d\dual{\Haar}(\omega).
\label{eq:reg_L2}
\end{dmath}
where $\inner{y', A(\omega)y}\rho(\omega)\colonequals\FT{\inner{y', K_e(\cdot)y}}(\omega)$.
\label{pr:regularization}
\end{proposition}
\begin{proof}
We first show how the \acl{FT} relates to the feature operator. Since $\mathcal{H}_K$ is embed into $\mathcal{H}=L^2(\dual{\mathcal{X}}, \probability_{\dual{\Haar},\rho}; \mathcal{Y})$ by mean of the feature operator $W$, we have for all $f\in\mathcal{H}_k$, for all $f\in\mathcal{H}$ and for all $x\in\mathcal{X}$
\begin{dgroup*}
\begin{dmath*}
\FT{\IFT{f}}(x)\hiderel{=}\int_{\dual{\mathcal{X}}}\overline{\pairing{x,\omega}}\IFT{f}(\omega)d\dual{\Haar}(\omega) = f(x)
\end{dmath*}
\begin{dmath*}
(Wg)(x)\hiderel{=}\int_{\dual{\mathcal{X}}}\conj{\pairing{x,\omega}}\rho(\omega)B(\omega)g(\omega)d\dual{\Haar}(\omega) = f(x).
\end{dmath*}
\end{dgroup*}
By injectivity of the \acl{FT}, $\IFT{f}(\omega)=\rho(\omega)B(\omega)g(\omega)$. From \cref{pr:feature_operator} we have
\begin{dmath*}
\norm{f}^2_{K} = \inf \Set{\norm{g}^2_{\mathcal{H}} | \forall g\hiderel{\in}\mathcal{H}, \enskip Wg\hiderel{=}f} = \inf \Set{\int_{\dual{\mathcal{X}}} \norm{g}^2_{\mathcal{Y}}d\probability_{\dual{\Haar},\rho} | \forall g\hiderel{\in}\mathcal{H},\enskip \IFT{f}\hiderel{=}\rho(\cdot)B(\cdot)g(\cdot)}.
\end{dmath*}
The pseudo inverse of the operator $B(\omega)$ -- noted $B(\omega)^\dagger$ -- is the unique solution of the system $\IFT{f}(\omega)=\rho(\omega)B(\omega)g(\omega)$ \wrt~$g(\omega)$ with minimal norm\mpar{Note that since $B(\omega)$ is bounded the pseudo inverse of $B(\omega)$ is well defined for $\dual{\Haar}$-almost all $\omega$. However if $B(\omega)$ is infinite dimensional, the pseudo inverse is continuous if and only if $A(\omega)$ has closed range. This is always true if $\mathcal{Y}$ is finite dimensional.}. Eventually,
\begin{dmath*}
\norm{f}^2_K = \int_{\dual{\mathcal{X}}} \frac{\norm{B(\omega)^\dagger\IFT{f}(\omega)}_{\mathcal{Y}}^2}{\rho(\omega)^2}d\probability_{\dual{\Haar},\rho}(\omega)
\end{dmath*}
Using the fact that $\IFT{\cdot}=\mathcal{F}\mathcal{R}[\cdot]$ and $\mathcal{F}^2[\cdot]=\mathcal{R}[\cdot]$,
\begin{dmath*}
\norm{f}^2_K= \displaystyle\int_{\dual{\mathcal{X}}} \frac{\norm{\mathcal{R}\left[B(\cdot)^\dagger\rho(\cdot)\right](\omega)\FT{f}(\omega)}^2_{\mathcal{Y}}}{\rho(\omega)^2}d\dual{\Haar}(\omega)
= \displaystyle\int_{\dual{\mathcal{X}}} \frac{\norm{B(\omega)^\dagger\rho(\omega)\FT{f}(\omega)}^2_{\mathcal{Y}}}{\rho(\omega)^2}d\dual{\Haar}(\omega)
= \displaystyle\int_{\dual{\mathcal{X}}} \frac{\inner{B(\omega)^\dagger\FT{f}(\omega),B(\omega)^\dagger\FT{f}(\omega)}_{\mathcal{Y}}}{\rho(\omega)}d\dual{\Haar}(\omega)
= \displaystyle\int_{\dual{\mathcal{X}}} \frac{\inner{\FT{f}(\omega),A(\omega)^\dagger\FT{f}(\omega)}_{\mathcal{Y}}}{\rho(\omega)}d\dual{\Haar}(\omega)
\end{dmath*}
\end{proof}
Note that if $K(x,z)=k(x,z)$ is a scalar kernel then for all $\omega$ in $\dual{\mathcal{X}}$, $A(\omega)=1$. Therefore we recover a well known results for kernels that is for any $f\in\mathcal{H}_k$ we have $\norm{f}_k=\int_{\dual{\mathcal{X}}}\FT{k_e}(\omega)^{-1}\FT{f}(\omega)^2d\dual{\Haar}(\omega)$ \citep{Yang2012,vertregularization,smola1998connection}. We also note that the regularization property in $\mathcal{H}_K$ does not depends (as expected) on the decomposition of $A(\omega)$ into $B(\omega)B(\omega)^\adjoint $. Therefore the decomposition should be chosen such that it optimizes the computation cost. For instance if $A(\omega)\in\mathcal{L}(\mathbb{R}^p)$ has rank $r$, one could find an operator $B(\omega)\in\mathcal{L}(\mathbb{R}^p, \mathbb{R}^r)$ such that $A(\omega)=B(\omega)B(\omega)^\adjoint$. Moreover, in light of \cref{pr:regularization} the regularization property of the kernel with respect to the \acl{FT}, it is also possible to define an approximate feature map of an \acl{OVK} from its regularization properties in the \acs{vv-RKHS} as proposed in \cref{alg:ORFF2_construction}.
\SetKwInOut{Input}{Input}
\SetKwInOut{Output}{Output}
\begin{center}
\begin{algorithm2e}[H]\label{alg:ORFF2_construction}
    \SetAlgoLined
    \Input{
    \begin{itemize}
    \item The pairing $\pairing{x, \omega}$ of the \acs{LCA} group $(\mathcal{X}, \groupop)$.
    \item A probability measure $\probability_{\dual{\Haar},\rho}$ with density $\rho$ \wrt~the haar measure $\dual{\Haar}$ on $\dual{\mathcal{X}}$.
    \item An operator-valued function $B:\dual{\mathcal{X}}\to\mathcal{L}(\mathcal{Y},\mathcal{Y}')$ such that for all $y$ $y'\in\mathcal{Y}$, $\inner{y', B(\cdot)B(\cdot)^\adjoint y}\in L^1(\dual{\mathcal{X}},\probability_{\dual{\Haar},\rho})$.
    \item $D$ the number of features.
    \end{itemize}}
    \Output{A random feature $\tildePhi{\omega}(x)$ such that $\tildePhi{\omega}(x)^\adjoint \tildePhi{\omega}(z) \approx K_e(x\groupop z^{-1})$.}
    \BlankLine
    Draw $D$ random vectors $(\omega_j)_{j=1}^D$ \iid~from the probability law $\probability_{\dual{\Haar},\rho}$\;
    \Return $\tildePhi{\omega}(x)=\frac{1}{\sqrt{D}}\Vect_{j=1}^D\pairing{x, \omega_j}B(\omega_j)^\adjoint $\;
   \caption{Construction of \acs{ORFF}}
   \label{al:ORFF_construction}
\end{algorithm2e}
\end{center}
\subsection{Operator Random Feature engineering}
As in the scalar case, it is possible to construct. We list some examples in the following.
\subsubsection{Sum of kernels}
\begin{proposition}[Sum of kernels]
Let $I$ be a countable set and let $(K^i)_{i\in I}$ be a familly of $\mathcal{Y}$-reproducing kernels such that for all $y\in\mathcal{Y}$
\begin{dmath*}
\sum_{i\in I}\inner{y, K^i(x,x)y} < \infty.
\end{dmath*}
Given $x$, $z\in\mathcal{X}$, the serie $\sum_{i\in I}K^i(x,z)$ converges to a bounded operator $K(x,z)$ in the strong operator topology, and the map $K:\mathcal{X}\times\mathcal{X}\to\mathcal{L}(\mathcal{Y})$ defined by
\begin{dmath*}
K(x,z)y=\sum_{i\in I}K^{i}(x,z)y
\end{dmath*}
is a $\mathcal{Y}$-reproducing kernel. The corresponding space $\mathcal{H}_K$ is embedded in $\oplus_{i\in I} \mathcal{H}_{K^i}$ by means of the feature operator
\begin{dmath*}
(Wf)(x)=\sum_{i\in I} f_i
\end{dmath*}
where $f=\oplus_{i\in I} f_i$, $f_i\in\mathcal{H}_{K^i}$, and the sum converges in norm. Moreover if each $K^i$ is a Mercer kernel --resp. $\mathcal{C}_0$-kernel-- and $x\mapsto \sum_{i\in I}\norm{K^i(x,x)}_{\mathcal{Y},\mathcal{Y}}$ is locally bounded --resp. bounded-- then $K$ is Mercer --resp. $\mathcal{C}_0$.
\end{proposition}

% Let $\Omega=\Set{(\omega_j)_{j=1}^D | \omega_j \sim \probability_{\dual{\Haar}, \rho}\enskip \text{\iid}}$. Then
% \begin{dmath*}
% \inner{y', K(x,z)y}=\frac{1}{D}\sum_{\omega\in\Omega} \inner{\Phi_xy
% (\omega), \Phi_xy(\omega)}
% \end{dmath*}

% \begin{corollary}

% \end{corollary}

%----------------------------------------------------------------------------------------
\section{Learning with ORFF}
\label{sec:learning_with_operator-valued_random-fourier_features}
We now turn our attention to learning function with an ORFF model that approximate an OVK model.
\subsection{Warm-up: supervised regression}
Let $\seq{s} = (x_i,y_i)_{i=1}^N\in\left(\mathcal{X}\times\mathcal{Y}\right)^N$ be a sequence of training samples. Given a local loss function $L: \mathcal{X}\times\mathcal{F}\times\mathcal{Y}\to \overline{\mathbb{R}}$ such that $L$ is proper, convex and lower semi-continous in $f$, we are interested in finding a \emph{vector-valued function} $f_{\seq{s}}:\mathcal{X}\to\mathcal{Y}$, that lives in a RKHS and minimize a tradeoff between a data fitting term $L$ and a regularization term to prevent from overfitting. Namely finding $f_{\seq{s}}\in\mathcal{H}_K$ such that
\begin{dmath}
f_{\seq{s}} = \argmin_{f\in\mathcal{H}_K}  \frac{1}{N}\displaystyle\sum_{i=1}^NL(x_i, f, y_i) + \frac{\lambda}{2}\norm{f}^2_{K},
\label{eq:learning_rkhs}
\end{dmath}
where $\lambda\in\mathbb{R}_+$ is a regularization\mpar{Tychonov regularization.} parameter. A common choice of data fitting term for regression is $L:(x_i, f, y_i) \mapsto \norm{f(x_i)-y_i}_{\mathcal{Y}}^2$.
We introduce a corollary from Mazur and Schauder proposed in 1936 (see \citet{kurdila2006convex, gorniewicz1999topological}) showing that \cref{eq:learning_rkhs} --and \cref{eq:learning_rkhs_gen}-- attains a unique mimimizer.
\begin{theorem}[Mazur-Schauder]
\label{cor:unique_minimizer}
Let $\mathcal{H}$ be a Hilbert space and $J:\mathcal{H}\to \overline{\mathbb{R}}$ be a proper, convex, lower semi-continuous and coercive function. Then $J$ is bounded from below and attains a minimizer. Moreover if $J$ is strictly convex the minimizer is unique.
\end{theorem}
This is easily verified for Ridge regression. Define
\begin{dmath}
\label{eq:ridge}
J_\lambda(f)=\frac{1}{2N}\sum_{i=1}^N\norm{f(x_i)-y_i}_{\mathcal{Y}}^2+\frac{\lambda}{2}\norm{f}_K^2,
\end{dmath}
where $f\in\mathcal{H}_K$ and $\lambda\in\mathbb{R}_{>0}$. $J_\lambda$ is continuous\mpar{Reminder, if $f\in\mathcal{H}_k, \text{ev}_x:f\mapsto f(x)$ is continuous, see \cref{pr:unique_rkhs}.} and strictly convex. Additionally $J_\lambda$ is coercive since $\norm{f}_K$ is coercive, $\lambda\in\mathbb{R}_{>0}$, and all the summands of $J_\lambda$ are positive. Hence for all positive $\lambda$, $f_{\seq{s}}=\argmin_{f\in\mathcal{H}_K}J_\lambda(f)$ exists, is unique and attained.
\begin{remark}
\label{rk:rkhs_bound}
We condider the optimization problem proposed in \cref{eq:ridge} where $L:(x_i, f, y_i) \mapsto \norm{f(x_i)-y_i}_{\mathcal{Y}}^2$. If given a training sample $\seq{s}$, we have
\begin{dmath*}
\frac{1}{N}\sum_{i=1}^N\norm{y_i}_{\mathcal{Y}}^2 \le \sigma_y^2,
\end{dmath*}
then $\lambda\norm{f_{\seq{s}}}_K\le 2\sigma_y^2$. Indeed, since $\mathcal{H}_K$ is a Hilbert space, $0\in\mathcal{H}_K$, thus
\begin{dmath*}
\frac{\lambda}{2}\norm{f_{\seq{s}}}^2_{K} \le \frac{1}{N}\displaystyle\sum_{i=1}^NL(x_i, f_{\seq{s}}, y_i) + \frac{\lambda}{2}\norm{f_{\seq{s}}}^2_{K}
\le \frac{1}{N}\displaystyle\sum_{i=1}^NL(x_i, 0, y_i) \hiderel{\le} \sigma_y^2 \condition{by optimality of $f_{\seq{s}}$.}
\end{dmath*}
Since for all $x\in\mathcal{X}$, $\norm{f(x)}_{\mathcal{Y}}\le \sqrt{\norm{K(x, x)}_{\mathcal{Y},\mathcal{Y}}}\norm{f}_{K}$, the maximum value that the solution $\norm{f_{\seq{s}}(x)}_{\mathcal{Y}}$ of \cref{eq:ridge} can reach is $2\sqrt{\norm{K(x, x)}}\frac{\sigma_y^2}{\lambda}$. Thus when solving a Ridge regression problem, given a shift-invariant kernel $K_e$, one should choose
\begin{dmath*}
0 \hiderel{<} \lambda \hiderel{\le} 2\frac{\sqrt{\norm{K_e(e)}}\sigma_y^2}{C}.
\end{dmath*}
with $C\in\mathbb{R}_{>0}$ to have a chance to fit all the $y_i$ with norm $\norm{y_i}_{\mathcal{Y}} \le C$ in the train set.
\end{remark}
\subsection{Semi-supervised regression}
Regression in \acl{vv-RKHS} has been well studied \citep{Alvarez2012, Minh_icml13,minh2016unifying,sangnier2016joint,kadri2015operator,Micchelli2005,Brouard2016_jmlr}, and a cornerstone of learning in \acs{vv-RKHS} is the representer theorem\mpar{Sometimes referred to as minimal norm interpolation theorem.}, which allows to replace the search of a minimizer in a infinite dimensional \acs{vv-RKHS} by a finite number of paramaters $(u_i)_{i=1}^N$, $u_i\in\mathcal{Y}$. We present here the very genreal form of \citet{minh2016unifying}. This framework encompass Vector-valued Manifold Regularization \citep{belkin2006manifold,Brouard2011,minh2013unifying} and Co-regularized Multi-view Learning \citep{brefeld2006efficient,sindhwani2008rkhs,rosenberg2009kernel,sun2011multi}.
\paragraph{}
In the following we suppose we are given a cost function $c:\mathcal{Y}\times\mathcal{Y}\to\overline{\mathbb{R}}$, such that $c(f(x),y)$ returns the error of the prediction $f(x)$ \wrt~the ground truth $y$. A loss function of a model $f$ with respect to an example $(x,y)\in\mathcal{X}\times\mathcal{Y}$ can be naturally defined from a cost function as $L(x,f,y)=c(f(x),y)$. Conceptually the function $c$ evaluate the quality of the prediction versus its ground truth $y\in\mathcal{Y}$ while the loss function $L$ evaluate the quality of the model $f$ at a training point $(x,y)\in\mathcal{X}\times\mathcal{Y}$. Moreover we suppose that we are given a training sample $\seq{u}=(x_i)_{i=N}^{N+U}\in\mathcal{X}^U$ of unlabelled exemple. We note $\seq{z}\in\left(\mathcal{X}\times\mathcal{Y}\right)^N\times\mathcal{X}^U$ the sequence $\seq{s}\seq{u}$ concatenating both labeled ($\seq{s}$) and unlabelled ($\seq{u}$) training examples.
\begin{theorem}[Representer \citep{minh2016unifying}]
\label{th:representer}
Let $K$ be a $\mathcal{W}$-Mercer \acl{OVK} and $\mathcal{H}_K$ its corresponding $\mathcal{W}$-Reproducing Kernel Hilbert space.
\paragraph{}
Let $V:\mathcal{W}\to\mathcal{Y}$ be a bounded linear operator and let $c:\mathcal{Y}\times\mathcal{Y}\to\overline{\mathbb{R}}$ be a cost function such that $L(x, f, y)=c(Vf(x), y)$ is a proper convex lower semi-continuous function in $f$ for all $x\in\mathcal{X}$ and all $y\in\mathcal{Y}$.
\paragraph{}
Eventually let $\lambda_K\in\mathbb{R}_{>0}$ and $\lambda_M \in \mathbb{R}_+$ be two regularization hyperparameters and $(M_{ik})_{i,k=1}^{N+U}$ be a sequence of data dependent bounded linear operators in $\mathcal{L}(\mathcal{W})$, such that
\begin{dmath*}
\sum_{i,j=1}^{N+U} \inner{w_i, M_{ik}w_k} \ge 0 \condition{$\forall (w_i)_{i=1}^{N+U}\in\mathcal{W}^{N+U}$ and $M_{ik}=M_{ki}^*$}.
\end{dmath*}
The solution $f_{\seq{z}}\in\mathcal{H}_K$ of the regularized optimization problem
\begin{dmath}
f_{\seq{z}} = \argmin_{f\in\mathcal{H}_K} \frac{1}{N}\displaystyle\sum_{i=1}^N c(Vf(x_i), y_i) + \frac{\lambda_K}{2}\norm{f}^2_{K} \\ + \frac{\lambda_M}{2(N+U)}\sum_{i,k=1}^{N+U}\inner{f(x_i), M_{ik}f(x_k)}_{\mathcal{W}}
\label{eq:learning_rkhs_gen}
\end{dmath}
has the form
\begin{dmath*}
f_{\seq{z}}=\sum_{i=1}^{N+U}K(\cdot,x_i)u_i \condition{where $u_i\in\mathcal{Y}$}.
\end{dmath*}
\end{theorem}
The first representer theorem was first introduced by \citet{Wahba90} in the case where $\mathcal{Y}=\mathbb{R}$. The extension to an arbitrary Hilbert space $\mathcal{Y}$ has been proved by many authors in different forms \citep{Brouard2011,kadri2015operator,Micchelli2005}. We present here the proof of the generic formulation proposed by \citet{minh2016unifying}. In the mean time we clarify some elements of the proof. Indeed the existence of a global minimizer is not trivial and we must invoke the Mazur-Schauder theorem. Moreover the coercivity of the objective function required by the Mazur-Schauder theorem is not obvious when we do not require the cost function to take only positive values. However a corollary of Hahn-Banach theorem linking strong convexity to coercivity gives the solution.
\begin{proof}
Since $f(x)=K_x^*f$ (see \cref{eq:reproducing_prop}), the optimization problem reads
\begin{dmath*}
f_{\seq{z}} = \argmin_{f\in\mathcal{H}_K} \frac{1}{N}\displaystyle\sum_{i=1}^N c(VK_{x_i}^\adjoint f, y_i) + \frac{\lambda_K}{2}\norm{f}^2_{K} \\ + \frac{\lambda_M}{2(N+U)}\sum_{i,k=1}^{N+U}\inner{K_{x_i}^\adjoint f, M_{ik}K_{x_k}^\adjoint f}_{\mathcal{W}}
\end{dmath*}
Let $E_{V,\seq{s}}:\mathcal{H}_K\to\Vect_{i=1}^N\mathcal{Y}$ be a linear operator defined as
\begin{dmath*}
E_{V,\seq{s}}f = \Vect_{i=1}^N CK_{x_i}^\adjoint f,
\end{dmath*}
with $VK_{x_i}^\adjoint:\mathcal{H}_K\to\mathcal{Y}$ and $K_{x_i}V^\adjoint:\mathcal{Y}\to\mathcal{H}_K$. Let $Y=\vect_{i=1}^Ny_i\in\mathcal{Y}^N$. We have
\begin{dmath*}
\inner{Y, E_{V,\seq{s}}f}_{\Vect_{i=1}^N\mathcal{Y}}=\sum_{i=1}^N\inner{y_i, VK_{x_i}^\adjoint f}_{\mathcal{Y}}
\hiderel{=}\sum_{i=1}^N\inner{K_{x_i}V^\adjoint y_i, f}_{\mathcal{H}_K}.
\end{dmath*}
Thus the adjoint operator $E_{V,\seq{s}}^\adjoint:\Vect_{i=1}^N\mathcal{Y}\to\mathcal{H}_K$ is
\begin{dmath*}
E_{V,\seq{s}}^\adjoint Y=\sum_{i=1}^NK_{x_i}V^\adjoint y_i,
\end{dmath*}
and the operator $E_{V,\seq{s}}^*E_{V,\seq{s}}:\mathcal{H}_K\to\mathcal{H}_K$ is
\begin{dmath*}
E_{V,\seq{s}}^\adjoint E_{V,\seq{s}}f = \sum_{i=1}^NK_{x_i}V^\adjoint VK_{x_i}^\adjoint f
\end{dmath*}
where $V^\adjoint V\in\mathcal{L}(\mathcal{W})$. Let
\begin{dmath*}
J_{\lambda_K}(f) = \underbrace{\frac{1}{N}\displaystyle\sum_{i=1}^N c(Vf(x_i), y_i)}_{=J_c} + \frac{\lambda_K}{2}\norm{f}^2_{K} \\ + \underbrace{\frac{\lambda_M}{2(N+U)}\sum_{i,k=1}^{N+U}\inner{f(x_i), M_{ik}f(x_k)}_{\mathcal{W}}}_{=J_M}
\end{dmath*}
To ensure that $J_{\lambda_K}$ has a global minimizer we need the following technical lemma (which is a consequence of the Hahn-Banach theorem for lower-semicontimuous functional, see \cite{kurdila2006convex}).
\begin{lemma}
\label{lm:strongly_convex_is_coercive}
Let $J$ be a proper, convex, lower semi-continuous functional, defined on a Hilbert space $\mathcal{H}$. If $J$ is strongly convex, then $J$ is coercive.
\end{lemma}
\begin{proof}
Consider the convex function function $G(f)\colonequals J(f)-\lambda\norm{f}^2$, for some $\lambda>0$. Since $J$ is by assumption proper, lower semi-continuous and strongly convex with parameter $\lambda$, $G$ is proper, lower semi-continuous and convex. Thus Hahn-Banach theorem apply, stating that $G$ is bounded by below by an affine functional. \Ie~there exists $f_0$ and $f_1\in\mathcal{H}$ such that
\begin{dmath*}
G(f)\ge G(f_0) + \inner{f - f_0, f_1} \condition{for all $f\in\mathcal{H}$.}
\end{dmath*}
Then substitute the definition of $G$ to obtain
\begin{dmath*}
J(f)\ge J(f_0) + \lambda\left(\norm{f}-\norm{f_0}\right) + \inner{f - f_0, f_1}.
\end{dmath*}
By the Cauchy-Schwartz inequality, $\inner{f, f_1}\ge - \norm{f}\norm{f_1}$, thus
\begin{dmath*}
J(f)\ge J(f_0) + \lambda\left(\norm{f}-\norm{f_0}\right)  - \norm{f}\norm{f_1} - \inner{f_0, f_1},
\end{dmath*}
which tends to infinity as $f$ tends to infinity. Hence $J$ is coercive
\end{proof}
for all $f\in\mathcal{H}_K$. Since $c$ is proper, lower semi-continuous and convex by assumption, thus the term $J_c$ is also proper, lower semi-continuous and convex. Moreover the term $J_M$ is always positive for any $f\in\mathcal{H}_K$ and $\frac{\lambda_K}{2}\norm{f}^2_{K}$ is strongly convex. Thus $J_{\lambda_K}$ is strongly convex. Apply \cref{lm:strongly_convex_is_coercive} to obtain the coercivity of $J_{\lambda_K}$, and then \cref{cor:unique_minimizer} to show that $J_{\lambda_K}$ has a unique minimizer and is attained. Then let $\mathcal{H}_{K, \seq{z}}=\Set{\sum_{i=1}^{N+U}K_{x_i}u_i| \forall u_i \in\mathcal{W}^{N+U}}$. For $f\in\mathcal{H}_{K, \seq{z}}^\perp$\mpar{$\mathcal{H}_{K, \seq{z}}^\perp\oplus\mathcal{H}_{K, \seq{z}}=\mathcal{H_K}$.}, the operator $E_{V,\seq{s}}$ satisfies
\begin{dmath*}
\inner{Y, E_{V,\seq{s}}f}_{\Vect_{i=1}^N\mathcal{Y}} = \inner{\underbrace{f}_{\in\mathcal{H}_{K, \seq{z}}^\perp}, \underbrace{\sum_{i=1}^{N+U}K_{x_i}V^\adjoint y_i}_{\in\mathcal{H}_{K, \seq{z}}}}_{\mathcal{H}_K} \hiderel{=} 0
\end{dmath*}
for all sequences $(y_i)_{i=1}^N$, since $V^\adjoint y_i\in\mathcal{W}$. Hence,
\begin{dmath}
\label{eq:null1}
(Vf(x_i))_{i=1}^{N}=0
\end{dmath}
In the same way,
\begin{dmath*}
\sum_{i=1}^{N+U}\inner{K_{x_i}^* f, u_i}_{\mathcal{W}} \hiderel{=} \inner{\underbrace{f}_{\in\mathcal{H}_{K, \seq{z}}^\perp}, \underbrace{\sum_{i=1}^{N+U}K_{x_i}u_i}_{\in\mathcal{H}_{K, \seq{z}}}}_{\mathcal{H}_K} \hiderel{=} 0.
\end{dmath*}
for all sequences $(u_i)_{i=1}^{N+U}\in\mathcal{W}^{N+U}$. As a result,
\begin{dmath}
\label{eq:null2}
(f(x_i))_{i=1}^{U+N}=0.
\end{dmath}
Now for an arbitrary $f\in\mathcal{H_K}$, consider the orthogonal decomposition $f=f^{\perp}+f^{\parallel}$, where $f^{\perp}\in\mathcal{H}_{K, \seq{z}}^\perp$ and $f^{\parallel}\in\mathcal{H}_{K, \seq{z}}$. Then since $\norm{f^{\perp}+f^{\parallel}}_{\mathcal{H}_K}^2=\norm{f^{\perp}}_{\mathcal{H}_K}^2+\norm{f^{\parallel}}_{\mathcal{H}_K}^2$, \cref{eq:null1} and \cref{eq:null2} shows that if $\lambda_K\ge 0$, clearly then
\begin{dmath*}
J_{\lambda_K}(f)=J_{\lambda_K}\left(f^{\perp}+f^{\parallel}\right) \hiderel{\ge} J_{\lambda_K}\left(f^{\parallel}\right)
\end{dmath*}
The last inequality holds only when $\norm{f^{\perp}}_{\mathcal{H}_K}=0$, that is when $f^{\perp}=0$. As a result since the minimizer of $J_{\lambda_K}$is unique and attained, it must lies in $\mathcal{H}_{K, \seq{z}}$.
\end{proof}
The representer theorem show that minimizing a functional in a \acs{vv-RKHS} yields a solution which depends on all the points in the training set. Assuming that for all $x_i$, $x\in\mathcal{X}$ and for all $u_i\in\mathcal{Y}$ it takes time $O(P)$, to compute $K(x_i, x)u_i$, making a prediction using the representer theorem take $O(2(N+U)P)$. Obviously If $\mathcal{Y}=\mathbb{R}^p$, Then $P=O(p^2)$ thus making a prediction cost $O(2(N+U)p^2)$ operations.
\paragraph{}
Instead learning a model $f$ that depends on all the points of the training set, we would like to learn a parametric model of the form
$\tildef{1:D}(x)=\tildePhi{\omega}(x)^\adjoint \theta$, where $\theta$ lives in some redescription space\mpar{If $V=1$, $\mathcal{W}=\mathbb{R}$ then $\mathcal{W}'=\mathbb{R}$ thus $\theta\in \mathbb{R}^D$ endowed with the euclidean inner product.} $\tildeH{\omega}\subseteq \Vect_{j=1}^D\mathcal{W}'$. We are interested in finding a parameter vector $\theta_{\seq{z}}$ such that
\begin{dmath}
\label{eq:argmin_applied}
\theta_{\seq{z}}=\argmin_{\theta\in \Vect_{j=1}^D\mathcal{W}'} \frac{1}{2N}\sum_{i=1}^Nc\left(V\tildePhi{\omega}(x_i)^\adjoint \theta, y_i\right) + \frac{\lambda_K}{2}\norm{\theta}^2_{\Vect_{j=1}^D\mathcal{W}'} \\ + \frac{\lambda_M}{2(N+U)}\sum_{i,k=1}^{N+U}\inner{\theta, \tildePhi{\omega}(x_i)M_{ik}\tildePhi{\omega}(x_k)^\adjoint \theta}_{\Vect_{j=1}^D\mathcal{W}'}
\end{dmath}

\begin{corollary}[\acs{ORFF} representer]
Let $\tildeK{\omega}$ be an \acl{OVK} such that for all $x$, $z\in\mathcal{X}$, $\tildePhi{\omega}(x)^\adjoint \tildePhi{\omega}(z) \approx K(x,z)$ where $K$ is a $\mathcal{W}$-Mercer \acs{OVK} and $\mathcal{H}_{\tildeK{\omega}}$ its corresponding $\mathcal{W}$-Reproducing kernel Hilbert space.
\paragraph{}
Let $V:\mathcal{W}\to\mathcal{Y}$ be a bounded linear operator and let $c:\mathcal{Y}\times\mathcal{Y}\to\overline{\mathbb{R}}$ be a cost function such that $L(x, \widetilde{f}, y)=c(V\widetilde{f}(x), y)$ is a proper convex lower semi-continuous function in $\widetilde{f}\in\mathcal{H}_{\tildeK{\omega}}$ for all $x\in\mathcal{X}$ and all $y\in\mathcal{Y}$.
\paragraph{}
Eventually let $\lambda_K\in\mathbb{R}_{>0}$ and $\lambda_M \in \mathbb{R}_+$ be two regularization hyperparameters and $(M_{ik})_{i,k=1}^{N+U}$ be a sequence of data dependent bounded linear operators in $\mathcal{L}(\mathcal{W})$, such that
\begin{dmath*}
\sum_{i,j=1}^{N+U} \inner{w_i, M_{ik}w_k} \ge 0 \condition{$\forall (w_i)_{i=1}^{N+U}\in\mathcal{W}^{N+U}$ and $M_{ik}=M_{ki}^*$}.
\end{dmath*}
The solution $f_{\seq{z}}\in\mathcal{H}_{\tildeK{\omega}}$ of the regularized optimization problem
\begin{dmath}
\label{eq:argmin_RKHS_rand}
\widetilde{f}_{\seq{z}} = \argmin_{\widetilde{f}\in\mathcal{H}_{\tildeK{\omega}}} \frac{1}{N}\displaystyle\sum_{i=1}^N c\left(V\widetilde{f}(x_i), y_i\right) + \frac{\lambda_K}{2}\norm{\widetilde{f}}^2_{\tildeK{\omega}} \\ + \frac{\lambda_M}{2(N+U)}\sum_{i,k=1}^{N+U}\inner{\widetilde{f}(x_i), M_{ik}\widetilde{f}(x_k)}_{\mathcal{W}}
\end{dmath}
has the form $\widetilde{f}_{\seq{z}}=\tildePhi{\omega}(\cdot)^\adjoint\theta_{\seq{z}}$, where $\theta_{\seq{z}}\in (\Ker \tildeW{\omega})^{\perp}$ and
\begin{dmath}
\theta_{\seq{z}}=\argmin_{\theta\in \Vect_{j=1}^D\mathcal{W}'} \frac{1}{2N}\sum_{i=1}^Nc\left(V\tildePhi{\omega}(x_i)^\adjoint \theta, y_i\right) + \frac{\lambda_K}{2}\norm{\theta}^2_{\Vect_{j=1}^D\mathcal{W}'} \\ + \frac{\lambda_M}{2(N+U)}\sum_{i,k=1}^{N+U}\inner{\theta, \tildePhi{\omega}(x_i)M_{ik}\tildePhi{\omega}(x_k)^\adjoint \theta}_{\Vect_{j=1}^D\mathcal{W}'}.
\end{dmath}
\end{corollary}
\begin{proof}
Since $\tildeK{\omega}$ is an operator-valued kernel, from \cref{th:representer}, \cref{eq:argmin_RKHS_rand} has a solution of the form
\begin{dmath*}
\widetilde{f}_{\seq{z}} = \sum_{i=1}^{N+U} \tildeK{\omega}(\cdot, x_i)u_i \condition{$u_i \hiderel{\in} \mathcal{W}$, $x_i \hiderel{\in}\mathcal{X}$}
= \sum_{i=1}^N \tildePhi{\omega}(\cdot)^\adjoint \tildePhi{\omega}(x_i)u_i
\hiderel{=} \tildePhi{\omega}(\cdot)^\adjoint \underbrace{\left(\sum_{i=1}^{N+U}\tildePhi{\omega}(x_i)u_i\right)}_{=\theta\in \left(\Ker \tildeW{\omega}\right)^\perp\subset \tildeH{\omega}}.
\end{dmath*}
Let
\begin{dmath*}
\theta_{\seq{z}}=\argmin_{\theta\in\left(\Ker \tildeW{\omega}\right)^\perp} \frac{1}{2N}\sum_{i=1}^Nc\left(V\tildePhi{\omega}(x_i)^\adjoint \theta, y_i\right) + \frac{\lambda_K}{2}\norm{\tildePhi{\omega}(\cdot)^\adjoint\theta}^2_{\tildeK{\omega}} \\ + \frac{\lambda_M}{2(N+U)}\sum_{i,k=1}^{N+U}\inner*{\tildePhi{\omega}(x_i)^\adjoint \theta, M_{ik}\tildePhi{\omega}(x_k)^\adjoint \theta}_{\mathcal{W}}.
\end{dmath*}
Since $\theta\in(\Ker \tildeW{\omega})^\perp$ and $W$ is an isometry from $(\Ker \tildeW{\omega})^\perp\subset \tildeH{\omega}$ onto $\mathcal{H}_{\tildeK{\omega}}$, we have $\norm{\tildePhi{\omega}(\cdot)^\adjoint\theta}^2_{\tildeK{\omega}} = \norm{\theta}^2_{\tildeH{\omega}}$. Hence
\begin{dmath*}
\theta_{\seq{z}}=\argmin_{\theta\in\left(\Ker \tildeW{\omega}\right)^\perp} \frac{1}{2N}\sum_{i=1}^Nc\left(V\tildePhi{\omega}(x_i)^\adjoint \theta, y_i\right) + \frac{\lambda_K}{2}\norm{\theta}^2_{\tildeH{\omega}} \\ + \frac{\lambda_M}{2(N+U)}\sum_{i,k=1}^{N+U}\inner{\tildePhi{\omega}(x_i)^\adjoint \theta, M_{ik}\tildePhi{\omega}(x_k)^\adjoint \theta}_{\mathcal{W}}.
\end{dmath*}
Finding a minimizer $\theta_{\seq{z}}$ over $\left(\Ker \tildeW{\omega}\right)^\perp$ is not the same than finding a minimizer over $\tildeH{\omega}$. Although in both cases Mazur-Schauder's theorem guarantee that the respective minimizers are unique, they might not be the same. Since $\tildeW{\omega}$ is bounded, $\Ker \tildeW{\omega}$ is closed, so that we can perform the decomposition $\tildeH{\omega}=\left(\Ker \tildeW{\omega}\right)^\perp\oplus \left(\Ker \tildeW{\omega}\right)$. Then clearly by linearity of $W$ and the fact that for all $\theta^{\parallel}\in\Ker \tildeW{\omega}$, $\tildeW{\omega}\theta^{\parallel}=0$, if $\lambda > 0$ we have
\begin{dmath*}
\theta_{\seq{z}}=\argmin_{\theta\in\tildeH{\omega}} \frac{1}{2N}\sum_{i=1}^Nc\left(V\tildePhi{\omega}(x_i)^\adjoint \theta, y_i\right) + \frac{\lambda_K}{2}\norm{\theta}^2_{\tildeH{\omega}} \\ + \frac{\lambda_M}{2(N+U)}\sum_{i,k=1}^{N+U}\inner*{\tildePhi{\omega}(x_i)^\adjoint \theta, M_{ik}\tildePhi{\omega}(x_k)^\adjoint \theta}_{\mathcal{W}}
=\argmin_{\substack{\theta^{\perp}\in\left(\Ker \tildeW{\omega}\right)^\perp, \\ \theta^{\parallel}\in\Ker \tildeW{\omega}}} \frac{1}{2N}\sum_{i=1}^Nc\left(V\left(\tildeW{\omega}\theta^{\perp}\right)(x)+\underbrace{V\left(\tildeW{\omega}\theta^{\parallel}\right)(x)}_{=0\enskip\text{for all}\enskip \theta^{\parallel} }, y_i\right) \\ + \frac{\lambda_K}{2}\norm{\theta^\perp}^2_{\tildeH{\omega}} + \underbrace{\frac{\lambda}{2}\norm{\theta^{\parallel}}^2_{\tildeH{\omega}}}_{=0 \enskip\text{only if}\enskip \theta^{\parallel}=0} \\ + \frac{\lambda_M}{2(N+U)}\sum_{i,k=1}^{N+U}\inner*{\tildePhi{\omega}(x_i)^\adjoint \theta^{\perp}, M_{ik}\left(\tildeW{\omega}\theta^{\perp}\right)(x_k)}_{\mathcal{W}} \\
+ \frac{\lambda_M}{2(N+U)}\sum_{i,k=1}^{N+U}\inner*{\underbrace{\left(\tildeW{\omega}\theta^{\parallel}\right)(x_i)}_{=0\enskip\text{for all}\enskip \theta^{\parallel} }, M_{ik}\left(\tildeW{\omega}\theta^{\perp}\right)(x_k)}_{\mathcal{W}}
\\ + \frac{\lambda_M}{2(N+U)}\sum_{i,k=1}^{N+U}\inner*{\left(\tildeW{\omega}\theta^{\perp}\right)(x_i), M_{ik}\underbrace{\left(\tildeW{\omega}\theta^{\parallel}\right)(x_k)}_{=0\enskip\text{for all}\enskip \theta^{\parallel} }}_{\mathcal{W}} \\ + \frac{\lambda_M}{2(N+U)}\sum_{i,k=1}^{N+U}\inner*{\underbrace{\left(\tildeW{\omega}\theta^{\parallel}\right)(x_i)}_{=0\enskip\text{for all}\enskip \theta^{\parallel} }, M_{ik}\underbrace{\left(\tildeW{\omega}\theta^{\parallel}\right)(x_k)}_{=0\enskip\text{for all}\enskip \theta^{\parallel} }}_{\mathcal{W}}.
\end{dmath*}
Thus
\begin{dmath*}
\theta_{\seq{z}}=\argmin_{\theta^{\perp}\in\left(\Ker \tildeW{\omega}\right)^\perp}
\frac{1}{2N}\sum_{i=1}^Nc\left(V\left(\tildeW{\omega}\theta^{\perp}\right)(x), y_i\right) + \frac{\lambda_K}{2}\norm{\theta^\perp}^2_{\tildeH{\omega}} \\ + \frac{\lambda_M}{2(N+U)}\sum_{i,k=1}^{N+U}\inner*{\tildePhi{\omega}(x_i)^\adjoint \theta^{\perp}, M_{ik}\left(\tildeW{\omega}\theta^{\perp}\right)(x_k)}_{\mathcal{W}}.
\end{dmath*}
Hence minimizing over $\left(\Ker \tildeW{\omega}\right)^\perp$ or $\tildeH{\omega}$ is the same. Eventually for all $g\in L^{2}\left(\dual{\mathcal{X}},\probability_{\dual{\Haar},\rho},\mathcal{W}'\right)$ and for any outcome of $\omega_j \sim \probability_{\dual{\Haar},\rho}$ \iid,
\begin{dmath*}
\theta_{\seq{z}}=\argmin_{\theta\in\tildeH{\omega}} \frac{1}{2N}\sum_{i=1}^Nc\left(V\tildePhi{\omega}(x_i)^\adjoint \theta, y_i\right) + \frac{\lambda_K}{2}\norm{\theta}^2_{\tildeH{\omega}} \\ + \frac{\lambda_M}{2(N+U)}\sum_{i,k=1}^{N+U}\inner*{\tildePhi{\omega}(x_i)^\adjoint\theta, M_{ik}\tildePhi{\omega}(x_k)^\adjoint \theta}_{\mathcal{W}}
=\argmin_{\theta \in \Vect_{j=1}^D\mathcal{W}'} \frac{1}{2N}\sum_{i=1}^Nc\left(V\tildePhi{\omega}(x_i)^\adjoint\theta, y_i\right) + \frac{\lambda_K}{2}\norm{\theta}^2_{\Vect_{j=1}^D\mathcal{W}'} \\ + \frac{\lambda_M}{2(N+U)}\sum_{i,k=1}^{N+U}\inner*{\theta, \tildePhi{\omega}(x_i)M_{ik}\tildePhi{\omega}(x_k)^\adjoint \theta}_{\Vect_{j=1}^D\mathcal{W}'}. \qed
\end{dmath*}
\end{proof}

% \afterpage{
\begin{landscape}
\begin{figure}[htb]
\centering
\resizebox{\textheight}{!}{%
%% Creator: Matplotlib, PGF backend
%%
%% To include the figure in your LaTeX document, write
%%   \input{<filename>.pgf}
%%
%% Make sure the required packages are loaded in your preamble
%%   \usepackage{pgf}
%%
%% Figures using additional raster images can only be included by \input if
%% they are in the same directory as the main LaTeX file. For loading figures
%% from other directories you can use the `import` package
%%   \usepackage{import}
%% and then include the figures with
%%   \import{<path to file>}{<filename>.pgf}
%%
%% Matplotlib used the following preamble
%%   \usepackage{fontspec}
%%   \setmainfont{Times New Roman}
%%   \setsansfont{Arial}
%%   \setmonofont{Andale Mono}
%%
\begingroup%
\makeatletter%
\begin{pgfpicture}%
\pgfpathrectangle{\pgfpointorigin}{\pgfqpoint{16.000000in}{6.000000in}}%
\pgfusepath{use as bounding box, clip}%
\begin{pgfscope}%
\pgfsetbuttcap%
\pgfsetmiterjoin%
\definecolor{currentfill}{rgb}{1.000000,1.000000,1.000000}%
\pgfsetfillcolor{currentfill}%
\pgfsetlinewidth{0.000000pt}%
\definecolor{currentstroke}{rgb}{1.000000,1.000000,1.000000}%
\pgfsetstrokecolor{currentstroke}%
\pgfsetdash{}{0pt}%
\pgfpathmoveto{\pgfqpoint{0.000000in}{0.000000in}}%
\pgfpathlineto{\pgfqpoint{16.000000in}{0.000000in}}%
\pgfpathlineto{\pgfqpoint{16.000000in}{6.000000in}}%
\pgfpathlineto{\pgfqpoint{0.000000in}{6.000000in}}%
\pgfpathclose%
\pgfusepath{fill}%
\end{pgfscope}%
\begin{pgfscope}%
\pgfsetbuttcap%
\pgfsetmiterjoin%
\definecolor{currentfill}{rgb}{1.000000,1.000000,1.000000}%
\pgfsetfillcolor{currentfill}%
\pgfsetlinewidth{0.000000pt}%
\definecolor{currentstroke}{rgb}{0.000000,0.000000,0.000000}%
\pgfsetstrokecolor{currentstroke}%
\pgfsetstrokeopacity{0.000000}%
\pgfsetdash{}{0pt}%
\pgfpathmoveto{\pgfqpoint{2.000000in}{4.421053in}}%
\pgfpathlineto{\pgfqpoint{6.376471in}{4.421053in}}%
\pgfpathlineto{\pgfqpoint{6.376471in}{5.400000in}}%
\pgfpathlineto{\pgfqpoint{2.000000in}{5.400000in}}%
\pgfpathclose%
\pgfusepath{fill}%
\end{pgfscope}%
\begin{pgfscope}%
\pgfpathrectangle{\pgfqpoint{2.000000in}{4.421053in}}{\pgfqpoint{4.376471in}{0.978947in}} %
\pgfusepath{clip}%
\pgfsetroundcap%
\pgfsetroundjoin%
\pgfsetlinewidth{1.003750pt}%
\definecolor{currentstroke}{rgb}{0.800000,0.800000,0.800000}%
\pgfsetstrokecolor{currentstroke}%
\pgfsetdash{}{0pt}%
\pgfpathmoveto{\pgfqpoint{2.000000in}{4.421053in}}%
\pgfpathlineto{\pgfqpoint{2.000000in}{5.400000in}}%
\pgfusepath{stroke}%
\end{pgfscope}%
\begin{pgfscope}%
\pgfpathrectangle{\pgfqpoint{2.000000in}{4.421053in}}{\pgfqpoint{4.376471in}{0.978947in}} %
\pgfusepath{clip}%
\pgfsetroundcap%
\pgfsetroundjoin%
\pgfsetlinewidth{1.003750pt}%
\definecolor{currentstroke}{rgb}{0.800000,0.800000,0.800000}%
\pgfsetstrokecolor{currentstroke}%
\pgfsetdash{}{0pt}%
\pgfpathmoveto{\pgfqpoint{2.875294in}{4.421053in}}%
\pgfpathlineto{\pgfqpoint{2.875294in}{5.400000in}}%
\pgfusepath{stroke}%
\end{pgfscope}%
\begin{pgfscope}%
\pgfpathrectangle{\pgfqpoint{2.000000in}{4.421053in}}{\pgfqpoint{4.376471in}{0.978947in}} %
\pgfusepath{clip}%
\pgfsetroundcap%
\pgfsetroundjoin%
\pgfsetlinewidth{1.003750pt}%
\definecolor{currentstroke}{rgb}{0.800000,0.800000,0.800000}%
\pgfsetstrokecolor{currentstroke}%
\pgfsetdash{}{0pt}%
\pgfpathmoveto{\pgfqpoint{3.750588in}{4.421053in}}%
\pgfpathlineto{\pgfqpoint{3.750588in}{5.400000in}}%
\pgfusepath{stroke}%
\end{pgfscope}%
\begin{pgfscope}%
\pgfpathrectangle{\pgfqpoint{2.000000in}{4.421053in}}{\pgfqpoint{4.376471in}{0.978947in}} %
\pgfusepath{clip}%
\pgfsetroundcap%
\pgfsetroundjoin%
\pgfsetlinewidth{1.003750pt}%
\definecolor{currentstroke}{rgb}{0.800000,0.800000,0.800000}%
\pgfsetstrokecolor{currentstroke}%
\pgfsetdash{}{0pt}%
\pgfpathmoveto{\pgfqpoint{4.625882in}{4.421053in}}%
\pgfpathlineto{\pgfqpoint{4.625882in}{5.400000in}}%
\pgfusepath{stroke}%
\end{pgfscope}%
\begin{pgfscope}%
\pgfpathrectangle{\pgfqpoint{2.000000in}{4.421053in}}{\pgfqpoint{4.376471in}{0.978947in}} %
\pgfusepath{clip}%
\pgfsetroundcap%
\pgfsetroundjoin%
\pgfsetlinewidth{1.003750pt}%
\definecolor{currentstroke}{rgb}{0.800000,0.800000,0.800000}%
\pgfsetstrokecolor{currentstroke}%
\pgfsetdash{}{0pt}%
\pgfpathmoveto{\pgfqpoint{5.501176in}{4.421053in}}%
\pgfpathlineto{\pgfqpoint{5.501176in}{5.400000in}}%
\pgfusepath{stroke}%
\end{pgfscope}%
\begin{pgfscope}%
\pgfpathrectangle{\pgfqpoint{2.000000in}{4.421053in}}{\pgfqpoint{4.376471in}{0.978947in}} %
\pgfusepath{clip}%
\pgfsetroundcap%
\pgfsetroundjoin%
\pgfsetlinewidth{1.003750pt}%
\definecolor{currentstroke}{rgb}{0.800000,0.800000,0.800000}%
\pgfsetstrokecolor{currentstroke}%
\pgfsetdash{}{0pt}%
\pgfpathmoveto{\pgfqpoint{6.376471in}{4.421053in}}%
\pgfpathlineto{\pgfqpoint{6.376471in}{5.400000in}}%
\pgfusepath{stroke}%
\end{pgfscope}%
\begin{pgfscope}%
\pgfpathrectangle{\pgfqpoint{2.000000in}{4.421053in}}{\pgfqpoint{4.376471in}{0.978947in}} %
\pgfusepath{clip}%
\pgfsetroundcap%
\pgfsetroundjoin%
\pgfsetlinewidth{1.003750pt}%
\definecolor{currentstroke}{rgb}{0.800000,0.800000,0.800000}%
\pgfsetstrokecolor{currentstroke}%
\pgfsetdash{}{0pt}%
\pgfpathmoveto{\pgfqpoint{2.000000in}{4.584211in}}%
\pgfpathlineto{\pgfqpoint{6.376471in}{4.584211in}}%
\pgfusepath{stroke}%
\end{pgfscope}%
\begin{pgfscope}%
\definecolor{textcolor}{rgb}{0.150000,0.150000,0.150000}%
\pgfsetstrokecolor{textcolor}%
\pgfsetfillcolor{textcolor}%
\pgftext[x=1.902778in,y=4.584211in,right,]{\color{textcolor}\sffamily\fontsize{10.000000}{12.000000}\selectfont \(\displaystyle -1\)}%
\end{pgfscope}%
\begin{pgfscope}%
\pgfpathrectangle{\pgfqpoint{2.000000in}{4.421053in}}{\pgfqpoint{4.376471in}{0.978947in}} %
\pgfusepath{clip}%
\pgfsetroundcap%
\pgfsetroundjoin%
\pgfsetlinewidth{1.003750pt}%
\definecolor{currentstroke}{rgb}{0.800000,0.800000,0.800000}%
\pgfsetstrokecolor{currentstroke}%
\pgfsetdash{}{0pt}%
\pgfpathmoveto{\pgfqpoint{2.000000in}{4.788158in}}%
\pgfpathlineto{\pgfqpoint{6.376471in}{4.788158in}}%
\pgfusepath{stroke}%
\end{pgfscope}%
\begin{pgfscope}%
\definecolor{textcolor}{rgb}{0.150000,0.150000,0.150000}%
\pgfsetstrokecolor{textcolor}%
\pgfsetfillcolor{textcolor}%
\pgftext[x=1.902778in,y=4.788158in,right,]{\color{textcolor}\sffamily\fontsize{10.000000}{12.000000}\selectfont \(\displaystyle 0\)}%
\end{pgfscope}%
\begin{pgfscope}%
\pgfpathrectangle{\pgfqpoint{2.000000in}{4.421053in}}{\pgfqpoint{4.376471in}{0.978947in}} %
\pgfusepath{clip}%
\pgfsetroundcap%
\pgfsetroundjoin%
\pgfsetlinewidth{1.003750pt}%
\definecolor{currentstroke}{rgb}{0.800000,0.800000,0.800000}%
\pgfsetstrokecolor{currentstroke}%
\pgfsetdash{}{0pt}%
\pgfpathmoveto{\pgfqpoint{2.000000in}{4.992105in}}%
\pgfpathlineto{\pgfqpoint{6.376471in}{4.992105in}}%
\pgfusepath{stroke}%
\end{pgfscope}%
\begin{pgfscope}%
\definecolor{textcolor}{rgb}{0.150000,0.150000,0.150000}%
\pgfsetstrokecolor{textcolor}%
\pgfsetfillcolor{textcolor}%
\pgftext[x=1.902778in,y=4.992105in,right,]{\color{textcolor}\sffamily\fontsize{10.000000}{12.000000}\selectfont \(\displaystyle 1\)}%
\end{pgfscope}%
\begin{pgfscope}%
\pgfpathrectangle{\pgfqpoint{2.000000in}{4.421053in}}{\pgfqpoint{4.376471in}{0.978947in}} %
\pgfusepath{clip}%
\pgfsetroundcap%
\pgfsetroundjoin%
\pgfsetlinewidth{1.003750pt}%
\definecolor{currentstroke}{rgb}{0.800000,0.800000,0.800000}%
\pgfsetstrokecolor{currentstroke}%
\pgfsetdash{}{0pt}%
\pgfpathmoveto{\pgfqpoint{2.000000in}{5.196053in}}%
\pgfpathlineto{\pgfqpoint{6.376471in}{5.196053in}}%
\pgfusepath{stroke}%
\end{pgfscope}%
\begin{pgfscope}%
\definecolor{textcolor}{rgb}{0.150000,0.150000,0.150000}%
\pgfsetstrokecolor{textcolor}%
\pgfsetfillcolor{textcolor}%
\pgftext[x=1.902778in,y=5.196053in,right,]{\color{textcolor}\sffamily\fontsize{10.000000}{12.000000}\selectfont \(\displaystyle 2\)}%
\end{pgfscope}%
\begin{pgfscope}%
\pgfpathrectangle{\pgfqpoint{2.000000in}{4.421053in}}{\pgfqpoint{4.376471in}{0.978947in}} %
\pgfusepath{clip}%
\pgfsetroundcap%
\pgfsetroundjoin%
\pgfsetlinewidth{1.003750pt}%
\definecolor{currentstroke}{rgb}{0.800000,0.800000,0.800000}%
\pgfsetstrokecolor{currentstroke}%
\pgfsetdash{}{0pt}%
\pgfpathmoveto{\pgfqpoint{2.000000in}{5.400000in}}%
\pgfpathlineto{\pgfqpoint{6.376471in}{5.400000in}}%
\pgfusepath{stroke}%
\end{pgfscope}%
\begin{pgfscope}%
\definecolor{textcolor}{rgb}{0.150000,0.150000,0.150000}%
\pgfsetstrokecolor{textcolor}%
\pgfsetfillcolor{textcolor}%
\pgftext[x=1.902778in,y=5.400000in,right,]{\color{textcolor}\sffamily\fontsize{10.000000}{12.000000}\selectfont \(\displaystyle 3\)}%
\end{pgfscope}%
\begin{pgfscope}%
\definecolor{textcolor}{rgb}{0.150000,0.150000,0.150000}%
\pgfsetstrokecolor{textcolor}%
\pgfsetfillcolor{textcolor}%
\pgftext[x=1.655864in,y=4.910526in,,bottom,rotate=90.000000]{\color{textcolor}\sffamily\fontsize{11.000000}{13.200000}\selectfont y}%
\end{pgfscope}%
\begin{pgfscope}%
\pgfpathrectangle{\pgfqpoint{2.000000in}{4.421053in}}{\pgfqpoint{4.376471in}{0.978947in}} %
\pgfusepath{clip}%
\pgfsetbuttcap%
\pgfsetroundjoin%
\definecolor{currentfill}{rgb}{1.000000,0.000000,0.000000}%
\pgfsetfillcolor{currentfill}%
\pgfsetlinewidth{2.007500pt}%
\definecolor{currentstroke}{rgb}{1.000000,0.000000,0.000000}%
\pgfsetstrokecolor{currentstroke}%
\pgfsetdash{}{0pt}%
\pgfpathmoveto{\pgfqpoint{4.765731in}{4.978628in}}%
\pgfpathlineto{\pgfqpoint{4.827844in}{4.978628in}}%
\pgfpathmoveto{\pgfqpoint{4.796787in}{4.947572in}}%
\pgfpathlineto{\pgfqpoint{4.796787in}{5.009685in}}%
\pgfusepath{stroke,fill}%
\end{pgfscope}%
\begin{pgfscope}%
\pgfpathrectangle{\pgfqpoint{2.000000in}{4.421053in}}{\pgfqpoint{4.376471in}{0.978947in}} %
\pgfusepath{clip}%
\pgfsetbuttcap%
\pgfsetroundjoin%
\definecolor{currentfill}{rgb}{1.000000,0.000000,0.000000}%
\pgfsetfillcolor{currentfill}%
\pgfsetlinewidth{2.007500pt}%
\definecolor{currentstroke}{rgb}{1.000000,0.000000,0.000000}%
\pgfsetstrokecolor{currentstroke}%
\pgfsetdash{}{0pt}%
\pgfpathmoveto{\pgfqpoint{5.348242in}{4.661416in}}%
\pgfpathlineto{\pgfqpoint{5.410355in}{4.661416in}}%
\pgfpathmoveto{\pgfqpoint{5.379298in}{4.630360in}}%
\pgfpathlineto{\pgfqpoint{5.379298in}{4.692473in}}%
\pgfusepath{stroke,fill}%
\end{pgfscope}%
\begin{pgfscope}%
\pgfpathrectangle{\pgfqpoint{2.000000in}{4.421053in}}{\pgfqpoint{4.376471in}{0.978947in}} %
\pgfusepath{clip}%
\pgfsetbuttcap%
\pgfsetroundjoin%
\definecolor{currentfill}{rgb}{1.000000,0.000000,0.000000}%
\pgfsetfillcolor{currentfill}%
\pgfsetlinewidth{2.007500pt}%
\definecolor{currentstroke}{rgb}{1.000000,0.000000,0.000000}%
\pgfsetstrokecolor{currentstroke}%
\pgfsetdash{}{0pt}%
\pgfpathmoveto{\pgfqpoint{4.954619in}{5.032156in}}%
\pgfpathlineto{\pgfqpoint{5.016732in}{5.032156in}}%
\pgfpathmoveto{\pgfqpoint{4.985675in}{5.001100in}}%
\pgfpathlineto{\pgfqpoint{4.985675in}{5.063213in}}%
\pgfusepath{stroke,fill}%
\end{pgfscope}%
\begin{pgfscope}%
\pgfpathrectangle{\pgfqpoint{2.000000in}{4.421053in}}{\pgfqpoint{4.376471in}{0.978947in}} %
\pgfusepath{clip}%
\pgfsetbuttcap%
\pgfsetroundjoin%
\definecolor{currentfill}{rgb}{1.000000,0.000000,0.000000}%
\pgfsetfillcolor{currentfill}%
\pgfsetlinewidth{2.007500pt}%
\definecolor{currentstroke}{rgb}{1.000000,0.000000,0.000000}%
\pgfsetstrokecolor{currentstroke}%
\pgfsetdash{}{0pt}%
\pgfpathmoveto{\pgfqpoint{4.751970in}{4.921244in}}%
\pgfpathlineto{\pgfqpoint{4.814083in}{4.921244in}}%
\pgfpathmoveto{\pgfqpoint{4.783026in}{4.890188in}}%
\pgfpathlineto{\pgfqpoint{4.783026in}{4.952301in}}%
\pgfusepath{stroke,fill}%
\end{pgfscope}%
\begin{pgfscope}%
\pgfpathrectangle{\pgfqpoint{2.000000in}{4.421053in}}{\pgfqpoint{4.376471in}{0.978947in}} %
\pgfusepath{clip}%
\pgfsetbuttcap%
\pgfsetroundjoin%
\definecolor{currentfill}{rgb}{1.000000,0.000000,0.000000}%
\pgfsetfillcolor{currentfill}%
\pgfsetlinewidth{2.007500pt}%
\definecolor{currentstroke}{rgb}{1.000000,0.000000,0.000000}%
\pgfsetstrokecolor{currentstroke}%
\pgfsetdash{}{0pt}%
\pgfpathmoveto{\pgfqpoint{4.327528in}{4.594850in}}%
\pgfpathlineto{\pgfqpoint{4.389641in}{4.594850in}}%
\pgfpathmoveto{\pgfqpoint{4.358584in}{4.563793in}}%
\pgfpathlineto{\pgfqpoint{4.358584in}{4.625906in}}%
\pgfusepath{stroke,fill}%
\end{pgfscope}%
\begin{pgfscope}%
\pgfpathrectangle{\pgfqpoint{2.000000in}{4.421053in}}{\pgfqpoint{4.376471in}{0.978947in}} %
\pgfusepath{clip}%
\pgfsetbuttcap%
\pgfsetroundjoin%
\definecolor{currentfill}{rgb}{1.000000,0.000000,0.000000}%
\pgfsetfillcolor{currentfill}%
\pgfsetlinewidth{2.007500pt}%
\definecolor{currentstroke}{rgb}{1.000000,0.000000,0.000000}%
\pgfsetstrokecolor{currentstroke}%
\pgfsetdash{}{0pt}%
\pgfpathmoveto{\pgfqpoint{5.105627in}{4.864314in}}%
\pgfpathlineto{\pgfqpoint{5.167740in}{4.864314in}}%
\pgfpathmoveto{\pgfqpoint{5.136683in}{4.833258in}}%
\pgfpathlineto{\pgfqpoint{5.136683in}{4.895371in}}%
\pgfusepath{stroke,fill}%
\end{pgfscope}%
\begin{pgfscope}%
\pgfpathrectangle{\pgfqpoint{2.000000in}{4.421053in}}{\pgfqpoint{4.376471in}{0.978947in}} %
\pgfusepath{clip}%
\pgfsetbuttcap%
\pgfsetroundjoin%
\definecolor{currentfill}{rgb}{1.000000,0.000000,0.000000}%
\pgfsetfillcolor{currentfill}%
\pgfsetlinewidth{2.007500pt}%
\definecolor{currentstroke}{rgb}{1.000000,0.000000,0.000000}%
\pgfsetstrokecolor{currentstroke}%
\pgfsetdash{}{0pt}%
\pgfpathmoveto{\pgfqpoint{4.376308in}{4.632344in}}%
\pgfpathlineto{\pgfqpoint{4.438421in}{4.632344in}}%
\pgfpathmoveto{\pgfqpoint{4.407364in}{4.601287in}}%
\pgfpathlineto{\pgfqpoint{4.407364in}{4.663400in}}%
\pgfusepath{stroke,fill}%
\end{pgfscope}%
\begin{pgfscope}%
\pgfpathrectangle{\pgfqpoint{2.000000in}{4.421053in}}{\pgfqpoint{4.376471in}{0.978947in}} %
\pgfusepath{clip}%
\pgfsetbuttcap%
\pgfsetroundjoin%
\definecolor{currentfill}{rgb}{1.000000,0.000000,0.000000}%
\pgfsetfillcolor{currentfill}%
\pgfsetlinewidth{2.007500pt}%
\definecolor{currentstroke}{rgb}{1.000000,0.000000,0.000000}%
\pgfsetstrokecolor{currentstroke}%
\pgfsetdash{}{0pt}%
\pgfpathmoveto{\pgfqpoint{5.966492in}{5.272462in}}%
\pgfpathlineto{\pgfqpoint{6.028605in}{5.272462in}}%
\pgfpathmoveto{\pgfqpoint{5.997549in}{5.241406in}}%
\pgfpathlineto{\pgfqpoint{5.997549in}{5.303519in}}%
\pgfusepath{stroke,fill}%
\end{pgfscope}%
\begin{pgfscope}%
\pgfpathrectangle{\pgfqpoint{2.000000in}{4.421053in}}{\pgfqpoint{4.376471in}{0.978947in}} %
\pgfusepath{clip}%
\pgfsetbuttcap%
\pgfsetroundjoin%
\definecolor{currentfill}{rgb}{1.000000,0.000000,0.000000}%
\pgfsetfillcolor{currentfill}%
\pgfsetlinewidth{2.007500pt}%
\definecolor{currentstroke}{rgb}{1.000000,0.000000,0.000000}%
\pgfsetstrokecolor{currentstroke}%
\pgfsetdash{}{0pt}%
\pgfpathmoveto{\pgfqpoint{6.218191in}{5.172865in}}%
\pgfpathlineto{\pgfqpoint{6.280304in}{5.172865in}}%
\pgfpathmoveto{\pgfqpoint{6.249248in}{5.141808in}}%
\pgfpathlineto{\pgfqpoint{6.249248in}{5.203921in}}%
\pgfusepath{stroke,fill}%
\end{pgfscope}%
\begin{pgfscope}%
\pgfpathrectangle{\pgfqpoint{2.000000in}{4.421053in}}{\pgfqpoint{4.376471in}{0.978947in}} %
\pgfusepath{clip}%
\pgfsetbuttcap%
\pgfsetroundjoin%
\definecolor{currentfill}{rgb}{1.000000,0.000000,0.000000}%
\pgfsetfillcolor{currentfill}%
\pgfsetlinewidth{2.007500pt}%
\definecolor{currentstroke}{rgb}{1.000000,0.000000,0.000000}%
\pgfsetstrokecolor{currentstroke}%
\pgfsetdash{}{0pt}%
\pgfpathmoveto{\pgfqpoint{4.186734in}{4.670261in}}%
\pgfpathlineto{\pgfqpoint{4.248847in}{4.670261in}}%
\pgfpathmoveto{\pgfqpoint{4.217791in}{4.639204in}}%
\pgfpathlineto{\pgfqpoint{4.217791in}{4.701317in}}%
\pgfusepath{stroke,fill}%
\end{pgfscope}%
\begin{pgfscope}%
\pgfpathrectangle{\pgfqpoint{2.000000in}{4.421053in}}{\pgfqpoint{4.376471in}{0.978947in}} %
\pgfusepath{clip}%
\pgfsetbuttcap%
\pgfsetroundjoin%
\definecolor{currentfill}{rgb}{1.000000,0.000000,0.000000}%
\pgfsetfillcolor{currentfill}%
\pgfsetlinewidth{2.007500pt}%
\definecolor{currentstroke}{rgb}{1.000000,0.000000,0.000000}%
\pgfsetstrokecolor{currentstroke}%
\pgfsetdash{}{0pt}%
\pgfpathmoveto{\pgfqpoint{5.616207in}{4.820436in}}%
\pgfpathlineto{\pgfqpoint{5.678320in}{4.820436in}}%
\pgfpathmoveto{\pgfqpoint{5.647263in}{4.789380in}}%
\pgfpathlineto{\pgfqpoint{5.647263in}{4.851493in}}%
\pgfusepath{stroke,fill}%
\end{pgfscope}%
\begin{pgfscope}%
\pgfpathrectangle{\pgfqpoint{2.000000in}{4.421053in}}{\pgfqpoint{4.376471in}{0.978947in}} %
\pgfusepath{clip}%
\pgfsetbuttcap%
\pgfsetroundjoin%
\definecolor{currentfill}{rgb}{1.000000,0.000000,0.000000}%
\pgfsetfillcolor{currentfill}%
\pgfsetlinewidth{2.007500pt}%
\definecolor{currentstroke}{rgb}{1.000000,0.000000,0.000000}%
\pgfsetstrokecolor{currentstroke}%
\pgfsetdash{}{0pt}%
\pgfpathmoveto{\pgfqpoint{4.695992in}{4.860532in}}%
\pgfpathlineto{\pgfqpoint{4.758105in}{4.860532in}}%
\pgfpathmoveto{\pgfqpoint{4.727049in}{4.829475in}}%
\pgfpathlineto{\pgfqpoint{4.727049in}{4.891588in}}%
\pgfusepath{stroke,fill}%
\end{pgfscope}%
\begin{pgfscope}%
\pgfpathrectangle{\pgfqpoint{2.000000in}{4.421053in}}{\pgfqpoint{4.376471in}{0.978947in}} %
\pgfusepath{clip}%
\pgfsetbuttcap%
\pgfsetroundjoin%
\definecolor{currentfill}{rgb}{1.000000,0.000000,0.000000}%
\pgfsetfillcolor{currentfill}%
\pgfsetlinewidth{2.007500pt}%
\definecolor{currentstroke}{rgb}{1.000000,0.000000,0.000000}%
\pgfsetstrokecolor{currentstroke}%
\pgfsetdash{}{0pt}%
\pgfpathmoveto{\pgfqpoint{4.833062in}{4.988087in}}%
\pgfpathlineto{\pgfqpoint{4.895175in}{4.988087in}}%
\pgfpathmoveto{\pgfqpoint{4.864118in}{4.957031in}}%
\pgfpathlineto{\pgfqpoint{4.864118in}{5.019144in}}%
\pgfusepath{stroke,fill}%
\end{pgfscope}%
\begin{pgfscope}%
\pgfpathrectangle{\pgfqpoint{2.000000in}{4.421053in}}{\pgfqpoint{4.376471in}{0.978947in}} %
\pgfusepath{clip}%
\pgfsetbuttcap%
\pgfsetroundjoin%
\definecolor{currentfill}{rgb}{1.000000,0.000000,0.000000}%
\pgfsetfillcolor{currentfill}%
\pgfsetlinewidth{2.007500pt}%
\definecolor{currentstroke}{rgb}{1.000000,0.000000,0.000000}%
\pgfsetstrokecolor{currentstroke}%
\pgfsetdash{}{0pt}%
\pgfpathmoveto{\pgfqpoint{6.084915in}{5.248251in}}%
\pgfpathlineto{\pgfqpoint{6.147028in}{5.248251in}}%
\pgfpathmoveto{\pgfqpoint{6.115971in}{5.217195in}}%
\pgfpathlineto{\pgfqpoint{6.115971in}{5.279308in}}%
\pgfusepath{stroke,fill}%
\end{pgfscope}%
\begin{pgfscope}%
\pgfpathrectangle{\pgfqpoint{2.000000in}{4.421053in}}{\pgfqpoint{4.376471in}{0.978947in}} %
\pgfusepath{clip}%
\pgfsetbuttcap%
\pgfsetroundjoin%
\definecolor{currentfill}{rgb}{1.000000,0.000000,0.000000}%
\pgfsetfillcolor{currentfill}%
\pgfsetlinewidth{2.007500pt}%
\definecolor{currentstroke}{rgb}{1.000000,0.000000,0.000000}%
\pgfsetstrokecolor{currentstroke}%
\pgfsetdash{}{0pt}%
\pgfpathmoveto{\pgfqpoint{3.092947in}{4.960862in}}%
\pgfpathlineto{\pgfqpoint{3.155060in}{4.960862in}}%
\pgfpathmoveto{\pgfqpoint{3.124004in}{4.929806in}}%
\pgfpathlineto{\pgfqpoint{3.124004in}{4.991919in}}%
\pgfusepath{stroke,fill}%
\end{pgfscope}%
\begin{pgfscope}%
\pgfpathrectangle{\pgfqpoint{2.000000in}{4.421053in}}{\pgfqpoint{4.376471in}{0.978947in}} %
\pgfusepath{clip}%
\pgfsetbuttcap%
\pgfsetroundjoin%
\definecolor{currentfill}{rgb}{1.000000,0.000000,0.000000}%
\pgfsetfillcolor{currentfill}%
\pgfsetlinewidth{2.007500pt}%
\definecolor{currentstroke}{rgb}{1.000000,0.000000,0.000000}%
\pgfsetstrokecolor{currentstroke}%
\pgfsetdash{}{0pt}%
\pgfpathmoveto{\pgfqpoint{3.149293in}{4.903228in}}%
\pgfpathlineto{\pgfqpoint{3.211406in}{4.903228in}}%
\pgfpathmoveto{\pgfqpoint{3.180349in}{4.872172in}}%
\pgfpathlineto{\pgfqpoint{3.180349in}{4.934285in}}%
\pgfusepath{stroke,fill}%
\end{pgfscope}%
\begin{pgfscope}%
\pgfpathrectangle{\pgfqpoint{2.000000in}{4.421053in}}{\pgfqpoint{4.376471in}{0.978947in}} %
\pgfusepath{clip}%
\pgfsetbuttcap%
\pgfsetroundjoin%
\definecolor{currentfill}{rgb}{1.000000,0.000000,0.000000}%
\pgfsetfillcolor{currentfill}%
\pgfsetlinewidth{2.007500pt}%
\definecolor{currentstroke}{rgb}{1.000000,0.000000,0.000000}%
\pgfsetstrokecolor{currentstroke}%
\pgfsetdash{}{0pt}%
\pgfpathmoveto{\pgfqpoint{2.915026in}{5.190508in}}%
\pgfpathlineto{\pgfqpoint{2.977139in}{5.190508in}}%
\pgfpathmoveto{\pgfqpoint{2.946082in}{5.159452in}}%
\pgfpathlineto{\pgfqpoint{2.946082in}{5.221565in}}%
\pgfusepath{stroke,fill}%
\end{pgfscope}%
\begin{pgfscope}%
\pgfpathrectangle{\pgfqpoint{2.000000in}{4.421053in}}{\pgfqpoint{4.376471in}{0.978947in}} %
\pgfusepath{clip}%
\pgfsetbuttcap%
\pgfsetroundjoin%
\definecolor{currentfill}{rgb}{1.000000,0.000000,0.000000}%
\pgfsetfillcolor{currentfill}%
\pgfsetlinewidth{2.007500pt}%
\definecolor{currentstroke}{rgb}{1.000000,0.000000,0.000000}%
\pgfsetstrokecolor{currentstroke}%
\pgfsetdash{}{0pt}%
\pgfpathmoveto{\pgfqpoint{5.759387in}{5.036104in}}%
\pgfpathlineto{\pgfqpoint{5.821500in}{5.036104in}}%
\pgfpathmoveto{\pgfqpoint{5.790443in}{5.005048in}}%
\pgfpathlineto{\pgfqpoint{5.790443in}{5.067161in}}%
\pgfusepath{stroke,fill}%
\end{pgfscope}%
\begin{pgfscope}%
\pgfpathrectangle{\pgfqpoint{2.000000in}{4.421053in}}{\pgfqpoint{4.376471in}{0.978947in}} %
\pgfusepath{clip}%
\pgfsetbuttcap%
\pgfsetroundjoin%
\definecolor{currentfill}{rgb}{1.000000,0.000000,0.000000}%
\pgfsetfillcolor{currentfill}%
\pgfsetlinewidth{2.007500pt}%
\definecolor{currentstroke}{rgb}{1.000000,0.000000,0.000000}%
\pgfsetstrokecolor{currentstroke}%
\pgfsetdash{}{0pt}%
\pgfpathmoveto{\pgfqpoint{5.568702in}{4.758523in}}%
\pgfpathlineto{\pgfqpoint{5.630815in}{4.758523in}}%
\pgfpathmoveto{\pgfqpoint{5.599758in}{4.727466in}}%
\pgfpathlineto{\pgfqpoint{5.599758in}{4.789579in}}%
\pgfusepath{stroke,fill}%
\end{pgfscope}%
\begin{pgfscope}%
\pgfpathrectangle{\pgfqpoint{2.000000in}{4.421053in}}{\pgfqpoint{4.376471in}{0.978947in}} %
\pgfusepath{clip}%
\pgfsetbuttcap%
\pgfsetroundjoin%
\definecolor{currentfill}{rgb}{1.000000,0.000000,0.000000}%
\pgfsetfillcolor{currentfill}%
\pgfsetlinewidth{2.007500pt}%
\definecolor{currentstroke}{rgb}{1.000000,0.000000,0.000000}%
\pgfsetstrokecolor{currentstroke}%
\pgfsetdash{}{0pt}%
\pgfpathmoveto{\pgfqpoint{5.890304in}{5.165886in}}%
\pgfpathlineto{\pgfqpoint{5.952417in}{5.165886in}}%
\pgfpathmoveto{\pgfqpoint{5.921360in}{5.134829in}}%
\pgfpathlineto{\pgfqpoint{5.921360in}{5.196942in}}%
\pgfusepath{stroke,fill}%
\end{pgfscope}%
\begin{pgfscope}%
\pgfpathrectangle{\pgfqpoint{2.000000in}{4.421053in}}{\pgfqpoint{4.376471in}{0.978947in}} %
\pgfusepath{clip}%
\pgfsetbuttcap%
\pgfsetroundjoin%
\definecolor{currentfill}{rgb}{1.000000,0.000000,0.000000}%
\pgfsetfillcolor{currentfill}%
\pgfsetlinewidth{2.007500pt}%
\definecolor{currentstroke}{rgb}{1.000000,0.000000,0.000000}%
\pgfsetstrokecolor{currentstroke}%
\pgfsetdash{}{0pt}%
\pgfpathmoveto{\pgfqpoint{6.270553in}{5.097123in}}%
\pgfpathlineto{\pgfqpoint{6.332666in}{5.097123in}}%
\pgfpathmoveto{\pgfqpoint{6.301610in}{5.066066in}}%
\pgfpathlineto{\pgfqpoint{6.301610in}{5.128179in}}%
\pgfusepath{stroke,fill}%
\end{pgfscope}%
\begin{pgfscope}%
\pgfpathrectangle{\pgfqpoint{2.000000in}{4.421053in}}{\pgfqpoint{4.376471in}{0.978947in}} %
\pgfusepath{clip}%
\pgfsetbuttcap%
\pgfsetroundjoin%
\definecolor{currentfill}{rgb}{1.000000,0.000000,0.000000}%
\pgfsetfillcolor{currentfill}%
\pgfsetlinewidth{2.007500pt}%
\definecolor{currentstroke}{rgb}{1.000000,0.000000,0.000000}%
\pgfsetstrokecolor{currentstroke}%
\pgfsetdash{}{0pt}%
\pgfpathmoveto{\pgfqpoint{5.642233in}{4.913690in}}%
\pgfpathlineto{\pgfqpoint{5.704346in}{4.913690in}}%
\pgfpathmoveto{\pgfqpoint{5.673289in}{4.882633in}}%
\pgfpathlineto{\pgfqpoint{5.673289in}{4.944746in}}%
\pgfusepath{stroke,fill}%
\end{pgfscope}%
\begin{pgfscope}%
\pgfpathrectangle{\pgfqpoint{2.000000in}{4.421053in}}{\pgfqpoint{4.376471in}{0.978947in}} %
\pgfusepath{clip}%
\pgfsetbuttcap%
\pgfsetroundjoin%
\definecolor{currentfill}{rgb}{1.000000,0.000000,0.000000}%
\pgfsetfillcolor{currentfill}%
\pgfsetlinewidth{2.007500pt}%
\definecolor{currentstroke}{rgb}{1.000000,0.000000,0.000000}%
\pgfsetstrokecolor{currentstroke}%
\pgfsetdash{}{0pt}%
\pgfpathmoveto{\pgfqpoint{4.459958in}{4.638149in}}%
\pgfpathlineto{\pgfqpoint{4.522071in}{4.638149in}}%
\pgfpathmoveto{\pgfqpoint{4.491015in}{4.607093in}}%
\pgfpathlineto{\pgfqpoint{4.491015in}{4.669206in}}%
\pgfusepath{stroke,fill}%
\end{pgfscope}%
\begin{pgfscope}%
\pgfpathrectangle{\pgfqpoint{2.000000in}{4.421053in}}{\pgfqpoint{4.376471in}{0.978947in}} %
\pgfusepath{clip}%
\pgfsetbuttcap%
\pgfsetroundjoin%
\definecolor{currentfill}{rgb}{1.000000,0.000000,0.000000}%
\pgfsetfillcolor{currentfill}%
\pgfsetlinewidth{2.007500pt}%
\definecolor{currentstroke}{rgb}{1.000000,0.000000,0.000000}%
\pgfsetstrokecolor{currentstroke}%
\pgfsetdash{}{0pt}%
\pgfpathmoveto{\pgfqpoint{5.577008in}{4.780559in}}%
\pgfpathlineto{\pgfqpoint{5.639121in}{4.780559in}}%
\pgfpathmoveto{\pgfqpoint{5.608065in}{4.749503in}}%
\pgfpathlineto{\pgfqpoint{5.608065in}{4.811616in}}%
\pgfusepath{stroke,fill}%
\end{pgfscope}%
\begin{pgfscope}%
\pgfpathrectangle{\pgfqpoint{2.000000in}{4.421053in}}{\pgfqpoint{4.376471in}{0.978947in}} %
\pgfusepath{clip}%
\pgfsetbuttcap%
\pgfsetroundjoin%
\definecolor{currentfill}{rgb}{1.000000,0.000000,0.000000}%
\pgfsetfillcolor{currentfill}%
\pgfsetlinewidth{2.007500pt}%
\definecolor{currentstroke}{rgb}{1.000000,0.000000,0.000000}%
\pgfsetstrokecolor{currentstroke}%
\pgfsetdash{}{0pt}%
\pgfpathmoveto{\pgfqpoint{3.258337in}{4.801297in}}%
\pgfpathlineto{\pgfqpoint{3.320450in}{4.801297in}}%
\pgfpathmoveto{\pgfqpoint{3.289394in}{4.770240in}}%
\pgfpathlineto{\pgfqpoint{3.289394in}{4.832353in}}%
\pgfusepath{stroke,fill}%
\end{pgfscope}%
\begin{pgfscope}%
\pgfpathrectangle{\pgfqpoint{2.000000in}{4.421053in}}{\pgfqpoint{4.376471in}{0.978947in}} %
\pgfusepath{clip}%
\pgfsetbuttcap%
\pgfsetroundjoin%
\definecolor{currentfill}{rgb}{1.000000,0.000000,0.000000}%
\pgfsetfillcolor{currentfill}%
\pgfsetlinewidth{2.007500pt}%
\definecolor{currentstroke}{rgb}{1.000000,0.000000,0.000000}%
\pgfsetstrokecolor{currentstroke}%
\pgfsetdash{}{0pt}%
\pgfpathmoveto{\pgfqpoint{5.084714in}{4.904581in}}%
\pgfpathlineto{\pgfqpoint{5.146827in}{4.904581in}}%
\pgfpathmoveto{\pgfqpoint{5.115771in}{4.873524in}}%
\pgfpathlineto{\pgfqpoint{5.115771in}{4.935637in}}%
\pgfusepath{stroke,fill}%
\end{pgfscope}%
\begin{pgfscope}%
\pgfpathrectangle{\pgfqpoint{2.000000in}{4.421053in}}{\pgfqpoint{4.376471in}{0.978947in}} %
\pgfusepath{clip}%
\pgfsetbuttcap%
\pgfsetroundjoin%
\definecolor{currentfill}{rgb}{1.000000,0.000000,0.000000}%
\pgfsetfillcolor{currentfill}%
\pgfsetlinewidth{2.007500pt}%
\definecolor{currentstroke}{rgb}{1.000000,0.000000,0.000000}%
\pgfsetstrokecolor{currentstroke}%
\pgfsetdash{}{0pt}%
\pgfpathmoveto{\pgfqpoint{3.346143in}{4.809587in}}%
\pgfpathlineto{\pgfqpoint{3.408256in}{4.809587in}}%
\pgfpathmoveto{\pgfqpoint{3.377199in}{4.778530in}}%
\pgfpathlineto{\pgfqpoint{3.377199in}{4.840643in}}%
\pgfusepath{stroke,fill}%
\end{pgfscope}%
\begin{pgfscope}%
\pgfpathrectangle{\pgfqpoint{2.000000in}{4.421053in}}{\pgfqpoint{4.376471in}{0.978947in}} %
\pgfusepath{clip}%
\pgfsetbuttcap%
\pgfsetroundjoin%
\definecolor{currentfill}{rgb}{1.000000,0.000000,0.000000}%
\pgfsetfillcolor{currentfill}%
\pgfsetlinewidth{2.007500pt}%
\definecolor{currentstroke}{rgb}{1.000000,0.000000,0.000000}%
\pgfsetstrokecolor{currentstroke}%
\pgfsetdash{}{0pt}%
\pgfpathmoveto{\pgfqpoint{6.151690in}{5.209780in}}%
\pgfpathlineto{\pgfqpoint{6.213803in}{5.209780in}}%
\pgfpathmoveto{\pgfqpoint{6.182747in}{5.178724in}}%
\pgfpathlineto{\pgfqpoint{6.182747in}{5.240837in}}%
\pgfusepath{stroke,fill}%
\end{pgfscope}%
\begin{pgfscope}%
\pgfpathrectangle{\pgfqpoint{2.000000in}{4.421053in}}{\pgfqpoint{4.376471in}{0.978947in}} %
\pgfusepath{clip}%
\pgfsetbuttcap%
\pgfsetroundjoin%
\definecolor{currentfill}{rgb}{1.000000,0.000000,0.000000}%
\pgfsetfillcolor{currentfill}%
\pgfsetlinewidth{2.007500pt}%
\definecolor{currentstroke}{rgb}{1.000000,0.000000,0.000000}%
\pgfsetstrokecolor{currentstroke}%
\pgfsetdash{}{0pt}%
\pgfpathmoveto{\pgfqpoint{4.671321in}{4.856983in}}%
\pgfpathlineto{\pgfqpoint{4.733434in}{4.856983in}}%
\pgfpathmoveto{\pgfqpoint{4.702377in}{4.825926in}}%
\pgfpathlineto{\pgfqpoint{4.702377in}{4.888039in}}%
\pgfusepath{stroke,fill}%
\end{pgfscope}%
\begin{pgfscope}%
\pgfpathrectangle{\pgfqpoint{2.000000in}{4.421053in}}{\pgfqpoint{4.376471in}{0.978947in}} %
\pgfusepath{clip}%
\pgfsetbuttcap%
\pgfsetroundjoin%
\definecolor{currentfill}{rgb}{1.000000,0.000000,0.000000}%
\pgfsetfillcolor{currentfill}%
\pgfsetlinewidth{2.007500pt}%
\definecolor{currentstroke}{rgb}{1.000000,0.000000,0.000000}%
\pgfsetstrokecolor{currentstroke}%
\pgfsetdash{}{0pt}%
\pgfpathmoveto{\pgfqpoint{4.296042in}{4.605866in}}%
\pgfpathlineto{\pgfqpoint{4.358155in}{4.605866in}}%
\pgfpathmoveto{\pgfqpoint{4.327099in}{4.574809in}}%
\pgfpathlineto{\pgfqpoint{4.327099in}{4.636922in}}%
\pgfusepath{stroke,fill}%
\end{pgfscope}%
\begin{pgfscope}%
\pgfpathrectangle{\pgfqpoint{2.000000in}{4.421053in}}{\pgfqpoint{4.376471in}{0.978947in}} %
\pgfusepath{clip}%
\pgfsetbuttcap%
\pgfsetroundjoin%
\definecolor{currentfill}{rgb}{0.000000,0.000000,0.000000}%
\pgfsetfillcolor{currentfill}%
\pgfsetlinewidth{0.301125pt}%
\definecolor{currentstroke}{rgb}{0.000000,0.000000,0.000000}%
\pgfsetstrokecolor{currentstroke}%
\pgfsetdash{}{0pt}%
\pgfsys@defobject{currentmarker}{\pgfqpoint{-0.015528in}{-0.015528in}}{\pgfqpoint{0.015528in}{0.015528in}}{%
\pgfpathmoveto{\pgfqpoint{0.000000in}{-0.015528in}}%
\pgfpathcurveto{\pgfqpoint{0.004118in}{-0.015528in}}{\pgfqpoint{0.008068in}{-0.013892in}}{\pgfqpoint{0.010980in}{-0.010980in}}%
\pgfpathcurveto{\pgfqpoint{0.013892in}{-0.008068in}}{\pgfqpoint{0.015528in}{-0.004118in}}{\pgfqpoint{0.015528in}{0.000000in}}%
\pgfpathcurveto{\pgfqpoint{0.015528in}{0.004118in}}{\pgfqpoint{0.013892in}{0.008068in}}{\pgfqpoint{0.010980in}{0.010980in}}%
\pgfpathcurveto{\pgfqpoint{0.008068in}{0.013892in}}{\pgfqpoint{0.004118in}{0.015528in}}{\pgfqpoint{0.000000in}{0.015528in}}%
\pgfpathcurveto{\pgfqpoint{-0.004118in}{0.015528in}}{\pgfqpoint{-0.008068in}{0.013892in}}{\pgfqpoint{-0.010980in}{0.010980in}}%
\pgfpathcurveto{\pgfqpoint{-0.013892in}{0.008068in}}{\pgfqpoint{-0.015528in}{0.004118in}}{\pgfqpoint{-0.015528in}{0.000000in}}%
\pgfpathcurveto{\pgfqpoint{-0.015528in}{-0.004118in}}{\pgfqpoint{-0.013892in}{-0.008068in}}{\pgfqpoint{-0.010980in}{-0.010980in}}%
\pgfpathcurveto{\pgfqpoint{-0.008068in}{-0.013892in}}{\pgfqpoint{-0.004118in}{-0.015528in}}{\pgfqpoint{0.000000in}{-0.015528in}}%
\pgfpathclose%
\pgfusepath{stroke,fill}%
}%
\begin{pgfscope}%
\pgfsys@transformshift{2.875294in}{5.254916in}%
\pgfsys@useobject{currentmarker}{}%
\end{pgfscope}%
\begin{pgfscope}%
\pgfsys@transformshift{2.892888in}{5.160724in}%
\pgfsys@useobject{currentmarker}{}%
\end{pgfscope}%
\begin{pgfscope}%
\pgfsys@transformshift{2.910482in}{5.251498in}%
\pgfsys@useobject{currentmarker}{}%
\end{pgfscope}%
\begin{pgfscope}%
\pgfsys@transformshift{2.928076in}{5.270634in}%
\pgfsys@useobject{currentmarker}{}%
\end{pgfscope}%
\begin{pgfscope}%
\pgfsys@transformshift{2.945670in}{5.205849in}%
\pgfsys@useobject{currentmarker}{}%
\end{pgfscope}%
\begin{pgfscope}%
\pgfsys@transformshift{2.963263in}{5.201748in}%
\pgfsys@useobject{currentmarker}{}%
\end{pgfscope}%
\begin{pgfscope}%
\pgfsys@transformshift{2.980857in}{5.077991in}%
\pgfsys@useobject{currentmarker}{}%
\end{pgfscope}%
\begin{pgfscope}%
\pgfsys@transformshift{2.998451in}{5.077595in}%
\pgfsys@useobject{currentmarker}{}%
\end{pgfscope}%
\begin{pgfscope}%
\pgfsys@transformshift{3.016045in}{5.018268in}%
\pgfsys@useobject{currentmarker}{}%
\end{pgfscope}%
\begin{pgfscope}%
\pgfsys@transformshift{3.033639in}{5.022982in}%
\pgfsys@useobject{currentmarker}{}%
\end{pgfscope}%
\begin{pgfscope}%
\pgfsys@transformshift{3.051233in}{4.950281in}%
\pgfsys@useobject{currentmarker}{}%
\end{pgfscope}%
\begin{pgfscope}%
\pgfsys@transformshift{3.068826in}{4.831659in}%
\pgfsys@useobject{currentmarker}{}%
\end{pgfscope}%
\begin{pgfscope}%
\pgfsys@transformshift{3.086420in}{5.001404in}%
\pgfsys@useobject{currentmarker}{}%
\end{pgfscope}%
\begin{pgfscope}%
\pgfsys@transformshift{3.104014in}{4.919253in}%
\pgfsys@useobject{currentmarker}{}%
\end{pgfscope}%
\begin{pgfscope}%
\pgfsys@transformshift{3.121608in}{4.772358in}%
\pgfsys@useobject{currentmarker}{}%
\end{pgfscope}%
\begin{pgfscope}%
\pgfsys@transformshift{3.139202in}{4.965761in}%
\pgfsys@useobject{currentmarker}{}%
\end{pgfscope}%
\begin{pgfscope}%
\pgfsys@transformshift{3.156796in}{4.807758in}%
\pgfsys@useobject{currentmarker}{}%
\end{pgfscope}%
\begin{pgfscope}%
\pgfsys@transformshift{3.174390in}{4.889135in}%
\pgfsys@useobject{currentmarker}{}%
\end{pgfscope}%
\begin{pgfscope}%
\pgfsys@transformshift{3.191983in}{4.943692in}%
\pgfsys@useobject{currentmarker}{}%
\end{pgfscope}%
\begin{pgfscope}%
\pgfsys@transformshift{3.209577in}{4.870028in}%
\pgfsys@useobject{currentmarker}{}%
\end{pgfscope}%
\begin{pgfscope}%
\pgfsys@transformshift{3.227171in}{4.962679in}%
\pgfsys@useobject{currentmarker}{}%
\end{pgfscope}%
\begin{pgfscope}%
\pgfsys@transformshift{3.244765in}{4.712306in}%
\pgfsys@useobject{currentmarker}{}%
\end{pgfscope}%
\begin{pgfscope}%
\pgfsys@transformshift{3.262359in}{4.873130in}%
\pgfsys@useobject{currentmarker}{}%
\end{pgfscope}%
\begin{pgfscope}%
\pgfsys@transformshift{3.279953in}{4.758281in}%
\pgfsys@useobject{currentmarker}{}%
\end{pgfscope}%
\begin{pgfscope}%
\pgfsys@transformshift{3.297547in}{4.737440in}%
\pgfsys@useobject{currentmarker}{}%
\end{pgfscope}%
\begin{pgfscope}%
\pgfsys@transformshift{3.315140in}{4.767404in}%
\pgfsys@useobject{currentmarker}{}%
\end{pgfscope}%
\begin{pgfscope}%
\pgfsys@transformshift{3.332734in}{4.796858in}%
\pgfsys@useobject{currentmarker}{}%
\end{pgfscope}%
\begin{pgfscope}%
\pgfsys@transformshift{3.350328in}{4.838472in}%
\pgfsys@useobject{currentmarker}{}%
\end{pgfscope}%
\begin{pgfscope}%
\pgfsys@transformshift{3.367922in}{4.719865in}%
\pgfsys@useobject{currentmarker}{}%
\end{pgfscope}%
\begin{pgfscope}%
\pgfsys@transformshift{3.385516in}{4.938174in}%
\pgfsys@useobject{currentmarker}{}%
\end{pgfscope}%
\begin{pgfscope}%
\pgfsys@transformshift{3.403110in}{4.902994in}%
\pgfsys@useobject{currentmarker}{}%
\end{pgfscope}%
\begin{pgfscope}%
\pgfsys@transformshift{3.420704in}{4.709459in}%
\pgfsys@useobject{currentmarker}{}%
\end{pgfscope}%
\begin{pgfscope}%
\pgfsys@transformshift{3.438297in}{5.029731in}%
\pgfsys@useobject{currentmarker}{}%
\end{pgfscope}%
\begin{pgfscope}%
\pgfsys@transformshift{3.455891in}{5.084223in}%
\pgfsys@useobject{currentmarker}{}%
\end{pgfscope}%
\begin{pgfscope}%
\pgfsys@transformshift{3.473485in}{5.024905in}%
\pgfsys@useobject{currentmarker}{}%
\end{pgfscope}%
\begin{pgfscope}%
\pgfsys@transformshift{3.491079in}{4.900851in}%
\pgfsys@useobject{currentmarker}{}%
\end{pgfscope}%
\begin{pgfscope}%
\pgfsys@transformshift{3.508673in}{4.824998in}%
\pgfsys@useobject{currentmarker}{}%
\end{pgfscope}%
\begin{pgfscope}%
\pgfsys@transformshift{3.526267in}{5.056986in}%
\pgfsys@useobject{currentmarker}{}%
\end{pgfscope}%
\begin{pgfscope}%
\pgfsys@transformshift{3.543860in}{4.923698in}%
\pgfsys@useobject{currentmarker}{}%
\end{pgfscope}%
\begin{pgfscope}%
\pgfsys@transformshift{3.561454in}{5.104693in}%
\pgfsys@useobject{currentmarker}{}%
\end{pgfscope}%
\begin{pgfscope}%
\pgfsys@transformshift{3.579048in}{5.016167in}%
\pgfsys@useobject{currentmarker}{}%
\end{pgfscope}%
\begin{pgfscope}%
\pgfsys@transformshift{3.596642in}{5.108880in}%
\pgfsys@useobject{currentmarker}{}%
\end{pgfscope}%
\begin{pgfscope}%
\pgfsys@transformshift{3.614236in}{5.059262in}%
\pgfsys@useobject{currentmarker}{}%
\end{pgfscope}%
\begin{pgfscope}%
\pgfsys@transformshift{3.631830in}{5.107695in}%
\pgfsys@useobject{currentmarker}{}%
\end{pgfscope}%
\begin{pgfscope}%
\pgfsys@transformshift{3.649424in}{5.048347in}%
\pgfsys@useobject{currentmarker}{}%
\end{pgfscope}%
\begin{pgfscope}%
\pgfsys@transformshift{3.667017in}{5.239770in}%
\pgfsys@useobject{currentmarker}{}%
\end{pgfscope}%
\begin{pgfscope}%
\pgfsys@transformshift{3.684611in}{5.079579in}%
\pgfsys@useobject{currentmarker}{}%
\end{pgfscope}%
\begin{pgfscope}%
\pgfsys@transformshift{3.702205in}{5.115070in}%
\pgfsys@useobject{currentmarker}{}%
\end{pgfscope}%
\begin{pgfscope}%
\pgfsys@transformshift{3.719799in}{5.271894in}%
\pgfsys@useobject{currentmarker}{}%
\end{pgfscope}%
\begin{pgfscope}%
\pgfsys@transformshift{3.737393in}{4.946452in}%
\pgfsys@useobject{currentmarker}{}%
\end{pgfscope}%
\begin{pgfscope}%
\pgfsys@transformshift{3.754987in}{4.956515in}%
\pgfsys@useobject{currentmarker}{}%
\end{pgfscope}%
\begin{pgfscope}%
\pgfsys@transformshift{3.772581in}{5.185202in}%
\pgfsys@useobject{currentmarker}{}%
\end{pgfscope}%
\begin{pgfscope}%
\pgfsys@transformshift{3.790174in}{4.965051in}%
\pgfsys@useobject{currentmarker}{}%
\end{pgfscope}%
\begin{pgfscope}%
\pgfsys@transformshift{3.807768in}{5.279234in}%
\pgfsys@useobject{currentmarker}{}%
\end{pgfscope}%
\begin{pgfscope}%
\pgfsys@transformshift{3.825362in}{5.033240in}%
\pgfsys@useobject{currentmarker}{}%
\end{pgfscope}%
\begin{pgfscope}%
\pgfsys@transformshift{3.842956in}{4.991625in}%
\pgfsys@useobject{currentmarker}{}%
\end{pgfscope}%
\begin{pgfscope}%
\pgfsys@transformshift{3.860550in}{5.254445in}%
\pgfsys@useobject{currentmarker}{}%
\end{pgfscope}%
\begin{pgfscope}%
\pgfsys@transformshift{3.878144in}{5.197982in}%
\pgfsys@useobject{currentmarker}{}%
\end{pgfscope}%
\begin{pgfscope}%
\pgfsys@transformshift{3.895738in}{5.224324in}%
\pgfsys@useobject{currentmarker}{}%
\end{pgfscope}%
\begin{pgfscope}%
\pgfsys@transformshift{3.913331in}{5.111466in}%
\pgfsys@useobject{currentmarker}{}%
\end{pgfscope}%
\begin{pgfscope}%
\pgfsys@transformshift{3.930925in}{4.914878in}%
\pgfsys@useobject{currentmarker}{}%
\end{pgfscope}%
\begin{pgfscope}%
\pgfsys@transformshift{3.948519in}{5.179668in}%
\pgfsys@useobject{currentmarker}{}%
\end{pgfscope}%
\begin{pgfscope}%
\pgfsys@transformshift{3.966113in}{4.938468in}%
\pgfsys@useobject{currentmarker}{}%
\end{pgfscope}%
\begin{pgfscope}%
\pgfsys@transformshift{3.983707in}{5.027406in}%
\pgfsys@useobject{currentmarker}{}%
\end{pgfscope}%
\begin{pgfscope}%
\pgfsys@transformshift{4.001301in}{5.020999in}%
\pgfsys@useobject{currentmarker}{}%
\end{pgfscope}%
\begin{pgfscope}%
\pgfsys@transformshift{4.018894in}{4.886656in}%
\pgfsys@useobject{currentmarker}{}%
\end{pgfscope}%
\begin{pgfscope}%
\pgfsys@transformshift{4.036488in}{4.942565in}%
\pgfsys@useobject{currentmarker}{}%
\end{pgfscope}%
\begin{pgfscope}%
\pgfsys@transformshift{4.054082in}{4.951091in}%
\pgfsys@useobject{currentmarker}{}%
\end{pgfscope}%
\begin{pgfscope}%
\pgfsys@transformshift{4.071676in}{4.872367in}%
\pgfsys@useobject{currentmarker}{}%
\end{pgfscope}%
\begin{pgfscope}%
\pgfsys@transformshift{4.089270in}{4.698835in}%
\pgfsys@useobject{currentmarker}{}%
\end{pgfscope}%
\begin{pgfscope}%
\pgfsys@transformshift{4.106864in}{4.818561in}%
\pgfsys@useobject{currentmarker}{}%
\end{pgfscope}%
\begin{pgfscope}%
\pgfsys@transformshift{4.124458in}{4.901053in}%
\pgfsys@useobject{currentmarker}{}%
\end{pgfscope}%
\begin{pgfscope}%
\pgfsys@transformshift{4.142051in}{4.673262in}%
\pgfsys@useobject{currentmarker}{}%
\end{pgfscope}%
\begin{pgfscope}%
\pgfsys@transformshift{4.159645in}{4.707965in}%
\pgfsys@useobject{currentmarker}{}%
\end{pgfscope}%
\begin{pgfscope}%
\pgfsys@transformshift{4.177239in}{4.659015in}%
\pgfsys@useobject{currentmarker}{}%
\end{pgfscope}%
\begin{pgfscope}%
\pgfsys@transformshift{4.194833in}{4.873338in}%
\pgfsys@useobject{currentmarker}{}%
\end{pgfscope}%
\begin{pgfscope}%
\pgfsys@transformshift{4.212427in}{4.736067in}%
\pgfsys@useobject{currentmarker}{}%
\end{pgfscope}%
\begin{pgfscope}%
\pgfsys@transformshift{4.230021in}{4.693339in}%
\pgfsys@useobject{currentmarker}{}%
\end{pgfscope}%
\begin{pgfscope}%
\pgfsys@transformshift{4.247615in}{4.559224in}%
\pgfsys@useobject{currentmarker}{}%
\end{pgfscope}%
\begin{pgfscope}%
\pgfsys@transformshift{4.265208in}{4.680465in}%
\pgfsys@useobject{currentmarker}{}%
\end{pgfscope}%
\begin{pgfscope}%
\pgfsys@transformshift{4.282802in}{4.546350in}%
\pgfsys@useobject{currentmarker}{}%
\end{pgfscope}%
\begin{pgfscope}%
\pgfsys@transformshift{4.300396in}{4.609986in}%
\pgfsys@useobject{currentmarker}{}%
\end{pgfscope}%
\begin{pgfscope}%
\pgfsys@transformshift{4.317990in}{4.535582in}%
\pgfsys@useobject{currentmarker}{}%
\end{pgfscope}%
\begin{pgfscope}%
\pgfsys@transformshift{4.335584in}{4.665186in}%
\pgfsys@useobject{currentmarker}{}%
\end{pgfscope}%
\begin{pgfscope}%
\pgfsys@transformshift{4.353178in}{4.652923in}%
\pgfsys@useobject{currentmarker}{}%
\end{pgfscope}%
\begin{pgfscope}%
\pgfsys@transformshift{4.370772in}{4.572944in}%
\pgfsys@useobject{currentmarker}{}%
\end{pgfscope}%
\begin{pgfscope}%
\pgfsys@transformshift{4.388365in}{4.636754in}%
\pgfsys@useobject{currentmarker}{}%
\end{pgfscope}%
\begin{pgfscope}%
\pgfsys@transformshift{4.405959in}{4.489189in}%
\pgfsys@useobject{currentmarker}{}%
\end{pgfscope}%
\begin{pgfscope}%
\pgfsys@transformshift{4.423553in}{4.454903in}%
\pgfsys@useobject{currentmarker}{}%
\end{pgfscope}%
\begin{pgfscope}%
\pgfsys@transformshift{4.441147in}{4.660065in}%
\pgfsys@useobject{currentmarker}{}%
\end{pgfscope}%
\begin{pgfscope}%
\pgfsys@transformshift{4.458741in}{4.642414in}%
\pgfsys@useobject{currentmarker}{}%
\end{pgfscope}%
\begin{pgfscope}%
\pgfsys@transformshift{4.476335in}{4.702096in}%
\pgfsys@useobject{currentmarker}{}%
\end{pgfscope}%
\begin{pgfscope}%
\pgfsys@transformshift{4.493928in}{4.893914in}%
\pgfsys@useobject{currentmarker}{}%
\end{pgfscope}%
\begin{pgfscope}%
\pgfsys@transformshift{4.511522in}{4.762255in}%
\pgfsys@useobject{currentmarker}{}%
\end{pgfscope}%
\begin{pgfscope}%
\pgfsys@transformshift{4.529116in}{4.589240in}%
\pgfsys@useobject{currentmarker}{}%
\end{pgfscope}%
\begin{pgfscope}%
\pgfsys@transformshift{4.546710in}{4.813773in}%
\pgfsys@useobject{currentmarker}{}%
\end{pgfscope}%
\begin{pgfscope}%
\pgfsys@transformshift{4.564304in}{4.584205in}%
\pgfsys@useobject{currentmarker}{}%
\end{pgfscope}%
\begin{pgfscope}%
\pgfsys@transformshift{4.581898in}{4.690640in}%
\pgfsys@useobject{currentmarker}{}%
\end{pgfscope}%
\begin{pgfscope}%
\pgfsys@transformshift{4.599492in}{4.750662in}%
\pgfsys@useobject{currentmarker}{}%
\end{pgfscope}%
\begin{pgfscope}%
\pgfsys@transformshift{4.617085in}{4.952640in}%
\pgfsys@useobject{currentmarker}{}%
\end{pgfscope}%
\begin{pgfscope}%
\pgfsys@transformshift{4.634679in}{4.722467in}%
\pgfsys@useobject{currentmarker}{}%
\end{pgfscope}%
\begin{pgfscope}%
\pgfsys@transformshift{4.652273in}{4.734605in}%
\pgfsys@useobject{currentmarker}{}%
\end{pgfscope}%
\begin{pgfscope}%
\pgfsys@transformshift{4.669867in}{4.829081in}%
\pgfsys@useobject{currentmarker}{}%
\end{pgfscope}%
\begin{pgfscope}%
\pgfsys@transformshift{4.687461in}{4.791275in}%
\pgfsys@useobject{currentmarker}{}%
\end{pgfscope}%
\begin{pgfscope}%
\pgfsys@transformshift{4.705055in}{4.993005in}%
\pgfsys@useobject{currentmarker}{}%
\end{pgfscope}%
\begin{pgfscope}%
\pgfsys@transformshift{4.722649in}{4.786360in}%
\pgfsys@useobject{currentmarker}{}%
\end{pgfscope}%
\begin{pgfscope}%
\pgfsys@transformshift{4.740242in}{4.796843in}%
\pgfsys@useobject{currentmarker}{}%
\end{pgfscope}%
\begin{pgfscope}%
\pgfsys@transformshift{4.757836in}{4.885409in}%
\pgfsys@useobject{currentmarker}{}%
\end{pgfscope}%
\begin{pgfscope}%
\pgfsys@transformshift{4.775430in}{4.894144in}%
\pgfsys@useobject{currentmarker}{}%
\end{pgfscope}%
\begin{pgfscope}%
\pgfsys@transformshift{4.793024in}{5.155096in}%
\pgfsys@useobject{currentmarker}{}%
\end{pgfscope}%
\begin{pgfscope}%
\pgfsys@transformshift{4.810618in}{5.066958in}%
\pgfsys@useobject{currentmarker}{}%
\end{pgfscope}%
\begin{pgfscope}%
\pgfsys@transformshift{4.828212in}{4.989169in}%
\pgfsys@useobject{currentmarker}{}%
\end{pgfscope}%
\begin{pgfscope}%
\pgfsys@transformshift{4.845805in}{4.863576in}%
\pgfsys@useobject{currentmarker}{}%
\end{pgfscope}%
\begin{pgfscope}%
\pgfsys@transformshift{4.863399in}{5.081064in}%
\pgfsys@useobject{currentmarker}{}%
\end{pgfscope}%
\begin{pgfscope}%
\pgfsys@transformshift{4.880993in}{4.897460in}%
\pgfsys@useobject{currentmarker}{}%
\end{pgfscope}%
\begin{pgfscope}%
\pgfsys@transformshift{4.898587in}{4.844460in}%
\pgfsys@useobject{currentmarker}{}%
\end{pgfscope}%
\begin{pgfscope}%
\pgfsys@transformshift{4.916181in}{5.123690in}%
\pgfsys@useobject{currentmarker}{}%
\end{pgfscope}%
\begin{pgfscope}%
\pgfsys@transformshift{4.933775in}{5.033449in}%
\pgfsys@useobject{currentmarker}{}%
\end{pgfscope}%
\begin{pgfscope}%
\pgfsys@transformshift{4.951369in}{5.091681in}%
\pgfsys@useobject{currentmarker}{}%
\end{pgfscope}%
\begin{pgfscope}%
\pgfsys@transformshift{4.968962in}{5.025036in}%
\pgfsys@useobject{currentmarker}{}%
\end{pgfscope}%
\begin{pgfscope}%
\pgfsys@transformshift{4.986556in}{5.072843in}%
\pgfsys@useobject{currentmarker}{}%
\end{pgfscope}%
\begin{pgfscope}%
\pgfsys@transformshift{5.004150in}{4.910283in}%
\pgfsys@useobject{currentmarker}{}%
\end{pgfscope}%
\begin{pgfscope}%
\pgfsys@transformshift{5.021744in}{4.860777in}%
\pgfsys@useobject{currentmarker}{}%
\end{pgfscope}%
\begin{pgfscope}%
\pgfsys@transformshift{5.039338in}{5.023816in}%
\pgfsys@useobject{currentmarker}{}%
\end{pgfscope}%
\begin{pgfscope}%
\pgfsys@transformshift{5.056932in}{4.859088in}%
\pgfsys@useobject{currentmarker}{}%
\end{pgfscope}%
\begin{pgfscope}%
\pgfsys@transformshift{5.074526in}{4.856192in}%
\pgfsys@useobject{currentmarker}{}%
\end{pgfscope}%
\begin{pgfscope}%
\pgfsys@transformshift{5.092119in}{4.864505in}%
\pgfsys@useobject{currentmarker}{}%
\end{pgfscope}%
\begin{pgfscope}%
\pgfsys@transformshift{5.109713in}{4.896325in}%
\pgfsys@useobject{currentmarker}{}%
\end{pgfscope}%
\begin{pgfscope}%
\pgfsys@transformshift{5.127307in}{4.841352in}%
\pgfsys@useobject{currentmarker}{}%
\end{pgfscope}%
\begin{pgfscope}%
\pgfsys@transformshift{5.144901in}{4.719668in}%
\pgfsys@useobject{currentmarker}{}%
\end{pgfscope}%
\begin{pgfscope}%
\pgfsys@transformshift{5.162495in}{4.776392in}%
\pgfsys@useobject{currentmarker}{}%
\end{pgfscope}%
\begin{pgfscope}%
\pgfsys@transformshift{5.180089in}{4.597368in}%
\pgfsys@useobject{currentmarker}{}%
\end{pgfscope}%
\begin{pgfscope}%
\pgfsys@transformshift{5.197683in}{4.870059in}%
\pgfsys@useobject{currentmarker}{}%
\end{pgfscope}%
\begin{pgfscope}%
\pgfsys@transformshift{5.215276in}{4.625477in}%
\pgfsys@useobject{currentmarker}{}%
\end{pgfscope}%
\begin{pgfscope}%
\pgfsys@transformshift{5.232870in}{4.659313in}%
\pgfsys@useobject{currentmarker}{}%
\end{pgfscope}%
\begin{pgfscope}%
\pgfsys@transformshift{5.250464in}{4.761077in}%
\pgfsys@useobject{currentmarker}{}%
\end{pgfscope}%
\begin{pgfscope}%
\pgfsys@transformshift{5.268058in}{4.665102in}%
\pgfsys@useobject{currentmarker}{}%
\end{pgfscope}%
\begin{pgfscope}%
\pgfsys@transformshift{5.285652in}{4.883742in}%
\pgfsys@useobject{currentmarker}{}%
\end{pgfscope}%
\begin{pgfscope}%
\pgfsys@transformshift{5.303246in}{4.581725in}%
\pgfsys@useobject{currentmarker}{}%
\end{pgfscope}%
\begin{pgfscope}%
\pgfsys@transformshift{5.320839in}{4.729414in}%
\pgfsys@useobject{currentmarker}{}%
\end{pgfscope}%
\begin{pgfscope}%
\pgfsys@transformshift{5.338433in}{4.688394in}%
\pgfsys@useobject{currentmarker}{}%
\end{pgfscope}%
\begin{pgfscope}%
\pgfsys@transformshift{5.356027in}{4.565234in}%
\pgfsys@useobject{currentmarker}{}%
\end{pgfscope}%
\begin{pgfscope}%
\pgfsys@transformshift{5.373621in}{4.731505in}%
\pgfsys@useobject{currentmarker}{}%
\end{pgfscope}%
\begin{pgfscope}%
\pgfsys@transformshift{5.391215in}{4.656390in}%
\pgfsys@useobject{currentmarker}{}%
\end{pgfscope}%
\begin{pgfscope}%
\pgfsys@transformshift{5.408809in}{4.750347in}%
\pgfsys@useobject{currentmarker}{}%
\end{pgfscope}%
\begin{pgfscope}%
\pgfsys@transformshift{5.426403in}{4.755463in}%
\pgfsys@useobject{currentmarker}{}%
\end{pgfscope}%
\begin{pgfscope}%
\pgfsys@transformshift{5.443996in}{4.894045in}%
\pgfsys@useobject{currentmarker}{}%
\end{pgfscope}%
\begin{pgfscope}%
\pgfsys@transformshift{5.461590in}{4.813845in}%
\pgfsys@useobject{currentmarker}{}%
\end{pgfscope}%
\begin{pgfscope}%
\pgfsys@transformshift{5.479184in}{4.646147in}%
\pgfsys@useobject{currentmarker}{}%
\end{pgfscope}%
\begin{pgfscope}%
\pgfsys@transformshift{5.496778in}{4.667737in}%
\pgfsys@useobject{currentmarker}{}%
\end{pgfscope}%
\begin{pgfscope}%
\pgfsys@transformshift{5.514372in}{4.814708in}%
\pgfsys@useobject{currentmarker}{}%
\end{pgfscope}%
\begin{pgfscope}%
\pgfsys@transformshift{5.531966in}{4.781821in}%
\pgfsys@useobject{currentmarker}{}%
\end{pgfscope}%
\begin{pgfscope}%
\pgfsys@transformshift{5.549560in}{4.794632in}%
\pgfsys@useobject{currentmarker}{}%
\end{pgfscope}%
\begin{pgfscope}%
\pgfsys@transformshift{5.567153in}{4.580644in}%
\pgfsys@useobject{currentmarker}{}%
\end{pgfscope}%
\begin{pgfscope}%
\pgfsys@transformshift{5.584747in}{4.760929in}%
\pgfsys@useobject{currentmarker}{}%
\end{pgfscope}%
\begin{pgfscope}%
\pgfsys@transformshift{5.602341in}{4.707595in}%
\pgfsys@useobject{currentmarker}{}%
\end{pgfscope}%
\begin{pgfscope}%
\pgfsys@transformshift{5.619935in}{4.832262in}%
\pgfsys@useobject{currentmarker}{}%
\end{pgfscope}%
\begin{pgfscope}%
\pgfsys@transformshift{5.637529in}{4.815823in}%
\pgfsys@useobject{currentmarker}{}%
\end{pgfscope}%
\begin{pgfscope}%
\pgfsys@transformshift{5.655123in}{4.941839in}%
\pgfsys@useobject{currentmarker}{}%
\end{pgfscope}%
\begin{pgfscope}%
\pgfsys@transformshift{5.672717in}{4.905455in}%
\pgfsys@useobject{currentmarker}{}%
\end{pgfscope}%
\begin{pgfscope}%
\pgfsys@transformshift{5.690310in}{4.978072in}%
\pgfsys@useobject{currentmarker}{}%
\end{pgfscope}%
\begin{pgfscope}%
\pgfsys@transformshift{5.707904in}{4.875602in}%
\pgfsys@useobject{currentmarker}{}%
\end{pgfscope}%
\begin{pgfscope}%
\pgfsys@transformshift{5.725498in}{4.852426in}%
\pgfsys@useobject{currentmarker}{}%
\end{pgfscope}%
\begin{pgfscope}%
\pgfsys@transformshift{5.743092in}{4.932576in}%
\pgfsys@useobject{currentmarker}{}%
\end{pgfscope}%
\begin{pgfscope}%
\pgfsys@transformshift{5.760686in}{4.998190in}%
\pgfsys@useobject{currentmarker}{}%
\end{pgfscope}%
\begin{pgfscope}%
\pgfsys@transformshift{5.778280in}{5.063799in}%
\pgfsys@useobject{currentmarker}{}%
\end{pgfscope}%
\begin{pgfscope}%
\pgfsys@transformshift{5.795873in}{5.280211in}%
\pgfsys@useobject{currentmarker}{}%
\end{pgfscope}%
\begin{pgfscope}%
\pgfsys@transformshift{5.813467in}{5.069477in}%
\pgfsys@useobject{currentmarker}{}%
\end{pgfscope}%
\begin{pgfscope}%
\pgfsys@transformshift{5.831061in}{4.999365in}%
\pgfsys@useobject{currentmarker}{}%
\end{pgfscope}%
\begin{pgfscope}%
\pgfsys@transformshift{5.848655in}{5.083543in}%
\pgfsys@useobject{currentmarker}{}%
\end{pgfscope}%
\begin{pgfscope}%
\pgfsys@transformshift{5.866249in}{5.092281in}%
\pgfsys@useobject{currentmarker}{}%
\end{pgfscope}%
\begin{pgfscope}%
\pgfsys@transformshift{5.883843in}{5.207988in}%
\pgfsys@useobject{currentmarker}{}%
\end{pgfscope}%
\begin{pgfscope}%
\pgfsys@transformshift{5.901437in}{5.019570in}%
\pgfsys@useobject{currentmarker}{}%
\end{pgfscope}%
\begin{pgfscope}%
\pgfsys@transformshift{5.919030in}{5.199289in}%
\pgfsys@useobject{currentmarker}{}%
\end{pgfscope}%
\begin{pgfscope}%
\pgfsys@transformshift{5.936624in}{5.223190in}%
\pgfsys@useobject{currentmarker}{}%
\end{pgfscope}%
\begin{pgfscope}%
\pgfsys@transformshift{5.954218in}{5.243459in}%
\pgfsys@useobject{currentmarker}{}%
\end{pgfscope}%
\begin{pgfscope}%
\pgfsys@transformshift{5.971812in}{5.169522in}%
\pgfsys@useobject{currentmarker}{}%
\end{pgfscope}%
\begin{pgfscope}%
\pgfsys@transformshift{5.989406in}{5.214873in}%
\pgfsys@useobject{currentmarker}{}%
\end{pgfscope}%
\begin{pgfscope}%
\pgfsys@transformshift{6.007000in}{5.100648in}%
\pgfsys@useobject{currentmarker}{}%
\end{pgfscope}%
\begin{pgfscope}%
\pgfsys@transformshift{6.024594in}{5.200314in}%
\pgfsys@useobject{currentmarker}{}%
\end{pgfscope}%
\begin{pgfscope}%
\pgfsys@transformshift{6.042187in}{5.198061in}%
\pgfsys@useobject{currentmarker}{}%
\end{pgfscope}%
\begin{pgfscope}%
\pgfsys@transformshift{6.059781in}{5.296725in}%
\pgfsys@useobject{currentmarker}{}%
\end{pgfscope}%
\begin{pgfscope}%
\pgfsys@transformshift{6.077375in}{5.135419in}%
\pgfsys@useobject{currentmarker}{}%
\end{pgfscope}%
\begin{pgfscope}%
\pgfsys@transformshift{6.094969in}{5.330218in}%
\pgfsys@useobject{currentmarker}{}%
\end{pgfscope}%
\begin{pgfscope}%
\pgfsys@transformshift{6.112563in}{5.398508in}%
\pgfsys@useobject{currentmarker}{}%
\end{pgfscope}%
\begin{pgfscope}%
\pgfsys@transformshift{6.130157in}{5.029008in}%
\pgfsys@useobject{currentmarker}{}%
\end{pgfscope}%
\begin{pgfscope}%
\pgfsys@transformshift{6.147751in}{5.276102in}%
\pgfsys@useobject{currentmarker}{}%
\end{pgfscope}%
\begin{pgfscope}%
\pgfsys@transformshift{6.165344in}{5.292913in}%
\pgfsys@useobject{currentmarker}{}%
\end{pgfscope}%
\begin{pgfscope}%
\pgfsys@transformshift{6.182938in}{5.149004in}%
\pgfsys@useobject{currentmarker}{}%
\end{pgfscope}%
\begin{pgfscope}%
\pgfsys@transformshift{6.200532in}{5.162656in}%
\pgfsys@useobject{currentmarker}{}%
\end{pgfscope}%
\begin{pgfscope}%
\pgfsys@transformshift{6.218126in}{5.178002in}%
\pgfsys@useobject{currentmarker}{}%
\end{pgfscope}%
\begin{pgfscope}%
\pgfsys@transformshift{6.235720in}{5.148988in}%
\pgfsys@useobject{currentmarker}{}%
\end{pgfscope}%
\begin{pgfscope}%
\pgfsys@transformshift{6.253314in}{5.135262in}%
\pgfsys@useobject{currentmarker}{}%
\end{pgfscope}%
\begin{pgfscope}%
\pgfsys@transformshift{6.270907in}{4.983102in}%
\pgfsys@useobject{currentmarker}{}%
\end{pgfscope}%
\begin{pgfscope}%
\pgfsys@transformshift{6.288501in}{5.258780in}%
\pgfsys@useobject{currentmarker}{}%
\end{pgfscope}%
\begin{pgfscope}%
\pgfsys@transformshift{6.306095in}{5.238859in}%
\pgfsys@useobject{currentmarker}{}%
\end{pgfscope}%
\begin{pgfscope}%
\pgfsys@transformshift{6.323689in}{5.033756in}%
\pgfsys@useobject{currentmarker}{}%
\end{pgfscope}%
\begin{pgfscope}%
\pgfsys@transformshift{6.341283in}{4.955731in}%
\pgfsys@useobject{currentmarker}{}%
\end{pgfscope}%
\begin{pgfscope}%
\pgfsys@transformshift{6.358877in}{5.147808in}%
\pgfsys@useobject{currentmarker}{}%
\end{pgfscope}%
\begin{pgfscope}%
\pgfsys@transformshift{6.376471in}{5.026386in}%
\pgfsys@useobject{currentmarker}{}%
\end{pgfscope}%
\end{pgfscope}%
\begin{pgfscope}%
\pgfpathrectangle{\pgfqpoint{2.000000in}{4.421053in}}{\pgfqpoint{4.376471in}{0.978947in}} %
\pgfusepath{clip}%
\pgfsetroundcap%
\pgfsetroundjoin%
\pgfsetlinewidth{1.756562pt}%
\definecolor{currentstroke}{rgb}{0.298039,0.447059,0.690196}%
\pgfsetstrokecolor{currentstroke}%
\pgfsetdash{}{0pt}%
\pgfpathmoveto{\pgfqpoint{2.875294in}{5.012215in}}%
\pgfpathlineto{\pgfqpoint{2.892888in}{5.088757in}}%
\pgfpathlineto{\pgfqpoint{2.910482in}{5.141372in}}%
\pgfpathlineto{\pgfqpoint{2.928076in}{5.173971in}}%
\pgfpathlineto{\pgfqpoint{2.945670in}{5.190063in}}%
\pgfpathlineto{\pgfqpoint{2.963263in}{5.192784in}}%
\pgfpathlineto{\pgfqpoint{2.980857in}{5.184906in}}%
\pgfpathlineto{\pgfqpoint{2.998451in}{5.168865in}}%
\pgfpathlineto{\pgfqpoint{3.016045in}{5.146780in}}%
\pgfpathlineto{\pgfqpoint{3.033639in}{5.120474in}}%
\pgfpathlineto{\pgfqpoint{3.068826in}{5.061153in}}%
\pgfpathlineto{\pgfqpoint{3.104014in}{5.000442in}}%
\pgfpathlineto{\pgfqpoint{3.139202in}{4.944588in}}%
\pgfpathlineto{\pgfqpoint{3.156796in}{4.919700in}}%
\pgfpathlineto{\pgfqpoint{3.174390in}{4.897217in}}%
\pgfpathlineto{\pgfqpoint{3.191983in}{4.877285in}}%
\pgfpathlineto{\pgfqpoint{3.209577in}{4.859960in}}%
\pgfpathlineto{\pgfqpoint{3.227171in}{4.845224in}}%
\pgfpathlineto{\pgfqpoint{3.244765in}{4.833000in}}%
\pgfpathlineto{\pgfqpoint{3.262359in}{4.823163in}}%
\pgfpathlineto{\pgfqpoint{3.279953in}{4.815553in}}%
\pgfpathlineto{\pgfqpoint{3.297547in}{4.809986in}}%
\pgfpathlineto{\pgfqpoint{3.315140in}{4.806261in}}%
\pgfpathlineto{\pgfqpoint{3.332734in}{4.804168in}}%
\pgfpathlineto{\pgfqpoint{3.350328in}{4.803496in}}%
\pgfpathlineto{\pgfqpoint{3.385516in}{4.805599in}}%
\pgfpathlineto{\pgfqpoint{3.420704in}{4.811030in}}%
\pgfpathlineto{\pgfqpoint{3.473485in}{4.822597in}}%
\pgfpathlineto{\pgfqpoint{3.579048in}{4.847219in}}%
\pgfpathlineto{\pgfqpoint{3.631830in}{4.856922in}}%
\pgfpathlineto{\pgfqpoint{3.684611in}{4.863781in}}%
\pgfpathlineto{\pgfqpoint{3.719799in}{4.866504in}}%
\pgfpathlineto{\pgfqpoint{3.754987in}{4.867540in}}%
\pgfpathlineto{\pgfqpoint{3.790174in}{4.866676in}}%
\pgfpathlineto{\pgfqpoint{3.825362in}{4.863669in}}%
\pgfpathlineto{\pgfqpoint{3.860550in}{4.858264in}}%
\pgfpathlineto{\pgfqpoint{3.895738in}{4.850221in}}%
\pgfpathlineto{\pgfqpoint{3.930925in}{4.839353in}}%
\pgfpathlineto{\pgfqpoint{3.966113in}{4.825562in}}%
\pgfpathlineto{\pgfqpoint{4.001301in}{4.808878in}}%
\pgfpathlineto{\pgfqpoint{4.036488in}{4.789491in}}%
\pgfpathlineto{\pgfqpoint{4.071676in}{4.767773in}}%
\pgfpathlineto{\pgfqpoint{4.124458in}{4.732104in}}%
\pgfpathlineto{\pgfqpoint{4.212427in}{4.671479in}}%
\pgfpathlineto{\pgfqpoint{4.247615in}{4.649868in}}%
\pgfpathlineto{\pgfqpoint{4.282802in}{4.631480in}}%
\pgfpathlineto{\pgfqpoint{4.300396in}{4.623845in}}%
\pgfpathlineto{\pgfqpoint{4.317990in}{4.617428in}}%
\pgfpathlineto{\pgfqpoint{4.335584in}{4.612349in}}%
\pgfpathlineto{\pgfqpoint{4.353178in}{4.608719in}}%
\pgfpathlineto{\pgfqpoint{4.370772in}{4.606636in}}%
\pgfpathlineto{\pgfqpoint{4.388365in}{4.606182in}}%
\pgfpathlineto{\pgfqpoint{4.405959in}{4.607424in}}%
\pgfpathlineto{\pgfqpoint{4.423553in}{4.610409in}}%
\pgfpathlineto{\pgfqpoint{4.441147in}{4.615167in}}%
\pgfpathlineto{\pgfqpoint{4.458741in}{4.621705in}}%
\pgfpathlineto{\pgfqpoint{4.476335in}{4.630010in}}%
\pgfpathlineto{\pgfqpoint{4.493928in}{4.640047in}}%
\pgfpathlineto{\pgfqpoint{4.511522in}{4.651760in}}%
\pgfpathlineto{\pgfqpoint{4.529116in}{4.665069in}}%
\pgfpathlineto{\pgfqpoint{4.564304in}{4.696053in}}%
\pgfpathlineto{\pgfqpoint{4.599492in}{4.731958in}}%
\pgfpathlineto{\pgfqpoint{4.634679in}{4.771446in}}%
\pgfpathlineto{\pgfqpoint{4.740242in}{4.894973in}}%
\pgfpathlineto{\pgfqpoint{4.775430in}{4.931879in}}%
\pgfpathlineto{\pgfqpoint{4.810618in}{4.963748in}}%
\pgfpathlineto{\pgfqpoint{4.828212in}{4.977315in}}%
\pgfpathlineto{\pgfqpoint{4.845805in}{4.989079in}}%
\pgfpathlineto{\pgfqpoint{4.863399in}{4.998897in}}%
\pgfpathlineto{\pgfqpoint{4.880993in}{5.006647in}}%
\pgfpathlineto{\pgfqpoint{4.898587in}{5.012234in}}%
\pgfpathlineto{\pgfqpoint{4.916181in}{5.015583in}}%
\pgfpathlineto{\pgfqpoint{4.933775in}{5.016651in}}%
\pgfpathlineto{\pgfqpoint{4.951369in}{5.015419in}}%
\pgfpathlineto{\pgfqpoint{4.968962in}{5.011898in}}%
\pgfpathlineto{\pgfqpoint{4.986556in}{5.006127in}}%
\pgfpathlineto{\pgfqpoint{5.004150in}{4.998173in}}%
\pgfpathlineto{\pgfqpoint{5.021744in}{4.988130in}}%
\pgfpathlineto{\pgfqpoint{5.039338in}{4.976119in}}%
\pgfpathlineto{\pgfqpoint{5.056932in}{4.962287in}}%
\pgfpathlineto{\pgfqpoint{5.092119in}{4.929859in}}%
\pgfpathlineto{\pgfqpoint{5.127307in}{4.892453in}}%
\pgfpathlineto{\pgfqpoint{5.180089in}{4.831159in}}%
\pgfpathlineto{\pgfqpoint{5.232870in}{4.769878in}}%
\pgfpathlineto{\pgfqpoint{5.268058in}{4.732540in}}%
\pgfpathlineto{\pgfqpoint{5.303246in}{4.700317in}}%
\pgfpathlineto{\pgfqpoint{5.320839in}{4.686672in}}%
\pgfpathlineto{\pgfqpoint{5.338433in}{4.674926in}}%
\pgfpathlineto{\pgfqpoint{5.356027in}{4.665245in}}%
\pgfpathlineto{\pgfqpoint{5.373621in}{4.657768in}}%
\pgfpathlineto{\pgfqpoint{5.391215in}{4.652607in}}%
\pgfpathlineto{\pgfqpoint{5.408809in}{4.649850in}}%
\pgfpathlineto{\pgfqpoint{5.426403in}{4.649555in}}%
\pgfpathlineto{\pgfqpoint{5.443996in}{4.651751in}}%
\pgfpathlineto{\pgfqpoint{5.461590in}{4.656437in}}%
\pgfpathlineto{\pgfqpoint{5.479184in}{4.663587in}}%
\pgfpathlineto{\pgfqpoint{5.496778in}{4.673145in}}%
\pgfpathlineto{\pgfqpoint{5.514372in}{4.685027in}}%
\pgfpathlineto{\pgfqpoint{5.531966in}{4.699126in}}%
\pgfpathlineto{\pgfqpoint{5.549560in}{4.715311in}}%
\pgfpathlineto{\pgfqpoint{5.567153in}{4.733428in}}%
\pgfpathlineto{\pgfqpoint{5.602341in}{4.774755in}}%
\pgfpathlineto{\pgfqpoint{5.637529in}{4.821551in}}%
\pgfpathlineto{\pgfqpoint{5.690310in}{4.898197in}}%
\pgfpathlineto{\pgfqpoint{5.778280in}{5.028318in}}%
\pgfpathlineto{\pgfqpoint{5.813467in}{5.076431in}}%
\pgfpathlineto{\pgfqpoint{5.848655in}{5.120284in}}%
\pgfpathlineto{\pgfqpoint{5.883843in}{5.158892in}}%
\pgfpathlineto{\pgfqpoint{5.901437in}{5.175995in}}%
\pgfpathlineto{\pgfqpoint{5.919030in}{5.191540in}}%
\pgfpathlineto{\pgfqpoint{5.936624in}{5.205475in}}%
\pgfpathlineto{\pgfqpoint{5.954218in}{5.217764in}}%
\pgfpathlineto{\pgfqpoint{5.971812in}{5.228380in}}%
\pgfpathlineto{\pgfqpoint{5.989406in}{5.237306in}}%
\pgfpathlineto{\pgfqpoint{6.007000in}{5.244533in}}%
\pgfpathlineto{\pgfqpoint{6.024594in}{5.250054in}}%
\pgfpathlineto{\pgfqpoint{6.042187in}{5.253865in}}%
\pgfpathlineto{\pgfqpoint{6.059781in}{5.255961in}}%
\pgfpathlineto{\pgfqpoint{6.077375in}{5.256336in}}%
\pgfpathlineto{\pgfqpoint{6.094969in}{5.254979in}}%
\pgfpathlineto{\pgfqpoint{6.112563in}{5.251871in}}%
\pgfpathlineto{\pgfqpoint{6.130157in}{5.246990in}}%
\pgfpathlineto{\pgfqpoint{6.147751in}{5.240304in}}%
\pgfpathlineto{\pgfqpoint{6.165344in}{5.231775in}}%
\pgfpathlineto{\pgfqpoint{6.182938in}{5.221359in}}%
\pgfpathlineto{\pgfqpoint{6.200532in}{5.209011in}}%
\pgfpathlineto{\pgfqpoint{6.218126in}{5.194685in}}%
\pgfpathlineto{\pgfqpoint{6.235720in}{5.178340in}}%
\pgfpathlineto{\pgfqpoint{6.253314in}{5.159946in}}%
\pgfpathlineto{\pgfqpoint{6.270907in}{5.139492in}}%
\pgfpathlineto{\pgfqpoint{6.288501in}{5.116994in}}%
\pgfpathlineto{\pgfqpoint{6.323689in}{5.066133in}}%
\pgfpathlineto{\pgfqpoint{6.358877in}{5.008482in}}%
\pgfpathlineto{\pgfqpoint{6.376471in}{4.977794in}}%
\pgfpathlineto{\pgfqpoint{6.376471in}{4.977794in}}%
\pgfusepath{stroke}%
\end{pgfscope}%
\begin{pgfscope}%
\pgfpathrectangle{\pgfqpoint{2.000000in}{4.421053in}}{\pgfqpoint{4.376471in}{0.978947in}} %
\pgfusepath{clip}%
\pgfsetbuttcap%
\pgfsetroundjoin%
\pgfsetlinewidth{1.756562pt}%
\definecolor{currentstroke}{rgb}{1.000000,0.647059,0.000000}%
\pgfsetstrokecolor{currentstroke}%
\pgfsetdash{{6.000000pt}{6.000000pt}}{0.000000pt}%
\pgfpathmoveto{\pgfqpoint{2.875294in}{5.012214in}}%
\pgfpathlineto{\pgfqpoint{2.892888in}{5.088756in}}%
\pgfpathlineto{\pgfqpoint{2.910482in}{5.141372in}}%
\pgfpathlineto{\pgfqpoint{2.928076in}{5.173971in}}%
\pgfpathlineto{\pgfqpoint{2.945670in}{5.190064in}}%
\pgfpathlineto{\pgfqpoint{2.963263in}{5.192784in}}%
\pgfpathlineto{\pgfqpoint{2.980857in}{5.184906in}}%
\pgfpathlineto{\pgfqpoint{2.998451in}{5.168865in}}%
\pgfpathlineto{\pgfqpoint{3.016045in}{5.146780in}}%
\pgfpathlineto{\pgfqpoint{3.033639in}{5.120474in}}%
\pgfpathlineto{\pgfqpoint{3.068826in}{5.061153in}}%
\pgfpathlineto{\pgfqpoint{3.104014in}{5.000442in}}%
\pgfpathlineto{\pgfqpoint{3.139202in}{4.944588in}}%
\pgfpathlineto{\pgfqpoint{3.156796in}{4.919700in}}%
\pgfpathlineto{\pgfqpoint{3.174390in}{4.897217in}}%
\pgfpathlineto{\pgfqpoint{3.191983in}{4.877285in}}%
\pgfpathlineto{\pgfqpoint{3.209577in}{4.859959in}}%
\pgfpathlineto{\pgfqpoint{3.227171in}{4.845223in}}%
\pgfpathlineto{\pgfqpoint{3.244765in}{4.832999in}}%
\pgfpathlineto{\pgfqpoint{3.262359in}{4.823162in}}%
\pgfpathlineto{\pgfqpoint{3.279953in}{4.815553in}}%
\pgfpathlineto{\pgfqpoint{3.297547in}{4.809986in}}%
\pgfpathlineto{\pgfqpoint{3.315140in}{4.806260in}}%
\pgfpathlineto{\pgfqpoint{3.332734in}{4.804168in}}%
\pgfpathlineto{\pgfqpoint{3.350328in}{4.803496in}}%
\pgfpathlineto{\pgfqpoint{3.385516in}{4.805599in}}%
\pgfpathlineto{\pgfqpoint{3.420704in}{4.811030in}}%
\pgfpathlineto{\pgfqpoint{3.473485in}{4.822597in}}%
\pgfpathlineto{\pgfqpoint{3.579048in}{4.847220in}}%
\pgfpathlineto{\pgfqpoint{3.631830in}{4.856923in}}%
\pgfpathlineto{\pgfqpoint{3.684611in}{4.863783in}}%
\pgfpathlineto{\pgfqpoint{3.719799in}{4.866505in}}%
\pgfpathlineto{\pgfqpoint{3.754987in}{4.867541in}}%
\pgfpathlineto{\pgfqpoint{3.790174in}{4.866677in}}%
\pgfpathlineto{\pgfqpoint{3.825362in}{4.863670in}}%
\pgfpathlineto{\pgfqpoint{3.860550in}{4.858265in}}%
\pgfpathlineto{\pgfqpoint{3.895738in}{4.850222in}}%
\pgfpathlineto{\pgfqpoint{3.930925in}{4.839354in}}%
\pgfpathlineto{\pgfqpoint{3.966113in}{4.825562in}}%
\pgfpathlineto{\pgfqpoint{4.001301in}{4.808879in}}%
\pgfpathlineto{\pgfqpoint{4.036488in}{4.789492in}}%
\pgfpathlineto{\pgfqpoint{4.071676in}{4.767773in}}%
\pgfpathlineto{\pgfqpoint{4.124458in}{4.732104in}}%
\pgfpathlineto{\pgfqpoint{4.212427in}{4.671479in}}%
\pgfpathlineto{\pgfqpoint{4.247615in}{4.649868in}}%
\pgfpathlineto{\pgfqpoint{4.282802in}{4.631480in}}%
\pgfpathlineto{\pgfqpoint{4.300396in}{4.623846in}}%
\pgfpathlineto{\pgfqpoint{4.317990in}{4.617428in}}%
\pgfpathlineto{\pgfqpoint{4.335584in}{4.612349in}}%
\pgfpathlineto{\pgfqpoint{4.353178in}{4.608719in}}%
\pgfpathlineto{\pgfqpoint{4.370772in}{4.606636in}}%
\pgfpathlineto{\pgfqpoint{4.388365in}{4.606182in}}%
\pgfpathlineto{\pgfqpoint{4.405959in}{4.607424in}}%
\pgfpathlineto{\pgfqpoint{4.423553in}{4.610409in}}%
\pgfpathlineto{\pgfqpoint{4.441147in}{4.615167in}}%
\pgfpathlineto{\pgfqpoint{4.458741in}{4.621705in}}%
\pgfpathlineto{\pgfqpoint{4.476335in}{4.630010in}}%
\pgfpathlineto{\pgfqpoint{4.493928in}{4.640047in}}%
\pgfpathlineto{\pgfqpoint{4.511522in}{4.651760in}}%
\pgfpathlineto{\pgfqpoint{4.529116in}{4.665069in}}%
\pgfpathlineto{\pgfqpoint{4.564304in}{4.696053in}}%
\pgfpathlineto{\pgfqpoint{4.599492in}{4.731958in}}%
\pgfpathlineto{\pgfqpoint{4.634679in}{4.771446in}}%
\pgfpathlineto{\pgfqpoint{4.740242in}{4.894973in}}%
\pgfpathlineto{\pgfqpoint{4.775430in}{4.931879in}}%
\pgfpathlineto{\pgfqpoint{4.810618in}{4.963749in}}%
\pgfpathlineto{\pgfqpoint{4.828212in}{4.977314in}}%
\pgfpathlineto{\pgfqpoint{4.845805in}{4.989078in}}%
\pgfpathlineto{\pgfqpoint{4.863399in}{4.998897in}}%
\pgfpathlineto{\pgfqpoint{4.880993in}{5.006647in}}%
\pgfpathlineto{\pgfqpoint{4.898587in}{5.012233in}}%
\pgfpathlineto{\pgfqpoint{4.916181in}{5.015583in}}%
\pgfpathlineto{\pgfqpoint{4.933775in}{5.016651in}}%
\pgfpathlineto{\pgfqpoint{4.951369in}{5.015419in}}%
\pgfpathlineto{\pgfqpoint{4.968962in}{5.011898in}}%
\pgfpathlineto{\pgfqpoint{4.986556in}{5.006127in}}%
\pgfpathlineto{\pgfqpoint{5.004150in}{4.998173in}}%
\pgfpathlineto{\pgfqpoint{5.021744in}{4.988130in}}%
\pgfpathlineto{\pgfqpoint{5.039338in}{4.976119in}}%
\pgfpathlineto{\pgfqpoint{5.056932in}{4.962287in}}%
\pgfpathlineto{\pgfqpoint{5.092119in}{4.929859in}}%
\pgfpathlineto{\pgfqpoint{5.127307in}{4.892453in}}%
\pgfpathlineto{\pgfqpoint{5.180089in}{4.831159in}}%
\pgfpathlineto{\pgfqpoint{5.232870in}{4.769878in}}%
\pgfpathlineto{\pgfqpoint{5.268058in}{4.732540in}}%
\pgfpathlineto{\pgfqpoint{5.303246in}{4.700317in}}%
\pgfpathlineto{\pgfqpoint{5.320839in}{4.686671in}}%
\pgfpathlineto{\pgfqpoint{5.338433in}{4.674926in}}%
\pgfpathlineto{\pgfqpoint{5.356027in}{4.665245in}}%
\pgfpathlineto{\pgfqpoint{5.373621in}{4.657767in}}%
\pgfpathlineto{\pgfqpoint{5.391215in}{4.652607in}}%
\pgfpathlineto{\pgfqpoint{5.408809in}{4.649850in}}%
\pgfpathlineto{\pgfqpoint{5.426403in}{4.649555in}}%
\pgfpathlineto{\pgfqpoint{5.443996in}{4.651750in}}%
\pgfpathlineto{\pgfqpoint{5.461590in}{4.656437in}}%
\pgfpathlineto{\pgfqpoint{5.479184in}{4.663587in}}%
\pgfpathlineto{\pgfqpoint{5.496778in}{4.673145in}}%
\pgfpathlineto{\pgfqpoint{5.514372in}{4.685027in}}%
\pgfpathlineto{\pgfqpoint{5.531966in}{4.699126in}}%
\pgfpathlineto{\pgfqpoint{5.549560in}{4.715310in}}%
\pgfpathlineto{\pgfqpoint{5.567153in}{4.733428in}}%
\pgfpathlineto{\pgfqpoint{5.602341in}{4.774754in}}%
\pgfpathlineto{\pgfqpoint{5.637529in}{4.821551in}}%
\pgfpathlineto{\pgfqpoint{5.690310in}{4.898197in}}%
\pgfpathlineto{\pgfqpoint{5.778280in}{5.028318in}}%
\pgfpathlineto{\pgfqpoint{5.813467in}{5.076431in}}%
\pgfpathlineto{\pgfqpoint{5.848655in}{5.120284in}}%
\pgfpathlineto{\pgfqpoint{5.883843in}{5.158892in}}%
\pgfpathlineto{\pgfqpoint{5.901437in}{5.175995in}}%
\pgfpathlineto{\pgfqpoint{5.919030in}{5.191540in}}%
\pgfpathlineto{\pgfqpoint{5.936624in}{5.205475in}}%
\pgfpathlineto{\pgfqpoint{5.954218in}{5.217764in}}%
\pgfpathlineto{\pgfqpoint{5.971812in}{5.228380in}}%
\pgfpathlineto{\pgfqpoint{5.989406in}{5.237306in}}%
\pgfpathlineto{\pgfqpoint{6.007000in}{5.244533in}}%
\pgfpathlineto{\pgfqpoint{6.024594in}{5.250054in}}%
\pgfpathlineto{\pgfqpoint{6.042187in}{5.253865in}}%
\pgfpathlineto{\pgfqpoint{6.059781in}{5.255961in}}%
\pgfpathlineto{\pgfqpoint{6.077375in}{5.256336in}}%
\pgfpathlineto{\pgfqpoint{6.094969in}{5.254979in}}%
\pgfpathlineto{\pgfqpoint{6.112563in}{5.251872in}}%
\pgfpathlineto{\pgfqpoint{6.130157in}{5.246990in}}%
\pgfpathlineto{\pgfqpoint{6.147751in}{5.240304in}}%
\pgfpathlineto{\pgfqpoint{6.165344in}{5.231775in}}%
\pgfpathlineto{\pgfqpoint{6.182938in}{5.221359in}}%
\pgfpathlineto{\pgfqpoint{6.200532in}{5.209011in}}%
\pgfpathlineto{\pgfqpoint{6.218126in}{5.194685in}}%
\pgfpathlineto{\pgfqpoint{6.235720in}{5.178340in}}%
\pgfpathlineto{\pgfqpoint{6.253314in}{5.159946in}}%
\pgfpathlineto{\pgfqpoint{6.270907in}{5.139492in}}%
\pgfpathlineto{\pgfqpoint{6.288501in}{5.116994in}}%
\pgfpathlineto{\pgfqpoint{6.323689in}{5.066133in}}%
\pgfpathlineto{\pgfqpoint{6.358877in}{5.008482in}}%
\pgfpathlineto{\pgfqpoint{6.376471in}{4.977795in}}%
\pgfpathlineto{\pgfqpoint{6.376471in}{4.977795in}}%
\pgfusepath{stroke}%
\end{pgfscope}%
\begin{pgfscope}%
\pgfsetrectcap%
\pgfsetmiterjoin%
\pgfsetlinewidth{1.003750pt}%
\definecolor{currentstroke}{rgb}{0.800000,0.800000,0.800000}%
\pgfsetstrokecolor{currentstroke}%
\pgfsetdash{}{0pt}%
\pgfpathmoveto{\pgfqpoint{2.000000in}{4.421053in}}%
\pgfpathlineto{\pgfqpoint{2.000000in}{5.400000in}}%
\pgfusepath{stroke}%
\end{pgfscope}%
\begin{pgfscope}%
\pgfsetrectcap%
\pgfsetmiterjoin%
\pgfsetlinewidth{1.003750pt}%
\definecolor{currentstroke}{rgb}{0.800000,0.800000,0.800000}%
\pgfsetstrokecolor{currentstroke}%
\pgfsetdash{}{0pt}%
\pgfpathmoveto{\pgfqpoint{6.376471in}{4.421053in}}%
\pgfpathlineto{\pgfqpoint{6.376471in}{5.400000in}}%
\pgfusepath{stroke}%
\end{pgfscope}%
\begin{pgfscope}%
\pgfsetrectcap%
\pgfsetmiterjoin%
\pgfsetlinewidth{1.003750pt}%
\definecolor{currentstroke}{rgb}{0.800000,0.800000,0.800000}%
\pgfsetstrokecolor{currentstroke}%
\pgfsetdash{}{0pt}%
\pgfpathmoveto{\pgfqpoint{2.000000in}{5.400000in}}%
\pgfpathlineto{\pgfqpoint{6.376471in}{5.400000in}}%
\pgfusepath{stroke}%
\end{pgfscope}%
\begin{pgfscope}%
\pgfsetrectcap%
\pgfsetmiterjoin%
\pgfsetlinewidth{1.003750pt}%
\definecolor{currentstroke}{rgb}{0.800000,0.800000,0.800000}%
\pgfsetstrokecolor{currentstroke}%
\pgfsetdash{}{0pt}%
\pgfpathmoveto{\pgfqpoint{2.000000in}{4.421053in}}%
\pgfpathlineto{\pgfqpoint{6.376471in}{4.421053in}}%
\pgfusepath{stroke}%
\end{pgfscope}%
\begin{pgfscope}%
\pgfsetroundcap%
\pgfsetroundjoin%
\pgfsetlinewidth{1.756562pt}%
\definecolor{currentstroke}{rgb}{0.298039,0.447059,0.690196}%
\pgfsetstrokecolor{currentstroke}%
\pgfsetdash{}{0pt}%
\pgfpathmoveto{\pgfqpoint{2.125000in}{5.220056in}}%
\pgfpathlineto{\pgfqpoint{2.402778in}{5.220056in}}%
\pgfusepath{stroke}%
\end{pgfscope}%
\begin{pgfscope}%
\definecolor{textcolor}{rgb}{0.150000,0.150000,0.150000}%
\pgfsetstrokecolor{textcolor}%
\pgfsetfillcolor{textcolor}%
\pgftext[x=2.513889in,y=5.171445in,left,base]{\color{textcolor}\sffamily\fontsize{10.000000}{12.000000}\selectfont \(\displaystyle \widetilde{\Phi}^* \theta\)}%
\end{pgfscope}%
\begin{pgfscope}%
\pgfsetbuttcap%
\pgfsetroundjoin%
\pgfsetlinewidth{1.756562pt}%
\definecolor{currentstroke}{rgb}{1.000000,0.647059,0.000000}%
\pgfsetstrokecolor{currentstroke}%
\pgfsetdash{{6.000000pt}{6.000000pt}}{0.000000pt}%
\pgfpathmoveto{\pgfqpoint{2.125000in}{5.015195in}}%
\pgfpathlineto{\pgfqpoint{2.402778in}{5.015195in}}%
\pgfusepath{stroke}%
\end{pgfscope}%
\begin{pgfscope}%
\definecolor{textcolor}{rgb}{0.150000,0.150000,0.150000}%
\pgfsetstrokecolor{textcolor}%
\pgfsetfillcolor{textcolor}%
\pgftext[x=2.513889in,y=4.966584in,left,base]{\color{textcolor}\sffamily\fontsize{10.000000}{12.000000}\selectfont \(\displaystyle \widetilde{K}u\)}%
\end{pgfscope}%
\begin{pgfscope}%
\pgfsetbuttcap%
\pgfsetroundjoin%
\definecolor{currentfill}{rgb}{1.000000,0.000000,0.000000}%
\pgfsetfillcolor{currentfill}%
\pgfsetlinewidth{2.007500pt}%
\definecolor{currentstroke}{rgb}{1.000000,0.000000,0.000000}%
\pgfsetstrokecolor{currentstroke}%
\pgfsetdash{}{0pt}%
\pgfpathmoveto{\pgfqpoint{2.232832in}{4.806577in}}%
\pgfpathlineto{\pgfqpoint{2.294945in}{4.806577in}}%
\pgfpathmoveto{\pgfqpoint{2.263889in}{4.775521in}}%
\pgfpathlineto{\pgfqpoint{2.263889in}{4.837634in}}%
\pgfusepath{stroke,fill}%
\end{pgfscope}%
\begin{pgfscope}%
\pgfsetbuttcap%
\pgfsetroundjoin%
\definecolor{currentfill}{rgb}{1.000000,0.000000,0.000000}%
\pgfsetfillcolor{currentfill}%
\pgfsetlinewidth{2.007500pt}%
\definecolor{currentstroke}{rgb}{1.000000,0.000000,0.000000}%
\pgfsetstrokecolor{currentstroke}%
\pgfsetdash{}{0pt}%
\pgfpathmoveto{\pgfqpoint{2.232832in}{4.806577in}}%
\pgfpathlineto{\pgfqpoint{2.294945in}{4.806577in}}%
\pgfpathmoveto{\pgfqpoint{2.263889in}{4.775521in}}%
\pgfpathlineto{\pgfqpoint{2.263889in}{4.837634in}}%
\pgfusepath{stroke,fill}%
\end{pgfscope}%
\begin{pgfscope}%
\pgfsetbuttcap%
\pgfsetroundjoin%
\definecolor{currentfill}{rgb}{1.000000,0.000000,0.000000}%
\pgfsetfillcolor{currentfill}%
\pgfsetlinewidth{2.007500pt}%
\definecolor{currentstroke}{rgb}{1.000000,0.000000,0.000000}%
\pgfsetstrokecolor{currentstroke}%
\pgfsetdash{}{0pt}%
\pgfpathmoveto{\pgfqpoint{2.232832in}{4.806577in}}%
\pgfpathlineto{\pgfqpoint{2.294945in}{4.806577in}}%
\pgfpathmoveto{\pgfqpoint{2.263889in}{4.775521in}}%
\pgfpathlineto{\pgfqpoint{2.263889in}{4.837634in}}%
\pgfusepath{stroke,fill}%
\end{pgfscope}%
\begin{pgfscope}%
\definecolor{textcolor}{rgb}{0.150000,0.150000,0.150000}%
\pgfsetstrokecolor{textcolor}%
\pgfsetfillcolor{textcolor}%
\pgftext[x=2.513889in,y=4.770119in,left,base]{\color{textcolor}\sffamily\fontsize{10.000000}{12.000000}\selectfont train}%
\end{pgfscope}%
\begin{pgfscope}%
\pgfsetbuttcap%
\pgfsetroundjoin%
\definecolor{currentfill}{rgb}{0.000000,0.000000,0.000000}%
\pgfsetfillcolor{currentfill}%
\pgfsetlinewidth{0.301125pt}%
\definecolor{currentstroke}{rgb}{0.000000,0.000000,0.000000}%
\pgfsetstrokecolor{currentstroke}%
\pgfsetdash{}{0pt}%
\pgfpathmoveto{\pgfqpoint{2.263889in}{4.594584in}}%
\pgfpathcurveto{\pgfqpoint{2.268007in}{4.594584in}}{\pgfqpoint{2.271957in}{4.596220in}}{\pgfqpoint{2.274869in}{4.599132in}}%
\pgfpathcurveto{\pgfqpoint{2.277781in}{4.602044in}}{\pgfqpoint{2.279417in}{4.605994in}}{\pgfqpoint{2.279417in}{4.610112in}}%
\pgfpathcurveto{\pgfqpoint{2.279417in}{4.614230in}}{\pgfqpoint{2.277781in}{4.618180in}}{\pgfqpoint{2.274869in}{4.621092in}}%
\pgfpathcurveto{\pgfqpoint{2.271957in}{4.624004in}}{\pgfqpoint{2.268007in}{4.625640in}}{\pgfqpoint{2.263889in}{4.625640in}}%
\pgfpathcurveto{\pgfqpoint{2.259771in}{4.625640in}}{\pgfqpoint{2.255821in}{4.624004in}}{\pgfqpoint{2.252909in}{4.621092in}}%
\pgfpathcurveto{\pgfqpoint{2.249997in}{4.618180in}}{\pgfqpoint{2.248361in}{4.614230in}}{\pgfqpoint{2.248361in}{4.610112in}}%
\pgfpathcurveto{\pgfqpoint{2.248361in}{4.605994in}}{\pgfqpoint{2.249997in}{4.602044in}}{\pgfqpoint{2.252909in}{4.599132in}}%
\pgfpathcurveto{\pgfqpoint{2.255821in}{4.596220in}}{\pgfqpoint{2.259771in}{4.594584in}}{\pgfqpoint{2.263889in}{4.594584in}}%
\pgfpathclose%
\pgfusepath{stroke,fill}%
\end{pgfscope}%
\begin{pgfscope}%
\pgfsetbuttcap%
\pgfsetroundjoin%
\definecolor{currentfill}{rgb}{0.000000,0.000000,0.000000}%
\pgfsetfillcolor{currentfill}%
\pgfsetlinewidth{0.301125pt}%
\definecolor{currentstroke}{rgb}{0.000000,0.000000,0.000000}%
\pgfsetstrokecolor{currentstroke}%
\pgfsetdash{}{0pt}%
\pgfpathmoveto{\pgfqpoint{2.263889in}{4.594584in}}%
\pgfpathcurveto{\pgfqpoint{2.268007in}{4.594584in}}{\pgfqpoint{2.271957in}{4.596220in}}{\pgfqpoint{2.274869in}{4.599132in}}%
\pgfpathcurveto{\pgfqpoint{2.277781in}{4.602044in}}{\pgfqpoint{2.279417in}{4.605994in}}{\pgfqpoint{2.279417in}{4.610112in}}%
\pgfpathcurveto{\pgfqpoint{2.279417in}{4.614230in}}{\pgfqpoint{2.277781in}{4.618180in}}{\pgfqpoint{2.274869in}{4.621092in}}%
\pgfpathcurveto{\pgfqpoint{2.271957in}{4.624004in}}{\pgfqpoint{2.268007in}{4.625640in}}{\pgfqpoint{2.263889in}{4.625640in}}%
\pgfpathcurveto{\pgfqpoint{2.259771in}{4.625640in}}{\pgfqpoint{2.255821in}{4.624004in}}{\pgfqpoint{2.252909in}{4.621092in}}%
\pgfpathcurveto{\pgfqpoint{2.249997in}{4.618180in}}{\pgfqpoint{2.248361in}{4.614230in}}{\pgfqpoint{2.248361in}{4.610112in}}%
\pgfpathcurveto{\pgfqpoint{2.248361in}{4.605994in}}{\pgfqpoint{2.249997in}{4.602044in}}{\pgfqpoint{2.252909in}{4.599132in}}%
\pgfpathcurveto{\pgfqpoint{2.255821in}{4.596220in}}{\pgfqpoint{2.259771in}{4.594584in}}{\pgfqpoint{2.263889in}{4.594584in}}%
\pgfpathclose%
\pgfusepath{stroke,fill}%
\end{pgfscope}%
\begin{pgfscope}%
\pgfsetbuttcap%
\pgfsetroundjoin%
\definecolor{currentfill}{rgb}{0.000000,0.000000,0.000000}%
\pgfsetfillcolor{currentfill}%
\pgfsetlinewidth{0.301125pt}%
\definecolor{currentstroke}{rgb}{0.000000,0.000000,0.000000}%
\pgfsetstrokecolor{currentstroke}%
\pgfsetdash{}{0pt}%
\pgfpathmoveto{\pgfqpoint{2.263889in}{4.594584in}}%
\pgfpathcurveto{\pgfqpoint{2.268007in}{4.594584in}}{\pgfqpoint{2.271957in}{4.596220in}}{\pgfqpoint{2.274869in}{4.599132in}}%
\pgfpathcurveto{\pgfqpoint{2.277781in}{4.602044in}}{\pgfqpoint{2.279417in}{4.605994in}}{\pgfqpoint{2.279417in}{4.610112in}}%
\pgfpathcurveto{\pgfqpoint{2.279417in}{4.614230in}}{\pgfqpoint{2.277781in}{4.618180in}}{\pgfqpoint{2.274869in}{4.621092in}}%
\pgfpathcurveto{\pgfqpoint{2.271957in}{4.624004in}}{\pgfqpoint{2.268007in}{4.625640in}}{\pgfqpoint{2.263889in}{4.625640in}}%
\pgfpathcurveto{\pgfqpoint{2.259771in}{4.625640in}}{\pgfqpoint{2.255821in}{4.624004in}}{\pgfqpoint{2.252909in}{4.621092in}}%
\pgfpathcurveto{\pgfqpoint{2.249997in}{4.618180in}}{\pgfqpoint{2.248361in}{4.614230in}}{\pgfqpoint{2.248361in}{4.610112in}}%
\pgfpathcurveto{\pgfqpoint{2.248361in}{4.605994in}}{\pgfqpoint{2.249997in}{4.602044in}}{\pgfqpoint{2.252909in}{4.599132in}}%
\pgfpathcurveto{\pgfqpoint{2.255821in}{4.596220in}}{\pgfqpoint{2.259771in}{4.594584in}}{\pgfqpoint{2.263889in}{4.594584in}}%
\pgfpathclose%
\pgfusepath{stroke,fill}%
\end{pgfscope}%
\begin{pgfscope}%
\definecolor{textcolor}{rgb}{0.150000,0.150000,0.150000}%
\pgfsetstrokecolor{textcolor}%
\pgfsetfillcolor{textcolor}%
\pgftext[x=2.513889in,y=4.573654in,left,base]{\color{textcolor}\sffamily\fontsize{10.000000}{12.000000}\selectfont test}%
\end{pgfscope}%
\begin{pgfscope}%
\pgfsetbuttcap%
\pgfsetmiterjoin%
\definecolor{currentfill}{rgb}{1.000000,1.000000,1.000000}%
\pgfsetfillcolor{currentfill}%
\pgfsetlinewidth{0.000000pt}%
\definecolor{currentstroke}{rgb}{0.000000,0.000000,0.000000}%
\pgfsetstrokecolor{currentstroke}%
\pgfsetstrokeopacity{0.000000}%
\pgfsetdash{}{0pt}%
\pgfpathmoveto{\pgfqpoint{7.105882in}{4.421053in}}%
\pgfpathlineto{\pgfqpoint{11.482353in}{4.421053in}}%
\pgfpathlineto{\pgfqpoint{11.482353in}{5.400000in}}%
\pgfpathlineto{\pgfqpoint{7.105882in}{5.400000in}}%
\pgfpathclose%
\pgfusepath{fill}%
\end{pgfscope}%
\begin{pgfscope}%
\pgfpathrectangle{\pgfqpoint{7.105882in}{4.421053in}}{\pgfqpoint{4.376471in}{0.978947in}} %
\pgfusepath{clip}%
\pgfsetroundcap%
\pgfsetroundjoin%
\pgfsetlinewidth{1.003750pt}%
\definecolor{currentstroke}{rgb}{0.800000,0.800000,0.800000}%
\pgfsetstrokecolor{currentstroke}%
\pgfsetdash{}{0pt}%
\pgfpathmoveto{\pgfqpoint{7.105882in}{4.421053in}}%
\pgfpathlineto{\pgfqpoint{7.105882in}{5.400000in}}%
\pgfusepath{stroke}%
\end{pgfscope}%
\begin{pgfscope}%
\pgfpathrectangle{\pgfqpoint{7.105882in}{4.421053in}}{\pgfqpoint{4.376471in}{0.978947in}} %
\pgfusepath{clip}%
\pgfsetroundcap%
\pgfsetroundjoin%
\pgfsetlinewidth{1.003750pt}%
\definecolor{currentstroke}{rgb}{0.800000,0.800000,0.800000}%
\pgfsetstrokecolor{currentstroke}%
\pgfsetdash{}{0pt}%
\pgfpathmoveto{\pgfqpoint{7.981176in}{4.421053in}}%
\pgfpathlineto{\pgfqpoint{7.981176in}{5.400000in}}%
\pgfusepath{stroke}%
\end{pgfscope}%
\begin{pgfscope}%
\pgfpathrectangle{\pgfqpoint{7.105882in}{4.421053in}}{\pgfqpoint{4.376471in}{0.978947in}} %
\pgfusepath{clip}%
\pgfsetroundcap%
\pgfsetroundjoin%
\pgfsetlinewidth{1.003750pt}%
\definecolor{currentstroke}{rgb}{0.800000,0.800000,0.800000}%
\pgfsetstrokecolor{currentstroke}%
\pgfsetdash{}{0pt}%
\pgfpathmoveto{\pgfqpoint{8.856471in}{4.421053in}}%
\pgfpathlineto{\pgfqpoint{8.856471in}{5.400000in}}%
\pgfusepath{stroke}%
\end{pgfscope}%
\begin{pgfscope}%
\pgfpathrectangle{\pgfqpoint{7.105882in}{4.421053in}}{\pgfqpoint{4.376471in}{0.978947in}} %
\pgfusepath{clip}%
\pgfsetroundcap%
\pgfsetroundjoin%
\pgfsetlinewidth{1.003750pt}%
\definecolor{currentstroke}{rgb}{0.800000,0.800000,0.800000}%
\pgfsetstrokecolor{currentstroke}%
\pgfsetdash{}{0pt}%
\pgfpathmoveto{\pgfqpoint{9.731765in}{4.421053in}}%
\pgfpathlineto{\pgfqpoint{9.731765in}{5.400000in}}%
\pgfusepath{stroke}%
\end{pgfscope}%
\begin{pgfscope}%
\pgfpathrectangle{\pgfqpoint{7.105882in}{4.421053in}}{\pgfqpoint{4.376471in}{0.978947in}} %
\pgfusepath{clip}%
\pgfsetroundcap%
\pgfsetroundjoin%
\pgfsetlinewidth{1.003750pt}%
\definecolor{currentstroke}{rgb}{0.800000,0.800000,0.800000}%
\pgfsetstrokecolor{currentstroke}%
\pgfsetdash{}{0pt}%
\pgfpathmoveto{\pgfqpoint{10.607059in}{4.421053in}}%
\pgfpathlineto{\pgfqpoint{10.607059in}{5.400000in}}%
\pgfusepath{stroke}%
\end{pgfscope}%
\begin{pgfscope}%
\pgfpathrectangle{\pgfqpoint{7.105882in}{4.421053in}}{\pgfqpoint{4.376471in}{0.978947in}} %
\pgfusepath{clip}%
\pgfsetroundcap%
\pgfsetroundjoin%
\pgfsetlinewidth{1.003750pt}%
\definecolor{currentstroke}{rgb}{0.800000,0.800000,0.800000}%
\pgfsetstrokecolor{currentstroke}%
\pgfsetdash{}{0pt}%
\pgfpathmoveto{\pgfqpoint{11.482353in}{4.421053in}}%
\pgfpathlineto{\pgfqpoint{11.482353in}{5.400000in}}%
\pgfusepath{stroke}%
\end{pgfscope}%
\begin{pgfscope}%
\pgfpathrectangle{\pgfqpoint{7.105882in}{4.421053in}}{\pgfqpoint{4.376471in}{0.978947in}} %
\pgfusepath{clip}%
\pgfsetroundcap%
\pgfsetroundjoin%
\pgfsetlinewidth{1.003750pt}%
\definecolor{currentstroke}{rgb}{0.800000,0.800000,0.800000}%
\pgfsetstrokecolor{currentstroke}%
\pgfsetdash{}{0pt}%
\pgfpathmoveto{\pgfqpoint{7.105882in}{4.584211in}}%
\pgfpathlineto{\pgfqpoint{11.482353in}{4.584211in}}%
\pgfusepath{stroke}%
\end{pgfscope}%
\begin{pgfscope}%
\definecolor{textcolor}{rgb}{0.150000,0.150000,0.150000}%
\pgfsetstrokecolor{textcolor}%
\pgfsetfillcolor{textcolor}%
\pgftext[x=7.008660in,y=4.584211in,right,]{\color{textcolor}\sffamily\fontsize{10.000000}{12.000000}\selectfont \(\displaystyle -1\)}%
\end{pgfscope}%
\begin{pgfscope}%
\pgfpathrectangle{\pgfqpoint{7.105882in}{4.421053in}}{\pgfqpoint{4.376471in}{0.978947in}} %
\pgfusepath{clip}%
\pgfsetroundcap%
\pgfsetroundjoin%
\pgfsetlinewidth{1.003750pt}%
\definecolor{currentstroke}{rgb}{0.800000,0.800000,0.800000}%
\pgfsetstrokecolor{currentstroke}%
\pgfsetdash{}{0pt}%
\pgfpathmoveto{\pgfqpoint{7.105882in}{4.788158in}}%
\pgfpathlineto{\pgfqpoint{11.482353in}{4.788158in}}%
\pgfusepath{stroke}%
\end{pgfscope}%
\begin{pgfscope}%
\definecolor{textcolor}{rgb}{0.150000,0.150000,0.150000}%
\pgfsetstrokecolor{textcolor}%
\pgfsetfillcolor{textcolor}%
\pgftext[x=7.008660in,y=4.788158in,right,]{\color{textcolor}\sffamily\fontsize{10.000000}{12.000000}\selectfont \(\displaystyle 0\)}%
\end{pgfscope}%
\begin{pgfscope}%
\pgfpathrectangle{\pgfqpoint{7.105882in}{4.421053in}}{\pgfqpoint{4.376471in}{0.978947in}} %
\pgfusepath{clip}%
\pgfsetroundcap%
\pgfsetroundjoin%
\pgfsetlinewidth{1.003750pt}%
\definecolor{currentstroke}{rgb}{0.800000,0.800000,0.800000}%
\pgfsetstrokecolor{currentstroke}%
\pgfsetdash{}{0pt}%
\pgfpathmoveto{\pgfqpoint{7.105882in}{4.992105in}}%
\pgfpathlineto{\pgfqpoint{11.482353in}{4.992105in}}%
\pgfusepath{stroke}%
\end{pgfscope}%
\begin{pgfscope}%
\definecolor{textcolor}{rgb}{0.150000,0.150000,0.150000}%
\pgfsetstrokecolor{textcolor}%
\pgfsetfillcolor{textcolor}%
\pgftext[x=7.008660in,y=4.992105in,right,]{\color{textcolor}\sffamily\fontsize{10.000000}{12.000000}\selectfont \(\displaystyle 1\)}%
\end{pgfscope}%
\begin{pgfscope}%
\pgfpathrectangle{\pgfqpoint{7.105882in}{4.421053in}}{\pgfqpoint{4.376471in}{0.978947in}} %
\pgfusepath{clip}%
\pgfsetroundcap%
\pgfsetroundjoin%
\pgfsetlinewidth{1.003750pt}%
\definecolor{currentstroke}{rgb}{0.800000,0.800000,0.800000}%
\pgfsetstrokecolor{currentstroke}%
\pgfsetdash{}{0pt}%
\pgfpathmoveto{\pgfqpoint{7.105882in}{5.196053in}}%
\pgfpathlineto{\pgfqpoint{11.482353in}{5.196053in}}%
\pgfusepath{stroke}%
\end{pgfscope}%
\begin{pgfscope}%
\definecolor{textcolor}{rgb}{0.150000,0.150000,0.150000}%
\pgfsetstrokecolor{textcolor}%
\pgfsetfillcolor{textcolor}%
\pgftext[x=7.008660in,y=5.196053in,right,]{\color{textcolor}\sffamily\fontsize{10.000000}{12.000000}\selectfont \(\displaystyle 2\)}%
\end{pgfscope}%
\begin{pgfscope}%
\pgfpathrectangle{\pgfqpoint{7.105882in}{4.421053in}}{\pgfqpoint{4.376471in}{0.978947in}} %
\pgfusepath{clip}%
\pgfsetroundcap%
\pgfsetroundjoin%
\pgfsetlinewidth{1.003750pt}%
\definecolor{currentstroke}{rgb}{0.800000,0.800000,0.800000}%
\pgfsetstrokecolor{currentstroke}%
\pgfsetdash{}{0pt}%
\pgfpathmoveto{\pgfqpoint{7.105882in}{5.400000in}}%
\pgfpathlineto{\pgfqpoint{11.482353in}{5.400000in}}%
\pgfusepath{stroke}%
\end{pgfscope}%
\begin{pgfscope}%
\definecolor{textcolor}{rgb}{0.150000,0.150000,0.150000}%
\pgfsetstrokecolor{textcolor}%
\pgfsetfillcolor{textcolor}%
\pgftext[x=7.008660in,y=5.400000in,right,]{\color{textcolor}\sffamily\fontsize{10.000000}{12.000000}\selectfont \(\displaystyle 3\)}%
\end{pgfscope}%
\begin{pgfscope}%
\pgfpathrectangle{\pgfqpoint{7.105882in}{4.421053in}}{\pgfqpoint{4.376471in}{0.978947in}} %
\pgfusepath{clip}%
\pgfsetbuttcap%
\pgfsetroundjoin%
\definecolor{currentfill}{rgb}{1.000000,0.000000,0.000000}%
\pgfsetfillcolor{currentfill}%
\pgfsetlinewidth{2.007500pt}%
\definecolor{currentstroke}{rgb}{1.000000,0.000000,0.000000}%
\pgfsetstrokecolor{currentstroke}%
\pgfsetdash{}{0pt}%
\pgfpathmoveto{\pgfqpoint{9.871613in}{4.978628in}}%
\pgfpathlineto{\pgfqpoint{9.933726in}{4.978628in}}%
\pgfpathmoveto{\pgfqpoint{9.902669in}{4.947572in}}%
\pgfpathlineto{\pgfqpoint{9.902669in}{5.009685in}}%
\pgfusepath{stroke,fill}%
\end{pgfscope}%
\begin{pgfscope}%
\pgfpathrectangle{\pgfqpoint{7.105882in}{4.421053in}}{\pgfqpoint{4.376471in}{0.978947in}} %
\pgfusepath{clip}%
\pgfsetbuttcap%
\pgfsetroundjoin%
\definecolor{currentfill}{rgb}{1.000000,0.000000,0.000000}%
\pgfsetfillcolor{currentfill}%
\pgfsetlinewidth{2.007500pt}%
\definecolor{currentstroke}{rgb}{1.000000,0.000000,0.000000}%
\pgfsetstrokecolor{currentstroke}%
\pgfsetdash{}{0pt}%
\pgfpathmoveto{\pgfqpoint{10.454124in}{4.661416in}}%
\pgfpathlineto{\pgfqpoint{10.516237in}{4.661416in}}%
\pgfpathmoveto{\pgfqpoint{10.485181in}{4.630360in}}%
\pgfpathlineto{\pgfqpoint{10.485181in}{4.692473in}}%
\pgfusepath{stroke,fill}%
\end{pgfscope}%
\begin{pgfscope}%
\pgfpathrectangle{\pgfqpoint{7.105882in}{4.421053in}}{\pgfqpoint{4.376471in}{0.978947in}} %
\pgfusepath{clip}%
\pgfsetbuttcap%
\pgfsetroundjoin%
\definecolor{currentfill}{rgb}{1.000000,0.000000,0.000000}%
\pgfsetfillcolor{currentfill}%
\pgfsetlinewidth{2.007500pt}%
\definecolor{currentstroke}{rgb}{1.000000,0.000000,0.000000}%
\pgfsetstrokecolor{currentstroke}%
\pgfsetdash{}{0pt}%
\pgfpathmoveto{\pgfqpoint{10.060501in}{5.032156in}}%
\pgfpathlineto{\pgfqpoint{10.122614in}{5.032156in}}%
\pgfpathmoveto{\pgfqpoint{10.091557in}{5.001100in}}%
\pgfpathlineto{\pgfqpoint{10.091557in}{5.063213in}}%
\pgfusepath{stroke,fill}%
\end{pgfscope}%
\begin{pgfscope}%
\pgfpathrectangle{\pgfqpoint{7.105882in}{4.421053in}}{\pgfqpoint{4.376471in}{0.978947in}} %
\pgfusepath{clip}%
\pgfsetbuttcap%
\pgfsetroundjoin%
\definecolor{currentfill}{rgb}{1.000000,0.000000,0.000000}%
\pgfsetfillcolor{currentfill}%
\pgfsetlinewidth{2.007500pt}%
\definecolor{currentstroke}{rgb}{1.000000,0.000000,0.000000}%
\pgfsetstrokecolor{currentstroke}%
\pgfsetdash{}{0pt}%
\pgfpathmoveto{\pgfqpoint{9.857852in}{4.921244in}}%
\pgfpathlineto{\pgfqpoint{9.919965in}{4.921244in}}%
\pgfpathmoveto{\pgfqpoint{9.888909in}{4.890188in}}%
\pgfpathlineto{\pgfqpoint{9.888909in}{4.952301in}}%
\pgfusepath{stroke,fill}%
\end{pgfscope}%
\begin{pgfscope}%
\pgfpathrectangle{\pgfqpoint{7.105882in}{4.421053in}}{\pgfqpoint{4.376471in}{0.978947in}} %
\pgfusepath{clip}%
\pgfsetbuttcap%
\pgfsetroundjoin%
\definecolor{currentfill}{rgb}{1.000000,0.000000,0.000000}%
\pgfsetfillcolor{currentfill}%
\pgfsetlinewidth{2.007500pt}%
\definecolor{currentstroke}{rgb}{1.000000,0.000000,0.000000}%
\pgfsetstrokecolor{currentstroke}%
\pgfsetdash{}{0pt}%
\pgfpathmoveto{\pgfqpoint{9.433410in}{4.594850in}}%
\pgfpathlineto{\pgfqpoint{9.495523in}{4.594850in}}%
\pgfpathmoveto{\pgfqpoint{9.464467in}{4.563793in}}%
\pgfpathlineto{\pgfqpoint{9.464467in}{4.625906in}}%
\pgfusepath{stroke,fill}%
\end{pgfscope}%
\begin{pgfscope}%
\pgfpathrectangle{\pgfqpoint{7.105882in}{4.421053in}}{\pgfqpoint{4.376471in}{0.978947in}} %
\pgfusepath{clip}%
\pgfsetbuttcap%
\pgfsetroundjoin%
\definecolor{currentfill}{rgb}{1.000000,0.000000,0.000000}%
\pgfsetfillcolor{currentfill}%
\pgfsetlinewidth{2.007500pt}%
\definecolor{currentstroke}{rgb}{1.000000,0.000000,0.000000}%
\pgfsetstrokecolor{currentstroke}%
\pgfsetdash{}{0pt}%
\pgfpathmoveto{\pgfqpoint{10.211509in}{4.864314in}}%
\pgfpathlineto{\pgfqpoint{10.273622in}{4.864314in}}%
\pgfpathmoveto{\pgfqpoint{10.242566in}{4.833258in}}%
\pgfpathlineto{\pgfqpoint{10.242566in}{4.895371in}}%
\pgfusepath{stroke,fill}%
\end{pgfscope}%
\begin{pgfscope}%
\pgfpathrectangle{\pgfqpoint{7.105882in}{4.421053in}}{\pgfqpoint{4.376471in}{0.978947in}} %
\pgfusepath{clip}%
\pgfsetbuttcap%
\pgfsetroundjoin%
\definecolor{currentfill}{rgb}{1.000000,0.000000,0.000000}%
\pgfsetfillcolor{currentfill}%
\pgfsetlinewidth{2.007500pt}%
\definecolor{currentstroke}{rgb}{1.000000,0.000000,0.000000}%
\pgfsetstrokecolor{currentstroke}%
\pgfsetdash{}{0pt}%
\pgfpathmoveto{\pgfqpoint{9.482190in}{4.632344in}}%
\pgfpathlineto{\pgfqpoint{9.544303in}{4.632344in}}%
\pgfpathmoveto{\pgfqpoint{9.513247in}{4.601287in}}%
\pgfpathlineto{\pgfqpoint{9.513247in}{4.663400in}}%
\pgfusepath{stroke,fill}%
\end{pgfscope}%
\begin{pgfscope}%
\pgfpathrectangle{\pgfqpoint{7.105882in}{4.421053in}}{\pgfqpoint{4.376471in}{0.978947in}} %
\pgfusepath{clip}%
\pgfsetbuttcap%
\pgfsetroundjoin%
\definecolor{currentfill}{rgb}{1.000000,0.000000,0.000000}%
\pgfsetfillcolor{currentfill}%
\pgfsetlinewidth{2.007500pt}%
\definecolor{currentstroke}{rgb}{1.000000,0.000000,0.000000}%
\pgfsetstrokecolor{currentstroke}%
\pgfsetdash{}{0pt}%
\pgfpathmoveto{\pgfqpoint{11.072375in}{5.272462in}}%
\pgfpathlineto{\pgfqpoint{11.134488in}{5.272462in}}%
\pgfpathmoveto{\pgfqpoint{11.103431in}{5.241406in}}%
\pgfpathlineto{\pgfqpoint{11.103431in}{5.303519in}}%
\pgfusepath{stroke,fill}%
\end{pgfscope}%
\begin{pgfscope}%
\pgfpathrectangle{\pgfqpoint{7.105882in}{4.421053in}}{\pgfqpoint{4.376471in}{0.978947in}} %
\pgfusepath{clip}%
\pgfsetbuttcap%
\pgfsetroundjoin%
\definecolor{currentfill}{rgb}{1.000000,0.000000,0.000000}%
\pgfsetfillcolor{currentfill}%
\pgfsetlinewidth{2.007500pt}%
\definecolor{currentstroke}{rgb}{1.000000,0.000000,0.000000}%
\pgfsetstrokecolor{currentstroke}%
\pgfsetdash{}{0pt}%
\pgfpathmoveto{\pgfqpoint{11.324073in}{5.172865in}}%
\pgfpathlineto{\pgfqpoint{11.386186in}{5.172865in}}%
\pgfpathmoveto{\pgfqpoint{11.355130in}{5.141808in}}%
\pgfpathlineto{\pgfqpoint{11.355130in}{5.203921in}}%
\pgfusepath{stroke,fill}%
\end{pgfscope}%
\begin{pgfscope}%
\pgfpathrectangle{\pgfqpoint{7.105882in}{4.421053in}}{\pgfqpoint{4.376471in}{0.978947in}} %
\pgfusepath{clip}%
\pgfsetbuttcap%
\pgfsetroundjoin%
\definecolor{currentfill}{rgb}{1.000000,0.000000,0.000000}%
\pgfsetfillcolor{currentfill}%
\pgfsetlinewidth{2.007500pt}%
\definecolor{currentstroke}{rgb}{1.000000,0.000000,0.000000}%
\pgfsetstrokecolor{currentstroke}%
\pgfsetdash{}{0pt}%
\pgfpathmoveto{\pgfqpoint{9.292616in}{4.670261in}}%
\pgfpathlineto{\pgfqpoint{9.354729in}{4.670261in}}%
\pgfpathmoveto{\pgfqpoint{9.323673in}{4.639204in}}%
\pgfpathlineto{\pgfqpoint{9.323673in}{4.701317in}}%
\pgfusepath{stroke,fill}%
\end{pgfscope}%
\begin{pgfscope}%
\pgfpathrectangle{\pgfqpoint{7.105882in}{4.421053in}}{\pgfqpoint{4.376471in}{0.978947in}} %
\pgfusepath{clip}%
\pgfsetbuttcap%
\pgfsetroundjoin%
\definecolor{currentfill}{rgb}{1.000000,0.000000,0.000000}%
\pgfsetfillcolor{currentfill}%
\pgfsetlinewidth{2.007500pt}%
\definecolor{currentstroke}{rgb}{1.000000,0.000000,0.000000}%
\pgfsetstrokecolor{currentstroke}%
\pgfsetdash{}{0pt}%
\pgfpathmoveto{\pgfqpoint{10.722089in}{4.820436in}}%
\pgfpathlineto{\pgfqpoint{10.784202in}{4.820436in}}%
\pgfpathmoveto{\pgfqpoint{10.753146in}{4.789380in}}%
\pgfpathlineto{\pgfqpoint{10.753146in}{4.851493in}}%
\pgfusepath{stroke,fill}%
\end{pgfscope}%
\begin{pgfscope}%
\pgfpathrectangle{\pgfqpoint{7.105882in}{4.421053in}}{\pgfqpoint{4.376471in}{0.978947in}} %
\pgfusepath{clip}%
\pgfsetbuttcap%
\pgfsetroundjoin%
\definecolor{currentfill}{rgb}{1.000000,0.000000,0.000000}%
\pgfsetfillcolor{currentfill}%
\pgfsetlinewidth{2.007500pt}%
\definecolor{currentstroke}{rgb}{1.000000,0.000000,0.000000}%
\pgfsetstrokecolor{currentstroke}%
\pgfsetdash{}{0pt}%
\pgfpathmoveto{\pgfqpoint{9.801874in}{4.860532in}}%
\pgfpathlineto{\pgfqpoint{9.863987in}{4.860532in}}%
\pgfpathmoveto{\pgfqpoint{9.832931in}{4.829475in}}%
\pgfpathlineto{\pgfqpoint{9.832931in}{4.891588in}}%
\pgfusepath{stroke,fill}%
\end{pgfscope}%
\begin{pgfscope}%
\pgfpathrectangle{\pgfqpoint{7.105882in}{4.421053in}}{\pgfqpoint{4.376471in}{0.978947in}} %
\pgfusepath{clip}%
\pgfsetbuttcap%
\pgfsetroundjoin%
\definecolor{currentfill}{rgb}{1.000000,0.000000,0.000000}%
\pgfsetfillcolor{currentfill}%
\pgfsetlinewidth{2.007500pt}%
\definecolor{currentstroke}{rgb}{1.000000,0.000000,0.000000}%
\pgfsetstrokecolor{currentstroke}%
\pgfsetdash{}{0pt}%
\pgfpathmoveto{\pgfqpoint{9.938944in}{4.988087in}}%
\pgfpathlineto{\pgfqpoint{10.001057in}{4.988087in}}%
\pgfpathmoveto{\pgfqpoint{9.970001in}{4.957031in}}%
\pgfpathlineto{\pgfqpoint{9.970001in}{5.019144in}}%
\pgfusepath{stroke,fill}%
\end{pgfscope}%
\begin{pgfscope}%
\pgfpathrectangle{\pgfqpoint{7.105882in}{4.421053in}}{\pgfqpoint{4.376471in}{0.978947in}} %
\pgfusepath{clip}%
\pgfsetbuttcap%
\pgfsetroundjoin%
\definecolor{currentfill}{rgb}{1.000000,0.000000,0.000000}%
\pgfsetfillcolor{currentfill}%
\pgfsetlinewidth{2.007500pt}%
\definecolor{currentstroke}{rgb}{1.000000,0.000000,0.000000}%
\pgfsetstrokecolor{currentstroke}%
\pgfsetdash{}{0pt}%
\pgfpathmoveto{\pgfqpoint{11.190797in}{5.248251in}}%
\pgfpathlineto{\pgfqpoint{11.252910in}{5.248251in}}%
\pgfpathmoveto{\pgfqpoint{11.221854in}{5.217195in}}%
\pgfpathlineto{\pgfqpoint{11.221854in}{5.279308in}}%
\pgfusepath{stroke,fill}%
\end{pgfscope}%
\begin{pgfscope}%
\pgfpathrectangle{\pgfqpoint{7.105882in}{4.421053in}}{\pgfqpoint{4.376471in}{0.978947in}} %
\pgfusepath{clip}%
\pgfsetbuttcap%
\pgfsetroundjoin%
\definecolor{currentfill}{rgb}{1.000000,0.000000,0.000000}%
\pgfsetfillcolor{currentfill}%
\pgfsetlinewidth{2.007500pt}%
\definecolor{currentstroke}{rgb}{1.000000,0.000000,0.000000}%
\pgfsetstrokecolor{currentstroke}%
\pgfsetdash{}{0pt}%
\pgfpathmoveto{\pgfqpoint{8.198830in}{4.960862in}}%
\pgfpathlineto{\pgfqpoint{8.260943in}{4.960862in}}%
\pgfpathmoveto{\pgfqpoint{8.229886in}{4.929806in}}%
\pgfpathlineto{\pgfqpoint{8.229886in}{4.991919in}}%
\pgfusepath{stroke,fill}%
\end{pgfscope}%
\begin{pgfscope}%
\pgfpathrectangle{\pgfqpoint{7.105882in}{4.421053in}}{\pgfqpoint{4.376471in}{0.978947in}} %
\pgfusepath{clip}%
\pgfsetbuttcap%
\pgfsetroundjoin%
\definecolor{currentfill}{rgb}{1.000000,0.000000,0.000000}%
\pgfsetfillcolor{currentfill}%
\pgfsetlinewidth{2.007500pt}%
\definecolor{currentstroke}{rgb}{1.000000,0.000000,0.000000}%
\pgfsetstrokecolor{currentstroke}%
\pgfsetdash{}{0pt}%
\pgfpathmoveto{\pgfqpoint{8.255175in}{4.903228in}}%
\pgfpathlineto{\pgfqpoint{8.317288in}{4.903228in}}%
\pgfpathmoveto{\pgfqpoint{8.286232in}{4.872172in}}%
\pgfpathlineto{\pgfqpoint{8.286232in}{4.934285in}}%
\pgfusepath{stroke,fill}%
\end{pgfscope}%
\begin{pgfscope}%
\pgfpathrectangle{\pgfqpoint{7.105882in}{4.421053in}}{\pgfqpoint{4.376471in}{0.978947in}} %
\pgfusepath{clip}%
\pgfsetbuttcap%
\pgfsetroundjoin%
\definecolor{currentfill}{rgb}{1.000000,0.000000,0.000000}%
\pgfsetfillcolor{currentfill}%
\pgfsetlinewidth{2.007500pt}%
\definecolor{currentstroke}{rgb}{1.000000,0.000000,0.000000}%
\pgfsetstrokecolor{currentstroke}%
\pgfsetdash{}{0pt}%
\pgfpathmoveto{\pgfqpoint{8.020908in}{5.190508in}}%
\pgfpathlineto{\pgfqpoint{8.083021in}{5.190508in}}%
\pgfpathmoveto{\pgfqpoint{8.051965in}{5.159452in}}%
\pgfpathlineto{\pgfqpoint{8.051965in}{5.221565in}}%
\pgfusepath{stroke,fill}%
\end{pgfscope}%
\begin{pgfscope}%
\pgfpathrectangle{\pgfqpoint{7.105882in}{4.421053in}}{\pgfqpoint{4.376471in}{0.978947in}} %
\pgfusepath{clip}%
\pgfsetbuttcap%
\pgfsetroundjoin%
\definecolor{currentfill}{rgb}{1.000000,0.000000,0.000000}%
\pgfsetfillcolor{currentfill}%
\pgfsetlinewidth{2.007500pt}%
\definecolor{currentstroke}{rgb}{1.000000,0.000000,0.000000}%
\pgfsetstrokecolor{currentstroke}%
\pgfsetdash{}{0pt}%
\pgfpathmoveto{\pgfqpoint{10.865269in}{5.036104in}}%
\pgfpathlineto{\pgfqpoint{10.927382in}{5.036104in}}%
\pgfpathmoveto{\pgfqpoint{10.896325in}{5.005048in}}%
\pgfpathlineto{\pgfqpoint{10.896325in}{5.067161in}}%
\pgfusepath{stroke,fill}%
\end{pgfscope}%
\begin{pgfscope}%
\pgfpathrectangle{\pgfqpoint{7.105882in}{4.421053in}}{\pgfqpoint{4.376471in}{0.978947in}} %
\pgfusepath{clip}%
\pgfsetbuttcap%
\pgfsetroundjoin%
\definecolor{currentfill}{rgb}{1.000000,0.000000,0.000000}%
\pgfsetfillcolor{currentfill}%
\pgfsetlinewidth{2.007500pt}%
\definecolor{currentstroke}{rgb}{1.000000,0.000000,0.000000}%
\pgfsetstrokecolor{currentstroke}%
\pgfsetdash{}{0pt}%
\pgfpathmoveto{\pgfqpoint{10.674584in}{4.758523in}}%
\pgfpathlineto{\pgfqpoint{10.736697in}{4.758523in}}%
\pgfpathmoveto{\pgfqpoint{10.705641in}{4.727466in}}%
\pgfpathlineto{\pgfqpoint{10.705641in}{4.789579in}}%
\pgfusepath{stroke,fill}%
\end{pgfscope}%
\begin{pgfscope}%
\pgfpathrectangle{\pgfqpoint{7.105882in}{4.421053in}}{\pgfqpoint{4.376471in}{0.978947in}} %
\pgfusepath{clip}%
\pgfsetbuttcap%
\pgfsetroundjoin%
\definecolor{currentfill}{rgb}{1.000000,0.000000,0.000000}%
\pgfsetfillcolor{currentfill}%
\pgfsetlinewidth{2.007500pt}%
\definecolor{currentstroke}{rgb}{1.000000,0.000000,0.000000}%
\pgfsetstrokecolor{currentstroke}%
\pgfsetdash{}{0pt}%
\pgfpathmoveto{\pgfqpoint{10.996186in}{5.165886in}}%
\pgfpathlineto{\pgfqpoint{11.058299in}{5.165886in}}%
\pgfpathmoveto{\pgfqpoint{11.027243in}{5.134829in}}%
\pgfpathlineto{\pgfqpoint{11.027243in}{5.196942in}}%
\pgfusepath{stroke,fill}%
\end{pgfscope}%
\begin{pgfscope}%
\pgfpathrectangle{\pgfqpoint{7.105882in}{4.421053in}}{\pgfqpoint{4.376471in}{0.978947in}} %
\pgfusepath{clip}%
\pgfsetbuttcap%
\pgfsetroundjoin%
\definecolor{currentfill}{rgb}{1.000000,0.000000,0.000000}%
\pgfsetfillcolor{currentfill}%
\pgfsetlinewidth{2.007500pt}%
\definecolor{currentstroke}{rgb}{1.000000,0.000000,0.000000}%
\pgfsetstrokecolor{currentstroke}%
\pgfsetdash{}{0pt}%
\pgfpathmoveto{\pgfqpoint{11.376435in}{5.097123in}}%
\pgfpathlineto{\pgfqpoint{11.438548in}{5.097123in}}%
\pgfpathmoveto{\pgfqpoint{11.407492in}{5.066066in}}%
\pgfpathlineto{\pgfqpoint{11.407492in}{5.128179in}}%
\pgfusepath{stroke,fill}%
\end{pgfscope}%
\begin{pgfscope}%
\pgfpathrectangle{\pgfqpoint{7.105882in}{4.421053in}}{\pgfqpoint{4.376471in}{0.978947in}} %
\pgfusepath{clip}%
\pgfsetbuttcap%
\pgfsetroundjoin%
\definecolor{currentfill}{rgb}{1.000000,0.000000,0.000000}%
\pgfsetfillcolor{currentfill}%
\pgfsetlinewidth{2.007500pt}%
\definecolor{currentstroke}{rgb}{1.000000,0.000000,0.000000}%
\pgfsetstrokecolor{currentstroke}%
\pgfsetdash{}{0pt}%
\pgfpathmoveto{\pgfqpoint{10.748115in}{4.913690in}}%
\pgfpathlineto{\pgfqpoint{10.810228in}{4.913690in}}%
\pgfpathmoveto{\pgfqpoint{10.779172in}{4.882633in}}%
\pgfpathlineto{\pgfqpoint{10.779172in}{4.944746in}}%
\pgfusepath{stroke,fill}%
\end{pgfscope}%
\begin{pgfscope}%
\pgfpathrectangle{\pgfqpoint{7.105882in}{4.421053in}}{\pgfqpoint{4.376471in}{0.978947in}} %
\pgfusepath{clip}%
\pgfsetbuttcap%
\pgfsetroundjoin%
\definecolor{currentfill}{rgb}{1.000000,0.000000,0.000000}%
\pgfsetfillcolor{currentfill}%
\pgfsetlinewidth{2.007500pt}%
\definecolor{currentstroke}{rgb}{1.000000,0.000000,0.000000}%
\pgfsetstrokecolor{currentstroke}%
\pgfsetdash{}{0pt}%
\pgfpathmoveto{\pgfqpoint{9.565841in}{4.638149in}}%
\pgfpathlineto{\pgfqpoint{9.627954in}{4.638149in}}%
\pgfpathmoveto{\pgfqpoint{9.596897in}{4.607093in}}%
\pgfpathlineto{\pgfqpoint{9.596897in}{4.669206in}}%
\pgfusepath{stroke,fill}%
\end{pgfscope}%
\begin{pgfscope}%
\pgfpathrectangle{\pgfqpoint{7.105882in}{4.421053in}}{\pgfqpoint{4.376471in}{0.978947in}} %
\pgfusepath{clip}%
\pgfsetbuttcap%
\pgfsetroundjoin%
\definecolor{currentfill}{rgb}{1.000000,0.000000,0.000000}%
\pgfsetfillcolor{currentfill}%
\pgfsetlinewidth{2.007500pt}%
\definecolor{currentstroke}{rgb}{1.000000,0.000000,0.000000}%
\pgfsetstrokecolor{currentstroke}%
\pgfsetdash{}{0pt}%
\pgfpathmoveto{\pgfqpoint{10.682890in}{4.780559in}}%
\pgfpathlineto{\pgfqpoint{10.745003in}{4.780559in}}%
\pgfpathmoveto{\pgfqpoint{10.713947in}{4.749503in}}%
\pgfpathlineto{\pgfqpoint{10.713947in}{4.811616in}}%
\pgfusepath{stroke,fill}%
\end{pgfscope}%
\begin{pgfscope}%
\pgfpathrectangle{\pgfqpoint{7.105882in}{4.421053in}}{\pgfqpoint{4.376471in}{0.978947in}} %
\pgfusepath{clip}%
\pgfsetbuttcap%
\pgfsetroundjoin%
\definecolor{currentfill}{rgb}{1.000000,0.000000,0.000000}%
\pgfsetfillcolor{currentfill}%
\pgfsetlinewidth{2.007500pt}%
\definecolor{currentstroke}{rgb}{1.000000,0.000000,0.000000}%
\pgfsetstrokecolor{currentstroke}%
\pgfsetdash{}{0pt}%
\pgfpathmoveto{\pgfqpoint{8.364220in}{4.801297in}}%
\pgfpathlineto{\pgfqpoint{8.426333in}{4.801297in}}%
\pgfpathmoveto{\pgfqpoint{8.395276in}{4.770240in}}%
\pgfpathlineto{\pgfqpoint{8.395276in}{4.832353in}}%
\pgfusepath{stroke,fill}%
\end{pgfscope}%
\begin{pgfscope}%
\pgfpathrectangle{\pgfqpoint{7.105882in}{4.421053in}}{\pgfqpoint{4.376471in}{0.978947in}} %
\pgfusepath{clip}%
\pgfsetbuttcap%
\pgfsetroundjoin%
\definecolor{currentfill}{rgb}{1.000000,0.000000,0.000000}%
\pgfsetfillcolor{currentfill}%
\pgfsetlinewidth{2.007500pt}%
\definecolor{currentstroke}{rgb}{1.000000,0.000000,0.000000}%
\pgfsetstrokecolor{currentstroke}%
\pgfsetdash{}{0pt}%
\pgfpathmoveto{\pgfqpoint{10.190596in}{4.904581in}}%
\pgfpathlineto{\pgfqpoint{10.252709in}{4.904581in}}%
\pgfpathmoveto{\pgfqpoint{10.221653in}{4.873524in}}%
\pgfpathlineto{\pgfqpoint{10.221653in}{4.935637in}}%
\pgfusepath{stroke,fill}%
\end{pgfscope}%
\begin{pgfscope}%
\pgfpathrectangle{\pgfqpoint{7.105882in}{4.421053in}}{\pgfqpoint{4.376471in}{0.978947in}} %
\pgfusepath{clip}%
\pgfsetbuttcap%
\pgfsetroundjoin%
\definecolor{currentfill}{rgb}{1.000000,0.000000,0.000000}%
\pgfsetfillcolor{currentfill}%
\pgfsetlinewidth{2.007500pt}%
\definecolor{currentstroke}{rgb}{1.000000,0.000000,0.000000}%
\pgfsetstrokecolor{currentstroke}%
\pgfsetdash{}{0pt}%
\pgfpathmoveto{\pgfqpoint{8.452025in}{4.809587in}}%
\pgfpathlineto{\pgfqpoint{8.514138in}{4.809587in}}%
\pgfpathmoveto{\pgfqpoint{8.483082in}{4.778530in}}%
\pgfpathlineto{\pgfqpoint{8.483082in}{4.840643in}}%
\pgfusepath{stroke,fill}%
\end{pgfscope}%
\begin{pgfscope}%
\pgfpathrectangle{\pgfqpoint{7.105882in}{4.421053in}}{\pgfqpoint{4.376471in}{0.978947in}} %
\pgfusepath{clip}%
\pgfsetbuttcap%
\pgfsetroundjoin%
\definecolor{currentfill}{rgb}{1.000000,0.000000,0.000000}%
\pgfsetfillcolor{currentfill}%
\pgfsetlinewidth{2.007500pt}%
\definecolor{currentstroke}{rgb}{1.000000,0.000000,0.000000}%
\pgfsetstrokecolor{currentstroke}%
\pgfsetdash{}{0pt}%
\pgfpathmoveto{\pgfqpoint{11.257573in}{5.209780in}}%
\pgfpathlineto{\pgfqpoint{11.319686in}{5.209780in}}%
\pgfpathmoveto{\pgfqpoint{11.288629in}{5.178724in}}%
\pgfpathlineto{\pgfqpoint{11.288629in}{5.240837in}}%
\pgfusepath{stroke,fill}%
\end{pgfscope}%
\begin{pgfscope}%
\pgfpathrectangle{\pgfqpoint{7.105882in}{4.421053in}}{\pgfqpoint{4.376471in}{0.978947in}} %
\pgfusepath{clip}%
\pgfsetbuttcap%
\pgfsetroundjoin%
\definecolor{currentfill}{rgb}{1.000000,0.000000,0.000000}%
\pgfsetfillcolor{currentfill}%
\pgfsetlinewidth{2.007500pt}%
\definecolor{currentstroke}{rgb}{1.000000,0.000000,0.000000}%
\pgfsetstrokecolor{currentstroke}%
\pgfsetdash{}{0pt}%
\pgfpathmoveto{\pgfqpoint{9.777203in}{4.856983in}}%
\pgfpathlineto{\pgfqpoint{9.839316in}{4.856983in}}%
\pgfpathmoveto{\pgfqpoint{9.808260in}{4.825926in}}%
\pgfpathlineto{\pgfqpoint{9.808260in}{4.888039in}}%
\pgfusepath{stroke,fill}%
\end{pgfscope}%
\begin{pgfscope}%
\pgfpathrectangle{\pgfqpoint{7.105882in}{4.421053in}}{\pgfqpoint{4.376471in}{0.978947in}} %
\pgfusepath{clip}%
\pgfsetbuttcap%
\pgfsetroundjoin%
\definecolor{currentfill}{rgb}{1.000000,0.000000,0.000000}%
\pgfsetfillcolor{currentfill}%
\pgfsetlinewidth{2.007500pt}%
\definecolor{currentstroke}{rgb}{1.000000,0.000000,0.000000}%
\pgfsetstrokecolor{currentstroke}%
\pgfsetdash{}{0pt}%
\pgfpathmoveto{\pgfqpoint{9.401925in}{4.605866in}}%
\pgfpathlineto{\pgfqpoint{9.464038in}{4.605866in}}%
\pgfpathmoveto{\pgfqpoint{9.432981in}{4.574809in}}%
\pgfpathlineto{\pgfqpoint{9.432981in}{4.636922in}}%
\pgfusepath{stroke,fill}%
\end{pgfscope}%
\begin{pgfscope}%
\pgfpathrectangle{\pgfqpoint{7.105882in}{4.421053in}}{\pgfqpoint{4.376471in}{0.978947in}} %
\pgfusepath{clip}%
\pgfsetbuttcap%
\pgfsetroundjoin%
\definecolor{currentfill}{rgb}{0.000000,0.000000,0.000000}%
\pgfsetfillcolor{currentfill}%
\pgfsetlinewidth{0.301125pt}%
\definecolor{currentstroke}{rgb}{0.000000,0.000000,0.000000}%
\pgfsetstrokecolor{currentstroke}%
\pgfsetdash{}{0pt}%
\pgfsys@defobject{currentmarker}{\pgfqpoint{-0.015528in}{-0.015528in}}{\pgfqpoint{0.015528in}{0.015528in}}{%
\pgfpathmoveto{\pgfqpoint{0.000000in}{-0.015528in}}%
\pgfpathcurveto{\pgfqpoint{0.004118in}{-0.015528in}}{\pgfqpoint{0.008068in}{-0.013892in}}{\pgfqpoint{0.010980in}{-0.010980in}}%
\pgfpathcurveto{\pgfqpoint{0.013892in}{-0.008068in}}{\pgfqpoint{0.015528in}{-0.004118in}}{\pgfqpoint{0.015528in}{0.000000in}}%
\pgfpathcurveto{\pgfqpoint{0.015528in}{0.004118in}}{\pgfqpoint{0.013892in}{0.008068in}}{\pgfqpoint{0.010980in}{0.010980in}}%
\pgfpathcurveto{\pgfqpoint{0.008068in}{0.013892in}}{\pgfqpoint{0.004118in}{0.015528in}}{\pgfqpoint{0.000000in}{0.015528in}}%
\pgfpathcurveto{\pgfqpoint{-0.004118in}{0.015528in}}{\pgfqpoint{-0.008068in}{0.013892in}}{\pgfqpoint{-0.010980in}{0.010980in}}%
\pgfpathcurveto{\pgfqpoint{-0.013892in}{0.008068in}}{\pgfqpoint{-0.015528in}{0.004118in}}{\pgfqpoint{-0.015528in}{0.000000in}}%
\pgfpathcurveto{\pgfqpoint{-0.015528in}{-0.004118in}}{\pgfqpoint{-0.013892in}{-0.008068in}}{\pgfqpoint{-0.010980in}{-0.010980in}}%
\pgfpathcurveto{\pgfqpoint{-0.008068in}{-0.013892in}}{\pgfqpoint{-0.004118in}{-0.015528in}}{\pgfqpoint{0.000000in}{-0.015528in}}%
\pgfpathclose%
\pgfusepath{stroke,fill}%
}%
\begin{pgfscope}%
\pgfsys@transformshift{7.981176in}{5.254916in}%
\pgfsys@useobject{currentmarker}{}%
\end{pgfscope}%
\begin{pgfscope}%
\pgfsys@transformshift{7.998770in}{5.160724in}%
\pgfsys@useobject{currentmarker}{}%
\end{pgfscope}%
\begin{pgfscope}%
\pgfsys@transformshift{8.016364in}{5.251498in}%
\pgfsys@useobject{currentmarker}{}%
\end{pgfscope}%
\begin{pgfscope}%
\pgfsys@transformshift{8.033958in}{5.270634in}%
\pgfsys@useobject{currentmarker}{}%
\end{pgfscope}%
\begin{pgfscope}%
\pgfsys@transformshift{8.051552in}{5.205849in}%
\pgfsys@useobject{currentmarker}{}%
\end{pgfscope}%
\begin{pgfscope}%
\pgfsys@transformshift{8.069146in}{5.201748in}%
\pgfsys@useobject{currentmarker}{}%
\end{pgfscope}%
\begin{pgfscope}%
\pgfsys@transformshift{8.086740in}{5.077991in}%
\pgfsys@useobject{currentmarker}{}%
\end{pgfscope}%
\begin{pgfscope}%
\pgfsys@transformshift{8.104333in}{5.077595in}%
\pgfsys@useobject{currentmarker}{}%
\end{pgfscope}%
\begin{pgfscope}%
\pgfsys@transformshift{8.121927in}{5.018268in}%
\pgfsys@useobject{currentmarker}{}%
\end{pgfscope}%
\begin{pgfscope}%
\pgfsys@transformshift{8.139521in}{5.022982in}%
\pgfsys@useobject{currentmarker}{}%
\end{pgfscope}%
\begin{pgfscope}%
\pgfsys@transformshift{8.157115in}{4.950281in}%
\pgfsys@useobject{currentmarker}{}%
\end{pgfscope}%
\begin{pgfscope}%
\pgfsys@transformshift{8.174709in}{4.831659in}%
\pgfsys@useobject{currentmarker}{}%
\end{pgfscope}%
\begin{pgfscope}%
\pgfsys@transformshift{8.192303in}{5.001404in}%
\pgfsys@useobject{currentmarker}{}%
\end{pgfscope}%
\begin{pgfscope}%
\pgfsys@transformshift{8.209897in}{4.919253in}%
\pgfsys@useobject{currentmarker}{}%
\end{pgfscope}%
\begin{pgfscope}%
\pgfsys@transformshift{8.227490in}{4.772358in}%
\pgfsys@useobject{currentmarker}{}%
\end{pgfscope}%
\begin{pgfscope}%
\pgfsys@transformshift{8.245084in}{4.965761in}%
\pgfsys@useobject{currentmarker}{}%
\end{pgfscope}%
\begin{pgfscope}%
\pgfsys@transformshift{8.262678in}{4.807758in}%
\pgfsys@useobject{currentmarker}{}%
\end{pgfscope}%
\begin{pgfscope}%
\pgfsys@transformshift{8.280272in}{4.889135in}%
\pgfsys@useobject{currentmarker}{}%
\end{pgfscope}%
\begin{pgfscope}%
\pgfsys@transformshift{8.297866in}{4.943692in}%
\pgfsys@useobject{currentmarker}{}%
\end{pgfscope}%
\begin{pgfscope}%
\pgfsys@transformshift{8.315460in}{4.870028in}%
\pgfsys@useobject{currentmarker}{}%
\end{pgfscope}%
\begin{pgfscope}%
\pgfsys@transformshift{8.333054in}{4.962679in}%
\pgfsys@useobject{currentmarker}{}%
\end{pgfscope}%
\begin{pgfscope}%
\pgfsys@transformshift{8.350647in}{4.712306in}%
\pgfsys@useobject{currentmarker}{}%
\end{pgfscope}%
\begin{pgfscope}%
\pgfsys@transformshift{8.368241in}{4.873130in}%
\pgfsys@useobject{currentmarker}{}%
\end{pgfscope}%
\begin{pgfscope}%
\pgfsys@transformshift{8.385835in}{4.758281in}%
\pgfsys@useobject{currentmarker}{}%
\end{pgfscope}%
\begin{pgfscope}%
\pgfsys@transformshift{8.403429in}{4.737440in}%
\pgfsys@useobject{currentmarker}{}%
\end{pgfscope}%
\begin{pgfscope}%
\pgfsys@transformshift{8.421023in}{4.767404in}%
\pgfsys@useobject{currentmarker}{}%
\end{pgfscope}%
\begin{pgfscope}%
\pgfsys@transformshift{8.438617in}{4.796858in}%
\pgfsys@useobject{currentmarker}{}%
\end{pgfscope}%
\begin{pgfscope}%
\pgfsys@transformshift{8.456210in}{4.838472in}%
\pgfsys@useobject{currentmarker}{}%
\end{pgfscope}%
\begin{pgfscope}%
\pgfsys@transformshift{8.473804in}{4.719865in}%
\pgfsys@useobject{currentmarker}{}%
\end{pgfscope}%
\begin{pgfscope}%
\pgfsys@transformshift{8.491398in}{4.938174in}%
\pgfsys@useobject{currentmarker}{}%
\end{pgfscope}%
\begin{pgfscope}%
\pgfsys@transformshift{8.508992in}{4.902994in}%
\pgfsys@useobject{currentmarker}{}%
\end{pgfscope}%
\begin{pgfscope}%
\pgfsys@transformshift{8.526586in}{4.709459in}%
\pgfsys@useobject{currentmarker}{}%
\end{pgfscope}%
\begin{pgfscope}%
\pgfsys@transformshift{8.544180in}{5.029731in}%
\pgfsys@useobject{currentmarker}{}%
\end{pgfscope}%
\begin{pgfscope}%
\pgfsys@transformshift{8.561774in}{5.084223in}%
\pgfsys@useobject{currentmarker}{}%
\end{pgfscope}%
\begin{pgfscope}%
\pgfsys@transformshift{8.579367in}{5.024905in}%
\pgfsys@useobject{currentmarker}{}%
\end{pgfscope}%
\begin{pgfscope}%
\pgfsys@transformshift{8.596961in}{4.900851in}%
\pgfsys@useobject{currentmarker}{}%
\end{pgfscope}%
\begin{pgfscope}%
\pgfsys@transformshift{8.614555in}{4.824998in}%
\pgfsys@useobject{currentmarker}{}%
\end{pgfscope}%
\begin{pgfscope}%
\pgfsys@transformshift{8.632149in}{5.056986in}%
\pgfsys@useobject{currentmarker}{}%
\end{pgfscope}%
\begin{pgfscope}%
\pgfsys@transformshift{8.649743in}{4.923698in}%
\pgfsys@useobject{currentmarker}{}%
\end{pgfscope}%
\begin{pgfscope}%
\pgfsys@transformshift{8.667337in}{5.104693in}%
\pgfsys@useobject{currentmarker}{}%
\end{pgfscope}%
\begin{pgfscope}%
\pgfsys@transformshift{8.684931in}{5.016167in}%
\pgfsys@useobject{currentmarker}{}%
\end{pgfscope}%
\begin{pgfscope}%
\pgfsys@transformshift{8.702524in}{5.108880in}%
\pgfsys@useobject{currentmarker}{}%
\end{pgfscope}%
\begin{pgfscope}%
\pgfsys@transformshift{8.720118in}{5.059262in}%
\pgfsys@useobject{currentmarker}{}%
\end{pgfscope}%
\begin{pgfscope}%
\pgfsys@transformshift{8.737712in}{5.107695in}%
\pgfsys@useobject{currentmarker}{}%
\end{pgfscope}%
\begin{pgfscope}%
\pgfsys@transformshift{8.755306in}{5.048347in}%
\pgfsys@useobject{currentmarker}{}%
\end{pgfscope}%
\begin{pgfscope}%
\pgfsys@transformshift{8.772900in}{5.239770in}%
\pgfsys@useobject{currentmarker}{}%
\end{pgfscope}%
\begin{pgfscope}%
\pgfsys@transformshift{8.790494in}{5.079579in}%
\pgfsys@useobject{currentmarker}{}%
\end{pgfscope}%
\begin{pgfscope}%
\pgfsys@transformshift{8.808087in}{5.115070in}%
\pgfsys@useobject{currentmarker}{}%
\end{pgfscope}%
\begin{pgfscope}%
\pgfsys@transformshift{8.825681in}{5.271894in}%
\pgfsys@useobject{currentmarker}{}%
\end{pgfscope}%
\begin{pgfscope}%
\pgfsys@transformshift{8.843275in}{4.946452in}%
\pgfsys@useobject{currentmarker}{}%
\end{pgfscope}%
\begin{pgfscope}%
\pgfsys@transformshift{8.860869in}{4.956515in}%
\pgfsys@useobject{currentmarker}{}%
\end{pgfscope}%
\begin{pgfscope}%
\pgfsys@transformshift{8.878463in}{5.185202in}%
\pgfsys@useobject{currentmarker}{}%
\end{pgfscope}%
\begin{pgfscope}%
\pgfsys@transformshift{8.896057in}{4.965051in}%
\pgfsys@useobject{currentmarker}{}%
\end{pgfscope}%
\begin{pgfscope}%
\pgfsys@transformshift{8.913651in}{5.279234in}%
\pgfsys@useobject{currentmarker}{}%
\end{pgfscope}%
\begin{pgfscope}%
\pgfsys@transformshift{8.931244in}{5.033240in}%
\pgfsys@useobject{currentmarker}{}%
\end{pgfscope}%
\begin{pgfscope}%
\pgfsys@transformshift{8.948838in}{4.991625in}%
\pgfsys@useobject{currentmarker}{}%
\end{pgfscope}%
\begin{pgfscope}%
\pgfsys@transformshift{8.966432in}{5.254445in}%
\pgfsys@useobject{currentmarker}{}%
\end{pgfscope}%
\begin{pgfscope}%
\pgfsys@transformshift{8.984026in}{5.197982in}%
\pgfsys@useobject{currentmarker}{}%
\end{pgfscope}%
\begin{pgfscope}%
\pgfsys@transformshift{9.001620in}{5.224324in}%
\pgfsys@useobject{currentmarker}{}%
\end{pgfscope}%
\begin{pgfscope}%
\pgfsys@transformshift{9.019214in}{5.111466in}%
\pgfsys@useobject{currentmarker}{}%
\end{pgfscope}%
\begin{pgfscope}%
\pgfsys@transformshift{9.036808in}{4.914878in}%
\pgfsys@useobject{currentmarker}{}%
\end{pgfscope}%
\begin{pgfscope}%
\pgfsys@transformshift{9.054401in}{5.179668in}%
\pgfsys@useobject{currentmarker}{}%
\end{pgfscope}%
\begin{pgfscope}%
\pgfsys@transformshift{9.071995in}{4.938468in}%
\pgfsys@useobject{currentmarker}{}%
\end{pgfscope}%
\begin{pgfscope}%
\pgfsys@transformshift{9.089589in}{5.027406in}%
\pgfsys@useobject{currentmarker}{}%
\end{pgfscope}%
\begin{pgfscope}%
\pgfsys@transformshift{9.107183in}{5.020999in}%
\pgfsys@useobject{currentmarker}{}%
\end{pgfscope}%
\begin{pgfscope}%
\pgfsys@transformshift{9.124777in}{4.886656in}%
\pgfsys@useobject{currentmarker}{}%
\end{pgfscope}%
\begin{pgfscope}%
\pgfsys@transformshift{9.142371in}{4.942565in}%
\pgfsys@useobject{currentmarker}{}%
\end{pgfscope}%
\begin{pgfscope}%
\pgfsys@transformshift{9.159965in}{4.951091in}%
\pgfsys@useobject{currentmarker}{}%
\end{pgfscope}%
\begin{pgfscope}%
\pgfsys@transformshift{9.177558in}{4.872367in}%
\pgfsys@useobject{currentmarker}{}%
\end{pgfscope}%
\begin{pgfscope}%
\pgfsys@transformshift{9.195152in}{4.698835in}%
\pgfsys@useobject{currentmarker}{}%
\end{pgfscope}%
\begin{pgfscope}%
\pgfsys@transformshift{9.212746in}{4.818561in}%
\pgfsys@useobject{currentmarker}{}%
\end{pgfscope}%
\begin{pgfscope}%
\pgfsys@transformshift{9.230340in}{4.901053in}%
\pgfsys@useobject{currentmarker}{}%
\end{pgfscope}%
\begin{pgfscope}%
\pgfsys@transformshift{9.247934in}{4.673262in}%
\pgfsys@useobject{currentmarker}{}%
\end{pgfscope}%
\begin{pgfscope}%
\pgfsys@transformshift{9.265528in}{4.707965in}%
\pgfsys@useobject{currentmarker}{}%
\end{pgfscope}%
\begin{pgfscope}%
\pgfsys@transformshift{9.283121in}{4.659015in}%
\pgfsys@useobject{currentmarker}{}%
\end{pgfscope}%
\begin{pgfscope}%
\pgfsys@transformshift{9.300715in}{4.873338in}%
\pgfsys@useobject{currentmarker}{}%
\end{pgfscope}%
\begin{pgfscope}%
\pgfsys@transformshift{9.318309in}{4.736067in}%
\pgfsys@useobject{currentmarker}{}%
\end{pgfscope}%
\begin{pgfscope}%
\pgfsys@transformshift{9.335903in}{4.693339in}%
\pgfsys@useobject{currentmarker}{}%
\end{pgfscope}%
\begin{pgfscope}%
\pgfsys@transformshift{9.353497in}{4.559224in}%
\pgfsys@useobject{currentmarker}{}%
\end{pgfscope}%
\begin{pgfscope}%
\pgfsys@transformshift{9.371091in}{4.680465in}%
\pgfsys@useobject{currentmarker}{}%
\end{pgfscope}%
\begin{pgfscope}%
\pgfsys@transformshift{9.388685in}{4.546350in}%
\pgfsys@useobject{currentmarker}{}%
\end{pgfscope}%
\begin{pgfscope}%
\pgfsys@transformshift{9.406278in}{4.609986in}%
\pgfsys@useobject{currentmarker}{}%
\end{pgfscope}%
\begin{pgfscope}%
\pgfsys@transformshift{9.423872in}{4.535582in}%
\pgfsys@useobject{currentmarker}{}%
\end{pgfscope}%
\begin{pgfscope}%
\pgfsys@transformshift{9.441466in}{4.665186in}%
\pgfsys@useobject{currentmarker}{}%
\end{pgfscope}%
\begin{pgfscope}%
\pgfsys@transformshift{9.459060in}{4.652923in}%
\pgfsys@useobject{currentmarker}{}%
\end{pgfscope}%
\begin{pgfscope}%
\pgfsys@transformshift{9.476654in}{4.572944in}%
\pgfsys@useobject{currentmarker}{}%
\end{pgfscope}%
\begin{pgfscope}%
\pgfsys@transformshift{9.494248in}{4.636754in}%
\pgfsys@useobject{currentmarker}{}%
\end{pgfscope}%
\begin{pgfscope}%
\pgfsys@transformshift{9.511842in}{4.489189in}%
\pgfsys@useobject{currentmarker}{}%
\end{pgfscope}%
\begin{pgfscope}%
\pgfsys@transformshift{9.529435in}{4.454903in}%
\pgfsys@useobject{currentmarker}{}%
\end{pgfscope}%
\begin{pgfscope}%
\pgfsys@transformshift{9.547029in}{4.660065in}%
\pgfsys@useobject{currentmarker}{}%
\end{pgfscope}%
\begin{pgfscope}%
\pgfsys@transformshift{9.564623in}{4.642414in}%
\pgfsys@useobject{currentmarker}{}%
\end{pgfscope}%
\begin{pgfscope}%
\pgfsys@transformshift{9.582217in}{4.702096in}%
\pgfsys@useobject{currentmarker}{}%
\end{pgfscope}%
\begin{pgfscope}%
\pgfsys@transformshift{9.599811in}{4.893914in}%
\pgfsys@useobject{currentmarker}{}%
\end{pgfscope}%
\begin{pgfscope}%
\pgfsys@transformshift{9.617405in}{4.762255in}%
\pgfsys@useobject{currentmarker}{}%
\end{pgfscope}%
\begin{pgfscope}%
\pgfsys@transformshift{9.634999in}{4.589240in}%
\pgfsys@useobject{currentmarker}{}%
\end{pgfscope}%
\begin{pgfscope}%
\pgfsys@transformshift{9.652592in}{4.813773in}%
\pgfsys@useobject{currentmarker}{}%
\end{pgfscope}%
\begin{pgfscope}%
\pgfsys@transformshift{9.670186in}{4.584205in}%
\pgfsys@useobject{currentmarker}{}%
\end{pgfscope}%
\begin{pgfscope}%
\pgfsys@transformshift{9.687780in}{4.690640in}%
\pgfsys@useobject{currentmarker}{}%
\end{pgfscope}%
\begin{pgfscope}%
\pgfsys@transformshift{9.705374in}{4.750662in}%
\pgfsys@useobject{currentmarker}{}%
\end{pgfscope}%
\begin{pgfscope}%
\pgfsys@transformshift{9.722968in}{4.952640in}%
\pgfsys@useobject{currentmarker}{}%
\end{pgfscope}%
\begin{pgfscope}%
\pgfsys@transformshift{9.740562in}{4.722467in}%
\pgfsys@useobject{currentmarker}{}%
\end{pgfscope}%
\begin{pgfscope}%
\pgfsys@transformshift{9.758155in}{4.734605in}%
\pgfsys@useobject{currentmarker}{}%
\end{pgfscope}%
\begin{pgfscope}%
\pgfsys@transformshift{9.775749in}{4.829081in}%
\pgfsys@useobject{currentmarker}{}%
\end{pgfscope}%
\begin{pgfscope}%
\pgfsys@transformshift{9.793343in}{4.791275in}%
\pgfsys@useobject{currentmarker}{}%
\end{pgfscope}%
\begin{pgfscope}%
\pgfsys@transformshift{9.810937in}{4.993005in}%
\pgfsys@useobject{currentmarker}{}%
\end{pgfscope}%
\begin{pgfscope}%
\pgfsys@transformshift{9.828531in}{4.786360in}%
\pgfsys@useobject{currentmarker}{}%
\end{pgfscope}%
\begin{pgfscope}%
\pgfsys@transformshift{9.846125in}{4.796843in}%
\pgfsys@useobject{currentmarker}{}%
\end{pgfscope}%
\begin{pgfscope}%
\pgfsys@transformshift{9.863719in}{4.885409in}%
\pgfsys@useobject{currentmarker}{}%
\end{pgfscope}%
\begin{pgfscope}%
\pgfsys@transformshift{9.881312in}{4.894144in}%
\pgfsys@useobject{currentmarker}{}%
\end{pgfscope}%
\begin{pgfscope}%
\pgfsys@transformshift{9.898906in}{5.155096in}%
\pgfsys@useobject{currentmarker}{}%
\end{pgfscope}%
\begin{pgfscope}%
\pgfsys@transformshift{9.916500in}{5.066958in}%
\pgfsys@useobject{currentmarker}{}%
\end{pgfscope}%
\begin{pgfscope}%
\pgfsys@transformshift{9.934094in}{4.989169in}%
\pgfsys@useobject{currentmarker}{}%
\end{pgfscope}%
\begin{pgfscope}%
\pgfsys@transformshift{9.951688in}{4.863576in}%
\pgfsys@useobject{currentmarker}{}%
\end{pgfscope}%
\begin{pgfscope}%
\pgfsys@transformshift{9.969282in}{5.081064in}%
\pgfsys@useobject{currentmarker}{}%
\end{pgfscope}%
\begin{pgfscope}%
\pgfsys@transformshift{9.986876in}{4.897460in}%
\pgfsys@useobject{currentmarker}{}%
\end{pgfscope}%
\begin{pgfscope}%
\pgfsys@transformshift{10.004469in}{4.844460in}%
\pgfsys@useobject{currentmarker}{}%
\end{pgfscope}%
\begin{pgfscope}%
\pgfsys@transformshift{10.022063in}{5.123690in}%
\pgfsys@useobject{currentmarker}{}%
\end{pgfscope}%
\begin{pgfscope}%
\pgfsys@transformshift{10.039657in}{5.033449in}%
\pgfsys@useobject{currentmarker}{}%
\end{pgfscope}%
\begin{pgfscope}%
\pgfsys@transformshift{10.057251in}{5.091681in}%
\pgfsys@useobject{currentmarker}{}%
\end{pgfscope}%
\begin{pgfscope}%
\pgfsys@transformshift{10.074845in}{5.025036in}%
\pgfsys@useobject{currentmarker}{}%
\end{pgfscope}%
\begin{pgfscope}%
\pgfsys@transformshift{10.092439in}{5.072843in}%
\pgfsys@useobject{currentmarker}{}%
\end{pgfscope}%
\begin{pgfscope}%
\pgfsys@transformshift{10.110033in}{4.910283in}%
\pgfsys@useobject{currentmarker}{}%
\end{pgfscope}%
\begin{pgfscope}%
\pgfsys@transformshift{10.127626in}{4.860777in}%
\pgfsys@useobject{currentmarker}{}%
\end{pgfscope}%
\begin{pgfscope}%
\pgfsys@transformshift{10.145220in}{5.023816in}%
\pgfsys@useobject{currentmarker}{}%
\end{pgfscope}%
\begin{pgfscope}%
\pgfsys@transformshift{10.162814in}{4.859088in}%
\pgfsys@useobject{currentmarker}{}%
\end{pgfscope}%
\begin{pgfscope}%
\pgfsys@transformshift{10.180408in}{4.856192in}%
\pgfsys@useobject{currentmarker}{}%
\end{pgfscope}%
\begin{pgfscope}%
\pgfsys@transformshift{10.198002in}{4.864505in}%
\pgfsys@useobject{currentmarker}{}%
\end{pgfscope}%
\begin{pgfscope}%
\pgfsys@transformshift{10.215596in}{4.896325in}%
\pgfsys@useobject{currentmarker}{}%
\end{pgfscope}%
\begin{pgfscope}%
\pgfsys@transformshift{10.233189in}{4.841352in}%
\pgfsys@useobject{currentmarker}{}%
\end{pgfscope}%
\begin{pgfscope}%
\pgfsys@transformshift{10.250783in}{4.719668in}%
\pgfsys@useobject{currentmarker}{}%
\end{pgfscope}%
\begin{pgfscope}%
\pgfsys@transformshift{10.268377in}{4.776392in}%
\pgfsys@useobject{currentmarker}{}%
\end{pgfscope}%
\begin{pgfscope}%
\pgfsys@transformshift{10.285971in}{4.597368in}%
\pgfsys@useobject{currentmarker}{}%
\end{pgfscope}%
\begin{pgfscope}%
\pgfsys@transformshift{10.303565in}{4.870059in}%
\pgfsys@useobject{currentmarker}{}%
\end{pgfscope}%
\begin{pgfscope}%
\pgfsys@transformshift{10.321159in}{4.625477in}%
\pgfsys@useobject{currentmarker}{}%
\end{pgfscope}%
\begin{pgfscope}%
\pgfsys@transformshift{10.338753in}{4.659313in}%
\pgfsys@useobject{currentmarker}{}%
\end{pgfscope}%
\begin{pgfscope}%
\pgfsys@transformshift{10.356346in}{4.761077in}%
\pgfsys@useobject{currentmarker}{}%
\end{pgfscope}%
\begin{pgfscope}%
\pgfsys@transformshift{10.373940in}{4.665102in}%
\pgfsys@useobject{currentmarker}{}%
\end{pgfscope}%
\begin{pgfscope}%
\pgfsys@transformshift{10.391534in}{4.883742in}%
\pgfsys@useobject{currentmarker}{}%
\end{pgfscope}%
\begin{pgfscope}%
\pgfsys@transformshift{10.409128in}{4.581725in}%
\pgfsys@useobject{currentmarker}{}%
\end{pgfscope}%
\begin{pgfscope}%
\pgfsys@transformshift{10.426722in}{4.729414in}%
\pgfsys@useobject{currentmarker}{}%
\end{pgfscope}%
\begin{pgfscope}%
\pgfsys@transformshift{10.444316in}{4.688394in}%
\pgfsys@useobject{currentmarker}{}%
\end{pgfscope}%
\begin{pgfscope}%
\pgfsys@transformshift{10.461910in}{4.565234in}%
\pgfsys@useobject{currentmarker}{}%
\end{pgfscope}%
\begin{pgfscope}%
\pgfsys@transformshift{10.479503in}{4.731505in}%
\pgfsys@useobject{currentmarker}{}%
\end{pgfscope}%
\begin{pgfscope}%
\pgfsys@transformshift{10.497097in}{4.656390in}%
\pgfsys@useobject{currentmarker}{}%
\end{pgfscope}%
\begin{pgfscope}%
\pgfsys@transformshift{10.514691in}{4.750347in}%
\pgfsys@useobject{currentmarker}{}%
\end{pgfscope}%
\begin{pgfscope}%
\pgfsys@transformshift{10.532285in}{4.755463in}%
\pgfsys@useobject{currentmarker}{}%
\end{pgfscope}%
\begin{pgfscope}%
\pgfsys@transformshift{10.549879in}{4.894045in}%
\pgfsys@useobject{currentmarker}{}%
\end{pgfscope}%
\begin{pgfscope}%
\pgfsys@transformshift{10.567473in}{4.813845in}%
\pgfsys@useobject{currentmarker}{}%
\end{pgfscope}%
\begin{pgfscope}%
\pgfsys@transformshift{10.585067in}{4.646147in}%
\pgfsys@useobject{currentmarker}{}%
\end{pgfscope}%
\begin{pgfscope}%
\pgfsys@transformshift{10.602660in}{4.667737in}%
\pgfsys@useobject{currentmarker}{}%
\end{pgfscope}%
\begin{pgfscope}%
\pgfsys@transformshift{10.620254in}{4.814708in}%
\pgfsys@useobject{currentmarker}{}%
\end{pgfscope}%
\begin{pgfscope}%
\pgfsys@transformshift{10.637848in}{4.781821in}%
\pgfsys@useobject{currentmarker}{}%
\end{pgfscope}%
\begin{pgfscope}%
\pgfsys@transformshift{10.655442in}{4.794632in}%
\pgfsys@useobject{currentmarker}{}%
\end{pgfscope}%
\begin{pgfscope}%
\pgfsys@transformshift{10.673036in}{4.580644in}%
\pgfsys@useobject{currentmarker}{}%
\end{pgfscope}%
\begin{pgfscope}%
\pgfsys@transformshift{10.690630in}{4.760929in}%
\pgfsys@useobject{currentmarker}{}%
\end{pgfscope}%
\begin{pgfscope}%
\pgfsys@transformshift{10.708223in}{4.707595in}%
\pgfsys@useobject{currentmarker}{}%
\end{pgfscope}%
\begin{pgfscope}%
\pgfsys@transformshift{10.725817in}{4.832262in}%
\pgfsys@useobject{currentmarker}{}%
\end{pgfscope}%
\begin{pgfscope}%
\pgfsys@transformshift{10.743411in}{4.815823in}%
\pgfsys@useobject{currentmarker}{}%
\end{pgfscope}%
\begin{pgfscope}%
\pgfsys@transformshift{10.761005in}{4.941839in}%
\pgfsys@useobject{currentmarker}{}%
\end{pgfscope}%
\begin{pgfscope}%
\pgfsys@transformshift{10.778599in}{4.905455in}%
\pgfsys@useobject{currentmarker}{}%
\end{pgfscope}%
\begin{pgfscope}%
\pgfsys@transformshift{10.796193in}{4.978072in}%
\pgfsys@useobject{currentmarker}{}%
\end{pgfscope}%
\begin{pgfscope}%
\pgfsys@transformshift{10.813787in}{4.875602in}%
\pgfsys@useobject{currentmarker}{}%
\end{pgfscope}%
\begin{pgfscope}%
\pgfsys@transformshift{10.831380in}{4.852426in}%
\pgfsys@useobject{currentmarker}{}%
\end{pgfscope}%
\begin{pgfscope}%
\pgfsys@transformshift{10.848974in}{4.932576in}%
\pgfsys@useobject{currentmarker}{}%
\end{pgfscope}%
\begin{pgfscope}%
\pgfsys@transformshift{10.866568in}{4.998190in}%
\pgfsys@useobject{currentmarker}{}%
\end{pgfscope}%
\begin{pgfscope}%
\pgfsys@transformshift{10.884162in}{5.063799in}%
\pgfsys@useobject{currentmarker}{}%
\end{pgfscope}%
\begin{pgfscope}%
\pgfsys@transformshift{10.901756in}{5.280211in}%
\pgfsys@useobject{currentmarker}{}%
\end{pgfscope}%
\begin{pgfscope}%
\pgfsys@transformshift{10.919350in}{5.069477in}%
\pgfsys@useobject{currentmarker}{}%
\end{pgfscope}%
\begin{pgfscope}%
\pgfsys@transformshift{10.936944in}{4.999365in}%
\pgfsys@useobject{currentmarker}{}%
\end{pgfscope}%
\begin{pgfscope}%
\pgfsys@transformshift{10.954537in}{5.083543in}%
\pgfsys@useobject{currentmarker}{}%
\end{pgfscope}%
\begin{pgfscope}%
\pgfsys@transformshift{10.972131in}{5.092281in}%
\pgfsys@useobject{currentmarker}{}%
\end{pgfscope}%
\begin{pgfscope}%
\pgfsys@transformshift{10.989725in}{5.207988in}%
\pgfsys@useobject{currentmarker}{}%
\end{pgfscope}%
\begin{pgfscope}%
\pgfsys@transformshift{11.007319in}{5.019570in}%
\pgfsys@useobject{currentmarker}{}%
\end{pgfscope}%
\begin{pgfscope}%
\pgfsys@transformshift{11.024913in}{5.199289in}%
\pgfsys@useobject{currentmarker}{}%
\end{pgfscope}%
\begin{pgfscope}%
\pgfsys@transformshift{11.042507in}{5.223190in}%
\pgfsys@useobject{currentmarker}{}%
\end{pgfscope}%
\begin{pgfscope}%
\pgfsys@transformshift{11.060101in}{5.243459in}%
\pgfsys@useobject{currentmarker}{}%
\end{pgfscope}%
\begin{pgfscope}%
\pgfsys@transformshift{11.077694in}{5.169522in}%
\pgfsys@useobject{currentmarker}{}%
\end{pgfscope}%
\begin{pgfscope}%
\pgfsys@transformshift{11.095288in}{5.214873in}%
\pgfsys@useobject{currentmarker}{}%
\end{pgfscope}%
\begin{pgfscope}%
\pgfsys@transformshift{11.112882in}{5.100648in}%
\pgfsys@useobject{currentmarker}{}%
\end{pgfscope}%
\begin{pgfscope}%
\pgfsys@transformshift{11.130476in}{5.200314in}%
\pgfsys@useobject{currentmarker}{}%
\end{pgfscope}%
\begin{pgfscope}%
\pgfsys@transformshift{11.148070in}{5.198061in}%
\pgfsys@useobject{currentmarker}{}%
\end{pgfscope}%
\begin{pgfscope}%
\pgfsys@transformshift{11.165664in}{5.296725in}%
\pgfsys@useobject{currentmarker}{}%
\end{pgfscope}%
\begin{pgfscope}%
\pgfsys@transformshift{11.183257in}{5.135419in}%
\pgfsys@useobject{currentmarker}{}%
\end{pgfscope}%
\begin{pgfscope}%
\pgfsys@transformshift{11.200851in}{5.330218in}%
\pgfsys@useobject{currentmarker}{}%
\end{pgfscope}%
\begin{pgfscope}%
\pgfsys@transformshift{11.218445in}{5.398508in}%
\pgfsys@useobject{currentmarker}{}%
\end{pgfscope}%
\begin{pgfscope}%
\pgfsys@transformshift{11.236039in}{5.029008in}%
\pgfsys@useobject{currentmarker}{}%
\end{pgfscope}%
\begin{pgfscope}%
\pgfsys@transformshift{11.253633in}{5.276102in}%
\pgfsys@useobject{currentmarker}{}%
\end{pgfscope}%
\begin{pgfscope}%
\pgfsys@transformshift{11.271227in}{5.292913in}%
\pgfsys@useobject{currentmarker}{}%
\end{pgfscope}%
\begin{pgfscope}%
\pgfsys@transformshift{11.288821in}{5.149004in}%
\pgfsys@useobject{currentmarker}{}%
\end{pgfscope}%
\begin{pgfscope}%
\pgfsys@transformshift{11.306414in}{5.162656in}%
\pgfsys@useobject{currentmarker}{}%
\end{pgfscope}%
\begin{pgfscope}%
\pgfsys@transformshift{11.324008in}{5.178002in}%
\pgfsys@useobject{currentmarker}{}%
\end{pgfscope}%
\begin{pgfscope}%
\pgfsys@transformshift{11.341602in}{5.148988in}%
\pgfsys@useobject{currentmarker}{}%
\end{pgfscope}%
\begin{pgfscope}%
\pgfsys@transformshift{11.359196in}{5.135262in}%
\pgfsys@useobject{currentmarker}{}%
\end{pgfscope}%
\begin{pgfscope}%
\pgfsys@transformshift{11.376790in}{4.983102in}%
\pgfsys@useobject{currentmarker}{}%
\end{pgfscope}%
\begin{pgfscope}%
\pgfsys@transformshift{11.394384in}{5.258780in}%
\pgfsys@useobject{currentmarker}{}%
\end{pgfscope}%
\begin{pgfscope}%
\pgfsys@transformshift{11.411978in}{5.238859in}%
\pgfsys@useobject{currentmarker}{}%
\end{pgfscope}%
\begin{pgfscope}%
\pgfsys@transformshift{11.429571in}{5.033756in}%
\pgfsys@useobject{currentmarker}{}%
\end{pgfscope}%
\begin{pgfscope}%
\pgfsys@transformshift{11.447165in}{4.955731in}%
\pgfsys@useobject{currentmarker}{}%
\end{pgfscope}%
\begin{pgfscope}%
\pgfsys@transformshift{11.464759in}{5.147808in}%
\pgfsys@useobject{currentmarker}{}%
\end{pgfscope}%
\begin{pgfscope}%
\pgfsys@transformshift{11.482353in}{5.026386in}%
\pgfsys@useobject{currentmarker}{}%
\end{pgfscope}%
\end{pgfscope}%
\begin{pgfscope}%
\pgfpathrectangle{\pgfqpoint{7.105882in}{4.421053in}}{\pgfqpoint{4.376471in}{0.978947in}} %
\pgfusepath{clip}%
\pgfsetroundcap%
\pgfsetroundjoin%
\pgfsetlinewidth{1.756562pt}%
\definecolor{currentstroke}{rgb}{0.298039,0.447059,0.690196}%
\pgfsetstrokecolor{currentstroke}%
\pgfsetdash{}{0pt}%
\pgfpathmoveto{\pgfqpoint{7.981176in}{4.788157in}}%
\pgfpathlineto{\pgfqpoint{11.482353in}{4.788159in}}%
\pgfpathlineto{\pgfqpoint{11.482353in}{4.788159in}}%
\pgfusepath{stroke}%
\end{pgfscope}%
\begin{pgfscope}%
\pgfsetrectcap%
\pgfsetmiterjoin%
\pgfsetlinewidth{1.003750pt}%
\definecolor{currentstroke}{rgb}{0.800000,0.800000,0.800000}%
\pgfsetstrokecolor{currentstroke}%
\pgfsetdash{}{0pt}%
\pgfpathmoveto{\pgfqpoint{7.105882in}{4.421053in}}%
\pgfpathlineto{\pgfqpoint{7.105882in}{5.400000in}}%
\pgfusepath{stroke}%
\end{pgfscope}%
\begin{pgfscope}%
\pgfsetrectcap%
\pgfsetmiterjoin%
\pgfsetlinewidth{1.003750pt}%
\definecolor{currentstroke}{rgb}{0.800000,0.800000,0.800000}%
\pgfsetstrokecolor{currentstroke}%
\pgfsetdash{}{0pt}%
\pgfpathmoveto{\pgfqpoint{11.482353in}{4.421053in}}%
\pgfpathlineto{\pgfqpoint{11.482353in}{5.400000in}}%
\pgfusepath{stroke}%
\end{pgfscope}%
\begin{pgfscope}%
\pgfsetrectcap%
\pgfsetmiterjoin%
\pgfsetlinewidth{1.003750pt}%
\definecolor{currentstroke}{rgb}{0.800000,0.800000,0.800000}%
\pgfsetstrokecolor{currentstroke}%
\pgfsetdash{}{0pt}%
\pgfpathmoveto{\pgfqpoint{7.105882in}{5.400000in}}%
\pgfpathlineto{\pgfqpoint{11.482353in}{5.400000in}}%
\pgfusepath{stroke}%
\end{pgfscope}%
\begin{pgfscope}%
\pgfsetrectcap%
\pgfsetmiterjoin%
\pgfsetlinewidth{1.003750pt}%
\definecolor{currentstroke}{rgb}{0.800000,0.800000,0.800000}%
\pgfsetstrokecolor{currentstroke}%
\pgfsetdash{}{0pt}%
\pgfpathmoveto{\pgfqpoint{7.105882in}{4.421053in}}%
\pgfpathlineto{\pgfqpoint{11.482353in}{4.421053in}}%
\pgfusepath{stroke}%
\end{pgfscope}%
\begin{pgfscope}%
\pgfsetroundcap%
\pgfsetroundjoin%
\pgfsetlinewidth{1.756562pt}%
\definecolor{currentstroke}{rgb}{0.298039,0.447059,0.690196}%
\pgfsetstrokecolor{currentstroke}%
\pgfsetdash{}{0pt}%
\pgfpathmoveto{\pgfqpoint{7.230882in}{5.015195in}}%
\pgfpathlineto{\pgfqpoint{7.508660in}{5.015195in}}%
\pgfusepath{stroke}%
\end{pgfscope}%
\begin{pgfscope}%
\definecolor{textcolor}{rgb}{0.150000,0.150000,0.150000}%
\pgfsetstrokecolor{textcolor}%
\pgfsetfillcolor{textcolor}%
\pgftext[x=7.619771in,y=4.966584in,left,base]{\color{textcolor}\sffamily\fontsize{10.000000}{12.000000}\selectfont \(\displaystyle \widetilde{\Phi}^* \theta^{\parallel}\)}%
\end{pgfscope}%
\begin{pgfscope}%
\pgfsetbuttcap%
\pgfsetroundjoin%
\definecolor{currentfill}{rgb}{1.000000,0.000000,0.000000}%
\pgfsetfillcolor{currentfill}%
\pgfsetlinewidth{2.007500pt}%
\definecolor{currentstroke}{rgb}{1.000000,0.000000,0.000000}%
\pgfsetstrokecolor{currentstroke}%
\pgfsetdash{}{0pt}%
\pgfpathmoveto{\pgfqpoint{7.338715in}{4.806577in}}%
\pgfpathlineto{\pgfqpoint{7.400828in}{4.806577in}}%
\pgfpathmoveto{\pgfqpoint{7.369771in}{4.775521in}}%
\pgfpathlineto{\pgfqpoint{7.369771in}{4.837634in}}%
\pgfusepath{stroke,fill}%
\end{pgfscope}%
\begin{pgfscope}%
\pgfsetbuttcap%
\pgfsetroundjoin%
\definecolor{currentfill}{rgb}{1.000000,0.000000,0.000000}%
\pgfsetfillcolor{currentfill}%
\pgfsetlinewidth{2.007500pt}%
\definecolor{currentstroke}{rgb}{1.000000,0.000000,0.000000}%
\pgfsetstrokecolor{currentstroke}%
\pgfsetdash{}{0pt}%
\pgfpathmoveto{\pgfqpoint{7.338715in}{4.806577in}}%
\pgfpathlineto{\pgfqpoint{7.400828in}{4.806577in}}%
\pgfpathmoveto{\pgfqpoint{7.369771in}{4.775521in}}%
\pgfpathlineto{\pgfqpoint{7.369771in}{4.837634in}}%
\pgfusepath{stroke,fill}%
\end{pgfscope}%
\begin{pgfscope}%
\pgfsetbuttcap%
\pgfsetroundjoin%
\definecolor{currentfill}{rgb}{1.000000,0.000000,0.000000}%
\pgfsetfillcolor{currentfill}%
\pgfsetlinewidth{2.007500pt}%
\definecolor{currentstroke}{rgb}{1.000000,0.000000,0.000000}%
\pgfsetstrokecolor{currentstroke}%
\pgfsetdash{}{0pt}%
\pgfpathmoveto{\pgfqpoint{7.338715in}{4.806577in}}%
\pgfpathlineto{\pgfqpoint{7.400828in}{4.806577in}}%
\pgfpathmoveto{\pgfqpoint{7.369771in}{4.775521in}}%
\pgfpathlineto{\pgfqpoint{7.369771in}{4.837634in}}%
\pgfusepath{stroke,fill}%
\end{pgfscope}%
\begin{pgfscope}%
\definecolor{textcolor}{rgb}{0.150000,0.150000,0.150000}%
\pgfsetstrokecolor{textcolor}%
\pgfsetfillcolor{textcolor}%
\pgftext[x=7.619771in,y=4.770119in,left,base]{\color{textcolor}\sffamily\fontsize{10.000000}{12.000000}\selectfont train}%
\end{pgfscope}%
\begin{pgfscope}%
\pgfsetbuttcap%
\pgfsetroundjoin%
\definecolor{currentfill}{rgb}{0.000000,0.000000,0.000000}%
\pgfsetfillcolor{currentfill}%
\pgfsetlinewidth{0.301125pt}%
\definecolor{currentstroke}{rgb}{0.000000,0.000000,0.000000}%
\pgfsetstrokecolor{currentstroke}%
\pgfsetdash{}{0pt}%
\pgfpathmoveto{\pgfqpoint{7.369771in}{4.594584in}}%
\pgfpathcurveto{\pgfqpoint{7.373889in}{4.594584in}}{\pgfqpoint{7.377839in}{4.596220in}}{\pgfqpoint{7.380751in}{4.599132in}}%
\pgfpathcurveto{\pgfqpoint{7.383663in}{4.602044in}}{\pgfqpoint{7.385299in}{4.605994in}}{\pgfqpoint{7.385299in}{4.610112in}}%
\pgfpathcurveto{\pgfqpoint{7.385299in}{4.614230in}}{\pgfqpoint{7.383663in}{4.618180in}}{\pgfqpoint{7.380751in}{4.621092in}}%
\pgfpathcurveto{\pgfqpoint{7.377839in}{4.624004in}}{\pgfqpoint{7.373889in}{4.625640in}}{\pgfqpoint{7.369771in}{4.625640in}}%
\pgfpathcurveto{\pgfqpoint{7.365653in}{4.625640in}}{\pgfqpoint{7.361703in}{4.624004in}}{\pgfqpoint{7.358791in}{4.621092in}}%
\pgfpathcurveto{\pgfqpoint{7.355879in}{4.618180in}}{\pgfqpoint{7.354243in}{4.614230in}}{\pgfqpoint{7.354243in}{4.610112in}}%
\pgfpathcurveto{\pgfqpoint{7.354243in}{4.605994in}}{\pgfqpoint{7.355879in}{4.602044in}}{\pgfqpoint{7.358791in}{4.599132in}}%
\pgfpathcurveto{\pgfqpoint{7.361703in}{4.596220in}}{\pgfqpoint{7.365653in}{4.594584in}}{\pgfqpoint{7.369771in}{4.594584in}}%
\pgfpathclose%
\pgfusepath{stroke,fill}%
\end{pgfscope}%
\begin{pgfscope}%
\pgfsetbuttcap%
\pgfsetroundjoin%
\definecolor{currentfill}{rgb}{0.000000,0.000000,0.000000}%
\pgfsetfillcolor{currentfill}%
\pgfsetlinewidth{0.301125pt}%
\definecolor{currentstroke}{rgb}{0.000000,0.000000,0.000000}%
\pgfsetstrokecolor{currentstroke}%
\pgfsetdash{}{0pt}%
\pgfpathmoveto{\pgfqpoint{7.369771in}{4.594584in}}%
\pgfpathcurveto{\pgfqpoint{7.373889in}{4.594584in}}{\pgfqpoint{7.377839in}{4.596220in}}{\pgfqpoint{7.380751in}{4.599132in}}%
\pgfpathcurveto{\pgfqpoint{7.383663in}{4.602044in}}{\pgfqpoint{7.385299in}{4.605994in}}{\pgfqpoint{7.385299in}{4.610112in}}%
\pgfpathcurveto{\pgfqpoint{7.385299in}{4.614230in}}{\pgfqpoint{7.383663in}{4.618180in}}{\pgfqpoint{7.380751in}{4.621092in}}%
\pgfpathcurveto{\pgfqpoint{7.377839in}{4.624004in}}{\pgfqpoint{7.373889in}{4.625640in}}{\pgfqpoint{7.369771in}{4.625640in}}%
\pgfpathcurveto{\pgfqpoint{7.365653in}{4.625640in}}{\pgfqpoint{7.361703in}{4.624004in}}{\pgfqpoint{7.358791in}{4.621092in}}%
\pgfpathcurveto{\pgfqpoint{7.355879in}{4.618180in}}{\pgfqpoint{7.354243in}{4.614230in}}{\pgfqpoint{7.354243in}{4.610112in}}%
\pgfpathcurveto{\pgfqpoint{7.354243in}{4.605994in}}{\pgfqpoint{7.355879in}{4.602044in}}{\pgfqpoint{7.358791in}{4.599132in}}%
\pgfpathcurveto{\pgfqpoint{7.361703in}{4.596220in}}{\pgfqpoint{7.365653in}{4.594584in}}{\pgfqpoint{7.369771in}{4.594584in}}%
\pgfpathclose%
\pgfusepath{stroke,fill}%
\end{pgfscope}%
\begin{pgfscope}%
\pgfsetbuttcap%
\pgfsetroundjoin%
\definecolor{currentfill}{rgb}{0.000000,0.000000,0.000000}%
\pgfsetfillcolor{currentfill}%
\pgfsetlinewidth{0.301125pt}%
\definecolor{currentstroke}{rgb}{0.000000,0.000000,0.000000}%
\pgfsetstrokecolor{currentstroke}%
\pgfsetdash{}{0pt}%
\pgfpathmoveto{\pgfqpoint{7.369771in}{4.594584in}}%
\pgfpathcurveto{\pgfqpoint{7.373889in}{4.594584in}}{\pgfqpoint{7.377839in}{4.596220in}}{\pgfqpoint{7.380751in}{4.599132in}}%
\pgfpathcurveto{\pgfqpoint{7.383663in}{4.602044in}}{\pgfqpoint{7.385299in}{4.605994in}}{\pgfqpoint{7.385299in}{4.610112in}}%
\pgfpathcurveto{\pgfqpoint{7.385299in}{4.614230in}}{\pgfqpoint{7.383663in}{4.618180in}}{\pgfqpoint{7.380751in}{4.621092in}}%
\pgfpathcurveto{\pgfqpoint{7.377839in}{4.624004in}}{\pgfqpoint{7.373889in}{4.625640in}}{\pgfqpoint{7.369771in}{4.625640in}}%
\pgfpathcurveto{\pgfqpoint{7.365653in}{4.625640in}}{\pgfqpoint{7.361703in}{4.624004in}}{\pgfqpoint{7.358791in}{4.621092in}}%
\pgfpathcurveto{\pgfqpoint{7.355879in}{4.618180in}}{\pgfqpoint{7.354243in}{4.614230in}}{\pgfqpoint{7.354243in}{4.610112in}}%
\pgfpathcurveto{\pgfqpoint{7.354243in}{4.605994in}}{\pgfqpoint{7.355879in}{4.602044in}}{\pgfqpoint{7.358791in}{4.599132in}}%
\pgfpathcurveto{\pgfqpoint{7.361703in}{4.596220in}}{\pgfqpoint{7.365653in}{4.594584in}}{\pgfqpoint{7.369771in}{4.594584in}}%
\pgfpathclose%
\pgfusepath{stroke,fill}%
\end{pgfscope}%
\begin{pgfscope}%
\definecolor{textcolor}{rgb}{0.150000,0.150000,0.150000}%
\pgfsetstrokecolor{textcolor}%
\pgfsetfillcolor{textcolor}%
\pgftext[x=7.619771in,y=4.573654in,left,base]{\color{textcolor}\sffamily\fontsize{10.000000}{12.000000}\selectfont test}%
\end{pgfscope}%
\begin{pgfscope}%
\pgfsetbuttcap%
\pgfsetmiterjoin%
\definecolor{currentfill}{rgb}{1.000000,1.000000,1.000000}%
\pgfsetfillcolor{currentfill}%
\pgfsetlinewidth{0.000000pt}%
\definecolor{currentstroke}{rgb}{0.000000,0.000000,0.000000}%
\pgfsetstrokecolor{currentstroke}%
\pgfsetstrokeopacity{0.000000}%
\pgfsetdash{}{0pt}%
\pgfpathmoveto{\pgfqpoint{12.211765in}{4.421053in}}%
\pgfpathlineto{\pgfqpoint{14.400000in}{4.421053in}}%
\pgfpathlineto{\pgfqpoint{14.400000in}{5.400000in}}%
\pgfpathlineto{\pgfqpoint{12.211765in}{5.400000in}}%
\pgfpathclose%
\pgfusepath{fill}%
\end{pgfscope}%
\begin{pgfscope}%
\pgfpathrectangle{\pgfqpoint{12.211765in}{4.421053in}}{\pgfqpoint{2.188235in}{0.978947in}} %
\pgfusepath{clip}%
\pgfsetroundcap%
\pgfsetroundjoin%
\pgfsetlinewidth{1.003750pt}%
\definecolor{currentstroke}{rgb}{0.800000,0.800000,0.800000}%
\pgfsetstrokecolor{currentstroke}%
\pgfsetdash{}{0pt}%
\pgfpathmoveto{\pgfqpoint{12.211765in}{4.421053in}}%
\pgfpathlineto{\pgfqpoint{12.211765in}{5.400000in}}%
\pgfusepath{stroke}%
\end{pgfscope}%
\begin{pgfscope}%
\pgfpathrectangle{\pgfqpoint{12.211765in}{4.421053in}}{\pgfqpoint{2.188235in}{0.978947in}} %
\pgfusepath{clip}%
\pgfsetroundcap%
\pgfsetroundjoin%
\pgfsetlinewidth{1.003750pt}%
\definecolor{currentstroke}{rgb}{0.800000,0.800000,0.800000}%
\pgfsetstrokecolor{currentstroke}%
\pgfsetdash{}{0pt}%
\pgfpathmoveto{\pgfqpoint{12.485294in}{4.421053in}}%
\pgfpathlineto{\pgfqpoint{12.485294in}{5.400000in}}%
\pgfusepath{stroke}%
\end{pgfscope}%
\begin{pgfscope}%
\pgfpathrectangle{\pgfqpoint{12.211765in}{4.421053in}}{\pgfqpoint{2.188235in}{0.978947in}} %
\pgfusepath{clip}%
\pgfsetroundcap%
\pgfsetroundjoin%
\pgfsetlinewidth{1.003750pt}%
\definecolor{currentstroke}{rgb}{0.800000,0.800000,0.800000}%
\pgfsetstrokecolor{currentstroke}%
\pgfsetdash{}{0pt}%
\pgfpathmoveto{\pgfqpoint{12.758824in}{4.421053in}}%
\pgfpathlineto{\pgfqpoint{12.758824in}{5.400000in}}%
\pgfusepath{stroke}%
\end{pgfscope}%
\begin{pgfscope}%
\pgfpathrectangle{\pgfqpoint{12.211765in}{4.421053in}}{\pgfqpoint{2.188235in}{0.978947in}} %
\pgfusepath{clip}%
\pgfsetroundcap%
\pgfsetroundjoin%
\pgfsetlinewidth{1.003750pt}%
\definecolor{currentstroke}{rgb}{0.800000,0.800000,0.800000}%
\pgfsetstrokecolor{currentstroke}%
\pgfsetdash{}{0pt}%
\pgfpathmoveto{\pgfqpoint{13.032353in}{4.421053in}}%
\pgfpathlineto{\pgfqpoint{13.032353in}{5.400000in}}%
\pgfusepath{stroke}%
\end{pgfscope}%
\begin{pgfscope}%
\pgfpathrectangle{\pgfqpoint{12.211765in}{4.421053in}}{\pgfqpoint{2.188235in}{0.978947in}} %
\pgfusepath{clip}%
\pgfsetroundcap%
\pgfsetroundjoin%
\pgfsetlinewidth{1.003750pt}%
\definecolor{currentstroke}{rgb}{0.800000,0.800000,0.800000}%
\pgfsetstrokecolor{currentstroke}%
\pgfsetdash{}{0pt}%
\pgfpathmoveto{\pgfqpoint{13.305882in}{4.421053in}}%
\pgfpathlineto{\pgfqpoint{13.305882in}{5.400000in}}%
\pgfusepath{stroke}%
\end{pgfscope}%
\begin{pgfscope}%
\pgfpathrectangle{\pgfqpoint{12.211765in}{4.421053in}}{\pgfqpoint{2.188235in}{0.978947in}} %
\pgfusepath{clip}%
\pgfsetroundcap%
\pgfsetroundjoin%
\pgfsetlinewidth{1.003750pt}%
\definecolor{currentstroke}{rgb}{0.800000,0.800000,0.800000}%
\pgfsetstrokecolor{currentstroke}%
\pgfsetdash{}{0pt}%
\pgfpathmoveto{\pgfqpoint{13.579412in}{4.421053in}}%
\pgfpathlineto{\pgfqpoint{13.579412in}{5.400000in}}%
\pgfusepath{stroke}%
\end{pgfscope}%
\begin{pgfscope}%
\pgfpathrectangle{\pgfqpoint{12.211765in}{4.421053in}}{\pgfqpoint{2.188235in}{0.978947in}} %
\pgfusepath{clip}%
\pgfsetroundcap%
\pgfsetroundjoin%
\pgfsetlinewidth{1.003750pt}%
\definecolor{currentstroke}{rgb}{0.800000,0.800000,0.800000}%
\pgfsetstrokecolor{currentstroke}%
\pgfsetdash{}{0pt}%
\pgfpathmoveto{\pgfqpoint{13.852941in}{4.421053in}}%
\pgfpathlineto{\pgfqpoint{13.852941in}{5.400000in}}%
\pgfusepath{stroke}%
\end{pgfscope}%
\begin{pgfscope}%
\pgfpathrectangle{\pgfqpoint{12.211765in}{4.421053in}}{\pgfqpoint{2.188235in}{0.978947in}} %
\pgfusepath{clip}%
\pgfsetroundcap%
\pgfsetroundjoin%
\pgfsetlinewidth{1.003750pt}%
\definecolor{currentstroke}{rgb}{0.800000,0.800000,0.800000}%
\pgfsetstrokecolor{currentstroke}%
\pgfsetdash{}{0pt}%
\pgfpathmoveto{\pgfqpoint{14.126471in}{4.421053in}}%
\pgfpathlineto{\pgfqpoint{14.126471in}{5.400000in}}%
\pgfusepath{stroke}%
\end{pgfscope}%
\begin{pgfscope}%
\pgfpathrectangle{\pgfqpoint{12.211765in}{4.421053in}}{\pgfqpoint{2.188235in}{0.978947in}} %
\pgfusepath{clip}%
\pgfsetroundcap%
\pgfsetroundjoin%
\pgfsetlinewidth{1.003750pt}%
\definecolor{currentstroke}{rgb}{0.800000,0.800000,0.800000}%
\pgfsetstrokecolor{currentstroke}%
\pgfsetdash{}{0pt}%
\pgfpathmoveto{\pgfqpoint{14.400000in}{4.421053in}}%
\pgfpathlineto{\pgfqpoint{14.400000in}{5.400000in}}%
\pgfusepath{stroke}%
\end{pgfscope}%
\begin{pgfscope}%
\pgfpathrectangle{\pgfqpoint{12.211765in}{4.421053in}}{\pgfqpoint{2.188235in}{0.978947in}} %
\pgfusepath{clip}%
\pgfsetroundcap%
\pgfsetroundjoin%
\pgfsetlinewidth{1.003750pt}%
\definecolor{currentstroke}{rgb}{0.800000,0.800000,0.800000}%
\pgfsetstrokecolor{currentstroke}%
\pgfsetdash{}{0pt}%
\pgfpathmoveto{\pgfqpoint{12.211765in}{4.421053in}}%
\pgfpathlineto{\pgfqpoint{14.400000in}{4.421053in}}%
\pgfusepath{stroke}%
\end{pgfscope}%
\begin{pgfscope}%
\definecolor{textcolor}{rgb}{0.150000,0.150000,0.150000}%
\pgfsetstrokecolor{textcolor}%
\pgfsetfillcolor{textcolor}%
\pgftext[x=12.114542in,y=4.421053in,right,]{\color{textcolor}\sffamily\fontsize{10.000000}{12.000000}\selectfont \(\displaystyle 0\)}%
\end{pgfscope}%
\begin{pgfscope}%
\pgfpathrectangle{\pgfqpoint{12.211765in}{4.421053in}}{\pgfqpoint{2.188235in}{0.978947in}} %
\pgfusepath{clip}%
\pgfsetroundcap%
\pgfsetroundjoin%
\pgfsetlinewidth{1.003750pt}%
\definecolor{currentstroke}{rgb}{0.800000,0.800000,0.800000}%
\pgfsetstrokecolor{currentstroke}%
\pgfsetdash{}{0pt}%
\pgfpathmoveto{\pgfqpoint{12.211765in}{4.665789in}}%
\pgfpathlineto{\pgfqpoint{14.400000in}{4.665789in}}%
\pgfusepath{stroke}%
\end{pgfscope}%
\begin{pgfscope}%
\definecolor{textcolor}{rgb}{0.150000,0.150000,0.150000}%
\pgfsetstrokecolor{textcolor}%
\pgfsetfillcolor{textcolor}%
\pgftext[x=12.114542in,y=4.665789in,right,]{\color{textcolor}\sffamily\fontsize{10.000000}{12.000000}\selectfont \(\displaystyle 50\)}%
\end{pgfscope}%
\begin{pgfscope}%
\pgfpathrectangle{\pgfqpoint{12.211765in}{4.421053in}}{\pgfqpoint{2.188235in}{0.978947in}} %
\pgfusepath{clip}%
\pgfsetroundcap%
\pgfsetroundjoin%
\pgfsetlinewidth{1.003750pt}%
\definecolor{currentstroke}{rgb}{0.800000,0.800000,0.800000}%
\pgfsetstrokecolor{currentstroke}%
\pgfsetdash{}{0pt}%
\pgfpathmoveto{\pgfqpoint{12.211765in}{4.910526in}}%
\pgfpathlineto{\pgfqpoint{14.400000in}{4.910526in}}%
\pgfusepath{stroke}%
\end{pgfscope}%
\begin{pgfscope}%
\definecolor{textcolor}{rgb}{0.150000,0.150000,0.150000}%
\pgfsetstrokecolor{textcolor}%
\pgfsetfillcolor{textcolor}%
\pgftext[x=12.114542in,y=4.910526in,right,]{\color{textcolor}\sffamily\fontsize{10.000000}{12.000000}\selectfont \(\displaystyle 100\)}%
\end{pgfscope}%
\begin{pgfscope}%
\pgfpathrectangle{\pgfqpoint{12.211765in}{4.421053in}}{\pgfqpoint{2.188235in}{0.978947in}} %
\pgfusepath{clip}%
\pgfsetroundcap%
\pgfsetroundjoin%
\pgfsetlinewidth{1.003750pt}%
\definecolor{currentstroke}{rgb}{0.800000,0.800000,0.800000}%
\pgfsetstrokecolor{currentstroke}%
\pgfsetdash{}{0pt}%
\pgfpathmoveto{\pgfqpoint{12.211765in}{5.155263in}}%
\pgfpathlineto{\pgfqpoint{14.400000in}{5.155263in}}%
\pgfusepath{stroke}%
\end{pgfscope}%
\begin{pgfscope}%
\definecolor{textcolor}{rgb}{0.150000,0.150000,0.150000}%
\pgfsetstrokecolor{textcolor}%
\pgfsetfillcolor{textcolor}%
\pgftext[x=12.114542in,y=5.155263in,right,]{\color{textcolor}\sffamily\fontsize{10.000000}{12.000000}\selectfont \(\displaystyle 150\)}%
\end{pgfscope}%
\begin{pgfscope}%
\pgfpathrectangle{\pgfqpoint{12.211765in}{4.421053in}}{\pgfqpoint{2.188235in}{0.978947in}} %
\pgfusepath{clip}%
\pgfsetroundcap%
\pgfsetroundjoin%
\pgfsetlinewidth{1.003750pt}%
\definecolor{currentstroke}{rgb}{0.800000,0.800000,0.800000}%
\pgfsetstrokecolor{currentstroke}%
\pgfsetdash{}{0pt}%
\pgfpathmoveto{\pgfqpoint{12.211765in}{5.400000in}}%
\pgfpathlineto{\pgfqpoint{14.400000in}{5.400000in}}%
\pgfusepath{stroke}%
\end{pgfscope}%
\begin{pgfscope}%
\definecolor{textcolor}{rgb}{0.150000,0.150000,0.150000}%
\pgfsetstrokecolor{textcolor}%
\pgfsetfillcolor{textcolor}%
\pgftext[x=12.114542in,y=5.400000in,right,]{\color{textcolor}\sffamily\fontsize{10.000000}{12.000000}\selectfont \(\displaystyle 200\)}%
\end{pgfscope}%
\begin{pgfscope}%
\definecolor{textcolor}{rgb}{0.150000,0.150000,0.150000}%
\pgfsetstrokecolor{textcolor}%
\pgfsetfillcolor{textcolor}%
\pgftext[x=11.836764in,y=4.910526in,,bottom,rotate=90.000000]{\color{textcolor}\sffamily\fontsize{11.000000}{13.200000}\selectfont \(\displaystyle \theta^{\parallel}_j\)}%
\end{pgfscope}%
\begin{pgfscope}%
\pgfpathrectangle{\pgfqpoint{12.211765in}{4.421053in}}{\pgfqpoint{2.188235in}{0.978947in}} %
\pgfusepath{clip}%
\pgfsetroundcap%
\pgfsetroundjoin%
\pgfsetlinewidth{1.756562pt}%
\definecolor{currentstroke}{rgb}{0.298039,0.447059,0.690196}%
\pgfsetstrokecolor{currentstroke}%
\pgfsetdash{}{0pt}%
\pgfpathmoveto{\pgfqpoint{13.305882in}{4.421053in}}%
\pgfpathlineto{\pgfqpoint{13.305882in}{5.395105in}}%
\pgfpathlineto{\pgfqpoint{13.305882in}{5.395105in}}%
\pgfusepath{stroke}%
\end{pgfscope}%
\begin{pgfscope}%
\pgfsetrectcap%
\pgfsetmiterjoin%
\pgfsetlinewidth{1.003750pt}%
\definecolor{currentstroke}{rgb}{0.800000,0.800000,0.800000}%
\pgfsetstrokecolor{currentstroke}%
\pgfsetdash{}{0pt}%
\pgfpathmoveto{\pgfqpoint{12.211765in}{4.421053in}}%
\pgfpathlineto{\pgfqpoint{12.211765in}{5.400000in}}%
\pgfusepath{stroke}%
\end{pgfscope}%
\begin{pgfscope}%
\pgfsetrectcap%
\pgfsetmiterjoin%
\pgfsetlinewidth{1.003750pt}%
\definecolor{currentstroke}{rgb}{0.800000,0.800000,0.800000}%
\pgfsetstrokecolor{currentstroke}%
\pgfsetdash{}{0pt}%
\pgfpathmoveto{\pgfqpoint{14.400000in}{4.421053in}}%
\pgfpathlineto{\pgfqpoint{14.400000in}{5.400000in}}%
\pgfusepath{stroke}%
\end{pgfscope}%
\begin{pgfscope}%
\pgfsetrectcap%
\pgfsetmiterjoin%
\pgfsetlinewidth{1.003750pt}%
\definecolor{currentstroke}{rgb}{0.800000,0.800000,0.800000}%
\pgfsetstrokecolor{currentstroke}%
\pgfsetdash{}{0pt}%
\pgfpathmoveto{\pgfqpoint{12.211765in}{5.400000in}}%
\pgfpathlineto{\pgfqpoint{14.400000in}{5.400000in}}%
\pgfusepath{stroke}%
\end{pgfscope}%
\begin{pgfscope}%
\pgfsetrectcap%
\pgfsetmiterjoin%
\pgfsetlinewidth{1.003750pt}%
\definecolor{currentstroke}{rgb}{0.800000,0.800000,0.800000}%
\pgfsetstrokecolor{currentstroke}%
\pgfsetdash{}{0pt}%
\pgfpathmoveto{\pgfqpoint{12.211765in}{4.421053in}}%
\pgfpathlineto{\pgfqpoint{14.400000in}{4.421053in}}%
\pgfusepath{stroke}%
\end{pgfscope}%
\begin{pgfscope}%
\pgfsetbuttcap%
\pgfsetmiterjoin%
\definecolor{currentfill}{rgb}{1.000000,1.000000,1.000000}%
\pgfsetfillcolor{currentfill}%
\pgfsetlinewidth{0.000000pt}%
\definecolor{currentstroke}{rgb}{0.000000,0.000000,0.000000}%
\pgfsetstrokecolor{currentstroke}%
\pgfsetstrokeopacity{0.000000}%
\pgfsetdash{}{0pt}%
\pgfpathmoveto{\pgfqpoint{2.000000in}{3.197368in}}%
\pgfpathlineto{\pgfqpoint{6.376471in}{3.197368in}}%
\pgfpathlineto{\pgfqpoint{6.376471in}{4.176316in}}%
\pgfpathlineto{\pgfqpoint{2.000000in}{4.176316in}}%
\pgfpathclose%
\pgfusepath{fill}%
\end{pgfscope}%
\begin{pgfscope}%
\pgfpathrectangle{\pgfqpoint{2.000000in}{3.197368in}}{\pgfqpoint{4.376471in}{0.978947in}} %
\pgfusepath{clip}%
\pgfsetroundcap%
\pgfsetroundjoin%
\pgfsetlinewidth{1.003750pt}%
\definecolor{currentstroke}{rgb}{0.800000,0.800000,0.800000}%
\pgfsetstrokecolor{currentstroke}%
\pgfsetdash{}{0pt}%
\pgfpathmoveto{\pgfqpoint{2.000000in}{3.197368in}}%
\pgfpathlineto{\pgfqpoint{2.000000in}{4.176316in}}%
\pgfusepath{stroke}%
\end{pgfscope}%
\begin{pgfscope}%
\pgfpathrectangle{\pgfqpoint{2.000000in}{3.197368in}}{\pgfqpoint{4.376471in}{0.978947in}} %
\pgfusepath{clip}%
\pgfsetroundcap%
\pgfsetroundjoin%
\pgfsetlinewidth{1.003750pt}%
\definecolor{currentstroke}{rgb}{0.800000,0.800000,0.800000}%
\pgfsetstrokecolor{currentstroke}%
\pgfsetdash{}{0pt}%
\pgfpathmoveto{\pgfqpoint{2.875294in}{3.197368in}}%
\pgfpathlineto{\pgfqpoint{2.875294in}{4.176316in}}%
\pgfusepath{stroke}%
\end{pgfscope}%
\begin{pgfscope}%
\pgfpathrectangle{\pgfqpoint{2.000000in}{3.197368in}}{\pgfqpoint{4.376471in}{0.978947in}} %
\pgfusepath{clip}%
\pgfsetroundcap%
\pgfsetroundjoin%
\pgfsetlinewidth{1.003750pt}%
\definecolor{currentstroke}{rgb}{0.800000,0.800000,0.800000}%
\pgfsetstrokecolor{currentstroke}%
\pgfsetdash{}{0pt}%
\pgfpathmoveto{\pgfqpoint{3.750588in}{3.197368in}}%
\pgfpathlineto{\pgfqpoint{3.750588in}{4.176316in}}%
\pgfusepath{stroke}%
\end{pgfscope}%
\begin{pgfscope}%
\pgfpathrectangle{\pgfqpoint{2.000000in}{3.197368in}}{\pgfqpoint{4.376471in}{0.978947in}} %
\pgfusepath{clip}%
\pgfsetroundcap%
\pgfsetroundjoin%
\pgfsetlinewidth{1.003750pt}%
\definecolor{currentstroke}{rgb}{0.800000,0.800000,0.800000}%
\pgfsetstrokecolor{currentstroke}%
\pgfsetdash{}{0pt}%
\pgfpathmoveto{\pgfqpoint{4.625882in}{3.197368in}}%
\pgfpathlineto{\pgfqpoint{4.625882in}{4.176316in}}%
\pgfusepath{stroke}%
\end{pgfscope}%
\begin{pgfscope}%
\pgfpathrectangle{\pgfqpoint{2.000000in}{3.197368in}}{\pgfqpoint{4.376471in}{0.978947in}} %
\pgfusepath{clip}%
\pgfsetroundcap%
\pgfsetroundjoin%
\pgfsetlinewidth{1.003750pt}%
\definecolor{currentstroke}{rgb}{0.800000,0.800000,0.800000}%
\pgfsetstrokecolor{currentstroke}%
\pgfsetdash{}{0pt}%
\pgfpathmoveto{\pgfqpoint{5.501176in}{3.197368in}}%
\pgfpathlineto{\pgfqpoint{5.501176in}{4.176316in}}%
\pgfusepath{stroke}%
\end{pgfscope}%
\begin{pgfscope}%
\pgfpathrectangle{\pgfqpoint{2.000000in}{3.197368in}}{\pgfqpoint{4.376471in}{0.978947in}} %
\pgfusepath{clip}%
\pgfsetroundcap%
\pgfsetroundjoin%
\pgfsetlinewidth{1.003750pt}%
\definecolor{currentstroke}{rgb}{0.800000,0.800000,0.800000}%
\pgfsetstrokecolor{currentstroke}%
\pgfsetdash{}{0pt}%
\pgfpathmoveto{\pgfqpoint{6.376471in}{3.197368in}}%
\pgfpathlineto{\pgfqpoint{6.376471in}{4.176316in}}%
\pgfusepath{stroke}%
\end{pgfscope}%
\begin{pgfscope}%
\pgfpathrectangle{\pgfqpoint{2.000000in}{3.197368in}}{\pgfqpoint{4.376471in}{0.978947in}} %
\pgfusepath{clip}%
\pgfsetroundcap%
\pgfsetroundjoin%
\pgfsetlinewidth{1.003750pt}%
\definecolor{currentstroke}{rgb}{0.800000,0.800000,0.800000}%
\pgfsetstrokecolor{currentstroke}%
\pgfsetdash{}{0pt}%
\pgfpathmoveto{\pgfqpoint{2.000000in}{3.360526in}}%
\pgfpathlineto{\pgfqpoint{6.376471in}{3.360526in}}%
\pgfusepath{stroke}%
\end{pgfscope}%
\begin{pgfscope}%
\definecolor{textcolor}{rgb}{0.150000,0.150000,0.150000}%
\pgfsetstrokecolor{textcolor}%
\pgfsetfillcolor{textcolor}%
\pgftext[x=1.902778in,y=3.360526in,right,]{\color{textcolor}\sffamily\fontsize{10.000000}{12.000000}\selectfont \(\displaystyle -1\)}%
\end{pgfscope}%
\begin{pgfscope}%
\pgfpathrectangle{\pgfqpoint{2.000000in}{3.197368in}}{\pgfqpoint{4.376471in}{0.978947in}} %
\pgfusepath{clip}%
\pgfsetroundcap%
\pgfsetroundjoin%
\pgfsetlinewidth{1.003750pt}%
\definecolor{currentstroke}{rgb}{0.800000,0.800000,0.800000}%
\pgfsetstrokecolor{currentstroke}%
\pgfsetdash{}{0pt}%
\pgfpathmoveto{\pgfqpoint{2.000000in}{3.564474in}}%
\pgfpathlineto{\pgfqpoint{6.376471in}{3.564474in}}%
\pgfusepath{stroke}%
\end{pgfscope}%
\begin{pgfscope}%
\definecolor{textcolor}{rgb}{0.150000,0.150000,0.150000}%
\pgfsetstrokecolor{textcolor}%
\pgfsetfillcolor{textcolor}%
\pgftext[x=1.902778in,y=3.564474in,right,]{\color{textcolor}\sffamily\fontsize{10.000000}{12.000000}\selectfont \(\displaystyle 0\)}%
\end{pgfscope}%
\begin{pgfscope}%
\pgfpathrectangle{\pgfqpoint{2.000000in}{3.197368in}}{\pgfqpoint{4.376471in}{0.978947in}} %
\pgfusepath{clip}%
\pgfsetroundcap%
\pgfsetroundjoin%
\pgfsetlinewidth{1.003750pt}%
\definecolor{currentstroke}{rgb}{0.800000,0.800000,0.800000}%
\pgfsetstrokecolor{currentstroke}%
\pgfsetdash{}{0pt}%
\pgfpathmoveto{\pgfqpoint{2.000000in}{3.768421in}}%
\pgfpathlineto{\pgfqpoint{6.376471in}{3.768421in}}%
\pgfusepath{stroke}%
\end{pgfscope}%
\begin{pgfscope}%
\definecolor{textcolor}{rgb}{0.150000,0.150000,0.150000}%
\pgfsetstrokecolor{textcolor}%
\pgfsetfillcolor{textcolor}%
\pgftext[x=1.902778in,y=3.768421in,right,]{\color{textcolor}\sffamily\fontsize{10.000000}{12.000000}\selectfont \(\displaystyle 1\)}%
\end{pgfscope}%
\begin{pgfscope}%
\pgfpathrectangle{\pgfqpoint{2.000000in}{3.197368in}}{\pgfqpoint{4.376471in}{0.978947in}} %
\pgfusepath{clip}%
\pgfsetroundcap%
\pgfsetroundjoin%
\pgfsetlinewidth{1.003750pt}%
\definecolor{currentstroke}{rgb}{0.800000,0.800000,0.800000}%
\pgfsetstrokecolor{currentstroke}%
\pgfsetdash{}{0pt}%
\pgfpathmoveto{\pgfqpoint{2.000000in}{3.972368in}}%
\pgfpathlineto{\pgfqpoint{6.376471in}{3.972368in}}%
\pgfusepath{stroke}%
\end{pgfscope}%
\begin{pgfscope}%
\definecolor{textcolor}{rgb}{0.150000,0.150000,0.150000}%
\pgfsetstrokecolor{textcolor}%
\pgfsetfillcolor{textcolor}%
\pgftext[x=1.902778in,y=3.972368in,right,]{\color{textcolor}\sffamily\fontsize{10.000000}{12.000000}\selectfont \(\displaystyle 2\)}%
\end{pgfscope}%
\begin{pgfscope}%
\pgfpathrectangle{\pgfqpoint{2.000000in}{3.197368in}}{\pgfqpoint{4.376471in}{0.978947in}} %
\pgfusepath{clip}%
\pgfsetroundcap%
\pgfsetroundjoin%
\pgfsetlinewidth{1.003750pt}%
\definecolor{currentstroke}{rgb}{0.800000,0.800000,0.800000}%
\pgfsetstrokecolor{currentstroke}%
\pgfsetdash{}{0pt}%
\pgfpathmoveto{\pgfqpoint{2.000000in}{4.176316in}}%
\pgfpathlineto{\pgfqpoint{6.376471in}{4.176316in}}%
\pgfusepath{stroke}%
\end{pgfscope}%
\begin{pgfscope}%
\definecolor{textcolor}{rgb}{0.150000,0.150000,0.150000}%
\pgfsetstrokecolor{textcolor}%
\pgfsetfillcolor{textcolor}%
\pgftext[x=1.902778in,y=4.176316in,right,]{\color{textcolor}\sffamily\fontsize{10.000000}{12.000000}\selectfont \(\displaystyle 3\)}%
\end{pgfscope}%
\begin{pgfscope}%
\definecolor{textcolor}{rgb}{0.150000,0.150000,0.150000}%
\pgfsetstrokecolor{textcolor}%
\pgfsetfillcolor{textcolor}%
\pgftext[x=1.655864in,y=3.686842in,,bottom,rotate=90.000000]{\color{textcolor}\sffamily\fontsize{11.000000}{13.200000}\selectfont y}%
\end{pgfscope}%
\begin{pgfscope}%
\pgfpathrectangle{\pgfqpoint{2.000000in}{3.197368in}}{\pgfqpoint{4.376471in}{0.978947in}} %
\pgfusepath{clip}%
\pgfsetbuttcap%
\pgfsetroundjoin%
\definecolor{currentfill}{rgb}{1.000000,0.000000,0.000000}%
\pgfsetfillcolor{currentfill}%
\pgfsetlinewidth{2.007500pt}%
\definecolor{currentstroke}{rgb}{1.000000,0.000000,0.000000}%
\pgfsetstrokecolor{currentstroke}%
\pgfsetdash{}{0pt}%
\pgfpathmoveto{\pgfqpoint{4.765731in}{3.754944in}}%
\pgfpathlineto{\pgfqpoint{4.827844in}{3.754944in}}%
\pgfpathmoveto{\pgfqpoint{4.796787in}{3.723888in}}%
\pgfpathlineto{\pgfqpoint{4.796787in}{3.786001in}}%
\pgfusepath{stroke,fill}%
\end{pgfscope}%
\begin{pgfscope}%
\pgfpathrectangle{\pgfqpoint{2.000000in}{3.197368in}}{\pgfqpoint{4.376471in}{0.978947in}} %
\pgfusepath{clip}%
\pgfsetbuttcap%
\pgfsetroundjoin%
\definecolor{currentfill}{rgb}{1.000000,0.000000,0.000000}%
\pgfsetfillcolor{currentfill}%
\pgfsetlinewidth{2.007500pt}%
\definecolor{currentstroke}{rgb}{1.000000,0.000000,0.000000}%
\pgfsetstrokecolor{currentstroke}%
\pgfsetdash{}{0pt}%
\pgfpathmoveto{\pgfqpoint{5.348242in}{3.437732in}}%
\pgfpathlineto{\pgfqpoint{5.410355in}{3.437732in}}%
\pgfpathmoveto{\pgfqpoint{5.379298in}{3.406676in}}%
\pgfpathlineto{\pgfqpoint{5.379298in}{3.468789in}}%
\pgfusepath{stroke,fill}%
\end{pgfscope}%
\begin{pgfscope}%
\pgfpathrectangle{\pgfqpoint{2.000000in}{3.197368in}}{\pgfqpoint{4.376471in}{0.978947in}} %
\pgfusepath{clip}%
\pgfsetbuttcap%
\pgfsetroundjoin%
\definecolor{currentfill}{rgb}{1.000000,0.000000,0.000000}%
\pgfsetfillcolor{currentfill}%
\pgfsetlinewidth{2.007500pt}%
\definecolor{currentstroke}{rgb}{1.000000,0.000000,0.000000}%
\pgfsetstrokecolor{currentstroke}%
\pgfsetdash{}{0pt}%
\pgfpathmoveto{\pgfqpoint{4.954619in}{3.808472in}}%
\pgfpathlineto{\pgfqpoint{5.016732in}{3.808472in}}%
\pgfpathmoveto{\pgfqpoint{4.985675in}{3.777416in}}%
\pgfpathlineto{\pgfqpoint{4.985675in}{3.839529in}}%
\pgfusepath{stroke,fill}%
\end{pgfscope}%
\begin{pgfscope}%
\pgfpathrectangle{\pgfqpoint{2.000000in}{3.197368in}}{\pgfqpoint{4.376471in}{0.978947in}} %
\pgfusepath{clip}%
\pgfsetbuttcap%
\pgfsetroundjoin%
\definecolor{currentfill}{rgb}{1.000000,0.000000,0.000000}%
\pgfsetfillcolor{currentfill}%
\pgfsetlinewidth{2.007500pt}%
\definecolor{currentstroke}{rgb}{1.000000,0.000000,0.000000}%
\pgfsetstrokecolor{currentstroke}%
\pgfsetdash{}{0pt}%
\pgfpathmoveto{\pgfqpoint{4.751970in}{3.697560in}}%
\pgfpathlineto{\pgfqpoint{4.814083in}{3.697560in}}%
\pgfpathmoveto{\pgfqpoint{4.783026in}{3.666503in}}%
\pgfpathlineto{\pgfqpoint{4.783026in}{3.728616in}}%
\pgfusepath{stroke,fill}%
\end{pgfscope}%
\begin{pgfscope}%
\pgfpathrectangle{\pgfqpoint{2.000000in}{3.197368in}}{\pgfqpoint{4.376471in}{0.978947in}} %
\pgfusepath{clip}%
\pgfsetbuttcap%
\pgfsetroundjoin%
\definecolor{currentfill}{rgb}{1.000000,0.000000,0.000000}%
\pgfsetfillcolor{currentfill}%
\pgfsetlinewidth{2.007500pt}%
\definecolor{currentstroke}{rgb}{1.000000,0.000000,0.000000}%
\pgfsetstrokecolor{currentstroke}%
\pgfsetdash{}{0pt}%
\pgfpathmoveto{\pgfqpoint{4.327528in}{3.371166in}}%
\pgfpathlineto{\pgfqpoint{4.389641in}{3.371166in}}%
\pgfpathmoveto{\pgfqpoint{4.358584in}{3.340109in}}%
\pgfpathlineto{\pgfqpoint{4.358584in}{3.402222in}}%
\pgfusepath{stroke,fill}%
\end{pgfscope}%
\begin{pgfscope}%
\pgfpathrectangle{\pgfqpoint{2.000000in}{3.197368in}}{\pgfqpoint{4.376471in}{0.978947in}} %
\pgfusepath{clip}%
\pgfsetbuttcap%
\pgfsetroundjoin%
\definecolor{currentfill}{rgb}{1.000000,0.000000,0.000000}%
\pgfsetfillcolor{currentfill}%
\pgfsetlinewidth{2.007500pt}%
\definecolor{currentstroke}{rgb}{1.000000,0.000000,0.000000}%
\pgfsetstrokecolor{currentstroke}%
\pgfsetdash{}{0pt}%
\pgfpathmoveto{\pgfqpoint{5.105627in}{3.640630in}}%
\pgfpathlineto{\pgfqpoint{5.167740in}{3.640630in}}%
\pgfpathmoveto{\pgfqpoint{5.136683in}{3.609573in}}%
\pgfpathlineto{\pgfqpoint{5.136683in}{3.671686in}}%
\pgfusepath{stroke,fill}%
\end{pgfscope}%
\begin{pgfscope}%
\pgfpathrectangle{\pgfqpoint{2.000000in}{3.197368in}}{\pgfqpoint{4.376471in}{0.978947in}} %
\pgfusepath{clip}%
\pgfsetbuttcap%
\pgfsetroundjoin%
\definecolor{currentfill}{rgb}{1.000000,0.000000,0.000000}%
\pgfsetfillcolor{currentfill}%
\pgfsetlinewidth{2.007500pt}%
\definecolor{currentstroke}{rgb}{1.000000,0.000000,0.000000}%
\pgfsetstrokecolor{currentstroke}%
\pgfsetdash{}{0pt}%
\pgfpathmoveto{\pgfqpoint{4.376308in}{3.408659in}}%
\pgfpathlineto{\pgfqpoint{4.438421in}{3.408659in}}%
\pgfpathmoveto{\pgfqpoint{4.407364in}{3.377603in}}%
\pgfpathlineto{\pgfqpoint{4.407364in}{3.439716in}}%
\pgfusepath{stroke,fill}%
\end{pgfscope}%
\begin{pgfscope}%
\pgfpathrectangle{\pgfqpoint{2.000000in}{3.197368in}}{\pgfqpoint{4.376471in}{0.978947in}} %
\pgfusepath{clip}%
\pgfsetbuttcap%
\pgfsetroundjoin%
\definecolor{currentfill}{rgb}{1.000000,0.000000,0.000000}%
\pgfsetfillcolor{currentfill}%
\pgfsetlinewidth{2.007500pt}%
\definecolor{currentstroke}{rgb}{1.000000,0.000000,0.000000}%
\pgfsetstrokecolor{currentstroke}%
\pgfsetdash{}{0pt}%
\pgfpathmoveto{\pgfqpoint{5.966492in}{4.048778in}}%
\pgfpathlineto{\pgfqpoint{6.028605in}{4.048778in}}%
\pgfpathmoveto{\pgfqpoint{5.997549in}{4.017721in}}%
\pgfpathlineto{\pgfqpoint{5.997549in}{4.079834in}}%
\pgfusepath{stroke,fill}%
\end{pgfscope}%
\begin{pgfscope}%
\pgfpathrectangle{\pgfqpoint{2.000000in}{3.197368in}}{\pgfqpoint{4.376471in}{0.978947in}} %
\pgfusepath{clip}%
\pgfsetbuttcap%
\pgfsetroundjoin%
\definecolor{currentfill}{rgb}{1.000000,0.000000,0.000000}%
\pgfsetfillcolor{currentfill}%
\pgfsetlinewidth{2.007500pt}%
\definecolor{currentstroke}{rgb}{1.000000,0.000000,0.000000}%
\pgfsetstrokecolor{currentstroke}%
\pgfsetdash{}{0pt}%
\pgfpathmoveto{\pgfqpoint{6.218191in}{3.949180in}}%
\pgfpathlineto{\pgfqpoint{6.280304in}{3.949180in}}%
\pgfpathmoveto{\pgfqpoint{6.249248in}{3.918124in}}%
\pgfpathlineto{\pgfqpoint{6.249248in}{3.980237in}}%
\pgfusepath{stroke,fill}%
\end{pgfscope}%
\begin{pgfscope}%
\pgfpathrectangle{\pgfqpoint{2.000000in}{3.197368in}}{\pgfqpoint{4.376471in}{0.978947in}} %
\pgfusepath{clip}%
\pgfsetbuttcap%
\pgfsetroundjoin%
\definecolor{currentfill}{rgb}{1.000000,0.000000,0.000000}%
\pgfsetfillcolor{currentfill}%
\pgfsetlinewidth{2.007500pt}%
\definecolor{currentstroke}{rgb}{1.000000,0.000000,0.000000}%
\pgfsetstrokecolor{currentstroke}%
\pgfsetdash{}{0pt}%
\pgfpathmoveto{\pgfqpoint{4.186734in}{3.446577in}}%
\pgfpathlineto{\pgfqpoint{4.248847in}{3.446577in}}%
\pgfpathmoveto{\pgfqpoint{4.217791in}{3.415520in}}%
\pgfpathlineto{\pgfqpoint{4.217791in}{3.477633in}}%
\pgfusepath{stroke,fill}%
\end{pgfscope}%
\begin{pgfscope}%
\pgfpathrectangle{\pgfqpoint{2.000000in}{3.197368in}}{\pgfqpoint{4.376471in}{0.978947in}} %
\pgfusepath{clip}%
\pgfsetbuttcap%
\pgfsetroundjoin%
\definecolor{currentfill}{rgb}{1.000000,0.000000,0.000000}%
\pgfsetfillcolor{currentfill}%
\pgfsetlinewidth{2.007500pt}%
\definecolor{currentstroke}{rgb}{1.000000,0.000000,0.000000}%
\pgfsetstrokecolor{currentstroke}%
\pgfsetdash{}{0pt}%
\pgfpathmoveto{\pgfqpoint{5.616207in}{3.596752in}}%
\pgfpathlineto{\pgfqpoint{5.678320in}{3.596752in}}%
\pgfpathmoveto{\pgfqpoint{5.647263in}{3.565696in}}%
\pgfpathlineto{\pgfqpoint{5.647263in}{3.627809in}}%
\pgfusepath{stroke,fill}%
\end{pgfscope}%
\begin{pgfscope}%
\pgfpathrectangle{\pgfqpoint{2.000000in}{3.197368in}}{\pgfqpoint{4.376471in}{0.978947in}} %
\pgfusepath{clip}%
\pgfsetbuttcap%
\pgfsetroundjoin%
\definecolor{currentfill}{rgb}{1.000000,0.000000,0.000000}%
\pgfsetfillcolor{currentfill}%
\pgfsetlinewidth{2.007500pt}%
\definecolor{currentstroke}{rgb}{1.000000,0.000000,0.000000}%
\pgfsetstrokecolor{currentstroke}%
\pgfsetdash{}{0pt}%
\pgfpathmoveto{\pgfqpoint{4.695992in}{3.636847in}}%
\pgfpathlineto{\pgfqpoint{4.758105in}{3.636847in}}%
\pgfpathmoveto{\pgfqpoint{4.727049in}{3.605791in}}%
\pgfpathlineto{\pgfqpoint{4.727049in}{3.667904in}}%
\pgfusepath{stroke,fill}%
\end{pgfscope}%
\begin{pgfscope}%
\pgfpathrectangle{\pgfqpoint{2.000000in}{3.197368in}}{\pgfqpoint{4.376471in}{0.978947in}} %
\pgfusepath{clip}%
\pgfsetbuttcap%
\pgfsetroundjoin%
\definecolor{currentfill}{rgb}{1.000000,0.000000,0.000000}%
\pgfsetfillcolor{currentfill}%
\pgfsetlinewidth{2.007500pt}%
\definecolor{currentstroke}{rgb}{1.000000,0.000000,0.000000}%
\pgfsetstrokecolor{currentstroke}%
\pgfsetdash{}{0pt}%
\pgfpathmoveto{\pgfqpoint{4.833062in}{3.764403in}}%
\pgfpathlineto{\pgfqpoint{4.895175in}{3.764403in}}%
\pgfpathmoveto{\pgfqpoint{4.864118in}{3.733347in}}%
\pgfpathlineto{\pgfqpoint{4.864118in}{3.795460in}}%
\pgfusepath{stroke,fill}%
\end{pgfscope}%
\begin{pgfscope}%
\pgfpathrectangle{\pgfqpoint{2.000000in}{3.197368in}}{\pgfqpoint{4.376471in}{0.978947in}} %
\pgfusepath{clip}%
\pgfsetbuttcap%
\pgfsetroundjoin%
\definecolor{currentfill}{rgb}{1.000000,0.000000,0.000000}%
\pgfsetfillcolor{currentfill}%
\pgfsetlinewidth{2.007500pt}%
\definecolor{currentstroke}{rgb}{1.000000,0.000000,0.000000}%
\pgfsetstrokecolor{currentstroke}%
\pgfsetdash{}{0pt}%
\pgfpathmoveto{\pgfqpoint{6.084915in}{4.024567in}}%
\pgfpathlineto{\pgfqpoint{6.147028in}{4.024567in}}%
\pgfpathmoveto{\pgfqpoint{6.115971in}{3.993511in}}%
\pgfpathlineto{\pgfqpoint{6.115971in}{4.055624in}}%
\pgfusepath{stroke,fill}%
\end{pgfscope}%
\begin{pgfscope}%
\pgfpathrectangle{\pgfqpoint{2.000000in}{3.197368in}}{\pgfqpoint{4.376471in}{0.978947in}} %
\pgfusepath{clip}%
\pgfsetbuttcap%
\pgfsetroundjoin%
\definecolor{currentfill}{rgb}{1.000000,0.000000,0.000000}%
\pgfsetfillcolor{currentfill}%
\pgfsetlinewidth{2.007500pt}%
\definecolor{currentstroke}{rgb}{1.000000,0.000000,0.000000}%
\pgfsetstrokecolor{currentstroke}%
\pgfsetdash{}{0pt}%
\pgfpathmoveto{\pgfqpoint{3.092947in}{3.737178in}}%
\pgfpathlineto{\pgfqpoint{3.155060in}{3.737178in}}%
\pgfpathmoveto{\pgfqpoint{3.124004in}{3.706121in}}%
\pgfpathlineto{\pgfqpoint{3.124004in}{3.768234in}}%
\pgfusepath{stroke,fill}%
\end{pgfscope}%
\begin{pgfscope}%
\pgfpathrectangle{\pgfqpoint{2.000000in}{3.197368in}}{\pgfqpoint{4.376471in}{0.978947in}} %
\pgfusepath{clip}%
\pgfsetbuttcap%
\pgfsetroundjoin%
\definecolor{currentfill}{rgb}{1.000000,0.000000,0.000000}%
\pgfsetfillcolor{currentfill}%
\pgfsetlinewidth{2.007500pt}%
\definecolor{currentstroke}{rgb}{1.000000,0.000000,0.000000}%
\pgfsetstrokecolor{currentstroke}%
\pgfsetdash{}{0pt}%
\pgfpathmoveto{\pgfqpoint{3.149293in}{3.679544in}}%
\pgfpathlineto{\pgfqpoint{3.211406in}{3.679544in}}%
\pgfpathmoveto{\pgfqpoint{3.180349in}{3.648488in}}%
\pgfpathlineto{\pgfqpoint{3.180349in}{3.710601in}}%
\pgfusepath{stroke,fill}%
\end{pgfscope}%
\begin{pgfscope}%
\pgfpathrectangle{\pgfqpoint{2.000000in}{3.197368in}}{\pgfqpoint{4.376471in}{0.978947in}} %
\pgfusepath{clip}%
\pgfsetbuttcap%
\pgfsetroundjoin%
\definecolor{currentfill}{rgb}{1.000000,0.000000,0.000000}%
\pgfsetfillcolor{currentfill}%
\pgfsetlinewidth{2.007500pt}%
\definecolor{currentstroke}{rgb}{1.000000,0.000000,0.000000}%
\pgfsetstrokecolor{currentstroke}%
\pgfsetdash{}{0pt}%
\pgfpathmoveto{\pgfqpoint{2.915026in}{3.966824in}}%
\pgfpathlineto{\pgfqpoint{2.977139in}{3.966824in}}%
\pgfpathmoveto{\pgfqpoint{2.946082in}{3.935768in}}%
\pgfpathlineto{\pgfqpoint{2.946082in}{3.997881in}}%
\pgfusepath{stroke,fill}%
\end{pgfscope}%
\begin{pgfscope}%
\pgfpathrectangle{\pgfqpoint{2.000000in}{3.197368in}}{\pgfqpoint{4.376471in}{0.978947in}} %
\pgfusepath{clip}%
\pgfsetbuttcap%
\pgfsetroundjoin%
\definecolor{currentfill}{rgb}{1.000000,0.000000,0.000000}%
\pgfsetfillcolor{currentfill}%
\pgfsetlinewidth{2.007500pt}%
\definecolor{currentstroke}{rgb}{1.000000,0.000000,0.000000}%
\pgfsetstrokecolor{currentstroke}%
\pgfsetdash{}{0pt}%
\pgfpathmoveto{\pgfqpoint{5.759387in}{3.812420in}}%
\pgfpathlineto{\pgfqpoint{5.821500in}{3.812420in}}%
\pgfpathmoveto{\pgfqpoint{5.790443in}{3.781364in}}%
\pgfpathlineto{\pgfqpoint{5.790443in}{3.843477in}}%
\pgfusepath{stroke,fill}%
\end{pgfscope}%
\begin{pgfscope}%
\pgfpathrectangle{\pgfqpoint{2.000000in}{3.197368in}}{\pgfqpoint{4.376471in}{0.978947in}} %
\pgfusepath{clip}%
\pgfsetbuttcap%
\pgfsetroundjoin%
\definecolor{currentfill}{rgb}{1.000000,0.000000,0.000000}%
\pgfsetfillcolor{currentfill}%
\pgfsetlinewidth{2.007500pt}%
\definecolor{currentstroke}{rgb}{1.000000,0.000000,0.000000}%
\pgfsetstrokecolor{currentstroke}%
\pgfsetdash{}{0pt}%
\pgfpathmoveto{\pgfqpoint{5.568702in}{3.534838in}}%
\pgfpathlineto{\pgfqpoint{5.630815in}{3.534838in}}%
\pgfpathmoveto{\pgfqpoint{5.599758in}{3.503782in}}%
\pgfpathlineto{\pgfqpoint{5.599758in}{3.565895in}}%
\pgfusepath{stroke,fill}%
\end{pgfscope}%
\begin{pgfscope}%
\pgfpathrectangle{\pgfqpoint{2.000000in}{3.197368in}}{\pgfqpoint{4.376471in}{0.978947in}} %
\pgfusepath{clip}%
\pgfsetbuttcap%
\pgfsetroundjoin%
\definecolor{currentfill}{rgb}{1.000000,0.000000,0.000000}%
\pgfsetfillcolor{currentfill}%
\pgfsetlinewidth{2.007500pt}%
\definecolor{currentstroke}{rgb}{1.000000,0.000000,0.000000}%
\pgfsetstrokecolor{currentstroke}%
\pgfsetdash{}{0pt}%
\pgfpathmoveto{\pgfqpoint{5.890304in}{3.942201in}}%
\pgfpathlineto{\pgfqpoint{5.952417in}{3.942201in}}%
\pgfpathmoveto{\pgfqpoint{5.921360in}{3.911145in}}%
\pgfpathlineto{\pgfqpoint{5.921360in}{3.973258in}}%
\pgfusepath{stroke,fill}%
\end{pgfscope}%
\begin{pgfscope}%
\pgfpathrectangle{\pgfqpoint{2.000000in}{3.197368in}}{\pgfqpoint{4.376471in}{0.978947in}} %
\pgfusepath{clip}%
\pgfsetbuttcap%
\pgfsetroundjoin%
\definecolor{currentfill}{rgb}{1.000000,0.000000,0.000000}%
\pgfsetfillcolor{currentfill}%
\pgfsetlinewidth{2.007500pt}%
\definecolor{currentstroke}{rgb}{1.000000,0.000000,0.000000}%
\pgfsetstrokecolor{currentstroke}%
\pgfsetdash{}{0pt}%
\pgfpathmoveto{\pgfqpoint{6.270553in}{3.873438in}}%
\pgfpathlineto{\pgfqpoint{6.332666in}{3.873438in}}%
\pgfpathmoveto{\pgfqpoint{6.301610in}{3.842382in}}%
\pgfpathlineto{\pgfqpoint{6.301610in}{3.904495in}}%
\pgfusepath{stroke,fill}%
\end{pgfscope}%
\begin{pgfscope}%
\pgfpathrectangle{\pgfqpoint{2.000000in}{3.197368in}}{\pgfqpoint{4.376471in}{0.978947in}} %
\pgfusepath{clip}%
\pgfsetbuttcap%
\pgfsetroundjoin%
\definecolor{currentfill}{rgb}{1.000000,0.000000,0.000000}%
\pgfsetfillcolor{currentfill}%
\pgfsetlinewidth{2.007500pt}%
\definecolor{currentstroke}{rgb}{1.000000,0.000000,0.000000}%
\pgfsetstrokecolor{currentstroke}%
\pgfsetdash{}{0pt}%
\pgfpathmoveto{\pgfqpoint{5.642233in}{3.690005in}}%
\pgfpathlineto{\pgfqpoint{5.704346in}{3.690005in}}%
\pgfpathmoveto{\pgfqpoint{5.673289in}{3.658949in}}%
\pgfpathlineto{\pgfqpoint{5.673289in}{3.721062in}}%
\pgfusepath{stroke,fill}%
\end{pgfscope}%
\begin{pgfscope}%
\pgfpathrectangle{\pgfqpoint{2.000000in}{3.197368in}}{\pgfqpoint{4.376471in}{0.978947in}} %
\pgfusepath{clip}%
\pgfsetbuttcap%
\pgfsetroundjoin%
\definecolor{currentfill}{rgb}{1.000000,0.000000,0.000000}%
\pgfsetfillcolor{currentfill}%
\pgfsetlinewidth{2.007500pt}%
\definecolor{currentstroke}{rgb}{1.000000,0.000000,0.000000}%
\pgfsetstrokecolor{currentstroke}%
\pgfsetdash{}{0pt}%
\pgfpathmoveto{\pgfqpoint{4.459958in}{3.414465in}}%
\pgfpathlineto{\pgfqpoint{4.522071in}{3.414465in}}%
\pgfpathmoveto{\pgfqpoint{4.491015in}{3.383409in}}%
\pgfpathlineto{\pgfqpoint{4.491015in}{3.445522in}}%
\pgfusepath{stroke,fill}%
\end{pgfscope}%
\begin{pgfscope}%
\pgfpathrectangle{\pgfqpoint{2.000000in}{3.197368in}}{\pgfqpoint{4.376471in}{0.978947in}} %
\pgfusepath{clip}%
\pgfsetbuttcap%
\pgfsetroundjoin%
\definecolor{currentfill}{rgb}{1.000000,0.000000,0.000000}%
\pgfsetfillcolor{currentfill}%
\pgfsetlinewidth{2.007500pt}%
\definecolor{currentstroke}{rgb}{1.000000,0.000000,0.000000}%
\pgfsetstrokecolor{currentstroke}%
\pgfsetdash{}{0pt}%
\pgfpathmoveto{\pgfqpoint{5.577008in}{3.556875in}}%
\pgfpathlineto{\pgfqpoint{5.639121in}{3.556875in}}%
\pgfpathmoveto{\pgfqpoint{5.608065in}{3.525819in}}%
\pgfpathlineto{\pgfqpoint{5.608065in}{3.587932in}}%
\pgfusepath{stroke,fill}%
\end{pgfscope}%
\begin{pgfscope}%
\pgfpathrectangle{\pgfqpoint{2.000000in}{3.197368in}}{\pgfqpoint{4.376471in}{0.978947in}} %
\pgfusepath{clip}%
\pgfsetbuttcap%
\pgfsetroundjoin%
\definecolor{currentfill}{rgb}{1.000000,0.000000,0.000000}%
\pgfsetfillcolor{currentfill}%
\pgfsetlinewidth{2.007500pt}%
\definecolor{currentstroke}{rgb}{1.000000,0.000000,0.000000}%
\pgfsetstrokecolor{currentstroke}%
\pgfsetdash{}{0pt}%
\pgfpathmoveto{\pgfqpoint{3.258337in}{3.577613in}}%
\pgfpathlineto{\pgfqpoint{3.320450in}{3.577613in}}%
\pgfpathmoveto{\pgfqpoint{3.289394in}{3.546556in}}%
\pgfpathlineto{\pgfqpoint{3.289394in}{3.608669in}}%
\pgfusepath{stroke,fill}%
\end{pgfscope}%
\begin{pgfscope}%
\pgfpathrectangle{\pgfqpoint{2.000000in}{3.197368in}}{\pgfqpoint{4.376471in}{0.978947in}} %
\pgfusepath{clip}%
\pgfsetbuttcap%
\pgfsetroundjoin%
\definecolor{currentfill}{rgb}{1.000000,0.000000,0.000000}%
\pgfsetfillcolor{currentfill}%
\pgfsetlinewidth{2.007500pt}%
\definecolor{currentstroke}{rgb}{1.000000,0.000000,0.000000}%
\pgfsetstrokecolor{currentstroke}%
\pgfsetdash{}{0pt}%
\pgfpathmoveto{\pgfqpoint{5.084714in}{3.680897in}}%
\pgfpathlineto{\pgfqpoint{5.146827in}{3.680897in}}%
\pgfpathmoveto{\pgfqpoint{5.115771in}{3.649840in}}%
\pgfpathlineto{\pgfqpoint{5.115771in}{3.711953in}}%
\pgfusepath{stroke,fill}%
\end{pgfscope}%
\begin{pgfscope}%
\pgfpathrectangle{\pgfqpoint{2.000000in}{3.197368in}}{\pgfqpoint{4.376471in}{0.978947in}} %
\pgfusepath{clip}%
\pgfsetbuttcap%
\pgfsetroundjoin%
\definecolor{currentfill}{rgb}{1.000000,0.000000,0.000000}%
\pgfsetfillcolor{currentfill}%
\pgfsetlinewidth{2.007500pt}%
\definecolor{currentstroke}{rgb}{1.000000,0.000000,0.000000}%
\pgfsetstrokecolor{currentstroke}%
\pgfsetdash{}{0pt}%
\pgfpathmoveto{\pgfqpoint{3.346143in}{3.585903in}}%
\pgfpathlineto{\pgfqpoint{3.408256in}{3.585903in}}%
\pgfpathmoveto{\pgfqpoint{3.377199in}{3.554846in}}%
\pgfpathlineto{\pgfqpoint{3.377199in}{3.616959in}}%
\pgfusepath{stroke,fill}%
\end{pgfscope}%
\begin{pgfscope}%
\pgfpathrectangle{\pgfqpoint{2.000000in}{3.197368in}}{\pgfqpoint{4.376471in}{0.978947in}} %
\pgfusepath{clip}%
\pgfsetbuttcap%
\pgfsetroundjoin%
\definecolor{currentfill}{rgb}{1.000000,0.000000,0.000000}%
\pgfsetfillcolor{currentfill}%
\pgfsetlinewidth{2.007500pt}%
\definecolor{currentstroke}{rgb}{1.000000,0.000000,0.000000}%
\pgfsetstrokecolor{currentstroke}%
\pgfsetdash{}{0pt}%
\pgfpathmoveto{\pgfqpoint{6.151690in}{3.986096in}}%
\pgfpathlineto{\pgfqpoint{6.213803in}{3.986096in}}%
\pgfpathmoveto{\pgfqpoint{6.182747in}{3.955040in}}%
\pgfpathlineto{\pgfqpoint{6.182747in}{4.017153in}}%
\pgfusepath{stroke,fill}%
\end{pgfscope}%
\begin{pgfscope}%
\pgfpathrectangle{\pgfqpoint{2.000000in}{3.197368in}}{\pgfqpoint{4.376471in}{0.978947in}} %
\pgfusepath{clip}%
\pgfsetbuttcap%
\pgfsetroundjoin%
\definecolor{currentfill}{rgb}{1.000000,0.000000,0.000000}%
\pgfsetfillcolor{currentfill}%
\pgfsetlinewidth{2.007500pt}%
\definecolor{currentstroke}{rgb}{1.000000,0.000000,0.000000}%
\pgfsetstrokecolor{currentstroke}%
\pgfsetdash{}{0pt}%
\pgfpathmoveto{\pgfqpoint{4.671321in}{3.633299in}}%
\pgfpathlineto{\pgfqpoint{4.733434in}{3.633299in}}%
\pgfpathmoveto{\pgfqpoint{4.702377in}{3.602242in}}%
\pgfpathlineto{\pgfqpoint{4.702377in}{3.664355in}}%
\pgfusepath{stroke,fill}%
\end{pgfscope}%
\begin{pgfscope}%
\pgfpathrectangle{\pgfqpoint{2.000000in}{3.197368in}}{\pgfqpoint{4.376471in}{0.978947in}} %
\pgfusepath{clip}%
\pgfsetbuttcap%
\pgfsetroundjoin%
\definecolor{currentfill}{rgb}{1.000000,0.000000,0.000000}%
\pgfsetfillcolor{currentfill}%
\pgfsetlinewidth{2.007500pt}%
\definecolor{currentstroke}{rgb}{1.000000,0.000000,0.000000}%
\pgfsetstrokecolor{currentstroke}%
\pgfsetdash{}{0pt}%
\pgfpathmoveto{\pgfqpoint{4.296042in}{3.382181in}}%
\pgfpathlineto{\pgfqpoint{4.358155in}{3.382181in}}%
\pgfpathmoveto{\pgfqpoint{4.327099in}{3.351125in}}%
\pgfpathlineto{\pgfqpoint{4.327099in}{3.413238in}}%
\pgfusepath{stroke,fill}%
\end{pgfscope}%
\begin{pgfscope}%
\pgfpathrectangle{\pgfqpoint{2.000000in}{3.197368in}}{\pgfqpoint{4.376471in}{0.978947in}} %
\pgfusepath{clip}%
\pgfsetbuttcap%
\pgfsetroundjoin%
\definecolor{currentfill}{rgb}{0.000000,0.000000,0.000000}%
\pgfsetfillcolor{currentfill}%
\pgfsetlinewidth{0.301125pt}%
\definecolor{currentstroke}{rgb}{0.000000,0.000000,0.000000}%
\pgfsetstrokecolor{currentstroke}%
\pgfsetdash{}{0pt}%
\pgfsys@defobject{currentmarker}{\pgfqpoint{-0.015528in}{-0.015528in}}{\pgfqpoint{0.015528in}{0.015528in}}{%
\pgfpathmoveto{\pgfqpoint{0.000000in}{-0.015528in}}%
\pgfpathcurveto{\pgfqpoint{0.004118in}{-0.015528in}}{\pgfqpoint{0.008068in}{-0.013892in}}{\pgfqpoint{0.010980in}{-0.010980in}}%
\pgfpathcurveto{\pgfqpoint{0.013892in}{-0.008068in}}{\pgfqpoint{0.015528in}{-0.004118in}}{\pgfqpoint{0.015528in}{0.000000in}}%
\pgfpathcurveto{\pgfqpoint{0.015528in}{0.004118in}}{\pgfqpoint{0.013892in}{0.008068in}}{\pgfqpoint{0.010980in}{0.010980in}}%
\pgfpathcurveto{\pgfqpoint{0.008068in}{0.013892in}}{\pgfqpoint{0.004118in}{0.015528in}}{\pgfqpoint{0.000000in}{0.015528in}}%
\pgfpathcurveto{\pgfqpoint{-0.004118in}{0.015528in}}{\pgfqpoint{-0.008068in}{0.013892in}}{\pgfqpoint{-0.010980in}{0.010980in}}%
\pgfpathcurveto{\pgfqpoint{-0.013892in}{0.008068in}}{\pgfqpoint{-0.015528in}{0.004118in}}{\pgfqpoint{-0.015528in}{0.000000in}}%
\pgfpathcurveto{\pgfqpoint{-0.015528in}{-0.004118in}}{\pgfqpoint{-0.013892in}{-0.008068in}}{\pgfqpoint{-0.010980in}{-0.010980in}}%
\pgfpathcurveto{\pgfqpoint{-0.008068in}{-0.013892in}}{\pgfqpoint{-0.004118in}{-0.015528in}}{\pgfqpoint{0.000000in}{-0.015528in}}%
\pgfpathclose%
\pgfusepath{stroke,fill}%
}%
\begin{pgfscope}%
\pgfsys@transformshift{2.875294in}{4.031231in}%
\pgfsys@useobject{currentmarker}{}%
\end{pgfscope}%
\begin{pgfscope}%
\pgfsys@transformshift{2.892888in}{3.937039in}%
\pgfsys@useobject{currentmarker}{}%
\end{pgfscope}%
\begin{pgfscope}%
\pgfsys@transformshift{2.910482in}{4.027813in}%
\pgfsys@useobject{currentmarker}{}%
\end{pgfscope}%
\begin{pgfscope}%
\pgfsys@transformshift{2.928076in}{4.046950in}%
\pgfsys@useobject{currentmarker}{}%
\end{pgfscope}%
\begin{pgfscope}%
\pgfsys@transformshift{2.945670in}{3.982165in}%
\pgfsys@useobject{currentmarker}{}%
\end{pgfscope}%
\begin{pgfscope}%
\pgfsys@transformshift{2.963263in}{3.978064in}%
\pgfsys@useobject{currentmarker}{}%
\end{pgfscope}%
\begin{pgfscope}%
\pgfsys@transformshift{2.980857in}{3.854307in}%
\pgfsys@useobject{currentmarker}{}%
\end{pgfscope}%
\begin{pgfscope}%
\pgfsys@transformshift{2.998451in}{3.853911in}%
\pgfsys@useobject{currentmarker}{}%
\end{pgfscope}%
\begin{pgfscope}%
\pgfsys@transformshift{3.016045in}{3.794584in}%
\pgfsys@useobject{currentmarker}{}%
\end{pgfscope}%
\begin{pgfscope}%
\pgfsys@transformshift{3.033639in}{3.799298in}%
\pgfsys@useobject{currentmarker}{}%
\end{pgfscope}%
\begin{pgfscope}%
\pgfsys@transformshift{3.051233in}{3.726597in}%
\pgfsys@useobject{currentmarker}{}%
\end{pgfscope}%
\begin{pgfscope}%
\pgfsys@transformshift{3.068826in}{3.607975in}%
\pgfsys@useobject{currentmarker}{}%
\end{pgfscope}%
\begin{pgfscope}%
\pgfsys@transformshift{3.086420in}{3.777720in}%
\pgfsys@useobject{currentmarker}{}%
\end{pgfscope}%
\begin{pgfscope}%
\pgfsys@transformshift{3.104014in}{3.695569in}%
\pgfsys@useobject{currentmarker}{}%
\end{pgfscope}%
\begin{pgfscope}%
\pgfsys@transformshift{3.121608in}{3.548673in}%
\pgfsys@useobject{currentmarker}{}%
\end{pgfscope}%
\begin{pgfscope}%
\pgfsys@transformshift{3.139202in}{3.742077in}%
\pgfsys@useobject{currentmarker}{}%
\end{pgfscope}%
\begin{pgfscope}%
\pgfsys@transformshift{3.156796in}{3.584074in}%
\pgfsys@useobject{currentmarker}{}%
\end{pgfscope}%
\begin{pgfscope}%
\pgfsys@transformshift{3.174390in}{3.665451in}%
\pgfsys@useobject{currentmarker}{}%
\end{pgfscope}%
\begin{pgfscope}%
\pgfsys@transformshift{3.191983in}{3.720008in}%
\pgfsys@useobject{currentmarker}{}%
\end{pgfscope}%
\begin{pgfscope}%
\pgfsys@transformshift{3.209577in}{3.646344in}%
\pgfsys@useobject{currentmarker}{}%
\end{pgfscope}%
\begin{pgfscope}%
\pgfsys@transformshift{3.227171in}{3.738995in}%
\pgfsys@useobject{currentmarker}{}%
\end{pgfscope}%
\begin{pgfscope}%
\pgfsys@transformshift{3.244765in}{3.488622in}%
\pgfsys@useobject{currentmarker}{}%
\end{pgfscope}%
\begin{pgfscope}%
\pgfsys@transformshift{3.262359in}{3.649446in}%
\pgfsys@useobject{currentmarker}{}%
\end{pgfscope}%
\begin{pgfscope}%
\pgfsys@transformshift{3.279953in}{3.534597in}%
\pgfsys@useobject{currentmarker}{}%
\end{pgfscope}%
\begin{pgfscope}%
\pgfsys@transformshift{3.297547in}{3.513756in}%
\pgfsys@useobject{currentmarker}{}%
\end{pgfscope}%
\begin{pgfscope}%
\pgfsys@transformshift{3.315140in}{3.543720in}%
\pgfsys@useobject{currentmarker}{}%
\end{pgfscope}%
\begin{pgfscope}%
\pgfsys@transformshift{3.332734in}{3.573174in}%
\pgfsys@useobject{currentmarker}{}%
\end{pgfscope}%
\begin{pgfscope}%
\pgfsys@transformshift{3.350328in}{3.614788in}%
\pgfsys@useobject{currentmarker}{}%
\end{pgfscope}%
\begin{pgfscope}%
\pgfsys@transformshift{3.367922in}{3.496181in}%
\pgfsys@useobject{currentmarker}{}%
\end{pgfscope}%
\begin{pgfscope}%
\pgfsys@transformshift{3.385516in}{3.714490in}%
\pgfsys@useobject{currentmarker}{}%
\end{pgfscope}%
\begin{pgfscope}%
\pgfsys@transformshift{3.403110in}{3.679310in}%
\pgfsys@useobject{currentmarker}{}%
\end{pgfscope}%
\begin{pgfscope}%
\pgfsys@transformshift{3.420704in}{3.485775in}%
\pgfsys@useobject{currentmarker}{}%
\end{pgfscope}%
\begin{pgfscope}%
\pgfsys@transformshift{3.438297in}{3.806047in}%
\pgfsys@useobject{currentmarker}{}%
\end{pgfscope}%
\begin{pgfscope}%
\pgfsys@transformshift{3.455891in}{3.860539in}%
\pgfsys@useobject{currentmarker}{}%
\end{pgfscope}%
\begin{pgfscope}%
\pgfsys@transformshift{3.473485in}{3.801221in}%
\pgfsys@useobject{currentmarker}{}%
\end{pgfscope}%
\begin{pgfscope}%
\pgfsys@transformshift{3.491079in}{3.677167in}%
\pgfsys@useobject{currentmarker}{}%
\end{pgfscope}%
\begin{pgfscope}%
\pgfsys@transformshift{3.508673in}{3.601314in}%
\pgfsys@useobject{currentmarker}{}%
\end{pgfscope}%
\begin{pgfscope}%
\pgfsys@transformshift{3.526267in}{3.833302in}%
\pgfsys@useobject{currentmarker}{}%
\end{pgfscope}%
\begin{pgfscope}%
\pgfsys@transformshift{3.543860in}{3.700014in}%
\pgfsys@useobject{currentmarker}{}%
\end{pgfscope}%
\begin{pgfscope}%
\pgfsys@transformshift{3.561454in}{3.881009in}%
\pgfsys@useobject{currentmarker}{}%
\end{pgfscope}%
\begin{pgfscope}%
\pgfsys@transformshift{3.579048in}{3.792483in}%
\pgfsys@useobject{currentmarker}{}%
\end{pgfscope}%
\begin{pgfscope}%
\pgfsys@transformshift{3.596642in}{3.885195in}%
\pgfsys@useobject{currentmarker}{}%
\end{pgfscope}%
\begin{pgfscope}%
\pgfsys@transformshift{3.614236in}{3.835577in}%
\pgfsys@useobject{currentmarker}{}%
\end{pgfscope}%
\begin{pgfscope}%
\pgfsys@transformshift{3.631830in}{3.884011in}%
\pgfsys@useobject{currentmarker}{}%
\end{pgfscope}%
\begin{pgfscope}%
\pgfsys@transformshift{3.649424in}{3.824663in}%
\pgfsys@useobject{currentmarker}{}%
\end{pgfscope}%
\begin{pgfscope}%
\pgfsys@transformshift{3.667017in}{4.016085in}%
\pgfsys@useobject{currentmarker}{}%
\end{pgfscope}%
\begin{pgfscope}%
\pgfsys@transformshift{3.684611in}{3.855894in}%
\pgfsys@useobject{currentmarker}{}%
\end{pgfscope}%
\begin{pgfscope}%
\pgfsys@transformshift{3.702205in}{3.891386in}%
\pgfsys@useobject{currentmarker}{}%
\end{pgfscope}%
\begin{pgfscope}%
\pgfsys@transformshift{3.719799in}{4.048210in}%
\pgfsys@useobject{currentmarker}{}%
\end{pgfscope}%
\begin{pgfscope}%
\pgfsys@transformshift{3.737393in}{3.722767in}%
\pgfsys@useobject{currentmarker}{}%
\end{pgfscope}%
\begin{pgfscope}%
\pgfsys@transformshift{3.754987in}{3.732831in}%
\pgfsys@useobject{currentmarker}{}%
\end{pgfscope}%
\begin{pgfscope}%
\pgfsys@transformshift{3.772581in}{3.961518in}%
\pgfsys@useobject{currentmarker}{}%
\end{pgfscope}%
\begin{pgfscope}%
\pgfsys@transformshift{3.790174in}{3.741367in}%
\pgfsys@useobject{currentmarker}{}%
\end{pgfscope}%
\begin{pgfscope}%
\pgfsys@transformshift{3.807768in}{4.055550in}%
\pgfsys@useobject{currentmarker}{}%
\end{pgfscope}%
\begin{pgfscope}%
\pgfsys@transformshift{3.825362in}{3.809556in}%
\pgfsys@useobject{currentmarker}{}%
\end{pgfscope}%
\begin{pgfscope}%
\pgfsys@transformshift{3.842956in}{3.767941in}%
\pgfsys@useobject{currentmarker}{}%
\end{pgfscope}%
\begin{pgfscope}%
\pgfsys@transformshift{3.860550in}{4.030761in}%
\pgfsys@useobject{currentmarker}{}%
\end{pgfscope}%
\begin{pgfscope}%
\pgfsys@transformshift{3.878144in}{3.974297in}%
\pgfsys@useobject{currentmarker}{}%
\end{pgfscope}%
\begin{pgfscope}%
\pgfsys@transformshift{3.895738in}{4.000639in}%
\pgfsys@useobject{currentmarker}{}%
\end{pgfscope}%
\begin{pgfscope}%
\pgfsys@transformshift{3.913331in}{3.887782in}%
\pgfsys@useobject{currentmarker}{}%
\end{pgfscope}%
\begin{pgfscope}%
\pgfsys@transformshift{3.930925in}{3.691193in}%
\pgfsys@useobject{currentmarker}{}%
\end{pgfscope}%
\begin{pgfscope}%
\pgfsys@transformshift{3.948519in}{3.955984in}%
\pgfsys@useobject{currentmarker}{}%
\end{pgfscope}%
\begin{pgfscope}%
\pgfsys@transformshift{3.966113in}{3.714784in}%
\pgfsys@useobject{currentmarker}{}%
\end{pgfscope}%
\begin{pgfscope}%
\pgfsys@transformshift{3.983707in}{3.803722in}%
\pgfsys@useobject{currentmarker}{}%
\end{pgfscope}%
\begin{pgfscope}%
\pgfsys@transformshift{4.001301in}{3.797315in}%
\pgfsys@useobject{currentmarker}{}%
\end{pgfscope}%
\begin{pgfscope}%
\pgfsys@transformshift{4.018894in}{3.662972in}%
\pgfsys@useobject{currentmarker}{}%
\end{pgfscope}%
\begin{pgfscope}%
\pgfsys@transformshift{4.036488in}{3.718881in}%
\pgfsys@useobject{currentmarker}{}%
\end{pgfscope}%
\begin{pgfscope}%
\pgfsys@transformshift{4.054082in}{3.727407in}%
\pgfsys@useobject{currentmarker}{}%
\end{pgfscope}%
\begin{pgfscope}%
\pgfsys@transformshift{4.071676in}{3.648683in}%
\pgfsys@useobject{currentmarker}{}%
\end{pgfscope}%
\begin{pgfscope}%
\pgfsys@transformshift{4.089270in}{3.475151in}%
\pgfsys@useobject{currentmarker}{}%
\end{pgfscope}%
\begin{pgfscope}%
\pgfsys@transformshift{4.106864in}{3.594877in}%
\pgfsys@useobject{currentmarker}{}%
\end{pgfscope}%
\begin{pgfscope}%
\pgfsys@transformshift{4.124458in}{3.677368in}%
\pgfsys@useobject{currentmarker}{}%
\end{pgfscope}%
\begin{pgfscope}%
\pgfsys@transformshift{4.142051in}{3.449577in}%
\pgfsys@useobject{currentmarker}{}%
\end{pgfscope}%
\begin{pgfscope}%
\pgfsys@transformshift{4.159645in}{3.484281in}%
\pgfsys@useobject{currentmarker}{}%
\end{pgfscope}%
\begin{pgfscope}%
\pgfsys@transformshift{4.177239in}{3.435330in}%
\pgfsys@useobject{currentmarker}{}%
\end{pgfscope}%
\begin{pgfscope}%
\pgfsys@transformshift{4.194833in}{3.649654in}%
\pgfsys@useobject{currentmarker}{}%
\end{pgfscope}%
\begin{pgfscope}%
\pgfsys@transformshift{4.212427in}{3.512382in}%
\pgfsys@useobject{currentmarker}{}%
\end{pgfscope}%
\begin{pgfscope}%
\pgfsys@transformshift{4.230021in}{3.469654in}%
\pgfsys@useobject{currentmarker}{}%
\end{pgfscope}%
\begin{pgfscope}%
\pgfsys@transformshift{4.247615in}{3.335540in}%
\pgfsys@useobject{currentmarker}{}%
\end{pgfscope}%
\begin{pgfscope}%
\pgfsys@transformshift{4.265208in}{3.456780in}%
\pgfsys@useobject{currentmarker}{}%
\end{pgfscope}%
\begin{pgfscope}%
\pgfsys@transformshift{4.282802in}{3.322666in}%
\pgfsys@useobject{currentmarker}{}%
\end{pgfscope}%
\begin{pgfscope}%
\pgfsys@transformshift{4.300396in}{3.386302in}%
\pgfsys@useobject{currentmarker}{}%
\end{pgfscope}%
\begin{pgfscope}%
\pgfsys@transformshift{4.317990in}{3.311898in}%
\pgfsys@useobject{currentmarker}{}%
\end{pgfscope}%
\begin{pgfscope}%
\pgfsys@transformshift{4.335584in}{3.441502in}%
\pgfsys@useobject{currentmarker}{}%
\end{pgfscope}%
\begin{pgfscope}%
\pgfsys@transformshift{4.353178in}{3.429239in}%
\pgfsys@useobject{currentmarker}{}%
\end{pgfscope}%
\begin{pgfscope}%
\pgfsys@transformshift{4.370772in}{3.349260in}%
\pgfsys@useobject{currentmarker}{}%
\end{pgfscope}%
\begin{pgfscope}%
\pgfsys@transformshift{4.388365in}{3.413070in}%
\pgfsys@useobject{currentmarker}{}%
\end{pgfscope}%
\begin{pgfscope}%
\pgfsys@transformshift{4.405959in}{3.265505in}%
\pgfsys@useobject{currentmarker}{}%
\end{pgfscope}%
\begin{pgfscope}%
\pgfsys@transformshift{4.423553in}{3.231219in}%
\pgfsys@useobject{currentmarker}{}%
\end{pgfscope}%
\begin{pgfscope}%
\pgfsys@transformshift{4.441147in}{3.436381in}%
\pgfsys@useobject{currentmarker}{}%
\end{pgfscope}%
\begin{pgfscope}%
\pgfsys@transformshift{4.458741in}{3.418730in}%
\pgfsys@useobject{currentmarker}{}%
\end{pgfscope}%
\begin{pgfscope}%
\pgfsys@transformshift{4.476335in}{3.478412in}%
\pgfsys@useobject{currentmarker}{}%
\end{pgfscope}%
\begin{pgfscope}%
\pgfsys@transformshift{4.493928in}{3.670229in}%
\pgfsys@useobject{currentmarker}{}%
\end{pgfscope}%
\begin{pgfscope}%
\pgfsys@transformshift{4.511522in}{3.538571in}%
\pgfsys@useobject{currentmarker}{}%
\end{pgfscope}%
\begin{pgfscope}%
\pgfsys@transformshift{4.529116in}{3.365555in}%
\pgfsys@useobject{currentmarker}{}%
\end{pgfscope}%
\begin{pgfscope}%
\pgfsys@transformshift{4.546710in}{3.590089in}%
\pgfsys@useobject{currentmarker}{}%
\end{pgfscope}%
\begin{pgfscope}%
\pgfsys@transformshift{4.564304in}{3.360521in}%
\pgfsys@useobject{currentmarker}{}%
\end{pgfscope}%
\begin{pgfscope}%
\pgfsys@transformshift{4.581898in}{3.466956in}%
\pgfsys@useobject{currentmarker}{}%
\end{pgfscope}%
\begin{pgfscope}%
\pgfsys@transformshift{4.599492in}{3.526978in}%
\pgfsys@useobject{currentmarker}{}%
\end{pgfscope}%
\begin{pgfscope}%
\pgfsys@transformshift{4.617085in}{3.728956in}%
\pgfsys@useobject{currentmarker}{}%
\end{pgfscope}%
\begin{pgfscope}%
\pgfsys@transformshift{4.634679in}{3.498783in}%
\pgfsys@useobject{currentmarker}{}%
\end{pgfscope}%
\begin{pgfscope}%
\pgfsys@transformshift{4.652273in}{3.510921in}%
\pgfsys@useobject{currentmarker}{}%
\end{pgfscope}%
\begin{pgfscope}%
\pgfsys@transformshift{4.669867in}{3.605397in}%
\pgfsys@useobject{currentmarker}{}%
\end{pgfscope}%
\begin{pgfscope}%
\pgfsys@transformshift{4.687461in}{3.567591in}%
\pgfsys@useobject{currentmarker}{}%
\end{pgfscope}%
\begin{pgfscope}%
\pgfsys@transformshift{4.705055in}{3.769320in}%
\pgfsys@useobject{currentmarker}{}%
\end{pgfscope}%
\begin{pgfscope}%
\pgfsys@transformshift{4.722649in}{3.562676in}%
\pgfsys@useobject{currentmarker}{}%
\end{pgfscope}%
\begin{pgfscope}%
\pgfsys@transformshift{4.740242in}{3.573159in}%
\pgfsys@useobject{currentmarker}{}%
\end{pgfscope}%
\begin{pgfscope}%
\pgfsys@transformshift{4.757836in}{3.661725in}%
\pgfsys@useobject{currentmarker}{}%
\end{pgfscope}%
\begin{pgfscope}%
\pgfsys@transformshift{4.775430in}{3.670459in}%
\pgfsys@useobject{currentmarker}{}%
\end{pgfscope}%
\begin{pgfscope}%
\pgfsys@transformshift{4.793024in}{3.931412in}%
\pgfsys@useobject{currentmarker}{}%
\end{pgfscope}%
\begin{pgfscope}%
\pgfsys@transformshift{4.810618in}{3.843274in}%
\pgfsys@useobject{currentmarker}{}%
\end{pgfscope}%
\begin{pgfscope}%
\pgfsys@transformshift{4.828212in}{3.765484in}%
\pgfsys@useobject{currentmarker}{}%
\end{pgfscope}%
\begin{pgfscope}%
\pgfsys@transformshift{4.845805in}{3.639892in}%
\pgfsys@useobject{currentmarker}{}%
\end{pgfscope}%
\begin{pgfscope}%
\pgfsys@transformshift{4.863399in}{3.857380in}%
\pgfsys@useobject{currentmarker}{}%
\end{pgfscope}%
\begin{pgfscope}%
\pgfsys@transformshift{4.880993in}{3.673775in}%
\pgfsys@useobject{currentmarker}{}%
\end{pgfscope}%
\begin{pgfscope}%
\pgfsys@transformshift{4.898587in}{3.620776in}%
\pgfsys@useobject{currentmarker}{}%
\end{pgfscope}%
\begin{pgfscope}%
\pgfsys@transformshift{4.916181in}{3.900005in}%
\pgfsys@useobject{currentmarker}{}%
\end{pgfscope}%
\begin{pgfscope}%
\pgfsys@transformshift{4.933775in}{3.809765in}%
\pgfsys@useobject{currentmarker}{}%
\end{pgfscope}%
\begin{pgfscope}%
\pgfsys@transformshift{4.951369in}{3.867996in}%
\pgfsys@useobject{currentmarker}{}%
\end{pgfscope}%
\begin{pgfscope}%
\pgfsys@transformshift{4.968962in}{3.801352in}%
\pgfsys@useobject{currentmarker}{}%
\end{pgfscope}%
\begin{pgfscope}%
\pgfsys@transformshift{4.986556in}{3.849159in}%
\pgfsys@useobject{currentmarker}{}%
\end{pgfscope}%
\begin{pgfscope}%
\pgfsys@transformshift{5.004150in}{3.686598in}%
\pgfsys@useobject{currentmarker}{}%
\end{pgfscope}%
\begin{pgfscope}%
\pgfsys@transformshift{5.021744in}{3.637093in}%
\pgfsys@useobject{currentmarker}{}%
\end{pgfscope}%
\begin{pgfscope}%
\pgfsys@transformshift{5.039338in}{3.800132in}%
\pgfsys@useobject{currentmarker}{}%
\end{pgfscope}%
\begin{pgfscope}%
\pgfsys@transformshift{5.056932in}{3.635404in}%
\pgfsys@useobject{currentmarker}{}%
\end{pgfscope}%
\begin{pgfscope}%
\pgfsys@transformshift{5.074526in}{3.632508in}%
\pgfsys@useobject{currentmarker}{}%
\end{pgfscope}%
\begin{pgfscope}%
\pgfsys@transformshift{5.092119in}{3.640821in}%
\pgfsys@useobject{currentmarker}{}%
\end{pgfscope}%
\begin{pgfscope}%
\pgfsys@transformshift{5.109713in}{3.672640in}%
\pgfsys@useobject{currentmarker}{}%
\end{pgfscope}%
\begin{pgfscope}%
\pgfsys@transformshift{5.127307in}{3.617668in}%
\pgfsys@useobject{currentmarker}{}%
\end{pgfscope}%
\begin{pgfscope}%
\pgfsys@transformshift{5.144901in}{3.495984in}%
\pgfsys@useobject{currentmarker}{}%
\end{pgfscope}%
\begin{pgfscope}%
\pgfsys@transformshift{5.162495in}{3.552707in}%
\pgfsys@useobject{currentmarker}{}%
\end{pgfscope}%
\begin{pgfscope}%
\pgfsys@transformshift{5.180089in}{3.373684in}%
\pgfsys@useobject{currentmarker}{}%
\end{pgfscope}%
\begin{pgfscope}%
\pgfsys@transformshift{5.197683in}{3.646375in}%
\pgfsys@useobject{currentmarker}{}%
\end{pgfscope}%
\begin{pgfscope}%
\pgfsys@transformshift{5.215276in}{3.401793in}%
\pgfsys@useobject{currentmarker}{}%
\end{pgfscope}%
\begin{pgfscope}%
\pgfsys@transformshift{5.232870in}{3.435629in}%
\pgfsys@useobject{currentmarker}{}%
\end{pgfscope}%
\begin{pgfscope}%
\pgfsys@transformshift{5.250464in}{3.537393in}%
\pgfsys@useobject{currentmarker}{}%
\end{pgfscope}%
\begin{pgfscope}%
\pgfsys@transformshift{5.268058in}{3.441417in}%
\pgfsys@useobject{currentmarker}{}%
\end{pgfscope}%
\begin{pgfscope}%
\pgfsys@transformshift{5.285652in}{3.660058in}%
\pgfsys@useobject{currentmarker}{}%
\end{pgfscope}%
\begin{pgfscope}%
\pgfsys@transformshift{5.303246in}{3.358041in}%
\pgfsys@useobject{currentmarker}{}%
\end{pgfscope}%
\begin{pgfscope}%
\pgfsys@transformshift{5.320839in}{3.505730in}%
\pgfsys@useobject{currentmarker}{}%
\end{pgfscope}%
\begin{pgfscope}%
\pgfsys@transformshift{5.338433in}{3.464710in}%
\pgfsys@useobject{currentmarker}{}%
\end{pgfscope}%
\begin{pgfscope}%
\pgfsys@transformshift{5.356027in}{3.341550in}%
\pgfsys@useobject{currentmarker}{}%
\end{pgfscope}%
\begin{pgfscope}%
\pgfsys@transformshift{5.373621in}{3.507820in}%
\pgfsys@useobject{currentmarker}{}%
\end{pgfscope}%
\begin{pgfscope}%
\pgfsys@transformshift{5.391215in}{3.432706in}%
\pgfsys@useobject{currentmarker}{}%
\end{pgfscope}%
\begin{pgfscope}%
\pgfsys@transformshift{5.408809in}{3.526663in}%
\pgfsys@useobject{currentmarker}{}%
\end{pgfscope}%
\begin{pgfscope}%
\pgfsys@transformshift{5.426403in}{3.531779in}%
\pgfsys@useobject{currentmarker}{}%
\end{pgfscope}%
\begin{pgfscope}%
\pgfsys@transformshift{5.443996in}{3.670360in}%
\pgfsys@useobject{currentmarker}{}%
\end{pgfscope}%
\begin{pgfscope}%
\pgfsys@transformshift{5.461590in}{3.590161in}%
\pgfsys@useobject{currentmarker}{}%
\end{pgfscope}%
\begin{pgfscope}%
\pgfsys@transformshift{5.479184in}{3.422463in}%
\pgfsys@useobject{currentmarker}{}%
\end{pgfscope}%
\begin{pgfscope}%
\pgfsys@transformshift{5.496778in}{3.444053in}%
\pgfsys@useobject{currentmarker}{}%
\end{pgfscope}%
\begin{pgfscope}%
\pgfsys@transformshift{5.514372in}{3.591023in}%
\pgfsys@useobject{currentmarker}{}%
\end{pgfscope}%
\begin{pgfscope}%
\pgfsys@transformshift{5.531966in}{3.558137in}%
\pgfsys@useobject{currentmarker}{}%
\end{pgfscope}%
\begin{pgfscope}%
\pgfsys@transformshift{5.549560in}{3.570947in}%
\pgfsys@useobject{currentmarker}{}%
\end{pgfscope}%
\begin{pgfscope}%
\pgfsys@transformshift{5.567153in}{3.356960in}%
\pgfsys@useobject{currentmarker}{}%
\end{pgfscope}%
\begin{pgfscope}%
\pgfsys@transformshift{5.584747in}{3.537244in}%
\pgfsys@useobject{currentmarker}{}%
\end{pgfscope}%
\begin{pgfscope}%
\pgfsys@transformshift{5.602341in}{3.483911in}%
\pgfsys@useobject{currentmarker}{}%
\end{pgfscope}%
\begin{pgfscope}%
\pgfsys@transformshift{5.619935in}{3.608578in}%
\pgfsys@useobject{currentmarker}{}%
\end{pgfscope}%
\begin{pgfscope}%
\pgfsys@transformshift{5.637529in}{3.592139in}%
\pgfsys@useobject{currentmarker}{}%
\end{pgfscope}%
\begin{pgfscope}%
\pgfsys@transformshift{5.655123in}{3.718154in}%
\pgfsys@useobject{currentmarker}{}%
\end{pgfscope}%
\begin{pgfscope}%
\pgfsys@transformshift{5.672717in}{3.681771in}%
\pgfsys@useobject{currentmarker}{}%
\end{pgfscope}%
\begin{pgfscope}%
\pgfsys@transformshift{5.690310in}{3.754388in}%
\pgfsys@useobject{currentmarker}{}%
\end{pgfscope}%
\begin{pgfscope}%
\pgfsys@transformshift{5.707904in}{3.651918in}%
\pgfsys@useobject{currentmarker}{}%
\end{pgfscope}%
\begin{pgfscope}%
\pgfsys@transformshift{5.725498in}{3.628742in}%
\pgfsys@useobject{currentmarker}{}%
\end{pgfscope}%
\begin{pgfscope}%
\pgfsys@transformshift{5.743092in}{3.708892in}%
\pgfsys@useobject{currentmarker}{}%
\end{pgfscope}%
\begin{pgfscope}%
\pgfsys@transformshift{5.760686in}{3.774506in}%
\pgfsys@useobject{currentmarker}{}%
\end{pgfscope}%
\begin{pgfscope}%
\pgfsys@transformshift{5.778280in}{3.840115in}%
\pgfsys@useobject{currentmarker}{}%
\end{pgfscope}%
\begin{pgfscope}%
\pgfsys@transformshift{5.795873in}{4.056527in}%
\pgfsys@useobject{currentmarker}{}%
\end{pgfscope}%
\begin{pgfscope}%
\pgfsys@transformshift{5.813467in}{3.845793in}%
\pgfsys@useobject{currentmarker}{}%
\end{pgfscope}%
\begin{pgfscope}%
\pgfsys@transformshift{5.831061in}{3.775681in}%
\pgfsys@useobject{currentmarker}{}%
\end{pgfscope}%
\begin{pgfscope}%
\pgfsys@transformshift{5.848655in}{3.859859in}%
\pgfsys@useobject{currentmarker}{}%
\end{pgfscope}%
\begin{pgfscope}%
\pgfsys@transformshift{5.866249in}{3.868597in}%
\pgfsys@useobject{currentmarker}{}%
\end{pgfscope}%
\begin{pgfscope}%
\pgfsys@transformshift{5.883843in}{3.984304in}%
\pgfsys@useobject{currentmarker}{}%
\end{pgfscope}%
\begin{pgfscope}%
\pgfsys@transformshift{5.901437in}{3.795885in}%
\pgfsys@useobject{currentmarker}{}%
\end{pgfscope}%
\begin{pgfscope}%
\pgfsys@transformshift{5.919030in}{3.975605in}%
\pgfsys@useobject{currentmarker}{}%
\end{pgfscope}%
\begin{pgfscope}%
\pgfsys@transformshift{5.936624in}{3.999505in}%
\pgfsys@useobject{currentmarker}{}%
\end{pgfscope}%
\begin{pgfscope}%
\pgfsys@transformshift{5.954218in}{4.019775in}%
\pgfsys@useobject{currentmarker}{}%
\end{pgfscope}%
\begin{pgfscope}%
\pgfsys@transformshift{5.971812in}{3.945838in}%
\pgfsys@useobject{currentmarker}{}%
\end{pgfscope}%
\begin{pgfscope}%
\pgfsys@transformshift{5.989406in}{3.991188in}%
\pgfsys@useobject{currentmarker}{}%
\end{pgfscope}%
\begin{pgfscope}%
\pgfsys@transformshift{6.007000in}{3.876964in}%
\pgfsys@useobject{currentmarker}{}%
\end{pgfscope}%
\begin{pgfscope}%
\pgfsys@transformshift{6.024594in}{3.976630in}%
\pgfsys@useobject{currentmarker}{}%
\end{pgfscope}%
\begin{pgfscope}%
\pgfsys@transformshift{6.042187in}{3.974377in}%
\pgfsys@useobject{currentmarker}{}%
\end{pgfscope}%
\begin{pgfscope}%
\pgfsys@transformshift{6.059781in}{4.073041in}%
\pgfsys@useobject{currentmarker}{}%
\end{pgfscope}%
\begin{pgfscope}%
\pgfsys@transformshift{6.077375in}{3.911734in}%
\pgfsys@useobject{currentmarker}{}%
\end{pgfscope}%
\begin{pgfscope}%
\pgfsys@transformshift{6.094969in}{4.106534in}%
\pgfsys@useobject{currentmarker}{}%
\end{pgfscope}%
\begin{pgfscope}%
\pgfsys@transformshift{6.112563in}{4.174823in}%
\pgfsys@useobject{currentmarker}{}%
\end{pgfscope}%
\begin{pgfscope}%
\pgfsys@transformshift{6.130157in}{3.805324in}%
\pgfsys@useobject{currentmarker}{}%
\end{pgfscope}%
\begin{pgfscope}%
\pgfsys@transformshift{6.147751in}{4.052418in}%
\pgfsys@useobject{currentmarker}{}%
\end{pgfscope}%
\begin{pgfscope}%
\pgfsys@transformshift{6.165344in}{4.069229in}%
\pgfsys@useobject{currentmarker}{}%
\end{pgfscope}%
\begin{pgfscope}%
\pgfsys@transformshift{6.182938in}{3.925320in}%
\pgfsys@useobject{currentmarker}{}%
\end{pgfscope}%
\begin{pgfscope}%
\pgfsys@transformshift{6.200532in}{3.938972in}%
\pgfsys@useobject{currentmarker}{}%
\end{pgfscope}%
\begin{pgfscope}%
\pgfsys@transformshift{6.218126in}{3.954318in}%
\pgfsys@useobject{currentmarker}{}%
\end{pgfscope}%
\begin{pgfscope}%
\pgfsys@transformshift{6.235720in}{3.925304in}%
\pgfsys@useobject{currentmarker}{}%
\end{pgfscope}%
\begin{pgfscope}%
\pgfsys@transformshift{6.253314in}{3.911578in}%
\pgfsys@useobject{currentmarker}{}%
\end{pgfscope}%
\begin{pgfscope}%
\pgfsys@transformshift{6.270907in}{3.759418in}%
\pgfsys@useobject{currentmarker}{}%
\end{pgfscope}%
\begin{pgfscope}%
\pgfsys@transformshift{6.288501in}{4.035096in}%
\pgfsys@useobject{currentmarker}{}%
\end{pgfscope}%
\begin{pgfscope}%
\pgfsys@transformshift{6.306095in}{4.015175in}%
\pgfsys@useobject{currentmarker}{}%
\end{pgfscope}%
\begin{pgfscope}%
\pgfsys@transformshift{6.323689in}{3.810072in}%
\pgfsys@useobject{currentmarker}{}%
\end{pgfscope}%
\begin{pgfscope}%
\pgfsys@transformshift{6.341283in}{3.732047in}%
\pgfsys@useobject{currentmarker}{}%
\end{pgfscope}%
\begin{pgfscope}%
\pgfsys@transformshift{6.358877in}{3.924124in}%
\pgfsys@useobject{currentmarker}{}%
\end{pgfscope}%
\begin{pgfscope}%
\pgfsys@transformshift{6.376471in}{3.802702in}%
\pgfsys@useobject{currentmarker}{}%
\end{pgfscope}%
\end{pgfscope}%
\begin{pgfscope}%
\pgfpathrectangle{\pgfqpoint{2.000000in}{3.197368in}}{\pgfqpoint{4.376471in}{0.978947in}} %
\pgfusepath{clip}%
\pgfsetroundcap%
\pgfsetroundjoin%
\pgfsetlinewidth{1.756562pt}%
\definecolor{currentstroke}{rgb}{0.298039,0.447059,0.690196}%
\pgfsetstrokecolor{currentstroke}%
\pgfsetdash{}{0pt}%
\pgfpathmoveto{\pgfqpoint{2.896410in}{4.186316in}}%
\pgfpathlineto{\pgfqpoint{2.910482in}{4.101386in}}%
\pgfpathlineto{\pgfqpoint{2.928076in}{4.023940in}}%
\pgfpathlineto{\pgfqpoint{2.945670in}{3.967808in}}%
\pgfpathlineto{\pgfqpoint{2.963263in}{3.926974in}}%
\pgfpathlineto{\pgfqpoint{2.980857in}{3.896652in}}%
\pgfpathlineto{\pgfqpoint{2.998451in}{3.873118in}}%
\pgfpathlineto{\pgfqpoint{3.016045in}{3.853557in}}%
\pgfpathlineto{\pgfqpoint{3.086420in}{3.782917in}}%
\pgfpathlineto{\pgfqpoint{3.121608in}{3.743296in}}%
\pgfpathlineto{\pgfqpoint{3.191983in}{3.660375in}}%
\pgfpathlineto{\pgfqpoint{3.209577in}{3.641501in}}%
\pgfpathlineto{\pgfqpoint{3.227171in}{3.624312in}}%
\pgfpathlineto{\pgfqpoint{3.244765in}{3.609211in}}%
\pgfpathlineto{\pgfqpoint{3.262359in}{3.596536in}}%
\pgfpathlineto{\pgfqpoint{3.279953in}{3.586547in}}%
\pgfpathlineto{\pgfqpoint{3.297547in}{3.579426in}}%
\pgfpathlineto{\pgfqpoint{3.315140in}{3.575272in}}%
\pgfpathlineto{\pgfqpoint{3.332734in}{3.574105in}}%
\pgfpathlineto{\pgfqpoint{3.350328in}{3.575869in}}%
\pgfpathlineto{\pgfqpoint{3.367922in}{3.580440in}}%
\pgfpathlineto{\pgfqpoint{3.385516in}{3.587631in}}%
\pgfpathlineto{\pgfqpoint{3.403110in}{3.597205in}}%
\pgfpathlineto{\pgfqpoint{3.420704in}{3.608879in}}%
\pgfpathlineto{\pgfqpoint{3.438297in}{3.622340in}}%
\pgfpathlineto{\pgfqpoint{3.473485in}{3.653249in}}%
\pgfpathlineto{\pgfqpoint{3.579048in}{3.752690in}}%
\pgfpathlineto{\pgfqpoint{3.614236in}{3.779750in}}%
\pgfpathlineto{\pgfqpoint{3.631830in}{3.791052in}}%
\pgfpathlineto{\pgfqpoint{3.649424in}{3.800654in}}%
\pgfpathlineto{\pgfqpoint{3.667017in}{3.808440in}}%
\pgfpathlineto{\pgfqpoint{3.684611in}{3.814326in}}%
\pgfpathlineto{\pgfqpoint{3.702205in}{3.818263in}}%
\pgfpathlineto{\pgfqpoint{3.719799in}{3.820229in}}%
\pgfpathlineto{\pgfqpoint{3.737393in}{3.820234in}}%
\pgfpathlineto{\pgfqpoint{3.754987in}{3.818311in}}%
\pgfpathlineto{\pgfqpoint{3.772581in}{3.814516in}}%
\pgfpathlineto{\pgfqpoint{3.790174in}{3.808926in}}%
\pgfpathlineto{\pgfqpoint{3.807768in}{3.801635in}}%
\pgfpathlineto{\pgfqpoint{3.825362in}{3.792749in}}%
\pgfpathlineto{\pgfqpoint{3.842956in}{3.782386in}}%
\pgfpathlineto{\pgfqpoint{3.878144in}{3.757739in}}%
\pgfpathlineto{\pgfqpoint{3.913331in}{3.728753in}}%
\pgfpathlineto{\pgfqpoint{3.948519in}{3.696514in}}%
\pgfpathlineto{\pgfqpoint{4.001301in}{3.644355in}}%
\pgfpathlineto{\pgfqpoint{4.106864in}{3.538095in}}%
\pgfpathlineto{\pgfqpoint{4.142051in}{3.505186in}}%
\pgfpathlineto{\pgfqpoint{4.177239in}{3.474796in}}%
\pgfpathlineto{\pgfqpoint{4.212427in}{3.447640in}}%
\pgfpathlineto{\pgfqpoint{4.247615in}{3.424407in}}%
\pgfpathlineto{\pgfqpoint{4.282802in}{3.405764in}}%
\pgfpathlineto{\pgfqpoint{4.300396in}{3.398362in}}%
\pgfpathlineto{\pgfqpoint{4.317990in}{3.392339in}}%
\pgfpathlineto{\pgfqpoint{4.335584in}{3.387761in}}%
\pgfpathlineto{\pgfqpoint{4.353178in}{3.384690in}}%
\pgfpathlineto{\pgfqpoint{4.370772in}{3.383182in}}%
\pgfpathlineto{\pgfqpoint{4.388365in}{3.383281in}}%
\pgfpathlineto{\pgfqpoint{4.405959in}{3.385022in}}%
\pgfpathlineto{\pgfqpoint{4.423553in}{3.388430in}}%
\pgfpathlineto{\pgfqpoint{4.441147in}{3.393516in}}%
\pgfpathlineto{\pgfqpoint{4.458741in}{3.400276in}}%
\pgfpathlineto{\pgfqpoint{4.476335in}{3.408693in}}%
\pgfpathlineto{\pgfqpoint{4.493928in}{3.418733in}}%
\pgfpathlineto{\pgfqpoint{4.511522in}{3.430344in}}%
\pgfpathlineto{\pgfqpoint{4.529116in}{3.443458in}}%
\pgfpathlineto{\pgfqpoint{4.564304in}{3.473831in}}%
\pgfpathlineto{\pgfqpoint{4.599492in}{3.508944in}}%
\pgfpathlineto{\pgfqpoint{4.634679in}{3.547618in}}%
\pgfpathlineto{\pgfqpoint{4.757836in}{3.688928in}}%
\pgfpathlineto{\pgfqpoint{4.793024in}{3.723894in}}%
\pgfpathlineto{\pgfqpoint{4.810618in}{3.739395in}}%
\pgfpathlineto{\pgfqpoint{4.828212in}{3.753315in}}%
\pgfpathlineto{\pgfqpoint{4.845805in}{3.765477in}}%
\pgfpathlineto{\pgfqpoint{4.863399in}{3.775721in}}%
\pgfpathlineto{\pgfqpoint{4.880993in}{3.783910in}}%
\pgfpathlineto{\pgfqpoint{4.898587in}{3.789928in}}%
\pgfpathlineto{\pgfqpoint{4.916181in}{3.793687in}}%
\pgfpathlineto{\pgfqpoint{4.933775in}{3.795123in}}%
\pgfpathlineto{\pgfqpoint{4.951369in}{3.794204in}}%
\pgfpathlineto{\pgfqpoint{4.968962in}{3.790925in}}%
\pgfpathlineto{\pgfqpoint{4.986556in}{3.785314in}}%
\pgfpathlineto{\pgfqpoint{5.004150in}{3.777426in}}%
\pgfpathlineto{\pgfqpoint{5.021744in}{3.767351in}}%
\pgfpathlineto{\pgfqpoint{5.039338in}{3.755205in}}%
\pgfpathlineto{\pgfqpoint{5.056932in}{3.741134in}}%
\pgfpathlineto{\pgfqpoint{5.074526in}{3.725311in}}%
\pgfpathlineto{\pgfqpoint{5.109713in}{3.689225in}}%
\pgfpathlineto{\pgfqpoint{5.144901in}{3.648785in}}%
\pgfpathlineto{\pgfqpoint{5.250464in}{3.522986in}}%
\pgfpathlineto{\pgfqpoint{5.285652in}{3.487041in}}%
\pgfpathlineto{\pgfqpoint{5.303246in}{3.471369in}}%
\pgfpathlineto{\pgfqpoint{5.320839in}{3.457526in}}%
\pgfpathlineto{\pgfqpoint{5.338433in}{3.445707in}}%
\pgfpathlineto{\pgfqpoint{5.356027in}{3.436077in}}%
\pgfpathlineto{\pgfqpoint{5.373621in}{3.428776in}}%
\pgfpathlineto{\pgfqpoint{5.391215in}{3.423912in}}%
\pgfpathlineto{\pgfqpoint{5.408809in}{3.421562in}}%
\pgfpathlineto{\pgfqpoint{5.426403in}{3.421771in}}%
\pgfpathlineto{\pgfqpoint{5.443996in}{3.424552in}}%
\pgfpathlineto{\pgfqpoint{5.461590in}{3.429886in}}%
\pgfpathlineto{\pgfqpoint{5.479184in}{3.437722in}}%
\pgfpathlineto{\pgfqpoint{5.496778in}{3.447982in}}%
\pgfpathlineto{\pgfqpoint{5.514372in}{3.460556in}}%
\pgfpathlineto{\pgfqpoint{5.531966in}{3.475312in}}%
\pgfpathlineto{\pgfqpoint{5.549560in}{3.492095in}}%
\pgfpathlineto{\pgfqpoint{5.567153in}{3.510728in}}%
\pgfpathlineto{\pgfqpoint{5.602341in}{3.552762in}}%
\pgfpathlineto{\pgfqpoint{5.637529in}{3.599740in}}%
\pgfpathlineto{\pgfqpoint{5.690310in}{3.675605in}}%
\pgfpathlineto{\pgfqpoint{5.760686in}{3.777950in}}%
\pgfpathlineto{\pgfqpoint{5.795873in}{3.826339in}}%
\pgfpathlineto{\pgfqpoint{5.831061in}{3.871346in}}%
\pgfpathlineto{\pgfqpoint{5.866249in}{3.912133in}}%
\pgfpathlineto{\pgfqpoint{5.901437in}{3.948064in}}%
\pgfpathlineto{\pgfqpoint{5.936624in}{3.978642in}}%
\pgfpathlineto{\pgfqpoint{5.954218in}{3.991784in}}%
\pgfpathlineto{\pgfqpoint{5.971812in}{4.003418in}}%
\pgfpathlineto{\pgfqpoint{5.989406in}{4.013483in}}%
\pgfpathlineto{\pgfqpoint{6.007000in}{4.021910in}}%
\pgfpathlineto{\pgfqpoint{6.024594in}{4.028625in}}%
\pgfpathlineto{\pgfqpoint{6.042187in}{4.033542in}}%
\pgfpathlineto{\pgfqpoint{6.059781in}{4.036573in}}%
\pgfpathlineto{\pgfqpoint{6.077375in}{4.037624in}}%
\pgfpathlineto{\pgfqpoint{6.094969in}{4.036599in}}%
\pgfpathlineto{\pgfqpoint{6.112563in}{4.033411in}}%
\pgfpathlineto{\pgfqpoint{6.130157in}{4.027987in}}%
\pgfpathlineto{\pgfqpoint{6.147751in}{4.020276in}}%
\pgfpathlineto{\pgfqpoint{6.165344in}{4.010267in}}%
\pgfpathlineto{\pgfqpoint{6.182938in}{3.998002in}}%
\pgfpathlineto{\pgfqpoint{6.200532in}{3.983595in}}%
\pgfpathlineto{\pgfqpoint{6.218126in}{3.967255in}}%
\pgfpathlineto{\pgfqpoint{6.253314in}{3.930229in}}%
\pgfpathlineto{\pgfqpoint{6.288501in}{3.891499in}}%
\pgfpathlineto{\pgfqpoint{6.306095in}{3.873828in}}%
\pgfpathlineto{\pgfqpoint{6.323689in}{3.859066in}}%
\pgfpathlineto{\pgfqpoint{6.341283in}{3.848953in}}%
\pgfpathlineto{\pgfqpoint{6.358877in}{3.845611in}}%
\pgfpathlineto{\pgfqpoint{6.376471in}{3.851585in}}%
\pgfpathlineto{\pgfqpoint{6.376471in}{3.851585in}}%
\pgfusepath{stroke}%
\end{pgfscope}%
\begin{pgfscope}%
\pgfpathrectangle{\pgfqpoint{2.000000in}{3.197368in}}{\pgfqpoint{4.376471in}{0.978947in}} %
\pgfusepath{clip}%
\pgfsetbuttcap%
\pgfsetroundjoin%
\pgfsetlinewidth{1.756562pt}%
\definecolor{currentstroke}{rgb}{1.000000,0.647059,0.000000}%
\pgfsetstrokecolor{currentstroke}%
\pgfsetdash{{6.000000pt}{6.000000pt}}{0.000000pt}%
\pgfpathmoveto{\pgfqpoint{2.896399in}{4.186316in}}%
\pgfpathlineto{\pgfqpoint{2.910482in}{4.101329in}}%
\pgfpathlineto{\pgfqpoint{2.928076in}{4.023905in}}%
\pgfpathlineto{\pgfqpoint{2.945670in}{3.967806in}}%
\pgfpathlineto{\pgfqpoint{2.963263in}{3.926994in}}%
\pgfpathlineto{\pgfqpoint{2.980857in}{3.896680in}}%
\pgfpathlineto{\pgfqpoint{2.998451in}{3.873132in}}%
\pgfpathlineto{\pgfqpoint{3.016045in}{3.853574in}}%
\pgfpathlineto{\pgfqpoint{3.086420in}{3.782890in}}%
\pgfpathlineto{\pgfqpoint{3.121608in}{3.743284in}}%
\pgfpathlineto{\pgfqpoint{3.191983in}{3.660340in}}%
\pgfpathlineto{\pgfqpoint{3.209577in}{3.641453in}}%
\pgfpathlineto{\pgfqpoint{3.227171in}{3.624204in}}%
\pgfpathlineto{\pgfqpoint{3.244765in}{3.609134in}}%
\pgfpathlineto{\pgfqpoint{3.262359in}{3.596463in}}%
\pgfpathlineto{\pgfqpoint{3.279953in}{3.586479in}}%
\pgfpathlineto{\pgfqpoint{3.297547in}{3.579404in}}%
\pgfpathlineto{\pgfqpoint{3.315140in}{3.575202in}}%
\pgfpathlineto{\pgfqpoint{3.332734in}{3.574091in}}%
\pgfpathlineto{\pgfqpoint{3.350328in}{3.575807in}}%
\pgfpathlineto{\pgfqpoint{3.367922in}{3.580378in}}%
\pgfpathlineto{\pgfqpoint{3.385516in}{3.587626in}}%
\pgfpathlineto{\pgfqpoint{3.403110in}{3.597183in}}%
\pgfpathlineto{\pgfqpoint{3.420704in}{3.608876in}}%
\pgfpathlineto{\pgfqpoint{3.438297in}{3.622390in}}%
\pgfpathlineto{\pgfqpoint{3.473485in}{3.653292in}}%
\pgfpathlineto{\pgfqpoint{3.579048in}{3.752853in}}%
\pgfpathlineto{\pgfqpoint{3.614236in}{3.780023in}}%
\pgfpathlineto{\pgfqpoint{3.631830in}{3.791294in}}%
\pgfpathlineto{\pgfqpoint{3.649424in}{3.800882in}}%
\pgfpathlineto{\pgfqpoint{3.667017in}{3.808712in}}%
\pgfpathlineto{\pgfqpoint{3.684611in}{3.814654in}}%
\pgfpathlineto{\pgfqpoint{3.702205in}{3.818597in}}%
\pgfpathlineto{\pgfqpoint{3.719799in}{3.820556in}}%
\pgfpathlineto{\pgfqpoint{3.737393in}{3.820534in}}%
\pgfpathlineto{\pgfqpoint{3.754987in}{3.818621in}}%
\pgfpathlineto{\pgfqpoint{3.772581in}{3.814871in}}%
\pgfpathlineto{\pgfqpoint{3.790174in}{3.809247in}}%
\pgfpathlineto{\pgfqpoint{3.807768in}{3.801962in}}%
\pgfpathlineto{\pgfqpoint{3.825362in}{3.793080in}}%
\pgfpathlineto{\pgfqpoint{3.842956in}{3.782727in}}%
\pgfpathlineto{\pgfqpoint{3.878144in}{3.758036in}}%
\pgfpathlineto{\pgfqpoint{3.913331in}{3.729042in}}%
\pgfpathlineto{\pgfqpoint{3.948519in}{3.696820in}}%
\pgfpathlineto{\pgfqpoint{4.001301in}{3.644601in}}%
\pgfpathlineto{\pgfqpoint{4.106864in}{3.538227in}}%
\pgfpathlineto{\pgfqpoint{4.142051in}{3.505284in}}%
\pgfpathlineto{\pgfqpoint{4.177239in}{3.474904in}}%
\pgfpathlineto{\pgfqpoint{4.212427in}{3.447662in}}%
\pgfpathlineto{\pgfqpoint{4.247615in}{3.424521in}}%
\pgfpathlineto{\pgfqpoint{4.265208in}{3.414538in}}%
\pgfpathlineto{\pgfqpoint{4.282802in}{3.405862in}}%
\pgfpathlineto{\pgfqpoint{4.300396in}{3.398455in}}%
\pgfpathlineto{\pgfqpoint{4.317990in}{3.392331in}}%
\pgfpathlineto{\pgfqpoint{4.335584in}{3.387862in}}%
\pgfpathlineto{\pgfqpoint{4.353178in}{3.384738in}}%
\pgfpathlineto{\pgfqpoint{4.370772in}{3.383281in}}%
\pgfpathlineto{\pgfqpoint{4.388365in}{3.383331in}}%
\pgfpathlineto{\pgfqpoint{4.405959in}{3.385074in}}%
\pgfpathlineto{\pgfqpoint{4.423553in}{3.388447in}}%
\pgfpathlineto{\pgfqpoint{4.441147in}{3.393588in}}%
\pgfpathlineto{\pgfqpoint{4.458741in}{3.400322in}}%
\pgfpathlineto{\pgfqpoint{4.476335in}{3.408812in}}%
\pgfpathlineto{\pgfqpoint{4.493928in}{3.418783in}}%
\pgfpathlineto{\pgfqpoint{4.511522in}{3.430384in}}%
\pgfpathlineto{\pgfqpoint{4.529116in}{3.443504in}}%
\pgfpathlineto{\pgfqpoint{4.564304in}{3.473840in}}%
\pgfpathlineto{\pgfqpoint{4.599492in}{3.509043in}}%
\pgfpathlineto{\pgfqpoint{4.634679in}{3.547644in}}%
\pgfpathlineto{\pgfqpoint{4.757836in}{3.688928in}}%
\pgfpathlineto{\pgfqpoint{4.793024in}{3.723857in}}%
\pgfpathlineto{\pgfqpoint{4.810618in}{3.739467in}}%
\pgfpathlineto{\pgfqpoint{4.828212in}{3.753272in}}%
\pgfpathlineto{\pgfqpoint{4.845805in}{3.765421in}}%
\pgfpathlineto{\pgfqpoint{4.863399in}{3.775765in}}%
\pgfpathlineto{\pgfqpoint{4.880993in}{3.783944in}}%
\pgfpathlineto{\pgfqpoint{4.898587in}{3.789919in}}%
\pgfpathlineto{\pgfqpoint{4.916181in}{3.793703in}}%
\pgfpathlineto{\pgfqpoint{4.933775in}{3.795209in}}%
\pgfpathlineto{\pgfqpoint{4.951369in}{3.794201in}}%
\pgfpathlineto{\pgfqpoint{4.968962in}{3.790989in}}%
\pgfpathlineto{\pgfqpoint{4.986556in}{3.785288in}}%
\pgfpathlineto{\pgfqpoint{5.004150in}{3.777471in}}%
\pgfpathlineto{\pgfqpoint{5.021744in}{3.767338in}}%
\pgfpathlineto{\pgfqpoint{5.039338in}{3.755176in}}%
\pgfpathlineto{\pgfqpoint{5.056932in}{3.741185in}}%
\pgfpathlineto{\pgfqpoint{5.074526in}{3.725364in}}%
\pgfpathlineto{\pgfqpoint{5.109713in}{3.689240in}}%
\pgfpathlineto{\pgfqpoint{5.144901in}{3.648765in}}%
\pgfpathlineto{\pgfqpoint{5.250464in}{3.522997in}}%
\pgfpathlineto{\pgfqpoint{5.285652in}{3.487035in}}%
\pgfpathlineto{\pgfqpoint{5.303246in}{3.471388in}}%
\pgfpathlineto{\pgfqpoint{5.320839in}{3.457543in}}%
\pgfpathlineto{\pgfqpoint{5.338433in}{3.445764in}}%
\pgfpathlineto{\pgfqpoint{5.356027in}{3.436048in}}%
\pgfpathlineto{\pgfqpoint{5.373621in}{3.428763in}}%
\pgfpathlineto{\pgfqpoint{5.391215in}{3.423925in}}%
\pgfpathlineto{\pgfqpoint{5.408809in}{3.421556in}}%
\pgfpathlineto{\pgfqpoint{5.426403in}{3.421768in}}%
\pgfpathlineto{\pgfqpoint{5.443996in}{3.424576in}}%
\pgfpathlineto{\pgfqpoint{5.461590in}{3.429874in}}%
\pgfpathlineto{\pgfqpoint{5.479184in}{3.437780in}}%
\pgfpathlineto{\pgfqpoint{5.496778in}{3.447972in}}%
\pgfpathlineto{\pgfqpoint{5.514372in}{3.460608in}}%
\pgfpathlineto{\pgfqpoint{5.531966in}{3.475359in}}%
\pgfpathlineto{\pgfqpoint{5.549560in}{3.492076in}}%
\pgfpathlineto{\pgfqpoint{5.584747in}{3.531020in}}%
\pgfpathlineto{\pgfqpoint{5.619935in}{3.575739in}}%
\pgfpathlineto{\pgfqpoint{5.655123in}{3.624476in}}%
\pgfpathlineto{\pgfqpoint{5.795873in}{3.826382in}}%
\pgfpathlineto{\pgfqpoint{5.831061in}{3.871384in}}%
\pgfpathlineto{\pgfqpoint{5.866249in}{3.912164in}}%
\pgfpathlineto{\pgfqpoint{5.901437in}{3.948114in}}%
\pgfpathlineto{\pgfqpoint{5.936624in}{3.978673in}}%
\pgfpathlineto{\pgfqpoint{5.954218in}{3.991781in}}%
\pgfpathlineto{\pgfqpoint{5.971812in}{4.003457in}}%
\pgfpathlineto{\pgfqpoint{5.989406in}{4.013500in}}%
\pgfpathlineto{\pgfqpoint{6.007000in}{4.021911in}}%
\pgfpathlineto{\pgfqpoint{6.024594in}{4.028608in}}%
\pgfpathlineto{\pgfqpoint{6.042187in}{4.033581in}}%
\pgfpathlineto{\pgfqpoint{6.059781in}{4.036622in}}%
\pgfpathlineto{\pgfqpoint{6.077375in}{4.037656in}}%
\pgfpathlineto{\pgfqpoint{6.094969in}{4.036629in}}%
\pgfpathlineto{\pgfqpoint{6.112563in}{4.033447in}}%
\pgfpathlineto{\pgfqpoint{6.130157in}{4.027995in}}%
\pgfpathlineto{\pgfqpoint{6.147751in}{4.020281in}}%
\pgfpathlineto{\pgfqpoint{6.165344in}{4.010293in}}%
\pgfpathlineto{\pgfqpoint{6.182938in}{3.997993in}}%
\pgfpathlineto{\pgfqpoint{6.200532in}{3.983634in}}%
\pgfpathlineto{\pgfqpoint{6.218126in}{3.967265in}}%
\pgfpathlineto{\pgfqpoint{6.253314in}{3.930234in}}%
\pgfpathlineto{\pgfqpoint{6.288501in}{3.891520in}}%
\pgfpathlineto{\pgfqpoint{6.306095in}{3.873875in}}%
\pgfpathlineto{\pgfqpoint{6.323689in}{3.859116in}}%
\pgfpathlineto{\pgfqpoint{6.341283in}{3.849078in}}%
\pgfpathlineto{\pgfqpoint{6.358877in}{3.845786in}}%
\pgfpathlineto{\pgfqpoint{6.376471in}{3.851849in}}%
\pgfpathlineto{\pgfqpoint{6.376471in}{3.851849in}}%
\pgfusepath{stroke}%
\end{pgfscope}%
\begin{pgfscope}%
\pgfsetrectcap%
\pgfsetmiterjoin%
\pgfsetlinewidth{1.003750pt}%
\definecolor{currentstroke}{rgb}{0.800000,0.800000,0.800000}%
\pgfsetstrokecolor{currentstroke}%
\pgfsetdash{}{0pt}%
\pgfpathmoveto{\pgfqpoint{2.000000in}{3.197368in}}%
\pgfpathlineto{\pgfqpoint{2.000000in}{4.176316in}}%
\pgfusepath{stroke}%
\end{pgfscope}%
\begin{pgfscope}%
\pgfsetrectcap%
\pgfsetmiterjoin%
\pgfsetlinewidth{1.003750pt}%
\definecolor{currentstroke}{rgb}{0.800000,0.800000,0.800000}%
\pgfsetstrokecolor{currentstroke}%
\pgfsetdash{}{0pt}%
\pgfpathmoveto{\pgfqpoint{6.376471in}{3.197368in}}%
\pgfpathlineto{\pgfqpoint{6.376471in}{4.176316in}}%
\pgfusepath{stroke}%
\end{pgfscope}%
\begin{pgfscope}%
\pgfsetrectcap%
\pgfsetmiterjoin%
\pgfsetlinewidth{1.003750pt}%
\definecolor{currentstroke}{rgb}{0.800000,0.800000,0.800000}%
\pgfsetstrokecolor{currentstroke}%
\pgfsetdash{}{0pt}%
\pgfpathmoveto{\pgfqpoint{2.000000in}{4.176316in}}%
\pgfpathlineto{\pgfqpoint{6.376471in}{4.176316in}}%
\pgfusepath{stroke}%
\end{pgfscope}%
\begin{pgfscope}%
\pgfsetrectcap%
\pgfsetmiterjoin%
\pgfsetlinewidth{1.003750pt}%
\definecolor{currentstroke}{rgb}{0.800000,0.800000,0.800000}%
\pgfsetstrokecolor{currentstroke}%
\pgfsetdash{}{0pt}%
\pgfpathmoveto{\pgfqpoint{2.000000in}{3.197368in}}%
\pgfpathlineto{\pgfqpoint{6.376471in}{3.197368in}}%
\pgfusepath{stroke}%
\end{pgfscope}%
\begin{pgfscope}%
\pgfsetroundcap%
\pgfsetroundjoin%
\pgfsetlinewidth{1.756562pt}%
\definecolor{currentstroke}{rgb}{0.298039,0.447059,0.690196}%
\pgfsetstrokecolor{currentstroke}%
\pgfsetdash{}{0pt}%
\pgfpathmoveto{\pgfqpoint{2.125000in}{3.996372in}}%
\pgfpathlineto{\pgfqpoint{2.402778in}{3.996372in}}%
\pgfusepath{stroke}%
\end{pgfscope}%
\begin{pgfscope}%
\definecolor{textcolor}{rgb}{0.150000,0.150000,0.150000}%
\pgfsetstrokecolor{textcolor}%
\pgfsetfillcolor{textcolor}%
\pgftext[x=2.513889in,y=3.947761in,left,base]{\color{textcolor}\sffamily\fontsize{10.000000}{12.000000}\selectfont \(\displaystyle \widetilde{\Phi}^* \theta\)}%
\end{pgfscope}%
\begin{pgfscope}%
\pgfsetbuttcap%
\pgfsetroundjoin%
\pgfsetlinewidth{1.756562pt}%
\definecolor{currentstroke}{rgb}{1.000000,0.647059,0.000000}%
\pgfsetstrokecolor{currentstroke}%
\pgfsetdash{{6.000000pt}{6.000000pt}}{0.000000pt}%
\pgfpathmoveto{\pgfqpoint{2.125000in}{3.791511in}}%
\pgfpathlineto{\pgfqpoint{2.402778in}{3.791511in}}%
\pgfusepath{stroke}%
\end{pgfscope}%
\begin{pgfscope}%
\definecolor{textcolor}{rgb}{0.150000,0.150000,0.150000}%
\pgfsetstrokecolor{textcolor}%
\pgfsetfillcolor{textcolor}%
\pgftext[x=2.513889in,y=3.742900in,left,base]{\color{textcolor}\sffamily\fontsize{10.000000}{12.000000}\selectfont \(\displaystyle \widetilde{K}u\)}%
\end{pgfscope}%
\begin{pgfscope}%
\pgfsetbuttcap%
\pgfsetroundjoin%
\definecolor{currentfill}{rgb}{1.000000,0.000000,0.000000}%
\pgfsetfillcolor{currentfill}%
\pgfsetlinewidth{2.007500pt}%
\definecolor{currentstroke}{rgb}{1.000000,0.000000,0.000000}%
\pgfsetstrokecolor{currentstroke}%
\pgfsetdash{}{0pt}%
\pgfpathmoveto{\pgfqpoint{2.232832in}{3.582893in}}%
\pgfpathlineto{\pgfqpoint{2.294945in}{3.582893in}}%
\pgfpathmoveto{\pgfqpoint{2.263889in}{3.551837in}}%
\pgfpathlineto{\pgfqpoint{2.263889in}{3.613950in}}%
\pgfusepath{stroke,fill}%
\end{pgfscope}%
\begin{pgfscope}%
\pgfsetbuttcap%
\pgfsetroundjoin%
\definecolor{currentfill}{rgb}{1.000000,0.000000,0.000000}%
\pgfsetfillcolor{currentfill}%
\pgfsetlinewidth{2.007500pt}%
\definecolor{currentstroke}{rgb}{1.000000,0.000000,0.000000}%
\pgfsetstrokecolor{currentstroke}%
\pgfsetdash{}{0pt}%
\pgfpathmoveto{\pgfqpoint{2.232832in}{3.582893in}}%
\pgfpathlineto{\pgfqpoint{2.294945in}{3.582893in}}%
\pgfpathmoveto{\pgfqpoint{2.263889in}{3.551837in}}%
\pgfpathlineto{\pgfqpoint{2.263889in}{3.613950in}}%
\pgfusepath{stroke,fill}%
\end{pgfscope}%
\begin{pgfscope}%
\pgfsetbuttcap%
\pgfsetroundjoin%
\definecolor{currentfill}{rgb}{1.000000,0.000000,0.000000}%
\pgfsetfillcolor{currentfill}%
\pgfsetlinewidth{2.007500pt}%
\definecolor{currentstroke}{rgb}{1.000000,0.000000,0.000000}%
\pgfsetstrokecolor{currentstroke}%
\pgfsetdash{}{0pt}%
\pgfpathmoveto{\pgfqpoint{2.232832in}{3.582893in}}%
\pgfpathlineto{\pgfqpoint{2.294945in}{3.582893in}}%
\pgfpathmoveto{\pgfqpoint{2.263889in}{3.551837in}}%
\pgfpathlineto{\pgfqpoint{2.263889in}{3.613950in}}%
\pgfusepath{stroke,fill}%
\end{pgfscope}%
\begin{pgfscope}%
\definecolor{textcolor}{rgb}{0.150000,0.150000,0.150000}%
\pgfsetstrokecolor{textcolor}%
\pgfsetfillcolor{textcolor}%
\pgftext[x=2.513889in,y=3.546435in,left,base]{\color{textcolor}\sffamily\fontsize{10.000000}{12.000000}\selectfont train}%
\end{pgfscope}%
\begin{pgfscope}%
\pgfsetbuttcap%
\pgfsetroundjoin%
\definecolor{currentfill}{rgb}{0.000000,0.000000,0.000000}%
\pgfsetfillcolor{currentfill}%
\pgfsetlinewidth{0.301125pt}%
\definecolor{currentstroke}{rgb}{0.000000,0.000000,0.000000}%
\pgfsetstrokecolor{currentstroke}%
\pgfsetdash{}{0pt}%
\pgfpathmoveto{\pgfqpoint{2.263889in}{3.370900in}}%
\pgfpathcurveto{\pgfqpoint{2.268007in}{3.370900in}}{\pgfqpoint{2.271957in}{3.372536in}}{\pgfqpoint{2.274869in}{3.375448in}}%
\pgfpathcurveto{\pgfqpoint{2.277781in}{3.378360in}}{\pgfqpoint{2.279417in}{3.382310in}}{\pgfqpoint{2.279417in}{3.386428in}}%
\pgfpathcurveto{\pgfqpoint{2.279417in}{3.390546in}}{\pgfqpoint{2.277781in}{3.394496in}}{\pgfqpoint{2.274869in}{3.397408in}}%
\pgfpathcurveto{\pgfqpoint{2.271957in}{3.400320in}}{\pgfqpoint{2.268007in}{3.401956in}}{\pgfqpoint{2.263889in}{3.401956in}}%
\pgfpathcurveto{\pgfqpoint{2.259771in}{3.401956in}}{\pgfqpoint{2.255821in}{3.400320in}}{\pgfqpoint{2.252909in}{3.397408in}}%
\pgfpathcurveto{\pgfqpoint{2.249997in}{3.394496in}}{\pgfqpoint{2.248361in}{3.390546in}}{\pgfqpoint{2.248361in}{3.386428in}}%
\pgfpathcurveto{\pgfqpoint{2.248361in}{3.382310in}}{\pgfqpoint{2.249997in}{3.378360in}}{\pgfqpoint{2.252909in}{3.375448in}}%
\pgfpathcurveto{\pgfqpoint{2.255821in}{3.372536in}}{\pgfqpoint{2.259771in}{3.370900in}}{\pgfqpoint{2.263889in}{3.370900in}}%
\pgfpathclose%
\pgfusepath{stroke,fill}%
\end{pgfscope}%
\begin{pgfscope}%
\pgfsetbuttcap%
\pgfsetroundjoin%
\definecolor{currentfill}{rgb}{0.000000,0.000000,0.000000}%
\pgfsetfillcolor{currentfill}%
\pgfsetlinewidth{0.301125pt}%
\definecolor{currentstroke}{rgb}{0.000000,0.000000,0.000000}%
\pgfsetstrokecolor{currentstroke}%
\pgfsetdash{}{0pt}%
\pgfpathmoveto{\pgfqpoint{2.263889in}{3.370900in}}%
\pgfpathcurveto{\pgfqpoint{2.268007in}{3.370900in}}{\pgfqpoint{2.271957in}{3.372536in}}{\pgfqpoint{2.274869in}{3.375448in}}%
\pgfpathcurveto{\pgfqpoint{2.277781in}{3.378360in}}{\pgfqpoint{2.279417in}{3.382310in}}{\pgfqpoint{2.279417in}{3.386428in}}%
\pgfpathcurveto{\pgfqpoint{2.279417in}{3.390546in}}{\pgfqpoint{2.277781in}{3.394496in}}{\pgfqpoint{2.274869in}{3.397408in}}%
\pgfpathcurveto{\pgfqpoint{2.271957in}{3.400320in}}{\pgfqpoint{2.268007in}{3.401956in}}{\pgfqpoint{2.263889in}{3.401956in}}%
\pgfpathcurveto{\pgfqpoint{2.259771in}{3.401956in}}{\pgfqpoint{2.255821in}{3.400320in}}{\pgfqpoint{2.252909in}{3.397408in}}%
\pgfpathcurveto{\pgfqpoint{2.249997in}{3.394496in}}{\pgfqpoint{2.248361in}{3.390546in}}{\pgfqpoint{2.248361in}{3.386428in}}%
\pgfpathcurveto{\pgfqpoint{2.248361in}{3.382310in}}{\pgfqpoint{2.249997in}{3.378360in}}{\pgfqpoint{2.252909in}{3.375448in}}%
\pgfpathcurveto{\pgfqpoint{2.255821in}{3.372536in}}{\pgfqpoint{2.259771in}{3.370900in}}{\pgfqpoint{2.263889in}{3.370900in}}%
\pgfpathclose%
\pgfusepath{stroke,fill}%
\end{pgfscope}%
\begin{pgfscope}%
\pgfsetbuttcap%
\pgfsetroundjoin%
\definecolor{currentfill}{rgb}{0.000000,0.000000,0.000000}%
\pgfsetfillcolor{currentfill}%
\pgfsetlinewidth{0.301125pt}%
\definecolor{currentstroke}{rgb}{0.000000,0.000000,0.000000}%
\pgfsetstrokecolor{currentstroke}%
\pgfsetdash{}{0pt}%
\pgfpathmoveto{\pgfqpoint{2.263889in}{3.370900in}}%
\pgfpathcurveto{\pgfqpoint{2.268007in}{3.370900in}}{\pgfqpoint{2.271957in}{3.372536in}}{\pgfqpoint{2.274869in}{3.375448in}}%
\pgfpathcurveto{\pgfqpoint{2.277781in}{3.378360in}}{\pgfqpoint{2.279417in}{3.382310in}}{\pgfqpoint{2.279417in}{3.386428in}}%
\pgfpathcurveto{\pgfqpoint{2.279417in}{3.390546in}}{\pgfqpoint{2.277781in}{3.394496in}}{\pgfqpoint{2.274869in}{3.397408in}}%
\pgfpathcurveto{\pgfqpoint{2.271957in}{3.400320in}}{\pgfqpoint{2.268007in}{3.401956in}}{\pgfqpoint{2.263889in}{3.401956in}}%
\pgfpathcurveto{\pgfqpoint{2.259771in}{3.401956in}}{\pgfqpoint{2.255821in}{3.400320in}}{\pgfqpoint{2.252909in}{3.397408in}}%
\pgfpathcurveto{\pgfqpoint{2.249997in}{3.394496in}}{\pgfqpoint{2.248361in}{3.390546in}}{\pgfqpoint{2.248361in}{3.386428in}}%
\pgfpathcurveto{\pgfqpoint{2.248361in}{3.382310in}}{\pgfqpoint{2.249997in}{3.378360in}}{\pgfqpoint{2.252909in}{3.375448in}}%
\pgfpathcurveto{\pgfqpoint{2.255821in}{3.372536in}}{\pgfqpoint{2.259771in}{3.370900in}}{\pgfqpoint{2.263889in}{3.370900in}}%
\pgfpathclose%
\pgfusepath{stroke,fill}%
\end{pgfscope}%
\begin{pgfscope}%
\definecolor{textcolor}{rgb}{0.150000,0.150000,0.150000}%
\pgfsetstrokecolor{textcolor}%
\pgfsetfillcolor{textcolor}%
\pgftext[x=2.513889in,y=3.349970in,left,base]{\color{textcolor}\sffamily\fontsize{10.000000}{12.000000}\selectfont test}%
\end{pgfscope}%
\begin{pgfscope}%
\pgfsetbuttcap%
\pgfsetmiterjoin%
\definecolor{currentfill}{rgb}{1.000000,1.000000,1.000000}%
\pgfsetfillcolor{currentfill}%
\pgfsetlinewidth{0.000000pt}%
\definecolor{currentstroke}{rgb}{0.000000,0.000000,0.000000}%
\pgfsetstrokecolor{currentstroke}%
\pgfsetstrokeopacity{0.000000}%
\pgfsetdash{}{0pt}%
\pgfpathmoveto{\pgfqpoint{7.105882in}{3.197368in}}%
\pgfpathlineto{\pgfqpoint{11.482353in}{3.197368in}}%
\pgfpathlineto{\pgfqpoint{11.482353in}{4.176316in}}%
\pgfpathlineto{\pgfqpoint{7.105882in}{4.176316in}}%
\pgfpathclose%
\pgfusepath{fill}%
\end{pgfscope}%
\begin{pgfscope}%
\pgfpathrectangle{\pgfqpoint{7.105882in}{3.197368in}}{\pgfqpoint{4.376471in}{0.978947in}} %
\pgfusepath{clip}%
\pgfsetroundcap%
\pgfsetroundjoin%
\pgfsetlinewidth{1.003750pt}%
\definecolor{currentstroke}{rgb}{0.800000,0.800000,0.800000}%
\pgfsetstrokecolor{currentstroke}%
\pgfsetdash{}{0pt}%
\pgfpathmoveto{\pgfqpoint{7.105882in}{3.197368in}}%
\pgfpathlineto{\pgfqpoint{7.105882in}{4.176316in}}%
\pgfusepath{stroke}%
\end{pgfscope}%
\begin{pgfscope}%
\pgfpathrectangle{\pgfqpoint{7.105882in}{3.197368in}}{\pgfqpoint{4.376471in}{0.978947in}} %
\pgfusepath{clip}%
\pgfsetroundcap%
\pgfsetroundjoin%
\pgfsetlinewidth{1.003750pt}%
\definecolor{currentstroke}{rgb}{0.800000,0.800000,0.800000}%
\pgfsetstrokecolor{currentstroke}%
\pgfsetdash{}{0pt}%
\pgfpathmoveto{\pgfqpoint{7.981176in}{3.197368in}}%
\pgfpathlineto{\pgfqpoint{7.981176in}{4.176316in}}%
\pgfusepath{stroke}%
\end{pgfscope}%
\begin{pgfscope}%
\pgfpathrectangle{\pgfqpoint{7.105882in}{3.197368in}}{\pgfqpoint{4.376471in}{0.978947in}} %
\pgfusepath{clip}%
\pgfsetroundcap%
\pgfsetroundjoin%
\pgfsetlinewidth{1.003750pt}%
\definecolor{currentstroke}{rgb}{0.800000,0.800000,0.800000}%
\pgfsetstrokecolor{currentstroke}%
\pgfsetdash{}{0pt}%
\pgfpathmoveto{\pgfqpoint{8.856471in}{3.197368in}}%
\pgfpathlineto{\pgfqpoint{8.856471in}{4.176316in}}%
\pgfusepath{stroke}%
\end{pgfscope}%
\begin{pgfscope}%
\pgfpathrectangle{\pgfqpoint{7.105882in}{3.197368in}}{\pgfqpoint{4.376471in}{0.978947in}} %
\pgfusepath{clip}%
\pgfsetroundcap%
\pgfsetroundjoin%
\pgfsetlinewidth{1.003750pt}%
\definecolor{currentstroke}{rgb}{0.800000,0.800000,0.800000}%
\pgfsetstrokecolor{currentstroke}%
\pgfsetdash{}{0pt}%
\pgfpathmoveto{\pgfqpoint{9.731765in}{3.197368in}}%
\pgfpathlineto{\pgfqpoint{9.731765in}{4.176316in}}%
\pgfusepath{stroke}%
\end{pgfscope}%
\begin{pgfscope}%
\pgfpathrectangle{\pgfqpoint{7.105882in}{3.197368in}}{\pgfqpoint{4.376471in}{0.978947in}} %
\pgfusepath{clip}%
\pgfsetroundcap%
\pgfsetroundjoin%
\pgfsetlinewidth{1.003750pt}%
\definecolor{currentstroke}{rgb}{0.800000,0.800000,0.800000}%
\pgfsetstrokecolor{currentstroke}%
\pgfsetdash{}{0pt}%
\pgfpathmoveto{\pgfqpoint{10.607059in}{3.197368in}}%
\pgfpathlineto{\pgfqpoint{10.607059in}{4.176316in}}%
\pgfusepath{stroke}%
\end{pgfscope}%
\begin{pgfscope}%
\pgfpathrectangle{\pgfqpoint{7.105882in}{3.197368in}}{\pgfqpoint{4.376471in}{0.978947in}} %
\pgfusepath{clip}%
\pgfsetroundcap%
\pgfsetroundjoin%
\pgfsetlinewidth{1.003750pt}%
\definecolor{currentstroke}{rgb}{0.800000,0.800000,0.800000}%
\pgfsetstrokecolor{currentstroke}%
\pgfsetdash{}{0pt}%
\pgfpathmoveto{\pgfqpoint{11.482353in}{3.197368in}}%
\pgfpathlineto{\pgfqpoint{11.482353in}{4.176316in}}%
\pgfusepath{stroke}%
\end{pgfscope}%
\begin{pgfscope}%
\pgfpathrectangle{\pgfqpoint{7.105882in}{3.197368in}}{\pgfqpoint{4.376471in}{0.978947in}} %
\pgfusepath{clip}%
\pgfsetroundcap%
\pgfsetroundjoin%
\pgfsetlinewidth{1.003750pt}%
\definecolor{currentstroke}{rgb}{0.800000,0.800000,0.800000}%
\pgfsetstrokecolor{currentstroke}%
\pgfsetdash{}{0pt}%
\pgfpathmoveto{\pgfqpoint{7.105882in}{3.360526in}}%
\pgfpathlineto{\pgfqpoint{11.482353in}{3.360526in}}%
\pgfusepath{stroke}%
\end{pgfscope}%
\begin{pgfscope}%
\definecolor{textcolor}{rgb}{0.150000,0.150000,0.150000}%
\pgfsetstrokecolor{textcolor}%
\pgfsetfillcolor{textcolor}%
\pgftext[x=7.008660in,y=3.360526in,right,]{\color{textcolor}\sffamily\fontsize{10.000000}{12.000000}\selectfont \(\displaystyle -1\)}%
\end{pgfscope}%
\begin{pgfscope}%
\pgfpathrectangle{\pgfqpoint{7.105882in}{3.197368in}}{\pgfqpoint{4.376471in}{0.978947in}} %
\pgfusepath{clip}%
\pgfsetroundcap%
\pgfsetroundjoin%
\pgfsetlinewidth{1.003750pt}%
\definecolor{currentstroke}{rgb}{0.800000,0.800000,0.800000}%
\pgfsetstrokecolor{currentstroke}%
\pgfsetdash{}{0pt}%
\pgfpathmoveto{\pgfqpoint{7.105882in}{3.564474in}}%
\pgfpathlineto{\pgfqpoint{11.482353in}{3.564474in}}%
\pgfusepath{stroke}%
\end{pgfscope}%
\begin{pgfscope}%
\definecolor{textcolor}{rgb}{0.150000,0.150000,0.150000}%
\pgfsetstrokecolor{textcolor}%
\pgfsetfillcolor{textcolor}%
\pgftext[x=7.008660in,y=3.564474in,right,]{\color{textcolor}\sffamily\fontsize{10.000000}{12.000000}\selectfont \(\displaystyle 0\)}%
\end{pgfscope}%
\begin{pgfscope}%
\pgfpathrectangle{\pgfqpoint{7.105882in}{3.197368in}}{\pgfqpoint{4.376471in}{0.978947in}} %
\pgfusepath{clip}%
\pgfsetroundcap%
\pgfsetroundjoin%
\pgfsetlinewidth{1.003750pt}%
\definecolor{currentstroke}{rgb}{0.800000,0.800000,0.800000}%
\pgfsetstrokecolor{currentstroke}%
\pgfsetdash{}{0pt}%
\pgfpathmoveto{\pgfqpoint{7.105882in}{3.768421in}}%
\pgfpathlineto{\pgfqpoint{11.482353in}{3.768421in}}%
\pgfusepath{stroke}%
\end{pgfscope}%
\begin{pgfscope}%
\definecolor{textcolor}{rgb}{0.150000,0.150000,0.150000}%
\pgfsetstrokecolor{textcolor}%
\pgfsetfillcolor{textcolor}%
\pgftext[x=7.008660in,y=3.768421in,right,]{\color{textcolor}\sffamily\fontsize{10.000000}{12.000000}\selectfont \(\displaystyle 1\)}%
\end{pgfscope}%
\begin{pgfscope}%
\pgfpathrectangle{\pgfqpoint{7.105882in}{3.197368in}}{\pgfqpoint{4.376471in}{0.978947in}} %
\pgfusepath{clip}%
\pgfsetroundcap%
\pgfsetroundjoin%
\pgfsetlinewidth{1.003750pt}%
\definecolor{currentstroke}{rgb}{0.800000,0.800000,0.800000}%
\pgfsetstrokecolor{currentstroke}%
\pgfsetdash{}{0pt}%
\pgfpathmoveto{\pgfqpoint{7.105882in}{3.972368in}}%
\pgfpathlineto{\pgfqpoint{11.482353in}{3.972368in}}%
\pgfusepath{stroke}%
\end{pgfscope}%
\begin{pgfscope}%
\definecolor{textcolor}{rgb}{0.150000,0.150000,0.150000}%
\pgfsetstrokecolor{textcolor}%
\pgfsetfillcolor{textcolor}%
\pgftext[x=7.008660in,y=3.972368in,right,]{\color{textcolor}\sffamily\fontsize{10.000000}{12.000000}\selectfont \(\displaystyle 2\)}%
\end{pgfscope}%
\begin{pgfscope}%
\pgfpathrectangle{\pgfqpoint{7.105882in}{3.197368in}}{\pgfqpoint{4.376471in}{0.978947in}} %
\pgfusepath{clip}%
\pgfsetroundcap%
\pgfsetroundjoin%
\pgfsetlinewidth{1.003750pt}%
\definecolor{currentstroke}{rgb}{0.800000,0.800000,0.800000}%
\pgfsetstrokecolor{currentstroke}%
\pgfsetdash{}{0pt}%
\pgfpathmoveto{\pgfqpoint{7.105882in}{4.176316in}}%
\pgfpathlineto{\pgfqpoint{11.482353in}{4.176316in}}%
\pgfusepath{stroke}%
\end{pgfscope}%
\begin{pgfscope}%
\definecolor{textcolor}{rgb}{0.150000,0.150000,0.150000}%
\pgfsetstrokecolor{textcolor}%
\pgfsetfillcolor{textcolor}%
\pgftext[x=7.008660in,y=4.176316in,right,]{\color{textcolor}\sffamily\fontsize{10.000000}{12.000000}\selectfont \(\displaystyle 3\)}%
\end{pgfscope}%
\begin{pgfscope}%
\pgfpathrectangle{\pgfqpoint{7.105882in}{3.197368in}}{\pgfqpoint{4.376471in}{0.978947in}} %
\pgfusepath{clip}%
\pgfsetbuttcap%
\pgfsetroundjoin%
\definecolor{currentfill}{rgb}{1.000000,0.000000,0.000000}%
\pgfsetfillcolor{currentfill}%
\pgfsetlinewidth{2.007500pt}%
\definecolor{currentstroke}{rgb}{1.000000,0.000000,0.000000}%
\pgfsetstrokecolor{currentstroke}%
\pgfsetdash{}{0pt}%
\pgfpathmoveto{\pgfqpoint{9.871613in}{3.754944in}}%
\pgfpathlineto{\pgfqpoint{9.933726in}{3.754944in}}%
\pgfpathmoveto{\pgfqpoint{9.902669in}{3.723888in}}%
\pgfpathlineto{\pgfqpoint{9.902669in}{3.786001in}}%
\pgfusepath{stroke,fill}%
\end{pgfscope}%
\begin{pgfscope}%
\pgfpathrectangle{\pgfqpoint{7.105882in}{3.197368in}}{\pgfqpoint{4.376471in}{0.978947in}} %
\pgfusepath{clip}%
\pgfsetbuttcap%
\pgfsetroundjoin%
\definecolor{currentfill}{rgb}{1.000000,0.000000,0.000000}%
\pgfsetfillcolor{currentfill}%
\pgfsetlinewidth{2.007500pt}%
\definecolor{currentstroke}{rgb}{1.000000,0.000000,0.000000}%
\pgfsetstrokecolor{currentstroke}%
\pgfsetdash{}{0pt}%
\pgfpathmoveto{\pgfqpoint{10.454124in}{3.437732in}}%
\pgfpathlineto{\pgfqpoint{10.516237in}{3.437732in}}%
\pgfpathmoveto{\pgfqpoint{10.485181in}{3.406676in}}%
\pgfpathlineto{\pgfqpoint{10.485181in}{3.468789in}}%
\pgfusepath{stroke,fill}%
\end{pgfscope}%
\begin{pgfscope}%
\pgfpathrectangle{\pgfqpoint{7.105882in}{3.197368in}}{\pgfqpoint{4.376471in}{0.978947in}} %
\pgfusepath{clip}%
\pgfsetbuttcap%
\pgfsetroundjoin%
\definecolor{currentfill}{rgb}{1.000000,0.000000,0.000000}%
\pgfsetfillcolor{currentfill}%
\pgfsetlinewidth{2.007500pt}%
\definecolor{currentstroke}{rgb}{1.000000,0.000000,0.000000}%
\pgfsetstrokecolor{currentstroke}%
\pgfsetdash{}{0pt}%
\pgfpathmoveto{\pgfqpoint{10.060501in}{3.808472in}}%
\pgfpathlineto{\pgfqpoint{10.122614in}{3.808472in}}%
\pgfpathmoveto{\pgfqpoint{10.091557in}{3.777416in}}%
\pgfpathlineto{\pgfqpoint{10.091557in}{3.839529in}}%
\pgfusepath{stroke,fill}%
\end{pgfscope}%
\begin{pgfscope}%
\pgfpathrectangle{\pgfqpoint{7.105882in}{3.197368in}}{\pgfqpoint{4.376471in}{0.978947in}} %
\pgfusepath{clip}%
\pgfsetbuttcap%
\pgfsetroundjoin%
\definecolor{currentfill}{rgb}{1.000000,0.000000,0.000000}%
\pgfsetfillcolor{currentfill}%
\pgfsetlinewidth{2.007500pt}%
\definecolor{currentstroke}{rgb}{1.000000,0.000000,0.000000}%
\pgfsetstrokecolor{currentstroke}%
\pgfsetdash{}{0pt}%
\pgfpathmoveto{\pgfqpoint{9.857852in}{3.697560in}}%
\pgfpathlineto{\pgfqpoint{9.919965in}{3.697560in}}%
\pgfpathmoveto{\pgfqpoint{9.888909in}{3.666503in}}%
\pgfpathlineto{\pgfqpoint{9.888909in}{3.728616in}}%
\pgfusepath{stroke,fill}%
\end{pgfscope}%
\begin{pgfscope}%
\pgfpathrectangle{\pgfqpoint{7.105882in}{3.197368in}}{\pgfqpoint{4.376471in}{0.978947in}} %
\pgfusepath{clip}%
\pgfsetbuttcap%
\pgfsetroundjoin%
\definecolor{currentfill}{rgb}{1.000000,0.000000,0.000000}%
\pgfsetfillcolor{currentfill}%
\pgfsetlinewidth{2.007500pt}%
\definecolor{currentstroke}{rgb}{1.000000,0.000000,0.000000}%
\pgfsetstrokecolor{currentstroke}%
\pgfsetdash{}{0pt}%
\pgfpathmoveto{\pgfqpoint{9.433410in}{3.371166in}}%
\pgfpathlineto{\pgfqpoint{9.495523in}{3.371166in}}%
\pgfpathmoveto{\pgfqpoint{9.464467in}{3.340109in}}%
\pgfpathlineto{\pgfqpoint{9.464467in}{3.402222in}}%
\pgfusepath{stroke,fill}%
\end{pgfscope}%
\begin{pgfscope}%
\pgfpathrectangle{\pgfqpoint{7.105882in}{3.197368in}}{\pgfqpoint{4.376471in}{0.978947in}} %
\pgfusepath{clip}%
\pgfsetbuttcap%
\pgfsetroundjoin%
\definecolor{currentfill}{rgb}{1.000000,0.000000,0.000000}%
\pgfsetfillcolor{currentfill}%
\pgfsetlinewidth{2.007500pt}%
\definecolor{currentstroke}{rgb}{1.000000,0.000000,0.000000}%
\pgfsetstrokecolor{currentstroke}%
\pgfsetdash{}{0pt}%
\pgfpathmoveto{\pgfqpoint{10.211509in}{3.640630in}}%
\pgfpathlineto{\pgfqpoint{10.273622in}{3.640630in}}%
\pgfpathmoveto{\pgfqpoint{10.242566in}{3.609573in}}%
\pgfpathlineto{\pgfqpoint{10.242566in}{3.671686in}}%
\pgfusepath{stroke,fill}%
\end{pgfscope}%
\begin{pgfscope}%
\pgfpathrectangle{\pgfqpoint{7.105882in}{3.197368in}}{\pgfqpoint{4.376471in}{0.978947in}} %
\pgfusepath{clip}%
\pgfsetbuttcap%
\pgfsetroundjoin%
\definecolor{currentfill}{rgb}{1.000000,0.000000,0.000000}%
\pgfsetfillcolor{currentfill}%
\pgfsetlinewidth{2.007500pt}%
\definecolor{currentstroke}{rgb}{1.000000,0.000000,0.000000}%
\pgfsetstrokecolor{currentstroke}%
\pgfsetdash{}{0pt}%
\pgfpathmoveto{\pgfqpoint{9.482190in}{3.408659in}}%
\pgfpathlineto{\pgfqpoint{9.544303in}{3.408659in}}%
\pgfpathmoveto{\pgfqpoint{9.513247in}{3.377603in}}%
\pgfpathlineto{\pgfqpoint{9.513247in}{3.439716in}}%
\pgfusepath{stroke,fill}%
\end{pgfscope}%
\begin{pgfscope}%
\pgfpathrectangle{\pgfqpoint{7.105882in}{3.197368in}}{\pgfqpoint{4.376471in}{0.978947in}} %
\pgfusepath{clip}%
\pgfsetbuttcap%
\pgfsetroundjoin%
\definecolor{currentfill}{rgb}{1.000000,0.000000,0.000000}%
\pgfsetfillcolor{currentfill}%
\pgfsetlinewidth{2.007500pt}%
\definecolor{currentstroke}{rgb}{1.000000,0.000000,0.000000}%
\pgfsetstrokecolor{currentstroke}%
\pgfsetdash{}{0pt}%
\pgfpathmoveto{\pgfqpoint{11.072375in}{4.048778in}}%
\pgfpathlineto{\pgfqpoint{11.134488in}{4.048778in}}%
\pgfpathmoveto{\pgfqpoint{11.103431in}{4.017721in}}%
\pgfpathlineto{\pgfqpoint{11.103431in}{4.079834in}}%
\pgfusepath{stroke,fill}%
\end{pgfscope}%
\begin{pgfscope}%
\pgfpathrectangle{\pgfqpoint{7.105882in}{3.197368in}}{\pgfqpoint{4.376471in}{0.978947in}} %
\pgfusepath{clip}%
\pgfsetbuttcap%
\pgfsetroundjoin%
\definecolor{currentfill}{rgb}{1.000000,0.000000,0.000000}%
\pgfsetfillcolor{currentfill}%
\pgfsetlinewidth{2.007500pt}%
\definecolor{currentstroke}{rgb}{1.000000,0.000000,0.000000}%
\pgfsetstrokecolor{currentstroke}%
\pgfsetdash{}{0pt}%
\pgfpathmoveto{\pgfqpoint{11.324073in}{3.949180in}}%
\pgfpathlineto{\pgfqpoint{11.386186in}{3.949180in}}%
\pgfpathmoveto{\pgfqpoint{11.355130in}{3.918124in}}%
\pgfpathlineto{\pgfqpoint{11.355130in}{3.980237in}}%
\pgfusepath{stroke,fill}%
\end{pgfscope}%
\begin{pgfscope}%
\pgfpathrectangle{\pgfqpoint{7.105882in}{3.197368in}}{\pgfqpoint{4.376471in}{0.978947in}} %
\pgfusepath{clip}%
\pgfsetbuttcap%
\pgfsetroundjoin%
\definecolor{currentfill}{rgb}{1.000000,0.000000,0.000000}%
\pgfsetfillcolor{currentfill}%
\pgfsetlinewidth{2.007500pt}%
\definecolor{currentstroke}{rgb}{1.000000,0.000000,0.000000}%
\pgfsetstrokecolor{currentstroke}%
\pgfsetdash{}{0pt}%
\pgfpathmoveto{\pgfqpoint{9.292616in}{3.446577in}}%
\pgfpathlineto{\pgfqpoint{9.354729in}{3.446577in}}%
\pgfpathmoveto{\pgfqpoint{9.323673in}{3.415520in}}%
\pgfpathlineto{\pgfqpoint{9.323673in}{3.477633in}}%
\pgfusepath{stroke,fill}%
\end{pgfscope}%
\begin{pgfscope}%
\pgfpathrectangle{\pgfqpoint{7.105882in}{3.197368in}}{\pgfqpoint{4.376471in}{0.978947in}} %
\pgfusepath{clip}%
\pgfsetbuttcap%
\pgfsetroundjoin%
\definecolor{currentfill}{rgb}{1.000000,0.000000,0.000000}%
\pgfsetfillcolor{currentfill}%
\pgfsetlinewidth{2.007500pt}%
\definecolor{currentstroke}{rgb}{1.000000,0.000000,0.000000}%
\pgfsetstrokecolor{currentstroke}%
\pgfsetdash{}{0pt}%
\pgfpathmoveto{\pgfqpoint{10.722089in}{3.596752in}}%
\pgfpathlineto{\pgfqpoint{10.784202in}{3.596752in}}%
\pgfpathmoveto{\pgfqpoint{10.753146in}{3.565696in}}%
\pgfpathlineto{\pgfqpoint{10.753146in}{3.627809in}}%
\pgfusepath{stroke,fill}%
\end{pgfscope}%
\begin{pgfscope}%
\pgfpathrectangle{\pgfqpoint{7.105882in}{3.197368in}}{\pgfqpoint{4.376471in}{0.978947in}} %
\pgfusepath{clip}%
\pgfsetbuttcap%
\pgfsetroundjoin%
\definecolor{currentfill}{rgb}{1.000000,0.000000,0.000000}%
\pgfsetfillcolor{currentfill}%
\pgfsetlinewidth{2.007500pt}%
\definecolor{currentstroke}{rgb}{1.000000,0.000000,0.000000}%
\pgfsetstrokecolor{currentstroke}%
\pgfsetdash{}{0pt}%
\pgfpathmoveto{\pgfqpoint{9.801874in}{3.636847in}}%
\pgfpathlineto{\pgfqpoint{9.863987in}{3.636847in}}%
\pgfpathmoveto{\pgfqpoint{9.832931in}{3.605791in}}%
\pgfpathlineto{\pgfqpoint{9.832931in}{3.667904in}}%
\pgfusepath{stroke,fill}%
\end{pgfscope}%
\begin{pgfscope}%
\pgfpathrectangle{\pgfqpoint{7.105882in}{3.197368in}}{\pgfqpoint{4.376471in}{0.978947in}} %
\pgfusepath{clip}%
\pgfsetbuttcap%
\pgfsetroundjoin%
\definecolor{currentfill}{rgb}{1.000000,0.000000,0.000000}%
\pgfsetfillcolor{currentfill}%
\pgfsetlinewidth{2.007500pt}%
\definecolor{currentstroke}{rgb}{1.000000,0.000000,0.000000}%
\pgfsetstrokecolor{currentstroke}%
\pgfsetdash{}{0pt}%
\pgfpathmoveto{\pgfqpoint{9.938944in}{3.764403in}}%
\pgfpathlineto{\pgfqpoint{10.001057in}{3.764403in}}%
\pgfpathmoveto{\pgfqpoint{9.970001in}{3.733347in}}%
\pgfpathlineto{\pgfqpoint{9.970001in}{3.795460in}}%
\pgfusepath{stroke,fill}%
\end{pgfscope}%
\begin{pgfscope}%
\pgfpathrectangle{\pgfqpoint{7.105882in}{3.197368in}}{\pgfqpoint{4.376471in}{0.978947in}} %
\pgfusepath{clip}%
\pgfsetbuttcap%
\pgfsetroundjoin%
\definecolor{currentfill}{rgb}{1.000000,0.000000,0.000000}%
\pgfsetfillcolor{currentfill}%
\pgfsetlinewidth{2.007500pt}%
\definecolor{currentstroke}{rgb}{1.000000,0.000000,0.000000}%
\pgfsetstrokecolor{currentstroke}%
\pgfsetdash{}{0pt}%
\pgfpathmoveto{\pgfqpoint{11.190797in}{4.024567in}}%
\pgfpathlineto{\pgfqpoint{11.252910in}{4.024567in}}%
\pgfpathmoveto{\pgfqpoint{11.221854in}{3.993511in}}%
\pgfpathlineto{\pgfqpoint{11.221854in}{4.055624in}}%
\pgfusepath{stroke,fill}%
\end{pgfscope}%
\begin{pgfscope}%
\pgfpathrectangle{\pgfqpoint{7.105882in}{3.197368in}}{\pgfqpoint{4.376471in}{0.978947in}} %
\pgfusepath{clip}%
\pgfsetbuttcap%
\pgfsetroundjoin%
\definecolor{currentfill}{rgb}{1.000000,0.000000,0.000000}%
\pgfsetfillcolor{currentfill}%
\pgfsetlinewidth{2.007500pt}%
\definecolor{currentstroke}{rgb}{1.000000,0.000000,0.000000}%
\pgfsetstrokecolor{currentstroke}%
\pgfsetdash{}{0pt}%
\pgfpathmoveto{\pgfqpoint{8.198830in}{3.737178in}}%
\pgfpathlineto{\pgfqpoint{8.260943in}{3.737178in}}%
\pgfpathmoveto{\pgfqpoint{8.229886in}{3.706121in}}%
\pgfpathlineto{\pgfqpoint{8.229886in}{3.768234in}}%
\pgfusepath{stroke,fill}%
\end{pgfscope}%
\begin{pgfscope}%
\pgfpathrectangle{\pgfqpoint{7.105882in}{3.197368in}}{\pgfqpoint{4.376471in}{0.978947in}} %
\pgfusepath{clip}%
\pgfsetbuttcap%
\pgfsetroundjoin%
\definecolor{currentfill}{rgb}{1.000000,0.000000,0.000000}%
\pgfsetfillcolor{currentfill}%
\pgfsetlinewidth{2.007500pt}%
\definecolor{currentstroke}{rgb}{1.000000,0.000000,0.000000}%
\pgfsetstrokecolor{currentstroke}%
\pgfsetdash{}{0pt}%
\pgfpathmoveto{\pgfqpoint{8.255175in}{3.679544in}}%
\pgfpathlineto{\pgfqpoint{8.317288in}{3.679544in}}%
\pgfpathmoveto{\pgfqpoint{8.286232in}{3.648488in}}%
\pgfpathlineto{\pgfqpoint{8.286232in}{3.710601in}}%
\pgfusepath{stroke,fill}%
\end{pgfscope}%
\begin{pgfscope}%
\pgfpathrectangle{\pgfqpoint{7.105882in}{3.197368in}}{\pgfqpoint{4.376471in}{0.978947in}} %
\pgfusepath{clip}%
\pgfsetbuttcap%
\pgfsetroundjoin%
\definecolor{currentfill}{rgb}{1.000000,0.000000,0.000000}%
\pgfsetfillcolor{currentfill}%
\pgfsetlinewidth{2.007500pt}%
\definecolor{currentstroke}{rgb}{1.000000,0.000000,0.000000}%
\pgfsetstrokecolor{currentstroke}%
\pgfsetdash{}{0pt}%
\pgfpathmoveto{\pgfqpoint{8.020908in}{3.966824in}}%
\pgfpathlineto{\pgfqpoint{8.083021in}{3.966824in}}%
\pgfpathmoveto{\pgfqpoint{8.051965in}{3.935768in}}%
\pgfpathlineto{\pgfqpoint{8.051965in}{3.997881in}}%
\pgfusepath{stroke,fill}%
\end{pgfscope}%
\begin{pgfscope}%
\pgfpathrectangle{\pgfqpoint{7.105882in}{3.197368in}}{\pgfqpoint{4.376471in}{0.978947in}} %
\pgfusepath{clip}%
\pgfsetbuttcap%
\pgfsetroundjoin%
\definecolor{currentfill}{rgb}{1.000000,0.000000,0.000000}%
\pgfsetfillcolor{currentfill}%
\pgfsetlinewidth{2.007500pt}%
\definecolor{currentstroke}{rgb}{1.000000,0.000000,0.000000}%
\pgfsetstrokecolor{currentstroke}%
\pgfsetdash{}{0pt}%
\pgfpathmoveto{\pgfqpoint{10.865269in}{3.812420in}}%
\pgfpathlineto{\pgfqpoint{10.927382in}{3.812420in}}%
\pgfpathmoveto{\pgfqpoint{10.896325in}{3.781364in}}%
\pgfpathlineto{\pgfqpoint{10.896325in}{3.843477in}}%
\pgfusepath{stroke,fill}%
\end{pgfscope}%
\begin{pgfscope}%
\pgfpathrectangle{\pgfqpoint{7.105882in}{3.197368in}}{\pgfqpoint{4.376471in}{0.978947in}} %
\pgfusepath{clip}%
\pgfsetbuttcap%
\pgfsetroundjoin%
\definecolor{currentfill}{rgb}{1.000000,0.000000,0.000000}%
\pgfsetfillcolor{currentfill}%
\pgfsetlinewidth{2.007500pt}%
\definecolor{currentstroke}{rgb}{1.000000,0.000000,0.000000}%
\pgfsetstrokecolor{currentstroke}%
\pgfsetdash{}{0pt}%
\pgfpathmoveto{\pgfqpoint{10.674584in}{3.534838in}}%
\pgfpathlineto{\pgfqpoint{10.736697in}{3.534838in}}%
\pgfpathmoveto{\pgfqpoint{10.705641in}{3.503782in}}%
\pgfpathlineto{\pgfqpoint{10.705641in}{3.565895in}}%
\pgfusepath{stroke,fill}%
\end{pgfscope}%
\begin{pgfscope}%
\pgfpathrectangle{\pgfqpoint{7.105882in}{3.197368in}}{\pgfqpoint{4.376471in}{0.978947in}} %
\pgfusepath{clip}%
\pgfsetbuttcap%
\pgfsetroundjoin%
\definecolor{currentfill}{rgb}{1.000000,0.000000,0.000000}%
\pgfsetfillcolor{currentfill}%
\pgfsetlinewidth{2.007500pt}%
\definecolor{currentstroke}{rgb}{1.000000,0.000000,0.000000}%
\pgfsetstrokecolor{currentstroke}%
\pgfsetdash{}{0pt}%
\pgfpathmoveto{\pgfqpoint{10.996186in}{3.942201in}}%
\pgfpathlineto{\pgfqpoint{11.058299in}{3.942201in}}%
\pgfpathmoveto{\pgfqpoint{11.027243in}{3.911145in}}%
\pgfpathlineto{\pgfqpoint{11.027243in}{3.973258in}}%
\pgfusepath{stroke,fill}%
\end{pgfscope}%
\begin{pgfscope}%
\pgfpathrectangle{\pgfqpoint{7.105882in}{3.197368in}}{\pgfqpoint{4.376471in}{0.978947in}} %
\pgfusepath{clip}%
\pgfsetbuttcap%
\pgfsetroundjoin%
\definecolor{currentfill}{rgb}{1.000000,0.000000,0.000000}%
\pgfsetfillcolor{currentfill}%
\pgfsetlinewidth{2.007500pt}%
\definecolor{currentstroke}{rgb}{1.000000,0.000000,0.000000}%
\pgfsetstrokecolor{currentstroke}%
\pgfsetdash{}{0pt}%
\pgfpathmoveto{\pgfqpoint{11.376435in}{3.873438in}}%
\pgfpathlineto{\pgfqpoint{11.438548in}{3.873438in}}%
\pgfpathmoveto{\pgfqpoint{11.407492in}{3.842382in}}%
\pgfpathlineto{\pgfqpoint{11.407492in}{3.904495in}}%
\pgfusepath{stroke,fill}%
\end{pgfscope}%
\begin{pgfscope}%
\pgfpathrectangle{\pgfqpoint{7.105882in}{3.197368in}}{\pgfqpoint{4.376471in}{0.978947in}} %
\pgfusepath{clip}%
\pgfsetbuttcap%
\pgfsetroundjoin%
\definecolor{currentfill}{rgb}{1.000000,0.000000,0.000000}%
\pgfsetfillcolor{currentfill}%
\pgfsetlinewidth{2.007500pt}%
\definecolor{currentstroke}{rgb}{1.000000,0.000000,0.000000}%
\pgfsetstrokecolor{currentstroke}%
\pgfsetdash{}{0pt}%
\pgfpathmoveto{\pgfqpoint{10.748115in}{3.690005in}}%
\pgfpathlineto{\pgfqpoint{10.810228in}{3.690005in}}%
\pgfpathmoveto{\pgfqpoint{10.779172in}{3.658949in}}%
\pgfpathlineto{\pgfqpoint{10.779172in}{3.721062in}}%
\pgfusepath{stroke,fill}%
\end{pgfscope}%
\begin{pgfscope}%
\pgfpathrectangle{\pgfqpoint{7.105882in}{3.197368in}}{\pgfqpoint{4.376471in}{0.978947in}} %
\pgfusepath{clip}%
\pgfsetbuttcap%
\pgfsetroundjoin%
\definecolor{currentfill}{rgb}{1.000000,0.000000,0.000000}%
\pgfsetfillcolor{currentfill}%
\pgfsetlinewidth{2.007500pt}%
\definecolor{currentstroke}{rgb}{1.000000,0.000000,0.000000}%
\pgfsetstrokecolor{currentstroke}%
\pgfsetdash{}{0pt}%
\pgfpathmoveto{\pgfqpoint{9.565841in}{3.414465in}}%
\pgfpathlineto{\pgfqpoint{9.627954in}{3.414465in}}%
\pgfpathmoveto{\pgfqpoint{9.596897in}{3.383409in}}%
\pgfpathlineto{\pgfqpoint{9.596897in}{3.445522in}}%
\pgfusepath{stroke,fill}%
\end{pgfscope}%
\begin{pgfscope}%
\pgfpathrectangle{\pgfqpoint{7.105882in}{3.197368in}}{\pgfqpoint{4.376471in}{0.978947in}} %
\pgfusepath{clip}%
\pgfsetbuttcap%
\pgfsetroundjoin%
\definecolor{currentfill}{rgb}{1.000000,0.000000,0.000000}%
\pgfsetfillcolor{currentfill}%
\pgfsetlinewidth{2.007500pt}%
\definecolor{currentstroke}{rgb}{1.000000,0.000000,0.000000}%
\pgfsetstrokecolor{currentstroke}%
\pgfsetdash{}{0pt}%
\pgfpathmoveto{\pgfqpoint{10.682890in}{3.556875in}}%
\pgfpathlineto{\pgfqpoint{10.745003in}{3.556875in}}%
\pgfpathmoveto{\pgfqpoint{10.713947in}{3.525819in}}%
\pgfpathlineto{\pgfqpoint{10.713947in}{3.587932in}}%
\pgfusepath{stroke,fill}%
\end{pgfscope}%
\begin{pgfscope}%
\pgfpathrectangle{\pgfqpoint{7.105882in}{3.197368in}}{\pgfqpoint{4.376471in}{0.978947in}} %
\pgfusepath{clip}%
\pgfsetbuttcap%
\pgfsetroundjoin%
\definecolor{currentfill}{rgb}{1.000000,0.000000,0.000000}%
\pgfsetfillcolor{currentfill}%
\pgfsetlinewidth{2.007500pt}%
\definecolor{currentstroke}{rgb}{1.000000,0.000000,0.000000}%
\pgfsetstrokecolor{currentstroke}%
\pgfsetdash{}{0pt}%
\pgfpathmoveto{\pgfqpoint{8.364220in}{3.577613in}}%
\pgfpathlineto{\pgfqpoint{8.426333in}{3.577613in}}%
\pgfpathmoveto{\pgfqpoint{8.395276in}{3.546556in}}%
\pgfpathlineto{\pgfqpoint{8.395276in}{3.608669in}}%
\pgfusepath{stroke,fill}%
\end{pgfscope}%
\begin{pgfscope}%
\pgfpathrectangle{\pgfqpoint{7.105882in}{3.197368in}}{\pgfqpoint{4.376471in}{0.978947in}} %
\pgfusepath{clip}%
\pgfsetbuttcap%
\pgfsetroundjoin%
\definecolor{currentfill}{rgb}{1.000000,0.000000,0.000000}%
\pgfsetfillcolor{currentfill}%
\pgfsetlinewidth{2.007500pt}%
\definecolor{currentstroke}{rgb}{1.000000,0.000000,0.000000}%
\pgfsetstrokecolor{currentstroke}%
\pgfsetdash{}{0pt}%
\pgfpathmoveto{\pgfqpoint{10.190596in}{3.680897in}}%
\pgfpathlineto{\pgfqpoint{10.252709in}{3.680897in}}%
\pgfpathmoveto{\pgfqpoint{10.221653in}{3.649840in}}%
\pgfpathlineto{\pgfqpoint{10.221653in}{3.711953in}}%
\pgfusepath{stroke,fill}%
\end{pgfscope}%
\begin{pgfscope}%
\pgfpathrectangle{\pgfqpoint{7.105882in}{3.197368in}}{\pgfqpoint{4.376471in}{0.978947in}} %
\pgfusepath{clip}%
\pgfsetbuttcap%
\pgfsetroundjoin%
\definecolor{currentfill}{rgb}{1.000000,0.000000,0.000000}%
\pgfsetfillcolor{currentfill}%
\pgfsetlinewidth{2.007500pt}%
\definecolor{currentstroke}{rgb}{1.000000,0.000000,0.000000}%
\pgfsetstrokecolor{currentstroke}%
\pgfsetdash{}{0pt}%
\pgfpathmoveto{\pgfqpoint{8.452025in}{3.585903in}}%
\pgfpathlineto{\pgfqpoint{8.514138in}{3.585903in}}%
\pgfpathmoveto{\pgfqpoint{8.483082in}{3.554846in}}%
\pgfpathlineto{\pgfqpoint{8.483082in}{3.616959in}}%
\pgfusepath{stroke,fill}%
\end{pgfscope}%
\begin{pgfscope}%
\pgfpathrectangle{\pgfqpoint{7.105882in}{3.197368in}}{\pgfqpoint{4.376471in}{0.978947in}} %
\pgfusepath{clip}%
\pgfsetbuttcap%
\pgfsetroundjoin%
\definecolor{currentfill}{rgb}{1.000000,0.000000,0.000000}%
\pgfsetfillcolor{currentfill}%
\pgfsetlinewidth{2.007500pt}%
\definecolor{currentstroke}{rgb}{1.000000,0.000000,0.000000}%
\pgfsetstrokecolor{currentstroke}%
\pgfsetdash{}{0pt}%
\pgfpathmoveto{\pgfqpoint{11.257573in}{3.986096in}}%
\pgfpathlineto{\pgfqpoint{11.319686in}{3.986096in}}%
\pgfpathmoveto{\pgfqpoint{11.288629in}{3.955040in}}%
\pgfpathlineto{\pgfqpoint{11.288629in}{4.017153in}}%
\pgfusepath{stroke,fill}%
\end{pgfscope}%
\begin{pgfscope}%
\pgfpathrectangle{\pgfqpoint{7.105882in}{3.197368in}}{\pgfqpoint{4.376471in}{0.978947in}} %
\pgfusepath{clip}%
\pgfsetbuttcap%
\pgfsetroundjoin%
\definecolor{currentfill}{rgb}{1.000000,0.000000,0.000000}%
\pgfsetfillcolor{currentfill}%
\pgfsetlinewidth{2.007500pt}%
\definecolor{currentstroke}{rgb}{1.000000,0.000000,0.000000}%
\pgfsetstrokecolor{currentstroke}%
\pgfsetdash{}{0pt}%
\pgfpathmoveto{\pgfqpoint{9.777203in}{3.633299in}}%
\pgfpathlineto{\pgfqpoint{9.839316in}{3.633299in}}%
\pgfpathmoveto{\pgfqpoint{9.808260in}{3.602242in}}%
\pgfpathlineto{\pgfqpoint{9.808260in}{3.664355in}}%
\pgfusepath{stroke,fill}%
\end{pgfscope}%
\begin{pgfscope}%
\pgfpathrectangle{\pgfqpoint{7.105882in}{3.197368in}}{\pgfqpoint{4.376471in}{0.978947in}} %
\pgfusepath{clip}%
\pgfsetbuttcap%
\pgfsetroundjoin%
\definecolor{currentfill}{rgb}{1.000000,0.000000,0.000000}%
\pgfsetfillcolor{currentfill}%
\pgfsetlinewidth{2.007500pt}%
\definecolor{currentstroke}{rgb}{1.000000,0.000000,0.000000}%
\pgfsetstrokecolor{currentstroke}%
\pgfsetdash{}{0pt}%
\pgfpathmoveto{\pgfqpoint{9.401925in}{3.382181in}}%
\pgfpathlineto{\pgfqpoint{9.464038in}{3.382181in}}%
\pgfpathmoveto{\pgfqpoint{9.432981in}{3.351125in}}%
\pgfpathlineto{\pgfqpoint{9.432981in}{3.413238in}}%
\pgfusepath{stroke,fill}%
\end{pgfscope}%
\begin{pgfscope}%
\pgfpathrectangle{\pgfqpoint{7.105882in}{3.197368in}}{\pgfqpoint{4.376471in}{0.978947in}} %
\pgfusepath{clip}%
\pgfsetbuttcap%
\pgfsetroundjoin%
\definecolor{currentfill}{rgb}{0.000000,0.000000,0.000000}%
\pgfsetfillcolor{currentfill}%
\pgfsetlinewidth{0.301125pt}%
\definecolor{currentstroke}{rgb}{0.000000,0.000000,0.000000}%
\pgfsetstrokecolor{currentstroke}%
\pgfsetdash{}{0pt}%
\pgfsys@defobject{currentmarker}{\pgfqpoint{-0.015528in}{-0.015528in}}{\pgfqpoint{0.015528in}{0.015528in}}{%
\pgfpathmoveto{\pgfqpoint{0.000000in}{-0.015528in}}%
\pgfpathcurveto{\pgfqpoint{0.004118in}{-0.015528in}}{\pgfqpoint{0.008068in}{-0.013892in}}{\pgfqpoint{0.010980in}{-0.010980in}}%
\pgfpathcurveto{\pgfqpoint{0.013892in}{-0.008068in}}{\pgfqpoint{0.015528in}{-0.004118in}}{\pgfqpoint{0.015528in}{0.000000in}}%
\pgfpathcurveto{\pgfqpoint{0.015528in}{0.004118in}}{\pgfqpoint{0.013892in}{0.008068in}}{\pgfqpoint{0.010980in}{0.010980in}}%
\pgfpathcurveto{\pgfqpoint{0.008068in}{0.013892in}}{\pgfqpoint{0.004118in}{0.015528in}}{\pgfqpoint{0.000000in}{0.015528in}}%
\pgfpathcurveto{\pgfqpoint{-0.004118in}{0.015528in}}{\pgfqpoint{-0.008068in}{0.013892in}}{\pgfqpoint{-0.010980in}{0.010980in}}%
\pgfpathcurveto{\pgfqpoint{-0.013892in}{0.008068in}}{\pgfqpoint{-0.015528in}{0.004118in}}{\pgfqpoint{-0.015528in}{0.000000in}}%
\pgfpathcurveto{\pgfqpoint{-0.015528in}{-0.004118in}}{\pgfqpoint{-0.013892in}{-0.008068in}}{\pgfqpoint{-0.010980in}{-0.010980in}}%
\pgfpathcurveto{\pgfqpoint{-0.008068in}{-0.013892in}}{\pgfqpoint{-0.004118in}{-0.015528in}}{\pgfqpoint{0.000000in}{-0.015528in}}%
\pgfpathclose%
\pgfusepath{stroke,fill}%
}%
\begin{pgfscope}%
\pgfsys@transformshift{7.981176in}{4.031231in}%
\pgfsys@useobject{currentmarker}{}%
\end{pgfscope}%
\begin{pgfscope}%
\pgfsys@transformshift{7.998770in}{3.937039in}%
\pgfsys@useobject{currentmarker}{}%
\end{pgfscope}%
\begin{pgfscope}%
\pgfsys@transformshift{8.016364in}{4.027813in}%
\pgfsys@useobject{currentmarker}{}%
\end{pgfscope}%
\begin{pgfscope}%
\pgfsys@transformshift{8.033958in}{4.046950in}%
\pgfsys@useobject{currentmarker}{}%
\end{pgfscope}%
\begin{pgfscope}%
\pgfsys@transformshift{8.051552in}{3.982165in}%
\pgfsys@useobject{currentmarker}{}%
\end{pgfscope}%
\begin{pgfscope}%
\pgfsys@transformshift{8.069146in}{3.978064in}%
\pgfsys@useobject{currentmarker}{}%
\end{pgfscope}%
\begin{pgfscope}%
\pgfsys@transformshift{8.086740in}{3.854307in}%
\pgfsys@useobject{currentmarker}{}%
\end{pgfscope}%
\begin{pgfscope}%
\pgfsys@transformshift{8.104333in}{3.853911in}%
\pgfsys@useobject{currentmarker}{}%
\end{pgfscope}%
\begin{pgfscope}%
\pgfsys@transformshift{8.121927in}{3.794584in}%
\pgfsys@useobject{currentmarker}{}%
\end{pgfscope}%
\begin{pgfscope}%
\pgfsys@transformshift{8.139521in}{3.799298in}%
\pgfsys@useobject{currentmarker}{}%
\end{pgfscope}%
\begin{pgfscope}%
\pgfsys@transformshift{8.157115in}{3.726597in}%
\pgfsys@useobject{currentmarker}{}%
\end{pgfscope}%
\begin{pgfscope}%
\pgfsys@transformshift{8.174709in}{3.607975in}%
\pgfsys@useobject{currentmarker}{}%
\end{pgfscope}%
\begin{pgfscope}%
\pgfsys@transformshift{8.192303in}{3.777720in}%
\pgfsys@useobject{currentmarker}{}%
\end{pgfscope}%
\begin{pgfscope}%
\pgfsys@transformshift{8.209897in}{3.695569in}%
\pgfsys@useobject{currentmarker}{}%
\end{pgfscope}%
\begin{pgfscope}%
\pgfsys@transformshift{8.227490in}{3.548673in}%
\pgfsys@useobject{currentmarker}{}%
\end{pgfscope}%
\begin{pgfscope}%
\pgfsys@transformshift{8.245084in}{3.742077in}%
\pgfsys@useobject{currentmarker}{}%
\end{pgfscope}%
\begin{pgfscope}%
\pgfsys@transformshift{8.262678in}{3.584074in}%
\pgfsys@useobject{currentmarker}{}%
\end{pgfscope}%
\begin{pgfscope}%
\pgfsys@transformshift{8.280272in}{3.665451in}%
\pgfsys@useobject{currentmarker}{}%
\end{pgfscope}%
\begin{pgfscope}%
\pgfsys@transformshift{8.297866in}{3.720008in}%
\pgfsys@useobject{currentmarker}{}%
\end{pgfscope}%
\begin{pgfscope}%
\pgfsys@transformshift{8.315460in}{3.646344in}%
\pgfsys@useobject{currentmarker}{}%
\end{pgfscope}%
\begin{pgfscope}%
\pgfsys@transformshift{8.333054in}{3.738995in}%
\pgfsys@useobject{currentmarker}{}%
\end{pgfscope}%
\begin{pgfscope}%
\pgfsys@transformshift{8.350647in}{3.488622in}%
\pgfsys@useobject{currentmarker}{}%
\end{pgfscope}%
\begin{pgfscope}%
\pgfsys@transformshift{8.368241in}{3.649446in}%
\pgfsys@useobject{currentmarker}{}%
\end{pgfscope}%
\begin{pgfscope}%
\pgfsys@transformshift{8.385835in}{3.534597in}%
\pgfsys@useobject{currentmarker}{}%
\end{pgfscope}%
\begin{pgfscope}%
\pgfsys@transformshift{8.403429in}{3.513756in}%
\pgfsys@useobject{currentmarker}{}%
\end{pgfscope}%
\begin{pgfscope}%
\pgfsys@transformshift{8.421023in}{3.543720in}%
\pgfsys@useobject{currentmarker}{}%
\end{pgfscope}%
\begin{pgfscope}%
\pgfsys@transformshift{8.438617in}{3.573174in}%
\pgfsys@useobject{currentmarker}{}%
\end{pgfscope}%
\begin{pgfscope}%
\pgfsys@transformshift{8.456210in}{3.614788in}%
\pgfsys@useobject{currentmarker}{}%
\end{pgfscope}%
\begin{pgfscope}%
\pgfsys@transformshift{8.473804in}{3.496181in}%
\pgfsys@useobject{currentmarker}{}%
\end{pgfscope}%
\begin{pgfscope}%
\pgfsys@transformshift{8.491398in}{3.714490in}%
\pgfsys@useobject{currentmarker}{}%
\end{pgfscope}%
\begin{pgfscope}%
\pgfsys@transformshift{8.508992in}{3.679310in}%
\pgfsys@useobject{currentmarker}{}%
\end{pgfscope}%
\begin{pgfscope}%
\pgfsys@transformshift{8.526586in}{3.485775in}%
\pgfsys@useobject{currentmarker}{}%
\end{pgfscope}%
\begin{pgfscope}%
\pgfsys@transformshift{8.544180in}{3.806047in}%
\pgfsys@useobject{currentmarker}{}%
\end{pgfscope}%
\begin{pgfscope}%
\pgfsys@transformshift{8.561774in}{3.860539in}%
\pgfsys@useobject{currentmarker}{}%
\end{pgfscope}%
\begin{pgfscope}%
\pgfsys@transformshift{8.579367in}{3.801221in}%
\pgfsys@useobject{currentmarker}{}%
\end{pgfscope}%
\begin{pgfscope}%
\pgfsys@transformshift{8.596961in}{3.677167in}%
\pgfsys@useobject{currentmarker}{}%
\end{pgfscope}%
\begin{pgfscope}%
\pgfsys@transformshift{8.614555in}{3.601314in}%
\pgfsys@useobject{currentmarker}{}%
\end{pgfscope}%
\begin{pgfscope}%
\pgfsys@transformshift{8.632149in}{3.833302in}%
\pgfsys@useobject{currentmarker}{}%
\end{pgfscope}%
\begin{pgfscope}%
\pgfsys@transformshift{8.649743in}{3.700014in}%
\pgfsys@useobject{currentmarker}{}%
\end{pgfscope}%
\begin{pgfscope}%
\pgfsys@transformshift{8.667337in}{3.881009in}%
\pgfsys@useobject{currentmarker}{}%
\end{pgfscope}%
\begin{pgfscope}%
\pgfsys@transformshift{8.684931in}{3.792483in}%
\pgfsys@useobject{currentmarker}{}%
\end{pgfscope}%
\begin{pgfscope}%
\pgfsys@transformshift{8.702524in}{3.885195in}%
\pgfsys@useobject{currentmarker}{}%
\end{pgfscope}%
\begin{pgfscope}%
\pgfsys@transformshift{8.720118in}{3.835577in}%
\pgfsys@useobject{currentmarker}{}%
\end{pgfscope}%
\begin{pgfscope}%
\pgfsys@transformshift{8.737712in}{3.884011in}%
\pgfsys@useobject{currentmarker}{}%
\end{pgfscope}%
\begin{pgfscope}%
\pgfsys@transformshift{8.755306in}{3.824663in}%
\pgfsys@useobject{currentmarker}{}%
\end{pgfscope}%
\begin{pgfscope}%
\pgfsys@transformshift{8.772900in}{4.016085in}%
\pgfsys@useobject{currentmarker}{}%
\end{pgfscope}%
\begin{pgfscope}%
\pgfsys@transformshift{8.790494in}{3.855894in}%
\pgfsys@useobject{currentmarker}{}%
\end{pgfscope}%
\begin{pgfscope}%
\pgfsys@transformshift{8.808087in}{3.891386in}%
\pgfsys@useobject{currentmarker}{}%
\end{pgfscope}%
\begin{pgfscope}%
\pgfsys@transformshift{8.825681in}{4.048210in}%
\pgfsys@useobject{currentmarker}{}%
\end{pgfscope}%
\begin{pgfscope}%
\pgfsys@transformshift{8.843275in}{3.722767in}%
\pgfsys@useobject{currentmarker}{}%
\end{pgfscope}%
\begin{pgfscope}%
\pgfsys@transformshift{8.860869in}{3.732831in}%
\pgfsys@useobject{currentmarker}{}%
\end{pgfscope}%
\begin{pgfscope}%
\pgfsys@transformshift{8.878463in}{3.961518in}%
\pgfsys@useobject{currentmarker}{}%
\end{pgfscope}%
\begin{pgfscope}%
\pgfsys@transformshift{8.896057in}{3.741367in}%
\pgfsys@useobject{currentmarker}{}%
\end{pgfscope}%
\begin{pgfscope}%
\pgfsys@transformshift{8.913651in}{4.055550in}%
\pgfsys@useobject{currentmarker}{}%
\end{pgfscope}%
\begin{pgfscope}%
\pgfsys@transformshift{8.931244in}{3.809556in}%
\pgfsys@useobject{currentmarker}{}%
\end{pgfscope}%
\begin{pgfscope}%
\pgfsys@transformshift{8.948838in}{3.767941in}%
\pgfsys@useobject{currentmarker}{}%
\end{pgfscope}%
\begin{pgfscope}%
\pgfsys@transformshift{8.966432in}{4.030761in}%
\pgfsys@useobject{currentmarker}{}%
\end{pgfscope}%
\begin{pgfscope}%
\pgfsys@transformshift{8.984026in}{3.974297in}%
\pgfsys@useobject{currentmarker}{}%
\end{pgfscope}%
\begin{pgfscope}%
\pgfsys@transformshift{9.001620in}{4.000639in}%
\pgfsys@useobject{currentmarker}{}%
\end{pgfscope}%
\begin{pgfscope}%
\pgfsys@transformshift{9.019214in}{3.887782in}%
\pgfsys@useobject{currentmarker}{}%
\end{pgfscope}%
\begin{pgfscope}%
\pgfsys@transformshift{9.036808in}{3.691193in}%
\pgfsys@useobject{currentmarker}{}%
\end{pgfscope}%
\begin{pgfscope}%
\pgfsys@transformshift{9.054401in}{3.955984in}%
\pgfsys@useobject{currentmarker}{}%
\end{pgfscope}%
\begin{pgfscope}%
\pgfsys@transformshift{9.071995in}{3.714784in}%
\pgfsys@useobject{currentmarker}{}%
\end{pgfscope}%
\begin{pgfscope}%
\pgfsys@transformshift{9.089589in}{3.803722in}%
\pgfsys@useobject{currentmarker}{}%
\end{pgfscope}%
\begin{pgfscope}%
\pgfsys@transformshift{9.107183in}{3.797315in}%
\pgfsys@useobject{currentmarker}{}%
\end{pgfscope}%
\begin{pgfscope}%
\pgfsys@transformshift{9.124777in}{3.662972in}%
\pgfsys@useobject{currentmarker}{}%
\end{pgfscope}%
\begin{pgfscope}%
\pgfsys@transformshift{9.142371in}{3.718881in}%
\pgfsys@useobject{currentmarker}{}%
\end{pgfscope}%
\begin{pgfscope}%
\pgfsys@transformshift{9.159965in}{3.727407in}%
\pgfsys@useobject{currentmarker}{}%
\end{pgfscope}%
\begin{pgfscope}%
\pgfsys@transformshift{9.177558in}{3.648683in}%
\pgfsys@useobject{currentmarker}{}%
\end{pgfscope}%
\begin{pgfscope}%
\pgfsys@transformshift{9.195152in}{3.475151in}%
\pgfsys@useobject{currentmarker}{}%
\end{pgfscope}%
\begin{pgfscope}%
\pgfsys@transformshift{9.212746in}{3.594877in}%
\pgfsys@useobject{currentmarker}{}%
\end{pgfscope}%
\begin{pgfscope}%
\pgfsys@transformshift{9.230340in}{3.677368in}%
\pgfsys@useobject{currentmarker}{}%
\end{pgfscope}%
\begin{pgfscope}%
\pgfsys@transformshift{9.247934in}{3.449577in}%
\pgfsys@useobject{currentmarker}{}%
\end{pgfscope}%
\begin{pgfscope}%
\pgfsys@transformshift{9.265528in}{3.484281in}%
\pgfsys@useobject{currentmarker}{}%
\end{pgfscope}%
\begin{pgfscope}%
\pgfsys@transformshift{9.283121in}{3.435330in}%
\pgfsys@useobject{currentmarker}{}%
\end{pgfscope}%
\begin{pgfscope}%
\pgfsys@transformshift{9.300715in}{3.649654in}%
\pgfsys@useobject{currentmarker}{}%
\end{pgfscope}%
\begin{pgfscope}%
\pgfsys@transformshift{9.318309in}{3.512382in}%
\pgfsys@useobject{currentmarker}{}%
\end{pgfscope}%
\begin{pgfscope}%
\pgfsys@transformshift{9.335903in}{3.469654in}%
\pgfsys@useobject{currentmarker}{}%
\end{pgfscope}%
\begin{pgfscope}%
\pgfsys@transformshift{9.353497in}{3.335540in}%
\pgfsys@useobject{currentmarker}{}%
\end{pgfscope}%
\begin{pgfscope}%
\pgfsys@transformshift{9.371091in}{3.456780in}%
\pgfsys@useobject{currentmarker}{}%
\end{pgfscope}%
\begin{pgfscope}%
\pgfsys@transformshift{9.388685in}{3.322666in}%
\pgfsys@useobject{currentmarker}{}%
\end{pgfscope}%
\begin{pgfscope}%
\pgfsys@transformshift{9.406278in}{3.386302in}%
\pgfsys@useobject{currentmarker}{}%
\end{pgfscope}%
\begin{pgfscope}%
\pgfsys@transformshift{9.423872in}{3.311898in}%
\pgfsys@useobject{currentmarker}{}%
\end{pgfscope}%
\begin{pgfscope}%
\pgfsys@transformshift{9.441466in}{3.441502in}%
\pgfsys@useobject{currentmarker}{}%
\end{pgfscope}%
\begin{pgfscope}%
\pgfsys@transformshift{9.459060in}{3.429239in}%
\pgfsys@useobject{currentmarker}{}%
\end{pgfscope}%
\begin{pgfscope}%
\pgfsys@transformshift{9.476654in}{3.349260in}%
\pgfsys@useobject{currentmarker}{}%
\end{pgfscope}%
\begin{pgfscope}%
\pgfsys@transformshift{9.494248in}{3.413070in}%
\pgfsys@useobject{currentmarker}{}%
\end{pgfscope}%
\begin{pgfscope}%
\pgfsys@transformshift{9.511842in}{3.265505in}%
\pgfsys@useobject{currentmarker}{}%
\end{pgfscope}%
\begin{pgfscope}%
\pgfsys@transformshift{9.529435in}{3.231219in}%
\pgfsys@useobject{currentmarker}{}%
\end{pgfscope}%
\begin{pgfscope}%
\pgfsys@transformshift{9.547029in}{3.436381in}%
\pgfsys@useobject{currentmarker}{}%
\end{pgfscope}%
\begin{pgfscope}%
\pgfsys@transformshift{9.564623in}{3.418730in}%
\pgfsys@useobject{currentmarker}{}%
\end{pgfscope}%
\begin{pgfscope}%
\pgfsys@transformshift{9.582217in}{3.478412in}%
\pgfsys@useobject{currentmarker}{}%
\end{pgfscope}%
\begin{pgfscope}%
\pgfsys@transformshift{9.599811in}{3.670229in}%
\pgfsys@useobject{currentmarker}{}%
\end{pgfscope}%
\begin{pgfscope}%
\pgfsys@transformshift{9.617405in}{3.538571in}%
\pgfsys@useobject{currentmarker}{}%
\end{pgfscope}%
\begin{pgfscope}%
\pgfsys@transformshift{9.634999in}{3.365555in}%
\pgfsys@useobject{currentmarker}{}%
\end{pgfscope}%
\begin{pgfscope}%
\pgfsys@transformshift{9.652592in}{3.590089in}%
\pgfsys@useobject{currentmarker}{}%
\end{pgfscope}%
\begin{pgfscope}%
\pgfsys@transformshift{9.670186in}{3.360521in}%
\pgfsys@useobject{currentmarker}{}%
\end{pgfscope}%
\begin{pgfscope}%
\pgfsys@transformshift{9.687780in}{3.466956in}%
\pgfsys@useobject{currentmarker}{}%
\end{pgfscope}%
\begin{pgfscope}%
\pgfsys@transformshift{9.705374in}{3.526978in}%
\pgfsys@useobject{currentmarker}{}%
\end{pgfscope}%
\begin{pgfscope}%
\pgfsys@transformshift{9.722968in}{3.728956in}%
\pgfsys@useobject{currentmarker}{}%
\end{pgfscope}%
\begin{pgfscope}%
\pgfsys@transformshift{9.740562in}{3.498783in}%
\pgfsys@useobject{currentmarker}{}%
\end{pgfscope}%
\begin{pgfscope}%
\pgfsys@transformshift{9.758155in}{3.510921in}%
\pgfsys@useobject{currentmarker}{}%
\end{pgfscope}%
\begin{pgfscope}%
\pgfsys@transformshift{9.775749in}{3.605397in}%
\pgfsys@useobject{currentmarker}{}%
\end{pgfscope}%
\begin{pgfscope}%
\pgfsys@transformshift{9.793343in}{3.567591in}%
\pgfsys@useobject{currentmarker}{}%
\end{pgfscope}%
\begin{pgfscope}%
\pgfsys@transformshift{9.810937in}{3.769320in}%
\pgfsys@useobject{currentmarker}{}%
\end{pgfscope}%
\begin{pgfscope}%
\pgfsys@transformshift{9.828531in}{3.562676in}%
\pgfsys@useobject{currentmarker}{}%
\end{pgfscope}%
\begin{pgfscope}%
\pgfsys@transformshift{9.846125in}{3.573159in}%
\pgfsys@useobject{currentmarker}{}%
\end{pgfscope}%
\begin{pgfscope}%
\pgfsys@transformshift{9.863719in}{3.661725in}%
\pgfsys@useobject{currentmarker}{}%
\end{pgfscope}%
\begin{pgfscope}%
\pgfsys@transformshift{9.881312in}{3.670459in}%
\pgfsys@useobject{currentmarker}{}%
\end{pgfscope}%
\begin{pgfscope}%
\pgfsys@transformshift{9.898906in}{3.931412in}%
\pgfsys@useobject{currentmarker}{}%
\end{pgfscope}%
\begin{pgfscope}%
\pgfsys@transformshift{9.916500in}{3.843274in}%
\pgfsys@useobject{currentmarker}{}%
\end{pgfscope}%
\begin{pgfscope}%
\pgfsys@transformshift{9.934094in}{3.765484in}%
\pgfsys@useobject{currentmarker}{}%
\end{pgfscope}%
\begin{pgfscope}%
\pgfsys@transformshift{9.951688in}{3.639892in}%
\pgfsys@useobject{currentmarker}{}%
\end{pgfscope}%
\begin{pgfscope}%
\pgfsys@transformshift{9.969282in}{3.857380in}%
\pgfsys@useobject{currentmarker}{}%
\end{pgfscope}%
\begin{pgfscope}%
\pgfsys@transformshift{9.986876in}{3.673775in}%
\pgfsys@useobject{currentmarker}{}%
\end{pgfscope}%
\begin{pgfscope}%
\pgfsys@transformshift{10.004469in}{3.620776in}%
\pgfsys@useobject{currentmarker}{}%
\end{pgfscope}%
\begin{pgfscope}%
\pgfsys@transformshift{10.022063in}{3.900005in}%
\pgfsys@useobject{currentmarker}{}%
\end{pgfscope}%
\begin{pgfscope}%
\pgfsys@transformshift{10.039657in}{3.809765in}%
\pgfsys@useobject{currentmarker}{}%
\end{pgfscope}%
\begin{pgfscope}%
\pgfsys@transformshift{10.057251in}{3.867996in}%
\pgfsys@useobject{currentmarker}{}%
\end{pgfscope}%
\begin{pgfscope}%
\pgfsys@transformshift{10.074845in}{3.801352in}%
\pgfsys@useobject{currentmarker}{}%
\end{pgfscope}%
\begin{pgfscope}%
\pgfsys@transformshift{10.092439in}{3.849159in}%
\pgfsys@useobject{currentmarker}{}%
\end{pgfscope}%
\begin{pgfscope}%
\pgfsys@transformshift{10.110033in}{3.686598in}%
\pgfsys@useobject{currentmarker}{}%
\end{pgfscope}%
\begin{pgfscope}%
\pgfsys@transformshift{10.127626in}{3.637093in}%
\pgfsys@useobject{currentmarker}{}%
\end{pgfscope}%
\begin{pgfscope}%
\pgfsys@transformshift{10.145220in}{3.800132in}%
\pgfsys@useobject{currentmarker}{}%
\end{pgfscope}%
\begin{pgfscope}%
\pgfsys@transformshift{10.162814in}{3.635404in}%
\pgfsys@useobject{currentmarker}{}%
\end{pgfscope}%
\begin{pgfscope}%
\pgfsys@transformshift{10.180408in}{3.632508in}%
\pgfsys@useobject{currentmarker}{}%
\end{pgfscope}%
\begin{pgfscope}%
\pgfsys@transformshift{10.198002in}{3.640821in}%
\pgfsys@useobject{currentmarker}{}%
\end{pgfscope}%
\begin{pgfscope}%
\pgfsys@transformshift{10.215596in}{3.672640in}%
\pgfsys@useobject{currentmarker}{}%
\end{pgfscope}%
\begin{pgfscope}%
\pgfsys@transformshift{10.233189in}{3.617668in}%
\pgfsys@useobject{currentmarker}{}%
\end{pgfscope}%
\begin{pgfscope}%
\pgfsys@transformshift{10.250783in}{3.495984in}%
\pgfsys@useobject{currentmarker}{}%
\end{pgfscope}%
\begin{pgfscope}%
\pgfsys@transformshift{10.268377in}{3.552707in}%
\pgfsys@useobject{currentmarker}{}%
\end{pgfscope}%
\begin{pgfscope}%
\pgfsys@transformshift{10.285971in}{3.373684in}%
\pgfsys@useobject{currentmarker}{}%
\end{pgfscope}%
\begin{pgfscope}%
\pgfsys@transformshift{10.303565in}{3.646375in}%
\pgfsys@useobject{currentmarker}{}%
\end{pgfscope}%
\begin{pgfscope}%
\pgfsys@transformshift{10.321159in}{3.401793in}%
\pgfsys@useobject{currentmarker}{}%
\end{pgfscope}%
\begin{pgfscope}%
\pgfsys@transformshift{10.338753in}{3.435629in}%
\pgfsys@useobject{currentmarker}{}%
\end{pgfscope}%
\begin{pgfscope}%
\pgfsys@transformshift{10.356346in}{3.537393in}%
\pgfsys@useobject{currentmarker}{}%
\end{pgfscope}%
\begin{pgfscope}%
\pgfsys@transformshift{10.373940in}{3.441417in}%
\pgfsys@useobject{currentmarker}{}%
\end{pgfscope}%
\begin{pgfscope}%
\pgfsys@transformshift{10.391534in}{3.660058in}%
\pgfsys@useobject{currentmarker}{}%
\end{pgfscope}%
\begin{pgfscope}%
\pgfsys@transformshift{10.409128in}{3.358041in}%
\pgfsys@useobject{currentmarker}{}%
\end{pgfscope}%
\begin{pgfscope}%
\pgfsys@transformshift{10.426722in}{3.505730in}%
\pgfsys@useobject{currentmarker}{}%
\end{pgfscope}%
\begin{pgfscope}%
\pgfsys@transformshift{10.444316in}{3.464710in}%
\pgfsys@useobject{currentmarker}{}%
\end{pgfscope}%
\begin{pgfscope}%
\pgfsys@transformshift{10.461910in}{3.341550in}%
\pgfsys@useobject{currentmarker}{}%
\end{pgfscope}%
\begin{pgfscope}%
\pgfsys@transformshift{10.479503in}{3.507820in}%
\pgfsys@useobject{currentmarker}{}%
\end{pgfscope}%
\begin{pgfscope}%
\pgfsys@transformshift{10.497097in}{3.432706in}%
\pgfsys@useobject{currentmarker}{}%
\end{pgfscope}%
\begin{pgfscope}%
\pgfsys@transformshift{10.514691in}{3.526663in}%
\pgfsys@useobject{currentmarker}{}%
\end{pgfscope}%
\begin{pgfscope}%
\pgfsys@transformshift{10.532285in}{3.531779in}%
\pgfsys@useobject{currentmarker}{}%
\end{pgfscope}%
\begin{pgfscope}%
\pgfsys@transformshift{10.549879in}{3.670360in}%
\pgfsys@useobject{currentmarker}{}%
\end{pgfscope}%
\begin{pgfscope}%
\pgfsys@transformshift{10.567473in}{3.590161in}%
\pgfsys@useobject{currentmarker}{}%
\end{pgfscope}%
\begin{pgfscope}%
\pgfsys@transformshift{10.585067in}{3.422463in}%
\pgfsys@useobject{currentmarker}{}%
\end{pgfscope}%
\begin{pgfscope}%
\pgfsys@transformshift{10.602660in}{3.444053in}%
\pgfsys@useobject{currentmarker}{}%
\end{pgfscope}%
\begin{pgfscope}%
\pgfsys@transformshift{10.620254in}{3.591023in}%
\pgfsys@useobject{currentmarker}{}%
\end{pgfscope}%
\begin{pgfscope}%
\pgfsys@transformshift{10.637848in}{3.558137in}%
\pgfsys@useobject{currentmarker}{}%
\end{pgfscope}%
\begin{pgfscope}%
\pgfsys@transformshift{10.655442in}{3.570947in}%
\pgfsys@useobject{currentmarker}{}%
\end{pgfscope}%
\begin{pgfscope}%
\pgfsys@transformshift{10.673036in}{3.356960in}%
\pgfsys@useobject{currentmarker}{}%
\end{pgfscope}%
\begin{pgfscope}%
\pgfsys@transformshift{10.690630in}{3.537244in}%
\pgfsys@useobject{currentmarker}{}%
\end{pgfscope}%
\begin{pgfscope}%
\pgfsys@transformshift{10.708223in}{3.483911in}%
\pgfsys@useobject{currentmarker}{}%
\end{pgfscope}%
\begin{pgfscope}%
\pgfsys@transformshift{10.725817in}{3.608578in}%
\pgfsys@useobject{currentmarker}{}%
\end{pgfscope}%
\begin{pgfscope}%
\pgfsys@transformshift{10.743411in}{3.592139in}%
\pgfsys@useobject{currentmarker}{}%
\end{pgfscope}%
\begin{pgfscope}%
\pgfsys@transformshift{10.761005in}{3.718154in}%
\pgfsys@useobject{currentmarker}{}%
\end{pgfscope}%
\begin{pgfscope}%
\pgfsys@transformshift{10.778599in}{3.681771in}%
\pgfsys@useobject{currentmarker}{}%
\end{pgfscope}%
\begin{pgfscope}%
\pgfsys@transformshift{10.796193in}{3.754388in}%
\pgfsys@useobject{currentmarker}{}%
\end{pgfscope}%
\begin{pgfscope}%
\pgfsys@transformshift{10.813787in}{3.651918in}%
\pgfsys@useobject{currentmarker}{}%
\end{pgfscope}%
\begin{pgfscope}%
\pgfsys@transformshift{10.831380in}{3.628742in}%
\pgfsys@useobject{currentmarker}{}%
\end{pgfscope}%
\begin{pgfscope}%
\pgfsys@transformshift{10.848974in}{3.708892in}%
\pgfsys@useobject{currentmarker}{}%
\end{pgfscope}%
\begin{pgfscope}%
\pgfsys@transformshift{10.866568in}{3.774506in}%
\pgfsys@useobject{currentmarker}{}%
\end{pgfscope}%
\begin{pgfscope}%
\pgfsys@transformshift{10.884162in}{3.840115in}%
\pgfsys@useobject{currentmarker}{}%
\end{pgfscope}%
\begin{pgfscope}%
\pgfsys@transformshift{10.901756in}{4.056527in}%
\pgfsys@useobject{currentmarker}{}%
\end{pgfscope}%
\begin{pgfscope}%
\pgfsys@transformshift{10.919350in}{3.845793in}%
\pgfsys@useobject{currentmarker}{}%
\end{pgfscope}%
\begin{pgfscope}%
\pgfsys@transformshift{10.936944in}{3.775681in}%
\pgfsys@useobject{currentmarker}{}%
\end{pgfscope}%
\begin{pgfscope}%
\pgfsys@transformshift{10.954537in}{3.859859in}%
\pgfsys@useobject{currentmarker}{}%
\end{pgfscope}%
\begin{pgfscope}%
\pgfsys@transformshift{10.972131in}{3.868597in}%
\pgfsys@useobject{currentmarker}{}%
\end{pgfscope}%
\begin{pgfscope}%
\pgfsys@transformshift{10.989725in}{3.984304in}%
\pgfsys@useobject{currentmarker}{}%
\end{pgfscope}%
\begin{pgfscope}%
\pgfsys@transformshift{11.007319in}{3.795885in}%
\pgfsys@useobject{currentmarker}{}%
\end{pgfscope}%
\begin{pgfscope}%
\pgfsys@transformshift{11.024913in}{3.975605in}%
\pgfsys@useobject{currentmarker}{}%
\end{pgfscope}%
\begin{pgfscope}%
\pgfsys@transformshift{11.042507in}{3.999505in}%
\pgfsys@useobject{currentmarker}{}%
\end{pgfscope}%
\begin{pgfscope}%
\pgfsys@transformshift{11.060101in}{4.019775in}%
\pgfsys@useobject{currentmarker}{}%
\end{pgfscope}%
\begin{pgfscope}%
\pgfsys@transformshift{11.077694in}{3.945838in}%
\pgfsys@useobject{currentmarker}{}%
\end{pgfscope}%
\begin{pgfscope}%
\pgfsys@transformshift{11.095288in}{3.991188in}%
\pgfsys@useobject{currentmarker}{}%
\end{pgfscope}%
\begin{pgfscope}%
\pgfsys@transformshift{11.112882in}{3.876964in}%
\pgfsys@useobject{currentmarker}{}%
\end{pgfscope}%
\begin{pgfscope}%
\pgfsys@transformshift{11.130476in}{3.976630in}%
\pgfsys@useobject{currentmarker}{}%
\end{pgfscope}%
\begin{pgfscope}%
\pgfsys@transformshift{11.148070in}{3.974377in}%
\pgfsys@useobject{currentmarker}{}%
\end{pgfscope}%
\begin{pgfscope}%
\pgfsys@transformshift{11.165664in}{4.073041in}%
\pgfsys@useobject{currentmarker}{}%
\end{pgfscope}%
\begin{pgfscope}%
\pgfsys@transformshift{11.183257in}{3.911734in}%
\pgfsys@useobject{currentmarker}{}%
\end{pgfscope}%
\begin{pgfscope}%
\pgfsys@transformshift{11.200851in}{4.106534in}%
\pgfsys@useobject{currentmarker}{}%
\end{pgfscope}%
\begin{pgfscope}%
\pgfsys@transformshift{11.218445in}{4.174823in}%
\pgfsys@useobject{currentmarker}{}%
\end{pgfscope}%
\begin{pgfscope}%
\pgfsys@transformshift{11.236039in}{3.805324in}%
\pgfsys@useobject{currentmarker}{}%
\end{pgfscope}%
\begin{pgfscope}%
\pgfsys@transformshift{11.253633in}{4.052418in}%
\pgfsys@useobject{currentmarker}{}%
\end{pgfscope}%
\begin{pgfscope}%
\pgfsys@transformshift{11.271227in}{4.069229in}%
\pgfsys@useobject{currentmarker}{}%
\end{pgfscope}%
\begin{pgfscope}%
\pgfsys@transformshift{11.288821in}{3.925320in}%
\pgfsys@useobject{currentmarker}{}%
\end{pgfscope}%
\begin{pgfscope}%
\pgfsys@transformshift{11.306414in}{3.938972in}%
\pgfsys@useobject{currentmarker}{}%
\end{pgfscope}%
\begin{pgfscope}%
\pgfsys@transformshift{11.324008in}{3.954318in}%
\pgfsys@useobject{currentmarker}{}%
\end{pgfscope}%
\begin{pgfscope}%
\pgfsys@transformshift{11.341602in}{3.925304in}%
\pgfsys@useobject{currentmarker}{}%
\end{pgfscope}%
\begin{pgfscope}%
\pgfsys@transformshift{11.359196in}{3.911578in}%
\pgfsys@useobject{currentmarker}{}%
\end{pgfscope}%
\begin{pgfscope}%
\pgfsys@transformshift{11.376790in}{3.759418in}%
\pgfsys@useobject{currentmarker}{}%
\end{pgfscope}%
\begin{pgfscope}%
\pgfsys@transformshift{11.394384in}{4.035096in}%
\pgfsys@useobject{currentmarker}{}%
\end{pgfscope}%
\begin{pgfscope}%
\pgfsys@transformshift{11.411978in}{4.015175in}%
\pgfsys@useobject{currentmarker}{}%
\end{pgfscope}%
\begin{pgfscope}%
\pgfsys@transformshift{11.429571in}{3.810072in}%
\pgfsys@useobject{currentmarker}{}%
\end{pgfscope}%
\begin{pgfscope}%
\pgfsys@transformshift{11.447165in}{3.732047in}%
\pgfsys@useobject{currentmarker}{}%
\end{pgfscope}%
\begin{pgfscope}%
\pgfsys@transformshift{11.464759in}{3.924124in}%
\pgfsys@useobject{currentmarker}{}%
\end{pgfscope}%
\begin{pgfscope}%
\pgfsys@transformshift{11.482353in}{3.802702in}%
\pgfsys@useobject{currentmarker}{}%
\end{pgfscope}%
\end{pgfscope}%
\begin{pgfscope}%
\pgfpathrectangle{\pgfqpoint{7.105882in}{3.197368in}}{\pgfqpoint{4.376471in}{0.978947in}} %
\pgfusepath{clip}%
\pgfsetroundcap%
\pgfsetroundjoin%
\pgfsetlinewidth{1.756562pt}%
\definecolor{currentstroke}{rgb}{0.298039,0.447059,0.690196}%
\pgfsetstrokecolor{currentstroke}%
\pgfsetdash{}{0pt}%
\pgfpathmoveto{\pgfqpoint{7.981176in}{3.564359in}}%
\pgfpathlineto{\pgfqpoint{8.508992in}{3.564469in}}%
\pgfpathlineto{\pgfqpoint{9.863719in}{3.564508in}}%
\pgfpathlineto{\pgfqpoint{11.482353in}{3.564727in}}%
\pgfpathlineto{\pgfqpoint{11.482353in}{3.564727in}}%
\pgfusepath{stroke}%
\end{pgfscope}%
\begin{pgfscope}%
\pgfsetrectcap%
\pgfsetmiterjoin%
\pgfsetlinewidth{1.003750pt}%
\definecolor{currentstroke}{rgb}{0.800000,0.800000,0.800000}%
\pgfsetstrokecolor{currentstroke}%
\pgfsetdash{}{0pt}%
\pgfpathmoveto{\pgfqpoint{7.105882in}{3.197368in}}%
\pgfpathlineto{\pgfqpoint{7.105882in}{4.176316in}}%
\pgfusepath{stroke}%
\end{pgfscope}%
\begin{pgfscope}%
\pgfsetrectcap%
\pgfsetmiterjoin%
\pgfsetlinewidth{1.003750pt}%
\definecolor{currentstroke}{rgb}{0.800000,0.800000,0.800000}%
\pgfsetstrokecolor{currentstroke}%
\pgfsetdash{}{0pt}%
\pgfpathmoveto{\pgfqpoint{11.482353in}{3.197368in}}%
\pgfpathlineto{\pgfqpoint{11.482353in}{4.176316in}}%
\pgfusepath{stroke}%
\end{pgfscope}%
\begin{pgfscope}%
\pgfsetrectcap%
\pgfsetmiterjoin%
\pgfsetlinewidth{1.003750pt}%
\definecolor{currentstroke}{rgb}{0.800000,0.800000,0.800000}%
\pgfsetstrokecolor{currentstroke}%
\pgfsetdash{}{0pt}%
\pgfpathmoveto{\pgfqpoint{7.105882in}{4.176316in}}%
\pgfpathlineto{\pgfqpoint{11.482353in}{4.176316in}}%
\pgfusepath{stroke}%
\end{pgfscope}%
\begin{pgfscope}%
\pgfsetrectcap%
\pgfsetmiterjoin%
\pgfsetlinewidth{1.003750pt}%
\definecolor{currentstroke}{rgb}{0.800000,0.800000,0.800000}%
\pgfsetstrokecolor{currentstroke}%
\pgfsetdash{}{0pt}%
\pgfpathmoveto{\pgfqpoint{7.105882in}{3.197368in}}%
\pgfpathlineto{\pgfqpoint{11.482353in}{3.197368in}}%
\pgfusepath{stroke}%
\end{pgfscope}%
\begin{pgfscope}%
\pgfsetroundcap%
\pgfsetroundjoin%
\pgfsetlinewidth{1.756562pt}%
\definecolor{currentstroke}{rgb}{0.298039,0.447059,0.690196}%
\pgfsetstrokecolor{currentstroke}%
\pgfsetdash{}{0pt}%
\pgfpathmoveto{\pgfqpoint{7.230882in}{3.791511in}}%
\pgfpathlineto{\pgfqpoint{7.508660in}{3.791511in}}%
\pgfusepath{stroke}%
\end{pgfscope}%
\begin{pgfscope}%
\definecolor{textcolor}{rgb}{0.150000,0.150000,0.150000}%
\pgfsetstrokecolor{textcolor}%
\pgfsetfillcolor{textcolor}%
\pgftext[x=7.619771in,y=3.742900in,left,base]{\color{textcolor}\sffamily\fontsize{10.000000}{12.000000}\selectfont \(\displaystyle \widetilde{\Phi}^* \theta^{\parallel}\)}%
\end{pgfscope}%
\begin{pgfscope}%
\pgfsetbuttcap%
\pgfsetroundjoin%
\definecolor{currentfill}{rgb}{1.000000,0.000000,0.000000}%
\pgfsetfillcolor{currentfill}%
\pgfsetlinewidth{2.007500pt}%
\definecolor{currentstroke}{rgb}{1.000000,0.000000,0.000000}%
\pgfsetstrokecolor{currentstroke}%
\pgfsetdash{}{0pt}%
\pgfpathmoveto{\pgfqpoint{7.338715in}{3.582893in}}%
\pgfpathlineto{\pgfqpoint{7.400828in}{3.582893in}}%
\pgfpathmoveto{\pgfqpoint{7.369771in}{3.551837in}}%
\pgfpathlineto{\pgfqpoint{7.369771in}{3.613950in}}%
\pgfusepath{stroke,fill}%
\end{pgfscope}%
\begin{pgfscope}%
\pgfsetbuttcap%
\pgfsetroundjoin%
\definecolor{currentfill}{rgb}{1.000000,0.000000,0.000000}%
\pgfsetfillcolor{currentfill}%
\pgfsetlinewidth{2.007500pt}%
\definecolor{currentstroke}{rgb}{1.000000,0.000000,0.000000}%
\pgfsetstrokecolor{currentstroke}%
\pgfsetdash{}{0pt}%
\pgfpathmoveto{\pgfqpoint{7.338715in}{3.582893in}}%
\pgfpathlineto{\pgfqpoint{7.400828in}{3.582893in}}%
\pgfpathmoveto{\pgfqpoint{7.369771in}{3.551837in}}%
\pgfpathlineto{\pgfqpoint{7.369771in}{3.613950in}}%
\pgfusepath{stroke,fill}%
\end{pgfscope}%
\begin{pgfscope}%
\pgfsetbuttcap%
\pgfsetroundjoin%
\definecolor{currentfill}{rgb}{1.000000,0.000000,0.000000}%
\pgfsetfillcolor{currentfill}%
\pgfsetlinewidth{2.007500pt}%
\definecolor{currentstroke}{rgb}{1.000000,0.000000,0.000000}%
\pgfsetstrokecolor{currentstroke}%
\pgfsetdash{}{0pt}%
\pgfpathmoveto{\pgfqpoint{7.338715in}{3.582893in}}%
\pgfpathlineto{\pgfqpoint{7.400828in}{3.582893in}}%
\pgfpathmoveto{\pgfqpoint{7.369771in}{3.551837in}}%
\pgfpathlineto{\pgfqpoint{7.369771in}{3.613950in}}%
\pgfusepath{stroke,fill}%
\end{pgfscope}%
\begin{pgfscope}%
\definecolor{textcolor}{rgb}{0.150000,0.150000,0.150000}%
\pgfsetstrokecolor{textcolor}%
\pgfsetfillcolor{textcolor}%
\pgftext[x=7.619771in,y=3.546435in,left,base]{\color{textcolor}\sffamily\fontsize{10.000000}{12.000000}\selectfont train}%
\end{pgfscope}%
\begin{pgfscope}%
\pgfsetbuttcap%
\pgfsetroundjoin%
\definecolor{currentfill}{rgb}{0.000000,0.000000,0.000000}%
\pgfsetfillcolor{currentfill}%
\pgfsetlinewidth{0.301125pt}%
\definecolor{currentstroke}{rgb}{0.000000,0.000000,0.000000}%
\pgfsetstrokecolor{currentstroke}%
\pgfsetdash{}{0pt}%
\pgfpathmoveto{\pgfqpoint{7.369771in}{3.370900in}}%
\pgfpathcurveto{\pgfqpoint{7.373889in}{3.370900in}}{\pgfqpoint{7.377839in}{3.372536in}}{\pgfqpoint{7.380751in}{3.375448in}}%
\pgfpathcurveto{\pgfqpoint{7.383663in}{3.378360in}}{\pgfqpoint{7.385299in}{3.382310in}}{\pgfqpoint{7.385299in}{3.386428in}}%
\pgfpathcurveto{\pgfqpoint{7.385299in}{3.390546in}}{\pgfqpoint{7.383663in}{3.394496in}}{\pgfqpoint{7.380751in}{3.397408in}}%
\pgfpathcurveto{\pgfqpoint{7.377839in}{3.400320in}}{\pgfqpoint{7.373889in}{3.401956in}}{\pgfqpoint{7.369771in}{3.401956in}}%
\pgfpathcurveto{\pgfqpoint{7.365653in}{3.401956in}}{\pgfqpoint{7.361703in}{3.400320in}}{\pgfqpoint{7.358791in}{3.397408in}}%
\pgfpathcurveto{\pgfqpoint{7.355879in}{3.394496in}}{\pgfqpoint{7.354243in}{3.390546in}}{\pgfqpoint{7.354243in}{3.386428in}}%
\pgfpathcurveto{\pgfqpoint{7.354243in}{3.382310in}}{\pgfqpoint{7.355879in}{3.378360in}}{\pgfqpoint{7.358791in}{3.375448in}}%
\pgfpathcurveto{\pgfqpoint{7.361703in}{3.372536in}}{\pgfqpoint{7.365653in}{3.370900in}}{\pgfqpoint{7.369771in}{3.370900in}}%
\pgfpathclose%
\pgfusepath{stroke,fill}%
\end{pgfscope}%
\begin{pgfscope}%
\pgfsetbuttcap%
\pgfsetroundjoin%
\definecolor{currentfill}{rgb}{0.000000,0.000000,0.000000}%
\pgfsetfillcolor{currentfill}%
\pgfsetlinewidth{0.301125pt}%
\definecolor{currentstroke}{rgb}{0.000000,0.000000,0.000000}%
\pgfsetstrokecolor{currentstroke}%
\pgfsetdash{}{0pt}%
\pgfpathmoveto{\pgfqpoint{7.369771in}{3.370900in}}%
\pgfpathcurveto{\pgfqpoint{7.373889in}{3.370900in}}{\pgfqpoint{7.377839in}{3.372536in}}{\pgfqpoint{7.380751in}{3.375448in}}%
\pgfpathcurveto{\pgfqpoint{7.383663in}{3.378360in}}{\pgfqpoint{7.385299in}{3.382310in}}{\pgfqpoint{7.385299in}{3.386428in}}%
\pgfpathcurveto{\pgfqpoint{7.385299in}{3.390546in}}{\pgfqpoint{7.383663in}{3.394496in}}{\pgfqpoint{7.380751in}{3.397408in}}%
\pgfpathcurveto{\pgfqpoint{7.377839in}{3.400320in}}{\pgfqpoint{7.373889in}{3.401956in}}{\pgfqpoint{7.369771in}{3.401956in}}%
\pgfpathcurveto{\pgfqpoint{7.365653in}{3.401956in}}{\pgfqpoint{7.361703in}{3.400320in}}{\pgfqpoint{7.358791in}{3.397408in}}%
\pgfpathcurveto{\pgfqpoint{7.355879in}{3.394496in}}{\pgfqpoint{7.354243in}{3.390546in}}{\pgfqpoint{7.354243in}{3.386428in}}%
\pgfpathcurveto{\pgfqpoint{7.354243in}{3.382310in}}{\pgfqpoint{7.355879in}{3.378360in}}{\pgfqpoint{7.358791in}{3.375448in}}%
\pgfpathcurveto{\pgfqpoint{7.361703in}{3.372536in}}{\pgfqpoint{7.365653in}{3.370900in}}{\pgfqpoint{7.369771in}{3.370900in}}%
\pgfpathclose%
\pgfusepath{stroke,fill}%
\end{pgfscope}%
\begin{pgfscope}%
\pgfsetbuttcap%
\pgfsetroundjoin%
\definecolor{currentfill}{rgb}{0.000000,0.000000,0.000000}%
\pgfsetfillcolor{currentfill}%
\pgfsetlinewidth{0.301125pt}%
\definecolor{currentstroke}{rgb}{0.000000,0.000000,0.000000}%
\pgfsetstrokecolor{currentstroke}%
\pgfsetdash{}{0pt}%
\pgfpathmoveto{\pgfqpoint{7.369771in}{3.370900in}}%
\pgfpathcurveto{\pgfqpoint{7.373889in}{3.370900in}}{\pgfqpoint{7.377839in}{3.372536in}}{\pgfqpoint{7.380751in}{3.375448in}}%
\pgfpathcurveto{\pgfqpoint{7.383663in}{3.378360in}}{\pgfqpoint{7.385299in}{3.382310in}}{\pgfqpoint{7.385299in}{3.386428in}}%
\pgfpathcurveto{\pgfqpoint{7.385299in}{3.390546in}}{\pgfqpoint{7.383663in}{3.394496in}}{\pgfqpoint{7.380751in}{3.397408in}}%
\pgfpathcurveto{\pgfqpoint{7.377839in}{3.400320in}}{\pgfqpoint{7.373889in}{3.401956in}}{\pgfqpoint{7.369771in}{3.401956in}}%
\pgfpathcurveto{\pgfqpoint{7.365653in}{3.401956in}}{\pgfqpoint{7.361703in}{3.400320in}}{\pgfqpoint{7.358791in}{3.397408in}}%
\pgfpathcurveto{\pgfqpoint{7.355879in}{3.394496in}}{\pgfqpoint{7.354243in}{3.390546in}}{\pgfqpoint{7.354243in}{3.386428in}}%
\pgfpathcurveto{\pgfqpoint{7.354243in}{3.382310in}}{\pgfqpoint{7.355879in}{3.378360in}}{\pgfqpoint{7.358791in}{3.375448in}}%
\pgfpathcurveto{\pgfqpoint{7.361703in}{3.372536in}}{\pgfqpoint{7.365653in}{3.370900in}}{\pgfqpoint{7.369771in}{3.370900in}}%
\pgfpathclose%
\pgfusepath{stroke,fill}%
\end{pgfscope}%
\begin{pgfscope}%
\definecolor{textcolor}{rgb}{0.150000,0.150000,0.150000}%
\pgfsetstrokecolor{textcolor}%
\pgfsetfillcolor{textcolor}%
\pgftext[x=7.619771in,y=3.349970in,left,base]{\color{textcolor}\sffamily\fontsize{10.000000}{12.000000}\selectfont test}%
\end{pgfscope}%
\begin{pgfscope}%
\pgfsetbuttcap%
\pgfsetmiterjoin%
\definecolor{currentfill}{rgb}{1.000000,1.000000,1.000000}%
\pgfsetfillcolor{currentfill}%
\pgfsetlinewidth{0.000000pt}%
\definecolor{currentstroke}{rgb}{0.000000,0.000000,0.000000}%
\pgfsetstrokecolor{currentstroke}%
\pgfsetstrokeopacity{0.000000}%
\pgfsetdash{}{0pt}%
\pgfpathmoveto{\pgfqpoint{12.211765in}{3.197368in}}%
\pgfpathlineto{\pgfqpoint{14.400000in}{3.197368in}}%
\pgfpathlineto{\pgfqpoint{14.400000in}{4.176316in}}%
\pgfpathlineto{\pgfqpoint{12.211765in}{4.176316in}}%
\pgfpathclose%
\pgfusepath{fill}%
\end{pgfscope}%
\begin{pgfscope}%
\pgfpathrectangle{\pgfqpoint{12.211765in}{3.197368in}}{\pgfqpoint{2.188235in}{0.978947in}} %
\pgfusepath{clip}%
\pgfsetroundcap%
\pgfsetroundjoin%
\pgfsetlinewidth{1.003750pt}%
\definecolor{currentstroke}{rgb}{0.800000,0.800000,0.800000}%
\pgfsetstrokecolor{currentstroke}%
\pgfsetdash{}{0pt}%
\pgfpathmoveto{\pgfqpoint{12.211765in}{3.197368in}}%
\pgfpathlineto{\pgfqpoint{12.211765in}{4.176316in}}%
\pgfusepath{stroke}%
\end{pgfscope}%
\begin{pgfscope}%
\pgfpathrectangle{\pgfqpoint{12.211765in}{3.197368in}}{\pgfqpoint{2.188235in}{0.978947in}} %
\pgfusepath{clip}%
\pgfsetroundcap%
\pgfsetroundjoin%
\pgfsetlinewidth{1.003750pt}%
\definecolor{currentstroke}{rgb}{0.800000,0.800000,0.800000}%
\pgfsetstrokecolor{currentstroke}%
\pgfsetdash{}{0pt}%
\pgfpathmoveto{\pgfqpoint{12.485294in}{3.197368in}}%
\pgfpathlineto{\pgfqpoint{12.485294in}{4.176316in}}%
\pgfusepath{stroke}%
\end{pgfscope}%
\begin{pgfscope}%
\pgfpathrectangle{\pgfqpoint{12.211765in}{3.197368in}}{\pgfqpoint{2.188235in}{0.978947in}} %
\pgfusepath{clip}%
\pgfsetroundcap%
\pgfsetroundjoin%
\pgfsetlinewidth{1.003750pt}%
\definecolor{currentstroke}{rgb}{0.800000,0.800000,0.800000}%
\pgfsetstrokecolor{currentstroke}%
\pgfsetdash{}{0pt}%
\pgfpathmoveto{\pgfqpoint{12.758824in}{3.197368in}}%
\pgfpathlineto{\pgfqpoint{12.758824in}{4.176316in}}%
\pgfusepath{stroke}%
\end{pgfscope}%
\begin{pgfscope}%
\pgfpathrectangle{\pgfqpoint{12.211765in}{3.197368in}}{\pgfqpoint{2.188235in}{0.978947in}} %
\pgfusepath{clip}%
\pgfsetroundcap%
\pgfsetroundjoin%
\pgfsetlinewidth{1.003750pt}%
\definecolor{currentstroke}{rgb}{0.800000,0.800000,0.800000}%
\pgfsetstrokecolor{currentstroke}%
\pgfsetdash{}{0pt}%
\pgfpathmoveto{\pgfqpoint{13.032353in}{3.197368in}}%
\pgfpathlineto{\pgfqpoint{13.032353in}{4.176316in}}%
\pgfusepath{stroke}%
\end{pgfscope}%
\begin{pgfscope}%
\pgfpathrectangle{\pgfqpoint{12.211765in}{3.197368in}}{\pgfqpoint{2.188235in}{0.978947in}} %
\pgfusepath{clip}%
\pgfsetroundcap%
\pgfsetroundjoin%
\pgfsetlinewidth{1.003750pt}%
\definecolor{currentstroke}{rgb}{0.800000,0.800000,0.800000}%
\pgfsetstrokecolor{currentstroke}%
\pgfsetdash{}{0pt}%
\pgfpathmoveto{\pgfqpoint{13.305882in}{3.197368in}}%
\pgfpathlineto{\pgfqpoint{13.305882in}{4.176316in}}%
\pgfusepath{stroke}%
\end{pgfscope}%
\begin{pgfscope}%
\pgfpathrectangle{\pgfqpoint{12.211765in}{3.197368in}}{\pgfqpoint{2.188235in}{0.978947in}} %
\pgfusepath{clip}%
\pgfsetroundcap%
\pgfsetroundjoin%
\pgfsetlinewidth{1.003750pt}%
\definecolor{currentstroke}{rgb}{0.800000,0.800000,0.800000}%
\pgfsetstrokecolor{currentstroke}%
\pgfsetdash{}{0pt}%
\pgfpathmoveto{\pgfqpoint{13.579412in}{3.197368in}}%
\pgfpathlineto{\pgfqpoint{13.579412in}{4.176316in}}%
\pgfusepath{stroke}%
\end{pgfscope}%
\begin{pgfscope}%
\pgfpathrectangle{\pgfqpoint{12.211765in}{3.197368in}}{\pgfqpoint{2.188235in}{0.978947in}} %
\pgfusepath{clip}%
\pgfsetroundcap%
\pgfsetroundjoin%
\pgfsetlinewidth{1.003750pt}%
\definecolor{currentstroke}{rgb}{0.800000,0.800000,0.800000}%
\pgfsetstrokecolor{currentstroke}%
\pgfsetdash{}{0pt}%
\pgfpathmoveto{\pgfqpoint{13.852941in}{3.197368in}}%
\pgfpathlineto{\pgfqpoint{13.852941in}{4.176316in}}%
\pgfusepath{stroke}%
\end{pgfscope}%
\begin{pgfscope}%
\pgfpathrectangle{\pgfqpoint{12.211765in}{3.197368in}}{\pgfqpoint{2.188235in}{0.978947in}} %
\pgfusepath{clip}%
\pgfsetroundcap%
\pgfsetroundjoin%
\pgfsetlinewidth{1.003750pt}%
\definecolor{currentstroke}{rgb}{0.800000,0.800000,0.800000}%
\pgfsetstrokecolor{currentstroke}%
\pgfsetdash{}{0pt}%
\pgfpathmoveto{\pgfqpoint{14.126471in}{3.197368in}}%
\pgfpathlineto{\pgfqpoint{14.126471in}{4.176316in}}%
\pgfusepath{stroke}%
\end{pgfscope}%
\begin{pgfscope}%
\pgfpathrectangle{\pgfqpoint{12.211765in}{3.197368in}}{\pgfqpoint{2.188235in}{0.978947in}} %
\pgfusepath{clip}%
\pgfsetroundcap%
\pgfsetroundjoin%
\pgfsetlinewidth{1.003750pt}%
\definecolor{currentstroke}{rgb}{0.800000,0.800000,0.800000}%
\pgfsetstrokecolor{currentstroke}%
\pgfsetdash{}{0pt}%
\pgfpathmoveto{\pgfqpoint{14.400000in}{3.197368in}}%
\pgfpathlineto{\pgfqpoint{14.400000in}{4.176316in}}%
\pgfusepath{stroke}%
\end{pgfscope}%
\begin{pgfscope}%
\pgfpathrectangle{\pgfqpoint{12.211765in}{3.197368in}}{\pgfqpoint{2.188235in}{0.978947in}} %
\pgfusepath{clip}%
\pgfsetroundcap%
\pgfsetroundjoin%
\pgfsetlinewidth{1.003750pt}%
\definecolor{currentstroke}{rgb}{0.800000,0.800000,0.800000}%
\pgfsetstrokecolor{currentstroke}%
\pgfsetdash{}{0pt}%
\pgfpathmoveto{\pgfqpoint{12.211765in}{3.197368in}}%
\pgfpathlineto{\pgfqpoint{14.400000in}{3.197368in}}%
\pgfusepath{stroke}%
\end{pgfscope}%
\begin{pgfscope}%
\definecolor{textcolor}{rgb}{0.150000,0.150000,0.150000}%
\pgfsetstrokecolor{textcolor}%
\pgfsetfillcolor{textcolor}%
\pgftext[x=12.114542in,y=3.197368in,right,]{\color{textcolor}\sffamily\fontsize{10.000000}{12.000000}\selectfont \(\displaystyle 0\)}%
\end{pgfscope}%
\begin{pgfscope}%
\pgfpathrectangle{\pgfqpoint{12.211765in}{3.197368in}}{\pgfqpoint{2.188235in}{0.978947in}} %
\pgfusepath{clip}%
\pgfsetroundcap%
\pgfsetroundjoin%
\pgfsetlinewidth{1.003750pt}%
\definecolor{currentstroke}{rgb}{0.800000,0.800000,0.800000}%
\pgfsetstrokecolor{currentstroke}%
\pgfsetdash{}{0pt}%
\pgfpathmoveto{\pgfqpoint{12.211765in}{3.442105in}}%
\pgfpathlineto{\pgfqpoint{14.400000in}{3.442105in}}%
\pgfusepath{stroke}%
\end{pgfscope}%
\begin{pgfscope}%
\definecolor{textcolor}{rgb}{0.150000,0.150000,0.150000}%
\pgfsetstrokecolor{textcolor}%
\pgfsetfillcolor{textcolor}%
\pgftext[x=12.114542in,y=3.442105in,right,]{\color{textcolor}\sffamily\fontsize{10.000000}{12.000000}\selectfont \(\displaystyle 50\)}%
\end{pgfscope}%
\begin{pgfscope}%
\pgfpathrectangle{\pgfqpoint{12.211765in}{3.197368in}}{\pgfqpoint{2.188235in}{0.978947in}} %
\pgfusepath{clip}%
\pgfsetroundcap%
\pgfsetroundjoin%
\pgfsetlinewidth{1.003750pt}%
\definecolor{currentstroke}{rgb}{0.800000,0.800000,0.800000}%
\pgfsetstrokecolor{currentstroke}%
\pgfsetdash{}{0pt}%
\pgfpathmoveto{\pgfqpoint{12.211765in}{3.686842in}}%
\pgfpathlineto{\pgfqpoint{14.400000in}{3.686842in}}%
\pgfusepath{stroke}%
\end{pgfscope}%
\begin{pgfscope}%
\definecolor{textcolor}{rgb}{0.150000,0.150000,0.150000}%
\pgfsetstrokecolor{textcolor}%
\pgfsetfillcolor{textcolor}%
\pgftext[x=12.114542in,y=3.686842in,right,]{\color{textcolor}\sffamily\fontsize{10.000000}{12.000000}\selectfont \(\displaystyle 100\)}%
\end{pgfscope}%
\begin{pgfscope}%
\pgfpathrectangle{\pgfqpoint{12.211765in}{3.197368in}}{\pgfqpoint{2.188235in}{0.978947in}} %
\pgfusepath{clip}%
\pgfsetroundcap%
\pgfsetroundjoin%
\pgfsetlinewidth{1.003750pt}%
\definecolor{currentstroke}{rgb}{0.800000,0.800000,0.800000}%
\pgfsetstrokecolor{currentstroke}%
\pgfsetdash{}{0pt}%
\pgfpathmoveto{\pgfqpoint{12.211765in}{3.931579in}}%
\pgfpathlineto{\pgfqpoint{14.400000in}{3.931579in}}%
\pgfusepath{stroke}%
\end{pgfscope}%
\begin{pgfscope}%
\definecolor{textcolor}{rgb}{0.150000,0.150000,0.150000}%
\pgfsetstrokecolor{textcolor}%
\pgfsetfillcolor{textcolor}%
\pgftext[x=12.114542in,y=3.931579in,right,]{\color{textcolor}\sffamily\fontsize{10.000000}{12.000000}\selectfont \(\displaystyle 150\)}%
\end{pgfscope}%
\begin{pgfscope}%
\pgfpathrectangle{\pgfqpoint{12.211765in}{3.197368in}}{\pgfqpoint{2.188235in}{0.978947in}} %
\pgfusepath{clip}%
\pgfsetroundcap%
\pgfsetroundjoin%
\pgfsetlinewidth{1.003750pt}%
\definecolor{currentstroke}{rgb}{0.800000,0.800000,0.800000}%
\pgfsetstrokecolor{currentstroke}%
\pgfsetdash{}{0pt}%
\pgfpathmoveto{\pgfqpoint{12.211765in}{4.176316in}}%
\pgfpathlineto{\pgfqpoint{14.400000in}{4.176316in}}%
\pgfusepath{stroke}%
\end{pgfscope}%
\begin{pgfscope}%
\definecolor{textcolor}{rgb}{0.150000,0.150000,0.150000}%
\pgfsetstrokecolor{textcolor}%
\pgfsetfillcolor{textcolor}%
\pgftext[x=12.114542in,y=4.176316in,right,]{\color{textcolor}\sffamily\fontsize{10.000000}{12.000000}\selectfont \(\displaystyle 200\)}%
\end{pgfscope}%
\begin{pgfscope}%
\definecolor{textcolor}{rgb}{0.150000,0.150000,0.150000}%
\pgfsetstrokecolor{textcolor}%
\pgfsetfillcolor{textcolor}%
\pgftext[x=11.836764in,y=3.686842in,,bottom,rotate=90.000000]{\color{textcolor}\sffamily\fontsize{11.000000}{13.200000}\selectfont \(\displaystyle \theta^{\parallel}_j\)}%
\end{pgfscope}%
\begin{pgfscope}%
\pgfpathrectangle{\pgfqpoint{12.211765in}{3.197368in}}{\pgfqpoint{2.188235in}{0.978947in}} %
\pgfusepath{clip}%
\pgfsetroundcap%
\pgfsetroundjoin%
\pgfsetlinewidth{1.756562pt}%
\definecolor{currentstroke}{rgb}{0.298039,0.447059,0.690196}%
\pgfsetstrokecolor{currentstroke}%
\pgfsetdash{}{0pt}%
\pgfpathmoveto{\pgfqpoint{13.305745in}{3.197368in}}%
\pgfpathlineto{\pgfqpoint{13.305836in}{3.970737in}}%
\pgfpathlineto{\pgfqpoint{13.305894in}{4.019684in}}%
\pgfpathlineto{\pgfqpoint{13.305865in}{4.142053in}}%
\pgfpathlineto{\pgfqpoint{13.305974in}{4.171421in}}%
\pgfpathlineto{\pgfqpoint{13.305974in}{4.171421in}}%
\pgfusepath{stroke}%
\end{pgfscope}%
\begin{pgfscope}%
\pgfsetrectcap%
\pgfsetmiterjoin%
\pgfsetlinewidth{1.003750pt}%
\definecolor{currentstroke}{rgb}{0.800000,0.800000,0.800000}%
\pgfsetstrokecolor{currentstroke}%
\pgfsetdash{}{0pt}%
\pgfpathmoveto{\pgfqpoint{12.211765in}{3.197368in}}%
\pgfpathlineto{\pgfqpoint{12.211765in}{4.176316in}}%
\pgfusepath{stroke}%
\end{pgfscope}%
\begin{pgfscope}%
\pgfsetrectcap%
\pgfsetmiterjoin%
\pgfsetlinewidth{1.003750pt}%
\definecolor{currentstroke}{rgb}{0.800000,0.800000,0.800000}%
\pgfsetstrokecolor{currentstroke}%
\pgfsetdash{}{0pt}%
\pgfpathmoveto{\pgfqpoint{14.400000in}{3.197368in}}%
\pgfpathlineto{\pgfqpoint{14.400000in}{4.176316in}}%
\pgfusepath{stroke}%
\end{pgfscope}%
\begin{pgfscope}%
\pgfsetrectcap%
\pgfsetmiterjoin%
\pgfsetlinewidth{1.003750pt}%
\definecolor{currentstroke}{rgb}{0.800000,0.800000,0.800000}%
\pgfsetstrokecolor{currentstroke}%
\pgfsetdash{}{0pt}%
\pgfpathmoveto{\pgfqpoint{12.211765in}{4.176316in}}%
\pgfpathlineto{\pgfqpoint{14.400000in}{4.176316in}}%
\pgfusepath{stroke}%
\end{pgfscope}%
\begin{pgfscope}%
\pgfsetrectcap%
\pgfsetmiterjoin%
\pgfsetlinewidth{1.003750pt}%
\definecolor{currentstroke}{rgb}{0.800000,0.800000,0.800000}%
\pgfsetstrokecolor{currentstroke}%
\pgfsetdash{}{0pt}%
\pgfpathmoveto{\pgfqpoint{12.211765in}{3.197368in}}%
\pgfpathlineto{\pgfqpoint{14.400000in}{3.197368in}}%
\pgfusepath{stroke}%
\end{pgfscope}%
\begin{pgfscope}%
\pgfsetbuttcap%
\pgfsetmiterjoin%
\definecolor{currentfill}{rgb}{1.000000,1.000000,1.000000}%
\pgfsetfillcolor{currentfill}%
\pgfsetlinewidth{0.000000pt}%
\definecolor{currentstroke}{rgb}{0.000000,0.000000,0.000000}%
\pgfsetstrokecolor{currentstroke}%
\pgfsetstrokeopacity{0.000000}%
\pgfsetdash{}{0pt}%
\pgfpathmoveto{\pgfqpoint{2.000000in}{1.973684in}}%
\pgfpathlineto{\pgfqpoint{6.376471in}{1.973684in}}%
\pgfpathlineto{\pgfqpoint{6.376471in}{2.952632in}}%
\pgfpathlineto{\pgfqpoint{2.000000in}{2.952632in}}%
\pgfpathclose%
\pgfusepath{fill}%
\end{pgfscope}%
\begin{pgfscope}%
\pgfpathrectangle{\pgfqpoint{2.000000in}{1.973684in}}{\pgfqpoint{4.376471in}{0.978947in}} %
\pgfusepath{clip}%
\pgfsetroundcap%
\pgfsetroundjoin%
\pgfsetlinewidth{1.003750pt}%
\definecolor{currentstroke}{rgb}{0.800000,0.800000,0.800000}%
\pgfsetstrokecolor{currentstroke}%
\pgfsetdash{}{0pt}%
\pgfpathmoveto{\pgfqpoint{2.000000in}{1.973684in}}%
\pgfpathlineto{\pgfqpoint{2.000000in}{2.952632in}}%
\pgfusepath{stroke}%
\end{pgfscope}%
\begin{pgfscope}%
\pgfpathrectangle{\pgfqpoint{2.000000in}{1.973684in}}{\pgfqpoint{4.376471in}{0.978947in}} %
\pgfusepath{clip}%
\pgfsetroundcap%
\pgfsetroundjoin%
\pgfsetlinewidth{1.003750pt}%
\definecolor{currentstroke}{rgb}{0.800000,0.800000,0.800000}%
\pgfsetstrokecolor{currentstroke}%
\pgfsetdash{}{0pt}%
\pgfpathmoveto{\pgfqpoint{2.875294in}{1.973684in}}%
\pgfpathlineto{\pgfqpoint{2.875294in}{2.952632in}}%
\pgfusepath{stroke}%
\end{pgfscope}%
\begin{pgfscope}%
\pgfpathrectangle{\pgfqpoint{2.000000in}{1.973684in}}{\pgfqpoint{4.376471in}{0.978947in}} %
\pgfusepath{clip}%
\pgfsetroundcap%
\pgfsetroundjoin%
\pgfsetlinewidth{1.003750pt}%
\definecolor{currentstroke}{rgb}{0.800000,0.800000,0.800000}%
\pgfsetstrokecolor{currentstroke}%
\pgfsetdash{}{0pt}%
\pgfpathmoveto{\pgfqpoint{3.750588in}{1.973684in}}%
\pgfpathlineto{\pgfqpoint{3.750588in}{2.952632in}}%
\pgfusepath{stroke}%
\end{pgfscope}%
\begin{pgfscope}%
\pgfpathrectangle{\pgfqpoint{2.000000in}{1.973684in}}{\pgfqpoint{4.376471in}{0.978947in}} %
\pgfusepath{clip}%
\pgfsetroundcap%
\pgfsetroundjoin%
\pgfsetlinewidth{1.003750pt}%
\definecolor{currentstroke}{rgb}{0.800000,0.800000,0.800000}%
\pgfsetstrokecolor{currentstroke}%
\pgfsetdash{}{0pt}%
\pgfpathmoveto{\pgfqpoint{4.625882in}{1.973684in}}%
\pgfpathlineto{\pgfqpoint{4.625882in}{2.952632in}}%
\pgfusepath{stroke}%
\end{pgfscope}%
\begin{pgfscope}%
\pgfpathrectangle{\pgfqpoint{2.000000in}{1.973684in}}{\pgfqpoint{4.376471in}{0.978947in}} %
\pgfusepath{clip}%
\pgfsetroundcap%
\pgfsetroundjoin%
\pgfsetlinewidth{1.003750pt}%
\definecolor{currentstroke}{rgb}{0.800000,0.800000,0.800000}%
\pgfsetstrokecolor{currentstroke}%
\pgfsetdash{}{0pt}%
\pgfpathmoveto{\pgfqpoint{5.501176in}{1.973684in}}%
\pgfpathlineto{\pgfqpoint{5.501176in}{2.952632in}}%
\pgfusepath{stroke}%
\end{pgfscope}%
\begin{pgfscope}%
\pgfpathrectangle{\pgfqpoint{2.000000in}{1.973684in}}{\pgfqpoint{4.376471in}{0.978947in}} %
\pgfusepath{clip}%
\pgfsetroundcap%
\pgfsetroundjoin%
\pgfsetlinewidth{1.003750pt}%
\definecolor{currentstroke}{rgb}{0.800000,0.800000,0.800000}%
\pgfsetstrokecolor{currentstroke}%
\pgfsetdash{}{0pt}%
\pgfpathmoveto{\pgfqpoint{6.376471in}{1.973684in}}%
\pgfpathlineto{\pgfqpoint{6.376471in}{2.952632in}}%
\pgfusepath{stroke}%
\end{pgfscope}%
\begin{pgfscope}%
\pgfpathrectangle{\pgfqpoint{2.000000in}{1.973684in}}{\pgfqpoint{4.376471in}{0.978947in}} %
\pgfusepath{clip}%
\pgfsetroundcap%
\pgfsetroundjoin%
\pgfsetlinewidth{1.003750pt}%
\definecolor{currentstroke}{rgb}{0.800000,0.800000,0.800000}%
\pgfsetstrokecolor{currentstroke}%
\pgfsetdash{}{0pt}%
\pgfpathmoveto{\pgfqpoint{2.000000in}{2.136842in}}%
\pgfpathlineto{\pgfqpoint{6.376471in}{2.136842in}}%
\pgfusepath{stroke}%
\end{pgfscope}%
\begin{pgfscope}%
\definecolor{textcolor}{rgb}{0.150000,0.150000,0.150000}%
\pgfsetstrokecolor{textcolor}%
\pgfsetfillcolor{textcolor}%
\pgftext[x=1.902778in,y=2.136842in,right,]{\color{textcolor}\sffamily\fontsize{10.000000}{12.000000}\selectfont \(\displaystyle -1\)}%
\end{pgfscope}%
\begin{pgfscope}%
\pgfpathrectangle{\pgfqpoint{2.000000in}{1.973684in}}{\pgfqpoint{4.376471in}{0.978947in}} %
\pgfusepath{clip}%
\pgfsetroundcap%
\pgfsetroundjoin%
\pgfsetlinewidth{1.003750pt}%
\definecolor{currentstroke}{rgb}{0.800000,0.800000,0.800000}%
\pgfsetstrokecolor{currentstroke}%
\pgfsetdash{}{0pt}%
\pgfpathmoveto{\pgfqpoint{2.000000in}{2.340789in}}%
\pgfpathlineto{\pgfqpoint{6.376471in}{2.340789in}}%
\pgfusepath{stroke}%
\end{pgfscope}%
\begin{pgfscope}%
\definecolor{textcolor}{rgb}{0.150000,0.150000,0.150000}%
\pgfsetstrokecolor{textcolor}%
\pgfsetfillcolor{textcolor}%
\pgftext[x=1.902778in,y=2.340789in,right,]{\color{textcolor}\sffamily\fontsize{10.000000}{12.000000}\selectfont \(\displaystyle 0\)}%
\end{pgfscope}%
\begin{pgfscope}%
\pgfpathrectangle{\pgfqpoint{2.000000in}{1.973684in}}{\pgfqpoint{4.376471in}{0.978947in}} %
\pgfusepath{clip}%
\pgfsetroundcap%
\pgfsetroundjoin%
\pgfsetlinewidth{1.003750pt}%
\definecolor{currentstroke}{rgb}{0.800000,0.800000,0.800000}%
\pgfsetstrokecolor{currentstroke}%
\pgfsetdash{}{0pt}%
\pgfpathmoveto{\pgfqpoint{2.000000in}{2.544737in}}%
\pgfpathlineto{\pgfqpoint{6.376471in}{2.544737in}}%
\pgfusepath{stroke}%
\end{pgfscope}%
\begin{pgfscope}%
\definecolor{textcolor}{rgb}{0.150000,0.150000,0.150000}%
\pgfsetstrokecolor{textcolor}%
\pgfsetfillcolor{textcolor}%
\pgftext[x=1.902778in,y=2.544737in,right,]{\color{textcolor}\sffamily\fontsize{10.000000}{12.000000}\selectfont \(\displaystyle 1\)}%
\end{pgfscope}%
\begin{pgfscope}%
\pgfpathrectangle{\pgfqpoint{2.000000in}{1.973684in}}{\pgfqpoint{4.376471in}{0.978947in}} %
\pgfusepath{clip}%
\pgfsetroundcap%
\pgfsetroundjoin%
\pgfsetlinewidth{1.003750pt}%
\definecolor{currentstroke}{rgb}{0.800000,0.800000,0.800000}%
\pgfsetstrokecolor{currentstroke}%
\pgfsetdash{}{0pt}%
\pgfpathmoveto{\pgfqpoint{2.000000in}{2.748684in}}%
\pgfpathlineto{\pgfqpoint{6.376471in}{2.748684in}}%
\pgfusepath{stroke}%
\end{pgfscope}%
\begin{pgfscope}%
\definecolor{textcolor}{rgb}{0.150000,0.150000,0.150000}%
\pgfsetstrokecolor{textcolor}%
\pgfsetfillcolor{textcolor}%
\pgftext[x=1.902778in,y=2.748684in,right,]{\color{textcolor}\sffamily\fontsize{10.000000}{12.000000}\selectfont \(\displaystyle 2\)}%
\end{pgfscope}%
\begin{pgfscope}%
\pgfpathrectangle{\pgfqpoint{2.000000in}{1.973684in}}{\pgfqpoint{4.376471in}{0.978947in}} %
\pgfusepath{clip}%
\pgfsetroundcap%
\pgfsetroundjoin%
\pgfsetlinewidth{1.003750pt}%
\definecolor{currentstroke}{rgb}{0.800000,0.800000,0.800000}%
\pgfsetstrokecolor{currentstroke}%
\pgfsetdash{}{0pt}%
\pgfpathmoveto{\pgfqpoint{2.000000in}{2.952632in}}%
\pgfpathlineto{\pgfqpoint{6.376471in}{2.952632in}}%
\pgfusepath{stroke}%
\end{pgfscope}%
\begin{pgfscope}%
\definecolor{textcolor}{rgb}{0.150000,0.150000,0.150000}%
\pgfsetstrokecolor{textcolor}%
\pgfsetfillcolor{textcolor}%
\pgftext[x=1.902778in,y=2.952632in,right,]{\color{textcolor}\sffamily\fontsize{10.000000}{12.000000}\selectfont \(\displaystyle 3\)}%
\end{pgfscope}%
\begin{pgfscope}%
\definecolor{textcolor}{rgb}{0.150000,0.150000,0.150000}%
\pgfsetstrokecolor{textcolor}%
\pgfsetfillcolor{textcolor}%
\pgftext[x=1.655864in,y=2.463158in,,bottom,rotate=90.000000]{\color{textcolor}\sffamily\fontsize{11.000000}{13.200000}\selectfont y}%
\end{pgfscope}%
\begin{pgfscope}%
\pgfpathrectangle{\pgfqpoint{2.000000in}{1.973684in}}{\pgfqpoint{4.376471in}{0.978947in}} %
\pgfusepath{clip}%
\pgfsetbuttcap%
\pgfsetroundjoin%
\definecolor{currentfill}{rgb}{1.000000,0.000000,0.000000}%
\pgfsetfillcolor{currentfill}%
\pgfsetlinewidth{2.007500pt}%
\definecolor{currentstroke}{rgb}{1.000000,0.000000,0.000000}%
\pgfsetstrokecolor{currentstroke}%
\pgfsetdash{}{0pt}%
\pgfpathmoveto{\pgfqpoint{4.765731in}{2.531260in}}%
\pgfpathlineto{\pgfqpoint{4.827844in}{2.531260in}}%
\pgfpathmoveto{\pgfqpoint{4.796787in}{2.500203in}}%
\pgfpathlineto{\pgfqpoint{4.796787in}{2.562316in}}%
\pgfusepath{stroke,fill}%
\end{pgfscope}%
\begin{pgfscope}%
\pgfpathrectangle{\pgfqpoint{2.000000in}{1.973684in}}{\pgfqpoint{4.376471in}{0.978947in}} %
\pgfusepath{clip}%
\pgfsetbuttcap%
\pgfsetroundjoin%
\definecolor{currentfill}{rgb}{1.000000,0.000000,0.000000}%
\pgfsetfillcolor{currentfill}%
\pgfsetlinewidth{2.007500pt}%
\definecolor{currentstroke}{rgb}{1.000000,0.000000,0.000000}%
\pgfsetstrokecolor{currentstroke}%
\pgfsetdash{}{0pt}%
\pgfpathmoveto{\pgfqpoint{5.348242in}{2.214048in}}%
\pgfpathlineto{\pgfqpoint{5.410355in}{2.214048in}}%
\pgfpathmoveto{\pgfqpoint{5.379298in}{2.182991in}}%
\pgfpathlineto{\pgfqpoint{5.379298in}{2.245104in}}%
\pgfusepath{stroke,fill}%
\end{pgfscope}%
\begin{pgfscope}%
\pgfpathrectangle{\pgfqpoint{2.000000in}{1.973684in}}{\pgfqpoint{4.376471in}{0.978947in}} %
\pgfusepath{clip}%
\pgfsetbuttcap%
\pgfsetroundjoin%
\definecolor{currentfill}{rgb}{1.000000,0.000000,0.000000}%
\pgfsetfillcolor{currentfill}%
\pgfsetlinewidth{2.007500pt}%
\definecolor{currentstroke}{rgb}{1.000000,0.000000,0.000000}%
\pgfsetstrokecolor{currentstroke}%
\pgfsetdash{}{0pt}%
\pgfpathmoveto{\pgfqpoint{4.954619in}{2.584788in}}%
\pgfpathlineto{\pgfqpoint{5.016732in}{2.584788in}}%
\pgfpathmoveto{\pgfqpoint{4.985675in}{2.553731in}}%
\pgfpathlineto{\pgfqpoint{4.985675in}{2.615844in}}%
\pgfusepath{stroke,fill}%
\end{pgfscope}%
\begin{pgfscope}%
\pgfpathrectangle{\pgfqpoint{2.000000in}{1.973684in}}{\pgfqpoint{4.376471in}{0.978947in}} %
\pgfusepath{clip}%
\pgfsetbuttcap%
\pgfsetroundjoin%
\definecolor{currentfill}{rgb}{1.000000,0.000000,0.000000}%
\pgfsetfillcolor{currentfill}%
\pgfsetlinewidth{2.007500pt}%
\definecolor{currentstroke}{rgb}{1.000000,0.000000,0.000000}%
\pgfsetstrokecolor{currentstroke}%
\pgfsetdash{}{0pt}%
\pgfpathmoveto{\pgfqpoint{4.751970in}{2.473876in}}%
\pgfpathlineto{\pgfqpoint{4.814083in}{2.473876in}}%
\pgfpathmoveto{\pgfqpoint{4.783026in}{2.442819in}}%
\pgfpathlineto{\pgfqpoint{4.783026in}{2.504932in}}%
\pgfusepath{stroke,fill}%
\end{pgfscope}%
\begin{pgfscope}%
\pgfpathrectangle{\pgfqpoint{2.000000in}{1.973684in}}{\pgfqpoint{4.376471in}{0.978947in}} %
\pgfusepath{clip}%
\pgfsetbuttcap%
\pgfsetroundjoin%
\definecolor{currentfill}{rgb}{1.000000,0.000000,0.000000}%
\pgfsetfillcolor{currentfill}%
\pgfsetlinewidth{2.007500pt}%
\definecolor{currentstroke}{rgb}{1.000000,0.000000,0.000000}%
\pgfsetstrokecolor{currentstroke}%
\pgfsetdash{}{0pt}%
\pgfpathmoveto{\pgfqpoint{4.327528in}{2.147482in}}%
\pgfpathlineto{\pgfqpoint{4.389641in}{2.147482in}}%
\pgfpathmoveto{\pgfqpoint{4.358584in}{2.116425in}}%
\pgfpathlineto{\pgfqpoint{4.358584in}{2.178538in}}%
\pgfusepath{stroke,fill}%
\end{pgfscope}%
\begin{pgfscope}%
\pgfpathrectangle{\pgfqpoint{2.000000in}{1.973684in}}{\pgfqpoint{4.376471in}{0.978947in}} %
\pgfusepath{clip}%
\pgfsetbuttcap%
\pgfsetroundjoin%
\definecolor{currentfill}{rgb}{1.000000,0.000000,0.000000}%
\pgfsetfillcolor{currentfill}%
\pgfsetlinewidth{2.007500pt}%
\definecolor{currentstroke}{rgb}{1.000000,0.000000,0.000000}%
\pgfsetstrokecolor{currentstroke}%
\pgfsetdash{}{0pt}%
\pgfpathmoveto{\pgfqpoint{5.105627in}{2.416946in}}%
\pgfpathlineto{\pgfqpoint{5.167740in}{2.416946in}}%
\pgfpathmoveto{\pgfqpoint{5.136683in}{2.385889in}}%
\pgfpathlineto{\pgfqpoint{5.136683in}{2.448002in}}%
\pgfusepath{stroke,fill}%
\end{pgfscope}%
\begin{pgfscope}%
\pgfpathrectangle{\pgfqpoint{2.000000in}{1.973684in}}{\pgfqpoint{4.376471in}{0.978947in}} %
\pgfusepath{clip}%
\pgfsetbuttcap%
\pgfsetroundjoin%
\definecolor{currentfill}{rgb}{1.000000,0.000000,0.000000}%
\pgfsetfillcolor{currentfill}%
\pgfsetlinewidth{2.007500pt}%
\definecolor{currentstroke}{rgb}{1.000000,0.000000,0.000000}%
\pgfsetstrokecolor{currentstroke}%
\pgfsetdash{}{0pt}%
\pgfpathmoveto{\pgfqpoint{4.376308in}{2.184975in}}%
\pgfpathlineto{\pgfqpoint{4.438421in}{2.184975in}}%
\pgfpathmoveto{\pgfqpoint{4.407364in}{2.153919in}}%
\pgfpathlineto{\pgfqpoint{4.407364in}{2.216032in}}%
\pgfusepath{stroke,fill}%
\end{pgfscope}%
\begin{pgfscope}%
\pgfpathrectangle{\pgfqpoint{2.000000in}{1.973684in}}{\pgfqpoint{4.376471in}{0.978947in}} %
\pgfusepath{clip}%
\pgfsetbuttcap%
\pgfsetroundjoin%
\definecolor{currentfill}{rgb}{1.000000,0.000000,0.000000}%
\pgfsetfillcolor{currentfill}%
\pgfsetlinewidth{2.007500pt}%
\definecolor{currentstroke}{rgb}{1.000000,0.000000,0.000000}%
\pgfsetstrokecolor{currentstroke}%
\pgfsetdash{}{0pt}%
\pgfpathmoveto{\pgfqpoint{5.966492in}{2.825094in}}%
\pgfpathlineto{\pgfqpoint{6.028605in}{2.825094in}}%
\pgfpathmoveto{\pgfqpoint{5.997549in}{2.794037in}}%
\pgfpathlineto{\pgfqpoint{5.997549in}{2.856150in}}%
\pgfusepath{stroke,fill}%
\end{pgfscope}%
\begin{pgfscope}%
\pgfpathrectangle{\pgfqpoint{2.000000in}{1.973684in}}{\pgfqpoint{4.376471in}{0.978947in}} %
\pgfusepath{clip}%
\pgfsetbuttcap%
\pgfsetroundjoin%
\definecolor{currentfill}{rgb}{1.000000,0.000000,0.000000}%
\pgfsetfillcolor{currentfill}%
\pgfsetlinewidth{2.007500pt}%
\definecolor{currentstroke}{rgb}{1.000000,0.000000,0.000000}%
\pgfsetstrokecolor{currentstroke}%
\pgfsetdash{}{0pt}%
\pgfpathmoveto{\pgfqpoint{6.218191in}{2.725496in}}%
\pgfpathlineto{\pgfqpoint{6.280304in}{2.725496in}}%
\pgfpathmoveto{\pgfqpoint{6.249248in}{2.694440in}}%
\pgfpathlineto{\pgfqpoint{6.249248in}{2.756553in}}%
\pgfusepath{stroke,fill}%
\end{pgfscope}%
\begin{pgfscope}%
\pgfpathrectangle{\pgfqpoint{2.000000in}{1.973684in}}{\pgfqpoint{4.376471in}{0.978947in}} %
\pgfusepath{clip}%
\pgfsetbuttcap%
\pgfsetroundjoin%
\definecolor{currentfill}{rgb}{1.000000,0.000000,0.000000}%
\pgfsetfillcolor{currentfill}%
\pgfsetlinewidth{2.007500pt}%
\definecolor{currentstroke}{rgb}{1.000000,0.000000,0.000000}%
\pgfsetstrokecolor{currentstroke}%
\pgfsetdash{}{0pt}%
\pgfpathmoveto{\pgfqpoint{4.186734in}{2.222892in}}%
\pgfpathlineto{\pgfqpoint{4.248847in}{2.222892in}}%
\pgfpathmoveto{\pgfqpoint{4.217791in}{2.191836in}}%
\pgfpathlineto{\pgfqpoint{4.217791in}{2.253949in}}%
\pgfusepath{stroke,fill}%
\end{pgfscope}%
\begin{pgfscope}%
\pgfpathrectangle{\pgfqpoint{2.000000in}{1.973684in}}{\pgfqpoint{4.376471in}{0.978947in}} %
\pgfusepath{clip}%
\pgfsetbuttcap%
\pgfsetroundjoin%
\definecolor{currentfill}{rgb}{1.000000,0.000000,0.000000}%
\pgfsetfillcolor{currentfill}%
\pgfsetlinewidth{2.007500pt}%
\definecolor{currentstroke}{rgb}{1.000000,0.000000,0.000000}%
\pgfsetstrokecolor{currentstroke}%
\pgfsetdash{}{0pt}%
\pgfpathmoveto{\pgfqpoint{5.616207in}{2.373068in}}%
\pgfpathlineto{\pgfqpoint{5.678320in}{2.373068in}}%
\pgfpathmoveto{\pgfqpoint{5.647263in}{2.342011in}}%
\pgfpathlineto{\pgfqpoint{5.647263in}{2.404124in}}%
\pgfusepath{stroke,fill}%
\end{pgfscope}%
\begin{pgfscope}%
\pgfpathrectangle{\pgfqpoint{2.000000in}{1.973684in}}{\pgfqpoint{4.376471in}{0.978947in}} %
\pgfusepath{clip}%
\pgfsetbuttcap%
\pgfsetroundjoin%
\definecolor{currentfill}{rgb}{1.000000,0.000000,0.000000}%
\pgfsetfillcolor{currentfill}%
\pgfsetlinewidth{2.007500pt}%
\definecolor{currentstroke}{rgb}{1.000000,0.000000,0.000000}%
\pgfsetstrokecolor{currentstroke}%
\pgfsetdash{}{0pt}%
\pgfpathmoveto{\pgfqpoint{4.695992in}{2.413163in}}%
\pgfpathlineto{\pgfqpoint{4.758105in}{2.413163in}}%
\pgfpathmoveto{\pgfqpoint{4.727049in}{2.382107in}}%
\pgfpathlineto{\pgfqpoint{4.727049in}{2.444220in}}%
\pgfusepath{stroke,fill}%
\end{pgfscope}%
\begin{pgfscope}%
\pgfpathrectangle{\pgfqpoint{2.000000in}{1.973684in}}{\pgfqpoint{4.376471in}{0.978947in}} %
\pgfusepath{clip}%
\pgfsetbuttcap%
\pgfsetroundjoin%
\definecolor{currentfill}{rgb}{1.000000,0.000000,0.000000}%
\pgfsetfillcolor{currentfill}%
\pgfsetlinewidth{2.007500pt}%
\definecolor{currentstroke}{rgb}{1.000000,0.000000,0.000000}%
\pgfsetstrokecolor{currentstroke}%
\pgfsetdash{}{0pt}%
\pgfpathmoveto{\pgfqpoint{4.833062in}{2.540719in}}%
\pgfpathlineto{\pgfqpoint{4.895175in}{2.540719in}}%
\pgfpathmoveto{\pgfqpoint{4.864118in}{2.509663in}}%
\pgfpathlineto{\pgfqpoint{4.864118in}{2.571776in}}%
\pgfusepath{stroke,fill}%
\end{pgfscope}%
\begin{pgfscope}%
\pgfpathrectangle{\pgfqpoint{2.000000in}{1.973684in}}{\pgfqpoint{4.376471in}{0.978947in}} %
\pgfusepath{clip}%
\pgfsetbuttcap%
\pgfsetroundjoin%
\definecolor{currentfill}{rgb}{1.000000,0.000000,0.000000}%
\pgfsetfillcolor{currentfill}%
\pgfsetlinewidth{2.007500pt}%
\definecolor{currentstroke}{rgb}{1.000000,0.000000,0.000000}%
\pgfsetstrokecolor{currentstroke}%
\pgfsetdash{}{0pt}%
\pgfpathmoveto{\pgfqpoint{6.084915in}{2.800883in}}%
\pgfpathlineto{\pgfqpoint{6.147028in}{2.800883in}}%
\pgfpathmoveto{\pgfqpoint{6.115971in}{2.769826in}}%
\pgfpathlineto{\pgfqpoint{6.115971in}{2.831939in}}%
\pgfusepath{stroke,fill}%
\end{pgfscope}%
\begin{pgfscope}%
\pgfpathrectangle{\pgfqpoint{2.000000in}{1.973684in}}{\pgfqpoint{4.376471in}{0.978947in}} %
\pgfusepath{clip}%
\pgfsetbuttcap%
\pgfsetroundjoin%
\definecolor{currentfill}{rgb}{1.000000,0.000000,0.000000}%
\pgfsetfillcolor{currentfill}%
\pgfsetlinewidth{2.007500pt}%
\definecolor{currentstroke}{rgb}{1.000000,0.000000,0.000000}%
\pgfsetstrokecolor{currentstroke}%
\pgfsetdash{}{0pt}%
\pgfpathmoveto{\pgfqpoint{3.092947in}{2.513494in}}%
\pgfpathlineto{\pgfqpoint{3.155060in}{2.513494in}}%
\pgfpathmoveto{\pgfqpoint{3.124004in}{2.482437in}}%
\pgfpathlineto{\pgfqpoint{3.124004in}{2.544550in}}%
\pgfusepath{stroke,fill}%
\end{pgfscope}%
\begin{pgfscope}%
\pgfpathrectangle{\pgfqpoint{2.000000in}{1.973684in}}{\pgfqpoint{4.376471in}{0.978947in}} %
\pgfusepath{clip}%
\pgfsetbuttcap%
\pgfsetroundjoin%
\definecolor{currentfill}{rgb}{1.000000,0.000000,0.000000}%
\pgfsetfillcolor{currentfill}%
\pgfsetlinewidth{2.007500pt}%
\definecolor{currentstroke}{rgb}{1.000000,0.000000,0.000000}%
\pgfsetstrokecolor{currentstroke}%
\pgfsetdash{}{0pt}%
\pgfpathmoveto{\pgfqpoint{3.149293in}{2.455860in}}%
\pgfpathlineto{\pgfqpoint{3.211406in}{2.455860in}}%
\pgfpathmoveto{\pgfqpoint{3.180349in}{2.424803in}}%
\pgfpathlineto{\pgfqpoint{3.180349in}{2.486916in}}%
\pgfusepath{stroke,fill}%
\end{pgfscope}%
\begin{pgfscope}%
\pgfpathrectangle{\pgfqpoint{2.000000in}{1.973684in}}{\pgfqpoint{4.376471in}{0.978947in}} %
\pgfusepath{clip}%
\pgfsetbuttcap%
\pgfsetroundjoin%
\definecolor{currentfill}{rgb}{1.000000,0.000000,0.000000}%
\pgfsetfillcolor{currentfill}%
\pgfsetlinewidth{2.007500pt}%
\definecolor{currentstroke}{rgb}{1.000000,0.000000,0.000000}%
\pgfsetstrokecolor{currentstroke}%
\pgfsetdash{}{0pt}%
\pgfpathmoveto{\pgfqpoint{2.915026in}{2.743140in}}%
\pgfpathlineto{\pgfqpoint{2.977139in}{2.743140in}}%
\pgfpathmoveto{\pgfqpoint{2.946082in}{2.712083in}}%
\pgfpathlineto{\pgfqpoint{2.946082in}{2.774196in}}%
\pgfusepath{stroke,fill}%
\end{pgfscope}%
\begin{pgfscope}%
\pgfpathrectangle{\pgfqpoint{2.000000in}{1.973684in}}{\pgfqpoint{4.376471in}{0.978947in}} %
\pgfusepath{clip}%
\pgfsetbuttcap%
\pgfsetroundjoin%
\definecolor{currentfill}{rgb}{1.000000,0.000000,0.000000}%
\pgfsetfillcolor{currentfill}%
\pgfsetlinewidth{2.007500pt}%
\definecolor{currentstroke}{rgb}{1.000000,0.000000,0.000000}%
\pgfsetstrokecolor{currentstroke}%
\pgfsetdash{}{0pt}%
\pgfpathmoveto{\pgfqpoint{5.759387in}{2.588736in}}%
\pgfpathlineto{\pgfqpoint{5.821500in}{2.588736in}}%
\pgfpathmoveto{\pgfqpoint{5.790443in}{2.557679in}}%
\pgfpathlineto{\pgfqpoint{5.790443in}{2.619792in}}%
\pgfusepath{stroke,fill}%
\end{pgfscope}%
\begin{pgfscope}%
\pgfpathrectangle{\pgfqpoint{2.000000in}{1.973684in}}{\pgfqpoint{4.376471in}{0.978947in}} %
\pgfusepath{clip}%
\pgfsetbuttcap%
\pgfsetroundjoin%
\definecolor{currentfill}{rgb}{1.000000,0.000000,0.000000}%
\pgfsetfillcolor{currentfill}%
\pgfsetlinewidth{2.007500pt}%
\definecolor{currentstroke}{rgb}{1.000000,0.000000,0.000000}%
\pgfsetstrokecolor{currentstroke}%
\pgfsetdash{}{0pt}%
\pgfpathmoveto{\pgfqpoint{5.568702in}{2.311154in}}%
\pgfpathlineto{\pgfqpoint{5.630815in}{2.311154in}}%
\pgfpathmoveto{\pgfqpoint{5.599758in}{2.280098in}}%
\pgfpathlineto{\pgfqpoint{5.599758in}{2.342211in}}%
\pgfusepath{stroke,fill}%
\end{pgfscope}%
\begin{pgfscope}%
\pgfpathrectangle{\pgfqpoint{2.000000in}{1.973684in}}{\pgfqpoint{4.376471in}{0.978947in}} %
\pgfusepath{clip}%
\pgfsetbuttcap%
\pgfsetroundjoin%
\definecolor{currentfill}{rgb}{1.000000,0.000000,0.000000}%
\pgfsetfillcolor{currentfill}%
\pgfsetlinewidth{2.007500pt}%
\definecolor{currentstroke}{rgb}{1.000000,0.000000,0.000000}%
\pgfsetstrokecolor{currentstroke}%
\pgfsetdash{}{0pt}%
\pgfpathmoveto{\pgfqpoint{5.890304in}{2.718517in}}%
\pgfpathlineto{\pgfqpoint{5.952417in}{2.718517in}}%
\pgfpathmoveto{\pgfqpoint{5.921360in}{2.687461in}}%
\pgfpathlineto{\pgfqpoint{5.921360in}{2.749574in}}%
\pgfusepath{stroke,fill}%
\end{pgfscope}%
\begin{pgfscope}%
\pgfpathrectangle{\pgfqpoint{2.000000in}{1.973684in}}{\pgfqpoint{4.376471in}{0.978947in}} %
\pgfusepath{clip}%
\pgfsetbuttcap%
\pgfsetroundjoin%
\definecolor{currentfill}{rgb}{1.000000,0.000000,0.000000}%
\pgfsetfillcolor{currentfill}%
\pgfsetlinewidth{2.007500pt}%
\definecolor{currentstroke}{rgb}{1.000000,0.000000,0.000000}%
\pgfsetstrokecolor{currentstroke}%
\pgfsetdash{}{0pt}%
\pgfpathmoveto{\pgfqpoint{6.270553in}{2.649754in}}%
\pgfpathlineto{\pgfqpoint{6.332666in}{2.649754in}}%
\pgfpathmoveto{\pgfqpoint{6.301610in}{2.618698in}}%
\pgfpathlineto{\pgfqpoint{6.301610in}{2.680811in}}%
\pgfusepath{stroke,fill}%
\end{pgfscope}%
\begin{pgfscope}%
\pgfpathrectangle{\pgfqpoint{2.000000in}{1.973684in}}{\pgfqpoint{4.376471in}{0.978947in}} %
\pgfusepath{clip}%
\pgfsetbuttcap%
\pgfsetroundjoin%
\definecolor{currentfill}{rgb}{1.000000,0.000000,0.000000}%
\pgfsetfillcolor{currentfill}%
\pgfsetlinewidth{2.007500pt}%
\definecolor{currentstroke}{rgb}{1.000000,0.000000,0.000000}%
\pgfsetstrokecolor{currentstroke}%
\pgfsetdash{}{0pt}%
\pgfpathmoveto{\pgfqpoint{5.642233in}{2.466321in}}%
\pgfpathlineto{\pgfqpoint{5.704346in}{2.466321in}}%
\pgfpathmoveto{\pgfqpoint{5.673289in}{2.435265in}}%
\pgfpathlineto{\pgfqpoint{5.673289in}{2.497378in}}%
\pgfusepath{stroke,fill}%
\end{pgfscope}%
\begin{pgfscope}%
\pgfpathrectangle{\pgfqpoint{2.000000in}{1.973684in}}{\pgfqpoint{4.376471in}{0.978947in}} %
\pgfusepath{clip}%
\pgfsetbuttcap%
\pgfsetroundjoin%
\definecolor{currentfill}{rgb}{1.000000,0.000000,0.000000}%
\pgfsetfillcolor{currentfill}%
\pgfsetlinewidth{2.007500pt}%
\definecolor{currentstroke}{rgb}{1.000000,0.000000,0.000000}%
\pgfsetstrokecolor{currentstroke}%
\pgfsetdash{}{0pt}%
\pgfpathmoveto{\pgfqpoint{4.459958in}{2.190781in}}%
\pgfpathlineto{\pgfqpoint{4.522071in}{2.190781in}}%
\pgfpathmoveto{\pgfqpoint{4.491015in}{2.159724in}}%
\pgfpathlineto{\pgfqpoint{4.491015in}{2.221837in}}%
\pgfusepath{stroke,fill}%
\end{pgfscope}%
\begin{pgfscope}%
\pgfpathrectangle{\pgfqpoint{2.000000in}{1.973684in}}{\pgfqpoint{4.376471in}{0.978947in}} %
\pgfusepath{clip}%
\pgfsetbuttcap%
\pgfsetroundjoin%
\definecolor{currentfill}{rgb}{1.000000,0.000000,0.000000}%
\pgfsetfillcolor{currentfill}%
\pgfsetlinewidth{2.007500pt}%
\definecolor{currentstroke}{rgb}{1.000000,0.000000,0.000000}%
\pgfsetstrokecolor{currentstroke}%
\pgfsetdash{}{0pt}%
\pgfpathmoveto{\pgfqpoint{5.577008in}{2.333191in}}%
\pgfpathlineto{\pgfqpoint{5.639121in}{2.333191in}}%
\pgfpathmoveto{\pgfqpoint{5.608065in}{2.302134in}}%
\pgfpathlineto{\pgfqpoint{5.608065in}{2.364247in}}%
\pgfusepath{stroke,fill}%
\end{pgfscope}%
\begin{pgfscope}%
\pgfpathrectangle{\pgfqpoint{2.000000in}{1.973684in}}{\pgfqpoint{4.376471in}{0.978947in}} %
\pgfusepath{clip}%
\pgfsetbuttcap%
\pgfsetroundjoin%
\definecolor{currentfill}{rgb}{1.000000,0.000000,0.000000}%
\pgfsetfillcolor{currentfill}%
\pgfsetlinewidth{2.007500pt}%
\definecolor{currentstroke}{rgb}{1.000000,0.000000,0.000000}%
\pgfsetstrokecolor{currentstroke}%
\pgfsetdash{}{0pt}%
\pgfpathmoveto{\pgfqpoint{3.258337in}{2.353928in}}%
\pgfpathlineto{\pgfqpoint{3.320450in}{2.353928in}}%
\pgfpathmoveto{\pgfqpoint{3.289394in}{2.322872in}}%
\pgfpathlineto{\pgfqpoint{3.289394in}{2.384985in}}%
\pgfusepath{stroke,fill}%
\end{pgfscope}%
\begin{pgfscope}%
\pgfpathrectangle{\pgfqpoint{2.000000in}{1.973684in}}{\pgfqpoint{4.376471in}{0.978947in}} %
\pgfusepath{clip}%
\pgfsetbuttcap%
\pgfsetroundjoin%
\definecolor{currentfill}{rgb}{1.000000,0.000000,0.000000}%
\pgfsetfillcolor{currentfill}%
\pgfsetlinewidth{2.007500pt}%
\definecolor{currentstroke}{rgb}{1.000000,0.000000,0.000000}%
\pgfsetstrokecolor{currentstroke}%
\pgfsetdash{}{0pt}%
\pgfpathmoveto{\pgfqpoint{5.084714in}{2.457212in}}%
\pgfpathlineto{\pgfqpoint{5.146827in}{2.457212in}}%
\pgfpathmoveto{\pgfqpoint{5.115771in}{2.426156in}}%
\pgfpathlineto{\pgfqpoint{5.115771in}{2.488269in}}%
\pgfusepath{stroke,fill}%
\end{pgfscope}%
\begin{pgfscope}%
\pgfpathrectangle{\pgfqpoint{2.000000in}{1.973684in}}{\pgfqpoint{4.376471in}{0.978947in}} %
\pgfusepath{clip}%
\pgfsetbuttcap%
\pgfsetroundjoin%
\definecolor{currentfill}{rgb}{1.000000,0.000000,0.000000}%
\pgfsetfillcolor{currentfill}%
\pgfsetlinewidth{2.007500pt}%
\definecolor{currentstroke}{rgb}{1.000000,0.000000,0.000000}%
\pgfsetstrokecolor{currentstroke}%
\pgfsetdash{}{0pt}%
\pgfpathmoveto{\pgfqpoint{3.346143in}{2.362218in}}%
\pgfpathlineto{\pgfqpoint{3.408256in}{2.362218in}}%
\pgfpathmoveto{\pgfqpoint{3.377199in}{2.331162in}}%
\pgfpathlineto{\pgfqpoint{3.377199in}{2.393275in}}%
\pgfusepath{stroke,fill}%
\end{pgfscope}%
\begin{pgfscope}%
\pgfpathrectangle{\pgfqpoint{2.000000in}{1.973684in}}{\pgfqpoint{4.376471in}{0.978947in}} %
\pgfusepath{clip}%
\pgfsetbuttcap%
\pgfsetroundjoin%
\definecolor{currentfill}{rgb}{1.000000,0.000000,0.000000}%
\pgfsetfillcolor{currentfill}%
\pgfsetlinewidth{2.007500pt}%
\definecolor{currentstroke}{rgb}{1.000000,0.000000,0.000000}%
\pgfsetstrokecolor{currentstroke}%
\pgfsetdash{}{0pt}%
\pgfpathmoveto{\pgfqpoint{6.151690in}{2.762412in}}%
\pgfpathlineto{\pgfqpoint{6.213803in}{2.762412in}}%
\pgfpathmoveto{\pgfqpoint{6.182747in}{2.731355in}}%
\pgfpathlineto{\pgfqpoint{6.182747in}{2.793468in}}%
\pgfusepath{stroke,fill}%
\end{pgfscope}%
\begin{pgfscope}%
\pgfpathrectangle{\pgfqpoint{2.000000in}{1.973684in}}{\pgfqpoint{4.376471in}{0.978947in}} %
\pgfusepath{clip}%
\pgfsetbuttcap%
\pgfsetroundjoin%
\definecolor{currentfill}{rgb}{1.000000,0.000000,0.000000}%
\pgfsetfillcolor{currentfill}%
\pgfsetlinewidth{2.007500pt}%
\definecolor{currentstroke}{rgb}{1.000000,0.000000,0.000000}%
\pgfsetstrokecolor{currentstroke}%
\pgfsetdash{}{0pt}%
\pgfpathmoveto{\pgfqpoint{4.671321in}{2.409615in}}%
\pgfpathlineto{\pgfqpoint{4.733434in}{2.409615in}}%
\pgfpathmoveto{\pgfqpoint{4.702377in}{2.378558in}}%
\pgfpathlineto{\pgfqpoint{4.702377in}{2.440671in}}%
\pgfusepath{stroke,fill}%
\end{pgfscope}%
\begin{pgfscope}%
\pgfpathrectangle{\pgfqpoint{2.000000in}{1.973684in}}{\pgfqpoint{4.376471in}{0.978947in}} %
\pgfusepath{clip}%
\pgfsetbuttcap%
\pgfsetroundjoin%
\definecolor{currentfill}{rgb}{1.000000,0.000000,0.000000}%
\pgfsetfillcolor{currentfill}%
\pgfsetlinewidth{2.007500pt}%
\definecolor{currentstroke}{rgb}{1.000000,0.000000,0.000000}%
\pgfsetstrokecolor{currentstroke}%
\pgfsetdash{}{0pt}%
\pgfpathmoveto{\pgfqpoint{4.296042in}{2.158497in}}%
\pgfpathlineto{\pgfqpoint{4.358155in}{2.158497in}}%
\pgfpathmoveto{\pgfqpoint{4.327099in}{2.127441in}}%
\pgfpathlineto{\pgfqpoint{4.327099in}{2.189554in}}%
\pgfusepath{stroke,fill}%
\end{pgfscope}%
\begin{pgfscope}%
\pgfpathrectangle{\pgfqpoint{2.000000in}{1.973684in}}{\pgfqpoint{4.376471in}{0.978947in}} %
\pgfusepath{clip}%
\pgfsetbuttcap%
\pgfsetroundjoin%
\definecolor{currentfill}{rgb}{0.000000,0.000000,0.000000}%
\pgfsetfillcolor{currentfill}%
\pgfsetlinewidth{0.301125pt}%
\definecolor{currentstroke}{rgb}{0.000000,0.000000,0.000000}%
\pgfsetstrokecolor{currentstroke}%
\pgfsetdash{}{0pt}%
\pgfsys@defobject{currentmarker}{\pgfqpoint{-0.015528in}{-0.015528in}}{\pgfqpoint{0.015528in}{0.015528in}}{%
\pgfpathmoveto{\pgfqpoint{0.000000in}{-0.015528in}}%
\pgfpathcurveto{\pgfqpoint{0.004118in}{-0.015528in}}{\pgfqpoint{0.008068in}{-0.013892in}}{\pgfqpoint{0.010980in}{-0.010980in}}%
\pgfpathcurveto{\pgfqpoint{0.013892in}{-0.008068in}}{\pgfqpoint{0.015528in}{-0.004118in}}{\pgfqpoint{0.015528in}{0.000000in}}%
\pgfpathcurveto{\pgfqpoint{0.015528in}{0.004118in}}{\pgfqpoint{0.013892in}{0.008068in}}{\pgfqpoint{0.010980in}{0.010980in}}%
\pgfpathcurveto{\pgfqpoint{0.008068in}{0.013892in}}{\pgfqpoint{0.004118in}{0.015528in}}{\pgfqpoint{0.000000in}{0.015528in}}%
\pgfpathcurveto{\pgfqpoint{-0.004118in}{0.015528in}}{\pgfqpoint{-0.008068in}{0.013892in}}{\pgfqpoint{-0.010980in}{0.010980in}}%
\pgfpathcurveto{\pgfqpoint{-0.013892in}{0.008068in}}{\pgfqpoint{-0.015528in}{0.004118in}}{\pgfqpoint{-0.015528in}{0.000000in}}%
\pgfpathcurveto{\pgfqpoint{-0.015528in}{-0.004118in}}{\pgfqpoint{-0.013892in}{-0.008068in}}{\pgfqpoint{-0.010980in}{-0.010980in}}%
\pgfpathcurveto{\pgfqpoint{-0.008068in}{-0.013892in}}{\pgfqpoint{-0.004118in}{-0.015528in}}{\pgfqpoint{0.000000in}{-0.015528in}}%
\pgfpathclose%
\pgfusepath{stroke,fill}%
}%
\begin{pgfscope}%
\pgfsys@transformshift{2.875294in}{2.807547in}%
\pgfsys@useobject{currentmarker}{}%
\end{pgfscope}%
\begin{pgfscope}%
\pgfsys@transformshift{2.892888in}{2.713355in}%
\pgfsys@useobject{currentmarker}{}%
\end{pgfscope}%
\begin{pgfscope}%
\pgfsys@transformshift{2.910482in}{2.804129in}%
\pgfsys@useobject{currentmarker}{}%
\end{pgfscope}%
\begin{pgfscope}%
\pgfsys@transformshift{2.928076in}{2.823266in}%
\pgfsys@useobject{currentmarker}{}%
\end{pgfscope}%
\begin{pgfscope}%
\pgfsys@transformshift{2.945670in}{2.758481in}%
\pgfsys@useobject{currentmarker}{}%
\end{pgfscope}%
\begin{pgfscope}%
\pgfsys@transformshift{2.963263in}{2.754380in}%
\pgfsys@useobject{currentmarker}{}%
\end{pgfscope}%
\begin{pgfscope}%
\pgfsys@transformshift{2.980857in}{2.630622in}%
\pgfsys@useobject{currentmarker}{}%
\end{pgfscope}%
\begin{pgfscope}%
\pgfsys@transformshift{2.998451in}{2.630227in}%
\pgfsys@useobject{currentmarker}{}%
\end{pgfscope}%
\begin{pgfscope}%
\pgfsys@transformshift{3.016045in}{2.570900in}%
\pgfsys@useobject{currentmarker}{}%
\end{pgfscope}%
\begin{pgfscope}%
\pgfsys@transformshift{3.033639in}{2.575614in}%
\pgfsys@useobject{currentmarker}{}%
\end{pgfscope}%
\begin{pgfscope}%
\pgfsys@transformshift{3.051233in}{2.502913in}%
\pgfsys@useobject{currentmarker}{}%
\end{pgfscope}%
\begin{pgfscope}%
\pgfsys@transformshift{3.068826in}{2.384290in}%
\pgfsys@useobject{currentmarker}{}%
\end{pgfscope}%
\begin{pgfscope}%
\pgfsys@transformshift{3.086420in}{2.554036in}%
\pgfsys@useobject{currentmarker}{}%
\end{pgfscope}%
\begin{pgfscope}%
\pgfsys@transformshift{3.104014in}{2.471885in}%
\pgfsys@useobject{currentmarker}{}%
\end{pgfscope}%
\begin{pgfscope}%
\pgfsys@transformshift{3.121608in}{2.324989in}%
\pgfsys@useobject{currentmarker}{}%
\end{pgfscope}%
\begin{pgfscope}%
\pgfsys@transformshift{3.139202in}{2.518392in}%
\pgfsys@useobject{currentmarker}{}%
\end{pgfscope}%
\begin{pgfscope}%
\pgfsys@transformshift{3.156796in}{2.360390in}%
\pgfsys@useobject{currentmarker}{}%
\end{pgfscope}%
\begin{pgfscope}%
\pgfsys@transformshift{3.174390in}{2.441767in}%
\pgfsys@useobject{currentmarker}{}%
\end{pgfscope}%
\begin{pgfscope}%
\pgfsys@transformshift{3.191983in}{2.496324in}%
\pgfsys@useobject{currentmarker}{}%
\end{pgfscope}%
\begin{pgfscope}%
\pgfsys@transformshift{3.209577in}{2.422660in}%
\pgfsys@useobject{currentmarker}{}%
\end{pgfscope}%
\begin{pgfscope}%
\pgfsys@transformshift{3.227171in}{2.515310in}%
\pgfsys@useobject{currentmarker}{}%
\end{pgfscope}%
\begin{pgfscope}%
\pgfsys@transformshift{3.244765in}{2.264938in}%
\pgfsys@useobject{currentmarker}{}%
\end{pgfscope}%
\begin{pgfscope}%
\pgfsys@transformshift{3.262359in}{2.425762in}%
\pgfsys@useobject{currentmarker}{}%
\end{pgfscope}%
\begin{pgfscope}%
\pgfsys@transformshift{3.279953in}{2.310913in}%
\pgfsys@useobject{currentmarker}{}%
\end{pgfscope}%
\begin{pgfscope}%
\pgfsys@transformshift{3.297547in}{2.290072in}%
\pgfsys@useobject{currentmarker}{}%
\end{pgfscope}%
\begin{pgfscope}%
\pgfsys@transformshift{3.315140in}{2.320035in}%
\pgfsys@useobject{currentmarker}{}%
\end{pgfscope}%
\begin{pgfscope}%
\pgfsys@transformshift{3.332734in}{2.349489in}%
\pgfsys@useobject{currentmarker}{}%
\end{pgfscope}%
\begin{pgfscope}%
\pgfsys@transformshift{3.350328in}{2.391104in}%
\pgfsys@useobject{currentmarker}{}%
\end{pgfscope}%
\begin{pgfscope}%
\pgfsys@transformshift{3.367922in}{2.272497in}%
\pgfsys@useobject{currentmarker}{}%
\end{pgfscope}%
\begin{pgfscope}%
\pgfsys@transformshift{3.385516in}{2.490806in}%
\pgfsys@useobject{currentmarker}{}%
\end{pgfscope}%
\begin{pgfscope}%
\pgfsys@transformshift{3.403110in}{2.455626in}%
\pgfsys@useobject{currentmarker}{}%
\end{pgfscope}%
\begin{pgfscope}%
\pgfsys@transformshift{3.420704in}{2.262090in}%
\pgfsys@useobject{currentmarker}{}%
\end{pgfscope}%
\begin{pgfscope}%
\pgfsys@transformshift{3.438297in}{2.582363in}%
\pgfsys@useobject{currentmarker}{}%
\end{pgfscope}%
\begin{pgfscope}%
\pgfsys@transformshift{3.455891in}{2.636855in}%
\pgfsys@useobject{currentmarker}{}%
\end{pgfscope}%
\begin{pgfscope}%
\pgfsys@transformshift{3.473485in}{2.577536in}%
\pgfsys@useobject{currentmarker}{}%
\end{pgfscope}%
\begin{pgfscope}%
\pgfsys@transformshift{3.491079in}{2.453483in}%
\pgfsys@useobject{currentmarker}{}%
\end{pgfscope}%
\begin{pgfscope}%
\pgfsys@transformshift{3.508673in}{2.377630in}%
\pgfsys@useobject{currentmarker}{}%
\end{pgfscope}%
\begin{pgfscope}%
\pgfsys@transformshift{3.526267in}{2.609618in}%
\pgfsys@useobject{currentmarker}{}%
\end{pgfscope}%
\begin{pgfscope}%
\pgfsys@transformshift{3.543860in}{2.476330in}%
\pgfsys@useobject{currentmarker}{}%
\end{pgfscope}%
\begin{pgfscope}%
\pgfsys@transformshift{3.561454in}{2.657325in}%
\pgfsys@useobject{currentmarker}{}%
\end{pgfscope}%
\begin{pgfscope}%
\pgfsys@transformshift{3.579048in}{2.568798in}%
\pgfsys@useobject{currentmarker}{}%
\end{pgfscope}%
\begin{pgfscope}%
\pgfsys@transformshift{3.596642in}{2.661511in}%
\pgfsys@useobject{currentmarker}{}%
\end{pgfscope}%
\begin{pgfscope}%
\pgfsys@transformshift{3.614236in}{2.611893in}%
\pgfsys@useobject{currentmarker}{}%
\end{pgfscope}%
\begin{pgfscope}%
\pgfsys@transformshift{3.631830in}{2.660326in}%
\pgfsys@useobject{currentmarker}{}%
\end{pgfscope}%
\begin{pgfscope}%
\pgfsys@transformshift{3.649424in}{2.600979in}%
\pgfsys@useobject{currentmarker}{}%
\end{pgfscope}%
\begin{pgfscope}%
\pgfsys@transformshift{3.667017in}{2.792401in}%
\pgfsys@useobject{currentmarker}{}%
\end{pgfscope}%
\begin{pgfscope}%
\pgfsys@transformshift{3.684611in}{2.632210in}%
\pgfsys@useobject{currentmarker}{}%
\end{pgfscope}%
\begin{pgfscope}%
\pgfsys@transformshift{3.702205in}{2.667702in}%
\pgfsys@useobject{currentmarker}{}%
\end{pgfscope}%
\begin{pgfscope}%
\pgfsys@transformshift{3.719799in}{2.824526in}%
\pgfsys@useobject{currentmarker}{}%
\end{pgfscope}%
\begin{pgfscope}%
\pgfsys@transformshift{3.737393in}{2.499083in}%
\pgfsys@useobject{currentmarker}{}%
\end{pgfscope}%
\begin{pgfscope}%
\pgfsys@transformshift{3.754987in}{2.509147in}%
\pgfsys@useobject{currentmarker}{}%
\end{pgfscope}%
\begin{pgfscope}%
\pgfsys@transformshift{3.772581in}{2.737834in}%
\pgfsys@useobject{currentmarker}{}%
\end{pgfscope}%
\begin{pgfscope}%
\pgfsys@transformshift{3.790174in}{2.517682in}%
\pgfsys@useobject{currentmarker}{}%
\end{pgfscope}%
\begin{pgfscope}%
\pgfsys@transformshift{3.807768in}{2.831865in}%
\pgfsys@useobject{currentmarker}{}%
\end{pgfscope}%
\begin{pgfscope}%
\pgfsys@transformshift{3.825362in}{2.585872in}%
\pgfsys@useobject{currentmarker}{}%
\end{pgfscope}%
\begin{pgfscope}%
\pgfsys@transformshift{3.842956in}{2.544256in}%
\pgfsys@useobject{currentmarker}{}%
\end{pgfscope}%
\begin{pgfscope}%
\pgfsys@transformshift{3.860550in}{2.807077in}%
\pgfsys@useobject{currentmarker}{}%
\end{pgfscope}%
\begin{pgfscope}%
\pgfsys@transformshift{3.878144in}{2.750613in}%
\pgfsys@useobject{currentmarker}{}%
\end{pgfscope}%
\begin{pgfscope}%
\pgfsys@transformshift{3.895738in}{2.776955in}%
\pgfsys@useobject{currentmarker}{}%
\end{pgfscope}%
\begin{pgfscope}%
\pgfsys@transformshift{3.913331in}{2.664098in}%
\pgfsys@useobject{currentmarker}{}%
\end{pgfscope}%
\begin{pgfscope}%
\pgfsys@transformshift{3.930925in}{2.467509in}%
\pgfsys@useobject{currentmarker}{}%
\end{pgfscope}%
\begin{pgfscope}%
\pgfsys@transformshift{3.948519in}{2.732300in}%
\pgfsys@useobject{currentmarker}{}%
\end{pgfscope}%
\begin{pgfscope}%
\pgfsys@transformshift{3.966113in}{2.491099in}%
\pgfsys@useobject{currentmarker}{}%
\end{pgfscope}%
\begin{pgfscope}%
\pgfsys@transformshift{3.983707in}{2.580038in}%
\pgfsys@useobject{currentmarker}{}%
\end{pgfscope}%
\begin{pgfscope}%
\pgfsys@transformshift{4.001301in}{2.573630in}%
\pgfsys@useobject{currentmarker}{}%
\end{pgfscope}%
\begin{pgfscope}%
\pgfsys@transformshift{4.018894in}{2.439287in}%
\pgfsys@useobject{currentmarker}{}%
\end{pgfscope}%
\begin{pgfscope}%
\pgfsys@transformshift{4.036488in}{2.495197in}%
\pgfsys@useobject{currentmarker}{}%
\end{pgfscope}%
\begin{pgfscope}%
\pgfsys@transformshift{4.054082in}{2.503723in}%
\pgfsys@useobject{currentmarker}{}%
\end{pgfscope}%
\begin{pgfscope}%
\pgfsys@transformshift{4.071676in}{2.424999in}%
\pgfsys@useobject{currentmarker}{}%
\end{pgfscope}%
\begin{pgfscope}%
\pgfsys@transformshift{4.089270in}{2.251467in}%
\pgfsys@useobject{currentmarker}{}%
\end{pgfscope}%
\begin{pgfscope}%
\pgfsys@transformshift{4.106864in}{2.371193in}%
\pgfsys@useobject{currentmarker}{}%
\end{pgfscope}%
\begin{pgfscope}%
\pgfsys@transformshift{4.124458in}{2.453684in}%
\pgfsys@useobject{currentmarker}{}%
\end{pgfscope}%
\begin{pgfscope}%
\pgfsys@transformshift{4.142051in}{2.225893in}%
\pgfsys@useobject{currentmarker}{}%
\end{pgfscope}%
\begin{pgfscope}%
\pgfsys@transformshift{4.159645in}{2.260597in}%
\pgfsys@useobject{currentmarker}{}%
\end{pgfscope}%
\begin{pgfscope}%
\pgfsys@transformshift{4.177239in}{2.211646in}%
\pgfsys@useobject{currentmarker}{}%
\end{pgfscope}%
\begin{pgfscope}%
\pgfsys@transformshift{4.194833in}{2.425970in}%
\pgfsys@useobject{currentmarker}{}%
\end{pgfscope}%
\begin{pgfscope}%
\pgfsys@transformshift{4.212427in}{2.288698in}%
\pgfsys@useobject{currentmarker}{}%
\end{pgfscope}%
\begin{pgfscope}%
\pgfsys@transformshift{4.230021in}{2.245970in}%
\pgfsys@useobject{currentmarker}{}%
\end{pgfscope}%
\begin{pgfscope}%
\pgfsys@transformshift{4.247615in}{2.111856in}%
\pgfsys@useobject{currentmarker}{}%
\end{pgfscope}%
\begin{pgfscope}%
\pgfsys@transformshift{4.265208in}{2.233096in}%
\pgfsys@useobject{currentmarker}{}%
\end{pgfscope}%
\begin{pgfscope}%
\pgfsys@transformshift{4.282802in}{2.098982in}%
\pgfsys@useobject{currentmarker}{}%
\end{pgfscope}%
\begin{pgfscope}%
\pgfsys@transformshift{4.300396in}{2.162618in}%
\pgfsys@useobject{currentmarker}{}%
\end{pgfscope}%
\begin{pgfscope}%
\pgfsys@transformshift{4.317990in}{2.088214in}%
\pgfsys@useobject{currentmarker}{}%
\end{pgfscope}%
\begin{pgfscope}%
\pgfsys@transformshift{4.335584in}{2.217817in}%
\pgfsys@useobject{currentmarker}{}%
\end{pgfscope}%
\begin{pgfscope}%
\pgfsys@transformshift{4.353178in}{2.205555in}%
\pgfsys@useobject{currentmarker}{}%
\end{pgfscope}%
\begin{pgfscope}%
\pgfsys@transformshift{4.370772in}{2.125576in}%
\pgfsys@useobject{currentmarker}{}%
\end{pgfscope}%
\begin{pgfscope}%
\pgfsys@transformshift{4.388365in}{2.189386in}%
\pgfsys@useobject{currentmarker}{}%
\end{pgfscope}%
\begin{pgfscope}%
\pgfsys@transformshift{4.405959in}{2.041820in}%
\pgfsys@useobject{currentmarker}{}%
\end{pgfscope}%
\begin{pgfscope}%
\pgfsys@transformshift{4.423553in}{2.007535in}%
\pgfsys@useobject{currentmarker}{}%
\end{pgfscope}%
\begin{pgfscope}%
\pgfsys@transformshift{4.441147in}{2.212697in}%
\pgfsys@useobject{currentmarker}{}%
\end{pgfscope}%
\begin{pgfscope}%
\pgfsys@transformshift{4.458741in}{2.195046in}%
\pgfsys@useobject{currentmarker}{}%
\end{pgfscope}%
\begin{pgfscope}%
\pgfsys@transformshift{4.476335in}{2.254728in}%
\pgfsys@useobject{currentmarker}{}%
\end{pgfscope}%
\begin{pgfscope}%
\pgfsys@transformshift{4.493928in}{2.446545in}%
\pgfsys@useobject{currentmarker}{}%
\end{pgfscope}%
\begin{pgfscope}%
\pgfsys@transformshift{4.511522in}{2.314886in}%
\pgfsys@useobject{currentmarker}{}%
\end{pgfscope}%
\begin{pgfscope}%
\pgfsys@transformshift{4.529116in}{2.141871in}%
\pgfsys@useobject{currentmarker}{}%
\end{pgfscope}%
\begin{pgfscope}%
\pgfsys@transformshift{4.546710in}{2.366405in}%
\pgfsys@useobject{currentmarker}{}%
\end{pgfscope}%
\begin{pgfscope}%
\pgfsys@transformshift{4.564304in}{2.136836in}%
\pgfsys@useobject{currentmarker}{}%
\end{pgfscope}%
\begin{pgfscope}%
\pgfsys@transformshift{4.581898in}{2.243272in}%
\pgfsys@useobject{currentmarker}{}%
\end{pgfscope}%
\begin{pgfscope}%
\pgfsys@transformshift{4.599492in}{2.303294in}%
\pgfsys@useobject{currentmarker}{}%
\end{pgfscope}%
\begin{pgfscope}%
\pgfsys@transformshift{4.617085in}{2.505271in}%
\pgfsys@useobject{currentmarker}{}%
\end{pgfscope}%
\begin{pgfscope}%
\pgfsys@transformshift{4.634679in}{2.275099in}%
\pgfsys@useobject{currentmarker}{}%
\end{pgfscope}%
\begin{pgfscope}%
\pgfsys@transformshift{4.652273in}{2.287237in}%
\pgfsys@useobject{currentmarker}{}%
\end{pgfscope}%
\begin{pgfscope}%
\pgfsys@transformshift{4.669867in}{2.381713in}%
\pgfsys@useobject{currentmarker}{}%
\end{pgfscope}%
\begin{pgfscope}%
\pgfsys@transformshift{4.687461in}{2.343907in}%
\pgfsys@useobject{currentmarker}{}%
\end{pgfscope}%
\begin{pgfscope}%
\pgfsys@transformshift{4.705055in}{2.545636in}%
\pgfsys@useobject{currentmarker}{}%
\end{pgfscope}%
\begin{pgfscope}%
\pgfsys@transformshift{4.722649in}{2.338992in}%
\pgfsys@useobject{currentmarker}{}%
\end{pgfscope}%
\begin{pgfscope}%
\pgfsys@transformshift{4.740242in}{2.349474in}%
\pgfsys@useobject{currentmarker}{}%
\end{pgfscope}%
\begin{pgfscope}%
\pgfsys@transformshift{4.757836in}{2.438041in}%
\pgfsys@useobject{currentmarker}{}%
\end{pgfscope}%
\begin{pgfscope}%
\pgfsys@transformshift{4.775430in}{2.446775in}%
\pgfsys@useobject{currentmarker}{}%
\end{pgfscope}%
\begin{pgfscope}%
\pgfsys@transformshift{4.793024in}{2.707728in}%
\pgfsys@useobject{currentmarker}{}%
\end{pgfscope}%
\begin{pgfscope}%
\pgfsys@transformshift{4.810618in}{2.619589in}%
\pgfsys@useobject{currentmarker}{}%
\end{pgfscope}%
\begin{pgfscope}%
\pgfsys@transformshift{4.828212in}{2.541800in}%
\pgfsys@useobject{currentmarker}{}%
\end{pgfscope}%
\begin{pgfscope}%
\pgfsys@transformshift{4.845805in}{2.416208in}%
\pgfsys@useobject{currentmarker}{}%
\end{pgfscope}%
\begin{pgfscope}%
\pgfsys@transformshift{4.863399in}{2.633696in}%
\pgfsys@useobject{currentmarker}{}%
\end{pgfscope}%
\begin{pgfscope}%
\pgfsys@transformshift{4.880993in}{2.450091in}%
\pgfsys@useobject{currentmarker}{}%
\end{pgfscope}%
\begin{pgfscope}%
\pgfsys@transformshift{4.898587in}{2.397092in}%
\pgfsys@useobject{currentmarker}{}%
\end{pgfscope}%
\begin{pgfscope}%
\pgfsys@transformshift{4.916181in}{2.676321in}%
\pgfsys@useobject{currentmarker}{}%
\end{pgfscope}%
\begin{pgfscope}%
\pgfsys@transformshift{4.933775in}{2.586081in}%
\pgfsys@useobject{currentmarker}{}%
\end{pgfscope}%
\begin{pgfscope}%
\pgfsys@transformshift{4.951369in}{2.644312in}%
\pgfsys@useobject{currentmarker}{}%
\end{pgfscope}%
\begin{pgfscope}%
\pgfsys@transformshift{4.968962in}{2.577668in}%
\pgfsys@useobject{currentmarker}{}%
\end{pgfscope}%
\begin{pgfscope}%
\pgfsys@transformshift{4.986556in}{2.625475in}%
\pgfsys@useobject{currentmarker}{}%
\end{pgfscope}%
\begin{pgfscope}%
\pgfsys@transformshift{5.004150in}{2.462914in}%
\pgfsys@useobject{currentmarker}{}%
\end{pgfscope}%
\begin{pgfscope}%
\pgfsys@transformshift{5.021744in}{2.413409in}%
\pgfsys@useobject{currentmarker}{}%
\end{pgfscope}%
\begin{pgfscope}%
\pgfsys@transformshift{5.039338in}{2.576448in}%
\pgfsys@useobject{currentmarker}{}%
\end{pgfscope}%
\begin{pgfscope}%
\pgfsys@transformshift{5.056932in}{2.411720in}%
\pgfsys@useobject{currentmarker}{}%
\end{pgfscope}%
\begin{pgfscope}%
\pgfsys@transformshift{5.074526in}{2.408824in}%
\pgfsys@useobject{currentmarker}{}%
\end{pgfscope}%
\begin{pgfscope}%
\pgfsys@transformshift{5.092119in}{2.417137in}%
\pgfsys@useobject{currentmarker}{}%
\end{pgfscope}%
\begin{pgfscope}%
\pgfsys@transformshift{5.109713in}{2.448956in}%
\pgfsys@useobject{currentmarker}{}%
\end{pgfscope}%
\begin{pgfscope}%
\pgfsys@transformshift{5.127307in}{2.393983in}%
\pgfsys@useobject{currentmarker}{}%
\end{pgfscope}%
\begin{pgfscope}%
\pgfsys@transformshift{5.144901in}{2.272299in}%
\pgfsys@useobject{currentmarker}{}%
\end{pgfscope}%
\begin{pgfscope}%
\pgfsys@transformshift{5.162495in}{2.329023in}%
\pgfsys@useobject{currentmarker}{}%
\end{pgfscope}%
\begin{pgfscope}%
\pgfsys@transformshift{5.180089in}{2.150000in}%
\pgfsys@useobject{currentmarker}{}%
\end{pgfscope}%
\begin{pgfscope}%
\pgfsys@transformshift{5.197683in}{2.422690in}%
\pgfsys@useobject{currentmarker}{}%
\end{pgfscope}%
\begin{pgfscope}%
\pgfsys@transformshift{5.215276in}{2.178109in}%
\pgfsys@useobject{currentmarker}{}%
\end{pgfscope}%
\begin{pgfscope}%
\pgfsys@transformshift{5.232870in}{2.211944in}%
\pgfsys@useobject{currentmarker}{}%
\end{pgfscope}%
\begin{pgfscope}%
\pgfsys@transformshift{5.250464in}{2.313709in}%
\pgfsys@useobject{currentmarker}{}%
\end{pgfscope}%
\begin{pgfscope}%
\pgfsys@transformshift{5.268058in}{2.217733in}%
\pgfsys@useobject{currentmarker}{}%
\end{pgfscope}%
\begin{pgfscope}%
\pgfsys@transformshift{5.285652in}{2.436373in}%
\pgfsys@useobject{currentmarker}{}%
\end{pgfscope}%
\begin{pgfscope}%
\pgfsys@transformshift{5.303246in}{2.134357in}%
\pgfsys@useobject{currentmarker}{}%
\end{pgfscope}%
\begin{pgfscope}%
\pgfsys@transformshift{5.320839in}{2.282046in}%
\pgfsys@useobject{currentmarker}{}%
\end{pgfscope}%
\begin{pgfscope}%
\pgfsys@transformshift{5.338433in}{2.241026in}%
\pgfsys@useobject{currentmarker}{}%
\end{pgfscope}%
\begin{pgfscope}%
\pgfsys@transformshift{5.356027in}{2.117866in}%
\pgfsys@useobject{currentmarker}{}%
\end{pgfscope}%
\begin{pgfscope}%
\pgfsys@transformshift{5.373621in}{2.284136in}%
\pgfsys@useobject{currentmarker}{}%
\end{pgfscope}%
\begin{pgfscope}%
\pgfsys@transformshift{5.391215in}{2.209022in}%
\pgfsys@useobject{currentmarker}{}%
\end{pgfscope}%
\begin{pgfscope}%
\pgfsys@transformshift{5.408809in}{2.302978in}%
\pgfsys@useobject{currentmarker}{}%
\end{pgfscope}%
\begin{pgfscope}%
\pgfsys@transformshift{5.426403in}{2.308095in}%
\pgfsys@useobject{currentmarker}{}%
\end{pgfscope}%
\begin{pgfscope}%
\pgfsys@transformshift{5.443996in}{2.446676in}%
\pgfsys@useobject{currentmarker}{}%
\end{pgfscope}%
\begin{pgfscope}%
\pgfsys@transformshift{5.461590in}{2.366477in}%
\pgfsys@useobject{currentmarker}{}%
\end{pgfscope}%
\begin{pgfscope}%
\pgfsys@transformshift{5.479184in}{2.198779in}%
\pgfsys@useobject{currentmarker}{}%
\end{pgfscope}%
\begin{pgfscope}%
\pgfsys@transformshift{5.496778in}{2.220369in}%
\pgfsys@useobject{currentmarker}{}%
\end{pgfscope}%
\begin{pgfscope}%
\pgfsys@transformshift{5.514372in}{2.367339in}%
\pgfsys@useobject{currentmarker}{}%
\end{pgfscope}%
\begin{pgfscope}%
\pgfsys@transformshift{5.531966in}{2.334453in}%
\pgfsys@useobject{currentmarker}{}%
\end{pgfscope}%
\begin{pgfscope}%
\pgfsys@transformshift{5.549560in}{2.347263in}%
\pgfsys@useobject{currentmarker}{}%
\end{pgfscope}%
\begin{pgfscope}%
\pgfsys@transformshift{5.567153in}{2.133275in}%
\pgfsys@useobject{currentmarker}{}%
\end{pgfscope}%
\begin{pgfscope}%
\pgfsys@transformshift{5.584747in}{2.313560in}%
\pgfsys@useobject{currentmarker}{}%
\end{pgfscope}%
\begin{pgfscope}%
\pgfsys@transformshift{5.602341in}{2.260227in}%
\pgfsys@useobject{currentmarker}{}%
\end{pgfscope}%
\begin{pgfscope}%
\pgfsys@transformshift{5.619935in}{2.384894in}%
\pgfsys@useobject{currentmarker}{}%
\end{pgfscope}%
\begin{pgfscope}%
\pgfsys@transformshift{5.637529in}{2.368455in}%
\pgfsys@useobject{currentmarker}{}%
\end{pgfscope}%
\begin{pgfscope}%
\pgfsys@transformshift{5.655123in}{2.494470in}%
\pgfsys@useobject{currentmarker}{}%
\end{pgfscope}%
\begin{pgfscope}%
\pgfsys@transformshift{5.672717in}{2.458087in}%
\pgfsys@useobject{currentmarker}{}%
\end{pgfscope}%
\begin{pgfscope}%
\pgfsys@transformshift{5.690310in}{2.530704in}%
\pgfsys@useobject{currentmarker}{}%
\end{pgfscope}%
\begin{pgfscope}%
\pgfsys@transformshift{5.707904in}{2.428234in}%
\pgfsys@useobject{currentmarker}{}%
\end{pgfscope}%
\begin{pgfscope}%
\pgfsys@transformshift{5.725498in}{2.405057in}%
\pgfsys@useobject{currentmarker}{}%
\end{pgfscope}%
\begin{pgfscope}%
\pgfsys@transformshift{5.743092in}{2.485208in}%
\pgfsys@useobject{currentmarker}{}%
\end{pgfscope}%
\begin{pgfscope}%
\pgfsys@transformshift{5.760686in}{2.550822in}%
\pgfsys@useobject{currentmarker}{}%
\end{pgfscope}%
\begin{pgfscope}%
\pgfsys@transformshift{5.778280in}{2.616431in}%
\pgfsys@useobject{currentmarker}{}%
\end{pgfscope}%
\begin{pgfscope}%
\pgfsys@transformshift{5.795873in}{2.832843in}%
\pgfsys@useobject{currentmarker}{}%
\end{pgfscope}%
\begin{pgfscope}%
\pgfsys@transformshift{5.813467in}{2.622109in}%
\pgfsys@useobject{currentmarker}{}%
\end{pgfscope}%
\begin{pgfscope}%
\pgfsys@transformshift{5.831061in}{2.551996in}%
\pgfsys@useobject{currentmarker}{}%
\end{pgfscope}%
\begin{pgfscope}%
\pgfsys@transformshift{5.848655in}{2.636175in}%
\pgfsys@useobject{currentmarker}{}%
\end{pgfscope}%
\begin{pgfscope}%
\pgfsys@transformshift{5.866249in}{2.644912in}%
\pgfsys@useobject{currentmarker}{}%
\end{pgfscope}%
\begin{pgfscope}%
\pgfsys@transformshift{5.883843in}{2.760620in}%
\pgfsys@useobject{currentmarker}{}%
\end{pgfscope}%
\begin{pgfscope}%
\pgfsys@transformshift{5.901437in}{2.572201in}%
\pgfsys@useobject{currentmarker}{}%
\end{pgfscope}%
\begin{pgfscope}%
\pgfsys@transformshift{5.919030in}{2.751921in}%
\pgfsys@useobject{currentmarker}{}%
\end{pgfscope}%
\begin{pgfscope}%
\pgfsys@transformshift{5.936624in}{2.775821in}%
\pgfsys@useobject{currentmarker}{}%
\end{pgfscope}%
\begin{pgfscope}%
\pgfsys@transformshift{5.954218in}{2.796091in}%
\pgfsys@useobject{currentmarker}{}%
\end{pgfscope}%
\begin{pgfscope}%
\pgfsys@transformshift{5.971812in}{2.722154in}%
\pgfsys@useobject{currentmarker}{}%
\end{pgfscope}%
\begin{pgfscope}%
\pgfsys@transformshift{5.989406in}{2.767504in}%
\pgfsys@useobject{currentmarker}{}%
\end{pgfscope}%
\begin{pgfscope}%
\pgfsys@transformshift{6.007000in}{2.653279in}%
\pgfsys@useobject{currentmarker}{}%
\end{pgfscope}%
\begin{pgfscope}%
\pgfsys@transformshift{6.024594in}{2.752946in}%
\pgfsys@useobject{currentmarker}{}%
\end{pgfscope}%
\begin{pgfscope}%
\pgfsys@transformshift{6.042187in}{2.750693in}%
\pgfsys@useobject{currentmarker}{}%
\end{pgfscope}%
\begin{pgfscope}%
\pgfsys@transformshift{6.059781in}{2.849357in}%
\pgfsys@useobject{currentmarker}{}%
\end{pgfscope}%
\begin{pgfscope}%
\pgfsys@transformshift{6.077375in}{2.688050in}%
\pgfsys@useobject{currentmarker}{}%
\end{pgfscope}%
\begin{pgfscope}%
\pgfsys@transformshift{6.094969in}{2.882850in}%
\pgfsys@useobject{currentmarker}{}%
\end{pgfscope}%
\begin{pgfscope}%
\pgfsys@transformshift{6.112563in}{2.951139in}%
\pgfsys@useobject{currentmarker}{}%
\end{pgfscope}%
\begin{pgfscope}%
\pgfsys@transformshift{6.130157in}{2.581640in}%
\pgfsys@useobject{currentmarker}{}%
\end{pgfscope}%
\begin{pgfscope}%
\pgfsys@transformshift{6.147751in}{2.828734in}%
\pgfsys@useobject{currentmarker}{}%
\end{pgfscope}%
\begin{pgfscope}%
\pgfsys@transformshift{6.165344in}{2.845545in}%
\pgfsys@useobject{currentmarker}{}%
\end{pgfscope}%
\begin{pgfscope}%
\pgfsys@transformshift{6.182938in}{2.701636in}%
\pgfsys@useobject{currentmarker}{}%
\end{pgfscope}%
\begin{pgfscope}%
\pgfsys@transformshift{6.200532in}{2.715287in}%
\pgfsys@useobject{currentmarker}{}%
\end{pgfscope}%
\begin{pgfscope}%
\pgfsys@transformshift{6.218126in}{2.730634in}%
\pgfsys@useobject{currentmarker}{}%
\end{pgfscope}%
\begin{pgfscope}%
\pgfsys@transformshift{6.235720in}{2.701620in}%
\pgfsys@useobject{currentmarker}{}%
\end{pgfscope}%
\begin{pgfscope}%
\pgfsys@transformshift{6.253314in}{2.687893in}%
\pgfsys@useobject{currentmarker}{}%
\end{pgfscope}%
\begin{pgfscope}%
\pgfsys@transformshift{6.270907in}{2.535733in}%
\pgfsys@useobject{currentmarker}{}%
\end{pgfscope}%
\begin{pgfscope}%
\pgfsys@transformshift{6.288501in}{2.811411in}%
\pgfsys@useobject{currentmarker}{}%
\end{pgfscope}%
\begin{pgfscope}%
\pgfsys@transformshift{6.306095in}{2.791491in}%
\pgfsys@useobject{currentmarker}{}%
\end{pgfscope}%
\begin{pgfscope}%
\pgfsys@transformshift{6.323689in}{2.586388in}%
\pgfsys@useobject{currentmarker}{}%
\end{pgfscope}%
\begin{pgfscope}%
\pgfsys@transformshift{6.341283in}{2.508363in}%
\pgfsys@useobject{currentmarker}{}%
\end{pgfscope}%
\begin{pgfscope}%
\pgfsys@transformshift{6.358877in}{2.700439in}%
\pgfsys@useobject{currentmarker}{}%
\end{pgfscope}%
\begin{pgfscope}%
\pgfsys@transformshift{6.376471in}{2.579017in}%
\pgfsys@useobject{currentmarker}{}%
\end{pgfscope}%
\end{pgfscope}%
\begin{pgfscope}%
\pgfpathrectangle{\pgfqpoint{2.000000in}{1.973684in}}{\pgfqpoint{4.376471in}{0.978947in}} %
\pgfusepath{clip}%
\pgfsetroundcap%
\pgfsetroundjoin%
\pgfsetlinewidth{1.756562pt}%
\definecolor{currentstroke}{rgb}{0.298039,0.447059,0.690196}%
\pgfsetstrokecolor{currentstroke}%
\pgfsetdash{}{0pt}%
\pgfpathmoveto{\pgfqpoint{2.923232in}{2.962632in}}%
\pgfpathlineto{\pgfqpoint{2.928076in}{2.899903in}}%
\pgfpathlineto{\pgfqpoint{2.945670in}{2.746061in}}%
\pgfpathlineto{\pgfqpoint{2.963263in}{2.647049in}}%
\pgfpathlineto{\pgfqpoint{2.980857in}{2.587324in}}%
\pgfpathlineto{\pgfqpoint{2.998451in}{2.554495in}}%
\pgfpathlineto{\pgfqpoint{3.016045in}{2.538893in}}%
\pgfpathlineto{\pgfqpoint{3.033639in}{2.533158in}}%
\pgfpathlineto{\pgfqpoint{3.086420in}{2.528852in}}%
\pgfpathlineto{\pgfqpoint{3.104014in}{2.523218in}}%
\pgfpathlineto{\pgfqpoint{3.121608in}{2.513785in}}%
\pgfpathlineto{\pgfqpoint{3.139202in}{2.500594in}}%
\pgfpathlineto{\pgfqpoint{3.156796in}{2.484141in}}%
\pgfpathlineto{\pgfqpoint{3.191983in}{2.444831in}}%
\pgfpathlineto{\pgfqpoint{3.227171in}{2.403859in}}%
\pgfpathlineto{\pgfqpoint{3.244765in}{2.385347in}}%
\pgfpathlineto{\pgfqpoint{3.262359in}{2.369373in}}%
\pgfpathlineto{\pgfqpoint{3.279953in}{2.356687in}}%
\pgfpathlineto{\pgfqpoint{3.297547in}{2.347873in}}%
\pgfpathlineto{\pgfqpoint{3.315140in}{2.343340in}}%
\pgfpathlineto{\pgfqpoint{3.332734in}{2.343324in}}%
\pgfpathlineto{\pgfqpoint{3.350328in}{2.347890in}}%
\pgfpathlineto{\pgfqpoint{3.367922in}{2.356945in}}%
\pgfpathlineto{\pgfqpoint{3.385516in}{2.370252in}}%
\pgfpathlineto{\pgfqpoint{3.403110in}{2.387453in}}%
\pgfpathlineto{\pgfqpoint{3.420704in}{2.408083in}}%
\pgfpathlineto{\pgfqpoint{3.438297in}{2.431600in}}%
\pgfpathlineto{\pgfqpoint{3.473485in}{2.484843in}}%
\pgfpathlineto{\pgfqpoint{3.543860in}{2.598025in}}%
\pgfpathlineto{\pgfqpoint{3.579048in}{2.648239in}}%
\pgfpathlineto{\pgfqpoint{3.596642in}{2.669981in}}%
\pgfpathlineto{\pgfqpoint{3.614236in}{2.688975in}}%
\pgfpathlineto{\pgfqpoint{3.631830in}{2.704936in}}%
\pgfpathlineto{\pgfqpoint{3.649424in}{2.717652in}}%
\pgfpathlineto{\pgfqpoint{3.667017in}{2.726985in}}%
\pgfpathlineto{\pgfqpoint{3.684611in}{2.732870in}}%
\pgfpathlineto{\pgfqpoint{3.702205in}{2.735309in}}%
\pgfpathlineto{\pgfqpoint{3.719799in}{2.734362in}}%
\pgfpathlineto{\pgfqpoint{3.737393in}{2.730150in}}%
\pgfpathlineto{\pgfqpoint{3.754987in}{2.722839in}}%
\pgfpathlineto{\pgfqpoint{3.772581in}{2.712635in}}%
\pgfpathlineto{\pgfqpoint{3.790174in}{2.699780in}}%
\pgfpathlineto{\pgfqpoint{3.807768in}{2.684538in}}%
\pgfpathlineto{\pgfqpoint{3.825362in}{2.667191in}}%
\pgfpathlineto{\pgfqpoint{3.860550in}{2.627358in}}%
\pgfpathlineto{\pgfqpoint{3.895738in}{2.582622in}}%
\pgfpathlineto{\pgfqpoint{3.966113in}{2.487030in}}%
\pgfpathlineto{\pgfqpoint{4.018894in}{2.416801in}}%
\pgfpathlineto{\pgfqpoint{4.054082in}{2.372994in}}%
\pgfpathlineto{\pgfqpoint{4.089270in}{2.332481in}}%
\pgfpathlineto{\pgfqpoint{4.124458in}{2.295715in}}%
\pgfpathlineto{\pgfqpoint{4.159645in}{2.262962in}}%
\pgfpathlineto{\pgfqpoint{4.194833in}{2.234391in}}%
\pgfpathlineto{\pgfqpoint{4.230021in}{2.210147in}}%
\pgfpathlineto{\pgfqpoint{4.265208in}{2.190412in}}%
\pgfpathlineto{\pgfqpoint{4.282802in}{2.182310in}}%
\pgfpathlineto{\pgfqpoint{4.300396in}{2.175435in}}%
\pgfpathlineto{\pgfqpoint{4.317990in}{2.169832in}}%
\pgfpathlineto{\pgfqpoint{4.335584in}{2.165545in}}%
\pgfpathlineto{\pgfqpoint{4.353178in}{2.162625in}}%
\pgfpathlineto{\pgfqpoint{4.370772in}{2.161120in}}%
\pgfpathlineto{\pgfqpoint{4.388365in}{2.161081in}}%
\pgfpathlineto{\pgfqpoint{4.405959in}{2.162550in}}%
\pgfpathlineto{\pgfqpoint{4.423553in}{2.165569in}}%
\pgfpathlineto{\pgfqpoint{4.441147in}{2.170168in}}%
\pgfpathlineto{\pgfqpoint{4.458741in}{2.176369in}}%
\pgfpathlineto{\pgfqpoint{4.476335in}{2.184180in}}%
\pgfpathlineto{\pgfqpoint{4.493928in}{2.193594in}}%
\pgfpathlineto{\pgfqpoint{4.511522in}{2.204590in}}%
\pgfpathlineto{\pgfqpoint{4.529116in}{2.217125in}}%
\pgfpathlineto{\pgfqpoint{4.564304in}{2.246547in}}%
\pgfpathlineto{\pgfqpoint{4.599492in}{2.281117in}}%
\pgfpathlineto{\pgfqpoint{4.634679in}{2.319753in}}%
\pgfpathlineto{\pgfqpoint{4.687461in}{2.382187in}}%
\pgfpathlineto{\pgfqpoint{4.740242in}{2.444761in}}%
\pgfpathlineto{\pgfqpoint{4.775430in}{2.483351in}}%
\pgfpathlineto{\pgfqpoint{4.810618in}{2.517185in}}%
\pgfpathlineto{\pgfqpoint{4.828212in}{2.531754in}}%
\pgfpathlineto{\pgfqpoint{4.845805in}{2.544486in}}%
\pgfpathlineto{\pgfqpoint{4.863399in}{2.555203in}}%
\pgfpathlineto{\pgfqpoint{4.880993in}{2.563751in}}%
\pgfpathlineto{\pgfqpoint{4.898587in}{2.570002in}}%
\pgfpathlineto{\pgfqpoint{4.916181in}{2.573858in}}%
\pgfpathlineto{\pgfqpoint{4.933775in}{2.575254in}}%
\pgfpathlineto{\pgfqpoint{4.951369in}{2.574157in}}%
\pgfpathlineto{\pgfqpoint{4.968962in}{2.570568in}}%
\pgfpathlineto{\pgfqpoint{4.986556in}{2.564526in}}%
\pgfpathlineto{\pgfqpoint{5.004150in}{2.556102in}}%
\pgfpathlineto{\pgfqpoint{5.021744in}{2.545401in}}%
\pgfpathlineto{\pgfqpoint{5.039338in}{2.532563in}}%
\pgfpathlineto{\pgfqpoint{5.056932in}{2.517758in}}%
\pgfpathlineto{\pgfqpoint{5.074526in}{2.501184in}}%
\pgfpathlineto{\pgfqpoint{5.109713in}{2.463650in}}%
\pgfpathlineto{\pgfqpoint{5.144901in}{2.422007in}}%
\pgfpathlineto{\pgfqpoint{5.232870in}{2.315080in}}%
\pgfpathlineto{\pgfqpoint{5.268058in}{2.277292in}}%
\pgfpathlineto{\pgfqpoint{5.285652in}{2.260535in}}%
\pgfpathlineto{\pgfqpoint{5.303246in}{2.245503in}}%
\pgfpathlineto{\pgfqpoint{5.320839in}{2.232392in}}%
\pgfpathlineto{\pgfqpoint{5.338433in}{2.221371in}}%
\pgfpathlineto{\pgfqpoint{5.356027in}{2.212578in}}%
\pgfpathlineto{\pgfqpoint{5.373621in}{2.206122in}}%
\pgfpathlineto{\pgfqpoint{5.391215in}{2.202079in}}%
\pgfpathlineto{\pgfqpoint{5.408809in}{2.200496in}}%
\pgfpathlineto{\pgfqpoint{5.426403in}{2.201388in}}%
\pgfpathlineto{\pgfqpoint{5.443996in}{2.204740in}}%
\pgfpathlineto{\pgfqpoint{5.461590in}{2.210509in}}%
\pgfpathlineto{\pgfqpoint{5.479184in}{2.218627in}}%
\pgfpathlineto{\pgfqpoint{5.496778in}{2.228999in}}%
\pgfpathlineto{\pgfqpoint{5.514372in}{2.241510in}}%
\pgfpathlineto{\pgfqpoint{5.531966in}{2.256027in}}%
\pgfpathlineto{\pgfqpoint{5.549560in}{2.272400in}}%
\pgfpathlineto{\pgfqpoint{5.584747in}{2.310053in}}%
\pgfpathlineto{\pgfqpoint{5.619935in}{2.353073in}}%
\pgfpathlineto{\pgfqpoint{5.655123in}{2.399994in}}%
\pgfpathlineto{\pgfqpoint{5.707904in}{2.474528in}}%
\pgfpathlineto{\pgfqpoint{5.778280in}{2.574596in}}%
\pgfpathlineto{\pgfqpoint{5.813467in}{2.622191in}}%
\pgfpathlineto{\pgfqpoint{5.848655in}{2.666689in}}%
\pgfpathlineto{\pgfqpoint{5.883843in}{2.707049in}}%
\pgfpathlineto{\pgfqpoint{5.919030in}{2.742262in}}%
\pgfpathlineto{\pgfqpoint{5.936624in}{2.757630in}}%
\pgfpathlineto{\pgfqpoint{5.954218in}{2.771350in}}%
\pgfpathlineto{\pgfqpoint{5.971812in}{2.783310in}}%
\pgfpathlineto{\pgfqpoint{5.989406in}{2.793411in}}%
\pgfpathlineto{\pgfqpoint{6.007000in}{2.801562in}}%
\pgfpathlineto{\pgfqpoint{6.024594in}{2.807690in}}%
\pgfpathlineto{\pgfqpoint{6.042187in}{2.811742in}}%
\pgfpathlineto{\pgfqpoint{6.059781in}{2.813688in}}%
\pgfpathlineto{\pgfqpoint{6.077375in}{2.813523in}}%
\pgfpathlineto{\pgfqpoint{6.094969in}{2.811272in}}%
\pgfpathlineto{\pgfqpoint{6.112563in}{2.806991in}}%
\pgfpathlineto{\pgfqpoint{6.130157in}{2.800764in}}%
\pgfpathlineto{\pgfqpoint{6.147751in}{2.792698in}}%
\pgfpathlineto{\pgfqpoint{6.165344in}{2.782919in}}%
\pgfpathlineto{\pgfqpoint{6.182938in}{2.771553in}}%
\pgfpathlineto{\pgfqpoint{6.200532in}{2.758713in}}%
\pgfpathlineto{\pgfqpoint{6.235720in}{2.728802in}}%
\pgfpathlineto{\pgfqpoint{6.253314in}{2.711584in}}%
\pgfpathlineto{\pgfqpoint{6.270907in}{2.692488in}}%
\pgfpathlineto{\pgfqpoint{6.288501in}{2.670931in}}%
\pgfpathlineto{\pgfqpoint{6.306095in}{2.645979in}}%
\pgfpathlineto{\pgfqpoint{6.323689in}{2.616237in}}%
\pgfpathlineto{\pgfqpoint{6.341283in}{2.579717in}}%
\pgfpathlineto{\pgfqpoint{6.358877in}{2.533686in}}%
\pgfpathlineto{\pgfqpoint{6.376471in}{2.474480in}}%
\pgfpathlineto{\pgfqpoint{6.376471in}{2.474480in}}%
\pgfusepath{stroke}%
\end{pgfscope}%
\begin{pgfscope}%
\pgfpathrectangle{\pgfqpoint{2.000000in}{1.973684in}}{\pgfqpoint{4.376471in}{0.978947in}} %
\pgfusepath{clip}%
\pgfsetbuttcap%
\pgfsetroundjoin%
\pgfsetlinewidth{1.756562pt}%
\definecolor{currentstroke}{rgb}{1.000000,0.647059,0.000000}%
\pgfsetstrokecolor{currentstroke}%
\pgfsetdash{{6.000000pt}{6.000000pt}}{0.000000pt}%
\pgfpathmoveto{\pgfqpoint{2.921777in}{2.962632in}}%
\pgfpathlineto{\pgfqpoint{2.928076in}{2.887902in}}%
\pgfpathlineto{\pgfqpoint{2.945670in}{2.747937in}}%
\pgfpathlineto{\pgfqpoint{2.963263in}{2.656768in}}%
\pgfpathlineto{\pgfqpoint{2.980857in}{2.601051in}}%
\pgfpathlineto{\pgfqpoint{2.998451in}{2.568786in}}%
\pgfpathlineto{\pgfqpoint{3.016045in}{2.553152in}}%
\pgfpathlineto{\pgfqpoint{3.033639in}{2.544090in}}%
\pgfpathlineto{\pgfqpoint{3.051233in}{2.543592in}}%
\pgfpathlineto{\pgfqpoint{3.086420in}{2.534081in}}%
\pgfpathlineto{\pgfqpoint{3.104014in}{2.527409in}}%
\pgfpathlineto{\pgfqpoint{3.121608in}{2.515708in}}%
\pgfpathlineto{\pgfqpoint{3.139202in}{2.502115in}}%
\pgfpathlineto{\pgfqpoint{3.156796in}{2.484489in}}%
\pgfpathlineto{\pgfqpoint{3.209577in}{2.423544in}}%
\pgfpathlineto{\pgfqpoint{3.227171in}{2.401038in}}%
\pgfpathlineto{\pgfqpoint{3.244765in}{2.383810in}}%
\pgfpathlineto{\pgfqpoint{3.262359in}{2.369071in}}%
\pgfpathlineto{\pgfqpoint{3.279953in}{2.356822in}}%
\pgfpathlineto{\pgfqpoint{3.315140in}{2.343080in}}%
\pgfpathlineto{\pgfqpoint{3.350328in}{2.347661in}}%
\pgfpathlineto{\pgfqpoint{3.367922in}{2.356424in}}%
\pgfpathlineto{\pgfqpoint{3.403110in}{2.386498in}}%
\pgfpathlineto{\pgfqpoint{3.420704in}{2.406216in}}%
\pgfpathlineto{\pgfqpoint{3.473485in}{2.480804in}}%
\pgfpathlineto{\pgfqpoint{3.491079in}{2.510281in}}%
\pgfpathlineto{\pgfqpoint{3.508673in}{2.535675in}}%
\pgfpathlineto{\pgfqpoint{3.561454in}{2.617632in}}%
\pgfpathlineto{\pgfqpoint{3.614236in}{2.682162in}}%
\pgfpathlineto{\pgfqpoint{3.631830in}{2.695108in}}%
\pgfpathlineto{\pgfqpoint{3.667017in}{2.716817in}}%
\pgfpathlineto{\pgfqpoint{3.684611in}{2.725382in}}%
\pgfpathlineto{\pgfqpoint{3.702205in}{2.727572in}}%
\pgfpathlineto{\pgfqpoint{3.719799in}{2.726178in}}%
\pgfpathlineto{\pgfqpoint{3.754987in}{2.714427in}}%
\pgfpathlineto{\pgfqpoint{3.772581in}{2.706461in}}%
\pgfpathlineto{\pgfqpoint{3.807768in}{2.679573in}}%
\pgfpathlineto{\pgfqpoint{3.825362in}{2.663441in}}%
\pgfpathlineto{\pgfqpoint{3.842956in}{2.645316in}}%
\pgfpathlineto{\pgfqpoint{3.878144in}{2.603690in}}%
\pgfpathlineto{\pgfqpoint{3.930925in}{2.535973in}}%
\pgfpathlineto{\pgfqpoint{3.948519in}{2.515260in}}%
\pgfpathlineto{\pgfqpoint{3.966113in}{2.486580in}}%
\pgfpathlineto{\pgfqpoint{3.983707in}{2.467062in}}%
\pgfpathlineto{\pgfqpoint{4.054082in}{2.374648in}}%
\pgfpathlineto{\pgfqpoint{4.071676in}{2.358316in}}%
\pgfpathlineto{\pgfqpoint{4.106864in}{2.316491in}}%
\pgfpathlineto{\pgfqpoint{4.159645in}{2.265903in}}%
\pgfpathlineto{\pgfqpoint{4.177239in}{2.251562in}}%
\pgfpathlineto{\pgfqpoint{4.194833in}{2.238816in}}%
\pgfpathlineto{\pgfqpoint{4.212427in}{2.221687in}}%
\pgfpathlineto{\pgfqpoint{4.230021in}{2.210534in}}%
\pgfpathlineto{\pgfqpoint{4.247615in}{2.204559in}}%
\pgfpathlineto{\pgfqpoint{4.265208in}{2.193406in}}%
\pgfpathlineto{\pgfqpoint{4.300396in}{2.176676in}}%
\pgfpathlineto{\pgfqpoint{4.317990in}{2.166717in}}%
\pgfpathlineto{\pgfqpoint{4.335584in}{2.168709in}}%
\pgfpathlineto{\pgfqpoint{4.353178in}{2.162734in}}%
\pgfpathlineto{\pgfqpoint{4.370772in}{2.163132in}}%
\pgfpathlineto{\pgfqpoint{4.388365in}{2.162336in}}%
\pgfpathlineto{\pgfqpoint{4.405959in}{2.162734in}}%
\pgfpathlineto{\pgfqpoint{4.441147in}{2.167514in}}%
\pgfpathlineto{\pgfqpoint{4.458741in}{2.174684in}}%
\pgfpathlineto{\pgfqpoint{4.493928in}{2.193406in}}%
\pgfpathlineto{\pgfqpoint{4.511522in}{2.203762in}}%
\pgfpathlineto{\pgfqpoint{4.529116in}{2.219696in}}%
\pgfpathlineto{\pgfqpoint{4.546710in}{2.230849in}}%
\pgfpathlineto{\pgfqpoint{4.564304in}{2.245986in}}%
\pgfpathlineto{\pgfqpoint{4.617085in}{2.301753in}}%
\pgfpathlineto{\pgfqpoint{4.652273in}{2.339993in}}%
\pgfpathlineto{\pgfqpoint{4.687461in}{2.379030in}}%
\pgfpathlineto{\pgfqpoint{4.705055in}{2.406913in}}%
\pgfpathlineto{\pgfqpoint{4.722649in}{2.422846in}}%
\pgfpathlineto{\pgfqpoint{4.757836in}{2.465070in}}%
\pgfpathlineto{\pgfqpoint{4.775430in}{2.484190in}}%
\pgfpathlineto{\pgfqpoint{4.793024in}{2.500123in}}%
\pgfpathlineto{\pgfqpoint{4.810618in}{2.519243in}}%
\pgfpathlineto{\pgfqpoint{4.828212in}{2.528803in}}%
\pgfpathlineto{\pgfqpoint{4.863399in}{2.555492in}}%
\pgfpathlineto{\pgfqpoint{4.880993in}{2.565849in}}%
\pgfpathlineto{\pgfqpoint{4.898587in}{2.566645in}}%
\pgfpathlineto{\pgfqpoint{4.916181in}{2.573019in}}%
\pgfpathlineto{\pgfqpoint{4.933775in}{2.577799in}}%
\pgfpathlineto{\pgfqpoint{4.951369in}{2.569434in}}%
\pgfpathlineto{\pgfqpoint{4.968962in}{2.570230in}}%
\pgfpathlineto{\pgfqpoint{4.986556in}{2.560670in}}%
\pgfpathlineto{\pgfqpoint{5.004150in}{2.556687in}}%
\pgfpathlineto{\pgfqpoint{5.021744in}{2.544339in}}%
\pgfpathlineto{\pgfqpoint{5.039338in}{2.529998in}}%
\pgfpathlineto{\pgfqpoint{5.056932in}{2.518447in}}%
\pgfpathlineto{\pgfqpoint{5.092119in}{2.485783in}}%
\pgfpathlineto{\pgfqpoint{5.127307in}{2.442763in}}%
\pgfpathlineto{\pgfqpoint{5.162495in}{2.398548in}}%
\pgfpathlineto{\pgfqpoint{5.197683in}{2.357719in}}%
\pgfpathlineto{\pgfqpoint{5.250464in}{2.295379in}}%
\pgfpathlineto{\pgfqpoint{5.268058in}{2.279247in}}%
\pgfpathlineto{\pgfqpoint{5.285652in}{2.261322in}}%
\pgfpathlineto{\pgfqpoint{5.320839in}{2.231945in}}%
\pgfpathlineto{\pgfqpoint{5.338433in}{2.223779in}}%
\pgfpathlineto{\pgfqpoint{5.356027in}{2.211878in}}%
\pgfpathlineto{\pgfqpoint{5.373621in}{2.205107in}}%
\pgfpathlineto{\pgfqpoint{5.408809in}{2.200103in}}%
\pgfpathlineto{\pgfqpoint{5.426403in}{2.201721in}}%
\pgfpathlineto{\pgfqpoint{5.461590in}{2.210385in}}%
\pgfpathlineto{\pgfqpoint{5.479184in}{2.220891in}}%
\pgfpathlineto{\pgfqpoint{5.496778in}{2.228559in}}%
\pgfpathlineto{\pgfqpoint{5.531966in}{2.258235in}}%
\pgfpathlineto{\pgfqpoint{5.549560in}{2.270683in}}%
\pgfpathlineto{\pgfqpoint{5.567153in}{2.292790in}}%
\pgfpathlineto{\pgfqpoint{5.584747in}{2.309919in}}%
\pgfpathlineto{\pgfqpoint{5.619935in}{2.351545in}}%
\pgfpathlineto{\pgfqpoint{5.637529in}{2.375644in}}%
\pgfpathlineto{\pgfqpoint{5.655123in}{2.396955in}}%
\pgfpathlineto{\pgfqpoint{5.672717in}{2.424241in}}%
\pgfpathlineto{\pgfqpoint{5.690310in}{2.447145in}}%
\pgfpathlineto{\pgfqpoint{5.707904in}{2.475227in}}%
\pgfpathlineto{\pgfqpoint{5.725498in}{2.498928in}}%
\pgfpathlineto{\pgfqpoint{5.743092in}{2.525019in}}%
\pgfpathlineto{\pgfqpoint{5.760686in}{2.553301in}}%
\pgfpathlineto{\pgfqpoint{5.778280in}{2.577400in}}%
\pgfpathlineto{\pgfqpoint{5.795873in}{2.599508in}}%
\pgfpathlineto{\pgfqpoint{5.813467in}{2.626196in}}%
\pgfpathlineto{\pgfqpoint{5.831061in}{2.646113in}}%
\pgfpathlineto{\pgfqpoint{5.866249in}{2.690328in}}%
\pgfpathlineto{\pgfqpoint{5.919030in}{2.743307in}}%
\pgfpathlineto{\pgfqpoint{5.936624in}{2.758643in}}%
\pgfpathlineto{\pgfqpoint{5.954218in}{2.771389in}}%
\pgfpathlineto{\pgfqpoint{5.971812in}{2.785729in}}%
\pgfpathlineto{\pgfqpoint{5.989406in}{2.795289in}}%
\pgfpathlineto{\pgfqpoint{6.042187in}{2.814011in}}%
\pgfpathlineto{\pgfqpoint{6.059781in}{2.816600in}}%
\pgfpathlineto{\pgfqpoint{6.077375in}{2.815604in}}%
\pgfpathlineto{\pgfqpoint{6.094969in}{2.813015in}}%
\pgfpathlineto{\pgfqpoint{6.130157in}{2.802459in}}%
\pgfpathlineto{\pgfqpoint{6.165344in}{2.784335in}}%
\pgfpathlineto{\pgfqpoint{6.200532in}{2.760136in}}%
\pgfpathlineto{\pgfqpoint{6.235720in}{2.730062in}}%
\pgfpathlineto{\pgfqpoint{6.270907in}{2.693017in}}%
\pgfpathlineto{\pgfqpoint{6.288501in}{2.671806in}}%
\pgfpathlineto{\pgfqpoint{6.306095in}{2.646910in}}%
\pgfpathlineto{\pgfqpoint{6.323689in}{2.617533in}}%
\pgfpathlineto{\pgfqpoint{6.341283in}{2.583873in}}%
\pgfpathlineto{\pgfqpoint{6.358877in}{2.538065in}}%
\pgfpathlineto{\pgfqpoint{6.376471in}{2.481800in}}%
\pgfpathlineto{\pgfqpoint{6.376471in}{2.481800in}}%
\pgfusepath{stroke}%
\end{pgfscope}%
\begin{pgfscope}%
\pgfsetrectcap%
\pgfsetmiterjoin%
\pgfsetlinewidth{1.003750pt}%
\definecolor{currentstroke}{rgb}{0.800000,0.800000,0.800000}%
\pgfsetstrokecolor{currentstroke}%
\pgfsetdash{}{0pt}%
\pgfpathmoveto{\pgfqpoint{2.000000in}{1.973684in}}%
\pgfpathlineto{\pgfqpoint{2.000000in}{2.952632in}}%
\pgfusepath{stroke}%
\end{pgfscope}%
\begin{pgfscope}%
\pgfsetrectcap%
\pgfsetmiterjoin%
\pgfsetlinewidth{1.003750pt}%
\definecolor{currentstroke}{rgb}{0.800000,0.800000,0.800000}%
\pgfsetstrokecolor{currentstroke}%
\pgfsetdash{}{0pt}%
\pgfpathmoveto{\pgfqpoint{6.376471in}{1.973684in}}%
\pgfpathlineto{\pgfqpoint{6.376471in}{2.952632in}}%
\pgfusepath{stroke}%
\end{pgfscope}%
\begin{pgfscope}%
\pgfsetrectcap%
\pgfsetmiterjoin%
\pgfsetlinewidth{1.003750pt}%
\definecolor{currentstroke}{rgb}{0.800000,0.800000,0.800000}%
\pgfsetstrokecolor{currentstroke}%
\pgfsetdash{}{0pt}%
\pgfpathmoveto{\pgfqpoint{2.000000in}{2.952632in}}%
\pgfpathlineto{\pgfqpoint{6.376471in}{2.952632in}}%
\pgfusepath{stroke}%
\end{pgfscope}%
\begin{pgfscope}%
\pgfsetrectcap%
\pgfsetmiterjoin%
\pgfsetlinewidth{1.003750pt}%
\definecolor{currentstroke}{rgb}{0.800000,0.800000,0.800000}%
\pgfsetstrokecolor{currentstroke}%
\pgfsetdash{}{0pt}%
\pgfpathmoveto{\pgfqpoint{2.000000in}{1.973684in}}%
\pgfpathlineto{\pgfqpoint{6.376471in}{1.973684in}}%
\pgfusepath{stroke}%
\end{pgfscope}%
\begin{pgfscope}%
\pgfsetroundcap%
\pgfsetroundjoin%
\pgfsetlinewidth{1.756562pt}%
\definecolor{currentstroke}{rgb}{0.298039,0.447059,0.690196}%
\pgfsetstrokecolor{currentstroke}%
\pgfsetdash{}{0pt}%
\pgfpathmoveto{\pgfqpoint{2.125000in}{2.772688in}}%
\pgfpathlineto{\pgfqpoint{2.402778in}{2.772688in}}%
\pgfusepath{stroke}%
\end{pgfscope}%
\begin{pgfscope}%
\definecolor{textcolor}{rgb}{0.150000,0.150000,0.150000}%
\pgfsetstrokecolor{textcolor}%
\pgfsetfillcolor{textcolor}%
\pgftext[x=2.513889in,y=2.724077in,left,base]{\color{textcolor}\sffamily\fontsize{10.000000}{12.000000}\selectfont \(\displaystyle \widetilde{\Phi}^* \theta\)}%
\end{pgfscope}%
\begin{pgfscope}%
\pgfsetbuttcap%
\pgfsetroundjoin%
\pgfsetlinewidth{1.756562pt}%
\definecolor{currentstroke}{rgb}{1.000000,0.647059,0.000000}%
\pgfsetstrokecolor{currentstroke}%
\pgfsetdash{{6.000000pt}{6.000000pt}}{0.000000pt}%
\pgfpathmoveto{\pgfqpoint{2.125000in}{2.567827in}}%
\pgfpathlineto{\pgfqpoint{2.402778in}{2.567827in}}%
\pgfusepath{stroke}%
\end{pgfscope}%
\begin{pgfscope}%
\definecolor{textcolor}{rgb}{0.150000,0.150000,0.150000}%
\pgfsetstrokecolor{textcolor}%
\pgfsetfillcolor{textcolor}%
\pgftext[x=2.513889in,y=2.519216in,left,base]{\color{textcolor}\sffamily\fontsize{10.000000}{12.000000}\selectfont \(\displaystyle \widetilde{K}u\)}%
\end{pgfscope}%
\begin{pgfscope}%
\pgfsetbuttcap%
\pgfsetroundjoin%
\definecolor{currentfill}{rgb}{1.000000,0.000000,0.000000}%
\pgfsetfillcolor{currentfill}%
\pgfsetlinewidth{2.007500pt}%
\definecolor{currentstroke}{rgb}{1.000000,0.000000,0.000000}%
\pgfsetstrokecolor{currentstroke}%
\pgfsetdash{}{0pt}%
\pgfpathmoveto{\pgfqpoint{2.232832in}{2.359209in}}%
\pgfpathlineto{\pgfqpoint{2.294945in}{2.359209in}}%
\pgfpathmoveto{\pgfqpoint{2.263889in}{2.328152in}}%
\pgfpathlineto{\pgfqpoint{2.263889in}{2.390265in}}%
\pgfusepath{stroke,fill}%
\end{pgfscope}%
\begin{pgfscope}%
\pgfsetbuttcap%
\pgfsetroundjoin%
\definecolor{currentfill}{rgb}{1.000000,0.000000,0.000000}%
\pgfsetfillcolor{currentfill}%
\pgfsetlinewidth{2.007500pt}%
\definecolor{currentstroke}{rgb}{1.000000,0.000000,0.000000}%
\pgfsetstrokecolor{currentstroke}%
\pgfsetdash{}{0pt}%
\pgfpathmoveto{\pgfqpoint{2.232832in}{2.359209in}}%
\pgfpathlineto{\pgfqpoint{2.294945in}{2.359209in}}%
\pgfpathmoveto{\pgfqpoint{2.263889in}{2.328152in}}%
\pgfpathlineto{\pgfqpoint{2.263889in}{2.390265in}}%
\pgfusepath{stroke,fill}%
\end{pgfscope}%
\begin{pgfscope}%
\pgfsetbuttcap%
\pgfsetroundjoin%
\definecolor{currentfill}{rgb}{1.000000,0.000000,0.000000}%
\pgfsetfillcolor{currentfill}%
\pgfsetlinewidth{2.007500pt}%
\definecolor{currentstroke}{rgb}{1.000000,0.000000,0.000000}%
\pgfsetstrokecolor{currentstroke}%
\pgfsetdash{}{0pt}%
\pgfpathmoveto{\pgfqpoint{2.232832in}{2.359209in}}%
\pgfpathlineto{\pgfqpoint{2.294945in}{2.359209in}}%
\pgfpathmoveto{\pgfqpoint{2.263889in}{2.328152in}}%
\pgfpathlineto{\pgfqpoint{2.263889in}{2.390265in}}%
\pgfusepath{stroke,fill}%
\end{pgfscope}%
\begin{pgfscope}%
\definecolor{textcolor}{rgb}{0.150000,0.150000,0.150000}%
\pgfsetstrokecolor{textcolor}%
\pgfsetfillcolor{textcolor}%
\pgftext[x=2.513889in,y=2.322751in,left,base]{\color{textcolor}\sffamily\fontsize{10.000000}{12.000000}\selectfont train}%
\end{pgfscope}%
\begin{pgfscope}%
\pgfsetbuttcap%
\pgfsetroundjoin%
\definecolor{currentfill}{rgb}{0.000000,0.000000,0.000000}%
\pgfsetfillcolor{currentfill}%
\pgfsetlinewidth{0.301125pt}%
\definecolor{currentstroke}{rgb}{0.000000,0.000000,0.000000}%
\pgfsetstrokecolor{currentstroke}%
\pgfsetdash{}{0pt}%
\pgfpathmoveto{\pgfqpoint{2.263889in}{2.147216in}}%
\pgfpathcurveto{\pgfqpoint{2.268007in}{2.147216in}}{\pgfqpoint{2.271957in}{2.148852in}}{\pgfqpoint{2.274869in}{2.151764in}}%
\pgfpathcurveto{\pgfqpoint{2.277781in}{2.154676in}}{\pgfqpoint{2.279417in}{2.158626in}}{\pgfqpoint{2.279417in}{2.162744in}}%
\pgfpathcurveto{\pgfqpoint{2.279417in}{2.166862in}}{\pgfqpoint{2.277781in}{2.170812in}}{\pgfqpoint{2.274869in}{2.173724in}}%
\pgfpathcurveto{\pgfqpoint{2.271957in}{2.176636in}}{\pgfqpoint{2.268007in}{2.178272in}}{\pgfqpoint{2.263889in}{2.178272in}}%
\pgfpathcurveto{\pgfqpoint{2.259771in}{2.178272in}}{\pgfqpoint{2.255821in}{2.176636in}}{\pgfqpoint{2.252909in}{2.173724in}}%
\pgfpathcurveto{\pgfqpoint{2.249997in}{2.170812in}}{\pgfqpoint{2.248361in}{2.166862in}}{\pgfqpoint{2.248361in}{2.162744in}}%
\pgfpathcurveto{\pgfqpoint{2.248361in}{2.158626in}}{\pgfqpoint{2.249997in}{2.154676in}}{\pgfqpoint{2.252909in}{2.151764in}}%
\pgfpathcurveto{\pgfqpoint{2.255821in}{2.148852in}}{\pgfqpoint{2.259771in}{2.147216in}}{\pgfqpoint{2.263889in}{2.147216in}}%
\pgfpathclose%
\pgfusepath{stroke,fill}%
\end{pgfscope}%
\begin{pgfscope}%
\pgfsetbuttcap%
\pgfsetroundjoin%
\definecolor{currentfill}{rgb}{0.000000,0.000000,0.000000}%
\pgfsetfillcolor{currentfill}%
\pgfsetlinewidth{0.301125pt}%
\definecolor{currentstroke}{rgb}{0.000000,0.000000,0.000000}%
\pgfsetstrokecolor{currentstroke}%
\pgfsetdash{}{0pt}%
\pgfpathmoveto{\pgfqpoint{2.263889in}{2.147216in}}%
\pgfpathcurveto{\pgfqpoint{2.268007in}{2.147216in}}{\pgfqpoint{2.271957in}{2.148852in}}{\pgfqpoint{2.274869in}{2.151764in}}%
\pgfpathcurveto{\pgfqpoint{2.277781in}{2.154676in}}{\pgfqpoint{2.279417in}{2.158626in}}{\pgfqpoint{2.279417in}{2.162744in}}%
\pgfpathcurveto{\pgfqpoint{2.279417in}{2.166862in}}{\pgfqpoint{2.277781in}{2.170812in}}{\pgfqpoint{2.274869in}{2.173724in}}%
\pgfpathcurveto{\pgfqpoint{2.271957in}{2.176636in}}{\pgfqpoint{2.268007in}{2.178272in}}{\pgfqpoint{2.263889in}{2.178272in}}%
\pgfpathcurveto{\pgfqpoint{2.259771in}{2.178272in}}{\pgfqpoint{2.255821in}{2.176636in}}{\pgfqpoint{2.252909in}{2.173724in}}%
\pgfpathcurveto{\pgfqpoint{2.249997in}{2.170812in}}{\pgfqpoint{2.248361in}{2.166862in}}{\pgfqpoint{2.248361in}{2.162744in}}%
\pgfpathcurveto{\pgfqpoint{2.248361in}{2.158626in}}{\pgfqpoint{2.249997in}{2.154676in}}{\pgfqpoint{2.252909in}{2.151764in}}%
\pgfpathcurveto{\pgfqpoint{2.255821in}{2.148852in}}{\pgfqpoint{2.259771in}{2.147216in}}{\pgfqpoint{2.263889in}{2.147216in}}%
\pgfpathclose%
\pgfusepath{stroke,fill}%
\end{pgfscope}%
\begin{pgfscope}%
\pgfsetbuttcap%
\pgfsetroundjoin%
\definecolor{currentfill}{rgb}{0.000000,0.000000,0.000000}%
\pgfsetfillcolor{currentfill}%
\pgfsetlinewidth{0.301125pt}%
\definecolor{currentstroke}{rgb}{0.000000,0.000000,0.000000}%
\pgfsetstrokecolor{currentstroke}%
\pgfsetdash{}{0pt}%
\pgfpathmoveto{\pgfqpoint{2.263889in}{2.147216in}}%
\pgfpathcurveto{\pgfqpoint{2.268007in}{2.147216in}}{\pgfqpoint{2.271957in}{2.148852in}}{\pgfqpoint{2.274869in}{2.151764in}}%
\pgfpathcurveto{\pgfqpoint{2.277781in}{2.154676in}}{\pgfqpoint{2.279417in}{2.158626in}}{\pgfqpoint{2.279417in}{2.162744in}}%
\pgfpathcurveto{\pgfqpoint{2.279417in}{2.166862in}}{\pgfqpoint{2.277781in}{2.170812in}}{\pgfqpoint{2.274869in}{2.173724in}}%
\pgfpathcurveto{\pgfqpoint{2.271957in}{2.176636in}}{\pgfqpoint{2.268007in}{2.178272in}}{\pgfqpoint{2.263889in}{2.178272in}}%
\pgfpathcurveto{\pgfqpoint{2.259771in}{2.178272in}}{\pgfqpoint{2.255821in}{2.176636in}}{\pgfqpoint{2.252909in}{2.173724in}}%
\pgfpathcurveto{\pgfqpoint{2.249997in}{2.170812in}}{\pgfqpoint{2.248361in}{2.166862in}}{\pgfqpoint{2.248361in}{2.162744in}}%
\pgfpathcurveto{\pgfqpoint{2.248361in}{2.158626in}}{\pgfqpoint{2.249997in}{2.154676in}}{\pgfqpoint{2.252909in}{2.151764in}}%
\pgfpathcurveto{\pgfqpoint{2.255821in}{2.148852in}}{\pgfqpoint{2.259771in}{2.147216in}}{\pgfqpoint{2.263889in}{2.147216in}}%
\pgfpathclose%
\pgfusepath{stroke,fill}%
\end{pgfscope}%
\begin{pgfscope}%
\definecolor{textcolor}{rgb}{0.150000,0.150000,0.150000}%
\pgfsetstrokecolor{textcolor}%
\pgfsetfillcolor{textcolor}%
\pgftext[x=2.513889in,y=2.126285in,left,base]{\color{textcolor}\sffamily\fontsize{10.000000}{12.000000}\selectfont test}%
\end{pgfscope}%
\begin{pgfscope}%
\pgfsetbuttcap%
\pgfsetmiterjoin%
\definecolor{currentfill}{rgb}{1.000000,1.000000,1.000000}%
\pgfsetfillcolor{currentfill}%
\pgfsetlinewidth{0.000000pt}%
\definecolor{currentstroke}{rgb}{0.000000,0.000000,0.000000}%
\pgfsetstrokecolor{currentstroke}%
\pgfsetstrokeopacity{0.000000}%
\pgfsetdash{}{0pt}%
\pgfpathmoveto{\pgfqpoint{7.105882in}{1.973684in}}%
\pgfpathlineto{\pgfqpoint{11.482353in}{1.973684in}}%
\pgfpathlineto{\pgfqpoint{11.482353in}{2.952632in}}%
\pgfpathlineto{\pgfqpoint{7.105882in}{2.952632in}}%
\pgfpathclose%
\pgfusepath{fill}%
\end{pgfscope}%
\begin{pgfscope}%
\pgfpathrectangle{\pgfqpoint{7.105882in}{1.973684in}}{\pgfqpoint{4.376471in}{0.978947in}} %
\pgfusepath{clip}%
\pgfsetroundcap%
\pgfsetroundjoin%
\pgfsetlinewidth{1.003750pt}%
\definecolor{currentstroke}{rgb}{0.800000,0.800000,0.800000}%
\pgfsetstrokecolor{currentstroke}%
\pgfsetdash{}{0pt}%
\pgfpathmoveto{\pgfqpoint{7.105882in}{1.973684in}}%
\pgfpathlineto{\pgfqpoint{7.105882in}{2.952632in}}%
\pgfusepath{stroke}%
\end{pgfscope}%
\begin{pgfscope}%
\pgfpathrectangle{\pgfqpoint{7.105882in}{1.973684in}}{\pgfqpoint{4.376471in}{0.978947in}} %
\pgfusepath{clip}%
\pgfsetroundcap%
\pgfsetroundjoin%
\pgfsetlinewidth{1.003750pt}%
\definecolor{currentstroke}{rgb}{0.800000,0.800000,0.800000}%
\pgfsetstrokecolor{currentstroke}%
\pgfsetdash{}{0pt}%
\pgfpathmoveto{\pgfqpoint{7.981176in}{1.973684in}}%
\pgfpathlineto{\pgfqpoint{7.981176in}{2.952632in}}%
\pgfusepath{stroke}%
\end{pgfscope}%
\begin{pgfscope}%
\pgfpathrectangle{\pgfqpoint{7.105882in}{1.973684in}}{\pgfqpoint{4.376471in}{0.978947in}} %
\pgfusepath{clip}%
\pgfsetroundcap%
\pgfsetroundjoin%
\pgfsetlinewidth{1.003750pt}%
\definecolor{currentstroke}{rgb}{0.800000,0.800000,0.800000}%
\pgfsetstrokecolor{currentstroke}%
\pgfsetdash{}{0pt}%
\pgfpathmoveto{\pgfqpoint{8.856471in}{1.973684in}}%
\pgfpathlineto{\pgfqpoint{8.856471in}{2.952632in}}%
\pgfusepath{stroke}%
\end{pgfscope}%
\begin{pgfscope}%
\pgfpathrectangle{\pgfqpoint{7.105882in}{1.973684in}}{\pgfqpoint{4.376471in}{0.978947in}} %
\pgfusepath{clip}%
\pgfsetroundcap%
\pgfsetroundjoin%
\pgfsetlinewidth{1.003750pt}%
\definecolor{currentstroke}{rgb}{0.800000,0.800000,0.800000}%
\pgfsetstrokecolor{currentstroke}%
\pgfsetdash{}{0pt}%
\pgfpathmoveto{\pgfqpoint{9.731765in}{1.973684in}}%
\pgfpathlineto{\pgfqpoint{9.731765in}{2.952632in}}%
\pgfusepath{stroke}%
\end{pgfscope}%
\begin{pgfscope}%
\pgfpathrectangle{\pgfqpoint{7.105882in}{1.973684in}}{\pgfqpoint{4.376471in}{0.978947in}} %
\pgfusepath{clip}%
\pgfsetroundcap%
\pgfsetroundjoin%
\pgfsetlinewidth{1.003750pt}%
\definecolor{currentstroke}{rgb}{0.800000,0.800000,0.800000}%
\pgfsetstrokecolor{currentstroke}%
\pgfsetdash{}{0pt}%
\pgfpathmoveto{\pgfqpoint{10.607059in}{1.973684in}}%
\pgfpathlineto{\pgfqpoint{10.607059in}{2.952632in}}%
\pgfusepath{stroke}%
\end{pgfscope}%
\begin{pgfscope}%
\pgfpathrectangle{\pgfqpoint{7.105882in}{1.973684in}}{\pgfqpoint{4.376471in}{0.978947in}} %
\pgfusepath{clip}%
\pgfsetroundcap%
\pgfsetroundjoin%
\pgfsetlinewidth{1.003750pt}%
\definecolor{currentstroke}{rgb}{0.800000,0.800000,0.800000}%
\pgfsetstrokecolor{currentstroke}%
\pgfsetdash{}{0pt}%
\pgfpathmoveto{\pgfqpoint{11.482353in}{1.973684in}}%
\pgfpathlineto{\pgfqpoint{11.482353in}{2.952632in}}%
\pgfusepath{stroke}%
\end{pgfscope}%
\begin{pgfscope}%
\pgfpathrectangle{\pgfqpoint{7.105882in}{1.973684in}}{\pgfqpoint{4.376471in}{0.978947in}} %
\pgfusepath{clip}%
\pgfsetroundcap%
\pgfsetroundjoin%
\pgfsetlinewidth{1.003750pt}%
\definecolor{currentstroke}{rgb}{0.800000,0.800000,0.800000}%
\pgfsetstrokecolor{currentstroke}%
\pgfsetdash{}{0pt}%
\pgfpathmoveto{\pgfqpoint{7.105882in}{2.136842in}}%
\pgfpathlineto{\pgfqpoint{11.482353in}{2.136842in}}%
\pgfusepath{stroke}%
\end{pgfscope}%
\begin{pgfscope}%
\definecolor{textcolor}{rgb}{0.150000,0.150000,0.150000}%
\pgfsetstrokecolor{textcolor}%
\pgfsetfillcolor{textcolor}%
\pgftext[x=7.008660in,y=2.136842in,right,]{\color{textcolor}\sffamily\fontsize{10.000000}{12.000000}\selectfont \(\displaystyle -1\)}%
\end{pgfscope}%
\begin{pgfscope}%
\pgfpathrectangle{\pgfqpoint{7.105882in}{1.973684in}}{\pgfqpoint{4.376471in}{0.978947in}} %
\pgfusepath{clip}%
\pgfsetroundcap%
\pgfsetroundjoin%
\pgfsetlinewidth{1.003750pt}%
\definecolor{currentstroke}{rgb}{0.800000,0.800000,0.800000}%
\pgfsetstrokecolor{currentstroke}%
\pgfsetdash{}{0pt}%
\pgfpathmoveto{\pgfqpoint{7.105882in}{2.340789in}}%
\pgfpathlineto{\pgfqpoint{11.482353in}{2.340789in}}%
\pgfusepath{stroke}%
\end{pgfscope}%
\begin{pgfscope}%
\definecolor{textcolor}{rgb}{0.150000,0.150000,0.150000}%
\pgfsetstrokecolor{textcolor}%
\pgfsetfillcolor{textcolor}%
\pgftext[x=7.008660in,y=2.340789in,right,]{\color{textcolor}\sffamily\fontsize{10.000000}{12.000000}\selectfont \(\displaystyle 0\)}%
\end{pgfscope}%
\begin{pgfscope}%
\pgfpathrectangle{\pgfqpoint{7.105882in}{1.973684in}}{\pgfqpoint{4.376471in}{0.978947in}} %
\pgfusepath{clip}%
\pgfsetroundcap%
\pgfsetroundjoin%
\pgfsetlinewidth{1.003750pt}%
\definecolor{currentstroke}{rgb}{0.800000,0.800000,0.800000}%
\pgfsetstrokecolor{currentstroke}%
\pgfsetdash{}{0pt}%
\pgfpathmoveto{\pgfqpoint{7.105882in}{2.544737in}}%
\pgfpathlineto{\pgfqpoint{11.482353in}{2.544737in}}%
\pgfusepath{stroke}%
\end{pgfscope}%
\begin{pgfscope}%
\definecolor{textcolor}{rgb}{0.150000,0.150000,0.150000}%
\pgfsetstrokecolor{textcolor}%
\pgfsetfillcolor{textcolor}%
\pgftext[x=7.008660in,y=2.544737in,right,]{\color{textcolor}\sffamily\fontsize{10.000000}{12.000000}\selectfont \(\displaystyle 1\)}%
\end{pgfscope}%
\begin{pgfscope}%
\pgfpathrectangle{\pgfqpoint{7.105882in}{1.973684in}}{\pgfqpoint{4.376471in}{0.978947in}} %
\pgfusepath{clip}%
\pgfsetroundcap%
\pgfsetroundjoin%
\pgfsetlinewidth{1.003750pt}%
\definecolor{currentstroke}{rgb}{0.800000,0.800000,0.800000}%
\pgfsetstrokecolor{currentstroke}%
\pgfsetdash{}{0pt}%
\pgfpathmoveto{\pgfqpoint{7.105882in}{2.748684in}}%
\pgfpathlineto{\pgfqpoint{11.482353in}{2.748684in}}%
\pgfusepath{stroke}%
\end{pgfscope}%
\begin{pgfscope}%
\definecolor{textcolor}{rgb}{0.150000,0.150000,0.150000}%
\pgfsetstrokecolor{textcolor}%
\pgfsetfillcolor{textcolor}%
\pgftext[x=7.008660in,y=2.748684in,right,]{\color{textcolor}\sffamily\fontsize{10.000000}{12.000000}\selectfont \(\displaystyle 2\)}%
\end{pgfscope}%
\begin{pgfscope}%
\pgfpathrectangle{\pgfqpoint{7.105882in}{1.973684in}}{\pgfqpoint{4.376471in}{0.978947in}} %
\pgfusepath{clip}%
\pgfsetroundcap%
\pgfsetroundjoin%
\pgfsetlinewidth{1.003750pt}%
\definecolor{currentstroke}{rgb}{0.800000,0.800000,0.800000}%
\pgfsetstrokecolor{currentstroke}%
\pgfsetdash{}{0pt}%
\pgfpathmoveto{\pgfqpoint{7.105882in}{2.952632in}}%
\pgfpathlineto{\pgfqpoint{11.482353in}{2.952632in}}%
\pgfusepath{stroke}%
\end{pgfscope}%
\begin{pgfscope}%
\definecolor{textcolor}{rgb}{0.150000,0.150000,0.150000}%
\pgfsetstrokecolor{textcolor}%
\pgfsetfillcolor{textcolor}%
\pgftext[x=7.008660in,y=2.952632in,right,]{\color{textcolor}\sffamily\fontsize{10.000000}{12.000000}\selectfont \(\displaystyle 3\)}%
\end{pgfscope}%
\begin{pgfscope}%
\pgfpathrectangle{\pgfqpoint{7.105882in}{1.973684in}}{\pgfqpoint{4.376471in}{0.978947in}} %
\pgfusepath{clip}%
\pgfsetbuttcap%
\pgfsetroundjoin%
\definecolor{currentfill}{rgb}{1.000000,0.000000,0.000000}%
\pgfsetfillcolor{currentfill}%
\pgfsetlinewidth{2.007500pt}%
\definecolor{currentstroke}{rgb}{1.000000,0.000000,0.000000}%
\pgfsetstrokecolor{currentstroke}%
\pgfsetdash{}{0pt}%
\pgfpathmoveto{\pgfqpoint{9.871613in}{2.531260in}}%
\pgfpathlineto{\pgfqpoint{9.933726in}{2.531260in}}%
\pgfpathmoveto{\pgfqpoint{9.902669in}{2.500203in}}%
\pgfpathlineto{\pgfqpoint{9.902669in}{2.562316in}}%
\pgfusepath{stroke,fill}%
\end{pgfscope}%
\begin{pgfscope}%
\pgfpathrectangle{\pgfqpoint{7.105882in}{1.973684in}}{\pgfqpoint{4.376471in}{0.978947in}} %
\pgfusepath{clip}%
\pgfsetbuttcap%
\pgfsetroundjoin%
\definecolor{currentfill}{rgb}{1.000000,0.000000,0.000000}%
\pgfsetfillcolor{currentfill}%
\pgfsetlinewidth{2.007500pt}%
\definecolor{currentstroke}{rgb}{1.000000,0.000000,0.000000}%
\pgfsetstrokecolor{currentstroke}%
\pgfsetdash{}{0pt}%
\pgfpathmoveto{\pgfqpoint{10.454124in}{2.214048in}}%
\pgfpathlineto{\pgfqpoint{10.516237in}{2.214048in}}%
\pgfpathmoveto{\pgfqpoint{10.485181in}{2.182991in}}%
\pgfpathlineto{\pgfqpoint{10.485181in}{2.245104in}}%
\pgfusepath{stroke,fill}%
\end{pgfscope}%
\begin{pgfscope}%
\pgfpathrectangle{\pgfqpoint{7.105882in}{1.973684in}}{\pgfqpoint{4.376471in}{0.978947in}} %
\pgfusepath{clip}%
\pgfsetbuttcap%
\pgfsetroundjoin%
\definecolor{currentfill}{rgb}{1.000000,0.000000,0.000000}%
\pgfsetfillcolor{currentfill}%
\pgfsetlinewidth{2.007500pt}%
\definecolor{currentstroke}{rgb}{1.000000,0.000000,0.000000}%
\pgfsetstrokecolor{currentstroke}%
\pgfsetdash{}{0pt}%
\pgfpathmoveto{\pgfqpoint{10.060501in}{2.584788in}}%
\pgfpathlineto{\pgfqpoint{10.122614in}{2.584788in}}%
\pgfpathmoveto{\pgfqpoint{10.091557in}{2.553731in}}%
\pgfpathlineto{\pgfqpoint{10.091557in}{2.615844in}}%
\pgfusepath{stroke,fill}%
\end{pgfscope}%
\begin{pgfscope}%
\pgfpathrectangle{\pgfqpoint{7.105882in}{1.973684in}}{\pgfqpoint{4.376471in}{0.978947in}} %
\pgfusepath{clip}%
\pgfsetbuttcap%
\pgfsetroundjoin%
\definecolor{currentfill}{rgb}{1.000000,0.000000,0.000000}%
\pgfsetfillcolor{currentfill}%
\pgfsetlinewidth{2.007500pt}%
\definecolor{currentstroke}{rgb}{1.000000,0.000000,0.000000}%
\pgfsetstrokecolor{currentstroke}%
\pgfsetdash{}{0pt}%
\pgfpathmoveto{\pgfqpoint{9.857852in}{2.473876in}}%
\pgfpathlineto{\pgfqpoint{9.919965in}{2.473876in}}%
\pgfpathmoveto{\pgfqpoint{9.888909in}{2.442819in}}%
\pgfpathlineto{\pgfqpoint{9.888909in}{2.504932in}}%
\pgfusepath{stroke,fill}%
\end{pgfscope}%
\begin{pgfscope}%
\pgfpathrectangle{\pgfqpoint{7.105882in}{1.973684in}}{\pgfqpoint{4.376471in}{0.978947in}} %
\pgfusepath{clip}%
\pgfsetbuttcap%
\pgfsetroundjoin%
\definecolor{currentfill}{rgb}{1.000000,0.000000,0.000000}%
\pgfsetfillcolor{currentfill}%
\pgfsetlinewidth{2.007500pt}%
\definecolor{currentstroke}{rgb}{1.000000,0.000000,0.000000}%
\pgfsetstrokecolor{currentstroke}%
\pgfsetdash{}{0pt}%
\pgfpathmoveto{\pgfqpoint{9.433410in}{2.147482in}}%
\pgfpathlineto{\pgfqpoint{9.495523in}{2.147482in}}%
\pgfpathmoveto{\pgfqpoint{9.464467in}{2.116425in}}%
\pgfpathlineto{\pgfqpoint{9.464467in}{2.178538in}}%
\pgfusepath{stroke,fill}%
\end{pgfscope}%
\begin{pgfscope}%
\pgfpathrectangle{\pgfqpoint{7.105882in}{1.973684in}}{\pgfqpoint{4.376471in}{0.978947in}} %
\pgfusepath{clip}%
\pgfsetbuttcap%
\pgfsetroundjoin%
\definecolor{currentfill}{rgb}{1.000000,0.000000,0.000000}%
\pgfsetfillcolor{currentfill}%
\pgfsetlinewidth{2.007500pt}%
\definecolor{currentstroke}{rgb}{1.000000,0.000000,0.000000}%
\pgfsetstrokecolor{currentstroke}%
\pgfsetdash{}{0pt}%
\pgfpathmoveto{\pgfqpoint{10.211509in}{2.416946in}}%
\pgfpathlineto{\pgfqpoint{10.273622in}{2.416946in}}%
\pgfpathmoveto{\pgfqpoint{10.242566in}{2.385889in}}%
\pgfpathlineto{\pgfqpoint{10.242566in}{2.448002in}}%
\pgfusepath{stroke,fill}%
\end{pgfscope}%
\begin{pgfscope}%
\pgfpathrectangle{\pgfqpoint{7.105882in}{1.973684in}}{\pgfqpoint{4.376471in}{0.978947in}} %
\pgfusepath{clip}%
\pgfsetbuttcap%
\pgfsetroundjoin%
\definecolor{currentfill}{rgb}{1.000000,0.000000,0.000000}%
\pgfsetfillcolor{currentfill}%
\pgfsetlinewidth{2.007500pt}%
\definecolor{currentstroke}{rgb}{1.000000,0.000000,0.000000}%
\pgfsetstrokecolor{currentstroke}%
\pgfsetdash{}{0pt}%
\pgfpathmoveto{\pgfqpoint{9.482190in}{2.184975in}}%
\pgfpathlineto{\pgfqpoint{9.544303in}{2.184975in}}%
\pgfpathmoveto{\pgfqpoint{9.513247in}{2.153919in}}%
\pgfpathlineto{\pgfqpoint{9.513247in}{2.216032in}}%
\pgfusepath{stroke,fill}%
\end{pgfscope}%
\begin{pgfscope}%
\pgfpathrectangle{\pgfqpoint{7.105882in}{1.973684in}}{\pgfqpoint{4.376471in}{0.978947in}} %
\pgfusepath{clip}%
\pgfsetbuttcap%
\pgfsetroundjoin%
\definecolor{currentfill}{rgb}{1.000000,0.000000,0.000000}%
\pgfsetfillcolor{currentfill}%
\pgfsetlinewidth{2.007500pt}%
\definecolor{currentstroke}{rgb}{1.000000,0.000000,0.000000}%
\pgfsetstrokecolor{currentstroke}%
\pgfsetdash{}{0pt}%
\pgfpathmoveto{\pgfqpoint{11.072375in}{2.825094in}}%
\pgfpathlineto{\pgfqpoint{11.134488in}{2.825094in}}%
\pgfpathmoveto{\pgfqpoint{11.103431in}{2.794037in}}%
\pgfpathlineto{\pgfqpoint{11.103431in}{2.856150in}}%
\pgfusepath{stroke,fill}%
\end{pgfscope}%
\begin{pgfscope}%
\pgfpathrectangle{\pgfqpoint{7.105882in}{1.973684in}}{\pgfqpoint{4.376471in}{0.978947in}} %
\pgfusepath{clip}%
\pgfsetbuttcap%
\pgfsetroundjoin%
\definecolor{currentfill}{rgb}{1.000000,0.000000,0.000000}%
\pgfsetfillcolor{currentfill}%
\pgfsetlinewidth{2.007500pt}%
\definecolor{currentstroke}{rgb}{1.000000,0.000000,0.000000}%
\pgfsetstrokecolor{currentstroke}%
\pgfsetdash{}{0pt}%
\pgfpathmoveto{\pgfqpoint{11.324073in}{2.725496in}}%
\pgfpathlineto{\pgfqpoint{11.386186in}{2.725496in}}%
\pgfpathmoveto{\pgfqpoint{11.355130in}{2.694440in}}%
\pgfpathlineto{\pgfqpoint{11.355130in}{2.756553in}}%
\pgfusepath{stroke,fill}%
\end{pgfscope}%
\begin{pgfscope}%
\pgfpathrectangle{\pgfqpoint{7.105882in}{1.973684in}}{\pgfqpoint{4.376471in}{0.978947in}} %
\pgfusepath{clip}%
\pgfsetbuttcap%
\pgfsetroundjoin%
\definecolor{currentfill}{rgb}{1.000000,0.000000,0.000000}%
\pgfsetfillcolor{currentfill}%
\pgfsetlinewidth{2.007500pt}%
\definecolor{currentstroke}{rgb}{1.000000,0.000000,0.000000}%
\pgfsetstrokecolor{currentstroke}%
\pgfsetdash{}{0pt}%
\pgfpathmoveto{\pgfqpoint{9.292616in}{2.222892in}}%
\pgfpathlineto{\pgfqpoint{9.354729in}{2.222892in}}%
\pgfpathmoveto{\pgfqpoint{9.323673in}{2.191836in}}%
\pgfpathlineto{\pgfqpoint{9.323673in}{2.253949in}}%
\pgfusepath{stroke,fill}%
\end{pgfscope}%
\begin{pgfscope}%
\pgfpathrectangle{\pgfqpoint{7.105882in}{1.973684in}}{\pgfqpoint{4.376471in}{0.978947in}} %
\pgfusepath{clip}%
\pgfsetbuttcap%
\pgfsetroundjoin%
\definecolor{currentfill}{rgb}{1.000000,0.000000,0.000000}%
\pgfsetfillcolor{currentfill}%
\pgfsetlinewidth{2.007500pt}%
\definecolor{currentstroke}{rgb}{1.000000,0.000000,0.000000}%
\pgfsetstrokecolor{currentstroke}%
\pgfsetdash{}{0pt}%
\pgfpathmoveto{\pgfqpoint{10.722089in}{2.373068in}}%
\pgfpathlineto{\pgfqpoint{10.784202in}{2.373068in}}%
\pgfpathmoveto{\pgfqpoint{10.753146in}{2.342011in}}%
\pgfpathlineto{\pgfqpoint{10.753146in}{2.404124in}}%
\pgfusepath{stroke,fill}%
\end{pgfscope}%
\begin{pgfscope}%
\pgfpathrectangle{\pgfqpoint{7.105882in}{1.973684in}}{\pgfqpoint{4.376471in}{0.978947in}} %
\pgfusepath{clip}%
\pgfsetbuttcap%
\pgfsetroundjoin%
\definecolor{currentfill}{rgb}{1.000000,0.000000,0.000000}%
\pgfsetfillcolor{currentfill}%
\pgfsetlinewidth{2.007500pt}%
\definecolor{currentstroke}{rgb}{1.000000,0.000000,0.000000}%
\pgfsetstrokecolor{currentstroke}%
\pgfsetdash{}{0pt}%
\pgfpathmoveto{\pgfqpoint{9.801874in}{2.413163in}}%
\pgfpathlineto{\pgfqpoint{9.863987in}{2.413163in}}%
\pgfpathmoveto{\pgfqpoint{9.832931in}{2.382107in}}%
\pgfpathlineto{\pgfqpoint{9.832931in}{2.444220in}}%
\pgfusepath{stroke,fill}%
\end{pgfscope}%
\begin{pgfscope}%
\pgfpathrectangle{\pgfqpoint{7.105882in}{1.973684in}}{\pgfqpoint{4.376471in}{0.978947in}} %
\pgfusepath{clip}%
\pgfsetbuttcap%
\pgfsetroundjoin%
\definecolor{currentfill}{rgb}{1.000000,0.000000,0.000000}%
\pgfsetfillcolor{currentfill}%
\pgfsetlinewidth{2.007500pt}%
\definecolor{currentstroke}{rgb}{1.000000,0.000000,0.000000}%
\pgfsetstrokecolor{currentstroke}%
\pgfsetdash{}{0pt}%
\pgfpathmoveto{\pgfqpoint{9.938944in}{2.540719in}}%
\pgfpathlineto{\pgfqpoint{10.001057in}{2.540719in}}%
\pgfpathmoveto{\pgfqpoint{9.970001in}{2.509663in}}%
\pgfpathlineto{\pgfqpoint{9.970001in}{2.571776in}}%
\pgfusepath{stroke,fill}%
\end{pgfscope}%
\begin{pgfscope}%
\pgfpathrectangle{\pgfqpoint{7.105882in}{1.973684in}}{\pgfqpoint{4.376471in}{0.978947in}} %
\pgfusepath{clip}%
\pgfsetbuttcap%
\pgfsetroundjoin%
\definecolor{currentfill}{rgb}{1.000000,0.000000,0.000000}%
\pgfsetfillcolor{currentfill}%
\pgfsetlinewidth{2.007500pt}%
\definecolor{currentstroke}{rgb}{1.000000,0.000000,0.000000}%
\pgfsetstrokecolor{currentstroke}%
\pgfsetdash{}{0pt}%
\pgfpathmoveto{\pgfqpoint{11.190797in}{2.800883in}}%
\pgfpathlineto{\pgfqpoint{11.252910in}{2.800883in}}%
\pgfpathmoveto{\pgfqpoint{11.221854in}{2.769826in}}%
\pgfpathlineto{\pgfqpoint{11.221854in}{2.831939in}}%
\pgfusepath{stroke,fill}%
\end{pgfscope}%
\begin{pgfscope}%
\pgfpathrectangle{\pgfqpoint{7.105882in}{1.973684in}}{\pgfqpoint{4.376471in}{0.978947in}} %
\pgfusepath{clip}%
\pgfsetbuttcap%
\pgfsetroundjoin%
\definecolor{currentfill}{rgb}{1.000000,0.000000,0.000000}%
\pgfsetfillcolor{currentfill}%
\pgfsetlinewidth{2.007500pt}%
\definecolor{currentstroke}{rgb}{1.000000,0.000000,0.000000}%
\pgfsetstrokecolor{currentstroke}%
\pgfsetdash{}{0pt}%
\pgfpathmoveto{\pgfqpoint{8.198830in}{2.513494in}}%
\pgfpathlineto{\pgfqpoint{8.260943in}{2.513494in}}%
\pgfpathmoveto{\pgfqpoint{8.229886in}{2.482437in}}%
\pgfpathlineto{\pgfqpoint{8.229886in}{2.544550in}}%
\pgfusepath{stroke,fill}%
\end{pgfscope}%
\begin{pgfscope}%
\pgfpathrectangle{\pgfqpoint{7.105882in}{1.973684in}}{\pgfqpoint{4.376471in}{0.978947in}} %
\pgfusepath{clip}%
\pgfsetbuttcap%
\pgfsetroundjoin%
\definecolor{currentfill}{rgb}{1.000000,0.000000,0.000000}%
\pgfsetfillcolor{currentfill}%
\pgfsetlinewidth{2.007500pt}%
\definecolor{currentstroke}{rgb}{1.000000,0.000000,0.000000}%
\pgfsetstrokecolor{currentstroke}%
\pgfsetdash{}{0pt}%
\pgfpathmoveto{\pgfqpoint{8.255175in}{2.455860in}}%
\pgfpathlineto{\pgfqpoint{8.317288in}{2.455860in}}%
\pgfpathmoveto{\pgfqpoint{8.286232in}{2.424803in}}%
\pgfpathlineto{\pgfqpoint{8.286232in}{2.486916in}}%
\pgfusepath{stroke,fill}%
\end{pgfscope}%
\begin{pgfscope}%
\pgfpathrectangle{\pgfqpoint{7.105882in}{1.973684in}}{\pgfqpoint{4.376471in}{0.978947in}} %
\pgfusepath{clip}%
\pgfsetbuttcap%
\pgfsetroundjoin%
\definecolor{currentfill}{rgb}{1.000000,0.000000,0.000000}%
\pgfsetfillcolor{currentfill}%
\pgfsetlinewidth{2.007500pt}%
\definecolor{currentstroke}{rgb}{1.000000,0.000000,0.000000}%
\pgfsetstrokecolor{currentstroke}%
\pgfsetdash{}{0pt}%
\pgfpathmoveto{\pgfqpoint{8.020908in}{2.743140in}}%
\pgfpathlineto{\pgfqpoint{8.083021in}{2.743140in}}%
\pgfpathmoveto{\pgfqpoint{8.051965in}{2.712083in}}%
\pgfpathlineto{\pgfqpoint{8.051965in}{2.774196in}}%
\pgfusepath{stroke,fill}%
\end{pgfscope}%
\begin{pgfscope}%
\pgfpathrectangle{\pgfqpoint{7.105882in}{1.973684in}}{\pgfqpoint{4.376471in}{0.978947in}} %
\pgfusepath{clip}%
\pgfsetbuttcap%
\pgfsetroundjoin%
\definecolor{currentfill}{rgb}{1.000000,0.000000,0.000000}%
\pgfsetfillcolor{currentfill}%
\pgfsetlinewidth{2.007500pt}%
\definecolor{currentstroke}{rgb}{1.000000,0.000000,0.000000}%
\pgfsetstrokecolor{currentstroke}%
\pgfsetdash{}{0pt}%
\pgfpathmoveto{\pgfqpoint{10.865269in}{2.588736in}}%
\pgfpathlineto{\pgfqpoint{10.927382in}{2.588736in}}%
\pgfpathmoveto{\pgfqpoint{10.896325in}{2.557679in}}%
\pgfpathlineto{\pgfqpoint{10.896325in}{2.619792in}}%
\pgfusepath{stroke,fill}%
\end{pgfscope}%
\begin{pgfscope}%
\pgfpathrectangle{\pgfqpoint{7.105882in}{1.973684in}}{\pgfqpoint{4.376471in}{0.978947in}} %
\pgfusepath{clip}%
\pgfsetbuttcap%
\pgfsetroundjoin%
\definecolor{currentfill}{rgb}{1.000000,0.000000,0.000000}%
\pgfsetfillcolor{currentfill}%
\pgfsetlinewidth{2.007500pt}%
\definecolor{currentstroke}{rgb}{1.000000,0.000000,0.000000}%
\pgfsetstrokecolor{currentstroke}%
\pgfsetdash{}{0pt}%
\pgfpathmoveto{\pgfqpoint{10.674584in}{2.311154in}}%
\pgfpathlineto{\pgfqpoint{10.736697in}{2.311154in}}%
\pgfpathmoveto{\pgfqpoint{10.705641in}{2.280098in}}%
\pgfpathlineto{\pgfqpoint{10.705641in}{2.342211in}}%
\pgfusepath{stroke,fill}%
\end{pgfscope}%
\begin{pgfscope}%
\pgfpathrectangle{\pgfqpoint{7.105882in}{1.973684in}}{\pgfqpoint{4.376471in}{0.978947in}} %
\pgfusepath{clip}%
\pgfsetbuttcap%
\pgfsetroundjoin%
\definecolor{currentfill}{rgb}{1.000000,0.000000,0.000000}%
\pgfsetfillcolor{currentfill}%
\pgfsetlinewidth{2.007500pt}%
\definecolor{currentstroke}{rgb}{1.000000,0.000000,0.000000}%
\pgfsetstrokecolor{currentstroke}%
\pgfsetdash{}{0pt}%
\pgfpathmoveto{\pgfqpoint{10.996186in}{2.718517in}}%
\pgfpathlineto{\pgfqpoint{11.058299in}{2.718517in}}%
\pgfpathmoveto{\pgfqpoint{11.027243in}{2.687461in}}%
\pgfpathlineto{\pgfqpoint{11.027243in}{2.749574in}}%
\pgfusepath{stroke,fill}%
\end{pgfscope}%
\begin{pgfscope}%
\pgfpathrectangle{\pgfqpoint{7.105882in}{1.973684in}}{\pgfqpoint{4.376471in}{0.978947in}} %
\pgfusepath{clip}%
\pgfsetbuttcap%
\pgfsetroundjoin%
\definecolor{currentfill}{rgb}{1.000000,0.000000,0.000000}%
\pgfsetfillcolor{currentfill}%
\pgfsetlinewidth{2.007500pt}%
\definecolor{currentstroke}{rgb}{1.000000,0.000000,0.000000}%
\pgfsetstrokecolor{currentstroke}%
\pgfsetdash{}{0pt}%
\pgfpathmoveto{\pgfqpoint{11.376435in}{2.649754in}}%
\pgfpathlineto{\pgfqpoint{11.438548in}{2.649754in}}%
\pgfpathmoveto{\pgfqpoint{11.407492in}{2.618698in}}%
\pgfpathlineto{\pgfqpoint{11.407492in}{2.680811in}}%
\pgfusepath{stroke,fill}%
\end{pgfscope}%
\begin{pgfscope}%
\pgfpathrectangle{\pgfqpoint{7.105882in}{1.973684in}}{\pgfqpoint{4.376471in}{0.978947in}} %
\pgfusepath{clip}%
\pgfsetbuttcap%
\pgfsetroundjoin%
\definecolor{currentfill}{rgb}{1.000000,0.000000,0.000000}%
\pgfsetfillcolor{currentfill}%
\pgfsetlinewidth{2.007500pt}%
\definecolor{currentstroke}{rgb}{1.000000,0.000000,0.000000}%
\pgfsetstrokecolor{currentstroke}%
\pgfsetdash{}{0pt}%
\pgfpathmoveto{\pgfqpoint{10.748115in}{2.466321in}}%
\pgfpathlineto{\pgfqpoint{10.810228in}{2.466321in}}%
\pgfpathmoveto{\pgfqpoint{10.779172in}{2.435265in}}%
\pgfpathlineto{\pgfqpoint{10.779172in}{2.497378in}}%
\pgfusepath{stroke,fill}%
\end{pgfscope}%
\begin{pgfscope}%
\pgfpathrectangle{\pgfqpoint{7.105882in}{1.973684in}}{\pgfqpoint{4.376471in}{0.978947in}} %
\pgfusepath{clip}%
\pgfsetbuttcap%
\pgfsetroundjoin%
\definecolor{currentfill}{rgb}{1.000000,0.000000,0.000000}%
\pgfsetfillcolor{currentfill}%
\pgfsetlinewidth{2.007500pt}%
\definecolor{currentstroke}{rgb}{1.000000,0.000000,0.000000}%
\pgfsetstrokecolor{currentstroke}%
\pgfsetdash{}{0pt}%
\pgfpathmoveto{\pgfqpoint{9.565841in}{2.190781in}}%
\pgfpathlineto{\pgfqpoint{9.627954in}{2.190781in}}%
\pgfpathmoveto{\pgfqpoint{9.596897in}{2.159724in}}%
\pgfpathlineto{\pgfqpoint{9.596897in}{2.221837in}}%
\pgfusepath{stroke,fill}%
\end{pgfscope}%
\begin{pgfscope}%
\pgfpathrectangle{\pgfqpoint{7.105882in}{1.973684in}}{\pgfqpoint{4.376471in}{0.978947in}} %
\pgfusepath{clip}%
\pgfsetbuttcap%
\pgfsetroundjoin%
\definecolor{currentfill}{rgb}{1.000000,0.000000,0.000000}%
\pgfsetfillcolor{currentfill}%
\pgfsetlinewidth{2.007500pt}%
\definecolor{currentstroke}{rgb}{1.000000,0.000000,0.000000}%
\pgfsetstrokecolor{currentstroke}%
\pgfsetdash{}{0pt}%
\pgfpathmoveto{\pgfqpoint{10.682890in}{2.333191in}}%
\pgfpathlineto{\pgfqpoint{10.745003in}{2.333191in}}%
\pgfpathmoveto{\pgfqpoint{10.713947in}{2.302134in}}%
\pgfpathlineto{\pgfqpoint{10.713947in}{2.364247in}}%
\pgfusepath{stroke,fill}%
\end{pgfscope}%
\begin{pgfscope}%
\pgfpathrectangle{\pgfqpoint{7.105882in}{1.973684in}}{\pgfqpoint{4.376471in}{0.978947in}} %
\pgfusepath{clip}%
\pgfsetbuttcap%
\pgfsetroundjoin%
\definecolor{currentfill}{rgb}{1.000000,0.000000,0.000000}%
\pgfsetfillcolor{currentfill}%
\pgfsetlinewidth{2.007500pt}%
\definecolor{currentstroke}{rgb}{1.000000,0.000000,0.000000}%
\pgfsetstrokecolor{currentstroke}%
\pgfsetdash{}{0pt}%
\pgfpathmoveto{\pgfqpoint{8.364220in}{2.353928in}}%
\pgfpathlineto{\pgfqpoint{8.426333in}{2.353928in}}%
\pgfpathmoveto{\pgfqpoint{8.395276in}{2.322872in}}%
\pgfpathlineto{\pgfqpoint{8.395276in}{2.384985in}}%
\pgfusepath{stroke,fill}%
\end{pgfscope}%
\begin{pgfscope}%
\pgfpathrectangle{\pgfqpoint{7.105882in}{1.973684in}}{\pgfqpoint{4.376471in}{0.978947in}} %
\pgfusepath{clip}%
\pgfsetbuttcap%
\pgfsetroundjoin%
\definecolor{currentfill}{rgb}{1.000000,0.000000,0.000000}%
\pgfsetfillcolor{currentfill}%
\pgfsetlinewidth{2.007500pt}%
\definecolor{currentstroke}{rgb}{1.000000,0.000000,0.000000}%
\pgfsetstrokecolor{currentstroke}%
\pgfsetdash{}{0pt}%
\pgfpathmoveto{\pgfqpoint{10.190596in}{2.457212in}}%
\pgfpathlineto{\pgfqpoint{10.252709in}{2.457212in}}%
\pgfpathmoveto{\pgfqpoint{10.221653in}{2.426156in}}%
\pgfpathlineto{\pgfqpoint{10.221653in}{2.488269in}}%
\pgfusepath{stroke,fill}%
\end{pgfscope}%
\begin{pgfscope}%
\pgfpathrectangle{\pgfqpoint{7.105882in}{1.973684in}}{\pgfqpoint{4.376471in}{0.978947in}} %
\pgfusepath{clip}%
\pgfsetbuttcap%
\pgfsetroundjoin%
\definecolor{currentfill}{rgb}{1.000000,0.000000,0.000000}%
\pgfsetfillcolor{currentfill}%
\pgfsetlinewidth{2.007500pt}%
\definecolor{currentstroke}{rgb}{1.000000,0.000000,0.000000}%
\pgfsetstrokecolor{currentstroke}%
\pgfsetdash{}{0pt}%
\pgfpathmoveto{\pgfqpoint{8.452025in}{2.362218in}}%
\pgfpathlineto{\pgfqpoint{8.514138in}{2.362218in}}%
\pgfpathmoveto{\pgfqpoint{8.483082in}{2.331162in}}%
\pgfpathlineto{\pgfqpoint{8.483082in}{2.393275in}}%
\pgfusepath{stroke,fill}%
\end{pgfscope}%
\begin{pgfscope}%
\pgfpathrectangle{\pgfqpoint{7.105882in}{1.973684in}}{\pgfqpoint{4.376471in}{0.978947in}} %
\pgfusepath{clip}%
\pgfsetbuttcap%
\pgfsetroundjoin%
\definecolor{currentfill}{rgb}{1.000000,0.000000,0.000000}%
\pgfsetfillcolor{currentfill}%
\pgfsetlinewidth{2.007500pt}%
\definecolor{currentstroke}{rgb}{1.000000,0.000000,0.000000}%
\pgfsetstrokecolor{currentstroke}%
\pgfsetdash{}{0pt}%
\pgfpathmoveto{\pgfqpoint{11.257573in}{2.762412in}}%
\pgfpathlineto{\pgfqpoint{11.319686in}{2.762412in}}%
\pgfpathmoveto{\pgfqpoint{11.288629in}{2.731355in}}%
\pgfpathlineto{\pgfqpoint{11.288629in}{2.793468in}}%
\pgfusepath{stroke,fill}%
\end{pgfscope}%
\begin{pgfscope}%
\pgfpathrectangle{\pgfqpoint{7.105882in}{1.973684in}}{\pgfqpoint{4.376471in}{0.978947in}} %
\pgfusepath{clip}%
\pgfsetbuttcap%
\pgfsetroundjoin%
\definecolor{currentfill}{rgb}{1.000000,0.000000,0.000000}%
\pgfsetfillcolor{currentfill}%
\pgfsetlinewidth{2.007500pt}%
\definecolor{currentstroke}{rgb}{1.000000,0.000000,0.000000}%
\pgfsetstrokecolor{currentstroke}%
\pgfsetdash{}{0pt}%
\pgfpathmoveto{\pgfqpoint{9.777203in}{2.409615in}}%
\pgfpathlineto{\pgfqpoint{9.839316in}{2.409615in}}%
\pgfpathmoveto{\pgfqpoint{9.808260in}{2.378558in}}%
\pgfpathlineto{\pgfqpoint{9.808260in}{2.440671in}}%
\pgfusepath{stroke,fill}%
\end{pgfscope}%
\begin{pgfscope}%
\pgfpathrectangle{\pgfqpoint{7.105882in}{1.973684in}}{\pgfqpoint{4.376471in}{0.978947in}} %
\pgfusepath{clip}%
\pgfsetbuttcap%
\pgfsetroundjoin%
\definecolor{currentfill}{rgb}{1.000000,0.000000,0.000000}%
\pgfsetfillcolor{currentfill}%
\pgfsetlinewidth{2.007500pt}%
\definecolor{currentstroke}{rgb}{1.000000,0.000000,0.000000}%
\pgfsetstrokecolor{currentstroke}%
\pgfsetdash{}{0pt}%
\pgfpathmoveto{\pgfqpoint{9.401925in}{2.158497in}}%
\pgfpathlineto{\pgfqpoint{9.464038in}{2.158497in}}%
\pgfpathmoveto{\pgfqpoint{9.432981in}{2.127441in}}%
\pgfpathlineto{\pgfqpoint{9.432981in}{2.189554in}}%
\pgfusepath{stroke,fill}%
\end{pgfscope}%
\begin{pgfscope}%
\pgfpathrectangle{\pgfqpoint{7.105882in}{1.973684in}}{\pgfqpoint{4.376471in}{0.978947in}} %
\pgfusepath{clip}%
\pgfsetbuttcap%
\pgfsetroundjoin%
\definecolor{currentfill}{rgb}{0.000000,0.000000,0.000000}%
\pgfsetfillcolor{currentfill}%
\pgfsetlinewidth{0.301125pt}%
\definecolor{currentstroke}{rgb}{0.000000,0.000000,0.000000}%
\pgfsetstrokecolor{currentstroke}%
\pgfsetdash{}{0pt}%
\pgfsys@defobject{currentmarker}{\pgfqpoint{-0.015528in}{-0.015528in}}{\pgfqpoint{0.015528in}{0.015528in}}{%
\pgfpathmoveto{\pgfqpoint{0.000000in}{-0.015528in}}%
\pgfpathcurveto{\pgfqpoint{0.004118in}{-0.015528in}}{\pgfqpoint{0.008068in}{-0.013892in}}{\pgfqpoint{0.010980in}{-0.010980in}}%
\pgfpathcurveto{\pgfqpoint{0.013892in}{-0.008068in}}{\pgfqpoint{0.015528in}{-0.004118in}}{\pgfqpoint{0.015528in}{0.000000in}}%
\pgfpathcurveto{\pgfqpoint{0.015528in}{0.004118in}}{\pgfqpoint{0.013892in}{0.008068in}}{\pgfqpoint{0.010980in}{0.010980in}}%
\pgfpathcurveto{\pgfqpoint{0.008068in}{0.013892in}}{\pgfqpoint{0.004118in}{0.015528in}}{\pgfqpoint{0.000000in}{0.015528in}}%
\pgfpathcurveto{\pgfqpoint{-0.004118in}{0.015528in}}{\pgfqpoint{-0.008068in}{0.013892in}}{\pgfqpoint{-0.010980in}{0.010980in}}%
\pgfpathcurveto{\pgfqpoint{-0.013892in}{0.008068in}}{\pgfqpoint{-0.015528in}{0.004118in}}{\pgfqpoint{-0.015528in}{0.000000in}}%
\pgfpathcurveto{\pgfqpoint{-0.015528in}{-0.004118in}}{\pgfqpoint{-0.013892in}{-0.008068in}}{\pgfqpoint{-0.010980in}{-0.010980in}}%
\pgfpathcurveto{\pgfqpoint{-0.008068in}{-0.013892in}}{\pgfqpoint{-0.004118in}{-0.015528in}}{\pgfqpoint{0.000000in}{-0.015528in}}%
\pgfpathclose%
\pgfusepath{stroke,fill}%
}%
\begin{pgfscope}%
\pgfsys@transformshift{7.981176in}{2.807547in}%
\pgfsys@useobject{currentmarker}{}%
\end{pgfscope}%
\begin{pgfscope}%
\pgfsys@transformshift{7.998770in}{2.713355in}%
\pgfsys@useobject{currentmarker}{}%
\end{pgfscope}%
\begin{pgfscope}%
\pgfsys@transformshift{8.016364in}{2.804129in}%
\pgfsys@useobject{currentmarker}{}%
\end{pgfscope}%
\begin{pgfscope}%
\pgfsys@transformshift{8.033958in}{2.823266in}%
\pgfsys@useobject{currentmarker}{}%
\end{pgfscope}%
\begin{pgfscope}%
\pgfsys@transformshift{8.051552in}{2.758481in}%
\pgfsys@useobject{currentmarker}{}%
\end{pgfscope}%
\begin{pgfscope}%
\pgfsys@transformshift{8.069146in}{2.754380in}%
\pgfsys@useobject{currentmarker}{}%
\end{pgfscope}%
\begin{pgfscope}%
\pgfsys@transformshift{8.086740in}{2.630622in}%
\pgfsys@useobject{currentmarker}{}%
\end{pgfscope}%
\begin{pgfscope}%
\pgfsys@transformshift{8.104333in}{2.630227in}%
\pgfsys@useobject{currentmarker}{}%
\end{pgfscope}%
\begin{pgfscope}%
\pgfsys@transformshift{8.121927in}{2.570900in}%
\pgfsys@useobject{currentmarker}{}%
\end{pgfscope}%
\begin{pgfscope}%
\pgfsys@transformshift{8.139521in}{2.575614in}%
\pgfsys@useobject{currentmarker}{}%
\end{pgfscope}%
\begin{pgfscope}%
\pgfsys@transformshift{8.157115in}{2.502913in}%
\pgfsys@useobject{currentmarker}{}%
\end{pgfscope}%
\begin{pgfscope}%
\pgfsys@transformshift{8.174709in}{2.384290in}%
\pgfsys@useobject{currentmarker}{}%
\end{pgfscope}%
\begin{pgfscope}%
\pgfsys@transformshift{8.192303in}{2.554036in}%
\pgfsys@useobject{currentmarker}{}%
\end{pgfscope}%
\begin{pgfscope}%
\pgfsys@transformshift{8.209897in}{2.471885in}%
\pgfsys@useobject{currentmarker}{}%
\end{pgfscope}%
\begin{pgfscope}%
\pgfsys@transformshift{8.227490in}{2.324989in}%
\pgfsys@useobject{currentmarker}{}%
\end{pgfscope}%
\begin{pgfscope}%
\pgfsys@transformshift{8.245084in}{2.518392in}%
\pgfsys@useobject{currentmarker}{}%
\end{pgfscope}%
\begin{pgfscope}%
\pgfsys@transformshift{8.262678in}{2.360390in}%
\pgfsys@useobject{currentmarker}{}%
\end{pgfscope}%
\begin{pgfscope}%
\pgfsys@transformshift{8.280272in}{2.441767in}%
\pgfsys@useobject{currentmarker}{}%
\end{pgfscope}%
\begin{pgfscope}%
\pgfsys@transformshift{8.297866in}{2.496324in}%
\pgfsys@useobject{currentmarker}{}%
\end{pgfscope}%
\begin{pgfscope}%
\pgfsys@transformshift{8.315460in}{2.422660in}%
\pgfsys@useobject{currentmarker}{}%
\end{pgfscope}%
\begin{pgfscope}%
\pgfsys@transformshift{8.333054in}{2.515310in}%
\pgfsys@useobject{currentmarker}{}%
\end{pgfscope}%
\begin{pgfscope}%
\pgfsys@transformshift{8.350647in}{2.264938in}%
\pgfsys@useobject{currentmarker}{}%
\end{pgfscope}%
\begin{pgfscope}%
\pgfsys@transformshift{8.368241in}{2.425762in}%
\pgfsys@useobject{currentmarker}{}%
\end{pgfscope}%
\begin{pgfscope}%
\pgfsys@transformshift{8.385835in}{2.310913in}%
\pgfsys@useobject{currentmarker}{}%
\end{pgfscope}%
\begin{pgfscope}%
\pgfsys@transformshift{8.403429in}{2.290072in}%
\pgfsys@useobject{currentmarker}{}%
\end{pgfscope}%
\begin{pgfscope}%
\pgfsys@transformshift{8.421023in}{2.320035in}%
\pgfsys@useobject{currentmarker}{}%
\end{pgfscope}%
\begin{pgfscope}%
\pgfsys@transformshift{8.438617in}{2.349489in}%
\pgfsys@useobject{currentmarker}{}%
\end{pgfscope}%
\begin{pgfscope}%
\pgfsys@transformshift{8.456210in}{2.391104in}%
\pgfsys@useobject{currentmarker}{}%
\end{pgfscope}%
\begin{pgfscope}%
\pgfsys@transformshift{8.473804in}{2.272497in}%
\pgfsys@useobject{currentmarker}{}%
\end{pgfscope}%
\begin{pgfscope}%
\pgfsys@transformshift{8.491398in}{2.490806in}%
\pgfsys@useobject{currentmarker}{}%
\end{pgfscope}%
\begin{pgfscope}%
\pgfsys@transformshift{8.508992in}{2.455626in}%
\pgfsys@useobject{currentmarker}{}%
\end{pgfscope}%
\begin{pgfscope}%
\pgfsys@transformshift{8.526586in}{2.262090in}%
\pgfsys@useobject{currentmarker}{}%
\end{pgfscope}%
\begin{pgfscope}%
\pgfsys@transformshift{8.544180in}{2.582363in}%
\pgfsys@useobject{currentmarker}{}%
\end{pgfscope}%
\begin{pgfscope}%
\pgfsys@transformshift{8.561774in}{2.636855in}%
\pgfsys@useobject{currentmarker}{}%
\end{pgfscope}%
\begin{pgfscope}%
\pgfsys@transformshift{8.579367in}{2.577536in}%
\pgfsys@useobject{currentmarker}{}%
\end{pgfscope}%
\begin{pgfscope}%
\pgfsys@transformshift{8.596961in}{2.453483in}%
\pgfsys@useobject{currentmarker}{}%
\end{pgfscope}%
\begin{pgfscope}%
\pgfsys@transformshift{8.614555in}{2.377630in}%
\pgfsys@useobject{currentmarker}{}%
\end{pgfscope}%
\begin{pgfscope}%
\pgfsys@transformshift{8.632149in}{2.609618in}%
\pgfsys@useobject{currentmarker}{}%
\end{pgfscope}%
\begin{pgfscope}%
\pgfsys@transformshift{8.649743in}{2.476330in}%
\pgfsys@useobject{currentmarker}{}%
\end{pgfscope}%
\begin{pgfscope}%
\pgfsys@transformshift{8.667337in}{2.657325in}%
\pgfsys@useobject{currentmarker}{}%
\end{pgfscope}%
\begin{pgfscope}%
\pgfsys@transformshift{8.684931in}{2.568798in}%
\pgfsys@useobject{currentmarker}{}%
\end{pgfscope}%
\begin{pgfscope}%
\pgfsys@transformshift{8.702524in}{2.661511in}%
\pgfsys@useobject{currentmarker}{}%
\end{pgfscope}%
\begin{pgfscope}%
\pgfsys@transformshift{8.720118in}{2.611893in}%
\pgfsys@useobject{currentmarker}{}%
\end{pgfscope}%
\begin{pgfscope}%
\pgfsys@transformshift{8.737712in}{2.660326in}%
\pgfsys@useobject{currentmarker}{}%
\end{pgfscope}%
\begin{pgfscope}%
\pgfsys@transformshift{8.755306in}{2.600979in}%
\pgfsys@useobject{currentmarker}{}%
\end{pgfscope}%
\begin{pgfscope}%
\pgfsys@transformshift{8.772900in}{2.792401in}%
\pgfsys@useobject{currentmarker}{}%
\end{pgfscope}%
\begin{pgfscope}%
\pgfsys@transformshift{8.790494in}{2.632210in}%
\pgfsys@useobject{currentmarker}{}%
\end{pgfscope}%
\begin{pgfscope}%
\pgfsys@transformshift{8.808087in}{2.667702in}%
\pgfsys@useobject{currentmarker}{}%
\end{pgfscope}%
\begin{pgfscope}%
\pgfsys@transformshift{8.825681in}{2.824526in}%
\pgfsys@useobject{currentmarker}{}%
\end{pgfscope}%
\begin{pgfscope}%
\pgfsys@transformshift{8.843275in}{2.499083in}%
\pgfsys@useobject{currentmarker}{}%
\end{pgfscope}%
\begin{pgfscope}%
\pgfsys@transformshift{8.860869in}{2.509147in}%
\pgfsys@useobject{currentmarker}{}%
\end{pgfscope}%
\begin{pgfscope}%
\pgfsys@transformshift{8.878463in}{2.737834in}%
\pgfsys@useobject{currentmarker}{}%
\end{pgfscope}%
\begin{pgfscope}%
\pgfsys@transformshift{8.896057in}{2.517682in}%
\pgfsys@useobject{currentmarker}{}%
\end{pgfscope}%
\begin{pgfscope}%
\pgfsys@transformshift{8.913651in}{2.831865in}%
\pgfsys@useobject{currentmarker}{}%
\end{pgfscope}%
\begin{pgfscope}%
\pgfsys@transformshift{8.931244in}{2.585872in}%
\pgfsys@useobject{currentmarker}{}%
\end{pgfscope}%
\begin{pgfscope}%
\pgfsys@transformshift{8.948838in}{2.544256in}%
\pgfsys@useobject{currentmarker}{}%
\end{pgfscope}%
\begin{pgfscope}%
\pgfsys@transformshift{8.966432in}{2.807077in}%
\pgfsys@useobject{currentmarker}{}%
\end{pgfscope}%
\begin{pgfscope}%
\pgfsys@transformshift{8.984026in}{2.750613in}%
\pgfsys@useobject{currentmarker}{}%
\end{pgfscope}%
\begin{pgfscope}%
\pgfsys@transformshift{9.001620in}{2.776955in}%
\pgfsys@useobject{currentmarker}{}%
\end{pgfscope}%
\begin{pgfscope}%
\pgfsys@transformshift{9.019214in}{2.664098in}%
\pgfsys@useobject{currentmarker}{}%
\end{pgfscope}%
\begin{pgfscope}%
\pgfsys@transformshift{9.036808in}{2.467509in}%
\pgfsys@useobject{currentmarker}{}%
\end{pgfscope}%
\begin{pgfscope}%
\pgfsys@transformshift{9.054401in}{2.732300in}%
\pgfsys@useobject{currentmarker}{}%
\end{pgfscope}%
\begin{pgfscope}%
\pgfsys@transformshift{9.071995in}{2.491099in}%
\pgfsys@useobject{currentmarker}{}%
\end{pgfscope}%
\begin{pgfscope}%
\pgfsys@transformshift{9.089589in}{2.580038in}%
\pgfsys@useobject{currentmarker}{}%
\end{pgfscope}%
\begin{pgfscope}%
\pgfsys@transformshift{9.107183in}{2.573630in}%
\pgfsys@useobject{currentmarker}{}%
\end{pgfscope}%
\begin{pgfscope}%
\pgfsys@transformshift{9.124777in}{2.439287in}%
\pgfsys@useobject{currentmarker}{}%
\end{pgfscope}%
\begin{pgfscope}%
\pgfsys@transformshift{9.142371in}{2.495197in}%
\pgfsys@useobject{currentmarker}{}%
\end{pgfscope}%
\begin{pgfscope}%
\pgfsys@transformshift{9.159965in}{2.503723in}%
\pgfsys@useobject{currentmarker}{}%
\end{pgfscope}%
\begin{pgfscope}%
\pgfsys@transformshift{9.177558in}{2.424999in}%
\pgfsys@useobject{currentmarker}{}%
\end{pgfscope}%
\begin{pgfscope}%
\pgfsys@transformshift{9.195152in}{2.251467in}%
\pgfsys@useobject{currentmarker}{}%
\end{pgfscope}%
\begin{pgfscope}%
\pgfsys@transformshift{9.212746in}{2.371193in}%
\pgfsys@useobject{currentmarker}{}%
\end{pgfscope}%
\begin{pgfscope}%
\pgfsys@transformshift{9.230340in}{2.453684in}%
\pgfsys@useobject{currentmarker}{}%
\end{pgfscope}%
\begin{pgfscope}%
\pgfsys@transformshift{9.247934in}{2.225893in}%
\pgfsys@useobject{currentmarker}{}%
\end{pgfscope}%
\begin{pgfscope}%
\pgfsys@transformshift{9.265528in}{2.260597in}%
\pgfsys@useobject{currentmarker}{}%
\end{pgfscope}%
\begin{pgfscope}%
\pgfsys@transformshift{9.283121in}{2.211646in}%
\pgfsys@useobject{currentmarker}{}%
\end{pgfscope}%
\begin{pgfscope}%
\pgfsys@transformshift{9.300715in}{2.425970in}%
\pgfsys@useobject{currentmarker}{}%
\end{pgfscope}%
\begin{pgfscope}%
\pgfsys@transformshift{9.318309in}{2.288698in}%
\pgfsys@useobject{currentmarker}{}%
\end{pgfscope}%
\begin{pgfscope}%
\pgfsys@transformshift{9.335903in}{2.245970in}%
\pgfsys@useobject{currentmarker}{}%
\end{pgfscope}%
\begin{pgfscope}%
\pgfsys@transformshift{9.353497in}{2.111856in}%
\pgfsys@useobject{currentmarker}{}%
\end{pgfscope}%
\begin{pgfscope}%
\pgfsys@transformshift{9.371091in}{2.233096in}%
\pgfsys@useobject{currentmarker}{}%
\end{pgfscope}%
\begin{pgfscope}%
\pgfsys@transformshift{9.388685in}{2.098982in}%
\pgfsys@useobject{currentmarker}{}%
\end{pgfscope}%
\begin{pgfscope}%
\pgfsys@transformshift{9.406278in}{2.162618in}%
\pgfsys@useobject{currentmarker}{}%
\end{pgfscope}%
\begin{pgfscope}%
\pgfsys@transformshift{9.423872in}{2.088214in}%
\pgfsys@useobject{currentmarker}{}%
\end{pgfscope}%
\begin{pgfscope}%
\pgfsys@transformshift{9.441466in}{2.217817in}%
\pgfsys@useobject{currentmarker}{}%
\end{pgfscope}%
\begin{pgfscope}%
\pgfsys@transformshift{9.459060in}{2.205555in}%
\pgfsys@useobject{currentmarker}{}%
\end{pgfscope}%
\begin{pgfscope}%
\pgfsys@transformshift{9.476654in}{2.125576in}%
\pgfsys@useobject{currentmarker}{}%
\end{pgfscope}%
\begin{pgfscope}%
\pgfsys@transformshift{9.494248in}{2.189386in}%
\pgfsys@useobject{currentmarker}{}%
\end{pgfscope}%
\begin{pgfscope}%
\pgfsys@transformshift{9.511842in}{2.041820in}%
\pgfsys@useobject{currentmarker}{}%
\end{pgfscope}%
\begin{pgfscope}%
\pgfsys@transformshift{9.529435in}{2.007535in}%
\pgfsys@useobject{currentmarker}{}%
\end{pgfscope}%
\begin{pgfscope}%
\pgfsys@transformshift{9.547029in}{2.212697in}%
\pgfsys@useobject{currentmarker}{}%
\end{pgfscope}%
\begin{pgfscope}%
\pgfsys@transformshift{9.564623in}{2.195046in}%
\pgfsys@useobject{currentmarker}{}%
\end{pgfscope}%
\begin{pgfscope}%
\pgfsys@transformshift{9.582217in}{2.254728in}%
\pgfsys@useobject{currentmarker}{}%
\end{pgfscope}%
\begin{pgfscope}%
\pgfsys@transformshift{9.599811in}{2.446545in}%
\pgfsys@useobject{currentmarker}{}%
\end{pgfscope}%
\begin{pgfscope}%
\pgfsys@transformshift{9.617405in}{2.314886in}%
\pgfsys@useobject{currentmarker}{}%
\end{pgfscope}%
\begin{pgfscope}%
\pgfsys@transformshift{9.634999in}{2.141871in}%
\pgfsys@useobject{currentmarker}{}%
\end{pgfscope}%
\begin{pgfscope}%
\pgfsys@transformshift{9.652592in}{2.366405in}%
\pgfsys@useobject{currentmarker}{}%
\end{pgfscope}%
\begin{pgfscope}%
\pgfsys@transformshift{9.670186in}{2.136836in}%
\pgfsys@useobject{currentmarker}{}%
\end{pgfscope}%
\begin{pgfscope}%
\pgfsys@transformshift{9.687780in}{2.243272in}%
\pgfsys@useobject{currentmarker}{}%
\end{pgfscope}%
\begin{pgfscope}%
\pgfsys@transformshift{9.705374in}{2.303294in}%
\pgfsys@useobject{currentmarker}{}%
\end{pgfscope}%
\begin{pgfscope}%
\pgfsys@transformshift{9.722968in}{2.505271in}%
\pgfsys@useobject{currentmarker}{}%
\end{pgfscope}%
\begin{pgfscope}%
\pgfsys@transformshift{9.740562in}{2.275099in}%
\pgfsys@useobject{currentmarker}{}%
\end{pgfscope}%
\begin{pgfscope}%
\pgfsys@transformshift{9.758155in}{2.287237in}%
\pgfsys@useobject{currentmarker}{}%
\end{pgfscope}%
\begin{pgfscope}%
\pgfsys@transformshift{9.775749in}{2.381713in}%
\pgfsys@useobject{currentmarker}{}%
\end{pgfscope}%
\begin{pgfscope}%
\pgfsys@transformshift{9.793343in}{2.343907in}%
\pgfsys@useobject{currentmarker}{}%
\end{pgfscope}%
\begin{pgfscope}%
\pgfsys@transformshift{9.810937in}{2.545636in}%
\pgfsys@useobject{currentmarker}{}%
\end{pgfscope}%
\begin{pgfscope}%
\pgfsys@transformshift{9.828531in}{2.338992in}%
\pgfsys@useobject{currentmarker}{}%
\end{pgfscope}%
\begin{pgfscope}%
\pgfsys@transformshift{9.846125in}{2.349474in}%
\pgfsys@useobject{currentmarker}{}%
\end{pgfscope}%
\begin{pgfscope}%
\pgfsys@transformshift{9.863719in}{2.438041in}%
\pgfsys@useobject{currentmarker}{}%
\end{pgfscope}%
\begin{pgfscope}%
\pgfsys@transformshift{9.881312in}{2.446775in}%
\pgfsys@useobject{currentmarker}{}%
\end{pgfscope}%
\begin{pgfscope}%
\pgfsys@transformshift{9.898906in}{2.707728in}%
\pgfsys@useobject{currentmarker}{}%
\end{pgfscope}%
\begin{pgfscope}%
\pgfsys@transformshift{9.916500in}{2.619589in}%
\pgfsys@useobject{currentmarker}{}%
\end{pgfscope}%
\begin{pgfscope}%
\pgfsys@transformshift{9.934094in}{2.541800in}%
\pgfsys@useobject{currentmarker}{}%
\end{pgfscope}%
\begin{pgfscope}%
\pgfsys@transformshift{9.951688in}{2.416208in}%
\pgfsys@useobject{currentmarker}{}%
\end{pgfscope}%
\begin{pgfscope}%
\pgfsys@transformshift{9.969282in}{2.633696in}%
\pgfsys@useobject{currentmarker}{}%
\end{pgfscope}%
\begin{pgfscope}%
\pgfsys@transformshift{9.986876in}{2.450091in}%
\pgfsys@useobject{currentmarker}{}%
\end{pgfscope}%
\begin{pgfscope}%
\pgfsys@transformshift{10.004469in}{2.397092in}%
\pgfsys@useobject{currentmarker}{}%
\end{pgfscope}%
\begin{pgfscope}%
\pgfsys@transformshift{10.022063in}{2.676321in}%
\pgfsys@useobject{currentmarker}{}%
\end{pgfscope}%
\begin{pgfscope}%
\pgfsys@transformshift{10.039657in}{2.586081in}%
\pgfsys@useobject{currentmarker}{}%
\end{pgfscope}%
\begin{pgfscope}%
\pgfsys@transformshift{10.057251in}{2.644312in}%
\pgfsys@useobject{currentmarker}{}%
\end{pgfscope}%
\begin{pgfscope}%
\pgfsys@transformshift{10.074845in}{2.577668in}%
\pgfsys@useobject{currentmarker}{}%
\end{pgfscope}%
\begin{pgfscope}%
\pgfsys@transformshift{10.092439in}{2.625475in}%
\pgfsys@useobject{currentmarker}{}%
\end{pgfscope}%
\begin{pgfscope}%
\pgfsys@transformshift{10.110033in}{2.462914in}%
\pgfsys@useobject{currentmarker}{}%
\end{pgfscope}%
\begin{pgfscope}%
\pgfsys@transformshift{10.127626in}{2.413409in}%
\pgfsys@useobject{currentmarker}{}%
\end{pgfscope}%
\begin{pgfscope}%
\pgfsys@transformshift{10.145220in}{2.576448in}%
\pgfsys@useobject{currentmarker}{}%
\end{pgfscope}%
\begin{pgfscope}%
\pgfsys@transformshift{10.162814in}{2.411720in}%
\pgfsys@useobject{currentmarker}{}%
\end{pgfscope}%
\begin{pgfscope}%
\pgfsys@transformshift{10.180408in}{2.408824in}%
\pgfsys@useobject{currentmarker}{}%
\end{pgfscope}%
\begin{pgfscope}%
\pgfsys@transformshift{10.198002in}{2.417137in}%
\pgfsys@useobject{currentmarker}{}%
\end{pgfscope}%
\begin{pgfscope}%
\pgfsys@transformshift{10.215596in}{2.448956in}%
\pgfsys@useobject{currentmarker}{}%
\end{pgfscope}%
\begin{pgfscope}%
\pgfsys@transformshift{10.233189in}{2.393983in}%
\pgfsys@useobject{currentmarker}{}%
\end{pgfscope}%
\begin{pgfscope}%
\pgfsys@transformshift{10.250783in}{2.272299in}%
\pgfsys@useobject{currentmarker}{}%
\end{pgfscope}%
\begin{pgfscope}%
\pgfsys@transformshift{10.268377in}{2.329023in}%
\pgfsys@useobject{currentmarker}{}%
\end{pgfscope}%
\begin{pgfscope}%
\pgfsys@transformshift{10.285971in}{2.150000in}%
\pgfsys@useobject{currentmarker}{}%
\end{pgfscope}%
\begin{pgfscope}%
\pgfsys@transformshift{10.303565in}{2.422690in}%
\pgfsys@useobject{currentmarker}{}%
\end{pgfscope}%
\begin{pgfscope}%
\pgfsys@transformshift{10.321159in}{2.178109in}%
\pgfsys@useobject{currentmarker}{}%
\end{pgfscope}%
\begin{pgfscope}%
\pgfsys@transformshift{10.338753in}{2.211944in}%
\pgfsys@useobject{currentmarker}{}%
\end{pgfscope}%
\begin{pgfscope}%
\pgfsys@transformshift{10.356346in}{2.313709in}%
\pgfsys@useobject{currentmarker}{}%
\end{pgfscope}%
\begin{pgfscope}%
\pgfsys@transformshift{10.373940in}{2.217733in}%
\pgfsys@useobject{currentmarker}{}%
\end{pgfscope}%
\begin{pgfscope}%
\pgfsys@transformshift{10.391534in}{2.436373in}%
\pgfsys@useobject{currentmarker}{}%
\end{pgfscope}%
\begin{pgfscope}%
\pgfsys@transformshift{10.409128in}{2.134357in}%
\pgfsys@useobject{currentmarker}{}%
\end{pgfscope}%
\begin{pgfscope}%
\pgfsys@transformshift{10.426722in}{2.282046in}%
\pgfsys@useobject{currentmarker}{}%
\end{pgfscope}%
\begin{pgfscope}%
\pgfsys@transformshift{10.444316in}{2.241026in}%
\pgfsys@useobject{currentmarker}{}%
\end{pgfscope}%
\begin{pgfscope}%
\pgfsys@transformshift{10.461910in}{2.117866in}%
\pgfsys@useobject{currentmarker}{}%
\end{pgfscope}%
\begin{pgfscope}%
\pgfsys@transformshift{10.479503in}{2.284136in}%
\pgfsys@useobject{currentmarker}{}%
\end{pgfscope}%
\begin{pgfscope}%
\pgfsys@transformshift{10.497097in}{2.209022in}%
\pgfsys@useobject{currentmarker}{}%
\end{pgfscope}%
\begin{pgfscope}%
\pgfsys@transformshift{10.514691in}{2.302978in}%
\pgfsys@useobject{currentmarker}{}%
\end{pgfscope}%
\begin{pgfscope}%
\pgfsys@transformshift{10.532285in}{2.308095in}%
\pgfsys@useobject{currentmarker}{}%
\end{pgfscope}%
\begin{pgfscope}%
\pgfsys@transformshift{10.549879in}{2.446676in}%
\pgfsys@useobject{currentmarker}{}%
\end{pgfscope}%
\begin{pgfscope}%
\pgfsys@transformshift{10.567473in}{2.366477in}%
\pgfsys@useobject{currentmarker}{}%
\end{pgfscope}%
\begin{pgfscope}%
\pgfsys@transformshift{10.585067in}{2.198779in}%
\pgfsys@useobject{currentmarker}{}%
\end{pgfscope}%
\begin{pgfscope}%
\pgfsys@transformshift{10.602660in}{2.220369in}%
\pgfsys@useobject{currentmarker}{}%
\end{pgfscope}%
\begin{pgfscope}%
\pgfsys@transformshift{10.620254in}{2.367339in}%
\pgfsys@useobject{currentmarker}{}%
\end{pgfscope}%
\begin{pgfscope}%
\pgfsys@transformshift{10.637848in}{2.334453in}%
\pgfsys@useobject{currentmarker}{}%
\end{pgfscope}%
\begin{pgfscope}%
\pgfsys@transformshift{10.655442in}{2.347263in}%
\pgfsys@useobject{currentmarker}{}%
\end{pgfscope}%
\begin{pgfscope}%
\pgfsys@transformshift{10.673036in}{2.133275in}%
\pgfsys@useobject{currentmarker}{}%
\end{pgfscope}%
\begin{pgfscope}%
\pgfsys@transformshift{10.690630in}{2.313560in}%
\pgfsys@useobject{currentmarker}{}%
\end{pgfscope}%
\begin{pgfscope}%
\pgfsys@transformshift{10.708223in}{2.260227in}%
\pgfsys@useobject{currentmarker}{}%
\end{pgfscope}%
\begin{pgfscope}%
\pgfsys@transformshift{10.725817in}{2.384894in}%
\pgfsys@useobject{currentmarker}{}%
\end{pgfscope}%
\begin{pgfscope}%
\pgfsys@transformshift{10.743411in}{2.368455in}%
\pgfsys@useobject{currentmarker}{}%
\end{pgfscope}%
\begin{pgfscope}%
\pgfsys@transformshift{10.761005in}{2.494470in}%
\pgfsys@useobject{currentmarker}{}%
\end{pgfscope}%
\begin{pgfscope}%
\pgfsys@transformshift{10.778599in}{2.458087in}%
\pgfsys@useobject{currentmarker}{}%
\end{pgfscope}%
\begin{pgfscope}%
\pgfsys@transformshift{10.796193in}{2.530704in}%
\pgfsys@useobject{currentmarker}{}%
\end{pgfscope}%
\begin{pgfscope}%
\pgfsys@transformshift{10.813787in}{2.428234in}%
\pgfsys@useobject{currentmarker}{}%
\end{pgfscope}%
\begin{pgfscope}%
\pgfsys@transformshift{10.831380in}{2.405057in}%
\pgfsys@useobject{currentmarker}{}%
\end{pgfscope}%
\begin{pgfscope}%
\pgfsys@transformshift{10.848974in}{2.485208in}%
\pgfsys@useobject{currentmarker}{}%
\end{pgfscope}%
\begin{pgfscope}%
\pgfsys@transformshift{10.866568in}{2.550822in}%
\pgfsys@useobject{currentmarker}{}%
\end{pgfscope}%
\begin{pgfscope}%
\pgfsys@transformshift{10.884162in}{2.616431in}%
\pgfsys@useobject{currentmarker}{}%
\end{pgfscope}%
\begin{pgfscope}%
\pgfsys@transformshift{10.901756in}{2.832843in}%
\pgfsys@useobject{currentmarker}{}%
\end{pgfscope}%
\begin{pgfscope}%
\pgfsys@transformshift{10.919350in}{2.622109in}%
\pgfsys@useobject{currentmarker}{}%
\end{pgfscope}%
\begin{pgfscope}%
\pgfsys@transformshift{10.936944in}{2.551996in}%
\pgfsys@useobject{currentmarker}{}%
\end{pgfscope}%
\begin{pgfscope}%
\pgfsys@transformshift{10.954537in}{2.636175in}%
\pgfsys@useobject{currentmarker}{}%
\end{pgfscope}%
\begin{pgfscope}%
\pgfsys@transformshift{10.972131in}{2.644912in}%
\pgfsys@useobject{currentmarker}{}%
\end{pgfscope}%
\begin{pgfscope}%
\pgfsys@transformshift{10.989725in}{2.760620in}%
\pgfsys@useobject{currentmarker}{}%
\end{pgfscope}%
\begin{pgfscope}%
\pgfsys@transformshift{11.007319in}{2.572201in}%
\pgfsys@useobject{currentmarker}{}%
\end{pgfscope}%
\begin{pgfscope}%
\pgfsys@transformshift{11.024913in}{2.751921in}%
\pgfsys@useobject{currentmarker}{}%
\end{pgfscope}%
\begin{pgfscope}%
\pgfsys@transformshift{11.042507in}{2.775821in}%
\pgfsys@useobject{currentmarker}{}%
\end{pgfscope}%
\begin{pgfscope}%
\pgfsys@transformshift{11.060101in}{2.796091in}%
\pgfsys@useobject{currentmarker}{}%
\end{pgfscope}%
\begin{pgfscope}%
\pgfsys@transformshift{11.077694in}{2.722154in}%
\pgfsys@useobject{currentmarker}{}%
\end{pgfscope}%
\begin{pgfscope}%
\pgfsys@transformshift{11.095288in}{2.767504in}%
\pgfsys@useobject{currentmarker}{}%
\end{pgfscope}%
\begin{pgfscope}%
\pgfsys@transformshift{11.112882in}{2.653279in}%
\pgfsys@useobject{currentmarker}{}%
\end{pgfscope}%
\begin{pgfscope}%
\pgfsys@transformshift{11.130476in}{2.752946in}%
\pgfsys@useobject{currentmarker}{}%
\end{pgfscope}%
\begin{pgfscope}%
\pgfsys@transformshift{11.148070in}{2.750693in}%
\pgfsys@useobject{currentmarker}{}%
\end{pgfscope}%
\begin{pgfscope}%
\pgfsys@transformshift{11.165664in}{2.849357in}%
\pgfsys@useobject{currentmarker}{}%
\end{pgfscope}%
\begin{pgfscope}%
\pgfsys@transformshift{11.183257in}{2.688050in}%
\pgfsys@useobject{currentmarker}{}%
\end{pgfscope}%
\begin{pgfscope}%
\pgfsys@transformshift{11.200851in}{2.882850in}%
\pgfsys@useobject{currentmarker}{}%
\end{pgfscope}%
\begin{pgfscope}%
\pgfsys@transformshift{11.218445in}{2.951139in}%
\pgfsys@useobject{currentmarker}{}%
\end{pgfscope}%
\begin{pgfscope}%
\pgfsys@transformshift{11.236039in}{2.581640in}%
\pgfsys@useobject{currentmarker}{}%
\end{pgfscope}%
\begin{pgfscope}%
\pgfsys@transformshift{11.253633in}{2.828734in}%
\pgfsys@useobject{currentmarker}{}%
\end{pgfscope}%
\begin{pgfscope}%
\pgfsys@transformshift{11.271227in}{2.845545in}%
\pgfsys@useobject{currentmarker}{}%
\end{pgfscope}%
\begin{pgfscope}%
\pgfsys@transformshift{11.288821in}{2.701636in}%
\pgfsys@useobject{currentmarker}{}%
\end{pgfscope}%
\begin{pgfscope}%
\pgfsys@transformshift{11.306414in}{2.715287in}%
\pgfsys@useobject{currentmarker}{}%
\end{pgfscope}%
\begin{pgfscope}%
\pgfsys@transformshift{11.324008in}{2.730634in}%
\pgfsys@useobject{currentmarker}{}%
\end{pgfscope}%
\begin{pgfscope}%
\pgfsys@transformshift{11.341602in}{2.701620in}%
\pgfsys@useobject{currentmarker}{}%
\end{pgfscope}%
\begin{pgfscope}%
\pgfsys@transformshift{11.359196in}{2.687893in}%
\pgfsys@useobject{currentmarker}{}%
\end{pgfscope}%
\begin{pgfscope}%
\pgfsys@transformshift{11.376790in}{2.535733in}%
\pgfsys@useobject{currentmarker}{}%
\end{pgfscope}%
\begin{pgfscope}%
\pgfsys@transformshift{11.394384in}{2.811411in}%
\pgfsys@useobject{currentmarker}{}%
\end{pgfscope}%
\begin{pgfscope}%
\pgfsys@transformshift{11.411978in}{2.791491in}%
\pgfsys@useobject{currentmarker}{}%
\end{pgfscope}%
\begin{pgfscope}%
\pgfsys@transformshift{11.429571in}{2.586388in}%
\pgfsys@useobject{currentmarker}{}%
\end{pgfscope}%
\begin{pgfscope}%
\pgfsys@transformshift{11.447165in}{2.508363in}%
\pgfsys@useobject{currentmarker}{}%
\end{pgfscope}%
\begin{pgfscope}%
\pgfsys@transformshift{11.464759in}{2.700439in}%
\pgfsys@useobject{currentmarker}{}%
\end{pgfscope}%
\begin{pgfscope}%
\pgfsys@transformshift{11.482353in}{2.579017in}%
\pgfsys@useobject{currentmarker}{}%
\end{pgfscope}%
\end{pgfscope}%
\begin{pgfscope}%
\pgfpathrectangle{\pgfqpoint{7.105882in}{1.973684in}}{\pgfqpoint{4.376471in}{0.978947in}} %
\pgfusepath{clip}%
\pgfsetroundcap%
\pgfsetroundjoin%
\pgfsetlinewidth{1.756562pt}%
\definecolor{currentstroke}{rgb}{0.298039,0.447059,0.690196}%
\pgfsetstrokecolor{currentstroke}%
\pgfsetdash{}{0pt}%
\pgfpathmoveto{\pgfqpoint{7.981176in}{2.242334in}}%
\pgfpathlineto{\pgfqpoint{7.998770in}{2.281612in}}%
\pgfpathlineto{\pgfqpoint{8.016364in}{2.309812in}}%
\pgfpathlineto{\pgfqpoint{8.033958in}{2.329302in}}%
\pgfpathlineto{\pgfqpoint{8.051552in}{2.342068in}}%
\pgfpathlineto{\pgfqpoint{8.069146in}{2.349759in}}%
\pgfpathlineto{\pgfqpoint{8.086740in}{2.353723in}}%
\pgfpathlineto{\pgfqpoint{8.104333in}{2.355053in}}%
\pgfpathlineto{\pgfqpoint{8.139521in}{2.353083in}}%
\pgfpathlineto{\pgfqpoint{8.245084in}{2.341488in}}%
\pgfpathlineto{\pgfqpoint{8.297866in}{2.339883in}}%
\pgfpathlineto{\pgfqpoint{8.368241in}{2.340892in}}%
\pgfpathlineto{\pgfqpoint{8.456210in}{2.341862in}}%
\pgfpathlineto{\pgfqpoint{8.526586in}{2.340105in}}%
\pgfpathlineto{\pgfqpoint{8.755306in}{2.331073in}}%
\pgfpathlineto{\pgfqpoint{8.825681in}{2.332118in}}%
\pgfpathlineto{\pgfqpoint{8.913651in}{2.336017in}}%
\pgfpathlineto{\pgfqpoint{9.071995in}{2.343717in}}%
\pgfpathlineto{\pgfqpoint{9.159965in}{2.345365in}}%
\pgfpathlineto{\pgfqpoint{9.265528in}{2.344555in}}%
\pgfpathlineto{\pgfqpoint{9.547029in}{2.340491in}}%
\pgfpathlineto{\pgfqpoint{10.285971in}{2.341272in}}%
\pgfpathlineto{\pgfqpoint{10.479503in}{2.341862in}}%
\pgfpathlineto{\pgfqpoint{10.848974in}{2.340874in}}%
\pgfpathlineto{\pgfqpoint{11.112882in}{2.342525in}}%
\pgfpathlineto{\pgfqpoint{11.411978in}{2.341986in}}%
\pgfpathlineto{\pgfqpoint{11.447165in}{2.344013in}}%
\pgfpathlineto{\pgfqpoint{11.482353in}{2.347970in}}%
\pgfpathlineto{\pgfqpoint{11.482353in}{2.347970in}}%
\pgfusepath{stroke}%
\end{pgfscope}%
\begin{pgfscope}%
\pgfsetrectcap%
\pgfsetmiterjoin%
\pgfsetlinewidth{1.003750pt}%
\definecolor{currentstroke}{rgb}{0.800000,0.800000,0.800000}%
\pgfsetstrokecolor{currentstroke}%
\pgfsetdash{}{0pt}%
\pgfpathmoveto{\pgfqpoint{7.105882in}{1.973684in}}%
\pgfpathlineto{\pgfqpoint{7.105882in}{2.952632in}}%
\pgfusepath{stroke}%
\end{pgfscope}%
\begin{pgfscope}%
\pgfsetrectcap%
\pgfsetmiterjoin%
\pgfsetlinewidth{1.003750pt}%
\definecolor{currentstroke}{rgb}{0.800000,0.800000,0.800000}%
\pgfsetstrokecolor{currentstroke}%
\pgfsetdash{}{0pt}%
\pgfpathmoveto{\pgfqpoint{11.482353in}{1.973684in}}%
\pgfpathlineto{\pgfqpoint{11.482353in}{2.952632in}}%
\pgfusepath{stroke}%
\end{pgfscope}%
\begin{pgfscope}%
\pgfsetrectcap%
\pgfsetmiterjoin%
\pgfsetlinewidth{1.003750pt}%
\definecolor{currentstroke}{rgb}{0.800000,0.800000,0.800000}%
\pgfsetstrokecolor{currentstroke}%
\pgfsetdash{}{0pt}%
\pgfpathmoveto{\pgfqpoint{7.105882in}{2.952632in}}%
\pgfpathlineto{\pgfqpoint{11.482353in}{2.952632in}}%
\pgfusepath{stroke}%
\end{pgfscope}%
\begin{pgfscope}%
\pgfsetrectcap%
\pgfsetmiterjoin%
\pgfsetlinewidth{1.003750pt}%
\definecolor{currentstroke}{rgb}{0.800000,0.800000,0.800000}%
\pgfsetstrokecolor{currentstroke}%
\pgfsetdash{}{0pt}%
\pgfpathmoveto{\pgfqpoint{7.105882in}{1.973684in}}%
\pgfpathlineto{\pgfqpoint{11.482353in}{1.973684in}}%
\pgfusepath{stroke}%
\end{pgfscope}%
\begin{pgfscope}%
\pgfsetroundcap%
\pgfsetroundjoin%
\pgfsetlinewidth{1.756562pt}%
\definecolor{currentstroke}{rgb}{0.298039,0.447059,0.690196}%
\pgfsetstrokecolor{currentstroke}%
\pgfsetdash{}{0pt}%
\pgfpathmoveto{\pgfqpoint{7.230882in}{2.567827in}}%
\pgfpathlineto{\pgfqpoint{7.508660in}{2.567827in}}%
\pgfusepath{stroke}%
\end{pgfscope}%
\begin{pgfscope}%
\definecolor{textcolor}{rgb}{0.150000,0.150000,0.150000}%
\pgfsetstrokecolor{textcolor}%
\pgfsetfillcolor{textcolor}%
\pgftext[x=7.619771in,y=2.519216in,left,base]{\color{textcolor}\sffamily\fontsize{10.000000}{12.000000}\selectfont \(\displaystyle \widetilde{\Phi}^* \theta^{\parallel}\)}%
\end{pgfscope}%
\begin{pgfscope}%
\pgfsetbuttcap%
\pgfsetroundjoin%
\definecolor{currentfill}{rgb}{1.000000,0.000000,0.000000}%
\pgfsetfillcolor{currentfill}%
\pgfsetlinewidth{2.007500pt}%
\definecolor{currentstroke}{rgb}{1.000000,0.000000,0.000000}%
\pgfsetstrokecolor{currentstroke}%
\pgfsetdash{}{0pt}%
\pgfpathmoveto{\pgfqpoint{7.338715in}{2.359209in}}%
\pgfpathlineto{\pgfqpoint{7.400828in}{2.359209in}}%
\pgfpathmoveto{\pgfqpoint{7.369771in}{2.328152in}}%
\pgfpathlineto{\pgfqpoint{7.369771in}{2.390265in}}%
\pgfusepath{stroke,fill}%
\end{pgfscope}%
\begin{pgfscope}%
\pgfsetbuttcap%
\pgfsetroundjoin%
\definecolor{currentfill}{rgb}{1.000000,0.000000,0.000000}%
\pgfsetfillcolor{currentfill}%
\pgfsetlinewidth{2.007500pt}%
\definecolor{currentstroke}{rgb}{1.000000,0.000000,0.000000}%
\pgfsetstrokecolor{currentstroke}%
\pgfsetdash{}{0pt}%
\pgfpathmoveto{\pgfqpoint{7.338715in}{2.359209in}}%
\pgfpathlineto{\pgfqpoint{7.400828in}{2.359209in}}%
\pgfpathmoveto{\pgfqpoint{7.369771in}{2.328152in}}%
\pgfpathlineto{\pgfqpoint{7.369771in}{2.390265in}}%
\pgfusepath{stroke,fill}%
\end{pgfscope}%
\begin{pgfscope}%
\pgfsetbuttcap%
\pgfsetroundjoin%
\definecolor{currentfill}{rgb}{1.000000,0.000000,0.000000}%
\pgfsetfillcolor{currentfill}%
\pgfsetlinewidth{2.007500pt}%
\definecolor{currentstroke}{rgb}{1.000000,0.000000,0.000000}%
\pgfsetstrokecolor{currentstroke}%
\pgfsetdash{}{0pt}%
\pgfpathmoveto{\pgfqpoint{7.338715in}{2.359209in}}%
\pgfpathlineto{\pgfqpoint{7.400828in}{2.359209in}}%
\pgfpathmoveto{\pgfqpoint{7.369771in}{2.328152in}}%
\pgfpathlineto{\pgfqpoint{7.369771in}{2.390265in}}%
\pgfusepath{stroke,fill}%
\end{pgfscope}%
\begin{pgfscope}%
\definecolor{textcolor}{rgb}{0.150000,0.150000,0.150000}%
\pgfsetstrokecolor{textcolor}%
\pgfsetfillcolor{textcolor}%
\pgftext[x=7.619771in,y=2.322751in,left,base]{\color{textcolor}\sffamily\fontsize{10.000000}{12.000000}\selectfont train}%
\end{pgfscope}%
\begin{pgfscope}%
\pgfsetbuttcap%
\pgfsetroundjoin%
\definecolor{currentfill}{rgb}{0.000000,0.000000,0.000000}%
\pgfsetfillcolor{currentfill}%
\pgfsetlinewidth{0.301125pt}%
\definecolor{currentstroke}{rgb}{0.000000,0.000000,0.000000}%
\pgfsetstrokecolor{currentstroke}%
\pgfsetdash{}{0pt}%
\pgfpathmoveto{\pgfqpoint{7.369771in}{2.147216in}}%
\pgfpathcurveto{\pgfqpoint{7.373889in}{2.147216in}}{\pgfqpoint{7.377839in}{2.148852in}}{\pgfqpoint{7.380751in}{2.151764in}}%
\pgfpathcurveto{\pgfqpoint{7.383663in}{2.154676in}}{\pgfqpoint{7.385299in}{2.158626in}}{\pgfqpoint{7.385299in}{2.162744in}}%
\pgfpathcurveto{\pgfqpoint{7.385299in}{2.166862in}}{\pgfqpoint{7.383663in}{2.170812in}}{\pgfqpoint{7.380751in}{2.173724in}}%
\pgfpathcurveto{\pgfqpoint{7.377839in}{2.176636in}}{\pgfqpoint{7.373889in}{2.178272in}}{\pgfqpoint{7.369771in}{2.178272in}}%
\pgfpathcurveto{\pgfqpoint{7.365653in}{2.178272in}}{\pgfqpoint{7.361703in}{2.176636in}}{\pgfqpoint{7.358791in}{2.173724in}}%
\pgfpathcurveto{\pgfqpoint{7.355879in}{2.170812in}}{\pgfqpoint{7.354243in}{2.166862in}}{\pgfqpoint{7.354243in}{2.162744in}}%
\pgfpathcurveto{\pgfqpoint{7.354243in}{2.158626in}}{\pgfqpoint{7.355879in}{2.154676in}}{\pgfqpoint{7.358791in}{2.151764in}}%
\pgfpathcurveto{\pgfqpoint{7.361703in}{2.148852in}}{\pgfqpoint{7.365653in}{2.147216in}}{\pgfqpoint{7.369771in}{2.147216in}}%
\pgfpathclose%
\pgfusepath{stroke,fill}%
\end{pgfscope}%
\begin{pgfscope}%
\pgfsetbuttcap%
\pgfsetroundjoin%
\definecolor{currentfill}{rgb}{0.000000,0.000000,0.000000}%
\pgfsetfillcolor{currentfill}%
\pgfsetlinewidth{0.301125pt}%
\definecolor{currentstroke}{rgb}{0.000000,0.000000,0.000000}%
\pgfsetstrokecolor{currentstroke}%
\pgfsetdash{}{0pt}%
\pgfpathmoveto{\pgfqpoint{7.369771in}{2.147216in}}%
\pgfpathcurveto{\pgfqpoint{7.373889in}{2.147216in}}{\pgfqpoint{7.377839in}{2.148852in}}{\pgfqpoint{7.380751in}{2.151764in}}%
\pgfpathcurveto{\pgfqpoint{7.383663in}{2.154676in}}{\pgfqpoint{7.385299in}{2.158626in}}{\pgfqpoint{7.385299in}{2.162744in}}%
\pgfpathcurveto{\pgfqpoint{7.385299in}{2.166862in}}{\pgfqpoint{7.383663in}{2.170812in}}{\pgfqpoint{7.380751in}{2.173724in}}%
\pgfpathcurveto{\pgfqpoint{7.377839in}{2.176636in}}{\pgfqpoint{7.373889in}{2.178272in}}{\pgfqpoint{7.369771in}{2.178272in}}%
\pgfpathcurveto{\pgfqpoint{7.365653in}{2.178272in}}{\pgfqpoint{7.361703in}{2.176636in}}{\pgfqpoint{7.358791in}{2.173724in}}%
\pgfpathcurveto{\pgfqpoint{7.355879in}{2.170812in}}{\pgfqpoint{7.354243in}{2.166862in}}{\pgfqpoint{7.354243in}{2.162744in}}%
\pgfpathcurveto{\pgfqpoint{7.354243in}{2.158626in}}{\pgfqpoint{7.355879in}{2.154676in}}{\pgfqpoint{7.358791in}{2.151764in}}%
\pgfpathcurveto{\pgfqpoint{7.361703in}{2.148852in}}{\pgfqpoint{7.365653in}{2.147216in}}{\pgfqpoint{7.369771in}{2.147216in}}%
\pgfpathclose%
\pgfusepath{stroke,fill}%
\end{pgfscope}%
\begin{pgfscope}%
\pgfsetbuttcap%
\pgfsetroundjoin%
\definecolor{currentfill}{rgb}{0.000000,0.000000,0.000000}%
\pgfsetfillcolor{currentfill}%
\pgfsetlinewidth{0.301125pt}%
\definecolor{currentstroke}{rgb}{0.000000,0.000000,0.000000}%
\pgfsetstrokecolor{currentstroke}%
\pgfsetdash{}{0pt}%
\pgfpathmoveto{\pgfqpoint{7.369771in}{2.147216in}}%
\pgfpathcurveto{\pgfqpoint{7.373889in}{2.147216in}}{\pgfqpoint{7.377839in}{2.148852in}}{\pgfqpoint{7.380751in}{2.151764in}}%
\pgfpathcurveto{\pgfqpoint{7.383663in}{2.154676in}}{\pgfqpoint{7.385299in}{2.158626in}}{\pgfqpoint{7.385299in}{2.162744in}}%
\pgfpathcurveto{\pgfqpoint{7.385299in}{2.166862in}}{\pgfqpoint{7.383663in}{2.170812in}}{\pgfqpoint{7.380751in}{2.173724in}}%
\pgfpathcurveto{\pgfqpoint{7.377839in}{2.176636in}}{\pgfqpoint{7.373889in}{2.178272in}}{\pgfqpoint{7.369771in}{2.178272in}}%
\pgfpathcurveto{\pgfqpoint{7.365653in}{2.178272in}}{\pgfqpoint{7.361703in}{2.176636in}}{\pgfqpoint{7.358791in}{2.173724in}}%
\pgfpathcurveto{\pgfqpoint{7.355879in}{2.170812in}}{\pgfqpoint{7.354243in}{2.166862in}}{\pgfqpoint{7.354243in}{2.162744in}}%
\pgfpathcurveto{\pgfqpoint{7.354243in}{2.158626in}}{\pgfqpoint{7.355879in}{2.154676in}}{\pgfqpoint{7.358791in}{2.151764in}}%
\pgfpathcurveto{\pgfqpoint{7.361703in}{2.148852in}}{\pgfqpoint{7.365653in}{2.147216in}}{\pgfqpoint{7.369771in}{2.147216in}}%
\pgfpathclose%
\pgfusepath{stroke,fill}%
\end{pgfscope}%
\begin{pgfscope}%
\definecolor{textcolor}{rgb}{0.150000,0.150000,0.150000}%
\pgfsetstrokecolor{textcolor}%
\pgfsetfillcolor{textcolor}%
\pgftext[x=7.619771in,y=2.126285in,left,base]{\color{textcolor}\sffamily\fontsize{10.000000}{12.000000}\selectfont test}%
\end{pgfscope}%
\begin{pgfscope}%
\pgfsetbuttcap%
\pgfsetmiterjoin%
\definecolor{currentfill}{rgb}{1.000000,1.000000,1.000000}%
\pgfsetfillcolor{currentfill}%
\pgfsetlinewidth{0.000000pt}%
\definecolor{currentstroke}{rgb}{0.000000,0.000000,0.000000}%
\pgfsetstrokecolor{currentstroke}%
\pgfsetstrokeopacity{0.000000}%
\pgfsetdash{}{0pt}%
\pgfpathmoveto{\pgfqpoint{12.211765in}{1.973684in}}%
\pgfpathlineto{\pgfqpoint{14.400000in}{1.973684in}}%
\pgfpathlineto{\pgfqpoint{14.400000in}{2.952632in}}%
\pgfpathlineto{\pgfqpoint{12.211765in}{2.952632in}}%
\pgfpathclose%
\pgfusepath{fill}%
\end{pgfscope}%
\begin{pgfscope}%
\pgfpathrectangle{\pgfqpoint{12.211765in}{1.973684in}}{\pgfqpoint{2.188235in}{0.978947in}} %
\pgfusepath{clip}%
\pgfsetroundcap%
\pgfsetroundjoin%
\pgfsetlinewidth{1.003750pt}%
\definecolor{currentstroke}{rgb}{0.800000,0.800000,0.800000}%
\pgfsetstrokecolor{currentstroke}%
\pgfsetdash{}{0pt}%
\pgfpathmoveto{\pgfqpoint{12.211765in}{1.973684in}}%
\pgfpathlineto{\pgfqpoint{12.211765in}{2.952632in}}%
\pgfusepath{stroke}%
\end{pgfscope}%
\begin{pgfscope}%
\pgfpathrectangle{\pgfqpoint{12.211765in}{1.973684in}}{\pgfqpoint{2.188235in}{0.978947in}} %
\pgfusepath{clip}%
\pgfsetroundcap%
\pgfsetroundjoin%
\pgfsetlinewidth{1.003750pt}%
\definecolor{currentstroke}{rgb}{0.800000,0.800000,0.800000}%
\pgfsetstrokecolor{currentstroke}%
\pgfsetdash{}{0pt}%
\pgfpathmoveto{\pgfqpoint{12.485294in}{1.973684in}}%
\pgfpathlineto{\pgfqpoint{12.485294in}{2.952632in}}%
\pgfusepath{stroke}%
\end{pgfscope}%
\begin{pgfscope}%
\pgfpathrectangle{\pgfqpoint{12.211765in}{1.973684in}}{\pgfqpoint{2.188235in}{0.978947in}} %
\pgfusepath{clip}%
\pgfsetroundcap%
\pgfsetroundjoin%
\pgfsetlinewidth{1.003750pt}%
\definecolor{currentstroke}{rgb}{0.800000,0.800000,0.800000}%
\pgfsetstrokecolor{currentstroke}%
\pgfsetdash{}{0pt}%
\pgfpathmoveto{\pgfqpoint{12.758824in}{1.973684in}}%
\pgfpathlineto{\pgfqpoint{12.758824in}{2.952632in}}%
\pgfusepath{stroke}%
\end{pgfscope}%
\begin{pgfscope}%
\pgfpathrectangle{\pgfqpoint{12.211765in}{1.973684in}}{\pgfqpoint{2.188235in}{0.978947in}} %
\pgfusepath{clip}%
\pgfsetroundcap%
\pgfsetroundjoin%
\pgfsetlinewidth{1.003750pt}%
\definecolor{currentstroke}{rgb}{0.800000,0.800000,0.800000}%
\pgfsetstrokecolor{currentstroke}%
\pgfsetdash{}{0pt}%
\pgfpathmoveto{\pgfqpoint{13.032353in}{1.973684in}}%
\pgfpathlineto{\pgfqpoint{13.032353in}{2.952632in}}%
\pgfusepath{stroke}%
\end{pgfscope}%
\begin{pgfscope}%
\pgfpathrectangle{\pgfqpoint{12.211765in}{1.973684in}}{\pgfqpoint{2.188235in}{0.978947in}} %
\pgfusepath{clip}%
\pgfsetroundcap%
\pgfsetroundjoin%
\pgfsetlinewidth{1.003750pt}%
\definecolor{currentstroke}{rgb}{0.800000,0.800000,0.800000}%
\pgfsetstrokecolor{currentstroke}%
\pgfsetdash{}{0pt}%
\pgfpathmoveto{\pgfqpoint{13.305882in}{1.973684in}}%
\pgfpathlineto{\pgfqpoint{13.305882in}{2.952632in}}%
\pgfusepath{stroke}%
\end{pgfscope}%
\begin{pgfscope}%
\pgfpathrectangle{\pgfqpoint{12.211765in}{1.973684in}}{\pgfqpoint{2.188235in}{0.978947in}} %
\pgfusepath{clip}%
\pgfsetroundcap%
\pgfsetroundjoin%
\pgfsetlinewidth{1.003750pt}%
\definecolor{currentstroke}{rgb}{0.800000,0.800000,0.800000}%
\pgfsetstrokecolor{currentstroke}%
\pgfsetdash{}{0pt}%
\pgfpathmoveto{\pgfqpoint{13.579412in}{1.973684in}}%
\pgfpathlineto{\pgfqpoint{13.579412in}{2.952632in}}%
\pgfusepath{stroke}%
\end{pgfscope}%
\begin{pgfscope}%
\pgfpathrectangle{\pgfqpoint{12.211765in}{1.973684in}}{\pgfqpoint{2.188235in}{0.978947in}} %
\pgfusepath{clip}%
\pgfsetroundcap%
\pgfsetroundjoin%
\pgfsetlinewidth{1.003750pt}%
\definecolor{currentstroke}{rgb}{0.800000,0.800000,0.800000}%
\pgfsetstrokecolor{currentstroke}%
\pgfsetdash{}{0pt}%
\pgfpathmoveto{\pgfqpoint{13.852941in}{1.973684in}}%
\pgfpathlineto{\pgfqpoint{13.852941in}{2.952632in}}%
\pgfusepath{stroke}%
\end{pgfscope}%
\begin{pgfscope}%
\pgfpathrectangle{\pgfqpoint{12.211765in}{1.973684in}}{\pgfqpoint{2.188235in}{0.978947in}} %
\pgfusepath{clip}%
\pgfsetroundcap%
\pgfsetroundjoin%
\pgfsetlinewidth{1.003750pt}%
\definecolor{currentstroke}{rgb}{0.800000,0.800000,0.800000}%
\pgfsetstrokecolor{currentstroke}%
\pgfsetdash{}{0pt}%
\pgfpathmoveto{\pgfqpoint{14.126471in}{1.973684in}}%
\pgfpathlineto{\pgfqpoint{14.126471in}{2.952632in}}%
\pgfusepath{stroke}%
\end{pgfscope}%
\begin{pgfscope}%
\pgfpathrectangle{\pgfqpoint{12.211765in}{1.973684in}}{\pgfqpoint{2.188235in}{0.978947in}} %
\pgfusepath{clip}%
\pgfsetroundcap%
\pgfsetroundjoin%
\pgfsetlinewidth{1.003750pt}%
\definecolor{currentstroke}{rgb}{0.800000,0.800000,0.800000}%
\pgfsetstrokecolor{currentstroke}%
\pgfsetdash{}{0pt}%
\pgfpathmoveto{\pgfqpoint{14.400000in}{1.973684in}}%
\pgfpathlineto{\pgfqpoint{14.400000in}{2.952632in}}%
\pgfusepath{stroke}%
\end{pgfscope}%
\begin{pgfscope}%
\pgfpathrectangle{\pgfqpoint{12.211765in}{1.973684in}}{\pgfqpoint{2.188235in}{0.978947in}} %
\pgfusepath{clip}%
\pgfsetroundcap%
\pgfsetroundjoin%
\pgfsetlinewidth{1.003750pt}%
\definecolor{currentstroke}{rgb}{0.800000,0.800000,0.800000}%
\pgfsetstrokecolor{currentstroke}%
\pgfsetdash{}{0pt}%
\pgfpathmoveto{\pgfqpoint{12.211765in}{1.973684in}}%
\pgfpathlineto{\pgfqpoint{14.400000in}{1.973684in}}%
\pgfusepath{stroke}%
\end{pgfscope}%
\begin{pgfscope}%
\definecolor{textcolor}{rgb}{0.150000,0.150000,0.150000}%
\pgfsetstrokecolor{textcolor}%
\pgfsetfillcolor{textcolor}%
\pgftext[x=12.114542in,y=1.973684in,right,]{\color{textcolor}\sffamily\fontsize{10.000000}{12.000000}\selectfont \(\displaystyle 0\)}%
\end{pgfscope}%
\begin{pgfscope}%
\pgfpathrectangle{\pgfqpoint{12.211765in}{1.973684in}}{\pgfqpoint{2.188235in}{0.978947in}} %
\pgfusepath{clip}%
\pgfsetroundcap%
\pgfsetroundjoin%
\pgfsetlinewidth{1.003750pt}%
\definecolor{currentstroke}{rgb}{0.800000,0.800000,0.800000}%
\pgfsetstrokecolor{currentstroke}%
\pgfsetdash{}{0pt}%
\pgfpathmoveto{\pgfqpoint{12.211765in}{2.218421in}}%
\pgfpathlineto{\pgfqpoint{14.400000in}{2.218421in}}%
\pgfusepath{stroke}%
\end{pgfscope}%
\begin{pgfscope}%
\definecolor{textcolor}{rgb}{0.150000,0.150000,0.150000}%
\pgfsetstrokecolor{textcolor}%
\pgfsetfillcolor{textcolor}%
\pgftext[x=12.114542in,y=2.218421in,right,]{\color{textcolor}\sffamily\fontsize{10.000000}{12.000000}\selectfont \(\displaystyle 50\)}%
\end{pgfscope}%
\begin{pgfscope}%
\pgfpathrectangle{\pgfqpoint{12.211765in}{1.973684in}}{\pgfqpoint{2.188235in}{0.978947in}} %
\pgfusepath{clip}%
\pgfsetroundcap%
\pgfsetroundjoin%
\pgfsetlinewidth{1.003750pt}%
\definecolor{currentstroke}{rgb}{0.800000,0.800000,0.800000}%
\pgfsetstrokecolor{currentstroke}%
\pgfsetdash{}{0pt}%
\pgfpathmoveto{\pgfqpoint{12.211765in}{2.463158in}}%
\pgfpathlineto{\pgfqpoint{14.400000in}{2.463158in}}%
\pgfusepath{stroke}%
\end{pgfscope}%
\begin{pgfscope}%
\definecolor{textcolor}{rgb}{0.150000,0.150000,0.150000}%
\pgfsetstrokecolor{textcolor}%
\pgfsetfillcolor{textcolor}%
\pgftext[x=12.114542in,y=2.463158in,right,]{\color{textcolor}\sffamily\fontsize{10.000000}{12.000000}\selectfont \(\displaystyle 100\)}%
\end{pgfscope}%
\begin{pgfscope}%
\pgfpathrectangle{\pgfqpoint{12.211765in}{1.973684in}}{\pgfqpoint{2.188235in}{0.978947in}} %
\pgfusepath{clip}%
\pgfsetroundcap%
\pgfsetroundjoin%
\pgfsetlinewidth{1.003750pt}%
\definecolor{currentstroke}{rgb}{0.800000,0.800000,0.800000}%
\pgfsetstrokecolor{currentstroke}%
\pgfsetdash{}{0pt}%
\pgfpathmoveto{\pgfqpoint{12.211765in}{2.707895in}}%
\pgfpathlineto{\pgfqpoint{14.400000in}{2.707895in}}%
\pgfusepath{stroke}%
\end{pgfscope}%
\begin{pgfscope}%
\definecolor{textcolor}{rgb}{0.150000,0.150000,0.150000}%
\pgfsetstrokecolor{textcolor}%
\pgfsetfillcolor{textcolor}%
\pgftext[x=12.114542in,y=2.707895in,right,]{\color{textcolor}\sffamily\fontsize{10.000000}{12.000000}\selectfont \(\displaystyle 150\)}%
\end{pgfscope}%
\begin{pgfscope}%
\pgfpathrectangle{\pgfqpoint{12.211765in}{1.973684in}}{\pgfqpoint{2.188235in}{0.978947in}} %
\pgfusepath{clip}%
\pgfsetroundcap%
\pgfsetroundjoin%
\pgfsetlinewidth{1.003750pt}%
\definecolor{currentstroke}{rgb}{0.800000,0.800000,0.800000}%
\pgfsetstrokecolor{currentstroke}%
\pgfsetdash{}{0pt}%
\pgfpathmoveto{\pgfqpoint{12.211765in}{2.952632in}}%
\pgfpathlineto{\pgfqpoint{14.400000in}{2.952632in}}%
\pgfusepath{stroke}%
\end{pgfscope}%
\begin{pgfscope}%
\definecolor{textcolor}{rgb}{0.150000,0.150000,0.150000}%
\pgfsetstrokecolor{textcolor}%
\pgfsetfillcolor{textcolor}%
\pgftext[x=12.114542in,y=2.952632in,right,]{\color{textcolor}\sffamily\fontsize{10.000000}{12.000000}\selectfont \(\displaystyle 200\)}%
\end{pgfscope}%
\begin{pgfscope}%
\definecolor{textcolor}{rgb}{0.150000,0.150000,0.150000}%
\pgfsetstrokecolor{textcolor}%
\pgfsetfillcolor{textcolor}%
\pgftext[x=11.836764in,y=2.463158in,,bottom,rotate=90.000000]{\color{textcolor}\sffamily\fontsize{11.000000}{13.200000}\selectfont \(\displaystyle \theta^{\parallel}_j\)}%
\end{pgfscope}%
\begin{pgfscope}%
\pgfpathrectangle{\pgfqpoint{12.211765in}{1.973684in}}{\pgfqpoint{2.188235in}{0.978947in}} %
\pgfusepath{clip}%
\pgfsetroundcap%
\pgfsetroundjoin%
\pgfsetlinewidth{1.756562pt}%
\definecolor{currentstroke}{rgb}{0.298039,0.447059,0.690196}%
\pgfsetstrokecolor{currentstroke}%
\pgfsetdash{}{0pt}%
\pgfpathmoveto{\pgfqpoint{13.254504in}{1.973684in}}%
\pgfpathlineto{\pgfqpoint{13.189266in}{1.978579in}}%
\pgfpathlineto{\pgfqpoint{13.312350in}{1.983474in}}%
\pgfpathlineto{\pgfqpoint{13.313084in}{1.988368in}}%
\pgfpathlineto{\pgfqpoint{13.260418in}{1.993263in}}%
\pgfpathlineto{\pgfqpoint{13.318237in}{1.998158in}}%
\pgfpathlineto{\pgfqpoint{13.308900in}{2.003053in}}%
\pgfpathlineto{\pgfqpoint{13.295380in}{2.007947in}}%
\pgfpathlineto{\pgfqpoint{13.301259in}{2.012842in}}%
\pgfpathlineto{\pgfqpoint{13.324132in}{2.017737in}}%
\pgfpathlineto{\pgfqpoint{13.308200in}{2.022632in}}%
\pgfpathlineto{\pgfqpoint{13.240112in}{2.027526in}}%
\pgfpathlineto{\pgfqpoint{13.294799in}{2.032421in}}%
\pgfpathlineto{\pgfqpoint{13.294049in}{2.037316in}}%
\pgfpathlineto{\pgfqpoint{13.270829in}{2.042211in}}%
\pgfpathlineto{\pgfqpoint{13.179538in}{2.047105in}}%
\pgfpathlineto{\pgfqpoint{13.312272in}{2.052000in}}%
\pgfpathlineto{\pgfqpoint{13.403988in}{2.056895in}}%
\pgfpathlineto{\pgfqpoint{13.399700in}{2.061789in}}%
\pgfpathlineto{\pgfqpoint{13.311738in}{2.066684in}}%
\pgfpathlineto{\pgfqpoint{13.303752in}{2.071579in}}%
\pgfpathlineto{\pgfqpoint{13.306042in}{2.076474in}}%
\pgfpathlineto{\pgfqpoint{13.245475in}{2.086263in}}%
\pgfpathlineto{\pgfqpoint{13.321097in}{2.091158in}}%
\pgfpathlineto{\pgfqpoint{13.272993in}{2.096053in}}%
\pgfpathlineto{\pgfqpoint{13.331668in}{2.100947in}}%
\pgfpathlineto{\pgfqpoint{13.272623in}{2.105842in}}%
\pgfpathlineto{\pgfqpoint{13.296570in}{2.110737in}}%
\pgfpathlineto{\pgfqpoint{13.335995in}{2.115632in}}%
\pgfpathlineto{\pgfqpoint{13.342260in}{2.120526in}}%
\pgfpathlineto{\pgfqpoint{13.326158in}{2.125421in}}%
\pgfpathlineto{\pgfqpoint{13.438286in}{2.130316in}}%
\pgfpathlineto{\pgfqpoint{13.387100in}{2.135211in}}%
\pgfpathlineto{\pgfqpoint{13.294939in}{2.140105in}}%
\pgfpathlineto{\pgfqpoint{13.326103in}{2.145000in}}%
\pgfpathlineto{\pgfqpoint{13.283532in}{2.149895in}}%
\pgfpathlineto{\pgfqpoint{13.298797in}{2.154789in}}%
\pgfpathlineto{\pgfqpoint{13.368892in}{2.159684in}}%
\pgfpathlineto{\pgfqpoint{13.268380in}{2.164579in}}%
\pgfpathlineto{\pgfqpoint{13.330643in}{2.169474in}}%
\pgfpathlineto{\pgfqpoint{13.322829in}{2.174368in}}%
\pgfpathlineto{\pgfqpoint{13.420742in}{2.179263in}}%
\pgfpathlineto{\pgfqpoint{13.286721in}{2.184158in}}%
\pgfpathlineto{\pgfqpoint{13.288931in}{2.189053in}}%
\pgfpathlineto{\pgfqpoint{13.267586in}{2.193947in}}%
\pgfpathlineto{\pgfqpoint{13.304105in}{2.198842in}}%
\pgfpathlineto{\pgfqpoint{13.301963in}{2.203737in}}%
\pgfpathlineto{\pgfqpoint{13.313267in}{2.208632in}}%
\pgfpathlineto{\pgfqpoint{13.268193in}{2.213526in}}%
\pgfpathlineto{\pgfqpoint{13.334556in}{2.218421in}}%
\pgfpathlineto{\pgfqpoint{13.244578in}{2.223316in}}%
\pgfpathlineto{\pgfqpoint{13.328795in}{2.228211in}}%
\pgfpathlineto{\pgfqpoint{13.304382in}{2.233105in}}%
\pgfpathlineto{\pgfqpoint{13.324976in}{2.238000in}}%
\pgfpathlineto{\pgfqpoint{13.324958in}{2.242895in}}%
\pgfpathlineto{\pgfqpoint{13.303177in}{2.247789in}}%
\pgfpathlineto{\pgfqpoint{13.226056in}{2.252684in}}%
\pgfpathlineto{\pgfqpoint{13.299865in}{2.257579in}}%
\pgfpathlineto{\pgfqpoint{13.308175in}{2.262474in}}%
\pgfpathlineto{\pgfqpoint{13.278465in}{2.267368in}}%
\pgfpathlineto{\pgfqpoint{13.307135in}{2.272263in}}%
\pgfpathlineto{\pgfqpoint{13.346703in}{2.277158in}}%
\pgfpathlineto{\pgfqpoint{13.236512in}{2.282053in}}%
\pgfpathlineto{\pgfqpoint{13.311820in}{2.286947in}}%
\pgfpathlineto{\pgfqpoint{13.313234in}{2.291842in}}%
\pgfpathlineto{\pgfqpoint{13.346895in}{2.296737in}}%
\pgfpathlineto{\pgfqpoint{13.365053in}{2.301632in}}%
\pgfpathlineto{\pgfqpoint{13.273398in}{2.311421in}}%
\pgfpathlineto{\pgfqpoint{13.321052in}{2.316316in}}%
\pgfpathlineto{\pgfqpoint{13.282483in}{2.321211in}}%
\pgfpathlineto{\pgfqpoint{13.275354in}{2.326105in}}%
\pgfpathlineto{\pgfqpoint{13.288075in}{2.331000in}}%
\pgfpathlineto{\pgfqpoint{13.273938in}{2.335895in}}%
\pgfpathlineto{\pgfqpoint{13.300924in}{2.340789in}}%
\pgfpathlineto{\pgfqpoint{13.300143in}{2.345684in}}%
\pgfpathlineto{\pgfqpoint{13.301660in}{2.350579in}}%
\pgfpathlineto{\pgfqpoint{13.317251in}{2.355474in}}%
\pgfpathlineto{\pgfqpoint{13.289428in}{2.360368in}}%
\pgfpathlineto{\pgfqpoint{13.310841in}{2.365263in}}%
\pgfpathlineto{\pgfqpoint{13.308695in}{2.370158in}}%
\pgfpathlineto{\pgfqpoint{13.293611in}{2.375053in}}%
\pgfpathlineto{\pgfqpoint{13.330902in}{2.379947in}}%
\pgfpathlineto{\pgfqpoint{13.321641in}{2.384842in}}%
\pgfpathlineto{\pgfqpoint{13.259575in}{2.389737in}}%
\pgfpathlineto{\pgfqpoint{13.316445in}{2.394632in}}%
\pgfpathlineto{\pgfqpoint{13.299725in}{2.399526in}}%
\pgfpathlineto{\pgfqpoint{13.289737in}{2.404421in}}%
\pgfpathlineto{\pgfqpoint{13.358679in}{2.409316in}}%
\pgfpathlineto{\pgfqpoint{13.292349in}{2.414211in}}%
\pgfpathlineto{\pgfqpoint{13.327800in}{2.419105in}}%
\pgfpathlineto{\pgfqpoint{13.256841in}{2.424000in}}%
\pgfpathlineto{\pgfqpoint{13.309998in}{2.428895in}}%
\pgfpathlineto{\pgfqpoint{13.322215in}{2.433789in}}%
\pgfpathlineto{\pgfqpoint{13.376373in}{2.438684in}}%
\pgfpathlineto{\pgfqpoint{13.312457in}{2.443579in}}%
\pgfpathlineto{\pgfqpoint{13.328567in}{2.448474in}}%
\pgfpathlineto{\pgfqpoint{13.324321in}{2.453368in}}%
\pgfpathlineto{\pgfqpoint{13.307728in}{2.458263in}}%
\pgfpathlineto{\pgfqpoint{13.319797in}{2.463158in}}%
\pgfpathlineto{\pgfqpoint{13.297347in}{2.468053in}}%
\pgfpathlineto{\pgfqpoint{13.256251in}{2.472947in}}%
\pgfpathlineto{\pgfqpoint{13.301587in}{2.477842in}}%
\pgfpathlineto{\pgfqpoint{13.314788in}{2.482737in}}%
\pgfpathlineto{\pgfqpoint{13.310430in}{2.487632in}}%
\pgfpathlineto{\pgfqpoint{13.326177in}{2.492526in}}%
\pgfpathlineto{\pgfqpoint{13.307544in}{2.497421in}}%
\pgfpathlineto{\pgfqpoint{13.282440in}{2.502316in}}%
\pgfpathlineto{\pgfqpoint{13.311880in}{2.512105in}}%
\pgfpathlineto{\pgfqpoint{13.293385in}{2.521895in}}%
\pgfpathlineto{\pgfqpoint{13.310972in}{2.526789in}}%
\pgfpathlineto{\pgfqpoint{13.258616in}{2.531684in}}%
\pgfpathlineto{\pgfqpoint{13.297900in}{2.536579in}}%
\pgfpathlineto{\pgfqpoint{13.310701in}{2.546368in}}%
\pgfpathlineto{\pgfqpoint{13.313635in}{2.556158in}}%
\pgfpathlineto{\pgfqpoint{13.325844in}{2.561053in}}%
\pgfpathlineto{\pgfqpoint{13.340612in}{2.570842in}}%
\pgfpathlineto{\pgfqpoint{13.284579in}{2.575737in}}%
\pgfpathlineto{\pgfqpoint{13.293754in}{2.580632in}}%
\pgfpathlineto{\pgfqpoint{13.300288in}{2.585526in}}%
\pgfpathlineto{\pgfqpoint{13.268401in}{2.590421in}}%
\pgfpathlineto{\pgfqpoint{13.327032in}{2.595316in}}%
\pgfpathlineto{\pgfqpoint{13.300683in}{2.600211in}}%
\pgfpathlineto{\pgfqpoint{13.315683in}{2.605105in}}%
\pgfpathlineto{\pgfqpoint{13.354184in}{2.610000in}}%
\pgfpathlineto{\pgfqpoint{13.288173in}{2.614895in}}%
\pgfpathlineto{\pgfqpoint{13.321392in}{2.624684in}}%
\pgfpathlineto{\pgfqpoint{13.300307in}{2.629579in}}%
\pgfpathlineto{\pgfqpoint{13.334820in}{2.634474in}}%
\pgfpathlineto{\pgfqpoint{13.293272in}{2.639368in}}%
\pgfpathlineto{\pgfqpoint{13.326724in}{2.649158in}}%
\pgfpathlineto{\pgfqpoint{13.280095in}{2.654053in}}%
\pgfpathlineto{\pgfqpoint{13.268983in}{2.658947in}}%
\pgfpathlineto{\pgfqpoint{13.290118in}{2.663842in}}%
\pgfpathlineto{\pgfqpoint{13.298482in}{2.668737in}}%
\pgfpathlineto{\pgfqpoint{13.288537in}{2.673632in}}%
\pgfpathlineto{\pgfqpoint{13.313720in}{2.678526in}}%
\pgfpathlineto{\pgfqpoint{13.321468in}{2.683421in}}%
\pgfpathlineto{\pgfqpoint{13.315744in}{2.688316in}}%
\pgfpathlineto{\pgfqpoint{13.322744in}{2.693211in}}%
\pgfpathlineto{\pgfqpoint{13.286898in}{2.703000in}}%
\pgfpathlineto{\pgfqpoint{13.299970in}{2.707895in}}%
\pgfpathlineto{\pgfqpoint{13.297043in}{2.712789in}}%
\pgfpathlineto{\pgfqpoint{13.290688in}{2.717684in}}%
\pgfpathlineto{\pgfqpoint{13.314533in}{2.722579in}}%
\pgfpathlineto{\pgfqpoint{13.309145in}{2.727474in}}%
\pgfpathlineto{\pgfqpoint{13.290034in}{2.732368in}}%
\pgfpathlineto{\pgfqpoint{13.289893in}{2.737263in}}%
\pgfpathlineto{\pgfqpoint{13.310068in}{2.742158in}}%
\pgfpathlineto{\pgfqpoint{13.296589in}{2.747053in}}%
\pgfpathlineto{\pgfqpoint{13.320897in}{2.751947in}}%
\pgfpathlineto{\pgfqpoint{13.297609in}{2.756842in}}%
\pgfpathlineto{\pgfqpoint{13.310892in}{2.766632in}}%
\pgfpathlineto{\pgfqpoint{13.304803in}{2.776421in}}%
\pgfpathlineto{\pgfqpoint{13.296531in}{2.781316in}}%
\pgfpathlineto{\pgfqpoint{13.303112in}{2.786211in}}%
\pgfpathlineto{\pgfqpoint{13.304446in}{2.791105in}}%
\pgfpathlineto{\pgfqpoint{13.301414in}{2.796000in}}%
\pgfpathlineto{\pgfqpoint{13.295928in}{2.800895in}}%
\pgfpathlineto{\pgfqpoint{13.313364in}{2.805789in}}%
\pgfpathlineto{\pgfqpoint{13.281741in}{2.810684in}}%
\pgfpathlineto{\pgfqpoint{13.314561in}{2.815579in}}%
\pgfpathlineto{\pgfqpoint{13.292118in}{2.820474in}}%
\pgfpathlineto{\pgfqpoint{13.295705in}{2.825368in}}%
\pgfpathlineto{\pgfqpoint{13.294487in}{2.830263in}}%
\pgfpathlineto{\pgfqpoint{13.308584in}{2.835158in}}%
\pgfpathlineto{\pgfqpoint{13.297720in}{2.840053in}}%
\pgfpathlineto{\pgfqpoint{13.292433in}{2.844947in}}%
\pgfpathlineto{\pgfqpoint{13.297799in}{2.849842in}}%
\pgfpathlineto{\pgfqpoint{13.311125in}{2.854737in}}%
\pgfpathlineto{\pgfqpoint{13.293557in}{2.859632in}}%
\pgfpathlineto{\pgfqpoint{13.287877in}{2.864526in}}%
\pgfpathlineto{\pgfqpoint{13.301068in}{2.869421in}}%
\pgfpathlineto{\pgfqpoint{13.329634in}{2.874316in}}%
\pgfpathlineto{\pgfqpoint{13.315084in}{2.879211in}}%
\pgfpathlineto{\pgfqpoint{13.314210in}{2.884105in}}%
\pgfpathlineto{\pgfqpoint{13.283960in}{2.889000in}}%
\pgfpathlineto{\pgfqpoint{13.323897in}{2.893895in}}%
\pgfpathlineto{\pgfqpoint{13.293169in}{2.898789in}}%
\pgfpathlineto{\pgfqpoint{13.316325in}{2.903684in}}%
\pgfpathlineto{\pgfqpoint{13.304179in}{2.908579in}}%
\pgfpathlineto{\pgfqpoint{13.305212in}{2.913474in}}%
\pgfpathlineto{\pgfqpoint{13.281762in}{2.918368in}}%
\pgfpathlineto{\pgfqpoint{13.320330in}{2.923263in}}%
\pgfpathlineto{\pgfqpoint{13.314057in}{2.928158in}}%
\pgfpathlineto{\pgfqpoint{13.338443in}{2.933053in}}%
\pgfpathlineto{\pgfqpoint{13.318058in}{2.937947in}}%
\pgfpathlineto{\pgfqpoint{13.321239in}{2.942842in}}%
\pgfpathlineto{\pgfqpoint{13.316399in}{2.947737in}}%
\pgfpathlineto{\pgfqpoint{13.316399in}{2.947737in}}%
\pgfusepath{stroke}%
\end{pgfscope}%
\begin{pgfscope}%
\pgfsetrectcap%
\pgfsetmiterjoin%
\pgfsetlinewidth{1.003750pt}%
\definecolor{currentstroke}{rgb}{0.800000,0.800000,0.800000}%
\pgfsetstrokecolor{currentstroke}%
\pgfsetdash{}{0pt}%
\pgfpathmoveto{\pgfqpoint{12.211765in}{1.973684in}}%
\pgfpathlineto{\pgfqpoint{12.211765in}{2.952632in}}%
\pgfusepath{stroke}%
\end{pgfscope}%
\begin{pgfscope}%
\pgfsetrectcap%
\pgfsetmiterjoin%
\pgfsetlinewidth{1.003750pt}%
\definecolor{currentstroke}{rgb}{0.800000,0.800000,0.800000}%
\pgfsetstrokecolor{currentstroke}%
\pgfsetdash{}{0pt}%
\pgfpathmoveto{\pgfqpoint{14.400000in}{1.973684in}}%
\pgfpathlineto{\pgfqpoint{14.400000in}{2.952632in}}%
\pgfusepath{stroke}%
\end{pgfscope}%
\begin{pgfscope}%
\pgfsetrectcap%
\pgfsetmiterjoin%
\pgfsetlinewidth{1.003750pt}%
\definecolor{currentstroke}{rgb}{0.800000,0.800000,0.800000}%
\pgfsetstrokecolor{currentstroke}%
\pgfsetdash{}{0pt}%
\pgfpathmoveto{\pgfqpoint{12.211765in}{2.952632in}}%
\pgfpathlineto{\pgfqpoint{14.400000in}{2.952632in}}%
\pgfusepath{stroke}%
\end{pgfscope}%
\begin{pgfscope}%
\pgfsetrectcap%
\pgfsetmiterjoin%
\pgfsetlinewidth{1.003750pt}%
\definecolor{currentstroke}{rgb}{0.800000,0.800000,0.800000}%
\pgfsetstrokecolor{currentstroke}%
\pgfsetdash{}{0pt}%
\pgfpathmoveto{\pgfqpoint{12.211765in}{1.973684in}}%
\pgfpathlineto{\pgfqpoint{14.400000in}{1.973684in}}%
\pgfusepath{stroke}%
\end{pgfscope}%
\begin{pgfscope}%
\pgfsetbuttcap%
\pgfsetmiterjoin%
\definecolor{currentfill}{rgb}{1.000000,1.000000,1.000000}%
\pgfsetfillcolor{currentfill}%
\pgfsetlinewidth{0.000000pt}%
\definecolor{currentstroke}{rgb}{0.000000,0.000000,0.000000}%
\pgfsetstrokecolor{currentstroke}%
\pgfsetstrokeopacity{0.000000}%
\pgfsetdash{}{0pt}%
\pgfpathmoveto{\pgfqpoint{2.000000in}{0.750000in}}%
\pgfpathlineto{\pgfqpoint{6.376471in}{0.750000in}}%
\pgfpathlineto{\pgfqpoint{6.376471in}{1.728947in}}%
\pgfpathlineto{\pgfqpoint{2.000000in}{1.728947in}}%
\pgfpathclose%
\pgfusepath{fill}%
\end{pgfscope}%
\begin{pgfscope}%
\pgfpathrectangle{\pgfqpoint{2.000000in}{0.750000in}}{\pgfqpoint{4.376471in}{0.978947in}} %
\pgfusepath{clip}%
\pgfsetroundcap%
\pgfsetroundjoin%
\pgfsetlinewidth{1.003750pt}%
\definecolor{currentstroke}{rgb}{0.800000,0.800000,0.800000}%
\pgfsetstrokecolor{currentstroke}%
\pgfsetdash{}{0pt}%
\pgfpathmoveto{\pgfqpoint{2.000000in}{0.750000in}}%
\pgfpathlineto{\pgfqpoint{2.000000in}{1.728947in}}%
\pgfusepath{stroke}%
\end{pgfscope}%
\begin{pgfscope}%
\definecolor{textcolor}{rgb}{0.150000,0.150000,0.150000}%
\pgfsetstrokecolor{textcolor}%
\pgfsetfillcolor{textcolor}%
\pgftext[x=2.000000in,y=0.652778in,,top]{\color{textcolor}\sffamily\fontsize{10.000000}{12.000000}\selectfont \(\displaystyle -1.5\)}%
\end{pgfscope}%
\begin{pgfscope}%
\pgfpathrectangle{\pgfqpoint{2.000000in}{0.750000in}}{\pgfqpoint{4.376471in}{0.978947in}} %
\pgfusepath{clip}%
\pgfsetroundcap%
\pgfsetroundjoin%
\pgfsetlinewidth{1.003750pt}%
\definecolor{currentstroke}{rgb}{0.800000,0.800000,0.800000}%
\pgfsetstrokecolor{currentstroke}%
\pgfsetdash{}{0pt}%
\pgfpathmoveto{\pgfqpoint{2.875294in}{0.750000in}}%
\pgfpathlineto{\pgfqpoint{2.875294in}{1.728947in}}%
\pgfusepath{stroke}%
\end{pgfscope}%
\begin{pgfscope}%
\definecolor{textcolor}{rgb}{0.150000,0.150000,0.150000}%
\pgfsetstrokecolor{textcolor}%
\pgfsetfillcolor{textcolor}%
\pgftext[x=2.875294in,y=0.652778in,,top]{\color{textcolor}\sffamily\fontsize{10.000000}{12.000000}\selectfont \(\displaystyle -1.0\)}%
\end{pgfscope}%
\begin{pgfscope}%
\pgfpathrectangle{\pgfqpoint{2.000000in}{0.750000in}}{\pgfqpoint{4.376471in}{0.978947in}} %
\pgfusepath{clip}%
\pgfsetroundcap%
\pgfsetroundjoin%
\pgfsetlinewidth{1.003750pt}%
\definecolor{currentstroke}{rgb}{0.800000,0.800000,0.800000}%
\pgfsetstrokecolor{currentstroke}%
\pgfsetdash{}{0pt}%
\pgfpathmoveto{\pgfqpoint{3.750588in}{0.750000in}}%
\pgfpathlineto{\pgfqpoint{3.750588in}{1.728947in}}%
\pgfusepath{stroke}%
\end{pgfscope}%
\begin{pgfscope}%
\definecolor{textcolor}{rgb}{0.150000,0.150000,0.150000}%
\pgfsetstrokecolor{textcolor}%
\pgfsetfillcolor{textcolor}%
\pgftext[x=3.750588in,y=0.652778in,,top]{\color{textcolor}\sffamily\fontsize{10.000000}{12.000000}\selectfont \(\displaystyle -0.5\)}%
\end{pgfscope}%
\begin{pgfscope}%
\pgfpathrectangle{\pgfqpoint{2.000000in}{0.750000in}}{\pgfqpoint{4.376471in}{0.978947in}} %
\pgfusepath{clip}%
\pgfsetroundcap%
\pgfsetroundjoin%
\pgfsetlinewidth{1.003750pt}%
\definecolor{currentstroke}{rgb}{0.800000,0.800000,0.800000}%
\pgfsetstrokecolor{currentstroke}%
\pgfsetdash{}{0pt}%
\pgfpathmoveto{\pgfqpoint{4.625882in}{0.750000in}}%
\pgfpathlineto{\pgfqpoint{4.625882in}{1.728947in}}%
\pgfusepath{stroke}%
\end{pgfscope}%
\begin{pgfscope}%
\definecolor{textcolor}{rgb}{0.150000,0.150000,0.150000}%
\pgfsetstrokecolor{textcolor}%
\pgfsetfillcolor{textcolor}%
\pgftext[x=4.625882in,y=0.652778in,,top]{\color{textcolor}\sffamily\fontsize{10.000000}{12.000000}\selectfont \(\displaystyle 0.0\)}%
\end{pgfscope}%
\begin{pgfscope}%
\pgfpathrectangle{\pgfqpoint{2.000000in}{0.750000in}}{\pgfqpoint{4.376471in}{0.978947in}} %
\pgfusepath{clip}%
\pgfsetroundcap%
\pgfsetroundjoin%
\pgfsetlinewidth{1.003750pt}%
\definecolor{currentstroke}{rgb}{0.800000,0.800000,0.800000}%
\pgfsetstrokecolor{currentstroke}%
\pgfsetdash{}{0pt}%
\pgfpathmoveto{\pgfqpoint{5.501176in}{0.750000in}}%
\pgfpathlineto{\pgfqpoint{5.501176in}{1.728947in}}%
\pgfusepath{stroke}%
\end{pgfscope}%
\begin{pgfscope}%
\definecolor{textcolor}{rgb}{0.150000,0.150000,0.150000}%
\pgfsetstrokecolor{textcolor}%
\pgfsetfillcolor{textcolor}%
\pgftext[x=5.501176in,y=0.652778in,,top]{\color{textcolor}\sffamily\fontsize{10.000000}{12.000000}\selectfont \(\displaystyle 0.5\)}%
\end{pgfscope}%
\begin{pgfscope}%
\pgfpathrectangle{\pgfqpoint{2.000000in}{0.750000in}}{\pgfqpoint{4.376471in}{0.978947in}} %
\pgfusepath{clip}%
\pgfsetroundcap%
\pgfsetroundjoin%
\pgfsetlinewidth{1.003750pt}%
\definecolor{currentstroke}{rgb}{0.800000,0.800000,0.800000}%
\pgfsetstrokecolor{currentstroke}%
\pgfsetdash{}{0pt}%
\pgfpathmoveto{\pgfqpoint{6.376471in}{0.750000in}}%
\pgfpathlineto{\pgfqpoint{6.376471in}{1.728947in}}%
\pgfusepath{stroke}%
\end{pgfscope}%
\begin{pgfscope}%
\definecolor{textcolor}{rgb}{0.150000,0.150000,0.150000}%
\pgfsetstrokecolor{textcolor}%
\pgfsetfillcolor{textcolor}%
\pgftext[x=6.376471in,y=0.652778in,,top]{\color{textcolor}\sffamily\fontsize{10.000000}{12.000000}\selectfont \(\displaystyle 1.0\)}%
\end{pgfscope}%
\begin{pgfscope}%
\definecolor{textcolor}{rgb}{0.150000,0.150000,0.150000}%
\pgfsetstrokecolor{textcolor}%
\pgfsetfillcolor{textcolor}%
\pgftext[x=4.188235in,y=0.456313in,,top]{\color{textcolor}\sffamily\fontsize{11.000000}{13.200000}\selectfont x}%
\end{pgfscope}%
\begin{pgfscope}%
\pgfpathrectangle{\pgfqpoint{2.000000in}{0.750000in}}{\pgfqpoint{4.376471in}{0.978947in}} %
\pgfusepath{clip}%
\pgfsetroundcap%
\pgfsetroundjoin%
\pgfsetlinewidth{1.003750pt}%
\definecolor{currentstroke}{rgb}{0.800000,0.800000,0.800000}%
\pgfsetstrokecolor{currentstroke}%
\pgfsetdash{}{0pt}%
\pgfpathmoveto{\pgfqpoint{2.000000in}{0.913158in}}%
\pgfpathlineto{\pgfqpoint{6.376471in}{0.913158in}}%
\pgfusepath{stroke}%
\end{pgfscope}%
\begin{pgfscope}%
\definecolor{textcolor}{rgb}{0.150000,0.150000,0.150000}%
\pgfsetstrokecolor{textcolor}%
\pgfsetfillcolor{textcolor}%
\pgftext[x=1.902778in,y=0.913158in,right,]{\color{textcolor}\sffamily\fontsize{10.000000}{12.000000}\selectfont \(\displaystyle -1\)}%
\end{pgfscope}%
\begin{pgfscope}%
\pgfpathrectangle{\pgfqpoint{2.000000in}{0.750000in}}{\pgfqpoint{4.376471in}{0.978947in}} %
\pgfusepath{clip}%
\pgfsetroundcap%
\pgfsetroundjoin%
\pgfsetlinewidth{1.003750pt}%
\definecolor{currentstroke}{rgb}{0.800000,0.800000,0.800000}%
\pgfsetstrokecolor{currentstroke}%
\pgfsetdash{}{0pt}%
\pgfpathmoveto{\pgfqpoint{2.000000in}{1.117105in}}%
\pgfpathlineto{\pgfqpoint{6.376471in}{1.117105in}}%
\pgfusepath{stroke}%
\end{pgfscope}%
\begin{pgfscope}%
\definecolor{textcolor}{rgb}{0.150000,0.150000,0.150000}%
\pgfsetstrokecolor{textcolor}%
\pgfsetfillcolor{textcolor}%
\pgftext[x=1.902778in,y=1.117105in,right,]{\color{textcolor}\sffamily\fontsize{10.000000}{12.000000}\selectfont \(\displaystyle 0\)}%
\end{pgfscope}%
\begin{pgfscope}%
\pgfpathrectangle{\pgfqpoint{2.000000in}{0.750000in}}{\pgfqpoint{4.376471in}{0.978947in}} %
\pgfusepath{clip}%
\pgfsetroundcap%
\pgfsetroundjoin%
\pgfsetlinewidth{1.003750pt}%
\definecolor{currentstroke}{rgb}{0.800000,0.800000,0.800000}%
\pgfsetstrokecolor{currentstroke}%
\pgfsetdash{}{0pt}%
\pgfpathmoveto{\pgfqpoint{2.000000in}{1.321053in}}%
\pgfpathlineto{\pgfqpoint{6.376471in}{1.321053in}}%
\pgfusepath{stroke}%
\end{pgfscope}%
\begin{pgfscope}%
\definecolor{textcolor}{rgb}{0.150000,0.150000,0.150000}%
\pgfsetstrokecolor{textcolor}%
\pgfsetfillcolor{textcolor}%
\pgftext[x=1.902778in,y=1.321053in,right,]{\color{textcolor}\sffamily\fontsize{10.000000}{12.000000}\selectfont \(\displaystyle 1\)}%
\end{pgfscope}%
\begin{pgfscope}%
\pgfpathrectangle{\pgfqpoint{2.000000in}{0.750000in}}{\pgfqpoint{4.376471in}{0.978947in}} %
\pgfusepath{clip}%
\pgfsetroundcap%
\pgfsetroundjoin%
\pgfsetlinewidth{1.003750pt}%
\definecolor{currentstroke}{rgb}{0.800000,0.800000,0.800000}%
\pgfsetstrokecolor{currentstroke}%
\pgfsetdash{}{0pt}%
\pgfpathmoveto{\pgfqpoint{2.000000in}{1.525000in}}%
\pgfpathlineto{\pgfqpoint{6.376471in}{1.525000in}}%
\pgfusepath{stroke}%
\end{pgfscope}%
\begin{pgfscope}%
\definecolor{textcolor}{rgb}{0.150000,0.150000,0.150000}%
\pgfsetstrokecolor{textcolor}%
\pgfsetfillcolor{textcolor}%
\pgftext[x=1.902778in,y=1.525000in,right,]{\color{textcolor}\sffamily\fontsize{10.000000}{12.000000}\selectfont \(\displaystyle 2\)}%
\end{pgfscope}%
\begin{pgfscope}%
\pgfpathrectangle{\pgfqpoint{2.000000in}{0.750000in}}{\pgfqpoint{4.376471in}{0.978947in}} %
\pgfusepath{clip}%
\pgfsetroundcap%
\pgfsetroundjoin%
\pgfsetlinewidth{1.003750pt}%
\definecolor{currentstroke}{rgb}{0.800000,0.800000,0.800000}%
\pgfsetstrokecolor{currentstroke}%
\pgfsetdash{}{0pt}%
\pgfpathmoveto{\pgfqpoint{2.000000in}{1.728947in}}%
\pgfpathlineto{\pgfqpoint{6.376471in}{1.728947in}}%
\pgfusepath{stroke}%
\end{pgfscope}%
\begin{pgfscope}%
\definecolor{textcolor}{rgb}{0.150000,0.150000,0.150000}%
\pgfsetstrokecolor{textcolor}%
\pgfsetfillcolor{textcolor}%
\pgftext[x=1.902778in,y=1.728947in,right,]{\color{textcolor}\sffamily\fontsize{10.000000}{12.000000}\selectfont \(\displaystyle 3\)}%
\end{pgfscope}%
\begin{pgfscope}%
\definecolor{textcolor}{rgb}{0.150000,0.150000,0.150000}%
\pgfsetstrokecolor{textcolor}%
\pgfsetfillcolor{textcolor}%
\pgftext[x=1.655864in,y=1.239474in,,bottom,rotate=90.000000]{\color{textcolor}\sffamily\fontsize{11.000000}{13.200000}\selectfont y}%
\end{pgfscope}%
\begin{pgfscope}%
\pgfpathrectangle{\pgfqpoint{2.000000in}{0.750000in}}{\pgfqpoint{4.376471in}{0.978947in}} %
\pgfusepath{clip}%
\pgfsetbuttcap%
\pgfsetroundjoin%
\definecolor{currentfill}{rgb}{1.000000,0.000000,0.000000}%
\pgfsetfillcolor{currentfill}%
\pgfsetlinewidth{2.007500pt}%
\definecolor{currentstroke}{rgb}{1.000000,0.000000,0.000000}%
\pgfsetstrokecolor{currentstroke}%
\pgfsetdash{}{0pt}%
\pgfpathmoveto{\pgfqpoint{4.765731in}{1.307576in}}%
\pgfpathlineto{\pgfqpoint{4.827844in}{1.307576in}}%
\pgfpathmoveto{\pgfqpoint{4.796787in}{1.276519in}}%
\pgfpathlineto{\pgfqpoint{4.796787in}{1.338632in}}%
\pgfusepath{stroke,fill}%
\end{pgfscope}%
\begin{pgfscope}%
\pgfpathrectangle{\pgfqpoint{2.000000in}{0.750000in}}{\pgfqpoint{4.376471in}{0.978947in}} %
\pgfusepath{clip}%
\pgfsetbuttcap%
\pgfsetroundjoin%
\definecolor{currentfill}{rgb}{1.000000,0.000000,0.000000}%
\pgfsetfillcolor{currentfill}%
\pgfsetlinewidth{2.007500pt}%
\definecolor{currentstroke}{rgb}{1.000000,0.000000,0.000000}%
\pgfsetstrokecolor{currentstroke}%
\pgfsetdash{}{0pt}%
\pgfpathmoveto{\pgfqpoint{5.348242in}{0.990364in}}%
\pgfpathlineto{\pgfqpoint{5.410355in}{0.990364in}}%
\pgfpathmoveto{\pgfqpoint{5.379298in}{0.959307in}}%
\pgfpathlineto{\pgfqpoint{5.379298in}{1.021420in}}%
\pgfusepath{stroke,fill}%
\end{pgfscope}%
\begin{pgfscope}%
\pgfpathrectangle{\pgfqpoint{2.000000in}{0.750000in}}{\pgfqpoint{4.376471in}{0.978947in}} %
\pgfusepath{clip}%
\pgfsetbuttcap%
\pgfsetroundjoin%
\definecolor{currentfill}{rgb}{1.000000,0.000000,0.000000}%
\pgfsetfillcolor{currentfill}%
\pgfsetlinewidth{2.007500pt}%
\definecolor{currentstroke}{rgb}{1.000000,0.000000,0.000000}%
\pgfsetstrokecolor{currentstroke}%
\pgfsetdash{}{0pt}%
\pgfpathmoveto{\pgfqpoint{4.954619in}{1.361104in}}%
\pgfpathlineto{\pgfqpoint{5.016732in}{1.361104in}}%
\pgfpathmoveto{\pgfqpoint{4.985675in}{1.330047in}}%
\pgfpathlineto{\pgfqpoint{4.985675in}{1.392160in}}%
\pgfusepath{stroke,fill}%
\end{pgfscope}%
\begin{pgfscope}%
\pgfpathrectangle{\pgfqpoint{2.000000in}{0.750000in}}{\pgfqpoint{4.376471in}{0.978947in}} %
\pgfusepath{clip}%
\pgfsetbuttcap%
\pgfsetroundjoin%
\definecolor{currentfill}{rgb}{1.000000,0.000000,0.000000}%
\pgfsetfillcolor{currentfill}%
\pgfsetlinewidth{2.007500pt}%
\definecolor{currentstroke}{rgb}{1.000000,0.000000,0.000000}%
\pgfsetstrokecolor{currentstroke}%
\pgfsetdash{}{0pt}%
\pgfpathmoveto{\pgfqpoint{4.751970in}{1.250192in}}%
\pgfpathlineto{\pgfqpoint{4.814083in}{1.250192in}}%
\pgfpathmoveto{\pgfqpoint{4.783026in}{1.219135in}}%
\pgfpathlineto{\pgfqpoint{4.783026in}{1.281248in}}%
\pgfusepath{stroke,fill}%
\end{pgfscope}%
\begin{pgfscope}%
\pgfpathrectangle{\pgfqpoint{2.000000in}{0.750000in}}{\pgfqpoint{4.376471in}{0.978947in}} %
\pgfusepath{clip}%
\pgfsetbuttcap%
\pgfsetroundjoin%
\definecolor{currentfill}{rgb}{1.000000,0.000000,0.000000}%
\pgfsetfillcolor{currentfill}%
\pgfsetlinewidth{2.007500pt}%
\definecolor{currentstroke}{rgb}{1.000000,0.000000,0.000000}%
\pgfsetstrokecolor{currentstroke}%
\pgfsetdash{}{0pt}%
\pgfpathmoveto{\pgfqpoint{4.327528in}{0.923797in}}%
\pgfpathlineto{\pgfqpoint{4.389641in}{0.923797in}}%
\pgfpathmoveto{\pgfqpoint{4.358584in}{0.892741in}}%
\pgfpathlineto{\pgfqpoint{4.358584in}{0.954854in}}%
\pgfusepath{stroke,fill}%
\end{pgfscope}%
\begin{pgfscope}%
\pgfpathrectangle{\pgfqpoint{2.000000in}{0.750000in}}{\pgfqpoint{4.376471in}{0.978947in}} %
\pgfusepath{clip}%
\pgfsetbuttcap%
\pgfsetroundjoin%
\definecolor{currentfill}{rgb}{1.000000,0.000000,0.000000}%
\pgfsetfillcolor{currentfill}%
\pgfsetlinewidth{2.007500pt}%
\definecolor{currentstroke}{rgb}{1.000000,0.000000,0.000000}%
\pgfsetstrokecolor{currentstroke}%
\pgfsetdash{}{0pt}%
\pgfpathmoveto{\pgfqpoint{5.105627in}{1.193262in}}%
\pgfpathlineto{\pgfqpoint{5.167740in}{1.193262in}}%
\pgfpathmoveto{\pgfqpoint{5.136683in}{1.162205in}}%
\pgfpathlineto{\pgfqpoint{5.136683in}{1.224318in}}%
\pgfusepath{stroke,fill}%
\end{pgfscope}%
\begin{pgfscope}%
\pgfpathrectangle{\pgfqpoint{2.000000in}{0.750000in}}{\pgfqpoint{4.376471in}{0.978947in}} %
\pgfusepath{clip}%
\pgfsetbuttcap%
\pgfsetroundjoin%
\definecolor{currentfill}{rgb}{1.000000,0.000000,0.000000}%
\pgfsetfillcolor{currentfill}%
\pgfsetlinewidth{2.007500pt}%
\definecolor{currentstroke}{rgb}{1.000000,0.000000,0.000000}%
\pgfsetstrokecolor{currentstroke}%
\pgfsetdash{}{0pt}%
\pgfpathmoveto{\pgfqpoint{4.376308in}{0.961291in}}%
\pgfpathlineto{\pgfqpoint{4.438421in}{0.961291in}}%
\pgfpathmoveto{\pgfqpoint{4.407364in}{0.930235in}}%
\pgfpathlineto{\pgfqpoint{4.407364in}{0.992348in}}%
\pgfusepath{stroke,fill}%
\end{pgfscope}%
\begin{pgfscope}%
\pgfpathrectangle{\pgfqpoint{2.000000in}{0.750000in}}{\pgfqpoint{4.376471in}{0.978947in}} %
\pgfusepath{clip}%
\pgfsetbuttcap%
\pgfsetroundjoin%
\definecolor{currentfill}{rgb}{1.000000,0.000000,0.000000}%
\pgfsetfillcolor{currentfill}%
\pgfsetlinewidth{2.007500pt}%
\definecolor{currentstroke}{rgb}{1.000000,0.000000,0.000000}%
\pgfsetstrokecolor{currentstroke}%
\pgfsetdash{}{0pt}%
\pgfpathmoveto{\pgfqpoint{5.966492in}{1.601409in}}%
\pgfpathlineto{\pgfqpoint{6.028605in}{1.601409in}}%
\pgfpathmoveto{\pgfqpoint{5.997549in}{1.570353in}}%
\pgfpathlineto{\pgfqpoint{5.997549in}{1.632466in}}%
\pgfusepath{stroke,fill}%
\end{pgfscope}%
\begin{pgfscope}%
\pgfpathrectangle{\pgfqpoint{2.000000in}{0.750000in}}{\pgfqpoint{4.376471in}{0.978947in}} %
\pgfusepath{clip}%
\pgfsetbuttcap%
\pgfsetroundjoin%
\definecolor{currentfill}{rgb}{1.000000,0.000000,0.000000}%
\pgfsetfillcolor{currentfill}%
\pgfsetlinewidth{2.007500pt}%
\definecolor{currentstroke}{rgb}{1.000000,0.000000,0.000000}%
\pgfsetstrokecolor{currentstroke}%
\pgfsetdash{}{0pt}%
\pgfpathmoveto{\pgfqpoint{6.218191in}{1.501812in}}%
\pgfpathlineto{\pgfqpoint{6.280304in}{1.501812in}}%
\pgfpathmoveto{\pgfqpoint{6.249248in}{1.470755in}}%
\pgfpathlineto{\pgfqpoint{6.249248in}{1.532868in}}%
\pgfusepath{stroke,fill}%
\end{pgfscope}%
\begin{pgfscope}%
\pgfpathrectangle{\pgfqpoint{2.000000in}{0.750000in}}{\pgfqpoint{4.376471in}{0.978947in}} %
\pgfusepath{clip}%
\pgfsetbuttcap%
\pgfsetroundjoin%
\definecolor{currentfill}{rgb}{1.000000,0.000000,0.000000}%
\pgfsetfillcolor{currentfill}%
\pgfsetlinewidth{2.007500pt}%
\definecolor{currentstroke}{rgb}{1.000000,0.000000,0.000000}%
\pgfsetstrokecolor{currentstroke}%
\pgfsetdash{}{0pt}%
\pgfpathmoveto{\pgfqpoint{4.186734in}{0.999208in}}%
\pgfpathlineto{\pgfqpoint{4.248847in}{0.999208in}}%
\pgfpathmoveto{\pgfqpoint{4.217791in}{0.968152in}}%
\pgfpathlineto{\pgfqpoint{4.217791in}{1.030265in}}%
\pgfusepath{stroke,fill}%
\end{pgfscope}%
\begin{pgfscope}%
\pgfpathrectangle{\pgfqpoint{2.000000in}{0.750000in}}{\pgfqpoint{4.376471in}{0.978947in}} %
\pgfusepath{clip}%
\pgfsetbuttcap%
\pgfsetroundjoin%
\definecolor{currentfill}{rgb}{1.000000,0.000000,0.000000}%
\pgfsetfillcolor{currentfill}%
\pgfsetlinewidth{2.007500pt}%
\definecolor{currentstroke}{rgb}{1.000000,0.000000,0.000000}%
\pgfsetstrokecolor{currentstroke}%
\pgfsetdash{}{0pt}%
\pgfpathmoveto{\pgfqpoint{5.616207in}{1.149384in}}%
\pgfpathlineto{\pgfqpoint{5.678320in}{1.149384in}}%
\pgfpathmoveto{\pgfqpoint{5.647263in}{1.118327in}}%
\pgfpathlineto{\pgfqpoint{5.647263in}{1.180440in}}%
\pgfusepath{stroke,fill}%
\end{pgfscope}%
\begin{pgfscope}%
\pgfpathrectangle{\pgfqpoint{2.000000in}{0.750000in}}{\pgfqpoint{4.376471in}{0.978947in}} %
\pgfusepath{clip}%
\pgfsetbuttcap%
\pgfsetroundjoin%
\definecolor{currentfill}{rgb}{1.000000,0.000000,0.000000}%
\pgfsetfillcolor{currentfill}%
\pgfsetlinewidth{2.007500pt}%
\definecolor{currentstroke}{rgb}{1.000000,0.000000,0.000000}%
\pgfsetstrokecolor{currentstroke}%
\pgfsetdash{}{0pt}%
\pgfpathmoveto{\pgfqpoint{4.695992in}{1.189479in}}%
\pgfpathlineto{\pgfqpoint{4.758105in}{1.189479in}}%
\pgfpathmoveto{\pgfqpoint{4.727049in}{1.158422in}}%
\pgfpathlineto{\pgfqpoint{4.727049in}{1.220535in}}%
\pgfusepath{stroke,fill}%
\end{pgfscope}%
\begin{pgfscope}%
\pgfpathrectangle{\pgfqpoint{2.000000in}{0.750000in}}{\pgfqpoint{4.376471in}{0.978947in}} %
\pgfusepath{clip}%
\pgfsetbuttcap%
\pgfsetroundjoin%
\definecolor{currentfill}{rgb}{1.000000,0.000000,0.000000}%
\pgfsetfillcolor{currentfill}%
\pgfsetlinewidth{2.007500pt}%
\definecolor{currentstroke}{rgb}{1.000000,0.000000,0.000000}%
\pgfsetstrokecolor{currentstroke}%
\pgfsetdash{}{0pt}%
\pgfpathmoveto{\pgfqpoint{4.833062in}{1.317035in}}%
\pgfpathlineto{\pgfqpoint{4.895175in}{1.317035in}}%
\pgfpathmoveto{\pgfqpoint{4.864118in}{1.285978in}}%
\pgfpathlineto{\pgfqpoint{4.864118in}{1.348091in}}%
\pgfusepath{stroke,fill}%
\end{pgfscope}%
\begin{pgfscope}%
\pgfpathrectangle{\pgfqpoint{2.000000in}{0.750000in}}{\pgfqpoint{4.376471in}{0.978947in}} %
\pgfusepath{clip}%
\pgfsetbuttcap%
\pgfsetroundjoin%
\definecolor{currentfill}{rgb}{1.000000,0.000000,0.000000}%
\pgfsetfillcolor{currentfill}%
\pgfsetlinewidth{2.007500pt}%
\definecolor{currentstroke}{rgb}{1.000000,0.000000,0.000000}%
\pgfsetstrokecolor{currentstroke}%
\pgfsetdash{}{0pt}%
\pgfpathmoveto{\pgfqpoint{6.084915in}{1.577199in}}%
\pgfpathlineto{\pgfqpoint{6.147028in}{1.577199in}}%
\pgfpathmoveto{\pgfqpoint{6.115971in}{1.546142in}}%
\pgfpathlineto{\pgfqpoint{6.115971in}{1.608255in}}%
\pgfusepath{stroke,fill}%
\end{pgfscope}%
\begin{pgfscope}%
\pgfpathrectangle{\pgfqpoint{2.000000in}{0.750000in}}{\pgfqpoint{4.376471in}{0.978947in}} %
\pgfusepath{clip}%
\pgfsetbuttcap%
\pgfsetroundjoin%
\definecolor{currentfill}{rgb}{1.000000,0.000000,0.000000}%
\pgfsetfillcolor{currentfill}%
\pgfsetlinewidth{2.007500pt}%
\definecolor{currentstroke}{rgb}{1.000000,0.000000,0.000000}%
\pgfsetstrokecolor{currentstroke}%
\pgfsetdash{}{0pt}%
\pgfpathmoveto{\pgfqpoint{3.092947in}{1.289809in}}%
\pgfpathlineto{\pgfqpoint{3.155060in}{1.289809in}}%
\pgfpathmoveto{\pgfqpoint{3.124004in}{1.258753in}}%
\pgfpathlineto{\pgfqpoint{3.124004in}{1.320866in}}%
\pgfusepath{stroke,fill}%
\end{pgfscope}%
\begin{pgfscope}%
\pgfpathrectangle{\pgfqpoint{2.000000in}{0.750000in}}{\pgfqpoint{4.376471in}{0.978947in}} %
\pgfusepath{clip}%
\pgfsetbuttcap%
\pgfsetroundjoin%
\definecolor{currentfill}{rgb}{1.000000,0.000000,0.000000}%
\pgfsetfillcolor{currentfill}%
\pgfsetlinewidth{2.007500pt}%
\definecolor{currentstroke}{rgb}{1.000000,0.000000,0.000000}%
\pgfsetstrokecolor{currentstroke}%
\pgfsetdash{}{0pt}%
\pgfpathmoveto{\pgfqpoint{3.149293in}{1.232176in}}%
\pgfpathlineto{\pgfqpoint{3.211406in}{1.232176in}}%
\pgfpathmoveto{\pgfqpoint{3.180349in}{1.201119in}}%
\pgfpathlineto{\pgfqpoint{3.180349in}{1.263232in}}%
\pgfusepath{stroke,fill}%
\end{pgfscope}%
\begin{pgfscope}%
\pgfpathrectangle{\pgfqpoint{2.000000in}{0.750000in}}{\pgfqpoint{4.376471in}{0.978947in}} %
\pgfusepath{clip}%
\pgfsetbuttcap%
\pgfsetroundjoin%
\definecolor{currentfill}{rgb}{1.000000,0.000000,0.000000}%
\pgfsetfillcolor{currentfill}%
\pgfsetlinewidth{2.007500pt}%
\definecolor{currentstroke}{rgb}{1.000000,0.000000,0.000000}%
\pgfsetstrokecolor{currentstroke}%
\pgfsetdash{}{0pt}%
\pgfpathmoveto{\pgfqpoint{2.915026in}{1.519456in}}%
\pgfpathlineto{\pgfqpoint{2.977139in}{1.519456in}}%
\pgfpathmoveto{\pgfqpoint{2.946082in}{1.488399in}}%
\pgfpathlineto{\pgfqpoint{2.946082in}{1.550512in}}%
\pgfusepath{stroke,fill}%
\end{pgfscope}%
\begin{pgfscope}%
\pgfpathrectangle{\pgfqpoint{2.000000in}{0.750000in}}{\pgfqpoint{4.376471in}{0.978947in}} %
\pgfusepath{clip}%
\pgfsetbuttcap%
\pgfsetroundjoin%
\definecolor{currentfill}{rgb}{1.000000,0.000000,0.000000}%
\pgfsetfillcolor{currentfill}%
\pgfsetlinewidth{2.007500pt}%
\definecolor{currentstroke}{rgb}{1.000000,0.000000,0.000000}%
\pgfsetstrokecolor{currentstroke}%
\pgfsetdash{}{0pt}%
\pgfpathmoveto{\pgfqpoint{5.759387in}{1.365052in}}%
\pgfpathlineto{\pgfqpoint{5.821500in}{1.365052in}}%
\pgfpathmoveto{\pgfqpoint{5.790443in}{1.333995in}}%
\pgfpathlineto{\pgfqpoint{5.790443in}{1.396108in}}%
\pgfusepath{stroke,fill}%
\end{pgfscope}%
\begin{pgfscope}%
\pgfpathrectangle{\pgfqpoint{2.000000in}{0.750000in}}{\pgfqpoint{4.376471in}{0.978947in}} %
\pgfusepath{clip}%
\pgfsetbuttcap%
\pgfsetroundjoin%
\definecolor{currentfill}{rgb}{1.000000,0.000000,0.000000}%
\pgfsetfillcolor{currentfill}%
\pgfsetlinewidth{2.007500pt}%
\definecolor{currentstroke}{rgb}{1.000000,0.000000,0.000000}%
\pgfsetstrokecolor{currentstroke}%
\pgfsetdash{}{0pt}%
\pgfpathmoveto{\pgfqpoint{5.568702in}{1.087470in}}%
\pgfpathlineto{\pgfqpoint{5.630815in}{1.087470in}}%
\pgfpathmoveto{\pgfqpoint{5.599758in}{1.056414in}}%
\pgfpathlineto{\pgfqpoint{5.599758in}{1.118527in}}%
\pgfusepath{stroke,fill}%
\end{pgfscope}%
\begin{pgfscope}%
\pgfpathrectangle{\pgfqpoint{2.000000in}{0.750000in}}{\pgfqpoint{4.376471in}{0.978947in}} %
\pgfusepath{clip}%
\pgfsetbuttcap%
\pgfsetroundjoin%
\definecolor{currentfill}{rgb}{1.000000,0.000000,0.000000}%
\pgfsetfillcolor{currentfill}%
\pgfsetlinewidth{2.007500pt}%
\definecolor{currentstroke}{rgb}{1.000000,0.000000,0.000000}%
\pgfsetstrokecolor{currentstroke}%
\pgfsetdash{}{0pt}%
\pgfpathmoveto{\pgfqpoint{5.890304in}{1.494833in}}%
\pgfpathlineto{\pgfqpoint{5.952417in}{1.494833in}}%
\pgfpathmoveto{\pgfqpoint{5.921360in}{1.463777in}}%
\pgfpathlineto{\pgfqpoint{5.921360in}{1.525890in}}%
\pgfusepath{stroke,fill}%
\end{pgfscope}%
\begin{pgfscope}%
\pgfpathrectangle{\pgfqpoint{2.000000in}{0.750000in}}{\pgfqpoint{4.376471in}{0.978947in}} %
\pgfusepath{clip}%
\pgfsetbuttcap%
\pgfsetroundjoin%
\definecolor{currentfill}{rgb}{1.000000,0.000000,0.000000}%
\pgfsetfillcolor{currentfill}%
\pgfsetlinewidth{2.007500pt}%
\definecolor{currentstroke}{rgb}{1.000000,0.000000,0.000000}%
\pgfsetstrokecolor{currentstroke}%
\pgfsetdash{}{0pt}%
\pgfpathmoveto{\pgfqpoint{6.270553in}{1.426070in}}%
\pgfpathlineto{\pgfqpoint{6.332666in}{1.426070in}}%
\pgfpathmoveto{\pgfqpoint{6.301610in}{1.395013in}}%
\pgfpathlineto{\pgfqpoint{6.301610in}{1.457126in}}%
\pgfusepath{stroke,fill}%
\end{pgfscope}%
\begin{pgfscope}%
\pgfpathrectangle{\pgfqpoint{2.000000in}{0.750000in}}{\pgfqpoint{4.376471in}{0.978947in}} %
\pgfusepath{clip}%
\pgfsetbuttcap%
\pgfsetroundjoin%
\definecolor{currentfill}{rgb}{1.000000,0.000000,0.000000}%
\pgfsetfillcolor{currentfill}%
\pgfsetlinewidth{2.007500pt}%
\definecolor{currentstroke}{rgb}{1.000000,0.000000,0.000000}%
\pgfsetstrokecolor{currentstroke}%
\pgfsetdash{}{0pt}%
\pgfpathmoveto{\pgfqpoint{5.642233in}{1.242637in}}%
\pgfpathlineto{\pgfqpoint{5.704346in}{1.242637in}}%
\pgfpathmoveto{\pgfqpoint{5.673289in}{1.211581in}}%
\pgfpathlineto{\pgfqpoint{5.673289in}{1.273694in}}%
\pgfusepath{stroke,fill}%
\end{pgfscope}%
\begin{pgfscope}%
\pgfpathrectangle{\pgfqpoint{2.000000in}{0.750000in}}{\pgfqpoint{4.376471in}{0.978947in}} %
\pgfusepath{clip}%
\pgfsetbuttcap%
\pgfsetroundjoin%
\definecolor{currentfill}{rgb}{1.000000,0.000000,0.000000}%
\pgfsetfillcolor{currentfill}%
\pgfsetlinewidth{2.007500pt}%
\definecolor{currentstroke}{rgb}{1.000000,0.000000,0.000000}%
\pgfsetstrokecolor{currentstroke}%
\pgfsetdash{}{0pt}%
\pgfpathmoveto{\pgfqpoint{4.459958in}{0.967097in}}%
\pgfpathlineto{\pgfqpoint{4.522071in}{0.967097in}}%
\pgfpathmoveto{\pgfqpoint{4.491015in}{0.936040in}}%
\pgfpathlineto{\pgfqpoint{4.491015in}{0.998153in}}%
\pgfusepath{stroke,fill}%
\end{pgfscope}%
\begin{pgfscope}%
\pgfpathrectangle{\pgfqpoint{2.000000in}{0.750000in}}{\pgfqpoint{4.376471in}{0.978947in}} %
\pgfusepath{clip}%
\pgfsetbuttcap%
\pgfsetroundjoin%
\definecolor{currentfill}{rgb}{1.000000,0.000000,0.000000}%
\pgfsetfillcolor{currentfill}%
\pgfsetlinewidth{2.007500pt}%
\definecolor{currentstroke}{rgb}{1.000000,0.000000,0.000000}%
\pgfsetstrokecolor{currentstroke}%
\pgfsetdash{}{0pt}%
\pgfpathmoveto{\pgfqpoint{5.577008in}{1.109507in}}%
\pgfpathlineto{\pgfqpoint{5.639121in}{1.109507in}}%
\pgfpathmoveto{\pgfqpoint{5.608065in}{1.078450in}}%
\pgfpathlineto{\pgfqpoint{5.608065in}{1.140563in}}%
\pgfusepath{stroke,fill}%
\end{pgfscope}%
\begin{pgfscope}%
\pgfpathrectangle{\pgfqpoint{2.000000in}{0.750000in}}{\pgfqpoint{4.376471in}{0.978947in}} %
\pgfusepath{clip}%
\pgfsetbuttcap%
\pgfsetroundjoin%
\definecolor{currentfill}{rgb}{1.000000,0.000000,0.000000}%
\pgfsetfillcolor{currentfill}%
\pgfsetlinewidth{2.007500pt}%
\definecolor{currentstroke}{rgb}{1.000000,0.000000,0.000000}%
\pgfsetstrokecolor{currentstroke}%
\pgfsetdash{}{0pt}%
\pgfpathmoveto{\pgfqpoint{3.258337in}{1.130244in}}%
\pgfpathlineto{\pgfqpoint{3.320450in}{1.130244in}}%
\pgfpathmoveto{\pgfqpoint{3.289394in}{1.099188in}}%
\pgfpathlineto{\pgfqpoint{3.289394in}{1.161301in}}%
\pgfusepath{stroke,fill}%
\end{pgfscope}%
\begin{pgfscope}%
\pgfpathrectangle{\pgfqpoint{2.000000in}{0.750000in}}{\pgfqpoint{4.376471in}{0.978947in}} %
\pgfusepath{clip}%
\pgfsetbuttcap%
\pgfsetroundjoin%
\definecolor{currentfill}{rgb}{1.000000,0.000000,0.000000}%
\pgfsetfillcolor{currentfill}%
\pgfsetlinewidth{2.007500pt}%
\definecolor{currentstroke}{rgb}{1.000000,0.000000,0.000000}%
\pgfsetstrokecolor{currentstroke}%
\pgfsetdash{}{0pt}%
\pgfpathmoveto{\pgfqpoint{5.084714in}{1.233528in}}%
\pgfpathlineto{\pgfqpoint{5.146827in}{1.233528in}}%
\pgfpathmoveto{\pgfqpoint{5.115771in}{1.202472in}}%
\pgfpathlineto{\pgfqpoint{5.115771in}{1.264585in}}%
\pgfusepath{stroke,fill}%
\end{pgfscope}%
\begin{pgfscope}%
\pgfpathrectangle{\pgfqpoint{2.000000in}{0.750000in}}{\pgfqpoint{4.376471in}{0.978947in}} %
\pgfusepath{clip}%
\pgfsetbuttcap%
\pgfsetroundjoin%
\definecolor{currentfill}{rgb}{1.000000,0.000000,0.000000}%
\pgfsetfillcolor{currentfill}%
\pgfsetlinewidth{2.007500pt}%
\definecolor{currentstroke}{rgb}{1.000000,0.000000,0.000000}%
\pgfsetstrokecolor{currentstroke}%
\pgfsetdash{}{0pt}%
\pgfpathmoveto{\pgfqpoint{3.346143in}{1.138534in}}%
\pgfpathlineto{\pgfqpoint{3.408256in}{1.138534in}}%
\pgfpathmoveto{\pgfqpoint{3.377199in}{1.107478in}}%
\pgfpathlineto{\pgfqpoint{3.377199in}{1.169591in}}%
\pgfusepath{stroke,fill}%
\end{pgfscope}%
\begin{pgfscope}%
\pgfpathrectangle{\pgfqpoint{2.000000in}{0.750000in}}{\pgfqpoint{4.376471in}{0.978947in}} %
\pgfusepath{clip}%
\pgfsetbuttcap%
\pgfsetroundjoin%
\definecolor{currentfill}{rgb}{1.000000,0.000000,0.000000}%
\pgfsetfillcolor{currentfill}%
\pgfsetlinewidth{2.007500pt}%
\definecolor{currentstroke}{rgb}{1.000000,0.000000,0.000000}%
\pgfsetstrokecolor{currentstroke}%
\pgfsetdash{}{0pt}%
\pgfpathmoveto{\pgfqpoint{6.151690in}{1.538728in}}%
\pgfpathlineto{\pgfqpoint{6.213803in}{1.538728in}}%
\pgfpathmoveto{\pgfqpoint{6.182747in}{1.507671in}}%
\pgfpathlineto{\pgfqpoint{6.182747in}{1.569784in}}%
\pgfusepath{stroke,fill}%
\end{pgfscope}%
\begin{pgfscope}%
\pgfpathrectangle{\pgfqpoint{2.000000in}{0.750000in}}{\pgfqpoint{4.376471in}{0.978947in}} %
\pgfusepath{clip}%
\pgfsetbuttcap%
\pgfsetroundjoin%
\definecolor{currentfill}{rgb}{1.000000,0.000000,0.000000}%
\pgfsetfillcolor{currentfill}%
\pgfsetlinewidth{2.007500pt}%
\definecolor{currentstroke}{rgb}{1.000000,0.000000,0.000000}%
\pgfsetstrokecolor{currentstroke}%
\pgfsetdash{}{0pt}%
\pgfpathmoveto{\pgfqpoint{4.671321in}{1.185930in}}%
\pgfpathlineto{\pgfqpoint{4.733434in}{1.185930in}}%
\pgfpathmoveto{\pgfqpoint{4.702377in}{1.154874in}}%
\pgfpathlineto{\pgfqpoint{4.702377in}{1.216987in}}%
\pgfusepath{stroke,fill}%
\end{pgfscope}%
\begin{pgfscope}%
\pgfpathrectangle{\pgfqpoint{2.000000in}{0.750000in}}{\pgfqpoint{4.376471in}{0.978947in}} %
\pgfusepath{clip}%
\pgfsetbuttcap%
\pgfsetroundjoin%
\definecolor{currentfill}{rgb}{1.000000,0.000000,0.000000}%
\pgfsetfillcolor{currentfill}%
\pgfsetlinewidth{2.007500pt}%
\definecolor{currentstroke}{rgb}{1.000000,0.000000,0.000000}%
\pgfsetstrokecolor{currentstroke}%
\pgfsetdash{}{0pt}%
\pgfpathmoveto{\pgfqpoint{4.296042in}{0.934813in}}%
\pgfpathlineto{\pgfqpoint{4.358155in}{0.934813in}}%
\pgfpathmoveto{\pgfqpoint{4.327099in}{0.903757in}}%
\pgfpathlineto{\pgfqpoint{4.327099in}{0.965870in}}%
\pgfusepath{stroke,fill}%
\end{pgfscope}%
\begin{pgfscope}%
\pgfpathrectangle{\pgfqpoint{2.000000in}{0.750000in}}{\pgfqpoint{4.376471in}{0.978947in}} %
\pgfusepath{clip}%
\pgfsetbuttcap%
\pgfsetroundjoin%
\definecolor{currentfill}{rgb}{0.000000,0.000000,0.000000}%
\pgfsetfillcolor{currentfill}%
\pgfsetlinewidth{0.301125pt}%
\definecolor{currentstroke}{rgb}{0.000000,0.000000,0.000000}%
\pgfsetstrokecolor{currentstroke}%
\pgfsetdash{}{0pt}%
\pgfsys@defobject{currentmarker}{\pgfqpoint{-0.015528in}{-0.015528in}}{\pgfqpoint{0.015528in}{0.015528in}}{%
\pgfpathmoveto{\pgfqpoint{0.000000in}{-0.015528in}}%
\pgfpathcurveto{\pgfqpoint{0.004118in}{-0.015528in}}{\pgfqpoint{0.008068in}{-0.013892in}}{\pgfqpoint{0.010980in}{-0.010980in}}%
\pgfpathcurveto{\pgfqpoint{0.013892in}{-0.008068in}}{\pgfqpoint{0.015528in}{-0.004118in}}{\pgfqpoint{0.015528in}{0.000000in}}%
\pgfpathcurveto{\pgfqpoint{0.015528in}{0.004118in}}{\pgfqpoint{0.013892in}{0.008068in}}{\pgfqpoint{0.010980in}{0.010980in}}%
\pgfpathcurveto{\pgfqpoint{0.008068in}{0.013892in}}{\pgfqpoint{0.004118in}{0.015528in}}{\pgfqpoint{0.000000in}{0.015528in}}%
\pgfpathcurveto{\pgfqpoint{-0.004118in}{0.015528in}}{\pgfqpoint{-0.008068in}{0.013892in}}{\pgfqpoint{-0.010980in}{0.010980in}}%
\pgfpathcurveto{\pgfqpoint{-0.013892in}{0.008068in}}{\pgfqpoint{-0.015528in}{0.004118in}}{\pgfqpoint{-0.015528in}{0.000000in}}%
\pgfpathcurveto{\pgfqpoint{-0.015528in}{-0.004118in}}{\pgfqpoint{-0.013892in}{-0.008068in}}{\pgfqpoint{-0.010980in}{-0.010980in}}%
\pgfpathcurveto{\pgfqpoint{-0.008068in}{-0.013892in}}{\pgfqpoint{-0.004118in}{-0.015528in}}{\pgfqpoint{0.000000in}{-0.015528in}}%
\pgfpathclose%
\pgfusepath{stroke,fill}%
}%
\begin{pgfscope}%
\pgfsys@transformshift{2.875294in}{1.583863in}%
\pgfsys@useobject{currentmarker}{}%
\end{pgfscope}%
\begin{pgfscope}%
\pgfsys@transformshift{2.892888in}{1.489671in}%
\pgfsys@useobject{currentmarker}{}%
\end{pgfscope}%
\begin{pgfscope}%
\pgfsys@transformshift{2.910482in}{1.580445in}%
\pgfsys@useobject{currentmarker}{}%
\end{pgfscope}%
\begin{pgfscope}%
\pgfsys@transformshift{2.928076in}{1.599582in}%
\pgfsys@useobject{currentmarker}{}%
\end{pgfscope}%
\begin{pgfscope}%
\pgfsys@transformshift{2.945670in}{1.534796in}%
\pgfsys@useobject{currentmarker}{}%
\end{pgfscope}%
\begin{pgfscope}%
\pgfsys@transformshift{2.963263in}{1.530696in}%
\pgfsys@useobject{currentmarker}{}%
\end{pgfscope}%
\begin{pgfscope}%
\pgfsys@transformshift{2.980857in}{1.406938in}%
\pgfsys@useobject{currentmarker}{}%
\end{pgfscope}%
\begin{pgfscope}%
\pgfsys@transformshift{2.998451in}{1.406543in}%
\pgfsys@useobject{currentmarker}{}%
\end{pgfscope}%
\begin{pgfscope}%
\pgfsys@transformshift{3.016045in}{1.347216in}%
\pgfsys@useobject{currentmarker}{}%
\end{pgfscope}%
\begin{pgfscope}%
\pgfsys@transformshift{3.033639in}{1.351930in}%
\pgfsys@useobject{currentmarker}{}%
\end{pgfscope}%
\begin{pgfscope}%
\pgfsys@transformshift{3.051233in}{1.279228in}%
\pgfsys@useobject{currentmarker}{}%
\end{pgfscope}%
\begin{pgfscope}%
\pgfsys@transformshift{3.068826in}{1.160606in}%
\pgfsys@useobject{currentmarker}{}%
\end{pgfscope}%
\begin{pgfscope}%
\pgfsys@transformshift{3.086420in}{1.330351in}%
\pgfsys@useobject{currentmarker}{}%
\end{pgfscope}%
\begin{pgfscope}%
\pgfsys@transformshift{3.104014in}{1.248201in}%
\pgfsys@useobject{currentmarker}{}%
\end{pgfscope}%
\begin{pgfscope}%
\pgfsys@transformshift{3.121608in}{1.101305in}%
\pgfsys@useobject{currentmarker}{}%
\end{pgfscope}%
\begin{pgfscope}%
\pgfsys@transformshift{3.139202in}{1.294708in}%
\pgfsys@useobject{currentmarker}{}%
\end{pgfscope}%
\begin{pgfscope}%
\pgfsys@transformshift{3.156796in}{1.136705in}%
\pgfsys@useobject{currentmarker}{}%
\end{pgfscope}%
\begin{pgfscope}%
\pgfsys@transformshift{3.174390in}{1.218082in}%
\pgfsys@useobject{currentmarker}{}%
\end{pgfscope}%
\begin{pgfscope}%
\pgfsys@transformshift{3.191983in}{1.272640in}%
\pgfsys@useobject{currentmarker}{}%
\end{pgfscope}%
\begin{pgfscope}%
\pgfsys@transformshift{3.209577in}{1.198976in}%
\pgfsys@useobject{currentmarker}{}%
\end{pgfscope}%
\begin{pgfscope}%
\pgfsys@transformshift{3.227171in}{1.291626in}%
\pgfsys@useobject{currentmarker}{}%
\end{pgfscope}%
\begin{pgfscope}%
\pgfsys@transformshift{3.244765in}{1.041254in}%
\pgfsys@useobject{currentmarker}{}%
\end{pgfscope}%
\begin{pgfscope}%
\pgfsys@transformshift{3.262359in}{1.202078in}%
\pgfsys@useobject{currentmarker}{}%
\end{pgfscope}%
\begin{pgfscope}%
\pgfsys@transformshift{3.279953in}{1.087229in}%
\pgfsys@useobject{currentmarker}{}%
\end{pgfscope}%
\begin{pgfscope}%
\pgfsys@transformshift{3.297547in}{1.066387in}%
\pgfsys@useobject{currentmarker}{}%
\end{pgfscope}%
\begin{pgfscope}%
\pgfsys@transformshift{3.315140in}{1.096351in}%
\pgfsys@useobject{currentmarker}{}%
\end{pgfscope}%
\begin{pgfscope}%
\pgfsys@transformshift{3.332734in}{1.125805in}%
\pgfsys@useobject{currentmarker}{}%
\end{pgfscope}%
\begin{pgfscope}%
\pgfsys@transformshift{3.350328in}{1.167420in}%
\pgfsys@useobject{currentmarker}{}%
\end{pgfscope}%
\begin{pgfscope}%
\pgfsys@transformshift{3.367922in}{1.048813in}%
\pgfsys@useobject{currentmarker}{}%
\end{pgfscope}%
\begin{pgfscope}%
\pgfsys@transformshift{3.385516in}{1.267121in}%
\pgfsys@useobject{currentmarker}{}%
\end{pgfscope}%
\begin{pgfscope}%
\pgfsys@transformshift{3.403110in}{1.231941in}%
\pgfsys@useobject{currentmarker}{}%
\end{pgfscope}%
\begin{pgfscope}%
\pgfsys@transformshift{3.420704in}{1.038406in}%
\pgfsys@useobject{currentmarker}{}%
\end{pgfscope}%
\begin{pgfscope}%
\pgfsys@transformshift{3.438297in}{1.358678in}%
\pgfsys@useobject{currentmarker}{}%
\end{pgfscope}%
\begin{pgfscope}%
\pgfsys@transformshift{3.455891in}{1.413170in}%
\pgfsys@useobject{currentmarker}{}%
\end{pgfscope}%
\begin{pgfscope}%
\pgfsys@transformshift{3.473485in}{1.353852in}%
\pgfsys@useobject{currentmarker}{}%
\end{pgfscope}%
\begin{pgfscope}%
\pgfsys@transformshift{3.491079in}{1.229798in}%
\pgfsys@useobject{currentmarker}{}%
\end{pgfscope}%
\begin{pgfscope}%
\pgfsys@transformshift{3.508673in}{1.153945in}%
\pgfsys@useobject{currentmarker}{}%
\end{pgfscope}%
\begin{pgfscope}%
\pgfsys@transformshift{3.526267in}{1.385934in}%
\pgfsys@useobject{currentmarker}{}%
\end{pgfscope}%
\begin{pgfscope}%
\pgfsys@transformshift{3.543860in}{1.252646in}%
\pgfsys@useobject{currentmarker}{}%
\end{pgfscope}%
\begin{pgfscope}%
\pgfsys@transformshift{3.561454in}{1.433641in}%
\pgfsys@useobject{currentmarker}{}%
\end{pgfscope}%
\begin{pgfscope}%
\pgfsys@transformshift{3.579048in}{1.345114in}%
\pgfsys@useobject{currentmarker}{}%
\end{pgfscope}%
\begin{pgfscope}%
\pgfsys@transformshift{3.596642in}{1.437827in}%
\pgfsys@useobject{currentmarker}{}%
\end{pgfscope}%
\begin{pgfscope}%
\pgfsys@transformshift{3.614236in}{1.388209in}%
\pgfsys@useobject{currentmarker}{}%
\end{pgfscope}%
\begin{pgfscope}%
\pgfsys@transformshift{3.631830in}{1.436642in}%
\pgfsys@useobject{currentmarker}{}%
\end{pgfscope}%
\begin{pgfscope}%
\pgfsys@transformshift{3.649424in}{1.377294in}%
\pgfsys@useobject{currentmarker}{}%
\end{pgfscope}%
\begin{pgfscope}%
\pgfsys@transformshift{3.667017in}{1.568717in}%
\pgfsys@useobject{currentmarker}{}%
\end{pgfscope}%
\begin{pgfscope}%
\pgfsys@transformshift{3.684611in}{1.408526in}%
\pgfsys@useobject{currentmarker}{}%
\end{pgfscope}%
\begin{pgfscope}%
\pgfsys@transformshift{3.702205in}{1.444018in}%
\pgfsys@useobject{currentmarker}{}%
\end{pgfscope}%
\begin{pgfscope}%
\pgfsys@transformshift{3.719799in}{1.600841in}%
\pgfsys@useobject{currentmarker}{}%
\end{pgfscope}%
\begin{pgfscope}%
\pgfsys@transformshift{3.737393in}{1.275399in}%
\pgfsys@useobject{currentmarker}{}%
\end{pgfscope}%
\begin{pgfscope}%
\pgfsys@transformshift{3.754987in}{1.285462in}%
\pgfsys@useobject{currentmarker}{}%
\end{pgfscope}%
\begin{pgfscope}%
\pgfsys@transformshift{3.772581in}{1.514150in}%
\pgfsys@useobject{currentmarker}{}%
\end{pgfscope}%
\begin{pgfscope}%
\pgfsys@transformshift{3.790174in}{1.293998in}%
\pgfsys@useobject{currentmarker}{}%
\end{pgfscope}%
\begin{pgfscope}%
\pgfsys@transformshift{3.807768in}{1.608181in}%
\pgfsys@useobject{currentmarker}{}%
\end{pgfscope}%
\begin{pgfscope}%
\pgfsys@transformshift{3.825362in}{1.362188in}%
\pgfsys@useobject{currentmarker}{}%
\end{pgfscope}%
\begin{pgfscope}%
\pgfsys@transformshift{3.842956in}{1.320572in}%
\pgfsys@useobject{currentmarker}{}%
\end{pgfscope}%
\begin{pgfscope}%
\pgfsys@transformshift{3.860550in}{1.583393in}%
\pgfsys@useobject{currentmarker}{}%
\end{pgfscope}%
\begin{pgfscope}%
\pgfsys@transformshift{3.878144in}{1.526929in}%
\pgfsys@useobject{currentmarker}{}%
\end{pgfscope}%
\begin{pgfscope}%
\pgfsys@transformshift{3.895738in}{1.553271in}%
\pgfsys@useobject{currentmarker}{}%
\end{pgfscope}%
\begin{pgfscope}%
\pgfsys@transformshift{3.913331in}{1.440414in}%
\pgfsys@useobject{currentmarker}{}%
\end{pgfscope}%
\begin{pgfscope}%
\pgfsys@transformshift{3.930925in}{1.243825in}%
\pgfsys@useobject{currentmarker}{}%
\end{pgfscope}%
\begin{pgfscope}%
\pgfsys@transformshift{3.948519in}{1.508616in}%
\pgfsys@useobject{currentmarker}{}%
\end{pgfscope}%
\begin{pgfscope}%
\pgfsys@transformshift{3.966113in}{1.267415in}%
\pgfsys@useobject{currentmarker}{}%
\end{pgfscope}%
\begin{pgfscope}%
\pgfsys@transformshift{3.983707in}{1.356354in}%
\pgfsys@useobject{currentmarker}{}%
\end{pgfscope}%
\begin{pgfscope}%
\pgfsys@transformshift{4.001301in}{1.349946in}%
\pgfsys@useobject{currentmarker}{}%
\end{pgfscope}%
\begin{pgfscope}%
\pgfsys@transformshift{4.018894in}{1.215603in}%
\pgfsys@useobject{currentmarker}{}%
\end{pgfscope}%
\begin{pgfscope}%
\pgfsys@transformshift{4.036488in}{1.271512in}%
\pgfsys@useobject{currentmarker}{}%
\end{pgfscope}%
\begin{pgfscope}%
\pgfsys@transformshift{4.054082in}{1.280038in}%
\pgfsys@useobject{currentmarker}{}%
\end{pgfscope}%
\begin{pgfscope}%
\pgfsys@transformshift{4.071676in}{1.201314in}%
\pgfsys@useobject{currentmarker}{}%
\end{pgfscope}%
\begin{pgfscope}%
\pgfsys@transformshift{4.089270in}{1.027783in}%
\pgfsys@useobject{currentmarker}{}%
\end{pgfscope}%
\begin{pgfscope}%
\pgfsys@transformshift{4.106864in}{1.147508in}%
\pgfsys@useobject{currentmarker}{}%
\end{pgfscope}%
\begin{pgfscope}%
\pgfsys@transformshift{4.124458in}{1.230000in}%
\pgfsys@useobject{currentmarker}{}%
\end{pgfscope}%
\begin{pgfscope}%
\pgfsys@transformshift{4.142051in}{1.002209in}%
\pgfsys@useobject{currentmarker}{}%
\end{pgfscope}%
\begin{pgfscope}%
\pgfsys@transformshift{4.159645in}{1.036913in}%
\pgfsys@useobject{currentmarker}{}%
\end{pgfscope}%
\begin{pgfscope}%
\pgfsys@transformshift{4.177239in}{0.987962in}%
\pgfsys@useobject{currentmarker}{}%
\end{pgfscope}%
\begin{pgfscope}%
\pgfsys@transformshift{4.194833in}{1.202285in}%
\pgfsys@useobject{currentmarker}{}%
\end{pgfscope}%
\begin{pgfscope}%
\pgfsys@transformshift{4.212427in}{1.065014in}%
\pgfsys@useobject{currentmarker}{}%
\end{pgfscope}%
\begin{pgfscope}%
\pgfsys@transformshift{4.230021in}{1.022286in}%
\pgfsys@useobject{currentmarker}{}%
\end{pgfscope}%
\begin{pgfscope}%
\pgfsys@transformshift{4.247615in}{0.888172in}%
\pgfsys@useobject{currentmarker}{}%
\end{pgfscope}%
\begin{pgfscope}%
\pgfsys@transformshift{4.265208in}{1.009412in}%
\pgfsys@useobject{currentmarker}{}%
\end{pgfscope}%
\begin{pgfscope}%
\pgfsys@transformshift{4.282802in}{0.875298in}%
\pgfsys@useobject{currentmarker}{}%
\end{pgfscope}%
\begin{pgfscope}%
\pgfsys@transformshift{4.300396in}{0.938933in}%
\pgfsys@useobject{currentmarker}{}%
\end{pgfscope}%
\begin{pgfscope}%
\pgfsys@transformshift{4.317990in}{0.864529in}%
\pgfsys@useobject{currentmarker}{}%
\end{pgfscope}%
\begin{pgfscope}%
\pgfsys@transformshift{4.335584in}{0.994133in}%
\pgfsys@useobject{currentmarker}{}%
\end{pgfscope}%
\begin{pgfscope}%
\pgfsys@transformshift{4.353178in}{0.981871in}%
\pgfsys@useobject{currentmarker}{}%
\end{pgfscope}%
\begin{pgfscope}%
\pgfsys@transformshift{4.370772in}{0.901892in}%
\pgfsys@useobject{currentmarker}{}%
\end{pgfscope}%
\begin{pgfscope}%
\pgfsys@transformshift{4.388365in}{0.965702in}%
\pgfsys@useobject{currentmarker}{}%
\end{pgfscope}%
\begin{pgfscope}%
\pgfsys@transformshift{4.405959in}{0.818136in}%
\pgfsys@useobject{currentmarker}{}%
\end{pgfscope}%
\begin{pgfscope}%
\pgfsys@transformshift{4.423553in}{0.783851in}%
\pgfsys@useobject{currentmarker}{}%
\end{pgfscope}%
\begin{pgfscope}%
\pgfsys@transformshift{4.441147in}{0.989012in}%
\pgfsys@useobject{currentmarker}{}%
\end{pgfscope}%
\begin{pgfscope}%
\pgfsys@transformshift{4.458741in}{0.971362in}%
\pgfsys@useobject{currentmarker}{}%
\end{pgfscope}%
\begin{pgfscope}%
\pgfsys@transformshift{4.476335in}{1.031043in}%
\pgfsys@useobject{currentmarker}{}%
\end{pgfscope}%
\begin{pgfscope}%
\pgfsys@transformshift{4.493928in}{1.222861in}%
\pgfsys@useobject{currentmarker}{}%
\end{pgfscope}%
\begin{pgfscope}%
\pgfsys@transformshift{4.511522in}{1.091202in}%
\pgfsys@useobject{currentmarker}{}%
\end{pgfscope}%
\begin{pgfscope}%
\pgfsys@transformshift{4.529116in}{0.918187in}%
\pgfsys@useobject{currentmarker}{}%
\end{pgfscope}%
\begin{pgfscope}%
\pgfsys@transformshift{4.546710in}{1.142721in}%
\pgfsys@useobject{currentmarker}{}%
\end{pgfscope}%
\begin{pgfscope}%
\pgfsys@transformshift{4.564304in}{0.913152in}%
\pgfsys@useobject{currentmarker}{}%
\end{pgfscope}%
\begin{pgfscope}%
\pgfsys@transformshift{4.581898in}{1.019588in}%
\pgfsys@useobject{currentmarker}{}%
\end{pgfscope}%
\begin{pgfscope}%
\pgfsys@transformshift{4.599492in}{1.079610in}%
\pgfsys@useobject{currentmarker}{}%
\end{pgfscope}%
\begin{pgfscope}%
\pgfsys@transformshift{4.617085in}{1.281587in}%
\pgfsys@useobject{currentmarker}{}%
\end{pgfscope}%
\begin{pgfscope}%
\pgfsys@transformshift{4.634679in}{1.051414in}%
\pgfsys@useobject{currentmarker}{}%
\end{pgfscope}%
\begin{pgfscope}%
\pgfsys@transformshift{4.652273in}{1.063552in}%
\pgfsys@useobject{currentmarker}{}%
\end{pgfscope}%
\begin{pgfscope}%
\pgfsys@transformshift{4.669867in}{1.158029in}%
\pgfsys@useobject{currentmarker}{}%
\end{pgfscope}%
\begin{pgfscope}%
\pgfsys@transformshift{4.687461in}{1.120223in}%
\pgfsys@useobject{currentmarker}{}%
\end{pgfscope}%
\begin{pgfscope}%
\pgfsys@transformshift{4.705055in}{1.321952in}%
\pgfsys@useobject{currentmarker}{}%
\end{pgfscope}%
\begin{pgfscope}%
\pgfsys@transformshift{4.722649in}{1.115308in}%
\pgfsys@useobject{currentmarker}{}%
\end{pgfscope}%
\begin{pgfscope}%
\pgfsys@transformshift{4.740242in}{1.125790in}%
\pgfsys@useobject{currentmarker}{}%
\end{pgfscope}%
\begin{pgfscope}%
\pgfsys@transformshift{4.757836in}{1.214357in}%
\pgfsys@useobject{currentmarker}{}%
\end{pgfscope}%
\begin{pgfscope}%
\pgfsys@transformshift{4.775430in}{1.223091in}%
\pgfsys@useobject{currentmarker}{}%
\end{pgfscope}%
\begin{pgfscope}%
\pgfsys@transformshift{4.793024in}{1.484043in}%
\pgfsys@useobject{currentmarker}{}%
\end{pgfscope}%
\begin{pgfscope}%
\pgfsys@transformshift{4.810618in}{1.395905in}%
\pgfsys@useobject{currentmarker}{}%
\end{pgfscope}%
\begin{pgfscope}%
\pgfsys@transformshift{4.828212in}{1.318116in}%
\pgfsys@useobject{currentmarker}{}%
\end{pgfscope}%
\begin{pgfscope}%
\pgfsys@transformshift{4.845805in}{1.192524in}%
\pgfsys@useobject{currentmarker}{}%
\end{pgfscope}%
\begin{pgfscope}%
\pgfsys@transformshift{4.863399in}{1.410011in}%
\pgfsys@useobject{currentmarker}{}%
\end{pgfscope}%
\begin{pgfscope}%
\pgfsys@transformshift{4.880993in}{1.226407in}%
\pgfsys@useobject{currentmarker}{}%
\end{pgfscope}%
\begin{pgfscope}%
\pgfsys@transformshift{4.898587in}{1.173408in}%
\pgfsys@useobject{currentmarker}{}%
\end{pgfscope}%
\begin{pgfscope}%
\pgfsys@transformshift{4.916181in}{1.452637in}%
\pgfsys@useobject{currentmarker}{}%
\end{pgfscope}%
\begin{pgfscope}%
\pgfsys@transformshift{4.933775in}{1.362397in}%
\pgfsys@useobject{currentmarker}{}%
\end{pgfscope}%
\begin{pgfscope}%
\pgfsys@transformshift{4.951369in}{1.420628in}%
\pgfsys@useobject{currentmarker}{}%
\end{pgfscope}%
\begin{pgfscope}%
\pgfsys@transformshift{4.968962in}{1.353984in}%
\pgfsys@useobject{currentmarker}{}%
\end{pgfscope}%
\begin{pgfscope}%
\pgfsys@transformshift{4.986556in}{1.401791in}%
\pgfsys@useobject{currentmarker}{}%
\end{pgfscope}%
\begin{pgfscope}%
\pgfsys@transformshift{5.004150in}{1.239230in}%
\pgfsys@useobject{currentmarker}{}%
\end{pgfscope}%
\begin{pgfscope}%
\pgfsys@transformshift{5.021744in}{1.189725in}%
\pgfsys@useobject{currentmarker}{}%
\end{pgfscope}%
\begin{pgfscope}%
\pgfsys@transformshift{5.039338in}{1.352764in}%
\pgfsys@useobject{currentmarker}{}%
\end{pgfscope}%
\begin{pgfscope}%
\pgfsys@transformshift{5.056932in}{1.188036in}%
\pgfsys@useobject{currentmarker}{}%
\end{pgfscope}%
\begin{pgfscope}%
\pgfsys@transformshift{5.074526in}{1.185140in}%
\pgfsys@useobject{currentmarker}{}%
\end{pgfscope}%
\begin{pgfscope}%
\pgfsys@transformshift{5.092119in}{1.193453in}%
\pgfsys@useobject{currentmarker}{}%
\end{pgfscope}%
\begin{pgfscope}%
\pgfsys@transformshift{5.109713in}{1.225272in}%
\pgfsys@useobject{currentmarker}{}%
\end{pgfscope}%
\begin{pgfscope}%
\pgfsys@transformshift{5.127307in}{1.170299in}%
\pgfsys@useobject{currentmarker}{}%
\end{pgfscope}%
\begin{pgfscope}%
\pgfsys@transformshift{5.144901in}{1.048615in}%
\pgfsys@useobject{currentmarker}{}%
\end{pgfscope}%
\begin{pgfscope}%
\pgfsys@transformshift{5.162495in}{1.105339in}%
\pgfsys@useobject{currentmarker}{}%
\end{pgfscope}%
\begin{pgfscope}%
\pgfsys@transformshift{5.180089in}{0.926315in}%
\pgfsys@useobject{currentmarker}{}%
\end{pgfscope}%
\begin{pgfscope}%
\pgfsys@transformshift{5.197683in}{1.199006in}%
\pgfsys@useobject{currentmarker}{}%
\end{pgfscope}%
\begin{pgfscope}%
\pgfsys@transformshift{5.215276in}{0.954425in}%
\pgfsys@useobject{currentmarker}{}%
\end{pgfscope}%
\begin{pgfscope}%
\pgfsys@transformshift{5.232870in}{0.988260in}%
\pgfsys@useobject{currentmarker}{}%
\end{pgfscope}%
\begin{pgfscope}%
\pgfsys@transformshift{5.250464in}{1.090024in}%
\pgfsys@useobject{currentmarker}{}%
\end{pgfscope}%
\begin{pgfscope}%
\pgfsys@transformshift{5.268058in}{0.994049in}%
\pgfsys@useobject{currentmarker}{}%
\end{pgfscope}%
\begin{pgfscope}%
\pgfsys@transformshift{5.285652in}{1.212689in}%
\pgfsys@useobject{currentmarker}{}%
\end{pgfscope}%
\begin{pgfscope}%
\pgfsys@transformshift{5.303246in}{0.910673in}%
\pgfsys@useobject{currentmarker}{}%
\end{pgfscope}%
\begin{pgfscope}%
\pgfsys@transformshift{5.320839in}{1.058361in}%
\pgfsys@useobject{currentmarker}{}%
\end{pgfscope}%
\begin{pgfscope}%
\pgfsys@transformshift{5.338433in}{1.017342in}%
\pgfsys@useobject{currentmarker}{}%
\end{pgfscope}%
\begin{pgfscope}%
\pgfsys@transformshift{5.356027in}{0.894182in}%
\pgfsys@useobject{currentmarker}{}%
\end{pgfscope}%
\begin{pgfscope}%
\pgfsys@transformshift{5.373621in}{1.060452in}%
\pgfsys@useobject{currentmarker}{}%
\end{pgfscope}%
\begin{pgfscope}%
\pgfsys@transformshift{5.391215in}{0.985337in}%
\pgfsys@useobject{currentmarker}{}%
\end{pgfscope}%
\begin{pgfscope}%
\pgfsys@transformshift{5.408809in}{1.079294in}%
\pgfsys@useobject{currentmarker}{}%
\end{pgfscope}%
\begin{pgfscope}%
\pgfsys@transformshift{5.426403in}{1.084411in}%
\pgfsys@useobject{currentmarker}{}%
\end{pgfscope}%
\begin{pgfscope}%
\pgfsys@transformshift{5.443996in}{1.222992in}%
\pgfsys@useobject{currentmarker}{}%
\end{pgfscope}%
\begin{pgfscope}%
\pgfsys@transformshift{5.461590in}{1.142792in}%
\pgfsys@useobject{currentmarker}{}%
\end{pgfscope}%
\begin{pgfscope}%
\pgfsys@transformshift{5.479184in}{0.975095in}%
\pgfsys@useobject{currentmarker}{}%
\end{pgfscope}%
\begin{pgfscope}%
\pgfsys@transformshift{5.496778in}{0.996685in}%
\pgfsys@useobject{currentmarker}{}%
\end{pgfscope}%
\begin{pgfscope}%
\pgfsys@transformshift{5.514372in}{1.143655in}%
\pgfsys@useobject{currentmarker}{}%
\end{pgfscope}%
\begin{pgfscope}%
\pgfsys@transformshift{5.531966in}{1.110768in}%
\pgfsys@useobject{currentmarker}{}%
\end{pgfscope}%
\begin{pgfscope}%
\pgfsys@transformshift{5.549560in}{1.123579in}%
\pgfsys@useobject{currentmarker}{}%
\end{pgfscope}%
\begin{pgfscope}%
\pgfsys@transformshift{5.567153in}{0.909591in}%
\pgfsys@useobject{currentmarker}{}%
\end{pgfscope}%
\begin{pgfscope}%
\pgfsys@transformshift{5.584747in}{1.089876in}%
\pgfsys@useobject{currentmarker}{}%
\end{pgfscope}%
\begin{pgfscope}%
\pgfsys@transformshift{5.602341in}{1.036542in}%
\pgfsys@useobject{currentmarker}{}%
\end{pgfscope}%
\begin{pgfscope}%
\pgfsys@transformshift{5.619935in}{1.161209in}%
\pgfsys@useobject{currentmarker}{}%
\end{pgfscope}%
\begin{pgfscope}%
\pgfsys@transformshift{5.637529in}{1.144771in}%
\pgfsys@useobject{currentmarker}{}%
\end{pgfscope}%
\begin{pgfscope}%
\pgfsys@transformshift{5.655123in}{1.270786in}%
\pgfsys@useobject{currentmarker}{}%
\end{pgfscope}%
\begin{pgfscope}%
\pgfsys@transformshift{5.672717in}{1.234403in}%
\pgfsys@useobject{currentmarker}{}%
\end{pgfscope}%
\begin{pgfscope}%
\pgfsys@transformshift{5.690310in}{1.307020in}%
\pgfsys@useobject{currentmarker}{}%
\end{pgfscope}%
\begin{pgfscope}%
\pgfsys@transformshift{5.707904in}{1.204550in}%
\pgfsys@useobject{currentmarker}{}%
\end{pgfscope}%
\begin{pgfscope}%
\pgfsys@transformshift{5.725498in}{1.181373in}%
\pgfsys@useobject{currentmarker}{}%
\end{pgfscope}%
\begin{pgfscope}%
\pgfsys@transformshift{5.743092in}{1.261523in}%
\pgfsys@useobject{currentmarker}{}%
\end{pgfscope}%
\begin{pgfscope}%
\pgfsys@transformshift{5.760686in}{1.327137in}%
\pgfsys@useobject{currentmarker}{}%
\end{pgfscope}%
\begin{pgfscope}%
\pgfsys@transformshift{5.778280in}{1.392747in}%
\pgfsys@useobject{currentmarker}{}%
\end{pgfscope}%
\begin{pgfscope}%
\pgfsys@transformshift{5.795873in}{1.609158in}%
\pgfsys@useobject{currentmarker}{}%
\end{pgfscope}%
\begin{pgfscope}%
\pgfsys@transformshift{5.813467in}{1.398425in}%
\pgfsys@useobject{currentmarker}{}%
\end{pgfscope}%
\begin{pgfscope}%
\pgfsys@transformshift{5.831061in}{1.328312in}%
\pgfsys@useobject{currentmarker}{}%
\end{pgfscope}%
\begin{pgfscope}%
\pgfsys@transformshift{5.848655in}{1.412490in}%
\pgfsys@useobject{currentmarker}{}%
\end{pgfscope}%
\begin{pgfscope}%
\pgfsys@transformshift{5.866249in}{1.421228in}%
\pgfsys@useobject{currentmarker}{}%
\end{pgfscope}%
\begin{pgfscope}%
\pgfsys@transformshift{5.883843in}{1.536935in}%
\pgfsys@useobject{currentmarker}{}%
\end{pgfscope}%
\begin{pgfscope}%
\pgfsys@transformshift{5.901437in}{1.348517in}%
\pgfsys@useobject{currentmarker}{}%
\end{pgfscope}%
\begin{pgfscope}%
\pgfsys@transformshift{5.919030in}{1.528236in}%
\pgfsys@useobject{currentmarker}{}%
\end{pgfscope}%
\begin{pgfscope}%
\pgfsys@transformshift{5.936624in}{1.552137in}%
\pgfsys@useobject{currentmarker}{}%
\end{pgfscope}%
\begin{pgfscope}%
\pgfsys@transformshift{5.954218in}{1.572407in}%
\pgfsys@useobject{currentmarker}{}%
\end{pgfscope}%
\begin{pgfscope}%
\pgfsys@transformshift{5.971812in}{1.498469in}%
\pgfsys@useobject{currentmarker}{}%
\end{pgfscope}%
\begin{pgfscope}%
\pgfsys@transformshift{5.989406in}{1.543820in}%
\pgfsys@useobject{currentmarker}{}%
\end{pgfscope}%
\begin{pgfscope}%
\pgfsys@transformshift{6.007000in}{1.429595in}%
\pgfsys@useobject{currentmarker}{}%
\end{pgfscope}%
\begin{pgfscope}%
\pgfsys@transformshift{6.024594in}{1.529262in}%
\pgfsys@useobject{currentmarker}{}%
\end{pgfscope}%
\begin{pgfscope}%
\pgfsys@transformshift{6.042187in}{1.527009in}%
\pgfsys@useobject{currentmarker}{}%
\end{pgfscope}%
\begin{pgfscope}%
\pgfsys@transformshift{6.059781in}{1.625672in}%
\pgfsys@useobject{currentmarker}{}%
\end{pgfscope}%
\begin{pgfscope}%
\pgfsys@transformshift{6.077375in}{1.464366in}%
\pgfsys@useobject{currentmarker}{}%
\end{pgfscope}%
\begin{pgfscope}%
\pgfsys@transformshift{6.094969in}{1.659166in}%
\pgfsys@useobject{currentmarker}{}%
\end{pgfscope}%
\begin{pgfscope}%
\pgfsys@transformshift{6.112563in}{1.727455in}%
\pgfsys@useobject{currentmarker}{}%
\end{pgfscope}%
\begin{pgfscope}%
\pgfsys@transformshift{6.130157in}{1.357955in}%
\pgfsys@useobject{currentmarker}{}%
\end{pgfscope}%
\begin{pgfscope}%
\pgfsys@transformshift{6.147751in}{1.605050in}%
\pgfsys@useobject{currentmarker}{}%
\end{pgfscope}%
\begin{pgfscope}%
\pgfsys@transformshift{6.165344in}{1.621861in}%
\pgfsys@useobject{currentmarker}{}%
\end{pgfscope}%
\begin{pgfscope}%
\pgfsys@transformshift{6.182938in}{1.477952in}%
\pgfsys@useobject{currentmarker}{}%
\end{pgfscope}%
\begin{pgfscope}%
\pgfsys@transformshift{6.200532in}{1.491603in}%
\pgfsys@useobject{currentmarker}{}%
\end{pgfscope}%
\begin{pgfscope}%
\pgfsys@transformshift{6.218126in}{1.506950in}%
\pgfsys@useobject{currentmarker}{}%
\end{pgfscope}%
\begin{pgfscope}%
\pgfsys@transformshift{6.235720in}{1.477936in}%
\pgfsys@useobject{currentmarker}{}%
\end{pgfscope}%
\begin{pgfscope}%
\pgfsys@transformshift{6.253314in}{1.464209in}%
\pgfsys@useobject{currentmarker}{}%
\end{pgfscope}%
\begin{pgfscope}%
\pgfsys@transformshift{6.270907in}{1.312049in}%
\pgfsys@useobject{currentmarker}{}%
\end{pgfscope}%
\begin{pgfscope}%
\pgfsys@transformshift{6.288501in}{1.587727in}%
\pgfsys@useobject{currentmarker}{}%
\end{pgfscope}%
\begin{pgfscope}%
\pgfsys@transformshift{6.306095in}{1.567807in}%
\pgfsys@useobject{currentmarker}{}%
\end{pgfscope}%
\begin{pgfscope}%
\pgfsys@transformshift{6.323689in}{1.362703in}%
\pgfsys@useobject{currentmarker}{}%
\end{pgfscope}%
\begin{pgfscope}%
\pgfsys@transformshift{6.341283in}{1.284679in}%
\pgfsys@useobject{currentmarker}{}%
\end{pgfscope}%
\begin{pgfscope}%
\pgfsys@transformshift{6.358877in}{1.476755in}%
\pgfsys@useobject{currentmarker}{}%
\end{pgfscope}%
\begin{pgfscope}%
\pgfsys@transformshift{6.376471in}{1.355333in}%
\pgfsys@useobject{currentmarker}{}%
\end{pgfscope}%
\end{pgfscope}%
\begin{pgfscope}%
\pgfpathrectangle{\pgfqpoint{2.000000in}{0.750000in}}{\pgfqpoint{4.376471in}{0.978947in}} %
\pgfusepath{clip}%
\pgfsetroundcap%
\pgfsetroundjoin%
\pgfsetlinewidth{1.756562pt}%
\definecolor{currentstroke}{rgb}{0.298039,0.447059,0.690196}%
\pgfsetstrokecolor{currentstroke}%
\pgfsetdash{}{0pt}%
\pgfpathmoveto{\pgfqpoint{2.925406in}{1.738947in}}%
\pgfpathlineto{\pgfqpoint{2.928076in}{1.698945in}}%
\pgfpathlineto{\pgfqpoint{2.945670in}{1.522794in}}%
\pgfpathlineto{\pgfqpoint{2.963263in}{1.411159in}}%
\pgfpathlineto{\pgfqpoint{2.980857in}{1.345555in}}%
\pgfpathlineto{\pgfqpoint{2.998451in}{1.311325in}}%
\pgfpathlineto{\pgfqpoint{3.016045in}{1.297080in}}%
\pgfpathlineto{\pgfqpoint{3.033639in}{1.294202in}}%
\pgfpathlineto{\pgfqpoint{3.086420in}{1.299926in}}%
\pgfpathlineto{\pgfqpoint{3.104014in}{1.296816in}}%
\pgfpathlineto{\pgfqpoint{3.121608in}{1.289253in}}%
\pgfpathlineto{\pgfqpoint{3.139202in}{1.277291in}}%
\pgfpathlineto{\pgfqpoint{3.156796in}{1.261497in}}%
\pgfpathlineto{\pgfqpoint{3.174390in}{1.242776in}}%
\pgfpathlineto{\pgfqpoint{3.244765in}{1.161402in}}%
\pgfpathlineto{\pgfqpoint{3.262359in}{1.145009in}}%
\pgfpathlineto{\pgfqpoint{3.279953in}{1.132032in}}%
\pgfpathlineto{\pgfqpoint{3.297547in}{1.123093in}}%
\pgfpathlineto{\pgfqpoint{3.315140in}{1.118619in}}%
\pgfpathlineto{\pgfqpoint{3.332734in}{1.118844in}}%
\pgfpathlineto{\pgfqpoint{3.350328in}{1.123817in}}%
\pgfpathlineto{\pgfqpoint{3.367922in}{1.133415in}}%
\pgfpathlineto{\pgfqpoint{3.385516in}{1.147365in}}%
\pgfpathlineto{\pgfqpoint{3.403110in}{1.165264in}}%
\pgfpathlineto{\pgfqpoint{3.420704in}{1.186602in}}%
\pgfpathlineto{\pgfqpoint{3.438297in}{1.210788in}}%
\pgfpathlineto{\pgfqpoint{3.473485in}{1.265080in}}%
\pgfpathlineto{\pgfqpoint{3.543860in}{1.378329in}}%
\pgfpathlineto{\pgfqpoint{3.561454in}{1.403910in}}%
\pgfpathlineto{\pgfqpoint{3.579048in}{1.427326in}}%
\pgfpathlineto{\pgfqpoint{3.596642in}{1.448160in}}%
\pgfpathlineto{\pgfqpoint{3.614236in}{1.466071in}}%
\pgfpathlineto{\pgfqpoint{3.631830in}{1.480799in}}%
\pgfpathlineto{\pgfqpoint{3.649424in}{1.492161in}}%
\pgfpathlineto{\pgfqpoint{3.667017in}{1.500055in}}%
\pgfpathlineto{\pgfqpoint{3.684611in}{1.504449in}}%
\pgfpathlineto{\pgfqpoint{3.702205in}{1.505381in}}%
\pgfpathlineto{\pgfqpoint{3.719799in}{1.502950in}}%
\pgfpathlineto{\pgfqpoint{3.737393in}{1.497308in}}%
\pgfpathlineto{\pgfqpoint{3.754987in}{1.488656in}}%
\pgfpathlineto{\pgfqpoint{3.772581in}{1.477228in}}%
\pgfpathlineto{\pgfqpoint{3.790174in}{1.463292in}}%
\pgfpathlineto{\pgfqpoint{3.807768in}{1.447132in}}%
\pgfpathlineto{\pgfqpoint{3.825362in}{1.429047in}}%
\pgfpathlineto{\pgfqpoint{3.860550in}{1.388318in}}%
\pgfpathlineto{\pgfqpoint{3.895738in}{1.343475in}}%
\pgfpathlineto{\pgfqpoint{4.018894in}{1.182333in}}%
\pgfpathlineto{\pgfqpoint{4.054082in}{1.140668in}}%
\pgfpathlineto{\pgfqpoint{4.089270in}{1.102324in}}%
\pgfpathlineto{\pgfqpoint{4.124458in}{1.067591in}}%
\pgfpathlineto{\pgfqpoint{4.159645in}{1.036605in}}%
\pgfpathlineto{\pgfqpoint{4.194833in}{1.009450in}}%
\pgfpathlineto{\pgfqpoint{4.230021in}{0.986226in}}%
\pgfpathlineto{\pgfqpoint{4.265208in}{0.967117in}}%
\pgfpathlineto{\pgfqpoint{4.300396in}{0.952409in}}%
\pgfpathlineto{\pgfqpoint{4.317990in}{0.946825in}}%
\pgfpathlineto{\pgfqpoint{4.335584in}{0.942495in}}%
\pgfpathlineto{\pgfqpoint{4.353178in}{0.939478in}}%
\pgfpathlineto{\pgfqpoint{4.370772in}{0.937837in}}%
\pgfpathlineto{\pgfqpoint{4.388365in}{0.937630in}}%
\pgfpathlineto{\pgfqpoint{4.405959in}{0.938914in}}%
\pgfpathlineto{\pgfqpoint{4.423553in}{0.941738in}}%
\pgfpathlineto{\pgfqpoint{4.441147in}{0.946144in}}%
\pgfpathlineto{\pgfqpoint{4.458741in}{0.952161in}}%
\pgfpathlineto{\pgfqpoint{4.476335in}{0.959806in}}%
\pgfpathlineto{\pgfqpoint{4.493928in}{0.969079in}}%
\pgfpathlineto{\pgfqpoint{4.511522in}{0.979960in}}%
\pgfpathlineto{\pgfqpoint{4.529116in}{0.992412in}}%
\pgfpathlineto{\pgfqpoint{4.546710in}{1.006376in}}%
\pgfpathlineto{\pgfqpoint{4.581898in}{1.038488in}}%
\pgfpathlineto{\pgfqpoint{4.617085in}{1.075371in}}%
\pgfpathlineto{\pgfqpoint{4.652273in}{1.115750in}}%
\pgfpathlineto{\pgfqpoint{4.757836in}{1.240935in}}%
\pgfpathlineto{\pgfqpoint{4.793024in}{1.277582in}}%
\pgfpathlineto{\pgfqpoint{4.810618in}{1.293859in}}%
\pgfpathlineto{\pgfqpoint{4.828212in}{1.308476in}}%
\pgfpathlineto{\pgfqpoint{4.845805in}{1.321233in}}%
\pgfpathlineto{\pgfqpoint{4.863399in}{1.331953in}}%
\pgfpathlineto{\pgfqpoint{4.880993in}{1.340482in}}%
\pgfpathlineto{\pgfqpoint{4.898587in}{1.346695in}}%
\pgfpathlineto{\pgfqpoint{4.916181in}{1.350496in}}%
\pgfpathlineto{\pgfqpoint{4.933775in}{1.351823in}}%
\pgfpathlineto{\pgfqpoint{4.951369in}{1.350647in}}%
\pgfpathlineto{\pgfqpoint{4.968962in}{1.346973in}}%
\pgfpathlineto{\pgfqpoint{4.986556in}{1.340843in}}%
\pgfpathlineto{\pgfqpoint{5.004150in}{1.332333in}}%
\pgfpathlineto{\pgfqpoint{5.021744in}{1.321552in}}%
\pgfpathlineto{\pgfqpoint{5.039338in}{1.308642in}}%
\pgfpathlineto{\pgfqpoint{5.056932in}{1.293778in}}%
\pgfpathlineto{\pgfqpoint{5.092119in}{1.259014in}}%
\pgfpathlineto{\pgfqpoint{5.127307in}{1.219153in}}%
\pgfpathlineto{\pgfqpoint{5.250464in}{1.071997in}}%
\pgfpathlineto{\pgfqpoint{5.285652in}{1.037174in}}%
\pgfpathlineto{\pgfqpoint{5.303246in}{1.022232in}}%
\pgfpathlineto{\pgfqpoint{5.320839in}{1.009202in}}%
\pgfpathlineto{\pgfqpoint{5.338433in}{0.998248in}}%
\pgfpathlineto{\pgfqpoint{5.356027in}{0.989506in}}%
\pgfpathlineto{\pgfqpoint{5.373621in}{0.983084in}}%
\pgfpathlineto{\pgfqpoint{5.391215in}{0.979057in}}%
\pgfpathlineto{\pgfqpoint{5.408809in}{0.977470in}}%
\pgfpathlineto{\pgfqpoint{5.426403in}{0.978340in}}%
\pgfpathlineto{\pgfqpoint{5.443996in}{0.981652in}}%
\pgfpathlineto{\pgfqpoint{5.461590in}{0.987366in}}%
\pgfpathlineto{\pgfqpoint{5.479184in}{0.995414in}}%
\pgfpathlineto{\pgfqpoint{5.496778in}{1.005707in}}%
\pgfpathlineto{\pgfqpoint{5.514372in}{1.018132in}}%
\pgfpathlineto{\pgfqpoint{5.531966in}{1.032560in}}%
\pgfpathlineto{\pgfqpoint{5.549560in}{1.048845in}}%
\pgfpathlineto{\pgfqpoint{5.584747in}{1.086341in}}%
\pgfpathlineto{\pgfqpoint{5.619935in}{1.129250in}}%
\pgfpathlineto{\pgfqpoint{5.655123in}{1.176127in}}%
\pgfpathlineto{\pgfqpoint{5.707904in}{1.250726in}}%
\pgfpathlineto{\pgfqpoint{5.778280in}{1.351050in}}%
\pgfpathlineto{\pgfqpoint{5.813467in}{1.398775in}}%
\pgfpathlineto{\pgfqpoint{5.848655in}{1.443354in}}%
\pgfpathlineto{\pgfqpoint{5.883843in}{1.483722in}}%
\pgfpathlineto{\pgfqpoint{5.919030in}{1.518855in}}%
\pgfpathlineto{\pgfqpoint{5.936624in}{1.534155in}}%
\pgfpathlineto{\pgfqpoint{5.954218in}{1.547791in}}%
\pgfpathlineto{\pgfqpoint{5.971812in}{1.559659in}}%
\pgfpathlineto{\pgfqpoint{5.989406in}{1.569664in}}%
\pgfpathlineto{\pgfqpoint{6.007000in}{1.577723in}}%
\pgfpathlineto{\pgfqpoint{6.024594in}{1.583771in}}%
\pgfpathlineto{\pgfqpoint{6.042187in}{1.587762in}}%
\pgfpathlineto{\pgfqpoint{6.059781in}{1.589671in}}%
\pgfpathlineto{\pgfqpoint{6.077375in}{1.589499in}}%
\pgfpathlineto{\pgfqpoint{6.094969in}{1.587272in}}%
\pgfpathlineto{\pgfqpoint{6.112563in}{1.583046in}}%
\pgfpathlineto{\pgfqpoint{6.130157in}{1.576899in}}%
\pgfpathlineto{\pgfqpoint{6.147751in}{1.568930in}}%
\pgfpathlineto{\pgfqpoint{6.165344in}{1.559253in}}%
\pgfpathlineto{\pgfqpoint{6.182938in}{1.547978in}}%
\pgfpathlineto{\pgfqpoint{6.200532in}{1.535203in}}%
\pgfpathlineto{\pgfqpoint{6.218126in}{1.520980in}}%
\pgfpathlineto{\pgfqpoint{6.235720in}{1.505290in}}%
\pgfpathlineto{\pgfqpoint{6.253314in}{1.487997in}}%
\pgfpathlineto{\pgfqpoint{6.270907in}{1.468800in}}%
\pgfpathlineto{\pgfqpoint{6.288501in}{1.447167in}}%
\pgfpathlineto{\pgfqpoint{6.306095in}{1.422255in}}%
\pgfpathlineto{\pgfqpoint{6.323689in}{1.392821in}}%
\pgfpathlineto{\pgfqpoint{6.341283in}{1.357107in}}%
\pgfpathlineto{\pgfqpoint{6.358877in}{1.312716in}}%
\pgfpathlineto{\pgfqpoint{6.376471in}{1.256451in}}%
\pgfpathlineto{\pgfqpoint{6.376471in}{1.256451in}}%
\pgfusepath{stroke}%
\end{pgfscope}%
\begin{pgfscope}%
\pgfpathrectangle{\pgfqpoint{2.000000in}{0.750000in}}{\pgfqpoint{4.376471in}{0.978947in}} %
\pgfusepath{clip}%
\pgfsetbuttcap%
\pgfsetroundjoin%
\pgfsetlinewidth{1.756562pt}%
\definecolor{currentstroke}{rgb}{1.000000,0.647059,0.000000}%
\pgfsetstrokecolor{currentstroke}%
\pgfsetdash{{6.000000pt}{6.000000pt}}{0.000000pt}%
\pgfpathmoveto{\pgfqpoint{2.925450in}{1.738947in}}%
\pgfpathlineto{\pgfqpoint{2.928076in}{1.699495in}}%
\pgfpathlineto{\pgfqpoint{2.945670in}{1.522784in}}%
\pgfpathlineto{\pgfqpoint{2.963263in}{1.410765in}}%
\pgfpathlineto{\pgfqpoint{2.980857in}{1.344940in}}%
\pgfpathlineto{\pgfqpoint{2.998451in}{1.310559in}}%
\pgfpathlineto{\pgfqpoint{3.016045in}{1.296319in}}%
\pgfpathlineto{\pgfqpoint{3.033639in}{1.293505in}}%
\pgfpathlineto{\pgfqpoint{3.086420in}{1.299505in}}%
\pgfpathlineto{\pgfqpoint{3.104014in}{1.296518in}}%
\pgfpathlineto{\pgfqpoint{3.121608in}{1.289198in}}%
\pgfpathlineto{\pgfqpoint{3.139202in}{1.277223in}}%
\pgfpathlineto{\pgfqpoint{3.156796in}{1.261502in}}%
\pgfpathlineto{\pgfqpoint{3.174390in}{1.242867in}}%
\pgfpathlineto{\pgfqpoint{3.244765in}{1.161395in}}%
\pgfpathlineto{\pgfqpoint{3.262359in}{1.144877in}}%
\pgfpathlineto{\pgfqpoint{3.279953in}{1.131856in}}%
\pgfpathlineto{\pgfqpoint{3.297547in}{1.122993in}}%
\pgfpathlineto{\pgfqpoint{3.315140in}{1.118487in}}%
\pgfpathlineto{\pgfqpoint{3.332734in}{1.118723in}}%
\pgfpathlineto{\pgfqpoint{3.350328in}{1.123740in}}%
\pgfpathlineto{\pgfqpoint{3.367922in}{1.133337in}}%
\pgfpathlineto{\pgfqpoint{3.385516in}{1.147453in}}%
\pgfpathlineto{\pgfqpoint{3.403110in}{1.165478in}}%
\pgfpathlineto{\pgfqpoint{3.420704in}{1.187001in}}%
\pgfpathlineto{\pgfqpoint{3.438297in}{1.211349in}}%
\pgfpathlineto{\pgfqpoint{3.473485in}{1.266020in}}%
\pgfpathlineto{\pgfqpoint{3.543860in}{1.380442in}}%
\pgfpathlineto{\pgfqpoint{3.561454in}{1.406259in}}%
\pgfpathlineto{\pgfqpoint{3.579048in}{1.429947in}}%
\pgfpathlineto{\pgfqpoint{3.596642in}{1.451047in}}%
\pgfpathlineto{\pgfqpoint{3.614236in}{1.469109in}}%
\pgfpathlineto{\pgfqpoint{3.631830in}{1.484084in}}%
\pgfpathlineto{\pgfqpoint{3.649424in}{1.495635in}}%
\pgfpathlineto{\pgfqpoint{3.667017in}{1.503590in}}%
\pgfpathlineto{\pgfqpoint{3.684611in}{1.508158in}}%
\pgfpathlineto{\pgfqpoint{3.702205in}{1.509204in}}%
\pgfpathlineto{\pgfqpoint{3.719799in}{1.506807in}}%
\pgfpathlineto{\pgfqpoint{3.737393in}{1.501100in}}%
\pgfpathlineto{\pgfqpoint{3.754987in}{1.492430in}}%
\pgfpathlineto{\pgfqpoint{3.772581in}{1.480996in}}%
\pgfpathlineto{\pgfqpoint{3.790174in}{1.466905in}}%
\pgfpathlineto{\pgfqpoint{3.807768in}{1.450605in}}%
\pgfpathlineto{\pgfqpoint{3.842956in}{1.412564in}}%
\pgfpathlineto{\pgfqpoint{3.878144in}{1.369040in}}%
\pgfpathlineto{\pgfqpoint{3.930925in}{1.298821in}}%
\pgfpathlineto{\pgfqpoint{4.001301in}{1.205514in}}%
\pgfpathlineto{\pgfqpoint{4.036488in}{1.161971in}}%
\pgfpathlineto{\pgfqpoint{4.071676in}{1.121630in}}%
\pgfpathlineto{\pgfqpoint{4.106864in}{1.084828in}}%
\pgfpathlineto{\pgfqpoint{4.142051in}{1.051791in}}%
\pgfpathlineto{\pgfqpoint{4.177239in}{1.022569in}}%
\pgfpathlineto{\pgfqpoint{4.212427in}{0.997303in}}%
\pgfpathlineto{\pgfqpoint{4.247615in}{0.976095in}}%
\pgfpathlineto{\pgfqpoint{4.265208in}{0.967014in}}%
\pgfpathlineto{\pgfqpoint{4.300396in}{0.952353in}}%
\pgfpathlineto{\pgfqpoint{4.317990in}{0.946808in}}%
\pgfpathlineto{\pgfqpoint{4.335584in}{0.942482in}}%
\pgfpathlineto{\pgfqpoint{4.353178in}{0.939467in}}%
\pgfpathlineto{\pgfqpoint{4.370772in}{0.937886in}}%
\pgfpathlineto{\pgfqpoint{4.388365in}{0.937680in}}%
\pgfpathlineto{\pgfqpoint{4.405959in}{0.938956in}}%
\pgfpathlineto{\pgfqpoint{4.423553in}{0.941732in}}%
\pgfpathlineto{\pgfqpoint{4.441147in}{0.946189in}}%
\pgfpathlineto{\pgfqpoint{4.458741in}{0.952188in}}%
\pgfpathlineto{\pgfqpoint{4.476335in}{0.959794in}}%
\pgfpathlineto{\pgfqpoint{4.493928in}{0.969080in}}%
\pgfpathlineto{\pgfqpoint{4.511522in}{0.979960in}}%
\pgfpathlineto{\pgfqpoint{4.529116in}{0.992377in}}%
\pgfpathlineto{\pgfqpoint{4.564304in}{1.021698in}}%
\pgfpathlineto{\pgfqpoint{4.599492in}{1.056365in}}%
\pgfpathlineto{\pgfqpoint{4.634679in}{1.095072in}}%
\pgfpathlineto{\pgfqpoint{4.687461in}{1.157959in}}%
\pgfpathlineto{\pgfqpoint{4.740242in}{1.220971in}}%
\pgfpathlineto{\pgfqpoint{4.775430in}{1.259858in}}%
\pgfpathlineto{\pgfqpoint{4.810618in}{1.293929in}}%
\pgfpathlineto{\pgfqpoint{4.828212in}{1.308505in}}%
\pgfpathlineto{\pgfqpoint{4.845805in}{1.321264in}}%
\pgfpathlineto{\pgfqpoint{4.863399in}{1.331982in}}%
\pgfpathlineto{\pgfqpoint{4.880993in}{1.340608in}}%
\pgfpathlineto{\pgfqpoint{4.898587in}{1.346783in}}%
\pgfpathlineto{\pgfqpoint{4.916181in}{1.350604in}}%
\pgfpathlineto{\pgfqpoint{4.933775in}{1.351936in}}%
\pgfpathlineto{\pgfqpoint{4.951369in}{1.350766in}}%
\pgfpathlineto{\pgfqpoint{4.968962in}{1.347156in}}%
\pgfpathlineto{\pgfqpoint{4.986556in}{1.340907in}}%
\pgfpathlineto{\pgfqpoint{5.004150in}{1.332455in}}%
\pgfpathlineto{\pgfqpoint{5.021744in}{1.321575in}}%
\pgfpathlineto{\pgfqpoint{5.039338in}{1.308742in}}%
\pgfpathlineto{\pgfqpoint{5.056932in}{1.293779in}}%
\pgfpathlineto{\pgfqpoint{5.074526in}{1.277211in}}%
\pgfpathlineto{\pgfqpoint{5.109713in}{1.239568in}}%
\pgfpathlineto{\pgfqpoint{5.144901in}{1.197967in}}%
\pgfpathlineto{\pgfqpoint{5.232870in}{1.091238in}}%
\pgfpathlineto{\pgfqpoint{5.268058in}{1.053733in}}%
\pgfpathlineto{\pgfqpoint{5.285652in}{1.037090in}}%
\pgfpathlineto{\pgfqpoint{5.303246in}{1.022115in}}%
\pgfpathlineto{\pgfqpoint{5.320839in}{1.009107in}}%
\pgfpathlineto{\pgfqpoint{5.338433in}{0.998227in}}%
\pgfpathlineto{\pgfqpoint{5.356027in}{0.989439in}}%
\pgfpathlineto{\pgfqpoint{5.373621in}{0.983115in}}%
\pgfpathlineto{\pgfqpoint{5.391215in}{0.979045in}}%
\pgfpathlineto{\pgfqpoint{5.408809in}{0.977477in}}%
\pgfpathlineto{\pgfqpoint{5.426403in}{0.978385in}}%
\pgfpathlineto{\pgfqpoint{5.443996in}{0.981721in}}%
\pgfpathlineto{\pgfqpoint{5.461590in}{0.987398in}}%
\pgfpathlineto{\pgfqpoint{5.479184in}{0.995563in}}%
\pgfpathlineto{\pgfqpoint{5.496778in}{1.005771in}}%
\pgfpathlineto{\pgfqpoint{5.514372in}{1.018231in}}%
\pgfpathlineto{\pgfqpoint{5.531966in}{1.032621in}}%
\pgfpathlineto{\pgfqpoint{5.549560in}{1.048915in}}%
\pgfpathlineto{\pgfqpoint{5.584747in}{1.086396in}}%
\pgfpathlineto{\pgfqpoint{5.619935in}{1.129317in}}%
\pgfpathlineto{\pgfqpoint{5.655123in}{1.176109in}}%
\pgfpathlineto{\pgfqpoint{5.707904in}{1.250734in}}%
\pgfpathlineto{\pgfqpoint{5.795873in}{1.375127in}}%
\pgfpathlineto{\pgfqpoint{5.831061in}{1.421445in}}%
\pgfpathlineto{\pgfqpoint{5.866249in}{1.464055in}}%
\pgfpathlineto{\pgfqpoint{5.901437in}{1.502009in}}%
\pgfpathlineto{\pgfqpoint{5.919030in}{1.518888in}}%
\pgfpathlineto{\pgfqpoint{5.936624in}{1.534211in}}%
\pgfpathlineto{\pgfqpoint{5.954218in}{1.547904in}}%
\pgfpathlineto{\pgfqpoint{5.971812in}{1.559655in}}%
\pgfpathlineto{\pgfqpoint{5.989406in}{1.569726in}}%
\pgfpathlineto{\pgfqpoint{6.007000in}{1.577792in}}%
\pgfpathlineto{\pgfqpoint{6.024594in}{1.583742in}}%
\pgfpathlineto{\pgfqpoint{6.042187in}{1.587850in}}%
\pgfpathlineto{\pgfqpoint{6.059781in}{1.589729in}}%
\pgfpathlineto{\pgfqpoint{6.077375in}{1.589505in}}%
\pgfpathlineto{\pgfqpoint{6.094969in}{1.587215in}}%
\pgfpathlineto{\pgfqpoint{6.112563in}{1.583045in}}%
\pgfpathlineto{\pgfqpoint{6.130157in}{1.576821in}}%
\pgfpathlineto{\pgfqpoint{6.147751in}{1.568792in}}%
\pgfpathlineto{\pgfqpoint{6.165344in}{1.559157in}}%
\pgfpathlineto{\pgfqpoint{6.182938in}{1.547904in}}%
\pgfpathlineto{\pgfqpoint{6.200532in}{1.535158in}}%
\pgfpathlineto{\pgfqpoint{6.218126in}{1.520992in}}%
\pgfpathlineto{\pgfqpoint{6.235720in}{1.505320in}}%
\pgfpathlineto{\pgfqpoint{6.253314in}{1.488129in}}%
\pgfpathlineto{\pgfqpoint{6.270907in}{1.468909in}}%
\pgfpathlineto{\pgfqpoint{6.288501in}{1.447225in}}%
\pgfpathlineto{\pgfqpoint{6.306095in}{1.422155in}}%
\pgfpathlineto{\pgfqpoint{6.323689in}{1.392274in}}%
\pgfpathlineto{\pgfqpoint{6.341283in}{1.355782in}}%
\pgfpathlineto{\pgfqpoint{6.358877in}{1.310235in}}%
\pgfpathlineto{\pgfqpoint{6.376471in}{1.252066in}}%
\pgfpathlineto{\pgfqpoint{6.376471in}{1.252066in}}%
\pgfusepath{stroke}%
\end{pgfscope}%
\begin{pgfscope}%
\pgfsetrectcap%
\pgfsetmiterjoin%
\pgfsetlinewidth{1.003750pt}%
\definecolor{currentstroke}{rgb}{0.800000,0.800000,0.800000}%
\pgfsetstrokecolor{currentstroke}%
\pgfsetdash{}{0pt}%
\pgfpathmoveto{\pgfqpoint{2.000000in}{0.750000in}}%
\pgfpathlineto{\pgfqpoint{2.000000in}{1.728947in}}%
\pgfusepath{stroke}%
\end{pgfscope}%
\begin{pgfscope}%
\pgfsetrectcap%
\pgfsetmiterjoin%
\pgfsetlinewidth{1.003750pt}%
\definecolor{currentstroke}{rgb}{0.800000,0.800000,0.800000}%
\pgfsetstrokecolor{currentstroke}%
\pgfsetdash{}{0pt}%
\pgfpathmoveto{\pgfqpoint{6.376471in}{0.750000in}}%
\pgfpathlineto{\pgfqpoint{6.376471in}{1.728947in}}%
\pgfusepath{stroke}%
\end{pgfscope}%
\begin{pgfscope}%
\pgfsetrectcap%
\pgfsetmiterjoin%
\pgfsetlinewidth{1.003750pt}%
\definecolor{currentstroke}{rgb}{0.800000,0.800000,0.800000}%
\pgfsetstrokecolor{currentstroke}%
\pgfsetdash{}{0pt}%
\pgfpathmoveto{\pgfqpoint{2.000000in}{1.728947in}}%
\pgfpathlineto{\pgfqpoint{6.376471in}{1.728947in}}%
\pgfusepath{stroke}%
\end{pgfscope}%
\begin{pgfscope}%
\pgfsetrectcap%
\pgfsetmiterjoin%
\pgfsetlinewidth{1.003750pt}%
\definecolor{currentstroke}{rgb}{0.800000,0.800000,0.800000}%
\pgfsetstrokecolor{currentstroke}%
\pgfsetdash{}{0pt}%
\pgfpathmoveto{\pgfqpoint{2.000000in}{0.750000in}}%
\pgfpathlineto{\pgfqpoint{6.376471in}{0.750000in}}%
\pgfusepath{stroke}%
\end{pgfscope}%
\begin{pgfscope}%
\pgfsetroundcap%
\pgfsetroundjoin%
\pgfsetlinewidth{1.756562pt}%
\definecolor{currentstroke}{rgb}{0.298039,0.447059,0.690196}%
\pgfsetstrokecolor{currentstroke}%
\pgfsetdash{}{0pt}%
\pgfpathmoveto{\pgfqpoint{2.125000in}{1.549004in}}%
\pgfpathlineto{\pgfqpoint{2.402778in}{1.549004in}}%
\pgfusepath{stroke}%
\end{pgfscope}%
\begin{pgfscope}%
\definecolor{textcolor}{rgb}{0.150000,0.150000,0.150000}%
\pgfsetstrokecolor{textcolor}%
\pgfsetfillcolor{textcolor}%
\pgftext[x=2.513889in,y=1.500393in,left,base]{\color{textcolor}\sffamily\fontsize{10.000000}{12.000000}\selectfont \(\displaystyle \widetilde{\Phi}^* \theta\)}%
\end{pgfscope}%
\begin{pgfscope}%
\pgfsetbuttcap%
\pgfsetroundjoin%
\pgfsetlinewidth{1.756562pt}%
\definecolor{currentstroke}{rgb}{1.000000,0.647059,0.000000}%
\pgfsetstrokecolor{currentstroke}%
\pgfsetdash{{6.000000pt}{6.000000pt}}{0.000000pt}%
\pgfpathmoveto{\pgfqpoint{2.125000in}{1.344143in}}%
\pgfpathlineto{\pgfqpoint{2.402778in}{1.344143in}}%
\pgfusepath{stroke}%
\end{pgfscope}%
\begin{pgfscope}%
\definecolor{textcolor}{rgb}{0.150000,0.150000,0.150000}%
\pgfsetstrokecolor{textcolor}%
\pgfsetfillcolor{textcolor}%
\pgftext[x=2.513889in,y=1.295532in,left,base]{\color{textcolor}\sffamily\fontsize{10.000000}{12.000000}\selectfont \(\displaystyle \widetilde{K}u\)}%
\end{pgfscope}%
\begin{pgfscope}%
\pgfsetbuttcap%
\pgfsetroundjoin%
\definecolor{currentfill}{rgb}{1.000000,0.000000,0.000000}%
\pgfsetfillcolor{currentfill}%
\pgfsetlinewidth{2.007500pt}%
\definecolor{currentstroke}{rgb}{1.000000,0.000000,0.000000}%
\pgfsetstrokecolor{currentstroke}%
\pgfsetdash{}{0pt}%
\pgfpathmoveto{\pgfqpoint{2.232832in}{1.135525in}}%
\pgfpathlineto{\pgfqpoint{2.294945in}{1.135525in}}%
\pgfpathmoveto{\pgfqpoint{2.263889in}{1.104468in}}%
\pgfpathlineto{\pgfqpoint{2.263889in}{1.166581in}}%
\pgfusepath{stroke,fill}%
\end{pgfscope}%
\begin{pgfscope}%
\pgfsetbuttcap%
\pgfsetroundjoin%
\definecolor{currentfill}{rgb}{1.000000,0.000000,0.000000}%
\pgfsetfillcolor{currentfill}%
\pgfsetlinewidth{2.007500pt}%
\definecolor{currentstroke}{rgb}{1.000000,0.000000,0.000000}%
\pgfsetstrokecolor{currentstroke}%
\pgfsetdash{}{0pt}%
\pgfpathmoveto{\pgfqpoint{2.232832in}{1.135525in}}%
\pgfpathlineto{\pgfqpoint{2.294945in}{1.135525in}}%
\pgfpathmoveto{\pgfqpoint{2.263889in}{1.104468in}}%
\pgfpathlineto{\pgfqpoint{2.263889in}{1.166581in}}%
\pgfusepath{stroke,fill}%
\end{pgfscope}%
\begin{pgfscope}%
\pgfsetbuttcap%
\pgfsetroundjoin%
\definecolor{currentfill}{rgb}{1.000000,0.000000,0.000000}%
\pgfsetfillcolor{currentfill}%
\pgfsetlinewidth{2.007500pt}%
\definecolor{currentstroke}{rgb}{1.000000,0.000000,0.000000}%
\pgfsetstrokecolor{currentstroke}%
\pgfsetdash{}{0pt}%
\pgfpathmoveto{\pgfqpoint{2.232832in}{1.135525in}}%
\pgfpathlineto{\pgfqpoint{2.294945in}{1.135525in}}%
\pgfpathmoveto{\pgfqpoint{2.263889in}{1.104468in}}%
\pgfpathlineto{\pgfqpoint{2.263889in}{1.166581in}}%
\pgfusepath{stroke,fill}%
\end{pgfscope}%
\begin{pgfscope}%
\definecolor{textcolor}{rgb}{0.150000,0.150000,0.150000}%
\pgfsetstrokecolor{textcolor}%
\pgfsetfillcolor{textcolor}%
\pgftext[x=2.513889in,y=1.099066in,left,base]{\color{textcolor}\sffamily\fontsize{10.000000}{12.000000}\selectfont train}%
\end{pgfscope}%
\begin{pgfscope}%
\pgfsetbuttcap%
\pgfsetroundjoin%
\definecolor{currentfill}{rgb}{0.000000,0.000000,0.000000}%
\pgfsetfillcolor{currentfill}%
\pgfsetlinewidth{0.301125pt}%
\definecolor{currentstroke}{rgb}{0.000000,0.000000,0.000000}%
\pgfsetstrokecolor{currentstroke}%
\pgfsetdash{}{0pt}%
\pgfpathmoveto{\pgfqpoint{2.263889in}{0.923531in}}%
\pgfpathcurveto{\pgfqpoint{2.268007in}{0.923531in}}{\pgfqpoint{2.271957in}{0.925167in}}{\pgfqpoint{2.274869in}{0.928079in}}%
\pgfpathcurveto{\pgfqpoint{2.277781in}{0.930991in}}{\pgfqpoint{2.279417in}{0.934941in}}{\pgfqpoint{2.279417in}{0.939060in}}%
\pgfpathcurveto{\pgfqpoint{2.279417in}{0.943178in}}{\pgfqpoint{2.277781in}{0.947128in}}{\pgfqpoint{2.274869in}{0.950040in}}%
\pgfpathcurveto{\pgfqpoint{2.271957in}{0.952952in}}{\pgfqpoint{2.268007in}{0.954588in}}{\pgfqpoint{2.263889in}{0.954588in}}%
\pgfpathcurveto{\pgfqpoint{2.259771in}{0.954588in}}{\pgfqpoint{2.255821in}{0.952952in}}{\pgfqpoint{2.252909in}{0.950040in}}%
\pgfpathcurveto{\pgfqpoint{2.249997in}{0.947128in}}{\pgfqpoint{2.248361in}{0.943178in}}{\pgfqpoint{2.248361in}{0.939060in}}%
\pgfpathcurveto{\pgfqpoint{2.248361in}{0.934941in}}{\pgfqpoint{2.249997in}{0.930991in}}{\pgfqpoint{2.252909in}{0.928079in}}%
\pgfpathcurveto{\pgfqpoint{2.255821in}{0.925167in}}{\pgfqpoint{2.259771in}{0.923531in}}{\pgfqpoint{2.263889in}{0.923531in}}%
\pgfpathclose%
\pgfusepath{stroke,fill}%
\end{pgfscope}%
\begin{pgfscope}%
\pgfsetbuttcap%
\pgfsetroundjoin%
\definecolor{currentfill}{rgb}{0.000000,0.000000,0.000000}%
\pgfsetfillcolor{currentfill}%
\pgfsetlinewidth{0.301125pt}%
\definecolor{currentstroke}{rgb}{0.000000,0.000000,0.000000}%
\pgfsetstrokecolor{currentstroke}%
\pgfsetdash{}{0pt}%
\pgfpathmoveto{\pgfqpoint{2.263889in}{0.923531in}}%
\pgfpathcurveto{\pgfqpoint{2.268007in}{0.923531in}}{\pgfqpoint{2.271957in}{0.925167in}}{\pgfqpoint{2.274869in}{0.928079in}}%
\pgfpathcurveto{\pgfqpoint{2.277781in}{0.930991in}}{\pgfqpoint{2.279417in}{0.934941in}}{\pgfqpoint{2.279417in}{0.939060in}}%
\pgfpathcurveto{\pgfqpoint{2.279417in}{0.943178in}}{\pgfqpoint{2.277781in}{0.947128in}}{\pgfqpoint{2.274869in}{0.950040in}}%
\pgfpathcurveto{\pgfqpoint{2.271957in}{0.952952in}}{\pgfqpoint{2.268007in}{0.954588in}}{\pgfqpoint{2.263889in}{0.954588in}}%
\pgfpathcurveto{\pgfqpoint{2.259771in}{0.954588in}}{\pgfqpoint{2.255821in}{0.952952in}}{\pgfqpoint{2.252909in}{0.950040in}}%
\pgfpathcurveto{\pgfqpoint{2.249997in}{0.947128in}}{\pgfqpoint{2.248361in}{0.943178in}}{\pgfqpoint{2.248361in}{0.939060in}}%
\pgfpathcurveto{\pgfqpoint{2.248361in}{0.934941in}}{\pgfqpoint{2.249997in}{0.930991in}}{\pgfqpoint{2.252909in}{0.928079in}}%
\pgfpathcurveto{\pgfqpoint{2.255821in}{0.925167in}}{\pgfqpoint{2.259771in}{0.923531in}}{\pgfqpoint{2.263889in}{0.923531in}}%
\pgfpathclose%
\pgfusepath{stroke,fill}%
\end{pgfscope}%
\begin{pgfscope}%
\pgfsetbuttcap%
\pgfsetroundjoin%
\definecolor{currentfill}{rgb}{0.000000,0.000000,0.000000}%
\pgfsetfillcolor{currentfill}%
\pgfsetlinewidth{0.301125pt}%
\definecolor{currentstroke}{rgb}{0.000000,0.000000,0.000000}%
\pgfsetstrokecolor{currentstroke}%
\pgfsetdash{}{0pt}%
\pgfpathmoveto{\pgfqpoint{2.263889in}{0.923531in}}%
\pgfpathcurveto{\pgfqpoint{2.268007in}{0.923531in}}{\pgfqpoint{2.271957in}{0.925167in}}{\pgfqpoint{2.274869in}{0.928079in}}%
\pgfpathcurveto{\pgfqpoint{2.277781in}{0.930991in}}{\pgfqpoint{2.279417in}{0.934941in}}{\pgfqpoint{2.279417in}{0.939060in}}%
\pgfpathcurveto{\pgfqpoint{2.279417in}{0.943178in}}{\pgfqpoint{2.277781in}{0.947128in}}{\pgfqpoint{2.274869in}{0.950040in}}%
\pgfpathcurveto{\pgfqpoint{2.271957in}{0.952952in}}{\pgfqpoint{2.268007in}{0.954588in}}{\pgfqpoint{2.263889in}{0.954588in}}%
\pgfpathcurveto{\pgfqpoint{2.259771in}{0.954588in}}{\pgfqpoint{2.255821in}{0.952952in}}{\pgfqpoint{2.252909in}{0.950040in}}%
\pgfpathcurveto{\pgfqpoint{2.249997in}{0.947128in}}{\pgfqpoint{2.248361in}{0.943178in}}{\pgfqpoint{2.248361in}{0.939060in}}%
\pgfpathcurveto{\pgfqpoint{2.248361in}{0.934941in}}{\pgfqpoint{2.249997in}{0.930991in}}{\pgfqpoint{2.252909in}{0.928079in}}%
\pgfpathcurveto{\pgfqpoint{2.255821in}{0.925167in}}{\pgfqpoint{2.259771in}{0.923531in}}{\pgfqpoint{2.263889in}{0.923531in}}%
\pgfpathclose%
\pgfusepath{stroke,fill}%
\end{pgfscope}%
\begin{pgfscope}%
\definecolor{textcolor}{rgb}{0.150000,0.150000,0.150000}%
\pgfsetstrokecolor{textcolor}%
\pgfsetfillcolor{textcolor}%
\pgftext[x=2.513889in,y=0.902601in,left,base]{\color{textcolor}\sffamily\fontsize{10.000000}{12.000000}\selectfont test}%
\end{pgfscope}%
\begin{pgfscope}%
\pgfsetbuttcap%
\pgfsetmiterjoin%
\definecolor{currentfill}{rgb}{1.000000,1.000000,1.000000}%
\pgfsetfillcolor{currentfill}%
\pgfsetlinewidth{0.000000pt}%
\definecolor{currentstroke}{rgb}{0.000000,0.000000,0.000000}%
\pgfsetstrokecolor{currentstroke}%
\pgfsetstrokeopacity{0.000000}%
\pgfsetdash{}{0pt}%
\pgfpathmoveto{\pgfqpoint{7.105882in}{0.750000in}}%
\pgfpathlineto{\pgfqpoint{11.482353in}{0.750000in}}%
\pgfpathlineto{\pgfqpoint{11.482353in}{1.728947in}}%
\pgfpathlineto{\pgfqpoint{7.105882in}{1.728947in}}%
\pgfpathclose%
\pgfusepath{fill}%
\end{pgfscope}%
\begin{pgfscope}%
\pgfpathrectangle{\pgfqpoint{7.105882in}{0.750000in}}{\pgfqpoint{4.376471in}{0.978947in}} %
\pgfusepath{clip}%
\pgfsetroundcap%
\pgfsetroundjoin%
\pgfsetlinewidth{1.003750pt}%
\definecolor{currentstroke}{rgb}{0.800000,0.800000,0.800000}%
\pgfsetstrokecolor{currentstroke}%
\pgfsetdash{}{0pt}%
\pgfpathmoveto{\pgfqpoint{7.105882in}{0.750000in}}%
\pgfpathlineto{\pgfqpoint{7.105882in}{1.728947in}}%
\pgfusepath{stroke}%
\end{pgfscope}%
\begin{pgfscope}%
\definecolor{textcolor}{rgb}{0.150000,0.150000,0.150000}%
\pgfsetstrokecolor{textcolor}%
\pgfsetfillcolor{textcolor}%
\pgftext[x=7.105882in,y=0.652778in,,top]{\color{textcolor}\sffamily\fontsize{10.000000}{12.000000}\selectfont \(\displaystyle -1.5\)}%
\end{pgfscope}%
\begin{pgfscope}%
\pgfpathrectangle{\pgfqpoint{7.105882in}{0.750000in}}{\pgfqpoint{4.376471in}{0.978947in}} %
\pgfusepath{clip}%
\pgfsetroundcap%
\pgfsetroundjoin%
\pgfsetlinewidth{1.003750pt}%
\definecolor{currentstroke}{rgb}{0.800000,0.800000,0.800000}%
\pgfsetstrokecolor{currentstroke}%
\pgfsetdash{}{0pt}%
\pgfpathmoveto{\pgfqpoint{7.981176in}{0.750000in}}%
\pgfpathlineto{\pgfqpoint{7.981176in}{1.728947in}}%
\pgfusepath{stroke}%
\end{pgfscope}%
\begin{pgfscope}%
\definecolor{textcolor}{rgb}{0.150000,0.150000,0.150000}%
\pgfsetstrokecolor{textcolor}%
\pgfsetfillcolor{textcolor}%
\pgftext[x=7.981176in,y=0.652778in,,top]{\color{textcolor}\sffamily\fontsize{10.000000}{12.000000}\selectfont \(\displaystyle -1.0\)}%
\end{pgfscope}%
\begin{pgfscope}%
\pgfpathrectangle{\pgfqpoint{7.105882in}{0.750000in}}{\pgfqpoint{4.376471in}{0.978947in}} %
\pgfusepath{clip}%
\pgfsetroundcap%
\pgfsetroundjoin%
\pgfsetlinewidth{1.003750pt}%
\definecolor{currentstroke}{rgb}{0.800000,0.800000,0.800000}%
\pgfsetstrokecolor{currentstroke}%
\pgfsetdash{}{0pt}%
\pgfpathmoveto{\pgfqpoint{8.856471in}{0.750000in}}%
\pgfpathlineto{\pgfqpoint{8.856471in}{1.728947in}}%
\pgfusepath{stroke}%
\end{pgfscope}%
\begin{pgfscope}%
\definecolor{textcolor}{rgb}{0.150000,0.150000,0.150000}%
\pgfsetstrokecolor{textcolor}%
\pgfsetfillcolor{textcolor}%
\pgftext[x=8.856471in,y=0.652778in,,top]{\color{textcolor}\sffamily\fontsize{10.000000}{12.000000}\selectfont \(\displaystyle -0.5\)}%
\end{pgfscope}%
\begin{pgfscope}%
\pgfpathrectangle{\pgfqpoint{7.105882in}{0.750000in}}{\pgfqpoint{4.376471in}{0.978947in}} %
\pgfusepath{clip}%
\pgfsetroundcap%
\pgfsetroundjoin%
\pgfsetlinewidth{1.003750pt}%
\definecolor{currentstroke}{rgb}{0.800000,0.800000,0.800000}%
\pgfsetstrokecolor{currentstroke}%
\pgfsetdash{}{0pt}%
\pgfpathmoveto{\pgfqpoint{9.731765in}{0.750000in}}%
\pgfpathlineto{\pgfqpoint{9.731765in}{1.728947in}}%
\pgfusepath{stroke}%
\end{pgfscope}%
\begin{pgfscope}%
\definecolor{textcolor}{rgb}{0.150000,0.150000,0.150000}%
\pgfsetstrokecolor{textcolor}%
\pgfsetfillcolor{textcolor}%
\pgftext[x=9.731765in,y=0.652778in,,top]{\color{textcolor}\sffamily\fontsize{10.000000}{12.000000}\selectfont \(\displaystyle 0.0\)}%
\end{pgfscope}%
\begin{pgfscope}%
\pgfpathrectangle{\pgfqpoint{7.105882in}{0.750000in}}{\pgfqpoint{4.376471in}{0.978947in}} %
\pgfusepath{clip}%
\pgfsetroundcap%
\pgfsetroundjoin%
\pgfsetlinewidth{1.003750pt}%
\definecolor{currentstroke}{rgb}{0.800000,0.800000,0.800000}%
\pgfsetstrokecolor{currentstroke}%
\pgfsetdash{}{0pt}%
\pgfpathmoveto{\pgfqpoint{10.607059in}{0.750000in}}%
\pgfpathlineto{\pgfqpoint{10.607059in}{1.728947in}}%
\pgfusepath{stroke}%
\end{pgfscope}%
\begin{pgfscope}%
\definecolor{textcolor}{rgb}{0.150000,0.150000,0.150000}%
\pgfsetstrokecolor{textcolor}%
\pgfsetfillcolor{textcolor}%
\pgftext[x=10.607059in,y=0.652778in,,top]{\color{textcolor}\sffamily\fontsize{10.000000}{12.000000}\selectfont \(\displaystyle 0.5\)}%
\end{pgfscope}%
\begin{pgfscope}%
\pgfpathrectangle{\pgfqpoint{7.105882in}{0.750000in}}{\pgfqpoint{4.376471in}{0.978947in}} %
\pgfusepath{clip}%
\pgfsetroundcap%
\pgfsetroundjoin%
\pgfsetlinewidth{1.003750pt}%
\definecolor{currentstroke}{rgb}{0.800000,0.800000,0.800000}%
\pgfsetstrokecolor{currentstroke}%
\pgfsetdash{}{0pt}%
\pgfpathmoveto{\pgfqpoint{11.482353in}{0.750000in}}%
\pgfpathlineto{\pgfqpoint{11.482353in}{1.728947in}}%
\pgfusepath{stroke}%
\end{pgfscope}%
\begin{pgfscope}%
\definecolor{textcolor}{rgb}{0.150000,0.150000,0.150000}%
\pgfsetstrokecolor{textcolor}%
\pgfsetfillcolor{textcolor}%
\pgftext[x=11.482353in,y=0.652778in,,top]{\color{textcolor}\sffamily\fontsize{10.000000}{12.000000}\selectfont \(\displaystyle 1.0\)}%
\end{pgfscope}%
\begin{pgfscope}%
\definecolor{textcolor}{rgb}{0.150000,0.150000,0.150000}%
\pgfsetstrokecolor{textcolor}%
\pgfsetfillcolor{textcolor}%
\pgftext[x=9.294118in,y=0.456313in,,top]{\color{textcolor}\sffamily\fontsize{11.000000}{13.200000}\selectfont x}%
\end{pgfscope}%
\begin{pgfscope}%
\pgfpathrectangle{\pgfqpoint{7.105882in}{0.750000in}}{\pgfqpoint{4.376471in}{0.978947in}} %
\pgfusepath{clip}%
\pgfsetroundcap%
\pgfsetroundjoin%
\pgfsetlinewidth{1.003750pt}%
\definecolor{currentstroke}{rgb}{0.800000,0.800000,0.800000}%
\pgfsetstrokecolor{currentstroke}%
\pgfsetdash{}{0pt}%
\pgfpathmoveto{\pgfqpoint{7.105882in}{0.913158in}}%
\pgfpathlineto{\pgfqpoint{11.482353in}{0.913158in}}%
\pgfusepath{stroke}%
\end{pgfscope}%
\begin{pgfscope}%
\definecolor{textcolor}{rgb}{0.150000,0.150000,0.150000}%
\pgfsetstrokecolor{textcolor}%
\pgfsetfillcolor{textcolor}%
\pgftext[x=7.008660in,y=0.913158in,right,]{\color{textcolor}\sffamily\fontsize{10.000000}{12.000000}\selectfont \(\displaystyle -1\)}%
\end{pgfscope}%
\begin{pgfscope}%
\pgfpathrectangle{\pgfqpoint{7.105882in}{0.750000in}}{\pgfqpoint{4.376471in}{0.978947in}} %
\pgfusepath{clip}%
\pgfsetroundcap%
\pgfsetroundjoin%
\pgfsetlinewidth{1.003750pt}%
\definecolor{currentstroke}{rgb}{0.800000,0.800000,0.800000}%
\pgfsetstrokecolor{currentstroke}%
\pgfsetdash{}{0pt}%
\pgfpathmoveto{\pgfqpoint{7.105882in}{1.117105in}}%
\pgfpathlineto{\pgfqpoint{11.482353in}{1.117105in}}%
\pgfusepath{stroke}%
\end{pgfscope}%
\begin{pgfscope}%
\definecolor{textcolor}{rgb}{0.150000,0.150000,0.150000}%
\pgfsetstrokecolor{textcolor}%
\pgfsetfillcolor{textcolor}%
\pgftext[x=7.008660in,y=1.117105in,right,]{\color{textcolor}\sffamily\fontsize{10.000000}{12.000000}\selectfont \(\displaystyle 0\)}%
\end{pgfscope}%
\begin{pgfscope}%
\pgfpathrectangle{\pgfqpoint{7.105882in}{0.750000in}}{\pgfqpoint{4.376471in}{0.978947in}} %
\pgfusepath{clip}%
\pgfsetroundcap%
\pgfsetroundjoin%
\pgfsetlinewidth{1.003750pt}%
\definecolor{currentstroke}{rgb}{0.800000,0.800000,0.800000}%
\pgfsetstrokecolor{currentstroke}%
\pgfsetdash{}{0pt}%
\pgfpathmoveto{\pgfqpoint{7.105882in}{1.321053in}}%
\pgfpathlineto{\pgfqpoint{11.482353in}{1.321053in}}%
\pgfusepath{stroke}%
\end{pgfscope}%
\begin{pgfscope}%
\definecolor{textcolor}{rgb}{0.150000,0.150000,0.150000}%
\pgfsetstrokecolor{textcolor}%
\pgfsetfillcolor{textcolor}%
\pgftext[x=7.008660in,y=1.321053in,right,]{\color{textcolor}\sffamily\fontsize{10.000000}{12.000000}\selectfont \(\displaystyle 1\)}%
\end{pgfscope}%
\begin{pgfscope}%
\pgfpathrectangle{\pgfqpoint{7.105882in}{0.750000in}}{\pgfqpoint{4.376471in}{0.978947in}} %
\pgfusepath{clip}%
\pgfsetroundcap%
\pgfsetroundjoin%
\pgfsetlinewidth{1.003750pt}%
\definecolor{currentstroke}{rgb}{0.800000,0.800000,0.800000}%
\pgfsetstrokecolor{currentstroke}%
\pgfsetdash{}{0pt}%
\pgfpathmoveto{\pgfqpoint{7.105882in}{1.525000in}}%
\pgfpathlineto{\pgfqpoint{11.482353in}{1.525000in}}%
\pgfusepath{stroke}%
\end{pgfscope}%
\begin{pgfscope}%
\definecolor{textcolor}{rgb}{0.150000,0.150000,0.150000}%
\pgfsetstrokecolor{textcolor}%
\pgfsetfillcolor{textcolor}%
\pgftext[x=7.008660in,y=1.525000in,right,]{\color{textcolor}\sffamily\fontsize{10.000000}{12.000000}\selectfont \(\displaystyle 2\)}%
\end{pgfscope}%
\begin{pgfscope}%
\pgfpathrectangle{\pgfqpoint{7.105882in}{0.750000in}}{\pgfqpoint{4.376471in}{0.978947in}} %
\pgfusepath{clip}%
\pgfsetroundcap%
\pgfsetroundjoin%
\pgfsetlinewidth{1.003750pt}%
\definecolor{currentstroke}{rgb}{0.800000,0.800000,0.800000}%
\pgfsetstrokecolor{currentstroke}%
\pgfsetdash{}{0pt}%
\pgfpathmoveto{\pgfqpoint{7.105882in}{1.728947in}}%
\pgfpathlineto{\pgfqpoint{11.482353in}{1.728947in}}%
\pgfusepath{stroke}%
\end{pgfscope}%
\begin{pgfscope}%
\definecolor{textcolor}{rgb}{0.150000,0.150000,0.150000}%
\pgfsetstrokecolor{textcolor}%
\pgfsetfillcolor{textcolor}%
\pgftext[x=7.008660in,y=1.728947in,right,]{\color{textcolor}\sffamily\fontsize{10.000000}{12.000000}\selectfont \(\displaystyle 3\)}%
\end{pgfscope}%
\begin{pgfscope}%
\pgfpathrectangle{\pgfqpoint{7.105882in}{0.750000in}}{\pgfqpoint{4.376471in}{0.978947in}} %
\pgfusepath{clip}%
\pgfsetbuttcap%
\pgfsetroundjoin%
\definecolor{currentfill}{rgb}{1.000000,0.000000,0.000000}%
\pgfsetfillcolor{currentfill}%
\pgfsetlinewidth{2.007500pt}%
\definecolor{currentstroke}{rgb}{1.000000,0.000000,0.000000}%
\pgfsetstrokecolor{currentstroke}%
\pgfsetdash{}{0pt}%
\pgfpathmoveto{\pgfqpoint{9.871613in}{1.307576in}}%
\pgfpathlineto{\pgfqpoint{9.933726in}{1.307576in}}%
\pgfpathmoveto{\pgfqpoint{9.902669in}{1.276519in}}%
\pgfpathlineto{\pgfqpoint{9.902669in}{1.338632in}}%
\pgfusepath{stroke,fill}%
\end{pgfscope}%
\begin{pgfscope}%
\pgfpathrectangle{\pgfqpoint{7.105882in}{0.750000in}}{\pgfqpoint{4.376471in}{0.978947in}} %
\pgfusepath{clip}%
\pgfsetbuttcap%
\pgfsetroundjoin%
\definecolor{currentfill}{rgb}{1.000000,0.000000,0.000000}%
\pgfsetfillcolor{currentfill}%
\pgfsetlinewidth{2.007500pt}%
\definecolor{currentstroke}{rgb}{1.000000,0.000000,0.000000}%
\pgfsetstrokecolor{currentstroke}%
\pgfsetdash{}{0pt}%
\pgfpathmoveto{\pgfqpoint{10.454124in}{0.990364in}}%
\pgfpathlineto{\pgfqpoint{10.516237in}{0.990364in}}%
\pgfpathmoveto{\pgfqpoint{10.485181in}{0.959307in}}%
\pgfpathlineto{\pgfqpoint{10.485181in}{1.021420in}}%
\pgfusepath{stroke,fill}%
\end{pgfscope}%
\begin{pgfscope}%
\pgfpathrectangle{\pgfqpoint{7.105882in}{0.750000in}}{\pgfqpoint{4.376471in}{0.978947in}} %
\pgfusepath{clip}%
\pgfsetbuttcap%
\pgfsetroundjoin%
\definecolor{currentfill}{rgb}{1.000000,0.000000,0.000000}%
\pgfsetfillcolor{currentfill}%
\pgfsetlinewidth{2.007500pt}%
\definecolor{currentstroke}{rgb}{1.000000,0.000000,0.000000}%
\pgfsetstrokecolor{currentstroke}%
\pgfsetdash{}{0pt}%
\pgfpathmoveto{\pgfqpoint{10.060501in}{1.361104in}}%
\pgfpathlineto{\pgfqpoint{10.122614in}{1.361104in}}%
\pgfpathmoveto{\pgfqpoint{10.091557in}{1.330047in}}%
\pgfpathlineto{\pgfqpoint{10.091557in}{1.392160in}}%
\pgfusepath{stroke,fill}%
\end{pgfscope}%
\begin{pgfscope}%
\pgfpathrectangle{\pgfqpoint{7.105882in}{0.750000in}}{\pgfqpoint{4.376471in}{0.978947in}} %
\pgfusepath{clip}%
\pgfsetbuttcap%
\pgfsetroundjoin%
\definecolor{currentfill}{rgb}{1.000000,0.000000,0.000000}%
\pgfsetfillcolor{currentfill}%
\pgfsetlinewidth{2.007500pt}%
\definecolor{currentstroke}{rgb}{1.000000,0.000000,0.000000}%
\pgfsetstrokecolor{currentstroke}%
\pgfsetdash{}{0pt}%
\pgfpathmoveto{\pgfqpoint{9.857852in}{1.250192in}}%
\pgfpathlineto{\pgfqpoint{9.919965in}{1.250192in}}%
\pgfpathmoveto{\pgfqpoint{9.888909in}{1.219135in}}%
\pgfpathlineto{\pgfqpoint{9.888909in}{1.281248in}}%
\pgfusepath{stroke,fill}%
\end{pgfscope}%
\begin{pgfscope}%
\pgfpathrectangle{\pgfqpoint{7.105882in}{0.750000in}}{\pgfqpoint{4.376471in}{0.978947in}} %
\pgfusepath{clip}%
\pgfsetbuttcap%
\pgfsetroundjoin%
\definecolor{currentfill}{rgb}{1.000000,0.000000,0.000000}%
\pgfsetfillcolor{currentfill}%
\pgfsetlinewidth{2.007500pt}%
\definecolor{currentstroke}{rgb}{1.000000,0.000000,0.000000}%
\pgfsetstrokecolor{currentstroke}%
\pgfsetdash{}{0pt}%
\pgfpathmoveto{\pgfqpoint{9.433410in}{0.923797in}}%
\pgfpathlineto{\pgfqpoint{9.495523in}{0.923797in}}%
\pgfpathmoveto{\pgfqpoint{9.464467in}{0.892741in}}%
\pgfpathlineto{\pgfqpoint{9.464467in}{0.954854in}}%
\pgfusepath{stroke,fill}%
\end{pgfscope}%
\begin{pgfscope}%
\pgfpathrectangle{\pgfqpoint{7.105882in}{0.750000in}}{\pgfqpoint{4.376471in}{0.978947in}} %
\pgfusepath{clip}%
\pgfsetbuttcap%
\pgfsetroundjoin%
\definecolor{currentfill}{rgb}{1.000000,0.000000,0.000000}%
\pgfsetfillcolor{currentfill}%
\pgfsetlinewidth{2.007500pt}%
\definecolor{currentstroke}{rgb}{1.000000,0.000000,0.000000}%
\pgfsetstrokecolor{currentstroke}%
\pgfsetdash{}{0pt}%
\pgfpathmoveto{\pgfqpoint{10.211509in}{1.193262in}}%
\pgfpathlineto{\pgfqpoint{10.273622in}{1.193262in}}%
\pgfpathmoveto{\pgfqpoint{10.242566in}{1.162205in}}%
\pgfpathlineto{\pgfqpoint{10.242566in}{1.224318in}}%
\pgfusepath{stroke,fill}%
\end{pgfscope}%
\begin{pgfscope}%
\pgfpathrectangle{\pgfqpoint{7.105882in}{0.750000in}}{\pgfqpoint{4.376471in}{0.978947in}} %
\pgfusepath{clip}%
\pgfsetbuttcap%
\pgfsetroundjoin%
\definecolor{currentfill}{rgb}{1.000000,0.000000,0.000000}%
\pgfsetfillcolor{currentfill}%
\pgfsetlinewidth{2.007500pt}%
\definecolor{currentstroke}{rgb}{1.000000,0.000000,0.000000}%
\pgfsetstrokecolor{currentstroke}%
\pgfsetdash{}{0pt}%
\pgfpathmoveto{\pgfqpoint{9.482190in}{0.961291in}}%
\pgfpathlineto{\pgfqpoint{9.544303in}{0.961291in}}%
\pgfpathmoveto{\pgfqpoint{9.513247in}{0.930235in}}%
\pgfpathlineto{\pgfqpoint{9.513247in}{0.992348in}}%
\pgfusepath{stroke,fill}%
\end{pgfscope}%
\begin{pgfscope}%
\pgfpathrectangle{\pgfqpoint{7.105882in}{0.750000in}}{\pgfqpoint{4.376471in}{0.978947in}} %
\pgfusepath{clip}%
\pgfsetbuttcap%
\pgfsetroundjoin%
\definecolor{currentfill}{rgb}{1.000000,0.000000,0.000000}%
\pgfsetfillcolor{currentfill}%
\pgfsetlinewidth{2.007500pt}%
\definecolor{currentstroke}{rgb}{1.000000,0.000000,0.000000}%
\pgfsetstrokecolor{currentstroke}%
\pgfsetdash{}{0pt}%
\pgfpathmoveto{\pgfqpoint{11.072375in}{1.601409in}}%
\pgfpathlineto{\pgfqpoint{11.134488in}{1.601409in}}%
\pgfpathmoveto{\pgfqpoint{11.103431in}{1.570353in}}%
\pgfpathlineto{\pgfqpoint{11.103431in}{1.632466in}}%
\pgfusepath{stroke,fill}%
\end{pgfscope}%
\begin{pgfscope}%
\pgfpathrectangle{\pgfqpoint{7.105882in}{0.750000in}}{\pgfqpoint{4.376471in}{0.978947in}} %
\pgfusepath{clip}%
\pgfsetbuttcap%
\pgfsetroundjoin%
\definecolor{currentfill}{rgb}{1.000000,0.000000,0.000000}%
\pgfsetfillcolor{currentfill}%
\pgfsetlinewidth{2.007500pt}%
\definecolor{currentstroke}{rgb}{1.000000,0.000000,0.000000}%
\pgfsetstrokecolor{currentstroke}%
\pgfsetdash{}{0pt}%
\pgfpathmoveto{\pgfqpoint{11.324073in}{1.501812in}}%
\pgfpathlineto{\pgfqpoint{11.386186in}{1.501812in}}%
\pgfpathmoveto{\pgfqpoint{11.355130in}{1.470755in}}%
\pgfpathlineto{\pgfqpoint{11.355130in}{1.532868in}}%
\pgfusepath{stroke,fill}%
\end{pgfscope}%
\begin{pgfscope}%
\pgfpathrectangle{\pgfqpoint{7.105882in}{0.750000in}}{\pgfqpoint{4.376471in}{0.978947in}} %
\pgfusepath{clip}%
\pgfsetbuttcap%
\pgfsetroundjoin%
\definecolor{currentfill}{rgb}{1.000000,0.000000,0.000000}%
\pgfsetfillcolor{currentfill}%
\pgfsetlinewidth{2.007500pt}%
\definecolor{currentstroke}{rgb}{1.000000,0.000000,0.000000}%
\pgfsetstrokecolor{currentstroke}%
\pgfsetdash{}{0pt}%
\pgfpathmoveto{\pgfqpoint{9.292616in}{0.999208in}}%
\pgfpathlineto{\pgfqpoint{9.354729in}{0.999208in}}%
\pgfpathmoveto{\pgfqpoint{9.323673in}{0.968152in}}%
\pgfpathlineto{\pgfqpoint{9.323673in}{1.030265in}}%
\pgfusepath{stroke,fill}%
\end{pgfscope}%
\begin{pgfscope}%
\pgfpathrectangle{\pgfqpoint{7.105882in}{0.750000in}}{\pgfqpoint{4.376471in}{0.978947in}} %
\pgfusepath{clip}%
\pgfsetbuttcap%
\pgfsetroundjoin%
\definecolor{currentfill}{rgb}{1.000000,0.000000,0.000000}%
\pgfsetfillcolor{currentfill}%
\pgfsetlinewidth{2.007500pt}%
\definecolor{currentstroke}{rgb}{1.000000,0.000000,0.000000}%
\pgfsetstrokecolor{currentstroke}%
\pgfsetdash{}{0pt}%
\pgfpathmoveto{\pgfqpoint{10.722089in}{1.149384in}}%
\pgfpathlineto{\pgfqpoint{10.784202in}{1.149384in}}%
\pgfpathmoveto{\pgfqpoint{10.753146in}{1.118327in}}%
\pgfpathlineto{\pgfqpoint{10.753146in}{1.180440in}}%
\pgfusepath{stroke,fill}%
\end{pgfscope}%
\begin{pgfscope}%
\pgfpathrectangle{\pgfqpoint{7.105882in}{0.750000in}}{\pgfqpoint{4.376471in}{0.978947in}} %
\pgfusepath{clip}%
\pgfsetbuttcap%
\pgfsetroundjoin%
\definecolor{currentfill}{rgb}{1.000000,0.000000,0.000000}%
\pgfsetfillcolor{currentfill}%
\pgfsetlinewidth{2.007500pt}%
\definecolor{currentstroke}{rgb}{1.000000,0.000000,0.000000}%
\pgfsetstrokecolor{currentstroke}%
\pgfsetdash{}{0pt}%
\pgfpathmoveto{\pgfqpoint{9.801874in}{1.189479in}}%
\pgfpathlineto{\pgfqpoint{9.863987in}{1.189479in}}%
\pgfpathmoveto{\pgfqpoint{9.832931in}{1.158422in}}%
\pgfpathlineto{\pgfqpoint{9.832931in}{1.220535in}}%
\pgfusepath{stroke,fill}%
\end{pgfscope}%
\begin{pgfscope}%
\pgfpathrectangle{\pgfqpoint{7.105882in}{0.750000in}}{\pgfqpoint{4.376471in}{0.978947in}} %
\pgfusepath{clip}%
\pgfsetbuttcap%
\pgfsetroundjoin%
\definecolor{currentfill}{rgb}{1.000000,0.000000,0.000000}%
\pgfsetfillcolor{currentfill}%
\pgfsetlinewidth{2.007500pt}%
\definecolor{currentstroke}{rgb}{1.000000,0.000000,0.000000}%
\pgfsetstrokecolor{currentstroke}%
\pgfsetdash{}{0pt}%
\pgfpathmoveto{\pgfqpoint{9.938944in}{1.317035in}}%
\pgfpathlineto{\pgfqpoint{10.001057in}{1.317035in}}%
\pgfpathmoveto{\pgfqpoint{9.970001in}{1.285978in}}%
\pgfpathlineto{\pgfqpoint{9.970001in}{1.348091in}}%
\pgfusepath{stroke,fill}%
\end{pgfscope}%
\begin{pgfscope}%
\pgfpathrectangle{\pgfqpoint{7.105882in}{0.750000in}}{\pgfqpoint{4.376471in}{0.978947in}} %
\pgfusepath{clip}%
\pgfsetbuttcap%
\pgfsetroundjoin%
\definecolor{currentfill}{rgb}{1.000000,0.000000,0.000000}%
\pgfsetfillcolor{currentfill}%
\pgfsetlinewidth{2.007500pt}%
\definecolor{currentstroke}{rgb}{1.000000,0.000000,0.000000}%
\pgfsetstrokecolor{currentstroke}%
\pgfsetdash{}{0pt}%
\pgfpathmoveto{\pgfqpoint{11.190797in}{1.577199in}}%
\pgfpathlineto{\pgfqpoint{11.252910in}{1.577199in}}%
\pgfpathmoveto{\pgfqpoint{11.221854in}{1.546142in}}%
\pgfpathlineto{\pgfqpoint{11.221854in}{1.608255in}}%
\pgfusepath{stroke,fill}%
\end{pgfscope}%
\begin{pgfscope}%
\pgfpathrectangle{\pgfqpoint{7.105882in}{0.750000in}}{\pgfqpoint{4.376471in}{0.978947in}} %
\pgfusepath{clip}%
\pgfsetbuttcap%
\pgfsetroundjoin%
\definecolor{currentfill}{rgb}{1.000000,0.000000,0.000000}%
\pgfsetfillcolor{currentfill}%
\pgfsetlinewidth{2.007500pt}%
\definecolor{currentstroke}{rgb}{1.000000,0.000000,0.000000}%
\pgfsetstrokecolor{currentstroke}%
\pgfsetdash{}{0pt}%
\pgfpathmoveto{\pgfqpoint{8.198830in}{1.289809in}}%
\pgfpathlineto{\pgfqpoint{8.260943in}{1.289809in}}%
\pgfpathmoveto{\pgfqpoint{8.229886in}{1.258753in}}%
\pgfpathlineto{\pgfqpoint{8.229886in}{1.320866in}}%
\pgfusepath{stroke,fill}%
\end{pgfscope}%
\begin{pgfscope}%
\pgfpathrectangle{\pgfqpoint{7.105882in}{0.750000in}}{\pgfqpoint{4.376471in}{0.978947in}} %
\pgfusepath{clip}%
\pgfsetbuttcap%
\pgfsetroundjoin%
\definecolor{currentfill}{rgb}{1.000000,0.000000,0.000000}%
\pgfsetfillcolor{currentfill}%
\pgfsetlinewidth{2.007500pt}%
\definecolor{currentstroke}{rgb}{1.000000,0.000000,0.000000}%
\pgfsetstrokecolor{currentstroke}%
\pgfsetdash{}{0pt}%
\pgfpathmoveto{\pgfqpoint{8.255175in}{1.232176in}}%
\pgfpathlineto{\pgfqpoint{8.317288in}{1.232176in}}%
\pgfpathmoveto{\pgfqpoint{8.286232in}{1.201119in}}%
\pgfpathlineto{\pgfqpoint{8.286232in}{1.263232in}}%
\pgfusepath{stroke,fill}%
\end{pgfscope}%
\begin{pgfscope}%
\pgfpathrectangle{\pgfqpoint{7.105882in}{0.750000in}}{\pgfqpoint{4.376471in}{0.978947in}} %
\pgfusepath{clip}%
\pgfsetbuttcap%
\pgfsetroundjoin%
\definecolor{currentfill}{rgb}{1.000000,0.000000,0.000000}%
\pgfsetfillcolor{currentfill}%
\pgfsetlinewidth{2.007500pt}%
\definecolor{currentstroke}{rgb}{1.000000,0.000000,0.000000}%
\pgfsetstrokecolor{currentstroke}%
\pgfsetdash{}{0pt}%
\pgfpathmoveto{\pgfqpoint{8.020908in}{1.519456in}}%
\pgfpathlineto{\pgfqpoint{8.083021in}{1.519456in}}%
\pgfpathmoveto{\pgfqpoint{8.051965in}{1.488399in}}%
\pgfpathlineto{\pgfqpoint{8.051965in}{1.550512in}}%
\pgfusepath{stroke,fill}%
\end{pgfscope}%
\begin{pgfscope}%
\pgfpathrectangle{\pgfqpoint{7.105882in}{0.750000in}}{\pgfqpoint{4.376471in}{0.978947in}} %
\pgfusepath{clip}%
\pgfsetbuttcap%
\pgfsetroundjoin%
\definecolor{currentfill}{rgb}{1.000000,0.000000,0.000000}%
\pgfsetfillcolor{currentfill}%
\pgfsetlinewidth{2.007500pt}%
\definecolor{currentstroke}{rgb}{1.000000,0.000000,0.000000}%
\pgfsetstrokecolor{currentstroke}%
\pgfsetdash{}{0pt}%
\pgfpathmoveto{\pgfqpoint{10.865269in}{1.365052in}}%
\pgfpathlineto{\pgfqpoint{10.927382in}{1.365052in}}%
\pgfpathmoveto{\pgfqpoint{10.896325in}{1.333995in}}%
\pgfpathlineto{\pgfqpoint{10.896325in}{1.396108in}}%
\pgfusepath{stroke,fill}%
\end{pgfscope}%
\begin{pgfscope}%
\pgfpathrectangle{\pgfqpoint{7.105882in}{0.750000in}}{\pgfqpoint{4.376471in}{0.978947in}} %
\pgfusepath{clip}%
\pgfsetbuttcap%
\pgfsetroundjoin%
\definecolor{currentfill}{rgb}{1.000000,0.000000,0.000000}%
\pgfsetfillcolor{currentfill}%
\pgfsetlinewidth{2.007500pt}%
\definecolor{currentstroke}{rgb}{1.000000,0.000000,0.000000}%
\pgfsetstrokecolor{currentstroke}%
\pgfsetdash{}{0pt}%
\pgfpathmoveto{\pgfqpoint{10.674584in}{1.087470in}}%
\pgfpathlineto{\pgfqpoint{10.736697in}{1.087470in}}%
\pgfpathmoveto{\pgfqpoint{10.705641in}{1.056414in}}%
\pgfpathlineto{\pgfqpoint{10.705641in}{1.118527in}}%
\pgfusepath{stroke,fill}%
\end{pgfscope}%
\begin{pgfscope}%
\pgfpathrectangle{\pgfqpoint{7.105882in}{0.750000in}}{\pgfqpoint{4.376471in}{0.978947in}} %
\pgfusepath{clip}%
\pgfsetbuttcap%
\pgfsetroundjoin%
\definecolor{currentfill}{rgb}{1.000000,0.000000,0.000000}%
\pgfsetfillcolor{currentfill}%
\pgfsetlinewidth{2.007500pt}%
\definecolor{currentstroke}{rgb}{1.000000,0.000000,0.000000}%
\pgfsetstrokecolor{currentstroke}%
\pgfsetdash{}{0pt}%
\pgfpathmoveto{\pgfqpoint{10.996186in}{1.494833in}}%
\pgfpathlineto{\pgfqpoint{11.058299in}{1.494833in}}%
\pgfpathmoveto{\pgfqpoint{11.027243in}{1.463777in}}%
\pgfpathlineto{\pgfqpoint{11.027243in}{1.525890in}}%
\pgfusepath{stroke,fill}%
\end{pgfscope}%
\begin{pgfscope}%
\pgfpathrectangle{\pgfqpoint{7.105882in}{0.750000in}}{\pgfqpoint{4.376471in}{0.978947in}} %
\pgfusepath{clip}%
\pgfsetbuttcap%
\pgfsetroundjoin%
\definecolor{currentfill}{rgb}{1.000000,0.000000,0.000000}%
\pgfsetfillcolor{currentfill}%
\pgfsetlinewidth{2.007500pt}%
\definecolor{currentstroke}{rgb}{1.000000,0.000000,0.000000}%
\pgfsetstrokecolor{currentstroke}%
\pgfsetdash{}{0pt}%
\pgfpathmoveto{\pgfqpoint{11.376435in}{1.426070in}}%
\pgfpathlineto{\pgfqpoint{11.438548in}{1.426070in}}%
\pgfpathmoveto{\pgfqpoint{11.407492in}{1.395013in}}%
\pgfpathlineto{\pgfqpoint{11.407492in}{1.457126in}}%
\pgfusepath{stroke,fill}%
\end{pgfscope}%
\begin{pgfscope}%
\pgfpathrectangle{\pgfqpoint{7.105882in}{0.750000in}}{\pgfqpoint{4.376471in}{0.978947in}} %
\pgfusepath{clip}%
\pgfsetbuttcap%
\pgfsetroundjoin%
\definecolor{currentfill}{rgb}{1.000000,0.000000,0.000000}%
\pgfsetfillcolor{currentfill}%
\pgfsetlinewidth{2.007500pt}%
\definecolor{currentstroke}{rgb}{1.000000,0.000000,0.000000}%
\pgfsetstrokecolor{currentstroke}%
\pgfsetdash{}{0pt}%
\pgfpathmoveto{\pgfqpoint{10.748115in}{1.242637in}}%
\pgfpathlineto{\pgfqpoint{10.810228in}{1.242637in}}%
\pgfpathmoveto{\pgfqpoint{10.779172in}{1.211581in}}%
\pgfpathlineto{\pgfqpoint{10.779172in}{1.273694in}}%
\pgfusepath{stroke,fill}%
\end{pgfscope}%
\begin{pgfscope}%
\pgfpathrectangle{\pgfqpoint{7.105882in}{0.750000in}}{\pgfqpoint{4.376471in}{0.978947in}} %
\pgfusepath{clip}%
\pgfsetbuttcap%
\pgfsetroundjoin%
\definecolor{currentfill}{rgb}{1.000000,0.000000,0.000000}%
\pgfsetfillcolor{currentfill}%
\pgfsetlinewidth{2.007500pt}%
\definecolor{currentstroke}{rgb}{1.000000,0.000000,0.000000}%
\pgfsetstrokecolor{currentstroke}%
\pgfsetdash{}{0pt}%
\pgfpathmoveto{\pgfqpoint{9.565841in}{0.967097in}}%
\pgfpathlineto{\pgfqpoint{9.627954in}{0.967097in}}%
\pgfpathmoveto{\pgfqpoint{9.596897in}{0.936040in}}%
\pgfpathlineto{\pgfqpoint{9.596897in}{0.998153in}}%
\pgfusepath{stroke,fill}%
\end{pgfscope}%
\begin{pgfscope}%
\pgfpathrectangle{\pgfqpoint{7.105882in}{0.750000in}}{\pgfqpoint{4.376471in}{0.978947in}} %
\pgfusepath{clip}%
\pgfsetbuttcap%
\pgfsetroundjoin%
\definecolor{currentfill}{rgb}{1.000000,0.000000,0.000000}%
\pgfsetfillcolor{currentfill}%
\pgfsetlinewidth{2.007500pt}%
\definecolor{currentstroke}{rgb}{1.000000,0.000000,0.000000}%
\pgfsetstrokecolor{currentstroke}%
\pgfsetdash{}{0pt}%
\pgfpathmoveto{\pgfqpoint{10.682890in}{1.109507in}}%
\pgfpathlineto{\pgfqpoint{10.745003in}{1.109507in}}%
\pgfpathmoveto{\pgfqpoint{10.713947in}{1.078450in}}%
\pgfpathlineto{\pgfqpoint{10.713947in}{1.140563in}}%
\pgfusepath{stroke,fill}%
\end{pgfscope}%
\begin{pgfscope}%
\pgfpathrectangle{\pgfqpoint{7.105882in}{0.750000in}}{\pgfqpoint{4.376471in}{0.978947in}} %
\pgfusepath{clip}%
\pgfsetbuttcap%
\pgfsetroundjoin%
\definecolor{currentfill}{rgb}{1.000000,0.000000,0.000000}%
\pgfsetfillcolor{currentfill}%
\pgfsetlinewidth{2.007500pt}%
\definecolor{currentstroke}{rgb}{1.000000,0.000000,0.000000}%
\pgfsetstrokecolor{currentstroke}%
\pgfsetdash{}{0pt}%
\pgfpathmoveto{\pgfqpoint{8.364220in}{1.130244in}}%
\pgfpathlineto{\pgfqpoint{8.426333in}{1.130244in}}%
\pgfpathmoveto{\pgfqpoint{8.395276in}{1.099188in}}%
\pgfpathlineto{\pgfqpoint{8.395276in}{1.161301in}}%
\pgfusepath{stroke,fill}%
\end{pgfscope}%
\begin{pgfscope}%
\pgfpathrectangle{\pgfqpoint{7.105882in}{0.750000in}}{\pgfqpoint{4.376471in}{0.978947in}} %
\pgfusepath{clip}%
\pgfsetbuttcap%
\pgfsetroundjoin%
\definecolor{currentfill}{rgb}{1.000000,0.000000,0.000000}%
\pgfsetfillcolor{currentfill}%
\pgfsetlinewidth{2.007500pt}%
\definecolor{currentstroke}{rgb}{1.000000,0.000000,0.000000}%
\pgfsetstrokecolor{currentstroke}%
\pgfsetdash{}{0pt}%
\pgfpathmoveto{\pgfqpoint{10.190596in}{1.233528in}}%
\pgfpathlineto{\pgfqpoint{10.252709in}{1.233528in}}%
\pgfpathmoveto{\pgfqpoint{10.221653in}{1.202472in}}%
\pgfpathlineto{\pgfqpoint{10.221653in}{1.264585in}}%
\pgfusepath{stroke,fill}%
\end{pgfscope}%
\begin{pgfscope}%
\pgfpathrectangle{\pgfqpoint{7.105882in}{0.750000in}}{\pgfqpoint{4.376471in}{0.978947in}} %
\pgfusepath{clip}%
\pgfsetbuttcap%
\pgfsetroundjoin%
\definecolor{currentfill}{rgb}{1.000000,0.000000,0.000000}%
\pgfsetfillcolor{currentfill}%
\pgfsetlinewidth{2.007500pt}%
\definecolor{currentstroke}{rgb}{1.000000,0.000000,0.000000}%
\pgfsetstrokecolor{currentstroke}%
\pgfsetdash{}{0pt}%
\pgfpathmoveto{\pgfqpoint{8.452025in}{1.138534in}}%
\pgfpathlineto{\pgfqpoint{8.514138in}{1.138534in}}%
\pgfpathmoveto{\pgfqpoint{8.483082in}{1.107478in}}%
\pgfpathlineto{\pgfqpoint{8.483082in}{1.169591in}}%
\pgfusepath{stroke,fill}%
\end{pgfscope}%
\begin{pgfscope}%
\pgfpathrectangle{\pgfqpoint{7.105882in}{0.750000in}}{\pgfqpoint{4.376471in}{0.978947in}} %
\pgfusepath{clip}%
\pgfsetbuttcap%
\pgfsetroundjoin%
\definecolor{currentfill}{rgb}{1.000000,0.000000,0.000000}%
\pgfsetfillcolor{currentfill}%
\pgfsetlinewidth{2.007500pt}%
\definecolor{currentstroke}{rgb}{1.000000,0.000000,0.000000}%
\pgfsetstrokecolor{currentstroke}%
\pgfsetdash{}{0pt}%
\pgfpathmoveto{\pgfqpoint{11.257573in}{1.538728in}}%
\pgfpathlineto{\pgfqpoint{11.319686in}{1.538728in}}%
\pgfpathmoveto{\pgfqpoint{11.288629in}{1.507671in}}%
\pgfpathlineto{\pgfqpoint{11.288629in}{1.569784in}}%
\pgfusepath{stroke,fill}%
\end{pgfscope}%
\begin{pgfscope}%
\pgfpathrectangle{\pgfqpoint{7.105882in}{0.750000in}}{\pgfqpoint{4.376471in}{0.978947in}} %
\pgfusepath{clip}%
\pgfsetbuttcap%
\pgfsetroundjoin%
\definecolor{currentfill}{rgb}{1.000000,0.000000,0.000000}%
\pgfsetfillcolor{currentfill}%
\pgfsetlinewidth{2.007500pt}%
\definecolor{currentstroke}{rgb}{1.000000,0.000000,0.000000}%
\pgfsetstrokecolor{currentstroke}%
\pgfsetdash{}{0pt}%
\pgfpathmoveto{\pgfqpoint{9.777203in}{1.185930in}}%
\pgfpathlineto{\pgfqpoint{9.839316in}{1.185930in}}%
\pgfpathmoveto{\pgfqpoint{9.808260in}{1.154874in}}%
\pgfpathlineto{\pgfqpoint{9.808260in}{1.216987in}}%
\pgfusepath{stroke,fill}%
\end{pgfscope}%
\begin{pgfscope}%
\pgfpathrectangle{\pgfqpoint{7.105882in}{0.750000in}}{\pgfqpoint{4.376471in}{0.978947in}} %
\pgfusepath{clip}%
\pgfsetbuttcap%
\pgfsetroundjoin%
\definecolor{currentfill}{rgb}{1.000000,0.000000,0.000000}%
\pgfsetfillcolor{currentfill}%
\pgfsetlinewidth{2.007500pt}%
\definecolor{currentstroke}{rgb}{1.000000,0.000000,0.000000}%
\pgfsetstrokecolor{currentstroke}%
\pgfsetdash{}{0pt}%
\pgfpathmoveto{\pgfqpoint{9.401925in}{0.934813in}}%
\pgfpathlineto{\pgfqpoint{9.464038in}{0.934813in}}%
\pgfpathmoveto{\pgfqpoint{9.432981in}{0.903757in}}%
\pgfpathlineto{\pgfqpoint{9.432981in}{0.965870in}}%
\pgfusepath{stroke,fill}%
\end{pgfscope}%
\begin{pgfscope}%
\pgfpathrectangle{\pgfqpoint{7.105882in}{0.750000in}}{\pgfqpoint{4.376471in}{0.978947in}} %
\pgfusepath{clip}%
\pgfsetbuttcap%
\pgfsetroundjoin%
\definecolor{currentfill}{rgb}{0.000000,0.000000,0.000000}%
\pgfsetfillcolor{currentfill}%
\pgfsetlinewidth{0.301125pt}%
\definecolor{currentstroke}{rgb}{0.000000,0.000000,0.000000}%
\pgfsetstrokecolor{currentstroke}%
\pgfsetdash{}{0pt}%
\pgfsys@defobject{currentmarker}{\pgfqpoint{-0.015528in}{-0.015528in}}{\pgfqpoint{0.015528in}{0.015528in}}{%
\pgfpathmoveto{\pgfqpoint{0.000000in}{-0.015528in}}%
\pgfpathcurveto{\pgfqpoint{0.004118in}{-0.015528in}}{\pgfqpoint{0.008068in}{-0.013892in}}{\pgfqpoint{0.010980in}{-0.010980in}}%
\pgfpathcurveto{\pgfqpoint{0.013892in}{-0.008068in}}{\pgfqpoint{0.015528in}{-0.004118in}}{\pgfqpoint{0.015528in}{0.000000in}}%
\pgfpathcurveto{\pgfqpoint{0.015528in}{0.004118in}}{\pgfqpoint{0.013892in}{0.008068in}}{\pgfqpoint{0.010980in}{0.010980in}}%
\pgfpathcurveto{\pgfqpoint{0.008068in}{0.013892in}}{\pgfqpoint{0.004118in}{0.015528in}}{\pgfqpoint{0.000000in}{0.015528in}}%
\pgfpathcurveto{\pgfqpoint{-0.004118in}{0.015528in}}{\pgfqpoint{-0.008068in}{0.013892in}}{\pgfqpoint{-0.010980in}{0.010980in}}%
\pgfpathcurveto{\pgfqpoint{-0.013892in}{0.008068in}}{\pgfqpoint{-0.015528in}{0.004118in}}{\pgfqpoint{-0.015528in}{0.000000in}}%
\pgfpathcurveto{\pgfqpoint{-0.015528in}{-0.004118in}}{\pgfqpoint{-0.013892in}{-0.008068in}}{\pgfqpoint{-0.010980in}{-0.010980in}}%
\pgfpathcurveto{\pgfqpoint{-0.008068in}{-0.013892in}}{\pgfqpoint{-0.004118in}{-0.015528in}}{\pgfqpoint{0.000000in}{-0.015528in}}%
\pgfpathclose%
\pgfusepath{stroke,fill}%
}%
\begin{pgfscope}%
\pgfsys@transformshift{7.981176in}{1.583863in}%
\pgfsys@useobject{currentmarker}{}%
\end{pgfscope}%
\begin{pgfscope}%
\pgfsys@transformshift{7.998770in}{1.489671in}%
\pgfsys@useobject{currentmarker}{}%
\end{pgfscope}%
\begin{pgfscope}%
\pgfsys@transformshift{8.016364in}{1.580445in}%
\pgfsys@useobject{currentmarker}{}%
\end{pgfscope}%
\begin{pgfscope}%
\pgfsys@transformshift{8.033958in}{1.599582in}%
\pgfsys@useobject{currentmarker}{}%
\end{pgfscope}%
\begin{pgfscope}%
\pgfsys@transformshift{8.051552in}{1.534796in}%
\pgfsys@useobject{currentmarker}{}%
\end{pgfscope}%
\begin{pgfscope}%
\pgfsys@transformshift{8.069146in}{1.530696in}%
\pgfsys@useobject{currentmarker}{}%
\end{pgfscope}%
\begin{pgfscope}%
\pgfsys@transformshift{8.086740in}{1.406938in}%
\pgfsys@useobject{currentmarker}{}%
\end{pgfscope}%
\begin{pgfscope}%
\pgfsys@transformshift{8.104333in}{1.406543in}%
\pgfsys@useobject{currentmarker}{}%
\end{pgfscope}%
\begin{pgfscope}%
\pgfsys@transformshift{8.121927in}{1.347216in}%
\pgfsys@useobject{currentmarker}{}%
\end{pgfscope}%
\begin{pgfscope}%
\pgfsys@transformshift{8.139521in}{1.351930in}%
\pgfsys@useobject{currentmarker}{}%
\end{pgfscope}%
\begin{pgfscope}%
\pgfsys@transformshift{8.157115in}{1.279228in}%
\pgfsys@useobject{currentmarker}{}%
\end{pgfscope}%
\begin{pgfscope}%
\pgfsys@transformshift{8.174709in}{1.160606in}%
\pgfsys@useobject{currentmarker}{}%
\end{pgfscope}%
\begin{pgfscope}%
\pgfsys@transformshift{8.192303in}{1.330351in}%
\pgfsys@useobject{currentmarker}{}%
\end{pgfscope}%
\begin{pgfscope}%
\pgfsys@transformshift{8.209897in}{1.248201in}%
\pgfsys@useobject{currentmarker}{}%
\end{pgfscope}%
\begin{pgfscope}%
\pgfsys@transformshift{8.227490in}{1.101305in}%
\pgfsys@useobject{currentmarker}{}%
\end{pgfscope}%
\begin{pgfscope}%
\pgfsys@transformshift{8.245084in}{1.294708in}%
\pgfsys@useobject{currentmarker}{}%
\end{pgfscope}%
\begin{pgfscope}%
\pgfsys@transformshift{8.262678in}{1.136705in}%
\pgfsys@useobject{currentmarker}{}%
\end{pgfscope}%
\begin{pgfscope}%
\pgfsys@transformshift{8.280272in}{1.218082in}%
\pgfsys@useobject{currentmarker}{}%
\end{pgfscope}%
\begin{pgfscope}%
\pgfsys@transformshift{8.297866in}{1.272640in}%
\pgfsys@useobject{currentmarker}{}%
\end{pgfscope}%
\begin{pgfscope}%
\pgfsys@transformshift{8.315460in}{1.198976in}%
\pgfsys@useobject{currentmarker}{}%
\end{pgfscope}%
\begin{pgfscope}%
\pgfsys@transformshift{8.333054in}{1.291626in}%
\pgfsys@useobject{currentmarker}{}%
\end{pgfscope}%
\begin{pgfscope}%
\pgfsys@transformshift{8.350647in}{1.041254in}%
\pgfsys@useobject{currentmarker}{}%
\end{pgfscope}%
\begin{pgfscope}%
\pgfsys@transformshift{8.368241in}{1.202078in}%
\pgfsys@useobject{currentmarker}{}%
\end{pgfscope}%
\begin{pgfscope}%
\pgfsys@transformshift{8.385835in}{1.087229in}%
\pgfsys@useobject{currentmarker}{}%
\end{pgfscope}%
\begin{pgfscope}%
\pgfsys@transformshift{8.403429in}{1.066387in}%
\pgfsys@useobject{currentmarker}{}%
\end{pgfscope}%
\begin{pgfscope}%
\pgfsys@transformshift{8.421023in}{1.096351in}%
\pgfsys@useobject{currentmarker}{}%
\end{pgfscope}%
\begin{pgfscope}%
\pgfsys@transformshift{8.438617in}{1.125805in}%
\pgfsys@useobject{currentmarker}{}%
\end{pgfscope}%
\begin{pgfscope}%
\pgfsys@transformshift{8.456210in}{1.167420in}%
\pgfsys@useobject{currentmarker}{}%
\end{pgfscope}%
\begin{pgfscope}%
\pgfsys@transformshift{8.473804in}{1.048813in}%
\pgfsys@useobject{currentmarker}{}%
\end{pgfscope}%
\begin{pgfscope}%
\pgfsys@transformshift{8.491398in}{1.267121in}%
\pgfsys@useobject{currentmarker}{}%
\end{pgfscope}%
\begin{pgfscope}%
\pgfsys@transformshift{8.508992in}{1.231941in}%
\pgfsys@useobject{currentmarker}{}%
\end{pgfscope}%
\begin{pgfscope}%
\pgfsys@transformshift{8.526586in}{1.038406in}%
\pgfsys@useobject{currentmarker}{}%
\end{pgfscope}%
\begin{pgfscope}%
\pgfsys@transformshift{8.544180in}{1.358678in}%
\pgfsys@useobject{currentmarker}{}%
\end{pgfscope}%
\begin{pgfscope}%
\pgfsys@transformshift{8.561774in}{1.413170in}%
\pgfsys@useobject{currentmarker}{}%
\end{pgfscope}%
\begin{pgfscope}%
\pgfsys@transformshift{8.579367in}{1.353852in}%
\pgfsys@useobject{currentmarker}{}%
\end{pgfscope}%
\begin{pgfscope}%
\pgfsys@transformshift{8.596961in}{1.229798in}%
\pgfsys@useobject{currentmarker}{}%
\end{pgfscope}%
\begin{pgfscope}%
\pgfsys@transformshift{8.614555in}{1.153945in}%
\pgfsys@useobject{currentmarker}{}%
\end{pgfscope}%
\begin{pgfscope}%
\pgfsys@transformshift{8.632149in}{1.385934in}%
\pgfsys@useobject{currentmarker}{}%
\end{pgfscope}%
\begin{pgfscope}%
\pgfsys@transformshift{8.649743in}{1.252646in}%
\pgfsys@useobject{currentmarker}{}%
\end{pgfscope}%
\begin{pgfscope}%
\pgfsys@transformshift{8.667337in}{1.433641in}%
\pgfsys@useobject{currentmarker}{}%
\end{pgfscope}%
\begin{pgfscope}%
\pgfsys@transformshift{8.684931in}{1.345114in}%
\pgfsys@useobject{currentmarker}{}%
\end{pgfscope}%
\begin{pgfscope}%
\pgfsys@transformshift{8.702524in}{1.437827in}%
\pgfsys@useobject{currentmarker}{}%
\end{pgfscope}%
\begin{pgfscope}%
\pgfsys@transformshift{8.720118in}{1.388209in}%
\pgfsys@useobject{currentmarker}{}%
\end{pgfscope}%
\begin{pgfscope}%
\pgfsys@transformshift{8.737712in}{1.436642in}%
\pgfsys@useobject{currentmarker}{}%
\end{pgfscope}%
\begin{pgfscope}%
\pgfsys@transformshift{8.755306in}{1.377294in}%
\pgfsys@useobject{currentmarker}{}%
\end{pgfscope}%
\begin{pgfscope}%
\pgfsys@transformshift{8.772900in}{1.568717in}%
\pgfsys@useobject{currentmarker}{}%
\end{pgfscope}%
\begin{pgfscope}%
\pgfsys@transformshift{8.790494in}{1.408526in}%
\pgfsys@useobject{currentmarker}{}%
\end{pgfscope}%
\begin{pgfscope}%
\pgfsys@transformshift{8.808087in}{1.444018in}%
\pgfsys@useobject{currentmarker}{}%
\end{pgfscope}%
\begin{pgfscope}%
\pgfsys@transformshift{8.825681in}{1.600841in}%
\pgfsys@useobject{currentmarker}{}%
\end{pgfscope}%
\begin{pgfscope}%
\pgfsys@transformshift{8.843275in}{1.275399in}%
\pgfsys@useobject{currentmarker}{}%
\end{pgfscope}%
\begin{pgfscope}%
\pgfsys@transformshift{8.860869in}{1.285462in}%
\pgfsys@useobject{currentmarker}{}%
\end{pgfscope}%
\begin{pgfscope}%
\pgfsys@transformshift{8.878463in}{1.514150in}%
\pgfsys@useobject{currentmarker}{}%
\end{pgfscope}%
\begin{pgfscope}%
\pgfsys@transformshift{8.896057in}{1.293998in}%
\pgfsys@useobject{currentmarker}{}%
\end{pgfscope}%
\begin{pgfscope}%
\pgfsys@transformshift{8.913651in}{1.608181in}%
\pgfsys@useobject{currentmarker}{}%
\end{pgfscope}%
\begin{pgfscope}%
\pgfsys@transformshift{8.931244in}{1.362188in}%
\pgfsys@useobject{currentmarker}{}%
\end{pgfscope}%
\begin{pgfscope}%
\pgfsys@transformshift{8.948838in}{1.320572in}%
\pgfsys@useobject{currentmarker}{}%
\end{pgfscope}%
\begin{pgfscope}%
\pgfsys@transformshift{8.966432in}{1.583393in}%
\pgfsys@useobject{currentmarker}{}%
\end{pgfscope}%
\begin{pgfscope}%
\pgfsys@transformshift{8.984026in}{1.526929in}%
\pgfsys@useobject{currentmarker}{}%
\end{pgfscope}%
\begin{pgfscope}%
\pgfsys@transformshift{9.001620in}{1.553271in}%
\pgfsys@useobject{currentmarker}{}%
\end{pgfscope}%
\begin{pgfscope}%
\pgfsys@transformshift{9.019214in}{1.440414in}%
\pgfsys@useobject{currentmarker}{}%
\end{pgfscope}%
\begin{pgfscope}%
\pgfsys@transformshift{9.036808in}{1.243825in}%
\pgfsys@useobject{currentmarker}{}%
\end{pgfscope}%
\begin{pgfscope}%
\pgfsys@transformshift{9.054401in}{1.508616in}%
\pgfsys@useobject{currentmarker}{}%
\end{pgfscope}%
\begin{pgfscope}%
\pgfsys@transformshift{9.071995in}{1.267415in}%
\pgfsys@useobject{currentmarker}{}%
\end{pgfscope}%
\begin{pgfscope}%
\pgfsys@transformshift{9.089589in}{1.356354in}%
\pgfsys@useobject{currentmarker}{}%
\end{pgfscope}%
\begin{pgfscope}%
\pgfsys@transformshift{9.107183in}{1.349946in}%
\pgfsys@useobject{currentmarker}{}%
\end{pgfscope}%
\begin{pgfscope}%
\pgfsys@transformshift{9.124777in}{1.215603in}%
\pgfsys@useobject{currentmarker}{}%
\end{pgfscope}%
\begin{pgfscope}%
\pgfsys@transformshift{9.142371in}{1.271512in}%
\pgfsys@useobject{currentmarker}{}%
\end{pgfscope}%
\begin{pgfscope}%
\pgfsys@transformshift{9.159965in}{1.280038in}%
\pgfsys@useobject{currentmarker}{}%
\end{pgfscope}%
\begin{pgfscope}%
\pgfsys@transformshift{9.177558in}{1.201314in}%
\pgfsys@useobject{currentmarker}{}%
\end{pgfscope}%
\begin{pgfscope}%
\pgfsys@transformshift{9.195152in}{1.027783in}%
\pgfsys@useobject{currentmarker}{}%
\end{pgfscope}%
\begin{pgfscope}%
\pgfsys@transformshift{9.212746in}{1.147508in}%
\pgfsys@useobject{currentmarker}{}%
\end{pgfscope}%
\begin{pgfscope}%
\pgfsys@transformshift{9.230340in}{1.230000in}%
\pgfsys@useobject{currentmarker}{}%
\end{pgfscope}%
\begin{pgfscope}%
\pgfsys@transformshift{9.247934in}{1.002209in}%
\pgfsys@useobject{currentmarker}{}%
\end{pgfscope}%
\begin{pgfscope}%
\pgfsys@transformshift{9.265528in}{1.036913in}%
\pgfsys@useobject{currentmarker}{}%
\end{pgfscope}%
\begin{pgfscope}%
\pgfsys@transformshift{9.283121in}{0.987962in}%
\pgfsys@useobject{currentmarker}{}%
\end{pgfscope}%
\begin{pgfscope}%
\pgfsys@transformshift{9.300715in}{1.202285in}%
\pgfsys@useobject{currentmarker}{}%
\end{pgfscope}%
\begin{pgfscope}%
\pgfsys@transformshift{9.318309in}{1.065014in}%
\pgfsys@useobject{currentmarker}{}%
\end{pgfscope}%
\begin{pgfscope}%
\pgfsys@transformshift{9.335903in}{1.022286in}%
\pgfsys@useobject{currentmarker}{}%
\end{pgfscope}%
\begin{pgfscope}%
\pgfsys@transformshift{9.353497in}{0.888172in}%
\pgfsys@useobject{currentmarker}{}%
\end{pgfscope}%
\begin{pgfscope}%
\pgfsys@transformshift{9.371091in}{1.009412in}%
\pgfsys@useobject{currentmarker}{}%
\end{pgfscope}%
\begin{pgfscope}%
\pgfsys@transformshift{9.388685in}{0.875298in}%
\pgfsys@useobject{currentmarker}{}%
\end{pgfscope}%
\begin{pgfscope}%
\pgfsys@transformshift{9.406278in}{0.938933in}%
\pgfsys@useobject{currentmarker}{}%
\end{pgfscope}%
\begin{pgfscope}%
\pgfsys@transformshift{9.423872in}{0.864529in}%
\pgfsys@useobject{currentmarker}{}%
\end{pgfscope}%
\begin{pgfscope}%
\pgfsys@transformshift{9.441466in}{0.994133in}%
\pgfsys@useobject{currentmarker}{}%
\end{pgfscope}%
\begin{pgfscope}%
\pgfsys@transformshift{9.459060in}{0.981871in}%
\pgfsys@useobject{currentmarker}{}%
\end{pgfscope}%
\begin{pgfscope}%
\pgfsys@transformshift{9.476654in}{0.901892in}%
\pgfsys@useobject{currentmarker}{}%
\end{pgfscope}%
\begin{pgfscope}%
\pgfsys@transformshift{9.494248in}{0.965702in}%
\pgfsys@useobject{currentmarker}{}%
\end{pgfscope}%
\begin{pgfscope}%
\pgfsys@transformshift{9.511842in}{0.818136in}%
\pgfsys@useobject{currentmarker}{}%
\end{pgfscope}%
\begin{pgfscope}%
\pgfsys@transformshift{9.529435in}{0.783851in}%
\pgfsys@useobject{currentmarker}{}%
\end{pgfscope}%
\begin{pgfscope}%
\pgfsys@transformshift{9.547029in}{0.989012in}%
\pgfsys@useobject{currentmarker}{}%
\end{pgfscope}%
\begin{pgfscope}%
\pgfsys@transformshift{9.564623in}{0.971362in}%
\pgfsys@useobject{currentmarker}{}%
\end{pgfscope}%
\begin{pgfscope}%
\pgfsys@transformshift{9.582217in}{1.031043in}%
\pgfsys@useobject{currentmarker}{}%
\end{pgfscope}%
\begin{pgfscope}%
\pgfsys@transformshift{9.599811in}{1.222861in}%
\pgfsys@useobject{currentmarker}{}%
\end{pgfscope}%
\begin{pgfscope}%
\pgfsys@transformshift{9.617405in}{1.091202in}%
\pgfsys@useobject{currentmarker}{}%
\end{pgfscope}%
\begin{pgfscope}%
\pgfsys@transformshift{9.634999in}{0.918187in}%
\pgfsys@useobject{currentmarker}{}%
\end{pgfscope}%
\begin{pgfscope}%
\pgfsys@transformshift{9.652592in}{1.142721in}%
\pgfsys@useobject{currentmarker}{}%
\end{pgfscope}%
\begin{pgfscope}%
\pgfsys@transformshift{9.670186in}{0.913152in}%
\pgfsys@useobject{currentmarker}{}%
\end{pgfscope}%
\begin{pgfscope}%
\pgfsys@transformshift{9.687780in}{1.019588in}%
\pgfsys@useobject{currentmarker}{}%
\end{pgfscope}%
\begin{pgfscope}%
\pgfsys@transformshift{9.705374in}{1.079610in}%
\pgfsys@useobject{currentmarker}{}%
\end{pgfscope}%
\begin{pgfscope}%
\pgfsys@transformshift{9.722968in}{1.281587in}%
\pgfsys@useobject{currentmarker}{}%
\end{pgfscope}%
\begin{pgfscope}%
\pgfsys@transformshift{9.740562in}{1.051414in}%
\pgfsys@useobject{currentmarker}{}%
\end{pgfscope}%
\begin{pgfscope}%
\pgfsys@transformshift{9.758155in}{1.063552in}%
\pgfsys@useobject{currentmarker}{}%
\end{pgfscope}%
\begin{pgfscope}%
\pgfsys@transformshift{9.775749in}{1.158029in}%
\pgfsys@useobject{currentmarker}{}%
\end{pgfscope}%
\begin{pgfscope}%
\pgfsys@transformshift{9.793343in}{1.120223in}%
\pgfsys@useobject{currentmarker}{}%
\end{pgfscope}%
\begin{pgfscope}%
\pgfsys@transformshift{9.810937in}{1.321952in}%
\pgfsys@useobject{currentmarker}{}%
\end{pgfscope}%
\begin{pgfscope}%
\pgfsys@transformshift{9.828531in}{1.115308in}%
\pgfsys@useobject{currentmarker}{}%
\end{pgfscope}%
\begin{pgfscope}%
\pgfsys@transformshift{9.846125in}{1.125790in}%
\pgfsys@useobject{currentmarker}{}%
\end{pgfscope}%
\begin{pgfscope}%
\pgfsys@transformshift{9.863719in}{1.214357in}%
\pgfsys@useobject{currentmarker}{}%
\end{pgfscope}%
\begin{pgfscope}%
\pgfsys@transformshift{9.881312in}{1.223091in}%
\pgfsys@useobject{currentmarker}{}%
\end{pgfscope}%
\begin{pgfscope}%
\pgfsys@transformshift{9.898906in}{1.484043in}%
\pgfsys@useobject{currentmarker}{}%
\end{pgfscope}%
\begin{pgfscope}%
\pgfsys@transformshift{9.916500in}{1.395905in}%
\pgfsys@useobject{currentmarker}{}%
\end{pgfscope}%
\begin{pgfscope}%
\pgfsys@transformshift{9.934094in}{1.318116in}%
\pgfsys@useobject{currentmarker}{}%
\end{pgfscope}%
\begin{pgfscope}%
\pgfsys@transformshift{9.951688in}{1.192524in}%
\pgfsys@useobject{currentmarker}{}%
\end{pgfscope}%
\begin{pgfscope}%
\pgfsys@transformshift{9.969282in}{1.410011in}%
\pgfsys@useobject{currentmarker}{}%
\end{pgfscope}%
\begin{pgfscope}%
\pgfsys@transformshift{9.986876in}{1.226407in}%
\pgfsys@useobject{currentmarker}{}%
\end{pgfscope}%
\begin{pgfscope}%
\pgfsys@transformshift{10.004469in}{1.173408in}%
\pgfsys@useobject{currentmarker}{}%
\end{pgfscope}%
\begin{pgfscope}%
\pgfsys@transformshift{10.022063in}{1.452637in}%
\pgfsys@useobject{currentmarker}{}%
\end{pgfscope}%
\begin{pgfscope}%
\pgfsys@transformshift{10.039657in}{1.362397in}%
\pgfsys@useobject{currentmarker}{}%
\end{pgfscope}%
\begin{pgfscope}%
\pgfsys@transformshift{10.057251in}{1.420628in}%
\pgfsys@useobject{currentmarker}{}%
\end{pgfscope}%
\begin{pgfscope}%
\pgfsys@transformshift{10.074845in}{1.353984in}%
\pgfsys@useobject{currentmarker}{}%
\end{pgfscope}%
\begin{pgfscope}%
\pgfsys@transformshift{10.092439in}{1.401791in}%
\pgfsys@useobject{currentmarker}{}%
\end{pgfscope}%
\begin{pgfscope}%
\pgfsys@transformshift{10.110033in}{1.239230in}%
\pgfsys@useobject{currentmarker}{}%
\end{pgfscope}%
\begin{pgfscope}%
\pgfsys@transformshift{10.127626in}{1.189725in}%
\pgfsys@useobject{currentmarker}{}%
\end{pgfscope}%
\begin{pgfscope}%
\pgfsys@transformshift{10.145220in}{1.352764in}%
\pgfsys@useobject{currentmarker}{}%
\end{pgfscope}%
\begin{pgfscope}%
\pgfsys@transformshift{10.162814in}{1.188036in}%
\pgfsys@useobject{currentmarker}{}%
\end{pgfscope}%
\begin{pgfscope}%
\pgfsys@transformshift{10.180408in}{1.185140in}%
\pgfsys@useobject{currentmarker}{}%
\end{pgfscope}%
\begin{pgfscope}%
\pgfsys@transformshift{10.198002in}{1.193453in}%
\pgfsys@useobject{currentmarker}{}%
\end{pgfscope}%
\begin{pgfscope}%
\pgfsys@transformshift{10.215596in}{1.225272in}%
\pgfsys@useobject{currentmarker}{}%
\end{pgfscope}%
\begin{pgfscope}%
\pgfsys@transformshift{10.233189in}{1.170299in}%
\pgfsys@useobject{currentmarker}{}%
\end{pgfscope}%
\begin{pgfscope}%
\pgfsys@transformshift{10.250783in}{1.048615in}%
\pgfsys@useobject{currentmarker}{}%
\end{pgfscope}%
\begin{pgfscope}%
\pgfsys@transformshift{10.268377in}{1.105339in}%
\pgfsys@useobject{currentmarker}{}%
\end{pgfscope}%
\begin{pgfscope}%
\pgfsys@transformshift{10.285971in}{0.926315in}%
\pgfsys@useobject{currentmarker}{}%
\end{pgfscope}%
\begin{pgfscope}%
\pgfsys@transformshift{10.303565in}{1.199006in}%
\pgfsys@useobject{currentmarker}{}%
\end{pgfscope}%
\begin{pgfscope}%
\pgfsys@transformshift{10.321159in}{0.954425in}%
\pgfsys@useobject{currentmarker}{}%
\end{pgfscope}%
\begin{pgfscope}%
\pgfsys@transformshift{10.338753in}{0.988260in}%
\pgfsys@useobject{currentmarker}{}%
\end{pgfscope}%
\begin{pgfscope}%
\pgfsys@transformshift{10.356346in}{1.090024in}%
\pgfsys@useobject{currentmarker}{}%
\end{pgfscope}%
\begin{pgfscope}%
\pgfsys@transformshift{10.373940in}{0.994049in}%
\pgfsys@useobject{currentmarker}{}%
\end{pgfscope}%
\begin{pgfscope}%
\pgfsys@transformshift{10.391534in}{1.212689in}%
\pgfsys@useobject{currentmarker}{}%
\end{pgfscope}%
\begin{pgfscope}%
\pgfsys@transformshift{10.409128in}{0.910673in}%
\pgfsys@useobject{currentmarker}{}%
\end{pgfscope}%
\begin{pgfscope}%
\pgfsys@transformshift{10.426722in}{1.058361in}%
\pgfsys@useobject{currentmarker}{}%
\end{pgfscope}%
\begin{pgfscope}%
\pgfsys@transformshift{10.444316in}{1.017342in}%
\pgfsys@useobject{currentmarker}{}%
\end{pgfscope}%
\begin{pgfscope}%
\pgfsys@transformshift{10.461910in}{0.894182in}%
\pgfsys@useobject{currentmarker}{}%
\end{pgfscope}%
\begin{pgfscope}%
\pgfsys@transformshift{10.479503in}{1.060452in}%
\pgfsys@useobject{currentmarker}{}%
\end{pgfscope}%
\begin{pgfscope}%
\pgfsys@transformshift{10.497097in}{0.985337in}%
\pgfsys@useobject{currentmarker}{}%
\end{pgfscope}%
\begin{pgfscope}%
\pgfsys@transformshift{10.514691in}{1.079294in}%
\pgfsys@useobject{currentmarker}{}%
\end{pgfscope}%
\begin{pgfscope}%
\pgfsys@transformshift{10.532285in}{1.084411in}%
\pgfsys@useobject{currentmarker}{}%
\end{pgfscope}%
\begin{pgfscope}%
\pgfsys@transformshift{10.549879in}{1.222992in}%
\pgfsys@useobject{currentmarker}{}%
\end{pgfscope}%
\begin{pgfscope}%
\pgfsys@transformshift{10.567473in}{1.142792in}%
\pgfsys@useobject{currentmarker}{}%
\end{pgfscope}%
\begin{pgfscope}%
\pgfsys@transformshift{10.585067in}{0.975095in}%
\pgfsys@useobject{currentmarker}{}%
\end{pgfscope}%
\begin{pgfscope}%
\pgfsys@transformshift{10.602660in}{0.996685in}%
\pgfsys@useobject{currentmarker}{}%
\end{pgfscope}%
\begin{pgfscope}%
\pgfsys@transformshift{10.620254in}{1.143655in}%
\pgfsys@useobject{currentmarker}{}%
\end{pgfscope}%
\begin{pgfscope}%
\pgfsys@transformshift{10.637848in}{1.110768in}%
\pgfsys@useobject{currentmarker}{}%
\end{pgfscope}%
\begin{pgfscope}%
\pgfsys@transformshift{10.655442in}{1.123579in}%
\pgfsys@useobject{currentmarker}{}%
\end{pgfscope}%
\begin{pgfscope}%
\pgfsys@transformshift{10.673036in}{0.909591in}%
\pgfsys@useobject{currentmarker}{}%
\end{pgfscope}%
\begin{pgfscope}%
\pgfsys@transformshift{10.690630in}{1.089876in}%
\pgfsys@useobject{currentmarker}{}%
\end{pgfscope}%
\begin{pgfscope}%
\pgfsys@transformshift{10.708223in}{1.036542in}%
\pgfsys@useobject{currentmarker}{}%
\end{pgfscope}%
\begin{pgfscope}%
\pgfsys@transformshift{10.725817in}{1.161209in}%
\pgfsys@useobject{currentmarker}{}%
\end{pgfscope}%
\begin{pgfscope}%
\pgfsys@transformshift{10.743411in}{1.144771in}%
\pgfsys@useobject{currentmarker}{}%
\end{pgfscope}%
\begin{pgfscope}%
\pgfsys@transformshift{10.761005in}{1.270786in}%
\pgfsys@useobject{currentmarker}{}%
\end{pgfscope}%
\begin{pgfscope}%
\pgfsys@transformshift{10.778599in}{1.234403in}%
\pgfsys@useobject{currentmarker}{}%
\end{pgfscope}%
\begin{pgfscope}%
\pgfsys@transformshift{10.796193in}{1.307020in}%
\pgfsys@useobject{currentmarker}{}%
\end{pgfscope}%
\begin{pgfscope}%
\pgfsys@transformshift{10.813787in}{1.204550in}%
\pgfsys@useobject{currentmarker}{}%
\end{pgfscope}%
\begin{pgfscope}%
\pgfsys@transformshift{10.831380in}{1.181373in}%
\pgfsys@useobject{currentmarker}{}%
\end{pgfscope}%
\begin{pgfscope}%
\pgfsys@transformshift{10.848974in}{1.261523in}%
\pgfsys@useobject{currentmarker}{}%
\end{pgfscope}%
\begin{pgfscope}%
\pgfsys@transformshift{10.866568in}{1.327137in}%
\pgfsys@useobject{currentmarker}{}%
\end{pgfscope}%
\begin{pgfscope}%
\pgfsys@transformshift{10.884162in}{1.392747in}%
\pgfsys@useobject{currentmarker}{}%
\end{pgfscope}%
\begin{pgfscope}%
\pgfsys@transformshift{10.901756in}{1.609158in}%
\pgfsys@useobject{currentmarker}{}%
\end{pgfscope}%
\begin{pgfscope}%
\pgfsys@transformshift{10.919350in}{1.398425in}%
\pgfsys@useobject{currentmarker}{}%
\end{pgfscope}%
\begin{pgfscope}%
\pgfsys@transformshift{10.936944in}{1.328312in}%
\pgfsys@useobject{currentmarker}{}%
\end{pgfscope}%
\begin{pgfscope}%
\pgfsys@transformshift{10.954537in}{1.412490in}%
\pgfsys@useobject{currentmarker}{}%
\end{pgfscope}%
\begin{pgfscope}%
\pgfsys@transformshift{10.972131in}{1.421228in}%
\pgfsys@useobject{currentmarker}{}%
\end{pgfscope}%
\begin{pgfscope}%
\pgfsys@transformshift{10.989725in}{1.536935in}%
\pgfsys@useobject{currentmarker}{}%
\end{pgfscope}%
\begin{pgfscope}%
\pgfsys@transformshift{11.007319in}{1.348517in}%
\pgfsys@useobject{currentmarker}{}%
\end{pgfscope}%
\begin{pgfscope}%
\pgfsys@transformshift{11.024913in}{1.528236in}%
\pgfsys@useobject{currentmarker}{}%
\end{pgfscope}%
\begin{pgfscope}%
\pgfsys@transformshift{11.042507in}{1.552137in}%
\pgfsys@useobject{currentmarker}{}%
\end{pgfscope}%
\begin{pgfscope}%
\pgfsys@transformshift{11.060101in}{1.572407in}%
\pgfsys@useobject{currentmarker}{}%
\end{pgfscope}%
\begin{pgfscope}%
\pgfsys@transformshift{11.077694in}{1.498469in}%
\pgfsys@useobject{currentmarker}{}%
\end{pgfscope}%
\begin{pgfscope}%
\pgfsys@transformshift{11.095288in}{1.543820in}%
\pgfsys@useobject{currentmarker}{}%
\end{pgfscope}%
\begin{pgfscope}%
\pgfsys@transformshift{11.112882in}{1.429595in}%
\pgfsys@useobject{currentmarker}{}%
\end{pgfscope}%
\begin{pgfscope}%
\pgfsys@transformshift{11.130476in}{1.529262in}%
\pgfsys@useobject{currentmarker}{}%
\end{pgfscope}%
\begin{pgfscope}%
\pgfsys@transformshift{11.148070in}{1.527009in}%
\pgfsys@useobject{currentmarker}{}%
\end{pgfscope}%
\begin{pgfscope}%
\pgfsys@transformshift{11.165664in}{1.625672in}%
\pgfsys@useobject{currentmarker}{}%
\end{pgfscope}%
\begin{pgfscope}%
\pgfsys@transformshift{11.183257in}{1.464366in}%
\pgfsys@useobject{currentmarker}{}%
\end{pgfscope}%
\begin{pgfscope}%
\pgfsys@transformshift{11.200851in}{1.659166in}%
\pgfsys@useobject{currentmarker}{}%
\end{pgfscope}%
\begin{pgfscope}%
\pgfsys@transformshift{11.218445in}{1.727455in}%
\pgfsys@useobject{currentmarker}{}%
\end{pgfscope}%
\begin{pgfscope}%
\pgfsys@transformshift{11.236039in}{1.357955in}%
\pgfsys@useobject{currentmarker}{}%
\end{pgfscope}%
\begin{pgfscope}%
\pgfsys@transformshift{11.253633in}{1.605050in}%
\pgfsys@useobject{currentmarker}{}%
\end{pgfscope}%
\begin{pgfscope}%
\pgfsys@transformshift{11.271227in}{1.621861in}%
\pgfsys@useobject{currentmarker}{}%
\end{pgfscope}%
\begin{pgfscope}%
\pgfsys@transformshift{11.288821in}{1.477952in}%
\pgfsys@useobject{currentmarker}{}%
\end{pgfscope}%
\begin{pgfscope}%
\pgfsys@transformshift{11.306414in}{1.491603in}%
\pgfsys@useobject{currentmarker}{}%
\end{pgfscope}%
\begin{pgfscope}%
\pgfsys@transformshift{11.324008in}{1.506950in}%
\pgfsys@useobject{currentmarker}{}%
\end{pgfscope}%
\begin{pgfscope}%
\pgfsys@transformshift{11.341602in}{1.477936in}%
\pgfsys@useobject{currentmarker}{}%
\end{pgfscope}%
\begin{pgfscope}%
\pgfsys@transformshift{11.359196in}{1.464209in}%
\pgfsys@useobject{currentmarker}{}%
\end{pgfscope}%
\begin{pgfscope}%
\pgfsys@transformshift{11.376790in}{1.312049in}%
\pgfsys@useobject{currentmarker}{}%
\end{pgfscope}%
\begin{pgfscope}%
\pgfsys@transformshift{11.394384in}{1.587727in}%
\pgfsys@useobject{currentmarker}{}%
\end{pgfscope}%
\begin{pgfscope}%
\pgfsys@transformshift{11.411978in}{1.567807in}%
\pgfsys@useobject{currentmarker}{}%
\end{pgfscope}%
\begin{pgfscope}%
\pgfsys@transformshift{11.429571in}{1.362703in}%
\pgfsys@useobject{currentmarker}{}%
\end{pgfscope}%
\begin{pgfscope}%
\pgfsys@transformshift{11.447165in}{1.284679in}%
\pgfsys@useobject{currentmarker}{}%
\end{pgfscope}%
\begin{pgfscope}%
\pgfsys@transformshift{11.464759in}{1.476755in}%
\pgfsys@useobject{currentmarker}{}%
\end{pgfscope}%
\begin{pgfscope}%
\pgfsys@transformshift{11.482353in}{1.355333in}%
\pgfsys@useobject{currentmarker}{}%
\end{pgfscope}%
\end{pgfscope}%
\begin{pgfscope}%
\pgfpathrectangle{\pgfqpoint{7.105882in}{0.750000in}}{\pgfqpoint{4.376471in}{0.978947in}} %
\pgfusepath{clip}%
\pgfsetroundcap%
\pgfsetroundjoin%
\pgfsetlinewidth{1.756562pt}%
\definecolor{currentstroke}{rgb}{0.298039,0.447059,0.690196}%
\pgfsetstrokecolor{currentstroke}%
\pgfsetdash{}{0pt}%
\pgfpathmoveto{\pgfqpoint{7.981176in}{1.120622in}}%
\pgfpathlineto{\pgfqpoint{8.033958in}{1.117688in}}%
\pgfpathlineto{\pgfqpoint{8.104333in}{1.116343in}}%
\pgfpathlineto{\pgfqpoint{8.473804in}{1.117072in}}%
\pgfpathlineto{\pgfqpoint{8.649743in}{1.119198in}}%
\pgfpathlineto{\pgfqpoint{8.825681in}{1.120911in}}%
\pgfpathlineto{\pgfqpoint{8.984026in}{1.119870in}}%
\pgfpathlineto{\pgfqpoint{9.300715in}{1.117123in}}%
\pgfpathlineto{\pgfqpoint{9.969282in}{1.117178in}}%
\pgfpathlineto{\pgfqpoint{11.218445in}{1.117051in}}%
\pgfpathlineto{\pgfqpoint{11.447165in}{1.115828in}}%
\pgfpathlineto{\pgfqpoint{11.482353in}{1.112722in}}%
\pgfpathlineto{\pgfqpoint{11.482353in}{1.112722in}}%
\pgfusepath{stroke}%
\end{pgfscope}%
\begin{pgfscope}%
\pgfsetrectcap%
\pgfsetmiterjoin%
\pgfsetlinewidth{1.003750pt}%
\definecolor{currentstroke}{rgb}{0.800000,0.800000,0.800000}%
\pgfsetstrokecolor{currentstroke}%
\pgfsetdash{}{0pt}%
\pgfpathmoveto{\pgfqpoint{7.105882in}{0.750000in}}%
\pgfpathlineto{\pgfqpoint{7.105882in}{1.728947in}}%
\pgfusepath{stroke}%
\end{pgfscope}%
\begin{pgfscope}%
\pgfsetrectcap%
\pgfsetmiterjoin%
\pgfsetlinewidth{1.003750pt}%
\definecolor{currentstroke}{rgb}{0.800000,0.800000,0.800000}%
\pgfsetstrokecolor{currentstroke}%
\pgfsetdash{}{0pt}%
\pgfpathmoveto{\pgfqpoint{11.482353in}{0.750000in}}%
\pgfpathlineto{\pgfqpoint{11.482353in}{1.728947in}}%
\pgfusepath{stroke}%
\end{pgfscope}%
\begin{pgfscope}%
\pgfsetrectcap%
\pgfsetmiterjoin%
\pgfsetlinewidth{1.003750pt}%
\definecolor{currentstroke}{rgb}{0.800000,0.800000,0.800000}%
\pgfsetstrokecolor{currentstroke}%
\pgfsetdash{}{0pt}%
\pgfpathmoveto{\pgfqpoint{7.105882in}{1.728947in}}%
\pgfpathlineto{\pgfqpoint{11.482353in}{1.728947in}}%
\pgfusepath{stroke}%
\end{pgfscope}%
\begin{pgfscope}%
\pgfsetrectcap%
\pgfsetmiterjoin%
\pgfsetlinewidth{1.003750pt}%
\definecolor{currentstroke}{rgb}{0.800000,0.800000,0.800000}%
\pgfsetstrokecolor{currentstroke}%
\pgfsetdash{}{0pt}%
\pgfpathmoveto{\pgfqpoint{7.105882in}{0.750000in}}%
\pgfpathlineto{\pgfqpoint{11.482353in}{0.750000in}}%
\pgfusepath{stroke}%
\end{pgfscope}%
\begin{pgfscope}%
\pgfsetroundcap%
\pgfsetroundjoin%
\pgfsetlinewidth{1.756562pt}%
\definecolor{currentstroke}{rgb}{0.298039,0.447059,0.690196}%
\pgfsetstrokecolor{currentstroke}%
\pgfsetdash{}{0pt}%
\pgfpathmoveto{\pgfqpoint{7.230882in}{1.344143in}}%
\pgfpathlineto{\pgfqpoint{7.508660in}{1.344143in}}%
\pgfusepath{stroke}%
\end{pgfscope}%
\begin{pgfscope}%
\definecolor{textcolor}{rgb}{0.150000,0.150000,0.150000}%
\pgfsetstrokecolor{textcolor}%
\pgfsetfillcolor{textcolor}%
\pgftext[x=7.619771in,y=1.295532in,left,base]{\color{textcolor}\sffamily\fontsize{10.000000}{12.000000}\selectfont \(\displaystyle \widetilde{\Phi}^* \theta^{\parallel}\)}%
\end{pgfscope}%
\begin{pgfscope}%
\pgfsetbuttcap%
\pgfsetroundjoin%
\definecolor{currentfill}{rgb}{1.000000,0.000000,0.000000}%
\pgfsetfillcolor{currentfill}%
\pgfsetlinewidth{2.007500pt}%
\definecolor{currentstroke}{rgb}{1.000000,0.000000,0.000000}%
\pgfsetstrokecolor{currentstroke}%
\pgfsetdash{}{0pt}%
\pgfpathmoveto{\pgfqpoint{7.338715in}{1.135525in}}%
\pgfpathlineto{\pgfqpoint{7.400828in}{1.135525in}}%
\pgfpathmoveto{\pgfqpoint{7.369771in}{1.104468in}}%
\pgfpathlineto{\pgfqpoint{7.369771in}{1.166581in}}%
\pgfusepath{stroke,fill}%
\end{pgfscope}%
\begin{pgfscope}%
\pgfsetbuttcap%
\pgfsetroundjoin%
\definecolor{currentfill}{rgb}{1.000000,0.000000,0.000000}%
\pgfsetfillcolor{currentfill}%
\pgfsetlinewidth{2.007500pt}%
\definecolor{currentstroke}{rgb}{1.000000,0.000000,0.000000}%
\pgfsetstrokecolor{currentstroke}%
\pgfsetdash{}{0pt}%
\pgfpathmoveto{\pgfqpoint{7.338715in}{1.135525in}}%
\pgfpathlineto{\pgfqpoint{7.400828in}{1.135525in}}%
\pgfpathmoveto{\pgfqpoint{7.369771in}{1.104468in}}%
\pgfpathlineto{\pgfqpoint{7.369771in}{1.166581in}}%
\pgfusepath{stroke,fill}%
\end{pgfscope}%
\begin{pgfscope}%
\pgfsetbuttcap%
\pgfsetroundjoin%
\definecolor{currentfill}{rgb}{1.000000,0.000000,0.000000}%
\pgfsetfillcolor{currentfill}%
\pgfsetlinewidth{2.007500pt}%
\definecolor{currentstroke}{rgb}{1.000000,0.000000,0.000000}%
\pgfsetstrokecolor{currentstroke}%
\pgfsetdash{}{0pt}%
\pgfpathmoveto{\pgfqpoint{7.338715in}{1.135525in}}%
\pgfpathlineto{\pgfqpoint{7.400828in}{1.135525in}}%
\pgfpathmoveto{\pgfqpoint{7.369771in}{1.104468in}}%
\pgfpathlineto{\pgfqpoint{7.369771in}{1.166581in}}%
\pgfusepath{stroke,fill}%
\end{pgfscope}%
\begin{pgfscope}%
\definecolor{textcolor}{rgb}{0.150000,0.150000,0.150000}%
\pgfsetstrokecolor{textcolor}%
\pgfsetfillcolor{textcolor}%
\pgftext[x=7.619771in,y=1.099066in,left,base]{\color{textcolor}\sffamily\fontsize{10.000000}{12.000000}\selectfont train}%
\end{pgfscope}%
\begin{pgfscope}%
\pgfsetbuttcap%
\pgfsetroundjoin%
\definecolor{currentfill}{rgb}{0.000000,0.000000,0.000000}%
\pgfsetfillcolor{currentfill}%
\pgfsetlinewidth{0.301125pt}%
\definecolor{currentstroke}{rgb}{0.000000,0.000000,0.000000}%
\pgfsetstrokecolor{currentstroke}%
\pgfsetdash{}{0pt}%
\pgfpathmoveto{\pgfqpoint{7.369771in}{0.923531in}}%
\pgfpathcurveto{\pgfqpoint{7.373889in}{0.923531in}}{\pgfqpoint{7.377839in}{0.925167in}}{\pgfqpoint{7.380751in}{0.928079in}}%
\pgfpathcurveto{\pgfqpoint{7.383663in}{0.930991in}}{\pgfqpoint{7.385299in}{0.934941in}}{\pgfqpoint{7.385299in}{0.939060in}}%
\pgfpathcurveto{\pgfqpoint{7.385299in}{0.943178in}}{\pgfqpoint{7.383663in}{0.947128in}}{\pgfqpoint{7.380751in}{0.950040in}}%
\pgfpathcurveto{\pgfqpoint{7.377839in}{0.952952in}}{\pgfqpoint{7.373889in}{0.954588in}}{\pgfqpoint{7.369771in}{0.954588in}}%
\pgfpathcurveto{\pgfqpoint{7.365653in}{0.954588in}}{\pgfqpoint{7.361703in}{0.952952in}}{\pgfqpoint{7.358791in}{0.950040in}}%
\pgfpathcurveto{\pgfqpoint{7.355879in}{0.947128in}}{\pgfqpoint{7.354243in}{0.943178in}}{\pgfqpoint{7.354243in}{0.939060in}}%
\pgfpathcurveto{\pgfqpoint{7.354243in}{0.934941in}}{\pgfqpoint{7.355879in}{0.930991in}}{\pgfqpoint{7.358791in}{0.928079in}}%
\pgfpathcurveto{\pgfqpoint{7.361703in}{0.925167in}}{\pgfqpoint{7.365653in}{0.923531in}}{\pgfqpoint{7.369771in}{0.923531in}}%
\pgfpathclose%
\pgfusepath{stroke,fill}%
\end{pgfscope}%
\begin{pgfscope}%
\pgfsetbuttcap%
\pgfsetroundjoin%
\definecolor{currentfill}{rgb}{0.000000,0.000000,0.000000}%
\pgfsetfillcolor{currentfill}%
\pgfsetlinewidth{0.301125pt}%
\definecolor{currentstroke}{rgb}{0.000000,0.000000,0.000000}%
\pgfsetstrokecolor{currentstroke}%
\pgfsetdash{}{0pt}%
\pgfpathmoveto{\pgfqpoint{7.369771in}{0.923531in}}%
\pgfpathcurveto{\pgfqpoint{7.373889in}{0.923531in}}{\pgfqpoint{7.377839in}{0.925167in}}{\pgfqpoint{7.380751in}{0.928079in}}%
\pgfpathcurveto{\pgfqpoint{7.383663in}{0.930991in}}{\pgfqpoint{7.385299in}{0.934941in}}{\pgfqpoint{7.385299in}{0.939060in}}%
\pgfpathcurveto{\pgfqpoint{7.385299in}{0.943178in}}{\pgfqpoint{7.383663in}{0.947128in}}{\pgfqpoint{7.380751in}{0.950040in}}%
\pgfpathcurveto{\pgfqpoint{7.377839in}{0.952952in}}{\pgfqpoint{7.373889in}{0.954588in}}{\pgfqpoint{7.369771in}{0.954588in}}%
\pgfpathcurveto{\pgfqpoint{7.365653in}{0.954588in}}{\pgfqpoint{7.361703in}{0.952952in}}{\pgfqpoint{7.358791in}{0.950040in}}%
\pgfpathcurveto{\pgfqpoint{7.355879in}{0.947128in}}{\pgfqpoint{7.354243in}{0.943178in}}{\pgfqpoint{7.354243in}{0.939060in}}%
\pgfpathcurveto{\pgfqpoint{7.354243in}{0.934941in}}{\pgfqpoint{7.355879in}{0.930991in}}{\pgfqpoint{7.358791in}{0.928079in}}%
\pgfpathcurveto{\pgfqpoint{7.361703in}{0.925167in}}{\pgfqpoint{7.365653in}{0.923531in}}{\pgfqpoint{7.369771in}{0.923531in}}%
\pgfpathclose%
\pgfusepath{stroke,fill}%
\end{pgfscope}%
\begin{pgfscope}%
\pgfsetbuttcap%
\pgfsetroundjoin%
\definecolor{currentfill}{rgb}{0.000000,0.000000,0.000000}%
\pgfsetfillcolor{currentfill}%
\pgfsetlinewidth{0.301125pt}%
\definecolor{currentstroke}{rgb}{0.000000,0.000000,0.000000}%
\pgfsetstrokecolor{currentstroke}%
\pgfsetdash{}{0pt}%
\pgfpathmoveto{\pgfqpoint{7.369771in}{0.923531in}}%
\pgfpathcurveto{\pgfqpoint{7.373889in}{0.923531in}}{\pgfqpoint{7.377839in}{0.925167in}}{\pgfqpoint{7.380751in}{0.928079in}}%
\pgfpathcurveto{\pgfqpoint{7.383663in}{0.930991in}}{\pgfqpoint{7.385299in}{0.934941in}}{\pgfqpoint{7.385299in}{0.939060in}}%
\pgfpathcurveto{\pgfqpoint{7.385299in}{0.943178in}}{\pgfqpoint{7.383663in}{0.947128in}}{\pgfqpoint{7.380751in}{0.950040in}}%
\pgfpathcurveto{\pgfqpoint{7.377839in}{0.952952in}}{\pgfqpoint{7.373889in}{0.954588in}}{\pgfqpoint{7.369771in}{0.954588in}}%
\pgfpathcurveto{\pgfqpoint{7.365653in}{0.954588in}}{\pgfqpoint{7.361703in}{0.952952in}}{\pgfqpoint{7.358791in}{0.950040in}}%
\pgfpathcurveto{\pgfqpoint{7.355879in}{0.947128in}}{\pgfqpoint{7.354243in}{0.943178in}}{\pgfqpoint{7.354243in}{0.939060in}}%
\pgfpathcurveto{\pgfqpoint{7.354243in}{0.934941in}}{\pgfqpoint{7.355879in}{0.930991in}}{\pgfqpoint{7.358791in}{0.928079in}}%
\pgfpathcurveto{\pgfqpoint{7.361703in}{0.925167in}}{\pgfqpoint{7.365653in}{0.923531in}}{\pgfqpoint{7.369771in}{0.923531in}}%
\pgfpathclose%
\pgfusepath{stroke,fill}%
\end{pgfscope}%
\begin{pgfscope}%
\definecolor{textcolor}{rgb}{0.150000,0.150000,0.150000}%
\pgfsetstrokecolor{textcolor}%
\pgfsetfillcolor{textcolor}%
\pgftext[x=7.619771in,y=0.902601in,left,base]{\color{textcolor}\sffamily\fontsize{10.000000}{12.000000}\selectfont test}%
\end{pgfscope}%
\begin{pgfscope}%
\pgfsetbuttcap%
\pgfsetmiterjoin%
\definecolor{currentfill}{rgb}{1.000000,1.000000,1.000000}%
\pgfsetfillcolor{currentfill}%
\pgfsetlinewidth{0.000000pt}%
\definecolor{currentstroke}{rgb}{0.000000,0.000000,0.000000}%
\pgfsetstrokecolor{currentstroke}%
\pgfsetstrokeopacity{0.000000}%
\pgfsetdash{}{0pt}%
\pgfpathmoveto{\pgfqpoint{12.211765in}{0.750000in}}%
\pgfpathlineto{\pgfqpoint{14.400000in}{0.750000in}}%
\pgfpathlineto{\pgfqpoint{14.400000in}{1.728947in}}%
\pgfpathlineto{\pgfqpoint{12.211765in}{1.728947in}}%
\pgfpathclose%
\pgfusepath{fill}%
\end{pgfscope}%
\begin{pgfscope}%
\pgfpathrectangle{\pgfqpoint{12.211765in}{0.750000in}}{\pgfqpoint{2.188235in}{0.978947in}} %
\pgfusepath{clip}%
\pgfsetroundcap%
\pgfsetroundjoin%
\pgfsetlinewidth{1.003750pt}%
\definecolor{currentstroke}{rgb}{0.800000,0.800000,0.800000}%
\pgfsetstrokecolor{currentstroke}%
\pgfsetdash{}{0pt}%
\pgfpathmoveto{\pgfqpoint{12.211765in}{0.750000in}}%
\pgfpathlineto{\pgfqpoint{12.211765in}{1.728947in}}%
\pgfusepath{stroke}%
\end{pgfscope}%
\begin{pgfscope}%
\definecolor{textcolor}{rgb}{0.150000,0.150000,0.150000}%
\pgfsetstrokecolor{textcolor}%
\pgfsetfillcolor{textcolor}%
\pgftext[x=12.211765in,y=0.652778in,,top]{\color{textcolor}\sffamily\fontsize{10.000000}{12.000000}\selectfont \(\displaystyle -2.0\)}%
\end{pgfscope}%
\begin{pgfscope}%
\pgfpathrectangle{\pgfqpoint{12.211765in}{0.750000in}}{\pgfqpoint{2.188235in}{0.978947in}} %
\pgfusepath{clip}%
\pgfsetroundcap%
\pgfsetroundjoin%
\pgfsetlinewidth{1.003750pt}%
\definecolor{currentstroke}{rgb}{0.800000,0.800000,0.800000}%
\pgfsetstrokecolor{currentstroke}%
\pgfsetdash{}{0pt}%
\pgfpathmoveto{\pgfqpoint{12.485294in}{0.750000in}}%
\pgfpathlineto{\pgfqpoint{12.485294in}{1.728947in}}%
\pgfusepath{stroke}%
\end{pgfscope}%
\begin{pgfscope}%
\definecolor{textcolor}{rgb}{0.150000,0.150000,0.150000}%
\pgfsetstrokecolor{textcolor}%
\pgfsetfillcolor{textcolor}%
\pgftext[x=12.485294in,y=0.652778in,,top]{\color{textcolor}\sffamily\fontsize{10.000000}{12.000000}\selectfont \(\displaystyle -1.5\)}%
\end{pgfscope}%
\begin{pgfscope}%
\pgfpathrectangle{\pgfqpoint{12.211765in}{0.750000in}}{\pgfqpoint{2.188235in}{0.978947in}} %
\pgfusepath{clip}%
\pgfsetroundcap%
\pgfsetroundjoin%
\pgfsetlinewidth{1.003750pt}%
\definecolor{currentstroke}{rgb}{0.800000,0.800000,0.800000}%
\pgfsetstrokecolor{currentstroke}%
\pgfsetdash{}{0pt}%
\pgfpathmoveto{\pgfqpoint{12.758824in}{0.750000in}}%
\pgfpathlineto{\pgfqpoint{12.758824in}{1.728947in}}%
\pgfusepath{stroke}%
\end{pgfscope}%
\begin{pgfscope}%
\definecolor{textcolor}{rgb}{0.150000,0.150000,0.150000}%
\pgfsetstrokecolor{textcolor}%
\pgfsetfillcolor{textcolor}%
\pgftext[x=12.758824in,y=0.652778in,,top]{\color{textcolor}\sffamily\fontsize{10.000000}{12.000000}\selectfont \(\displaystyle -1.0\)}%
\end{pgfscope}%
\begin{pgfscope}%
\pgfpathrectangle{\pgfqpoint{12.211765in}{0.750000in}}{\pgfqpoint{2.188235in}{0.978947in}} %
\pgfusepath{clip}%
\pgfsetroundcap%
\pgfsetroundjoin%
\pgfsetlinewidth{1.003750pt}%
\definecolor{currentstroke}{rgb}{0.800000,0.800000,0.800000}%
\pgfsetstrokecolor{currentstroke}%
\pgfsetdash{}{0pt}%
\pgfpathmoveto{\pgfqpoint{13.032353in}{0.750000in}}%
\pgfpathlineto{\pgfqpoint{13.032353in}{1.728947in}}%
\pgfusepath{stroke}%
\end{pgfscope}%
\begin{pgfscope}%
\definecolor{textcolor}{rgb}{0.150000,0.150000,0.150000}%
\pgfsetstrokecolor{textcolor}%
\pgfsetfillcolor{textcolor}%
\pgftext[x=13.032353in,y=0.652778in,,top]{\color{textcolor}\sffamily\fontsize{10.000000}{12.000000}\selectfont \(\displaystyle -0.5\)}%
\end{pgfscope}%
\begin{pgfscope}%
\pgfpathrectangle{\pgfqpoint{12.211765in}{0.750000in}}{\pgfqpoint{2.188235in}{0.978947in}} %
\pgfusepath{clip}%
\pgfsetroundcap%
\pgfsetroundjoin%
\pgfsetlinewidth{1.003750pt}%
\definecolor{currentstroke}{rgb}{0.800000,0.800000,0.800000}%
\pgfsetstrokecolor{currentstroke}%
\pgfsetdash{}{0pt}%
\pgfpathmoveto{\pgfqpoint{13.305882in}{0.750000in}}%
\pgfpathlineto{\pgfqpoint{13.305882in}{1.728947in}}%
\pgfusepath{stroke}%
\end{pgfscope}%
\begin{pgfscope}%
\definecolor{textcolor}{rgb}{0.150000,0.150000,0.150000}%
\pgfsetstrokecolor{textcolor}%
\pgfsetfillcolor{textcolor}%
\pgftext[x=13.305882in,y=0.652778in,,top]{\color{textcolor}\sffamily\fontsize{10.000000}{12.000000}\selectfont \(\displaystyle 0.0\)}%
\end{pgfscope}%
\begin{pgfscope}%
\pgfpathrectangle{\pgfqpoint{12.211765in}{0.750000in}}{\pgfqpoint{2.188235in}{0.978947in}} %
\pgfusepath{clip}%
\pgfsetroundcap%
\pgfsetroundjoin%
\pgfsetlinewidth{1.003750pt}%
\definecolor{currentstroke}{rgb}{0.800000,0.800000,0.800000}%
\pgfsetstrokecolor{currentstroke}%
\pgfsetdash{}{0pt}%
\pgfpathmoveto{\pgfqpoint{13.579412in}{0.750000in}}%
\pgfpathlineto{\pgfqpoint{13.579412in}{1.728947in}}%
\pgfusepath{stroke}%
\end{pgfscope}%
\begin{pgfscope}%
\definecolor{textcolor}{rgb}{0.150000,0.150000,0.150000}%
\pgfsetstrokecolor{textcolor}%
\pgfsetfillcolor{textcolor}%
\pgftext[x=13.579412in,y=0.652778in,,top]{\color{textcolor}\sffamily\fontsize{10.000000}{12.000000}\selectfont \(\displaystyle 0.5\)}%
\end{pgfscope}%
\begin{pgfscope}%
\pgfpathrectangle{\pgfqpoint{12.211765in}{0.750000in}}{\pgfqpoint{2.188235in}{0.978947in}} %
\pgfusepath{clip}%
\pgfsetroundcap%
\pgfsetroundjoin%
\pgfsetlinewidth{1.003750pt}%
\definecolor{currentstroke}{rgb}{0.800000,0.800000,0.800000}%
\pgfsetstrokecolor{currentstroke}%
\pgfsetdash{}{0pt}%
\pgfpathmoveto{\pgfqpoint{13.852941in}{0.750000in}}%
\pgfpathlineto{\pgfqpoint{13.852941in}{1.728947in}}%
\pgfusepath{stroke}%
\end{pgfscope}%
\begin{pgfscope}%
\definecolor{textcolor}{rgb}{0.150000,0.150000,0.150000}%
\pgfsetstrokecolor{textcolor}%
\pgfsetfillcolor{textcolor}%
\pgftext[x=13.852941in,y=0.652778in,,top]{\color{textcolor}\sffamily\fontsize{10.000000}{12.000000}\selectfont \(\displaystyle 1.0\)}%
\end{pgfscope}%
\begin{pgfscope}%
\pgfpathrectangle{\pgfqpoint{12.211765in}{0.750000in}}{\pgfqpoint{2.188235in}{0.978947in}} %
\pgfusepath{clip}%
\pgfsetroundcap%
\pgfsetroundjoin%
\pgfsetlinewidth{1.003750pt}%
\definecolor{currentstroke}{rgb}{0.800000,0.800000,0.800000}%
\pgfsetstrokecolor{currentstroke}%
\pgfsetdash{}{0pt}%
\pgfpathmoveto{\pgfqpoint{14.126471in}{0.750000in}}%
\pgfpathlineto{\pgfqpoint{14.126471in}{1.728947in}}%
\pgfusepath{stroke}%
\end{pgfscope}%
\begin{pgfscope}%
\definecolor{textcolor}{rgb}{0.150000,0.150000,0.150000}%
\pgfsetstrokecolor{textcolor}%
\pgfsetfillcolor{textcolor}%
\pgftext[x=14.126471in,y=0.652778in,,top]{\color{textcolor}\sffamily\fontsize{10.000000}{12.000000}\selectfont \(\displaystyle 1.5\)}%
\end{pgfscope}%
\begin{pgfscope}%
\pgfpathrectangle{\pgfqpoint{12.211765in}{0.750000in}}{\pgfqpoint{2.188235in}{0.978947in}} %
\pgfusepath{clip}%
\pgfsetroundcap%
\pgfsetroundjoin%
\pgfsetlinewidth{1.003750pt}%
\definecolor{currentstroke}{rgb}{0.800000,0.800000,0.800000}%
\pgfsetstrokecolor{currentstroke}%
\pgfsetdash{}{0pt}%
\pgfpathmoveto{\pgfqpoint{14.400000in}{0.750000in}}%
\pgfpathlineto{\pgfqpoint{14.400000in}{1.728947in}}%
\pgfusepath{stroke}%
\end{pgfscope}%
\begin{pgfscope}%
\definecolor{textcolor}{rgb}{0.150000,0.150000,0.150000}%
\pgfsetstrokecolor{textcolor}%
\pgfsetfillcolor{textcolor}%
\pgftext[x=14.400000in,y=0.652778in,,top]{\color{textcolor}\sffamily\fontsize{10.000000}{12.000000}\selectfont \(\displaystyle 2.0\)}%
\end{pgfscope}%
\begin{pgfscope}%
\definecolor{textcolor}{rgb}{0.150000,0.150000,0.150000}%
\pgfsetstrokecolor{textcolor}%
\pgfsetfillcolor{textcolor}%
\pgftext[x=13.305882in,y=0.456313in,,top]{\color{textcolor}\sffamily\fontsize{11.000000}{13.200000}\selectfont \(\displaystyle j\)}%
\end{pgfscope}%
\begin{pgfscope}%
\definecolor{textcolor}{rgb}{0.150000,0.150000,0.150000}%
\pgfsetstrokecolor{textcolor}%
\pgfsetfillcolor{textcolor}%
\pgftext[x=14.400000in,y=0.484090in,right,top]{\color{textcolor}\sffamily\fontsize{10.000000}{12.000000}\selectfont \(\displaystyle \times10^{5}\)}%
\end{pgfscope}%
\begin{pgfscope}%
\pgfpathrectangle{\pgfqpoint{12.211765in}{0.750000in}}{\pgfqpoint{2.188235in}{0.978947in}} %
\pgfusepath{clip}%
\pgfsetroundcap%
\pgfsetroundjoin%
\pgfsetlinewidth{1.003750pt}%
\definecolor{currentstroke}{rgb}{0.800000,0.800000,0.800000}%
\pgfsetstrokecolor{currentstroke}%
\pgfsetdash{}{0pt}%
\pgfpathmoveto{\pgfqpoint{12.211765in}{0.750000in}}%
\pgfpathlineto{\pgfqpoint{14.400000in}{0.750000in}}%
\pgfusepath{stroke}%
\end{pgfscope}%
\begin{pgfscope}%
\definecolor{textcolor}{rgb}{0.150000,0.150000,0.150000}%
\pgfsetstrokecolor{textcolor}%
\pgfsetfillcolor{textcolor}%
\pgftext[x=12.114542in,y=0.750000in,right,]{\color{textcolor}\sffamily\fontsize{10.000000}{12.000000}\selectfont \(\displaystyle 0\)}%
\end{pgfscope}%
\begin{pgfscope}%
\pgfpathrectangle{\pgfqpoint{12.211765in}{0.750000in}}{\pgfqpoint{2.188235in}{0.978947in}} %
\pgfusepath{clip}%
\pgfsetroundcap%
\pgfsetroundjoin%
\pgfsetlinewidth{1.003750pt}%
\definecolor{currentstroke}{rgb}{0.800000,0.800000,0.800000}%
\pgfsetstrokecolor{currentstroke}%
\pgfsetdash{}{0pt}%
\pgfpathmoveto{\pgfqpoint{12.211765in}{0.994737in}}%
\pgfpathlineto{\pgfqpoint{14.400000in}{0.994737in}}%
\pgfusepath{stroke}%
\end{pgfscope}%
\begin{pgfscope}%
\definecolor{textcolor}{rgb}{0.150000,0.150000,0.150000}%
\pgfsetstrokecolor{textcolor}%
\pgfsetfillcolor{textcolor}%
\pgftext[x=12.114542in,y=0.994737in,right,]{\color{textcolor}\sffamily\fontsize{10.000000}{12.000000}\selectfont \(\displaystyle 50\)}%
\end{pgfscope}%
\begin{pgfscope}%
\pgfpathrectangle{\pgfqpoint{12.211765in}{0.750000in}}{\pgfqpoint{2.188235in}{0.978947in}} %
\pgfusepath{clip}%
\pgfsetroundcap%
\pgfsetroundjoin%
\pgfsetlinewidth{1.003750pt}%
\definecolor{currentstroke}{rgb}{0.800000,0.800000,0.800000}%
\pgfsetstrokecolor{currentstroke}%
\pgfsetdash{}{0pt}%
\pgfpathmoveto{\pgfqpoint{12.211765in}{1.239474in}}%
\pgfpathlineto{\pgfqpoint{14.400000in}{1.239474in}}%
\pgfusepath{stroke}%
\end{pgfscope}%
\begin{pgfscope}%
\definecolor{textcolor}{rgb}{0.150000,0.150000,0.150000}%
\pgfsetstrokecolor{textcolor}%
\pgfsetfillcolor{textcolor}%
\pgftext[x=12.114542in,y=1.239474in,right,]{\color{textcolor}\sffamily\fontsize{10.000000}{12.000000}\selectfont \(\displaystyle 100\)}%
\end{pgfscope}%
\begin{pgfscope}%
\pgfpathrectangle{\pgfqpoint{12.211765in}{0.750000in}}{\pgfqpoint{2.188235in}{0.978947in}} %
\pgfusepath{clip}%
\pgfsetroundcap%
\pgfsetroundjoin%
\pgfsetlinewidth{1.003750pt}%
\definecolor{currentstroke}{rgb}{0.800000,0.800000,0.800000}%
\pgfsetstrokecolor{currentstroke}%
\pgfsetdash{}{0pt}%
\pgfpathmoveto{\pgfqpoint{12.211765in}{1.484211in}}%
\pgfpathlineto{\pgfqpoint{14.400000in}{1.484211in}}%
\pgfusepath{stroke}%
\end{pgfscope}%
\begin{pgfscope}%
\definecolor{textcolor}{rgb}{0.150000,0.150000,0.150000}%
\pgfsetstrokecolor{textcolor}%
\pgfsetfillcolor{textcolor}%
\pgftext[x=12.114542in,y=1.484211in,right,]{\color{textcolor}\sffamily\fontsize{10.000000}{12.000000}\selectfont \(\displaystyle 150\)}%
\end{pgfscope}%
\begin{pgfscope}%
\pgfpathrectangle{\pgfqpoint{12.211765in}{0.750000in}}{\pgfqpoint{2.188235in}{0.978947in}} %
\pgfusepath{clip}%
\pgfsetroundcap%
\pgfsetroundjoin%
\pgfsetlinewidth{1.003750pt}%
\definecolor{currentstroke}{rgb}{0.800000,0.800000,0.800000}%
\pgfsetstrokecolor{currentstroke}%
\pgfsetdash{}{0pt}%
\pgfpathmoveto{\pgfqpoint{12.211765in}{1.728947in}}%
\pgfpathlineto{\pgfqpoint{14.400000in}{1.728947in}}%
\pgfusepath{stroke}%
\end{pgfscope}%
\begin{pgfscope}%
\definecolor{textcolor}{rgb}{0.150000,0.150000,0.150000}%
\pgfsetstrokecolor{textcolor}%
\pgfsetfillcolor{textcolor}%
\pgftext[x=12.114542in,y=1.728947in,right,]{\color{textcolor}\sffamily\fontsize{10.000000}{12.000000}\selectfont \(\displaystyle 200\)}%
\end{pgfscope}%
\begin{pgfscope}%
\definecolor{textcolor}{rgb}{0.150000,0.150000,0.150000}%
\pgfsetstrokecolor{textcolor}%
\pgfsetfillcolor{textcolor}%
\pgftext[x=11.836764in,y=1.239474in,,bottom,rotate=90.000000]{\color{textcolor}\sffamily\fontsize{11.000000}{13.200000}\selectfont \(\displaystyle \theta^{\parallel}_j\)}%
\end{pgfscope}%
\begin{pgfscope}%
\pgfpathrectangle{\pgfqpoint{12.211765in}{0.750000in}}{\pgfqpoint{2.188235in}{0.978947in}} %
\pgfusepath{clip}%
\pgfsetroundcap%
\pgfsetroundjoin%
\pgfsetlinewidth{1.756562pt}%
\definecolor{currentstroke}{rgb}{0.298039,0.447059,0.690196}%
\pgfsetstrokecolor{currentstroke}%
\pgfsetdash{}{0pt}%
\pgfpathmoveto{\pgfqpoint{13.318583in}{0.750000in}}%
\pgfpathlineto{\pgfqpoint{13.319716in}{0.754895in}}%
\pgfpathlineto{\pgfqpoint{13.505278in}{0.759789in}}%
\pgfpathlineto{\pgfqpoint{13.088970in}{0.764684in}}%
\pgfpathlineto{\pgfqpoint{13.009376in}{0.769579in}}%
\pgfpathlineto{\pgfqpoint{13.306678in}{0.774474in}}%
\pgfpathlineto{\pgfqpoint{12.835495in}{0.779368in}}%
\pgfpathlineto{\pgfqpoint{13.427294in}{0.789158in}}%
\pgfpathlineto{\pgfqpoint{12.873136in}{0.794053in}}%
\pgfpathlineto{\pgfqpoint{13.296390in}{0.798947in}}%
\pgfpathlineto{\pgfqpoint{13.125624in}{0.803842in}}%
\pgfpathlineto{\pgfqpoint{13.722766in}{0.813632in}}%
\pgfpathlineto{\pgfqpoint{13.222671in}{0.818526in}}%
\pgfpathlineto{\pgfqpoint{13.009751in}{0.823421in}}%
\pgfpathlineto{\pgfqpoint{14.262259in}{0.828316in}}%
\pgfpathlineto{\pgfqpoint{12.589164in}{0.833211in}}%
\pgfpathlineto{\pgfqpoint{12.347675in}{0.838105in}}%
\pgfpathlineto{\pgfqpoint{13.300604in}{0.843000in}}%
\pgfpathlineto{\pgfqpoint{13.283917in}{0.847895in}}%
\pgfpathlineto{\pgfqpoint{13.643233in}{0.852789in}}%
\pgfpathlineto{\pgfqpoint{13.609389in}{0.857684in}}%
\pgfpathlineto{\pgfqpoint{14.079650in}{0.862579in}}%
\pgfpathlineto{\pgfqpoint{13.331759in}{0.867474in}}%
\pgfpathlineto{\pgfqpoint{13.431527in}{0.872368in}}%
\pgfpathlineto{\pgfqpoint{13.283798in}{0.877263in}}%
\pgfpathlineto{\pgfqpoint{13.311260in}{0.882158in}}%
\pgfpathlineto{\pgfqpoint{13.028419in}{0.887053in}}%
\pgfpathlineto{\pgfqpoint{13.205100in}{0.891947in}}%
\pgfpathlineto{\pgfqpoint{13.464490in}{0.896842in}}%
\pgfpathlineto{\pgfqpoint{13.372658in}{0.901737in}}%
\pgfpathlineto{\pgfqpoint{13.379991in}{0.906632in}}%
\pgfpathlineto{\pgfqpoint{13.275185in}{0.911526in}}%
\pgfpathlineto{\pgfqpoint{13.307792in}{0.916421in}}%
\pgfpathlineto{\pgfqpoint{13.302084in}{0.921316in}}%
\pgfpathlineto{\pgfqpoint{13.321845in}{0.926211in}}%
\pgfpathlineto{\pgfqpoint{13.306216in}{0.931105in}}%
\pgfpathlineto{\pgfqpoint{13.320266in}{0.936000in}}%
\pgfpathlineto{\pgfqpoint{13.269674in}{0.945789in}}%
\pgfpathlineto{\pgfqpoint{13.293961in}{0.950684in}}%
\pgfpathlineto{\pgfqpoint{13.253555in}{0.955579in}}%
\pgfpathlineto{\pgfqpoint{13.284425in}{0.960474in}}%
\pgfpathlineto{\pgfqpoint{13.264993in}{0.965368in}}%
\pgfpathlineto{\pgfqpoint{13.326918in}{0.970263in}}%
\pgfpathlineto{\pgfqpoint{13.341694in}{0.975158in}}%
\pgfpathlineto{\pgfqpoint{13.375413in}{0.980053in}}%
\pgfpathlineto{\pgfqpoint{13.264599in}{0.984947in}}%
\pgfpathlineto{\pgfqpoint{13.368765in}{0.989842in}}%
\pgfpathlineto{\pgfqpoint{13.347897in}{0.994737in}}%
\pgfpathlineto{\pgfqpoint{13.239606in}{0.999632in}}%
\pgfpathlineto{\pgfqpoint{13.315040in}{1.004526in}}%
\pgfpathlineto{\pgfqpoint{13.198593in}{1.009421in}}%
\pgfpathlineto{\pgfqpoint{13.411475in}{1.014316in}}%
\pgfpathlineto{\pgfqpoint{13.348855in}{1.019211in}}%
\pgfpathlineto{\pgfqpoint{13.365114in}{1.024105in}}%
\pgfpathlineto{\pgfqpoint{13.250677in}{1.029000in}}%
\pgfpathlineto{\pgfqpoint{13.195021in}{1.033895in}}%
\pgfpathlineto{\pgfqpoint{13.350873in}{1.038789in}}%
\pgfpathlineto{\pgfqpoint{13.319205in}{1.043684in}}%
\pgfpathlineto{\pgfqpoint{13.250544in}{1.048579in}}%
\pgfpathlineto{\pgfqpoint{13.212046in}{1.053474in}}%
\pgfpathlineto{\pgfqpoint{13.578251in}{1.063263in}}%
\pgfpathlineto{\pgfqpoint{13.405082in}{1.068158in}}%
\pgfpathlineto{\pgfqpoint{13.336883in}{1.077947in}}%
\pgfpathlineto{\pgfqpoint{13.262717in}{1.082842in}}%
\pgfpathlineto{\pgfqpoint{13.292784in}{1.087737in}}%
\pgfpathlineto{\pgfqpoint{13.277665in}{1.092632in}}%
\pgfpathlineto{\pgfqpoint{13.279471in}{1.097526in}}%
\pgfpathlineto{\pgfqpoint{13.299348in}{1.102421in}}%
\pgfpathlineto{\pgfqpoint{13.426340in}{1.107316in}}%
\pgfpathlineto{\pgfqpoint{13.360732in}{1.112211in}}%
\pgfpathlineto{\pgfqpoint{13.319368in}{1.117105in}}%
\pgfpathlineto{\pgfqpoint{13.315122in}{1.122000in}}%
\pgfpathlineto{\pgfqpoint{13.347821in}{1.126895in}}%
\pgfpathlineto{\pgfqpoint{13.310284in}{1.131789in}}%
\pgfpathlineto{\pgfqpoint{13.394929in}{1.136684in}}%
\pgfpathlineto{\pgfqpoint{13.386711in}{1.141579in}}%
\pgfpathlineto{\pgfqpoint{13.312265in}{1.146474in}}%
\pgfpathlineto{\pgfqpoint{13.318126in}{1.151368in}}%
\pgfpathlineto{\pgfqpoint{13.321795in}{1.156263in}}%
\pgfpathlineto{\pgfqpoint{13.338412in}{1.161158in}}%
\pgfpathlineto{\pgfqpoint{13.289559in}{1.166053in}}%
\pgfpathlineto{\pgfqpoint{13.292515in}{1.170947in}}%
\pgfpathlineto{\pgfqpoint{13.373984in}{1.175842in}}%
\pgfpathlineto{\pgfqpoint{13.197679in}{1.180737in}}%
\pgfpathlineto{\pgfqpoint{13.194352in}{1.185632in}}%
\pgfpathlineto{\pgfqpoint{13.279855in}{1.190526in}}%
\pgfpathlineto{\pgfqpoint{13.304310in}{1.195421in}}%
\pgfpathlineto{\pgfqpoint{13.381147in}{1.200316in}}%
\pgfpathlineto{\pgfqpoint{13.309790in}{1.205211in}}%
\pgfpathlineto{\pgfqpoint{13.253022in}{1.215000in}}%
\pgfpathlineto{\pgfqpoint{13.337301in}{1.219895in}}%
\pgfpathlineto{\pgfqpoint{13.235095in}{1.224789in}}%
\pgfpathlineto{\pgfqpoint{13.346844in}{1.229684in}}%
\pgfpathlineto{\pgfqpoint{13.379070in}{1.234579in}}%
\pgfpathlineto{\pgfqpoint{13.232800in}{1.249263in}}%
\pgfpathlineto{\pgfqpoint{13.290115in}{1.254158in}}%
\pgfpathlineto{\pgfqpoint{13.262587in}{1.259053in}}%
\pgfpathlineto{\pgfqpoint{13.352596in}{1.263947in}}%
\pgfpathlineto{\pgfqpoint{13.323984in}{1.268842in}}%
\pgfpathlineto{\pgfqpoint{13.309931in}{1.273737in}}%
\pgfpathlineto{\pgfqpoint{13.329651in}{1.278632in}}%
\pgfpathlineto{\pgfqpoint{13.327334in}{1.283526in}}%
\pgfpathlineto{\pgfqpoint{13.295086in}{1.288421in}}%
\pgfpathlineto{\pgfqpoint{13.304888in}{1.293316in}}%
\pgfpathlineto{\pgfqpoint{13.325003in}{1.298211in}}%
\pgfpathlineto{\pgfqpoint{13.260809in}{1.303105in}}%
\pgfpathlineto{\pgfqpoint{13.319008in}{1.308000in}}%
\pgfpathlineto{\pgfqpoint{13.314798in}{1.312895in}}%
\pgfpathlineto{\pgfqpoint{13.259183in}{1.317789in}}%
\pgfpathlineto{\pgfqpoint{13.321789in}{1.322684in}}%
\pgfpathlineto{\pgfqpoint{13.287730in}{1.327579in}}%
\pgfpathlineto{\pgfqpoint{13.305615in}{1.332474in}}%
\pgfpathlineto{\pgfqpoint{13.288911in}{1.337368in}}%
\pgfpathlineto{\pgfqpoint{13.352404in}{1.342263in}}%
\pgfpathlineto{\pgfqpoint{13.267461in}{1.347158in}}%
\pgfpathlineto{\pgfqpoint{13.289447in}{1.352053in}}%
\pgfpathlineto{\pgfqpoint{13.273537in}{1.356947in}}%
\pgfpathlineto{\pgfqpoint{13.286382in}{1.361842in}}%
\pgfpathlineto{\pgfqpoint{13.284816in}{1.366737in}}%
\pgfpathlineto{\pgfqpoint{13.307654in}{1.371632in}}%
\pgfpathlineto{\pgfqpoint{13.313491in}{1.376526in}}%
\pgfpathlineto{\pgfqpoint{13.334709in}{1.381421in}}%
\pgfpathlineto{\pgfqpoint{13.376685in}{1.386316in}}%
\pgfpathlineto{\pgfqpoint{13.335922in}{1.391211in}}%
\pgfpathlineto{\pgfqpoint{13.311644in}{1.396105in}}%
\pgfpathlineto{\pgfqpoint{13.310785in}{1.401000in}}%
\pgfpathlineto{\pgfqpoint{13.299466in}{1.405895in}}%
\pgfpathlineto{\pgfqpoint{13.308094in}{1.410789in}}%
\pgfpathlineto{\pgfqpoint{13.351926in}{1.415684in}}%
\pgfpathlineto{\pgfqpoint{13.276298in}{1.420579in}}%
\pgfpathlineto{\pgfqpoint{13.301670in}{1.425474in}}%
\pgfpathlineto{\pgfqpoint{13.339814in}{1.430368in}}%
\pgfpathlineto{\pgfqpoint{13.308825in}{1.435263in}}%
\pgfpathlineto{\pgfqpoint{13.334439in}{1.440158in}}%
\pgfpathlineto{\pgfqpoint{13.316049in}{1.445053in}}%
\pgfpathlineto{\pgfqpoint{13.326517in}{1.449947in}}%
\pgfpathlineto{\pgfqpoint{13.312145in}{1.454842in}}%
\pgfpathlineto{\pgfqpoint{13.337528in}{1.459737in}}%
\pgfpathlineto{\pgfqpoint{13.338402in}{1.464632in}}%
\pgfpathlineto{\pgfqpoint{13.310576in}{1.469526in}}%
\pgfpathlineto{\pgfqpoint{13.309865in}{1.474421in}}%
\pgfpathlineto{\pgfqpoint{13.299039in}{1.479316in}}%
\pgfpathlineto{\pgfqpoint{13.302382in}{1.484211in}}%
\pgfpathlineto{\pgfqpoint{13.366004in}{1.489105in}}%
\pgfpathlineto{\pgfqpoint{13.380912in}{1.494000in}}%
\pgfpathlineto{\pgfqpoint{13.349933in}{1.498895in}}%
\pgfpathlineto{\pgfqpoint{13.296258in}{1.503789in}}%
\pgfpathlineto{\pgfqpoint{13.285974in}{1.508684in}}%
\pgfpathlineto{\pgfqpoint{13.299521in}{1.513579in}}%
\pgfpathlineto{\pgfqpoint{13.295012in}{1.518474in}}%
\pgfpathlineto{\pgfqpoint{13.313393in}{1.523368in}}%
\pgfpathlineto{\pgfqpoint{13.316415in}{1.528263in}}%
\pgfpathlineto{\pgfqpoint{13.317807in}{1.533158in}}%
\pgfpathlineto{\pgfqpoint{13.266680in}{1.538053in}}%
\pgfpathlineto{\pgfqpoint{13.289385in}{1.542947in}}%
\pgfpathlineto{\pgfqpoint{13.321376in}{1.547842in}}%
\pgfpathlineto{\pgfqpoint{13.274531in}{1.557632in}}%
\pgfpathlineto{\pgfqpoint{13.314857in}{1.562526in}}%
\pgfpathlineto{\pgfqpoint{13.303647in}{1.567421in}}%
\pgfpathlineto{\pgfqpoint{13.348105in}{1.572316in}}%
\pgfpathlineto{\pgfqpoint{13.273644in}{1.577211in}}%
\pgfpathlineto{\pgfqpoint{13.296569in}{1.582105in}}%
\pgfpathlineto{\pgfqpoint{13.366367in}{1.587000in}}%
\pgfpathlineto{\pgfqpoint{13.264048in}{1.591895in}}%
\pgfpathlineto{\pgfqpoint{13.323509in}{1.596789in}}%
\pgfpathlineto{\pgfqpoint{13.314195in}{1.601684in}}%
\pgfpathlineto{\pgfqpoint{13.302483in}{1.606579in}}%
\pgfpathlineto{\pgfqpoint{13.367083in}{1.611474in}}%
\pgfpathlineto{\pgfqpoint{13.357510in}{1.616368in}}%
\pgfpathlineto{\pgfqpoint{13.362290in}{1.621263in}}%
\pgfpathlineto{\pgfqpoint{13.294772in}{1.626158in}}%
\pgfpathlineto{\pgfqpoint{13.287740in}{1.631053in}}%
\pgfpathlineto{\pgfqpoint{13.321781in}{1.635947in}}%
\pgfpathlineto{\pgfqpoint{13.306540in}{1.640842in}}%
\pgfpathlineto{\pgfqpoint{13.297179in}{1.645737in}}%
\pgfpathlineto{\pgfqpoint{13.302243in}{1.650632in}}%
\pgfpathlineto{\pgfqpoint{13.378972in}{1.655526in}}%
\pgfpathlineto{\pgfqpoint{13.307493in}{1.660421in}}%
\pgfpathlineto{\pgfqpoint{13.332046in}{1.665316in}}%
\pgfpathlineto{\pgfqpoint{13.292913in}{1.670211in}}%
\pgfpathlineto{\pgfqpoint{13.306141in}{1.675105in}}%
\pgfpathlineto{\pgfqpoint{13.359208in}{1.680000in}}%
\pgfpathlineto{\pgfqpoint{13.333258in}{1.684895in}}%
\pgfpathlineto{\pgfqpoint{13.274190in}{1.689789in}}%
\pgfpathlineto{\pgfqpoint{13.281137in}{1.694684in}}%
\pgfpathlineto{\pgfqpoint{13.375858in}{1.699579in}}%
\pgfpathlineto{\pgfqpoint{13.297677in}{1.704474in}}%
\pgfpathlineto{\pgfqpoint{13.310534in}{1.709368in}}%
\pgfpathlineto{\pgfqpoint{13.240434in}{1.714263in}}%
\pgfpathlineto{\pgfqpoint{13.245646in}{1.719158in}}%
\pgfpathlineto{\pgfqpoint{13.285100in}{1.724053in}}%
\pgfpathlineto{\pgfqpoint{13.285100in}{1.724053in}}%
\pgfusepath{stroke}%
\end{pgfscope}%
\begin{pgfscope}%
\pgfsetrectcap%
\pgfsetmiterjoin%
\pgfsetlinewidth{1.003750pt}%
\definecolor{currentstroke}{rgb}{0.800000,0.800000,0.800000}%
\pgfsetstrokecolor{currentstroke}%
\pgfsetdash{}{0pt}%
\pgfpathmoveto{\pgfqpoint{12.211765in}{0.750000in}}%
\pgfpathlineto{\pgfqpoint{12.211765in}{1.728947in}}%
\pgfusepath{stroke}%
\end{pgfscope}%
\begin{pgfscope}%
\pgfsetrectcap%
\pgfsetmiterjoin%
\pgfsetlinewidth{1.003750pt}%
\definecolor{currentstroke}{rgb}{0.800000,0.800000,0.800000}%
\pgfsetstrokecolor{currentstroke}%
\pgfsetdash{}{0pt}%
\pgfpathmoveto{\pgfqpoint{14.400000in}{0.750000in}}%
\pgfpathlineto{\pgfqpoint{14.400000in}{1.728947in}}%
\pgfusepath{stroke}%
\end{pgfscope}%
\begin{pgfscope}%
\pgfsetrectcap%
\pgfsetmiterjoin%
\pgfsetlinewidth{1.003750pt}%
\definecolor{currentstroke}{rgb}{0.800000,0.800000,0.800000}%
\pgfsetstrokecolor{currentstroke}%
\pgfsetdash{}{0pt}%
\pgfpathmoveto{\pgfqpoint{12.211765in}{1.728947in}}%
\pgfpathlineto{\pgfqpoint{14.400000in}{1.728947in}}%
\pgfusepath{stroke}%
\end{pgfscope}%
\begin{pgfscope}%
\pgfsetrectcap%
\pgfsetmiterjoin%
\pgfsetlinewidth{1.003750pt}%
\definecolor{currentstroke}{rgb}{0.800000,0.800000,0.800000}%
\pgfsetstrokecolor{currentstroke}%
\pgfsetdash{}{0pt}%
\pgfpathmoveto{\pgfqpoint{12.211765in}{0.750000in}}%
\pgfpathlineto{\pgfqpoint{14.400000in}{0.750000in}}%
\pgfusepath{stroke}%
\end{pgfscope}%
\end{pgfpicture}%
\makeatother%
\endgroup%

}
\caption[ORFF Representer theorem]{ORFF Representer theorem. We trained a first model named $\tildeK{\omega}$ following}
\label{fig:rel_features}
\end{figure}
\end{landscape}


% Since $\theta\in\tildeH{\omega}\left(\mathcal{Y}'\right)$ lives in a different space than $f\in\mathcal{H}_K$ is makes no sense to show the convergence of $\theta_*$ to $f_*$. However we can verify that $\tildef{1:D}$ converges to $f\in\mathcal{H}_K$ pointwise. We first introduce the following lemma and assumptions. First we characterize extremum estimators. Let $\mathcal{H}$ be any Hilbert space.
% \begin{assumption}[EE]
% \label{ass:ee}
% The parameter $\tilde{g}_*\in\mathcal{H}$ is an extremum estimator of $Q_D$ if $\tilde{g}_*=\argmin_{g\in\mathcal{H}} Q_D(g)$.
% \end{assumption}
% To derive consistency we must ensure that the stochastic objective $Q_D$ function converges uniformly-weakly to some target function $Q_0$. Let $B(g,\epsilon)$ be the open ball in $\mathcal{H}$ with radius $\epsilon$ and $\mathcal{H}\setminus{B(g,\epsilon)}$ denotes its complement in $\mathcal{H}$.
% \begin{assumption}[U-SCON]
% \label{ass:uwcon}
% A \emph{stochastic} function $Q_D$ converges uniformly-strongly to some \emph{non-stochastic} function $Q_0$ if \begin{dmath*}
% \sup_{g\in\mathcal{H}}\abs{Q_D(g)-Q_0(g)}\converges{\asurely}{D\to\infty}0
% \end{dmath*}
% \end{assumption}
% This is equivalent to say that $Q_D$ converges to $Q_0$ in probability in the uniform norm topology. Eventually the mimizers of $Q$ must be uniquely identifiable.
% \begin{assumption}[ID]
% \label{ass:id}
% A function has unique identifiable minimizer if there exist $g_*\in\mathcal{H}$ such that for all $\epsilon\ge 0$, $\inf_{g\in \mathcal{H}\setminus{B(g_*,\epsilon)}} Q(g)>Q(g_*)$.
% \end{assumption}
% Note that if $Q$ is a strictly convex function of $g$, then \cref{ass:id} holds. These three assumptions combined yields the consistency of an estimator.
% \begin{proposition} If
% \cref{ass:ee}, \cref{ass:uwcon} and \cref{ass:id} hold then $\tilde{g}_* \converges{\asurely}{D\to\infty} g_*.$
% \end{proposition}
% \begin{proof}

% \end{proof}
% We are now ready to prove the consistency of ORFF with respect to the minimization in the RKHS, that is if we replace the model $f(x)=\sum_{i=1}^NK(\cdot, x_i)c_i$ by a model $f(x)=\tildePhi{\omega}(x)^\adjoint \theta$ and find the minimizer of the ridge regression problem, then when $D$ tends to infinity, both model return the \say{same} function.
% \begin{proposition}[Consitensy of ORFF \wrt~\acs{vv-RKHS}] Let $K$ be a $\mathcal{Y}$-Mercer Kernel. If $\theta_*\in\tildeH{\omega}\left(\mathcal{Y}'\right)$ is the minimizer of \cref{eq:argmin_applied} and $f_*\in\mathcal{H}_K$ is the minimizer of \cref{eq:argmin_RKHS}. Then
% \begin{dmath}
% \tildePhi{\omega}(x)^\adjoint\theta_* \converges{\asurely}{D\to\infty}f_*(x),
% \end{dmath}
% for all $x\in\mathcal{X}$.
% \end{proposition}
% \begin{proof}
% % Let $(Wg)(x)=\Phi_x^\adjoint g$ be the feature operator associated to $\Phi$ and $\tildePhi{\omega}(x)y=\frac{1}{\sqrt{D}}\Vect_{j=1}^D(\Phi_x y)(\omega_j)$, where $\omega_j$ are \iid~random vectors following the law $\mu$. Since $W$ is a partial isometry

% % We have that for all $f\in\mathcal{H_K}$, there exists a $g\in\mathcal{H}$ such that $(W^\adjoint f)(x)=g(x)=\Phi(x)^\adjoint g$ where $W$ is a partial isometry with initial subspace $(\Ker W)^\perp\subset\mathcal{H}$ and final subspace $\mathcal{H}_K$.

% % \begin{dmath}
% % \label{eq:argmin_HS}
% % g_*=\argmin_{g\in(\Ker W)^\perp\subset\mathcal{H}} \frac{1}{2N}\sum_{i=1}^N\norm{(Wg)(x_i)-y_i}_{\mathcal{Y}}^2 + \frac{\lambda}{2}\norm{g}^2_{\mathcal{H}}
% % \end{dmath}

% Define
% \begin{dmath}
% Q_0(g)=\frac{1}{2N}\sum_{i=1}^N\norm{(Wg)(x_i)-y_i}_{\mathcal{Y}}^2 + \frac{\lambda}{2}\norm{g}^2_{\mathcal{H}}.
% \end{dmath}
% Moreover let $\tildePhi{\omega}(x_i)y=\Vect_{j=1}^D(\Phi_{x_i}y)(\omega_j)$ for all $y\in\mathcal{Y}$, and $\theta_g=\vect_{j=1}^Dg(\omega_j)$, where $\omega_j$ are \iid~random vectors following a law $\mu$. We define the empirical estimator of $Q_0$ as
% \begin{dmath}
% Q_D(g)=\frac{1}{2N}\sum_{i=1}^N\norm{\tildePhi{\omega}(x_i)^\adjoint \theta_g-y_i}_{\mathcal{Y}}^2 + \frac{\lambda}{2D}\norm{\theta_g}^2_{\tildeH{\omega}\left(\mathcal{Y}'\right)}
% \end{dmath}
% Let $g_*=\argmin_{g\in\mathcal{H}}Q_0(g)$ and $\tilde{g}_*=\argmin_{g\in\mathcal{H}}Q_D(g)$ thus $\tilde{g}_*$ is an extremum estimator (\cref{ass:ee} holds). Notice that since $W$ is a bounded linear application, $Q_0$ is continuous strongly convex and coercive. By \cref{cor:unique_minimizer} attains an identifiable unique minimizer (\cref{ass:id} holds).

% Since W is bounded, $\Ker W$ is closed, so that we can perform the decomposition $\mathcal{H}=(\Ker W)^\perp\oplus \Ker W$. Then clearly by linearity of $W$ and the fact that for all $g\in\Ker W$, $Wg=0$,
% \begin{dmath*}
% g_*=\argmin_{g\in\mathcal{H}}Q_0(g)
% \hiderel{=}\argmin_{g\in(\Ker W)^\perp\oplus \Ker W}Q_0(g)
% \hiderel{=}\argmin_{g\in(\Ker W)^\perp}Q_0(g).
% \end{dmath*}
% Besides with similar arguments as in \cref{rk:rkhs_bound} we deduce that $\lambda\norm{g_*}_{\mathcal{H}}^2$ $\le$ $2\sigma^2_y$. As a result $g_*\in\mathcal{H}_\lambda\colonequals\Set{g\in(\Ker W)^\perp|\lambda\norm{g}_{\mathcal{H}}^2\le 2\sigma^2_y}$. Eventually
% \begin{dmath*}
% g_*=\argmin_{g\in\mathcal{H}}Q_0(g)\hiderel{=}\argmin_{g\in\mathcal{H}_\lambda}Q_0(g)
% \end{dmath*}
% \begin{enumerate}
% \item By assumption $\sup_{y\in\mathcal{Y}} \norm{y}\le \sigma_y$ thus from \cref{rk:rkhs_bound} we have that $\lambda\norm{f_*}^2_K$ is bounded by $2\sigma_y^2$ and so is $\lambda\norm{g_*}_{\mathcal{H}}^2$. We note $\mathcal{H}_K^{\lambda}=\Set{f\in\mathcal{H}_K|\lambda\norm{f}^2_K\le 2\sigma_y}$ and $\mathcal{H}^\lambda=\Set{g\in(\Ker W)^\perp|\lambda\norm{g}^2_{\mathcal{H}}\le 2\sigma_y}$.
% Then
% \begin{dmath*}
% f_*=\argmin_{f\in\mathcal{H}_K} \frac{1}{2N}\displaystyle\sum_{i=1}^N\norm{f(x_i) - y_i}_{\mathcal{Y}}^2 + \frac{\lambda}{2}\norm{f}^2_{\mathcal{H}_K}
% =\argmin_{f\in\mathcal{H}_K^\lambda} \frac{1}{2N}\displaystyle\sum_{i=1}^N\norm{f(x_i) - y_i}_{\mathcal{Y}}^2 + \frac{\lambda}{2}\norm{f}^2_{\mathcal{H}_K}
% \end{dmath*}
% and $g_*=\argmin_{g\in(Ker W)^\perp} Q_0(g)
% \hiderel{=}\argmin_{g\in\mathcal{H}^\lambda} Q_0(g)$.
% Besides the inclusion $\iota_K:\mathcal{H}_K\hookrightarrow\mathcal{C}(\mathcal{X};\mathcal{Y})$ is compact, thus the closed set $\mathcal{H}_K^{\lambda}=\Set{f\in\mathcal{H}_K|\lambda\norm{f}^2_K\le 2\sigma_y}$ is compact with respect to the compact-open topology. $W$ is a partial isometry with initial space $(\Ker W)^\perp$, the restriction $W|_{(\Ker W)^\perp}:(\Ker W)^\perp\to\mathcal{H}_K$ is an isometry. Thus the set $\mathcal{H}^\lambda=\Set{g\in(\Ker W)^\perp|\lambda\norm{g}^2_{\mathcal{H}}\le 2\sigma_y}=(W|_{(\Ker W)^\perp})^\adjoint \mathcal{H}_K^{\lambda}$ is compact.
% \item It is easy to note that $Q_0$ is continuous and measureable. Besides since $f_*$ is unique and $W|_{(\Ker W)^\perp}$ is an isometry, $Q_0$ has unique minimizer $g_*$ such that $W|_{(\Ker W)^\perp}g_*=f_*$.
% \item Suppose that $\norm{A(\omega)}_{\mathcal{Y},\mathcal{Y}}$ is bounded for all $\omega\in\dual{\mathcal{X}}$ as a consequence $\sup_{g\in\mathcal{H}^\lambda} \abs{Q_D(g)-Q_0(g)}$ is bounded thus from the strong law of large numbers $Q_D(g)\converges{\asurely}{D\to\infty}Q_0(g)$ uniformly.
% \end{enumerate}
% As a result, the extremum estimator $Q_D$ is consistent
% \begin{dmath}
% \label{eq:consitency1}
% \tilde{g}_* \converges{\proba}{D\to\infty} g_*
% \end{dmath}
% Let $\theta_{g,*}=\vect_{j=1}^D\tilde{g}_*(\omega_j)$. since $\tilde{g}_*$ minimizes $Q_D$, $\theta_{g,*}$ minimizes \cref{eq:argmin_applied}. Besides from \cref{pr:phitilde_phi_rel} we have
% \begin{dmath}
% \tilde{W}\theta_{g,*} \converges{\asurely}{D\to\infty} W\tilde{g}_*.
% \end{dmath}
% Finally \cref{eq:consitency1} shows that $W\tilde{g}_*\converges{\proba}{D\to\infty}Wg_*=f_*$, hence
% \begin{dmath*}
% \tildePhi{\omega}\theta_{g,*}\hiderel{=}\tilde{W}\theta_{g,*}\converges{\proba}{D\to\infty} f_*\hiderel{\in}\mathcal{H}_K.
% \end{dmath*}
% with $\theta_{g,*}=\theta_*$ defined in \cref{eq:argmin_RKHS}.
% \end{proof}

\clearpage

%----------------------------------------------------------------------------------------
% \section{Consistency and generalization bounds}
% \label{sec:consistency and generalization bounds}
% \subsection{Operator-valued kernels}
% In this section are interested in finding a minimizer $f_*:\mathcal{X}\to\mathcal{Y}$, where $\mathcal{X}$ is a Polish space and $\mathcal{Y}$ a separable Hilbert space such that for all $x_i$ in $\mathcal{X}$ and all $y_i$ in $\mathcal{Y}$,
% \begin{dmath}
% f_* = \argmin_{f\in\mathcal{H}_K} \sum_{i=1}^NL_{y_i}(f(x_i)) \hiderel{=} \argmin_{f\in\mathcal{H}_K} \mathcal{R}_{emp}(f, L),
% \label{eq:pbOVK}
% \end{dmath}
% where $\mathcal{H}_K$ is a \acs{vv-RKHS} and $\forall y_i \in \mathcal{Y}$, $L_{y_i}$ a $L$-Lipschitz cost function. How does a function trained as in \cref{eq:pbOVK} generalizes on a test set?
% \begin{proposition} Suppose that $f\in\mathcal{H}_K$ a \acs{vv-RKHS} where $\sup_{x\in\mathcal{X}} \Tr[K(x,x)] < T$ and $\norm{f}_{\mathcal{H}_K}<B$. Moreover let $L:\mathcal{Y}\times \mathcal{Y}\to[0, C]$ be a $L$-Lipschitz cost function. Then if we are given $N$ training points, we have with at least probability $1-2\delta$ that
% \begin{dmath}
% \mathcal{R}_{true}(f, L) \le \mathcal{R}_{emp}(f, L)  + 2\sqrt{\frac{2}{N}}\left( LBT^{1/2} + \frac{3C}{4}\sqrt{\ln(1/\delta)}\right).
% \end{dmath}
% \label{pr:ovk_gen}
% \end{proposition}
% \begin{proof}
% This proof is due to Mauer \mpar{See \url{https://arxiv.org/pdf/1605.00251.pdf}}. We do not claim any originality for this proof. First let us introduce the notion of Rade\-macher complexity of a class of function $F$.
% \begin{definition}
% Let $\mathcal{X}$ be any set. Let $\epsilon_1$, $\hdots$, $\epsilon_N$ be $N$ independent Rade\-macher random variables, identically uniformly distributed on $\{-1;1\}$. For any class of functions $F:~\mathcal{X}\to\mathbb{R}$, then for all $x_1, \hdots x_N\in \mathcal{X}$ the quantity
% \begin{dmath}
% \mathcal{R}_N(F) \colonequals \expectation\left[ \sup_{f\in F} \sum_{i=1}^N \epsilon_i f(x_i) \right]
% \end{dmath}
% is called Rademacher complexity of the class $F$.
% \end{definition}
% In generalization bounds the Rademacher complexity of a class of function often involves a composition between a target function to be learn and a cost function, part of the risk we want to minimize. The idea is to bound the Rademacher complexity with a term that does not depends on the cost function, but only on the target function. Such a bound has be recently proposed by Mauer:
% \begin{proposition}
% \label{pr:radswap}
% Let $\mathcal{X}$ be any set, $x_1, \hdots, x_N$ in $\mathcal{X}$, let $F$ be a class of function $f:~\mathcal{X}\to\mathcal{Y}$ and $h_i:~\mathcal{Y}\to\mathbb{R}$ be a $L$-Lipschitz function, where $\mathcal{Y}$ is a second countable Hilbert space endowed with euclidean inner product. Then
% \begin{dmath}
% \expectation \sup_{f\in F} \sum_{i=1}^N \epsilon_i h_i(x_i) \le \sqrt{2}L\sup_{f\in F}\sum_{i=1, k}^{i=N} \epsilon_{ik}f_k(x_i),
% \end{dmath}
% where $\epsilon_{ik}$ is a doubly indexed independent Rademacher sequence and $f_k(x_i)$ is the $k$-th component of $f(x_i)$
% \end{proposition}
% From now on we consider functions $f\in\mathcal{H}_K$ a vv-RKHS. Then there exists an induced feature-map $\Phi:~\mathcal{X}\to \mathcal{L}(\mathcal{Y}, \mathcal{H})$ such that for all $y, y'\in\mathcal{Y}$ the kernel is given by
% \begin{dmath}
% \inner{y, K(x,z)y'} = \inner{\Phi_xy, \Phi_zy'}.
% \end{dmath}
% We say that the feature space $\mathcal{H}$ is embed into the RKHS $\mathcal{H}_K$ by mean of the \emph{feature operator} $(W\theta)(x):=(\Phi_x^\adjoint \theta)$. Indeed $W$ defines a partial isometry between $\mathcal{H}$ and $\mathcal{H}_K$. Remember that $\mathcal{Y}$ a is supposed to be a separable Hilbert space so it has a countable basis and let the class of $\mathcal{Y}$-valued functions $F$ be
% \begin{dmath}
% F=\Set{ f | f: x \mapsto \Phi_x^\adjoint  \theta, \enskip \norm{\theta} < B }\hiderel{\subset} \mathcal{H}_K.
% \end{dmath}
% Then from \cref{pr:radswap} and if $K$ is trace class, we have
% \begin{dmath}
% \label{eq:ker_bound}
% \expectation \sup_{\norm{\theta}<B} \sum_{i=1}^N \epsilon_i L_{y_i}(\Phi_{x_i}^\adjoint  \theta) \le \sqrt{2}L\expectation \sup_{\norm{\theta}<B} \sum_{i=1,k}^{i=N}\epsilon_{ik}\inner{\Phi_{x_i} \theta, e_k}
% = \sqrt{2}L\expectation \sup_{\norm{\theta}<B} \inner*{ \theta, \sum_{i,k}^{i=N}\epsilon_{ik}\Phi_{x_i}e_k}
% \le \sqrt{2}LB\expectation \norm{\sum_{i=1,k}^{i=N}\epsilon_{ik}\Phi_{x_i}e_k}
% \le \sqrt{2}LB\sqrt{\sum_{i=1,k}^{i=N} \norm{\Phi_{x_i}e_k}^2 }
% \le \sqrt{2}LB\sqrt{\sum_{i=1}^{i=N} \Tr\left[ K(x_i, x_i) \right] }.
% \end{dmath}
% From \url{http://www.jmlr.org/papers/volume3/bartlett02a/bartlett02a.pdf}:
% \begin{theorem}
% \label{th:gen_rad_bound}
% Let $\mathcal{X}$ be any set, $F$ a class of functions $f:\mathcal{X}\to[0, C]$ and let $X_1, \hdots, X_N$ be a sequence of iid random variable with value in $\mathcal{X}$. Then for $\delta > 0$, with probability at least $1-\delta$, we have for all $f\in F$ that
% \begin{dmath*}
% \expectation f(X) \le \frac{1}{N} \sum_{i=1}^Nf(X_i) + \frac{2}{N}\mathcal{R}_N(F) + C\sqrt{\frac{9\ln(2/\delta)}{2N}}
% \end{dmath*}
% \end{theorem}
% Conclude by plugin \cref{eq:ker_bound} in \cref{th:gen_rad_bound}.
% \end{proof}

% \subsection{Operator-valued Random Fourier Features}
% Suppose we want to learn an approximate function such that for all $x\in\mathcal{X}$, $\tilde{f}_*(x) \approx f_*(x)$, where $f_*$ is defined as in \cref{eq:pbOVK}. In this section we suppose that $K$ is a shift invariant $\mathcal{Y}$-Mercer kernel on the \acs{LCA} group $(\mathcal{X},\star)$. Brault et al. showed that such a Kernel can be written for all $y$, $y'$ in $\mathcal{Y}$
% \begin{dmath*}
% \inner{\tilde{\Phi}(x)y, \tilde{\Phi}(z)y'}\approx \inner{y, K(x,z)y'}
% \end{dmath*}
% where
% \begin{dmath*}
% \tilde{\Phi}(x)=\Vect_{j=1}^D \exp\left( -i\inner{x, \omega_j}B(\omega_j) \right), \enskip \omega_j \hiderel{\sim} \mu
% \label{eq:ORFF}
% \end{dmath*}
% such that $\inner{y, B(\omega)B(\omega)^\adjoint y'}\frac{d\mu}{d\omega}=\FT{\inner{y, K(\cdot y')})}(\omega)$. We are interested in solving \cref{eq:pbOVK} with the feature map defined from the kernel approximation defined in \cref{eq:ORFF}. Namely:
% \begin{dmath*}
% f_* = \argmin_{f\in\tilde{F}} \sum_{i=1}^NL_{y_i}(\tilde{f}(x_i))
% \hiderel{=} \argmin_{f\in\tilde{F}} \mathcal{R}_{emp}(\tilde{f}, L)
% = \argmin_{\theta\in\mathcal{Y}^D} \sum_{i=1}^NL_{y_i}(\tilde{\Phi}(x_i)^\adjoint \theta).
% \end{dmath*}
% \begin{proposition} Suppose that $\Tr[\tilde{K}_e(0)] \le T$ and $\norm{\theta}_2<\frac{B}{D}$. Let $L:\mathcal{Y}\times \mathcal{Y}\to[0, C]$ be a $L$-Lipschitz cost function. Then if we are given $N$ training points, we have with at least probability $1-2\delta$ that
% \begin{dmath*}
% \mathcal{R}_{true}(\tilde{f}, L) - \min_{f\in F}\mathcal{R}_{true}(f, L) \le O\left(\underbrace{\frac{1}{\sqrt{N}}\left(LBT^{1/2}+\frac{3C}{4}\sqrt{\ln\left(\frac{1}{\delta}\right)}\right)}_{\text{estimation error}}+\underbrace{\frac{LB}{\sqrt{D}}\left(1+\sqrt{\ln\left(\frac{1}{\delta}\right)}\right)}_{\text{approximation error}}\right)
% \end{dmath*}
% \label{pr:ovk_gen}
% \end{proposition}



% \begin{proof}
% We follow the proof of \url{https://people.eecs.berkeley.edu/~brecht/papers/08.rah.rec.nips.pdf}. It is an adaptation of the original proof in the light of the recent results of Mauer. We first define the two following sets:
% \begin{dmath}
% F=\Set{ f | f:\enskip x \mapsto \Phi_x^\adjoint  \theta, \enskip \norm{\theta(\omega)} < B\mu(\omega) }\hiderel{\subset} \mathcal{H}_K.
% \end{dmath}
% and
% \begin{dmath}
% \tilde{F}=\Set{ f | f: x \mapsto \tilde{\Phi}_x^\adjoint  \theta, \enskip \norm{\theta_j} < \frac{B}{D}}\hiderel{\subset} \mathcal{H}.
% \end{dmath}
% \begin{proposition}[Existence of approximation function] Let $\nu$ be a measure on $\mathcal{X}$, and $f_*$ a function in $F$. If $\omega_1, \hdots, \omega_D$ are drawn i.i.d. from $\mu$ then with probability at least $1 - \delta$, there exists a function $\tilde{f}\in \tilde{F}$ such that
% \begin{dmath}
% \sqrt{\int_{\mathcal{X}}\norm{f(x)-\tilde{f}(x)}^2_{\mathcal{Y}}d\mu(x)}\le \frac{B}{\sqrt{D}}\left(1 + \sqrt{2\log(1/\delta)} \right)
% \end{dmath}
% \label{pr:existence_app}
% \end{proposition}

% We use the following lemma of Rahimi and Recht:
% \begin{lemma}
% \label{lm:concentration_hilbert}
% Let $X_1, \hdots X_D$ be random variables with value in a ball $\mathcal{H}$ with radius $M$ centered around the origin in a Hilbert space. Denote the sample average $\bar{X}=\frac{1}{D}\sum_{j=1}^DX_j$. Then with probability $1-\delta$,
% \begin{dmath}
% \norm{\bar{X}-\expectation\bar{X}}\le \frac{M}{\sqrt{D}}\left(1 + \sqrt{2\log(1/\delta)} \right).
% \end{dmath}
% \end{lemma}
% \begin{proof}
% See \url{https://people.eecs.berkeley.edu/~brecht/papers/08.rah.rec.nips.pdf}, lemma 4 of the appendix.
% \end{proof}
% Since $f_*\in F$, we can write $f_*(x)=\int_{\omega\in\dual{\mathcal{X}}}\pairing{x,\omega}A(\omega)\theta(\omega)d\omega$. Construct the functions $f_j=\pairing{\cdot, \omega_j}A(\omega_j)\theta_j$, $j=1,\dots, D$ with $\theta_j:=\theta(\omega_j)/\mu(\omega_j)$. Let $\tilde{f}=\frac{1}{D}\sum_{j=1}^Df_j$ be the sample average of these functions. Then $\tilde{f}\in \tilde{F}$ because $\norm{\theta_j}/D\le B/D$. Also under the inner product $\inner{f,g}=\int_{\mathcal{X}} f(x)g(x)d\mu(x)$ we have that $\norm{\pairing{\cdot, \omega_j}A(\omega_j)\theta_j}\le B$. The claim follow by applying \cref{lm:concentration_hilbert} to $f_1, \hdots f_D$ under this inner product.
% \end{proof}

% \begin{proposition}[Bound on the approximation error] Suppose that $f\in F$ and $L_{y_i}$ is a $L$-Lipschitz cost function. With probability $1-\delta$ there exist a function $\tilde{f}\in \tilde{F}$ such that
% \begin{dmath}
% \mathcal{R}(\tilde{f}, L_{y_i})\le\mathcal{R}(f, L_{y_i}) + \frac{LB}{\sqrt{D}}\left(1 + \sqrt{2\log(1/\delta)} \right)
% \end{dmath}
% \label{pr:bound_app_error}
% \end{proposition}
% \begin{proof} For any two functions $f$ and $g$, the Lipschitz condition on $L_{y_i}$ followed by the concavity of square root gives
% \begin{dmath}
% \mathcal{R}(f, L) - \mathcal{R}(f, L) = \expectation \left[L(f(x)) - L(g(x))\right]
% \le \expectation \norm{L(f(x)) - L(g(x))}
% \le L \expectation \norm{f(x) - g(x)}
% \le L \sqrt{\expectation \norm{f(x) - g(x)}^2}
% \end{dmath}
% Eventually apply \cref{pr:existence_app} to conclude.
% \end{proof}
% \begin{proposition}[Bound on the estimation error]
% \label{pr:bound_est_error}
% Suppose $L: \mathcal{Y}\times\mathcal{Y}\to [0; C]$ is $L$-Lipschitz. Let $\omega_1, \hdots, \omega_D$ be fixed. If $\{(x_i, y_i)\}\subset(\mathcal{X}\times \mathcal{Y})^N$ are drawn i.i.d. from a fixed distribution, then for $\delta>0$ with probability $1-\delta$ over the dataset we have
% \begin{dmath}
% \forall\tilde{f}\in\mathcal{F}, \enskip \mathcal{R}_{true}(\tilde{f}, L) \le \mathcal{R}_{emp}(\tilde{f}, L) + \frac{1}{\sqrt{N}}\left( BL\sqrt{2\Tr\left[ \tilde{K}_e(0) \right]}+C\sqrt{\frac{9\ln(2/\delta)}{2}}\right)
% \end{dmath}
% \end{proposition}
% \begin{proof}
% Apply \cref{th:gen_rad_bound}. We get
% \begin{equation}
% \mathcal{R}_{true}(f, L) \le \mathcal{R}_{emp}(f, L) + \frac{2}{N}\mathcal{R}_N(F) + C\sqrt{\frac{9\ln(2/\delta)}{2N}}.
% \label{eq:est_bound}
% \end{dmath}
% Since $f\in F$, $\forall j\inrange{1}{D},\enskip\norm{\theta_j}_\mathcal{Y}<\frac{B}{D}$; and recall that $\tilde{\Phi}(x)=\vect_{j=1}^D\Phi_x(\omega_j)$. Now with similar arguments as in \cref{eq:ker_bound} we have
% \begin{dmath}
% \mathcal{R}_N(F) \le \sqrt{2}BL\sqrt{N\Tr\left[K(0)\right]}
% \end{dmath}
% Plugin back in \cref{eq:est_bound} yields the desired result.
% \end{proof}

\clearpage
%----------------------------------------------------------------------------------------
\section{Conclusions}
\label{sec:conclusions}

\chapterend
