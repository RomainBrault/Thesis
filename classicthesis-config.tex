%!TEX root = ./ThesisRomainbrault.tex
%!TeX program = xelatex

%%%%%%%%%%%%%%%%%%%%%%%%%%%%%%%%%%%%%%%%%
% Classicthesis Typographic Thesis
% Configuration File
%
% This file has been downloaded from:
% http://www.LaTeXTemplates.com
%
% Original author:
% André Miede (http://www.miede.de) with extensive commenting changes by:
% Vel (vel@LaTeXTemplates.com)
%
% License:
% GNU General Public License (v2)
%
% Important note:
% The main lines to change in this file are in the thref VARIABLES
% section, the rest of the file is for advanced configuration.
%
%%%%%%%%%%%%%%%%%%%%%%%%%%%%%%%%%%%%%%%%%

\listfiles

\usepackage{silence}

\usepackage[utf8]{inputenc}
\usepackage{varioref}
\if@PDFX@
\usepackage[a-1b]{pdfx}
\else
\usepackage{hyperref}
\fi



%------------------------------------------------------------------------------

\usepackage{adforn}
\usepackage{pythontex}
\usepackage{multicol}
\usepackage{multirow}
\usepackage{cmap} % copy-paste pdf
\usepackage{ragged2e}
\usepackage{minitoc}

%------------------------------------------------------------------------------
%  USEFUL COMMANDS
%------------------------------------------------------------------------------
\newcounter{dummy} % Necessary for correct hyperlinks (to index, bib, etc.)
\providecommand{\mLyX}{L\kern-.1667em\lower.25em\hbox{Y}\kern-.125emX\@}
\newlength{\abcd} % for ab..z string length calculation


%------------------------------------------------------------------------------
%  PACKAGES
%------------------------------------------------------------------------------

% Used for inserting dummy 'Lorem ipsum' text into the template
% \usepackage{lipsum}

%------------------------------------------------------------------------------

% Spanish languages need extra options in order to work with this template
%\PassOptionsToPackage{spanish,es-lcroman}{babel}
\usepackage[american]{babel}

%------------------------------------------------------------------------------

\usepackage[absolute]{textpos}
\usepackage{afterpage}
\usepackage{pdflscape}
\usepackage{rotating}
\usepackage{placeins}

%------------------------------------------------------------------------------

\usepackage{csquotes}
\PassOptionsToPackage{%
backend=biber, % Instead of bibtex
% backend=bibtex8, %
bibencoding=ascii,%
language=auto,%
style=numeric-comp,%
firstinits=true, %
% style=authoryear-comp, % Author 1999, 2010
% bibstyle=authoryear,dashed=false, % dashed: substitute rep. author with ---
sorting=nyt, % name, year, title
maxbibnames=10, % default: 3, et al.
backref=true, %
block=space, %
natbib=true % natbib compatibility mode (\citep and \citet still work)
}{biblatex}
\usepackage{biblatex}

%------------------------------------------------------------------------------

\PassOptionsToPackage{fleqn}{amsmath} % Math environments and more by the AMS
\usepackage{amsmath}
\usepackage{amssymb}
\usepackage{dsfont}
%------------------------------------------------------------------------------

\PassOptionsToPackage{T1}{fontenc} % T2A for cyrillics
\usepackage{fontenc}

%------------------------------------------------------------------------------

\usepackage{textcomp} % Fix warning with missing font shapes

%------------------------------------------------------------------------------

\usepackage{scrhack} % Fix warnings when using KOMA with listings package

%------------------------------------------------------------------------------

\usepackage{xspace} % To get the spacing after macros right

%------------------------------------------------------------------------------

\usepackage{mparhack} % To get marginpar right

%------------------------------------------------------------------------------

% Not necessary for overleaf's LaTeX version.
% \usepackage{fixltx2e} % Fixes some LaTeX stuff

%------------------------------------------------------------------------------

\usepackage{graphicx}
\usepackage[export]{adjustbox}
\usepackage{tikz}
\usetikzlibrary{cd}
\usetikzlibrary{matrix}

\makeatletter
\tikzset{
  column sep/.code=\def\pgfmatrixcolumnsep{\pgf@matrix@xscale*(#1)},
  row sep/.code   =\def\pgfmatrixrowsep{\pgf@matrix@yscale*(#1)},
  matrix xscale/.code=%
    \pgfmathsetmacro\pgf@matrix@xscale{\pgf@matrix@xscale*(#1)},
  matrix yscale/.code=%
    \pgfmathsetmacro\pgf@matrix@yscale{\pgf@matrix@yscale*(#1)},
  matrix scale/.style={/tikz/matrix xscale={#1},/tikz/matrix yscale={#1}}}
\def\pgf@matrix@xscale{1}
\def\pgf@matrix@yscale{1}
\makeatother

%------------------------------------------------------------------------------
%	FLOATS: TABLES, FIGURES AND CAPTIONS SETUP
%------------------------------------------------------------------------------
\usepackage{ltablex}
\usepackage{threeparttablex}
\setlength{\extrarowheight}{3pt} % Increase table row height
\newcommand{\tableheadline}[1]{\multicolumn{1}{c}{\spacedlowsmallcaps{#1}}}
\newcommand{\myfloatalign}{\centering} % To be used with each float for alignment
\usepackage{caption}
\captionsetup{font=small}
\usepackage{subfig}

%------------------------------------------------------------------------------
%	CODE LISTINGS SETUP
%------------------------------------------------------------------------------

\usepackage{listings}
%\lstset{emph={trueIndex,root},emphstyle=\color{BlueViolet}}%\underbar} % For special keywords
\lstset{language=[LaTeX]Tex,%C++ % Specify the language(s) for listings here
morekeywords={PassOptionsToPackage,selectlanguage},
keywordstyle=\color{RoyalBlue}, % Add \bfseries for bold
basicstyle=\small\ttfamily, % Makes listings a smaller font size and a different font
identifierstyle=\color{NavyBlue}, % Color of text inside brackets
commentstyle=\color{Green}\ttfamily, % Color of comments
stringstyle=\rmfamily, % Font type to use for strings
numbers=left, % Change left to none to remove line numbers
numberstyle=\scriptsize, % Font size of the line numbers
stepnumber=5, % Increment of line numbers
numbersep=8pt, % Distance of line numbers from code listing
showstringspaces=false, % Sets whether spaces in strings should appear underlined
breaklines=true, % Force the code to stay in the confines of the listing box
%frameround=ftff, % Uncomment for rounded frame
frame=single, % Frame border - none/leftline/topline/bottomline/lines/single/shadowbox/L
belowcaptionskip=.75\baselineskip % Space after the "Listing #: Desciption" text and the listing box
}

%------------------------------------------------------------------------------
%	HYPERREFERENCES
%------------------------------------------------------------------------------

% \usepackage{varioref}

\pdfminorversion=9
\pdfcompresslevel=0
% \pdfadjustspacing=1
\hypersetup{%
% Uncomment the line below to remove all links (to references, figures, tables,
% etc), useful for b/w printouts draft,
pdflang={en-US},
pdfauthor={Romain Raymond Brault},
xetex,
hyperfootnotes=true,
colorlinks=true, linktocpage=true, pdfstartpage=3, pdfstartview=FitV,
% Uncomment the line below if you want to have black links (e.g. for printing
% black and white)
%colorlinks=false, linktocpage=false, pdfborder={0 0 0}, pdfstartpage=3,
pdfstartview=FitV,
pdfpagelayout=TwoColumnRight,
breaklinks=true, pdfpagemode=UseNone, pageanchor=true, pdfpagemode=UseOutlines,
plainpages=false, bookmarksnumbered, bookmarksopen=true, bookmarksopenlevel=1,
hypertexnames=true, pdfhighlight=/O, nesting=true, %frenchlinks,%
urlcolor=PSaclay, linkcolor=RoyalBlue, citecolor=webgreen, 
%pagecolor=Black,%
}

\VerbatimFootnotes




%------------------------------------------------------------------------------
\usepackage[shortlabels]{enumitem}
\usepackage{braket}
\usepackage{dirtytalk}
\usepackage{numprint}
% Remove drafting to get rid of the '[ Date - classicthesis version 4.0 ]' text
% at the bottom of every page
\PassOptionsToPackage{eulerchapternumbers, listings, subfig, parts,
floatperchapter, linedheaders}{classicthesis}
% Available options: drafting parts nochapters linedheaders eulerchapternumbers
% beramono eulermath pdfspacing minionprospacing tocaligned dottedtoc
% manychapters listings floatperchapter subfig
\usepackage{classicthesis}
\usepackage{breqn}

%------------------------------------------------------------------------------

\usepackage{colonequals}
\usepackage{amssymb,amsmath}
\usepackage{nameref}
\usepackage[amsmath,thmmarks,hyperref]{ntheorem}
\usepackage[nameinlink,noabbrev]{cleveref}
\usepackage{mathtools}
\usepackage[algo2e,ruled,linesnumbered]{algorithm2e}
\newcommand{\theHalgorithm}{\arabic{algorithm}}
\usepackage{commands}

%\crefname{algocfline}{algorithm}{algorithms}
%\Crefname{algocfline}{Algorithm}{Algorithms}
%\crefname{algocf}{algorithm}{algorithm}
%\Crefname{algocf}{Algorithm}{Algorithms}
%\crefname{table}{table}{tables}
%\Crefname{table}{Table}{Tables}
%\crefname{figure}{figure}{figures}
%\Crefname{figure}{Figure}{Figures}
%\crefname{chapter}{chapter}{chapters}
%\Crefname{chapter}{Chapter}{Chapters}
%\crefname{section}{section}{sections}
%\Crefname{section}{Section}{Sections}
%\crefname{subsection}{subsection}{subsections}
%\Crefname{subsection}{Subsection}{Subsections}
%\crefname{theorem}{theorem}{theorems}
%\Crefname{theorem}{Theorem}{Theorems}

\crefname{algocfline}{Algorithm}{Algorithms}
\Crefname{algocfline}{Algorithm}{Algorithms}
\crefname{algocf}{Algorithm}{Algorithm}
\Crefname{algocf}{Algorithm}{Algorithms}
\crefname{table}{Table}{Tables}
\Crefname{table}{Table}{Tables}
\crefname{figure}{Figure}{Figures}
\Crefname{figure}{Figure}{Figures}
\crefname{chapter}{Chapter}{Chapters}
\Crefname{chapter}{Chapter}{Chapters}
\crefname{section}{Section}{Sections}
\Crefname{section}{Section}{Sections}
\crefname{subsection}{Subsection}{Subsections}
\Crefname{subsection}{Subsection}{Subsections}
\crefname{theorem}{Theorem}{Theorems}
\Crefname{theorem}{Theorem}{Theorems}
\crefname{proposition}{Proposition}{Propositions}
\Crefname{proposition}{Proposition}{Propositions}
\crefname{lemma}{Lemma}{Lemmas}
\Crefname{lemma}{Lemma}{Lemmas}
\crefname{corollary}{Corollary}{Corollaries}
\Crefname{corollary}{corollary}{Corollaries}
\crefname{equation}{Equation}{Equations}
\Crefname{equation}{Equation}{Equations}
\crefname{remark}{Remark}{Remarks}
\Crefname{remark}{Remark}{Remarks}
\crefname{example}{Example}{Examples}
\Crefname{example}{Example}{Examples}
\creflabelformat{equation}{#2\textup{#1}#3}

\newlist{propenum}{enumerate}{1} % also creates a counter called 'propenumi'
\setlist[propenum]{label=\arabic*., ref=\theproposition~item~\arabic*}
\crefalias{propenumi}{proposition}


\newlist{lemmaenum}{enumerate}{1} % also creates a counter called 'propenumi'
\setlist[lemmaenum]{label=\arabic*., ref=\thelemma~item~\arabic*}
\crefalias{lemmai}{lemma}

\newcommand{\chapterendsymbol}{%
    \par
    \begin{center}\adforn{60}\end{center}
    }
\newcommand{\chapterend}{\chapterendsymbol}

\DeclareFlexSymbol{\Gamma}  {Var}{latin}{00}
\DeclareFlexSymbol{\Delta}  {Var}{latin}{01}
\DeclareFlexSymbol{\Theta}  {Var}{latin}{02}
\DeclareFlexSymbol{\Lambda} {Var}{latin}{03}
\DeclareFlexSymbol{\Xi}     {Var}{latin}{04}
\DeclareFlexSymbol{\Pi}     {Var}{latin}{05}
\DeclareFlexSymbol{\Sigma}  {Var}{latin}{06}
\DeclareFlexSymbol{\Upsilon}{Var}{latin}{07}
\DeclareFlexSymbol{\Phi}    {Var}{latin}{08}
\DeclareFlexSymbol{\Psi}    {Var}{latin}{09}
\DeclareFlexSymbol{\Omega}  {Var}{latin}{0A}
\DeclareFlexSymbol{0}{Var}{digit}{30}
\DeclareFlexSymbol{1}{Var}{digit}{31}
\DeclareFlexSymbol{2}{Var}{digit}{32}
\DeclareFlexSymbol{3}{Var}{digit}{33}
\DeclareFlexSymbol{4}{Var}{digit}{34}
\DeclareFlexSymbol{5}{Var}{digit}{35}
\DeclareFlexSymbol{6}{Var}{digit}{36}
\DeclareFlexSymbol{7}{Var}{digit}{37}
\DeclareFlexSymbol{8}{Var}{digit}{38}
\DeclareFlexSymbol{9}{Var}{digit}{39}


%------------------------------------------------------------------------------
%	CHANGING TEXT AREA
%------------------------------------------------------------------------------

%\linespread{1.0} % a bit more for Palatino
% 686 (factor 2.2) + 33 head + 42 head \the\footskip
%\areaset[current]{312pt}{761pt} 
%\setlength{\marginparwidth}{7em}%
%\setlength{\marginparsep}{2em}%

%------------------------------------------------------------------------------
% Include printonlyused in the first bracket to only show acronyms used in the
% text
\usepackage{acro} 
% Nice macros for handling all acronyms in the thesis
%\renewcommand*{\acsfont}[1]{\textssc{#1}} % For MinionPro
% \newcommand*{\aclabelfont}[1]{\acsfont{#1}}

%------------------------------------------------------------------------------
%	USING DIFFERENT FONTS
%------------------------------------------------------------------------------
\usepackage[quiet]{mathspec}
\usepackage{xltxtra} %also loads xunicode
\setmathfont(Digits,Latin)[
    Numbers={Lining,Proportional},
    Mapping=tex-text,
    Ligatures={Common, Rare, Historic},
]{Hoefler Text}
\setmathfont(Greek)[
    Mapping=tex-text,
    Lowercase=Regular,
    Uppercase=Regular
]{Hoefler Text}
\setmainfont[
    Mapping=tex-text,
    Ligatures={TeX, Common, Rare, Historic}
]{Hoefler Text}
