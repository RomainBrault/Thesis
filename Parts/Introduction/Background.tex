%----------------------------------------------------------------------------------------
\section{Notations}
\label{sec:notations}
The euclidean inner product in $\mathbb{R}^d$ is denoted $\inner{\cdot, \cdot}$ and the euclidean norm is denoted $\norm{\cdot}$. The unit pure imaginary number $\sqrt{-1}$ is denoted $\iu$.
$\mathcal{B}(\mathbb{R}^d)$ is the Borel $\sigma$-algebra on $\mathbb{R}^d$.
If $\mathcal{X}$ and $\mathcal{Y}$ are two vector spaces, we denote by $\mathcal{F}(\mathcal{X};\mathcal{Y})$ the vector space of functions $f:\mathcal{X}\to\mathcal{Y}$ and $\mathcal{C}(\mathcal{X};\mathcal{Y})\subset\mathcal{F}(\mathcal{X};\mathcal{Y})$ the subspace of continuous functions.
If $\mathcal{H}$ is an Hilbert space we denote its scalar product by $\inner{.,.}_\mathcal{H}$ and its norm by $\norm{.}_\mathcal{H}$.
We set $\mathcal{L}(\mathcal{H})=\mathcal{L}(\mathcal{H};\mathcal{H})$ to be the space of linear operators from $\mathcal{H}$ to itself. If $W\in\mathcal{L}(\mathcal{H})$, $\Ker W$ denotes the nullspace, $\Ima W$ the image and $W^\adjoint \in \mathcal{L}(\mathcal{H})$ the adjoint operator (transpose when $W$ is a real matrix). All these notations are summarized in \cref{table:notations}.

\begin{table}[!ht]
\centering
\caption{Mathematical symbols used throughout the parper and their signification.}
\begin{tabularx}{\textwidth}{cX}
\toprule
Symbol & \multicolumn{1}{c}{Meaning} \\
\cmidrule{1-2}
$\iu$ & Unit pure imaginary number $\sqrt{-1}$. \\
$\ec$ & Euler constant. \\
$\inner{\cdot,\cdot}$ & Euclidean inner product. \\
$\norm{\cdot}$ & Euclidean norm. \\
$\mathcal{X}$ & Input space (LCA). \\
$\dual{\mathcal{X}}$ & The Pontryagin dual of $\mathcal{X}$. \\
$\mathcal{Y}$ & Output space (Hilbert space). \\
$\mathcal{H}$ & Feature space (Hilbert space). \\
$\inner{\cdot,\cdot}_{\mathcal{Y}}$ & The canonical inner product of the Hilbert space $\mathcal{Y}$. \\
$\norm{\cdot}_{\mathcal{Y}}$ & The canonical norm induced by the inner product of the Hilbert space $\mathcal{Y}$. \\
$\mathcal{F}(\mathcal{X};\mathcal{Y})$ & Vector space of function from $\mathcal{X}$ to $\mathcal{Y}$. \\
$\mathcal{C}(\mathcal{X};\mathcal{Y})$ & The vector subspace of $\mathcal{F}$ of continuous function from $\mathcal{X}$ to $\mathcal{Y}$. \\
$\mathcal{L}(\mathcal{H};\mathcal{Y})$ & The set of bounded linear operator from a Hilbert space $\mathcal{H}$ to a Hilbert space $\mathcal{Y}$. \\
$\mathcal{L}(\mathcal{Y})$ & The set of bounded linear operator from a Hilbert space $\mathcal{H}$ to itself. \\
$\mathcal{B}(\mathcal{X})$ & Borel $\sigma$-algebra on $\mathcal{X}$. \\
$\mu(\mathcal{X})$ & A scalar positive measure of $\mathcal{X}$. \\
$p_\mu(x)$ & The Radon-Nikodym derivative of $\mu$ w.r.t the Lebesgue measure. \\
$dx$, $d\omega$ & The canonical Haar measure of the LCA $(\mathcal{X},\mathcal{B}(\mathcal{X}))$. (resp. $(\dual{\mathcal{X}}, \mathcal{B}(\dual{\mathcal{X}})$). \\
$L^p(\mathcal{X},dx)$ & The Banach space of $\abs{\cdot}^p$-integrable function from $(\mathcal{X},\mathcal{B}(\mathcal{X},dx))$ to $\mathbb{C}$. \\
$L^p(\mathcal{X},dx;\mathcal{Y})$ & The Banach space of  $\norm{\cdot}_\mathcal{Y}^p$ (Bochner)-integrable function from $(\mathcal{X},\mathcal{B}(\mathcal{X}), dx)$ to $\mathcal{Y}$. \\
\bottomrule
\end{tabularx}
\label{table:notations}
\end{table}

%----------------------------------------------------------------------------------------
\section{About statistical learning}
\label{sec:about_statistical_learning}

%----------------------------------------------------------------------------------------
\section{On large-scale learning}
\label{sec:on_large-scale_learning}

%----------------------------------------------------------------------------------------
\section{On operator-valued kernels}
\label{sec:background_on_operator-valued_kernels}

%----------------------------------------------------------------------------------------
\section{Elements of abstract harmonic analysis}
\label{sec:abstract_harmonic}

\subsection{Locally compact Abelian groups}
\begin{definition}{\acl{LCA}.}
A group $(\mathcal{X}, \groupop)$ is said to be Locally Compact Abelian if it is a topological \emph{commutative} group $\mathcal{X}$ for which every point has a compact neighborhood and is Hausdorff. 
\end{definition}
\paragraph{}
\acf{LCA} groups are central to the general definition of Fourier Transform which is related to the concept of Pontryagin duality \citep{folland1994course}.
Let $(\mathcal{X}, \groupop)$ be a \acs{LCA} group with $e$ its neutral element and the notation, $\inv{x}$, for the inverse of $x \in \mathcal{X}$. A \emph{character} is a complex continuous homomorphism $\omega:\mathcal{X}\to\mathbb{U}$ from $\mathcal{X}$ to the set of complex numbers of unit module $\mathbb{U}$. The set of all characters of $\mathcal{X}$ forms the Pontryagin \emph{dual  group} $\dual{\mathcal{X}}$. The dual group of an \acs{LCA} group is an \acs{LCA} group and the dual group operation is defined by 
\begin{equation*}
(\omega_1 \groupop \omega_2)(x)=\omega_1(x)\omega_2(x) \in \mathbb{U}.
\end{equation*}
\paragraph{}
The Pontryagin duality theorem states that $\dual{\dual{\mathcal{X}}}\cong \mathcal{X}$. \Ie~there is a canonical isomorphism between any \acs{LCA} group and its double dual. To emphasize this duality the following notation is usually adopted: $\omega(x)=\pairing{x,\omega}=\pairing{\omega,x}$, where $x\in\mathcal{X}$, $\omega\in\dual{\mathcal{X}}$. Another important property involves the complex conjugate of the pairing which is defined as $\conj{\pairing{x,\omega}} = \pairing{\inv{x},\omega}$.
\begin{table}[!ht]\label{table:pairings}
\caption{Classification of Fourier transforms in terms of their domain and transform domain.}
\label{tab:dual_and_pairing}
\centering
\begin{tabularx}{\textwidth}{cccX}
\toprule
\multicolumn{1}{c}{$\mathcal{X}$} & \multicolumn{1}{c}{$\dual{\mathcal{X}}$} & \multicolumn{1}{c}{Operation} & \multicolumn{1}{c}{Pairing} \\
\cmidrule{1-4}
$\mathbb{R}^d$ & $\mathbb{R}^d$ & $+$ & $\pairing{x,\omega} = \exp\left(\iu \inner{x, \omega}\right)$ \\
$\mathbb{R}^d_{*,+}$ & $\mathbb{R}^d$ & $\cdot$ & $\pairing{x,\omega} =\exp\left( \iu \inner{\log(x), \omega} \right)$ \\
$(-c;+\infty)^d$ & $\mathbb{R}^d$ & $\odot$ & $\pairing{x,\omega} =\exp\left( \iu \inner{\log(x+c), \omega} \right)$ \\
\bottomrule
\end{tabularx}
\end{table}
\paragraph{}
We notice that for any pairing depending of $\omega$, there exists a function $h_{\omega}: \mathcal{X} \to \mathbb{R}$ such that: $(x,\omega)= \exp(-i h_{\omega}(x))$ since any pairing maps into $\mathbb{U}$. Moreover, 
\begin{equation*}
\begin{aligned}
(x \groupop \inv{z},\omega) = \omega(x)\omega(\inv{z}) &=\exp(-i h_{\omega}(x))\exp(-i h_{\omega}(\inv{z})) \\
&=\exp(-i h_{\omega}(x))\exp(+i h_{\omega}(z)).
\end{aligned}
\end{equation*}
\Cref{tab:dual_and_pairing} provide an explicit list of pairings for various groups based on $\mathbb{R}^d$ or its subsets. We especially mention the duality pairing associated to the skewed multiplicative LCA group $\mathcal{X}=((-c;+\infty)^d, \odot)$
% where the operation $\odot$ is defined component-wise as follows:
% \begin{equation}
% \forall k\inrange{1}{d}, \enskip (x \odot z)_k := (x_k+c)(z_k+c) - c,
% \end{equation}
% Then the pairing is defined by:
% \begin{equation}\label{eq:pairing-skewed}
% \pairing{x, \omega} = \exp\left(i\sum_{k=1}^d\omega_k \log(x_k+c)\right).
% \end{equation}
Hence $h_\omega(x)=\sum_{k=1}^d\omega_k \log(x_k+c)$. This group together with the operation $\odot$ has  been proposed by \cite{li2010random} to handle histograms features especially useful in image recognition applications.
% % \begin{example}
% % $\dual{\mathbb{R}}\equiv\mathbb{R}$.
% % \begin{proof}
% % If $\omega\in\dual{\mathbb{R}}$ then $\omega(0)=1$ since $\omega$ is an homeomorphism from $\mathbb{R}$ to $\mathbb{U}$. Therefore there exists $a>0$ such that $\int_0^a\omega(t)dt\neq0$. Setting $A\omega=\int_0^a\omega(t)dt$ we have
% % \begin{equation*}
% % (A\omega)(x)=\int_0^a\omega(x+t)dt=\int_x^{a+x}\omega(t)dt.
% % \end{equation*}
% % so $\omega$ is differentiable and
% % \begin{equation*}
% % \omega'(x)=A^{-1}(\omega(a+x)-\omega(x))=c\omega(x) \quad\text{where}\quad c=A^{-1}(\omega(a)-1).
% % \end{equation*}
% % It follow that $\omega(x)=e^{cx}$, and since $\abs{\omega}=1$, one can take $c=i\omega$ for some $\omega\in\mathbb{R}$. Hence we can identify $\omega$ with $\omega$ and $\dual{\mathbb{R}}$ with $\mathbb{R}$ since $\omega$ uniquely determines $\omega$.
% % \end{proof}
% % \end{example}
\subsection{The Fourier transform}
For a function with values in a separable Hilbert space $f\in L^1(\mathcal{X},dx,\mathcal{Y})$, where $dx$ is the Haar measure on $\mathcal{X}$, we denote $\FT{f}$ its Fourier transform which is defined by
\begin{equation*}
    \forall \omega \in \dual{\mathcal{X}},\enskip \FT{f}(\omega)=\int_{\dual{\mathcal{X}}} \conj{\pairing{x,\omega}}f(x)dx.
\end{equation*} 
For a measure defined on $\mathcal{X}$, there exists a unique suitably normalized measure $d\omega$ on $\dual{\mathcal{X}}$ such that $\forall f \in L^1(\mathcal{X}, dx, \mathcal{Y})$ and if $\FT{f} \in L^1(\dual{\mathcal{X}}, d\omega, \mathbb{R})$ we have
 \begin{equation}\label{fourier-l1}
 \forall x \in \mathcal{X},\enskip f(x)=\int_{\dual{\mathcal{X}}}\FT{f}(\omega)(x,\omega) d\omega.
\end{equation}
Moreover if $d\omega$ is normalized, $\mathcal{F}$ extends to a unitary operator from $L^2(\mathcal{X}, dx, \mathcal{Y})$ onto $L^2(\dual{\mathcal{X}}, d\omega, \mathcal{Y})$ Then the inverse Fourier transform of a function $g\in L^1(\dual{\mathcal{X}},d\omega,\mathcal{Y})$ (where $d\omega$ is a Haar measure on $\dual{\mathcal{X}}$ suitably normalize w.r.t. the Haar measure $dx$) is noted $\IFT{g}$ defined by
\begin{equation*}
    \forall x \in \mathcal{X},\enskip \IFT{g}(x) = \int_{\dual{\mathcal{X}}} \pairing{x,\omega}g(\omega) d\omega,
\end{equation*}
% The inverse Fourier transform of $g$, an integrable function on $\dual{X}$ is defined as:
% where the Haar measure $\mu=\dual{\nu}$ is called the {\it dual measure} to $\nu$.
%\Cref{tab:dual_and_pairing} gives some examples of real Abelian groups with their associated dual and pairing. 
%See \cite{folland1994course} for a more detailed construction of LCA, Pontryagin duality and Fourier transforms on LCA.
 For the familiar case of a scalar-valued function $f$ on the LCA $(\mathbb{R}^d, +)$, we have: 
 \begin{equation}\label{fourier-R-plus}
 \forall \omega \in \dual{\mathcal{X}},\enskip \FT{f}(\omega)=\int_{\mathbb{R}^d} e^{-i \inner{\omega,x - z}} dx,
\end{equation}
the Haar measure being here the Lebesgue measure.