%----------------------------------------------------------------------------------------
\section{Motivations}
\label{sec:motivations}
Random Fourier Features have been proved useful to implement efficiently kernel methods in the scalar case, allowing to learn a linear model based on an approximated feature map. In this work, we are interested to construct approximated operator-valued feature maps to learn vector-valued functions. With an explicit (approximated) feature map, one converts the problem of learning a function $f$ in the vector-valued Reproducing Kernel Hilbert Space $\mathcal{H}_K$ into the learning of a linear model $\tilde{f}$ defined by:
 \begin{equation*}
 \tilde{f}(x) = \tildePhi(x)^* \theta,
 \end{equation*}
 where $\Phi: \mathcal{X} \to \mathcal{L}(\mathcal{H},\mathcal{Y})$ and $\theta \in \mathcal{H}$. The methodology we propose works for operator-valued kernels defined on any \acf{LCA} group, noted ($\mathcal{X}, \groupop)$, for some operation noted $\groupop$. This allows us to use the general context of Pontryagin duality for \acl{FT} of functions on \acs{LCA} groups. Building upon a generalization of Bochner's theorem for operator-valued measures, an operator-valued kernel is seen as the \emph{\acl{FT}} of an operator-valued positive measure. From that result, we extend the principle of Random Fourier Feature for scalar-valued kernels and derive a general methodology to build Operator Random Fourier Feature when operator-valued kernels are shift-invariant according to the chosen group operation.

%----------------------------------------------------------------------------------------
\section{Construction}
\label{sec:construction}
We present a construction of \acf{ORFF} such that $f: x\mapsto \tildePhi(x)^*\theta$ is a continuous function that maps an arbitrary \acs{LCA} group $\mathcal{X}$ as input space to an arbitrary output Hilbert space $\mathcal{Y}$. First we define a functional \emph{Fourier feature map}, and then propose a Monte-Carlo sampling from this feature map to construct an approximation of a shift-invariant $\mathcal{Y}$-Mercer kernel.
Then, we prove the convergence of the kernel approximation $\tilde{K}(x,z)=\tildePhi(x)^*\tildePhi(z)$ with high probability on \emph{compact} subsets of the \acs{LCA} $\mathcal{X}$, when $\mathcal{Y}$ is \emph{finite dimensional}. Eventually we conclude with some numerical experiments.
\subsection{Theoretical study}
The following proposition of \citet{Zhang2012,Carmeli2010} extends Bochner's theorem to any shift-invariant $\mathcal{Y}$-Mercer kernel. 
\begin{proposition}[Operator-valued Bochner's theorem \citep{Zhang2012}]
If a continuous function $K$ from $\mathcal{X} \times \mathcal{X}$ to $\mathcal{Y}$ is a shift-invariant $\mathcal{Y}$-Mercer kernel on $\mathcal{X}$, then there exists a unique positive operator-valued measure $M: \mathcal{B}(\mathcal{X}) \to \mathcal{L}_+(\mathcal{Y})$ such that:
\begin{equation}\label{eq:bochner-gen}
\forall x, z \in \mathcal{X}, K(x, z) = \int_{\hat{\mathcal{X}}} \conj{\pairing{x \groupop \inv{z}, \omega}} dM(\omega),
\end{equation}
where $M$ belongs to the set of all the $\mathcal{L}_+(\mathcal{Y})$-valued measures of bounded variation on the $\sigma$-algebra of Borel subsets of $\dual{\mathcal{X}}$. Conversely, from any positive operator-valued measure $M$, a shift-invariant kernel $K$ can be defined by \cref{eq:bochner-gen}. 
\end{proposition}
Although this theorem is central to the spectral decomposition of shift-invariant $\mathcal{Y}$-Mercer \acs{OVK}, the following results proved by \citet{Carmeli2010} provides insights about this decomposition that are more relevant in practise. It first shows how to build shift-invariant $\mathcal{Y}$-Mercer kernel but more importantly, also states that any operator-valued spectral decomposition of such \acs{OVK}s when $\mathcal{Y}$ is finite dimensional or $\mathcal{X}$ is compact can be written using a pair $(A, \mu)$ where $A$ is an operator-valued function on $\dual{\mathcal{X}}$ and $\mu$ is a real-valued positive measure on $\dual{\mathcal{X}}$. Note that obviously such a pair is not unique ad the choice of this paper may have an impact on theoretical properties as well as practical computations.
\begin{proposition}[\citet{Carmeli2010}]\label{pr:mercer_kernel_bochner}
Let $\mu$ be a positive measure on $\mathcal{B}(\mathcal{\dual{\mathcal{X}}})$ and $A: \dual{\mathcal{X}}\to \mathcal{L}(\mathcal{Y})$ such that $\inner{A(.)y,y'}\in L^1(\mathcal{X},d\mu)$ for all $y,y'\in\mathcal{Y}$ and $A(\omega)\succcurlyeq 0$ for $\mu$-almost all $\omega$. Then, for all $\delta \in \mathcal{X}$ and for all $y, y' \in \mathcal{Y}$,
\begin{equation}
\label{eq:AK0}
\inner{y,K_e(\delta)y}_{\mathcal{Y}}=\int_{\dual{\mathcal{X}}}\conj{\pairing{\delta,\omega}}\inner{y,A(\omega)y'}_{\mathcal{Y}}d\mu(\omega)
\end{equation}
is the kernel signature of a shift-invariant $\mathcal{Y}$-Mercer kernel $K$ such that $K(x,z)=K_e(x \groupop \inv{z})$. In other terms, each function $K_e(\cdot)$ is the \acl{FT} of $A(\cdot)\density(\cdot)$ where the integral converges in the weak operator topology and $\density(\omega)=\frac{d\mu}{d\omega}$ is the Radon-Nikodym derivative (density) of the measure $\mu$. If $\mathcal{Y}$ is finite dimensional or $\mathcal{X}$ is compact, any shift-invariant kernel is of the above form for some pair $(A(\omega),\mu (\omega))$. 
\end{proposition}
This theorem is more interesting than \cref{eq:bochner-gen} in the sense that it shows that we are certain of the existence of a \emph{scalar} measure $\mu$ and a positive operator $A(\omega)$, provided that $\mathcal{X}$ is compact or $\mathcal{Y}$ is finite dimensional. When $p=1$ one can always assume $A$ is reduced to the scalar $1$, $\mu$ is still a bounded positive measure and we retrieve the Bochner theorem applied to the scalar case (\cref{th:bochner-scalar}).
\paragraph{}
While \cref{pr:mercer_kernel_bochner} gives some insights on how to build an approximation of a $\mathcal{Y}$-Mercer kernel, we need a theorem that provides an explicit construction of the pair $A(\omega), \mu(\omega)$ from the kernel signature. Proposition 14 in \citet{Carmeli2010} gives the solution, and also provide a sufficient condition for \cref{pr:mercer_kernel_bochner} to apply.
\begin{proposition}[\citet{Carmeli2010}]
\label{pr:inverse_ovk_Fourier_decomposition}
Let $K$ be a shift-invariant $\mathcal{Y}$-Mercer kernel. %
Suppose that $\forall z \in \mathcal{X}$ and $\forall y ,y' \in\mathcal{Y}$, $\inner{K_e(.)y,y'}\in L^1(\mathcal{X},dx)$ where $dx$ denotes the Haar measure on $(\mathcal{X}, \groupop)$. %
Define $C$ such that for all $\omega \in \dual{\mathcal{X}}$ and for all $y$, $y'$ in $\mathcal{Y}$,
\begin{equation}\label{eq:CK0}
\begin{aligned}
\inner{y,C(\omega)y'} &= \int_{\mathcal{X}} \pairing{\delta, \omega}\inner{y, K_e(\delta)y'}d\delta \\
	&=\IFT{\inner{y, K_e(\cdot)y'}}(\omega)
\end{aligned}
\end{equation} %
Then
\begin{enumerate}[i)]
\item $C(\omega)$ is a bounded non-negative operator for all $\omega \in \dual{\mathcal{X}}$,
\item $\inner{y, C(.)y'}\in L^1(\dual{\mathcal{X}},d\omega)$ for all $y,y'\in\mathcal{X}$,
\item for all $\delta\in\mathcal{X}$ and for all $y$, $y'$ in $\mathcal{Y}$,
\begin{equation*}
\inner{y, K_e(\delta)y'}= \int_{\dual{\mathcal{X}}}\conj{\pairing{\delta,\omega}}\inner{y, C(\omega)y'}d\omega.
\end{equation*} %
\end{enumerate}
\end{proposition} %
Gathering the two propositions, we present now the following property that allows to build a spectral decomposition of a shift-invariant $\mathcal{Y}$-Mercer kernel on a \acs{LCA} group $(\mathcal{X},\groupop)$.
\begin{proposition}[Spectral decomposition of shift-invariant $\mathcal{Y}$-Mercer kernel]
\label{pr:spectral}
Let $K_e$ be the signature of a shift-invariant $\mathcal{Y}$-Mercer kernel on $(\mathcal{X}, \groupop)$ and suppose that for all $y$, $y' \in\mathcal{Y}$, $\inner{K_e(.)y,y'}\in L^1(\mathcal{X},dx)$. 
\paragraph{}
If $\mathcal{Y}$ is of finite dimension or $\mathcal{X}$ is compact, then there exists $\mu$, a positive measure on $\mathcal{B}(\mathcal{\dual{\mathcal{X}}})$, and $A:\dual{\mathcal{X}}\to \mathcal{L}_+(\mathcal{Y})$, a operator-valued functions such that for all $y,$ $y'\in\mathcal{Y}$ $\inner{A(.)y,y'}\in L^1(\mathcal{X},d\mu)$ and
\begin{equation*}
\forall (y,y')\in\mathcal{Y}^2, \enskip \inner{y, K_e(\delta)y'}=\int_{\dual{\mathcal{X}}}\conj{\pairing{\delta, \omega}}\inner{y, A(\omega)y'}p_{\mu}(\omega)d\omega.
\end{equation*}
where $\inner{y, A(\omega)y'}p_{\mu}(\omega) = \IFT{\inner{y, K_e(x\groupop \inv{z})y'}}$.
\end{proposition}
\begin{proof}
From \cref{eq:AK0} and \cref{eq:CK0}, if $\mathcal{X}$ is compact or $\mathcal{Y}$ is finite dimensional, we can write the following equality concerning the \acs{OVK} signature $K_e$. For all $\delta \in \mathcal{X}$ and for all $y, y'$ in $\mathcal{Y}$
\begin{equation*}
\int_{\dual{\mathcal{X}}}\conj{\pairing{\delta, \omega}}\inner{y, C(\omega)y'}d\omega=\int_{\dual{\mathcal{X}}}\conj{\pairing{\delta, \omega}}\inner{y, A(\omega)y'}d\mu(\omega).
\end{equation*} %
Since both sides of the equation define continuous functions, the following equality holds $\mu$-almost everywhere. For all $\omega \in \dual{\mathcal{X}}$ and for all $y$, $y'\in\mathcal{Y}$,
\begin{equation}
\label{eq:operator_identification}
\inner{y, C(\omega)y'}=\inner{y, A(\omega)y'}p_{\mu}(\omega)=\IFT{\inner{y, K_e(\cdot)y'}}(\omega),
\end{equation}
where $p_{\mu}(\omega)=\frac{d\mu}{d\omega}$ is the Radon-Nikodym derivative of the measure $\mu$, e.g. its density. 
\end{proof}
In the case where $\mathcal{Y}=\mathbb{R}^p$, we rewrite \cref{eq:operator_identification} coefficient-wisely by choosing the orthonormal basis of $\mathcal{Y}$, $(e_1,\ldots, e_p)$, such that for all $ i,j\inrange{1}{p}$,
\begin{equation}
\label{eq:operator_identification_real}\inner{e_i,C(\omega)e_j}=C(\omega)_{ij}=A(\omega)_{ij}p_{\mu}(\omega)=\IFT{K_e(\delta)_{ij}}.
\end{equation}
It follows that
\begin{equation}\label{eq:matrix-exp}
\forall i,j \inrange{1}{p}, \enskip K_e(x\groupop \inv{z})_{ij}= \FT{A(\cdot)_{ij}}
%\expectation_{\mu}[\conj{\pairing{x\myop \myinv{z},\omega}} A(\omega)_{ij}].
\end{equation}
%This leads to the following corollary of \cref{pr:spectral}:
%\begin{proposition}\label{pr:inverseTF-kernel}
%Let $\mathcal{Y}=\mathbb{R}^p$. In the same conditions as \ref{pr:spectral}, we have:

\begin{remark}
Note that although the \acl{IFT} of $K_e$ yields a unique operator-valued function $C(\cdot)$, the decomposition of $C(\omega)$ into $A(\omega)p_\mu(\omega)$ is not unique. The choice of the decomposition may be justified by the computational cost or by the nature of the constants involved in the uniform convergence of the estimator.
\end{remark}

\subsection{Functional Fourier feature map}

\subsection{Regularization property}
We have shown so far that it is always possible to construct a feature map that allows to approximate a shift-invariant $\mathcal{Y}$-Mercer kernel. However we could also propose a construction of such map by studying the regularization induced with respect to the Fourier transform of a target function $f\in \mathcal{H}_K$. In other words, what is the norm in $L^2(\dual{\mathcal{X}}, d\omega, \mathcal{Y}')$ induced by $\norm{\cdot}_K$?
\begin{proposition}
Let $K$ be a shift-invariant $\mathcal{Y}$-Mercer Kernel such that for all $y$, $y'$ in $\mathcal{Y}$, $\inner{y, K_e(\cdot)y'}\in L^1(\mathcal{X}, dx)$. 
\paragraph{}
If $\forall y$, $y'\in\mathcal{Y}$, $\inner{y, A(\omega)y'}p_\mu(\omega):=\IFT{\inner{y, K_e(\cdot)y'}}(\omega)$. Let $f\in\mathcal{H}_K$ then
\begin{equation}
\norm{f}^2_K=\displaystyle\int_{\dual{\mathcal{X}}}\frac{\inner{\FT{f}(\omega), A(\omega)^\dagger\FT{f}(\omega)}_\mathcal{Y}}{p_\mu(\omega)}d\omega.
\label{eq:reg_L2}
\end{equation}
\label{pr:regularization}
\end{proposition}
\begin{proof}
We first show how the Fourier transform relates to the feature operator. Since $\mathcal{Y}$ is separable and $\mathcal{H}_K$ is embed into $L^2(\dual{\mathcal{X}}, \mu, \mathcal{Y})$ by mean of the feature operator $W$, we have:
\begin{equation*}
\begin{aligned}
\FT{\IFT{g}}(x)=\int_{\dual{\mathcal{X}}}\overline{\pairing{x,\omega}}\IFT{g}(\omega)d\omega &=g(x) \\
(Wg)(x)=\int_{\dual{\mathcal{X}}}\overline{\pairing{x,\omega}}p_\mu(\omega)B(\omega)g(\omega)d\omega&=g(x).
\end{aligned}
\end{equation*}
Hence, $\IFT{f}(\omega)=p_\mu(\omega)B(\omega)g(\omega)$ $\mu$-almost everywhere. From \cref{pr:feature_operator} we have,
\begin{equation*}
\begin{aligned}
\norm{f}^2_K &= \inf \left\{\norm{g}^2_\mathcal{H} \enskip \middle| \forall g\in\mathcal{H}, \enskip Wg=f \right\} \\
&= \begin{multlined}[t][0.8\textwidth]\inf \big\{\int_{\dual{\mathcal{X}}} \norm{g(\omega)}^2_\mathcal{Y}d\mu(\omega) \enskip \big| \\ \enskip \forall g\in\mathcal{H},\enskip \IFT{f}(\omega)=p_\mu(\omega)B(\omega)g(\omega) \big\}.\end{multlined}
\end{aligned}
\end{equation*}
The pseudo inverse of the operator $B(\omega)$ (noted $B(\omega)^\dagger$) is the unique solution of the system $\IFT{f}(\omega)=p_\mu(\omega)B(\omega)g(\omega)$ w.r.t. to $g(\omega)$ with minimal norm. Eventually,
\begin{equation}
\norm{f}^2_K = \displaystyle\int_{\dual{\mathcal{X}}} \frac{\norm{B(\omega)^\dagger\IFT{f}(\omega)}^2_\mathcal{Y}}{p_\mu(\omega)^2}d\mu(\omega)
= \displaystyle\int_{\dual{\mathcal{X}}} \frac{\norm{B(\omega)^\dagger\FT{f}(\omega)}^2_\mathcal{Y}}{p_\mu(\omega)^2}d\mu(\omega)
\end{equation}
Conclude the proof by taking $d\mu(\omega)=p_\mu(\omega)d\omega$.
\end{proof}
Note that if $K(x,z)=k(x,z)$ is a scalar kernel then for all $\omega$ in $\dual{\mathcal{X}}$, $A(\omega)=1$. Therefore we recover a well known results for kernels that is for any $f\in\mathcal{H}_k$ we have $\norm{f}_k=\int_{\dual{\mathcal{X}}}\FT{k_e}(\omega)^{-1}\FT{f}(\omega)^2d\omega$. We also note that the regularization property in $\mathcal{H}_K$ does not depends (as expected) on the decomposition of $A(\omega)$ into $B(\omega)B(\omega)^*$. Therefore the decomposition should be chosen such that it optimizes the computation cost. For instance if $A(\omega)\in\mathcal{L}(\mathbb{R}^p)$ has rank $r$, one could find an operator $B(\omega)\in\mathcal{L}(\mathbb{R}^p, \mathbb{R}^r)$ such that $A(\omega)=B(\omega)B(\omega)^*$.

\subsection{Building Operator-valued Random Fourier Features}
Without loss of generality we assume that $\int_{\mathcal{X}} d\mu(\omega)=1$ and thus, $\mu$ is a probability distribution and $\density$, a probability density. Note that this is always possible through an appropriate rescaling of the kernel.

Given a $\mathcal{Y}$-Mercer shift-invariant kernel $K$ on $\mathcal{X}$, an approximation of $K$ can be obtained using a decomposition $(A, \mu)$ and a plug-in Monte-Carlo estimator instead of the expectation. However, for efficient computations, as motivated in the introduction, we are interested in finding an approximated feature map more than a kernel approximation. Indeed, an approximated feature map will allow to build linear models in regression tasks. The following proposition provides the general form of an Operator Random Fourier Feature map:
\begin{proposition}\label{pr:ORFF-map}If one can find $B: \dual{\mathcal{X}} \to \mathcal{L}(Y)$, such that for all $y$, $y'\in\mathcal{Y}$, $\inner{y, A(\omega)y'}=\inner{y, B(\omega)B(\omega)^*y'} \in L^1(\dual{\mathcal{X}}, d\mu)$, then the operator-valued function
\begin{equation}\label{eq:phitilde}
\tildePhi(x)= \frac{1}{\sqrt{D}}\Vect_{j=1}^D\pairing{x, \omega_j}B(\omega_j)^*, \qquad \omega_j \sim \mu
\end{equation}
is an approximated feature map of kernel $K$
\end{proposition}

\begin{proof}
Let $\omega_1, \ldots, \omega_D$ be $D$ \iid~random vectors following the law $\mu$. For all $(x,z) \in \mathcal{X}^2$, 
\begin{equation*}
\begin{aligned}
\tildePhi(x)^*\tildePhi(z)&= \!\begin{multlined}[t][.8\textwidth] \left(\frac{1}{\sqrt{D}}\Vect_{j=1}^D \exp\left(ih_{\omega_j}(x)B(\omega_j)^*\right)\right)^* {}\\ \left(\frac{1}{\sqrt{D}}\Vect_{j=1}^D \exp\left(ih_{\omega_j}(z)B(\omega_j)^*\right)\right) \end{multlined}\\
&= \frac{1}{D} \sum_{j=1}^D \exp\left(-i(h_{\omega_j}(x)-h_{\omega_j}(z)\right) A(\omega_j)\\
&= \frac{1}{D} \sum_{j=1}^D \conj{\pairing{x,z}}A(\omega_j)
\end{aligned}
\end{equation*}
From the strong law of large numbers, $ \frac{1}{D} \sum_{j=1}^D \conj{\pairing{x,z}}A(\omega_j)$ 
converges almost-surely in the weak operator topology to $\expectation_{\mu}[\conj{\pairing{x\groupop z^{-1},\omega_j}}A(\omega) ]$ when $D$ tends to infinity.
\end{proof}
\begin{remark}
We find a decomposition such that for all $j=1, \ldots, D$, $A(\omega_j)=B(\omega_j)B(\omega_j)^*$ either by exhibiting an analytic closed-form or using a numerical decomposition. 
\end{remark}
This proposition leads to the following construction algorithm.
\SetKwInOut{Input}{Input}
\SetKwInOut{Output}{Output}
\begin{center}
\begin{algorithm2e}[H]
	\SetAlgoLined
    \Input{$K(x, z)=K_e(\delta)$ a $\mathcal{Y}$-shift-invariant Mercer kernel such that $\forall y,y'\in\mathcal{Y},$ $\inner{y, K_e(\delta)y'}\in L^1(\mathbb{R}^d, dx)$.}
    \Output{A random feature $\tildePhi(x)$ such that $\tildePhi(x)^\adjoint \tildePhi(z) \approx K(x,z)$}
    \BlankLine
	Define the pairing $\pairing{x, \omega}$ from the \acs{LCA} group $(\mathcal{X}, \groupop)$\;
	Find a decomposition $(B(\omega)$,$\density(\omega))$ such that $B(\omega)B(\omega)^*\density(\omega)=\IFT{K_e}(\omega)$\;
% 	Build an randomized feature map via Monte-Carlo sampling from the probability measure $\mu$ and the application $B$\;
	Draw $D$ random vectors $\omega_j$, $j=1, \hdots, D$ from the probability distribution $\mu$\;
   \Return $\tildePhi(x)=\frac{1}{\sqrt{D}}\Vect_{j=1}^D\pairing{x, \omega_j}B(\omega_j)^*$\;
   \caption{Construction of ORFF}
   \label{al:ORFF_construction}
\end{algorithm2e}
\end{center}

%----------------------------------------------------------------------------------------
\section{Uniform bound on the approximation}
\label{sec:uniform_bound_on_the_approximation}

%----------------------------------------------------------------------------------------
\section{Learning with operator-valued random-Fourier features}
\label{sec:learning_with_operator-valued_random-fourier_features}

%----------------------------------------------------------------------------------------
\section{Consistency and generalization bounds}
\label{sec:consistency and generalization bounds}

%----------------------------------------------------------------------------------------
\section{Conclusions}
\label{sec:conclusions}
