%!TEX root = ../../ThesisRomainbrault.tex

\section{Learning with ORFF}
\label{sec:learning_with_operator-valued_random-fourier_features}
We now turn our attention to learning function with an ORFF model that approximate an OVK model.
\subsection{Warm-up: supervised regression}
Let $\seq{s} = (x_i,y_i)_{i=1}^N\in\left(\mathcal{X}\times\mathcal{Y}\right)^N$ be a sequence of training samples. Given a local loss function $L: \mathcal{X}\times\mathcal{F}\times\mathcal{Y}\to \overline{\mathbb{R}}$ such that $L$ is proper, convex and lower semi-continous in $f$, we are interested in finding a \emph{vector-valued function} $f_{\seq{s}}:\mathcal{X}\to\mathcal{Y}$, that lives in a \acs{vv-RKHS} and minimize a tradeoff between a data fitting term $L$ and a regularization term to prevent from overfitting. Namely finding $f_{\seq{s}}\in\mathcal{H}_K$ such that
\begin{dmath}
f_{\seq{s}} = \argmin_{f\in\mathcal{H}_K}  \frac{1}{N}\displaystyle\sum_{i=1}^NL(x_i, f, y_i) + \frac{\lambda}{2}\norm{f}^2_{K},
\label{eq:learning_rkhs}
\end{dmath}
where $\lambda\in\mathbb{R}_+$ is a regularization\mpar{Tychonov regularization.} parameter. We call the quantity
\begin{dmath*}
\mathcal{R}(f)=\frac{1}{N}\displaystyle\sum_{i=1}^NL(x_i, f, y_i) \condition{$\forall f\in\mathcal{H}_K$, $\forall \seq{s}\in\left(\mathcal{X}\times\mathcal{Y}\right)^N$.}
\end{dmath*}
The empirical risk of the model $f\in\mathcal{H}_K$. A common choice of data fitting term for regression is $L:(x_i, f, y_i) \mapsto \norm{f(x_i)-y_i}_{\mathcal{Y}}^2$.
We introduce a corollary from Mazur and Schauder proposed in 1936 (see \citet{kurdila2006convex, gorniewicz1999topological}) showing that \cref{eq:learning_rkhs} --and \cref{eq:learning_rkhs_gen}-- attains a unique mimimizer.
\begin{theorem}[Mazur-Schauder]
\label{cor:unique_minimizer}
Let $\mathcal{H}$ be a Hilbert space and $J:\mathcal{H}\to \overline{\mathbb{R}}$ be a proper, convex, lower semi-continuous and coercive function. Then $J$ is bounded from below and attains a minimizer. Moreover if $J$ is strictly convex the minimizer is unique.
\end{theorem}
This is easily verified for Ridge regression. Define
\begin{dmath}
\label{eq:ridge}
J_\lambda(f)=\frac{1}{N}\sum_{i=1}^N\norm{f(x_i)-y_i}_{\mathcal{Y}}^2+\frac{\lambda}{2}\norm{f}_K^2,
\end{dmath}
where $f\in\mathcal{H}_K$ and $\lambda\in\mathbb{R}_{>0}$. $J_\lambda$ is continuous\mpar{Reminder, if $f\in\mathcal{H}_k, \text{ev}_x:f\mapsto f(x)$ is continuous, see \cref{pr:unique_rkhs}.} and strictly convex. Additionally $J_\lambda$ is coercive since $\norm{f}_K$ is coercive, $\lambda\in\mathbb{R}_{>0}$, and all the summands of $J_\lambda$ are positive. Hence for all positive $\lambda$, $f_{\seq{s}}=\argmin_{f\in\mathcal{H}_K}J_\lambda(f)$ exists, is unique and attained.
\begin{remark}
\label{rk:rkhs_bound}
We condider the optimization problem proposed in \cref{eq:ridge} where $L:(x_i, f, y_i) \mapsto \norm{f(x_i)-y_i}_{\mathcal{Y}}^2$. If given a training sample $\seq{s}$, we have
\begin{dmath*}
\frac{1}{N}\sum_{i=1}^N\norm{y_i}_{\mathcal{Y}}^2 \le \sigma_y^2,
\end{dmath*}
then $\lambda\norm{f_{\seq{s}}}_K\le 2\sigma_y^2$. Indeed, since $\mathcal{H}_K$ is a Hilbert space, $0\in\mathcal{H}_K$, thus
\begin{dmath*}
\frac{\lambda}{2}\norm{f_{\seq{s}}}^2_{K} \le \frac{1}{N}\displaystyle\sum_{i=1}^NL(x_i, f_{\seq{s}}, y_i) + \frac{\lambda}{2}\norm{f_{\seq{s}}}^2_{K}
\le \frac{1}{N}\displaystyle\sum_{i=1}^NL(x_i, 0, y_i) \hiderel{\le} \sigma_y^2 \condition{by optimality of $f_{\seq{s}}$.}
\end{dmath*}
Since for all $x\in\mathcal{X}$, $\norm{f(x)}_{\mathcal{Y}}\le \sqrt{\norm{K(x, x)}_{\mathcal{Y},\mathcal{Y}}}\norm{f}_{K}$, the maximum value that the solution $\norm{f_{\seq{s}}(x)}_{\mathcal{Y}}$ of \cref{eq:ridge} can reach is $2\sqrt{\norm{K(x, x)}}\frac{\sigma_y^2}{\lambda}$. Thus when solving a Ridge regression problem, given a shift-invariant kernel $K_e$, one should choose
\begin{dmath*}
0 \hiderel{<} \lambda \hiderel{\le} 2\frac{\sqrt{\norm{K_e(e)}}\sigma_y^2}{C}.
\end{dmath*}
with $C\in\mathbb{R}_{>0}$ to have a chance to fit all the $y_i$ with norm $\norm{y_i}_{\mathcal{Y}} \le C$ in the train set.
\end{remark}
\subsection{Semi-supervised regression}
Regression in \acl{vv-RKHS} has been well studied \citep{Alvarez2012, Minh_icml13,minh2016unifying,sangnier2016joint,kadri2015operator,Micchelli2005,Brouard2016_jmlr}, and a cornerstone of learning in \acs{vv-RKHS} is the representer theorem\mpar{Sometimes referred to as minimal norm interpolation theorem.}, which allows to replace the search of a minimizer in a infinite dimensional \acs{vv-RKHS} by a finite number of paramaters $(u_i)_{i=1}^N$, $u_i\in\mathcal{Y}$. We present here the very genreal form of \citet{minh2016unifying}. This framework encompass Vector-valued Manifold Regularization \citep{belkin2006manifold,Brouard2011,minh2013unifying} and Co-regularized Multi-view Learning \citep{brefeld2006efficient,sindhwani2008rkhs,rosenberg2009kernel,sun2011multi}.
\paragraph{}
In the following we suppose we are given a cost function $c:\mathcal{Y}\times\mathcal{Y}\to\overline{\mathbb{R}}$, such that $c(f(x),y)$ returns the error of the prediction $f(x)$ \wrt~the ground truth $y$. A loss function of a model $f$ with respect to an example $(x,y)\in\mathcal{X}\times\mathcal{Y}$ can be naturally defined from a cost function as $L(x,f,y)=c(f(x),y)$. Conceptually the function $c$ evaluate the quality of the prediction versus its ground truth $y\in\mathcal{Y}$ while the loss function $L$ evaluate the quality of the model $f$ at a training point $(x,y)\in\mathcal{X}\times\mathcal{Y}$. Moreover we suppose that we are given a training sample $\seq{u}=(x_i)_{i=N}^{N+U}\in\mathcal{X}^U$ of unlabelled exemple. We note $\seq{z}\in\left(\mathcal{X}\times\mathcal{Y}\right)^N\times\mathcal{X}^U$ the sequence $\seq{z}=\seq{s}\seq{u}$ concatenating both labeled ($\seq{s}$) and unlabelled ($\seq{u}$) training examples.
\begin{theorem}[Representer \citep{minh2016unifying}]
\label{th:representer}
Let $K$ be a $\mathcal{U}$-Mercer \acl{OVK} and $\mathcal{H}_K$ its corresponding $\mathcal{U}$-Reproducing Kernel Hilbert space.
\paragraph{}
Let $V:\mathcal{U}\to\mathcal{Y}$ be a bounded linear operator and let $c:\mathcal{Y}\times\mathcal{Y}\to\overline{\mathbb{R}}$ be a cost function such that $L(x, f, y)=c(Vf(x), y)$ is a proper convex lower semi-continuous function in $f$ for all $x\in\mathcal{X}$ and all $y\in\mathcal{Y}$.
\paragraph{}
Eventually let $\lambda_K\in\mathbb{R}_{>0}$ and $\lambda_M \in \mathbb{R}_+$ be two regularization hyperparameters and $(M_{ik})_{i,k=1}^{N+U}$ be a sequence of data dependent bounded linear operators in $\mathcal{L}(\mathcal{U})$, such that
\begin{dmath*}
\sum_{i,j=1}^{N+U} \inner{u_i, M_{ik}u_k} \ge 0 \condition{$\forall (u_i)_{i=1}^{N+U}\in\mathcal{U}^{N+U}$ and $M_{ik}=M_{ki}^*$}.
\end{dmath*}
The solution $f_{\seq{z}}\in\mathcal{H}_K$ of the regularized optimization problem
\begin{dmath}
f_{\seq{z}} = \argmin_{f\in\mathcal{H}_K} \frac{1}{N}\displaystyle\sum_{i=1}^N c(Vf(x_i), y_i) + \frac{\lambda_K}{2}\norm{f}^2_{K} \\ + \frac{\lambda_M}{2}\sum_{i,k=1}^{N+U}\inner{f(x_i), M_{ik}f(x_k)}_{\mathcal{U}}
\label{eq:learning_rkhs_gen}
\end{dmath}
has the form $f_{\seq{z}}=\sum_{j=1}^{N+U}K(\cdot,x_j)u_{\seq{z},j}$ where $u_{\seq{z},j}\in\mathcal{U}$ and
\begin{dmath}
    \label{eq:argmin_u}
    u_{\seq{z}} = \argmin_{u\in\Vect_{i=1}^{N+U}\mathcal{U}}\frac{1}{N}\displaystyle\sum_{i=1}^N c\left(V\sum_{k=1}^{N+U}K(x_i,x_j)u_j, y_i\right) + \frac{\lambda_K}{2}\sum_{k=1}^{N+U}u_i^\adjoint K(x_i,x_k)u_k \\ +
    \frac{\lambda_M}{2}\sum_{i,k=1}^{N+U} \inner*{\sum_{j=1}^{N+U}K(x_i,x_j)u_j, M_{ik}\sum_{j=1}^{N+U}K(x_k,x_j)u_j}_{\mathcal{U}}
\end{dmath}
\end{theorem}
The first representer theorem was first introduced by \citet{Wahba90} in the case where $\mathcal{Y}=\mathbb{R}$. The extension to an arbitrary Hilbert space $\mathcal{Y}$ has been proved by many authors in different forms \citep{Brouard2011,kadri2015operator,Micchelli2005}. The idea behind the representer theorem is that eventhough we minimize over the whole space $\mathcal{H}_K$, when $\lambda_K>0$, the solution of \cref{eq:learning_rkhs_gen} falls inevitably into the set $\mathcal{H}_{K, \seq{z}}=\Set{\sum_{j=1}^{N+U}K_{x_j}u_j| \forall (u_i)_{i=1}^{N+U} \in\mathcal{U}^{N+U}}$. Therefore the result can be expressed as a finite linear combination of basis functions of the form $K(\cdot,x_k)$. Remark that we can perform the kernel expansion of $f_{\seq{z}}=\sum_{j=1}^{N+U}K(\cdot,x_j)u_{\seq{z},j}$ eventhough $\lambda_K=0$. However $f_{\seq{z}}$ is no longer the solution of \cref{eq:learning_rkhs_gen} over the whole space $\mathcal{H}_K$ but a projection on the subspace $\mathcal{H}_{K, \seq{z}}$. While this is in general not a problem for practical applications, it might raise issues for further theoretical investigations. In particular, it makes it difficult to perform theoretical comparison the \say{exact} solution of \cref{eq:learning_rkhs_gen} with respect to the \acs{ORFF} approximation solution given in \cref{cr:orff_representer}.
\paragraph{}
We present here the proof of the generic formulation proposed by \citet{minh2016unifying}. In the mean time we clarify some elements of the proof. Indeed the existence of a global minimizer is not trivial and we must invoke the Mazur-Schauder theorem. Moreover the coercivity of the objective function required by the Mazur-Schauder theorem is not obvious when we do not require the cost function to take only positive values. However a corollary of Hahn-Banach theorem linking strong convexity to coercivity gives the solution.
\begin{proof}
Since $f(x)=K_x^*f$ (see \cref{eq:reproducing_prop}), the optimization problem reads
\begin{dmath*}
f_{\seq{z}} = \argmin_{f\in\mathcal{H}_K} \frac{1}{N}\displaystyle\sum_{i=1}^N c(VK_{x_i}^\adjoint f, y_i) + \frac{\lambda_K}{2}\norm{f}^2_{K} \\ + \frac{\lambda_M}{2}\sum_{i,k=1}^{N+U}\inner{K_{x_i}^\adjoint f, M_{ik}K_{x_k}^\adjoint f}_{\mathcal{U}}
\end{dmath*}
Let $W_{V,\seq{s}}:\mathcal{H}_K\to\Vect_{i=1}^N\mathcal{Y}$ be a linear operator defined as
\begin{dmath*}
W_{V,\seq{s}}f = \Vect_{i=1}^N CK_{x_i}^\adjoint f,
\end{dmath*}
with $VK_{x_i}^\adjoint:\mathcal{H}_K\to\mathcal{Y}$ and $K_{x_i}V^\adjoint:\mathcal{Y}\to\mathcal{H}_K$. Let $Y=\vect_{i=1}^Ny_i\in\mathcal{Y}^N$. We have
\begin{dmath*}
\inner{Y, W_{V,\seq{s}}f}_{\Vect_{i=1}^N\mathcal{Y}}=\sum_{i=1}^N\inner{y_i, VK_{x_i}^\adjoint f}_{\mathcal{Y}}
\hiderel{=}\sum_{i=1}^N\inner{K_{x_i}V^\adjoint y_i, f}_{\mathcal{H}_K}.
\end{dmath*}
Thus the adjoint operator $W_{V,\seq{s}}^\adjoint:\Vect_{i=1}^N\mathcal{Y}\to\mathcal{H}_K$ is
\begin{dmath*}
W_{V,\seq{s}}^\adjoint Y=\sum_{i=1}^NK_{x_i}V^\adjoint y_i,
\end{dmath*}
and the operator $W_{V,\seq{s}}^*W_{V,\seq{s}}:\mathcal{H}_K\to\mathcal{H}_K$ is
\begin{dmath*}
W_{V,\seq{s}}^\adjoint W_{V,\seq{s}}f = \sum_{i=1}^NK_{x_i}V^\adjoint VK_{x_i}^\adjoint f
\end{dmath*}
where $V^\adjoint V\in\mathcal{L}(\mathcal{U})$. Let
\begin{dmath*}
J_{\lambda_K}(f) = \underbrace{\frac{1}{N}\displaystyle\sum_{i=1}^N c(Vf(x_i), y_i)}_{=J_c} + \frac{\lambda_K}{2}\norm{f}^2_{K} \\ + \underbrace{\frac{\lambda_M}{2}\sum_{i,k=1}^{N+U}\inner{f(x_i), M_{ik}f(x_k)}_{\mathcal{U}}}_{=J_M}
\end{dmath*}
To ensure that $J_{\lambda_K}$ has a global minimizer we need the following technical lemma (which is a consequence of the Hahn-Banach theorem for lower-semicontimuous functional, see \citet{kurdila2006convex}).
\begin{lemma}
\label{lm:strongly_convex_is_coercive}
Let $J$ be a proper, convex, lower semi-continuous functional, defined on a Hilbert space $\mathcal{H}$. If $J$ is strongly convex, then $J$ is coercive.
\end{lemma}
\begin{proof}
Consider the convex function function $G(f)\colonequals J(f)-\lambda\norm{f}^2$, for some $\lambda>0$. Since $J$ is by assumption proper, lower semi-continuous and strongly convex with parameter $\lambda$, $G$ is proper, lower semi-continuous and convex. Thus Hahn-Banach theorem apply, stating that $G$ is bounded by below by an affine functional. \Ie~there exists $f_0$ and $f_1\in\mathcal{H}$ such that
\begin{dmath*}
G(f)\ge G(f_0) + \inner{f - f_0, f_1} \condition{for all $f\in\mathcal{H}$.}
\end{dmath*}
Then substitute the definition of $G$ to obtain
\begin{dmath*}
J(f)\ge J(f_0) + \lambda\left(\norm{f}-\norm{f_0}\right) + \inner{f - f_0, f_1}.
\end{dmath*}
By the Cauchy-Schwartz inequality, $\inner{f, f_1}\ge - \norm{f}\norm{f_1}$, thus
\begin{dmath*}
J(f)\ge J(f_0) + \lambda\left(\norm{f}-\norm{f_0}\right) - \norm{f}\norm{f_1} - \inner{f_0, f_1},
\end{dmath*}
which tends to infinity as $f$ tends to infinity. Hence $J$ is coercive
\end{proof}
for all $f\in\mathcal{H}_K$. Since $c$ is proper, lower semi-continuous and convex by assumption, thus the term $J_c$ is also proper, lower semi-continuous and convex. Moreover the term $J_M$ is always positive for any $f\in\mathcal{H}_K$ and $\frac{\lambda_K}{2}\norm{f}^2_{K}$ is strongly convex. Thus $J_{\lambda_K}$ is strongly convex. Apply \cref{lm:strongly_convex_is_coercive} to obtain the coercivity of $J_{\lambda_K}$, and then \cref{cor:unique_minimizer} to show that $J_{\lambda_K}$ has a unique minimizer and is attained. Then let $\mathcal{H}_{K, \seq{z}}=\Set{\sum_{j=1}^{N+U}K_{x_j}u_j| \forall (u_i)_{i=1}^{N+U} \in\mathcal{U}^{N+U}}$. For $f\in\mathcal{H}_{K, \seq{z}}^\perp$\mpar{$\mathcal{H}_{K, \seq{z}}^\perp\oplus\mathcal{H}_{K, \seq{z}}=\mathcal{H_K}$.}, the operator $W_{V,\seq{s}}$ satisfies
\begin{dmath*}
\inner{Y, W_{V,\seq{s}}f}_{\Vect_{i=1}^N\mathcal{Y}} = \inner{\underbrace{f}_{\in\mathcal{H}_{K, \seq{z}}^\perp}, \underbrace{\sum_{i=1}^{N+U}K_{x_i}V^\adjoint y_i}_{\in\mathcal{H}_{K, \seq{z}}}}_{\mathcal{H}_K} \hiderel{=} 0
\end{dmath*}
for all sequences $(y_i)_{i=1}^N$, since $V^\adjoint y_i\in\mathcal{U}$. Hence,
\begin{dmath}
\label{eq:null1}
(Vf(x_i))_{i=1}^{N}=0
\end{dmath}
In the same way,
\begin{dmath*}
\sum_{i=1}^{N+U}\inner{K_{x_i}^* f, u_i}_{\mathcal{U}} \hiderel{=} \inner{\underbrace{f}_{\in\mathcal{H}_{K, \seq{z}}^\perp}, \underbrace{\sum_{j=1}^{N+U}K_{x_j}u_j}_{\in\mathcal{H}_{K, \seq{z}}}}_{\mathcal{H}_K} \hiderel{=} 0.
\end{dmath*}
for all sequences $(u_i)_{i=1}^{N+U}\in\mathcal{U}^{N+U}$. As a result,
\begin{dmath}
\label{eq:null2}
(f(x_i))_{i=1}^{U+N}=0.
\end{dmath}
Now for an arbitrary $f\in\mathcal{H_K}$, consider the orthogonal decomposition $f=f^{\perp}+f^{\parallel}$, where $f^{\perp}\in\mathcal{H}_{K, \seq{z}}^\perp$ and $f^{\parallel}\in\mathcal{H}_{K, \seq{z}}$. Then since $\norm{f^{\perp}+f^{\parallel}}_{\mathcal{H}_K}^2=\norm{f^{\perp}}_{\mathcal{H}_K}^2+\norm{f^{\parallel}}_{\mathcal{H}_K}^2$, \cref{eq:null1} and \cref{eq:null2} shows that if $\lambda_K\ge 0$, clearly then
\begin{dmath*}
J_{\lambda_K}(f)=J_{\lambda_K}\left(f^{\perp}+f^{\parallel}\right) \hiderel{\ge} J_{\lambda_K}\left(f^{\parallel}\right)
\end{dmath*}
The last inequality holds only when $\norm{f^{\perp}}_{\mathcal{H}_K}=0$, that is when $f^{\perp}=0$. As a result since the minimizer of $J_{\lambda_K}$is unique and attained, it must lies in $\mathcal{H}_{K, \seq{z}}$.
\end{proof}
The representer theorem show that minimizing a functional in a \acs{vv-RKHS} yields a solution which depends on all the points in the training set. Assuming that for all $x_i$, $x\in\mathcal{X}$ and for all $u_i\in\mathcal{Y}$ it takes time $O(P)$, to compute $K(x_i, x)u_i$, making a prediction using the representer theorem take $O(2P)$. Obviously If $\mathcal{Y}=\mathbb{R}^p$, Then $P=O(p^2)$ thus making a prediction cost $O(2p^2)$ operations.
\paragraph{}
Instead learning a model $f$ that depends on all the points of the training set, we would like to learn a parametric model of the form
$\tildef{\omega}(x)=\tildePhi{\omega}(x)^\adjoint \theta$, where $\theta$ lives in some redescription space $\tildeH{\omega}$. We are interested in finding a parameter vector $\theta_{\seq{z}}$ such that
\begin{dmath}
\label{eq:argmin_applied}
\theta_{\seq{z}}=\argmin_{\theta\in \tildeH{\omega}} \frac{1}{N}\sum_{i=1}^Nc\left(V\tildePhi{\omega}(x_i)^\adjoint \theta, y_i\right) + \frac{\lambda_K}{2}\norm{\theta}^2_{\tildeH{\omega}} \\ + \frac{\lambda_M}{2}\sum_{i,k=1}^{N+U}\inner{\theta, \tildePhi{\omega}(x_i)M_{ik}\tildePhi{\omega}(x_k)^\adjoint \theta}_{\tildeH{\omega}}
\end{dmath}

\begin{corollary}[\acs{ORFF} representer]
\label{cr:orff_representer}
Let $\tildeK{\omega}$ be an \acl{OVK} such that for all $x$, $z\in\mathcal{X}$, $\tildePhi{\omega}(x)^\adjoint \tildePhi{\omega}(z) = \widetilde{K}(x,z)$ where $\widetilde{K}$ is a $\mathcal{U}$-Mercer \acs{OVK} and $\mathcal{H}_{\tildeK{\omega}}$ its corresponding $\mathcal{U}$-Reproducing kernel Hilbert space.
\paragraph{}
Let $V:\mathcal{U}\to\mathcal{Y}$ be a bounded linear operator and let $c:\mathcal{Y}\times\mathcal{Y}\to\overline{\mathbb{R}}$ be a cost function such that $L(x, \widetilde{f}, y)=c(V\widetilde{f}(x), y)$ is a proper convex lower semi-continuous function in $\widetilde{f}\in\mathcal{H}_{\tildeK{\omega}}$ for all $x\in\mathcal{X}$ and all $y\in\mathcal{Y}$.
\paragraph{}
Eventually let $\lambda_K\in\mathbb{R}_{>0}$ and $\lambda_M \in \mathbb{R}_+$ be two regularization hyperparameters and $(M_{ik})_{i,k=1}^{N+U}$ be a sequence of data dependent bounded linear operators in $\mathcal{L}(\mathcal{U})$, such that
\begin{dmath*}
\sum_{i,j=1}^{N+U} \inner{u_i, M_{ik}u_k} \ge 0 \condition{$\forall (u_i)_{i=1}^{N+U}\in\mathcal{U}^{N+U}$ and $M_{ik}=M_{ki}^*$}.
\end{dmath*}
The solution $f_{\seq{z}}\in\mathcal{H}_{\tildeK{\omega}}$ of the regularized optimization problem
\begin{dmath}
\label{eq:argmin_RKHS_rand}
\widetilde{f}_{\seq{z}} = \argmin_{\widetilde{f}\in\mathcal{H}_{\tildeK{\omega}}} \frac{1}{N}\displaystyle\sum_{i=1}^N c\left(V\widetilde{f}(x_i), y_i\right) + \frac{\lambda_K}{2}\norm{\widetilde{f}}^2_{\tildeK{\omega}} \\ + \frac{\lambda_M}{2}\sum_{i,k=1}^{N+U}\inner{\widetilde{f}(x_i), M_{ik}\widetilde{f}(x_k)}_{\mathcal{U}}
\end{dmath}
has the form $\widetilde{f}_{\seq{z}}=\tildePhi{\omega}(\cdot)^\adjoint\theta_{\seq{z}}$, where $\theta_{\seq{z}}\in (\Ker \tildeW{\omega})^{\perp}$ and
\begin{dmath}
\theta_{\seq{z}}=\argmin_{\theta\in \tildeH{\omega}} \frac{1}{N}\sum_{i=1}^Nc\left(V\tildePhi{\omega}(x_i)^\adjoint \theta, y_i\right) + \frac{\lambda_K}{2}\norm{\theta}^2_{\tildeH{\omega}} \\ + \frac{\lambda_M}{2}\sum_{i,k=1}^{N+U}\inner{\theta, \tildePhi{\omega}(x_i)M_{ik}\tildePhi{\omega}(x_k)^\adjoint \theta}_{\tildeH{\omega}}.
\end{dmath}
\end{corollary}
\begin{proof}
Since $\tildeK{\omega}$ is an operator-valued kernel, from \cref{th:representer}, \cref{eq:argmin_RKHS_rand} has a solution of the form
\begin{dmath*}
\widetilde{f}_{\seq{z}} = \sum_{i=1}^{N+U} \tildeK{\omega}(\cdot, x_i)u_i \condition{$u_i \hiderel{\in} \mathcal{U}$, $x_i \hiderel{\in}\mathcal{X}$}
= \sum_{i=1}^N \tildePhi{\omega}(\cdot)^\adjoint \tildePhi{\omega}(x_i)u_i
\hiderel{=} \tildePhi{\omega}(\cdot)^\adjoint \underbrace{\left(\sum_{i=1}^{N+U}\tildePhi{\omega}(x_i)u_i\right)}_{=\theta\in \left(\Ker \tildeW{\omega}\right)^\perp\subset \tildeH{\omega}}.
\end{dmath*}
Let
\begin{dmath*}
\label{eq:argmin_theta}
\theta_{\seq{z}}=\argmin_{\theta\in\left(\Ker \tildeW{\omega}\right)^\perp} \frac{1}{N}\sum_{i=1}^Nc\left(V\tildePhi{\omega}(x_i)^\adjoint \theta, y_i\right) + \frac{\lambda_K}{2}\norm{\tildePhi{\omega}(\cdot)^\adjoint\theta}^2_{\tildeK{\omega}} \\ + \frac{\lambda_M}{2}\sum_{i,k=1}^{N+U}\inner*{\tildePhi{\omega}(x_i)^\adjoint \theta, M_{ik}\tildePhi{\omega}(x_k)^\adjoint \theta}_{\mathcal{U}}.
\end{dmath*}
Since $\theta\in(\Ker \tildeW{\omega})^\perp$ and $W$ is an isometry from $(\Ker \tildeW{\omega})^\perp\subset \tildeH{\omega}$ onto $\mathcal{H}_{\tildeK{\omega}}$, we have $\norm{\tildePhi{\omega}(\cdot)^\adjoint\theta}^2_{\tildeK{\omega}} = \norm{\theta}^2_{\tildeH{\omega}}$. Hence
\begin{dmath*}
\theta_{\seq{z}}=\argmin_{\theta\in\left(\Ker \tildeW{\omega}\right)^\perp} \frac{1}{N}\sum_{i=1}^Nc\left(V\tildePhi{\omega}(x_i)^\adjoint \theta, y_i\right) + \frac{\lambda_K}{2}\norm{\theta}^2_{\tildeH{\omega}} \\ + \frac{\lambda_M}{2}\sum_{i,k=1}^{N+U}\inner{\tildePhi{\omega}(x_i)^\adjoint \theta, M_{ik}\tildePhi{\omega}(x_k)^\adjoint \theta}_{\mathcal{U}}.
\end{dmath*}
Finding a minimizer $\theta_{\seq{z}}$ over $\left(\Ker \tildeW{\omega}\right)^\perp$ is not the same than finding a minimizer over $\tildeH{\omega}$. Although in both cases Mazur-Schauder's theorem guarantee that the respective minimizers are unique, they might not be the same. Since $\tildeW{\omega}$ is bounded, $\Ker \tildeW{\omega}$ is closed, so that we can perform the decomposition $\tildeH{\omega}=\left(\Ker \tildeW{\omega}\right)^\perp\oplus \left(\Ker \tildeW{\omega}\right)$. Then clearly by linearity of $W$ and the fact that for all $\theta^{\parallel}\in\Ker \tildeW{\omega}$, $\tildeW{\omega}\theta^{\parallel}=0$, if $\lambda > 0$ we have
\begin{dmath*}
\theta_{\seq{z}}=\argmin_{\theta\in\tildeH{\omega}} \frac{1}{N}\sum_{i=1}^Nc\left(V\tildePhi{\omega}(x_i)^\adjoint \theta, y_i\right) + \frac{\lambda_K}{2}\norm{\theta}^2_{\tildeH{\omega}} \\ + \frac{\lambda_M}{2}\sum_{i,k=1}^{N+U}\inner*{\tildePhi{\omega}(x_i)^\adjoint \theta, M_{ik}\tildePhi{\omega}(x_k)^\adjoint \theta}_{\mathcal{U}}
\end{dmath*}
Thus
\begin{dmath*}
\theta_{\seq{z}}=\argmin_{\substack{\theta^{\perp}\in\left(\Ker \tildeW{\omega}\right)^\perp, \\ \theta^{\parallel}\in\Ker \tildeW{\omega}}} \frac{1}{N}\sum_{i=1}^Nc\left(V\left(\tildeW{\omega}\theta^{\perp}\right)(x)+\underbrace{V\left(\tildeW{\omega}\theta^{\parallel}\right)(x)}_{=0\enskip\text{for all}\enskip \theta^{\parallel} }, y_i\right) \\ + \frac{\lambda_K}{2}\norm{\theta^\perp}^2_{\tildeH{\omega}} + \underbrace{\frac{\lambda}{2}\norm{\theta^{\parallel}}^2_{\tildeH{\omega}}}_{=0 \enskip\text{only if}\enskip \theta^{\parallel}=0} \\ + \frac{\lambda_M}{2}\sum_{i,k=1}^{N+U}\inner*{\tildePhi{\omega}(x_i)^\adjoint \theta^{\perp}, M_{ik}\left(\tildeW{\omega}\theta^{\perp}\right)(x_k)}_{\mathcal{U}} \\
+ \frac{\lambda_M}{2}\sum_{i,k=1}^{N+U}\inner*{\underbrace{\left(\tildeW{\omega}\theta^{\parallel}\right)(x_i)}_{=0\enskip\text{for all}\enskip \theta^{\parallel} }, M_{ik}\left(\tildeW{\omega}\theta^{\perp}\right)(x_k)}_{\mathcal{U}}
\\ + \frac{\lambda_M}{2}\sum_{i,k=1}^{N+U}\inner*{\left(\tildeW{\omega}\theta^{\perp}\right)(x_i), M_{ik}\underbrace{\left(\tildeW{\omega}\theta^{\parallel}\right)(x_k)}_{=0\enskip\text{for all}\enskip \theta^{\parallel} }}_{\mathcal{U}} \\ + \frac{\lambda_M}{2}\sum_{i,k=1}^{N+U}\inner*{\underbrace{\left(\tildeW{\omega}\theta^{\parallel}\right)(x_i)}_{=0\enskip\text{for all}\enskip \theta^{\parallel} }, M_{ik}\underbrace{\left(\tildeW{\omega}\theta^{\parallel}\right)(x_k)}_{=0\enskip\text{for all}\enskip \theta^{\parallel} }}_{\mathcal{U}}.
\end{dmath*}
Thus
\begin{dmath*}
\theta_{\seq{z}}=\argmin_{\theta^{\perp}\in\left(\Ker \tildeW{\omega}\right)^\perp}
\frac{1}{N}\sum_{i=1}^Nc\left(V\left(\tildeW{\omega}\theta^{\perp}\right)(x), y_i\right) + \frac{\lambda_K}{2}\norm{\theta^\perp}^2_{\tildeH{\omega}} \\ + \frac{\lambda_M}{2}\sum_{i,k=1}^{N+U}\inner*{\tildePhi{\omega}(x_i)^\adjoint \theta^{\perp}, M_{ik}\left(\tildeW{\omega}\theta^{\perp}\right)(x_k)}_{\mathcal{U}}.
\end{dmath*}
Hence minimizing over $\left(\Ker \tildeW{\omega}\right)^\perp$ or $\tildeH{\omega}$ is the same when $\lambda_K > 0$. Eventually,
% Eventually for any outcome of $\omega_j \sim \probability_{\dual{\Haar},\rho}$ \iid,
\begin{dmath*}
\theta_{\seq{z}}=\argmin_{\theta\in\tildeH{\omega}} \frac{1}{N}\sum_{i=1}^Nc\left(V\tildePhi{\omega}(x_i)^\adjoint \theta, y_i\right) + \frac{\lambda_K}{2}\norm{\theta}^2_{\tildeH{\omega}} \\ + \frac{\lambda_M}{2}\sum_{i,k=1}^{N+U}\inner*{\tildePhi{\omega}(x_i)^\adjoint\theta, M_{ik}\tildePhi{\omega}(x_k)^\adjoint \theta}_{\mathcal{U}}
=\argmin_{\theta \in \tildeH{\omega}} \frac{1}{N}\sum_{i=1}^Nc\left(V\tildePhi{\omega}(x_i)^\adjoint\theta, y_i\right) + \frac{\lambda_K}{2}\norm{\theta}^2_{\tildeH{\omega}} \\ + \frac{\lambda_M}{2}\sum_{i,k=1}^{N+U}\inner*{\theta, \tildePhi{\omega}(x_i)M_{ik}\tildePhi{\omega}(x_k)^\adjoint \theta}_{\tildeH{\omega}}. \qed
\end{dmath*}
\end{proof}
This shows that when $\lambda_K>0$ the solution of \cref{eq:argmin_u} with the approximated kernel $K(x,z) \approx \tildeK{\omega}(x,z) = \tildePhi{\omega}(x)^\adjoint\tildePhi{\omega}(z)$ is the same than the solution of \cref{eq:argmin_theta} up to a transformation. Namely, if $u_{\seq{z}}$ is the solution of \cref{eq:argmin_u}, $\theta_{\seq{z}}$ is the solution of \cref{eq:argmin_theta} and $\lambda_K>0$ we have
\begin{dmath*}
\theta_{\seq{z}} = \sum_{j=1}^{N+U} \tildePhi{\omega}(x_j) u_{\seq{z}} \hiderel{\in}\tildeH{\omega}.
\end{dmath*}
If $\lambda_K=0$ we can still find a solution $u_{\seq{z}}$ of \cref{eq:argmin_u}. By construction of the kernel expansion, we have $u_{\seq{z}}\in(\Ker W)^\bot$. However looking at the proof of \cref{cr:orff_representer} we see that $\theta_{\seq{z}}$ might \emph{not} g belong to $(\Ker W)^\bot$. Then we can compute a residual vector
\begin{dmath*}
r_{\seq{z}} = \theta_{\seq{z}} - \sum_{j=1}^{N+U} \tildePhi{\omega}(x_j) u_{\seq{z}} \hiderel{\in}\tildeH{\omega}.
\end{dmath*}
Then if $r_{\seq{z}}=0$, it mkeans that $\lambda_K$ is large enough for both reprensenter theorem and \acs{ORFF} representer theorem to apply. If $r_{\seq{z}}\neq0$ but $\tildePhi{\omega}(\cdot)^\adjoint r_{\seq{z}} = 0$ means that both $\theta_{\seq{z}}$ and $\sum_{j=1}^{N+U} \tildePhi{\omega}(x_j) u_{\seq{z}}$ are in $(\Ker W)^\bot$, thus the representer theorem fails to find the \say{true} solution over the whole space $\mathcal{H}_K$ but returns a projection onto $\mathcal{H}_{\tildeK{\omega},\seq{z}}$ of the solution. If $r_{\seq{z}} \neq 0$ and $\tildePhi{\omega}(\cdot)^\adjoint r_{\seq{z}} \neq 0$ means that $\theta_{\seq{z}}$ is \emph{not} in $(\Ker W)^\bot$, thus the \acs{ORFF} representer theorem fails to apply. This remark is illustrated by \cref{fig:representer}.

\section{Solution of the empirical risk minimization}
% We illustrate the ORFF representer theorem (\cref{cr:orff_representer}) on two experiment involving scalar valued kernels.

\subsection{Gradient methods}
\label{subsec:gradient_methods}
In order to find a solution to \cref{eq:argmin_theta}, we turn our attention to gradient descent methods. In the following we let
\begin{dmath}
\label{eq:cost_functional}
J_{\lambda_K}(\theta) = \frac{1}{N}\sum_{i=1}^Nc\left(V\tildePhi{\omega}(x_i)^\adjoint \theta, y_i\right) + \frac{\lambda_K}{2}\norm{\theta}^2_{\tildeH{\omega}} \\ + \frac{\lambda_M}{2}\sum_{i,k=1}^{N+U}\inner{\theta, \tildePhi{\omega}(x_i)M_{ik}\tildePhi{\omega}(x_k)^\adjoint \theta}_{\tildeH{\omega}}.
\end{dmath}
Since the solution of \cref{eq:argmin_theta} is unique when $\lambda_K>0$, a sufficient and necessary condition is that the gradient of $J_{\lambda_K}$ at the minimizer $\theta_{\seq{z}}$ is zero. We use the Frechet derivative, the strongest notion of derivative in Banach spaces\mpar{Here we view the Hilbert space $\mathcal{H}$ (feature space) as a reflexive Banach space.} \cite{conway2013course, kurdila2006convex} wich directly generalizes the notion of gradient to Banach spaces. A function $f:\mathcal{H}_0\to\mathcal{H}_1$ is call Frechet differentiable at $\theta_0\in \mathcal{H}_0$ if there exist a bounded linear operator $A:\mathcal{H}_0\to \mathcal{H}_1$ such that
\begin{dmath*}
\lim_{h\to 0} \frac{\norm{f(\theta_0+h)-f(\theta_0)-Ah}_{\mathcal{H}_1}}{\norm{h}_{\mathcal{H}_0}}=0
\end{dmath*}
We write
\begin{dmath*}
(D_Ff)(\theta_0)\hiderel{=}\derivativeat{f(\theta)}{\theta}{\theta_0}\hiderel{=}A
\end{dmath*}
and call it Frechet derivative of $f$ with respect to $\theta$ at $\theta_0$. With mild abuse of notation we write $\derivative{f(\theta_0)}{\theta_0} = \derivativeat{f(\theta)}{\theta}{\theta_0}$. The Frechet derivative is an unbounded linear operator. The chain rule is valid in this context. Namely if a function $f:\mathcal{H}_0\to\mathcal{H}_1$ is Frechet differentiable at $\theta$ and $g:\mathcal{H}_1\to \mathcal{H}_2$ is Frechet differentiable at $f(\theta)$ then $g\circ f$ is Frechet differentiable at $\theta$ and
\begin{dmath*}
\lderivative{(g\circ f)(\theta)}{\theta} = \derivative{g(f(\theta))}{f(\theta)} \circ \derivative{f(\theta)}{\theta}.
\end{dmath*}
If $f:\mathcal{H}\to\mathbb{R}$ we define the gradient of $f$ as
\begin{dmath*}
\nabla_{\theta} f(\theta) = \left(\derivative{f(\theta)}{\theta}\right)^\adjoint
\end{dmath*}
and in the same way, for a function $f:\mathcal{H}_0\to\mathcal{H}_1$ the jacobian of $f$ as
\begin{dmath*}
\jacobian_{\theta} f(\theta) = \left(\derivative{f(\theta)}{\theta}\right)^\adjoint.
\end{dmath*}
In this context if $f:\mathcal{H}_0\to\mathcal{H}_1$ and $g:\mathcal{H}_1\to\mathbb{R}$ the chain rule reads
\begin{dmath*}
\nabla_{\theta} (g\circ f)(\theta) = \jacobian_{\theta}f(\theta) \circ \nabla_{f(\theta)}g(f(\theta)).
\end{dmath*}
We consider that $\mathcal{U}$ and $\mathcal{U}'$ are both isometrically isomorphic to some $\ell^2$ such that $\nabla_{\theta} \norm{\theta}^2 = 2\theta$. By linearity and applying the chaine rule to \cref{eq:argmin_theta} and since $M_{ik}^\adjoint = M_{ki}$ for all $i$, $k\in\mathbb{N}_{N+U}$, we have
\begin{dgroup*}
\begin{dmath*}
\nabla_{\theta}c\left(V\tildePhi{\omega}(x_i)^\adjoint \theta, y_i\right)= \tildePhi{\omega}(x_i)V^\adjoint \left(\lderivativeat{c\left(y, y_i\right)}{y}{V\tildePhi{\omega}(x_i)^\adjoint \theta}\right)^\adjoint,
\end{dmath*}
\begin{dmath*}
\nabla_{\theta}\inner*{\tildePhi{\omega}(x_i)^\adjoint \theta, M_{ik}\tildePhi{\omega}(x_k)^\adjoint \theta}_{\mathcal{U}}=\tildePhi{\omega}(x_i)\left(M_{ik}+M_{ki}\right)\tildePhi{\omega}(x_k)^\adjoint \theta,
\end{dmath*}
\begin{dmath*}
\nabla_{\theta}\norm{\theta}^2_{\tildeH{\omega}}=2\theta.
\end{dmath*}
\end{dgroup*}
Provided that $c(y,y_i)$ is Frechet differentiable \wrt~$y$, for all $y$ and $y_i\in\mathcal{Y}$ we have
\begin{dmath*}
\nabla_{\theta} J_{\lambda_K}(\theta) = \frac{1}{N}\sum_{i=1}^N \tildePhi{\omega}(x_i)V^\adjoint \left(\lderivativeat{c\left(y, y_i\right)}{y}{V\tildePhi{\omega}(x_i)^\adjoint \theta}\right)^\adjoint + \lambda_K\theta + \lambda_M\sum_{i,k=1}^{N+U}\tildePhi{\omega}(x_i)M_{ik}\tildePhi{\omega}(x_k)^\adjoint \theta
\end{dmath*}
Therefore after factorization, considering $\lambda_K > 0$,
\begin{dmath*}
\nabla_{\theta} J_{\lambda_K}(\theta) = \frac{1}{N}\sum_{i=1}^N \tildePhi{\omega}(x_i)V^\adjoint \left(\lderivativeat{c\left(y, y_i\right)}{y}{V\tildePhi{\omega}(x_i)^\adjoint \theta}\right)^\adjoint + \lambda_K\left(I_{\tildeH{\omega}} + \frac{\lambda_M}{\lambda_K}\sum_{i,k=1}^{N+U}\tildePhi{\omega}(x_i)M_{ik}\tildePhi{\omega}(x_k)^\adjoint \right)\theta
\end{dmath*}
We note the quantity $\mathbf{\widetilde{M}}_{\left(\lambda_K,\lambda_M\right)}=I_{\tildeH{\omega}} + \frac{\lambda_M}{\lambda_K}\sum_{i,k=1}^{N+U}\tildePhi{\omega}(x_i)M_{ik}\tildePhi{\omega}(x_k)^\adjoint \in \mathcal{L}(\Vect_{j=1}^D\mathcal{U}')$ so that
\begin{dmath*}
\nabla_{\theta} J_{\lambda_K}(\theta) = \frac{1}{N}\sum_{i=1}^N \tildePhi{\omega}(x_i)V^\adjoint \left(\lderivativeat{c\left(y, y_i\right)}{y}{V\tildePhi{\omega}(x_i)^\adjoint \theta}\right)^\adjoint + \lambda_K\mathbf{\widetilde{M}}_{\left(\lambda_K,\lambda_M\right)}\theta
\end{dmath*}
\begin{example}[Naive close form for the squared error cost]
Consider the cost function defined for all $y$, $y'\in\mathcal{Y}$ by $c(y,y')=\norm{y-y}_2^2$. Then
\begin{dmath*}
\left(\lderivativeat{c\left(y, y_i\right)}{y}{V\tildePhi{\omega}(x_i)^\adjoint \theta}\right)^\adjoint = 2\left(V\tildePhi{\omega}(x_i)^\adjoint \theta-y_i\right).
\end{dmath*}
Thus, since the optimal solution $\theta_{\seq{z}}$ verifies $\nabla_{\theta_{\seq{z}}} J_{\lambda_K}(\theta_{\seq{z}}) = 0$ we have
\begin{dmath*}
\frac{1}{N}\sum_{i=1}^N \tildePhi{\omega}(x_i)V^\adjoint\left(V\tildePhi{\omega}(x_i)^\adjoint \theta_{\seq{z}}-y_i\right) + \lambda_K\mathbf{\widetilde{M}}_{\left(\lambda_K,\lambda_M\right)}\theta_{\seq{z}}= 0.
\end{dmath*}
Therefore,
\begin{dmath*}
\left(\frac{1}{N}\sum_{i=1}^N \tildePhi{\omega}(x_i)V^\adjoint V\tildePhi{\omega}(x_i)^\adjoint + \lambda_K\mathbf{\widetilde{M}}_{\left(\lambda_K,\lambda_M\right)}\right)\theta_{\seq{z}} = \frac{1}{N}\sum_{i=1}^N \tildePhi{\omega}(x_i)V^\adjoint y_i.
\end{dmath*}
Suppose that $\mathcal{Y}\subseteq\mathbb{R}^p$, $V:\mathcal{U}\to\mathcal{Y}$ where $\mathcal{U}\subseteq\mathbb{R}^u$ and for all $x\in\mathcal{X}$, $\tildePhi{\omega}(x): \mathbb{R}^{r}\to\mathbb{R}^u$. From this we can derive \cref{alg:close_form} which return the close form solution of \cref{eq:cost_functional} for $c(y,y')=\norm{y-y'}_2^2$.
\end{example}
\paragraph{Complexity analysis:}
\Cref{alg:close_form} constitute our first step toward large-scale learning with operator-valued kernels. We can easily compute the time complexity of \cref{alg:close_form} when all the operators act on finite dimensional Hilbert spaces. Suppose that $u=\dim(\mathcal{U})<+\infty$ and $u'=\dim(\mathcal{U}')<+\infty$ and for all $x\in\mathcal{X}$, $\tildePhi{\omega}(x):\mathcal{U}'\to\tildeH{\omega}$ where $r=\dim(\tildeH{\omega})<+\infty$ is the dimension of the redescription space $\tildeH{\omega}=\mathbb{R}^{r}$. Since $u$, $u'$, and $r<+\infty$, we view the operators $\tildePhi{\omega}(x)$, $V$ and $\mathbf{\widetilde{M}}_{\left(\lambda_K,\lambda_M\right)}$ as matrices. Computing $V^\adjoint V$ cost $O_t(u^2p)$. Step 1 costs $O_t(r^2u + ru^2)$. Steps 5 (optional) has the same cost except that the sum is done over all pair of $N+U$ points thus it costs $O_t((N+U)^2(r^2u + r u^2))$. Steps 7 costs $O_t(N(ru + up))$. For step 8, the naive inversion of the operator costs $O_t(r^3)$. Eventually the overall complexity of \cref{alg:close_form} is
\begin{dmath*}
O_t\left(ru(r + u) \begin{cases} (N+U)^2 & \text{if $\lambda_M > 0$} \\ N & \text{if $\lambda_M = 0$} \end{cases}+ r^3 + Nu(r+p)
\right),
\end{dmath*}
while the space complexity is $O_s(r^2)$.
\afterpage{%
\begin{center}
\begin{algorithm2e}[H]
    \label{alg:close_form}
    \SetAlgoLined
    \Input{\begin{itemize}
    \item $\seq{s}=(x_i, y_i)_{i=1}^N\in\left(\mathcal{X}\times\mathbb{R}^p\right)^N$ a sequence of supervised training points,
    \item $\seq{u}=(x_i)_{i=N+1}^{N+U}\in\mathcal{X}^{U}$ a sequence of unsupervised training points,
    \item $\tildePhi{\omega}(x_i) \in \left(\mathbb{R}^u, \mathbb{R}^{r}\right)$ a feature map defined for all $x_i\in\mathcal{X}$,
    \item $(M_{ik})_{i,k=1}^{N+U}$ a sequence of data dependent operators (see \cref{cr:orff_representer}),
    \item $V \in \mathcal{L}\left(\mathbb{R}^u, \to \mathbb{R}^p\right)$ a combination operator,
    \item $\lambda_K \in\mathbb{R}_{>0}$ the Tychonov regularization term,
    \item $\lambda_M \in\mathbb{R}_+$ the manifold regularization term.
    \end{itemize}}
    \Output{A model
    \begin{dmath*}
    h:\begin{cases} \mathcal{X} \to \mathbb{R}^p \\ x\mapsto\tildePhi{\omega}(x)^T\theta_{\seq{z}},\end{cases}
    \end{dmath*}
    such that $\theta_{\seq{z}}$ minimize \cref{eq:cost_functional}, where $c(y,y')=\norm{y-y'}_2^2$.}
    $\mathbf{P} \gets \frac{1}{N}\sum_{i=1}^N \tildePhi{\omega}(x_i)V^T V\tildePhi{\omega}(x_i)^T \in\mathcal{L}(\mathbb{R}^{r}, \mathbb{R}^{r})  $\;
    \uIf{$\lambda_M = 0$}{
        $\mathbf{\widetilde{M}}_{\left(\lambda_K,\lambda_M\right)} \gets I_{r} \in\mathcal{L}(\mathbb{R}^{r}, \mathbb{R}^{r})$\;
    }
    \Else{
        $\mathbf{\widetilde{M}}_{\left(\lambda_K,\lambda_M\right)} \gets \left(I_{r} + \frac{\lambda_M}{\lambda_K}\sum_{i,k=1}^{N+U}\tildePhi{\omega}(x_i)M_{ik}\tildePhi{\omega}(x_k)^T \right) \in\mathcal{L}(\mathbb{R}^{r}, \mathbb{R}^{r}) $\;
    }
    $\mathbf{Y} \gets \frac{1}{N}\sum_{i=1}^N \tildePhi{\omega}(x_i)V^T y_i \in \mathbb{R}^{r} $\;
    $\theta_{\seq{z}} \gets \left(P + \lambda_K\mathbf{\widetilde{M}}_{\left(\lambda_K,\lambda_M\right)}\right)^{-1}Y \in \mathbb{R}^{r} $\;
    \Return $h: x \mapsto \tildePhi{\omega}(x)^T\theta_{\seq{z}}$\;
    \caption{Naive close form for the squared error cost.}
\end{algorithm2e}
\end{center}}
This complexity is to compare with the kernelized solution proposed by \citet{minh2016unifying}. Let
\begin{dmath*}
\mathbf{K}:\begin{cases}
\mathcal{U}^{N+U} \to \mathcal{U}^{N+U} \\
u\mapsto\Vect_{i=1}^{N+U}\sum_{j=1}^{N+U}K(x_i, x_j)u_j
\end{cases}
\end{dmath*}
and
\begin{dmath*}
\mathbf{M}:\begin{cases}
\mathcal{U}^{N+U} \to \mathcal{U}^{N+U} \\
u\mapsto\Vect_{i=1}^{N+U}\sum_{k=1}^{N+U}M_{ik}u_k.
\end{cases}
\end{dmath*}
When $\mathcal{U}=\mathbb{R}$,
\begin{dmath*}
    \mathbf{K}=\begin{pmatrix} K(x_1, x_1) & \hdots & K(x_1, x_{N+U})
    \\ \vdots & \ddots & \vdots \\  K(x_{N+U}, x_1) & \hdots & K(x_{N+U}, x_{N+U}) \end{pmatrix}
\end{dmath*}
is called the Gram matrix of $K$. When $\mathcal{U}=\mathbb{R}^p$, $\mathbf{K}$ is a matrix-valued Gram matrix of size $u(N+U)\times u(N+U)$ where each entry $\mathbf{K}_{ij}\in\mathcal{L}(\mathbb{R}^u)$. When $\mathcal{U}=\mathbb{R}^u$, $\mathbf{M}$ can also be seen as a matrix-valued matrix where each entry is $M_{ik}\in\mathcal{L}(\mathbb{R}^u)$. We also introduce the operators $\mathbf{C}^T \mathbf{C}\colonequals I_{N+U} \otimes (C^T C)$ and
\begin{dmath*}
\mathbf{P}:\begin{cases}
\mathcal{U}^{N+U} \to \mathcal{U}^{N+U} \\
u\mapsto \left(\Vect_{j=1}^Nu_j\right) \oplus \left(\Vect_{j=N+1}^{N+U}0\right)
\end{cases}
\end{dmath*}
The operator $\mathbf{P}$ is a projection that sets all the terms $u_j$, $N < j \le N + U$ of $u$ to zero. When $\mathcal{U}=\mathbb{R}^u$ it can also be seen as the block matrix of size $u(N+U) \times u(N + U)$ and
\begin{dmath*}
    \mathbf{P}=\begin{pmatrix}  & & & 0 & \hdots & 0 \\ & I_u \otimes I_{N} & & \vdots & \ddots & \vdots \\ & & & 0 & \hdots & 0 \\
    0 & \hdots & 0 & 0 & \hdots & 0 \\
    \vdots & \ddots & \vdots & \vdots & \ddots & \vdots \\
    0 & \hdots & 0 & 0 & \hdots & 0
    \end{pmatrix}
\end{dmath*}
Then the equivalent kernelized solution $u_{\seq{z}}$ of \cref{th:representer} is given by \cite{minh2016unifying}
\begin{dmath*}
\left(\frac{1}{N}\mathbf{C}^T \mathbf{C} \mathbf{P} \mathbf{K} + \lambda_M \mathbf{M} \mathbf{K} + \lambda_K I_{\Vect_{i=1}^{N+U}\mathcal{U}}\right)u_{\seq{z}}=\left(\Vect_{i=1}^N C^T y_i\right) \oplus \left(\Vect_{i=N+1}^{N+U} 0 \right).
\end{dmath*}
which has time complexity $O_t(((N+U)u)^3+ Nup)$ and space complexity $O_s(((N+U)u)^2)$. Hence \cref{alg:close_form} is better that its kernelized counterpart when $r=2Du'$ is small compare to $(N+U)u$. Roughly speaking it is better to use \cref{alg:close_form} when the number of features, $r$, required is small compared to the number of training point. Notice that computing the data dependent norm (manifold regularization) is expensive. Indeed when $\lambda_M=0$, \cref{alg:close_form} has a linear complexity with respect to the number of supervised training points $N$ while the complexity becomes quatratic in the number of supervised and unsupervised training points $N+U$ when $\lambda_M>0$. Moreover suppose that $\lambda_M=0$, $\mathcal{U}=\mathbb{R}^p$ and $\mathcal{U}'=\mathbb{R}^{p}$ and the combination operator is $V=I_{p}$. Then the complexity of \cref{alg:close_form} boils down to
\begin{dmath*}
O_t(p^3(ND^2+D^3)),
\end{dmath*}
which is annoying. Indeed learning $p$ independent models with scalar Random Fourier Features would cost $O_t(p(ND^2+D^3))$, meaning that learning vector-valued function has increase the (expected) complexity from $p$ to $p^3$. However in some cases we can drastically reduce the complexity by viewing the feature-maps as linear operators rather than matrices.

\begin{pycode}[representer]
sys.path.append('../src/')
import representer

err = representer.main()
\end{pycode}

\afterpage{%
\begin{landscape}
\begin{figure}[tb]
\pyc{print(r'\centering\resizebox{\textwidth}{!}{%% Creator: Matplotlib, PGF backend
%%
%% To include the figure in your LaTeX document, write
%%   \input{<filename>.pgf}
%%
%% Make sure the required packages are loaded in your preamble
%%   \usepackage{pgf}
%%
%% Figures using additional raster images can only be included by \input if
%% they are in the same directory as the main LaTeX file. For loading figures
%% from other directories you can use the `import` package
%%   \usepackage{import}
%% and then include the figures with
%%   \import{<path to file>}{<filename>.pgf}
%%
%% Matplotlib used the following preamble
%%   \usepackage{fontspec}
%%   \setmainfont{Times New Roman}
%%   \setsansfont{Arial}
%%   \setmonofont{Andale Mono}
%%
\begingroup%
\makeatletter%
\begin{pgfpicture}%
\pgfpathrectangle{\pgfpointorigin}{\pgfqpoint{16.000000in}{6.000000in}}%
\pgfusepath{use as bounding box, clip}%
\begin{pgfscope}%
\pgfsetbuttcap%
\pgfsetmiterjoin%
\definecolor{currentfill}{rgb}{1.000000,1.000000,1.000000}%
\pgfsetfillcolor{currentfill}%
\pgfsetlinewidth{0.000000pt}%
\definecolor{currentstroke}{rgb}{1.000000,1.000000,1.000000}%
\pgfsetstrokecolor{currentstroke}%
\pgfsetdash{}{0pt}%
\pgfpathmoveto{\pgfqpoint{0.000000in}{0.000000in}}%
\pgfpathlineto{\pgfqpoint{16.000000in}{0.000000in}}%
\pgfpathlineto{\pgfqpoint{16.000000in}{6.000000in}}%
\pgfpathlineto{\pgfqpoint{0.000000in}{6.000000in}}%
\pgfpathclose%
\pgfusepath{fill}%
\end{pgfscope}%
\begin{pgfscope}%
\pgfsetbuttcap%
\pgfsetmiterjoin%
\definecolor{currentfill}{rgb}{1.000000,1.000000,1.000000}%
\pgfsetfillcolor{currentfill}%
\pgfsetlinewidth{0.000000pt}%
\definecolor{currentstroke}{rgb}{0.000000,0.000000,0.000000}%
\pgfsetstrokecolor{currentstroke}%
\pgfsetstrokeopacity{0.000000}%
\pgfsetdash{}{0pt}%
\pgfpathmoveto{\pgfqpoint{2.000000in}{4.421053in}}%
\pgfpathlineto{\pgfqpoint{6.376471in}{4.421053in}}%
\pgfpathlineto{\pgfqpoint{6.376471in}{5.400000in}}%
\pgfpathlineto{\pgfqpoint{2.000000in}{5.400000in}}%
\pgfpathclose%
\pgfusepath{fill}%
\end{pgfscope}%
\begin{pgfscope}%
\pgfpathrectangle{\pgfqpoint{2.000000in}{4.421053in}}{\pgfqpoint{4.376471in}{0.978947in}} %
\pgfusepath{clip}%
\pgfsetroundcap%
\pgfsetroundjoin%
\pgfsetlinewidth{1.003750pt}%
\definecolor{currentstroke}{rgb}{0.800000,0.800000,0.800000}%
\pgfsetstrokecolor{currentstroke}%
\pgfsetdash{}{0pt}%
\pgfpathmoveto{\pgfqpoint{2.000000in}{4.421053in}}%
\pgfpathlineto{\pgfqpoint{2.000000in}{5.400000in}}%
\pgfusepath{stroke}%
\end{pgfscope}%
\begin{pgfscope}%
\pgfpathrectangle{\pgfqpoint{2.000000in}{4.421053in}}{\pgfqpoint{4.376471in}{0.978947in}} %
\pgfusepath{clip}%
\pgfsetroundcap%
\pgfsetroundjoin%
\pgfsetlinewidth{1.003750pt}%
\definecolor{currentstroke}{rgb}{0.800000,0.800000,0.800000}%
\pgfsetstrokecolor{currentstroke}%
\pgfsetdash{}{0pt}%
\pgfpathmoveto{\pgfqpoint{2.875294in}{4.421053in}}%
\pgfpathlineto{\pgfqpoint{2.875294in}{5.400000in}}%
\pgfusepath{stroke}%
\end{pgfscope}%
\begin{pgfscope}%
\pgfpathrectangle{\pgfqpoint{2.000000in}{4.421053in}}{\pgfqpoint{4.376471in}{0.978947in}} %
\pgfusepath{clip}%
\pgfsetroundcap%
\pgfsetroundjoin%
\pgfsetlinewidth{1.003750pt}%
\definecolor{currentstroke}{rgb}{0.800000,0.800000,0.800000}%
\pgfsetstrokecolor{currentstroke}%
\pgfsetdash{}{0pt}%
\pgfpathmoveto{\pgfqpoint{3.750588in}{4.421053in}}%
\pgfpathlineto{\pgfqpoint{3.750588in}{5.400000in}}%
\pgfusepath{stroke}%
\end{pgfscope}%
\begin{pgfscope}%
\pgfpathrectangle{\pgfqpoint{2.000000in}{4.421053in}}{\pgfqpoint{4.376471in}{0.978947in}} %
\pgfusepath{clip}%
\pgfsetroundcap%
\pgfsetroundjoin%
\pgfsetlinewidth{1.003750pt}%
\definecolor{currentstroke}{rgb}{0.800000,0.800000,0.800000}%
\pgfsetstrokecolor{currentstroke}%
\pgfsetdash{}{0pt}%
\pgfpathmoveto{\pgfqpoint{4.625882in}{4.421053in}}%
\pgfpathlineto{\pgfqpoint{4.625882in}{5.400000in}}%
\pgfusepath{stroke}%
\end{pgfscope}%
\begin{pgfscope}%
\pgfpathrectangle{\pgfqpoint{2.000000in}{4.421053in}}{\pgfqpoint{4.376471in}{0.978947in}} %
\pgfusepath{clip}%
\pgfsetroundcap%
\pgfsetroundjoin%
\pgfsetlinewidth{1.003750pt}%
\definecolor{currentstroke}{rgb}{0.800000,0.800000,0.800000}%
\pgfsetstrokecolor{currentstroke}%
\pgfsetdash{}{0pt}%
\pgfpathmoveto{\pgfqpoint{5.501176in}{4.421053in}}%
\pgfpathlineto{\pgfqpoint{5.501176in}{5.400000in}}%
\pgfusepath{stroke}%
\end{pgfscope}%
\begin{pgfscope}%
\pgfpathrectangle{\pgfqpoint{2.000000in}{4.421053in}}{\pgfqpoint{4.376471in}{0.978947in}} %
\pgfusepath{clip}%
\pgfsetroundcap%
\pgfsetroundjoin%
\pgfsetlinewidth{1.003750pt}%
\definecolor{currentstroke}{rgb}{0.800000,0.800000,0.800000}%
\pgfsetstrokecolor{currentstroke}%
\pgfsetdash{}{0pt}%
\pgfpathmoveto{\pgfqpoint{6.376471in}{4.421053in}}%
\pgfpathlineto{\pgfqpoint{6.376471in}{5.400000in}}%
\pgfusepath{stroke}%
\end{pgfscope}%
\begin{pgfscope}%
\pgfpathrectangle{\pgfqpoint{2.000000in}{4.421053in}}{\pgfqpoint{4.376471in}{0.978947in}} %
\pgfusepath{clip}%
\pgfsetroundcap%
\pgfsetroundjoin%
\pgfsetlinewidth{1.003750pt}%
\definecolor{currentstroke}{rgb}{0.800000,0.800000,0.800000}%
\pgfsetstrokecolor{currentstroke}%
\pgfsetdash{}{0pt}%
\pgfpathmoveto{\pgfqpoint{2.000000in}{4.584211in}}%
\pgfpathlineto{\pgfqpoint{6.376471in}{4.584211in}}%
\pgfusepath{stroke}%
\end{pgfscope}%
\begin{pgfscope}%
\definecolor{textcolor}{rgb}{0.150000,0.150000,0.150000}%
\pgfsetstrokecolor{textcolor}%
\pgfsetfillcolor{textcolor}%
\pgftext[x=1.902778in,y=4.584211in,right,]{\color{textcolor}\sffamily\fontsize{10.000000}{12.000000}\selectfont \(\displaystyle -1\)}%
\end{pgfscope}%
\begin{pgfscope}%
\pgfpathrectangle{\pgfqpoint{2.000000in}{4.421053in}}{\pgfqpoint{4.376471in}{0.978947in}} %
\pgfusepath{clip}%
\pgfsetroundcap%
\pgfsetroundjoin%
\pgfsetlinewidth{1.003750pt}%
\definecolor{currentstroke}{rgb}{0.800000,0.800000,0.800000}%
\pgfsetstrokecolor{currentstroke}%
\pgfsetdash{}{0pt}%
\pgfpathmoveto{\pgfqpoint{2.000000in}{4.788158in}}%
\pgfpathlineto{\pgfqpoint{6.376471in}{4.788158in}}%
\pgfusepath{stroke}%
\end{pgfscope}%
\begin{pgfscope}%
\definecolor{textcolor}{rgb}{0.150000,0.150000,0.150000}%
\pgfsetstrokecolor{textcolor}%
\pgfsetfillcolor{textcolor}%
\pgftext[x=1.902778in,y=4.788158in,right,]{\color{textcolor}\sffamily\fontsize{10.000000}{12.000000}\selectfont \(\displaystyle 0\)}%
\end{pgfscope}%
\begin{pgfscope}%
\pgfpathrectangle{\pgfqpoint{2.000000in}{4.421053in}}{\pgfqpoint{4.376471in}{0.978947in}} %
\pgfusepath{clip}%
\pgfsetroundcap%
\pgfsetroundjoin%
\pgfsetlinewidth{1.003750pt}%
\definecolor{currentstroke}{rgb}{0.800000,0.800000,0.800000}%
\pgfsetstrokecolor{currentstroke}%
\pgfsetdash{}{0pt}%
\pgfpathmoveto{\pgfqpoint{2.000000in}{4.992105in}}%
\pgfpathlineto{\pgfqpoint{6.376471in}{4.992105in}}%
\pgfusepath{stroke}%
\end{pgfscope}%
\begin{pgfscope}%
\definecolor{textcolor}{rgb}{0.150000,0.150000,0.150000}%
\pgfsetstrokecolor{textcolor}%
\pgfsetfillcolor{textcolor}%
\pgftext[x=1.902778in,y=4.992105in,right,]{\color{textcolor}\sffamily\fontsize{10.000000}{12.000000}\selectfont \(\displaystyle 1\)}%
\end{pgfscope}%
\begin{pgfscope}%
\pgfpathrectangle{\pgfqpoint{2.000000in}{4.421053in}}{\pgfqpoint{4.376471in}{0.978947in}} %
\pgfusepath{clip}%
\pgfsetroundcap%
\pgfsetroundjoin%
\pgfsetlinewidth{1.003750pt}%
\definecolor{currentstroke}{rgb}{0.800000,0.800000,0.800000}%
\pgfsetstrokecolor{currentstroke}%
\pgfsetdash{}{0pt}%
\pgfpathmoveto{\pgfqpoint{2.000000in}{5.196053in}}%
\pgfpathlineto{\pgfqpoint{6.376471in}{5.196053in}}%
\pgfusepath{stroke}%
\end{pgfscope}%
\begin{pgfscope}%
\definecolor{textcolor}{rgb}{0.150000,0.150000,0.150000}%
\pgfsetstrokecolor{textcolor}%
\pgfsetfillcolor{textcolor}%
\pgftext[x=1.902778in,y=5.196053in,right,]{\color{textcolor}\sffamily\fontsize{10.000000}{12.000000}\selectfont \(\displaystyle 2\)}%
\end{pgfscope}%
\begin{pgfscope}%
\pgfpathrectangle{\pgfqpoint{2.000000in}{4.421053in}}{\pgfqpoint{4.376471in}{0.978947in}} %
\pgfusepath{clip}%
\pgfsetroundcap%
\pgfsetroundjoin%
\pgfsetlinewidth{1.003750pt}%
\definecolor{currentstroke}{rgb}{0.800000,0.800000,0.800000}%
\pgfsetstrokecolor{currentstroke}%
\pgfsetdash{}{0pt}%
\pgfpathmoveto{\pgfqpoint{2.000000in}{5.400000in}}%
\pgfpathlineto{\pgfqpoint{6.376471in}{5.400000in}}%
\pgfusepath{stroke}%
\end{pgfscope}%
\begin{pgfscope}%
\definecolor{textcolor}{rgb}{0.150000,0.150000,0.150000}%
\pgfsetstrokecolor{textcolor}%
\pgfsetfillcolor{textcolor}%
\pgftext[x=1.902778in,y=5.400000in,right,]{\color{textcolor}\sffamily\fontsize{10.000000}{12.000000}\selectfont \(\displaystyle 3\)}%
\end{pgfscope}%
\begin{pgfscope}%
\definecolor{textcolor}{rgb}{0.150000,0.150000,0.150000}%
\pgfsetstrokecolor{textcolor}%
\pgfsetfillcolor{textcolor}%
\pgftext[x=1.655864in,y=4.910526in,,bottom,rotate=90.000000]{\color{textcolor}\sffamily\fontsize{11.000000}{13.200000}\selectfont y}%
\end{pgfscope}%
\begin{pgfscope}%
\pgfpathrectangle{\pgfqpoint{2.000000in}{4.421053in}}{\pgfqpoint{4.376471in}{0.978947in}} %
\pgfusepath{clip}%
\pgfsetbuttcap%
\pgfsetroundjoin%
\definecolor{currentfill}{rgb}{1.000000,0.000000,0.000000}%
\pgfsetfillcolor{currentfill}%
\pgfsetlinewidth{2.007500pt}%
\definecolor{currentstroke}{rgb}{1.000000,0.000000,0.000000}%
\pgfsetstrokecolor{currentstroke}%
\pgfsetdash{}{0pt}%
\pgfpathmoveto{\pgfqpoint{4.765731in}{4.978628in}}%
\pgfpathlineto{\pgfqpoint{4.827844in}{4.978628in}}%
\pgfpathmoveto{\pgfqpoint{4.796787in}{4.947572in}}%
\pgfpathlineto{\pgfqpoint{4.796787in}{5.009685in}}%
\pgfusepath{stroke,fill}%
\end{pgfscope}%
\begin{pgfscope}%
\pgfpathrectangle{\pgfqpoint{2.000000in}{4.421053in}}{\pgfqpoint{4.376471in}{0.978947in}} %
\pgfusepath{clip}%
\pgfsetbuttcap%
\pgfsetroundjoin%
\definecolor{currentfill}{rgb}{1.000000,0.000000,0.000000}%
\pgfsetfillcolor{currentfill}%
\pgfsetlinewidth{2.007500pt}%
\definecolor{currentstroke}{rgb}{1.000000,0.000000,0.000000}%
\pgfsetstrokecolor{currentstroke}%
\pgfsetdash{}{0pt}%
\pgfpathmoveto{\pgfqpoint{5.348242in}{4.661416in}}%
\pgfpathlineto{\pgfqpoint{5.410355in}{4.661416in}}%
\pgfpathmoveto{\pgfqpoint{5.379298in}{4.630360in}}%
\pgfpathlineto{\pgfqpoint{5.379298in}{4.692473in}}%
\pgfusepath{stroke,fill}%
\end{pgfscope}%
\begin{pgfscope}%
\pgfpathrectangle{\pgfqpoint{2.000000in}{4.421053in}}{\pgfqpoint{4.376471in}{0.978947in}} %
\pgfusepath{clip}%
\pgfsetbuttcap%
\pgfsetroundjoin%
\definecolor{currentfill}{rgb}{1.000000,0.000000,0.000000}%
\pgfsetfillcolor{currentfill}%
\pgfsetlinewidth{2.007500pt}%
\definecolor{currentstroke}{rgb}{1.000000,0.000000,0.000000}%
\pgfsetstrokecolor{currentstroke}%
\pgfsetdash{}{0pt}%
\pgfpathmoveto{\pgfqpoint{4.954619in}{5.032156in}}%
\pgfpathlineto{\pgfqpoint{5.016732in}{5.032156in}}%
\pgfpathmoveto{\pgfqpoint{4.985675in}{5.001100in}}%
\pgfpathlineto{\pgfqpoint{4.985675in}{5.063213in}}%
\pgfusepath{stroke,fill}%
\end{pgfscope}%
\begin{pgfscope}%
\pgfpathrectangle{\pgfqpoint{2.000000in}{4.421053in}}{\pgfqpoint{4.376471in}{0.978947in}} %
\pgfusepath{clip}%
\pgfsetbuttcap%
\pgfsetroundjoin%
\definecolor{currentfill}{rgb}{1.000000,0.000000,0.000000}%
\pgfsetfillcolor{currentfill}%
\pgfsetlinewidth{2.007500pt}%
\definecolor{currentstroke}{rgb}{1.000000,0.000000,0.000000}%
\pgfsetstrokecolor{currentstroke}%
\pgfsetdash{}{0pt}%
\pgfpathmoveto{\pgfqpoint{4.751970in}{4.921244in}}%
\pgfpathlineto{\pgfqpoint{4.814083in}{4.921244in}}%
\pgfpathmoveto{\pgfqpoint{4.783026in}{4.890188in}}%
\pgfpathlineto{\pgfqpoint{4.783026in}{4.952301in}}%
\pgfusepath{stroke,fill}%
\end{pgfscope}%
\begin{pgfscope}%
\pgfpathrectangle{\pgfqpoint{2.000000in}{4.421053in}}{\pgfqpoint{4.376471in}{0.978947in}} %
\pgfusepath{clip}%
\pgfsetbuttcap%
\pgfsetroundjoin%
\definecolor{currentfill}{rgb}{1.000000,0.000000,0.000000}%
\pgfsetfillcolor{currentfill}%
\pgfsetlinewidth{2.007500pt}%
\definecolor{currentstroke}{rgb}{1.000000,0.000000,0.000000}%
\pgfsetstrokecolor{currentstroke}%
\pgfsetdash{}{0pt}%
\pgfpathmoveto{\pgfqpoint{4.327528in}{4.594850in}}%
\pgfpathlineto{\pgfqpoint{4.389641in}{4.594850in}}%
\pgfpathmoveto{\pgfqpoint{4.358584in}{4.563793in}}%
\pgfpathlineto{\pgfqpoint{4.358584in}{4.625906in}}%
\pgfusepath{stroke,fill}%
\end{pgfscope}%
\begin{pgfscope}%
\pgfpathrectangle{\pgfqpoint{2.000000in}{4.421053in}}{\pgfqpoint{4.376471in}{0.978947in}} %
\pgfusepath{clip}%
\pgfsetbuttcap%
\pgfsetroundjoin%
\definecolor{currentfill}{rgb}{1.000000,0.000000,0.000000}%
\pgfsetfillcolor{currentfill}%
\pgfsetlinewidth{2.007500pt}%
\definecolor{currentstroke}{rgb}{1.000000,0.000000,0.000000}%
\pgfsetstrokecolor{currentstroke}%
\pgfsetdash{}{0pt}%
\pgfpathmoveto{\pgfqpoint{5.105627in}{4.864314in}}%
\pgfpathlineto{\pgfqpoint{5.167740in}{4.864314in}}%
\pgfpathmoveto{\pgfqpoint{5.136683in}{4.833258in}}%
\pgfpathlineto{\pgfqpoint{5.136683in}{4.895371in}}%
\pgfusepath{stroke,fill}%
\end{pgfscope}%
\begin{pgfscope}%
\pgfpathrectangle{\pgfqpoint{2.000000in}{4.421053in}}{\pgfqpoint{4.376471in}{0.978947in}} %
\pgfusepath{clip}%
\pgfsetbuttcap%
\pgfsetroundjoin%
\definecolor{currentfill}{rgb}{1.000000,0.000000,0.000000}%
\pgfsetfillcolor{currentfill}%
\pgfsetlinewidth{2.007500pt}%
\definecolor{currentstroke}{rgb}{1.000000,0.000000,0.000000}%
\pgfsetstrokecolor{currentstroke}%
\pgfsetdash{}{0pt}%
\pgfpathmoveto{\pgfqpoint{4.376308in}{4.632344in}}%
\pgfpathlineto{\pgfqpoint{4.438421in}{4.632344in}}%
\pgfpathmoveto{\pgfqpoint{4.407364in}{4.601287in}}%
\pgfpathlineto{\pgfqpoint{4.407364in}{4.663400in}}%
\pgfusepath{stroke,fill}%
\end{pgfscope}%
\begin{pgfscope}%
\pgfpathrectangle{\pgfqpoint{2.000000in}{4.421053in}}{\pgfqpoint{4.376471in}{0.978947in}} %
\pgfusepath{clip}%
\pgfsetbuttcap%
\pgfsetroundjoin%
\definecolor{currentfill}{rgb}{1.000000,0.000000,0.000000}%
\pgfsetfillcolor{currentfill}%
\pgfsetlinewidth{2.007500pt}%
\definecolor{currentstroke}{rgb}{1.000000,0.000000,0.000000}%
\pgfsetstrokecolor{currentstroke}%
\pgfsetdash{}{0pt}%
\pgfpathmoveto{\pgfqpoint{5.966492in}{5.272462in}}%
\pgfpathlineto{\pgfqpoint{6.028605in}{5.272462in}}%
\pgfpathmoveto{\pgfqpoint{5.997549in}{5.241406in}}%
\pgfpathlineto{\pgfqpoint{5.997549in}{5.303519in}}%
\pgfusepath{stroke,fill}%
\end{pgfscope}%
\begin{pgfscope}%
\pgfpathrectangle{\pgfqpoint{2.000000in}{4.421053in}}{\pgfqpoint{4.376471in}{0.978947in}} %
\pgfusepath{clip}%
\pgfsetbuttcap%
\pgfsetroundjoin%
\definecolor{currentfill}{rgb}{1.000000,0.000000,0.000000}%
\pgfsetfillcolor{currentfill}%
\pgfsetlinewidth{2.007500pt}%
\definecolor{currentstroke}{rgb}{1.000000,0.000000,0.000000}%
\pgfsetstrokecolor{currentstroke}%
\pgfsetdash{}{0pt}%
\pgfpathmoveto{\pgfqpoint{6.218191in}{5.172865in}}%
\pgfpathlineto{\pgfqpoint{6.280304in}{5.172865in}}%
\pgfpathmoveto{\pgfqpoint{6.249248in}{5.141808in}}%
\pgfpathlineto{\pgfqpoint{6.249248in}{5.203921in}}%
\pgfusepath{stroke,fill}%
\end{pgfscope}%
\begin{pgfscope}%
\pgfpathrectangle{\pgfqpoint{2.000000in}{4.421053in}}{\pgfqpoint{4.376471in}{0.978947in}} %
\pgfusepath{clip}%
\pgfsetbuttcap%
\pgfsetroundjoin%
\definecolor{currentfill}{rgb}{1.000000,0.000000,0.000000}%
\pgfsetfillcolor{currentfill}%
\pgfsetlinewidth{2.007500pt}%
\definecolor{currentstroke}{rgb}{1.000000,0.000000,0.000000}%
\pgfsetstrokecolor{currentstroke}%
\pgfsetdash{}{0pt}%
\pgfpathmoveto{\pgfqpoint{4.186734in}{4.670261in}}%
\pgfpathlineto{\pgfqpoint{4.248847in}{4.670261in}}%
\pgfpathmoveto{\pgfqpoint{4.217791in}{4.639204in}}%
\pgfpathlineto{\pgfqpoint{4.217791in}{4.701317in}}%
\pgfusepath{stroke,fill}%
\end{pgfscope}%
\begin{pgfscope}%
\pgfpathrectangle{\pgfqpoint{2.000000in}{4.421053in}}{\pgfqpoint{4.376471in}{0.978947in}} %
\pgfusepath{clip}%
\pgfsetbuttcap%
\pgfsetroundjoin%
\definecolor{currentfill}{rgb}{1.000000,0.000000,0.000000}%
\pgfsetfillcolor{currentfill}%
\pgfsetlinewidth{2.007500pt}%
\definecolor{currentstroke}{rgb}{1.000000,0.000000,0.000000}%
\pgfsetstrokecolor{currentstroke}%
\pgfsetdash{}{0pt}%
\pgfpathmoveto{\pgfqpoint{5.616207in}{4.820436in}}%
\pgfpathlineto{\pgfqpoint{5.678320in}{4.820436in}}%
\pgfpathmoveto{\pgfqpoint{5.647263in}{4.789380in}}%
\pgfpathlineto{\pgfqpoint{5.647263in}{4.851493in}}%
\pgfusepath{stroke,fill}%
\end{pgfscope}%
\begin{pgfscope}%
\pgfpathrectangle{\pgfqpoint{2.000000in}{4.421053in}}{\pgfqpoint{4.376471in}{0.978947in}} %
\pgfusepath{clip}%
\pgfsetbuttcap%
\pgfsetroundjoin%
\definecolor{currentfill}{rgb}{1.000000,0.000000,0.000000}%
\pgfsetfillcolor{currentfill}%
\pgfsetlinewidth{2.007500pt}%
\definecolor{currentstroke}{rgb}{1.000000,0.000000,0.000000}%
\pgfsetstrokecolor{currentstroke}%
\pgfsetdash{}{0pt}%
\pgfpathmoveto{\pgfqpoint{4.695992in}{4.860532in}}%
\pgfpathlineto{\pgfqpoint{4.758105in}{4.860532in}}%
\pgfpathmoveto{\pgfqpoint{4.727049in}{4.829475in}}%
\pgfpathlineto{\pgfqpoint{4.727049in}{4.891588in}}%
\pgfusepath{stroke,fill}%
\end{pgfscope}%
\begin{pgfscope}%
\pgfpathrectangle{\pgfqpoint{2.000000in}{4.421053in}}{\pgfqpoint{4.376471in}{0.978947in}} %
\pgfusepath{clip}%
\pgfsetbuttcap%
\pgfsetroundjoin%
\definecolor{currentfill}{rgb}{1.000000,0.000000,0.000000}%
\pgfsetfillcolor{currentfill}%
\pgfsetlinewidth{2.007500pt}%
\definecolor{currentstroke}{rgb}{1.000000,0.000000,0.000000}%
\pgfsetstrokecolor{currentstroke}%
\pgfsetdash{}{0pt}%
\pgfpathmoveto{\pgfqpoint{4.833062in}{4.988087in}}%
\pgfpathlineto{\pgfqpoint{4.895175in}{4.988087in}}%
\pgfpathmoveto{\pgfqpoint{4.864118in}{4.957031in}}%
\pgfpathlineto{\pgfqpoint{4.864118in}{5.019144in}}%
\pgfusepath{stroke,fill}%
\end{pgfscope}%
\begin{pgfscope}%
\pgfpathrectangle{\pgfqpoint{2.000000in}{4.421053in}}{\pgfqpoint{4.376471in}{0.978947in}} %
\pgfusepath{clip}%
\pgfsetbuttcap%
\pgfsetroundjoin%
\definecolor{currentfill}{rgb}{1.000000,0.000000,0.000000}%
\pgfsetfillcolor{currentfill}%
\pgfsetlinewidth{2.007500pt}%
\definecolor{currentstroke}{rgb}{1.000000,0.000000,0.000000}%
\pgfsetstrokecolor{currentstroke}%
\pgfsetdash{}{0pt}%
\pgfpathmoveto{\pgfqpoint{6.084915in}{5.248251in}}%
\pgfpathlineto{\pgfqpoint{6.147028in}{5.248251in}}%
\pgfpathmoveto{\pgfqpoint{6.115971in}{5.217195in}}%
\pgfpathlineto{\pgfqpoint{6.115971in}{5.279308in}}%
\pgfusepath{stroke,fill}%
\end{pgfscope}%
\begin{pgfscope}%
\pgfpathrectangle{\pgfqpoint{2.000000in}{4.421053in}}{\pgfqpoint{4.376471in}{0.978947in}} %
\pgfusepath{clip}%
\pgfsetbuttcap%
\pgfsetroundjoin%
\definecolor{currentfill}{rgb}{1.000000,0.000000,0.000000}%
\pgfsetfillcolor{currentfill}%
\pgfsetlinewidth{2.007500pt}%
\definecolor{currentstroke}{rgb}{1.000000,0.000000,0.000000}%
\pgfsetstrokecolor{currentstroke}%
\pgfsetdash{}{0pt}%
\pgfpathmoveto{\pgfqpoint{3.092947in}{4.960862in}}%
\pgfpathlineto{\pgfqpoint{3.155060in}{4.960862in}}%
\pgfpathmoveto{\pgfqpoint{3.124004in}{4.929806in}}%
\pgfpathlineto{\pgfqpoint{3.124004in}{4.991919in}}%
\pgfusepath{stroke,fill}%
\end{pgfscope}%
\begin{pgfscope}%
\pgfpathrectangle{\pgfqpoint{2.000000in}{4.421053in}}{\pgfqpoint{4.376471in}{0.978947in}} %
\pgfusepath{clip}%
\pgfsetbuttcap%
\pgfsetroundjoin%
\definecolor{currentfill}{rgb}{1.000000,0.000000,0.000000}%
\pgfsetfillcolor{currentfill}%
\pgfsetlinewidth{2.007500pt}%
\definecolor{currentstroke}{rgb}{1.000000,0.000000,0.000000}%
\pgfsetstrokecolor{currentstroke}%
\pgfsetdash{}{0pt}%
\pgfpathmoveto{\pgfqpoint{3.149293in}{4.903228in}}%
\pgfpathlineto{\pgfqpoint{3.211406in}{4.903228in}}%
\pgfpathmoveto{\pgfqpoint{3.180349in}{4.872172in}}%
\pgfpathlineto{\pgfqpoint{3.180349in}{4.934285in}}%
\pgfusepath{stroke,fill}%
\end{pgfscope}%
\begin{pgfscope}%
\pgfpathrectangle{\pgfqpoint{2.000000in}{4.421053in}}{\pgfqpoint{4.376471in}{0.978947in}} %
\pgfusepath{clip}%
\pgfsetbuttcap%
\pgfsetroundjoin%
\definecolor{currentfill}{rgb}{1.000000,0.000000,0.000000}%
\pgfsetfillcolor{currentfill}%
\pgfsetlinewidth{2.007500pt}%
\definecolor{currentstroke}{rgb}{1.000000,0.000000,0.000000}%
\pgfsetstrokecolor{currentstroke}%
\pgfsetdash{}{0pt}%
\pgfpathmoveto{\pgfqpoint{2.915026in}{5.190508in}}%
\pgfpathlineto{\pgfqpoint{2.977139in}{5.190508in}}%
\pgfpathmoveto{\pgfqpoint{2.946082in}{5.159452in}}%
\pgfpathlineto{\pgfqpoint{2.946082in}{5.221565in}}%
\pgfusepath{stroke,fill}%
\end{pgfscope}%
\begin{pgfscope}%
\pgfpathrectangle{\pgfqpoint{2.000000in}{4.421053in}}{\pgfqpoint{4.376471in}{0.978947in}} %
\pgfusepath{clip}%
\pgfsetbuttcap%
\pgfsetroundjoin%
\definecolor{currentfill}{rgb}{1.000000,0.000000,0.000000}%
\pgfsetfillcolor{currentfill}%
\pgfsetlinewidth{2.007500pt}%
\definecolor{currentstroke}{rgb}{1.000000,0.000000,0.000000}%
\pgfsetstrokecolor{currentstroke}%
\pgfsetdash{}{0pt}%
\pgfpathmoveto{\pgfqpoint{5.759387in}{5.036104in}}%
\pgfpathlineto{\pgfqpoint{5.821500in}{5.036104in}}%
\pgfpathmoveto{\pgfqpoint{5.790443in}{5.005048in}}%
\pgfpathlineto{\pgfqpoint{5.790443in}{5.067161in}}%
\pgfusepath{stroke,fill}%
\end{pgfscope}%
\begin{pgfscope}%
\pgfpathrectangle{\pgfqpoint{2.000000in}{4.421053in}}{\pgfqpoint{4.376471in}{0.978947in}} %
\pgfusepath{clip}%
\pgfsetbuttcap%
\pgfsetroundjoin%
\definecolor{currentfill}{rgb}{1.000000,0.000000,0.000000}%
\pgfsetfillcolor{currentfill}%
\pgfsetlinewidth{2.007500pt}%
\definecolor{currentstroke}{rgb}{1.000000,0.000000,0.000000}%
\pgfsetstrokecolor{currentstroke}%
\pgfsetdash{}{0pt}%
\pgfpathmoveto{\pgfqpoint{5.568702in}{4.758523in}}%
\pgfpathlineto{\pgfqpoint{5.630815in}{4.758523in}}%
\pgfpathmoveto{\pgfqpoint{5.599758in}{4.727466in}}%
\pgfpathlineto{\pgfqpoint{5.599758in}{4.789579in}}%
\pgfusepath{stroke,fill}%
\end{pgfscope}%
\begin{pgfscope}%
\pgfpathrectangle{\pgfqpoint{2.000000in}{4.421053in}}{\pgfqpoint{4.376471in}{0.978947in}} %
\pgfusepath{clip}%
\pgfsetbuttcap%
\pgfsetroundjoin%
\definecolor{currentfill}{rgb}{1.000000,0.000000,0.000000}%
\pgfsetfillcolor{currentfill}%
\pgfsetlinewidth{2.007500pt}%
\definecolor{currentstroke}{rgb}{1.000000,0.000000,0.000000}%
\pgfsetstrokecolor{currentstroke}%
\pgfsetdash{}{0pt}%
\pgfpathmoveto{\pgfqpoint{5.890304in}{5.165886in}}%
\pgfpathlineto{\pgfqpoint{5.952417in}{5.165886in}}%
\pgfpathmoveto{\pgfqpoint{5.921360in}{5.134829in}}%
\pgfpathlineto{\pgfqpoint{5.921360in}{5.196942in}}%
\pgfusepath{stroke,fill}%
\end{pgfscope}%
\begin{pgfscope}%
\pgfpathrectangle{\pgfqpoint{2.000000in}{4.421053in}}{\pgfqpoint{4.376471in}{0.978947in}} %
\pgfusepath{clip}%
\pgfsetbuttcap%
\pgfsetroundjoin%
\definecolor{currentfill}{rgb}{1.000000,0.000000,0.000000}%
\pgfsetfillcolor{currentfill}%
\pgfsetlinewidth{2.007500pt}%
\definecolor{currentstroke}{rgb}{1.000000,0.000000,0.000000}%
\pgfsetstrokecolor{currentstroke}%
\pgfsetdash{}{0pt}%
\pgfpathmoveto{\pgfqpoint{6.270553in}{5.097123in}}%
\pgfpathlineto{\pgfqpoint{6.332666in}{5.097123in}}%
\pgfpathmoveto{\pgfqpoint{6.301610in}{5.066066in}}%
\pgfpathlineto{\pgfqpoint{6.301610in}{5.128179in}}%
\pgfusepath{stroke,fill}%
\end{pgfscope}%
\begin{pgfscope}%
\pgfpathrectangle{\pgfqpoint{2.000000in}{4.421053in}}{\pgfqpoint{4.376471in}{0.978947in}} %
\pgfusepath{clip}%
\pgfsetbuttcap%
\pgfsetroundjoin%
\definecolor{currentfill}{rgb}{1.000000,0.000000,0.000000}%
\pgfsetfillcolor{currentfill}%
\pgfsetlinewidth{2.007500pt}%
\definecolor{currentstroke}{rgb}{1.000000,0.000000,0.000000}%
\pgfsetstrokecolor{currentstroke}%
\pgfsetdash{}{0pt}%
\pgfpathmoveto{\pgfqpoint{5.642233in}{4.913690in}}%
\pgfpathlineto{\pgfqpoint{5.704346in}{4.913690in}}%
\pgfpathmoveto{\pgfqpoint{5.673289in}{4.882633in}}%
\pgfpathlineto{\pgfqpoint{5.673289in}{4.944746in}}%
\pgfusepath{stroke,fill}%
\end{pgfscope}%
\begin{pgfscope}%
\pgfpathrectangle{\pgfqpoint{2.000000in}{4.421053in}}{\pgfqpoint{4.376471in}{0.978947in}} %
\pgfusepath{clip}%
\pgfsetbuttcap%
\pgfsetroundjoin%
\definecolor{currentfill}{rgb}{1.000000,0.000000,0.000000}%
\pgfsetfillcolor{currentfill}%
\pgfsetlinewidth{2.007500pt}%
\definecolor{currentstroke}{rgb}{1.000000,0.000000,0.000000}%
\pgfsetstrokecolor{currentstroke}%
\pgfsetdash{}{0pt}%
\pgfpathmoveto{\pgfqpoint{4.459958in}{4.638149in}}%
\pgfpathlineto{\pgfqpoint{4.522071in}{4.638149in}}%
\pgfpathmoveto{\pgfqpoint{4.491015in}{4.607093in}}%
\pgfpathlineto{\pgfqpoint{4.491015in}{4.669206in}}%
\pgfusepath{stroke,fill}%
\end{pgfscope}%
\begin{pgfscope}%
\pgfpathrectangle{\pgfqpoint{2.000000in}{4.421053in}}{\pgfqpoint{4.376471in}{0.978947in}} %
\pgfusepath{clip}%
\pgfsetbuttcap%
\pgfsetroundjoin%
\definecolor{currentfill}{rgb}{1.000000,0.000000,0.000000}%
\pgfsetfillcolor{currentfill}%
\pgfsetlinewidth{2.007500pt}%
\definecolor{currentstroke}{rgb}{1.000000,0.000000,0.000000}%
\pgfsetstrokecolor{currentstroke}%
\pgfsetdash{}{0pt}%
\pgfpathmoveto{\pgfqpoint{5.577008in}{4.780559in}}%
\pgfpathlineto{\pgfqpoint{5.639121in}{4.780559in}}%
\pgfpathmoveto{\pgfqpoint{5.608065in}{4.749503in}}%
\pgfpathlineto{\pgfqpoint{5.608065in}{4.811616in}}%
\pgfusepath{stroke,fill}%
\end{pgfscope}%
\begin{pgfscope}%
\pgfpathrectangle{\pgfqpoint{2.000000in}{4.421053in}}{\pgfqpoint{4.376471in}{0.978947in}} %
\pgfusepath{clip}%
\pgfsetbuttcap%
\pgfsetroundjoin%
\definecolor{currentfill}{rgb}{1.000000,0.000000,0.000000}%
\pgfsetfillcolor{currentfill}%
\pgfsetlinewidth{2.007500pt}%
\definecolor{currentstroke}{rgb}{1.000000,0.000000,0.000000}%
\pgfsetstrokecolor{currentstroke}%
\pgfsetdash{}{0pt}%
\pgfpathmoveto{\pgfqpoint{3.258337in}{4.801297in}}%
\pgfpathlineto{\pgfqpoint{3.320450in}{4.801297in}}%
\pgfpathmoveto{\pgfqpoint{3.289394in}{4.770240in}}%
\pgfpathlineto{\pgfqpoint{3.289394in}{4.832353in}}%
\pgfusepath{stroke,fill}%
\end{pgfscope}%
\begin{pgfscope}%
\pgfpathrectangle{\pgfqpoint{2.000000in}{4.421053in}}{\pgfqpoint{4.376471in}{0.978947in}} %
\pgfusepath{clip}%
\pgfsetbuttcap%
\pgfsetroundjoin%
\definecolor{currentfill}{rgb}{1.000000,0.000000,0.000000}%
\pgfsetfillcolor{currentfill}%
\pgfsetlinewidth{2.007500pt}%
\definecolor{currentstroke}{rgb}{1.000000,0.000000,0.000000}%
\pgfsetstrokecolor{currentstroke}%
\pgfsetdash{}{0pt}%
\pgfpathmoveto{\pgfqpoint{5.084714in}{4.904581in}}%
\pgfpathlineto{\pgfqpoint{5.146827in}{4.904581in}}%
\pgfpathmoveto{\pgfqpoint{5.115771in}{4.873524in}}%
\pgfpathlineto{\pgfqpoint{5.115771in}{4.935637in}}%
\pgfusepath{stroke,fill}%
\end{pgfscope}%
\begin{pgfscope}%
\pgfpathrectangle{\pgfqpoint{2.000000in}{4.421053in}}{\pgfqpoint{4.376471in}{0.978947in}} %
\pgfusepath{clip}%
\pgfsetbuttcap%
\pgfsetroundjoin%
\definecolor{currentfill}{rgb}{1.000000,0.000000,0.000000}%
\pgfsetfillcolor{currentfill}%
\pgfsetlinewidth{2.007500pt}%
\definecolor{currentstroke}{rgb}{1.000000,0.000000,0.000000}%
\pgfsetstrokecolor{currentstroke}%
\pgfsetdash{}{0pt}%
\pgfpathmoveto{\pgfqpoint{3.346143in}{4.809587in}}%
\pgfpathlineto{\pgfqpoint{3.408256in}{4.809587in}}%
\pgfpathmoveto{\pgfqpoint{3.377199in}{4.778530in}}%
\pgfpathlineto{\pgfqpoint{3.377199in}{4.840643in}}%
\pgfusepath{stroke,fill}%
\end{pgfscope}%
\begin{pgfscope}%
\pgfpathrectangle{\pgfqpoint{2.000000in}{4.421053in}}{\pgfqpoint{4.376471in}{0.978947in}} %
\pgfusepath{clip}%
\pgfsetbuttcap%
\pgfsetroundjoin%
\definecolor{currentfill}{rgb}{1.000000,0.000000,0.000000}%
\pgfsetfillcolor{currentfill}%
\pgfsetlinewidth{2.007500pt}%
\definecolor{currentstroke}{rgb}{1.000000,0.000000,0.000000}%
\pgfsetstrokecolor{currentstroke}%
\pgfsetdash{}{0pt}%
\pgfpathmoveto{\pgfqpoint{6.151690in}{5.209780in}}%
\pgfpathlineto{\pgfqpoint{6.213803in}{5.209780in}}%
\pgfpathmoveto{\pgfqpoint{6.182747in}{5.178724in}}%
\pgfpathlineto{\pgfqpoint{6.182747in}{5.240837in}}%
\pgfusepath{stroke,fill}%
\end{pgfscope}%
\begin{pgfscope}%
\pgfpathrectangle{\pgfqpoint{2.000000in}{4.421053in}}{\pgfqpoint{4.376471in}{0.978947in}} %
\pgfusepath{clip}%
\pgfsetbuttcap%
\pgfsetroundjoin%
\definecolor{currentfill}{rgb}{1.000000,0.000000,0.000000}%
\pgfsetfillcolor{currentfill}%
\pgfsetlinewidth{2.007500pt}%
\definecolor{currentstroke}{rgb}{1.000000,0.000000,0.000000}%
\pgfsetstrokecolor{currentstroke}%
\pgfsetdash{}{0pt}%
\pgfpathmoveto{\pgfqpoint{4.671321in}{4.856983in}}%
\pgfpathlineto{\pgfqpoint{4.733434in}{4.856983in}}%
\pgfpathmoveto{\pgfqpoint{4.702377in}{4.825926in}}%
\pgfpathlineto{\pgfqpoint{4.702377in}{4.888039in}}%
\pgfusepath{stroke,fill}%
\end{pgfscope}%
\begin{pgfscope}%
\pgfpathrectangle{\pgfqpoint{2.000000in}{4.421053in}}{\pgfqpoint{4.376471in}{0.978947in}} %
\pgfusepath{clip}%
\pgfsetbuttcap%
\pgfsetroundjoin%
\definecolor{currentfill}{rgb}{1.000000,0.000000,0.000000}%
\pgfsetfillcolor{currentfill}%
\pgfsetlinewidth{2.007500pt}%
\definecolor{currentstroke}{rgb}{1.000000,0.000000,0.000000}%
\pgfsetstrokecolor{currentstroke}%
\pgfsetdash{}{0pt}%
\pgfpathmoveto{\pgfqpoint{4.296042in}{4.605866in}}%
\pgfpathlineto{\pgfqpoint{4.358155in}{4.605866in}}%
\pgfpathmoveto{\pgfqpoint{4.327099in}{4.574809in}}%
\pgfpathlineto{\pgfqpoint{4.327099in}{4.636922in}}%
\pgfusepath{stroke,fill}%
\end{pgfscope}%
\begin{pgfscope}%
\pgfpathrectangle{\pgfqpoint{2.000000in}{4.421053in}}{\pgfqpoint{4.376471in}{0.978947in}} %
\pgfusepath{clip}%
\pgfsetbuttcap%
\pgfsetroundjoin%
\definecolor{currentfill}{rgb}{0.000000,0.000000,0.000000}%
\pgfsetfillcolor{currentfill}%
\pgfsetlinewidth{0.301125pt}%
\definecolor{currentstroke}{rgb}{0.000000,0.000000,0.000000}%
\pgfsetstrokecolor{currentstroke}%
\pgfsetdash{}{0pt}%
\pgfsys@defobject{currentmarker}{\pgfqpoint{-0.015528in}{-0.015528in}}{\pgfqpoint{0.015528in}{0.015528in}}{%
\pgfpathmoveto{\pgfqpoint{0.000000in}{-0.015528in}}%
\pgfpathcurveto{\pgfqpoint{0.004118in}{-0.015528in}}{\pgfqpoint{0.008068in}{-0.013892in}}{\pgfqpoint{0.010980in}{-0.010980in}}%
\pgfpathcurveto{\pgfqpoint{0.013892in}{-0.008068in}}{\pgfqpoint{0.015528in}{-0.004118in}}{\pgfqpoint{0.015528in}{0.000000in}}%
\pgfpathcurveto{\pgfqpoint{0.015528in}{0.004118in}}{\pgfqpoint{0.013892in}{0.008068in}}{\pgfqpoint{0.010980in}{0.010980in}}%
\pgfpathcurveto{\pgfqpoint{0.008068in}{0.013892in}}{\pgfqpoint{0.004118in}{0.015528in}}{\pgfqpoint{0.000000in}{0.015528in}}%
\pgfpathcurveto{\pgfqpoint{-0.004118in}{0.015528in}}{\pgfqpoint{-0.008068in}{0.013892in}}{\pgfqpoint{-0.010980in}{0.010980in}}%
\pgfpathcurveto{\pgfqpoint{-0.013892in}{0.008068in}}{\pgfqpoint{-0.015528in}{0.004118in}}{\pgfqpoint{-0.015528in}{0.000000in}}%
\pgfpathcurveto{\pgfqpoint{-0.015528in}{-0.004118in}}{\pgfqpoint{-0.013892in}{-0.008068in}}{\pgfqpoint{-0.010980in}{-0.010980in}}%
\pgfpathcurveto{\pgfqpoint{-0.008068in}{-0.013892in}}{\pgfqpoint{-0.004118in}{-0.015528in}}{\pgfqpoint{0.000000in}{-0.015528in}}%
\pgfpathclose%
\pgfusepath{stroke,fill}%
}%
\begin{pgfscope}%
\pgfsys@transformshift{2.875294in}{5.254916in}%
\pgfsys@useobject{currentmarker}{}%
\end{pgfscope}%
\begin{pgfscope}%
\pgfsys@transformshift{2.892888in}{5.160724in}%
\pgfsys@useobject{currentmarker}{}%
\end{pgfscope}%
\begin{pgfscope}%
\pgfsys@transformshift{2.910482in}{5.251498in}%
\pgfsys@useobject{currentmarker}{}%
\end{pgfscope}%
\begin{pgfscope}%
\pgfsys@transformshift{2.928076in}{5.270634in}%
\pgfsys@useobject{currentmarker}{}%
\end{pgfscope}%
\begin{pgfscope}%
\pgfsys@transformshift{2.945670in}{5.205849in}%
\pgfsys@useobject{currentmarker}{}%
\end{pgfscope}%
\begin{pgfscope}%
\pgfsys@transformshift{2.963263in}{5.201748in}%
\pgfsys@useobject{currentmarker}{}%
\end{pgfscope}%
\begin{pgfscope}%
\pgfsys@transformshift{2.980857in}{5.077991in}%
\pgfsys@useobject{currentmarker}{}%
\end{pgfscope}%
\begin{pgfscope}%
\pgfsys@transformshift{2.998451in}{5.077595in}%
\pgfsys@useobject{currentmarker}{}%
\end{pgfscope}%
\begin{pgfscope}%
\pgfsys@transformshift{3.016045in}{5.018268in}%
\pgfsys@useobject{currentmarker}{}%
\end{pgfscope}%
\begin{pgfscope}%
\pgfsys@transformshift{3.033639in}{5.022982in}%
\pgfsys@useobject{currentmarker}{}%
\end{pgfscope}%
\begin{pgfscope}%
\pgfsys@transformshift{3.051233in}{4.950281in}%
\pgfsys@useobject{currentmarker}{}%
\end{pgfscope}%
\begin{pgfscope}%
\pgfsys@transformshift{3.068826in}{4.831659in}%
\pgfsys@useobject{currentmarker}{}%
\end{pgfscope}%
\begin{pgfscope}%
\pgfsys@transformshift{3.086420in}{5.001404in}%
\pgfsys@useobject{currentmarker}{}%
\end{pgfscope}%
\begin{pgfscope}%
\pgfsys@transformshift{3.104014in}{4.919253in}%
\pgfsys@useobject{currentmarker}{}%
\end{pgfscope}%
\begin{pgfscope}%
\pgfsys@transformshift{3.121608in}{4.772358in}%
\pgfsys@useobject{currentmarker}{}%
\end{pgfscope}%
\begin{pgfscope}%
\pgfsys@transformshift{3.139202in}{4.965761in}%
\pgfsys@useobject{currentmarker}{}%
\end{pgfscope}%
\begin{pgfscope}%
\pgfsys@transformshift{3.156796in}{4.807758in}%
\pgfsys@useobject{currentmarker}{}%
\end{pgfscope}%
\begin{pgfscope}%
\pgfsys@transformshift{3.174390in}{4.889135in}%
\pgfsys@useobject{currentmarker}{}%
\end{pgfscope}%
\begin{pgfscope}%
\pgfsys@transformshift{3.191983in}{4.943692in}%
\pgfsys@useobject{currentmarker}{}%
\end{pgfscope}%
\begin{pgfscope}%
\pgfsys@transformshift{3.209577in}{4.870028in}%
\pgfsys@useobject{currentmarker}{}%
\end{pgfscope}%
\begin{pgfscope}%
\pgfsys@transformshift{3.227171in}{4.962679in}%
\pgfsys@useobject{currentmarker}{}%
\end{pgfscope}%
\begin{pgfscope}%
\pgfsys@transformshift{3.244765in}{4.712306in}%
\pgfsys@useobject{currentmarker}{}%
\end{pgfscope}%
\begin{pgfscope}%
\pgfsys@transformshift{3.262359in}{4.873130in}%
\pgfsys@useobject{currentmarker}{}%
\end{pgfscope}%
\begin{pgfscope}%
\pgfsys@transformshift{3.279953in}{4.758281in}%
\pgfsys@useobject{currentmarker}{}%
\end{pgfscope}%
\begin{pgfscope}%
\pgfsys@transformshift{3.297547in}{4.737440in}%
\pgfsys@useobject{currentmarker}{}%
\end{pgfscope}%
\begin{pgfscope}%
\pgfsys@transformshift{3.315140in}{4.767404in}%
\pgfsys@useobject{currentmarker}{}%
\end{pgfscope}%
\begin{pgfscope}%
\pgfsys@transformshift{3.332734in}{4.796858in}%
\pgfsys@useobject{currentmarker}{}%
\end{pgfscope}%
\begin{pgfscope}%
\pgfsys@transformshift{3.350328in}{4.838472in}%
\pgfsys@useobject{currentmarker}{}%
\end{pgfscope}%
\begin{pgfscope}%
\pgfsys@transformshift{3.367922in}{4.719865in}%
\pgfsys@useobject{currentmarker}{}%
\end{pgfscope}%
\begin{pgfscope}%
\pgfsys@transformshift{3.385516in}{4.938174in}%
\pgfsys@useobject{currentmarker}{}%
\end{pgfscope}%
\begin{pgfscope}%
\pgfsys@transformshift{3.403110in}{4.902994in}%
\pgfsys@useobject{currentmarker}{}%
\end{pgfscope}%
\begin{pgfscope}%
\pgfsys@transformshift{3.420704in}{4.709459in}%
\pgfsys@useobject{currentmarker}{}%
\end{pgfscope}%
\begin{pgfscope}%
\pgfsys@transformshift{3.438297in}{5.029731in}%
\pgfsys@useobject{currentmarker}{}%
\end{pgfscope}%
\begin{pgfscope}%
\pgfsys@transformshift{3.455891in}{5.084223in}%
\pgfsys@useobject{currentmarker}{}%
\end{pgfscope}%
\begin{pgfscope}%
\pgfsys@transformshift{3.473485in}{5.024905in}%
\pgfsys@useobject{currentmarker}{}%
\end{pgfscope}%
\begin{pgfscope}%
\pgfsys@transformshift{3.491079in}{4.900851in}%
\pgfsys@useobject{currentmarker}{}%
\end{pgfscope}%
\begin{pgfscope}%
\pgfsys@transformshift{3.508673in}{4.824998in}%
\pgfsys@useobject{currentmarker}{}%
\end{pgfscope}%
\begin{pgfscope}%
\pgfsys@transformshift{3.526267in}{5.056986in}%
\pgfsys@useobject{currentmarker}{}%
\end{pgfscope}%
\begin{pgfscope}%
\pgfsys@transformshift{3.543860in}{4.923698in}%
\pgfsys@useobject{currentmarker}{}%
\end{pgfscope}%
\begin{pgfscope}%
\pgfsys@transformshift{3.561454in}{5.104693in}%
\pgfsys@useobject{currentmarker}{}%
\end{pgfscope}%
\begin{pgfscope}%
\pgfsys@transformshift{3.579048in}{5.016167in}%
\pgfsys@useobject{currentmarker}{}%
\end{pgfscope}%
\begin{pgfscope}%
\pgfsys@transformshift{3.596642in}{5.108880in}%
\pgfsys@useobject{currentmarker}{}%
\end{pgfscope}%
\begin{pgfscope}%
\pgfsys@transformshift{3.614236in}{5.059262in}%
\pgfsys@useobject{currentmarker}{}%
\end{pgfscope}%
\begin{pgfscope}%
\pgfsys@transformshift{3.631830in}{5.107695in}%
\pgfsys@useobject{currentmarker}{}%
\end{pgfscope}%
\begin{pgfscope}%
\pgfsys@transformshift{3.649424in}{5.048347in}%
\pgfsys@useobject{currentmarker}{}%
\end{pgfscope}%
\begin{pgfscope}%
\pgfsys@transformshift{3.667017in}{5.239770in}%
\pgfsys@useobject{currentmarker}{}%
\end{pgfscope}%
\begin{pgfscope}%
\pgfsys@transformshift{3.684611in}{5.079579in}%
\pgfsys@useobject{currentmarker}{}%
\end{pgfscope}%
\begin{pgfscope}%
\pgfsys@transformshift{3.702205in}{5.115070in}%
\pgfsys@useobject{currentmarker}{}%
\end{pgfscope}%
\begin{pgfscope}%
\pgfsys@transformshift{3.719799in}{5.271894in}%
\pgfsys@useobject{currentmarker}{}%
\end{pgfscope}%
\begin{pgfscope}%
\pgfsys@transformshift{3.737393in}{4.946452in}%
\pgfsys@useobject{currentmarker}{}%
\end{pgfscope}%
\begin{pgfscope}%
\pgfsys@transformshift{3.754987in}{4.956515in}%
\pgfsys@useobject{currentmarker}{}%
\end{pgfscope}%
\begin{pgfscope}%
\pgfsys@transformshift{3.772581in}{5.185202in}%
\pgfsys@useobject{currentmarker}{}%
\end{pgfscope}%
\begin{pgfscope}%
\pgfsys@transformshift{3.790174in}{4.965051in}%
\pgfsys@useobject{currentmarker}{}%
\end{pgfscope}%
\begin{pgfscope}%
\pgfsys@transformshift{3.807768in}{5.279234in}%
\pgfsys@useobject{currentmarker}{}%
\end{pgfscope}%
\begin{pgfscope}%
\pgfsys@transformshift{3.825362in}{5.033240in}%
\pgfsys@useobject{currentmarker}{}%
\end{pgfscope}%
\begin{pgfscope}%
\pgfsys@transformshift{3.842956in}{4.991625in}%
\pgfsys@useobject{currentmarker}{}%
\end{pgfscope}%
\begin{pgfscope}%
\pgfsys@transformshift{3.860550in}{5.254445in}%
\pgfsys@useobject{currentmarker}{}%
\end{pgfscope}%
\begin{pgfscope}%
\pgfsys@transformshift{3.878144in}{5.197982in}%
\pgfsys@useobject{currentmarker}{}%
\end{pgfscope}%
\begin{pgfscope}%
\pgfsys@transformshift{3.895738in}{5.224324in}%
\pgfsys@useobject{currentmarker}{}%
\end{pgfscope}%
\begin{pgfscope}%
\pgfsys@transformshift{3.913331in}{5.111466in}%
\pgfsys@useobject{currentmarker}{}%
\end{pgfscope}%
\begin{pgfscope}%
\pgfsys@transformshift{3.930925in}{4.914878in}%
\pgfsys@useobject{currentmarker}{}%
\end{pgfscope}%
\begin{pgfscope}%
\pgfsys@transformshift{3.948519in}{5.179668in}%
\pgfsys@useobject{currentmarker}{}%
\end{pgfscope}%
\begin{pgfscope}%
\pgfsys@transformshift{3.966113in}{4.938468in}%
\pgfsys@useobject{currentmarker}{}%
\end{pgfscope}%
\begin{pgfscope}%
\pgfsys@transformshift{3.983707in}{5.027406in}%
\pgfsys@useobject{currentmarker}{}%
\end{pgfscope}%
\begin{pgfscope}%
\pgfsys@transformshift{4.001301in}{5.020999in}%
\pgfsys@useobject{currentmarker}{}%
\end{pgfscope}%
\begin{pgfscope}%
\pgfsys@transformshift{4.018894in}{4.886656in}%
\pgfsys@useobject{currentmarker}{}%
\end{pgfscope}%
\begin{pgfscope}%
\pgfsys@transformshift{4.036488in}{4.942565in}%
\pgfsys@useobject{currentmarker}{}%
\end{pgfscope}%
\begin{pgfscope}%
\pgfsys@transformshift{4.054082in}{4.951091in}%
\pgfsys@useobject{currentmarker}{}%
\end{pgfscope}%
\begin{pgfscope}%
\pgfsys@transformshift{4.071676in}{4.872367in}%
\pgfsys@useobject{currentmarker}{}%
\end{pgfscope}%
\begin{pgfscope}%
\pgfsys@transformshift{4.089270in}{4.698835in}%
\pgfsys@useobject{currentmarker}{}%
\end{pgfscope}%
\begin{pgfscope}%
\pgfsys@transformshift{4.106864in}{4.818561in}%
\pgfsys@useobject{currentmarker}{}%
\end{pgfscope}%
\begin{pgfscope}%
\pgfsys@transformshift{4.124458in}{4.901053in}%
\pgfsys@useobject{currentmarker}{}%
\end{pgfscope}%
\begin{pgfscope}%
\pgfsys@transformshift{4.142051in}{4.673262in}%
\pgfsys@useobject{currentmarker}{}%
\end{pgfscope}%
\begin{pgfscope}%
\pgfsys@transformshift{4.159645in}{4.707965in}%
\pgfsys@useobject{currentmarker}{}%
\end{pgfscope}%
\begin{pgfscope}%
\pgfsys@transformshift{4.177239in}{4.659015in}%
\pgfsys@useobject{currentmarker}{}%
\end{pgfscope}%
\begin{pgfscope}%
\pgfsys@transformshift{4.194833in}{4.873338in}%
\pgfsys@useobject{currentmarker}{}%
\end{pgfscope}%
\begin{pgfscope}%
\pgfsys@transformshift{4.212427in}{4.736067in}%
\pgfsys@useobject{currentmarker}{}%
\end{pgfscope}%
\begin{pgfscope}%
\pgfsys@transformshift{4.230021in}{4.693339in}%
\pgfsys@useobject{currentmarker}{}%
\end{pgfscope}%
\begin{pgfscope}%
\pgfsys@transformshift{4.247615in}{4.559224in}%
\pgfsys@useobject{currentmarker}{}%
\end{pgfscope}%
\begin{pgfscope}%
\pgfsys@transformshift{4.265208in}{4.680465in}%
\pgfsys@useobject{currentmarker}{}%
\end{pgfscope}%
\begin{pgfscope}%
\pgfsys@transformshift{4.282802in}{4.546350in}%
\pgfsys@useobject{currentmarker}{}%
\end{pgfscope}%
\begin{pgfscope}%
\pgfsys@transformshift{4.300396in}{4.609986in}%
\pgfsys@useobject{currentmarker}{}%
\end{pgfscope}%
\begin{pgfscope}%
\pgfsys@transformshift{4.317990in}{4.535582in}%
\pgfsys@useobject{currentmarker}{}%
\end{pgfscope}%
\begin{pgfscope}%
\pgfsys@transformshift{4.335584in}{4.665186in}%
\pgfsys@useobject{currentmarker}{}%
\end{pgfscope}%
\begin{pgfscope}%
\pgfsys@transformshift{4.353178in}{4.652923in}%
\pgfsys@useobject{currentmarker}{}%
\end{pgfscope}%
\begin{pgfscope}%
\pgfsys@transformshift{4.370772in}{4.572944in}%
\pgfsys@useobject{currentmarker}{}%
\end{pgfscope}%
\begin{pgfscope}%
\pgfsys@transformshift{4.388365in}{4.636754in}%
\pgfsys@useobject{currentmarker}{}%
\end{pgfscope}%
\begin{pgfscope}%
\pgfsys@transformshift{4.405959in}{4.489189in}%
\pgfsys@useobject{currentmarker}{}%
\end{pgfscope}%
\begin{pgfscope}%
\pgfsys@transformshift{4.423553in}{4.454903in}%
\pgfsys@useobject{currentmarker}{}%
\end{pgfscope}%
\begin{pgfscope}%
\pgfsys@transformshift{4.441147in}{4.660065in}%
\pgfsys@useobject{currentmarker}{}%
\end{pgfscope}%
\begin{pgfscope}%
\pgfsys@transformshift{4.458741in}{4.642414in}%
\pgfsys@useobject{currentmarker}{}%
\end{pgfscope}%
\begin{pgfscope}%
\pgfsys@transformshift{4.476335in}{4.702096in}%
\pgfsys@useobject{currentmarker}{}%
\end{pgfscope}%
\begin{pgfscope}%
\pgfsys@transformshift{4.493928in}{4.893914in}%
\pgfsys@useobject{currentmarker}{}%
\end{pgfscope}%
\begin{pgfscope}%
\pgfsys@transformshift{4.511522in}{4.762255in}%
\pgfsys@useobject{currentmarker}{}%
\end{pgfscope}%
\begin{pgfscope}%
\pgfsys@transformshift{4.529116in}{4.589240in}%
\pgfsys@useobject{currentmarker}{}%
\end{pgfscope}%
\begin{pgfscope}%
\pgfsys@transformshift{4.546710in}{4.813773in}%
\pgfsys@useobject{currentmarker}{}%
\end{pgfscope}%
\begin{pgfscope}%
\pgfsys@transformshift{4.564304in}{4.584205in}%
\pgfsys@useobject{currentmarker}{}%
\end{pgfscope}%
\begin{pgfscope}%
\pgfsys@transformshift{4.581898in}{4.690640in}%
\pgfsys@useobject{currentmarker}{}%
\end{pgfscope}%
\begin{pgfscope}%
\pgfsys@transformshift{4.599492in}{4.750662in}%
\pgfsys@useobject{currentmarker}{}%
\end{pgfscope}%
\begin{pgfscope}%
\pgfsys@transformshift{4.617085in}{4.952640in}%
\pgfsys@useobject{currentmarker}{}%
\end{pgfscope}%
\begin{pgfscope}%
\pgfsys@transformshift{4.634679in}{4.722467in}%
\pgfsys@useobject{currentmarker}{}%
\end{pgfscope}%
\begin{pgfscope}%
\pgfsys@transformshift{4.652273in}{4.734605in}%
\pgfsys@useobject{currentmarker}{}%
\end{pgfscope}%
\begin{pgfscope}%
\pgfsys@transformshift{4.669867in}{4.829081in}%
\pgfsys@useobject{currentmarker}{}%
\end{pgfscope}%
\begin{pgfscope}%
\pgfsys@transformshift{4.687461in}{4.791275in}%
\pgfsys@useobject{currentmarker}{}%
\end{pgfscope}%
\begin{pgfscope}%
\pgfsys@transformshift{4.705055in}{4.993005in}%
\pgfsys@useobject{currentmarker}{}%
\end{pgfscope}%
\begin{pgfscope}%
\pgfsys@transformshift{4.722649in}{4.786360in}%
\pgfsys@useobject{currentmarker}{}%
\end{pgfscope}%
\begin{pgfscope}%
\pgfsys@transformshift{4.740242in}{4.796843in}%
\pgfsys@useobject{currentmarker}{}%
\end{pgfscope}%
\begin{pgfscope}%
\pgfsys@transformshift{4.757836in}{4.885409in}%
\pgfsys@useobject{currentmarker}{}%
\end{pgfscope}%
\begin{pgfscope}%
\pgfsys@transformshift{4.775430in}{4.894144in}%
\pgfsys@useobject{currentmarker}{}%
\end{pgfscope}%
\begin{pgfscope}%
\pgfsys@transformshift{4.793024in}{5.155096in}%
\pgfsys@useobject{currentmarker}{}%
\end{pgfscope}%
\begin{pgfscope}%
\pgfsys@transformshift{4.810618in}{5.066958in}%
\pgfsys@useobject{currentmarker}{}%
\end{pgfscope}%
\begin{pgfscope}%
\pgfsys@transformshift{4.828212in}{4.989169in}%
\pgfsys@useobject{currentmarker}{}%
\end{pgfscope}%
\begin{pgfscope}%
\pgfsys@transformshift{4.845805in}{4.863576in}%
\pgfsys@useobject{currentmarker}{}%
\end{pgfscope}%
\begin{pgfscope}%
\pgfsys@transformshift{4.863399in}{5.081064in}%
\pgfsys@useobject{currentmarker}{}%
\end{pgfscope}%
\begin{pgfscope}%
\pgfsys@transformshift{4.880993in}{4.897460in}%
\pgfsys@useobject{currentmarker}{}%
\end{pgfscope}%
\begin{pgfscope}%
\pgfsys@transformshift{4.898587in}{4.844460in}%
\pgfsys@useobject{currentmarker}{}%
\end{pgfscope}%
\begin{pgfscope}%
\pgfsys@transformshift{4.916181in}{5.123690in}%
\pgfsys@useobject{currentmarker}{}%
\end{pgfscope}%
\begin{pgfscope}%
\pgfsys@transformshift{4.933775in}{5.033449in}%
\pgfsys@useobject{currentmarker}{}%
\end{pgfscope}%
\begin{pgfscope}%
\pgfsys@transformshift{4.951369in}{5.091681in}%
\pgfsys@useobject{currentmarker}{}%
\end{pgfscope}%
\begin{pgfscope}%
\pgfsys@transformshift{4.968962in}{5.025036in}%
\pgfsys@useobject{currentmarker}{}%
\end{pgfscope}%
\begin{pgfscope}%
\pgfsys@transformshift{4.986556in}{5.072843in}%
\pgfsys@useobject{currentmarker}{}%
\end{pgfscope}%
\begin{pgfscope}%
\pgfsys@transformshift{5.004150in}{4.910283in}%
\pgfsys@useobject{currentmarker}{}%
\end{pgfscope}%
\begin{pgfscope}%
\pgfsys@transformshift{5.021744in}{4.860777in}%
\pgfsys@useobject{currentmarker}{}%
\end{pgfscope}%
\begin{pgfscope}%
\pgfsys@transformshift{5.039338in}{5.023816in}%
\pgfsys@useobject{currentmarker}{}%
\end{pgfscope}%
\begin{pgfscope}%
\pgfsys@transformshift{5.056932in}{4.859088in}%
\pgfsys@useobject{currentmarker}{}%
\end{pgfscope}%
\begin{pgfscope}%
\pgfsys@transformshift{5.074526in}{4.856192in}%
\pgfsys@useobject{currentmarker}{}%
\end{pgfscope}%
\begin{pgfscope}%
\pgfsys@transformshift{5.092119in}{4.864505in}%
\pgfsys@useobject{currentmarker}{}%
\end{pgfscope}%
\begin{pgfscope}%
\pgfsys@transformshift{5.109713in}{4.896325in}%
\pgfsys@useobject{currentmarker}{}%
\end{pgfscope}%
\begin{pgfscope}%
\pgfsys@transformshift{5.127307in}{4.841352in}%
\pgfsys@useobject{currentmarker}{}%
\end{pgfscope}%
\begin{pgfscope}%
\pgfsys@transformshift{5.144901in}{4.719668in}%
\pgfsys@useobject{currentmarker}{}%
\end{pgfscope}%
\begin{pgfscope}%
\pgfsys@transformshift{5.162495in}{4.776392in}%
\pgfsys@useobject{currentmarker}{}%
\end{pgfscope}%
\begin{pgfscope}%
\pgfsys@transformshift{5.180089in}{4.597368in}%
\pgfsys@useobject{currentmarker}{}%
\end{pgfscope}%
\begin{pgfscope}%
\pgfsys@transformshift{5.197683in}{4.870059in}%
\pgfsys@useobject{currentmarker}{}%
\end{pgfscope}%
\begin{pgfscope}%
\pgfsys@transformshift{5.215276in}{4.625477in}%
\pgfsys@useobject{currentmarker}{}%
\end{pgfscope}%
\begin{pgfscope}%
\pgfsys@transformshift{5.232870in}{4.659313in}%
\pgfsys@useobject{currentmarker}{}%
\end{pgfscope}%
\begin{pgfscope}%
\pgfsys@transformshift{5.250464in}{4.761077in}%
\pgfsys@useobject{currentmarker}{}%
\end{pgfscope}%
\begin{pgfscope}%
\pgfsys@transformshift{5.268058in}{4.665102in}%
\pgfsys@useobject{currentmarker}{}%
\end{pgfscope}%
\begin{pgfscope}%
\pgfsys@transformshift{5.285652in}{4.883742in}%
\pgfsys@useobject{currentmarker}{}%
\end{pgfscope}%
\begin{pgfscope}%
\pgfsys@transformshift{5.303246in}{4.581725in}%
\pgfsys@useobject{currentmarker}{}%
\end{pgfscope}%
\begin{pgfscope}%
\pgfsys@transformshift{5.320839in}{4.729414in}%
\pgfsys@useobject{currentmarker}{}%
\end{pgfscope}%
\begin{pgfscope}%
\pgfsys@transformshift{5.338433in}{4.688394in}%
\pgfsys@useobject{currentmarker}{}%
\end{pgfscope}%
\begin{pgfscope}%
\pgfsys@transformshift{5.356027in}{4.565234in}%
\pgfsys@useobject{currentmarker}{}%
\end{pgfscope}%
\begin{pgfscope}%
\pgfsys@transformshift{5.373621in}{4.731505in}%
\pgfsys@useobject{currentmarker}{}%
\end{pgfscope}%
\begin{pgfscope}%
\pgfsys@transformshift{5.391215in}{4.656390in}%
\pgfsys@useobject{currentmarker}{}%
\end{pgfscope}%
\begin{pgfscope}%
\pgfsys@transformshift{5.408809in}{4.750347in}%
\pgfsys@useobject{currentmarker}{}%
\end{pgfscope}%
\begin{pgfscope}%
\pgfsys@transformshift{5.426403in}{4.755463in}%
\pgfsys@useobject{currentmarker}{}%
\end{pgfscope}%
\begin{pgfscope}%
\pgfsys@transformshift{5.443996in}{4.894045in}%
\pgfsys@useobject{currentmarker}{}%
\end{pgfscope}%
\begin{pgfscope}%
\pgfsys@transformshift{5.461590in}{4.813845in}%
\pgfsys@useobject{currentmarker}{}%
\end{pgfscope}%
\begin{pgfscope}%
\pgfsys@transformshift{5.479184in}{4.646147in}%
\pgfsys@useobject{currentmarker}{}%
\end{pgfscope}%
\begin{pgfscope}%
\pgfsys@transformshift{5.496778in}{4.667737in}%
\pgfsys@useobject{currentmarker}{}%
\end{pgfscope}%
\begin{pgfscope}%
\pgfsys@transformshift{5.514372in}{4.814708in}%
\pgfsys@useobject{currentmarker}{}%
\end{pgfscope}%
\begin{pgfscope}%
\pgfsys@transformshift{5.531966in}{4.781821in}%
\pgfsys@useobject{currentmarker}{}%
\end{pgfscope}%
\begin{pgfscope}%
\pgfsys@transformshift{5.549560in}{4.794632in}%
\pgfsys@useobject{currentmarker}{}%
\end{pgfscope}%
\begin{pgfscope}%
\pgfsys@transformshift{5.567153in}{4.580644in}%
\pgfsys@useobject{currentmarker}{}%
\end{pgfscope}%
\begin{pgfscope}%
\pgfsys@transformshift{5.584747in}{4.760929in}%
\pgfsys@useobject{currentmarker}{}%
\end{pgfscope}%
\begin{pgfscope}%
\pgfsys@transformshift{5.602341in}{4.707595in}%
\pgfsys@useobject{currentmarker}{}%
\end{pgfscope}%
\begin{pgfscope}%
\pgfsys@transformshift{5.619935in}{4.832262in}%
\pgfsys@useobject{currentmarker}{}%
\end{pgfscope}%
\begin{pgfscope}%
\pgfsys@transformshift{5.637529in}{4.815823in}%
\pgfsys@useobject{currentmarker}{}%
\end{pgfscope}%
\begin{pgfscope}%
\pgfsys@transformshift{5.655123in}{4.941839in}%
\pgfsys@useobject{currentmarker}{}%
\end{pgfscope}%
\begin{pgfscope}%
\pgfsys@transformshift{5.672717in}{4.905455in}%
\pgfsys@useobject{currentmarker}{}%
\end{pgfscope}%
\begin{pgfscope}%
\pgfsys@transformshift{5.690310in}{4.978072in}%
\pgfsys@useobject{currentmarker}{}%
\end{pgfscope}%
\begin{pgfscope}%
\pgfsys@transformshift{5.707904in}{4.875602in}%
\pgfsys@useobject{currentmarker}{}%
\end{pgfscope}%
\begin{pgfscope}%
\pgfsys@transformshift{5.725498in}{4.852426in}%
\pgfsys@useobject{currentmarker}{}%
\end{pgfscope}%
\begin{pgfscope}%
\pgfsys@transformshift{5.743092in}{4.932576in}%
\pgfsys@useobject{currentmarker}{}%
\end{pgfscope}%
\begin{pgfscope}%
\pgfsys@transformshift{5.760686in}{4.998190in}%
\pgfsys@useobject{currentmarker}{}%
\end{pgfscope}%
\begin{pgfscope}%
\pgfsys@transformshift{5.778280in}{5.063799in}%
\pgfsys@useobject{currentmarker}{}%
\end{pgfscope}%
\begin{pgfscope}%
\pgfsys@transformshift{5.795873in}{5.280211in}%
\pgfsys@useobject{currentmarker}{}%
\end{pgfscope}%
\begin{pgfscope}%
\pgfsys@transformshift{5.813467in}{5.069477in}%
\pgfsys@useobject{currentmarker}{}%
\end{pgfscope}%
\begin{pgfscope}%
\pgfsys@transformshift{5.831061in}{4.999365in}%
\pgfsys@useobject{currentmarker}{}%
\end{pgfscope}%
\begin{pgfscope}%
\pgfsys@transformshift{5.848655in}{5.083543in}%
\pgfsys@useobject{currentmarker}{}%
\end{pgfscope}%
\begin{pgfscope}%
\pgfsys@transformshift{5.866249in}{5.092281in}%
\pgfsys@useobject{currentmarker}{}%
\end{pgfscope}%
\begin{pgfscope}%
\pgfsys@transformshift{5.883843in}{5.207988in}%
\pgfsys@useobject{currentmarker}{}%
\end{pgfscope}%
\begin{pgfscope}%
\pgfsys@transformshift{5.901437in}{5.019570in}%
\pgfsys@useobject{currentmarker}{}%
\end{pgfscope}%
\begin{pgfscope}%
\pgfsys@transformshift{5.919030in}{5.199289in}%
\pgfsys@useobject{currentmarker}{}%
\end{pgfscope}%
\begin{pgfscope}%
\pgfsys@transformshift{5.936624in}{5.223190in}%
\pgfsys@useobject{currentmarker}{}%
\end{pgfscope}%
\begin{pgfscope}%
\pgfsys@transformshift{5.954218in}{5.243459in}%
\pgfsys@useobject{currentmarker}{}%
\end{pgfscope}%
\begin{pgfscope}%
\pgfsys@transformshift{5.971812in}{5.169522in}%
\pgfsys@useobject{currentmarker}{}%
\end{pgfscope}%
\begin{pgfscope}%
\pgfsys@transformshift{5.989406in}{5.214873in}%
\pgfsys@useobject{currentmarker}{}%
\end{pgfscope}%
\begin{pgfscope}%
\pgfsys@transformshift{6.007000in}{5.100648in}%
\pgfsys@useobject{currentmarker}{}%
\end{pgfscope}%
\begin{pgfscope}%
\pgfsys@transformshift{6.024594in}{5.200314in}%
\pgfsys@useobject{currentmarker}{}%
\end{pgfscope}%
\begin{pgfscope}%
\pgfsys@transformshift{6.042187in}{5.198061in}%
\pgfsys@useobject{currentmarker}{}%
\end{pgfscope}%
\begin{pgfscope}%
\pgfsys@transformshift{6.059781in}{5.296725in}%
\pgfsys@useobject{currentmarker}{}%
\end{pgfscope}%
\begin{pgfscope}%
\pgfsys@transformshift{6.077375in}{5.135419in}%
\pgfsys@useobject{currentmarker}{}%
\end{pgfscope}%
\begin{pgfscope}%
\pgfsys@transformshift{6.094969in}{5.330218in}%
\pgfsys@useobject{currentmarker}{}%
\end{pgfscope}%
\begin{pgfscope}%
\pgfsys@transformshift{6.112563in}{5.398508in}%
\pgfsys@useobject{currentmarker}{}%
\end{pgfscope}%
\begin{pgfscope}%
\pgfsys@transformshift{6.130157in}{5.029008in}%
\pgfsys@useobject{currentmarker}{}%
\end{pgfscope}%
\begin{pgfscope}%
\pgfsys@transformshift{6.147751in}{5.276102in}%
\pgfsys@useobject{currentmarker}{}%
\end{pgfscope}%
\begin{pgfscope}%
\pgfsys@transformshift{6.165344in}{5.292913in}%
\pgfsys@useobject{currentmarker}{}%
\end{pgfscope}%
\begin{pgfscope}%
\pgfsys@transformshift{6.182938in}{5.149004in}%
\pgfsys@useobject{currentmarker}{}%
\end{pgfscope}%
\begin{pgfscope}%
\pgfsys@transformshift{6.200532in}{5.162656in}%
\pgfsys@useobject{currentmarker}{}%
\end{pgfscope}%
\begin{pgfscope}%
\pgfsys@transformshift{6.218126in}{5.178002in}%
\pgfsys@useobject{currentmarker}{}%
\end{pgfscope}%
\begin{pgfscope}%
\pgfsys@transformshift{6.235720in}{5.148988in}%
\pgfsys@useobject{currentmarker}{}%
\end{pgfscope}%
\begin{pgfscope}%
\pgfsys@transformshift{6.253314in}{5.135262in}%
\pgfsys@useobject{currentmarker}{}%
\end{pgfscope}%
\begin{pgfscope}%
\pgfsys@transformshift{6.270907in}{4.983102in}%
\pgfsys@useobject{currentmarker}{}%
\end{pgfscope}%
\begin{pgfscope}%
\pgfsys@transformshift{6.288501in}{5.258780in}%
\pgfsys@useobject{currentmarker}{}%
\end{pgfscope}%
\begin{pgfscope}%
\pgfsys@transformshift{6.306095in}{5.238859in}%
\pgfsys@useobject{currentmarker}{}%
\end{pgfscope}%
\begin{pgfscope}%
\pgfsys@transformshift{6.323689in}{5.033756in}%
\pgfsys@useobject{currentmarker}{}%
\end{pgfscope}%
\begin{pgfscope}%
\pgfsys@transformshift{6.341283in}{4.955731in}%
\pgfsys@useobject{currentmarker}{}%
\end{pgfscope}%
\begin{pgfscope}%
\pgfsys@transformshift{6.358877in}{5.147808in}%
\pgfsys@useobject{currentmarker}{}%
\end{pgfscope}%
\begin{pgfscope}%
\pgfsys@transformshift{6.376471in}{5.026386in}%
\pgfsys@useobject{currentmarker}{}%
\end{pgfscope}%
\end{pgfscope}%
\begin{pgfscope}%
\pgfpathrectangle{\pgfqpoint{2.000000in}{4.421053in}}{\pgfqpoint{4.376471in}{0.978947in}} %
\pgfusepath{clip}%
\pgfsetroundcap%
\pgfsetroundjoin%
\pgfsetlinewidth{1.756562pt}%
\definecolor{currentstroke}{rgb}{0.298039,0.447059,0.690196}%
\pgfsetstrokecolor{currentstroke}%
\pgfsetdash{}{0pt}%
\pgfpathmoveto{\pgfqpoint{2.875294in}{5.012215in}}%
\pgfpathlineto{\pgfqpoint{2.892888in}{5.088757in}}%
\pgfpathlineto{\pgfqpoint{2.910482in}{5.141372in}}%
\pgfpathlineto{\pgfqpoint{2.928076in}{5.173971in}}%
\pgfpathlineto{\pgfqpoint{2.945670in}{5.190063in}}%
\pgfpathlineto{\pgfqpoint{2.963263in}{5.192784in}}%
\pgfpathlineto{\pgfqpoint{2.980857in}{5.184906in}}%
\pgfpathlineto{\pgfqpoint{2.998451in}{5.168865in}}%
\pgfpathlineto{\pgfqpoint{3.016045in}{5.146780in}}%
\pgfpathlineto{\pgfqpoint{3.033639in}{5.120474in}}%
\pgfpathlineto{\pgfqpoint{3.068826in}{5.061153in}}%
\pgfpathlineto{\pgfqpoint{3.104014in}{5.000442in}}%
\pgfpathlineto{\pgfqpoint{3.139202in}{4.944588in}}%
\pgfpathlineto{\pgfqpoint{3.156796in}{4.919700in}}%
\pgfpathlineto{\pgfqpoint{3.174390in}{4.897217in}}%
\pgfpathlineto{\pgfqpoint{3.191983in}{4.877285in}}%
\pgfpathlineto{\pgfqpoint{3.209577in}{4.859960in}}%
\pgfpathlineto{\pgfqpoint{3.227171in}{4.845224in}}%
\pgfpathlineto{\pgfqpoint{3.244765in}{4.833000in}}%
\pgfpathlineto{\pgfqpoint{3.262359in}{4.823163in}}%
\pgfpathlineto{\pgfqpoint{3.279953in}{4.815553in}}%
\pgfpathlineto{\pgfqpoint{3.297547in}{4.809986in}}%
\pgfpathlineto{\pgfqpoint{3.315140in}{4.806261in}}%
\pgfpathlineto{\pgfqpoint{3.332734in}{4.804168in}}%
\pgfpathlineto{\pgfqpoint{3.350328in}{4.803496in}}%
\pgfpathlineto{\pgfqpoint{3.385516in}{4.805599in}}%
\pgfpathlineto{\pgfqpoint{3.420704in}{4.811030in}}%
\pgfpathlineto{\pgfqpoint{3.473485in}{4.822597in}}%
\pgfpathlineto{\pgfqpoint{3.579048in}{4.847219in}}%
\pgfpathlineto{\pgfqpoint{3.631830in}{4.856922in}}%
\pgfpathlineto{\pgfqpoint{3.684611in}{4.863781in}}%
\pgfpathlineto{\pgfqpoint{3.719799in}{4.866504in}}%
\pgfpathlineto{\pgfqpoint{3.754987in}{4.867540in}}%
\pgfpathlineto{\pgfqpoint{3.790174in}{4.866676in}}%
\pgfpathlineto{\pgfqpoint{3.825362in}{4.863669in}}%
\pgfpathlineto{\pgfqpoint{3.860550in}{4.858264in}}%
\pgfpathlineto{\pgfqpoint{3.895738in}{4.850221in}}%
\pgfpathlineto{\pgfqpoint{3.930925in}{4.839353in}}%
\pgfpathlineto{\pgfqpoint{3.966113in}{4.825562in}}%
\pgfpathlineto{\pgfqpoint{4.001301in}{4.808878in}}%
\pgfpathlineto{\pgfqpoint{4.036488in}{4.789491in}}%
\pgfpathlineto{\pgfqpoint{4.071676in}{4.767773in}}%
\pgfpathlineto{\pgfqpoint{4.124458in}{4.732104in}}%
\pgfpathlineto{\pgfqpoint{4.212427in}{4.671479in}}%
\pgfpathlineto{\pgfqpoint{4.247615in}{4.649868in}}%
\pgfpathlineto{\pgfqpoint{4.282802in}{4.631480in}}%
\pgfpathlineto{\pgfqpoint{4.300396in}{4.623845in}}%
\pgfpathlineto{\pgfqpoint{4.317990in}{4.617428in}}%
\pgfpathlineto{\pgfqpoint{4.335584in}{4.612349in}}%
\pgfpathlineto{\pgfqpoint{4.353178in}{4.608719in}}%
\pgfpathlineto{\pgfqpoint{4.370772in}{4.606636in}}%
\pgfpathlineto{\pgfqpoint{4.388365in}{4.606182in}}%
\pgfpathlineto{\pgfqpoint{4.405959in}{4.607424in}}%
\pgfpathlineto{\pgfqpoint{4.423553in}{4.610409in}}%
\pgfpathlineto{\pgfqpoint{4.441147in}{4.615167in}}%
\pgfpathlineto{\pgfqpoint{4.458741in}{4.621705in}}%
\pgfpathlineto{\pgfqpoint{4.476335in}{4.630010in}}%
\pgfpathlineto{\pgfqpoint{4.493928in}{4.640047in}}%
\pgfpathlineto{\pgfqpoint{4.511522in}{4.651760in}}%
\pgfpathlineto{\pgfqpoint{4.529116in}{4.665069in}}%
\pgfpathlineto{\pgfqpoint{4.564304in}{4.696053in}}%
\pgfpathlineto{\pgfqpoint{4.599492in}{4.731958in}}%
\pgfpathlineto{\pgfqpoint{4.634679in}{4.771446in}}%
\pgfpathlineto{\pgfqpoint{4.740242in}{4.894973in}}%
\pgfpathlineto{\pgfqpoint{4.775430in}{4.931879in}}%
\pgfpathlineto{\pgfqpoint{4.810618in}{4.963748in}}%
\pgfpathlineto{\pgfqpoint{4.828212in}{4.977315in}}%
\pgfpathlineto{\pgfqpoint{4.845805in}{4.989079in}}%
\pgfpathlineto{\pgfqpoint{4.863399in}{4.998897in}}%
\pgfpathlineto{\pgfqpoint{4.880993in}{5.006647in}}%
\pgfpathlineto{\pgfqpoint{4.898587in}{5.012234in}}%
\pgfpathlineto{\pgfqpoint{4.916181in}{5.015583in}}%
\pgfpathlineto{\pgfqpoint{4.933775in}{5.016651in}}%
\pgfpathlineto{\pgfqpoint{4.951369in}{5.015419in}}%
\pgfpathlineto{\pgfqpoint{4.968962in}{5.011898in}}%
\pgfpathlineto{\pgfqpoint{4.986556in}{5.006127in}}%
\pgfpathlineto{\pgfqpoint{5.004150in}{4.998173in}}%
\pgfpathlineto{\pgfqpoint{5.021744in}{4.988130in}}%
\pgfpathlineto{\pgfqpoint{5.039338in}{4.976119in}}%
\pgfpathlineto{\pgfqpoint{5.056932in}{4.962287in}}%
\pgfpathlineto{\pgfqpoint{5.092119in}{4.929859in}}%
\pgfpathlineto{\pgfqpoint{5.127307in}{4.892453in}}%
\pgfpathlineto{\pgfqpoint{5.180089in}{4.831159in}}%
\pgfpathlineto{\pgfqpoint{5.232870in}{4.769878in}}%
\pgfpathlineto{\pgfqpoint{5.268058in}{4.732540in}}%
\pgfpathlineto{\pgfqpoint{5.303246in}{4.700317in}}%
\pgfpathlineto{\pgfqpoint{5.320839in}{4.686672in}}%
\pgfpathlineto{\pgfqpoint{5.338433in}{4.674926in}}%
\pgfpathlineto{\pgfqpoint{5.356027in}{4.665245in}}%
\pgfpathlineto{\pgfqpoint{5.373621in}{4.657768in}}%
\pgfpathlineto{\pgfqpoint{5.391215in}{4.652607in}}%
\pgfpathlineto{\pgfqpoint{5.408809in}{4.649850in}}%
\pgfpathlineto{\pgfqpoint{5.426403in}{4.649555in}}%
\pgfpathlineto{\pgfqpoint{5.443996in}{4.651751in}}%
\pgfpathlineto{\pgfqpoint{5.461590in}{4.656437in}}%
\pgfpathlineto{\pgfqpoint{5.479184in}{4.663587in}}%
\pgfpathlineto{\pgfqpoint{5.496778in}{4.673145in}}%
\pgfpathlineto{\pgfqpoint{5.514372in}{4.685027in}}%
\pgfpathlineto{\pgfqpoint{5.531966in}{4.699126in}}%
\pgfpathlineto{\pgfqpoint{5.549560in}{4.715311in}}%
\pgfpathlineto{\pgfqpoint{5.567153in}{4.733428in}}%
\pgfpathlineto{\pgfqpoint{5.602341in}{4.774755in}}%
\pgfpathlineto{\pgfqpoint{5.637529in}{4.821551in}}%
\pgfpathlineto{\pgfqpoint{5.690310in}{4.898197in}}%
\pgfpathlineto{\pgfqpoint{5.778280in}{5.028318in}}%
\pgfpathlineto{\pgfqpoint{5.813467in}{5.076431in}}%
\pgfpathlineto{\pgfqpoint{5.848655in}{5.120284in}}%
\pgfpathlineto{\pgfqpoint{5.883843in}{5.158892in}}%
\pgfpathlineto{\pgfqpoint{5.901437in}{5.175995in}}%
\pgfpathlineto{\pgfqpoint{5.919030in}{5.191540in}}%
\pgfpathlineto{\pgfqpoint{5.936624in}{5.205475in}}%
\pgfpathlineto{\pgfqpoint{5.954218in}{5.217764in}}%
\pgfpathlineto{\pgfqpoint{5.971812in}{5.228380in}}%
\pgfpathlineto{\pgfqpoint{5.989406in}{5.237306in}}%
\pgfpathlineto{\pgfqpoint{6.007000in}{5.244533in}}%
\pgfpathlineto{\pgfqpoint{6.024594in}{5.250054in}}%
\pgfpathlineto{\pgfqpoint{6.042187in}{5.253865in}}%
\pgfpathlineto{\pgfqpoint{6.059781in}{5.255961in}}%
\pgfpathlineto{\pgfqpoint{6.077375in}{5.256336in}}%
\pgfpathlineto{\pgfqpoint{6.094969in}{5.254979in}}%
\pgfpathlineto{\pgfqpoint{6.112563in}{5.251871in}}%
\pgfpathlineto{\pgfqpoint{6.130157in}{5.246990in}}%
\pgfpathlineto{\pgfqpoint{6.147751in}{5.240304in}}%
\pgfpathlineto{\pgfqpoint{6.165344in}{5.231775in}}%
\pgfpathlineto{\pgfqpoint{6.182938in}{5.221359in}}%
\pgfpathlineto{\pgfqpoint{6.200532in}{5.209011in}}%
\pgfpathlineto{\pgfqpoint{6.218126in}{5.194685in}}%
\pgfpathlineto{\pgfqpoint{6.235720in}{5.178340in}}%
\pgfpathlineto{\pgfqpoint{6.253314in}{5.159946in}}%
\pgfpathlineto{\pgfqpoint{6.270907in}{5.139492in}}%
\pgfpathlineto{\pgfqpoint{6.288501in}{5.116994in}}%
\pgfpathlineto{\pgfqpoint{6.323689in}{5.066133in}}%
\pgfpathlineto{\pgfqpoint{6.358877in}{5.008482in}}%
\pgfpathlineto{\pgfqpoint{6.376471in}{4.977794in}}%
\pgfpathlineto{\pgfqpoint{6.376471in}{4.977794in}}%
\pgfusepath{stroke}%
\end{pgfscope}%
\begin{pgfscope}%
\pgfpathrectangle{\pgfqpoint{2.000000in}{4.421053in}}{\pgfqpoint{4.376471in}{0.978947in}} %
\pgfusepath{clip}%
\pgfsetbuttcap%
\pgfsetroundjoin%
\pgfsetlinewidth{1.756562pt}%
\definecolor{currentstroke}{rgb}{1.000000,0.647059,0.000000}%
\pgfsetstrokecolor{currentstroke}%
\pgfsetdash{{6.000000pt}{6.000000pt}}{0.000000pt}%
\pgfpathmoveto{\pgfqpoint{2.875294in}{5.012214in}}%
\pgfpathlineto{\pgfqpoint{2.892888in}{5.088756in}}%
\pgfpathlineto{\pgfqpoint{2.910482in}{5.141372in}}%
\pgfpathlineto{\pgfqpoint{2.928076in}{5.173971in}}%
\pgfpathlineto{\pgfqpoint{2.945670in}{5.190064in}}%
\pgfpathlineto{\pgfqpoint{2.963263in}{5.192784in}}%
\pgfpathlineto{\pgfqpoint{2.980857in}{5.184906in}}%
\pgfpathlineto{\pgfqpoint{2.998451in}{5.168865in}}%
\pgfpathlineto{\pgfqpoint{3.016045in}{5.146780in}}%
\pgfpathlineto{\pgfqpoint{3.033639in}{5.120474in}}%
\pgfpathlineto{\pgfqpoint{3.068826in}{5.061153in}}%
\pgfpathlineto{\pgfqpoint{3.104014in}{5.000442in}}%
\pgfpathlineto{\pgfqpoint{3.139202in}{4.944588in}}%
\pgfpathlineto{\pgfqpoint{3.156796in}{4.919700in}}%
\pgfpathlineto{\pgfqpoint{3.174390in}{4.897217in}}%
\pgfpathlineto{\pgfqpoint{3.191983in}{4.877285in}}%
\pgfpathlineto{\pgfqpoint{3.209577in}{4.859959in}}%
\pgfpathlineto{\pgfqpoint{3.227171in}{4.845223in}}%
\pgfpathlineto{\pgfqpoint{3.244765in}{4.832999in}}%
\pgfpathlineto{\pgfqpoint{3.262359in}{4.823162in}}%
\pgfpathlineto{\pgfqpoint{3.279953in}{4.815553in}}%
\pgfpathlineto{\pgfqpoint{3.297547in}{4.809986in}}%
\pgfpathlineto{\pgfqpoint{3.315140in}{4.806260in}}%
\pgfpathlineto{\pgfqpoint{3.332734in}{4.804168in}}%
\pgfpathlineto{\pgfqpoint{3.350328in}{4.803496in}}%
\pgfpathlineto{\pgfqpoint{3.385516in}{4.805599in}}%
\pgfpathlineto{\pgfqpoint{3.420704in}{4.811030in}}%
\pgfpathlineto{\pgfqpoint{3.473485in}{4.822597in}}%
\pgfpathlineto{\pgfqpoint{3.579048in}{4.847220in}}%
\pgfpathlineto{\pgfqpoint{3.631830in}{4.856923in}}%
\pgfpathlineto{\pgfqpoint{3.684611in}{4.863783in}}%
\pgfpathlineto{\pgfqpoint{3.719799in}{4.866505in}}%
\pgfpathlineto{\pgfqpoint{3.754987in}{4.867541in}}%
\pgfpathlineto{\pgfqpoint{3.790174in}{4.866677in}}%
\pgfpathlineto{\pgfqpoint{3.825362in}{4.863670in}}%
\pgfpathlineto{\pgfqpoint{3.860550in}{4.858265in}}%
\pgfpathlineto{\pgfqpoint{3.895738in}{4.850222in}}%
\pgfpathlineto{\pgfqpoint{3.930925in}{4.839354in}}%
\pgfpathlineto{\pgfqpoint{3.966113in}{4.825562in}}%
\pgfpathlineto{\pgfqpoint{4.001301in}{4.808879in}}%
\pgfpathlineto{\pgfqpoint{4.036488in}{4.789492in}}%
\pgfpathlineto{\pgfqpoint{4.071676in}{4.767773in}}%
\pgfpathlineto{\pgfqpoint{4.124458in}{4.732104in}}%
\pgfpathlineto{\pgfqpoint{4.212427in}{4.671479in}}%
\pgfpathlineto{\pgfqpoint{4.247615in}{4.649868in}}%
\pgfpathlineto{\pgfqpoint{4.282802in}{4.631480in}}%
\pgfpathlineto{\pgfqpoint{4.300396in}{4.623846in}}%
\pgfpathlineto{\pgfqpoint{4.317990in}{4.617428in}}%
\pgfpathlineto{\pgfqpoint{4.335584in}{4.612349in}}%
\pgfpathlineto{\pgfqpoint{4.353178in}{4.608719in}}%
\pgfpathlineto{\pgfqpoint{4.370772in}{4.606636in}}%
\pgfpathlineto{\pgfqpoint{4.388365in}{4.606182in}}%
\pgfpathlineto{\pgfqpoint{4.405959in}{4.607424in}}%
\pgfpathlineto{\pgfqpoint{4.423553in}{4.610409in}}%
\pgfpathlineto{\pgfqpoint{4.441147in}{4.615167in}}%
\pgfpathlineto{\pgfqpoint{4.458741in}{4.621705in}}%
\pgfpathlineto{\pgfqpoint{4.476335in}{4.630010in}}%
\pgfpathlineto{\pgfqpoint{4.493928in}{4.640047in}}%
\pgfpathlineto{\pgfqpoint{4.511522in}{4.651760in}}%
\pgfpathlineto{\pgfqpoint{4.529116in}{4.665069in}}%
\pgfpathlineto{\pgfqpoint{4.564304in}{4.696053in}}%
\pgfpathlineto{\pgfqpoint{4.599492in}{4.731958in}}%
\pgfpathlineto{\pgfqpoint{4.634679in}{4.771446in}}%
\pgfpathlineto{\pgfqpoint{4.740242in}{4.894973in}}%
\pgfpathlineto{\pgfqpoint{4.775430in}{4.931879in}}%
\pgfpathlineto{\pgfqpoint{4.810618in}{4.963749in}}%
\pgfpathlineto{\pgfqpoint{4.828212in}{4.977314in}}%
\pgfpathlineto{\pgfqpoint{4.845805in}{4.989078in}}%
\pgfpathlineto{\pgfqpoint{4.863399in}{4.998897in}}%
\pgfpathlineto{\pgfqpoint{4.880993in}{5.006647in}}%
\pgfpathlineto{\pgfqpoint{4.898587in}{5.012233in}}%
\pgfpathlineto{\pgfqpoint{4.916181in}{5.015583in}}%
\pgfpathlineto{\pgfqpoint{4.933775in}{5.016651in}}%
\pgfpathlineto{\pgfqpoint{4.951369in}{5.015419in}}%
\pgfpathlineto{\pgfqpoint{4.968962in}{5.011898in}}%
\pgfpathlineto{\pgfqpoint{4.986556in}{5.006127in}}%
\pgfpathlineto{\pgfqpoint{5.004150in}{4.998173in}}%
\pgfpathlineto{\pgfqpoint{5.021744in}{4.988130in}}%
\pgfpathlineto{\pgfqpoint{5.039338in}{4.976119in}}%
\pgfpathlineto{\pgfqpoint{5.056932in}{4.962287in}}%
\pgfpathlineto{\pgfqpoint{5.092119in}{4.929859in}}%
\pgfpathlineto{\pgfqpoint{5.127307in}{4.892453in}}%
\pgfpathlineto{\pgfqpoint{5.180089in}{4.831159in}}%
\pgfpathlineto{\pgfqpoint{5.232870in}{4.769878in}}%
\pgfpathlineto{\pgfqpoint{5.268058in}{4.732540in}}%
\pgfpathlineto{\pgfqpoint{5.303246in}{4.700317in}}%
\pgfpathlineto{\pgfqpoint{5.320839in}{4.686671in}}%
\pgfpathlineto{\pgfqpoint{5.338433in}{4.674926in}}%
\pgfpathlineto{\pgfqpoint{5.356027in}{4.665245in}}%
\pgfpathlineto{\pgfqpoint{5.373621in}{4.657767in}}%
\pgfpathlineto{\pgfqpoint{5.391215in}{4.652607in}}%
\pgfpathlineto{\pgfqpoint{5.408809in}{4.649850in}}%
\pgfpathlineto{\pgfqpoint{5.426403in}{4.649555in}}%
\pgfpathlineto{\pgfqpoint{5.443996in}{4.651750in}}%
\pgfpathlineto{\pgfqpoint{5.461590in}{4.656437in}}%
\pgfpathlineto{\pgfqpoint{5.479184in}{4.663587in}}%
\pgfpathlineto{\pgfqpoint{5.496778in}{4.673145in}}%
\pgfpathlineto{\pgfqpoint{5.514372in}{4.685027in}}%
\pgfpathlineto{\pgfqpoint{5.531966in}{4.699126in}}%
\pgfpathlineto{\pgfqpoint{5.549560in}{4.715310in}}%
\pgfpathlineto{\pgfqpoint{5.567153in}{4.733428in}}%
\pgfpathlineto{\pgfqpoint{5.602341in}{4.774754in}}%
\pgfpathlineto{\pgfqpoint{5.637529in}{4.821551in}}%
\pgfpathlineto{\pgfqpoint{5.690310in}{4.898197in}}%
\pgfpathlineto{\pgfqpoint{5.778280in}{5.028318in}}%
\pgfpathlineto{\pgfqpoint{5.813467in}{5.076431in}}%
\pgfpathlineto{\pgfqpoint{5.848655in}{5.120284in}}%
\pgfpathlineto{\pgfqpoint{5.883843in}{5.158892in}}%
\pgfpathlineto{\pgfqpoint{5.901437in}{5.175995in}}%
\pgfpathlineto{\pgfqpoint{5.919030in}{5.191540in}}%
\pgfpathlineto{\pgfqpoint{5.936624in}{5.205475in}}%
\pgfpathlineto{\pgfqpoint{5.954218in}{5.217764in}}%
\pgfpathlineto{\pgfqpoint{5.971812in}{5.228380in}}%
\pgfpathlineto{\pgfqpoint{5.989406in}{5.237306in}}%
\pgfpathlineto{\pgfqpoint{6.007000in}{5.244533in}}%
\pgfpathlineto{\pgfqpoint{6.024594in}{5.250054in}}%
\pgfpathlineto{\pgfqpoint{6.042187in}{5.253865in}}%
\pgfpathlineto{\pgfqpoint{6.059781in}{5.255961in}}%
\pgfpathlineto{\pgfqpoint{6.077375in}{5.256336in}}%
\pgfpathlineto{\pgfqpoint{6.094969in}{5.254979in}}%
\pgfpathlineto{\pgfqpoint{6.112563in}{5.251872in}}%
\pgfpathlineto{\pgfqpoint{6.130157in}{5.246990in}}%
\pgfpathlineto{\pgfqpoint{6.147751in}{5.240304in}}%
\pgfpathlineto{\pgfqpoint{6.165344in}{5.231775in}}%
\pgfpathlineto{\pgfqpoint{6.182938in}{5.221359in}}%
\pgfpathlineto{\pgfqpoint{6.200532in}{5.209011in}}%
\pgfpathlineto{\pgfqpoint{6.218126in}{5.194685in}}%
\pgfpathlineto{\pgfqpoint{6.235720in}{5.178340in}}%
\pgfpathlineto{\pgfqpoint{6.253314in}{5.159946in}}%
\pgfpathlineto{\pgfqpoint{6.270907in}{5.139492in}}%
\pgfpathlineto{\pgfqpoint{6.288501in}{5.116994in}}%
\pgfpathlineto{\pgfqpoint{6.323689in}{5.066133in}}%
\pgfpathlineto{\pgfqpoint{6.358877in}{5.008482in}}%
\pgfpathlineto{\pgfqpoint{6.376471in}{4.977795in}}%
\pgfpathlineto{\pgfqpoint{6.376471in}{4.977795in}}%
\pgfusepath{stroke}%
\end{pgfscope}%
\begin{pgfscope}%
\pgfsetrectcap%
\pgfsetmiterjoin%
\pgfsetlinewidth{1.003750pt}%
\definecolor{currentstroke}{rgb}{0.800000,0.800000,0.800000}%
\pgfsetstrokecolor{currentstroke}%
\pgfsetdash{}{0pt}%
\pgfpathmoveto{\pgfqpoint{2.000000in}{4.421053in}}%
\pgfpathlineto{\pgfqpoint{2.000000in}{5.400000in}}%
\pgfusepath{stroke}%
\end{pgfscope}%
\begin{pgfscope}%
\pgfsetrectcap%
\pgfsetmiterjoin%
\pgfsetlinewidth{1.003750pt}%
\definecolor{currentstroke}{rgb}{0.800000,0.800000,0.800000}%
\pgfsetstrokecolor{currentstroke}%
\pgfsetdash{}{0pt}%
\pgfpathmoveto{\pgfqpoint{6.376471in}{4.421053in}}%
\pgfpathlineto{\pgfqpoint{6.376471in}{5.400000in}}%
\pgfusepath{stroke}%
\end{pgfscope}%
\begin{pgfscope}%
\pgfsetrectcap%
\pgfsetmiterjoin%
\pgfsetlinewidth{1.003750pt}%
\definecolor{currentstroke}{rgb}{0.800000,0.800000,0.800000}%
\pgfsetstrokecolor{currentstroke}%
\pgfsetdash{}{0pt}%
\pgfpathmoveto{\pgfqpoint{2.000000in}{5.400000in}}%
\pgfpathlineto{\pgfqpoint{6.376471in}{5.400000in}}%
\pgfusepath{stroke}%
\end{pgfscope}%
\begin{pgfscope}%
\pgfsetrectcap%
\pgfsetmiterjoin%
\pgfsetlinewidth{1.003750pt}%
\definecolor{currentstroke}{rgb}{0.800000,0.800000,0.800000}%
\pgfsetstrokecolor{currentstroke}%
\pgfsetdash{}{0pt}%
\pgfpathmoveto{\pgfqpoint{2.000000in}{4.421053in}}%
\pgfpathlineto{\pgfqpoint{6.376471in}{4.421053in}}%
\pgfusepath{stroke}%
\end{pgfscope}%
\begin{pgfscope}%
\pgfsetroundcap%
\pgfsetroundjoin%
\pgfsetlinewidth{1.756562pt}%
\definecolor{currentstroke}{rgb}{0.298039,0.447059,0.690196}%
\pgfsetstrokecolor{currentstroke}%
\pgfsetdash{}{0pt}%
\pgfpathmoveto{\pgfqpoint{2.125000in}{5.220056in}}%
\pgfpathlineto{\pgfqpoint{2.402778in}{5.220056in}}%
\pgfusepath{stroke}%
\end{pgfscope}%
\begin{pgfscope}%
\definecolor{textcolor}{rgb}{0.150000,0.150000,0.150000}%
\pgfsetstrokecolor{textcolor}%
\pgfsetfillcolor{textcolor}%
\pgftext[x=2.513889in,y=5.171445in,left,base]{\color{textcolor}\sffamily\fontsize{10.000000}{12.000000}\selectfont \(\displaystyle \widetilde{\Phi}^* \theta\)}%
\end{pgfscope}%
\begin{pgfscope}%
\pgfsetbuttcap%
\pgfsetroundjoin%
\pgfsetlinewidth{1.756562pt}%
\definecolor{currentstroke}{rgb}{1.000000,0.647059,0.000000}%
\pgfsetstrokecolor{currentstroke}%
\pgfsetdash{{6.000000pt}{6.000000pt}}{0.000000pt}%
\pgfpathmoveto{\pgfqpoint{2.125000in}{5.015195in}}%
\pgfpathlineto{\pgfqpoint{2.402778in}{5.015195in}}%
\pgfusepath{stroke}%
\end{pgfscope}%
\begin{pgfscope}%
\definecolor{textcolor}{rgb}{0.150000,0.150000,0.150000}%
\pgfsetstrokecolor{textcolor}%
\pgfsetfillcolor{textcolor}%
\pgftext[x=2.513889in,y=4.966584in,left,base]{\color{textcolor}\sffamily\fontsize{10.000000}{12.000000}\selectfont \(\displaystyle \widetilde{K}u\)}%
\end{pgfscope}%
\begin{pgfscope}%
\pgfsetbuttcap%
\pgfsetroundjoin%
\definecolor{currentfill}{rgb}{1.000000,0.000000,0.000000}%
\pgfsetfillcolor{currentfill}%
\pgfsetlinewidth{2.007500pt}%
\definecolor{currentstroke}{rgb}{1.000000,0.000000,0.000000}%
\pgfsetstrokecolor{currentstroke}%
\pgfsetdash{}{0pt}%
\pgfpathmoveto{\pgfqpoint{2.232832in}{4.806577in}}%
\pgfpathlineto{\pgfqpoint{2.294945in}{4.806577in}}%
\pgfpathmoveto{\pgfqpoint{2.263889in}{4.775521in}}%
\pgfpathlineto{\pgfqpoint{2.263889in}{4.837634in}}%
\pgfusepath{stroke,fill}%
\end{pgfscope}%
\begin{pgfscope}%
\pgfsetbuttcap%
\pgfsetroundjoin%
\definecolor{currentfill}{rgb}{1.000000,0.000000,0.000000}%
\pgfsetfillcolor{currentfill}%
\pgfsetlinewidth{2.007500pt}%
\definecolor{currentstroke}{rgb}{1.000000,0.000000,0.000000}%
\pgfsetstrokecolor{currentstroke}%
\pgfsetdash{}{0pt}%
\pgfpathmoveto{\pgfqpoint{2.232832in}{4.806577in}}%
\pgfpathlineto{\pgfqpoint{2.294945in}{4.806577in}}%
\pgfpathmoveto{\pgfqpoint{2.263889in}{4.775521in}}%
\pgfpathlineto{\pgfqpoint{2.263889in}{4.837634in}}%
\pgfusepath{stroke,fill}%
\end{pgfscope}%
\begin{pgfscope}%
\pgfsetbuttcap%
\pgfsetroundjoin%
\definecolor{currentfill}{rgb}{1.000000,0.000000,0.000000}%
\pgfsetfillcolor{currentfill}%
\pgfsetlinewidth{2.007500pt}%
\definecolor{currentstroke}{rgb}{1.000000,0.000000,0.000000}%
\pgfsetstrokecolor{currentstroke}%
\pgfsetdash{}{0pt}%
\pgfpathmoveto{\pgfqpoint{2.232832in}{4.806577in}}%
\pgfpathlineto{\pgfqpoint{2.294945in}{4.806577in}}%
\pgfpathmoveto{\pgfqpoint{2.263889in}{4.775521in}}%
\pgfpathlineto{\pgfqpoint{2.263889in}{4.837634in}}%
\pgfusepath{stroke,fill}%
\end{pgfscope}%
\begin{pgfscope}%
\definecolor{textcolor}{rgb}{0.150000,0.150000,0.150000}%
\pgfsetstrokecolor{textcolor}%
\pgfsetfillcolor{textcolor}%
\pgftext[x=2.513889in,y=4.770119in,left,base]{\color{textcolor}\sffamily\fontsize{10.000000}{12.000000}\selectfont train}%
\end{pgfscope}%
\begin{pgfscope}%
\pgfsetbuttcap%
\pgfsetroundjoin%
\definecolor{currentfill}{rgb}{0.000000,0.000000,0.000000}%
\pgfsetfillcolor{currentfill}%
\pgfsetlinewidth{0.301125pt}%
\definecolor{currentstroke}{rgb}{0.000000,0.000000,0.000000}%
\pgfsetstrokecolor{currentstroke}%
\pgfsetdash{}{0pt}%
\pgfpathmoveto{\pgfqpoint{2.263889in}{4.594584in}}%
\pgfpathcurveto{\pgfqpoint{2.268007in}{4.594584in}}{\pgfqpoint{2.271957in}{4.596220in}}{\pgfqpoint{2.274869in}{4.599132in}}%
\pgfpathcurveto{\pgfqpoint{2.277781in}{4.602044in}}{\pgfqpoint{2.279417in}{4.605994in}}{\pgfqpoint{2.279417in}{4.610112in}}%
\pgfpathcurveto{\pgfqpoint{2.279417in}{4.614230in}}{\pgfqpoint{2.277781in}{4.618180in}}{\pgfqpoint{2.274869in}{4.621092in}}%
\pgfpathcurveto{\pgfqpoint{2.271957in}{4.624004in}}{\pgfqpoint{2.268007in}{4.625640in}}{\pgfqpoint{2.263889in}{4.625640in}}%
\pgfpathcurveto{\pgfqpoint{2.259771in}{4.625640in}}{\pgfqpoint{2.255821in}{4.624004in}}{\pgfqpoint{2.252909in}{4.621092in}}%
\pgfpathcurveto{\pgfqpoint{2.249997in}{4.618180in}}{\pgfqpoint{2.248361in}{4.614230in}}{\pgfqpoint{2.248361in}{4.610112in}}%
\pgfpathcurveto{\pgfqpoint{2.248361in}{4.605994in}}{\pgfqpoint{2.249997in}{4.602044in}}{\pgfqpoint{2.252909in}{4.599132in}}%
\pgfpathcurveto{\pgfqpoint{2.255821in}{4.596220in}}{\pgfqpoint{2.259771in}{4.594584in}}{\pgfqpoint{2.263889in}{4.594584in}}%
\pgfpathclose%
\pgfusepath{stroke,fill}%
\end{pgfscope}%
\begin{pgfscope}%
\pgfsetbuttcap%
\pgfsetroundjoin%
\definecolor{currentfill}{rgb}{0.000000,0.000000,0.000000}%
\pgfsetfillcolor{currentfill}%
\pgfsetlinewidth{0.301125pt}%
\definecolor{currentstroke}{rgb}{0.000000,0.000000,0.000000}%
\pgfsetstrokecolor{currentstroke}%
\pgfsetdash{}{0pt}%
\pgfpathmoveto{\pgfqpoint{2.263889in}{4.594584in}}%
\pgfpathcurveto{\pgfqpoint{2.268007in}{4.594584in}}{\pgfqpoint{2.271957in}{4.596220in}}{\pgfqpoint{2.274869in}{4.599132in}}%
\pgfpathcurveto{\pgfqpoint{2.277781in}{4.602044in}}{\pgfqpoint{2.279417in}{4.605994in}}{\pgfqpoint{2.279417in}{4.610112in}}%
\pgfpathcurveto{\pgfqpoint{2.279417in}{4.614230in}}{\pgfqpoint{2.277781in}{4.618180in}}{\pgfqpoint{2.274869in}{4.621092in}}%
\pgfpathcurveto{\pgfqpoint{2.271957in}{4.624004in}}{\pgfqpoint{2.268007in}{4.625640in}}{\pgfqpoint{2.263889in}{4.625640in}}%
\pgfpathcurveto{\pgfqpoint{2.259771in}{4.625640in}}{\pgfqpoint{2.255821in}{4.624004in}}{\pgfqpoint{2.252909in}{4.621092in}}%
\pgfpathcurveto{\pgfqpoint{2.249997in}{4.618180in}}{\pgfqpoint{2.248361in}{4.614230in}}{\pgfqpoint{2.248361in}{4.610112in}}%
\pgfpathcurveto{\pgfqpoint{2.248361in}{4.605994in}}{\pgfqpoint{2.249997in}{4.602044in}}{\pgfqpoint{2.252909in}{4.599132in}}%
\pgfpathcurveto{\pgfqpoint{2.255821in}{4.596220in}}{\pgfqpoint{2.259771in}{4.594584in}}{\pgfqpoint{2.263889in}{4.594584in}}%
\pgfpathclose%
\pgfusepath{stroke,fill}%
\end{pgfscope}%
\begin{pgfscope}%
\pgfsetbuttcap%
\pgfsetroundjoin%
\definecolor{currentfill}{rgb}{0.000000,0.000000,0.000000}%
\pgfsetfillcolor{currentfill}%
\pgfsetlinewidth{0.301125pt}%
\definecolor{currentstroke}{rgb}{0.000000,0.000000,0.000000}%
\pgfsetstrokecolor{currentstroke}%
\pgfsetdash{}{0pt}%
\pgfpathmoveto{\pgfqpoint{2.263889in}{4.594584in}}%
\pgfpathcurveto{\pgfqpoint{2.268007in}{4.594584in}}{\pgfqpoint{2.271957in}{4.596220in}}{\pgfqpoint{2.274869in}{4.599132in}}%
\pgfpathcurveto{\pgfqpoint{2.277781in}{4.602044in}}{\pgfqpoint{2.279417in}{4.605994in}}{\pgfqpoint{2.279417in}{4.610112in}}%
\pgfpathcurveto{\pgfqpoint{2.279417in}{4.614230in}}{\pgfqpoint{2.277781in}{4.618180in}}{\pgfqpoint{2.274869in}{4.621092in}}%
\pgfpathcurveto{\pgfqpoint{2.271957in}{4.624004in}}{\pgfqpoint{2.268007in}{4.625640in}}{\pgfqpoint{2.263889in}{4.625640in}}%
\pgfpathcurveto{\pgfqpoint{2.259771in}{4.625640in}}{\pgfqpoint{2.255821in}{4.624004in}}{\pgfqpoint{2.252909in}{4.621092in}}%
\pgfpathcurveto{\pgfqpoint{2.249997in}{4.618180in}}{\pgfqpoint{2.248361in}{4.614230in}}{\pgfqpoint{2.248361in}{4.610112in}}%
\pgfpathcurveto{\pgfqpoint{2.248361in}{4.605994in}}{\pgfqpoint{2.249997in}{4.602044in}}{\pgfqpoint{2.252909in}{4.599132in}}%
\pgfpathcurveto{\pgfqpoint{2.255821in}{4.596220in}}{\pgfqpoint{2.259771in}{4.594584in}}{\pgfqpoint{2.263889in}{4.594584in}}%
\pgfpathclose%
\pgfusepath{stroke,fill}%
\end{pgfscope}%
\begin{pgfscope}%
\definecolor{textcolor}{rgb}{0.150000,0.150000,0.150000}%
\pgfsetstrokecolor{textcolor}%
\pgfsetfillcolor{textcolor}%
\pgftext[x=2.513889in,y=4.573654in,left,base]{\color{textcolor}\sffamily\fontsize{10.000000}{12.000000}\selectfont test}%
\end{pgfscope}%
\begin{pgfscope}%
\pgfsetbuttcap%
\pgfsetmiterjoin%
\definecolor{currentfill}{rgb}{1.000000,1.000000,1.000000}%
\pgfsetfillcolor{currentfill}%
\pgfsetlinewidth{0.000000pt}%
\definecolor{currentstroke}{rgb}{0.000000,0.000000,0.000000}%
\pgfsetstrokecolor{currentstroke}%
\pgfsetstrokeopacity{0.000000}%
\pgfsetdash{}{0pt}%
\pgfpathmoveto{\pgfqpoint{7.105882in}{4.421053in}}%
\pgfpathlineto{\pgfqpoint{11.482353in}{4.421053in}}%
\pgfpathlineto{\pgfqpoint{11.482353in}{5.400000in}}%
\pgfpathlineto{\pgfqpoint{7.105882in}{5.400000in}}%
\pgfpathclose%
\pgfusepath{fill}%
\end{pgfscope}%
\begin{pgfscope}%
\pgfpathrectangle{\pgfqpoint{7.105882in}{4.421053in}}{\pgfqpoint{4.376471in}{0.978947in}} %
\pgfusepath{clip}%
\pgfsetroundcap%
\pgfsetroundjoin%
\pgfsetlinewidth{1.003750pt}%
\definecolor{currentstroke}{rgb}{0.800000,0.800000,0.800000}%
\pgfsetstrokecolor{currentstroke}%
\pgfsetdash{}{0pt}%
\pgfpathmoveto{\pgfqpoint{7.105882in}{4.421053in}}%
\pgfpathlineto{\pgfqpoint{7.105882in}{5.400000in}}%
\pgfusepath{stroke}%
\end{pgfscope}%
\begin{pgfscope}%
\pgfpathrectangle{\pgfqpoint{7.105882in}{4.421053in}}{\pgfqpoint{4.376471in}{0.978947in}} %
\pgfusepath{clip}%
\pgfsetroundcap%
\pgfsetroundjoin%
\pgfsetlinewidth{1.003750pt}%
\definecolor{currentstroke}{rgb}{0.800000,0.800000,0.800000}%
\pgfsetstrokecolor{currentstroke}%
\pgfsetdash{}{0pt}%
\pgfpathmoveto{\pgfqpoint{7.981176in}{4.421053in}}%
\pgfpathlineto{\pgfqpoint{7.981176in}{5.400000in}}%
\pgfusepath{stroke}%
\end{pgfscope}%
\begin{pgfscope}%
\pgfpathrectangle{\pgfqpoint{7.105882in}{4.421053in}}{\pgfqpoint{4.376471in}{0.978947in}} %
\pgfusepath{clip}%
\pgfsetroundcap%
\pgfsetroundjoin%
\pgfsetlinewidth{1.003750pt}%
\definecolor{currentstroke}{rgb}{0.800000,0.800000,0.800000}%
\pgfsetstrokecolor{currentstroke}%
\pgfsetdash{}{0pt}%
\pgfpathmoveto{\pgfqpoint{8.856471in}{4.421053in}}%
\pgfpathlineto{\pgfqpoint{8.856471in}{5.400000in}}%
\pgfusepath{stroke}%
\end{pgfscope}%
\begin{pgfscope}%
\pgfpathrectangle{\pgfqpoint{7.105882in}{4.421053in}}{\pgfqpoint{4.376471in}{0.978947in}} %
\pgfusepath{clip}%
\pgfsetroundcap%
\pgfsetroundjoin%
\pgfsetlinewidth{1.003750pt}%
\definecolor{currentstroke}{rgb}{0.800000,0.800000,0.800000}%
\pgfsetstrokecolor{currentstroke}%
\pgfsetdash{}{0pt}%
\pgfpathmoveto{\pgfqpoint{9.731765in}{4.421053in}}%
\pgfpathlineto{\pgfqpoint{9.731765in}{5.400000in}}%
\pgfusepath{stroke}%
\end{pgfscope}%
\begin{pgfscope}%
\pgfpathrectangle{\pgfqpoint{7.105882in}{4.421053in}}{\pgfqpoint{4.376471in}{0.978947in}} %
\pgfusepath{clip}%
\pgfsetroundcap%
\pgfsetroundjoin%
\pgfsetlinewidth{1.003750pt}%
\definecolor{currentstroke}{rgb}{0.800000,0.800000,0.800000}%
\pgfsetstrokecolor{currentstroke}%
\pgfsetdash{}{0pt}%
\pgfpathmoveto{\pgfqpoint{10.607059in}{4.421053in}}%
\pgfpathlineto{\pgfqpoint{10.607059in}{5.400000in}}%
\pgfusepath{stroke}%
\end{pgfscope}%
\begin{pgfscope}%
\pgfpathrectangle{\pgfqpoint{7.105882in}{4.421053in}}{\pgfqpoint{4.376471in}{0.978947in}} %
\pgfusepath{clip}%
\pgfsetroundcap%
\pgfsetroundjoin%
\pgfsetlinewidth{1.003750pt}%
\definecolor{currentstroke}{rgb}{0.800000,0.800000,0.800000}%
\pgfsetstrokecolor{currentstroke}%
\pgfsetdash{}{0pt}%
\pgfpathmoveto{\pgfqpoint{11.482353in}{4.421053in}}%
\pgfpathlineto{\pgfqpoint{11.482353in}{5.400000in}}%
\pgfusepath{stroke}%
\end{pgfscope}%
\begin{pgfscope}%
\pgfpathrectangle{\pgfqpoint{7.105882in}{4.421053in}}{\pgfqpoint{4.376471in}{0.978947in}} %
\pgfusepath{clip}%
\pgfsetroundcap%
\pgfsetroundjoin%
\pgfsetlinewidth{1.003750pt}%
\definecolor{currentstroke}{rgb}{0.800000,0.800000,0.800000}%
\pgfsetstrokecolor{currentstroke}%
\pgfsetdash{}{0pt}%
\pgfpathmoveto{\pgfqpoint{7.105882in}{4.584211in}}%
\pgfpathlineto{\pgfqpoint{11.482353in}{4.584211in}}%
\pgfusepath{stroke}%
\end{pgfscope}%
\begin{pgfscope}%
\definecolor{textcolor}{rgb}{0.150000,0.150000,0.150000}%
\pgfsetstrokecolor{textcolor}%
\pgfsetfillcolor{textcolor}%
\pgftext[x=7.008660in,y=4.584211in,right,]{\color{textcolor}\sffamily\fontsize{10.000000}{12.000000}\selectfont \(\displaystyle -1\)}%
\end{pgfscope}%
\begin{pgfscope}%
\pgfpathrectangle{\pgfqpoint{7.105882in}{4.421053in}}{\pgfqpoint{4.376471in}{0.978947in}} %
\pgfusepath{clip}%
\pgfsetroundcap%
\pgfsetroundjoin%
\pgfsetlinewidth{1.003750pt}%
\definecolor{currentstroke}{rgb}{0.800000,0.800000,0.800000}%
\pgfsetstrokecolor{currentstroke}%
\pgfsetdash{}{0pt}%
\pgfpathmoveto{\pgfqpoint{7.105882in}{4.788158in}}%
\pgfpathlineto{\pgfqpoint{11.482353in}{4.788158in}}%
\pgfusepath{stroke}%
\end{pgfscope}%
\begin{pgfscope}%
\definecolor{textcolor}{rgb}{0.150000,0.150000,0.150000}%
\pgfsetstrokecolor{textcolor}%
\pgfsetfillcolor{textcolor}%
\pgftext[x=7.008660in,y=4.788158in,right,]{\color{textcolor}\sffamily\fontsize{10.000000}{12.000000}\selectfont \(\displaystyle 0\)}%
\end{pgfscope}%
\begin{pgfscope}%
\pgfpathrectangle{\pgfqpoint{7.105882in}{4.421053in}}{\pgfqpoint{4.376471in}{0.978947in}} %
\pgfusepath{clip}%
\pgfsetroundcap%
\pgfsetroundjoin%
\pgfsetlinewidth{1.003750pt}%
\definecolor{currentstroke}{rgb}{0.800000,0.800000,0.800000}%
\pgfsetstrokecolor{currentstroke}%
\pgfsetdash{}{0pt}%
\pgfpathmoveto{\pgfqpoint{7.105882in}{4.992105in}}%
\pgfpathlineto{\pgfqpoint{11.482353in}{4.992105in}}%
\pgfusepath{stroke}%
\end{pgfscope}%
\begin{pgfscope}%
\definecolor{textcolor}{rgb}{0.150000,0.150000,0.150000}%
\pgfsetstrokecolor{textcolor}%
\pgfsetfillcolor{textcolor}%
\pgftext[x=7.008660in,y=4.992105in,right,]{\color{textcolor}\sffamily\fontsize{10.000000}{12.000000}\selectfont \(\displaystyle 1\)}%
\end{pgfscope}%
\begin{pgfscope}%
\pgfpathrectangle{\pgfqpoint{7.105882in}{4.421053in}}{\pgfqpoint{4.376471in}{0.978947in}} %
\pgfusepath{clip}%
\pgfsetroundcap%
\pgfsetroundjoin%
\pgfsetlinewidth{1.003750pt}%
\definecolor{currentstroke}{rgb}{0.800000,0.800000,0.800000}%
\pgfsetstrokecolor{currentstroke}%
\pgfsetdash{}{0pt}%
\pgfpathmoveto{\pgfqpoint{7.105882in}{5.196053in}}%
\pgfpathlineto{\pgfqpoint{11.482353in}{5.196053in}}%
\pgfusepath{stroke}%
\end{pgfscope}%
\begin{pgfscope}%
\definecolor{textcolor}{rgb}{0.150000,0.150000,0.150000}%
\pgfsetstrokecolor{textcolor}%
\pgfsetfillcolor{textcolor}%
\pgftext[x=7.008660in,y=5.196053in,right,]{\color{textcolor}\sffamily\fontsize{10.000000}{12.000000}\selectfont \(\displaystyle 2\)}%
\end{pgfscope}%
\begin{pgfscope}%
\pgfpathrectangle{\pgfqpoint{7.105882in}{4.421053in}}{\pgfqpoint{4.376471in}{0.978947in}} %
\pgfusepath{clip}%
\pgfsetroundcap%
\pgfsetroundjoin%
\pgfsetlinewidth{1.003750pt}%
\definecolor{currentstroke}{rgb}{0.800000,0.800000,0.800000}%
\pgfsetstrokecolor{currentstroke}%
\pgfsetdash{}{0pt}%
\pgfpathmoveto{\pgfqpoint{7.105882in}{5.400000in}}%
\pgfpathlineto{\pgfqpoint{11.482353in}{5.400000in}}%
\pgfusepath{stroke}%
\end{pgfscope}%
\begin{pgfscope}%
\definecolor{textcolor}{rgb}{0.150000,0.150000,0.150000}%
\pgfsetstrokecolor{textcolor}%
\pgfsetfillcolor{textcolor}%
\pgftext[x=7.008660in,y=5.400000in,right,]{\color{textcolor}\sffamily\fontsize{10.000000}{12.000000}\selectfont \(\displaystyle 3\)}%
\end{pgfscope}%
\begin{pgfscope}%
\pgfpathrectangle{\pgfqpoint{7.105882in}{4.421053in}}{\pgfqpoint{4.376471in}{0.978947in}} %
\pgfusepath{clip}%
\pgfsetbuttcap%
\pgfsetroundjoin%
\definecolor{currentfill}{rgb}{1.000000,0.000000,0.000000}%
\pgfsetfillcolor{currentfill}%
\pgfsetlinewidth{2.007500pt}%
\definecolor{currentstroke}{rgb}{1.000000,0.000000,0.000000}%
\pgfsetstrokecolor{currentstroke}%
\pgfsetdash{}{0pt}%
\pgfpathmoveto{\pgfqpoint{9.871613in}{4.978628in}}%
\pgfpathlineto{\pgfqpoint{9.933726in}{4.978628in}}%
\pgfpathmoveto{\pgfqpoint{9.902669in}{4.947572in}}%
\pgfpathlineto{\pgfqpoint{9.902669in}{5.009685in}}%
\pgfusepath{stroke,fill}%
\end{pgfscope}%
\begin{pgfscope}%
\pgfpathrectangle{\pgfqpoint{7.105882in}{4.421053in}}{\pgfqpoint{4.376471in}{0.978947in}} %
\pgfusepath{clip}%
\pgfsetbuttcap%
\pgfsetroundjoin%
\definecolor{currentfill}{rgb}{1.000000,0.000000,0.000000}%
\pgfsetfillcolor{currentfill}%
\pgfsetlinewidth{2.007500pt}%
\definecolor{currentstroke}{rgb}{1.000000,0.000000,0.000000}%
\pgfsetstrokecolor{currentstroke}%
\pgfsetdash{}{0pt}%
\pgfpathmoveto{\pgfqpoint{10.454124in}{4.661416in}}%
\pgfpathlineto{\pgfqpoint{10.516237in}{4.661416in}}%
\pgfpathmoveto{\pgfqpoint{10.485181in}{4.630360in}}%
\pgfpathlineto{\pgfqpoint{10.485181in}{4.692473in}}%
\pgfusepath{stroke,fill}%
\end{pgfscope}%
\begin{pgfscope}%
\pgfpathrectangle{\pgfqpoint{7.105882in}{4.421053in}}{\pgfqpoint{4.376471in}{0.978947in}} %
\pgfusepath{clip}%
\pgfsetbuttcap%
\pgfsetroundjoin%
\definecolor{currentfill}{rgb}{1.000000,0.000000,0.000000}%
\pgfsetfillcolor{currentfill}%
\pgfsetlinewidth{2.007500pt}%
\definecolor{currentstroke}{rgb}{1.000000,0.000000,0.000000}%
\pgfsetstrokecolor{currentstroke}%
\pgfsetdash{}{0pt}%
\pgfpathmoveto{\pgfqpoint{10.060501in}{5.032156in}}%
\pgfpathlineto{\pgfqpoint{10.122614in}{5.032156in}}%
\pgfpathmoveto{\pgfqpoint{10.091557in}{5.001100in}}%
\pgfpathlineto{\pgfqpoint{10.091557in}{5.063213in}}%
\pgfusepath{stroke,fill}%
\end{pgfscope}%
\begin{pgfscope}%
\pgfpathrectangle{\pgfqpoint{7.105882in}{4.421053in}}{\pgfqpoint{4.376471in}{0.978947in}} %
\pgfusepath{clip}%
\pgfsetbuttcap%
\pgfsetroundjoin%
\definecolor{currentfill}{rgb}{1.000000,0.000000,0.000000}%
\pgfsetfillcolor{currentfill}%
\pgfsetlinewidth{2.007500pt}%
\definecolor{currentstroke}{rgb}{1.000000,0.000000,0.000000}%
\pgfsetstrokecolor{currentstroke}%
\pgfsetdash{}{0pt}%
\pgfpathmoveto{\pgfqpoint{9.857852in}{4.921244in}}%
\pgfpathlineto{\pgfqpoint{9.919965in}{4.921244in}}%
\pgfpathmoveto{\pgfqpoint{9.888909in}{4.890188in}}%
\pgfpathlineto{\pgfqpoint{9.888909in}{4.952301in}}%
\pgfusepath{stroke,fill}%
\end{pgfscope}%
\begin{pgfscope}%
\pgfpathrectangle{\pgfqpoint{7.105882in}{4.421053in}}{\pgfqpoint{4.376471in}{0.978947in}} %
\pgfusepath{clip}%
\pgfsetbuttcap%
\pgfsetroundjoin%
\definecolor{currentfill}{rgb}{1.000000,0.000000,0.000000}%
\pgfsetfillcolor{currentfill}%
\pgfsetlinewidth{2.007500pt}%
\definecolor{currentstroke}{rgb}{1.000000,0.000000,0.000000}%
\pgfsetstrokecolor{currentstroke}%
\pgfsetdash{}{0pt}%
\pgfpathmoveto{\pgfqpoint{9.433410in}{4.594850in}}%
\pgfpathlineto{\pgfqpoint{9.495523in}{4.594850in}}%
\pgfpathmoveto{\pgfqpoint{9.464467in}{4.563793in}}%
\pgfpathlineto{\pgfqpoint{9.464467in}{4.625906in}}%
\pgfusepath{stroke,fill}%
\end{pgfscope}%
\begin{pgfscope}%
\pgfpathrectangle{\pgfqpoint{7.105882in}{4.421053in}}{\pgfqpoint{4.376471in}{0.978947in}} %
\pgfusepath{clip}%
\pgfsetbuttcap%
\pgfsetroundjoin%
\definecolor{currentfill}{rgb}{1.000000,0.000000,0.000000}%
\pgfsetfillcolor{currentfill}%
\pgfsetlinewidth{2.007500pt}%
\definecolor{currentstroke}{rgb}{1.000000,0.000000,0.000000}%
\pgfsetstrokecolor{currentstroke}%
\pgfsetdash{}{0pt}%
\pgfpathmoveto{\pgfqpoint{10.211509in}{4.864314in}}%
\pgfpathlineto{\pgfqpoint{10.273622in}{4.864314in}}%
\pgfpathmoveto{\pgfqpoint{10.242566in}{4.833258in}}%
\pgfpathlineto{\pgfqpoint{10.242566in}{4.895371in}}%
\pgfusepath{stroke,fill}%
\end{pgfscope}%
\begin{pgfscope}%
\pgfpathrectangle{\pgfqpoint{7.105882in}{4.421053in}}{\pgfqpoint{4.376471in}{0.978947in}} %
\pgfusepath{clip}%
\pgfsetbuttcap%
\pgfsetroundjoin%
\definecolor{currentfill}{rgb}{1.000000,0.000000,0.000000}%
\pgfsetfillcolor{currentfill}%
\pgfsetlinewidth{2.007500pt}%
\definecolor{currentstroke}{rgb}{1.000000,0.000000,0.000000}%
\pgfsetstrokecolor{currentstroke}%
\pgfsetdash{}{0pt}%
\pgfpathmoveto{\pgfqpoint{9.482190in}{4.632344in}}%
\pgfpathlineto{\pgfqpoint{9.544303in}{4.632344in}}%
\pgfpathmoveto{\pgfqpoint{9.513247in}{4.601287in}}%
\pgfpathlineto{\pgfqpoint{9.513247in}{4.663400in}}%
\pgfusepath{stroke,fill}%
\end{pgfscope}%
\begin{pgfscope}%
\pgfpathrectangle{\pgfqpoint{7.105882in}{4.421053in}}{\pgfqpoint{4.376471in}{0.978947in}} %
\pgfusepath{clip}%
\pgfsetbuttcap%
\pgfsetroundjoin%
\definecolor{currentfill}{rgb}{1.000000,0.000000,0.000000}%
\pgfsetfillcolor{currentfill}%
\pgfsetlinewidth{2.007500pt}%
\definecolor{currentstroke}{rgb}{1.000000,0.000000,0.000000}%
\pgfsetstrokecolor{currentstroke}%
\pgfsetdash{}{0pt}%
\pgfpathmoveto{\pgfqpoint{11.072375in}{5.272462in}}%
\pgfpathlineto{\pgfqpoint{11.134488in}{5.272462in}}%
\pgfpathmoveto{\pgfqpoint{11.103431in}{5.241406in}}%
\pgfpathlineto{\pgfqpoint{11.103431in}{5.303519in}}%
\pgfusepath{stroke,fill}%
\end{pgfscope}%
\begin{pgfscope}%
\pgfpathrectangle{\pgfqpoint{7.105882in}{4.421053in}}{\pgfqpoint{4.376471in}{0.978947in}} %
\pgfusepath{clip}%
\pgfsetbuttcap%
\pgfsetroundjoin%
\definecolor{currentfill}{rgb}{1.000000,0.000000,0.000000}%
\pgfsetfillcolor{currentfill}%
\pgfsetlinewidth{2.007500pt}%
\definecolor{currentstroke}{rgb}{1.000000,0.000000,0.000000}%
\pgfsetstrokecolor{currentstroke}%
\pgfsetdash{}{0pt}%
\pgfpathmoveto{\pgfqpoint{11.324073in}{5.172865in}}%
\pgfpathlineto{\pgfqpoint{11.386186in}{5.172865in}}%
\pgfpathmoveto{\pgfqpoint{11.355130in}{5.141808in}}%
\pgfpathlineto{\pgfqpoint{11.355130in}{5.203921in}}%
\pgfusepath{stroke,fill}%
\end{pgfscope}%
\begin{pgfscope}%
\pgfpathrectangle{\pgfqpoint{7.105882in}{4.421053in}}{\pgfqpoint{4.376471in}{0.978947in}} %
\pgfusepath{clip}%
\pgfsetbuttcap%
\pgfsetroundjoin%
\definecolor{currentfill}{rgb}{1.000000,0.000000,0.000000}%
\pgfsetfillcolor{currentfill}%
\pgfsetlinewidth{2.007500pt}%
\definecolor{currentstroke}{rgb}{1.000000,0.000000,0.000000}%
\pgfsetstrokecolor{currentstroke}%
\pgfsetdash{}{0pt}%
\pgfpathmoveto{\pgfqpoint{9.292616in}{4.670261in}}%
\pgfpathlineto{\pgfqpoint{9.354729in}{4.670261in}}%
\pgfpathmoveto{\pgfqpoint{9.323673in}{4.639204in}}%
\pgfpathlineto{\pgfqpoint{9.323673in}{4.701317in}}%
\pgfusepath{stroke,fill}%
\end{pgfscope}%
\begin{pgfscope}%
\pgfpathrectangle{\pgfqpoint{7.105882in}{4.421053in}}{\pgfqpoint{4.376471in}{0.978947in}} %
\pgfusepath{clip}%
\pgfsetbuttcap%
\pgfsetroundjoin%
\definecolor{currentfill}{rgb}{1.000000,0.000000,0.000000}%
\pgfsetfillcolor{currentfill}%
\pgfsetlinewidth{2.007500pt}%
\definecolor{currentstroke}{rgb}{1.000000,0.000000,0.000000}%
\pgfsetstrokecolor{currentstroke}%
\pgfsetdash{}{0pt}%
\pgfpathmoveto{\pgfqpoint{10.722089in}{4.820436in}}%
\pgfpathlineto{\pgfqpoint{10.784202in}{4.820436in}}%
\pgfpathmoveto{\pgfqpoint{10.753146in}{4.789380in}}%
\pgfpathlineto{\pgfqpoint{10.753146in}{4.851493in}}%
\pgfusepath{stroke,fill}%
\end{pgfscope}%
\begin{pgfscope}%
\pgfpathrectangle{\pgfqpoint{7.105882in}{4.421053in}}{\pgfqpoint{4.376471in}{0.978947in}} %
\pgfusepath{clip}%
\pgfsetbuttcap%
\pgfsetroundjoin%
\definecolor{currentfill}{rgb}{1.000000,0.000000,0.000000}%
\pgfsetfillcolor{currentfill}%
\pgfsetlinewidth{2.007500pt}%
\definecolor{currentstroke}{rgb}{1.000000,0.000000,0.000000}%
\pgfsetstrokecolor{currentstroke}%
\pgfsetdash{}{0pt}%
\pgfpathmoveto{\pgfqpoint{9.801874in}{4.860532in}}%
\pgfpathlineto{\pgfqpoint{9.863987in}{4.860532in}}%
\pgfpathmoveto{\pgfqpoint{9.832931in}{4.829475in}}%
\pgfpathlineto{\pgfqpoint{9.832931in}{4.891588in}}%
\pgfusepath{stroke,fill}%
\end{pgfscope}%
\begin{pgfscope}%
\pgfpathrectangle{\pgfqpoint{7.105882in}{4.421053in}}{\pgfqpoint{4.376471in}{0.978947in}} %
\pgfusepath{clip}%
\pgfsetbuttcap%
\pgfsetroundjoin%
\definecolor{currentfill}{rgb}{1.000000,0.000000,0.000000}%
\pgfsetfillcolor{currentfill}%
\pgfsetlinewidth{2.007500pt}%
\definecolor{currentstroke}{rgb}{1.000000,0.000000,0.000000}%
\pgfsetstrokecolor{currentstroke}%
\pgfsetdash{}{0pt}%
\pgfpathmoveto{\pgfqpoint{9.938944in}{4.988087in}}%
\pgfpathlineto{\pgfqpoint{10.001057in}{4.988087in}}%
\pgfpathmoveto{\pgfqpoint{9.970001in}{4.957031in}}%
\pgfpathlineto{\pgfqpoint{9.970001in}{5.019144in}}%
\pgfusepath{stroke,fill}%
\end{pgfscope}%
\begin{pgfscope}%
\pgfpathrectangle{\pgfqpoint{7.105882in}{4.421053in}}{\pgfqpoint{4.376471in}{0.978947in}} %
\pgfusepath{clip}%
\pgfsetbuttcap%
\pgfsetroundjoin%
\definecolor{currentfill}{rgb}{1.000000,0.000000,0.000000}%
\pgfsetfillcolor{currentfill}%
\pgfsetlinewidth{2.007500pt}%
\definecolor{currentstroke}{rgb}{1.000000,0.000000,0.000000}%
\pgfsetstrokecolor{currentstroke}%
\pgfsetdash{}{0pt}%
\pgfpathmoveto{\pgfqpoint{11.190797in}{5.248251in}}%
\pgfpathlineto{\pgfqpoint{11.252910in}{5.248251in}}%
\pgfpathmoveto{\pgfqpoint{11.221854in}{5.217195in}}%
\pgfpathlineto{\pgfqpoint{11.221854in}{5.279308in}}%
\pgfusepath{stroke,fill}%
\end{pgfscope}%
\begin{pgfscope}%
\pgfpathrectangle{\pgfqpoint{7.105882in}{4.421053in}}{\pgfqpoint{4.376471in}{0.978947in}} %
\pgfusepath{clip}%
\pgfsetbuttcap%
\pgfsetroundjoin%
\definecolor{currentfill}{rgb}{1.000000,0.000000,0.000000}%
\pgfsetfillcolor{currentfill}%
\pgfsetlinewidth{2.007500pt}%
\definecolor{currentstroke}{rgb}{1.000000,0.000000,0.000000}%
\pgfsetstrokecolor{currentstroke}%
\pgfsetdash{}{0pt}%
\pgfpathmoveto{\pgfqpoint{8.198830in}{4.960862in}}%
\pgfpathlineto{\pgfqpoint{8.260943in}{4.960862in}}%
\pgfpathmoveto{\pgfqpoint{8.229886in}{4.929806in}}%
\pgfpathlineto{\pgfqpoint{8.229886in}{4.991919in}}%
\pgfusepath{stroke,fill}%
\end{pgfscope}%
\begin{pgfscope}%
\pgfpathrectangle{\pgfqpoint{7.105882in}{4.421053in}}{\pgfqpoint{4.376471in}{0.978947in}} %
\pgfusepath{clip}%
\pgfsetbuttcap%
\pgfsetroundjoin%
\definecolor{currentfill}{rgb}{1.000000,0.000000,0.000000}%
\pgfsetfillcolor{currentfill}%
\pgfsetlinewidth{2.007500pt}%
\definecolor{currentstroke}{rgb}{1.000000,0.000000,0.000000}%
\pgfsetstrokecolor{currentstroke}%
\pgfsetdash{}{0pt}%
\pgfpathmoveto{\pgfqpoint{8.255175in}{4.903228in}}%
\pgfpathlineto{\pgfqpoint{8.317288in}{4.903228in}}%
\pgfpathmoveto{\pgfqpoint{8.286232in}{4.872172in}}%
\pgfpathlineto{\pgfqpoint{8.286232in}{4.934285in}}%
\pgfusepath{stroke,fill}%
\end{pgfscope}%
\begin{pgfscope}%
\pgfpathrectangle{\pgfqpoint{7.105882in}{4.421053in}}{\pgfqpoint{4.376471in}{0.978947in}} %
\pgfusepath{clip}%
\pgfsetbuttcap%
\pgfsetroundjoin%
\definecolor{currentfill}{rgb}{1.000000,0.000000,0.000000}%
\pgfsetfillcolor{currentfill}%
\pgfsetlinewidth{2.007500pt}%
\definecolor{currentstroke}{rgb}{1.000000,0.000000,0.000000}%
\pgfsetstrokecolor{currentstroke}%
\pgfsetdash{}{0pt}%
\pgfpathmoveto{\pgfqpoint{8.020908in}{5.190508in}}%
\pgfpathlineto{\pgfqpoint{8.083021in}{5.190508in}}%
\pgfpathmoveto{\pgfqpoint{8.051965in}{5.159452in}}%
\pgfpathlineto{\pgfqpoint{8.051965in}{5.221565in}}%
\pgfusepath{stroke,fill}%
\end{pgfscope}%
\begin{pgfscope}%
\pgfpathrectangle{\pgfqpoint{7.105882in}{4.421053in}}{\pgfqpoint{4.376471in}{0.978947in}} %
\pgfusepath{clip}%
\pgfsetbuttcap%
\pgfsetroundjoin%
\definecolor{currentfill}{rgb}{1.000000,0.000000,0.000000}%
\pgfsetfillcolor{currentfill}%
\pgfsetlinewidth{2.007500pt}%
\definecolor{currentstroke}{rgb}{1.000000,0.000000,0.000000}%
\pgfsetstrokecolor{currentstroke}%
\pgfsetdash{}{0pt}%
\pgfpathmoveto{\pgfqpoint{10.865269in}{5.036104in}}%
\pgfpathlineto{\pgfqpoint{10.927382in}{5.036104in}}%
\pgfpathmoveto{\pgfqpoint{10.896325in}{5.005048in}}%
\pgfpathlineto{\pgfqpoint{10.896325in}{5.067161in}}%
\pgfusepath{stroke,fill}%
\end{pgfscope}%
\begin{pgfscope}%
\pgfpathrectangle{\pgfqpoint{7.105882in}{4.421053in}}{\pgfqpoint{4.376471in}{0.978947in}} %
\pgfusepath{clip}%
\pgfsetbuttcap%
\pgfsetroundjoin%
\definecolor{currentfill}{rgb}{1.000000,0.000000,0.000000}%
\pgfsetfillcolor{currentfill}%
\pgfsetlinewidth{2.007500pt}%
\definecolor{currentstroke}{rgb}{1.000000,0.000000,0.000000}%
\pgfsetstrokecolor{currentstroke}%
\pgfsetdash{}{0pt}%
\pgfpathmoveto{\pgfqpoint{10.674584in}{4.758523in}}%
\pgfpathlineto{\pgfqpoint{10.736697in}{4.758523in}}%
\pgfpathmoveto{\pgfqpoint{10.705641in}{4.727466in}}%
\pgfpathlineto{\pgfqpoint{10.705641in}{4.789579in}}%
\pgfusepath{stroke,fill}%
\end{pgfscope}%
\begin{pgfscope}%
\pgfpathrectangle{\pgfqpoint{7.105882in}{4.421053in}}{\pgfqpoint{4.376471in}{0.978947in}} %
\pgfusepath{clip}%
\pgfsetbuttcap%
\pgfsetroundjoin%
\definecolor{currentfill}{rgb}{1.000000,0.000000,0.000000}%
\pgfsetfillcolor{currentfill}%
\pgfsetlinewidth{2.007500pt}%
\definecolor{currentstroke}{rgb}{1.000000,0.000000,0.000000}%
\pgfsetstrokecolor{currentstroke}%
\pgfsetdash{}{0pt}%
\pgfpathmoveto{\pgfqpoint{10.996186in}{5.165886in}}%
\pgfpathlineto{\pgfqpoint{11.058299in}{5.165886in}}%
\pgfpathmoveto{\pgfqpoint{11.027243in}{5.134829in}}%
\pgfpathlineto{\pgfqpoint{11.027243in}{5.196942in}}%
\pgfusepath{stroke,fill}%
\end{pgfscope}%
\begin{pgfscope}%
\pgfpathrectangle{\pgfqpoint{7.105882in}{4.421053in}}{\pgfqpoint{4.376471in}{0.978947in}} %
\pgfusepath{clip}%
\pgfsetbuttcap%
\pgfsetroundjoin%
\definecolor{currentfill}{rgb}{1.000000,0.000000,0.000000}%
\pgfsetfillcolor{currentfill}%
\pgfsetlinewidth{2.007500pt}%
\definecolor{currentstroke}{rgb}{1.000000,0.000000,0.000000}%
\pgfsetstrokecolor{currentstroke}%
\pgfsetdash{}{0pt}%
\pgfpathmoveto{\pgfqpoint{11.376435in}{5.097123in}}%
\pgfpathlineto{\pgfqpoint{11.438548in}{5.097123in}}%
\pgfpathmoveto{\pgfqpoint{11.407492in}{5.066066in}}%
\pgfpathlineto{\pgfqpoint{11.407492in}{5.128179in}}%
\pgfusepath{stroke,fill}%
\end{pgfscope}%
\begin{pgfscope}%
\pgfpathrectangle{\pgfqpoint{7.105882in}{4.421053in}}{\pgfqpoint{4.376471in}{0.978947in}} %
\pgfusepath{clip}%
\pgfsetbuttcap%
\pgfsetroundjoin%
\definecolor{currentfill}{rgb}{1.000000,0.000000,0.000000}%
\pgfsetfillcolor{currentfill}%
\pgfsetlinewidth{2.007500pt}%
\definecolor{currentstroke}{rgb}{1.000000,0.000000,0.000000}%
\pgfsetstrokecolor{currentstroke}%
\pgfsetdash{}{0pt}%
\pgfpathmoveto{\pgfqpoint{10.748115in}{4.913690in}}%
\pgfpathlineto{\pgfqpoint{10.810228in}{4.913690in}}%
\pgfpathmoveto{\pgfqpoint{10.779172in}{4.882633in}}%
\pgfpathlineto{\pgfqpoint{10.779172in}{4.944746in}}%
\pgfusepath{stroke,fill}%
\end{pgfscope}%
\begin{pgfscope}%
\pgfpathrectangle{\pgfqpoint{7.105882in}{4.421053in}}{\pgfqpoint{4.376471in}{0.978947in}} %
\pgfusepath{clip}%
\pgfsetbuttcap%
\pgfsetroundjoin%
\definecolor{currentfill}{rgb}{1.000000,0.000000,0.000000}%
\pgfsetfillcolor{currentfill}%
\pgfsetlinewidth{2.007500pt}%
\definecolor{currentstroke}{rgb}{1.000000,0.000000,0.000000}%
\pgfsetstrokecolor{currentstroke}%
\pgfsetdash{}{0pt}%
\pgfpathmoveto{\pgfqpoint{9.565841in}{4.638149in}}%
\pgfpathlineto{\pgfqpoint{9.627954in}{4.638149in}}%
\pgfpathmoveto{\pgfqpoint{9.596897in}{4.607093in}}%
\pgfpathlineto{\pgfqpoint{9.596897in}{4.669206in}}%
\pgfusepath{stroke,fill}%
\end{pgfscope}%
\begin{pgfscope}%
\pgfpathrectangle{\pgfqpoint{7.105882in}{4.421053in}}{\pgfqpoint{4.376471in}{0.978947in}} %
\pgfusepath{clip}%
\pgfsetbuttcap%
\pgfsetroundjoin%
\definecolor{currentfill}{rgb}{1.000000,0.000000,0.000000}%
\pgfsetfillcolor{currentfill}%
\pgfsetlinewidth{2.007500pt}%
\definecolor{currentstroke}{rgb}{1.000000,0.000000,0.000000}%
\pgfsetstrokecolor{currentstroke}%
\pgfsetdash{}{0pt}%
\pgfpathmoveto{\pgfqpoint{10.682890in}{4.780559in}}%
\pgfpathlineto{\pgfqpoint{10.745003in}{4.780559in}}%
\pgfpathmoveto{\pgfqpoint{10.713947in}{4.749503in}}%
\pgfpathlineto{\pgfqpoint{10.713947in}{4.811616in}}%
\pgfusepath{stroke,fill}%
\end{pgfscope}%
\begin{pgfscope}%
\pgfpathrectangle{\pgfqpoint{7.105882in}{4.421053in}}{\pgfqpoint{4.376471in}{0.978947in}} %
\pgfusepath{clip}%
\pgfsetbuttcap%
\pgfsetroundjoin%
\definecolor{currentfill}{rgb}{1.000000,0.000000,0.000000}%
\pgfsetfillcolor{currentfill}%
\pgfsetlinewidth{2.007500pt}%
\definecolor{currentstroke}{rgb}{1.000000,0.000000,0.000000}%
\pgfsetstrokecolor{currentstroke}%
\pgfsetdash{}{0pt}%
\pgfpathmoveto{\pgfqpoint{8.364220in}{4.801297in}}%
\pgfpathlineto{\pgfqpoint{8.426333in}{4.801297in}}%
\pgfpathmoveto{\pgfqpoint{8.395276in}{4.770240in}}%
\pgfpathlineto{\pgfqpoint{8.395276in}{4.832353in}}%
\pgfusepath{stroke,fill}%
\end{pgfscope}%
\begin{pgfscope}%
\pgfpathrectangle{\pgfqpoint{7.105882in}{4.421053in}}{\pgfqpoint{4.376471in}{0.978947in}} %
\pgfusepath{clip}%
\pgfsetbuttcap%
\pgfsetroundjoin%
\definecolor{currentfill}{rgb}{1.000000,0.000000,0.000000}%
\pgfsetfillcolor{currentfill}%
\pgfsetlinewidth{2.007500pt}%
\definecolor{currentstroke}{rgb}{1.000000,0.000000,0.000000}%
\pgfsetstrokecolor{currentstroke}%
\pgfsetdash{}{0pt}%
\pgfpathmoveto{\pgfqpoint{10.190596in}{4.904581in}}%
\pgfpathlineto{\pgfqpoint{10.252709in}{4.904581in}}%
\pgfpathmoveto{\pgfqpoint{10.221653in}{4.873524in}}%
\pgfpathlineto{\pgfqpoint{10.221653in}{4.935637in}}%
\pgfusepath{stroke,fill}%
\end{pgfscope}%
\begin{pgfscope}%
\pgfpathrectangle{\pgfqpoint{7.105882in}{4.421053in}}{\pgfqpoint{4.376471in}{0.978947in}} %
\pgfusepath{clip}%
\pgfsetbuttcap%
\pgfsetroundjoin%
\definecolor{currentfill}{rgb}{1.000000,0.000000,0.000000}%
\pgfsetfillcolor{currentfill}%
\pgfsetlinewidth{2.007500pt}%
\definecolor{currentstroke}{rgb}{1.000000,0.000000,0.000000}%
\pgfsetstrokecolor{currentstroke}%
\pgfsetdash{}{0pt}%
\pgfpathmoveto{\pgfqpoint{8.452025in}{4.809587in}}%
\pgfpathlineto{\pgfqpoint{8.514138in}{4.809587in}}%
\pgfpathmoveto{\pgfqpoint{8.483082in}{4.778530in}}%
\pgfpathlineto{\pgfqpoint{8.483082in}{4.840643in}}%
\pgfusepath{stroke,fill}%
\end{pgfscope}%
\begin{pgfscope}%
\pgfpathrectangle{\pgfqpoint{7.105882in}{4.421053in}}{\pgfqpoint{4.376471in}{0.978947in}} %
\pgfusepath{clip}%
\pgfsetbuttcap%
\pgfsetroundjoin%
\definecolor{currentfill}{rgb}{1.000000,0.000000,0.000000}%
\pgfsetfillcolor{currentfill}%
\pgfsetlinewidth{2.007500pt}%
\definecolor{currentstroke}{rgb}{1.000000,0.000000,0.000000}%
\pgfsetstrokecolor{currentstroke}%
\pgfsetdash{}{0pt}%
\pgfpathmoveto{\pgfqpoint{11.257573in}{5.209780in}}%
\pgfpathlineto{\pgfqpoint{11.319686in}{5.209780in}}%
\pgfpathmoveto{\pgfqpoint{11.288629in}{5.178724in}}%
\pgfpathlineto{\pgfqpoint{11.288629in}{5.240837in}}%
\pgfusepath{stroke,fill}%
\end{pgfscope}%
\begin{pgfscope}%
\pgfpathrectangle{\pgfqpoint{7.105882in}{4.421053in}}{\pgfqpoint{4.376471in}{0.978947in}} %
\pgfusepath{clip}%
\pgfsetbuttcap%
\pgfsetroundjoin%
\definecolor{currentfill}{rgb}{1.000000,0.000000,0.000000}%
\pgfsetfillcolor{currentfill}%
\pgfsetlinewidth{2.007500pt}%
\definecolor{currentstroke}{rgb}{1.000000,0.000000,0.000000}%
\pgfsetstrokecolor{currentstroke}%
\pgfsetdash{}{0pt}%
\pgfpathmoveto{\pgfqpoint{9.777203in}{4.856983in}}%
\pgfpathlineto{\pgfqpoint{9.839316in}{4.856983in}}%
\pgfpathmoveto{\pgfqpoint{9.808260in}{4.825926in}}%
\pgfpathlineto{\pgfqpoint{9.808260in}{4.888039in}}%
\pgfusepath{stroke,fill}%
\end{pgfscope}%
\begin{pgfscope}%
\pgfpathrectangle{\pgfqpoint{7.105882in}{4.421053in}}{\pgfqpoint{4.376471in}{0.978947in}} %
\pgfusepath{clip}%
\pgfsetbuttcap%
\pgfsetroundjoin%
\definecolor{currentfill}{rgb}{1.000000,0.000000,0.000000}%
\pgfsetfillcolor{currentfill}%
\pgfsetlinewidth{2.007500pt}%
\definecolor{currentstroke}{rgb}{1.000000,0.000000,0.000000}%
\pgfsetstrokecolor{currentstroke}%
\pgfsetdash{}{0pt}%
\pgfpathmoveto{\pgfqpoint{9.401925in}{4.605866in}}%
\pgfpathlineto{\pgfqpoint{9.464038in}{4.605866in}}%
\pgfpathmoveto{\pgfqpoint{9.432981in}{4.574809in}}%
\pgfpathlineto{\pgfqpoint{9.432981in}{4.636922in}}%
\pgfusepath{stroke,fill}%
\end{pgfscope}%
\begin{pgfscope}%
\pgfpathrectangle{\pgfqpoint{7.105882in}{4.421053in}}{\pgfqpoint{4.376471in}{0.978947in}} %
\pgfusepath{clip}%
\pgfsetbuttcap%
\pgfsetroundjoin%
\definecolor{currentfill}{rgb}{0.000000,0.000000,0.000000}%
\pgfsetfillcolor{currentfill}%
\pgfsetlinewidth{0.301125pt}%
\definecolor{currentstroke}{rgb}{0.000000,0.000000,0.000000}%
\pgfsetstrokecolor{currentstroke}%
\pgfsetdash{}{0pt}%
\pgfsys@defobject{currentmarker}{\pgfqpoint{-0.015528in}{-0.015528in}}{\pgfqpoint{0.015528in}{0.015528in}}{%
\pgfpathmoveto{\pgfqpoint{0.000000in}{-0.015528in}}%
\pgfpathcurveto{\pgfqpoint{0.004118in}{-0.015528in}}{\pgfqpoint{0.008068in}{-0.013892in}}{\pgfqpoint{0.010980in}{-0.010980in}}%
\pgfpathcurveto{\pgfqpoint{0.013892in}{-0.008068in}}{\pgfqpoint{0.015528in}{-0.004118in}}{\pgfqpoint{0.015528in}{0.000000in}}%
\pgfpathcurveto{\pgfqpoint{0.015528in}{0.004118in}}{\pgfqpoint{0.013892in}{0.008068in}}{\pgfqpoint{0.010980in}{0.010980in}}%
\pgfpathcurveto{\pgfqpoint{0.008068in}{0.013892in}}{\pgfqpoint{0.004118in}{0.015528in}}{\pgfqpoint{0.000000in}{0.015528in}}%
\pgfpathcurveto{\pgfqpoint{-0.004118in}{0.015528in}}{\pgfqpoint{-0.008068in}{0.013892in}}{\pgfqpoint{-0.010980in}{0.010980in}}%
\pgfpathcurveto{\pgfqpoint{-0.013892in}{0.008068in}}{\pgfqpoint{-0.015528in}{0.004118in}}{\pgfqpoint{-0.015528in}{0.000000in}}%
\pgfpathcurveto{\pgfqpoint{-0.015528in}{-0.004118in}}{\pgfqpoint{-0.013892in}{-0.008068in}}{\pgfqpoint{-0.010980in}{-0.010980in}}%
\pgfpathcurveto{\pgfqpoint{-0.008068in}{-0.013892in}}{\pgfqpoint{-0.004118in}{-0.015528in}}{\pgfqpoint{0.000000in}{-0.015528in}}%
\pgfpathclose%
\pgfusepath{stroke,fill}%
}%
\begin{pgfscope}%
\pgfsys@transformshift{7.981176in}{5.254916in}%
\pgfsys@useobject{currentmarker}{}%
\end{pgfscope}%
\begin{pgfscope}%
\pgfsys@transformshift{7.998770in}{5.160724in}%
\pgfsys@useobject{currentmarker}{}%
\end{pgfscope}%
\begin{pgfscope}%
\pgfsys@transformshift{8.016364in}{5.251498in}%
\pgfsys@useobject{currentmarker}{}%
\end{pgfscope}%
\begin{pgfscope}%
\pgfsys@transformshift{8.033958in}{5.270634in}%
\pgfsys@useobject{currentmarker}{}%
\end{pgfscope}%
\begin{pgfscope}%
\pgfsys@transformshift{8.051552in}{5.205849in}%
\pgfsys@useobject{currentmarker}{}%
\end{pgfscope}%
\begin{pgfscope}%
\pgfsys@transformshift{8.069146in}{5.201748in}%
\pgfsys@useobject{currentmarker}{}%
\end{pgfscope}%
\begin{pgfscope}%
\pgfsys@transformshift{8.086740in}{5.077991in}%
\pgfsys@useobject{currentmarker}{}%
\end{pgfscope}%
\begin{pgfscope}%
\pgfsys@transformshift{8.104333in}{5.077595in}%
\pgfsys@useobject{currentmarker}{}%
\end{pgfscope}%
\begin{pgfscope}%
\pgfsys@transformshift{8.121927in}{5.018268in}%
\pgfsys@useobject{currentmarker}{}%
\end{pgfscope}%
\begin{pgfscope}%
\pgfsys@transformshift{8.139521in}{5.022982in}%
\pgfsys@useobject{currentmarker}{}%
\end{pgfscope}%
\begin{pgfscope}%
\pgfsys@transformshift{8.157115in}{4.950281in}%
\pgfsys@useobject{currentmarker}{}%
\end{pgfscope}%
\begin{pgfscope}%
\pgfsys@transformshift{8.174709in}{4.831659in}%
\pgfsys@useobject{currentmarker}{}%
\end{pgfscope}%
\begin{pgfscope}%
\pgfsys@transformshift{8.192303in}{5.001404in}%
\pgfsys@useobject{currentmarker}{}%
\end{pgfscope}%
\begin{pgfscope}%
\pgfsys@transformshift{8.209897in}{4.919253in}%
\pgfsys@useobject{currentmarker}{}%
\end{pgfscope}%
\begin{pgfscope}%
\pgfsys@transformshift{8.227490in}{4.772358in}%
\pgfsys@useobject{currentmarker}{}%
\end{pgfscope}%
\begin{pgfscope}%
\pgfsys@transformshift{8.245084in}{4.965761in}%
\pgfsys@useobject{currentmarker}{}%
\end{pgfscope}%
\begin{pgfscope}%
\pgfsys@transformshift{8.262678in}{4.807758in}%
\pgfsys@useobject{currentmarker}{}%
\end{pgfscope}%
\begin{pgfscope}%
\pgfsys@transformshift{8.280272in}{4.889135in}%
\pgfsys@useobject{currentmarker}{}%
\end{pgfscope}%
\begin{pgfscope}%
\pgfsys@transformshift{8.297866in}{4.943692in}%
\pgfsys@useobject{currentmarker}{}%
\end{pgfscope}%
\begin{pgfscope}%
\pgfsys@transformshift{8.315460in}{4.870028in}%
\pgfsys@useobject{currentmarker}{}%
\end{pgfscope}%
\begin{pgfscope}%
\pgfsys@transformshift{8.333054in}{4.962679in}%
\pgfsys@useobject{currentmarker}{}%
\end{pgfscope}%
\begin{pgfscope}%
\pgfsys@transformshift{8.350647in}{4.712306in}%
\pgfsys@useobject{currentmarker}{}%
\end{pgfscope}%
\begin{pgfscope}%
\pgfsys@transformshift{8.368241in}{4.873130in}%
\pgfsys@useobject{currentmarker}{}%
\end{pgfscope}%
\begin{pgfscope}%
\pgfsys@transformshift{8.385835in}{4.758281in}%
\pgfsys@useobject{currentmarker}{}%
\end{pgfscope}%
\begin{pgfscope}%
\pgfsys@transformshift{8.403429in}{4.737440in}%
\pgfsys@useobject{currentmarker}{}%
\end{pgfscope}%
\begin{pgfscope}%
\pgfsys@transformshift{8.421023in}{4.767404in}%
\pgfsys@useobject{currentmarker}{}%
\end{pgfscope}%
\begin{pgfscope}%
\pgfsys@transformshift{8.438617in}{4.796858in}%
\pgfsys@useobject{currentmarker}{}%
\end{pgfscope}%
\begin{pgfscope}%
\pgfsys@transformshift{8.456210in}{4.838472in}%
\pgfsys@useobject{currentmarker}{}%
\end{pgfscope}%
\begin{pgfscope}%
\pgfsys@transformshift{8.473804in}{4.719865in}%
\pgfsys@useobject{currentmarker}{}%
\end{pgfscope}%
\begin{pgfscope}%
\pgfsys@transformshift{8.491398in}{4.938174in}%
\pgfsys@useobject{currentmarker}{}%
\end{pgfscope}%
\begin{pgfscope}%
\pgfsys@transformshift{8.508992in}{4.902994in}%
\pgfsys@useobject{currentmarker}{}%
\end{pgfscope}%
\begin{pgfscope}%
\pgfsys@transformshift{8.526586in}{4.709459in}%
\pgfsys@useobject{currentmarker}{}%
\end{pgfscope}%
\begin{pgfscope}%
\pgfsys@transformshift{8.544180in}{5.029731in}%
\pgfsys@useobject{currentmarker}{}%
\end{pgfscope}%
\begin{pgfscope}%
\pgfsys@transformshift{8.561774in}{5.084223in}%
\pgfsys@useobject{currentmarker}{}%
\end{pgfscope}%
\begin{pgfscope}%
\pgfsys@transformshift{8.579367in}{5.024905in}%
\pgfsys@useobject{currentmarker}{}%
\end{pgfscope}%
\begin{pgfscope}%
\pgfsys@transformshift{8.596961in}{4.900851in}%
\pgfsys@useobject{currentmarker}{}%
\end{pgfscope}%
\begin{pgfscope}%
\pgfsys@transformshift{8.614555in}{4.824998in}%
\pgfsys@useobject{currentmarker}{}%
\end{pgfscope}%
\begin{pgfscope}%
\pgfsys@transformshift{8.632149in}{5.056986in}%
\pgfsys@useobject{currentmarker}{}%
\end{pgfscope}%
\begin{pgfscope}%
\pgfsys@transformshift{8.649743in}{4.923698in}%
\pgfsys@useobject{currentmarker}{}%
\end{pgfscope}%
\begin{pgfscope}%
\pgfsys@transformshift{8.667337in}{5.104693in}%
\pgfsys@useobject{currentmarker}{}%
\end{pgfscope}%
\begin{pgfscope}%
\pgfsys@transformshift{8.684931in}{5.016167in}%
\pgfsys@useobject{currentmarker}{}%
\end{pgfscope}%
\begin{pgfscope}%
\pgfsys@transformshift{8.702524in}{5.108880in}%
\pgfsys@useobject{currentmarker}{}%
\end{pgfscope}%
\begin{pgfscope}%
\pgfsys@transformshift{8.720118in}{5.059262in}%
\pgfsys@useobject{currentmarker}{}%
\end{pgfscope}%
\begin{pgfscope}%
\pgfsys@transformshift{8.737712in}{5.107695in}%
\pgfsys@useobject{currentmarker}{}%
\end{pgfscope}%
\begin{pgfscope}%
\pgfsys@transformshift{8.755306in}{5.048347in}%
\pgfsys@useobject{currentmarker}{}%
\end{pgfscope}%
\begin{pgfscope}%
\pgfsys@transformshift{8.772900in}{5.239770in}%
\pgfsys@useobject{currentmarker}{}%
\end{pgfscope}%
\begin{pgfscope}%
\pgfsys@transformshift{8.790494in}{5.079579in}%
\pgfsys@useobject{currentmarker}{}%
\end{pgfscope}%
\begin{pgfscope}%
\pgfsys@transformshift{8.808087in}{5.115070in}%
\pgfsys@useobject{currentmarker}{}%
\end{pgfscope}%
\begin{pgfscope}%
\pgfsys@transformshift{8.825681in}{5.271894in}%
\pgfsys@useobject{currentmarker}{}%
\end{pgfscope}%
\begin{pgfscope}%
\pgfsys@transformshift{8.843275in}{4.946452in}%
\pgfsys@useobject{currentmarker}{}%
\end{pgfscope}%
\begin{pgfscope}%
\pgfsys@transformshift{8.860869in}{4.956515in}%
\pgfsys@useobject{currentmarker}{}%
\end{pgfscope}%
\begin{pgfscope}%
\pgfsys@transformshift{8.878463in}{5.185202in}%
\pgfsys@useobject{currentmarker}{}%
\end{pgfscope}%
\begin{pgfscope}%
\pgfsys@transformshift{8.896057in}{4.965051in}%
\pgfsys@useobject{currentmarker}{}%
\end{pgfscope}%
\begin{pgfscope}%
\pgfsys@transformshift{8.913651in}{5.279234in}%
\pgfsys@useobject{currentmarker}{}%
\end{pgfscope}%
\begin{pgfscope}%
\pgfsys@transformshift{8.931244in}{5.033240in}%
\pgfsys@useobject{currentmarker}{}%
\end{pgfscope}%
\begin{pgfscope}%
\pgfsys@transformshift{8.948838in}{4.991625in}%
\pgfsys@useobject{currentmarker}{}%
\end{pgfscope}%
\begin{pgfscope}%
\pgfsys@transformshift{8.966432in}{5.254445in}%
\pgfsys@useobject{currentmarker}{}%
\end{pgfscope}%
\begin{pgfscope}%
\pgfsys@transformshift{8.984026in}{5.197982in}%
\pgfsys@useobject{currentmarker}{}%
\end{pgfscope}%
\begin{pgfscope}%
\pgfsys@transformshift{9.001620in}{5.224324in}%
\pgfsys@useobject{currentmarker}{}%
\end{pgfscope}%
\begin{pgfscope}%
\pgfsys@transformshift{9.019214in}{5.111466in}%
\pgfsys@useobject{currentmarker}{}%
\end{pgfscope}%
\begin{pgfscope}%
\pgfsys@transformshift{9.036808in}{4.914878in}%
\pgfsys@useobject{currentmarker}{}%
\end{pgfscope}%
\begin{pgfscope}%
\pgfsys@transformshift{9.054401in}{5.179668in}%
\pgfsys@useobject{currentmarker}{}%
\end{pgfscope}%
\begin{pgfscope}%
\pgfsys@transformshift{9.071995in}{4.938468in}%
\pgfsys@useobject{currentmarker}{}%
\end{pgfscope}%
\begin{pgfscope}%
\pgfsys@transformshift{9.089589in}{5.027406in}%
\pgfsys@useobject{currentmarker}{}%
\end{pgfscope}%
\begin{pgfscope}%
\pgfsys@transformshift{9.107183in}{5.020999in}%
\pgfsys@useobject{currentmarker}{}%
\end{pgfscope}%
\begin{pgfscope}%
\pgfsys@transformshift{9.124777in}{4.886656in}%
\pgfsys@useobject{currentmarker}{}%
\end{pgfscope}%
\begin{pgfscope}%
\pgfsys@transformshift{9.142371in}{4.942565in}%
\pgfsys@useobject{currentmarker}{}%
\end{pgfscope}%
\begin{pgfscope}%
\pgfsys@transformshift{9.159965in}{4.951091in}%
\pgfsys@useobject{currentmarker}{}%
\end{pgfscope}%
\begin{pgfscope}%
\pgfsys@transformshift{9.177558in}{4.872367in}%
\pgfsys@useobject{currentmarker}{}%
\end{pgfscope}%
\begin{pgfscope}%
\pgfsys@transformshift{9.195152in}{4.698835in}%
\pgfsys@useobject{currentmarker}{}%
\end{pgfscope}%
\begin{pgfscope}%
\pgfsys@transformshift{9.212746in}{4.818561in}%
\pgfsys@useobject{currentmarker}{}%
\end{pgfscope}%
\begin{pgfscope}%
\pgfsys@transformshift{9.230340in}{4.901053in}%
\pgfsys@useobject{currentmarker}{}%
\end{pgfscope}%
\begin{pgfscope}%
\pgfsys@transformshift{9.247934in}{4.673262in}%
\pgfsys@useobject{currentmarker}{}%
\end{pgfscope}%
\begin{pgfscope}%
\pgfsys@transformshift{9.265528in}{4.707965in}%
\pgfsys@useobject{currentmarker}{}%
\end{pgfscope}%
\begin{pgfscope}%
\pgfsys@transformshift{9.283121in}{4.659015in}%
\pgfsys@useobject{currentmarker}{}%
\end{pgfscope}%
\begin{pgfscope}%
\pgfsys@transformshift{9.300715in}{4.873338in}%
\pgfsys@useobject{currentmarker}{}%
\end{pgfscope}%
\begin{pgfscope}%
\pgfsys@transformshift{9.318309in}{4.736067in}%
\pgfsys@useobject{currentmarker}{}%
\end{pgfscope}%
\begin{pgfscope}%
\pgfsys@transformshift{9.335903in}{4.693339in}%
\pgfsys@useobject{currentmarker}{}%
\end{pgfscope}%
\begin{pgfscope}%
\pgfsys@transformshift{9.353497in}{4.559224in}%
\pgfsys@useobject{currentmarker}{}%
\end{pgfscope}%
\begin{pgfscope}%
\pgfsys@transformshift{9.371091in}{4.680465in}%
\pgfsys@useobject{currentmarker}{}%
\end{pgfscope}%
\begin{pgfscope}%
\pgfsys@transformshift{9.388685in}{4.546350in}%
\pgfsys@useobject{currentmarker}{}%
\end{pgfscope}%
\begin{pgfscope}%
\pgfsys@transformshift{9.406278in}{4.609986in}%
\pgfsys@useobject{currentmarker}{}%
\end{pgfscope}%
\begin{pgfscope}%
\pgfsys@transformshift{9.423872in}{4.535582in}%
\pgfsys@useobject{currentmarker}{}%
\end{pgfscope}%
\begin{pgfscope}%
\pgfsys@transformshift{9.441466in}{4.665186in}%
\pgfsys@useobject{currentmarker}{}%
\end{pgfscope}%
\begin{pgfscope}%
\pgfsys@transformshift{9.459060in}{4.652923in}%
\pgfsys@useobject{currentmarker}{}%
\end{pgfscope}%
\begin{pgfscope}%
\pgfsys@transformshift{9.476654in}{4.572944in}%
\pgfsys@useobject{currentmarker}{}%
\end{pgfscope}%
\begin{pgfscope}%
\pgfsys@transformshift{9.494248in}{4.636754in}%
\pgfsys@useobject{currentmarker}{}%
\end{pgfscope}%
\begin{pgfscope}%
\pgfsys@transformshift{9.511842in}{4.489189in}%
\pgfsys@useobject{currentmarker}{}%
\end{pgfscope}%
\begin{pgfscope}%
\pgfsys@transformshift{9.529435in}{4.454903in}%
\pgfsys@useobject{currentmarker}{}%
\end{pgfscope}%
\begin{pgfscope}%
\pgfsys@transformshift{9.547029in}{4.660065in}%
\pgfsys@useobject{currentmarker}{}%
\end{pgfscope}%
\begin{pgfscope}%
\pgfsys@transformshift{9.564623in}{4.642414in}%
\pgfsys@useobject{currentmarker}{}%
\end{pgfscope}%
\begin{pgfscope}%
\pgfsys@transformshift{9.582217in}{4.702096in}%
\pgfsys@useobject{currentmarker}{}%
\end{pgfscope}%
\begin{pgfscope}%
\pgfsys@transformshift{9.599811in}{4.893914in}%
\pgfsys@useobject{currentmarker}{}%
\end{pgfscope}%
\begin{pgfscope}%
\pgfsys@transformshift{9.617405in}{4.762255in}%
\pgfsys@useobject{currentmarker}{}%
\end{pgfscope}%
\begin{pgfscope}%
\pgfsys@transformshift{9.634999in}{4.589240in}%
\pgfsys@useobject{currentmarker}{}%
\end{pgfscope}%
\begin{pgfscope}%
\pgfsys@transformshift{9.652592in}{4.813773in}%
\pgfsys@useobject{currentmarker}{}%
\end{pgfscope}%
\begin{pgfscope}%
\pgfsys@transformshift{9.670186in}{4.584205in}%
\pgfsys@useobject{currentmarker}{}%
\end{pgfscope}%
\begin{pgfscope}%
\pgfsys@transformshift{9.687780in}{4.690640in}%
\pgfsys@useobject{currentmarker}{}%
\end{pgfscope}%
\begin{pgfscope}%
\pgfsys@transformshift{9.705374in}{4.750662in}%
\pgfsys@useobject{currentmarker}{}%
\end{pgfscope}%
\begin{pgfscope}%
\pgfsys@transformshift{9.722968in}{4.952640in}%
\pgfsys@useobject{currentmarker}{}%
\end{pgfscope}%
\begin{pgfscope}%
\pgfsys@transformshift{9.740562in}{4.722467in}%
\pgfsys@useobject{currentmarker}{}%
\end{pgfscope}%
\begin{pgfscope}%
\pgfsys@transformshift{9.758155in}{4.734605in}%
\pgfsys@useobject{currentmarker}{}%
\end{pgfscope}%
\begin{pgfscope}%
\pgfsys@transformshift{9.775749in}{4.829081in}%
\pgfsys@useobject{currentmarker}{}%
\end{pgfscope}%
\begin{pgfscope}%
\pgfsys@transformshift{9.793343in}{4.791275in}%
\pgfsys@useobject{currentmarker}{}%
\end{pgfscope}%
\begin{pgfscope}%
\pgfsys@transformshift{9.810937in}{4.993005in}%
\pgfsys@useobject{currentmarker}{}%
\end{pgfscope}%
\begin{pgfscope}%
\pgfsys@transformshift{9.828531in}{4.786360in}%
\pgfsys@useobject{currentmarker}{}%
\end{pgfscope}%
\begin{pgfscope}%
\pgfsys@transformshift{9.846125in}{4.796843in}%
\pgfsys@useobject{currentmarker}{}%
\end{pgfscope}%
\begin{pgfscope}%
\pgfsys@transformshift{9.863719in}{4.885409in}%
\pgfsys@useobject{currentmarker}{}%
\end{pgfscope}%
\begin{pgfscope}%
\pgfsys@transformshift{9.881312in}{4.894144in}%
\pgfsys@useobject{currentmarker}{}%
\end{pgfscope}%
\begin{pgfscope}%
\pgfsys@transformshift{9.898906in}{5.155096in}%
\pgfsys@useobject{currentmarker}{}%
\end{pgfscope}%
\begin{pgfscope}%
\pgfsys@transformshift{9.916500in}{5.066958in}%
\pgfsys@useobject{currentmarker}{}%
\end{pgfscope}%
\begin{pgfscope}%
\pgfsys@transformshift{9.934094in}{4.989169in}%
\pgfsys@useobject{currentmarker}{}%
\end{pgfscope}%
\begin{pgfscope}%
\pgfsys@transformshift{9.951688in}{4.863576in}%
\pgfsys@useobject{currentmarker}{}%
\end{pgfscope}%
\begin{pgfscope}%
\pgfsys@transformshift{9.969282in}{5.081064in}%
\pgfsys@useobject{currentmarker}{}%
\end{pgfscope}%
\begin{pgfscope}%
\pgfsys@transformshift{9.986876in}{4.897460in}%
\pgfsys@useobject{currentmarker}{}%
\end{pgfscope}%
\begin{pgfscope}%
\pgfsys@transformshift{10.004469in}{4.844460in}%
\pgfsys@useobject{currentmarker}{}%
\end{pgfscope}%
\begin{pgfscope}%
\pgfsys@transformshift{10.022063in}{5.123690in}%
\pgfsys@useobject{currentmarker}{}%
\end{pgfscope}%
\begin{pgfscope}%
\pgfsys@transformshift{10.039657in}{5.033449in}%
\pgfsys@useobject{currentmarker}{}%
\end{pgfscope}%
\begin{pgfscope}%
\pgfsys@transformshift{10.057251in}{5.091681in}%
\pgfsys@useobject{currentmarker}{}%
\end{pgfscope}%
\begin{pgfscope}%
\pgfsys@transformshift{10.074845in}{5.025036in}%
\pgfsys@useobject{currentmarker}{}%
\end{pgfscope}%
\begin{pgfscope}%
\pgfsys@transformshift{10.092439in}{5.072843in}%
\pgfsys@useobject{currentmarker}{}%
\end{pgfscope}%
\begin{pgfscope}%
\pgfsys@transformshift{10.110033in}{4.910283in}%
\pgfsys@useobject{currentmarker}{}%
\end{pgfscope}%
\begin{pgfscope}%
\pgfsys@transformshift{10.127626in}{4.860777in}%
\pgfsys@useobject{currentmarker}{}%
\end{pgfscope}%
\begin{pgfscope}%
\pgfsys@transformshift{10.145220in}{5.023816in}%
\pgfsys@useobject{currentmarker}{}%
\end{pgfscope}%
\begin{pgfscope}%
\pgfsys@transformshift{10.162814in}{4.859088in}%
\pgfsys@useobject{currentmarker}{}%
\end{pgfscope}%
\begin{pgfscope}%
\pgfsys@transformshift{10.180408in}{4.856192in}%
\pgfsys@useobject{currentmarker}{}%
\end{pgfscope}%
\begin{pgfscope}%
\pgfsys@transformshift{10.198002in}{4.864505in}%
\pgfsys@useobject{currentmarker}{}%
\end{pgfscope}%
\begin{pgfscope}%
\pgfsys@transformshift{10.215596in}{4.896325in}%
\pgfsys@useobject{currentmarker}{}%
\end{pgfscope}%
\begin{pgfscope}%
\pgfsys@transformshift{10.233189in}{4.841352in}%
\pgfsys@useobject{currentmarker}{}%
\end{pgfscope}%
\begin{pgfscope}%
\pgfsys@transformshift{10.250783in}{4.719668in}%
\pgfsys@useobject{currentmarker}{}%
\end{pgfscope}%
\begin{pgfscope}%
\pgfsys@transformshift{10.268377in}{4.776392in}%
\pgfsys@useobject{currentmarker}{}%
\end{pgfscope}%
\begin{pgfscope}%
\pgfsys@transformshift{10.285971in}{4.597368in}%
\pgfsys@useobject{currentmarker}{}%
\end{pgfscope}%
\begin{pgfscope}%
\pgfsys@transformshift{10.303565in}{4.870059in}%
\pgfsys@useobject{currentmarker}{}%
\end{pgfscope}%
\begin{pgfscope}%
\pgfsys@transformshift{10.321159in}{4.625477in}%
\pgfsys@useobject{currentmarker}{}%
\end{pgfscope}%
\begin{pgfscope}%
\pgfsys@transformshift{10.338753in}{4.659313in}%
\pgfsys@useobject{currentmarker}{}%
\end{pgfscope}%
\begin{pgfscope}%
\pgfsys@transformshift{10.356346in}{4.761077in}%
\pgfsys@useobject{currentmarker}{}%
\end{pgfscope}%
\begin{pgfscope}%
\pgfsys@transformshift{10.373940in}{4.665102in}%
\pgfsys@useobject{currentmarker}{}%
\end{pgfscope}%
\begin{pgfscope}%
\pgfsys@transformshift{10.391534in}{4.883742in}%
\pgfsys@useobject{currentmarker}{}%
\end{pgfscope}%
\begin{pgfscope}%
\pgfsys@transformshift{10.409128in}{4.581725in}%
\pgfsys@useobject{currentmarker}{}%
\end{pgfscope}%
\begin{pgfscope}%
\pgfsys@transformshift{10.426722in}{4.729414in}%
\pgfsys@useobject{currentmarker}{}%
\end{pgfscope}%
\begin{pgfscope}%
\pgfsys@transformshift{10.444316in}{4.688394in}%
\pgfsys@useobject{currentmarker}{}%
\end{pgfscope}%
\begin{pgfscope}%
\pgfsys@transformshift{10.461910in}{4.565234in}%
\pgfsys@useobject{currentmarker}{}%
\end{pgfscope}%
\begin{pgfscope}%
\pgfsys@transformshift{10.479503in}{4.731505in}%
\pgfsys@useobject{currentmarker}{}%
\end{pgfscope}%
\begin{pgfscope}%
\pgfsys@transformshift{10.497097in}{4.656390in}%
\pgfsys@useobject{currentmarker}{}%
\end{pgfscope}%
\begin{pgfscope}%
\pgfsys@transformshift{10.514691in}{4.750347in}%
\pgfsys@useobject{currentmarker}{}%
\end{pgfscope}%
\begin{pgfscope}%
\pgfsys@transformshift{10.532285in}{4.755463in}%
\pgfsys@useobject{currentmarker}{}%
\end{pgfscope}%
\begin{pgfscope}%
\pgfsys@transformshift{10.549879in}{4.894045in}%
\pgfsys@useobject{currentmarker}{}%
\end{pgfscope}%
\begin{pgfscope}%
\pgfsys@transformshift{10.567473in}{4.813845in}%
\pgfsys@useobject{currentmarker}{}%
\end{pgfscope}%
\begin{pgfscope}%
\pgfsys@transformshift{10.585067in}{4.646147in}%
\pgfsys@useobject{currentmarker}{}%
\end{pgfscope}%
\begin{pgfscope}%
\pgfsys@transformshift{10.602660in}{4.667737in}%
\pgfsys@useobject{currentmarker}{}%
\end{pgfscope}%
\begin{pgfscope}%
\pgfsys@transformshift{10.620254in}{4.814708in}%
\pgfsys@useobject{currentmarker}{}%
\end{pgfscope}%
\begin{pgfscope}%
\pgfsys@transformshift{10.637848in}{4.781821in}%
\pgfsys@useobject{currentmarker}{}%
\end{pgfscope}%
\begin{pgfscope}%
\pgfsys@transformshift{10.655442in}{4.794632in}%
\pgfsys@useobject{currentmarker}{}%
\end{pgfscope}%
\begin{pgfscope}%
\pgfsys@transformshift{10.673036in}{4.580644in}%
\pgfsys@useobject{currentmarker}{}%
\end{pgfscope}%
\begin{pgfscope}%
\pgfsys@transformshift{10.690630in}{4.760929in}%
\pgfsys@useobject{currentmarker}{}%
\end{pgfscope}%
\begin{pgfscope}%
\pgfsys@transformshift{10.708223in}{4.707595in}%
\pgfsys@useobject{currentmarker}{}%
\end{pgfscope}%
\begin{pgfscope}%
\pgfsys@transformshift{10.725817in}{4.832262in}%
\pgfsys@useobject{currentmarker}{}%
\end{pgfscope}%
\begin{pgfscope}%
\pgfsys@transformshift{10.743411in}{4.815823in}%
\pgfsys@useobject{currentmarker}{}%
\end{pgfscope}%
\begin{pgfscope}%
\pgfsys@transformshift{10.761005in}{4.941839in}%
\pgfsys@useobject{currentmarker}{}%
\end{pgfscope}%
\begin{pgfscope}%
\pgfsys@transformshift{10.778599in}{4.905455in}%
\pgfsys@useobject{currentmarker}{}%
\end{pgfscope}%
\begin{pgfscope}%
\pgfsys@transformshift{10.796193in}{4.978072in}%
\pgfsys@useobject{currentmarker}{}%
\end{pgfscope}%
\begin{pgfscope}%
\pgfsys@transformshift{10.813787in}{4.875602in}%
\pgfsys@useobject{currentmarker}{}%
\end{pgfscope}%
\begin{pgfscope}%
\pgfsys@transformshift{10.831380in}{4.852426in}%
\pgfsys@useobject{currentmarker}{}%
\end{pgfscope}%
\begin{pgfscope}%
\pgfsys@transformshift{10.848974in}{4.932576in}%
\pgfsys@useobject{currentmarker}{}%
\end{pgfscope}%
\begin{pgfscope}%
\pgfsys@transformshift{10.866568in}{4.998190in}%
\pgfsys@useobject{currentmarker}{}%
\end{pgfscope}%
\begin{pgfscope}%
\pgfsys@transformshift{10.884162in}{5.063799in}%
\pgfsys@useobject{currentmarker}{}%
\end{pgfscope}%
\begin{pgfscope}%
\pgfsys@transformshift{10.901756in}{5.280211in}%
\pgfsys@useobject{currentmarker}{}%
\end{pgfscope}%
\begin{pgfscope}%
\pgfsys@transformshift{10.919350in}{5.069477in}%
\pgfsys@useobject{currentmarker}{}%
\end{pgfscope}%
\begin{pgfscope}%
\pgfsys@transformshift{10.936944in}{4.999365in}%
\pgfsys@useobject{currentmarker}{}%
\end{pgfscope}%
\begin{pgfscope}%
\pgfsys@transformshift{10.954537in}{5.083543in}%
\pgfsys@useobject{currentmarker}{}%
\end{pgfscope}%
\begin{pgfscope}%
\pgfsys@transformshift{10.972131in}{5.092281in}%
\pgfsys@useobject{currentmarker}{}%
\end{pgfscope}%
\begin{pgfscope}%
\pgfsys@transformshift{10.989725in}{5.207988in}%
\pgfsys@useobject{currentmarker}{}%
\end{pgfscope}%
\begin{pgfscope}%
\pgfsys@transformshift{11.007319in}{5.019570in}%
\pgfsys@useobject{currentmarker}{}%
\end{pgfscope}%
\begin{pgfscope}%
\pgfsys@transformshift{11.024913in}{5.199289in}%
\pgfsys@useobject{currentmarker}{}%
\end{pgfscope}%
\begin{pgfscope}%
\pgfsys@transformshift{11.042507in}{5.223190in}%
\pgfsys@useobject{currentmarker}{}%
\end{pgfscope}%
\begin{pgfscope}%
\pgfsys@transformshift{11.060101in}{5.243459in}%
\pgfsys@useobject{currentmarker}{}%
\end{pgfscope}%
\begin{pgfscope}%
\pgfsys@transformshift{11.077694in}{5.169522in}%
\pgfsys@useobject{currentmarker}{}%
\end{pgfscope}%
\begin{pgfscope}%
\pgfsys@transformshift{11.095288in}{5.214873in}%
\pgfsys@useobject{currentmarker}{}%
\end{pgfscope}%
\begin{pgfscope}%
\pgfsys@transformshift{11.112882in}{5.100648in}%
\pgfsys@useobject{currentmarker}{}%
\end{pgfscope}%
\begin{pgfscope}%
\pgfsys@transformshift{11.130476in}{5.200314in}%
\pgfsys@useobject{currentmarker}{}%
\end{pgfscope}%
\begin{pgfscope}%
\pgfsys@transformshift{11.148070in}{5.198061in}%
\pgfsys@useobject{currentmarker}{}%
\end{pgfscope}%
\begin{pgfscope}%
\pgfsys@transformshift{11.165664in}{5.296725in}%
\pgfsys@useobject{currentmarker}{}%
\end{pgfscope}%
\begin{pgfscope}%
\pgfsys@transformshift{11.183257in}{5.135419in}%
\pgfsys@useobject{currentmarker}{}%
\end{pgfscope}%
\begin{pgfscope}%
\pgfsys@transformshift{11.200851in}{5.330218in}%
\pgfsys@useobject{currentmarker}{}%
\end{pgfscope}%
\begin{pgfscope}%
\pgfsys@transformshift{11.218445in}{5.398508in}%
\pgfsys@useobject{currentmarker}{}%
\end{pgfscope}%
\begin{pgfscope}%
\pgfsys@transformshift{11.236039in}{5.029008in}%
\pgfsys@useobject{currentmarker}{}%
\end{pgfscope}%
\begin{pgfscope}%
\pgfsys@transformshift{11.253633in}{5.276102in}%
\pgfsys@useobject{currentmarker}{}%
\end{pgfscope}%
\begin{pgfscope}%
\pgfsys@transformshift{11.271227in}{5.292913in}%
\pgfsys@useobject{currentmarker}{}%
\end{pgfscope}%
\begin{pgfscope}%
\pgfsys@transformshift{11.288821in}{5.149004in}%
\pgfsys@useobject{currentmarker}{}%
\end{pgfscope}%
\begin{pgfscope}%
\pgfsys@transformshift{11.306414in}{5.162656in}%
\pgfsys@useobject{currentmarker}{}%
\end{pgfscope}%
\begin{pgfscope}%
\pgfsys@transformshift{11.324008in}{5.178002in}%
\pgfsys@useobject{currentmarker}{}%
\end{pgfscope}%
\begin{pgfscope}%
\pgfsys@transformshift{11.341602in}{5.148988in}%
\pgfsys@useobject{currentmarker}{}%
\end{pgfscope}%
\begin{pgfscope}%
\pgfsys@transformshift{11.359196in}{5.135262in}%
\pgfsys@useobject{currentmarker}{}%
\end{pgfscope}%
\begin{pgfscope}%
\pgfsys@transformshift{11.376790in}{4.983102in}%
\pgfsys@useobject{currentmarker}{}%
\end{pgfscope}%
\begin{pgfscope}%
\pgfsys@transformshift{11.394384in}{5.258780in}%
\pgfsys@useobject{currentmarker}{}%
\end{pgfscope}%
\begin{pgfscope}%
\pgfsys@transformshift{11.411978in}{5.238859in}%
\pgfsys@useobject{currentmarker}{}%
\end{pgfscope}%
\begin{pgfscope}%
\pgfsys@transformshift{11.429571in}{5.033756in}%
\pgfsys@useobject{currentmarker}{}%
\end{pgfscope}%
\begin{pgfscope}%
\pgfsys@transformshift{11.447165in}{4.955731in}%
\pgfsys@useobject{currentmarker}{}%
\end{pgfscope}%
\begin{pgfscope}%
\pgfsys@transformshift{11.464759in}{5.147808in}%
\pgfsys@useobject{currentmarker}{}%
\end{pgfscope}%
\begin{pgfscope}%
\pgfsys@transformshift{11.482353in}{5.026386in}%
\pgfsys@useobject{currentmarker}{}%
\end{pgfscope}%
\end{pgfscope}%
\begin{pgfscope}%
\pgfpathrectangle{\pgfqpoint{7.105882in}{4.421053in}}{\pgfqpoint{4.376471in}{0.978947in}} %
\pgfusepath{clip}%
\pgfsetroundcap%
\pgfsetroundjoin%
\pgfsetlinewidth{1.756562pt}%
\definecolor{currentstroke}{rgb}{0.298039,0.447059,0.690196}%
\pgfsetstrokecolor{currentstroke}%
\pgfsetdash{}{0pt}%
\pgfpathmoveto{\pgfqpoint{7.981176in}{4.788157in}}%
\pgfpathlineto{\pgfqpoint{11.482353in}{4.788159in}}%
\pgfpathlineto{\pgfqpoint{11.482353in}{4.788159in}}%
\pgfusepath{stroke}%
\end{pgfscope}%
\begin{pgfscope}%
\pgfsetrectcap%
\pgfsetmiterjoin%
\pgfsetlinewidth{1.003750pt}%
\definecolor{currentstroke}{rgb}{0.800000,0.800000,0.800000}%
\pgfsetstrokecolor{currentstroke}%
\pgfsetdash{}{0pt}%
\pgfpathmoveto{\pgfqpoint{7.105882in}{4.421053in}}%
\pgfpathlineto{\pgfqpoint{7.105882in}{5.400000in}}%
\pgfusepath{stroke}%
\end{pgfscope}%
\begin{pgfscope}%
\pgfsetrectcap%
\pgfsetmiterjoin%
\pgfsetlinewidth{1.003750pt}%
\definecolor{currentstroke}{rgb}{0.800000,0.800000,0.800000}%
\pgfsetstrokecolor{currentstroke}%
\pgfsetdash{}{0pt}%
\pgfpathmoveto{\pgfqpoint{11.482353in}{4.421053in}}%
\pgfpathlineto{\pgfqpoint{11.482353in}{5.400000in}}%
\pgfusepath{stroke}%
\end{pgfscope}%
\begin{pgfscope}%
\pgfsetrectcap%
\pgfsetmiterjoin%
\pgfsetlinewidth{1.003750pt}%
\definecolor{currentstroke}{rgb}{0.800000,0.800000,0.800000}%
\pgfsetstrokecolor{currentstroke}%
\pgfsetdash{}{0pt}%
\pgfpathmoveto{\pgfqpoint{7.105882in}{5.400000in}}%
\pgfpathlineto{\pgfqpoint{11.482353in}{5.400000in}}%
\pgfusepath{stroke}%
\end{pgfscope}%
\begin{pgfscope}%
\pgfsetrectcap%
\pgfsetmiterjoin%
\pgfsetlinewidth{1.003750pt}%
\definecolor{currentstroke}{rgb}{0.800000,0.800000,0.800000}%
\pgfsetstrokecolor{currentstroke}%
\pgfsetdash{}{0pt}%
\pgfpathmoveto{\pgfqpoint{7.105882in}{4.421053in}}%
\pgfpathlineto{\pgfqpoint{11.482353in}{4.421053in}}%
\pgfusepath{stroke}%
\end{pgfscope}%
\begin{pgfscope}%
\pgfsetroundcap%
\pgfsetroundjoin%
\pgfsetlinewidth{1.756562pt}%
\definecolor{currentstroke}{rgb}{0.298039,0.447059,0.690196}%
\pgfsetstrokecolor{currentstroke}%
\pgfsetdash{}{0pt}%
\pgfpathmoveto{\pgfqpoint{7.230882in}{5.015195in}}%
\pgfpathlineto{\pgfqpoint{7.508660in}{5.015195in}}%
\pgfusepath{stroke}%
\end{pgfscope}%
\begin{pgfscope}%
\definecolor{textcolor}{rgb}{0.150000,0.150000,0.150000}%
\pgfsetstrokecolor{textcolor}%
\pgfsetfillcolor{textcolor}%
\pgftext[x=7.619771in,y=4.966584in,left,base]{\color{textcolor}\sffamily\fontsize{10.000000}{12.000000}\selectfont \(\displaystyle \widetilde{\Phi}^* \theta^{\parallel}\)}%
\end{pgfscope}%
\begin{pgfscope}%
\pgfsetbuttcap%
\pgfsetroundjoin%
\definecolor{currentfill}{rgb}{1.000000,0.000000,0.000000}%
\pgfsetfillcolor{currentfill}%
\pgfsetlinewidth{2.007500pt}%
\definecolor{currentstroke}{rgb}{1.000000,0.000000,0.000000}%
\pgfsetstrokecolor{currentstroke}%
\pgfsetdash{}{0pt}%
\pgfpathmoveto{\pgfqpoint{7.338715in}{4.806577in}}%
\pgfpathlineto{\pgfqpoint{7.400828in}{4.806577in}}%
\pgfpathmoveto{\pgfqpoint{7.369771in}{4.775521in}}%
\pgfpathlineto{\pgfqpoint{7.369771in}{4.837634in}}%
\pgfusepath{stroke,fill}%
\end{pgfscope}%
\begin{pgfscope}%
\pgfsetbuttcap%
\pgfsetroundjoin%
\definecolor{currentfill}{rgb}{1.000000,0.000000,0.000000}%
\pgfsetfillcolor{currentfill}%
\pgfsetlinewidth{2.007500pt}%
\definecolor{currentstroke}{rgb}{1.000000,0.000000,0.000000}%
\pgfsetstrokecolor{currentstroke}%
\pgfsetdash{}{0pt}%
\pgfpathmoveto{\pgfqpoint{7.338715in}{4.806577in}}%
\pgfpathlineto{\pgfqpoint{7.400828in}{4.806577in}}%
\pgfpathmoveto{\pgfqpoint{7.369771in}{4.775521in}}%
\pgfpathlineto{\pgfqpoint{7.369771in}{4.837634in}}%
\pgfusepath{stroke,fill}%
\end{pgfscope}%
\begin{pgfscope}%
\pgfsetbuttcap%
\pgfsetroundjoin%
\definecolor{currentfill}{rgb}{1.000000,0.000000,0.000000}%
\pgfsetfillcolor{currentfill}%
\pgfsetlinewidth{2.007500pt}%
\definecolor{currentstroke}{rgb}{1.000000,0.000000,0.000000}%
\pgfsetstrokecolor{currentstroke}%
\pgfsetdash{}{0pt}%
\pgfpathmoveto{\pgfqpoint{7.338715in}{4.806577in}}%
\pgfpathlineto{\pgfqpoint{7.400828in}{4.806577in}}%
\pgfpathmoveto{\pgfqpoint{7.369771in}{4.775521in}}%
\pgfpathlineto{\pgfqpoint{7.369771in}{4.837634in}}%
\pgfusepath{stroke,fill}%
\end{pgfscope}%
\begin{pgfscope}%
\definecolor{textcolor}{rgb}{0.150000,0.150000,0.150000}%
\pgfsetstrokecolor{textcolor}%
\pgfsetfillcolor{textcolor}%
\pgftext[x=7.619771in,y=4.770119in,left,base]{\color{textcolor}\sffamily\fontsize{10.000000}{12.000000}\selectfont train}%
\end{pgfscope}%
\begin{pgfscope}%
\pgfsetbuttcap%
\pgfsetroundjoin%
\definecolor{currentfill}{rgb}{0.000000,0.000000,0.000000}%
\pgfsetfillcolor{currentfill}%
\pgfsetlinewidth{0.301125pt}%
\definecolor{currentstroke}{rgb}{0.000000,0.000000,0.000000}%
\pgfsetstrokecolor{currentstroke}%
\pgfsetdash{}{0pt}%
\pgfpathmoveto{\pgfqpoint{7.369771in}{4.594584in}}%
\pgfpathcurveto{\pgfqpoint{7.373889in}{4.594584in}}{\pgfqpoint{7.377839in}{4.596220in}}{\pgfqpoint{7.380751in}{4.599132in}}%
\pgfpathcurveto{\pgfqpoint{7.383663in}{4.602044in}}{\pgfqpoint{7.385299in}{4.605994in}}{\pgfqpoint{7.385299in}{4.610112in}}%
\pgfpathcurveto{\pgfqpoint{7.385299in}{4.614230in}}{\pgfqpoint{7.383663in}{4.618180in}}{\pgfqpoint{7.380751in}{4.621092in}}%
\pgfpathcurveto{\pgfqpoint{7.377839in}{4.624004in}}{\pgfqpoint{7.373889in}{4.625640in}}{\pgfqpoint{7.369771in}{4.625640in}}%
\pgfpathcurveto{\pgfqpoint{7.365653in}{4.625640in}}{\pgfqpoint{7.361703in}{4.624004in}}{\pgfqpoint{7.358791in}{4.621092in}}%
\pgfpathcurveto{\pgfqpoint{7.355879in}{4.618180in}}{\pgfqpoint{7.354243in}{4.614230in}}{\pgfqpoint{7.354243in}{4.610112in}}%
\pgfpathcurveto{\pgfqpoint{7.354243in}{4.605994in}}{\pgfqpoint{7.355879in}{4.602044in}}{\pgfqpoint{7.358791in}{4.599132in}}%
\pgfpathcurveto{\pgfqpoint{7.361703in}{4.596220in}}{\pgfqpoint{7.365653in}{4.594584in}}{\pgfqpoint{7.369771in}{4.594584in}}%
\pgfpathclose%
\pgfusepath{stroke,fill}%
\end{pgfscope}%
\begin{pgfscope}%
\pgfsetbuttcap%
\pgfsetroundjoin%
\definecolor{currentfill}{rgb}{0.000000,0.000000,0.000000}%
\pgfsetfillcolor{currentfill}%
\pgfsetlinewidth{0.301125pt}%
\definecolor{currentstroke}{rgb}{0.000000,0.000000,0.000000}%
\pgfsetstrokecolor{currentstroke}%
\pgfsetdash{}{0pt}%
\pgfpathmoveto{\pgfqpoint{7.369771in}{4.594584in}}%
\pgfpathcurveto{\pgfqpoint{7.373889in}{4.594584in}}{\pgfqpoint{7.377839in}{4.596220in}}{\pgfqpoint{7.380751in}{4.599132in}}%
\pgfpathcurveto{\pgfqpoint{7.383663in}{4.602044in}}{\pgfqpoint{7.385299in}{4.605994in}}{\pgfqpoint{7.385299in}{4.610112in}}%
\pgfpathcurveto{\pgfqpoint{7.385299in}{4.614230in}}{\pgfqpoint{7.383663in}{4.618180in}}{\pgfqpoint{7.380751in}{4.621092in}}%
\pgfpathcurveto{\pgfqpoint{7.377839in}{4.624004in}}{\pgfqpoint{7.373889in}{4.625640in}}{\pgfqpoint{7.369771in}{4.625640in}}%
\pgfpathcurveto{\pgfqpoint{7.365653in}{4.625640in}}{\pgfqpoint{7.361703in}{4.624004in}}{\pgfqpoint{7.358791in}{4.621092in}}%
\pgfpathcurveto{\pgfqpoint{7.355879in}{4.618180in}}{\pgfqpoint{7.354243in}{4.614230in}}{\pgfqpoint{7.354243in}{4.610112in}}%
\pgfpathcurveto{\pgfqpoint{7.354243in}{4.605994in}}{\pgfqpoint{7.355879in}{4.602044in}}{\pgfqpoint{7.358791in}{4.599132in}}%
\pgfpathcurveto{\pgfqpoint{7.361703in}{4.596220in}}{\pgfqpoint{7.365653in}{4.594584in}}{\pgfqpoint{7.369771in}{4.594584in}}%
\pgfpathclose%
\pgfusepath{stroke,fill}%
\end{pgfscope}%
\begin{pgfscope}%
\pgfsetbuttcap%
\pgfsetroundjoin%
\definecolor{currentfill}{rgb}{0.000000,0.000000,0.000000}%
\pgfsetfillcolor{currentfill}%
\pgfsetlinewidth{0.301125pt}%
\definecolor{currentstroke}{rgb}{0.000000,0.000000,0.000000}%
\pgfsetstrokecolor{currentstroke}%
\pgfsetdash{}{0pt}%
\pgfpathmoveto{\pgfqpoint{7.369771in}{4.594584in}}%
\pgfpathcurveto{\pgfqpoint{7.373889in}{4.594584in}}{\pgfqpoint{7.377839in}{4.596220in}}{\pgfqpoint{7.380751in}{4.599132in}}%
\pgfpathcurveto{\pgfqpoint{7.383663in}{4.602044in}}{\pgfqpoint{7.385299in}{4.605994in}}{\pgfqpoint{7.385299in}{4.610112in}}%
\pgfpathcurveto{\pgfqpoint{7.385299in}{4.614230in}}{\pgfqpoint{7.383663in}{4.618180in}}{\pgfqpoint{7.380751in}{4.621092in}}%
\pgfpathcurveto{\pgfqpoint{7.377839in}{4.624004in}}{\pgfqpoint{7.373889in}{4.625640in}}{\pgfqpoint{7.369771in}{4.625640in}}%
\pgfpathcurveto{\pgfqpoint{7.365653in}{4.625640in}}{\pgfqpoint{7.361703in}{4.624004in}}{\pgfqpoint{7.358791in}{4.621092in}}%
\pgfpathcurveto{\pgfqpoint{7.355879in}{4.618180in}}{\pgfqpoint{7.354243in}{4.614230in}}{\pgfqpoint{7.354243in}{4.610112in}}%
\pgfpathcurveto{\pgfqpoint{7.354243in}{4.605994in}}{\pgfqpoint{7.355879in}{4.602044in}}{\pgfqpoint{7.358791in}{4.599132in}}%
\pgfpathcurveto{\pgfqpoint{7.361703in}{4.596220in}}{\pgfqpoint{7.365653in}{4.594584in}}{\pgfqpoint{7.369771in}{4.594584in}}%
\pgfpathclose%
\pgfusepath{stroke,fill}%
\end{pgfscope}%
\begin{pgfscope}%
\definecolor{textcolor}{rgb}{0.150000,0.150000,0.150000}%
\pgfsetstrokecolor{textcolor}%
\pgfsetfillcolor{textcolor}%
\pgftext[x=7.619771in,y=4.573654in,left,base]{\color{textcolor}\sffamily\fontsize{10.000000}{12.000000}\selectfont test}%
\end{pgfscope}%
\begin{pgfscope}%
\pgfsetbuttcap%
\pgfsetmiterjoin%
\definecolor{currentfill}{rgb}{1.000000,1.000000,1.000000}%
\pgfsetfillcolor{currentfill}%
\pgfsetlinewidth{0.000000pt}%
\definecolor{currentstroke}{rgb}{0.000000,0.000000,0.000000}%
\pgfsetstrokecolor{currentstroke}%
\pgfsetstrokeopacity{0.000000}%
\pgfsetdash{}{0pt}%
\pgfpathmoveto{\pgfqpoint{12.211765in}{4.421053in}}%
\pgfpathlineto{\pgfqpoint{14.400000in}{4.421053in}}%
\pgfpathlineto{\pgfqpoint{14.400000in}{5.400000in}}%
\pgfpathlineto{\pgfqpoint{12.211765in}{5.400000in}}%
\pgfpathclose%
\pgfusepath{fill}%
\end{pgfscope}%
\begin{pgfscope}%
\pgfpathrectangle{\pgfqpoint{12.211765in}{4.421053in}}{\pgfqpoint{2.188235in}{0.978947in}} %
\pgfusepath{clip}%
\pgfsetroundcap%
\pgfsetroundjoin%
\pgfsetlinewidth{1.003750pt}%
\definecolor{currentstroke}{rgb}{0.800000,0.800000,0.800000}%
\pgfsetstrokecolor{currentstroke}%
\pgfsetdash{}{0pt}%
\pgfpathmoveto{\pgfqpoint{12.211765in}{4.421053in}}%
\pgfpathlineto{\pgfqpoint{12.211765in}{5.400000in}}%
\pgfusepath{stroke}%
\end{pgfscope}%
\begin{pgfscope}%
\pgfpathrectangle{\pgfqpoint{12.211765in}{4.421053in}}{\pgfqpoint{2.188235in}{0.978947in}} %
\pgfusepath{clip}%
\pgfsetroundcap%
\pgfsetroundjoin%
\pgfsetlinewidth{1.003750pt}%
\definecolor{currentstroke}{rgb}{0.800000,0.800000,0.800000}%
\pgfsetstrokecolor{currentstroke}%
\pgfsetdash{}{0pt}%
\pgfpathmoveto{\pgfqpoint{12.485294in}{4.421053in}}%
\pgfpathlineto{\pgfqpoint{12.485294in}{5.400000in}}%
\pgfusepath{stroke}%
\end{pgfscope}%
\begin{pgfscope}%
\pgfpathrectangle{\pgfqpoint{12.211765in}{4.421053in}}{\pgfqpoint{2.188235in}{0.978947in}} %
\pgfusepath{clip}%
\pgfsetroundcap%
\pgfsetroundjoin%
\pgfsetlinewidth{1.003750pt}%
\definecolor{currentstroke}{rgb}{0.800000,0.800000,0.800000}%
\pgfsetstrokecolor{currentstroke}%
\pgfsetdash{}{0pt}%
\pgfpathmoveto{\pgfqpoint{12.758824in}{4.421053in}}%
\pgfpathlineto{\pgfqpoint{12.758824in}{5.400000in}}%
\pgfusepath{stroke}%
\end{pgfscope}%
\begin{pgfscope}%
\pgfpathrectangle{\pgfqpoint{12.211765in}{4.421053in}}{\pgfqpoint{2.188235in}{0.978947in}} %
\pgfusepath{clip}%
\pgfsetroundcap%
\pgfsetroundjoin%
\pgfsetlinewidth{1.003750pt}%
\definecolor{currentstroke}{rgb}{0.800000,0.800000,0.800000}%
\pgfsetstrokecolor{currentstroke}%
\pgfsetdash{}{0pt}%
\pgfpathmoveto{\pgfqpoint{13.032353in}{4.421053in}}%
\pgfpathlineto{\pgfqpoint{13.032353in}{5.400000in}}%
\pgfusepath{stroke}%
\end{pgfscope}%
\begin{pgfscope}%
\pgfpathrectangle{\pgfqpoint{12.211765in}{4.421053in}}{\pgfqpoint{2.188235in}{0.978947in}} %
\pgfusepath{clip}%
\pgfsetroundcap%
\pgfsetroundjoin%
\pgfsetlinewidth{1.003750pt}%
\definecolor{currentstroke}{rgb}{0.800000,0.800000,0.800000}%
\pgfsetstrokecolor{currentstroke}%
\pgfsetdash{}{0pt}%
\pgfpathmoveto{\pgfqpoint{13.305882in}{4.421053in}}%
\pgfpathlineto{\pgfqpoint{13.305882in}{5.400000in}}%
\pgfusepath{stroke}%
\end{pgfscope}%
\begin{pgfscope}%
\pgfpathrectangle{\pgfqpoint{12.211765in}{4.421053in}}{\pgfqpoint{2.188235in}{0.978947in}} %
\pgfusepath{clip}%
\pgfsetroundcap%
\pgfsetroundjoin%
\pgfsetlinewidth{1.003750pt}%
\definecolor{currentstroke}{rgb}{0.800000,0.800000,0.800000}%
\pgfsetstrokecolor{currentstroke}%
\pgfsetdash{}{0pt}%
\pgfpathmoveto{\pgfqpoint{13.579412in}{4.421053in}}%
\pgfpathlineto{\pgfqpoint{13.579412in}{5.400000in}}%
\pgfusepath{stroke}%
\end{pgfscope}%
\begin{pgfscope}%
\pgfpathrectangle{\pgfqpoint{12.211765in}{4.421053in}}{\pgfqpoint{2.188235in}{0.978947in}} %
\pgfusepath{clip}%
\pgfsetroundcap%
\pgfsetroundjoin%
\pgfsetlinewidth{1.003750pt}%
\definecolor{currentstroke}{rgb}{0.800000,0.800000,0.800000}%
\pgfsetstrokecolor{currentstroke}%
\pgfsetdash{}{0pt}%
\pgfpathmoveto{\pgfqpoint{13.852941in}{4.421053in}}%
\pgfpathlineto{\pgfqpoint{13.852941in}{5.400000in}}%
\pgfusepath{stroke}%
\end{pgfscope}%
\begin{pgfscope}%
\pgfpathrectangle{\pgfqpoint{12.211765in}{4.421053in}}{\pgfqpoint{2.188235in}{0.978947in}} %
\pgfusepath{clip}%
\pgfsetroundcap%
\pgfsetroundjoin%
\pgfsetlinewidth{1.003750pt}%
\definecolor{currentstroke}{rgb}{0.800000,0.800000,0.800000}%
\pgfsetstrokecolor{currentstroke}%
\pgfsetdash{}{0pt}%
\pgfpathmoveto{\pgfqpoint{14.126471in}{4.421053in}}%
\pgfpathlineto{\pgfqpoint{14.126471in}{5.400000in}}%
\pgfusepath{stroke}%
\end{pgfscope}%
\begin{pgfscope}%
\pgfpathrectangle{\pgfqpoint{12.211765in}{4.421053in}}{\pgfqpoint{2.188235in}{0.978947in}} %
\pgfusepath{clip}%
\pgfsetroundcap%
\pgfsetroundjoin%
\pgfsetlinewidth{1.003750pt}%
\definecolor{currentstroke}{rgb}{0.800000,0.800000,0.800000}%
\pgfsetstrokecolor{currentstroke}%
\pgfsetdash{}{0pt}%
\pgfpathmoveto{\pgfqpoint{14.400000in}{4.421053in}}%
\pgfpathlineto{\pgfqpoint{14.400000in}{5.400000in}}%
\pgfusepath{stroke}%
\end{pgfscope}%
\begin{pgfscope}%
\pgfpathrectangle{\pgfqpoint{12.211765in}{4.421053in}}{\pgfqpoint{2.188235in}{0.978947in}} %
\pgfusepath{clip}%
\pgfsetroundcap%
\pgfsetroundjoin%
\pgfsetlinewidth{1.003750pt}%
\definecolor{currentstroke}{rgb}{0.800000,0.800000,0.800000}%
\pgfsetstrokecolor{currentstroke}%
\pgfsetdash{}{0pt}%
\pgfpathmoveto{\pgfqpoint{12.211765in}{4.421053in}}%
\pgfpathlineto{\pgfqpoint{14.400000in}{4.421053in}}%
\pgfusepath{stroke}%
\end{pgfscope}%
\begin{pgfscope}%
\definecolor{textcolor}{rgb}{0.150000,0.150000,0.150000}%
\pgfsetstrokecolor{textcolor}%
\pgfsetfillcolor{textcolor}%
\pgftext[x=12.114542in,y=4.421053in,right,]{\color{textcolor}\sffamily\fontsize{10.000000}{12.000000}\selectfont \(\displaystyle 0\)}%
\end{pgfscope}%
\begin{pgfscope}%
\pgfpathrectangle{\pgfqpoint{12.211765in}{4.421053in}}{\pgfqpoint{2.188235in}{0.978947in}} %
\pgfusepath{clip}%
\pgfsetroundcap%
\pgfsetroundjoin%
\pgfsetlinewidth{1.003750pt}%
\definecolor{currentstroke}{rgb}{0.800000,0.800000,0.800000}%
\pgfsetstrokecolor{currentstroke}%
\pgfsetdash{}{0pt}%
\pgfpathmoveto{\pgfqpoint{12.211765in}{4.665789in}}%
\pgfpathlineto{\pgfqpoint{14.400000in}{4.665789in}}%
\pgfusepath{stroke}%
\end{pgfscope}%
\begin{pgfscope}%
\definecolor{textcolor}{rgb}{0.150000,0.150000,0.150000}%
\pgfsetstrokecolor{textcolor}%
\pgfsetfillcolor{textcolor}%
\pgftext[x=12.114542in,y=4.665789in,right,]{\color{textcolor}\sffamily\fontsize{10.000000}{12.000000}\selectfont \(\displaystyle 50\)}%
\end{pgfscope}%
\begin{pgfscope}%
\pgfpathrectangle{\pgfqpoint{12.211765in}{4.421053in}}{\pgfqpoint{2.188235in}{0.978947in}} %
\pgfusepath{clip}%
\pgfsetroundcap%
\pgfsetroundjoin%
\pgfsetlinewidth{1.003750pt}%
\definecolor{currentstroke}{rgb}{0.800000,0.800000,0.800000}%
\pgfsetstrokecolor{currentstroke}%
\pgfsetdash{}{0pt}%
\pgfpathmoveto{\pgfqpoint{12.211765in}{4.910526in}}%
\pgfpathlineto{\pgfqpoint{14.400000in}{4.910526in}}%
\pgfusepath{stroke}%
\end{pgfscope}%
\begin{pgfscope}%
\definecolor{textcolor}{rgb}{0.150000,0.150000,0.150000}%
\pgfsetstrokecolor{textcolor}%
\pgfsetfillcolor{textcolor}%
\pgftext[x=12.114542in,y=4.910526in,right,]{\color{textcolor}\sffamily\fontsize{10.000000}{12.000000}\selectfont \(\displaystyle 100\)}%
\end{pgfscope}%
\begin{pgfscope}%
\pgfpathrectangle{\pgfqpoint{12.211765in}{4.421053in}}{\pgfqpoint{2.188235in}{0.978947in}} %
\pgfusepath{clip}%
\pgfsetroundcap%
\pgfsetroundjoin%
\pgfsetlinewidth{1.003750pt}%
\definecolor{currentstroke}{rgb}{0.800000,0.800000,0.800000}%
\pgfsetstrokecolor{currentstroke}%
\pgfsetdash{}{0pt}%
\pgfpathmoveto{\pgfqpoint{12.211765in}{5.155263in}}%
\pgfpathlineto{\pgfqpoint{14.400000in}{5.155263in}}%
\pgfusepath{stroke}%
\end{pgfscope}%
\begin{pgfscope}%
\definecolor{textcolor}{rgb}{0.150000,0.150000,0.150000}%
\pgfsetstrokecolor{textcolor}%
\pgfsetfillcolor{textcolor}%
\pgftext[x=12.114542in,y=5.155263in,right,]{\color{textcolor}\sffamily\fontsize{10.000000}{12.000000}\selectfont \(\displaystyle 150\)}%
\end{pgfscope}%
\begin{pgfscope}%
\pgfpathrectangle{\pgfqpoint{12.211765in}{4.421053in}}{\pgfqpoint{2.188235in}{0.978947in}} %
\pgfusepath{clip}%
\pgfsetroundcap%
\pgfsetroundjoin%
\pgfsetlinewidth{1.003750pt}%
\definecolor{currentstroke}{rgb}{0.800000,0.800000,0.800000}%
\pgfsetstrokecolor{currentstroke}%
\pgfsetdash{}{0pt}%
\pgfpathmoveto{\pgfqpoint{12.211765in}{5.400000in}}%
\pgfpathlineto{\pgfqpoint{14.400000in}{5.400000in}}%
\pgfusepath{stroke}%
\end{pgfscope}%
\begin{pgfscope}%
\definecolor{textcolor}{rgb}{0.150000,0.150000,0.150000}%
\pgfsetstrokecolor{textcolor}%
\pgfsetfillcolor{textcolor}%
\pgftext[x=12.114542in,y=5.400000in,right,]{\color{textcolor}\sffamily\fontsize{10.000000}{12.000000}\selectfont \(\displaystyle 200\)}%
\end{pgfscope}%
\begin{pgfscope}%
\definecolor{textcolor}{rgb}{0.150000,0.150000,0.150000}%
\pgfsetstrokecolor{textcolor}%
\pgfsetfillcolor{textcolor}%
\pgftext[x=11.836764in,y=4.910526in,,bottom,rotate=90.000000]{\color{textcolor}\sffamily\fontsize{11.000000}{13.200000}\selectfont \(\displaystyle \theta^{\parallel}_j\)}%
\end{pgfscope}%
\begin{pgfscope}%
\pgfpathrectangle{\pgfqpoint{12.211765in}{4.421053in}}{\pgfqpoint{2.188235in}{0.978947in}} %
\pgfusepath{clip}%
\pgfsetroundcap%
\pgfsetroundjoin%
\pgfsetlinewidth{1.756562pt}%
\definecolor{currentstroke}{rgb}{0.298039,0.447059,0.690196}%
\pgfsetstrokecolor{currentstroke}%
\pgfsetdash{}{0pt}%
\pgfpathmoveto{\pgfqpoint{13.305882in}{4.421053in}}%
\pgfpathlineto{\pgfqpoint{13.305882in}{5.395105in}}%
\pgfpathlineto{\pgfqpoint{13.305882in}{5.395105in}}%
\pgfusepath{stroke}%
\end{pgfscope}%
\begin{pgfscope}%
\pgfsetrectcap%
\pgfsetmiterjoin%
\pgfsetlinewidth{1.003750pt}%
\definecolor{currentstroke}{rgb}{0.800000,0.800000,0.800000}%
\pgfsetstrokecolor{currentstroke}%
\pgfsetdash{}{0pt}%
\pgfpathmoveto{\pgfqpoint{12.211765in}{4.421053in}}%
\pgfpathlineto{\pgfqpoint{12.211765in}{5.400000in}}%
\pgfusepath{stroke}%
\end{pgfscope}%
\begin{pgfscope}%
\pgfsetrectcap%
\pgfsetmiterjoin%
\pgfsetlinewidth{1.003750pt}%
\definecolor{currentstroke}{rgb}{0.800000,0.800000,0.800000}%
\pgfsetstrokecolor{currentstroke}%
\pgfsetdash{}{0pt}%
\pgfpathmoveto{\pgfqpoint{14.400000in}{4.421053in}}%
\pgfpathlineto{\pgfqpoint{14.400000in}{5.400000in}}%
\pgfusepath{stroke}%
\end{pgfscope}%
\begin{pgfscope}%
\pgfsetrectcap%
\pgfsetmiterjoin%
\pgfsetlinewidth{1.003750pt}%
\definecolor{currentstroke}{rgb}{0.800000,0.800000,0.800000}%
\pgfsetstrokecolor{currentstroke}%
\pgfsetdash{}{0pt}%
\pgfpathmoveto{\pgfqpoint{12.211765in}{5.400000in}}%
\pgfpathlineto{\pgfqpoint{14.400000in}{5.400000in}}%
\pgfusepath{stroke}%
\end{pgfscope}%
\begin{pgfscope}%
\pgfsetrectcap%
\pgfsetmiterjoin%
\pgfsetlinewidth{1.003750pt}%
\definecolor{currentstroke}{rgb}{0.800000,0.800000,0.800000}%
\pgfsetstrokecolor{currentstroke}%
\pgfsetdash{}{0pt}%
\pgfpathmoveto{\pgfqpoint{12.211765in}{4.421053in}}%
\pgfpathlineto{\pgfqpoint{14.400000in}{4.421053in}}%
\pgfusepath{stroke}%
\end{pgfscope}%
\begin{pgfscope}%
\pgfsetbuttcap%
\pgfsetmiterjoin%
\definecolor{currentfill}{rgb}{1.000000,1.000000,1.000000}%
\pgfsetfillcolor{currentfill}%
\pgfsetlinewidth{0.000000pt}%
\definecolor{currentstroke}{rgb}{0.000000,0.000000,0.000000}%
\pgfsetstrokecolor{currentstroke}%
\pgfsetstrokeopacity{0.000000}%
\pgfsetdash{}{0pt}%
\pgfpathmoveto{\pgfqpoint{2.000000in}{3.197368in}}%
\pgfpathlineto{\pgfqpoint{6.376471in}{3.197368in}}%
\pgfpathlineto{\pgfqpoint{6.376471in}{4.176316in}}%
\pgfpathlineto{\pgfqpoint{2.000000in}{4.176316in}}%
\pgfpathclose%
\pgfusepath{fill}%
\end{pgfscope}%
\begin{pgfscope}%
\pgfpathrectangle{\pgfqpoint{2.000000in}{3.197368in}}{\pgfqpoint{4.376471in}{0.978947in}} %
\pgfusepath{clip}%
\pgfsetroundcap%
\pgfsetroundjoin%
\pgfsetlinewidth{1.003750pt}%
\definecolor{currentstroke}{rgb}{0.800000,0.800000,0.800000}%
\pgfsetstrokecolor{currentstroke}%
\pgfsetdash{}{0pt}%
\pgfpathmoveto{\pgfqpoint{2.000000in}{3.197368in}}%
\pgfpathlineto{\pgfqpoint{2.000000in}{4.176316in}}%
\pgfusepath{stroke}%
\end{pgfscope}%
\begin{pgfscope}%
\pgfpathrectangle{\pgfqpoint{2.000000in}{3.197368in}}{\pgfqpoint{4.376471in}{0.978947in}} %
\pgfusepath{clip}%
\pgfsetroundcap%
\pgfsetroundjoin%
\pgfsetlinewidth{1.003750pt}%
\definecolor{currentstroke}{rgb}{0.800000,0.800000,0.800000}%
\pgfsetstrokecolor{currentstroke}%
\pgfsetdash{}{0pt}%
\pgfpathmoveto{\pgfqpoint{2.875294in}{3.197368in}}%
\pgfpathlineto{\pgfqpoint{2.875294in}{4.176316in}}%
\pgfusepath{stroke}%
\end{pgfscope}%
\begin{pgfscope}%
\pgfpathrectangle{\pgfqpoint{2.000000in}{3.197368in}}{\pgfqpoint{4.376471in}{0.978947in}} %
\pgfusepath{clip}%
\pgfsetroundcap%
\pgfsetroundjoin%
\pgfsetlinewidth{1.003750pt}%
\definecolor{currentstroke}{rgb}{0.800000,0.800000,0.800000}%
\pgfsetstrokecolor{currentstroke}%
\pgfsetdash{}{0pt}%
\pgfpathmoveto{\pgfqpoint{3.750588in}{3.197368in}}%
\pgfpathlineto{\pgfqpoint{3.750588in}{4.176316in}}%
\pgfusepath{stroke}%
\end{pgfscope}%
\begin{pgfscope}%
\pgfpathrectangle{\pgfqpoint{2.000000in}{3.197368in}}{\pgfqpoint{4.376471in}{0.978947in}} %
\pgfusepath{clip}%
\pgfsetroundcap%
\pgfsetroundjoin%
\pgfsetlinewidth{1.003750pt}%
\definecolor{currentstroke}{rgb}{0.800000,0.800000,0.800000}%
\pgfsetstrokecolor{currentstroke}%
\pgfsetdash{}{0pt}%
\pgfpathmoveto{\pgfqpoint{4.625882in}{3.197368in}}%
\pgfpathlineto{\pgfqpoint{4.625882in}{4.176316in}}%
\pgfusepath{stroke}%
\end{pgfscope}%
\begin{pgfscope}%
\pgfpathrectangle{\pgfqpoint{2.000000in}{3.197368in}}{\pgfqpoint{4.376471in}{0.978947in}} %
\pgfusepath{clip}%
\pgfsetroundcap%
\pgfsetroundjoin%
\pgfsetlinewidth{1.003750pt}%
\definecolor{currentstroke}{rgb}{0.800000,0.800000,0.800000}%
\pgfsetstrokecolor{currentstroke}%
\pgfsetdash{}{0pt}%
\pgfpathmoveto{\pgfqpoint{5.501176in}{3.197368in}}%
\pgfpathlineto{\pgfqpoint{5.501176in}{4.176316in}}%
\pgfusepath{stroke}%
\end{pgfscope}%
\begin{pgfscope}%
\pgfpathrectangle{\pgfqpoint{2.000000in}{3.197368in}}{\pgfqpoint{4.376471in}{0.978947in}} %
\pgfusepath{clip}%
\pgfsetroundcap%
\pgfsetroundjoin%
\pgfsetlinewidth{1.003750pt}%
\definecolor{currentstroke}{rgb}{0.800000,0.800000,0.800000}%
\pgfsetstrokecolor{currentstroke}%
\pgfsetdash{}{0pt}%
\pgfpathmoveto{\pgfqpoint{6.376471in}{3.197368in}}%
\pgfpathlineto{\pgfqpoint{6.376471in}{4.176316in}}%
\pgfusepath{stroke}%
\end{pgfscope}%
\begin{pgfscope}%
\pgfpathrectangle{\pgfqpoint{2.000000in}{3.197368in}}{\pgfqpoint{4.376471in}{0.978947in}} %
\pgfusepath{clip}%
\pgfsetroundcap%
\pgfsetroundjoin%
\pgfsetlinewidth{1.003750pt}%
\definecolor{currentstroke}{rgb}{0.800000,0.800000,0.800000}%
\pgfsetstrokecolor{currentstroke}%
\pgfsetdash{}{0pt}%
\pgfpathmoveto{\pgfqpoint{2.000000in}{3.360526in}}%
\pgfpathlineto{\pgfqpoint{6.376471in}{3.360526in}}%
\pgfusepath{stroke}%
\end{pgfscope}%
\begin{pgfscope}%
\definecolor{textcolor}{rgb}{0.150000,0.150000,0.150000}%
\pgfsetstrokecolor{textcolor}%
\pgfsetfillcolor{textcolor}%
\pgftext[x=1.902778in,y=3.360526in,right,]{\color{textcolor}\sffamily\fontsize{10.000000}{12.000000}\selectfont \(\displaystyle -1\)}%
\end{pgfscope}%
\begin{pgfscope}%
\pgfpathrectangle{\pgfqpoint{2.000000in}{3.197368in}}{\pgfqpoint{4.376471in}{0.978947in}} %
\pgfusepath{clip}%
\pgfsetroundcap%
\pgfsetroundjoin%
\pgfsetlinewidth{1.003750pt}%
\definecolor{currentstroke}{rgb}{0.800000,0.800000,0.800000}%
\pgfsetstrokecolor{currentstroke}%
\pgfsetdash{}{0pt}%
\pgfpathmoveto{\pgfqpoint{2.000000in}{3.564474in}}%
\pgfpathlineto{\pgfqpoint{6.376471in}{3.564474in}}%
\pgfusepath{stroke}%
\end{pgfscope}%
\begin{pgfscope}%
\definecolor{textcolor}{rgb}{0.150000,0.150000,0.150000}%
\pgfsetstrokecolor{textcolor}%
\pgfsetfillcolor{textcolor}%
\pgftext[x=1.902778in,y=3.564474in,right,]{\color{textcolor}\sffamily\fontsize{10.000000}{12.000000}\selectfont \(\displaystyle 0\)}%
\end{pgfscope}%
\begin{pgfscope}%
\pgfpathrectangle{\pgfqpoint{2.000000in}{3.197368in}}{\pgfqpoint{4.376471in}{0.978947in}} %
\pgfusepath{clip}%
\pgfsetroundcap%
\pgfsetroundjoin%
\pgfsetlinewidth{1.003750pt}%
\definecolor{currentstroke}{rgb}{0.800000,0.800000,0.800000}%
\pgfsetstrokecolor{currentstroke}%
\pgfsetdash{}{0pt}%
\pgfpathmoveto{\pgfqpoint{2.000000in}{3.768421in}}%
\pgfpathlineto{\pgfqpoint{6.376471in}{3.768421in}}%
\pgfusepath{stroke}%
\end{pgfscope}%
\begin{pgfscope}%
\definecolor{textcolor}{rgb}{0.150000,0.150000,0.150000}%
\pgfsetstrokecolor{textcolor}%
\pgfsetfillcolor{textcolor}%
\pgftext[x=1.902778in,y=3.768421in,right,]{\color{textcolor}\sffamily\fontsize{10.000000}{12.000000}\selectfont \(\displaystyle 1\)}%
\end{pgfscope}%
\begin{pgfscope}%
\pgfpathrectangle{\pgfqpoint{2.000000in}{3.197368in}}{\pgfqpoint{4.376471in}{0.978947in}} %
\pgfusepath{clip}%
\pgfsetroundcap%
\pgfsetroundjoin%
\pgfsetlinewidth{1.003750pt}%
\definecolor{currentstroke}{rgb}{0.800000,0.800000,0.800000}%
\pgfsetstrokecolor{currentstroke}%
\pgfsetdash{}{0pt}%
\pgfpathmoveto{\pgfqpoint{2.000000in}{3.972368in}}%
\pgfpathlineto{\pgfqpoint{6.376471in}{3.972368in}}%
\pgfusepath{stroke}%
\end{pgfscope}%
\begin{pgfscope}%
\definecolor{textcolor}{rgb}{0.150000,0.150000,0.150000}%
\pgfsetstrokecolor{textcolor}%
\pgfsetfillcolor{textcolor}%
\pgftext[x=1.902778in,y=3.972368in,right,]{\color{textcolor}\sffamily\fontsize{10.000000}{12.000000}\selectfont \(\displaystyle 2\)}%
\end{pgfscope}%
\begin{pgfscope}%
\pgfpathrectangle{\pgfqpoint{2.000000in}{3.197368in}}{\pgfqpoint{4.376471in}{0.978947in}} %
\pgfusepath{clip}%
\pgfsetroundcap%
\pgfsetroundjoin%
\pgfsetlinewidth{1.003750pt}%
\definecolor{currentstroke}{rgb}{0.800000,0.800000,0.800000}%
\pgfsetstrokecolor{currentstroke}%
\pgfsetdash{}{0pt}%
\pgfpathmoveto{\pgfqpoint{2.000000in}{4.176316in}}%
\pgfpathlineto{\pgfqpoint{6.376471in}{4.176316in}}%
\pgfusepath{stroke}%
\end{pgfscope}%
\begin{pgfscope}%
\definecolor{textcolor}{rgb}{0.150000,0.150000,0.150000}%
\pgfsetstrokecolor{textcolor}%
\pgfsetfillcolor{textcolor}%
\pgftext[x=1.902778in,y=4.176316in,right,]{\color{textcolor}\sffamily\fontsize{10.000000}{12.000000}\selectfont \(\displaystyle 3\)}%
\end{pgfscope}%
\begin{pgfscope}%
\definecolor{textcolor}{rgb}{0.150000,0.150000,0.150000}%
\pgfsetstrokecolor{textcolor}%
\pgfsetfillcolor{textcolor}%
\pgftext[x=1.655864in,y=3.686842in,,bottom,rotate=90.000000]{\color{textcolor}\sffamily\fontsize{11.000000}{13.200000}\selectfont y}%
\end{pgfscope}%
\begin{pgfscope}%
\pgfpathrectangle{\pgfqpoint{2.000000in}{3.197368in}}{\pgfqpoint{4.376471in}{0.978947in}} %
\pgfusepath{clip}%
\pgfsetbuttcap%
\pgfsetroundjoin%
\definecolor{currentfill}{rgb}{1.000000,0.000000,0.000000}%
\pgfsetfillcolor{currentfill}%
\pgfsetlinewidth{2.007500pt}%
\definecolor{currentstroke}{rgb}{1.000000,0.000000,0.000000}%
\pgfsetstrokecolor{currentstroke}%
\pgfsetdash{}{0pt}%
\pgfpathmoveto{\pgfqpoint{4.765731in}{3.754944in}}%
\pgfpathlineto{\pgfqpoint{4.827844in}{3.754944in}}%
\pgfpathmoveto{\pgfqpoint{4.796787in}{3.723888in}}%
\pgfpathlineto{\pgfqpoint{4.796787in}{3.786001in}}%
\pgfusepath{stroke,fill}%
\end{pgfscope}%
\begin{pgfscope}%
\pgfpathrectangle{\pgfqpoint{2.000000in}{3.197368in}}{\pgfqpoint{4.376471in}{0.978947in}} %
\pgfusepath{clip}%
\pgfsetbuttcap%
\pgfsetroundjoin%
\definecolor{currentfill}{rgb}{1.000000,0.000000,0.000000}%
\pgfsetfillcolor{currentfill}%
\pgfsetlinewidth{2.007500pt}%
\definecolor{currentstroke}{rgb}{1.000000,0.000000,0.000000}%
\pgfsetstrokecolor{currentstroke}%
\pgfsetdash{}{0pt}%
\pgfpathmoveto{\pgfqpoint{5.348242in}{3.437732in}}%
\pgfpathlineto{\pgfqpoint{5.410355in}{3.437732in}}%
\pgfpathmoveto{\pgfqpoint{5.379298in}{3.406676in}}%
\pgfpathlineto{\pgfqpoint{5.379298in}{3.468789in}}%
\pgfusepath{stroke,fill}%
\end{pgfscope}%
\begin{pgfscope}%
\pgfpathrectangle{\pgfqpoint{2.000000in}{3.197368in}}{\pgfqpoint{4.376471in}{0.978947in}} %
\pgfusepath{clip}%
\pgfsetbuttcap%
\pgfsetroundjoin%
\definecolor{currentfill}{rgb}{1.000000,0.000000,0.000000}%
\pgfsetfillcolor{currentfill}%
\pgfsetlinewidth{2.007500pt}%
\definecolor{currentstroke}{rgb}{1.000000,0.000000,0.000000}%
\pgfsetstrokecolor{currentstroke}%
\pgfsetdash{}{0pt}%
\pgfpathmoveto{\pgfqpoint{4.954619in}{3.808472in}}%
\pgfpathlineto{\pgfqpoint{5.016732in}{3.808472in}}%
\pgfpathmoveto{\pgfqpoint{4.985675in}{3.777416in}}%
\pgfpathlineto{\pgfqpoint{4.985675in}{3.839529in}}%
\pgfusepath{stroke,fill}%
\end{pgfscope}%
\begin{pgfscope}%
\pgfpathrectangle{\pgfqpoint{2.000000in}{3.197368in}}{\pgfqpoint{4.376471in}{0.978947in}} %
\pgfusepath{clip}%
\pgfsetbuttcap%
\pgfsetroundjoin%
\definecolor{currentfill}{rgb}{1.000000,0.000000,0.000000}%
\pgfsetfillcolor{currentfill}%
\pgfsetlinewidth{2.007500pt}%
\definecolor{currentstroke}{rgb}{1.000000,0.000000,0.000000}%
\pgfsetstrokecolor{currentstroke}%
\pgfsetdash{}{0pt}%
\pgfpathmoveto{\pgfqpoint{4.751970in}{3.697560in}}%
\pgfpathlineto{\pgfqpoint{4.814083in}{3.697560in}}%
\pgfpathmoveto{\pgfqpoint{4.783026in}{3.666503in}}%
\pgfpathlineto{\pgfqpoint{4.783026in}{3.728616in}}%
\pgfusepath{stroke,fill}%
\end{pgfscope}%
\begin{pgfscope}%
\pgfpathrectangle{\pgfqpoint{2.000000in}{3.197368in}}{\pgfqpoint{4.376471in}{0.978947in}} %
\pgfusepath{clip}%
\pgfsetbuttcap%
\pgfsetroundjoin%
\definecolor{currentfill}{rgb}{1.000000,0.000000,0.000000}%
\pgfsetfillcolor{currentfill}%
\pgfsetlinewidth{2.007500pt}%
\definecolor{currentstroke}{rgb}{1.000000,0.000000,0.000000}%
\pgfsetstrokecolor{currentstroke}%
\pgfsetdash{}{0pt}%
\pgfpathmoveto{\pgfqpoint{4.327528in}{3.371166in}}%
\pgfpathlineto{\pgfqpoint{4.389641in}{3.371166in}}%
\pgfpathmoveto{\pgfqpoint{4.358584in}{3.340109in}}%
\pgfpathlineto{\pgfqpoint{4.358584in}{3.402222in}}%
\pgfusepath{stroke,fill}%
\end{pgfscope}%
\begin{pgfscope}%
\pgfpathrectangle{\pgfqpoint{2.000000in}{3.197368in}}{\pgfqpoint{4.376471in}{0.978947in}} %
\pgfusepath{clip}%
\pgfsetbuttcap%
\pgfsetroundjoin%
\definecolor{currentfill}{rgb}{1.000000,0.000000,0.000000}%
\pgfsetfillcolor{currentfill}%
\pgfsetlinewidth{2.007500pt}%
\definecolor{currentstroke}{rgb}{1.000000,0.000000,0.000000}%
\pgfsetstrokecolor{currentstroke}%
\pgfsetdash{}{0pt}%
\pgfpathmoveto{\pgfqpoint{5.105627in}{3.640630in}}%
\pgfpathlineto{\pgfqpoint{5.167740in}{3.640630in}}%
\pgfpathmoveto{\pgfqpoint{5.136683in}{3.609573in}}%
\pgfpathlineto{\pgfqpoint{5.136683in}{3.671686in}}%
\pgfusepath{stroke,fill}%
\end{pgfscope}%
\begin{pgfscope}%
\pgfpathrectangle{\pgfqpoint{2.000000in}{3.197368in}}{\pgfqpoint{4.376471in}{0.978947in}} %
\pgfusepath{clip}%
\pgfsetbuttcap%
\pgfsetroundjoin%
\definecolor{currentfill}{rgb}{1.000000,0.000000,0.000000}%
\pgfsetfillcolor{currentfill}%
\pgfsetlinewidth{2.007500pt}%
\definecolor{currentstroke}{rgb}{1.000000,0.000000,0.000000}%
\pgfsetstrokecolor{currentstroke}%
\pgfsetdash{}{0pt}%
\pgfpathmoveto{\pgfqpoint{4.376308in}{3.408659in}}%
\pgfpathlineto{\pgfqpoint{4.438421in}{3.408659in}}%
\pgfpathmoveto{\pgfqpoint{4.407364in}{3.377603in}}%
\pgfpathlineto{\pgfqpoint{4.407364in}{3.439716in}}%
\pgfusepath{stroke,fill}%
\end{pgfscope}%
\begin{pgfscope}%
\pgfpathrectangle{\pgfqpoint{2.000000in}{3.197368in}}{\pgfqpoint{4.376471in}{0.978947in}} %
\pgfusepath{clip}%
\pgfsetbuttcap%
\pgfsetroundjoin%
\definecolor{currentfill}{rgb}{1.000000,0.000000,0.000000}%
\pgfsetfillcolor{currentfill}%
\pgfsetlinewidth{2.007500pt}%
\definecolor{currentstroke}{rgb}{1.000000,0.000000,0.000000}%
\pgfsetstrokecolor{currentstroke}%
\pgfsetdash{}{0pt}%
\pgfpathmoveto{\pgfqpoint{5.966492in}{4.048778in}}%
\pgfpathlineto{\pgfqpoint{6.028605in}{4.048778in}}%
\pgfpathmoveto{\pgfqpoint{5.997549in}{4.017721in}}%
\pgfpathlineto{\pgfqpoint{5.997549in}{4.079834in}}%
\pgfusepath{stroke,fill}%
\end{pgfscope}%
\begin{pgfscope}%
\pgfpathrectangle{\pgfqpoint{2.000000in}{3.197368in}}{\pgfqpoint{4.376471in}{0.978947in}} %
\pgfusepath{clip}%
\pgfsetbuttcap%
\pgfsetroundjoin%
\definecolor{currentfill}{rgb}{1.000000,0.000000,0.000000}%
\pgfsetfillcolor{currentfill}%
\pgfsetlinewidth{2.007500pt}%
\definecolor{currentstroke}{rgb}{1.000000,0.000000,0.000000}%
\pgfsetstrokecolor{currentstroke}%
\pgfsetdash{}{0pt}%
\pgfpathmoveto{\pgfqpoint{6.218191in}{3.949180in}}%
\pgfpathlineto{\pgfqpoint{6.280304in}{3.949180in}}%
\pgfpathmoveto{\pgfqpoint{6.249248in}{3.918124in}}%
\pgfpathlineto{\pgfqpoint{6.249248in}{3.980237in}}%
\pgfusepath{stroke,fill}%
\end{pgfscope}%
\begin{pgfscope}%
\pgfpathrectangle{\pgfqpoint{2.000000in}{3.197368in}}{\pgfqpoint{4.376471in}{0.978947in}} %
\pgfusepath{clip}%
\pgfsetbuttcap%
\pgfsetroundjoin%
\definecolor{currentfill}{rgb}{1.000000,0.000000,0.000000}%
\pgfsetfillcolor{currentfill}%
\pgfsetlinewidth{2.007500pt}%
\definecolor{currentstroke}{rgb}{1.000000,0.000000,0.000000}%
\pgfsetstrokecolor{currentstroke}%
\pgfsetdash{}{0pt}%
\pgfpathmoveto{\pgfqpoint{4.186734in}{3.446577in}}%
\pgfpathlineto{\pgfqpoint{4.248847in}{3.446577in}}%
\pgfpathmoveto{\pgfqpoint{4.217791in}{3.415520in}}%
\pgfpathlineto{\pgfqpoint{4.217791in}{3.477633in}}%
\pgfusepath{stroke,fill}%
\end{pgfscope}%
\begin{pgfscope}%
\pgfpathrectangle{\pgfqpoint{2.000000in}{3.197368in}}{\pgfqpoint{4.376471in}{0.978947in}} %
\pgfusepath{clip}%
\pgfsetbuttcap%
\pgfsetroundjoin%
\definecolor{currentfill}{rgb}{1.000000,0.000000,0.000000}%
\pgfsetfillcolor{currentfill}%
\pgfsetlinewidth{2.007500pt}%
\definecolor{currentstroke}{rgb}{1.000000,0.000000,0.000000}%
\pgfsetstrokecolor{currentstroke}%
\pgfsetdash{}{0pt}%
\pgfpathmoveto{\pgfqpoint{5.616207in}{3.596752in}}%
\pgfpathlineto{\pgfqpoint{5.678320in}{3.596752in}}%
\pgfpathmoveto{\pgfqpoint{5.647263in}{3.565696in}}%
\pgfpathlineto{\pgfqpoint{5.647263in}{3.627809in}}%
\pgfusepath{stroke,fill}%
\end{pgfscope}%
\begin{pgfscope}%
\pgfpathrectangle{\pgfqpoint{2.000000in}{3.197368in}}{\pgfqpoint{4.376471in}{0.978947in}} %
\pgfusepath{clip}%
\pgfsetbuttcap%
\pgfsetroundjoin%
\definecolor{currentfill}{rgb}{1.000000,0.000000,0.000000}%
\pgfsetfillcolor{currentfill}%
\pgfsetlinewidth{2.007500pt}%
\definecolor{currentstroke}{rgb}{1.000000,0.000000,0.000000}%
\pgfsetstrokecolor{currentstroke}%
\pgfsetdash{}{0pt}%
\pgfpathmoveto{\pgfqpoint{4.695992in}{3.636847in}}%
\pgfpathlineto{\pgfqpoint{4.758105in}{3.636847in}}%
\pgfpathmoveto{\pgfqpoint{4.727049in}{3.605791in}}%
\pgfpathlineto{\pgfqpoint{4.727049in}{3.667904in}}%
\pgfusepath{stroke,fill}%
\end{pgfscope}%
\begin{pgfscope}%
\pgfpathrectangle{\pgfqpoint{2.000000in}{3.197368in}}{\pgfqpoint{4.376471in}{0.978947in}} %
\pgfusepath{clip}%
\pgfsetbuttcap%
\pgfsetroundjoin%
\definecolor{currentfill}{rgb}{1.000000,0.000000,0.000000}%
\pgfsetfillcolor{currentfill}%
\pgfsetlinewidth{2.007500pt}%
\definecolor{currentstroke}{rgb}{1.000000,0.000000,0.000000}%
\pgfsetstrokecolor{currentstroke}%
\pgfsetdash{}{0pt}%
\pgfpathmoveto{\pgfqpoint{4.833062in}{3.764403in}}%
\pgfpathlineto{\pgfqpoint{4.895175in}{3.764403in}}%
\pgfpathmoveto{\pgfqpoint{4.864118in}{3.733347in}}%
\pgfpathlineto{\pgfqpoint{4.864118in}{3.795460in}}%
\pgfusepath{stroke,fill}%
\end{pgfscope}%
\begin{pgfscope}%
\pgfpathrectangle{\pgfqpoint{2.000000in}{3.197368in}}{\pgfqpoint{4.376471in}{0.978947in}} %
\pgfusepath{clip}%
\pgfsetbuttcap%
\pgfsetroundjoin%
\definecolor{currentfill}{rgb}{1.000000,0.000000,0.000000}%
\pgfsetfillcolor{currentfill}%
\pgfsetlinewidth{2.007500pt}%
\definecolor{currentstroke}{rgb}{1.000000,0.000000,0.000000}%
\pgfsetstrokecolor{currentstroke}%
\pgfsetdash{}{0pt}%
\pgfpathmoveto{\pgfqpoint{6.084915in}{4.024567in}}%
\pgfpathlineto{\pgfqpoint{6.147028in}{4.024567in}}%
\pgfpathmoveto{\pgfqpoint{6.115971in}{3.993511in}}%
\pgfpathlineto{\pgfqpoint{6.115971in}{4.055624in}}%
\pgfusepath{stroke,fill}%
\end{pgfscope}%
\begin{pgfscope}%
\pgfpathrectangle{\pgfqpoint{2.000000in}{3.197368in}}{\pgfqpoint{4.376471in}{0.978947in}} %
\pgfusepath{clip}%
\pgfsetbuttcap%
\pgfsetroundjoin%
\definecolor{currentfill}{rgb}{1.000000,0.000000,0.000000}%
\pgfsetfillcolor{currentfill}%
\pgfsetlinewidth{2.007500pt}%
\definecolor{currentstroke}{rgb}{1.000000,0.000000,0.000000}%
\pgfsetstrokecolor{currentstroke}%
\pgfsetdash{}{0pt}%
\pgfpathmoveto{\pgfqpoint{3.092947in}{3.737178in}}%
\pgfpathlineto{\pgfqpoint{3.155060in}{3.737178in}}%
\pgfpathmoveto{\pgfqpoint{3.124004in}{3.706121in}}%
\pgfpathlineto{\pgfqpoint{3.124004in}{3.768234in}}%
\pgfusepath{stroke,fill}%
\end{pgfscope}%
\begin{pgfscope}%
\pgfpathrectangle{\pgfqpoint{2.000000in}{3.197368in}}{\pgfqpoint{4.376471in}{0.978947in}} %
\pgfusepath{clip}%
\pgfsetbuttcap%
\pgfsetroundjoin%
\definecolor{currentfill}{rgb}{1.000000,0.000000,0.000000}%
\pgfsetfillcolor{currentfill}%
\pgfsetlinewidth{2.007500pt}%
\definecolor{currentstroke}{rgb}{1.000000,0.000000,0.000000}%
\pgfsetstrokecolor{currentstroke}%
\pgfsetdash{}{0pt}%
\pgfpathmoveto{\pgfqpoint{3.149293in}{3.679544in}}%
\pgfpathlineto{\pgfqpoint{3.211406in}{3.679544in}}%
\pgfpathmoveto{\pgfqpoint{3.180349in}{3.648488in}}%
\pgfpathlineto{\pgfqpoint{3.180349in}{3.710601in}}%
\pgfusepath{stroke,fill}%
\end{pgfscope}%
\begin{pgfscope}%
\pgfpathrectangle{\pgfqpoint{2.000000in}{3.197368in}}{\pgfqpoint{4.376471in}{0.978947in}} %
\pgfusepath{clip}%
\pgfsetbuttcap%
\pgfsetroundjoin%
\definecolor{currentfill}{rgb}{1.000000,0.000000,0.000000}%
\pgfsetfillcolor{currentfill}%
\pgfsetlinewidth{2.007500pt}%
\definecolor{currentstroke}{rgb}{1.000000,0.000000,0.000000}%
\pgfsetstrokecolor{currentstroke}%
\pgfsetdash{}{0pt}%
\pgfpathmoveto{\pgfqpoint{2.915026in}{3.966824in}}%
\pgfpathlineto{\pgfqpoint{2.977139in}{3.966824in}}%
\pgfpathmoveto{\pgfqpoint{2.946082in}{3.935768in}}%
\pgfpathlineto{\pgfqpoint{2.946082in}{3.997881in}}%
\pgfusepath{stroke,fill}%
\end{pgfscope}%
\begin{pgfscope}%
\pgfpathrectangle{\pgfqpoint{2.000000in}{3.197368in}}{\pgfqpoint{4.376471in}{0.978947in}} %
\pgfusepath{clip}%
\pgfsetbuttcap%
\pgfsetroundjoin%
\definecolor{currentfill}{rgb}{1.000000,0.000000,0.000000}%
\pgfsetfillcolor{currentfill}%
\pgfsetlinewidth{2.007500pt}%
\definecolor{currentstroke}{rgb}{1.000000,0.000000,0.000000}%
\pgfsetstrokecolor{currentstroke}%
\pgfsetdash{}{0pt}%
\pgfpathmoveto{\pgfqpoint{5.759387in}{3.812420in}}%
\pgfpathlineto{\pgfqpoint{5.821500in}{3.812420in}}%
\pgfpathmoveto{\pgfqpoint{5.790443in}{3.781364in}}%
\pgfpathlineto{\pgfqpoint{5.790443in}{3.843477in}}%
\pgfusepath{stroke,fill}%
\end{pgfscope}%
\begin{pgfscope}%
\pgfpathrectangle{\pgfqpoint{2.000000in}{3.197368in}}{\pgfqpoint{4.376471in}{0.978947in}} %
\pgfusepath{clip}%
\pgfsetbuttcap%
\pgfsetroundjoin%
\definecolor{currentfill}{rgb}{1.000000,0.000000,0.000000}%
\pgfsetfillcolor{currentfill}%
\pgfsetlinewidth{2.007500pt}%
\definecolor{currentstroke}{rgb}{1.000000,0.000000,0.000000}%
\pgfsetstrokecolor{currentstroke}%
\pgfsetdash{}{0pt}%
\pgfpathmoveto{\pgfqpoint{5.568702in}{3.534838in}}%
\pgfpathlineto{\pgfqpoint{5.630815in}{3.534838in}}%
\pgfpathmoveto{\pgfqpoint{5.599758in}{3.503782in}}%
\pgfpathlineto{\pgfqpoint{5.599758in}{3.565895in}}%
\pgfusepath{stroke,fill}%
\end{pgfscope}%
\begin{pgfscope}%
\pgfpathrectangle{\pgfqpoint{2.000000in}{3.197368in}}{\pgfqpoint{4.376471in}{0.978947in}} %
\pgfusepath{clip}%
\pgfsetbuttcap%
\pgfsetroundjoin%
\definecolor{currentfill}{rgb}{1.000000,0.000000,0.000000}%
\pgfsetfillcolor{currentfill}%
\pgfsetlinewidth{2.007500pt}%
\definecolor{currentstroke}{rgb}{1.000000,0.000000,0.000000}%
\pgfsetstrokecolor{currentstroke}%
\pgfsetdash{}{0pt}%
\pgfpathmoveto{\pgfqpoint{5.890304in}{3.942201in}}%
\pgfpathlineto{\pgfqpoint{5.952417in}{3.942201in}}%
\pgfpathmoveto{\pgfqpoint{5.921360in}{3.911145in}}%
\pgfpathlineto{\pgfqpoint{5.921360in}{3.973258in}}%
\pgfusepath{stroke,fill}%
\end{pgfscope}%
\begin{pgfscope}%
\pgfpathrectangle{\pgfqpoint{2.000000in}{3.197368in}}{\pgfqpoint{4.376471in}{0.978947in}} %
\pgfusepath{clip}%
\pgfsetbuttcap%
\pgfsetroundjoin%
\definecolor{currentfill}{rgb}{1.000000,0.000000,0.000000}%
\pgfsetfillcolor{currentfill}%
\pgfsetlinewidth{2.007500pt}%
\definecolor{currentstroke}{rgb}{1.000000,0.000000,0.000000}%
\pgfsetstrokecolor{currentstroke}%
\pgfsetdash{}{0pt}%
\pgfpathmoveto{\pgfqpoint{6.270553in}{3.873438in}}%
\pgfpathlineto{\pgfqpoint{6.332666in}{3.873438in}}%
\pgfpathmoveto{\pgfqpoint{6.301610in}{3.842382in}}%
\pgfpathlineto{\pgfqpoint{6.301610in}{3.904495in}}%
\pgfusepath{stroke,fill}%
\end{pgfscope}%
\begin{pgfscope}%
\pgfpathrectangle{\pgfqpoint{2.000000in}{3.197368in}}{\pgfqpoint{4.376471in}{0.978947in}} %
\pgfusepath{clip}%
\pgfsetbuttcap%
\pgfsetroundjoin%
\definecolor{currentfill}{rgb}{1.000000,0.000000,0.000000}%
\pgfsetfillcolor{currentfill}%
\pgfsetlinewidth{2.007500pt}%
\definecolor{currentstroke}{rgb}{1.000000,0.000000,0.000000}%
\pgfsetstrokecolor{currentstroke}%
\pgfsetdash{}{0pt}%
\pgfpathmoveto{\pgfqpoint{5.642233in}{3.690005in}}%
\pgfpathlineto{\pgfqpoint{5.704346in}{3.690005in}}%
\pgfpathmoveto{\pgfqpoint{5.673289in}{3.658949in}}%
\pgfpathlineto{\pgfqpoint{5.673289in}{3.721062in}}%
\pgfusepath{stroke,fill}%
\end{pgfscope}%
\begin{pgfscope}%
\pgfpathrectangle{\pgfqpoint{2.000000in}{3.197368in}}{\pgfqpoint{4.376471in}{0.978947in}} %
\pgfusepath{clip}%
\pgfsetbuttcap%
\pgfsetroundjoin%
\definecolor{currentfill}{rgb}{1.000000,0.000000,0.000000}%
\pgfsetfillcolor{currentfill}%
\pgfsetlinewidth{2.007500pt}%
\definecolor{currentstroke}{rgb}{1.000000,0.000000,0.000000}%
\pgfsetstrokecolor{currentstroke}%
\pgfsetdash{}{0pt}%
\pgfpathmoveto{\pgfqpoint{4.459958in}{3.414465in}}%
\pgfpathlineto{\pgfqpoint{4.522071in}{3.414465in}}%
\pgfpathmoveto{\pgfqpoint{4.491015in}{3.383409in}}%
\pgfpathlineto{\pgfqpoint{4.491015in}{3.445522in}}%
\pgfusepath{stroke,fill}%
\end{pgfscope}%
\begin{pgfscope}%
\pgfpathrectangle{\pgfqpoint{2.000000in}{3.197368in}}{\pgfqpoint{4.376471in}{0.978947in}} %
\pgfusepath{clip}%
\pgfsetbuttcap%
\pgfsetroundjoin%
\definecolor{currentfill}{rgb}{1.000000,0.000000,0.000000}%
\pgfsetfillcolor{currentfill}%
\pgfsetlinewidth{2.007500pt}%
\definecolor{currentstroke}{rgb}{1.000000,0.000000,0.000000}%
\pgfsetstrokecolor{currentstroke}%
\pgfsetdash{}{0pt}%
\pgfpathmoveto{\pgfqpoint{5.577008in}{3.556875in}}%
\pgfpathlineto{\pgfqpoint{5.639121in}{3.556875in}}%
\pgfpathmoveto{\pgfqpoint{5.608065in}{3.525819in}}%
\pgfpathlineto{\pgfqpoint{5.608065in}{3.587932in}}%
\pgfusepath{stroke,fill}%
\end{pgfscope}%
\begin{pgfscope}%
\pgfpathrectangle{\pgfqpoint{2.000000in}{3.197368in}}{\pgfqpoint{4.376471in}{0.978947in}} %
\pgfusepath{clip}%
\pgfsetbuttcap%
\pgfsetroundjoin%
\definecolor{currentfill}{rgb}{1.000000,0.000000,0.000000}%
\pgfsetfillcolor{currentfill}%
\pgfsetlinewidth{2.007500pt}%
\definecolor{currentstroke}{rgb}{1.000000,0.000000,0.000000}%
\pgfsetstrokecolor{currentstroke}%
\pgfsetdash{}{0pt}%
\pgfpathmoveto{\pgfqpoint{3.258337in}{3.577613in}}%
\pgfpathlineto{\pgfqpoint{3.320450in}{3.577613in}}%
\pgfpathmoveto{\pgfqpoint{3.289394in}{3.546556in}}%
\pgfpathlineto{\pgfqpoint{3.289394in}{3.608669in}}%
\pgfusepath{stroke,fill}%
\end{pgfscope}%
\begin{pgfscope}%
\pgfpathrectangle{\pgfqpoint{2.000000in}{3.197368in}}{\pgfqpoint{4.376471in}{0.978947in}} %
\pgfusepath{clip}%
\pgfsetbuttcap%
\pgfsetroundjoin%
\definecolor{currentfill}{rgb}{1.000000,0.000000,0.000000}%
\pgfsetfillcolor{currentfill}%
\pgfsetlinewidth{2.007500pt}%
\definecolor{currentstroke}{rgb}{1.000000,0.000000,0.000000}%
\pgfsetstrokecolor{currentstroke}%
\pgfsetdash{}{0pt}%
\pgfpathmoveto{\pgfqpoint{5.084714in}{3.680897in}}%
\pgfpathlineto{\pgfqpoint{5.146827in}{3.680897in}}%
\pgfpathmoveto{\pgfqpoint{5.115771in}{3.649840in}}%
\pgfpathlineto{\pgfqpoint{5.115771in}{3.711953in}}%
\pgfusepath{stroke,fill}%
\end{pgfscope}%
\begin{pgfscope}%
\pgfpathrectangle{\pgfqpoint{2.000000in}{3.197368in}}{\pgfqpoint{4.376471in}{0.978947in}} %
\pgfusepath{clip}%
\pgfsetbuttcap%
\pgfsetroundjoin%
\definecolor{currentfill}{rgb}{1.000000,0.000000,0.000000}%
\pgfsetfillcolor{currentfill}%
\pgfsetlinewidth{2.007500pt}%
\definecolor{currentstroke}{rgb}{1.000000,0.000000,0.000000}%
\pgfsetstrokecolor{currentstroke}%
\pgfsetdash{}{0pt}%
\pgfpathmoveto{\pgfqpoint{3.346143in}{3.585903in}}%
\pgfpathlineto{\pgfqpoint{3.408256in}{3.585903in}}%
\pgfpathmoveto{\pgfqpoint{3.377199in}{3.554846in}}%
\pgfpathlineto{\pgfqpoint{3.377199in}{3.616959in}}%
\pgfusepath{stroke,fill}%
\end{pgfscope}%
\begin{pgfscope}%
\pgfpathrectangle{\pgfqpoint{2.000000in}{3.197368in}}{\pgfqpoint{4.376471in}{0.978947in}} %
\pgfusepath{clip}%
\pgfsetbuttcap%
\pgfsetroundjoin%
\definecolor{currentfill}{rgb}{1.000000,0.000000,0.000000}%
\pgfsetfillcolor{currentfill}%
\pgfsetlinewidth{2.007500pt}%
\definecolor{currentstroke}{rgb}{1.000000,0.000000,0.000000}%
\pgfsetstrokecolor{currentstroke}%
\pgfsetdash{}{0pt}%
\pgfpathmoveto{\pgfqpoint{6.151690in}{3.986096in}}%
\pgfpathlineto{\pgfqpoint{6.213803in}{3.986096in}}%
\pgfpathmoveto{\pgfqpoint{6.182747in}{3.955040in}}%
\pgfpathlineto{\pgfqpoint{6.182747in}{4.017153in}}%
\pgfusepath{stroke,fill}%
\end{pgfscope}%
\begin{pgfscope}%
\pgfpathrectangle{\pgfqpoint{2.000000in}{3.197368in}}{\pgfqpoint{4.376471in}{0.978947in}} %
\pgfusepath{clip}%
\pgfsetbuttcap%
\pgfsetroundjoin%
\definecolor{currentfill}{rgb}{1.000000,0.000000,0.000000}%
\pgfsetfillcolor{currentfill}%
\pgfsetlinewidth{2.007500pt}%
\definecolor{currentstroke}{rgb}{1.000000,0.000000,0.000000}%
\pgfsetstrokecolor{currentstroke}%
\pgfsetdash{}{0pt}%
\pgfpathmoveto{\pgfqpoint{4.671321in}{3.633299in}}%
\pgfpathlineto{\pgfqpoint{4.733434in}{3.633299in}}%
\pgfpathmoveto{\pgfqpoint{4.702377in}{3.602242in}}%
\pgfpathlineto{\pgfqpoint{4.702377in}{3.664355in}}%
\pgfusepath{stroke,fill}%
\end{pgfscope}%
\begin{pgfscope}%
\pgfpathrectangle{\pgfqpoint{2.000000in}{3.197368in}}{\pgfqpoint{4.376471in}{0.978947in}} %
\pgfusepath{clip}%
\pgfsetbuttcap%
\pgfsetroundjoin%
\definecolor{currentfill}{rgb}{1.000000,0.000000,0.000000}%
\pgfsetfillcolor{currentfill}%
\pgfsetlinewidth{2.007500pt}%
\definecolor{currentstroke}{rgb}{1.000000,0.000000,0.000000}%
\pgfsetstrokecolor{currentstroke}%
\pgfsetdash{}{0pt}%
\pgfpathmoveto{\pgfqpoint{4.296042in}{3.382181in}}%
\pgfpathlineto{\pgfqpoint{4.358155in}{3.382181in}}%
\pgfpathmoveto{\pgfqpoint{4.327099in}{3.351125in}}%
\pgfpathlineto{\pgfqpoint{4.327099in}{3.413238in}}%
\pgfusepath{stroke,fill}%
\end{pgfscope}%
\begin{pgfscope}%
\pgfpathrectangle{\pgfqpoint{2.000000in}{3.197368in}}{\pgfqpoint{4.376471in}{0.978947in}} %
\pgfusepath{clip}%
\pgfsetbuttcap%
\pgfsetroundjoin%
\definecolor{currentfill}{rgb}{0.000000,0.000000,0.000000}%
\pgfsetfillcolor{currentfill}%
\pgfsetlinewidth{0.301125pt}%
\definecolor{currentstroke}{rgb}{0.000000,0.000000,0.000000}%
\pgfsetstrokecolor{currentstroke}%
\pgfsetdash{}{0pt}%
\pgfsys@defobject{currentmarker}{\pgfqpoint{-0.015528in}{-0.015528in}}{\pgfqpoint{0.015528in}{0.015528in}}{%
\pgfpathmoveto{\pgfqpoint{0.000000in}{-0.015528in}}%
\pgfpathcurveto{\pgfqpoint{0.004118in}{-0.015528in}}{\pgfqpoint{0.008068in}{-0.013892in}}{\pgfqpoint{0.010980in}{-0.010980in}}%
\pgfpathcurveto{\pgfqpoint{0.013892in}{-0.008068in}}{\pgfqpoint{0.015528in}{-0.004118in}}{\pgfqpoint{0.015528in}{0.000000in}}%
\pgfpathcurveto{\pgfqpoint{0.015528in}{0.004118in}}{\pgfqpoint{0.013892in}{0.008068in}}{\pgfqpoint{0.010980in}{0.010980in}}%
\pgfpathcurveto{\pgfqpoint{0.008068in}{0.013892in}}{\pgfqpoint{0.004118in}{0.015528in}}{\pgfqpoint{0.000000in}{0.015528in}}%
\pgfpathcurveto{\pgfqpoint{-0.004118in}{0.015528in}}{\pgfqpoint{-0.008068in}{0.013892in}}{\pgfqpoint{-0.010980in}{0.010980in}}%
\pgfpathcurveto{\pgfqpoint{-0.013892in}{0.008068in}}{\pgfqpoint{-0.015528in}{0.004118in}}{\pgfqpoint{-0.015528in}{0.000000in}}%
\pgfpathcurveto{\pgfqpoint{-0.015528in}{-0.004118in}}{\pgfqpoint{-0.013892in}{-0.008068in}}{\pgfqpoint{-0.010980in}{-0.010980in}}%
\pgfpathcurveto{\pgfqpoint{-0.008068in}{-0.013892in}}{\pgfqpoint{-0.004118in}{-0.015528in}}{\pgfqpoint{0.000000in}{-0.015528in}}%
\pgfpathclose%
\pgfusepath{stroke,fill}%
}%
\begin{pgfscope}%
\pgfsys@transformshift{2.875294in}{4.031231in}%
\pgfsys@useobject{currentmarker}{}%
\end{pgfscope}%
\begin{pgfscope}%
\pgfsys@transformshift{2.892888in}{3.937039in}%
\pgfsys@useobject{currentmarker}{}%
\end{pgfscope}%
\begin{pgfscope}%
\pgfsys@transformshift{2.910482in}{4.027813in}%
\pgfsys@useobject{currentmarker}{}%
\end{pgfscope}%
\begin{pgfscope}%
\pgfsys@transformshift{2.928076in}{4.046950in}%
\pgfsys@useobject{currentmarker}{}%
\end{pgfscope}%
\begin{pgfscope}%
\pgfsys@transformshift{2.945670in}{3.982165in}%
\pgfsys@useobject{currentmarker}{}%
\end{pgfscope}%
\begin{pgfscope}%
\pgfsys@transformshift{2.963263in}{3.978064in}%
\pgfsys@useobject{currentmarker}{}%
\end{pgfscope}%
\begin{pgfscope}%
\pgfsys@transformshift{2.980857in}{3.854307in}%
\pgfsys@useobject{currentmarker}{}%
\end{pgfscope}%
\begin{pgfscope}%
\pgfsys@transformshift{2.998451in}{3.853911in}%
\pgfsys@useobject{currentmarker}{}%
\end{pgfscope}%
\begin{pgfscope}%
\pgfsys@transformshift{3.016045in}{3.794584in}%
\pgfsys@useobject{currentmarker}{}%
\end{pgfscope}%
\begin{pgfscope}%
\pgfsys@transformshift{3.033639in}{3.799298in}%
\pgfsys@useobject{currentmarker}{}%
\end{pgfscope}%
\begin{pgfscope}%
\pgfsys@transformshift{3.051233in}{3.726597in}%
\pgfsys@useobject{currentmarker}{}%
\end{pgfscope}%
\begin{pgfscope}%
\pgfsys@transformshift{3.068826in}{3.607975in}%
\pgfsys@useobject{currentmarker}{}%
\end{pgfscope}%
\begin{pgfscope}%
\pgfsys@transformshift{3.086420in}{3.777720in}%
\pgfsys@useobject{currentmarker}{}%
\end{pgfscope}%
\begin{pgfscope}%
\pgfsys@transformshift{3.104014in}{3.695569in}%
\pgfsys@useobject{currentmarker}{}%
\end{pgfscope}%
\begin{pgfscope}%
\pgfsys@transformshift{3.121608in}{3.548673in}%
\pgfsys@useobject{currentmarker}{}%
\end{pgfscope}%
\begin{pgfscope}%
\pgfsys@transformshift{3.139202in}{3.742077in}%
\pgfsys@useobject{currentmarker}{}%
\end{pgfscope}%
\begin{pgfscope}%
\pgfsys@transformshift{3.156796in}{3.584074in}%
\pgfsys@useobject{currentmarker}{}%
\end{pgfscope}%
\begin{pgfscope}%
\pgfsys@transformshift{3.174390in}{3.665451in}%
\pgfsys@useobject{currentmarker}{}%
\end{pgfscope}%
\begin{pgfscope}%
\pgfsys@transformshift{3.191983in}{3.720008in}%
\pgfsys@useobject{currentmarker}{}%
\end{pgfscope}%
\begin{pgfscope}%
\pgfsys@transformshift{3.209577in}{3.646344in}%
\pgfsys@useobject{currentmarker}{}%
\end{pgfscope}%
\begin{pgfscope}%
\pgfsys@transformshift{3.227171in}{3.738995in}%
\pgfsys@useobject{currentmarker}{}%
\end{pgfscope}%
\begin{pgfscope}%
\pgfsys@transformshift{3.244765in}{3.488622in}%
\pgfsys@useobject{currentmarker}{}%
\end{pgfscope}%
\begin{pgfscope}%
\pgfsys@transformshift{3.262359in}{3.649446in}%
\pgfsys@useobject{currentmarker}{}%
\end{pgfscope}%
\begin{pgfscope}%
\pgfsys@transformshift{3.279953in}{3.534597in}%
\pgfsys@useobject{currentmarker}{}%
\end{pgfscope}%
\begin{pgfscope}%
\pgfsys@transformshift{3.297547in}{3.513756in}%
\pgfsys@useobject{currentmarker}{}%
\end{pgfscope}%
\begin{pgfscope}%
\pgfsys@transformshift{3.315140in}{3.543720in}%
\pgfsys@useobject{currentmarker}{}%
\end{pgfscope}%
\begin{pgfscope}%
\pgfsys@transformshift{3.332734in}{3.573174in}%
\pgfsys@useobject{currentmarker}{}%
\end{pgfscope}%
\begin{pgfscope}%
\pgfsys@transformshift{3.350328in}{3.614788in}%
\pgfsys@useobject{currentmarker}{}%
\end{pgfscope}%
\begin{pgfscope}%
\pgfsys@transformshift{3.367922in}{3.496181in}%
\pgfsys@useobject{currentmarker}{}%
\end{pgfscope}%
\begin{pgfscope}%
\pgfsys@transformshift{3.385516in}{3.714490in}%
\pgfsys@useobject{currentmarker}{}%
\end{pgfscope}%
\begin{pgfscope}%
\pgfsys@transformshift{3.403110in}{3.679310in}%
\pgfsys@useobject{currentmarker}{}%
\end{pgfscope}%
\begin{pgfscope}%
\pgfsys@transformshift{3.420704in}{3.485775in}%
\pgfsys@useobject{currentmarker}{}%
\end{pgfscope}%
\begin{pgfscope}%
\pgfsys@transformshift{3.438297in}{3.806047in}%
\pgfsys@useobject{currentmarker}{}%
\end{pgfscope}%
\begin{pgfscope}%
\pgfsys@transformshift{3.455891in}{3.860539in}%
\pgfsys@useobject{currentmarker}{}%
\end{pgfscope}%
\begin{pgfscope}%
\pgfsys@transformshift{3.473485in}{3.801221in}%
\pgfsys@useobject{currentmarker}{}%
\end{pgfscope}%
\begin{pgfscope}%
\pgfsys@transformshift{3.491079in}{3.677167in}%
\pgfsys@useobject{currentmarker}{}%
\end{pgfscope}%
\begin{pgfscope}%
\pgfsys@transformshift{3.508673in}{3.601314in}%
\pgfsys@useobject{currentmarker}{}%
\end{pgfscope}%
\begin{pgfscope}%
\pgfsys@transformshift{3.526267in}{3.833302in}%
\pgfsys@useobject{currentmarker}{}%
\end{pgfscope}%
\begin{pgfscope}%
\pgfsys@transformshift{3.543860in}{3.700014in}%
\pgfsys@useobject{currentmarker}{}%
\end{pgfscope}%
\begin{pgfscope}%
\pgfsys@transformshift{3.561454in}{3.881009in}%
\pgfsys@useobject{currentmarker}{}%
\end{pgfscope}%
\begin{pgfscope}%
\pgfsys@transformshift{3.579048in}{3.792483in}%
\pgfsys@useobject{currentmarker}{}%
\end{pgfscope}%
\begin{pgfscope}%
\pgfsys@transformshift{3.596642in}{3.885195in}%
\pgfsys@useobject{currentmarker}{}%
\end{pgfscope}%
\begin{pgfscope}%
\pgfsys@transformshift{3.614236in}{3.835577in}%
\pgfsys@useobject{currentmarker}{}%
\end{pgfscope}%
\begin{pgfscope}%
\pgfsys@transformshift{3.631830in}{3.884011in}%
\pgfsys@useobject{currentmarker}{}%
\end{pgfscope}%
\begin{pgfscope}%
\pgfsys@transformshift{3.649424in}{3.824663in}%
\pgfsys@useobject{currentmarker}{}%
\end{pgfscope}%
\begin{pgfscope}%
\pgfsys@transformshift{3.667017in}{4.016085in}%
\pgfsys@useobject{currentmarker}{}%
\end{pgfscope}%
\begin{pgfscope}%
\pgfsys@transformshift{3.684611in}{3.855894in}%
\pgfsys@useobject{currentmarker}{}%
\end{pgfscope}%
\begin{pgfscope}%
\pgfsys@transformshift{3.702205in}{3.891386in}%
\pgfsys@useobject{currentmarker}{}%
\end{pgfscope}%
\begin{pgfscope}%
\pgfsys@transformshift{3.719799in}{4.048210in}%
\pgfsys@useobject{currentmarker}{}%
\end{pgfscope}%
\begin{pgfscope}%
\pgfsys@transformshift{3.737393in}{3.722767in}%
\pgfsys@useobject{currentmarker}{}%
\end{pgfscope}%
\begin{pgfscope}%
\pgfsys@transformshift{3.754987in}{3.732831in}%
\pgfsys@useobject{currentmarker}{}%
\end{pgfscope}%
\begin{pgfscope}%
\pgfsys@transformshift{3.772581in}{3.961518in}%
\pgfsys@useobject{currentmarker}{}%
\end{pgfscope}%
\begin{pgfscope}%
\pgfsys@transformshift{3.790174in}{3.741367in}%
\pgfsys@useobject{currentmarker}{}%
\end{pgfscope}%
\begin{pgfscope}%
\pgfsys@transformshift{3.807768in}{4.055550in}%
\pgfsys@useobject{currentmarker}{}%
\end{pgfscope}%
\begin{pgfscope}%
\pgfsys@transformshift{3.825362in}{3.809556in}%
\pgfsys@useobject{currentmarker}{}%
\end{pgfscope}%
\begin{pgfscope}%
\pgfsys@transformshift{3.842956in}{3.767941in}%
\pgfsys@useobject{currentmarker}{}%
\end{pgfscope}%
\begin{pgfscope}%
\pgfsys@transformshift{3.860550in}{4.030761in}%
\pgfsys@useobject{currentmarker}{}%
\end{pgfscope}%
\begin{pgfscope}%
\pgfsys@transformshift{3.878144in}{3.974297in}%
\pgfsys@useobject{currentmarker}{}%
\end{pgfscope}%
\begin{pgfscope}%
\pgfsys@transformshift{3.895738in}{4.000639in}%
\pgfsys@useobject{currentmarker}{}%
\end{pgfscope}%
\begin{pgfscope}%
\pgfsys@transformshift{3.913331in}{3.887782in}%
\pgfsys@useobject{currentmarker}{}%
\end{pgfscope}%
\begin{pgfscope}%
\pgfsys@transformshift{3.930925in}{3.691193in}%
\pgfsys@useobject{currentmarker}{}%
\end{pgfscope}%
\begin{pgfscope}%
\pgfsys@transformshift{3.948519in}{3.955984in}%
\pgfsys@useobject{currentmarker}{}%
\end{pgfscope}%
\begin{pgfscope}%
\pgfsys@transformshift{3.966113in}{3.714784in}%
\pgfsys@useobject{currentmarker}{}%
\end{pgfscope}%
\begin{pgfscope}%
\pgfsys@transformshift{3.983707in}{3.803722in}%
\pgfsys@useobject{currentmarker}{}%
\end{pgfscope}%
\begin{pgfscope}%
\pgfsys@transformshift{4.001301in}{3.797315in}%
\pgfsys@useobject{currentmarker}{}%
\end{pgfscope}%
\begin{pgfscope}%
\pgfsys@transformshift{4.018894in}{3.662972in}%
\pgfsys@useobject{currentmarker}{}%
\end{pgfscope}%
\begin{pgfscope}%
\pgfsys@transformshift{4.036488in}{3.718881in}%
\pgfsys@useobject{currentmarker}{}%
\end{pgfscope}%
\begin{pgfscope}%
\pgfsys@transformshift{4.054082in}{3.727407in}%
\pgfsys@useobject{currentmarker}{}%
\end{pgfscope}%
\begin{pgfscope}%
\pgfsys@transformshift{4.071676in}{3.648683in}%
\pgfsys@useobject{currentmarker}{}%
\end{pgfscope}%
\begin{pgfscope}%
\pgfsys@transformshift{4.089270in}{3.475151in}%
\pgfsys@useobject{currentmarker}{}%
\end{pgfscope}%
\begin{pgfscope}%
\pgfsys@transformshift{4.106864in}{3.594877in}%
\pgfsys@useobject{currentmarker}{}%
\end{pgfscope}%
\begin{pgfscope}%
\pgfsys@transformshift{4.124458in}{3.677368in}%
\pgfsys@useobject{currentmarker}{}%
\end{pgfscope}%
\begin{pgfscope}%
\pgfsys@transformshift{4.142051in}{3.449577in}%
\pgfsys@useobject{currentmarker}{}%
\end{pgfscope}%
\begin{pgfscope}%
\pgfsys@transformshift{4.159645in}{3.484281in}%
\pgfsys@useobject{currentmarker}{}%
\end{pgfscope}%
\begin{pgfscope}%
\pgfsys@transformshift{4.177239in}{3.435330in}%
\pgfsys@useobject{currentmarker}{}%
\end{pgfscope}%
\begin{pgfscope}%
\pgfsys@transformshift{4.194833in}{3.649654in}%
\pgfsys@useobject{currentmarker}{}%
\end{pgfscope}%
\begin{pgfscope}%
\pgfsys@transformshift{4.212427in}{3.512382in}%
\pgfsys@useobject{currentmarker}{}%
\end{pgfscope}%
\begin{pgfscope}%
\pgfsys@transformshift{4.230021in}{3.469654in}%
\pgfsys@useobject{currentmarker}{}%
\end{pgfscope}%
\begin{pgfscope}%
\pgfsys@transformshift{4.247615in}{3.335540in}%
\pgfsys@useobject{currentmarker}{}%
\end{pgfscope}%
\begin{pgfscope}%
\pgfsys@transformshift{4.265208in}{3.456780in}%
\pgfsys@useobject{currentmarker}{}%
\end{pgfscope}%
\begin{pgfscope}%
\pgfsys@transformshift{4.282802in}{3.322666in}%
\pgfsys@useobject{currentmarker}{}%
\end{pgfscope}%
\begin{pgfscope}%
\pgfsys@transformshift{4.300396in}{3.386302in}%
\pgfsys@useobject{currentmarker}{}%
\end{pgfscope}%
\begin{pgfscope}%
\pgfsys@transformshift{4.317990in}{3.311898in}%
\pgfsys@useobject{currentmarker}{}%
\end{pgfscope}%
\begin{pgfscope}%
\pgfsys@transformshift{4.335584in}{3.441502in}%
\pgfsys@useobject{currentmarker}{}%
\end{pgfscope}%
\begin{pgfscope}%
\pgfsys@transformshift{4.353178in}{3.429239in}%
\pgfsys@useobject{currentmarker}{}%
\end{pgfscope}%
\begin{pgfscope}%
\pgfsys@transformshift{4.370772in}{3.349260in}%
\pgfsys@useobject{currentmarker}{}%
\end{pgfscope}%
\begin{pgfscope}%
\pgfsys@transformshift{4.388365in}{3.413070in}%
\pgfsys@useobject{currentmarker}{}%
\end{pgfscope}%
\begin{pgfscope}%
\pgfsys@transformshift{4.405959in}{3.265505in}%
\pgfsys@useobject{currentmarker}{}%
\end{pgfscope}%
\begin{pgfscope}%
\pgfsys@transformshift{4.423553in}{3.231219in}%
\pgfsys@useobject{currentmarker}{}%
\end{pgfscope}%
\begin{pgfscope}%
\pgfsys@transformshift{4.441147in}{3.436381in}%
\pgfsys@useobject{currentmarker}{}%
\end{pgfscope}%
\begin{pgfscope}%
\pgfsys@transformshift{4.458741in}{3.418730in}%
\pgfsys@useobject{currentmarker}{}%
\end{pgfscope}%
\begin{pgfscope}%
\pgfsys@transformshift{4.476335in}{3.478412in}%
\pgfsys@useobject{currentmarker}{}%
\end{pgfscope}%
\begin{pgfscope}%
\pgfsys@transformshift{4.493928in}{3.670229in}%
\pgfsys@useobject{currentmarker}{}%
\end{pgfscope}%
\begin{pgfscope}%
\pgfsys@transformshift{4.511522in}{3.538571in}%
\pgfsys@useobject{currentmarker}{}%
\end{pgfscope}%
\begin{pgfscope}%
\pgfsys@transformshift{4.529116in}{3.365555in}%
\pgfsys@useobject{currentmarker}{}%
\end{pgfscope}%
\begin{pgfscope}%
\pgfsys@transformshift{4.546710in}{3.590089in}%
\pgfsys@useobject{currentmarker}{}%
\end{pgfscope}%
\begin{pgfscope}%
\pgfsys@transformshift{4.564304in}{3.360521in}%
\pgfsys@useobject{currentmarker}{}%
\end{pgfscope}%
\begin{pgfscope}%
\pgfsys@transformshift{4.581898in}{3.466956in}%
\pgfsys@useobject{currentmarker}{}%
\end{pgfscope}%
\begin{pgfscope}%
\pgfsys@transformshift{4.599492in}{3.526978in}%
\pgfsys@useobject{currentmarker}{}%
\end{pgfscope}%
\begin{pgfscope}%
\pgfsys@transformshift{4.617085in}{3.728956in}%
\pgfsys@useobject{currentmarker}{}%
\end{pgfscope}%
\begin{pgfscope}%
\pgfsys@transformshift{4.634679in}{3.498783in}%
\pgfsys@useobject{currentmarker}{}%
\end{pgfscope}%
\begin{pgfscope}%
\pgfsys@transformshift{4.652273in}{3.510921in}%
\pgfsys@useobject{currentmarker}{}%
\end{pgfscope}%
\begin{pgfscope}%
\pgfsys@transformshift{4.669867in}{3.605397in}%
\pgfsys@useobject{currentmarker}{}%
\end{pgfscope}%
\begin{pgfscope}%
\pgfsys@transformshift{4.687461in}{3.567591in}%
\pgfsys@useobject{currentmarker}{}%
\end{pgfscope}%
\begin{pgfscope}%
\pgfsys@transformshift{4.705055in}{3.769320in}%
\pgfsys@useobject{currentmarker}{}%
\end{pgfscope}%
\begin{pgfscope}%
\pgfsys@transformshift{4.722649in}{3.562676in}%
\pgfsys@useobject{currentmarker}{}%
\end{pgfscope}%
\begin{pgfscope}%
\pgfsys@transformshift{4.740242in}{3.573159in}%
\pgfsys@useobject{currentmarker}{}%
\end{pgfscope}%
\begin{pgfscope}%
\pgfsys@transformshift{4.757836in}{3.661725in}%
\pgfsys@useobject{currentmarker}{}%
\end{pgfscope}%
\begin{pgfscope}%
\pgfsys@transformshift{4.775430in}{3.670459in}%
\pgfsys@useobject{currentmarker}{}%
\end{pgfscope}%
\begin{pgfscope}%
\pgfsys@transformshift{4.793024in}{3.931412in}%
\pgfsys@useobject{currentmarker}{}%
\end{pgfscope}%
\begin{pgfscope}%
\pgfsys@transformshift{4.810618in}{3.843274in}%
\pgfsys@useobject{currentmarker}{}%
\end{pgfscope}%
\begin{pgfscope}%
\pgfsys@transformshift{4.828212in}{3.765484in}%
\pgfsys@useobject{currentmarker}{}%
\end{pgfscope}%
\begin{pgfscope}%
\pgfsys@transformshift{4.845805in}{3.639892in}%
\pgfsys@useobject{currentmarker}{}%
\end{pgfscope}%
\begin{pgfscope}%
\pgfsys@transformshift{4.863399in}{3.857380in}%
\pgfsys@useobject{currentmarker}{}%
\end{pgfscope}%
\begin{pgfscope}%
\pgfsys@transformshift{4.880993in}{3.673775in}%
\pgfsys@useobject{currentmarker}{}%
\end{pgfscope}%
\begin{pgfscope}%
\pgfsys@transformshift{4.898587in}{3.620776in}%
\pgfsys@useobject{currentmarker}{}%
\end{pgfscope}%
\begin{pgfscope}%
\pgfsys@transformshift{4.916181in}{3.900005in}%
\pgfsys@useobject{currentmarker}{}%
\end{pgfscope}%
\begin{pgfscope}%
\pgfsys@transformshift{4.933775in}{3.809765in}%
\pgfsys@useobject{currentmarker}{}%
\end{pgfscope}%
\begin{pgfscope}%
\pgfsys@transformshift{4.951369in}{3.867996in}%
\pgfsys@useobject{currentmarker}{}%
\end{pgfscope}%
\begin{pgfscope}%
\pgfsys@transformshift{4.968962in}{3.801352in}%
\pgfsys@useobject{currentmarker}{}%
\end{pgfscope}%
\begin{pgfscope}%
\pgfsys@transformshift{4.986556in}{3.849159in}%
\pgfsys@useobject{currentmarker}{}%
\end{pgfscope}%
\begin{pgfscope}%
\pgfsys@transformshift{5.004150in}{3.686598in}%
\pgfsys@useobject{currentmarker}{}%
\end{pgfscope}%
\begin{pgfscope}%
\pgfsys@transformshift{5.021744in}{3.637093in}%
\pgfsys@useobject{currentmarker}{}%
\end{pgfscope}%
\begin{pgfscope}%
\pgfsys@transformshift{5.039338in}{3.800132in}%
\pgfsys@useobject{currentmarker}{}%
\end{pgfscope}%
\begin{pgfscope}%
\pgfsys@transformshift{5.056932in}{3.635404in}%
\pgfsys@useobject{currentmarker}{}%
\end{pgfscope}%
\begin{pgfscope}%
\pgfsys@transformshift{5.074526in}{3.632508in}%
\pgfsys@useobject{currentmarker}{}%
\end{pgfscope}%
\begin{pgfscope}%
\pgfsys@transformshift{5.092119in}{3.640821in}%
\pgfsys@useobject{currentmarker}{}%
\end{pgfscope}%
\begin{pgfscope}%
\pgfsys@transformshift{5.109713in}{3.672640in}%
\pgfsys@useobject{currentmarker}{}%
\end{pgfscope}%
\begin{pgfscope}%
\pgfsys@transformshift{5.127307in}{3.617668in}%
\pgfsys@useobject{currentmarker}{}%
\end{pgfscope}%
\begin{pgfscope}%
\pgfsys@transformshift{5.144901in}{3.495984in}%
\pgfsys@useobject{currentmarker}{}%
\end{pgfscope}%
\begin{pgfscope}%
\pgfsys@transformshift{5.162495in}{3.552707in}%
\pgfsys@useobject{currentmarker}{}%
\end{pgfscope}%
\begin{pgfscope}%
\pgfsys@transformshift{5.180089in}{3.373684in}%
\pgfsys@useobject{currentmarker}{}%
\end{pgfscope}%
\begin{pgfscope}%
\pgfsys@transformshift{5.197683in}{3.646375in}%
\pgfsys@useobject{currentmarker}{}%
\end{pgfscope}%
\begin{pgfscope}%
\pgfsys@transformshift{5.215276in}{3.401793in}%
\pgfsys@useobject{currentmarker}{}%
\end{pgfscope}%
\begin{pgfscope}%
\pgfsys@transformshift{5.232870in}{3.435629in}%
\pgfsys@useobject{currentmarker}{}%
\end{pgfscope}%
\begin{pgfscope}%
\pgfsys@transformshift{5.250464in}{3.537393in}%
\pgfsys@useobject{currentmarker}{}%
\end{pgfscope}%
\begin{pgfscope}%
\pgfsys@transformshift{5.268058in}{3.441417in}%
\pgfsys@useobject{currentmarker}{}%
\end{pgfscope}%
\begin{pgfscope}%
\pgfsys@transformshift{5.285652in}{3.660058in}%
\pgfsys@useobject{currentmarker}{}%
\end{pgfscope}%
\begin{pgfscope}%
\pgfsys@transformshift{5.303246in}{3.358041in}%
\pgfsys@useobject{currentmarker}{}%
\end{pgfscope}%
\begin{pgfscope}%
\pgfsys@transformshift{5.320839in}{3.505730in}%
\pgfsys@useobject{currentmarker}{}%
\end{pgfscope}%
\begin{pgfscope}%
\pgfsys@transformshift{5.338433in}{3.464710in}%
\pgfsys@useobject{currentmarker}{}%
\end{pgfscope}%
\begin{pgfscope}%
\pgfsys@transformshift{5.356027in}{3.341550in}%
\pgfsys@useobject{currentmarker}{}%
\end{pgfscope}%
\begin{pgfscope}%
\pgfsys@transformshift{5.373621in}{3.507820in}%
\pgfsys@useobject{currentmarker}{}%
\end{pgfscope}%
\begin{pgfscope}%
\pgfsys@transformshift{5.391215in}{3.432706in}%
\pgfsys@useobject{currentmarker}{}%
\end{pgfscope}%
\begin{pgfscope}%
\pgfsys@transformshift{5.408809in}{3.526663in}%
\pgfsys@useobject{currentmarker}{}%
\end{pgfscope}%
\begin{pgfscope}%
\pgfsys@transformshift{5.426403in}{3.531779in}%
\pgfsys@useobject{currentmarker}{}%
\end{pgfscope}%
\begin{pgfscope}%
\pgfsys@transformshift{5.443996in}{3.670360in}%
\pgfsys@useobject{currentmarker}{}%
\end{pgfscope}%
\begin{pgfscope}%
\pgfsys@transformshift{5.461590in}{3.590161in}%
\pgfsys@useobject{currentmarker}{}%
\end{pgfscope}%
\begin{pgfscope}%
\pgfsys@transformshift{5.479184in}{3.422463in}%
\pgfsys@useobject{currentmarker}{}%
\end{pgfscope}%
\begin{pgfscope}%
\pgfsys@transformshift{5.496778in}{3.444053in}%
\pgfsys@useobject{currentmarker}{}%
\end{pgfscope}%
\begin{pgfscope}%
\pgfsys@transformshift{5.514372in}{3.591023in}%
\pgfsys@useobject{currentmarker}{}%
\end{pgfscope}%
\begin{pgfscope}%
\pgfsys@transformshift{5.531966in}{3.558137in}%
\pgfsys@useobject{currentmarker}{}%
\end{pgfscope}%
\begin{pgfscope}%
\pgfsys@transformshift{5.549560in}{3.570947in}%
\pgfsys@useobject{currentmarker}{}%
\end{pgfscope}%
\begin{pgfscope}%
\pgfsys@transformshift{5.567153in}{3.356960in}%
\pgfsys@useobject{currentmarker}{}%
\end{pgfscope}%
\begin{pgfscope}%
\pgfsys@transformshift{5.584747in}{3.537244in}%
\pgfsys@useobject{currentmarker}{}%
\end{pgfscope}%
\begin{pgfscope}%
\pgfsys@transformshift{5.602341in}{3.483911in}%
\pgfsys@useobject{currentmarker}{}%
\end{pgfscope}%
\begin{pgfscope}%
\pgfsys@transformshift{5.619935in}{3.608578in}%
\pgfsys@useobject{currentmarker}{}%
\end{pgfscope}%
\begin{pgfscope}%
\pgfsys@transformshift{5.637529in}{3.592139in}%
\pgfsys@useobject{currentmarker}{}%
\end{pgfscope}%
\begin{pgfscope}%
\pgfsys@transformshift{5.655123in}{3.718154in}%
\pgfsys@useobject{currentmarker}{}%
\end{pgfscope}%
\begin{pgfscope}%
\pgfsys@transformshift{5.672717in}{3.681771in}%
\pgfsys@useobject{currentmarker}{}%
\end{pgfscope}%
\begin{pgfscope}%
\pgfsys@transformshift{5.690310in}{3.754388in}%
\pgfsys@useobject{currentmarker}{}%
\end{pgfscope}%
\begin{pgfscope}%
\pgfsys@transformshift{5.707904in}{3.651918in}%
\pgfsys@useobject{currentmarker}{}%
\end{pgfscope}%
\begin{pgfscope}%
\pgfsys@transformshift{5.725498in}{3.628742in}%
\pgfsys@useobject{currentmarker}{}%
\end{pgfscope}%
\begin{pgfscope}%
\pgfsys@transformshift{5.743092in}{3.708892in}%
\pgfsys@useobject{currentmarker}{}%
\end{pgfscope}%
\begin{pgfscope}%
\pgfsys@transformshift{5.760686in}{3.774506in}%
\pgfsys@useobject{currentmarker}{}%
\end{pgfscope}%
\begin{pgfscope}%
\pgfsys@transformshift{5.778280in}{3.840115in}%
\pgfsys@useobject{currentmarker}{}%
\end{pgfscope}%
\begin{pgfscope}%
\pgfsys@transformshift{5.795873in}{4.056527in}%
\pgfsys@useobject{currentmarker}{}%
\end{pgfscope}%
\begin{pgfscope}%
\pgfsys@transformshift{5.813467in}{3.845793in}%
\pgfsys@useobject{currentmarker}{}%
\end{pgfscope}%
\begin{pgfscope}%
\pgfsys@transformshift{5.831061in}{3.775681in}%
\pgfsys@useobject{currentmarker}{}%
\end{pgfscope}%
\begin{pgfscope}%
\pgfsys@transformshift{5.848655in}{3.859859in}%
\pgfsys@useobject{currentmarker}{}%
\end{pgfscope}%
\begin{pgfscope}%
\pgfsys@transformshift{5.866249in}{3.868597in}%
\pgfsys@useobject{currentmarker}{}%
\end{pgfscope}%
\begin{pgfscope}%
\pgfsys@transformshift{5.883843in}{3.984304in}%
\pgfsys@useobject{currentmarker}{}%
\end{pgfscope}%
\begin{pgfscope}%
\pgfsys@transformshift{5.901437in}{3.795885in}%
\pgfsys@useobject{currentmarker}{}%
\end{pgfscope}%
\begin{pgfscope}%
\pgfsys@transformshift{5.919030in}{3.975605in}%
\pgfsys@useobject{currentmarker}{}%
\end{pgfscope}%
\begin{pgfscope}%
\pgfsys@transformshift{5.936624in}{3.999505in}%
\pgfsys@useobject{currentmarker}{}%
\end{pgfscope}%
\begin{pgfscope}%
\pgfsys@transformshift{5.954218in}{4.019775in}%
\pgfsys@useobject{currentmarker}{}%
\end{pgfscope}%
\begin{pgfscope}%
\pgfsys@transformshift{5.971812in}{3.945838in}%
\pgfsys@useobject{currentmarker}{}%
\end{pgfscope}%
\begin{pgfscope}%
\pgfsys@transformshift{5.989406in}{3.991188in}%
\pgfsys@useobject{currentmarker}{}%
\end{pgfscope}%
\begin{pgfscope}%
\pgfsys@transformshift{6.007000in}{3.876964in}%
\pgfsys@useobject{currentmarker}{}%
\end{pgfscope}%
\begin{pgfscope}%
\pgfsys@transformshift{6.024594in}{3.976630in}%
\pgfsys@useobject{currentmarker}{}%
\end{pgfscope}%
\begin{pgfscope}%
\pgfsys@transformshift{6.042187in}{3.974377in}%
\pgfsys@useobject{currentmarker}{}%
\end{pgfscope}%
\begin{pgfscope}%
\pgfsys@transformshift{6.059781in}{4.073041in}%
\pgfsys@useobject{currentmarker}{}%
\end{pgfscope}%
\begin{pgfscope}%
\pgfsys@transformshift{6.077375in}{3.911734in}%
\pgfsys@useobject{currentmarker}{}%
\end{pgfscope}%
\begin{pgfscope}%
\pgfsys@transformshift{6.094969in}{4.106534in}%
\pgfsys@useobject{currentmarker}{}%
\end{pgfscope}%
\begin{pgfscope}%
\pgfsys@transformshift{6.112563in}{4.174823in}%
\pgfsys@useobject{currentmarker}{}%
\end{pgfscope}%
\begin{pgfscope}%
\pgfsys@transformshift{6.130157in}{3.805324in}%
\pgfsys@useobject{currentmarker}{}%
\end{pgfscope}%
\begin{pgfscope}%
\pgfsys@transformshift{6.147751in}{4.052418in}%
\pgfsys@useobject{currentmarker}{}%
\end{pgfscope}%
\begin{pgfscope}%
\pgfsys@transformshift{6.165344in}{4.069229in}%
\pgfsys@useobject{currentmarker}{}%
\end{pgfscope}%
\begin{pgfscope}%
\pgfsys@transformshift{6.182938in}{3.925320in}%
\pgfsys@useobject{currentmarker}{}%
\end{pgfscope}%
\begin{pgfscope}%
\pgfsys@transformshift{6.200532in}{3.938972in}%
\pgfsys@useobject{currentmarker}{}%
\end{pgfscope}%
\begin{pgfscope}%
\pgfsys@transformshift{6.218126in}{3.954318in}%
\pgfsys@useobject{currentmarker}{}%
\end{pgfscope}%
\begin{pgfscope}%
\pgfsys@transformshift{6.235720in}{3.925304in}%
\pgfsys@useobject{currentmarker}{}%
\end{pgfscope}%
\begin{pgfscope}%
\pgfsys@transformshift{6.253314in}{3.911578in}%
\pgfsys@useobject{currentmarker}{}%
\end{pgfscope}%
\begin{pgfscope}%
\pgfsys@transformshift{6.270907in}{3.759418in}%
\pgfsys@useobject{currentmarker}{}%
\end{pgfscope}%
\begin{pgfscope}%
\pgfsys@transformshift{6.288501in}{4.035096in}%
\pgfsys@useobject{currentmarker}{}%
\end{pgfscope}%
\begin{pgfscope}%
\pgfsys@transformshift{6.306095in}{4.015175in}%
\pgfsys@useobject{currentmarker}{}%
\end{pgfscope}%
\begin{pgfscope}%
\pgfsys@transformshift{6.323689in}{3.810072in}%
\pgfsys@useobject{currentmarker}{}%
\end{pgfscope}%
\begin{pgfscope}%
\pgfsys@transformshift{6.341283in}{3.732047in}%
\pgfsys@useobject{currentmarker}{}%
\end{pgfscope}%
\begin{pgfscope}%
\pgfsys@transformshift{6.358877in}{3.924124in}%
\pgfsys@useobject{currentmarker}{}%
\end{pgfscope}%
\begin{pgfscope}%
\pgfsys@transformshift{6.376471in}{3.802702in}%
\pgfsys@useobject{currentmarker}{}%
\end{pgfscope}%
\end{pgfscope}%
\begin{pgfscope}%
\pgfpathrectangle{\pgfqpoint{2.000000in}{3.197368in}}{\pgfqpoint{4.376471in}{0.978947in}} %
\pgfusepath{clip}%
\pgfsetroundcap%
\pgfsetroundjoin%
\pgfsetlinewidth{1.756562pt}%
\definecolor{currentstroke}{rgb}{0.298039,0.447059,0.690196}%
\pgfsetstrokecolor{currentstroke}%
\pgfsetdash{}{0pt}%
\pgfpathmoveto{\pgfqpoint{2.896410in}{4.186316in}}%
\pgfpathlineto{\pgfqpoint{2.910482in}{4.101386in}}%
\pgfpathlineto{\pgfqpoint{2.928076in}{4.023940in}}%
\pgfpathlineto{\pgfqpoint{2.945670in}{3.967808in}}%
\pgfpathlineto{\pgfqpoint{2.963263in}{3.926974in}}%
\pgfpathlineto{\pgfqpoint{2.980857in}{3.896652in}}%
\pgfpathlineto{\pgfqpoint{2.998451in}{3.873118in}}%
\pgfpathlineto{\pgfqpoint{3.016045in}{3.853557in}}%
\pgfpathlineto{\pgfqpoint{3.086420in}{3.782917in}}%
\pgfpathlineto{\pgfqpoint{3.121608in}{3.743296in}}%
\pgfpathlineto{\pgfqpoint{3.191983in}{3.660375in}}%
\pgfpathlineto{\pgfqpoint{3.209577in}{3.641501in}}%
\pgfpathlineto{\pgfqpoint{3.227171in}{3.624312in}}%
\pgfpathlineto{\pgfqpoint{3.244765in}{3.609211in}}%
\pgfpathlineto{\pgfqpoint{3.262359in}{3.596536in}}%
\pgfpathlineto{\pgfqpoint{3.279953in}{3.586547in}}%
\pgfpathlineto{\pgfqpoint{3.297547in}{3.579426in}}%
\pgfpathlineto{\pgfqpoint{3.315140in}{3.575272in}}%
\pgfpathlineto{\pgfqpoint{3.332734in}{3.574105in}}%
\pgfpathlineto{\pgfqpoint{3.350328in}{3.575869in}}%
\pgfpathlineto{\pgfqpoint{3.367922in}{3.580440in}}%
\pgfpathlineto{\pgfqpoint{3.385516in}{3.587631in}}%
\pgfpathlineto{\pgfqpoint{3.403110in}{3.597205in}}%
\pgfpathlineto{\pgfqpoint{3.420704in}{3.608879in}}%
\pgfpathlineto{\pgfqpoint{3.438297in}{3.622340in}}%
\pgfpathlineto{\pgfqpoint{3.473485in}{3.653249in}}%
\pgfpathlineto{\pgfqpoint{3.579048in}{3.752690in}}%
\pgfpathlineto{\pgfqpoint{3.614236in}{3.779750in}}%
\pgfpathlineto{\pgfqpoint{3.631830in}{3.791052in}}%
\pgfpathlineto{\pgfqpoint{3.649424in}{3.800654in}}%
\pgfpathlineto{\pgfqpoint{3.667017in}{3.808440in}}%
\pgfpathlineto{\pgfqpoint{3.684611in}{3.814326in}}%
\pgfpathlineto{\pgfqpoint{3.702205in}{3.818263in}}%
\pgfpathlineto{\pgfqpoint{3.719799in}{3.820229in}}%
\pgfpathlineto{\pgfqpoint{3.737393in}{3.820234in}}%
\pgfpathlineto{\pgfqpoint{3.754987in}{3.818311in}}%
\pgfpathlineto{\pgfqpoint{3.772581in}{3.814516in}}%
\pgfpathlineto{\pgfqpoint{3.790174in}{3.808926in}}%
\pgfpathlineto{\pgfqpoint{3.807768in}{3.801635in}}%
\pgfpathlineto{\pgfqpoint{3.825362in}{3.792749in}}%
\pgfpathlineto{\pgfqpoint{3.842956in}{3.782386in}}%
\pgfpathlineto{\pgfqpoint{3.878144in}{3.757739in}}%
\pgfpathlineto{\pgfqpoint{3.913331in}{3.728753in}}%
\pgfpathlineto{\pgfqpoint{3.948519in}{3.696514in}}%
\pgfpathlineto{\pgfqpoint{4.001301in}{3.644355in}}%
\pgfpathlineto{\pgfqpoint{4.106864in}{3.538095in}}%
\pgfpathlineto{\pgfqpoint{4.142051in}{3.505186in}}%
\pgfpathlineto{\pgfqpoint{4.177239in}{3.474796in}}%
\pgfpathlineto{\pgfqpoint{4.212427in}{3.447640in}}%
\pgfpathlineto{\pgfqpoint{4.247615in}{3.424407in}}%
\pgfpathlineto{\pgfqpoint{4.282802in}{3.405764in}}%
\pgfpathlineto{\pgfqpoint{4.300396in}{3.398362in}}%
\pgfpathlineto{\pgfqpoint{4.317990in}{3.392339in}}%
\pgfpathlineto{\pgfqpoint{4.335584in}{3.387761in}}%
\pgfpathlineto{\pgfqpoint{4.353178in}{3.384690in}}%
\pgfpathlineto{\pgfqpoint{4.370772in}{3.383182in}}%
\pgfpathlineto{\pgfqpoint{4.388365in}{3.383281in}}%
\pgfpathlineto{\pgfqpoint{4.405959in}{3.385022in}}%
\pgfpathlineto{\pgfqpoint{4.423553in}{3.388430in}}%
\pgfpathlineto{\pgfqpoint{4.441147in}{3.393516in}}%
\pgfpathlineto{\pgfqpoint{4.458741in}{3.400276in}}%
\pgfpathlineto{\pgfqpoint{4.476335in}{3.408693in}}%
\pgfpathlineto{\pgfqpoint{4.493928in}{3.418733in}}%
\pgfpathlineto{\pgfqpoint{4.511522in}{3.430344in}}%
\pgfpathlineto{\pgfqpoint{4.529116in}{3.443458in}}%
\pgfpathlineto{\pgfqpoint{4.564304in}{3.473831in}}%
\pgfpathlineto{\pgfqpoint{4.599492in}{3.508944in}}%
\pgfpathlineto{\pgfqpoint{4.634679in}{3.547618in}}%
\pgfpathlineto{\pgfqpoint{4.757836in}{3.688928in}}%
\pgfpathlineto{\pgfqpoint{4.793024in}{3.723894in}}%
\pgfpathlineto{\pgfqpoint{4.810618in}{3.739395in}}%
\pgfpathlineto{\pgfqpoint{4.828212in}{3.753315in}}%
\pgfpathlineto{\pgfqpoint{4.845805in}{3.765477in}}%
\pgfpathlineto{\pgfqpoint{4.863399in}{3.775721in}}%
\pgfpathlineto{\pgfqpoint{4.880993in}{3.783910in}}%
\pgfpathlineto{\pgfqpoint{4.898587in}{3.789928in}}%
\pgfpathlineto{\pgfqpoint{4.916181in}{3.793687in}}%
\pgfpathlineto{\pgfqpoint{4.933775in}{3.795123in}}%
\pgfpathlineto{\pgfqpoint{4.951369in}{3.794204in}}%
\pgfpathlineto{\pgfqpoint{4.968962in}{3.790925in}}%
\pgfpathlineto{\pgfqpoint{4.986556in}{3.785314in}}%
\pgfpathlineto{\pgfqpoint{5.004150in}{3.777426in}}%
\pgfpathlineto{\pgfqpoint{5.021744in}{3.767351in}}%
\pgfpathlineto{\pgfqpoint{5.039338in}{3.755205in}}%
\pgfpathlineto{\pgfqpoint{5.056932in}{3.741134in}}%
\pgfpathlineto{\pgfqpoint{5.074526in}{3.725311in}}%
\pgfpathlineto{\pgfqpoint{5.109713in}{3.689225in}}%
\pgfpathlineto{\pgfqpoint{5.144901in}{3.648785in}}%
\pgfpathlineto{\pgfqpoint{5.250464in}{3.522986in}}%
\pgfpathlineto{\pgfqpoint{5.285652in}{3.487041in}}%
\pgfpathlineto{\pgfqpoint{5.303246in}{3.471369in}}%
\pgfpathlineto{\pgfqpoint{5.320839in}{3.457526in}}%
\pgfpathlineto{\pgfqpoint{5.338433in}{3.445707in}}%
\pgfpathlineto{\pgfqpoint{5.356027in}{3.436077in}}%
\pgfpathlineto{\pgfqpoint{5.373621in}{3.428776in}}%
\pgfpathlineto{\pgfqpoint{5.391215in}{3.423912in}}%
\pgfpathlineto{\pgfqpoint{5.408809in}{3.421562in}}%
\pgfpathlineto{\pgfqpoint{5.426403in}{3.421771in}}%
\pgfpathlineto{\pgfqpoint{5.443996in}{3.424552in}}%
\pgfpathlineto{\pgfqpoint{5.461590in}{3.429886in}}%
\pgfpathlineto{\pgfqpoint{5.479184in}{3.437722in}}%
\pgfpathlineto{\pgfqpoint{5.496778in}{3.447982in}}%
\pgfpathlineto{\pgfqpoint{5.514372in}{3.460556in}}%
\pgfpathlineto{\pgfqpoint{5.531966in}{3.475312in}}%
\pgfpathlineto{\pgfqpoint{5.549560in}{3.492095in}}%
\pgfpathlineto{\pgfqpoint{5.567153in}{3.510728in}}%
\pgfpathlineto{\pgfqpoint{5.602341in}{3.552762in}}%
\pgfpathlineto{\pgfqpoint{5.637529in}{3.599740in}}%
\pgfpathlineto{\pgfqpoint{5.690310in}{3.675605in}}%
\pgfpathlineto{\pgfqpoint{5.760686in}{3.777950in}}%
\pgfpathlineto{\pgfqpoint{5.795873in}{3.826339in}}%
\pgfpathlineto{\pgfqpoint{5.831061in}{3.871346in}}%
\pgfpathlineto{\pgfqpoint{5.866249in}{3.912133in}}%
\pgfpathlineto{\pgfqpoint{5.901437in}{3.948064in}}%
\pgfpathlineto{\pgfqpoint{5.936624in}{3.978642in}}%
\pgfpathlineto{\pgfqpoint{5.954218in}{3.991784in}}%
\pgfpathlineto{\pgfqpoint{5.971812in}{4.003418in}}%
\pgfpathlineto{\pgfqpoint{5.989406in}{4.013483in}}%
\pgfpathlineto{\pgfqpoint{6.007000in}{4.021910in}}%
\pgfpathlineto{\pgfqpoint{6.024594in}{4.028625in}}%
\pgfpathlineto{\pgfqpoint{6.042187in}{4.033542in}}%
\pgfpathlineto{\pgfqpoint{6.059781in}{4.036573in}}%
\pgfpathlineto{\pgfqpoint{6.077375in}{4.037624in}}%
\pgfpathlineto{\pgfqpoint{6.094969in}{4.036599in}}%
\pgfpathlineto{\pgfqpoint{6.112563in}{4.033411in}}%
\pgfpathlineto{\pgfqpoint{6.130157in}{4.027987in}}%
\pgfpathlineto{\pgfqpoint{6.147751in}{4.020276in}}%
\pgfpathlineto{\pgfqpoint{6.165344in}{4.010267in}}%
\pgfpathlineto{\pgfqpoint{6.182938in}{3.998002in}}%
\pgfpathlineto{\pgfqpoint{6.200532in}{3.983595in}}%
\pgfpathlineto{\pgfqpoint{6.218126in}{3.967255in}}%
\pgfpathlineto{\pgfqpoint{6.253314in}{3.930229in}}%
\pgfpathlineto{\pgfqpoint{6.288501in}{3.891499in}}%
\pgfpathlineto{\pgfqpoint{6.306095in}{3.873828in}}%
\pgfpathlineto{\pgfqpoint{6.323689in}{3.859066in}}%
\pgfpathlineto{\pgfqpoint{6.341283in}{3.848953in}}%
\pgfpathlineto{\pgfqpoint{6.358877in}{3.845611in}}%
\pgfpathlineto{\pgfqpoint{6.376471in}{3.851585in}}%
\pgfpathlineto{\pgfqpoint{6.376471in}{3.851585in}}%
\pgfusepath{stroke}%
\end{pgfscope}%
\begin{pgfscope}%
\pgfpathrectangle{\pgfqpoint{2.000000in}{3.197368in}}{\pgfqpoint{4.376471in}{0.978947in}} %
\pgfusepath{clip}%
\pgfsetbuttcap%
\pgfsetroundjoin%
\pgfsetlinewidth{1.756562pt}%
\definecolor{currentstroke}{rgb}{1.000000,0.647059,0.000000}%
\pgfsetstrokecolor{currentstroke}%
\pgfsetdash{{6.000000pt}{6.000000pt}}{0.000000pt}%
\pgfpathmoveto{\pgfqpoint{2.896399in}{4.186316in}}%
\pgfpathlineto{\pgfqpoint{2.910482in}{4.101329in}}%
\pgfpathlineto{\pgfqpoint{2.928076in}{4.023905in}}%
\pgfpathlineto{\pgfqpoint{2.945670in}{3.967806in}}%
\pgfpathlineto{\pgfqpoint{2.963263in}{3.926994in}}%
\pgfpathlineto{\pgfqpoint{2.980857in}{3.896680in}}%
\pgfpathlineto{\pgfqpoint{2.998451in}{3.873132in}}%
\pgfpathlineto{\pgfqpoint{3.016045in}{3.853574in}}%
\pgfpathlineto{\pgfqpoint{3.086420in}{3.782890in}}%
\pgfpathlineto{\pgfqpoint{3.121608in}{3.743284in}}%
\pgfpathlineto{\pgfqpoint{3.191983in}{3.660340in}}%
\pgfpathlineto{\pgfqpoint{3.209577in}{3.641453in}}%
\pgfpathlineto{\pgfqpoint{3.227171in}{3.624204in}}%
\pgfpathlineto{\pgfqpoint{3.244765in}{3.609134in}}%
\pgfpathlineto{\pgfqpoint{3.262359in}{3.596463in}}%
\pgfpathlineto{\pgfqpoint{3.279953in}{3.586479in}}%
\pgfpathlineto{\pgfqpoint{3.297547in}{3.579404in}}%
\pgfpathlineto{\pgfqpoint{3.315140in}{3.575202in}}%
\pgfpathlineto{\pgfqpoint{3.332734in}{3.574091in}}%
\pgfpathlineto{\pgfqpoint{3.350328in}{3.575807in}}%
\pgfpathlineto{\pgfqpoint{3.367922in}{3.580378in}}%
\pgfpathlineto{\pgfqpoint{3.385516in}{3.587626in}}%
\pgfpathlineto{\pgfqpoint{3.403110in}{3.597183in}}%
\pgfpathlineto{\pgfqpoint{3.420704in}{3.608876in}}%
\pgfpathlineto{\pgfqpoint{3.438297in}{3.622390in}}%
\pgfpathlineto{\pgfqpoint{3.473485in}{3.653292in}}%
\pgfpathlineto{\pgfqpoint{3.579048in}{3.752853in}}%
\pgfpathlineto{\pgfqpoint{3.614236in}{3.780023in}}%
\pgfpathlineto{\pgfqpoint{3.631830in}{3.791294in}}%
\pgfpathlineto{\pgfqpoint{3.649424in}{3.800882in}}%
\pgfpathlineto{\pgfqpoint{3.667017in}{3.808712in}}%
\pgfpathlineto{\pgfqpoint{3.684611in}{3.814654in}}%
\pgfpathlineto{\pgfqpoint{3.702205in}{3.818597in}}%
\pgfpathlineto{\pgfqpoint{3.719799in}{3.820556in}}%
\pgfpathlineto{\pgfqpoint{3.737393in}{3.820534in}}%
\pgfpathlineto{\pgfqpoint{3.754987in}{3.818621in}}%
\pgfpathlineto{\pgfqpoint{3.772581in}{3.814871in}}%
\pgfpathlineto{\pgfqpoint{3.790174in}{3.809247in}}%
\pgfpathlineto{\pgfqpoint{3.807768in}{3.801962in}}%
\pgfpathlineto{\pgfqpoint{3.825362in}{3.793080in}}%
\pgfpathlineto{\pgfqpoint{3.842956in}{3.782727in}}%
\pgfpathlineto{\pgfqpoint{3.878144in}{3.758036in}}%
\pgfpathlineto{\pgfqpoint{3.913331in}{3.729042in}}%
\pgfpathlineto{\pgfqpoint{3.948519in}{3.696820in}}%
\pgfpathlineto{\pgfqpoint{4.001301in}{3.644601in}}%
\pgfpathlineto{\pgfqpoint{4.106864in}{3.538227in}}%
\pgfpathlineto{\pgfqpoint{4.142051in}{3.505284in}}%
\pgfpathlineto{\pgfqpoint{4.177239in}{3.474904in}}%
\pgfpathlineto{\pgfqpoint{4.212427in}{3.447662in}}%
\pgfpathlineto{\pgfqpoint{4.247615in}{3.424521in}}%
\pgfpathlineto{\pgfqpoint{4.265208in}{3.414538in}}%
\pgfpathlineto{\pgfqpoint{4.282802in}{3.405862in}}%
\pgfpathlineto{\pgfqpoint{4.300396in}{3.398455in}}%
\pgfpathlineto{\pgfqpoint{4.317990in}{3.392331in}}%
\pgfpathlineto{\pgfqpoint{4.335584in}{3.387862in}}%
\pgfpathlineto{\pgfqpoint{4.353178in}{3.384738in}}%
\pgfpathlineto{\pgfqpoint{4.370772in}{3.383281in}}%
\pgfpathlineto{\pgfqpoint{4.388365in}{3.383331in}}%
\pgfpathlineto{\pgfqpoint{4.405959in}{3.385074in}}%
\pgfpathlineto{\pgfqpoint{4.423553in}{3.388447in}}%
\pgfpathlineto{\pgfqpoint{4.441147in}{3.393588in}}%
\pgfpathlineto{\pgfqpoint{4.458741in}{3.400322in}}%
\pgfpathlineto{\pgfqpoint{4.476335in}{3.408812in}}%
\pgfpathlineto{\pgfqpoint{4.493928in}{3.418783in}}%
\pgfpathlineto{\pgfqpoint{4.511522in}{3.430384in}}%
\pgfpathlineto{\pgfqpoint{4.529116in}{3.443504in}}%
\pgfpathlineto{\pgfqpoint{4.564304in}{3.473840in}}%
\pgfpathlineto{\pgfqpoint{4.599492in}{3.509043in}}%
\pgfpathlineto{\pgfqpoint{4.634679in}{3.547644in}}%
\pgfpathlineto{\pgfqpoint{4.757836in}{3.688928in}}%
\pgfpathlineto{\pgfqpoint{4.793024in}{3.723857in}}%
\pgfpathlineto{\pgfqpoint{4.810618in}{3.739467in}}%
\pgfpathlineto{\pgfqpoint{4.828212in}{3.753272in}}%
\pgfpathlineto{\pgfqpoint{4.845805in}{3.765421in}}%
\pgfpathlineto{\pgfqpoint{4.863399in}{3.775765in}}%
\pgfpathlineto{\pgfqpoint{4.880993in}{3.783944in}}%
\pgfpathlineto{\pgfqpoint{4.898587in}{3.789919in}}%
\pgfpathlineto{\pgfqpoint{4.916181in}{3.793703in}}%
\pgfpathlineto{\pgfqpoint{4.933775in}{3.795209in}}%
\pgfpathlineto{\pgfqpoint{4.951369in}{3.794201in}}%
\pgfpathlineto{\pgfqpoint{4.968962in}{3.790989in}}%
\pgfpathlineto{\pgfqpoint{4.986556in}{3.785288in}}%
\pgfpathlineto{\pgfqpoint{5.004150in}{3.777471in}}%
\pgfpathlineto{\pgfqpoint{5.021744in}{3.767338in}}%
\pgfpathlineto{\pgfqpoint{5.039338in}{3.755176in}}%
\pgfpathlineto{\pgfqpoint{5.056932in}{3.741185in}}%
\pgfpathlineto{\pgfqpoint{5.074526in}{3.725364in}}%
\pgfpathlineto{\pgfqpoint{5.109713in}{3.689240in}}%
\pgfpathlineto{\pgfqpoint{5.144901in}{3.648765in}}%
\pgfpathlineto{\pgfqpoint{5.250464in}{3.522997in}}%
\pgfpathlineto{\pgfqpoint{5.285652in}{3.487035in}}%
\pgfpathlineto{\pgfqpoint{5.303246in}{3.471388in}}%
\pgfpathlineto{\pgfqpoint{5.320839in}{3.457543in}}%
\pgfpathlineto{\pgfqpoint{5.338433in}{3.445764in}}%
\pgfpathlineto{\pgfqpoint{5.356027in}{3.436048in}}%
\pgfpathlineto{\pgfqpoint{5.373621in}{3.428763in}}%
\pgfpathlineto{\pgfqpoint{5.391215in}{3.423925in}}%
\pgfpathlineto{\pgfqpoint{5.408809in}{3.421556in}}%
\pgfpathlineto{\pgfqpoint{5.426403in}{3.421768in}}%
\pgfpathlineto{\pgfqpoint{5.443996in}{3.424576in}}%
\pgfpathlineto{\pgfqpoint{5.461590in}{3.429874in}}%
\pgfpathlineto{\pgfqpoint{5.479184in}{3.437780in}}%
\pgfpathlineto{\pgfqpoint{5.496778in}{3.447972in}}%
\pgfpathlineto{\pgfqpoint{5.514372in}{3.460608in}}%
\pgfpathlineto{\pgfqpoint{5.531966in}{3.475359in}}%
\pgfpathlineto{\pgfqpoint{5.549560in}{3.492076in}}%
\pgfpathlineto{\pgfqpoint{5.584747in}{3.531020in}}%
\pgfpathlineto{\pgfqpoint{5.619935in}{3.575739in}}%
\pgfpathlineto{\pgfqpoint{5.655123in}{3.624476in}}%
\pgfpathlineto{\pgfqpoint{5.795873in}{3.826382in}}%
\pgfpathlineto{\pgfqpoint{5.831061in}{3.871384in}}%
\pgfpathlineto{\pgfqpoint{5.866249in}{3.912164in}}%
\pgfpathlineto{\pgfqpoint{5.901437in}{3.948114in}}%
\pgfpathlineto{\pgfqpoint{5.936624in}{3.978673in}}%
\pgfpathlineto{\pgfqpoint{5.954218in}{3.991781in}}%
\pgfpathlineto{\pgfqpoint{5.971812in}{4.003457in}}%
\pgfpathlineto{\pgfqpoint{5.989406in}{4.013500in}}%
\pgfpathlineto{\pgfqpoint{6.007000in}{4.021911in}}%
\pgfpathlineto{\pgfqpoint{6.024594in}{4.028608in}}%
\pgfpathlineto{\pgfqpoint{6.042187in}{4.033581in}}%
\pgfpathlineto{\pgfqpoint{6.059781in}{4.036622in}}%
\pgfpathlineto{\pgfqpoint{6.077375in}{4.037656in}}%
\pgfpathlineto{\pgfqpoint{6.094969in}{4.036629in}}%
\pgfpathlineto{\pgfqpoint{6.112563in}{4.033447in}}%
\pgfpathlineto{\pgfqpoint{6.130157in}{4.027995in}}%
\pgfpathlineto{\pgfqpoint{6.147751in}{4.020281in}}%
\pgfpathlineto{\pgfqpoint{6.165344in}{4.010293in}}%
\pgfpathlineto{\pgfqpoint{6.182938in}{3.997993in}}%
\pgfpathlineto{\pgfqpoint{6.200532in}{3.983634in}}%
\pgfpathlineto{\pgfqpoint{6.218126in}{3.967265in}}%
\pgfpathlineto{\pgfqpoint{6.253314in}{3.930234in}}%
\pgfpathlineto{\pgfqpoint{6.288501in}{3.891520in}}%
\pgfpathlineto{\pgfqpoint{6.306095in}{3.873875in}}%
\pgfpathlineto{\pgfqpoint{6.323689in}{3.859116in}}%
\pgfpathlineto{\pgfqpoint{6.341283in}{3.849078in}}%
\pgfpathlineto{\pgfqpoint{6.358877in}{3.845786in}}%
\pgfpathlineto{\pgfqpoint{6.376471in}{3.851849in}}%
\pgfpathlineto{\pgfqpoint{6.376471in}{3.851849in}}%
\pgfusepath{stroke}%
\end{pgfscope}%
\begin{pgfscope}%
\pgfsetrectcap%
\pgfsetmiterjoin%
\pgfsetlinewidth{1.003750pt}%
\definecolor{currentstroke}{rgb}{0.800000,0.800000,0.800000}%
\pgfsetstrokecolor{currentstroke}%
\pgfsetdash{}{0pt}%
\pgfpathmoveto{\pgfqpoint{2.000000in}{3.197368in}}%
\pgfpathlineto{\pgfqpoint{2.000000in}{4.176316in}}%
\pgfusepath{stroke}%
\end{pgfscope}%
\begin{pgfscope}%
\pgfsetrectcap%
\pgfsetmiterjoin%
\pgfsetlinewidth{1.003750pt}%
\definecolor{currentstroke}{rgb}{0.800000,0.800000,0.800000}%
\pgfsetstrokecolor{currentstroke}%
\pgfsetdash{}{0pt}%
\pgfpathmoveto{\pgfqpoint{6.376471in}{3.197368in}}%
\pgfpathlineto{\pgfqpoint{6.376471in}{4.176316in}}%
\pgfusepath{stroke}%
\end{pgfscope}%
\begin{pgfscope}%
\pgfsetrectcap%
\pgfsetmiterjoin%
\pgfsetlinewidth{1.003750pt}%
\definecolor{currentstroke}{rgb}{0.800000,0.800000,0.800000}%
\pgfsetstrokecolor{currentstroke}%
\pgfsetdash{}{0pt}%
\pgfpathmoveto{\pgfqpoint{2.000000in}{4.176316in}}%
\pgfpathlineto{\pgfqpoint{6.376471in}{4.176316in}}%
\pgfusepath{stroke}%
\end{pgfscope}%
\begin{pgfscope}%
\pgfsetrectcap%
\pgfsetmiterjoin%
\pgfsetlinewidth{1.003750pt}%
\definecolor{currentstroke}{rgb}{0.800000,0.800000,0.800000}%
\pgfsetstrokecolor{currentstroke}%
\pgfsetdash{}{0pt}%
\pgfpathmoveto{\pgfqpoint{2.000000in}{3.197368in}}%
\pgfpathlineto{\pgfqpoint{6.376471in}{3.197368in}}%
\pgfusepath{stroke}%
\end{pgfscope}%
\begin{pgfscope}%
\pgfsetroundcap%
\pgfsetroundjoin%
\pgfsetlinewidth{1.756562pt}%
\definecolor{currentstroke}{rgb}{0.298039,0.447059,0.690196}%
\pgfsetstrokecolor{currentstroke}%
\pgfsetdash{}{0pt}%
\pgfpathmoveto{\pgfqpoint{2.125000in}{3.996372in}}%
\pgfpathlineto{\pgfqpoint{2.402778in}{3.996372in}}%
\pgfusepath{stroke}%
\end{pgfscope}%
\begin{pgfscope}%
\definecolor{textcolor}{rgb}{0.150000,0.150000,0.150000}%
\pgfsetstrokecolor{textcolor}%
\pgfsetfillcolor{textcolor}%
\pgftext[x=2.513889in,y=3.947761in,left,base]{\color{textcolor}\sffamily\fontsize{10.000000}{12.000000}\selectfont \(\displaystyle \widetilde{\Phi}^* \theta\)}%
\end{pgfscope}%
\begin{pgfscope}%
\pgfsetbuttcap%
\pgfsetroundjoin%
\pgfsetlinewidth{1.756562pt}%
\definecolor{currentstroke}{rgb}{1.000000,0.647059,0.000000}%
\pgfsetstrokecolor{currentstroke}%
\pgfsetdash{{6.000000pt}{6.000000pt}}{0.000000pt}%
\pgfpathmoveto{\pgfqpoint{2.125000in}{3.791511in}}%
\pgfpathlineto{\pgfqpoint{2.402778in}{3.791511in}}%
\pgfusepath{stroke}%
\end{pgfscope}%
\begin{pgfscope}%
\definecolor{textcolor}{rgb}{0.150000,0.150000,0.150000}%
\pgfsetstrokecolor{textcolor}%
\pgfsetfillcolor{textcolor}%
\pgftext[x=2.513889in,y=3.742900in,left,base]{\color{textcolor}\sffamily\fontsize{10.000000}{12.000000}\selectfont \(\displaystyle \widetilde{K}u\)}%
\end{pgfscope}%
\begin{pgfscope}%
\pgfsetbuttcap%
\pgfsetroundjoin%
\definecolor{currentfill}{rgb}{1.000000,0.000000,0.000000}%
\pgfsetfillcolor{currentfill}%
\pgfsetlinewidth{2.007500pt}%
\definecolor{currentstroke}{rgb}{1.000000,0.000000,0.000000}%
\pgfsetstrokecolor{currentstroke}%
\pgfsetdash{}{0pt}%
\pgfpathmoveto{\pgfqpoint{2.232832in}{3.582893in}}%
\pgfpathlineto{\pgfqpoint{2.294945in}{3.582893in}}%
\pgfpathmoveto{\pgfqpoint{2.263889in}{3.551837in}}%
\pgfpathlineto{\pgfqpoint{2.263889in}{3.613950in}}%
\pgfusepath{stroke,fill}%
\end{pgfscope}%
\begin{pgfscope}%
\pgfsetbuttcap%
\pgfsetroundjoin%
\definecolor{currentfill}{rgb}{1.000000,0.000000,0.000000}%
\pgfsetfillcolor{currentfill}%
\pgfsetlinewidth{2.007500pt}%
\definecolor{currentstroke}{rgb}{1.000000,0.000000,0.000000}%
\pgfsetstrokecolor{currentstroke}%
\pgfsetdash{}{0pt}%
\pgfpathmoveto{\pgfqpoint{2.232832in}{3.582893in}}%
\pgfpathlineto{\pgfqpoint{2.294945in}{3.582893in}}%
\pgfpathmoveto{\pgfqpoint{2.263889in}{3.551837in}}%
\pgfpathlineto{\pgfqpoint{2.263889in}{3.613950in}}%
\pgfusepath{stroke,fill}%
\end{pgfscope}%
\begin{pgfscope}%
\pgfsetbuttcap%
\pgfsetroundjoin%
\definecolor{currentfill}{rgb}{1.000000,0.000000,0.000000}%
\pgfsetfillcolor{currentfill}%
\pgfsetlinewidth{2.007500pt}%
\definecolor{currentstroke}{rgb}{1.000000,0.000000,0.000000}%
\pgfsetstrokecolor{currentstroke}%
\pgfsetdash{}{0pt}%
\pgfpathmoveto{\pgfqpoint{2.232832in}{3.582893in}}%
\pgfpathlineto{\pgfqpoint{2.294945in}{3.582893in}}%
\pgfpathmoveto{\pgfqpoint{2.263889in}{3.551837in}}%
\pgfpathlineto{\pgfqpoint{2.263889in}{3.613950in}}%
\pgfusepath{stroke,fill}%
\end{pgfscope}%
\begin{pgfscope}%
\definecolor{textcolor}{rgb}{0.150000,0.150000,0.150000}%
\pgfsetstrokecolor{textcolor}%
\pgfsetfillcolor{textcolor}%
\pgftext[x=2.513889in,y=3.546435in,left,base]{\color{textcolor}\sffamily\fontsize{10.000000}{12.000000}\selectfont train}%
\end{pgfscope}%
\begin{pgfscope}%
\pgfsetbuttcap%
\pgfsetroundjoin%
\definecolor{currentfill}{rgb}{0.000000,0.000000,0.000000}%
\pgfsetfillcolor{currentfill}%
\pgfsetlinewidth{0.301125pt}%
\definecolor{currentstroke}{rgb}{0.000000,0.000000,0.000000}%
\pgfsetstrokecolor{currentstroke}%
\pgfsetdash{}{0pt}%
\pgfpathmoveto{\pgfqpoint{2.263889in}{3.370900in}}%
\pgfpathcurveto{\pgfqpoint{2.268007in}{3.370900in}}{\pgfqpoint{2.271957in}{3.372536in}}{\pgfqpoint{2.274869in}{3.375448in}}%
\pgfpathcurveto{\pgfqpoint{2.277781in}{3.378360in}}{\pgfqpoint{2.279417in}{3.382310in}}{\pgfqpoint{2.279417in}{3.386428in}}%
\pgfpathcurveto{\pgfqpoint{2.279417in}{3.390546in}}{\pgfqpoint{2.277781in}{3.394496in}}{\pgfqpoint{2.274869in}{3.397408in}}%
\pgfpathcurveto{\pgfqpoint{2.271957in}{3.400320in}}{\pgfqpoint{2.268007in}{3.401956in}}{\pgfqpoint{2.263889in}{3.401956in}}%
\pgfpathcurveto{\pgfqpoint{2.259771in}{3.401956in}}{\pgfqpoint{2.255821in}{3.400320in}}{\pgfqpoint{2.252909in}{3.397408in}}%
\pgfpathcurveto{\pgfqpoint{2.249997in}{3.394496in}}{\pgfqpoint{2.248361in}{3.390546in}}{\pgfqpoint{2.248361in}{3.386428in}}%
\pgfpathcurveto{\pgfqpoint{2.248361in}{3.382310in}}{\pgfqpoint{2.249997in}{3.378360in}}{\pgfqpoint{2.252909in}{3.375448in}}%
\pgfpathcurveto{\pgfqpoint{2.255821in}{3.372536in}}{\pgfqpoint{2.259771in}{3.370900in}}{\pgfqpoint{2.263889in}{3.370900in}}%
\pgfpathclose%
\pgfusepath{stroke,fill}%
\end{pgfscope}%
\begin{pgfscope}%
\pgfsetbuttcap%
\pgfsetroundjoin%
\definecolor{currentfill}{rgb}{0.000000,0.000000,0.000000}%
\pgfsetfillcolor{currentfill}%
\pgfsetlinewidth{0.301125pt}%
\definecolor{currentstroke}{rgb}{0.000000,0.000000,0.000000}%
\pgfsetstrokecolor{currentstroke}%
\pgfsetdash{}{0pt}%
\pgfpathmoveto{\pgfqpoint{2.263889in}{3.370900in}}%
\pgfpathcurveto{\pgfqpoint{2.268007in}{3.370900in}}{\pgfqpoint{2.271957in}{3.372536in}}{\pgfqpoint{2.274869in}{3.375448in}}%
\pgfpathcurveto{\pgfqpoint{2.277781in}{3.378360in}}{\pgfqpoint{2.279417in}{3.382310in}}{\pgfqpoint{2.279417in}{3.386428in}}%
\pgfpathcurveto{\pgfqpoint{2.279417in}{3.390546in}}{\pgfqpoint{2.277781in}{3.394496in}}{\pgfqpoint{2.274869in}{3.397408in}}%
\pgfpathcurveto{\pgfqpoint{2.271957in}{3.400320in}}{\pgfqpoint{2.268007in}{3.401956in}}{\pgfqpoint{2.263889in}{3.401956in}}%
\pgfpathcurveto{\pgfqpoint{2.259771in}{3.401956in}}{\pgfqpoint{2.255821in}{3.400320in}}{\pgfqpoint{2.252909in}{3.397408in}}%
\pgfpathcurveto{\pgfqpoint{2.249997in}{3.394496in}}{\pgfqpoint{2.248361in}{3.390546in}}{\pgfqpoint{2.248361in}{3.386428in}}%
\pgfpathcurveto{\pgfqpoint{2.248361in}{3.382310in}}{\pgfqpoint{2.249997in}{3.378360in}}{\pgfqpoint{2.252909in}{3.375448in}}%
\pgfpathcurveto{\pgfqpoint{2.255821in}{3.372536in}}{\pgfqpoint{2.259771in}{3.370900in}}{\pgfqpoint{2.263889in}{3.370900in}}%
\pgfpathclose%
\pgfusepath{stroke,fill}%
\end{pgfscope}%
\begin{pgfscope}%
\pgfsetbuttcap%
\pgfsetroundjoin%
\definecolor{currentfill}{rgb}{0.000000,0.000000,0.000000}%
\pgfsetfillcolor{currentfill}%
\pgfsetlinewidth{0.301125pt}%
\definecolor{currentstroke}{rgb}{0.000000,0.000000,0.000000}%
\pgfsetstrokecolor{currentstroke}%
\pgfsetdash{}{0pt}%
\pgfpathmoveto{\pgfqpoint{2.263889in}{3.370900in}}%
\pgfpathcurveto{\pgfqpoint{2.268007in}{3.370900in}}{\pgfqpoint{2.271957in}{3.372536in}}{\pgfqpoint{2.274869in}{3.375448in}}%
\pgfpathcurveto{\pgfqpoint{2.277781in}{3.378360in}}{\pgfqpoint{2.279417in}{3.382310in}}{\pgfqpoint{2.279417in}{3.386428in}}%
\pgfpathcurveto{\pgfqpoint{2.279417in}{3.390546in}}{\pgfqpoint{2.277781in}{3.394496in}}{\pgfqpoint{2.274869in}{3.397408in}}%
\pgfpathcurveto{\pgfqpoint{2.271957in}{3.400320in}}{\pgfqpoint{2.268007in}{3.401956in}}{\pgfqpoint{2.263889in}{3.401956in}}%
\pgfpathcurveto{\pgfqpoint{2.259771in}{3.401956in}}{\pgfqpoint{2.255821in}{3.400320in}}{\pgfqpoint{2.252909in}{3.397408in}}%
\pgfpathcurveto{\pgfqpoint{2.249997in}{3.394496in}}{\pgfqpoint{2.248361in}{3.390546in}}{\pgfqpoint{2.248361in}{3.386428in}}%
\pgfpathcurveto{\pgfqpoint{2.248361in}{3.382310in}}{\pgfqpoint{2.249997in}{3.378360in}}{\pgfqpoint{2.252909in}{3.375448in}}%
\pgfpathcurveto{\pgfqpoint{2.255821in}{3.372536in}}{\pgfqpoint{2.259771in}{3.370900in}}{\pgfqpoint{2.263889in}{3.370900in}}%
\pgfpathclose%
\pgfusepath{stroke,fill}%
\end{pgfscope}%
\begin{pgfscope}%
\definecolor{textcolor}{rgb}{0.150000,0.150000,0.150000}%
\pgfsetstrokecolor{textcolor}%
\pgfsetfillcolor{textcolor}%
\pgftext[x=2.513889in,y=3.349970in,left,base]{\color{textcolor}\sffamily\fontsize{10.000000}{12.000000}\selectfont test}%
\end{pgfscope}%
\begin{pgfscope}%
\pgfsetbuttcap%
\pgfsetmiterjoin%
\definecolor{currentfill}{rgb}{1.000000,1.000000,1.000000}%
\pgfsetfillcolor{currentfill}%
\pgfsetlinewidth{0.000000pt}%
\definecolor{currentstroke}{rgb}{0.000000,0.000000,0.000000}%
\pgfsetstrokecolor{currentstroke}%
\pgfsetstrokeopacity{0.000000}%
\pgfsetdash{}{0pt}%
\pgfpathmoveto{\pgfqpoint{7.105882in}{3.197368in}}%
\pgfpathlineto{\pgfqpoint{11.482353in}{3.197368in}}%
\pgfpathlineto{\pgfqpoint{11.482353in}{4.176316in}}%
\pgfpathlineto{\pgfqpoint{7.105882in}{4.176316in}}%
\pgfpathclose%
\pgfusepath{fill}%
\end{pgfscope}%
\begin{pgfscope}%
\pgfpathrectangle{\pgfqpoint{7.105882in}{3.197368in}}{\pgfqpoint{4.376471in}{0.978947in}} %
\pgfusepath{clip}%
\pgfsetroundcap%
\pgfsetroundjoin%
\pgfsetlinewidth{1.003750pt}%
\definecolor{currentstroke}{rgb}{0.800000,0.800000,0.800000}%
\pgfsetstrokecolor{currentstroke}%
\pgfsetdash{}{0pt}%
\pgfpathmoveto{\pgfqpoint{7.105882in}{3.197368in}}%
\pgfpathlineto{\pgfqpoint{7.105882in}{4.176316in}}%
\pgfusepath{stroke}%
\end{pgfscope}%
\begin{pgfscope}%
\pgfpathrectangle{\pgfqpoint{7.105882in}{3.197368in}}{\pgfqpoint{4.376471in}{0.978947in}} %
\pgfusepath{clip}%
\pgfsetroundcap%
\pgfsetroundjoin%
\pgfsetlinewidth{1.003750pt}%
\definecolor{currentstroke}{rgb}{0.800000,0.800000,0.800000}%
\pgfsetstrokecolor{currentstroke}%
\pgfsetdash{}{0pt}%
\pgfpathmoveto{\pgfqpoint{7.981176in}{3.197368in}}%
\pgfpathlineto{\pgfqpoint{7.981176in}{4.176316in}}%
\pgfusepath{stroke}%
\end{pgfscope}%
\begin{pgfscope}%
\pgfpathrectangle{\pgfqpoint{7.105882in}{3.197368in}}{\pgfqpoint{4.376471in}{0.978947in}} %
\pgfusepath{clip}%
\pgfsetroundcap%
\pgfsetroundjoin%
\pgfsetlinewidth{1.003750pt}%
\definecolor{currentstroke}{rgb}{0.800000,0.800000,0.800000}%
\pgfsetstrokecolor{currentstroke}%
\pgfsetdash{}{0pt}%
\pgfpathmoveto{\pgfqpoint{8.856471in}{3.197368in}}%
\pgfpathlineto{\pgfqpoint{8.856471in}{4.176316in}}%
\pgfusepath{stroke}%
\end{pgfscope}%
\begin{pgfscope}%
\pgfpathrectangle{\pgfqpoint{7.105882in}{3.197368in}}{\pgfqpoint{4.376471in}{0.978947in}} %
\pgfusepath{clip}%
\pgfsetroundcap%
\pgfsetroundjoin%
\pgfsetlinewidth{1.003750pt}%
\definecolor{currentstroke}{rgb}{0.800000,0.800000,0.800000}%
\pgfsetstrokecolor{currentstroke}%
\pgfsetdash{}{0pt}%
\pgfpathmoveto{\pgfqpoint{9.731765in}{3.197368in}}%
\pgfpathlineto{\pgfqpoint{9.731765in}{4.176316in}}%
\pgfusepath{stroke}%
\end{pgfscope}%
\begin{pgfscope}%
\pgfpathrectangle{\pgfqpoint{7.105882in}{3.197368in}}{\pgfqpoint{4.376471in}{0.978947in}} %
\pgfusepath{clip}%
\pgfsetroundcap%
\pgfsetroundjoin%
\pgfsetlinewidth{1.003750pt}%
\definecolor{currentstroke}{rgb}{0.800000,0.800000,0.800000}%
\pgfsetstrokecolor{currentstroke}%
\pgfsetdash{}{0pt}%
\pgfpathmoveto{\pgfqpoint{10.607059in}{3.197368in}}%
\pgfpathlineto{\pgfqpoint{10.607059in}{4.176316in}}%
\pgfusepath{stroke}%
\end{pgfscope}%
\begin{pgfscope}%
\pgfpathrectangle{\pgfqpoint{7.105882in}{3.197368in}}{\pgfqpoint{4.376471in}{0.978947in}} %
\pgfusepath{clip}%
\pgfsetroundcap%
\pgfsetroundjoin%
\pgfsetlinewidth{1.003750pt}%
\definecolor{currentstroke}{rgb}{0.800000,0.800000,0.800000}%
\pgfsetstrokecolor{currentstroke}%
\pgfsetdash{}{0pt}%
\pgfpathmoveto{\pgfqpoint{11.482353in}{3.197368in}}%
\pgfpathlineto{\pgfqpoint{11.482353in}{4.176316in}}%
\pgfusepath{stroke}%
\end{pgfscope}%
\begin{pgfscope}%
\pgfpathrectangle{\pgfqpoint{7.105882in}{3.197368in}}{\pgfqpoint{4.376471in}{0.978947in}} %
\pgfusepath{clip}%
\pgfsetroundcap%
\pgfsetroundjoin%
\pgfsetlinewidth{1.003750pt}%
\definecolor{currentstroke}{rgb}{0.800000,0.800000,0.800000}%
\pgfsetstrokecolor{currentstroke}%
\pgfsetdash{}{0pt}%
\pgfpathmoveto{\pgfqpoint{7.105882in}{3.360526in}}%
\pgfpathlineto{\pgfqpoint{11.482353in}{3.360526in}}%
\pgfusepath{stroke}%
\end{pgfscope}%
\begin{pgfscope}%
\definecolor{textcolor}{rgb}{0.150000,0.150000,0.150000}%
\pgfsetstrokecolor{textcolor}%
\pgfsetfillcolor{textcolor}%
\pgftext[x=7.008660in,y=3.360526in,right,]{\color{textcolor}\sffamily\fontsize{10.000000}{12.000000}\selectfont \(\displaystyle -1\)}%
\end{pgfscope}%
\begin{pgfscope}%
\pgfpathrectangle{\pgfqpoint{7.105882in}{3.197368in}}{\pgfqpoint{4.376471in}{0.978947in}} %
\pgfusepath{clip}%
\pgfsetroundcap%
\pgfsetroundjoin%
\pgfsetlinewidth{1.003750pt}%
\definecolor{currentstroke}{rgb}{0.800000,0.800000,0.800000}%
\pgfsetstrokecolor{currentstroke}%
\pgfsetdash{}{0pt}%
\pgfpathmoveto{\pgfqpoint{7.105882in}{3.564474in}}%
\pgfpathlineto{\pgfqpoint{11.482353in}{3.564474in}}%
\pgfusepath{stroke}%
\end{pgfscope}%
\begin{pgfscope}%
\definecolor{textcolor}{rgb}{0.150000,0.150000,0.150000}%
\pgfsetstrokecolor{textcolor}%
\pgfsetfillcolor{textcolor}%
\pgftext[x=7.008660in,y=3.564474in,right,]{\color{textcolor}\sffamily\fontsize{10.000000}{12.000000}\selectfont \(\displaystyle 0\)}%
\end{pgfscope}%
\begin{pgfscope}%
\pgfpathrectangle{\pgfqpoint{7.105882in}{3.197368in}}{\pgfqpoint{4.376471in}{0.978947in}} %
\pgfusepath{clip}%
\pgfsetroundcap%
\pgfsetroundjoin%
\pgfsetlinewidth{1.003750pt}%
\definecolor{currentstroke}{rgb}{0.800000,0.800000,0.800000}%
\pgfsetstrokecolor{currentstroke}%
\pgfsetdash{}{0pt}%
\pgfpathmoveto{\pgfqpoint{7.105882in}{3.768421in}}%
\pgfpathlineto{\pgfqpoint{11.482353in}{3.768421in}}%
\pgfusepath{stroke}%
\end{pgfscope}%
\begin{pgfscope}%
\definecolor{textcolor}{rgb}{0.150000,0.150000,0.150000}%
\pgfsetstrokecolor{textcolor}%
\pgfsetfillcolor{textcolor}%
\pgftext[x=7.008660in,y=3.768421in,right,]{\color{textcolor}\sffamily\fontsize{10.000000}{12.000000}\selectfont \(\displaystyle 1\)}%
\end{pgfscope}%
\begin{pgfscope}%
\pgfpathrectangle{\pgfqpoint{7.105882in}{3.197368in}}{\pgfqpoint{4.376471in}{0.978947in}} %
\pgfusepath{clip}%
\pgfsetroundcap%
\pgfsetroundjoin%
\pgfsetlinewidth{1.003750pt}%
\definecolor{currentstroke}{rgb}{0.800000,0.800000,0.800000}%
\pgfsetstrokecolor{currentstroke}%
\pgfsetdash{}{0pt}%
\pgfpathmoveto{\pgfqpoint{7.105882in}{3.972368in}}%
\pgfpathlineto{\pgfqpoint{11.482353in}{3.972368in}}%
\pgfusepath{stroke}%
\end{pgfscope}%
\begin{pgfscope}%
\definecolor{textcolor}{rgb}{0.150000,0.150000,0.150000}%
\pgfsetstrokecolor{textcolor}%
\pgfsetfillcolor{textcolor}%
\pgftext[x=7.008660in,y=3.972368in,right,]{\color{textcolor}\sffamily\fontsize{10.000000}{12.000000}\selectfont \(\displaystyle 2\)}%
\end{pgfscope}%
\begin{pgfscope}%
\pgfpathrectangle{\pgfqpoint{7.105882in}{3.197368in}}{\pgfqpoint{4.376471in}{0.978947in}} %
\pgfusepath{clip}%
\pgfsetroundcap%
\pgfsetroundjoin%
\pgfsetlinewidth{1.003750pt}%
\definecolor{currentstroke}{rgb}{0.800000,0.800000,0.800000}%
\pgfsetstrokecolor{currentstroke}%
\pgfsetdash{}{0pt}%
\pgfpathmoveto{\pgfqpoint{7.105882in}{4.176316in}}%
\pgfpathlineto{\pgfqpoint{11.482353in}{4.176316in}}%
\pgfusepath{stroke}%
\end{pgfscope}%
\begin{pgfscope}%
\definecolor{textcolor}{rgb}{0.150000,0.150000,0.150000}%
\pgfsetstrokecolor{textcolor}%
\pgfsetfillcolor{textcolor}%
\pgftext[x=7.008660in,y=4.176316in,right,]{\color{textcolor}\sffamily\fontsize{10.000000}{12.000000}\selectfont \(\displaystyle 3\)}%
\end{pgfscope}%
\begin{pgfscope}%
\pgfpathrectangle{\pgfqpoint{7.105882in}{3.197368in}}{\pgfqpoint{4.376471in}{0.978947in}} %
\pgfusepath{clip}%
\pgfsetbuttcap%
\pgfsetroundjoin%
\definecolor{currentfill}{rgb}{1.000000,0.000000,0.000000}%
\pgfsetfillcolor{currentfill}%
\pgfsetlinewidth{2.007500pt}%
\definecolor{currentstroke}{rgb}{1.000000,0.000000,0.000000}%
\pgfsetstrokecolor{currentstroke}%
\pgfsetdash{}{0pt}%
\pgfpathmoveto{\pgfqpoint{9.871613in}{3.754944in}}%
\pgfpathlineto{\pgfqpoint{9.933726in}{3.754944in}}%
\pgfpathmoveto{\pgfqpoint{9.902669in}{3.723888in}}%
\pgfpathlineto{\pgfqpoint{9.902669in}{3.786001in}}%
\pgfusepath{stroke,fill}%
\end{pgfscope}%
\begin{pgfscope}%
\pgfpathrectangle{\pgfqpoint{7.105882in}{3.197368in}}{\pgfqpoint{4.376471in}{0.978947in}} %
\pgfusepath{clip}%
\pgfsetbuttcap%
\pgfsetroundjoin%
\definecolor{currentfill}{rgb}{1.000000,0.000000,0.000000}%
\pgfsetfillcolor{currentfill}%
\pgfsetlinewidth{2.007500pt}%
\definecolor{currentstroke}{rgb}{1.000000,0.000000,0.000000}%
\pgfsetstrokecolor{currentstroke}%
\pgfsetdash{}{0pt}%
\pgfpathmoveto{\pgfqpoint{10.454124in}{3.437732in}}%
\pgfpathlineto{\pgfqpoint{10.516237in}{3.437732in}}%
\pgfpathmoveto{\pgfqpoint{10.485181in}{3.406676in}}%
\pgfpathlineto{\pgfqpoint{10.485181in}{3.468789in}}%
\pgfusepath{stroke,fill}%
\end{pgfscope}%
\begin{pgfscope}%
\pgfpathrectangle{\pgfqpoint{7.105882in}{3.197368in}}{\pgfqpoint{4.376471in}{0.978947in}} %
\pgfusepath{clip}%
\pgfsetbuttcap%
\pgfsetroundjoin%
\definecolor{currentfill}{rgb}{1.000000,0.000000,0.000000}%
\pgfsetfillcolor{currentfill}%
\pgfsetlinewidth{2.007500pt}%
\definecolor{currentstroke}{rgb}{1.000000,0.000000,0.000000}%
\pgfsetstrokecolor{currentstroke}%
\pgfsetdash{}{0pt}%
\pgfpathmoveto{\pgfqpoint{10.060501in}{3.808472in}}%
\pgfpathlineto{\pgfqpoint{10.122614in}{3.808472in}}%
\pgfpathmoveto{\pgfqpoint{10.091557in}{3.777416in}}%
\pgfpathlineto{\pgfqpoint{10.091557in}{3.839529in}}%
\pgfusepath{stroke,fill}%
\end{pgfscope}%
\begin{pgfscope}%
\pgfpathrectangle{\pgfqpoint{7.105882in}{3.197368in}}{\pgfqpoint{4.376471in}{0.978947in}} %
\pgfusepath{clip}%
\pgfsetbuttcap%
\pgfsetroundjoin%
\definecolor{currentfill}{rgb}{1.000000,0.000000,0.000000}%
\pgfsetfillcolor{currentfill}%
\pgfsetlinewidth{2.007500pt}%
\definecolor{currentstroke}{rgb}{1.000000,0.000000,0.000000}%
\pgfsetstrokecolor{currentstroke}%
\pgfsetdash{}{0pt}%
\pgfpathmoveto{\pgfqpoint{9.857852in}{3.697560in}}%
\pgfpathlineto{\pgfqpoint{9.919965in}{3.697560in}}%
\pgfpathmoveto{\pgfqpoint{9.888909in}{3.666503in}}%
\pgfpathlineto{\pgfqpoint{9.888909in}{3.728616in}}%
\pgfusepath{stroke,fill}%
\end{pgfscope}%
\begin{pgfscope}%
\pgfpathrectangle{\pgfqpoint{7.105882in}{3.197368in}}{\pgfqpoint{4.376471in}{0.978947in}} %
\pgfusepath{clip}%
\pgfsetbuttcap%
\pgfsetroundjoin%
\definecolor{currentfill}{rgb}{1.000000,0.000000,0.000000}%
\pgfsetfillcolor{currentfill}%
\pgfsetlinewidth{2.007500pt}%
\definecolor{currentstroke}{rgb}{1.000000,0.000000,0.000000}%
\pgfsetstrokecolor{currentstroke}%
\pgfsetdash{}{0pt}%
\pgfpathmoveto{\pgfqpoint{9.433410in}{3.371166in}}%
\pgfpathlineto{\pgfqpoint{9.495523in}{3.371166in}}%
\pgfpathmoveto{\pgfqpoint{9.464467in}{3.340109in}}%
\pgfpathlineto{\pgfqpoint{9.464467in}{3.402222in}}%
\pgfusepath{stroke,fill}%
\end{pgfscope}%
\begin{pgfscope}%
\pgfpathrectangle{\pgfqpoint{7.105882in}{3.197368in}}{\pgfqpoint{4.376471in}{0.978947in}} %
\pgfusepath{clip}%
\pgfsetbuttcap%
\pgfsetroundjoin%
\definecolor{currentfill}{rgb}{1.000000,0.000000,0.000000}%
\pgfsetfillcolor{currentfill}%
\pgfsetlinewidth{2.007500pt}%
\definecolor{currentstroke}{rgb}{1.000000,0.000000,0.000000}%
\pgfsetstrokecolor{currentstroke}%
\pgfsetdash{}{0pt}%
\pgfpathmoveto{\pgfqpoint{10.211509in}{3.640630in}}%
\pgfpathlineto{\pgfqpoint{10.273622in}{3.640630in}}%
\pgfpathmoveto{\pgfqpoint{10.242566in}{3.609573in}}%
\pgfpathlineto{\pgfqpoint{10.242566in}{3.671686in}}%
\pgfusepath{stroke,fill}%
\end{pgfscope}%
\begin{pgfscope}%
\pgfpathrectangle{\pgfqpoint{7.105882in}{3.197368in}}{\pgfqpoint{4.376471in}{0.978947in}} %
\pgfusepath{clip}%
\pgfsetbuttcap%
\pgfsetroundjoin%
\definecolor{currentfill}{rgb}{1.000000,0.000000,0.000000}%
\pgfsetfillcolor{currentfill}%
\pgfsetlinewidth{2.007500pt}%
\definecolor{currentstroke}{rgb}{1.000000,0.000000,0.000000}%
\pgfsetstrokecolor{currentstroke}%
\pgfsetdash{}{0pt}%
\pgfpathmoveto{\pgfqpoint{9.482190in}{3.408659in}}%
\pgfpathlineto{\pgfqpoint{9.544303in}{3.408659in}}%
\pgfpathmoveto{\pgfqpoint{9.513247in}{3.377603in}}%
\pgfpathlineto{\pgfqpoint{9.513247in}{3.439716in}}%
\pgfusepath{stroke,fill}%
\end{pgfscope}%
\begin{pgfscope}%
\pgfpathrectangle{\pgfqpoint{7.105882in}{3.197368in}}{\pgfqpoint{4.376471in}{0.978947in}} %
\pgfusepath{clip}%
\pgfsetbuttcap%
\pgfsetroundjoin%
\definecolor{currentfill}{rgb}{1.000000,0.000000,0.000000}%
\pgfsetfillcolor{currentfill}%
\pgfsetlinewidth{2.007500pt}%
\definecolor{currentstroke}{rgb}{1.000000,0.000000,0.000000}%
\pgfsetstrokecolor{currentstroke}%
\pgfsetdash{}{0pt}%
\pgfpathmoveto{\pgfqpoint{11.072375in}{4.048778in}}%
\pgfpathlineto{\pgfqpoint{11.134488in}{4.048778in}}%
\pgfpathmoveto{\pgfqpoint{11.103431in}{4.017721in}}%
\pgfpathlineto{\pgfqpoint{11.103431in}{4.079834in}}%
\pgfusepath{stroke,fill}%
\end{pgfscope}%
\begin{pgfscope}%
\pgfpathrectangle{\pgfqpoint{7.105882in}{3.197368in}}{\pgfqpoint{4.376471in}{0.978947in}} %
\pgfusepath{clip}%
\pgfsetbuttcap%
\pgfsetroundjoin%
\definecolor{currentfill}{rgb}{1.000000,0.000000,0.000000}%
\pgfsetfillcolor{currentfill}%
\pgfsetlinewidth{2.007500pt}%
\definecolor{currentstroke}{rgb}{1.000000,0.000000,0.000000}%
\pgfsetstrokecolor{currentstroke}%
\pgfsetdash{}{0pt}%
\pgfpathmoveto{\pgfqpoint{11.324073in}{3.949180in}}%
\pgfpathlineto{\pgfqpoint{11.386186in}{3.949180in}}%
\pgfpathmoveto{\pgfqpoint{11.355130in}{3.918124in}}%
\pgfpathlineto{\pgfqpoint{11.355130in}{3.980237in}}%
\pgfusepath{stroke,fill}%
\end{pgfscope}%
\begin{pgfscope}%
\pgfpathrectangle{\pgfqpoint{7.105882in}{3.197368in}}{\pgfqpoint{4.376471in}{0.978947in}} %
\pgfusepath{clip}%
\pgfsetbuttcap%
\pgfsetroundjoin%
\definecolor{currentfill}{rgb}{1.000000,0.000000,0.000000}%
\pgfsetfillcolor{currentfill}%
\pgfsetlinewidth{2.007500pt}%
\definecolor{currentstroke}{rgb}{1.000000,0.000000,0.000000}%
\pgfsetstrokecolor{currentstroke}%
\pgfsetdash{}{0pt}%
\pgfpathmoveto{\pgfqpoint{9.292616in}{3.446577in}}%
\pgfpathlineto{\pgfqpoint{9.354729in}{3.446577in}}%
\pgfpathmoveto{\pgfqpoint{9.323673in}{3.415520in}}%
\pgfpathlineto{\pgfqpoint{9.323673in}{3.477633in}}%
\pgfusepath{stroke,fill}%
\end{pgfscope}%
\begin{pgfscope}%
\pgfpathrectangle{\pgfqpoint{7.105882in}{3.197368in}}{\pgfqpoint{4.376471in}{0.978947in}} %
\pgfusepath{clip}%
\pgfsetbuttcap%
\pgfsetroundjoin%
\definecolor{currentfill}{rgb}{1.000000,0.000000,0.000000}%
\pgfsetfillcolor{currentfill}%
\pgfsetlinewidth{2.007500pt}%
\definecolor{currentstroke}{rgb}{1.000000,0.000000,0.000000}%
\pgfsetstrokecolor{currentstroke}%
\pgfsetdash{}{0pt}%
\pgfpathmoveto{\pgfqpoint{10.722089in}{3.596752in}}%
\pgfpathlineto{\pgfqpoint{10.784202in}{3.596752in}}%
\pgfpathmoveto{\pgfqpoint{10.753146in}{3.565696in}}%
\pgfpathlineto{\pgfqpoint{10.753146in}{3.627809in}}%
\pgfusepath{stroke,fill}%
\end{pgfscope}%
\begin{pgfscope}%
\pgfpathrectangle{\pgfqpoint{7.105882in}{3.197368in}}{\pgfqpoint{4.376471in}{0.978947in}} %
\pgfusepath{clip}%
\pgfsetbuttcap%
\pgfsetroundjoin%
\definecolor{currentfill}{rgb}{1.000000,0.000000,0.000000}%
\pgfsetfillcolor{currentfill}%
\pgfsetlinewidth{2.007500pt}%
\definecolor{currentstroke}{rgb}{1.000000,0.000000,0.000000}%
\pgfsetstrokecolor{currentstroke}%
\pgfsetdash{}{0pt}%
\pgfpathmoveto{\pgfqpoint{9.801874in}{3.636847in}}%
\pgfpathlineto{\pgfqpoint{9.863987in}{3.636847in}}%
\pgfpathmoveto{\pgfqpoint{9.832931in}{3.605791in}}%
\pgfpathlineto{\pgfqpoint{9.832931in}{3.667904in}}%
\pgfusepath{stroke,fill}%
\end{pgfscope}%
\begin{pgfscope}%
\pgfpathrectangle{\pgfqpoint{7.105882in}{3.197368in}}{\pgfqpoint{4.376471in}{0.978947in}} %
\pgfusepath{clip}%
\pgfsetbuttcap%
\pgfsetroundjoin%
\definecolor{currentfill}{rgb}{1.000000,0.000000,0.000000}%
\pgfsetfillcolor{currentfill}%
\pgfsetlinewidth{2.007500pt}%
\definecolor{currentstroke}{rgb}{1.000000,0.000000,0.000000}%
\pgfsetstrokecolor{currentstroke}%
\pgfsetdash{}{0pt}%
\pgfpathmoveto{\pgfqpoint{9.938944in}{3.764403in}}%
\pgfpathlineto{\pgfqpoint{10.001057in}{3.764403in}}%
\pgfpathmoveto{\pgfqpoint{9.970001in}{3.733347in}}%
\pgfpathlineto{\pgfqpoint{9.970001in}{3.795460in}}%
\pgfusepath{stroke,fill}%
\end{pgfscope}%
\begin{pgfscope}%
\pgfpathrectangle{\pgfqpoint{7.105882in}{3.197368in}}{\pgfqpoint{4.376471in}{0.978947in}} %
\pgfusepath{clip}%
\pgfsetbuttcap%
\pgfsetroundjoin%
\definecolor{currentfill}{rgb}{1.000000,0.000000,0.000000}%
\pgfsetfillcolor{currentfill}%
\pgfsetlinewidth{2.007500pt}%
\definecolor{currentstroke}{rgb}{1.000000,0.000000,0.000000}%
\pgfsetstrokecolor{currentstroke}%
\pgfsetdash{}{0pt}%
\pgfpathmoveto{\pgfqpoint{11.190797in}{4.024567in}}%
\pgfpathlineto{\pgfqpoint{11.252910in}{4.024567in}}%
\pgfpathmoveto{\pgfqpoint{11.221854in}{3.993511in}}%
\pgfpathlineto{\pgfqpoint{11.221854in}{4.055624in}}%
\pgfusepath{stroke,fill}%
\end{pgfscope}%
\begin{pgfscope}%
\pgfpathrectangle{\pgfqpoint{7.105882in}{3.197368in}}{\pgfqpoint{4.376471in}{0.978947in}} %
\pgfusepath{clip}%
\pgfsetbuttcap%
\pgfsetroundjoin%
\definecolor{currentfill}{rgb}{1.000000,0.000000,0.000000}%
\pgfsetfillcolor{currentfill}%
\pgfsetlinewidth{2.007500pt}%
\definecolor{currentstroke}{rgb}{1.000000,0.000000,0.000000}%
\pgfsetstrokecolor{currentstroke}%
\pgfsetdash{}{0pt}%
\pgfpathmoveto{\pgfqpoint{8.198830in}{3.737178in}}%
\pgfpathlineto{\pgfqpoint{8.260943in}{3.737178in}}%
\pgfpathmoveto{\pgfqpoint{8.229886in}{3.706121in}}%
\pgfpathlineto{\pgfqpoint{8.229886in}{3.768234in}}%
\pgfusepath{stroke,fill}%
\end{pgfscope}%
\begin{pgfscope}%
\pgfpathrectangle{\pgfqpoint{7.105882in}{3.197368in}}{\pgfqpoint{4.376471in}{0.978947in}} %
\pgfusepath{clip}%
\pgfsetbuttcap%
\pgfsetroundjoin%
\definecolor{currentfill}{rgb}{1.000000,0.000000,0.000000}%
\pgfsetfillcolor{currentfill}%
\pgfsetlinewidth{2.007500pt}%
\definecolor{currentstroke}{rgb}{1.000000,0.000000,0.000000}%
\pgfsetstrokecolor{currentstroke}%
\pgfsetdash{}{0pt}%
\pgfpathmoveto{\pgfqpoint{8.255175in}{3.679544in}}%
\pgfpathlineto{\pgfqpoint{8.317288in}{3.679544in}}%
\pgfpathmoveto{\pgfqpoint{8.286232in}{3.648488in}}%
\pgfpathlineto{\pgfqpoint{8.286232in}{3.710601in}}%
\pgfusepath{stroke,fill}%
\end{pgfscope}%
\begin{pgfscope}%
\pgfpathrectangle{\pgfqpoint{7.105882in}{3.197368in}}{\pgfqpoint{4.376471in}{0.978947in}} %
\pgfusepath{clip}%
\pgfsetbuttcap%
\pgfsetroundjoin%
\definecolor{currentfill}{rgb}{1.000000,0.000000,0.000000}%
\pgfsetfillcolor{currentfill}%
\pgfsetlinewidth{2.007500pt}%
\definecolor{currentstroke}{rgb}{1.000000,0.000000,0.000000}%
\pgfsetstrokecolor{currentstroke}%
\pgfsetdash{}{0pt}%
\pgfpathmoveto{\pgfqpoint{8.020908in}{3.966824in}}%
\pgfpathlineto{\pgfqpoint{8.083021in}{3.966824in}}%
\pgfpathmoveto{\pgfqpoint{8.051965in}{3.935768in}}%
\pgfpathlineto{\pgfqpoint{8.051965in}{3.997881in}}%
\pgfusepath{stroke,fill}%
\end{pgfscope}%
\begin{pgfscope}%
\pgfpathrectangle{\pgfqpoint{7.105882in}{3.197368in}}{\pgfqpoint{4.376471in}{0.978947in}} %
\pgfusepath{clip}%
\pgfsetbuttcap%
\pgfsetroundjoin%
\definecolor{currentfill}{rgb}{1.000000,0.000000,0.000000}%
\pgfsetfillcolor{currentfill}%
\pgfsetlinewidth{2.007500pt}%
\definecolor{currentstroke}{rgb}{1.000000,0.000000,0.000000}%
\pgfsetstrokecolor{currentstroke}%
\pgfsetdash{}{0pt}%
\pgfpathmoveto{\pgfqpoint{10.865269in}{3.812420in}}%
\pgfpathlineto{\pgfqpoint{10.927382in}{3.812420in}}%
\pgfpathmoveto{\pgfqpoint{10.896325in}{3.781364in}}%
\pgfpathlineto{\pgfqpoint{10.896325in}{3.843477in}}%
\pgfusepath{stroke,fill}%
\end{pgfscope}%
\begin{pgfscope}%
\pgfpathrectangle{\pgfqpoint{7.105882in}{3.197368in}}{\pgfqpoint{4.376471in}{0.978947in}} %
\pgfusepath{clip}%
\pgfsetbuttcap%
\pgfsetroundjoin%
\definecolor{currentfill}{rgb}{1.000000,0.000000,0.000000}%
\pgfsetfillcolor{currentfill}%
\pgfsetlinewidth{2.007500pt}%
\definecolor{currentstroke}{rgb}{1.000000,0.000000,0.000000}%
\pgfsetstrokecolor{currentstroke}%
\pgfsetdash{}{0pt}%
\pgfpathmoveto{\pgfqpoint{10.674584in}{3.534838in}}%
\pgfpathlineto{\pgfqpoint{10.736697in}{3.534838in}}%
\pgfpathmoveto{\pgfqpoint{10.705641in}{3.503782in}}%
\pgfpathlineto{\pgfqpoint{10.705641in}{3.565895in}}%
\pgfusepath{stroke,fill}%
\end{pgfscope}%
\begin{pgfscope}%
\pgfpathrectangle{\pgfqpoint{7.105882in}{3.197368in}}{\pgfqpoint{4.376471in}{0.978947in}} %
\pgfusepath{clip}%
\pgfsetbuttcap%
\pgfsetroundjoin%
\definecolor{currentfill}{rgb}{1.000000,0.000000,0.000000}%
\pgfsetfillcolor{currentfill}%
\pgfsetlinewidth{2.007500pt}%
\definecolor{currentstroke}{rgb}{1.000000,0.000000,0.000000}%
\pgfsetstrokecolor{currentstroke}%
\pgfsetdash{}{0pt}%
\pgfpathmoveto{\pgfqpoint{10.996186in}{3.942201in}}%
\pgfpathlineto{\pgfqpoint{11.058299in}{3.942201in}}%
\pgfpathmoveto{\pgfqpoint{11.027243in}{3.911145in}}%
\pgfpathlineto{\pgfqpoint{11.027243in}{3.973258in}}%
\pgfusepath{stroke,fill}%
\end{pgfscope}%
\begin{pgfscope}%
\pgfpathrectangle{\pgfqpoint{7.105882in}{3.197368in}}{\pgfqpoint{4.376471in}{0.978947in}} %
\pgfusepath{clip}%
\pgfsetbuttcap%
\pgfsetroundjoin%
\definecolor{currentfill}{rgb}{1.000000,0.000000,0.000000}%
\pgfsetfillcolor{currentfill}%
\pgfsetlinewidth{2.007500pt}%
\definecolor{currentstroke}{rgb}{1.000000,0.000000,0.000000}%
\pgfsetstrokecolor{currentstroke}%
\pgfsetdash{}{0pt}%
\pgfpathmoveto{\pgfqpoint{11.376435in}{3.873438in}}%
\pgfpathlineto{\pgfqpoint{11.438548in}{3.873438in}}%
\pgfpathmoveto{\pgfqpoint{11.407492in}{3.842382in}}%
\pgfpathlineto{\pgfqpoint{11.407492in}{3.904495in}}%
\pgfusepath{stroke,fill}%
\end{pgfscope}%
\begin{pgfscope}%
\pgfpathrectangle{\pgfqpoint{7.105882in}{3.197368in}}{\pgfqpoint{4.376471in}{0.978947in}} %
\pgfusepath{clip}%
\pgfsetbuttcap%
\pgfsetroundjoin%
\definecolor{currentfill}{rgb}{1.000000,0.000000,0.000000}%
\pgfsetfillcolor{currentfill}%
\pgfsetlinewidth{2.007500pt}%
\definecolor{currentstroke}{rgb}{1.000000,0.000000,0.000000}%
\pgfsetstrokecolor{currentstroke}%
\pgfsetdash{}{0pt}%
\pgfpathmoveto{\pgfqpoint{10.748115in}{3.690005in}}%
\pgfpathlineto{\pgfqpoint{10.810228in}{3.690005in}}%
\pgfpathmoveto{\pgfqpoint{10.779172in}{3.658949in}}%
\pgfpathlineto{\pgfqpoint{10.779172in}{3.721062in}}%
\pgfusepath{stroke,fill}%
\end{pgfscope}%
\begin{pgfscope}%
\pgfpathrectangle{\pgfqpoint{7.105882in}{3.197368in}}{\pgfqpoint{4.376471in}{0.978947in}} %
\pgfusepath{clip}%
\pgfsetbuttcap%
\pgfsetroundjoin%
\definecolor{currentfill}{rgb}{1.000000,0.000000,0.000000}%
\pgfsetfillcolor{currentfill}%
\pgfsetlinewidth{2.007500pt}%
\definecolor{currentstroke}{rgb}{1.000000,0.000000,0.000000}%
\pgfsetstrokecolor{currentstroke}%
\pgfsetdash{}{0pt}%
\pgfpathmoveto{\pgfqpoint{9.565841in}{3.414465in}}%
\pgfpathlineto{\pgfqpoint{9.627954in}{3.414465in}}%
\pgfpathmoveto{\pgfqpoint{9.596897in}{3.383409in}}%
\pgfpathlineto{\pgfqpoint{9.596897in}{3.445522in}}%
\pgfusepath{stroke,fill}%
\end{pgfscope}%
\begin{pgfscope}%
\pgfpathrectangle{\pgfqpoint{7.105882in}{3.197368in}}{\pgfqpoint{4.376471in}{0.978947in}} %
\pgfusepath{clip}%
\pgfsetbuttcap%
\pgfsetroundjoin%
\definecolor{currentfill}{rgb}{1.000000,0.000000,0.000000}%
\pgfsetfillcolor{currentfill}%
\pgfsetlinewidth{2.007500pt}%
\definecolor{currentstroke}{rgb}{1.000000,0.000000,0.000000}%
\pgfsetstrokecolor{currentstroke}%
\pgfsetdash{}{0pt}%
\pgfpathmoveto{\pgfqpoint{10.682890in}{3.556875in}}%
\pgfpathlineto{\pgfqpoint{10.745003in}{3.556875in}}%
\pgfpathmoveto{\pgfqpoint{10.713947in}{3.525819in}}%
\pgfpathlineto{\pgfqpoint{10.713947in}{3.587932in}}%
\pgfusepath{stroke,fill}%
\end{pgfscope}%
\begin{pgfscope}%
\pgfpathrectangle{\pgfqpoint{7.105882in}{3.197368in}}{\pgfqpoint{4.376471in}{0.978947in}} %
\pgfusepath{clip}%
\pgfsetbuttcap%
\pgfsetroundjoin%
\definecolor{currentfill}{rgb}{1.000000,0.000000,0.000000}%
\pgfsetfillcolor{currentfill}%
\pgfsetlinewidth{2.007500pt}%
\definecolor{currentstroke}{rgb}{1.000000,0.000000,0.000000}%
\pgfsetstrokecolor{currentstroke}%
\pgfsetdash{}{0pt}%
\pgfpathmoveto{\pgfqpoint{8.364220in}{3.577613in}}%
\pgfpathlineto{\pgfqpoint{8.426333in}{3.577613in}}%
\pgfpathmoveto{\pgfqpoint{8.395276in}{3.546556in}}%
\pgfpathlineto{\pgfqpoint{8.395276in}{3.608669in}}%
\pgfusepath{stroke,fill}%
\end{pgfscope}%
\begin{pgfscope}%
\pgfpathrectangle{\pgfqpoint{7.105882in}{3.197368in}}{\pgfqpoint{4.376471in}{0.978947in}} %
\pgfusepath{clip}%
\pgfsetbuttcap%
\pgfsetroundjoin%
\definecolor{currentfill}{rgb}{1.000000,0.000000,0.000000}%
\pgfsetfillcolor{currentfill}%
\pgfsetlinewidth{2.007500pt}%
\definecolor{currentstroke}{rgb}{1.000000,0.000000,0.000000}%
\pgfsetstrokecolor{currentstroke}%
\pgfsetdash{}{0pt}%
\pgfpathmoveto{\pgfqpoint{10.190596in}{3.680897in}}%
\pgfpathlineto{\pgfqpoint{10.252709in}{3.680897in}}%
\pgfpathmoveto{\pgfqpoint{10.221653in}{3.649840in}}%
\pgfpathlineto{\pgfqpoint{10.221653in}{3.711953in}}%
\pgfusepath{stroke,fill}%
\end{pgfscope}%
\begin{pgfscope}%
\pgfpathrectangle{\pgfqpoint{7.105882in}{3.197368in}}{\pgfqpoint{4.376471in}{0.978947in}} %
\pgfusepath{clip}%
\pgfsetbuttcap%
\pgfsetroundjoin%
\definecolor{currentfill}{rgb}{1.000000,0.000000,0.000000}%
\pgfsetfillcolor{currentfill}%
\pgfsetlinewidth{2.007500pt}%
\definecolor{currentstroke}{rgb}{1.000000,0.000000,0.000000}%
\pgfsetstrokecolor{currentstroke}%
\pgfsetdash{}{0pt}%
\pgfpathmoveto{\pgfqpoint{8.452025in}{3.585903in}}%
\pgfpathlineto{\pgfqpoint{8.514138in}{3.585903in}}%
\pgfpathmoveto{\pgfqpoint{8.483082in}{3.554846in}}%
\pgfpathlineto{\pgfqpoint{8.483082in}{3.616959in}}%
\pgfusepath{stroke,fill}%
\end{pgfscope}%
\begin{pgfscope}%
\pgfpathrectangle{\pgfqpoint{7.105882in}{3.197368in}}{\pgfqpoint{4.376471in}{0.978947in}} %
\pgfusepath{clip}%
\pgfsetbuttcap%
\pgfsetroundjoin%
\definecolor{currentfill}{rgb}{1.000000,0.000000,0.000000}%
\pgfsetfillcolor{currentfill}%
\pgfsetlinewidth{2.007500pt}%
\definecolor{currentstroke}{rgb}{1.000000,0.000000,0.000000}%
\pgfsetstrokecolor{currentstroke}%
\pgfsetdash{}{0pt}%
\pgfpathmoveto{\pgfqpoint{11.257573in}{3.986096in}}%
\pgfpathlineto{\pgfqpoint{11.319686in}{3.986096in}}%
\pgfpathmoveto{\pgfqpoint{11.288629in}{3.955040in}}%
\pgfpathlineto{\pgfqpoint{11.288629in}{4.017153in}}%
\pgfusepath{stroke,fill}%
\end{pgfscope}%
\begin{pgfscope}%
\pgfpathrectangle{\pgfqpoint{7.105882in}{3.197368in}}{\pgfqpoint{4.376471in}{0.978947in}} %
\pgfusepath{clip}%
\pgfsetbuttcap%
\pgfsetroundjoin%
\definecolor{currentfill}{rgb}{1.000000,0.000000,0.000000}%
\pgfsetfillcolor{currentfill}%
\pgfsetlinewidth{2.007500pt}%
\definecolor{currentstroke}{rgb}{1.000000,0.000000,0.000000}%
\pgfsetstrokecolor{currentstroke}%
\pgfsetdash{}{0pt}%
\pgfpathmoveto{\pgfqpoint{9.777203in}{3.633299in}}%
\pgfpathlineto{\pgfqpoint{9.839316in}{3.633299in}}%
\pgfpathmoveto{\pgfqpoint{9.808260in}{3.602242in}}%
\pgfpathlineto{\pgfqpoint{9.808260in}{3.664355in}}%
\pgfusepath{stroke,fill}%
\end{pgfscope}%
\begin{pgfscope}%
\pgfpathrectangle{\pgfqpoint{7.105882in}{3.197368in}}{\pgfqpoint{4.376471in}{0.978947in}} %
\pgfusepath{clip}%
\pgfsetbuttcap%
\pgfsetroundjoin%
\definecolor{currentfill}{rgb}{1.000000,0.000000,0.000000}%
\pgfsetfillcolor{currentfill}%
\pgfsetlinewidth{2.007500pt}%
\definecolor{currentstroke}{rgb}{1.000000,0.000000,0.000000}%
\pgfsetstrokecolor{currentstroke}%
\pgfsetdash{}{0pt}%
\pgfpathmoveto{\pgfqpoint{9.401925in}{3.382181in}}%
\pgfpathlineto{\pgfqpoint{9.464038in}{3.382181in}}%
\pgfpathmoveto{\pgfqpoint{9.432981in}{3.351125in}}%
\pgfpathlineto{\pgfqpoint{9.432981in}{3.413238in}}%
\pgfusepath{stroke,fill}%
\end{pgfscope}%
\begin{pgfscope}%
\pgfpathrectangle{\pgfqpoint{7.105882in}{3.197368in}}{\pgfqpoint{4.376471in}{0.978947in}} %
\pgfusepath{clip}%
\pgfsetbuttcap%
\pgfsetroundjoin%
\definecolor{currentfill}{rgb}{0.000000,0.000000,0.000000}%
\pgfsetfillcolor{currentfill}%
\pgfsetlinewidth{0.301125pt}%
\definecolor{currentstroke}{rgb}{0.000000,0.000000,0.000000}%
\pgfsetstrokecolor{currentstroke}%
\pgfsetdash{}{0pt}%
\pgfsys@defobject{currentmarker}{\pgfqpoint{-0.015528in}{-0.015528in}}{\pgfqpoint{0.015528in}{0.015528in}}{%
\pgfpathmoveto{\pgfqpoint{0.000000in}{-0.015528in}}%
\pgfpathcurveto{\pgfqpoint{0.004118in}{-0.015528in}}{\pgfqpoint{0.008068in}{-0.013892in}}{\pgfqpoint{0.010980in}{-0.010980in}}%
\pgfpathcurveto{\pgfqpoint{0.013892in}{-0.008068in}}{\pgfqpoint{0.015528in}{-0.004118in}}{\pgfqpoint{0.015528in}{0.000000in}}%
\pgfpathcurveto{\pgfqpoint{0.015528in}{0.004118in}}{\pgfqpoint{0.013892in}{0.008068in}}{\pgfqpoint{0.010980in}{0.010980in}}%
\pgfpathcurveto{\pgfqpoint{0.008068in}{0.013892in}}{\pgfqpoint{0.004118in}{0.015528in}}{\pgfqpoint{0.000000in}{0.015528in}}%
\pgfpathcurveto{\pgfqpoint{-0.004118in}{0.015528in}}{\pgfqpoint{-0.008068in}{0.013892in}}{\pgfqpoint{-0.010980in}{0.010980in}}%
\pgfpathcurveto{\pgfqpoint{-0.013892in}{0.008068in}}{\pgfqpoint{-0.015528in}{0.004118in}}{\pgfqpoint{-0.015528in}{0.000000in}}%
\pgfpathcurveto{\pgfqpoint{-0.015528in}{-0.004118in}}{\pgfqpoint{-0.013892in}{-0.008068in}}{\pgfqpoint{-0.010980in}{-0.010980in}}%
\pgfpathcurveto{\pgfqpoint{-0.008068in}{-0.013892in}}{\pgfqpoint{-0.004118in}{-0.015528in}}{\pgfqpoint{0.000000in}{-0.015528in}}%
\pgfpathclose%
\pgfusepath{stroke,fill}%
}%
\begin{pgfscope}%
\pgfsys@transformshift{7.981176in}{4.031231in}%
\pgfsys@useobject{currentmarker}{}%
\end{pgfscope}%
\begin{pgfscope}%
\pgfsys@transformshift{7.998770in}{3.937039in}%
\pgfsys@useobject{currentmarker}{}%
\end{pgfscope}%
\begin{pgfscope}%
\pgfsys@transformshift{8.016364in}{4.027813in}%
\pgfsys@useobject{currentmarker}{}%
\end{pgfscope}%
\begin{pgfscope}%
\pgfsys@transformshift{8.033958in}{4.046950in}%
\pgfsys@useobject{currentmarker}{}%
\end{pgfscope}%
\begin{pgfscope}%
\pgfsys@transformshift{8.051552in}{3.982165in}%
\pgfsys@useobject{currentmarker}{}%
\end{pgfscope}%
\begin{pgfscope}%
\pgfsys@transformshift{8.069146in}{3.978064in}%
\pgfsys@useobject{currentmarker}{}%
\end{pgfscope}%
\begin{pgfscope}%
\pgfsys@transformshift{8.086740in}{3.854307in}%
\pgfsys@useobject{currentmarker}{}%
\end{pgfscope}%
\begin{pgfscope}%
\pgfsys@transformshift{8.104333in}{3.853911in}%
\pgfsys@useobject{currentmarker}{}%
\end{pgfscope}%
\begin{pgfscope}%
\pgfsys@transformshift{8.121927in}{3.794584in}%
\pgfsys@useobject{currentmarker}{}%
\end{pgfscope}%
\begin{pgfscope}%
\pgfsys@transformshift{8.139521in}{3.799298in}%
\pgfsys@useobject{currentmarker}{}%
\end{pgfscope}%
\begin{pgfscope}%
\pgfsys@transformshift{8.157115in}{3.726597in}%
\pgfsys@useobject{currentmarker}{}%
\end{pgfscope}%
\begin{pgfscope}%
\pgfsys@transformshift{8.174709in}{3.607975in}%
\pgfsys@useobject{currentmarker}{}%
\end{pgfscope}%
\begin{pgfscope}%
\pgfsys@transformshift{8.192303in}{3.777720in}%
\pgfsys@useobject{currentmarker}{}%
\end{pgfscope}%
\begin{pgfscope}%
\pgfsys@transformshift{8.209897in}{3.695569in}%
\pgfsys@useobject{currentmarker}{}%
\end{pgfscope}%
\begin{pgfscope}%
\pgfsys@transformshift{8.227490in}{3.548673in}%
\pgfsys@useobject{currentmarker}{}%
\end{pgfscope}%
\begin{pgfscope}%
\pgfsys@transformshift{8.245084in}{3.742077in}%
\pgfsys@useobject{currentmarker}{}%
\end{pgfscope}%
\begin{pgfscope}%
\pgfsys@transformshift{8.262678in}{3.584074in}%
\pgfsys@useobject{currentmarker}{}%
\end{pgfscope}%
\begin{pgfscope}%
\pgfsys@transformshift{8.280272in}{3.665451in}%
\pgfsys@useobject{currentmarker}{}%
\end{pgfscope}%
\begin{pgfscope}%
\pgfsys@transformshift{8.297866in}{3.720008in}%
\pgfsys@useobject{currentmarker}{}%
\end{pgfscope}%
\begin{pgfscope}%
\pgfsys@transformshift{8.315460in}{3.646344in}%
\pgfsys@useobject{currentmarker}{}%
\end{pgfscope}%
\begin{pgfscope}%
\pgfsys@transformshift{8.333054in}{3.738995in}%
\pgfsys@useobject{currentmarker}{}%
\end{pgfscope}%
\begin{pgfscope}%
\pgfsys@transformshift{8.350647in}{3.488622in}%
\pgfsys@useobject{currentmarker}{}%
\end{pgfscope}%
\begin{pgfscope}%
\pgfsys@transformshift{8.368241in}{3.649446in}%
\pgfsys@useobject{currentmarker}{}%
\end{pgfscope}%
\begin{pgfscope}%
\pgfsys@transformshift{8.385835in}{3.534597in}%
\pgfsys@useobject{currentmarker}{}%
\end{pgfscope}%
\begin{pgfscope}%
\pgfsys@transformshift{8.403429in}{3.513756in}%
\pgfsys@useobject{currentmarker}{}%
\end{pgfscope}%
\begin{pgfscope}%
\pgfsys@transformshift{8.421023in}{3.543720in}%
\pgfsys@useobject{currentmarker}{}%
\end{pgfscope}%
\begin{pgfscope}%
\pgfsys@transformshift{8.438617in}{3.573174in}%
\pgfsys@useobject{currentmarker}{}%
\end{pgfscope}%
\begin{pgfscope}%
\pgfsys@transformshift{8.456210in}{3.614788in}%
\pgfsys@useobject{currentmarker}{}%
\end{pgfscope}%
\begin{pgfscope}%
\pgfsys@transformshift{8.473804in}{3.496181in}%
\pgfsys@useobject{currentmarker}{}%
\end{pgfscope}%
\begin{pgfscope}%
\pgfsys@transformshift{8.491398in}{3.714490in}%
\pgfsys@useobject{currentmarker}{}%
\end{pgfscope}%
\begin{pgfscope}%
\pgfsys@transformshift{8.508992in}{3.679310in}%
\pgfsys@useobject{currentmarker}{}%
\end{pgfscope}%
\begin{pgfscope}%
\pgfsys@transformshift{8.526586in}{3.485775in}%
\pgfsys@useobject{currentmarker}{}%
\end{pgfscope}%
\begin{pgfscope}%
\pgfsys@transformshift{8.544180in}{3.806047in}%
\pgfsys@useobject{currentmarker}{}%
\end{pgfscope}%
\begin{pgfscope}%
\pgfsys@transformshift{8.561774in}{3.860539in}%
\pgfsys@useobject{currentmarker}{}%
\end{pgfscope}%
\begin{pgfscope}%
\pgfsys@transformshift{8.579367in}{3.801221in}%
\pgfsys@useobject{currentmarker}{}%
\end{pgfscope}%
\begin{pgfscope}%
\pgfsys@transformshift{8.596961in}{3.677167in}%
\pgfsys@useobject{currentmarker}{}%
\end{pgfscope}%
\begin{pgfscope}%
\pgfsys@transformshift{8.614555in}{3.601314in}%
\pgfsys@useobject{currentmarker}{}%
\end{pgfscope}%
\begin{pgfscope}%
\pgfsys@transformshift{8.632149in}{3.833302in}%
\pgfsys@useobject{currentmarker}{}%
\end{pgfscope}%
\begin{pgfscope}%
\pgfsys@transformshift{8.649743in}{3.700014in}%
\pgfsys@useobject{currentmarker}{}%
\end{pgfscope}%
\begin{pgfscope}%
\pgfsys@transformshift{8.667337in}{3.881009in}%
\pgfsys@useobject{currentmarker}{}%
\end{pgfscope}%
\begin{pgfscope}%
\pgfsys@transformshift{8.684931in}{3.792483in}%
\pgfsys@useobject{currentmarker}{}%
\end{pgfscope}%
\begin{pgfscope}%
\pgfsys@transformshift{8.702524in}{3.885195in}%
\pgfsys@useobject{currentmarker}{}%
\end{pgfscope}%
\begin{pgfscope}%
\pgfsys@transformshift{8.720118in}{3.835577in}%
\pgfsys@useobject{currentmarker}{}%
\end{pgfscope}%
\begin{pgfscope}%
\pgfsys@transformshift{8.737712in}{3.884011in}%
\pgfsys@useobject{currentmarker}{}%
\end{pgfscope}%
\begin{pgfscope}%
\pgfsys@transformshift{8.755306in}{3.824663in}%
\pgfsys@useobject{currentmarker}{}%
\end{pgfscope}%
\begin{pgfscope}%
\pgfsys@transformshift{8.772900in}{4.016085in}%
\pgfsys@useobject{currentmarker}{}%
\end{pgfscope}%
\begin{pgfscope}%
\pgfsys@transformshift{8.790494in}{3.855894in}%
\pgfsys@useobject{currentmarker}{}%
\end{pgfscope}%
\begin{pgfscope}%
\pgfsys@transformshift{8.808087in}{3.891386in}%
\pgfsys@useobject{currentmarker}{}%
\end{pgfscope}%
\begin{pgfscope}%
\pgfsys@transformshift{8.825681in}{4.048210in}%
\pgfsys@useobject{currentmarker}{}%
\end{pgfscope}%
\begin{pgfscope}%
\pgfsys@transformshift{8.843275in}{3.722767in}%
\pgfsys@useobject{currentmarker}{}%
\end{pgfscope}%
\begin{pgfscope}%
\pgfsys@transformshift{8.860869in}{3.732831in}%
\pgfsys@useobject{currentmarker}{}%
\end{pgfscope}%
\begin{pgfscope}%
\pgfsys@transformshift{8.878463in}{3.961518in}%
\pgfsys@useobject{currentmarker}{}%
\end{pgfscope}%
\begin{pgfscope}%
\pgfsys@transformshift{8.896057in}{3.741367in}%
\pgfsys@useobject{currentmarker}{}%
\end{pgfscope}%
\begin{pgfscope}%
\pgfsys@transformshift{8.913651in}{4.055550in}%
\pgfsys@useobject{currentmarker}{}%
\end{pgfscope}%
\begin{pgfscope}%
\pgfsys@transformshift{8.931244in}{3.809556in}%
\pgfsys@useobject{currentmarker}{}%
\end{pgfscope}%
\begin{pgfscope}%
\pgfsys@transformshift{8.948838in}{3.767941in}%
\pgfsys@useobject{currentmarker}{}%
\end{pgfscope}%
\begin{pgfscope}%
\pgfsys@transformshift{8.966432in}{4.030761in}%
\pgfsys@useobject{currentmarker}{}%
\end{pgfscope}%
\begin{pgfscope}%
\pgfsys@transformshift{8.984026in}{3.974297in}%
\pgfsys@useobject{currentmarker}{}%
\end{pgfscope}%
\begin{pgfscope}%
\pgfsys@transformshift{9.001620in}{4.000639in}%
\pgfsys@useobject{currentmarker}{}%
\end{pgfscope}%
\begin{pgfscope}%
\pgfsys@transformshift{9.019214in}{3.887782in}%
\pgfsys@useobject{currentmarker}{}%
\end{pgfscope}%
\begin{pgfscope}%
\pgfsys@transformshift{9.036808in}{3.691193in}%
\pgfsys@useobject{currentmarker}{}%
\end{pgfscope}%
\begin{pgfscope}%
\pgfsys@transformshift{9.054401in}{3.955984in}%
\pgfsys@useobject{currentmarker}{}%
\end{pgfscope}%
\begin{pgfscope}%
\pgfsys@transformshift{9.071995in}{3.714784in}%
\pgfsys@useobject{currentmarker}{}%
\end{pgfscope}%
\begin{pgfscope}%
\pgfsys@transformshift{9.089589in}{3.803722in}%
\pgfsys@useobject{currentmarker}{}%
\end{pgfscope}%
\begin{pgfscope}%
\pgfsys@transformshift{9.107183in}{3.797315in}%
\pgfsys@useobject{currentmarker}{}%
\end{pgfscope}%
\begin{pgfscope}%
\pgfsys@transformshift{9.124777in}{3.662972in}%
\pgfsys@useobject{currentmarker}{}%
\end{pgfscope}%
\begin{pgfscope}%
\pgfsys@transformshift{9.142371in}{3.718881in}%
\pgfsys@useobject{currentmarker}{}%
\end{pgfscope}%
\begin{pgfscope}%
\pgfsys@transformshift{9.159965in}{3.727407in}%
\pgfsys@useobject{currentmarker}{}%
\end{pgfscope}%
\begin{pgfscope}%
\pgfsys@transformshift{9.177558in}{3.648683in}%
\pgfsys@useobject{currentmarker}{}%
\end{pgfscope}%
\begin{pgfscope}%
\pgfsys@transformshift{9.195152in}{3.475151in}%
\pgfsys@useobject{currentmarker}{}%
\end{pgfscope}%
\begin{pgfscope}%
\pgfsys@transformshift{9.212746in}{3.594877in}%
\pgfsys@useobject{currentmarker}{}%
\end{pgfscope}%
\begin{pgfscope}%
\pgfsys@transformshift{9.230340in}{3.677368in}%
\pgfsys@useobject{currentmarker}{}%
\end{pgfscope}%
\begin{pgfscope}%
\pgfsys@transformshift{9.247934in}{3.449577in}%
\pgfsys@useobject{currentmarker}{}%
\end{pgfscope}%
\begin{pgfscope}%
\pgfsys@transformshift{9.265528in}{3.484281in}%
\pgfsys@useobject{currentmarker}{}%
\end{pgfscope}%
\begin{pgfscope}%
\pgfsys@transformshift{9.283121in}{3.435330in}%
\pgfsys@useobject{currentmarker}{}%
\end{pgfscope}%
\begin{pgfscope}%
\pgfsys@transformshift{9.300715in}{3.649654in}%
\pgfsys@useobject{currentmarker}{}%
\end{pgfscope}%
\begin{pgfscope}%
\pgfsys@transformshift{9.318309in}{3.512382in}%
\pgfsys@useobject{currentmarker}{}%
\end{pgfscope}%
\begin{pgfscope}%
\pgfsys@transformshift{9.335903in}{3.469654in}%
\pgfsys@useobject{currentmarker}{}%
\end{pgfscope}%
\begin{pgfscope}%
\pgfsys@transformshift{9.353497in}{3.335540in}%
\pgfsys@useobject{currentmarker}{}%
\end{pgfscope}%
\begin{pgfscope}%
\pgfsys@transformshift{9.371091in}{3.456780in}%
\pgfsys@useobject{currentmarker}{}%
\end{pgfscope}%
\begin{pgfscope}%
\pgfsys@transformshift{9.388685in}{3.322666in}%
\pgfsys@useobject{currentmarker}{}%
\end{pgfscope}%
\begin{pgfscope}%
\pgfsys@transformshift{9.406278in}{3.386302in}%
\pgfsys@useobject{currentmarker}{}%
\end{pgfscope}%
\begin{pgfscope}%
\pgfsys@transformshift{9.423872in}{3.311898in}%
\pgfsys@useobject{currentmarker}{}%
\end{pgfscope}%
\begin{pgfscope}%
\pgfsys@transformshift{9.441466in}{3.441502in}%
\pgfsys@useobject{currentmarker}{}%
\end{pgfscope}%
\begin{pgfscope}%
\pgfsys@transformshift{9.459060in}{3.429239in}%
\pgfsys@useobject{currentmarker}{}%
\end{pgfscope}%
\begin{pgfscope}%
\pgfsys@transformshift{9.476654in}{3.349260in}%
\pgfsys@useobject{currentmarker}{}%
\end{pgfscope}%
\begin{pgfscope}%
\pgfsys@transformshift{9.494248in}{3.413070in}%
\pgfsys@useobject{currentmarker}{}%
\end{pgfscope}%
\begin{pgfscope}%
\pgfsys@transformshift{9.511842in}{3.265505in}%
\pgfsys@useobject{currentmarker}{}%
\end{pgfscope}%
\begin{pgfscope}%
\pgfsys@transformshift{9.529435in}{3.231219in}%
\pgfsys@useobject{currentmarker}{}%
\end{pgfscope}%
\begin{pgfscope}%
\pgfsys@transformshift{9.547029in}{3.436381in}%
\pgfsys@useobject{currentmarker}{}%
\end{pgfscope}%
\begin{pgfscope}%
\pgfsys@transformshift{9.564623in}{3.418730in}%
\pgfsys@useobject{currentmarker}{}%
\end{pgfscope}%
\begin{pgfscope}%
\pgfsys@transformshift{9.582217in}{3.478412in}%
\pgfsys@useobject{currentmarker}{}%
\end{pgfscope}%
\begin{pgfscope}%
\pgfsys@transformshift{9.599811in}{3.670229in}%
\pgfsys@useobject{currentmarker}{}%
\end{pgfscope}%
\begin{pgfscope}%
\pgfsys@transformshift{9.617405in}{3.538571in}%
\pgfsys@useobject{currentmarker}{}%
\end{pgfscope}%
\begin{pgfscope}%
\pgfsys@transformshift{9.634999in}{3.365555in}%
\pgfsys@useobject{currentmarker}{}%
\end{pgfscope}%
\begin{pgfscope}%
\pgfsys@transformshift{9.652592in}{3.590089in}%
\pgfsys@useobject{currentmarker}{}%
\end{pgfscope}%
\begin{pgfscope}%
\pgfsys@transformshift{9.670186in}{3.360521in}%
\pgfsys@useobject{currentmarker}{}%
\end{pgfscope}%
\begin{pgfscope}%
\pgfsys@transformshift{9.687780in}{3.466956in}%
\pgfsys@useobject{currentmarker}{}%
\end{pgfscope}%
\begin{pgfscope}%
\pgfsys@transformshift{9.705374in}{3.526978in}%
\pgfsys@useobject{currentmarker}{}%
\end{pgfscope}%
\begin{pgfscope}%
\pgfsys@transformshift{9.722968in}{3.728956in}%
\pgfsys@useobject{currentmarker}{}%
\end{pgfscope}%
\begin{pgfscope}%
\pgfsys@transformshift{9.740562in}{3.498783in}%
\pgfsys@useobject{currentmarker}{}%
\end{pgfscope}%
\begin{pgfscope}%
\pgfsys@transformshift{9.758155in}{3.510921in}%
\pgfsys@useobject{currentmarker}{}%
\end{pgfscope}%
\begin{pgfscope}%
\pgfsys@transformshift{9.775749in}{3.605397in}%
\pgfsys@useobject{currentmarker}{}%
\end{pgfscope}%
\begin{pgfscope}%
\pgfsys@transformshift{9.793343in}{3.567591in}%
\pgfsys@useobject{currentmarker}{}%
\end{pgfscope}%
\begin{pgfscope}%
\pgfsys@transformshift{9.810937in}{3.769320in}%
\pgfsys@useobject{currentmarker}{}%
\end{pgfscope}%
\begin{pgfscope}%
\pgfsys@transformshift{9.828531in}{3.562676in}%
\pgfsys@useobject{currentmarker}{}%
\end{pgfscope}%
\begin{pgfscope}%
\pgfsys@transformshift{9.846125in}{3.573159in}%
\pgfsys@useobject{currentmarker}{}%
\end{pgfscope}%
\begin{pgfscope}%
\pgfsys@transformshift{9.863719in}{3.661725in}%
\pgfsys@useobject{currentmarker}{}%
\end{pgfscope}%
\begin{pgfscope}%
\pgfsys@transformshift{9.881312in}{3.670459in}%
\pgfsys@useobject{currentmarker}{}%
\end{pgfscope}%
\begin{pgfscope}%
\pgfsys@transformshift{9.898906in}{3.931412in}%
\pgfsys@useobject{currentmarker}{}%
\end{pgfscope}%
\begin{pgfscope}%
\pgfsys@transformshift{9.916500in}{3.843274in}%
\pgfsys@useobject{currentmarker}{}%
\end{pgfscope}%
\begin{pgfscope}%
\pgfsys@transformshift{9.934094in}{3.765484in}%
\pgfsys@useobject{currentmarker}{}%
\end{pgfscope}%
\begin{pgfscope}%
\pgfsys@transformshift{9.951688in}{3.639892in}%
\pgfsys@useobject{currentmarker}{}%
\end{pgfscope}%
\begin{pgfscope}%
\pgfsys@transformshift{9.969282in}{3.857380in}%
\pgfsys@useobject{currentmarker}{}%
\end{pgfscope}%
\begin{pgfscope}%
\pgfsys@transformshift{9.986876in}{3.673775in}%
\pgfsys@useobject{currentmarker}{}%
\end{pgfscope}%
\begin{pgfscope}%
\pgfsys@transformshift{10.004469in}{3.620776in}%
\pgfsys@useobject{currentmarker}{}%
\end{pgfscope}%
\begin{pgfscope}%
\pgfsys@transformshift{10.022063in}{3.900005in}%
\pgfsys@useobject{currentmarker}{}%
\end{pgfscope}%
\begin{pgfscope}%
\pgfsys@transformshift{10.039657in}{3.809765in}%
\pgfsys@useobject{currentmarker}{}%
\end{pgfscope}%
\begin{pgfscope}%
\pgfsys@transformshift{10.057251in}{3.867996in}%
\pgfsys@useobject{currentmarker}{}%
\end{pgfscope}%
\begin{pgfscope}%
\pgfsys@transformshift{10.074845in}{3.801352in}%
\pgfsys@useobject{currentmarker}{}%
\end{pgfscope}%
\begin{pgfscope}%
\pgfsys@transformshift{10.092439in}{3.849159in}%
\pgfsys@useobject{currentmarker}{}%
\end{pgfscope}%
\begin{pgfscope}%
\pgfsys@transformshift{10.110033in}{3.686598in}%
\pgfsys@useobject{currentmarker}{}%
\end{pgfscope}%
\begin{pgfscope}%
\pgfsys@transformshift{10.127626in}{3.637093in}%
\pgfsys@useobject{currentmarker}{}%
\end{pgfscope}%
\begin{pgfscope}%
\pgfsys@transformshift{10.145220in}{3.800132in}%
\pgfsys@useobject{currentmarker}{}%
\end{pgfscope}%
\begin{pgfscope}%
\pgfsys@transformshift{10.162814in}{3.635404in}%
\pgfsys@useobject{currentmarker}{}%
\end{pgfscope}%
\begin{pgfscope}%
\pgfsys@transformshift{10.180408in}{3.632508in}%
\pgfsys@useobject{currentmarker}{}%
\end{pgfscope}%
\begin{pgfscope}%
\pgfsys@transformshift{10.198002in}{3.640821in}%
\pgfsys@useobject{currentmarker}{}%
\end{pgfscope}%
\begin{pgfscope}%
\pgfsys@transformshift{10.215596in}{3.672640in}%
\pgfsys@useobject{currentmarker}{}%
\end{pgfscope}%
\begin{pgfscope}%
\pgfsys@transformshift{10.233189in}{3.617668in}%
\pgfsys@useobject{currentmarker}{}%
\end{pgfscope}%
\begin{pgfscope}%
\pgfsys@transformshift{10.250783in}{3.495984in}%
\pgfsys@useobject{currentmarker}{}%
\end{pgfscope}%
\begin{pgfscope}%
\pgfsys@transformshift{10.268377in}{3.552707in}%
\pgfsys@useobject{currentmarker}{}%
\end{pgfscope}%
\begin{pgfscope}%
\pgfsys@transformshift{10.285971in}{3.373684in}%
\pgfsys@useobject{currentmarker}{}%
\end{pgfscope}%
\begin{pgfscope}%
\pgfsys@transformshift{10.303565in}{3.646375in}%
\pgfsys@useobject{currentmarker}{}%
\end{pgfscope}%
\begin{pgfscope}%
\pgfsys@transformshift{10.321159in}{3.401793in}%
\pgfsys@useobject{currentmarker}{}%
\end{pgfscope}%
\begin{pgfscope}%
\pgfsys@transformshift{10.338753in}{3.435629in}%
\pgfsys@useobject{currentmarker}{}%
\end{pgfscope}%
\begin{pgfscope}%
\pgfsys@transformshift{10.356346in}{3.537393in}%
\pgfsys@useobject{currentmarker}{}%
\end{pgfscope}%
\begin{pgfscope}%
\pgfsys@transformshift{10.373940in}{3.441417in}%
\pgfsys@useobject{currentmarker}{}%
\end{pgfscope}%
\begin{pgfscope}%
\pgfsys@transformshift{10.391534in}{3.660058in}%
\pgfsys@useobject{currentmarker}{}%
\end{pgfscope}%
\begin{pgfscope}%
\pgfsys@transformshift{10.409128in}{3.358041in}%
\pgfsys@useobject{currentmarker}{}%
\end{pgfscope}%
\begin{pgfscope}%
\pgfsys@transformshift{10.426722in}{3.505730in}%
\pgfsys@useobject{currentmarker}{}%
\end{pgfscope}%
\begin{pgfscope}%
\pgfsys@transformshift{10.444316in}{3.464710in}%
\pgfsys@useobject{currentmarker}{}%
\end{pgfscope}%
\begin{pgfscope}%
\pgfsys@transformshift{10.461910in}{3.341550in}%
\pgfsys@useobject{currentmarker}{}%
\end{pgfscope}%
\begin{pgfscope}%
\pgfsys@transformshift{10.479503in}{3.507820in}%
\pgfsys@useobject{currentmarker}{}%
\end{pgfscope}%
\begin{pgfscope}%
\pgfsys@transformshift{10.497097in}{3.432706in}%
\pgfsys@useobject{currentmarker}{}%
\end{pgfscope}%
\begin{pgfscope}%
\pgfsys@transformshift{10.514691in}{3.526663in}%
\pgfsys@useobject{currentmarker}{}%
\end{pgfscope}%
\begin{pgfscope}%
\pgfsys@transformshift{10.532285in}{3.531779in}%
\pgfsys@useobject{currentmarker}{}%
\end{pgfscope}%
\begin{pgfscope}%
\pgfsys@transformshift{10.549879in}{3.670360in}%
\pgfsys@useobject{currentmarker}{}%
\end{pgfscope}%
\begin{pgfscope}%
\pgfsys@transformshift{10.567473in}{3.590161in}%
\pgfsys@useobject{currentmarker}{}%
\end{pgfscope}%
\begin{pgfscope}%
\pgfsys@transformshift{10.585067in}{3.422463in}%
\pgfsys@useobject{currentmarker}{}%
\end{pgfscope}%
\begin{pgfscope}%
\pgfsys@transformshift{10.602660in}{3.444053in}%
\pgfsys@useobject{currentmarker}{}%
\end{pgfscope}%
\begin{pgfscope}%
\pgfsys@transformshift{10.620254in}{3.591023in}%
\pgfsys@useobject{currentmarker}{}%
\end{pgfscope}%
\begin{pgfscope}%
\pgfsys@transformshift{10.637848in}{3.558137in}%
\pgfsys@useobject{currentmarker}{}%
\end{pgfscope}%
\begin{pgfscope}%
\pgfsys@transformshift{10.655442in}{3.570947in}%
\pgfsys@useobject{currentmarker}{}%
\end{pgfscope}%
\begin{pgfscope}%
\pgfsys@transformshift{10.673036in}{3.356960in}%
\pgfsys@useobject{currentmarker}{}%
\end{pgfscope}%
\begin{pgfscope}%
\pgfsys@transformshift{10.690630in}{3.537244in}%
\pgfsys@useobject{currentmarker}{}%
\end{pgfscope}%
\begin{pgfscope}%
\pgfsys@transformshift{10.708223in}{3.483911in}%
\pgfsys@useobject{currentmarker}{}%
\end{pgfscope}%
\begin{pgfscope}%
\pgfsys@transformshift{10.725817in}{3.608578in}%
\pgfsys@useobject{currentmarker}{}%
\end{pgfscope}%
\begin{pgfscope}%
\pgfsys@transformshift{10.743411in}{3.592139in}%
\pgfsys@useobject{currentmarker}{}%
\end{pgfscope}%
\begin{pgfscope}%
\pgfsys@transformshift{10.761005in}{3.718154in}%
\pgfsys@useobject{currentmarker}{}%
\end{pgfscope}%
\begin{pgfscope}%
\pgfsys@transformshift{10.778599in}{3.681771in}%
\pgfsys@useobject{currentmarker}{}%
\end{pgfscope}%
\begin{pgfscope}%
\pgfsys@transformshift{10.796193in}{3.754388in}%
\pgfsys@useobject{currentmarker}{}%
\end{pgfscope}%
\begin{pgfscope}%
\pgfsys@transformshift{10.813787in}{3.651918in}%
\pgfsys@useobject{currentmarker}{}%
\end{pgfscope}%
\begin{pgfscope}%
\pgfsys@transformshift{10.831380in}{3.628742in}%
\pgfsys@useobject{currentmarker}{}%
\end{pgfscope}%
\begin{pgfscope}%
\pgfsys@transformshift{10.848974in}{3.708892in}%
\pgfsys@useobject{currentmarker}{}%
\end{pgfscope}%
\begin{pgfscope}%
\pgfsys@transformshift{10.866568in}{3.774506in}%
\pgfsys@useobject{currentmarker}{}%
\end{pgfscope}%
\begin{pgfscope}%
\pgfsys@transformshift{10.884162in}{3.840115in}%
\pgfsys@useobject{currentmarker}{}%
\end{pgfscope}%
\begin{pgfscope}%
\pgfsys@transformshift{10.901756in}{4.056527in}%
\pgfsys@useobject{currentmarker}{}%
\end{pgfscope}%
\begin{pgfscope}%
\pgfsys@transformshift{10.919350in}{3.845793in}%
\pgfsys@useobject{currentmarker}{}%
\end{pgfscope}%
\begin{pgfscope}%
\pgfsys@transformshift{10.936944in}{3.775681in}%
\pgfsys@useobject{currentmarker}{}%
\end{pgfscope}%
\begin{pgfscope}%
\pgfsys@transformshift{10.954537in}{3.859859in}%
\pgfsys@useobject{currentmarker}{}%
\end{pgfscope}%
\begin{pgfscope}%
\pgfsys@transformshift{10.972131in}{3.868597in}%
\pgfsys@useobject{currentmarker}{}%
\end{pgfscope}%
\begin{pgfscope}%
\pgfsys@transformshift{10.989725in}{3.984304in}%
\pgfsys@useobject{currentmarker}{}%
\end{pgfscope}%
\begin{pgfscope}%
\pgfsys@transformshift{11.007319in}{3.795885in}%
\pgfsys@useobject{currentmarker}{}%
\end{pgfscope}%
\begin{pgfscope}%
\pgfsys@transformshift{11.024913in}{3.975605in}%
\pgfsys@useobject{currentmarker}{}%
\end{pgfscope}%
\begin{pgfscope}%
\pgfsys@transformshift{11.042507in}{3.999505in}%
\pgfsys@useobject{currentmarker}{}%
\end{pgfscope}%
\begin{pgfscope}%
\pgfsys@transformshift{11.060101in}{4.019775in}%
\pgfsys@useobject{currentmarker}{}%
\end{pgfscope}%
\begin{pgfscope}%
\pgfsys@transformshift{11.077694in}{3.945838in}%
\pgfsys@useobject{currentmarker}{}%
\end{pgfscope}%
\begin{pgfscope}%
\pgfsys@transformshift{11.095288in}{3.991188in}%
\pgfsys@useobject{currentmarker}{}%
\end{pgfscope}%
\begin{pgfscope}%
\pgfsys@transformshift{11.112882in}{3.876964in}%
\pgfsys@useobject{currentmarker}{}%
\end{pgfscope}%
\begin{pgfscope}%
\pgfsys@transformshift{11.130476in}{3.976630in}%
\pgfsys@useobject{currentmarker}{}%
\end{pgfscope}%
\begin{pgfscope}%
\pgfsys@transformshift{11.148070in}{3.974377in}%
\pgfsys@useobject{currentmarker}{}%
\end{pgfscope}%
\begin{pgfscope}%
\pgfsys@transformshift{11.165664in}{4.073041in}%
\pgfsys@useobject{currentmarker}{}%
\end{pgfscope}%
\begin{pgfscope}%
\pgfsys@transformshift{11.183257in}{3.911734in}%
\pgfsys@useobject{currentmarker}{}%
\end{pgfscope}%
\begin{pgfscope}%
\pgfsys@transformshift{11.200851in}{4.106534in}%
\pgfsys@useobject{currentmarker}{}%
\end{pgfscope}%
\begin{pgfscope}%
\pgfsys@transformshift{11.218445in}{4.174823in}%
\pgfsys@useobject{currentmarker}{}%
\end{pgfscope}%
\begin{pgfscope}%
\pgfsys@transformshift{11.236039in}{3.805324in}%
\pgfsys@useobject{currentmarker}{}%
\end{pgfscope}%
\begin{pgfscope}%
\pgfsys@transformshift{11.253633in}{4.052418in}%
\pgfsys@useobject{currentmarker}{}%
\end{pgfscope}%
\begin{pgfscope}%
\pgfsys@transformshift{11.271227in}{4.069229in}%
\pgfsys@useobject{currentmarker}{}%
\end{pgfscope}%
\begin{pgfscope}%
\pgfsys@transformshift{11.288821in}{3.925320in}%
\pgfsys@useobject{currentmarker}{}%
\end{pgfscope}%
\begin{pgfscope}%
\pgfsys@transformshift{11.306414in}{3.938972in}%
\pgfsys@useobject{currentmarker}{}%
\end{pgfscope}%
\begin{pgfscope}%
\pgfsys@transformshift{11.324008in}{3.954318in}%
\pgfsys@useobject{currentmarker}{}%
\end{pgfscope}%
\begin{pgfscope}%
\pgfsys@transformshift{11.341602in}{3.925304in}%
\pgfsys@useobject{currentmarker}{}%
\end{pgfscope}%
\begin{pgfscope}%
\pgfsys@transformshift{11.359196in}{3.911578in}%
\pgfsys@useobject{currentmarker}{}%
\end{pgfscope}%
\begin{pgfscope}%
\pgfsys@transformshift{11.376790in}{3.759418in}%
\pgfsys@useobject{currentmarker}{}%
\end{pgfscope}%
\begin{pgfscope}%
\pgfsys@transformshift{11.394384in}{4.035096in}%
\pgfsys@useobject{currentmarker}{}%
\end{pgfscope}%
\begin{pgfscope}%
\pgfsys@transformshift{11.411978in}{4.015175in}%
\pgfsys@useobject{currentmarker}{}%
\end{pgfscope}%
\begin{pgfscope}%
\pgfsys@transformshift{11.429571in}{3.810072in}%
\pgfsys@useobject{currentmarker}{}%
\end{pgfscope}%
\begin{pgfscope}%
\pgfsys@transformshift{11.447165in}{3.732047in}%
\pgfsys@useobject{currentmarker}{}%
\end{pgfscope}%
\begin{pgfscope}%
\pgfsys@transformshift{11.464759in}{3.924124in}%
\pgfsys@useobject{currentmarker}{}%
\end{pgfscope}%
\begin{pgfscope}%
\pgfsys@transformshift{11.482353in}{3.802702in}%
\pgfsys@useobject{currentmarker}{}%
\end{pgfscope}%
\end{pgfscope}%
\begin{pgfscope}%
\pgfpathrectangle{\pgfqpoint{7.105882in}{3.197368in}}{\pgfqpoint{4.376471in}{0.978947in}} %
\pgfusepath{clip}%
\pgfsetroundcap%
\pgfsetroundjoin%
\pgfsetlinewidth{1.756562pt}%
\definecolor{currentstroke}{rgb}{0.298039,0.447059,0.690196}%
\pgfsetstrokecolor{currentstroke}%
\pgfsetdash{}{0pt}%
\pgfpathmoveto{\pgfqpoint{7.981176in}{3.564359in}}%
\pgfpathlineto{\pgfqpoint{8.508992in}{3.564469in}}%
\pgfpathlineto{\pgfqpoint{9.863719in}{3.564508in}}%
\pgfpathlineto{\pgfqpoint{11.482353in}{3.564727in}}%
\pgfpathlineto{\pgfqpoint{11.482353in}{3.564727in}}%
\pgfusepath{stroke}%
\end{pgfscope}%
\begin{pgfscope}%
\pgfsetrectcap%
\pgfsetmiterjoin%
\pgfsetlinewidth{1.003750pt}%
\definecolor{currentstroke}{rgb}{0.800000,0.800000,0.800000}%
\pgfsetstrokecolor{currentstroke}%
\pgfsetdash{}{0pt}%
\pgfpathmoveto{\pgfqpoint{7.105882in}{3.197368in}}%
\pgfpathlineto{\pgfqpoint{7.105882in}{4.176316in}}%
\pgfusepath{stroke}%
\end{pgfscope}%
\begin{pgfscope}%
\pgfsetrectcap%
\pgfsetmiterjoin%
\pgfsetlinewidth{1.003750pt}%
\definecolor{currentstroke}{rgb}{0.800000,0.800000,0.800000}%
\pgfsetstrokecolor{currentstroke}%
\pgfsetdash{}{0pt}%
\pgfpathmoveto{\pgfqpoint{11.482353in}{3.197368in}}%
\pgfpathlineto{\pgfqpoint{11.482353in}{4.176316in}}%
\pgfusepath{stroke}%
\end{pgfscope}%
\begin{pgfscope}%
\pgfsetrectcap%
\pgfsetmiterjoin%
\pgfsetlinewidth{1.003750pt}%
\definecolor{currentstroke}{rgb}{0.800000,0.800000,0.800000}%
\pgfsetstrokecolor{currentstroke}%
\pgfsetdash{}{0pt}%
\pgfpathmoveto{\pgfqpoint{7.105882in}{4.176316in}}%
\pgfpathlineto{\pgfqpoint{11.482353in}{4.176316in}}%
\pgfusepath{stroke}%
\end{pgfscope}%
\begin{pgfscope}%
\pgfsetrectcap%
\pgfsetmiterjoin%
\pgfsetlinewidth{1.003750pt}%
\definecolor{currentstroke}{rgb}{0.800000,0.800000,0.800000}%
\pgfsetstrokecolor{currentstroke}%
\pgfsetdash{}{0pt}%
\pgfpathmoveto{\pgfqpoint{7.105882in}{3.197368in}}%
\pgfpathlineto{\pgfqpoint{11.482353in}{3.197368in}}%
\pgfusepath{stroke}%
\end{pgfscope}%
\begin{pgfscope}%
\pgfsetroundcap%
\pgfsetroundjoin%
\pgfsetlinewidth{1.756562pt}%
\definecolor{currentstroke}{rgb}{0.298039,0.447059,0.690196}%
\pgfsetstrokecolor{currentstroke}%
\pgfsetdash{}{0pt}%
\pgfpathmoveto{\pgfqpoint{7.230882in}{3.791511in}}%
\pgfpathlineto{\pgfqpoint{7.508660in}{3.791511in}}%
\pgfusepath{stroke}%
\end{pgfscope}%
\begin{pgfscope}%
\definecolor{textcolor}{rgb}{0.150000,0.150000,0.150000}%
\pgfsetstrokecolor{textcolor}%
\pgfsetfillcolor{textcolor}%
\pgftext[x=7.619771in,y=3.742900in,left,base]{\color{textcolor}\sffamily\fontsize{10.000000}{12.000000}\selectfont \(\displaystyle \widetilde{\Phi}^* \theta^{\parallel}\)}%
\end{pgfscope}%
\begin{pgfscope}%
\pgfsetbuttcap%
\pgfsetroundjoin%
\definecolor{currentfill}{rgb}{1.000000,0.000000,0.000000}%
\pgfsetfillcolor{currentfill}%
\pgfsetlinewidth{2.007500pt}%
\definecolor{currentstroke}{rgb}{1.000000,0.000000,0.000000}%
\pgfsetstrokecolor{currentstroke}%
\pgfsetdash{}{0pt}%
\pgfpathmoveto{\pgfqpoint{7.338715in}{3.582893in}}%
\pgfpathlineto{\pgfqpoint{7.400828in}{3.582893in}}%
\pgfpathmoveto{\pgfqpoint{7.369771in}{3.551837in}}%
\pgfpathlineto{\pgfqpoint{7.369771in}{3.613950in}}%
\pgfusepath{stroke,fill}%
\end{pgfscope}%
\begin{pgfscope}%
\pgfsetbuttcap%
\pgfsetroundjoin%
\definecolor{currentfill}{rgb}{1.000000,0.000000,0.000000}%
\pgfsetfillcolor{currentfill}%
\pgfsetlinewidth{2.007500pt}%
\definecolor{currentstroke}{rgb}{1.000000,0.000000,0.000000}%
\pgfsetstrokecolor{currentstroke}%
\pgfsetdash{}{0pt}%
\pgfpathmoveto{\pgfqpoint{7.338715in}{3.582893in}}%
\pgfpathlineto{\pgfqpoint{7.400828in}{3.582893in}}%
\pgfpathmoveto{\pgfqpoint{7.369771in}{3.551837in}}%
\pgfpathlineto{\pgfqpoint{7.369771in}{3.613950in}}%
\pgfusepath{stroke,fill}%
\end{pgfscope}%
\begin{pgfscope}%
\pgfsetbuttcap%
\pgfsetroundjoin%
\definecolor{currentfill}{rgb}{1.000000,0.000000,0.000000}%
\pgfsetfillcolor{currentfill}%
\pgfsetlinewidth{2.007500pt}%
\definecolor{currentstroke}{rgb}{1.000000,0.000000,0.000000}%
\pgfsetstrokecolor{currentstroke}%
\pgfsetdash{}{0pt}%
\pgfpathmoveto{\pgfqpoint{7.338715in}{3.582893in}}%
\pgfpathlineto{\pgfqpoint{7.400828in}{3.582893in}}%
\pgfpathmoveto{\pgfqpoint{7.369771in}{3.551837in}}%
\pgfpathlineto{\pgfqpoint{7.369771in}{3.613950in}}%
\pgfusepath{stroke,fill}%
\end{pgfscope}%
\begin{pgfscope}%
\definecolor{textcolor}{rgb}{0.150000,0.150000,0.150000}%
\pgfsetstrokecolor{textcolor}%
\pgfsetfillcolor{textcolor}%
\pgftext[x=7.619771in,y=3.546435in,left,base]{\color{textcolor}\sffamily\fontsize{10.000000}{12.000000}\selectfont train}%
\end{pgfscope}%
\begin{pgfscope}%
\pgfsetbuttcap%
\pgfsetroundjoin%
\definecolor{currentfill}{rgb}{0.000000,0.000000,0.000000}%
\pgfsetfillcolor{currentfill}%
\pgfsetlinewidth{0.301125pt}%
\definecolor{currentstroke}{rgb}{0.000000,0.000000,0.000000}%
\pgfsetstrokecolor{currentstroke}%
\pgfsetdash{}{0pt}%
\pgfpathmoveto{\pgfqpoint{7.369771in}{3.370900in}}%
\pgfpathcurveto{\pgfqpoint{7.373889in}{3.370900in}}{\pgfqpoint{7.377839in}{3.372536in}}{\pgfqpoint{7.380751in}{3.375448in}}%
\pgfpathcurveto{\pgfqpoint{7.383663in}{3.378360in}}{\pgfqpoint{7.385299in}{3.382310in}}{\pgfqpoint{7.385299in}{3.386428in}}%
\pgfpathcurveto{\pgfqpoint{7.385299in}{3.390546in}}{\pgfqpoint{7.383663in}{3.394496in}}{\pgfqpoint{7.380751in}{3.397408in}}%
\pgfpathcurveto{\pgfqpoint{7.377839in}{3.400320in}}{\pgfqpoint{7.373889in}{3.401956in}}{\pgfqpoint{7.369771in}{3.401956in}}%
\pgfpathcurveto{\pgfqpoint{7.365653in}{3.401956in}}{\pgfqpoint{7.361703in}{3.400320in}}{\pgfqpoint{7.358791in}{3.397408in}}%
\pgfpathcurveto{\pgfqpoint{7.355879in}{3.394496in}}{\pgfqpoint{7.354243in}{3.390546in}}{\pgfqpoint{7.354243in}{3.386428in}}%
\pgfpathcurveto{\pgfqpoint{7.354243in}{3.382310in}}{\pgfqpoint{7.355879in}{3.378360in}}{\pgfqpoint{7.358791in}{3.375448in}}%
\pgfpathcurveto{\pgfqpoint{7.361703in}{3.372536in}}{\pgfqpoint{7.365653in}{3.370900in}}{\pgfqpoint{7.369771in}{3.370900in}}%
\pgfpathclose%
\pgfusepath{stroke,fill}%
\end{pgfscope}%
\begin{pgfscope}%
\pgfsetbuttcap%
\pgfsetroundjoin%
\definecolor{currentfill}{rgb}{0.000000,0.000000,0.000000}%
\pgfsetfillcolor{currentfill}%
\pgfsetlinewidth{0.301125pt}%
\definecolor{currentstroke}{rgb}{0.000000,0.000000,0.000000}%
\pgfsetstrokecolor{currentstroke}%
\pgfsetdash{}{0pt}%
\pgfpathmoveto{\pgfqpoint{7.369771in}{3.370900in}}%
\pgfpathcurveto{\pgfqpoint{7.373889in}{3.370900in}}{\pgfqpoint{7.377839in}{3.372536in}}{\pgfqpoint{7.380751in}{3.375448in}}%
\pgfpathcurveto{\pgfqpoint{7.383663in}{3.378360in}}{\pgfqpoint{7.385299in}{3.382310in}}{\pgfqpoint{7.385299in}{3.386428in}}%
\pgfpathcurveto{\pgfqpoint{7.385299in}{3.390546in}}{\pgfqpoint{7.383663in}{3.394496in}}{\pgfqpoint{7.380751in}{3.397408in}}%
\pgfpathcurveto{\pgfqpoint{7.377839in}{3.400320in}}{\pgfqpoint{7.373889in}{3.401956in}}{\pgfqpoint{7.369771in}{3.401956in}}%
\pgfpathcurveto{\pgfqpoint{7.365653in}{3.401956in}}{\pgfqpoint{7.361703in}{3.400320in}}{\pgfqpoint{7.358791in}{3.397408in}}%
\pgfpathcurveto{\pgfqpoint{7.355879in}{3.394496in}}{\pgfqpoint{7.354243in}{3.390546in}}{\pgfqpoint{7.354243in}{3.386428in}}%
\pgfpathcurveto{\pgfqpoint{7.354243in}{3.382310in}}{\pgfqpoint{7.355879in}{3.378360in}}{\pgfqpoint{7.358791in}{3.375448in}}%
\pgfpathcurveto{\pgfqpoint{7.361703in}{3.372536in}}{\pgfqpoint{7.365653in}{3.370900in}}{\pgfqpoint{7.369771in}{3.370900in}}%
\pgfpathclose%
\pgfusepath{stroke,fill}%
\end{pgfscope}%
\begin{pgfscope}%
\pgfsetbuttcap%
\pgfsetroundjoin%
\definecolor{currentfill}{rgb}{0.000000,0.000000,0.000000}%
\pgfsetfillcolor{currentfill}%
\pgfsetlinewidth{0.301125pt}%
\definecolor{currentstroke}{rgb}{0.000000,0.000000,0.000000}%
\pgfsetstrokecolor{currentstroke}%
\pgfsetdash{}{0pt}%
\pgfpathmoveto{\pgfqpoint{7.369771in}{3.370900in}}%
\pgfpathcurveto{\pgfqpoint{7.373889in}{3.370900in}}{\pgfqpoint{7.377839in}{3.372536in}}{\pgfqpoint{7.380751in}{3.375448in}}%
\pgfpathcurveto{\pgfqpoint{7.383663in}{3.378360in}}{\pgfqpoint{7.385299in}{3.382310in}}{\pgfqpoint{7.385299in}{3.386428in}}%
\pgfpathcurveto{\pgfqpoint{7.385299in}{3.390546in}}{\pgfqpoint{7.383663in}{3.394496in}}{\pgfqpoint{7.380751in}{3.397408in}}%
\pgfpathcurveto{\pgfqpoint{7.377839in}{3.400320in}}{\pgfqpoint{7.373889in}{3.401956in}}{\pgfqpoint{7.369771in}{3.401956in}}%
\pgfpathcurveto{\pgfqpoint{7.365653in}{3.401956in}}{\pgfqpoint{7.361703in}{3.400320in}}{\pgfqpoint{7.358791in}{3.397408in}}%
\pgfpathcurveto{\pgfqpoint{7.355879in}{3.394496in}}{\pgfqpoint{7.354243in}{3.390546in}}{\pgfqpoint{7.354243in}{3.386428in}}%
\pgfpathcurveto{\pgfqpoint{7.354243in}{3.382310in}}{\pgfqpoint{7.355879in}{3.378360in}}{\pgfqpoint{7.358791in}{3.375448in}}%
\pgfpathcurveto{\pgfqpoint{7.361703in}{3.372536in}}{\pgfqpoint{7.365653in}{3.370900in}}{\pgfqpoint{7.369771in}{3.370900in}}%
\pgfpathclose%
\pgfusepath{stroke,fill}%
\end{pgfscope}%
\begin{pgfscope}%
\definecolor{textcolor}{rgb}{0.150000,0.150000,0.150000}%
\pgfsetstrokecolor{textcolor}%
\pgfsetfillcolor{textcolor}%
\pgftext[x=7.619771in,y=3.349970in,left,base]{\color{textcolor}\sffamily\fontsize{10.000000}{12.000000}\selectfont test}%
\end{pgfscope}%
\begin{pgfscope}%
\pgfsetbuttcap%
\pgfsetmiterjoin%
\definecolor{currentfill}{rgb}{1.000000,1.000000,1.000000}%
\pgfsetfillcolor{currentfill}%
\pgfsetlinewidth{0.000000pt}%
\definecolor{currentstroke}{rgb}{0.000000,0.000000,0.000000}%
\pgfsetstrokecolor{currentstroke}%
\pgfsetstrokeopacity{0.000000}%
\pgfsetdash{}{0pt}%
\pgfpathmoveto{\pgfqpoint{12.211765in}{3.197368in}}%
\pgfpathlineto{\pgfqpoint{14.400000in}{3.197368in}}%
\pgfpathlineto{\pgfqpoint{14.400000in}{4.176316in}}%
\pgfpathlineto{\pgfqpoint{12.211765in}{4.176316in}}%
\pgfpathclose%
\pgfusepath{fill}%
\end{pgfscope}%
\begin{pgfscope}%
\pgfpathrectangle{\pgfqpoint{12.211765in}{3.197368in}}{\pgfqpoint{2.188235in}{0.978947in}} %
\pgfusepath{clip}%
\pgfsetroundcap%
\pgfsetroundjoin%
\pgfsetlinewidth{1.003750pt}%
\definecolor{currentstroke}{rgb}{0.800000,0.800000,0.800000}%
\pgfsetstrokecolor{currentstroke}%
\pgfsetdash{}{0pt}%
\pgfpathmoveto{\pgfqpoint{12.211765in}{3.197368in}}%
\pgfpathlineto{\pgfqpoint{12.211765in}{4.176316in}}%
\pgfusepath{stroke}%
\end{pgfscope}%
\begin{pgfscope}%
\pgfpathrectangle{\pgfqpoint{12.211765in}{3.197368in}}{\pgfqpoint{2.188235in}{0.978947in}} %
\pgfusepath{clip}%
\pgfsetroundcap%
\pgfsetroundjoin%
\pgfsetlinewidth{1.003750pt}%
\definecolor{currentstroke}{rgb}{0.800000,0.800000,0.800000}%
\pgfsetstrokecolor{currentstroke}%
\pgfsetdash{}{0pt}%
\pgfpathmoveto{\pgfqpoint{12.485294in}{3.197368in}}%
\pgfpathlineto{\pgfqpoint{12.485294in}{4.176316in}}%
\pgfusepath{stroke}%
\end{pgfscope}%
\begin{pgfscope}%
\pgfpathrectangle{\pgfqpoint{12.211765in}{3.197368in}}{\pgfqpoint{2.188235in}{0.978947in}} %
\pgfusepath{clip}%
\pgfsetroundcap%
\pgfsetroundjoin%
\pgfsetlinewidth{1.003750pt}%
\definecolor{currentstroke}{rgb}{0.800000,0.800000,0.800000}%
\pgfsetstrokecolor{currentstroke}%
\pgfsetdash{}{0pt}%
\pgfpathmoveto{\pgfqpoint{12.758824in}{3.197368in}}%
\pgfpathlineto{\pgfqpoint{12.758824in}{4.176316in}}%
\pgfusepath{stroke}%
\end{pgfscope}%
\begin{pgfscope}%
\pgfpathrectangle{\pgfqpoint{12.211765in}{3.197368in}}{\pgfqpoint{2.188235in}{0.978947in}} %
\pgfusepath{clip}%
\pgfsetroundcap%
\pgfsetroundjoin%
\pgfsetlinewidth{1.003750pt}%
\definecolor{currentstroke}{rgb}{0.800000,0.800000,0.800000}%
\pgfsetstrokecolor{currentstroke}%
\pgfsetdash{}{0pt}%
\pgfpathmoveto{\pgfqpoint{13.032353in}{3.197368in}}%
\pgfpathlineto{\pgfqpoint{13.032353in}{4.176316in}}%
\pgfusepath{stroke}%
\end{pgfscope}%
\begin{pgfscope}%
\pgfpathrectangle{\pgfqpoint{12.211765in}{3.197368in}}{\pgfqpoint{2.188235in}{0.978947in}} %
\pgfusepath{clip}%
\pgfsetroundcap%
\pgfsetroundjoin%
\pgfsetlinewidth{1.003750pt}%
\definecolor{currentstroke}{rgb}{0.800000,0.800000,0.800000}%
\pgfsetstrokecolor{currentstroke}%
\pgfsetdash{}{0pt}%
\pgfpathmoveto{\pgfqpoint{13.305882in}{3.197368in}}%
\pgfpathlineto{\pgfqpoint{13.305882in}{4.176316in}}%
\pgfusepath{stroke}%
\end{pgfscope}%
\begin{pgfscope}%
\pgfpathrectangle{\pgfqpoint{12.211765in}{3.197368in}}{\pgfqpoint{2.188235in}{0.978947in}} %
\pgfusepath{clip}%
\pgfsetroundcap%
\pgfsetroundjoin%
\pgfsetlinewidth{1.003750pt}%
\definecolor{currentstroke}{rgb}{0.800000,0.800000,0.800000}%
\pgfsetstrokecolor{currentstroke}%
\pgfsetdash{}{0pt}%
\pgfpathmoveto{\pgfqpoint{13.579412in}{3.197368in}}%
\pgfpathlineto{\pgfqpoint{13.579412in}{4.176316in}}%
\pgfusepath{stroke}%
\end{pgfscope}%
\begin{pgfscope}%
\pgfpathrectangle{\pgfqpoint{12.211765in}{3.197368in}}{\pgfqpoint{2.188235in}{0.978947in}} %
\pgfusepath{clip}%
\pgfsetroundcap%
\pgfsetroundjoin%
\pgfsetlinewidth{1.003750pt}%
\definecolor{currentstroke}{rgb}{0.800000,0.800000,0.800000}%
\pgfsetstrokecolor{currentstroke}%
\pgfsetdash{}{0pt}%
\pgfpathmoveto{\pgfqpoint{13.852941in}{3.197368in}}%
\pgfpathlineto{\pgfqpoint{13.852941in}{4.176316in}}%
\pgfusepath{stroke}%
\end{pgfscope}%
\begin{pgfscope}%
\pgfpathrectangle{\pgfqpoint{12.211765in}{3.197368in}}{\pgfqpoint{2.188235in}{0.978947in}} %
\pgfusepath{clip}%
\pgfsetroundcap%
\pgfsetroundjoin%
\pgfsetlinewidth{1.003750pt}%
\definecolor{currentstroke}{rgb}{0.800000,0.800000,0.800000}%
\pgfsetstrokecolor{currentstroke}%
\pgfsetdash{}{0pt}%
\pgfpathmoveto{\pgfqpoint{14.126471in}{3.197368in}}%
\pgfpathlineto{\pgfqpoint{14.126471in}{4.176316in}}%
\pgfusepath{stroke}%
\end{pgfscope}%
\begin{pgfscope}%
\pgfpathrectangle{\pgfqpoint{12.211765in}{3.197368in}}{\pgfqpoint{2.188235in}{0.978947in}} %
\pgfusepath{clip}%
\pgfsetroundcap%
\pgfsetroundjoin%
\pgfsetlinewidth{1.003750pt}%
\definecolor{currentstroke}{rgb}{0.800000,0.800000,0.800000}%
\pgfsetstrokecolor{currentstroke}%
\pgfsetdash{}{0pt}%
\pgfpathmoveto{\pgfqpoint{14.400000in}{3.197368in}}%
\pgfpathlineto{\pgfqpoint{14.400000in}{4.176316in}}%
\pgfusepath{stroke}%
\end{pgfscope}%
\begin{pgfscope}%
\pgfpathrectangle{\pgfqpoint{12.211765in}{3.197368in}}{\pgfqpoint{2.188235in}{0.978947in}} %
\pgfusepath{clip}%
\pgfsetroundcap%
\pgfsetroundjoin%
\pgfsetlinewidth{1.003750pt}%
\definecolor{currentstroke}{rgb}{0.800000,0.800000,0.800000}%
\pgfsetstrokecolor{currentstroke}%
\pgfsetdash{}{0pt}%
\pgfpathmoveto{\pgfqpoint{12.211765in}{3.197368in}}%
\pgfpathlineto{\pgfqpoint{14.400000in}{3.197368in}}%
\pgfusepath{stroke}%
\end{pgfscope}%
\begin{pgfscope}%
\definecolor{textcolor}{rgb}{0.150000,0.150000,0.150000}%
\pgfsetstrokecolor{textcolor}%
\pgfsetfillcolor{textcolor}%
\pgftext[x=12.114542in,y=3.197368in,right,]{\color{textcolor}\sffamily\fontsize{10.000000}{12.000000}\selectfont \(\displaystyle 0\)}%
\end{pgfscope}%
\begin{pgfscope}%
\pgfpathrectangle{\pgfqpoint{12.211765in}{3.197368in}}{\pgfqpoint{2.188235in}{0.978947in}} %
\pgfusepath{clip}%
\pgfsetroundcap%
\pgfsetroundjoin%
\pgfsetlinewidth{1.003750pt}%
\definecolor{currentstroke}{rgb}{0.800000,0.800000,0.800000}%
\pgfsetstrokecolor{currentstroke}%
\pgfsetdash{}{0pt}%
\pgfpathmoveto{\pgfqpoint{12.211765in}{3.442105in}}%
\pgfpathlineto{\pgfqpoint{14.400000in}{3.442105in}}%
\pgfusepath{stroke}%
\end{pgfscope}%
\begin{pgfscope}%
\definecolor{textcolor}{rgb}{0.150000,0.150000,0.150000}%
\pgfsetstrokecolor{textcolor}%
\pgfsetfillcolor{textcolor}%
\pgftext[x=12.114542in,y=3.442105in,right,]{\color{textcolor}\sffamily\fontsize{10.000000}{12.000000}\selectfont \(\displaystyle 50\)}%
\end{pgfscope}%
\begin{pgfscope}%
\pgfpathrectangle{\pgfqpoint{12.211765in}{3.197368in}}{\pgfqpoint{2.188235in}{0.978947in}} %
\pgfusepath{clip}%
\pgfsetroundcap%
\pgfsetroundjoin%
\pgfsetlinewidth{1.003750pt}%
\definecolor{currentstroke}{rgb}{0.800000,0.800000,0.800000}%
\pgfsetstrokecolor{currentstroke}%
\pgfsetdash{}{0pt}%
\pgfpathmoveto{\pgfqpoint{12.211765in}{3.686842in}}%
\pgfpathlineto{\pgfqpoint{14.400000in}{3.686842in}}%
\pgfusepath{stroke}%
\end{pgfscope}%
\begin{pgfscope}%
\definecolor{textcolor}{rgb}{0.150000,0.150000,0.150000}%
\pgfsetstrokecolor{textcolor}%
\pgfsetfillcolor{textcolor}%
\pgftext[x=12.114542in,y=3.686842in,right,]{\color{textcolor}\sffamily\fontsize{10.000000}{12.000000}\selectfont \(\displaystyle 100\)}%
\end{pgfscope}%
\begin{pgfscope}%
\pgfpathrectangle{\pgfqpoint{12.211765in}{3.197368in}}{\pgfqpoint{2.188235in}{0.978947in}} %
\pgfusepath{clip}%
\pgfsetroundcap%
\pgfsetroundjoin%
\pgfsetlinewidth{1.003750pt}%
\definecolor{currentstroke}{rgb}{0.800000,0.800000,0.800000}%
\pgfsetstrokecolor{currentstroke}%
\pgfsetdash{}{0pt}%
\pgfpathmoveto{\pgfqpoint{12.211765in}{3.931579in}}%
\pgfpathlineto{\pgfqpoint{14.400000in}{3.931579in}}%
\pgfusepath{stroke}%
\end{pgfscope}%
\begin{pgfscope}%
\definecolor{textcolor}{rgb}{0.150000,0.150000,0.150000}%
\pgfsetstrokecolor{textcolor}%
\pgfsetfillcolor{textcolor}%
\pgftext[x=12.114542in,y=3.931579in,right,]{\color{textcolor}\sffamily\fontsize{10.000000}{12.000000}\selectfont \(\displaystyle 150\)}%
\end{pgfscope}%
\begin{pgfscope}%
\pgfpathrectangle{\pgfqpoint{12.211765in}{3.197368in}}{\pgfqpoint{2.188235in}{0.978947in}} %
\pgfusepath{clip}%
\pgfsetroundcap%
\pgfsetroundjoin%
\pgfsetlinewidth{1.003750pt}%
\definecolor{currentstroke}{rgb}{0.800000,0.800000,0.800000}%
\pgfsetstrokecolor{currentstroke}%
\pgfsetdash{}{0pt}%
\pgfpathmoveto{\pgfqpoint{12.211765in}{4.176316in}}%
\pgfpathlineto{\pgfqpoint{14.400000in}{4.176316in}}%
\pgfusepath{stroke}%
\end{pgfscope}%
\begin{pgfscope}%
\definecolor{textcolor}{rgb}{0.150000,0.150000,0.150000}%
\pgfsetstrokecolor{textcolor}%
\pgfsetfillcolor{textcolor}%
\pgftext[x=12.114542in,y=4.176316in,right,]{\color{textcolor}\sffamily\fontsize{10.000000}{12.000000}\selectfont \(\displaystyle 200\)}%
\end{pgfscope}%
\begin{pgfscope}%
\definecolor{textcolor}{rgb}{0.150000,0.150000,0.150000}%
\pgfsetstrokecolor{textcolor}%
\pgfsetfillcolor{textcolor}%
\pgftext[x=11.836764in,y=3.686842in,,bottom,rotate=90.000000]{\color{textcolor}\sffamily\fontsize{11.000000}{13.200000}\selectfont \(\displaystyle \theta^{\parallel}_j\)}%
\end{pgfscope}%
\begin{pgfscope}%
\pgfpathrectangle{\pgfqpoint{12.211765in}{3.197368in}}{\pgfqpoint{2.188235in}{0.978947in}} %
\pgfusepath{clip}%
\pgfsetroundcap%
\pgfsetroundjoin%
\pgfsetlinewidth{1.756562pt}%
\definecolor{currentstroke}{rgb}{0.298039,0.447059,0.690196}%
\pgfsetstrokecolor{currentstroke}%
\pgfsetdash{}{0pt}%
\pgfpathmoveto{\pgfqpoint{13.305745in}{3.197368in}}%
\pgfpathlineto{\pgfqpoint{13.305836in}{3.970737in}}%
\pgfpathlineto{\pgfqpoint{13.305894in}{4.019684in}}%
\pgfpathlineto{\pgfqpoint{13.305865in}{4.142053in}}%
\pgfpathlineto{\pgfqpoint{13.305974in}{4.171421in}}%
\pgfpathlineto{\pgfqpoint{13.305974in}{4.171421in}}%
\pgfusepath{stroke}%
\end{pgfscope}%
\begin{pgfscope}%
\pgfsetrectcap%
\pgfsetmiterjoin%
\pgfsetlinewidth{1.003750pt}%
\definecolor{currentstroke}{rgb}{0.800000,0.800000,0.800000}%
\pgfsetstrokecolor{currentstroke}%
\pgfsetdash{}{0pt}%
\pgfpathmoveto{\pgfqpoint{12.211765in}{3.197368in}}%
\pgfpathlineto{\pgfqpoint{12.211765in}{4.176316in}}%
\pgfusepath{stroke}%
\end{pgfscope}%
\begin{pgfscope}%
\pgfsetrectcap%
\pgfsetmiterjoin%
\pgfsetlinewidth{1.003750pt}%
\definecolor{currentstroke}{rgb}{0.800000,0.800000,0.800000}%
\pgfsetstrokecolor{currentstroke}%
\pgfsetdash{}{0pt}%
\pgfpathmoveto{\pgfqpoint{14.400000in}{3.197368in}}%
\pgfpathlineto{\pgfqpoint{14.400000in}{4.176316in}}%
\pgfusepath{stroke}%
\end{pgfscope}%
\begin{pgfscope}%
\pgfsetrectcap%
\pgfsetmiterjoin%
\pgfsetlinewidth{1.003750pt}%
\definecolor{currentstroke}{rgb}{0.800000,0.800000,0.800000}%
\pgfsetstrokecolor{currentstroke}%
\pgfsetdash{}{0pt}%
\pgfpathmoveto{\pgfqpoint{12.211765in}{4.176316in}}%
\pgfpathlineto{\pgfqpoint{14.400000in}{4.176316in}}%
\pgfusepath{stroke}%
\end{pgfscope}%
\begin{pgfscope}%
\pgfsetrectcap%
\pgfsetmiterjoin%
\pgfsetlinewidth{1.003750pt}%
\definecolor{currentstroke}{rgb}{0.800000,0.800000,0.800000}%
\pgfsetstrokecolor{currentstroke}%
\pgfsetdash{}{0pt}%
\pgfpathmoveto{\pgfqpoint{12.211765in}{3.197368in}}%
\pgfpathlineto{\pgfqpoint{14.400000in}{3.197368in}}%
\pgfusepath{stroke}%
\end{pgfscope}%
\begin{pgfscope}%
\pgfsetbuttcap%
\pgfsetmiterjoin%
\definecolor{currentfill}{rgb}{1.000000,1.000000,1.000000}%
\pgfsetfillcolor{currentfill}%
\pgfsetlinewidth{0.000000pt}%
\definecolor{currentstroke}{rgb}{0.000000,0.000000,0.000000}%
\pgfsetstrokecolor{currentstroke}%
\pgfsetstrokeopacity{0.000000}%
\pgfsetdash{}{0pt}%
\pgfpathmoveto{\pgfqpoint{2.000000in}{1.973684in}}%
\pgfpathlineto{\pgfqpoint{6.376471in}{1.973684in}}%
\pgfpathlineto{\pgfqpoint{6.376471in}{2.952632in}}%
\pgfpathlineto{\pgfqpoint{2.000000in}{2.952632in}}%
\pgfpathclose%
\pgfusepath{fill}%
\end{pgfscope}%
\begin{pgfscope}%
\pgfpathrectangle{\pgfqpoint{2.000000in}{1.973684in}}{\pgfqpoint{4.376471in}{0.978947in}} %
\pgfusepath{clip}%
\pgfsetroundcap%
\pgfsetroundjoin%
\pgfsetlinewidth{1.003750pt}%
\definecolor{currentstroke}{rgb}{0.800000,0.800000,0.800000}%
\pgfsetstrokecolor{currentstroke}%
\pgfsetdash{}{0pt}%
\pgfpathmoveto{\pgfqpoint{2.000000in}{1.973684in}}%
\pgfpathlineto{\pgfqpoint{2.000000in}{2.952632in}}%
\pgfusepath{stroke}%
\end{pgfscope}%
\begin{pgfscope}%
\pgfpathrectangle{\pgfqpoint{2.000000in}{1.973684in}}{\pgfqpoint{4.376471in}{0.978947in}} %
\pgfusepath{clip}%
\pgfsetroundcap%
\pgfsetroundjoin%
\pgfsetlinewidth{1.003750pt}%
\definecolor{currentstroke}{rgb}{0.800000,0.800000,0.800000}%
\pgfsetstrokecolor{currentstroke}%
\pgfsetdash{}{0pt}%
\pgfpathmoveto{\pgfqpoint{2.875294in}{1.973684in}}%
\pgfpathlineto{\pgfqpoint{2.875294in}{2.952632in}}%
\pgfusepath{stroke}%
\end{pgfscope}%
\begin{pgfscope}%
\pgfpathrectangle{\pgfqpoint{2.000000in}{1.973684in}}{\pgfqpoint{4.376471in}{0.978947in}} %
\pgfusepath{clip}%
\pgfsetroundcap%
\pgfsetroundjoin%
\pgfsetlinewidth{1.003750pt}%
\definecolor{currentstroke}{rgb}{0.800000,0.800000,0.800000}%
\pgfsetstrokecolor{currentstroke}%
\pgfsetdash{}{0pt}%
\pgfpathmoveto{\pgfqpoint{3.750588in}{1.973684in}}%
\pgfpathlineto{\pgfqpoint{3.750588in}{2.952632in}}%
\pgfusepath{stroke}%
\end{pgfscope}%
\begin{pgfscope}%
\pgfpathrectangle{\pgfqpoint{2.000000in}{1.973684in}}{\pgfqpoint{4.376471in}{0.978947in}} %
\pgfusepath{clip}%
\pgfsetroundcap%
\pgfsetroundjoin%
\pgfsetlinewidth{1.003750pt}%
\definecolor{currentstroke}{rgb}{0.800000,0.800000,0.800000}%
\pgfsetstrokecolor{currentstroke}%
\pgfsetdash{}{0pt}%
\pgfpathmoveto{\pgfqpoint{4.625882in}{1.973684in}}%
\pgfpathlineto{\pgfqpoint{4.625882in}{2.952632in}}%
\pgfusepath{stroke}%
\end{pgfscope}%
\begin{pgfscope}%
\pgfpathrectangle{\pgfqpoint{2.000000in}{1.973684in}}{\pgfqpoint{4.376471in}{0.978947in}} %
\pgfusepath{clip}%
\pgfsetroundcap%
\pgfsetroundjoin%
\pgfsetlinewidth{1.003750pt}%
\definecolor{currentstroke}{rgb}{0.800000,0.800000,0.800000}%
\pgfsetstrokecolor{currentstroke}%
\pgfsetdash{}{0pt}%
\pgfpathmoveto{\pgfqpoint{5.501176in}{1.973684in}}%
\pgfpathlineto{\pgfqpoint{5.501176in}{2.952632in}}%
\pgfusepath{stroke}%
\end{pgfscope}%
\begin{pgfscope}%
\pgfpathrectangle{\pgfqpoint{2.000000in}{1.973684in}}{\pgfqpoint{4.376471in}{0.978947in}} %
\pgfusepath{clip}%
\pgfsetroundcap%
\pgfsetroundjoin%
\pgfsetlinewidth{1.003750pt}%
\definecolor{currentstroke}{rgb}{0.800000,0.800000,0.800000}%
\pgfsetstrokecolor{currentstroke}%
\pgfsetdash{}{0pt}%
\pgfpathmoveto{\pgfqpoint{6.376471in}{1.973684in}}%
\pgfpathlineto{\pgfqpoint{6.376471in}{2.952632in}}%
\pgfusepath{stroke}%
\end{pgfscope}%
\begin{pgfscope}%
\pgfpathrectangle{\pgfqpoint{2.000000in}{1.973684in}}{\pgfqpoint{4.376471in}{0.978947in}} %
\pgfusepath{clip}%
\pgfsetroundcap%
\pgfsetroundjoin%
\pgfsetlinewidth{1.003750pt}%
\definecolor{currentstroke}{rgb}{0.800000,0.800000,0.800000}%
\pgfsetstrokecolor{currentstroke}%
\pgfsetdash{}{0pt}%
\pgfpathmoveto{\pgfqpoint{2.000000in}{2.136842in}}%
\pgfpathlineto{\pgfqpoint{6.376471in}{2.136842in}}%
\pgfusepath{stroke}%
\end{pgfscope}%
\begin{pgfscope}%
\definecolor{textcolor}{rgb}{0.150000,0.150000,0.150000}%
\pgfsetstrokecolor{textcolor}%
\pgfsetfillcolor{textcolor}%
\pgftext[x=1.902778in,y=2.136842in,right,]{\color{textcolor}\sffamily\fontsize{10.000000}{12.000000}\selectfont \(\displaystyle -1\)}%
\end{pgfscope}%
\begin{pgfscope}%
\pgfpathrectangle{\pgfqpoint{2.000000in}{1.973684in}}{\pgfqpoint{4.376471in}{0.978947in}} %
\pgfusepath{clip}%
\pgfsetroundcap%
\pgfsetroundjoin%
\pgfsetlinewidth{1.003750pt}%
\definecolor{currentstroke}{rgb}{0.800000,0.800000,0.800000}%
\pgfsetstrokecolor{currentstroke}%
\pgfsetdash{}{0pt}%
\pgfpathmoveto{\pgfqpoint{2.000000in}{2.340789in}}%
\pgfpathlineto{\pgfqpoint{6.376471in}{2.340789in}}%
\pgfusepath{stroke}%
\end{pgfscope}%
\begin{pgfscope}%
\definecolor{textcolor}{rgb}{0.150000,0.150000,0.150000}%
\pgfsetstrokecolor{textcolor}%
\pgfsetfillcolor{textcolor}%
\pgftext[x=1.902778in,y=2.340789in,right,]{\color{textcolor}\sffamily\fontsize{10.000000}{12.000000}\selectfont \(\displaystyle 0\)}%
\end{pgfscope}%
\begin{pgfscope}%
\pgfpathrectangle{\pgfqpoint{2.000000in}{1.973684in}}{\pgfqpoint{4.376471in}{0.978947in}} %
\pgfusepath{clip}%
\pgfsetroundcap%
\pgfsetroundjoin%
\pgfsetlinewidth{1.003750pt}%
\definecolor{currentstroke}{rgb}{0.800000,0.800000,0.800000}%
\pgfsetstrokecolor{currentstroke}%
\pgfsetdash{}{0pt}%
\pgfpathmoveto{\pgfqpoint{2.000000in}{2.544737in}}%
\pgfpathlineto{\pgfqpoint{6.376471in}{2.544737in}}%
\pgfusepath{stroke}%
\end{pgfscope}%
\begin{pgfscope}%
\definecolor{textcolor}{rgb}{0.150000,0.150000,0.150000}%
\pgfsetstrokecolor{textcolor}%
\pgfsetfillcolor{textcolor}%
\pgftext[x=1.902778in,y=2.544737in,right,]{\color{textcolor}\sffamily\fontsize{10.000000}{12.000000}\selectfont \(\displaystyle 1\)}%
\end{pgfscope}%
\begin{pgfscope}%
\pgfpathrectangle{\pgfqpoint{2.000000in}{1.973684in}}{\pgfqpoint{4.376471in}{0.978947in}} %
\pgfusepath{clip}%
\pgfsetroundcap%
\pgfsetroundjoin%
\pgfsetlinewidth{1.003750pt}%
\definecolor{currentstroke}{rgb}{0.800000,0.800000,0.800000}%
\pgfsetstrokecolor{currentstroke}%
\pgfsetdash{}{0pt}%
\pgfpathmoveto{\pgfqpoint{2.000000in}{2.748684in}}%
\pgfpathlineto{\pgfqpoint{6.376471in}{2.748684in}}%
\pgfusepath{stroke}%
\end{pgfscope}%
\begin{pgfscope}%
\definecolor{textcolor}{rgb}{0.150000,0.150000,0.150000}%
\pgfsetstrokecolor{textcolor}%
\pgfsetfillcolor{textcolor}%
\pgftext[x=1.902778in,y=2.748684in,right,]{\color{textcolor}\sffamily\fontsize{10.000000}{12.000000}\selectfont \(\displaystyle 2\)}%
\end{pgfscope}%
\begin{pgfscope}%
\pgfpathrectangle{\pgfqpoint{2.000000in}{1.973684in}}{\pgfqpoint{4.376471in}{0.978947in}} %
\pgfusepath{clip}%
\pgfsetroundcap%
\pgfsetroundjoin%
\pgfsetlinewidth{1.003750pt}%
\definecolor{currentstroke}{rgb}{0.800000,0.800000,0.800000}%
\pgfsetstrokecolor{currentstroke}%
\pgfsetdash{}{0pt}%
\pgfpathmoveto{\pgfqpoint{2.000000in}{2.952632in}}%
\pgfpathlineto{\pgfqpoint{6.376471in}{2.952632in}}%
\pgfusepath{stroke}%
\end{pgfscope}%
\begin{pgfscope}%
\definecolor{textcolor}{rgb}{0.150000,0.150000,0.150000}%
\pgfsetstrokecolor{textcolor}%
\pgfsetfillcolor{textcolor}%
\pgftext[x=1.902778in,y=2.952632in,right,]{\color{textcolor}\sffamily\fontsize{10.000000}{12.000000}\selectfont \(\displaystyle 3\)}%
\end{pgfscope}%
\begin{pgfscope}%
\definecolor{textcolor}{rgb}{0.150000,0.150000,0.150000}%
\pgfsetstrokecolor{textcolor}%
\pgfsetfillcolor{textcolor}%
\pgftext[x=1.655864in,y=2.463158in,,bottom,rotate=90.000000]{\color{textcolor}\sffamily\fontsize{11.000000}{13.200000}\selectfont y}%
\end{pgfscope}%
\begin{pgfscope}%
\pgfpathrectangle{\pgfqpoint{2.000000in}{1.973684in}}{\pgfqpoint{4.376471in}{0.978947in}} %
\pgfusepath{clip}%
\pgfsetbuttcap%
\pgfsetroundjoin%
\definecolor{currentfill}{rgb}{1.000000,0.000000,0.000000}%
\pgfsetfillcolor{currentfill}%
\pgfsetlinewidth{2.007500pt}%
\definecolor{currentstroke}{rgb}{1.000000,0.000000,0.000000}%
\pgfsetstrokecolor{currentstroke}%
\pgfsetdash{}{0pt}%
\pgfpathmoveto{\pgfqpoint{4.765731in}{2.531260in}}%
\pgfpathlineto{\pgfqpoint{4.827844in}{2.531260in}}%
\pgfpathmoveto{\pgfqpoint{4.796787in}{2.500203in}}%
\pgfpathlineto{\pgfqpoint{4.796787in}{2.562316in}}%
\pgfusepath{stroke,fill}%
\end{pgfscope}%
\begin{pgfscope}%
\pgfpathrectangle{\pgfqpoint{2.000000in}{1.973684in}}{\pgfqpoint{4.376471in}{0.978947in}} %
\pgfusepath{clip}%
\pgfsetbuttcap%
\pgfsetroundjoin%
\definecolor{currentfill}{rgb}{1.000000,0.000000,0.000000}%
\pgfsetfillcolor{currentfill}%
\pgfsetlinewidth{2.007500pt}%
\definecolor{currentstroke}{rgb}{1.000000,0.000000,0.000000}%
\pgfsetstrokecolor{currentstroke}%
\pgfsetdash{}{0pt}%
\pgfpathmoveto{\pgfqpoint{5.348242in}{2.214048in}}%
\pgfpathlineto{\pgfqpoint{5.410355in}{2.214048in}}%
\pgfpathmoveto{\pgfqpoint{5.379298in}{2.182991in}}%
\pgfpathlineto{\pgfqpoint{5.379298in}{2.245104in}}%
\pgfusepath{stroke,fill}%
\end{pgfscope}%
\begin{pgfscope}%
\pgfpathrectangle{\pgfqpoint{2.000000in}{1.973684in}}{\pgfqpoint{4.376471in}{0.978947in}} %
\pgfusepath{clip}%
\pgfsetbuttcap%
\pgfsetroundjoin%
\definecolor{currentfill}{rgb}{1.000000,0.000000,0.000000}%
\pgfsetfillcolor{currentfill}%
\pgfsetlinewidth{2.007500pt}%
\definecolor{currentstroke}{rgb}{1.000000,0.000000,0.000000}%
\pgfsetstrokecolor{currentstroke}%
\pgfsetdash{}{0pt}%
\pgfpathmoveto{\pgfqpoint{4.954619in}{2.584788in}}%
\pgfpathlineto{\pgfqpoint{5.016732in}{2.584788in}}%
\pgfpathmoveto{\pgfqpoint{4.985675in}{2.553731in}}%
\pgfpathlineto{\pgfqpoint{4.985675in}{2.615844in}}%
\pgfusepath{stroke,fill}%
\end{pgfscope}%
\begin{pgfscope}%
\pgfpathrectangle{\pgfqpoint{2.000000in}{1.973684in}}{\pgfqpoint{4.376471in}{0.978947in}} %
\pgfusepath{clip}%
\pgfsetbuttcap%
\pgfsetroundjoin%
\definecolor{currentfill}{rgb}{1.000000,0.000000,0.000000}%
\pgfsetfillcolor{currentfill}%
\pgfsetlinewidth{2.007500pt}%
\definecolor{currentstroke}{rgb}{1.000000,0.000000,0.000000}%
\pgfsetstrokecolor{currentstroke}%
\pgfsetdash{}{0pt}%
\pgfpathmoveto{\pgfqpoint{4.751970in}{2.473876in}}%
\pgfpathlineto{\pgfqpoint{4.814083in}{2.473876in}}%
\pgfpathmoveto{\pgfqpoint{4.783026in}{2.442819in}}%
\pgfpathlineto{\pgfqpoint{4.783026in}{2.504932in}}%
\pgfusepath{stroke,fill}%
\end{pgfscope}%
\begin{pgfscope}%
\pgfpathrectangle{\pgfqpoint{2.000000in}{1.973684in}}{\pgfqpoint{4.376471in}{0.978947in}} %
\pgfusepath{clip}%
\pgfsetbuttcap%
\pgfsetroundjoin%
\definecolor{currentfill}{rgb}{1.000000,0.000000,0.000000}%
\pgfsetfillcolor{currentfill}%
\pgfsetlinewidth{2.007500pt}%
\definecolor{currentstroke}{rgb}{1.000000,0.000000,0.000000}%
\pgfsetstrokecolor{currentstroke}%
\pgfsetdash{}{0pt}%
\pgfpathmoveto{\pgfqpoint{4.327528in}{2.147482in}}%
\pgfpathlineto{\pgfqpoint{4.389641in}{2.147482in}}%
\pgfpathmoveto{\pgfqpoint{4.358584in}{2.116425in}}%
\pgfpathlineto{\pgfqpoint{4.358584in}{2.178538in}}%
\pgfusepath{stroke,fill}%
\end{pgfscope}%
\begin{pgfscope}%
\pgfpathrectangle{\pgfqpoint{2.000000in}{1.973684in}}{\pgfqpoint{4.376471in}{0.978947in}} %
\pgfusepath{clip}%
\pgfsetbuttcap%
\pgfsetroundjoin%
\definecolor{currentfill}{rgb}{1.000000,0.000000,0.000000}%
\pgfsetfillcolor{currentfill}%
\pgfsetlinewidth{2.007500pt}%
\definecolor{currentstroke}{rgb}{1.000000,0.000000,0.000000}%
\pgfsetstrokecolor{currentstroke}%
\pgfsetdash{}{0pt}%
\pgfpathmoveto{\pgfqpoint{5.105627in}{2.416946in}}%
\pgfpathlineto{\pgfqpoint{5.167740in}{2.416946in}}%
\pgfpathmoveto{\pgfqpoint{5.136683in}{2.385889in}}%
\pgfpathlineto{\pgfqpoint{5.136683in}{2.448002in}}%
\pgfusepath{stroke,fill}%
\end{pgfscope}%
\begin{pgfscope}%
\pgfpathrectangle{\pgfqpoint{2.000000in}{1.973684in}}{\pgfqpoint{4.376471in}{0.978947in}} %
\pgfusepath{clip}%
\pgfsetbuttcap%
\pgfsetroundjoin%
\definecolor{currentfill}{rgb}{1.000000,0.000000,0.000000}%
\pgfsetfillcolor{currentfill}%
\pgfsetlinewidth{2.007500pt}%
\definecolor{currentstroke}{rgb}{1.000000,0.000000,0.000000}%
\pgfsetstrokecolor{currentstroke}%
\pgfsetdash{}{0pt}%
\pgfpathmoveto{\pgfqpoint{4.376308in}{2.184975in}}%
\pgfpathlineto{\pgfqpoint{4.438421in}{2.184975in}}%
\pgfpathmoveto{\pgfqpoint{4.407364in}{2.153919in}}%
\pgfpathlineto{\pgfqpoint{4.407364in}{2.216032in}}%
\pgfusepath{stroke,fill}%
\end{pgfscope}%
\begin{pgfscope}%
\pgfpathrectangle{\pgfqpoint{2.000000in}{1.973684in}}{\pgfqpoint{4.376471in}{0.978947in}} %
\pgfusepath{clip}%
\pgfsetbuttcap%
\pgfsetroundjoin%
\definecolor{currentfill}{rgb}{1.000000,0.000000,0.000000}%
\pgfsetfillcolor{currentfill}%
\pgfsetlinewidth{2.007500pt}%
\definecolor{currentstroke}{rgb}{1.000000,0.000000,0.000000}%
\pgfsetstrokecolor{currentstroke}%
\pgfsetdash{}{0pt}%
\pgfpathmoveto{\pgfqpoint{5.966492in}{2.825094in}}%
\pgfpathlineto{\pgfqpoint{6.028605in}{2.825094in}}%
\pgfpathmoveto{\pgfqpoint{5.997549in}{2.794037in}}%
\pgfpathlineto{\pgfqpoint{5.997549in}{2.856150in}}%
\pgfusepath{stroke,fill}%
\end{pgfscope}%
\begin{pgfscope}%
\pgfpathrectangle{\pgfqpoint{2.000000in}{1.973684in}}{\pgfqpoint{4.376471in}{0.978947in}} %
\pgfusepath{clip}%
\pgfsetbuttcap%
\pgfsetroundjoin%
\definecolor{currentfill}{rgb}{1.000000,0.000000,0.000000}%
\pgfsetfillcolor{currentfill}%
\pgfsetlinewidth{2.007500pt}%
\definecolor{currentstroke}{rgb}{1.000000,0.000000,0.000000}%
\pgfsetstrokecolor{currentstroke}%
\pgfsetdash{}{0pt}%
\pgfpathmoveto{\pgfqpoint{6.218191in}{2.725496in}}%
\pgfpathlineto{\pgfqpoint{6.280304in}{2.725496in}}%
\pgfpathmoveto{\pgfqpoint{6.249248in}{2.694440in}}%
\pgfpathlineto{\pgfqpoint{6.249248in}{2.756553in}}%
\pgfusepath{stroke,fill}%
\end{pgfscope}%
\begin{pgfscope}%
\pgfpathrectangle{\pgfqpoint{2.000000in}{1.973684in}}{\pgfqpoint{4.376471in}{0.978947in}} %
\pgfusepath{clip}%
\pgfsetbuttcap%
\pgfsetroundjoin%
\definecolor{currentfill}{rgb}{1.000000,0.000000,0.000000}%
\pgfsetfillcolor{currentfill}%
\pgfsetlinewidth{2.007500pt}%
\definecolor{currentstroke}{rgb}{1.000000,0.000000,0.000000}%
\pgfsetstrokecolor{currentstroke}%
\pgfsetdash{}{0pt}%
\pgfpathmoveto{\pgfqpoint{4.186734in}{2.222892in}}%
\pgfpathlineto{\pgfqpoint{4.248847in}{2.222892in}}%
\pgfpathmoveto{\pgfqpoint{4.217791in}{2.191836in}}%
\pgfpathlineto{\pgfqpoint{4.217791in}{2.253949in}}%
\pgfusepath{stroke,fill}%
\end{pgfscope}%
\begin{pgfscope}%
\pgfpathrectangle{\pgfqpoint{2.000000in}{1.973684in}}{\pgfqpoint{4.376471in}{0.978947in}} %
\pgfusepath{clip}%
\pgfsetbuttcap%
\pgfsetroundjoin%
\definecolor{currentfill}{rgb}{1.000000,0.000000,0.000000}%
\pgfsetfillcolor{currentfill}%
\pgfsetlinewidth{2.007500pt}%
\definecolor{currentstroke}{rgb}{1.000000,0.000000,0.000000}%
\pgfsetstrokecolor{currentstroke}%
\pgfsetdash{}{0pt}%
\pgfpathmoveto{\pgfqpoint{5.616207in}{2.373068in}}%
\pgfpathlineto{\pgfqpoint{5.678320in}{2.373068in}}%
\pgfpathmoveto{\pgfqpoint{5.647263in}{2.342011in}}%
\pgfpathlineto{\pgfqpoint{5.647263in}{2.404124in}}%
\pgfusepath{stroke,fill}%
\end{pgfscope}%
\begin{pgfscope}%
\pgfpathrectangle{\pgfqpoint{2.000000in}{1.973684in}}{\pgfqpoint{4.376471in}{0.978947in}} %
\pgfusepath{clip}%
\pgfsetbuttcap%
\pgfsetroundjoin%
\definecolor{currentfill}{rgb}{1.000000,0.000000,0.000000}%
\pgfsetfillcolor{currentfill}%
\pgfsetlinewidth{2.007500pt}%
\definecolor{currentstroke}{rgb}{1.000000,0.000000,0.000000}%
\pgfsetstrokecolor{currentstroke}%
\pgfsetdash{}{0pt}%
\pgfpathmoveto{\pgfqpoint{4.695992in}{2.413163in}}%
\pgfpathlineto{\pgfqpoint{4.758105in}{2.413163in}}%
\pgfpathmoveto{\pgfqpoint{4.727049in}{2.382107in}}%
\pgfpathlineto{\pgfqpoint{4.727049in}{2.444220in}}%
\pgfusepath{stroke,fill}%
\end{pgfscope}%
\begin{pgfscope}%
\pgfpathrectangle{\pgfqpoint{2.000000in}{1.973684in}}{\pgfqpoint{4.376471in}{0.978947in}} %
\pgfusepath{clip}%
\pgfsetbuttcap%
\pgfsetroundjoin%
\definecolor{currentfill}{rgb}{1.000000,0.000000,0.000000}%
\pgfsetfillcolor{currentfill}%
\pgfsetlinewidth{2.007500pt}%
\definecolor{currentstroke}{rgb}{1.000000,0.000000,0.000000}%
\pgfsetstrokecolor{currentstroke}%
\pgfsetdash{}{0pt}%
\pgfpathmoveto{\pgfqpoint{4.833062in}{2.540719in}}%
\pgfpathlineto{\pgfqpoint{4.895175in}{2.540719in}}%
\pgfpathmoveto{\pgfqpoint{4.864118in}{2.509663in}}%
\pgfpathlineto{\pgfqpoint{4.864118in}{2.571776in}}%
\pgfusepath{stroke,fill}%
\end{pgfscope}%
\begin{pgfscope}%
\pgfpathrectangle{\pgfqpoint{2.000000in}{1.973684in}}{\pgfqpoint{4.376471in}{0.978947in}} %
\pgfusepath{clip}%
\pgfsetbuttcap%
\pgfsetroundjoin%
\definecolor{currentfill}{rgb}{1.000000,0.000000,0.000000}%
\pgfsetfillcolor{currentfill}%
\pgfsetlinewidth{2.007500pt}%
\definecolor{currentstroke}{rgb}{1.000000,0.000000,0.000000}%
\pgfsetstrokecolor{currentstroke}%
\pgfsetdash{}{0pt}%
\pgfpathmoveto{\pgfqpoint{6.084915in}{2.800883in}}%
\pgfpathlineto{\pgfqpoint{6.147028in}{2.800883in}}%
\pgfpathmoveto{\pgfqpoint{6.115971in}{2.769826in}}%
\pgfpathlineto{\pgfqpoint{6.115971in}{2.831939in}}%
\pgfusepath{stroke,fill}%
\end{pgfscope}%
\begin{pgfscope}%
\pgfpathrectangle{\pgfqpoint{2.000000in}{1.973684in}}{\pgfqpoint{4.376471in}{0.978947in}} %
\pgfusepath{clip}%
\pgfsetbuttcap%
\pgfsetroundjoin%
\definecolor{currentfill}{rgb}{1.000000,0.000000,0.000000}%
\pgfsetfillcolor{currentfill}%
\pgfsetlinewidth{2.007500pt}%
\definecolor{currentstroke}{rgb}{1.000000,0.000000,0.000000}%
\pgfsetstrokecolor{currentstroke}%
\pgfsetdash{}{0pt}%
\pgfpathmoveto{\pgfqpoint{3.092947in}{2.513494in}}%
\pgfpathlineto{\pgfqpoint{3.155060in}{2.513494in}}%
\pgfpathmoveto{\pgfqpoint{3.124004in}{2.482437in}}%
\pgfpathlineto{\pgfqpoint{3.124004in}{2.544550in}}%
\pgfusepath{stroke,fill}%
\end{pgfscope}%
\begin{pgfscope}%
\pgfpathrectangle{\pgfqpoint{2.000000in}{1.973684in}}{\pgfqpoint{4.376471in}{0.978947in}} %
\pgfusepath{clip}%
\pgfsetbuttcap%
\pgfsetroundjoin%
\definecolor{currentfill}{rgb}{1.000000,0.000000,0.000000}%
\pgfsetfillcolor{currentfill}%
\pgfsetlinewidth{2.007500pt}%
\definecolor{currentstroke}{rgb}{1.000000,0.000000,0.000000}%
\pgfsetstrokecolor{currentstroke}%
\pgfsetdash{}{0pt}%
\pgfpathmoveto{\pgfqpoint{3.149293in}{2.455860in}}%
\pgfpathlineto{\pgfqpoint{3.211406in}{2.455860in}}%
\pgfpathmoveto{\pgfqpoint{3.180349in}{2.424803in}}%
\pgfpathlineto{\pgfqpoint{3.180349in}{2.486916in}}%
\pgfusepath{stroke,fill}%
\end{pgfscope}%
\begin{pgfscope}%
\pgfpathrectangle{\pgfqpoint{2.000000in}{1.973684in}}{\pgfqpoint{4.376471in}{0.978947in}} %
\pgfusepath{clip}%
\pgfsetbuttcap%
\pgfsetroundjoin%
\definecolor{currentfill}{rgb}{1.000000,0.000000,0.000000}%
\pgfsetfillcolor{currentfill}%
\pgfsetlinewidth{2.007500pt}%
\definecolor{currentstroke}{rgb}{1.000000,0.000000,0.000000}%
\pgfsetstrokecolor{currentstroke}%
\pgfsetdash{}{0pt}%
\pgfpathmoveto{\pgfqpoint{2.915026in}{2.743140in}}%
\pgfpathlineto{\pgfqpoint{2.977139in}{2.743140in}}%
\pgfpathmoveto{\pgfqpoint{2.946082in}{2.712083in}}%
\pgfpathlineto{\pgfqpoint{2.946082in}{2.774196in}}%
\pgfusepath{stroke,fill}%
\end{pgfscope}%
\begin{pgfscope}%
\pgfpathrectangle{\pgfqpoint{2.000000in}{1.973684in}}{\pgfqpoint{4.376471in}{0.978947in}} %
\pgfusepath{clip}%
\pgfsetbuttcap%
\pgfsetroundjoin%
\definecolor{currentfill}{rgb}{1.000000,0.000000,0.000000}%
\pgfsetfillcolor{currentfill}%
\pgfsetlinewidth{2.007500pt}%
\definecolor{currentstroke}{rgb}{1.000000,0.000000,0.000000}%
\pgfsetstrokecolor{currentstroke}%
\pgfsetdash{}{0pt}%
\pgfpathmoveto{\pgfqpoint{5.759387in}{2.588736in}}%
\pgfpathlineto{\pgfqpoint{5.821500in}{2.588736in}}%
\pgfpathmoveto{\pgfqpoint{5.790443in}{2.557679in}}%
\pgfpathlineto{\pgfqpoint{5.790443in}{2.619792in}}%
\pgfusepath{stroke,fill}%
\end{pgfscope}%
\begin{pgfscope}%
\pgfpathrectangle{\pgfqpoint{2.000000in}{1.973684in}}{\pgfqpoint{4.376471in}{0.978947in}} %
\pgfusepath{clip}%
\pgfsetbuttcap%
\pgfsetroundjoin%
\definecolor{currentfill}{rgb}{1.000000,0.000000,0.000000}%
\pgfsetfillcolor{currentfill}%
\pgfsetlinewidth{2.007500pt}%
\definecolor{currentstroke}{rgb}{1.000000,0.000000,0.000000}%
\pgfsetstrokecolor{currentstroke}%
\pgfsetdash{}{0pt}%
\pgfpathmoveto{\pgfqpoint{5.568702in}{2.311154in}}%
\pgfpathlineto{\pgfqpoint{5.630815in}{2.311154in}}%
\pgfpathmoveto{\pgfqpoint{5.599758in}{2.280098in}}%
\pgfpathlineto{\pgfqpoint{5.599758in}{2.342211in}}%
\pgfusepath{stroke,fill}%
\end{pgfscope}%
\begin{pgfscope}%
\pgfpathrectangle{\pgfqpoint{2.000000in}{1.973684in}}{\pgfqpoint{4.376471in}{0.978947in}} %
\pgfusepath{clip}%
\pgfsetbuttcap%
\pgfsetroundjoin%
\definecolor{currentfill}{rgb}{1.000000,0.000000,0.000000}%
\pgfsetfillcolor{currentfill}%
\pgfsetlinewidth{2.007500pt}%
\definecolor{currentstroke}{rgb}{1.000000,0.000000,0.000000}%
\pgfsetstrokecolor{currentstroke}%
\pgfsetdash{}{0pt}%
\pgfpathmoveto{\pgfqpoint{5.890304in}{2.718517in}}%
\pgfpathlineto{\pgfqpoint{5.952417in}{2.718517in}}%
\pgfpathmoveto{\pgfqpoint{5.921360in}{2.687461in}}%
\pgfpathlineto{\pgfqpoint{5.921360in}{2.749574in}}%
\pgfusepath{stroke,fill}%
\end{pgfscope}%
\begin{pgfscope}%
\pgfpathrectangle{\pgfqpoint{2.000000in}{1.973684in}}{\pgfqpoint{4.376471in}{0.978947in}} %
\pgfusepath{clip}%
\pgfsetbuttcap%
\pgfsetroundjoin%
\definecolor{currentfill}{rgb}{1.000000,0.000000,0.000000}%
\pgfsetfillcolor{currentfill}%
\pgfsetlinewidth{2.007500pt}%
\definecolor{currentstroke}{rgb}{1.000000,0.000000,0.000000}%
\pgfsetstrokecolor{currentstroke}%
\pgfsetdash{}{0pt}%
\pgfpathmoveto{\pgfqpoint{6.270553in}{2.649754in}}%
\pgfpathlineto{\pgfqpoint{6.332666in}{2.649754in}}%
\pgfpathmoveto{\pgfqpoint{6.301610in}{2.618698in}}%
\pgfpathlineto{\pgfqpoint{6.301610in}{2.680811in}}%
\pgfusepath{stroke,fill}%
\end{pgfscope}%
\begin{pgfscope}%
\pgfpathrectangle{\pgfqpoint{2.000000in}{1.973684in}}{\pgfqpoint{4.376471in}{0.978947in}} %
\pgfusepath{clip}%
\pgfsetbuttcap%
\pgfsetroundjoin%
\definecolor{currentfill}{rgb}{1.000000,0.000000,0.000000}%
\pgfsetfillcolor{currentfill}%
\pgfsetlinewidth{2.007500pt}%
\definecolor{currentstroke}{rgb}{1.000000,0.000000,0.000000}%
\pgfsetstrokecolor{currentstroke}%
\pgfsetdash{}{0pt}%
\pgfpathmoveto{\pgfqpoint{5.642233in}{2.466321in}}%
\pgfpathlineto{\pgfqpoint{5.704346in}{2.466321in}}%
\pgfpathmoveto{\pgfqpoint{5.673289in}{2.435265in}}%
\pgfpathlineto{\pgfqpoint{5.673289in}{2.497378in}}%
\pgfusepath{stroke,fill}%
\end{pgfscope}%
\begin{pgfscope}%
\pgfpathrectangle{\pgfqpoint{2.000000in}{1.973684in}}{\pgfqpoint{4.376471in}{0.978947in}} %
\pgfusepath{clip}%
\pgfsetbuttcap%
\pgfsetroundjoin%
\definecolor{currentfill}{rgb}{1.000000,0.000000,0.000000}%
\pgfsetfillcolor{currentfill}%
\pgfsetlinewidth{2.007500pt}%
\definecolor{currentstroke}{rgb}{1.000000,0.000000,0.000000}%
\pgfsetstrokecolor{currentstroke}%
\pgfsetdash{}{0pt}%
\pgfpathmoveto{\pgfqpoint{4.459958in}{2.190781in}}%
\pgfpathlineto{\pgfqpoint{4.522071in}{2.190781in}}%
\pgfpathmoveto{\pgfqpoint{4.491015in}{2.159724in}}%
\pgfpathlineto{\pgfqpoint{4.491015in}{2.221837in}}%
\pgfusepath{stroke,fill}%
\end{pgfscope}%
\begin{pgfscope}%
\pgfpathrectangle{\pgfqpoint{2.000000in}{1.973684in}}{\pgfqpoint{4.376471in}{0.978947in}} %
\pgfusepath{clip}%
\pgfsetbuttcap%
\pgfsetroundjoin%
\definecolor{currentfill}{rgb}{1.000000,0.000000,0.000000}%
\pgfsetfillcolor{currentfill}%
\pgfsetlinewidth{2.007500pt}%
\definecolor{currentstroke}{rgb}{1.000000,0.000000,0.000000}%
\pgfsetstrokecolor{currentstroke}%
\pgfsetdash{}{0pt}%
\pgfpathmoveto{\pgfqpoint{5.577008in}{2.333191in}}%
\pgfpathlineto{\pgfqpoint{5.639121in}{2.333191in}}%
\pgfpathmoveto{\pgfqpoint{5.608065in}{2.302134in}}%
\pgfpathlineto{\pgfqpoint{5.608065in}{2.364247in}}%
\pgfusepath{stroke,fill}%
\end{pgfscope}%
\begin{pgfscope}%
\pgfpathrectangle{\pgfqpoint{2.000000in}{1.973684in}}{\pgfqpoint{4.376471in}{0.978947in}} %
\pgfusepath{clip}%
\pgfsetbuttcap%
\pgfsetroundjoin%
\definecolor{currentfill}{rgb}{1.000000,0.000000,0.000000}%
\pgfsetfillcolor{currentfill}%
\pgfsetlinewidth{2.007500pt}%
\definecolor{currentstroke}{rgb}{1.000000,0.000000,0.000000}%
\pgfsetstrokecolor{currentstroke}%
\pgfsetdash{}{0pt}%
\pgfpathmoveto{\pgfqpoint{3.258337in}{2.353928in}}%
\pgfpathlineto{\pgfqpoint{3.320450in}{2.353928in}}%
\pgfpathmoveto{\pgfqpoint{3.289394in}{2.322872in}}%
\pgfpathlineto{\pgfqpoint{3.289394in}{2.384985in}}%
\pgfusepath{stroke,fill}%
\end{pgfscope}%
\begin{pgfscope}%
\pgfpathrectangle{\pgfqpoint{2.000000in}{1.973684in}}{\pgfqpoint{4.376471in}{0.978947in}} %
\pgfusepath{clip}%
\pgfsetbuttcap%
\pgfsetroundjoin%
\definecolor{currentfill}{rgb}{1.000000,0.000000,0.000000}%
\pgfsetfillcolor{currentfill}%
\pgfsetlinewidth{2.007500pt}%
\definecolor{currentstroke}{rgb}{1.000000,0.000000,0.000000}%
\pgfsetstrokecolor{currentstroke}%
\pgfsetdash{}{0pt}%
\pgfpathmoveto{\pgfqpoint{5.084714in}{2.457212in}}%
\pgfpathlineto{\pgfqpoint{5.146827in}{2.457212in}}%
\pgfpathmoveto{\pgfqpoint{5.115771in}{2.426156in}}%
\pgfpathlineto{\pgfqpoint{5.115771in}{2.488269in}}%
\pgfusepath{stroke,fill}%
\end{pgfscope}%
\begin{pgfscope}%
\pgfpathrectangle{\pgfqpoint{2.000000in}{1.973684in}}{\pgfqpoint{4.376471in}{0.978947in}} %
\pgfusepath{clip}%
\pgfsetbuttcap%
\pgfsetroundjoin%
\definecolor{currentfill}{rgb}{1.000000,0.000000,0.000000}%
\pgfsetfillcolor{currentfill}%
\pgfsetlinewidth{2.007500pt}%
\definecolor{currentstroke}{rgb}{1.000000,0.000000,0.000000}%
\pgfsetstrokecolor{currentstroke}%
\pgfsetdash{}{0pt}%
\pgfpathmoveto{\pgfqpoint{3.346143in}{2.362218in}}%
\pgfpathlineto{\pgfqpoint{3.408256in}{2.362218in}}%
\pgfpathmoveto{\pgfqpoint{3.377199in}{2.331162in}}%
\pgfpathlineto{\pgfqpoint{3.377199in}{2.393275in}}%
\pgfusepath{stroke,fill}%
\end{pgfscope}%
\begin{pgfscope}%
\pgfpathrectangle{\pgfqpoint{2.000000in}{1.973684in}}{\pgfqpoint{4.376471in}{0.978947in}} %
\pgfusepath{clip}%
\pgfsetbuttcap%
\pgfsetroundjoin%
\definecolor{currentfill}{rgb}{1.000000,0.000000,0.000000}%
\pgfsetfillcolor{currentfill}%
\pgfsetlinewidth{2.007500pt}%
\definecolor{currentstroke}{rgb}{1.000000,0.000000,0.000000}%
\pgfsetstrokecolor{currentstroke}%
\pgfsetdash{}{0pt}%
\pgfpathmoveto{\pgfqpoint{6.151690in}{2.762412in}}%
\pgfpathlineto{\pgfqpoint{6.213803in}{2.762412in}}%
\pgfpathmoveto{\pgfqpoint{6.182747in}{2.731355in}}%
\pgfpathlineto{\pgfqpoint{6.182747in}{2.793468in}}%
\pgfusepath{stroke,fill}%
\end{pgfscope}%
\begin{pgfscope}%
\pgfpathrectangle{\pgfqpoint{2.000000in}{1.973684in}}{\pgfqpoint{4.376471in}{0.978947in}} %
\pgfusepath{clip}%
\pgfsetbuttcap%
\pgfsetroundjoin%
\definecolor{currentfill}{rgb}{1.000000,0.000000,0.000000}%
\pgfsetfillcolor{currentfill}%
\pgfsetlinewidth{2.007500pt}%
\definecolor{currentstroke}{rgb}{1.000000,0.000000,0.000000}%
\pgfsetstrokecolor{currentstroke}%
\pgfsetdash{}{0pt}%
\pgfpathmoveto{\pgfqpoint{4.671321in}{2.409615in}}%
\pgfpathlineto{\pgfqpoint{4.733434in}{2.409615in}}%
\pgfpathmoveto{\pgfqpoint{4.702377in}{2.378558in}}%
\pgfpathlineto{\pgfqpoint{4.702377in}{2.440671in}}%
\pgfusepath{stroke,fill}%
\end{pgfscope}%
\begin{pgfscope}%
\pgfpathrectangle{\pgfqpoint{2.000000in}{1.973684in}}{\pgfqpoint{4.376471in}{0.978947in}} %
\pgfusepath{clip}%
\pgfsetbuttcap%
\pgfsetroundjoin%
\definecolor{currentfill}{rgb}{1.000000,0.000000,0.000000}%
\pgfsetfillcolor{currentfill}%
\pgfsetlinewidth{2.007500pt}%
\definecolor{currentstroke}{rgb}{1.000000,0.000000,0.000000}%
\pgfsetstrokecolor{currentstroke}%
\pgfsetdash{}{0pt}%
\pgfpathmoveto{\pgfqpoint{4.296042in}{2.158497in}}%
\pgfpathlineto{\pgfqpoint{4.358155in}{2.158497in}}%
\pgfpathmoveto{\pgfqpoint{4.327099in}{2.127441in}}%
\pgfpathlineto{\pgfqpoint{4.327099in}{2.189554in}}%
\pgfusepath{stroke,fill}%
\end{pgfscope}%
\begin{pgfscope}%
\pgfpathrectangle{\pgfqpoint{2.000000in}{1.973684in}}{\pgfqpoint{4.376471in}{0.978947in}} %
\pgfusepath{clip}%
\pgfsetbuttcap%
\pgfsetroundjoin%
\definecolor{currentfill}{rgb}{0.000000,0.000000,0.000000}%
\pgfsetfillcolor{currentfill}%
\pgfsetlinewidth{0.301125pt}%
\definecolor{currentstroke}{rgb}{0.000000,0.000000,0.000000}%
\pgfsetstrokecolor{currentstroke}%
\pgfsetdash{}{0pt}%
\pgfsys@defobject{currentmarker}{\pgfqpoint{-0.015528in}{-0.015528in}}{\pgfqpoint{0.015528in}{0.015528in}}{%
\pgfpathmoveto{\pgfqpoint{0.000000in}{-0.015528in}}%
\pgfpathcurveto{\pgfqpoint{0.004118in}{-0.015528in}}{\pgfqpoint{0.008068in}{-0.013892in}}{\pgfqpoint{0.010980in}{-0.010980in}}%
\pgfpathcurveto{\pgfqpoint{0.013892in}{-0.008068in}}{\pgfqpoint{0.015528in}{-0.004118in}}{\pgfqpoint{0.015528in}{0.000000in}}%
\pgfpathcurveto{\pgfqpoint{0.015528in}{0.004118in}}{\pgfqpoint{0.013892in}{0.008068in}}{\pgfqpoint{0.010980in}{0.010980in}}%
\pgfpathcurveto{\pgfqpoint{0.008068in}{0.013892in}}{\pgfqpoint{0.004118in}{0.015528in}}{\pgfqpoint{0.000000in}{0.015528in}}%
\pgfpathcurveto{\pgfqpoint{-0.004118in}{0.015528in}}{\pgfqpoint{-0.008068in}{0.013892in}}{\pgfqpoint{-0.010980in}{0.010980in}}%
\pgfpathcurveto{\pgfqpoint{-0.013892in}{0.008068in}}{\pgfqpoint{-0.015528in}{0.004118in}}{\pgfqpoint{-0.015528in}{0.000000in}}%
\pgfpathcurveto{\pgfqpoint{-0.015528in}{-0.004118in}}{\pgfqpoint{-0.013892in}{-0.008068in}}{\pgfqpoint{-0.010980in}{-0.010980in}}%
\pgfpathcurveto{\pgfqpoint{-0.008068in}{-0.013892in}}{\pgfqpoint{-0.004118in}{-0.015528in}}{\pgfqpoint{0.000000in}{-0.015528in}}%
\pgfpathclose%
\pgfusepath{stroke,fill}%
}%
\begin{pgfscope}%
\pgfsys@transformshift{2.875294in}{2.807547in}%
\pgfsys@useobject{currentmarker}{}%
\end{pgfscope}%
\begin{pgfscope}%
\pgfsys@transformshift{2.892888in}{2.713355in}%
\pgfsys@useobject{currentmarker}{}%
\end{pgfscope}%
\begin{pgfscope}%
\pgfsys@transformshift{2.910482in}{2.804129in}%
\pgfsys@useobject{currentmarker}{}%
\end{pgfscope}%
\begin{pgfscope}%
\pgfsys@transformshift{2.928076in}{2.823266in}%
\pgfsys@useobject{currentmarker}{}%
\end{pgfscope}%
\begin{pgfscope}%
\pgfsys@transformshift{2.945670in}{2.758481in}%
\pgfsys@useobject{currentmarker}{}%
\end{pgfscope}%
\begin{pgfscope}%
\pgfsys@transformshift{2.963263in}{2.754380in}%
\pgfsys@useobject{currentmarker}{}%
\end{pgfscope}%
\begin{pgfscope}%
\pgfsys@transformshift{2.980857in}{2.630622in}%
\pgfsys@useobject{currentmarker}{}%
\end{pgfscope}%
\begin{pgfscope}%
\pgfsys@transformshift{2.998451in}{2.630227in}%
\pgfsys@useobject{currentmarker}{}%
\end{pgfscope}%
\begin{pgfscope}%
\pgfsys@transformshift{3.016045in}{2.570900in}%
\pgfsys@useobject{currentmarker}{}%
\end{pgfscope}%
\begin{pgfscope}%
\pgfsys@transformshift{3.033639in}{2.575614in}%
\pgfsys@useobject{currentmarker}{}%
\end{pgfscope}%
\begin{pgfscope}%
\pgfsys@transformshift{3.051233in}{2.502913in}%
\pgfsys@useobject{currentmarker}{}%
\end{pgfscope}%
\begin{pgfscope}%
\pgfsys@transformshift{3.068826in}{2.384290in}%
\pgfsys@useobject{currentmarker}{}%
\end{pgfscope}%
\begin{pgfscope}%
\pgfsys@transformshift{3.086420in}{2.554036in}%
\pgfsys@useobject{currentmarker}{}%
\end{pgfscope}%
\begin{pgfscope}%
\pgfsys@transformshift{3.104014in}{2.471885in}%
\pgfsys@useobject{currentmarker}{}%
\end{pgfscope}%
\begin{pgfscope}%
\pgfsys@transformshift{3.121608in}{2.324989in}%
\pgfsys@useobject{currentmarker}{}%
\end{pgfscope}%
\begin{pgfscope}%
\pgfsys@transformshift{3.139202in}{2.518392in}%
\pgfsys@useobject{currentmarker}{}%
\end{pgfscope}%
\begin{pgfscope}%
\pgfsys@transformshift{3.156796in}{2.360390in}%
\pgfsys@useobject{currentmarker}{}%
\end{pgfscope}%
\begin{pgfscope}%
\pgfsys@transformshift{3.174390in}{2.441767in}%
\pgfsys@useobject{currentmarker}{}%
\end{pgfscope}%
\begin{pgfscope}%
\pgfsys@transformshift{3.191983in}{2.496324in}%
\pgfsys@useobject{currentmarker}{}%
\end{pgfscope}%
\begin{pgfscope}%
\pgfsys@transformshift{3.209577in}{2.422660in}%
\pgfsys@useobject{currentmarker}{}%
\end{pgfscope}%
\begin{pgfscope}%
\pgfsys@transformshift{3.227171in}{2.515310in}%
\pgfsys@useobject{currentmarker}{}%
\end{pgfscope}%
\begin{pgfscope}%
\pgfsys@transformshift{3.244765in}{2.264938in}%
\pgfsys@useobject{currentmarker}{}%
\end{pgfscope}%
\begin{pgfscope}%
\pgfsys@transformshift{3.262359in}{2.425762in}%
\pgfsys@useobject{currentmarker}{}%
\end{pgfscope}%
\begin{pgfscope}%
\pgfsys@transformshift{3.279953in}{2.310913in}%
\pgfsys@useobject{currentmarker}{}%
\end{pgfscope}%
\begin{pgfscope}%
\pgfsys@transformshift{3.297547in}{2.290072in}%
\pgfsys@useobject{currentmarker}{}%
\end{pgfscope}%
\begin{pgfscope}%
\pgfsys@transformshift{3.315140in}{2.320035in}%
\pgfsys@useobject{currentmarker}{}%
\end{pgfscope}%
\begin{pgfscope}%
\pgfsys@transformshift{3.332734in}{2.349489in}%
\pgfsys@useobject{currentmarker}{}%
\end{pgfscope}%
\begin{pgfscope}%
\pgfsys@transformshift{3.350328in}{2.391104in}%
\pgfsys@useobject{currentmarker}{}%
\end{pgfscope}%
\begin{pgfscope}%
\pgfsys@transformshift{3.367922in}{2.272497in}%
\pgfsys@useobject{currentmarker}{}%
\end{pgfscope}%
\begin{pgfscope}%
\pgfsys@transformshift{3.385516in}{2.490806in}%
\pgfsys@useobject{currentmarker}{}%
\end{pgfscope}%
\begin{pgfscope}%
\pgfsys@transformshift{3.403110in}{2.455626in}%
\pgfsys@useobject{currentmarker}{}%
\end{pgfscope}%
\begin{pgfscope}%
\pgfsys@transformshift{3.420704in}{2.262090in}%
\pgfsys@useobject{currentmarker}{}%
\end{pgfscope}%
\begin{pgfscope}%
\pgfsys@transformshift{3.438297in}{2.582363in}%
\pgfsys@useobject{currentmarker}{}%
\end{pgfscope}%
\begin{pgfscope}%
\pgfsys@transformshift{3.455891in}{2.636855in}%
\pgfsys@useobject{currentmarker}{}%
\end{pgfscope}%
\begin{pgfscope}%
\pgfsys@transformshift{3.473485in}{2.577536in}%
\pgfsys@useobject{currentmarker}{}%
\end{pgfscope}%
\begin{pgfscope}%
\pgfsys@transformshift{3.491079in}{2.453483in}%
\pgfsys@useobject{currentmarker}{}%
\end{pgfscope}%
\begin{pgfscope}%
\pgfsys@transformshift{3.508673in}{2.377630in}%
\pgfsys@useobject{currentmarker}{}%
\end{pgfscope}%
\begin{pgfscope}%
\pgfsys@transformshift{3.526267in}{2.609618in}%
\pgfsys@useobject{currentmarker}{}%
\end{pgfscope}%
\begin{pgfscope}%
\pgfsys@transformshift{3.543860in}{2.476330in}%
\pgfsys@useobject{currentmarker}{}%
\end{pgfscope}%
\begin{pgfscope}%
\pgfsys@transformshift{3.561454in}{2.657325in}%
\pgfsys@useobject{currentmarker}{}%
\end{pgfscope}%
\begin{pgfscope}%
\pgfsys@transformshift{3.579048in}{2.568798in}%
\pgfsys@useobject{currentmarker}{}%
\end{pgfscope}%
\begin{pgfscope}%
\pgfsys@transformshift{3.596642in}{2.661511in}%
\pgfsys@useobject{currentmarker}{}%
\end{pgfscope}%
\begin{pgfscope}%
\pgfsys@transformshift{3.614236in}{2.611893in}%
\pgfsys@useobject{currentmarker}{}%
\end{pgfscope}%
\begin{pgfscope}%
\pgfsys@transformshift{3.631830in}{2.660326in}%
\pgfsys@useobject{currentmarker}{}%
\end{pgfscope}%
\begin{pgfscope}%
\pgfsys@transformshift{3.649424in}{2.600979in}%
\pgfsys@useobject{currentmarker}{}%
\end{pgfscope}%
\begin{pgfscope}%
\pgfsys@transformshift{3.667017in}{2.792401in}%
\pgfsys@useobject{currentmarker}{}%
\end{pgfscope}%
\begin{pgfscope}%
\pgfsys@transformshift{3.684611in}{2.632210in}%
\pgfsys@useobject{currentmarker}{}%
\end{pgfscope}%
\begin{pgfscope}%
\pgfsys@transformshift{3.702205in}{2.667702in}%
\pgfsys@useobject{currentmarker}{}%
\end{pgfscope}%
\begin{pgfscope}%
\pgfsys@transformshift{3.719799in}{2.824526in}%
\pgfsys@useobject{currentmarker}{}%
\end{pgfscope}%
\begin{pgfscope}%
\pgfsys@transformshift{3.737393in}{2.499083in}%
\pgfsys@useobject{currentmarker}{}%
\end{pgfscope}%
\begin{pgfscope}%
\pgfsys@transformshift{3.754987in}{2.509147in}%
\pgfsys@useobject{currentmarker}{}%
\end{pgfscope}%
\begin{pgfscope}%
\pgfsys@transformshift{3.772581in}{2.737834in}%
\pgfsys@useobject{currentmarker}{}%
\end{pgfscope}%
\begin{pgfscope}%
\pgfsys@transformshift{3.790174in}{2.517682in}%
\pgfsys@useobject{currentmarker}{}%
\end{pgfscope}%
\begin{pgfscope}%
\pgfsys@transformshift{3.807768in}{2.831865in}%
\pgfsys@useobject{currentmarker}{}%
\end{pgfscope}%
\begin{pgfscope}%
\pgfsys@transformshift{3.825362in}{2.585872in}%
\pgfsys@useobject{currentmarker}{}%
\end{pgfscope}%
\begin{pgfscope}%
\pgfsys@transformshift{3.842956in}{2.544256in}%
\pgfsys@useobject{currentmarker}{}%
\end{pgfscope}%
\begin{pgfscope}%
\pgfsys@transformshift{3.860550in}{2.807077in}%
\pgfsys@useobject{currentmarker}{}%
\end{pgfscope}%
\begin{pgfscope}%
\pgfsys@transformshift{3.878144in}{2.750613in}%
\pgfsys@useobject{currentmarker}{}%
\end{pgfscope}%
\begin{pgfscope}%
\pgfsys@transformshift{3.895738in}{2.776955in}%
\pgfsys@useobject{currentmarker}{}%
\end{pgfscope}%
\begin{pgfscope}%
\pgfsys@transformshift{3.913331in}{2.664098in}%
\pgfsys@useobject{currentmarker}{}%
\end{pgfscope}%
\begin{pgfscope}%
\pgfsys@transformshift{3.930925in}{2.467509in}%
\pgfsys@useobject{currentmarker}{}%
\end{pgfscope}%
\begin{pgfscope}%
\pgfsys@transformshift{3.948519in}{2.732300in}%
\pgfsys@useobject{currentmarker}{}%
\end{pgfscope}%
\begin{pgfscope}%
\pgfsys@transformshift{3.966113in}{2.491099in}%
\pgfsys@useobject{currentmarker}{}%
\end{pgfscope}%
\begin{pgfscope}%
\pgfsys@transformshift{3.983707in}{2.580038in}%
\pgfsys@useobject{currentmarker}{}%
\end{pgfscope}%
\begin{pgfscope}%
\pgfsys@transformshift{4.001301in}{2.573630in}%
\pgfsys@useobject{currentmarker}{}%
\end{pgfscope}%
\begin{pgfscope}%
\pgfsys@transformshift{4.018894in}{2.439287in}%
\pgfsys@useobject{currentmarker}{}%
\end{pgfscope}%
\begin{pgfscope}%
\pgfsys@transformshift{4.036488in}{2.495197in}%
\pgfsys@useobject{currentmarker}{}%
\end{pgfscope}%
\begin{pgfscope}%
\pgfsys@transformshift{4.054082in}{2.503723in}%
\pgfsys@useobject{currentmarker}{}%
\end{pgfscope}%
\begin{pgfscope}%
\pgfsys@transformshift{4.071676in}{2.424999in}%
\pgfsys@useobject{currentmarker}{}%
\end{pgfscope}%
\begin{pgfscope}%
\pgfsys@transformshift{4.089270in}{2.251467in}%
\pgfsys@useobject{currentmarker}{}%
\end{pgfscope}%
\begin{pgfscope}%
\pgfsys@transformshift{4.106864in}{2.371193in}%
\pgfsys@useobject{currentmarker}{}%
\end{pgfscope}%
\begin{pgfscope}%
\pgfsys@transformshift{4.124458in}{2.453684in}%
\pgfsys@useobject{currentmarker}{}%
\end{pgfscope}%
\begin{pgfscope}%
\pgfsys@transformshift{4.142051in}{2.225893in}%
\pgfsys@useobject{currentmarker}{}%
\end{pgfscope}%
\begin{pgfscope}%
\pgfsys@transformshift{4.159645in}{2.260597in}%
\pgfsys@useobject{currentmarker}{}%
\end{pgfscope}%
\begin{pgfscope}%
\pgfsys@transformshift{4.177239in}{2.211646in}%
\pgfsys@useobject{currentmarker}{}%
\end{pgfscope}%
\begin{pgfscope}%
\pgfsys@transformshift{4.194833in}{2.425970in}%
\pgfsys@useobject{currentmarker}{}%
\end{pgfscope}%
\begin{pgfscope}%
\pgfsys@transformshift{4.212427in}{2.288698in}%
\pgfsys@useobject{currentmarker}{}%
\end{pgfscope}%
\begin{pgfscope}%
\pgfsys@transformshift{4.230021in}{2.245970in}%
\pgfsys@useobject{currentmarker}{}%
\end{pgfscope}%
\begin{pgfscope}%
\pgfsys@transformshift{4.247615in}{2.111856in}%
\pgfsys@useobject{currentmarker}{}%
\end{pgfscope}%
\begin{pgfscope}%
\pgfsys@transformshift{4.265208in}{2.233096in}%
\pgfsys@useobject{currentmarker}{}%
\end{pgfscope}%
\begin{pgfscope}%
\pgfsys@transformshift{4.282802in}{2.098982in}%
\pgfsys@useobject{currentmarker}{}%
\end{pgfscope}%
\begin{pgfscope}%
\pgfsys@transformshift{4.300396in}{2.162618in}%
\pgfsys@useobject{currentmarker}{}%
\end{pgfscope}%
\begin{pgfscope}%
\pgfsys@transformshift{4.317990in}{2.088214in}%
\pgfsys@useobject{currentmarker}{}%
\end{pgfscope}%
\begin{pgfscope}%
\pgfsys@transformshift{4.335584in}{2.217817in}%
\pgfsys@useobject{currentmarker}{}%
\end{pgfscope}%
\begin{pgfscope}%
\pgfsys@transformshift{4.353178in}{2.205555in}%
\pgfsys@useobject{currentmarker}{}%
\end{pgfscope}%
\begin{pgfscope}%
\pgfsys@transformshift{4.370772in}{2.125576in}%
\pgfsys@useobject{currentmarker}{}%
\end{pgfscope}%
\begin{pgfscope}%
\pgfsys@transformshift{4.388365in}{2.189386in}%
\pgfsys@useobject{currentmarker}{}%
\end{pgfscope}%
\begin{pgfscope}%
\pgfsys@transformshift{4.405959in}{2.041820in}%
\pgfsys@useobject{currentmarker}{}%
\end{pgfscope}%
\begin{pgfscope}%
\pgfsys@transformshift{4.423553in}{2.007535in}%
\pgfsys@useobject{currentmarker}{}%
\end{pgfscope}%
\begin{pgfscope}%
\pgfsys@transformshift{4.441147in}{2.212697in}%
\pgfsys@useobject{currentmarker}{}%
\end{pgfscope}%
\begin{pgfscope}%
\pgfsys@transformshift{4.458741in}{2.195046in}%
\pgfsys@useobject{currentmarker}{}%
\end{pgfscope}%
\begin{pgfscope}%
\pgfsys@transformshift{4.476335in}{2.254728in}%
\pgfsys@useobject{currentmarker}{}%
\end{pgfscope}%
\begin{pgfscope}%
\pgfsys@transformshift{4.493928in}{2.446545in}%
\pgfsys@useobject{currentmarker}{}%
\end{pgfscope}%
\begin{pgfscope}%
\pgfsys@transformshift{4.511522in}{2.314886in}%
\pgfsys@useobject{currentmarker}{}%
\end{pgfscope}%
\begin{pgfscope}%
\pgfsys@transformshift{4.529116in}{2.141871in}%
\pgfsys@useobject{currentmarker}{}%
\end{pgfscope}%
\begin{pgfscope}%
\pgfsys@transformshift{4.546710in}{2.366405in}%
\pgfsys@useobject{currentmarker}{}%
\end{pgfscope}%
\begin{pgfscope}%
\pgfsys@transformshift{4.564304in}{2.136836in}%
\pgfsys@useobject{currentmarker}{}%
\end{pgfscope}%
\begin{pgfscope}%
\pgfsys@transformshift{4.581898in}{2.243272in}%
\pgfsys@useobject{currentmarker}{}%
\end{pgfscope}%
\begin{pgfscope}%
\pgfsys@transformshift{4.599492in}{2.303294in}%
\pgfsys@useobject{currentmarker}{}%
\end{pgfscope}%
\begin{pgfscope}%
\pgfsys@transformshift{4.617085in}{2.505271in}%
\pgfsys@useobject{currentmarker}{}%
\end{pgfscope}%
\begin{pgfscope}%
\pgfsys@transformshift{4.634679in}{2.275099in}%
\pgfsys@useobject{currentmarker}{}%
\end{pgfscope}%
\begin{pgfscope}%
\pgfsys@transformshift{4.652273in}{2.287237in}%
\pgfsys@useobject{currentmarker}{}%
\end{pgfscope}%
\begin{pgfscope}%
\pgfsys@transformshift{4.669867in}{2.381713in}%
\pgfsys@useobject{currentmarker}{}%
\end{pgfscope}%
\begin{pgfscope}%
\pgfsys@transformshift{4.687461in}{2.343907in}%
\pgfsys@useobject{currentmarker}{}%
\end{pgfscope}%
\begin{pgfscope}%
\pgfsys@transformshift{4.705055in}{2.545636in}%
\pgfsys@useobject{currentmarker}{}%
\end{pgfscope}%
\begin{pgfscope}%
\pgfsys@transformshift{4.722649in}{2.338992in}%
\pgfsys@useobject{currentmarker}{}%
\end{pgfscope}%
\begin{pgfscope}%
\pgfsys@transformshift{4.740242in}{2.349474in}%
\pgfsys@useobject{currentmarker}{}%
\end{pgfscope}%
\begin{pgfscope}%
\pgfsys@transformshift{4.757836in}{2.438041in}%
\pgfsys@useobject{currentmarker}{}%
\end{pgfscope}%
\begin{pgfscope}%
\pgfsys@transformshift{4.775430in}{2.446775in}%
\pgfsys@useobject{currentmarker}{}%
\end{pgfscope}%
\begin{pgfscope}%
\pgfsys@transformshift{4.793024in}{2.707728in}%
\pgfsys@useobject{currentmarker}{}%
\end{pgfscope}%
\begin{pgfscope}%
\pgfsys@transformshift{4.810618in}{2.619589in}%
\pgfsys@useobject{currentmarker}{}%
\end{pgfscope}%
\begin{pgfscope}%
\pgfsys@transformshift{4.828212in}{2.541800in}%
\pgfsys@useobject{currentmarker}{}%
\end{pgfscope}%
\begin{pgfscope}%
\pgfsys@transformshift{4.845805in}{2.416208in}%
\pgfsys@useobject{currentmarker}{}%
\end{pgfscope}%
\begin{pgfscope}%
\pgfsys@transformshift{4.863399in}{2.633696in}%
\pgfsys@useobject{currentmarker}{}%
\end{pgfscope}%
\begin{pgfscope}%
\pgfsys@transformshift{4.880993in}{2.450091in}%
\pgfsys@useobject{currentmarker}{}%
\end{pgfscope}%
\begin{pgfscope}%
\pgfsys@transformshift{4.898587in}{2.397092in}%
\pgfsys@useobject{currentmarker}{}%
\end{pgfscope}%
\begin{pgfscope}%
\pgfsys@transformshift{4.916181in}{2.676321in}%
\pgfsys@useobject{currentmarker}{}%
\end{pgfscope}%
\begin{pgfscope}%
\pgfsys@transformshift{4.933775in}{2.586081in}%
\pgfsys@useobject{currentmarker}{}%
\end{pgfscope}%
\begin{pgfscope}%
\pgfsys@transformshift{4.951369in}{2.644312in}%
\pgfsys@useobject{currentmarker}{}%
\end{pgfscope}%
\begin{pgfscope}%
\pgfsys@transformshift{4.968962in}{2.577668in}%
\pgfsys@useobject{currentmarker}{}%
\end{pgfscope}%
\begin{pgfscope}%
\pgfsys@transformshift{4.986556in}{2.625475in}%
\pgfsys@useobject{currentmarker}{}%
\end{pgfscope}%
\begin{pgfscope}%
\pgfsys@transformshift{5.004150in}{2.462914in}%
\pgfsys@useobject{currentmarker}{}%
\end{pgfscope}%
\begin{pgfscope}%
\pgfsys@transformshift{5.021744in}{2.413409in}%
\pgfsys@useobject{currentmarker}{}%
\end{pgfscope}%
\begin{pgfscope}%
\pgfsys@transformshift{5.039338in}{2.576448in}%
\pgfsys@useobject{currentmarker}{}%
\end{pgfscope}%
\begin{pgfscope}%
\pgfsys@transformshift{5.056932in}{2.411720in}%
\pgfsys@useobject{currentmarker}{}%
\end{pgfscope}%
\begin{pgfscope}%
\pgfsys@transformshift{5.074526in}{2.408824in}%
\pgfsys@useobject{currentmarker}{}%
\end{pgfscope}%
\begin{pgfscope}%
\pgfsys@transformshift{5.092119in}{2.417137in}%
\pgfsys@useobject{currentmarker}{}%
\end{pgfscope}%
\begin{pgfscope}%
\pgfsys@transformshift{5.109713in}{2.448956in}%
\pgfsys@useobject{currentmarker}{}%
\end{pgfscope}%
\begin{pgfscope}%
\pgfsys@transformshift{5.127307in}{2.393983in}%
\pgfsys@useobject{currentmarker}{}%
\end{pgfscope}%
\begin{pgfscope}%
\pgfsys@transformshift{5.144901in}{2.272299in}%
\pgfsys@useobject{currentmarker}{}%
\end{pgfscope}%
\begin{pgfscope}%
\pgfsys@transformshift{5.162495in}{2.329023in}%
\pgfsys@useobject{currentmarker}{}%
\end{pgfscope}%
\begin{pgfscope}%
\pgfsys@transformshift{5.180089in}{2.150000in}%
\pgfsys@useobject{currentmarker}{}%
\end{pgfscope}%
\begin{pgfscope}%
\pgfsys@transformshift{5.197683in}{2.422690in}%
\pgfsys@useobject{currentmarker}{}%
\end{pgfscope}%
\begin{pgfscope}%
\pgfsys@transformshift{5.215276in}{2.178109in}%
\pgfsys@useobject{currentmarker}{}%
\end{pgfscope}%
\begin{pgfscope}%
\pgfsys@transformshift{5.232870in}{2.211944in}%
\pgfsys@useobject{currentmarker}{}%
\end{pgfscope}%
\begin{pgfscope}%
\pgfsys@transformshift{5.250464in}{2.313709in}%
\pgfsys@useobject{currentmarker}{}%
\end{pgfscope}%
\begin{pgfscope}%
\pgfsys@transformshift{5.268058in}{2.217733in}%
\pgfsys@useobject{currentmarker}{}%
\end{pgfscope}%
\begin{pgfscope}%
\pgfsys@transformshift{5.285652in}{2.436373in}%
\pgfsys@useobject{currentmarker}{}%
\end{pgfscope}%
\begin{pgfscope}%
\pgfsys@transformshift{5.303246in}{2.134357in}%
\pgfsys@useobject{currentmarker}{}%
\end{pgfscope}%
\begin{pgfscope}%
\pgfsys@transformshift{5.320839in}{2.282046in}%
\pgfsys@useobject{currentmarker}{}%
\end{pgfscope}%
\begin{pgfscope}%
\pgfsys@transformshift{5.338433in}{2.241026in}%
\pgfsys@useobject{currentmarker}{}%
\end{pgfscope}%
\begin{pgfscope}%
\pgfsys@transformshift{5.356027in}{2.117866in}%
\pgfsys@useobject{currentmarker}{}%
\end{pgfscope}%
\begin{pgfscope}%
\pgfsys@transformshift{5.373621in}{2.284136in}%
\pgfsys@useobject{currentmarker}{}%
\end{pgfscope}%
\begin{pgfscope}%
\pgfsys@transformshift{5.391215in}{2.209022in}%
\pgfsys@useobject{currentmarker}{}%
\end{pgfscope}%
\begin{pgfscope}%
\pgfsys@transformshift{5.408809in}{2.302978in}%
\pgfsys@useobject{currentmarker}{}%
\end{pgfscope}%
\begin{pgfscope}%
\pgfsys@transformshift{5.426403in}{2.308095in}%
\pgfsys@useobject{currentmarker}{}%
\end{pgfscope}%
\begin{pgfscope}%
\pgfsys@transformshift{5.443996in}{2.446676in}%
\pgfsys@useobject{currentmarker}{}%
\end{pgfscope}%
\begin{pgfscope}%
\pgfsys@transformshift{5.461590in}{2.366477in}%
\pgfsys@useobject{currentmarker}{}%
\end{pgfscope}%
\begin{pgfscope}%
\pgfsys@transformshift{5.479184in}{2.198779in}%
\pgfsys@useobject{currentmarker}{}%
\end{pgfscope}%
\begin{pgfscope}%
\pgfsys@transformshift{5.496778in}{2.220369in}%
\pgfsys@useobject{currentmarker}{}%
\end{pgfscope}%
\begin{pgfscope}%
\pgfsys@transformshift{5.514372in}{2.367339in}%
\pgfsys@useobject{currentmarker}{}%
\end{pgfscope}%
\begin{pgfscope}%
\pgfsys@transformshift{5.531966in}{2.334453in}%
\pgfsys@useobject{currentmarker}{}%
\end{pgfscope}%
\begin{pgfscope}%
\pgfsys@transformshift{5.549560in}{2.347263in}%
\pgfsys@useobject{currentmarker}{}%
\end{pgfscope}%
\begin{pgfscope}%
\pgfsys@transformshift{5.567153in}{2.133275in}%
\pgfsys@useobject{currentmarker}{}%
\end{pgfscope}%
\begin{pgfscope}%
\pgfsys@transformshift{5.584747in}{2.313560in}%
\pgfsys@useobject{currentmarker}{}%
\end{pgfscope}%
\begin{pgfscope}%
\pgfsys@transformshift{5.602341in}{2.260227in}%
\pgfsys@useobject{currentmarker}{}%
\end{pgfscope}%
\begin{pgfscope}%
\pgfsys@transformshift{5.619935in}{2.384894in}%
\pgfsys@useobject{currentmarker}{}%
\end{pgfscope}%
\begin{pgfscope}%
\pgfsys@transformshift{5.637529in}{2.368455in}%
\pgfsys@useobject{currentmarker}{}%
\end{pgfscope}%
\begin{pgfscope}%
\pgfsys@transformshift{5.655123in}{2.494470in}%
\pgfsys@useobject{currentmarker}{}%
\end{pgfscope}%
\begin{pgfscope}%
\pgfsys@transformshift{5.672717in}{2.458087in}%
\pgfsys@useobject{currentmarker}{}%
\end{pgfscope}%
\begin{pgfscope}%
\pgfsys@transformshift{5.690310in}{2.530704in}%
\pgfsys@useobject{currentmarker}{}%
\end{pgfscope}%
\begin{pgfscope}%
\pgfsys@transformshift{5.707904in}{2.428234in}%
\pgfsys@useobject{currentmarker}{}%
\end{pgfscope}%
\begin{pgfscope}%
\pgfsys@transformshift{5.725498in}{2.405057in}%
\pgfsys@useobject{currentmarker}{}%
\end{pgfscope}%
\begin{pgfscope}%
\pgfsys@transformshift{5.743092in}{2.485208in}%
\pgfsys@useobject{currentmarker}{}%
\end{pgfscope}%
\begin{pgfscope}%
\pgfsys@transformshift{5.760686in}{2.550822in}%
\pgfsys@useobject{currentmarker}{}%
\end{pgfscope}%
\begin{pgfscope}%
\pgfsys@transformshift{5.778280in}{2.616431in}%
\pgfsys@useobject{currentmarker}{}%
\end{pgfscope}%
\begin{pgfscope}%
\pgfsys@transformshift{5.795873in}{2.832843in}%
\pgfsys@useobject{currentmarker}{}%
\end{pgfscope}%
\begin{pgfscope}%
\pgfsys@transformshift{5.813467in}{2.622109in}%
\pgfsys@useobject{currentmarker}{}%
\end{pgfscope}%
\begin{pgfscope}%
\pgfsys@transformshift{5.831061in}{2.551996in}%
\pgfsys@useobject{currentmarker}{}%
\end{pgfscope}%
\begin{pgfscope}%
\pgfsys@transformshift{5.848655in}{2.636175in}%
\pgfsys@useobject{currentmarker}{}%
\end{pgfscope}%
\begin{pgfscope}%
\pgfsys@transformshift{5.866249in}{2.644912in}%
\pgfsys@useobject{currentmarker}{}%
\end{pgfscope}%
\begin{pgfscope}%
\pgfsys@transformshift{5.883843in}{2.760620in}%
\pgfsys@useobject{currentmarker}{}%
\end{pgfscope}%
\begin{pgfscope}%
\pgfsys@transformshift{5.901437in}{2.572201in}%
\pgfsys@useobject{currentmarker}{}%
\end{pgfscope}%
\begin{pgfscope}%
\pgfsys@transformshift{5.919030in}{2.751921in}%
\pgfsys@useobject{currentmarker}{}%
\end{pgfscope}%
\begin{pgfscope}%
\pgfsys@transformshift{5.936624in}{2.775821in}%
\pgfsys@useobject{currentmarker}{}%
\end{pgfscope}%
\begin{pgfscope}%
\pgfsys@transformshift{5.954218in}{2.796091in}%
\pgfsys@useobject{currentmarker}{}%
\end{pgfscope}%
\begin{pgfscope}%
\pgfsys@transformshift{5.971812in}{2.722154in}%
\pgfsys@useobject{currentmarker}{}%
\end{pgfscope}%
\begin{pgfscope}%
\pgfsys@transformshift{5.989406in}{2.767504in}%
\pgfsys@useobject{currentmarker}{}%
\end{pgfscope}%
\begin{pgfscope}%
\pgfsys@transformshift{6.007000in}{2.653279in}%
\pgfsys@useobject{currentmarker}{}%
\end{pgfscope}%
\begin{pgfscope}%
\pgfsys@transformshift{6.024594in}{2.752946in}%
\pgfsys@useobject{currentmarker}{}%
\end{pgfscope}%
\begin{pgfscope}%
\pgfsys@transformshift{6.042187in}{2.750693in}%
\pgfsys@useobject{currentmarker}{}%
\end{pgfscope}%
\begin{pgfscope}%
\pgfsys@transformshift{6.059781in}{2.849357in}%
\pgfsys@useobject{currentmarker}{}%
\end{pgfscope}%
\begin{pgfscope}%
\pgfsys@transformshift{6.077375in}{2.688050in}%
\pgfsys@useobject{currentmarker}{}%
\end{pgfscope}%
\begin{pgfscope}%
\pgfsys@transformshift{6.094969in}{2.882850in}%
\pgfsys@useobject{currentmarker}{}%
\end{pgfscope}%
\begin{pgfscope}%
\pgfsys@transformshift{6.112563in}{2.951139in}%
\pgfsys@useobject{currentmarker}{}%
\end{pgfscope}%
\begin{pgfscope}%
\pgfsys@transformshift{6.130157in}{2.581640in}%
\pgfsys@useobject{currentmarker}{}%
\end{pgfscope}%
\begin{pgfscope}%
\pgfsys@transformshift{6.147751in}{2.828734in}%
\pgfsys@useobject{currentmarker}{}%
\end{pgfscope}%
\begin{pgfscope}%
\pgfsys@transformshift{6.165344in}{2.845545in}%
\pgfsys@useobject{currentmarker}{}%
\end{pgfscope}%
\begin{pgfscope}%
\pgfsys@transformshift{6.182938in}{2.701636in}%
\pgfsys@useobject{currentmarker}{}%
\end{pgfscope}%
\begin{pgfscope}%
\pgfsys@transformshift{6.200532in}{2.715287in}%
\pgfsys@useobject{currentmarker}{}%
\end{pgfscope}%
\begin{pgfscope}%
\pgfsys@transformshift{6.218126in}{2.730634in}%
\pgfsys@useobject{currentmarker}{}%
\end{pgfscope}%
\begin{pgfscope}%
\pgfsys@transformshift{6.235720in}{2.701620in}%
\pgfsys@useobject{currentmarker}{}%
\end{pgfscope}%
\begin{pgfscope}%
\pgfsys@transformshift{6.253314in}{2.687893in}%
\pgfsys@useobject{currentmarker}{}%
\end{pgfscope}%
\begin{pgfscope}%
\pgfsys@transformshift{6.270907in}{2.535733in}%
\pgfsys@useobject{currentmarker}{}%
\end{pgfscope}%
\begin{pgfscope}%
\pgfsys@transformshift{6.288501in}{2.811411in}%
\pgfsys@useobject{currentmarker}{}%
\end{pgfscope}%
\begin{pgfscope}%
\pgfsys@transformshift{6.306095in}{2.791491in}%
\pgfsys@useobject{currentmarker}{}%
\end{pgfscope}%
\begin{pgfscope}%
\pgfsys@transformshift{6.323689in}{2.586388in}%
\pgfsys@useobject{currentmarker}{}%
\end{pgfscope}%
\begin{pgfscope}%
\pgfsys@transformshift{6.341283in}{2.508363in}%
\pgfsys@useobject{currentmarker}{}%
\end{pgfscope}%
\begin{pgfscope}%
\pgfsys@transformshift{6.358877in}{2.700439in}%
\pgfsys@useobject{currentmarker}{}%
\end{pgfscope}%
\begin{pgfscope}%
\pgfsys@transformshift{6.376471in}{2.579017in}%
\pgfsys@useobject{currentmarker}{}%
\end{pgfscope}%
\end{pgfscope}%
\begin{pgfscope}%
\pgfpathrectangle{\pgfqpoint{2.000000in}{1.973684in}}{\pgfqpoint{4.376471in}{0.978947in}} %
\pgfusepath{clip}%
\pgfsetroundcap%
\pgfsetroundjoin%
\pgfsetlinewidth{1.756562pt}%
\definecolor{currentstroke}{rgb}{0.298039,0.447059,0.690196}%
\pgfsetstrokecolor{currentstroke}%
\pgfsetdash{}{0pt}%
\pgfpathmoveto{\pgfqpoint{2.923232in}{2.962632in}}%
\pgfpathlineto{\pgfqpoint{2.928076in}{2.899903in}}%
\pgfpathlineto{\pgfqpoint{2.945670in}{2.746061in}}%
\pgfpathlineto{\pgfqpoint{2.963263in}{2.647049in}}%
\pgfpathlineto{\pgfqpoint{2.980857in}{2.587324in}}%
\pgfpathlineto{\pgfqpoint{2.998451in}{2.554495in}}%
\pgfpathlineto{\pgfqpoint{3.016045in}{2.538893in}}%
\pgfpathlineto{\pgfqpoint{3.033639in}{2.533158in}}%
\pgfpathlineto{\pgfqpoint{3.086420in}{2.528852in}}%
\pgfpathlineto{\pgfqpoint{3.104014in}{2.523218in}}%
\pgfpathlineto{\pgfqpoint{3.121608in}{2.513785in}}%
\pgfpathlineto{\pgfqpoint{3.139202in}{2.500594in}}%
\pgfpathlineto{\pgfqpoint{3.156796in}{2.484141in}}%
\pgfpathlineto{\pgfqpoint{3.191983in}{2.444831in}}%
\pgfpathlineto{\pgfqpoint{3.227171in}{2.403859in}}%
\pgfpathlineto{\pgfqpoint{3.244765in}{2.385347in}}%
\pgfpathlineto{\pgfqpoint{3.262359in}{2.369373in}}%
\pgfpathlineto{\pgfqpoint{3.279953in}{2.356687in}}%
\pgfpathlineto{\pgfqpoint{3.297547in}{2.347873in}}%
\pgfpathlineto{\pgfqpoint{3.315140in}{2.343340in}}%
\pgfpathlineto{\pgfqpoint{3.332734in}{2.343324in}}%
\pgfpathlineto{\pgfqpoint{3.350328in}{2.347890in}}%
\pgfpathlineto{\pgfqpoint{3.367922in}{2.356945in}}%
\pgfpathlineto{\pgfqpoint{3.385516in}{2.370252in}}%
\pgfpathlineto{\pgfqpoint{3.403110in}{2.387453in}}%
\pgfpathlineto{\pgfqpoint{3.420704in}{2.408083in}}%
\pgfpathlineto{\pgfqpoint{3.438297in}{2.431600in}}%
\pgfpathlineto{\pgfqpoint{3.473485in}{2.484843in}}%
\pgfpathlineto{\pgfqpoint{3.543860in}{2.598025in}}%
\pgfpathlineto{\pgfqpoint{3.579048in}{2.648239in}}%
\pgfpathlineto{\pgfqpoint{3.596642in}{2.669981in}}%
\pgfpathlineto{\pgfqpoint{3.614236in}{2.688975in}}%
\pgfpathlineto{\pgfqpoint{3.631830in}{2.704936in}}%
\pgfpathlineto{\pgfqpoint{3.649424in}{2.717652in}}%
\pgfpathlineto{\pgfqpoint{3.667017in}{2.726985in}}%
\pgfpathlineto{\pgfqpoint{3.684611in}{2.732870in}}%
\pgfpathlineto{\pgfqpoint{3.702205in}{2.735309in}}%
\pgfpathlineto{\pgfqpoint{3.719799in}{2.734362in}}%
\pgfpathlineto{\pgfqpoint{3.737393in}{2.730150in}}%
\pgfpathlineto{\pgfqpoint{3.754987in}{2.722839in}}%
\pgfpathlineto{\pgfqpoint{3.772581in}{2.712635in}}%
\pgfpathlineto{\pgfqpoint{3.790174in}{2.699780in}}%
\pgfpathlineto{\pgfqpoint{3.807768in}{2.684538in}}%
\pgfpathlineto{\pgfqpoint{3.825362in}{2.667191in}}%
\pgfpathlineto{\pgfqpoint{3.860550in}{2.627358in}}%
\pgfpathlineto{\pgfqpoint{3.895738in}{2.582622in}}%
\pgfpathlineto{\pgfqpoint{3.966113in}{2.487030in}}%
\pgfpathlineto{\pgfqpoint{4.018894in}{2.416801in}}%
\pgfpathlineto{\pgfqpoint{4.054082in}{2.372994in}}%
\pgfpathlineto{\pgfqpoint{4.089270in}{2.332481in}}%
\pgfpathlineto{\pgfqpoint{4.124458in}{2.295715in}}%
\pgfpathlineto{\pgfqpoint{4.159645in}{2.262962in}}%
\pgfpathlineto{\pgfqpoint{4.194833in}{2.234391in}}%
\pgfpathlineto{\pgfqpoint{4.230021in}{2.210147in}}%
\pgfpathlineto{\pgfqpoint{4.265208in}{2.190412in}}%
\pgfpathlineto{\pgfqpoint{4.282802in}{2.182310in}}%
\pgfpathlineto{\pgfqpoint{4.300396in}{2.175435in}}%
\pgfpathlineto{\pgfqpoint{4.317990in}{2.169832in}}%
\pgfpathlineto{\pgfqpoint{4.335584in}{2.165545in}}%
\pgfpathlineto{\pgfqpoint{4.353178in}{2.162625in}}%
\pgfpathlineto{\pgfqpoint{4.370772in}{2.161120in}}%
\pgfpathlineto{\pgfqpoint{4.388365in}{2.161081in}}%
\pgfpathlineto{\pgfqpoint{4.405959in}{2.162550in}}%
\pgfpathlineto{\pgfqpoint{4.423553in}{2.165569in}}%
\pgfpathlineto{\pgfqpoint{4.441147in}{2.170168in}}%
\pgfpathlineto{\pgfqpoint{4.458741in}{2.176369in}}%
\pgfpathlineto{\pgfqpoint{4.476335in}{2.184180in}}%
\pgfpathlineto{\pgfqpoint{4.493928in}{2.193594in}}%
\pgfpathlineto{\pgfqpoint{4.511522in}{2.204590in}}%
\pgfpathlineto{\pgfqpoint{4.529116in}{2.217125in}}%
\pgfpathlineto{\pgfqpoint{4.564304in}{2.246547in}}%
\pgfpathlineto{\pgfqpoint{4.599492in}{2.281117in}}%
\pgfpathlineto{\pgfqpoint{4.634679in}{2.319753in}}%
\pgfpathlineto{\pgfqpoint{4.687461in}{2.382187in}}%
\pgfpathlineto{\pgfqpoint{4.740242in}{2.444761in}}%
\pgfpathlineto{\pgfqpoint{4.775430in}{2.483351in}}%
\pgfpathlineto{\pgfqpoint{4.810618in}{2.517185in}}%
\pgfpathlineto{\pgfqpoint{4.828212in}{2.531754in}}%
\pgfpathlineto{\pgfqpoint{4.845805in}{2.544486in}}%
\pgfpathlineto{\pgfqpoint{4.863399in}{2.555203in}}%
\pgfpathlineto{\pgfqpoint{4.880993in}{2.563751in}}%
\pgfpathlineto{\pgfqpoint{4.898587in}{2.570002in}}%
\pgfpathlineto{\pgfqpoint{4.916181in}{2.573858in}}%
\pgfpathlineto{\pgfqpoint{4.933775in}{2.575254in}}%
\pgfpathlineto{\pgfqpoint{4.951369in}{2.574157in}}%
\pgfpathlineto{\pgfqpoint{4.968962in}{2.570568in}}%
\pgfpathlineto{\pgfqpoint{4.986556in}{2.564526in}}%
\pgfpathlineto{\pgfqpoint{5.004150in}{2.556102in}}%
\pgfpathlineto{\pgfqpoint{5.021744in}{2.545401in}}%
\pgfpathlineto{\pgfqpoint{5.039338in}{2.532563in}}%
\pgfpathlineto{\pgfqpoint{5.056932in}{2.517758in}}%
\pgfpathlineto{\pgfqpoint{5.074526in}{2.501184in}}%
\pgfpathlineto{\pgfqpoint{5.109713in}{2.463650in}}%
\pgfpathlineto{\pgfqpoint{5.144901in}{2.422007in}}%
\pgfpathlineto{\pgfqpoint{5.232870in}{2.315080in}}%
\pgfpathlineto{\pgfqpoint{5.268058in}{2.277292in}}%
\pgfpathlineto{\pgfqpoint{5.285652in}{2.260535in}}%
\pgfpathlineto{\pgfqpoint{5.303246in}{2.245503in}}%
\pgfpathlineto{\pgfqpoint{5.320839in}{2.232392in}}%
\pgfpathlineto{\pgfqpoint{5.338433in}{2.221371in}}%
\pgfpathlineto{\pgfqpoint{5.356027in}{2.212578in}}%
\pgfpathlineto{\pgfqpoint{5.373621in}{2.206122in}}%
\pgfpathlineto{\pgfqpoint{5.391215in}{2.202079in}}%
\pgfpathlineto{\pgfqpoint{5.408809in}{2.200496in}}%
\pgfpathlineto{\pgfqpoint{5.426403in}{2.201388in}}%
\pgfpathlineto{\pgfqpoint{5.443996in}{2.204740in}}%
\pgfpathlineto{\pgfqpoint{5.461590in}{2.210509in}}%
\pgfpathlineto{\pgfqpoint{5.479184in}{2.218627in}}%
\pgfpathlineto{\pgfqpoint{5.496778in}{2.228999in}}%
\pgfpathlineto{\pgfqpoint{5.514372in}{2.241510in}}%
\pgfpathlineto{\pgfqpoint{5.531966in}{2.256027in}}%
\pgfpathlineto{\pgfqpoint{5.549560in}{2.272400in}}%
\pgfpathlineto{\pgfqpoint{5.584747in}{2.310053in}}%
\pgfpathlineto{\pgfqpoint{5.619935in}{2.353073in}}%
\pgfpathlineto{\pgfqpoint{5.655123in}{2.399994in}}%
\pgfpathlineto{\pgfqpoint{5.707904in}{2.474528in}}%
\pgfpathlineto{\pgfqpoint{5.778280in}{2.574596in}}%
\pgfpathlineto{\pgfqpoint{5.813467in}{2.622191in}}%
\pgfpathlineto{\pgfqpoint{5.848655in}{2.666689in}}%
\pgfpathlineto{\pgfqpoint{5.883843in}{2.707049in}}%
\pgfpathlineto{\pgfqpoint{5.919030in}{2.742262in}}%
\pgfpathlineto{\pgfqpoint{5.936624in}{2.757630in}}%
\pgfpathlineto{\pgfqpoint{5.954218in}{2.771350in}}%
\pgfpathlineto{\pgfqpoint{5.971812in}{2.783310in}}%
\pgfpathlineto{\pgfqpoint{5.989406in}{2.793411in}}%
\pgfpathlineto{\pgfqpoint{6.007000in}{2.801562in}}%
\pgfpathlineto{\pgfqpoint{6.024594in}{2.807690in}}%
\pgfpathlineto{\pgfqpoint{6.042187in}{2.811742in}}%
\pgfpathlineto{\pgfqpoint{6.059781in}{2.813688in}}%
\pgfpathlineto{\pgfqpoint{6.077375in}{2.813523in}}%
\pgfpathlineto{\pgfqpoint{6.094969in}{2.811272in}}%
\pgfpathlineto{\pgfqpoint{6.112563in}{2.806991in}}%
\pgfpathlineto{\pgfqpoint{6.130157in}{2.800764in}}%
\pgfpathlineto{\pgfqpoint{6.147751in}{2.792698in}}%
\pgfpathlineto{\pgfqpoint{6.165344in}{2.782919in}}%
\pgfpathlineto{\pgfqpoint{6.182938in}{2.771553in}}%
\pgfpathlineto{\pgfqpoint{6.200532in}{2.758713in}}%
\pgfpathlineto{\pgfqpoint{6.235720in}{2.728802in}}%
\pgfpathlineto{\pgfqpoint{6.253314in}{2.711584in}}%
\pgfpathlineto{\pgfqpoint{6.270907in}{2.692488in}}%
\pgfpathlineto{\pgfqpoint{6.288501in}{2.670931in}}%
\pgfpathlineto{\pgfqpoint{6.306095in}{2.645979in}}%
\pgfpathlineto{\pgfqpoint{6.323689in}{2.616237in}}%
\pgfpathlineto{\pgfqpoint{6.341283in}{2.579717in}}%
\pgfpathlineto{\pgfqpoint{6.358877in}{2.533686in}}%
\pgfpathlineto{\pgfqpoint{6.376471in}{2.474480in}}%
\pgfpathlineto{\pgfqpoint{6.376471in}{2.474480in}}%
\pgfusepath{stroke}%
\end{pgfscope}%
\begin{pgfscope}%
\pgfpathrectangle{\pgfqpoint{2.000000in}{1.973684in}}{\pgfqpoint{4.376471in}{0.978947in}} %
\pgfusepath{clip}%
\pgfsetbuttcap%
\pgfsetroundjoin%
\pgfsetlinewidth{1.756562pt}%
\definecolor{currentstroke}{rgb}{1.000000,0.647059,0.000000}%
\pgfsetstrokecolor{currentstroke}%
\pgfsetdash{{6.000000pt}{6.000000pt}}{0.000000pt}%
\pgfpathmoveto{\pgfqpoint{2.921777in}{2.962632in}}%
\pgfpathlineto{\pgfqpoint{2.928076in}{2.887902in}}%
\pgfpathlineto{\pgfqpoint{2.945670in}{2.747937in}}%
\pgfpathlineto{\pgfqpoint{2.963263in}{2.656768in}}%
\pgfpathlineto{\pgfqpoint{2.980857in}{2.601051in}}%
\pgfpathlineto{\pgfqpoint{2.998451in}{2.568786in}}%
\pgfpathlineto{\pgfqpoint{3.016045in}{2.553152in}}%
\pgfpathlineto{\pgfqpoint{3.033639in}{2.544090in}}%
\pgfpathlineto{\pgfqpoint{3.051233in}{2.543592in}}%
\pgfpathlineto{\pgfqpoint{3.086420in}{2.534081in}}%
\pgfpathlineto{\pgfqpoint{3.104014in}{2.527409in}}%
\pgfpathlineto{\pgfqpoint{3.121608in}{2.515708in}}%
\pgfpathlineto{\pgfqpoint{3.139202in}{2.502115in}}%
\pgfpathlineto{\pgfqpoint{3.156796in}{2.484489in}}%
\pgfpathlineto{\pgfqpoint{3.209577in}{2.423544in}}%
\pgfpathlineto{\pgfqpoint{3.227171in}{2.401038in}}%
\pgfpathlineto{\pgfqpoint{3.244765in}{2.383810in}}%
\pgfpathlineto{\pgfqpoint{3.262359in}{2.369071in}}%
\pgfpathlineto{\pgfqpoint{3.279953in}{2.356822in}}%
\pgfpathlineto{\pgfqpoint{3.315140in}{2.343080in}}%
\pgfpathlineto{\pgfqpoint{3.350328in}{2.347661in}}%
\pgfpathlineto{\pgfqpoint{3.367922in}{2.356424in}}%
\pgfpathlineto{\pgfqpoint{3.403110in}{2.386498in}}%
\pgfpathlineto{\pgfqpoint{3.420704in}{2.406216in}}%
\pgfpathlineto{\pgfqpoint{3.473485in}{2.480804in}}%
\pgfpathlineto{\pgfqpoint{3.491079in}{2.510281in}}%
\pgfpathlineto{\pgfqpoint{3.508673in}{2.535675in}}%
\pgfpathlineto{\pgfqpoint{3.561454in}{2.617632in}}%
\pgfpathlineto{\pgfqpoint{3.614236in}{2.682162in}}%
\pgfpathlineto{\pgfqpoint{3.631830in}{2.695108in}}%
\pgfpathlineto{\pgfqpoint{3.667017in}{2.716817in}}%
\pgfpathlineto{\pgfqpoint{3.684611in}{2.725382in}}%
\pgfpathlineto{\pgfqpoint{3.702205in}{2.727572in}}%
\pgfpathlineto{\pgfqpoint{3.719799in}{2.726178in}}%
\pgfpathlineto{\pgfqpoint{3.754987in}{2.714427in}}%
\pgfpathlineto{\pgfqpoint{3.772581in}{2.706461in}}%
\pgfpathlineto{\pgfqpoint{3.807768in}{2.679573in}}%
\pgfpathlineto{\pgfqpoint{3.825362in}{2.663441in}}%
\pgfpathlineto{\pgfqpoint{3.842956in}{2.645316in}}%
\pgfpathlineto{\pgfqpoint{3.878144in}{2.603690in}}%
\pgfpathlineto{\pgfqpoint{3.930925in}{2.535973in}}%
\pgfpathlineto{\pgfqpoint{3.948519in}{2.515260in}}%
\pgfpathlineto{\pgfqpoint{3.966113in}{2.486580in}}%
\pgfpathlineto{\pgfqpoint{3.983707in}{2.467062in}}%
\pgfpathlineto{\pgfqpoint{4.054082in}{2.374648in}}%
\pgfpathlineto{\pgfqpoint{4.071676in}{2.358316in}}%
\pgfpathlineto{\pgfqpoint{4.106864in}{2.316491in}}%
\pgfpathlineto{\pgfqpoint{4.159645in}{2.265903in}}%
\pgfpathlineto{\pgfqpoint{4.177239in}{2.251562in}}%
\pgfpathlineto{\pgfqpoint{4.194833in}{2.238816in}}%
\pgfpathlineto{\pgfqpoint{4.212427in}{2.221687in}}%
\pgfpathlineto{\pgfqpoint{4.230021in}{2.210534in}}%
\pgfpathlineto{\pgfqpoint{4.247615in}{2.204559in}}%
\pgfpathlineto{\pgfqpoint{4.265208in}{2.193406in}}%
\pgfpathlineto{\pgfqpoint{4.300396in}{2.176676in}}%
\pgfpathlineto{\pgfqpoint{4.317990in}{2.166717in}}%
\pgfpathlineto{\pgfqpoint{4.335584in}{2.168709in}}%
\pgfpathlineto{\pgfqpoint{4.353178in}{2.162734in}}%
\pgfpathlineto{\pgfqpoint{4.370772in}{2.163132in}}%
\pgfpathlineto{\pgfqpoint{4.388365in}{2.162336in}}%
\pgfpathlineto{\pgfqpoint{4.405959in}{2.162734in}}%
\pgfpathlineto{\pgfqpoint{4.441147in}{2.167514in}}%
\pgfpathlineto{\pgfqpoint{4.458741in}{2.174684in}}%
\pgfpathlineto{\pgfqpoint{4.493928in}{2.193406in}}%
\pgfpathlineto{\pgfqpoint{4.511522in}{2.203762in}}%
\pgfpathlineto{\pgfqpoint{4.529116in}{2.219696in}}%
\pgfpathlineto{\pgfqpoint{4.546710in}{2.230849in}}%
\pgfpathlineto{\pgfqpoint{4.564304in}{2.245986in}}%
\pgfpathlineto{\pgfqpoint{4.617085in}{2.301753in}}%
\pgfpathlineto{\pgfqpoint{4.652273in}{2.339993in}}%
\pgfpathlineto{\pgfqpoint{4.687461in}{2.379030in}}%
\pgfpathlineto{\pgfqpoint{4.705055in}{2.406913in}}%
\pgfpathlineto{\pgfqpoint{4.722649in}{2.422846in}}%
\pgfpathlineto{\pgfqpoint{4.757836in}{2.465070in}}%
\pgfpathlineto{\pgfqpoint{4.775430in}{2.484190in}}%
\pgfpathlineto{\pgfqpoint{4.793024in}{2.500123in}}%
\pgfpathlineto{\pgfqpoint{4.810618in}{2.519243in}}%
\pgfpathlineto{\pgfqpoint{4.828212in}{2.528803in}}%
\pgfpathlineto{\pgfqpoint{4.863399in}{2.555492in}}%
\pgfpathlineto{\pgfqpoint{4.880993in}{2.565849in}}%
\pgfpathlineto{\pgfqpoint{4.898587in}{2.566645in}}%
\pgfpathlineto{\pgfqpoint{4.916181in}{2.573019in}}%
\pgfpathlineto{\pgfqpoint{4.933775in}{2.577799in}}%
\pgfpathlineto{\pgfqpoint{4.951369in}{2.569434in}}%
\pgfpathlineto{\pgfqpoint{4.968962in}{2.570230in}}%
\pgfpathlineto{\pgfqpoint{4.986556in}{2.560670in}}%
\pgfpathlineto{\pgfqpoint{5.004150in}{2.556687in}}%
\pgfpathlineto{\pgfqpoint{5.021744in}{2.544339in}}%
\pgfpathlineto{\pgfqpoint{5.039338in}{2.529998in}}%
\pgfpathlineto{\pgfqpoint{5.056932in}{2.518447in}}%
\pgfpathlineto{\pgfqpoint{5.092119in}{2.485783in}}%
\pgfpathlineto{\pgfqpoint{5.127307in}{2.442763in}}%
\pgfpathlineto{\pgfqpoint{5.162495in}{2.398548in}}%
\pgfpathlineto{\pgfqpoint{5.197683in}{2.357719in}}%
\pgfpathlineto{\pgfqpoint{5.250464in}{2.295379in}}%
\pgfpathlineto{\pgfqpoint{5.268058in}{2.279247in}}%
\pgfpathlineto{\pgfqpoint{5.285652in}{2.261322in}}%
\pgfpathlineto{\pgfqpoint{5.320839in}{2.231945in}}%
\pgfpathlineto{\pgfqpoint{5.338433in}{2.223779in}}%
\pgfpathlineto{\pgfqpoint{5.356027in}{2.211878in}}%
\pgfpathlineto{\pgfqpoint{5.373621in}{2.205107in}}%
\pgfpathlineto{\pgfqpoint{5.408809in}{2.200103in}}%
\pgfpathlineto{\pgfqpoint{5.426403in}{2.201721in}}%
\pgfpathlineto{\pgfqpoint{5.461590in}{2.210385in}}%
\pgfpathlineto{\pgfqpoint{5.479184in}{2.220891in}}%
\pgfpathlineto{\pgfqpoint{5.496778in}{2.228559in}}%
\pgfpathlineto{\pgfqpoint{5.531966in}{2.258235in}}%
\pgfpathlineto{\pgfqpoint{5.549560in}{2.270683in}}%
\pgfpathlineto{\pgfqpoint{5.567153in}{2.292790in}}%
\pgfpathlineto{\pgfqpoint{5.584747in}{2.309919in}}%
\pgfpathlineto{\pgfqpoint{5.619935in}{2.351545in}}%
\pgfpathlineto{\pgfqpoint{5.637529in}{2.375644in}}%
\pgfpathlineto{\pgfqpoint{5.655123in}{2.396955in}}%
\pgfpathlineto{\pgfqpoint{5.672717in}{2.424241in}}%
\pgfpathlineto{\pgfqpoint{5.690310in}{2.447145in}}%
\pgfpathlineto{\pgfqpoint{5.707904in}{2.475227in}}%
\pgfpathlineto{\pgfqpoint{5.725498in}{2.498928in}}%
\pgfpathlineto{\pgfqpoint{5.743092in}{2.525019in}}%
\pgfpathlineto{\pgfqpoint{5.760686in}{2.553301in}}%
\pgfpathlineto{\pgfqpoint{5.778280in}{2.577400in}}%
\pgfpathlineto{\pgfqpoint{5.795873in}{2.599508in}}%
\pgfpathlineto{\pgfqpoint{5.813467in}{2.626196in}}%
\pgfpathlineto{\pgfqpoint{5.831061in}{2.646113in}}%
\pgfpathlineto{\pgfqpoint{5.866249in}{2.690328in}}%
\pgfpathlineto{\pgfqpoint{5.919030in}{2.743307in}}%
\pgfpathlineto{\pgfqpoint{5.936624in}{2.758643in}}%
\pgfpathlineto{\pgfqpoint{5.954218in}{2.771389in}}%
\pgfpathlineto{\pgfqpoint{5.971812in}{2.785729in}}%
\pgfpathlineto{\pgfqpoint{5.989406in}{2.795289in}}%
\pgfpathlineto{\pgfqpoint{6.042187in}{2.814011in}}%
\pgfpathlineto{\pgfqpoint{6.059781in}{2.816600in}}%
\pgfpathlineto{\pgfqpoint{6.077375in}{2.815604in}}%
\pgfpathlineto{\pgfqpoint{6.094969in}{2.813015in}}%
\pgfpathlineto{\pgfqpoint{6.130157in}{2.802459in}}%
\pgfpathlineto{\pgfqpoint{6.165344in}{2.784335in}}%
\pgfpathlineto{\pgfqpoint{6.200532in}{2.760136in}}%
\pgfpathlineto{\pgfqpoint{6.235720in}{2.730062in}}%
\pgfpathlineto{\pgfqpoint{6.270907in}{2.693017in}}%
\pgfpathlineto{\pgfqpoint{6.288501in}{2.671806in}}%
\pgfpathlineto{\pgfqpoint{6.306095in}{2.646910in}}%
\pgfpathlineto{\pgfqpoint{6.323689in}{2.617533in}}%
\pgfpathlineto{\pgfqpoint{6.341283in}{2.583873in}}%
\pgfpathlineto{\pgfqpoint{6.358877in}{2.538065in}}%
\pgfpathlineto{\pgfqpoint{6.376471in}{2.481800in}}%
\pgfpathlineto{\pgfqpoint{6.376471in}{2.481800in}}%
\pgfusepath{stroke}%
\end{pgfscope}%
\begin{pgfscope}%
\pgfsetrectcap%
\pgfsetmiterjoin%
\pgfsetlinewidth{1.003750pt}%
\definecolor{currentstroke}{rgb}{0.800000,0.800000,0.800000}%
\pgfsetstrokecolor{currentstroke}%
\pgfsetdash{}{0pt}%
\pgfpathmoveto{\pgfqpoint{2.000000in}{1.973684in}}%
\pgfpathlineto{\pgfqpoint{2.000000in}{2.952632in}}%
\pgfusepath{stroke}%
\end{pgfscope}%
\begin{pgfscope}%
\pgfsetrectcap%
\pgfsetmiterjoin%
\pgfsetlinewidth{1.003750pt}%
\definecolor{currentstroke}{rgb}{0.800000,0.800000,0.800000}%
\pgfsetstrokecolor{currentstroke}%
\pgfsetdash{}{0pt}%
\pgfpathmoveto{\pgfqpoint{6.376471in}{1.973684in}}%
\pgfpathlineto{\pgfqpoint{6.376471in}{2.952632in}}%
\pgfusepath{stroke}%
\end{pgfscope}%
\begin{pgfscope}%
\pgfsetrectcap%
\pgfsetmiterjoin%
\pgfsetlinewidth{1.003750pt}%
\definecolor{currentstroke}{rgb}{0.800000,0.800000,0.800000}%
\pgfsetstrokecolor{currentstroke}%
\pgfsetdash{}{0pt}%
\pgfpathmoveto{\pgfqpoint{2.000000in}{2.952632in}}%
\pgfpathlineto{\pgfqpoint{6.376471in}{2.952632in}}%
\pgfusepath{stroke}%
\end{pgfscope}%
\begin{pgfscope}%
\pgfsetrectcap%
\pgfsetmiterjoin%
\pgfsetlinewidth{1.003750pt}%
\definecolor{currentstroke}{rgb}{0.800000,0.800000,0.800000}%
\pgfsetstrokecolor{currentstroke}%
\pgfsetdash{}{0pt}%
\pgfpathmoveto{\pgfqpoint{2.000000in}{1.973684in}}%
\pgfpathlineto{\pgfqpoint{6.376471in}{1.973684in}}%
\pgfusepath{stroke}%
\end{pgfscope}%
\begin{pgfscope}%
\pgfsetroundcap%
\pgfsetroundjoin%
\pgfsetlinewidth{1.756562pt}%
\definecolor{currentstroke}{rgb}{0.298039,0.447059,0.690196}%
\pgfsetstrokecolor{currentstroke}%
\pgfsetdash{}{0pt}%
\pgfpathmoveto{\pgfqpoint{2.125000in}{2.772688in}}%
\pgfpathlineto{\pgfqpoint{2.402778in}{2.772688in}}%
\pgfusepath{stroke}%
\end{pgfscope}%
\begin{pgfscope}%
\definecolor{textcolor}{rgb}{0.150000,0.150000,0.150000}%
\pgfsetstrokecolor{textcolor}%
\pgfsetfillcolor{textcolor}%
\pgftext[x=2.513889in,y=2.724077in,left,base]{\color{textcolor}\sffamily\fontsize{10.000000}{12.000000}\selectfont \(\displaystyle \widetilde{\Phi}^* \theta\)}%
\end{pgfscope}%
\begin{pgfscope}%
\pgfsetbuttcap%
\pgfsetroundjoin%
\pgfsetlinewidth{1.756562pt}%
\definecolor{currentstroke}{rgb}{1.000000,0.647059,0.000000}%
\pgfsetstrokecolor{currentstroke}%
\pgfsetdash{{6.000000pt}{6.000000pt}}{0.000000pt}%
\pgfpathmoveto{\pgfqpoint{2.125000in}{2.567827in}}%
\pgfpathlineto{\pgfqpoint{2.402778in}{2.567827in}}%
\pgfusepath{stroke}%
\end{pgfscope}%
\begin{pgfscope}%
\definecolor{textcolor}{rgb}{0.150000,0.150000,0.150000}%
\pgfsetstrokecolor{textcolor}%
\pgfsetfillcolor{textcolor}%
\pgftext[x=2.513889in,y=2.519216in,left,base]{\color{textcolor}\sffamily\fontsize{10.000000}{12.000000}\selectfont \(\displaystyle \widetilde{K}u\)}%
\end{pgfscope}%
\begin{pgfscope}%
\pgfsetbuttcap%
\pgfsetroundjoin%
\definecolor{currentfill}{rgb}{1.000000,0.000000,0.000000}%
\pgfsetfillcolor{currentfill}%
\pgfsetlinewidth{2.007500pt}%
\definecolor{currentstroke}{rgb}{1.000000,0.000000,0.000000}%
\pgfsetstrokecolor{currentstroke}%
\pgfsetdash{}{0pt}%
\pgfpathmoveto{\pgfqpoint{2.232832in}{2.359209in}}%
\pgfpathlineto{\pgfqpoint{2.294945in}{2.359209in}}%
\pgfpathmoveto{\pgfqpoint{2.263889in}{2.328152in}}%
\pgfpathlineto{\pgfqpoint{2.263889in}{2.390265in}}%
\pgfusepath{stroke,fill}%
\end{pgfscope}%
\begin{pgfscope}%
\pgfsetbuttcap%
\pgfsetroundjoin%
\definecolor{currentfill}{rgb}{1.000000,0.000000,0.000000}%
\pgfsetfillcolor{currentfill}%
\pgfsetlinewidth{2.007500pt}%
\definecolor{currentstroke}{rgb}{1.000000,0.000000,0.000000}%
\pgfsetstrokecolor{currentstroke}%
\pgfsetdash{}{0pt}%
\pgfpathmoveto{\pgfqpoint{2.232832in}{2.359209in}}%
\pgfpathlineto{\pgfqpoint{2.294945in}{2.359209in}}%
\pgfpathmoveto{\pgfqpoint{2.263889in}{2.328152in}}%
\pgfpathlineto{\pgfqpoint{2.263889in}{2.390265in}}%
\pgfusepath{stroke,fill}%
\end{pgfscope}%
\begin{pgfscope}%
\pgfsetbuttcap%
\pgfsetroundjoin%
\definecolor{currentfill}{rgb}{1.000000,0.000000,0.000000}%
\pgfsetfillcolor{currentfill}%
\pgfsetlinewidth{2.007500pt}%
\definecolor{currentstroke}{rgb}{1.000000,0.000000,0.000000}%
\pgfsetstrokecolor{currentstroke}%
\pgfsetdash{}{0pt}%
\pgfpathmoveto{\pgfqpoint{2.232832in}{2.359209in}}%
\pgfpathlineto{\pgfqpoint{2.294945in}{2.359209in}}%
\pgfpathmoveto{\pgfqpoint{2.263889in}{2.328152in}}%
\pgfpathlineto{\pgfqpoint{2.263889in}{2.390265in}}%
\pgfusepath{stroke,fill}%
\end{pgfscope}%
\begin{pgfscope}%
\definecolor{textcolor}{rgb}{0.150000,0.150000,0.150000}%
\pgfsetstrokecolor{textcolor}%
\pgfsetfillcolor{textcolor}%
\pgftext[x=2.513889in,y=2.322751in,left,base]{\color{textcolor}\sffamily\fontsize{10.000000}{12.000000}\selectfont train}%
\end{pgfscope}%
\begin{pgfscope}%
\pgfsetbuttcap%
\pgfsetroundjoin%
\definecolor{currentfill}{rgb}{0.000000,0.000000,0.000000}%
\pgfsetfillcolor{currentfill}%
\pgfsetlinewidth{0.301125pt}%
\definecolor{currentstroke}{rgb}{0.000000,0.000000,0.000000}%
\pgfsetstrokecolor{currentstroke}%
\pgfsetdash{}{0pt}%
\pgfpathmoveto{\pgfqpoint{2.263889in}{2.147216in}}%
\pgfpathcurveto{\pgfqpoint{2.268007in}{2.147216in}}{\pgfqpoint{2.271957in}{2.148852in}}{\pgfqpoint{2.274869in}{2.151764in}}%
\pgfpathcurveto{\pgfqpoint{2.277781in}{2.154676in}}{\pgfqpoint{2.279417in}{2.158626in}}{\pgfqpoint{2.279417in}{2.162744in}}%
\pgfpathcurveto{\pgfqpoint{2.279417in}{2.166862in}}{\pgfqpoint{2.277781in}{2.170812in}}{\pgfqpoint{2.274869in}{2.173724in}}%
\pgfpathcurveto{\pgfqpoint{2.271957in}{2.176636in}}{\pgfqpoint{2.268007in}{2.178272in}}{\pgfqpoint{2.263889in}{2.178272in}}%
\pgfpathcurveto{\pgfqpoint{2.259771in}{2.178272in}}{\pgfqpoint{2.255821in}{2.176636in}}{\pgfqpoint{2.252909in}{2.173724in}}%
\pgfpathcurveto{\pgfqpoint{2.249997in}{2.170812in}}{\pgfqpoint{2.248361in}{2.166862in}}{\pgfqpoint{2.248361in}{2.162744in}}%
\pgfpathcurveto{\pgfqpoint{2.248361in}{2.158626in}}{\pgfqpoint{2.249997in}{2.154676in}}{\pgfqpoint{2.252909in}{2.151764in}}%
\pgfpathcurveto{\pgfqpoint{2.255821in}{2.148852in}}{\pgfqpoint{2.259771in}{2.147216in}}{\pgfqpoint{2.263889in}{2.147216in}}%
\pgfpathclose%
\pgfusepath{stroke,fill}%
\end{pgfscope}%
\begin{pgfscope}%
\pgfsetbuttcap%
\pgfsetroundjoin%
\definecolor{currentfill}{rgb}{0.000000,0.000000,0.000000}%
\pgfsetfillcolor{currentfill}%
\pgfsetlinewidth{0.301125pt}%
\definecolor{currentstroke}{rgb}{0.000000,0.000000,0.000000}%
\pgfsetstrokecolor{currentstroke}%
\pgfsetdash{}{0pt}%
\pgfpathmoveto{\pgfqpoint{2.263889in}{2.147216in}}%
\pgfpathcurveto{\pgfqpoint{2.268007in}{2.147216in}}{\pgfqpoint{2.271957in}{2.148852in}}{\pgfqpoint{2.274869in}{2.151764in}}%
\pgfpathcurveto{\pgfqpoint{2.277781in}{2.154676in}}{\pgfqpoint{2.279417in}{2.158626in}}{\pgfqpoint{2.279417in}{2.162744in}}%
\pgfpathcurveto{\pgfqpoint{2.279417in}{2.166862in}}{\pgfqpoint{2.277781in}{2.170812in}}{\pgfqpoint{2.274869in}{2.173724in}}%
\pgfpathcurveto{\pgfqpoint{2.271957in}{2.176636in}}{\pgfqpoint{2.268007in}{2.178272in}}{\pgfqpoint{2.263889in}{2.178272in}}%
\pgfpathcurveto{\pgfqpoint{2.259771in}{2.178272in}}{\pgfqpoint{2.255821in}{2.176636in}}{\pgfqpoint{2.252909in}{2.173724in}}%
\pgfpathcurveto{\pgfqpoint{2.249997in}{2.170812in}}{\pgfqpoint{2.248361in}{2.166862in}}{\pgfqpoint{2.248361in}{2.162744in}}%
\pgfpathcurveto{\pgfqpoint{2.248361in}{2.158626in}}{\pgfqpoint{2.249997in}{2.154676in}}{\pgfqpoint{2.252909in}{2.151764in}}%
\pgfpathcurveto{\pgfqpoint{2.255821in}{2.148852in}}{\pgfqpoint{2.259771in}{2.147216in}}{\pgfqpoint{2.263889in}{2.147216in}}%
\pgfpathclose%
\pgfusepath{stroke,fill}%
\end{pgfscope}%
\begin{pgfscope}%
\pgfsetbuttcap%
\pgfsetroundjoin%
\definecolor{currentfill}{rgb}{0.000000,0.000000,0.000000}%
\pgfsetfillcolor{currentfill}%
\pgfsetlinewidth{0.301125pt}%
\definecolor{currentstroke}{rgb}{0.000000,0.000000,0.000000}%
\pgfsetstrokecolor{currentstroke}%
\pgfsetdash{}{0pt}%
\pgfpathmoveto{\pgfqpoint{2.263889in}{2.147216in}}%
\pgfpathcurveto{\pgfqpoint{2.268007in}{2.147216in}}{\pgfqpoint{2.271957in}{2.148852in}}{\pgfqpoint{2.274869in}{2.151764in}}%
\pgfpathcurveto{\pgfqpoint{2.277781in}{2.154676in}}{\pgfqpoint{2.279417in}{2.158626in}}{\pgfqpoint{2.279417in}{2.162744in}}%
\pgfpathcurveto{\pgfqpoint{2.279417in}{2.166862in}}{\pgfqpoint{2.277781in}{2.170812in}}{\pgfqpoint{2.274869in}{2.173724in}}%
\pgfpathcurveto{\pgfqpoint{2.271957in}{2.176636in}}{\pgfqpoint{2.268007in}{2.178272in}}{\pgfqpoint{2.263889in}{2.178272in}}%
\pgfpathcurveto{\pgfqpoint{2.259771in}{2.178272in}}{\pgfqpoint{2.255821in}{2.176636in}}{\pgfqpoint{2.252909in}{2.173724in}}%
\pgfpathcurveto{\pgfqpoint{2.249997in}{2.170812in}}{\pgfqpoint{2.248361in}{2.166862in}}{\pgfqpoint{2.248361in}{2.162744in}}%
\pgfpathcurveto{\pgfqpoint{2.248361in}{2.158626in}}{\pgfqpoint{2.249997in}{2.154676in}}{\pgfqpoint{2.252909in}{2.151764in}}%
\pgfpathcurveto{\pgfqpoint{2.255821in}{2.148852in}}{\pgfqpoint{2.259771in}{2.147216in}}{\pgfqpoint{2.263889in}{2.147216in}}%
\pgfpathclose%
\pgfusepath{stroke,fill}%
\end{pgfscope}%
\begin{pgfscope}%
\definecolor{textcolor}{rgb}{0.150000,0.150000,0.150000}%
\pgfsetstrokecolor{textcolor}%
\pgfsetfillcolor{textcolor}%
\pgftext[x=2.513889in,y=2.126285in,left,base]{\color{textcolor}\sffamily\fontsize{10.000000}{12.000000}\selectfont test}%
\end{pgfscope}%
\begin{pgfscope}%
\pgfsetbuttcap%
\pgfsetmiterjoin%
\definecolor{currentfill}{rgb}{1.000000,1.000000,1.000000}%
\pgfsetfillcolor{currentfill}%
\pgfsetlinewidth{0.000000pt}%
\definecolor{currentstroke}{rgb}{0.000000,0.000000,0.000000}%
\pgfsetstrokecolor{currentstroke}%
\pgfsetstrokeopacity{0.000000}%
\pgfsetdash{}{0pt}%
\pgfpathmoveto{\pgfqpoint{7.105882in}{1.973684in}}%
\pgfpathlineto{\pgfqpoint{11.482353in}{1.973684in}}%
\pgfpathlineto{\pgfqpoint{11.482353in}{2.952632in}}%
\pgfpathlineto{\pgfqpoint{7.105882in}{2.952632in}}%
\pgfpathclose%
\pgfusepath{fill}%
\end{pgfscope}%
\begin{pgfscope}%
\pgfpathrectangle{\pgfqpoint{7.105882in}{1.973684in}}{\pgfqpoint{4.376471in}{0.978947in}} %
\pgfusepath{clip}%
\pgfsetroundcap%
\pgfsetroundjoin%
\pgfsetlinewidth{1.003750pt}%
\definecolor{currentstroke}{rgb}{0.800000,0.800000,0.800000}%
\pgfsetstrokecolor{currentstroke}%
\pgfsetdash{}{0pt}%
\pgfpathmoveto{\pgfqpoint{7.105882in}{1.973684in}}%
\pgfpathlineto{\pgfqpoint{7.105882in}{2.952632in}}%
\pgfusepath{stroke}%
\end{pgfscope}%
\begin{pgfscope}%
\pgfpathrectangle{\pgfqpoint{7.105882in}{1.973684in}}{\pgfqpoint{4.376471in}{0.978947in}} %
\pgfusepath{clip}%
\pgfsetroundcap%
\pgfsetroundjoin%
\pgfsetlinewidth{1.003750pt}%
\definecolor{currentstroke}{rgb}{0.800000,0.800000,0.800000}%
\pgfsetstrokecolor{currentstroke}%
\pgfsetdash{}{0pt}%
\pgfpathmoveto{\pgfqpoint{7.981176in}{1.973684in}}%
\pgfpathlineto{\pgfqpoint{7.981176in}{2.952632in}}%
\pgfusepath{stroke}%
\end{pgfscope}%
\begin{pgfscope}%
\pgfpathrectangle{\pgfqpoint{7.105882in}{1.973684in}}{\pgfqpoint{4.376471in}{0.978947in}} %
\pgfusepath{clip}%
\pgfsetroundcap%
\pgfsetroundjoin%
\pgfsetlinewidth{1.003750pt}%
\definecolor{currentstroke}{rgb}{0.800000,0.800000,0.800000}%
\pgfsetstrokecolor{currentstroke}%
\pgfsetdash{}{0pt}%
\pgfpathmoveto{\pgfqpoint{8.856471in}{1.973684in}}%
\pgfpathlineto{\pgfqpoint{8.856471in}{2.952632in}}%
\pgfusepath{stroke}%
\end{pgfscope}%
\begin{pgfscope}%
\pgfpathrectangle{\pgfqpoint{7.105882in}{1.973684in}}{\pgfqpoint{4.376471in}{0.978947in}} %
\pgfusepath{clip}%
\pgfsetroundcap%
\pgfsetroundjoin%
\pgfsetlinewidth{1.003750pt}%
\definecolor{currentstroke}{rgb}{0.800000,0.800000,0.800000}%
\pgfsetstrokecolor{currentstroke}%
\pgfsetdash{}{0pt}%
\pgfpathmoveto{\pgfqpoint{9.731765in}{1.973684in}}%
\pgfpathlineto{\pgfqpoint{9.731765in}{2.952632in}}%
\pgfusepath{stroke}%
\end{pgfscope}%
\begin{pgfscope}%
\pgfpathrectangle{\pgfqpoint{7.105882in}{1.973684in}}{\pgfqpoint{4.376471in}{0.978947in}} %
\pgfusepath{clip}%
\pgfsetroundcap%
\pgfsetroundjoin%
\pgfsetlinewidth{1.003750pt}%
\definecolor{currentstroke}{rgb}{0.800000,0.800000,0.800000}%
\pgfsetstrokecolor{currentstroke}%
\pgfsetdash{}{0pt}%
\pgfpathmoveto{\pgfqpoint{10.607059in}{1.973684in}}%
\pgfpathlineto{\pgfqpoint{10.607059in}{2.952632in}}%
\pgfusepath{stroke}%
\end{pgfscope}%
\begin{pgfscope}%
\pgfpathrectangle{\pgfqpoint{7.105882in}{1.973684in}}{\pgfqpoint{4.376471in}{0.978947in}} %
\pgfusepath{clip}%
\pgfsetroundcap%
\pgfsetroundjoin%
\pgfsetlinewidth{1.003750pt}%
\definecolor{currentstroke}{rgb}{0.800000,0.800000,0.800000}%
\pgfsetstrokecolor{currentstroke}%
\pgfsetdash{}{0pt}%
\pgfpathmoveto{\pgfqpoint{11.482353in}{1.973684in}}%
\pgfpathlineto{\pgfqpoint{11.482353in}{2.952632in}}%
\pgfusepath{stroke}%
\end{pgfscope}%
\begin{pgfscope}%
\pgfpathrectangle{\pgfqpoint{7.105882in}{1.973684in}}{\pgfqpoint{4.376471in}{0.978947in}} %
\pgfusepath{clip}%
\pgfsetroundcap%
\pgfsetroundjoin%
\pgfsetlinewidth{1.003750pt}%
\definecolor{currentstroke}{rgb}{0.800000,0.800000,0.800000}%
\pgfsetstrokecolor{currentstroke}%
\pgfsetdash{}{0pt}%
\pgfpathmoveto{\pgfqpoint{7.105882in}{2.136842in}}%
\pgfpathlineto{\pgfqpoint{11.482353in}{2.136842in}}%
\pgfusepath{stroke}%
\end{pgfscope}%
\begin{pgfscope}%
\definecolor{textcolor}{rgb}{0.150000,0.150000,0.150000}%
\pgfsetstrokecolor{textcolor}%
\pgfsetfillcolor{textcolor}%
\pgftext[x=7.008660in,y=2.136842in,right,]{\color{textcolor}\sffamily\fontsize{10.000000}{12.000000}\selectfont \(\displaystyle -1\)}%
\end{pgfscope}%
\begin{pgfscope}%
\pgfpathrectangle{\pgfqpoint{7.105882in}{1.973684in}}{\pgfqpoint{4.376471in}{0.978947in}} %
\pgfusepath{clip}%
\pgfsetroundcap%
\pgfsetroundjoin%
\pgfsetlinewidth{1.003750pt}%
\definecolor{currentstroke}{rgb}{0.800000,0.800000,0.800000}%
\pgfsetstrokecolor{currentstroke}%
\pgfsetdash{}{0pt}%
\pgfpathmoveto{\pgfqpoint{7.105882in}{2.340789in}}%
\pgfpathlineto{\pgfqpoint{11.482353in}{2.340789in}}%
\pgfusepath{stroke}%
\end{pgfscope}%
\begin{pgfscope}%
\definecolor{textcolor}{rgb}{0.150000,0.150000,0.150000}%
\pgfsetstrokecolor{textcolor}%
\pgfsetfillcolor{textcolor}%
\pgftext[x=7.008660in,y=2.340789in,right,]{\color{textcolor}\sffamily\fontsize{10.000000}{12.000000}\selectfont \(\displaystyle 0\)}%
\end{pgfscope}%
\begin{pgfscope}%
\pgfpathrectangle{\pgfqpoint{7.105882in}{1.973684in}}{\pgfqpoint{4.376471in}{0.978947in}} %
\pgfusepath{clip}%
\pgfsetroundcap%
\pgfsetroundjoin%
\pgfsetlinewidth{1.003750pt}%
\definecolor{currentstroke}{rgb}{0.800000,0.800000,0.800000}%
\pgfsetstrokecolor{currentstroke}%
\pgfsetdash{}{0pt}%
\pgfpathmoveto{\pgfqpoint{7.105882in}{2.544737in}}%
\pgfpathlineto{\pgfqpoint{11.482353in}{2.544737in}}%
\pgfusepath{stroke}%
\end{pgfscope}%
\begin{pgfscope}%
\definecolor{textcolor}{rgb}{0.150000,0.150000,0.150000}%
\pgfsetstrokecolor{textcolor}%
\pgfsetfillcolor{textcolor}%
\pgftext[x=7.008660in,y=2.544737in,right,]{\color{textcolor}\sffamily\fontsize{10.000000}{12.000000}\selectfont \(\displaystyle 1\)}%
\end{pgfscope}%
\begin{pgfscope}%
\pgfpathrectangle{\pgfqpoint{7.105882in}{1.973684in}}{\pgfqpoint{4.376471in}{0.978947in}} %
\pgfusepath{clip}%
\pgfsetroundcap%
\pgfsetroundjoin%
\pgfsetlinewidth{1.003750pt}%
\definecolor{currentstroke}{rgb}{0.800000,0.800000,0.800000}%
\pgfsetstrokecolor{currentstroke}%
\pgfsetdash{}{0pt}%
\pgfpathmoveto{\pgfqpoint{7.105882in}{2.748684in}}%
\pgfpathlineto{\pgfqpoint{11.482353in}{2.748684in}}%
\pgfusepath{stroke}%
\end{pgfscope}%
\begin{pgfscope}%
\definecolor{textcolor}{rgb}{0.150000,0.150000,0.150000}%
\pgfsetstrokecolor{textcolor}%
\pgfsetfillcolor{textcolor}%
\pgftext[x=7.008660in,y=2.748684in,right,]{\color{textcolor}\sffamily\fontsize{10.000000}{12.000000}\selectfont \(\displaystyle 2\)}%
\end{pgfscope}%
\begin{pgfscope}%
\pgfpathrectangle{\pgfqpoint{7.105882in}{1.973684in}}{\pgfqpoint{4.376471in}{0.978947in}} %
\pgfusepath{clip}%
\pgfsetroundcap%
\pgfsetroundjoin%
\pgfsetlinewidth{1.003750pt}%
\definecolor{currentstroke}{rgb}{0.800000,0.800000,0.800000}%
\pgfsetstrokecolor{currentstroke}%
\pgfsetdash{}{0pt}%
\pgfpathmoveto{\pgfqpoint{7.105882in}{2.952632in}}%
\pgfpathlineto{\pgfqpoint{11.482353in}{2.952632in}}%
\pgfusepath{stroke}%
\end{pgfscope}%
\begin{pgfscope}%
\definecolor{textcolor}{rgb}{0.150000,0.150000,0.150000}%
\pgfsetstrokecolor{textcolor}%
\pgfsetfillcolor{textcolor}%
\pgftext[x=7.008660in,y=2.952632in,right,]{\color{textcolor}\sffamily\fontsize{10.000000}{12.000000}\selectfont \(\displaystyle 3\)}%
\end{pgfscope}%
\begin{pgfscope}%
\pgfpathrectangle{\pgfqpoint{7.105882in}{1.973684in}}{\pgfqpoint{4.376471in}{0.978947in}} %
\pgfusepath{clip}%
\pgfsetbuttcap%
\pgfsetroundjoin%
\definecolor{currentfill}{rgb}{1.000000,0.000000,0.000000}%
\pgfsetfillcolor{currentfill}%
\pgfsetlinewidth{2.007500pt}%
\definecolor{currentstroke}{rgb}{1.000000,0.000000,0.000000}%
\pgfsetstrokecolor{currentstroke}%
\pgfsetdash{}{0pt}%
\pgfpathmoveto{\pgfqpoint{9.871613in}{2.531260in}}%
\pgfpathlineto{\pgfqpoint{9.933726in}{2.531260in}}%
\pgfpathmoveto{\pgfqpoint{9.902669in}{2.500203in}}%
\pgfpathlineto{\pgfqpoint{9.902669in}{2.562316in}}%
\pgfusepath{stroke,fill}%
\end{pgfscope}%
\begin{pgfscope}%
\pgfpathrectangle{\pgfqpoint{7.105882in}{1.973684in}}{\pgfqpoint{4.376471in}{0.978947in}} %
\pgfusepath{clip}%
\pgfsetbuttcap%
\pgfsetroundjoin%
\definecolor{currentfill}{rgb}{1.000000,0.000000,0.000000}%
\pgfsetfillcolor{currentfill}%
\pgfsetlinewidth{2.007500pt}%
\definecolor{currentstroke}{rgb}{1.000000,0.000000,0.000000}%
\pgfsetstrokecolor{currentstroke}%
\pgfsetdash{}{0pt}%
\pgfpathmoveto{\pgfqpoint{10.454124in}{2.214048in}}%
\pgfpathlineto{\pgfqpoint{10.516237in}{2.214048in}}%
\pgfpathmoveto{\pgfqpoint{10.485181in}{2.182991in}}%
\pgfpathlineto{\pgfqpoint{10.485181in}{2.245104in}}%
\pgfusepath{stroke,fill}%
\end{pgfscope}%
\begin{pgfscope}%
\pgfpathrectangle{\pgfqpoint{7.105882in}{1.973684in}}{\pgfqpoint{4.376471in}{0.978947in}} %
\pgfusepath{clip}%
\pgfsetbuttcap%
\pgfsetroundjoin%
\definecolor{currentfill}{rgb}{1.000000,0.000000,0.000000}%
\pgfsetfillcolor{currentfill}%
\pgfsetlinewidth{2.007500pt}%
\definecolor{currentstroke}{rgb}{1.000000,0.000000,0.000000}%
\pgfsetstrokecolor{currentstroke}%
\pgfsetdash{}{0pt}%
\pgfpathmoveto{\pgfqpoint{10.060501in}{2.584788in}}%
\pgfpathlineto{\pgfqpoint{10.122614in}{2.584788in}}%
\pgfpathmoveto{\pgfqpoint{10.091557in}{2.553731in}}%
\pgfpathlineto{\pgfqpoint{10.091557in}{2.615844in}}%
\pgfusepath{stroke,fill}%
\end{pgfscope}%
\begin{pgfscope}%
\pgfpathrectangle{\pgfqpoint{7.105882in}{1.973684in}}{\pgfqpoint{4.376471in}{0.978947in}} %
\pgfusepath{clip}%
\pgfsetbuttcap%
\pgfsetroundjoin%
\definecolor{currentfill}{rgb}{1.000000,0.000000,0.000000}%
\pgfsetfillcolor{currentfill}%
\pgfsetlinewidth{2.007500pt}%
\definecolor{currentstroke}{rgb}{1.000000,0.000000,0.000000}%
\pgfsetstrokecolor{currentstroke}%
\pgfsetdash{}{0pt}%
\pgfpathmoveto{\pgfqpoint{9.857852in}{2.473876in}}%
\pgfpathlineto{\pgfqpoint{9.919965in}{2.473876in}}%
\pgfpathmoveto{\pgfqpoint{9.888909in}{2.442819in}}%
\pgfpathlineto{\pgfqpoint{9.888909in}{2.504932in}}%
\pgfusepath{stroke,fill}%
\end{pgfscope}%
\begin{pgfscope}%
\pgfpathrectangle{\pgfqpoint{7.105882in}{1.973684in}}{\pgfqpoint{4.376471in}{0.978947in}} %
\pgfusepath{clip}%
\pgfsetbuttcap%
\pgfsetroundjoin%
\definecolor{currentfill}{rgb}{1.000000,0.000000,0.000000}%
\pgfsetfillcolor{currentfill}%
\pgfsetlinewidth{2.007500pt}%
\definecolor{currentstroke}{rgb}{1.000000,0.000000,0.000000}%
\pgfsetstrokecolor{currentstroke}%
\pgfsetdash{}{0pt}%
\pgfpathmoveto{\pgfqpoint{9.433410in}{2.147482in}}%
\pgfpathlineto{\pgfqpoint{9.495523in}{2.147482in}}%
\pgfpathmoveto{\pgfqpoint{9.464467in}{2.116425in}}%
\pgfpathlineto{\pgfqpoint{9.464467in}{2.178538in}}%
\pgfusepath{stroke,fill}%
\end{pgfscope}%
\begin{pgfscope}%
\pgfpathrectangle{\pgfqpoint{7.105882in}{1.973684in}}{\pgfqpoint{4.376471in}{0.978947in}} %
\pgfusepath{clip}%
\pgfsetbuttcap%
\pgfsetroundjoin%
\definecolor{currentfill}{rgb}{1.000000,0.000000,0.000000}%
\pgfsetfillcolor{currentfill}%
\pgfsetlinewidth{2.007500pt}%
\definecolor{currentstroke}{rgb}{1.000000,0.000000,0.000000}%
\pgfsetstrokecolor{currentstroke}%
\pgfsetdash{}{0pt}%
\pgfpathmoveto{\pgfqpoint{10.211509in}{2.416946in}}%
\pgfpathlineto{\pgfqpoint{10.273622in}{2.416946in}}%
\pgfpathmoveto{\pgfqpoint{10.242566in}{2.385889in}}%
\pgfpathlineto{\pgfqpoint{10.242566in}{2.448002in}}%
\pgfusepath{stroke,fill}%
\end{pgfscope}%
\begin{pgfscope}%
\pgfpathrectangle{\pgfqpoint{7.105882in}{1.973684in}}{\pgfqpoint{4.376471in}{0.978947in}} %
\pgfusepath{clip}%
\pgfsetbuttcap%
\pgfsetroundjoin%
\definecolor{currentfill}{rgb}{1.000000,0.000000,0.000000}%
\pgfsetfillcolor{currentfill}%
\pgfsetlinewidth{2.007500pt}%
\definecolor{currentstroke}{rgb}{1.000000,0.000000,0.000000}%
\pgfsetstrokecolor{currentstroke}%
\pgfsetdash{}{0pt}%
\pgfpathmoveto{\pgfqpoint{9.482190in}{2.184975in}}%
\pgfpathlineto{\pgfqpoint{9.544303in}{2.184975in}}%
\pgfpathmoveto{\pgfqpoint{9.513247in}{2.153919in}}%
\pgfpathlineto{\pgfqpoint{9.513247in}{2.216032in}}%
\pgfusepath{stroke,fill}%
\end{pgfscope}%
\begin{pgfscope}%
\pgfpathrectangle{\pgfqpoint{7.105882in}{1.973684in}}{\pgfqpoint{4.376471in}{0.978947in}} %
\pgfusepath{clip}%
\pgfsetbuttcap%
\pgfsetroundjoin%
\definecolor{currentfill}{rgb}{1.000000,0.000000,0.000000}%
\pgfsetfillcolor{currentfill}%
\pgfsetlinewidth{2.007500pt}%
\definecolor{currentstroke}{rgb}{1.000000,0.000000,0.000000}%
\pgfsetstrokecolor{currentstroke}%
\pgfsetdash{}{0pt}%
\pgfpathmoveto{\pgfqpoint{11.072375in}{2.825094in}}%
\pgfpathlineto{\pgfqpoint{11.134488in}{2.825094in}}%
\pgfpathmoveto{\pgfqpoint{11.103431in}{2.794037in}}%
\pgfpathlineto{\pgfqpoint{11.103431in}{2.856150in}}%
\pgfusepath{stroke,fill}%
\end{pgfscope}%
\begin{pgfscope}%
\pgfpathrectangle{\pgfqpoint{7.105882in}{1.973684in}}{\pgfqpoint{4.376471in}{0.978947in}} %
\pgfusepath{clip}%
\pgfsetbuttcap%
\pgfsetroundjoin%
\definecolor{currentfill}{rgb}{1.000000,0.000000,0.000000}%
\pgfsetfillcolor{currentfill}%
\pgfsetlinewidth{2.007500pt}%
\definecolor{currentstroke}{rgb}{1.000000,0.000000,0.000000}%
\pgfsetstrokecolor{currentstroke}%
\pgfsetdash{}{0pt}%
\pgfpathmoveto{\pgfqpoint{11.324073in}{2.725496in}}%
\pgfpathlineto{\pgfqpoint{11.386186in}{2.725496in}}%
\pgfpathmoveto{\pgfqpoint{11.355130in}{2.694440in}}%
\pgfpathlineto{\pgfqpoint{11.355130in}{2.756553in}}%
\pgfusepath{stroke,fill}%
\end{pgfscope}%
\begin{pgfscope}%
\pgfpathrectangle{\pgfqpoint{7.105882in}{1.973684in}}{\pgfqpoint{4.376471in}{0.978947in}} %
\pgfusepath{clip}%
\pgfsetbuttcap%
\pgfsetroundjoin%
\definecolor{currentfill}{rgb}{1.000000,0.000000,0.000000}%
\pgfsetfillcolor{currentfill}%
\pgfsetlinewidth{2.007500pt}%
\definecolor{currentstroke}{rgb}{1.000000,0.000000,0.000000}%
\pgfsetstrokecolor{currentstroke}%
\pgfsetdash{}{0pt}%
\pgfpathmoveto{\pgfqpoint{9.292616in}{2.222892in}}%
\pgfpathlineto{\pgfqpoint{9.354729in}{2.222892in}}%
\pgfpathmoveto{\pgfqpoint{9.323673in}{2.191836in}}%
\pgfpathlineto{\pgfqpoint{9.323673in}{2.253949in}}%
\pgfusepath{stroke,fill}%
\end{pgfscope}%
\begin{pgfscope}%
\pgfpathrectangle{\pgfqpoint{7.105882in}{1.973684in}}{\pgfqpoint{4.376471in}{0.978947in}} %
\pgfusepath{clip}%
\pgfsetbuttcap%
\pgfsetroundjoin%
\definecolor{currentfill}{rgb}{1.000000,0.000000,0.000000}%
\pgfsetfillcolor{currentfill}%
\pgfsetlinewidth{2.007500pt}%
\definecolor{currentstroke}{rgb}{1.000000,0.000000,0.000000}%
\pgfsetstrokecolor{currentstroke}%
\pgfsetdash{}{0pt}%
\pgfpathmoveto{\pgfqpoint{10.722089in}{2.373068in}}%
\pgfpathlineto{\pgfqpoint{10.784202in}{2.373068in}}%
\pgfpathmoveto{\pgfqpoint{10.753146in}{2.342011in}}%
\pgfpathlineto{\pgfqpoint{10.753146in}{2.404124in}}%
\pgfusepath{stroke,fill}%
\end{pgfscope}%
\begin{pgfscope}%
\pgfpathrectangle{\pgfqpoint{7.105882in}{1.973684in}}{\pgfqpoint{4.376471in}{0.978947in}} %
\pgfusepath{clip}%
\pgfsetbuttcap%
\pgfsetroundjoin%
\definecolor{currentfill}{rgb}{1.000000,0.000000,0.000000}%
\pgfsetfillcolor{currentfill}%
\pgfsetlinewidth{2.007500pt}%
\definecolor{currentstroke}{rgb}{1.000000,0.000000,0.000000}%
\pgfsetstrokecolor{currentstroke}%
\pgfsetdash{}{0pt}%
\pgfpathmoveto{\pgfqpoint{9.801874in}{2.413163in}}%
\pgfpathlineto{\pgfqpoint{9.863987in}{2.413163in}}%
\pgfpathmoveto{\pgfqpoint{9.832931in}{2.382107in}}%
\pgfpathlineto{\pgfqpoint{9.832931in}{2.444220in}}%
\pgfusepath{stroke,fill}%
\end{pgfscope}%
\begin{pgfscope}%
\pgfpathrectangle{\pgfqpoint{7.105882in}{1.973684in}}{\pgfqpoint{4.376471in}{0.978947in}} %
\pgfusepath{clip}%
\pgfsetbuttcap%
\pgfsetroundjoin%
\definecolor{currentfill}{rgb}{1.000000,0.000000,0.000000}%
\pgfsetfillcolor{currentfill}%
\pgfsetlinewidth{2.007500pt}%
\definecolor{currentstroke}{rgb}{1.000000,0.000000,0.000000}%
\pgfsetstrokecolor{currentstroke}%
\pgfsetdash{}{0pt}%
\pgfpathmoveto{\pgfqpoint{9.938944in}{2.540719in}}%
\pgfpathlineto{\pgfqpoint{10.001057in}{2.540719in}}%
\pgfpathmoveto{\pgfqpoint{9.970001in}{2.509663in}}%
\pgfpathlineto{\pgfqpoint{9.970001in}{2.571776in}}%
\pgfusepath{stroke,fill}%
\end{pgfscope}%
\begin{pgfscope}%
\pgfpathrectangle{\pgfqpoint{7.105882in}{1.973684in}}{\pgfqpoint{4.376471in}{0.978947in}} %
\pgfusepath{clip}%
\pgfsetbuttcap%
\pgfsetroundjoin%
\definecolor{currentfill}{rgb}{1.000000,0.000000,0.000000}%
\pgfsetfillcolor{currentfill}%
\pgfsetlinewidth{2.007500pt}%
\definecolor{currentstroke}{rgb}{1.000000,0.000000,0.000000}%
\pgfsetstrokecolor{currentstroke}%
\pgfsetdash{}{0pt}%
\pgfpathmoveto{\pgfqpoint{11.190797in}{2.800883in}}%
\pgfpathlineto{\pgfqpoint{11.252910in}{2.800883in}}%
\pgfpathmoveto{\pgfqpoint{11.221854in}{2.769826in}}%
\pgfpathlineto{\pgfqpoint{11.221854in}{2.831939in}}%
\pgfusepath{stroke,fill}%
\end{pgfscope}%
\begin{pgfscope}%
\pgfpathrectangle{\pgfqpoint{7.105882in}{1.973684in}}{\pgfqpoint{4.376471in}{0.978947in}} %
\pgfusepath{clip}%
\pgfsetbuttcap%
\pgfsetroundjoin%
\definecolor{currentfill}{rgb}{1.000000,0.000000,0.000000}%
\pgfsetfillcolor{currentfill}%
\pgfsetlinewidth{2.007500pt}%
\definecolor{currentstroke}{rgb}{1.000000,0.000000,0.000000}%
\pgfsetstrokecolor{currentstroke}%
\pgfsetdash{}{0pt}%
\pgfpathmoveto{\pgfqpoint{8.198830in}{2.513494in}}%
\pgfpathlineto{\pgfqpoint{8.260943in}{2.513494in}}%
\pgfpathmoveto{\pgfqpoint{8.229886in}{2.482437in}}%
\pgfpathlineto{\pgfqpoint{8.229886in}{2.544550in}}%
\pgfusepath{stroke,fill}%
\end{pgfscope}%
\begin{pgfscope}%
\pgfpathrectangle{\pgfqpoint{7.105882in}{1.973684in}}{\pgfqpoint{4.376471in}{0.978947in}} %
\pgfusepath{clip}%
\pgfsetbuttcap%
\pgfsetroundjoin%
\definecolor{currentfill}{rgb}{1.000000,0.000000,0.000000}%
\pgfsetfillcolor{currentfill}%
\pgfsetlinewidth{2.007500pt}%
\definecolor{currentstroke}{rgb}{1.000000,0.000000,0.000000}%
\pgfsetstrokecolor{currentstroke}%
\pgfsetdash{}{0pt}%
\pgfpathmoveto{\pgfqpoint{8.255175in}{2.455860in}}%
\pgfpathlineto{\pgfqpoint{8.317288in}{2.455860in}}%
\pgfpathmoveto{\pgfqpoint{8.286232in}{2.424803in}}%
\pgfpathlineto{\pgfqpoint{8.286232in}{2.486916in}}%
\pgfusepath{stroke,fill}%
\end{pgfscope}%
\begin{pgfscope}%
\pgfpathrectangle{\pgfqpoint{7.105882in}{1.973684in}}{\pgfqpoint{4.376471in}{0.978947in}} %
\pgfusepath{clip}%
\pgfsetbuttcap%
\pgfsetroundjoin%
\definecolor{currentfill}{rgb}{1.000000,0.000000,0.000000}%
\pgfsetfillcolor{currentfill}%
\pgfsetlinewidth{2.007500pt}%
\definecolor{currentstroke}{rgb}{1.000000,0.000000,0.000000}%
\pgfsetstrokecolor{currentstroke}%
\pgfsetdash{}{0pt}%
\pgfpathmoveto{\pgfqpoint{8.020908in}{2.743140in}}%
\pgfpathlineto{\pgfqpoint{8.083021in}{2.743140in}}%
\pgfpathmoveto{\pgfqpoint{8.051965in}{2.712083in}}%
\pgfpathlineto{\pgfqpoint{8.051965in}{2.774196in}}%
\pgfusepath{stroke,fill}%
\end{pgfscope}%
\begin{pgfscope}%
\pgfpathrectangle{\pgfqpoint{7.105882in}{1.973684in}}{\pgfqpoint{4.376471in}{0.978947in}} %
\pgfusepath{clip}%
\pgfsetbuttcap%
\pgfsetroundjoin%
\definecolor{currentfill}{rgb}{1.000000,0.000000,0.000000}%
\pgfsetfillcolor{currentfill}%
\pgfsetlinewidth{2.007500pt}%
\definecolor{currentstroke}{rgb}{1.000000,0.000000,0.000000}%
\pgfsetstrokecolor{currentstroke}%
\pgfsetdash{}{0pt}%
\pgfpathmoveto{\pgfqpoint{10.865269in}{2.588736in}}%
\pgfpathlineto{\pgfqpoint{10.927382in}{2.588736in}}%
\pgfpathmoveto{\pgfqpoint{10.896325in}{2.557679in}}%
\pgfpathlineto{\pgfqpoint{10.896325in}{2.619792in}}%
\pgfusepath{stroke,fill}%
\end{pgfscope}%
\begin{pgfscope}%
\pgfpathrectangle{\pgfqpoint{7.105882in}{1.973684in}}{\pgfqpoint{4.376471in}{0.978947in}} %
\pgfusepath{clip}%
\pgfsetbuttcap%
\pgfsetroundjoin%
\definecolor{currentfill}{rgb}{1.000000,0.000000,0.000000}%
\pgfsetfillcolor{currentfill}%
\pgfsetlinewidth{2.007500pt}%
\definecolor{currentstroke}{rgb}{1.000000,0.000000,0.000000}%
\pgfsetstrokecolor{currentstroke}%
\pgfsetdash{}{0pt}%
\pgfpathmoveto{\pgfqpoint{10.674584in}{2.311154in}}%
\pgfpathlineto{\pgfqpoint{10.736697in}{2.311154in}}%
\pgfpathmoveto{\pgfqpoint{10.705641in}{2.280098in}}%
\pgfpathlineto{\pgfqpoint{10.705641in}{2.342211in}}%
\pgfusepath{stroke,fill}%
\end{pgfscope}%
\begin{pgfscope}%
\pgfpathrectangle{\pgfqpoint{7.105882in}{1.973684in}}{\pgfqpoint{4.376471in}{0.978947in}} %
\pgfusepath{clip}%
\pgfsetbuttcap%
\pgfsetroundjoin%
\definecolor{currentfill}{rgb}{1.000000,0.000000,0.000000}%
\pgfsetfillcolor{currentfill}%
\pgfsetlinewidth{2.007500pt}%
\definecolor{currentstroke}{rgb}{1.000000,0.000000,0.000000}%
\pgfsetstrokecolor{currentstroke}%
\pgfsetdash{}{0pt}%
\pgfpathmoveto{\pgfqpoint{10.996186in}{2.718517in}}%
\pgfpathlineto{\pgfqpoint{11.058299in}{2.718517in}}%
\pgfpathmoveto{\pgfqpoint{11.027243in}{2.687461in}}%
\pgfpathlineto{\pgfqpoint{11.027243in}{2.749574in}}%
\pgfusepath{stroke,fill}%
\end{pgfscope}%
\begin{pgfscope}%
\pgfpathrectangle{\pgfqpoint{7.105882in}{1.973684in}}{\pgfqpoint{4.376471in}{0.978947in}} %
\pgfusepath{clip}%
\pgfsetbuttcap%
\pgfsetroundjoin%
\definecolor{currentfill}{rgb}{1.000000,0.000000,0.000000}%
\pgfsetfillcolor{currentfill}%
\pgfsetlinewidth{2.007500pt}%
\definecolor{currentstroke}{rgb}{1.000000,0.000000,0.000000}%
\pgfsetstrokecolor{currentstroke}%
\pgfsetdash{}{0pt}%
\pgfpathmoveto{\pgfqpoint{11.376435in}{2.649754in}}%
\pgfpathlineto{\pgfqpoint{11.438548in}{2.649754in}}%
\pgfpathmoveto{\pgfqpoint{11.407492in}{2.618698in}}%
\pgfpathlineto{\pgfqpoint{11.407492in}{2.680811in}}%
\pgfusepath{stroke,fill}%
\end{pgfscope}%
\begin{pgfscope}%
\pgfpathrectangle{\pgfqpoint{7.105882in}{1.973684in}}{\pgfqpoint{4.376471in}{0.978947in}} %
\pgfusepath{clip}%
\pgfsetbuttcap%
\pgfsetroundjoin%
\definecolor{currentfill}{rgb}{1.000000,0.000000,0.000000}%
\pgfsetfillcolor{currentfill}%
\pgfsetlinewidth{2.007500pt}%
\definecolor{currentstroke}{rgb}{1.000000,0.000000,0.000000}%
\pgfsetstrokecolor{currentstroke}%
\pgfsetdash{}{0pt}%
\pgfpathmoveto{\pgfqpoint{10.748115in}{2.466321in}}%
\pgfpathlineto{\pgfqpoint{10.810228in}{2.466321in}}%
\pgfpathmoveto{\pgfqpoint{10.779172in}{2.435265in}}%
\pgfpathlineto{\pgfqpoint{10.779172in}{2.497378in}}%
\pgfusepath{stroke,fill}%
\end{pgfscope}%
\begin{pgfscope}%
\pgfpathrectangle{\pgfqpoint{7.105882in}{1.973684in}}{\pgfqpoint{4.376471in}{0.978947in}} %
\pgfusepath{clip}%
\pgfsetbuttcap%
\pgfsetroundjoin%
\definecolor{currentfill}{rgb}{1.000000,0.000000,0.000000}%
\pgfsetfillcolor{currentfill}%
\pgfsetlinewidth{2.007500pt}%
\definecolor{currentstroke}{rgb}{1.000000,0.000000,0.000000}%
\pgfsetstrokecolor{currentstroke}%
\pgfsetdash{}{0pt}%
\pgfpathmoveto{\pgfqpoint{9.565841in}{2.190781in}}%
\pgfpathlineto{\pgfqpoint{9.627954in}{2.190781in}}%
\pgfpathmoveto{\pgfqpoint{9.596897in}{2.159724in}}%
\pgfpathlineto{\pgfqpoint{9.596897in}{2.221837in}}%
\pgfusepath{stroke,fill}%
\end{pgfscope}%
\begin{pgfscope}%
\pgfpathrectangle{\pgfqpoint{7.105882in}{1.973684in}}{\pgfqpoint{4.376471in}{0.978947in}} %
\pgfusepath{clip}%
\pgfsetbuttcap%
\pgfsetroundjoin%
\definecolor{currentfill}{rgb}{1.000000,0.000000,0.000000}%
\pgfsetfillcolor{currentfill}%
\pgfsetlinewidth{2.007500pt}%
\definecolor{currentstroke}{rgb}{1.000000,0.000000,0.000000}%
\pgfsetstrokecolor{currentstroke}%
\pgfsetdash{}{0pt}%
\pgfpathmoveto{\pgfqpoint{10.682890in}{2.333191in}}%
\pgfpathlineto{\pgfqpoint{10.745003in}{2.333191in}}%
\pgfpathmoveto{\pgfqpoint{10.713947in}{2.302134in}}%
\pgfpathlineto{\pgfqpoint{10.713947in}{2.364247in}}%
\pgfusepath{stroke,fill}%
\end{pgfscope}%
\begin{pgfscope}%
\pgfpathrectangle{\pgfqpoint{7.105882in}{1.973684in}}{\pgfqpoint{4.376471in}{0.978947in}} %
\pgfusepath{clip}%
\pgfsetbuttcap%
\pgfsetroundjoin%
\definecolor{currentfill}{rgb}{1.000000,0.000000,0.000000}%
\pgfsetfillcolor{currentfill}%
\pgfsetlinewidth{2.007500pt}%
\definecolor{currentstroke}{rgb}{1.000000,0.000000,0.000000}%
\pgfsetstrokecolor{currentstroke}%
\pgfsetdash{}{0pt}%
\pgfpathmoveto{\pgfqpoint{8.364220in}{2.353928in}}%
\pgfpathlineto{\pgfqpoint{8.426333in}{2.353928in}}%
\pgfpathmoveto{\pgfqpoint{8.395276in}{2.322872in}}%
\pgfpathlineto{\pgfqpoint{8.395276in}{2.384985in}}%
\pgfusepath{stroke,fill}%
\end{pgfscope}%
\begin{pgfscope}%
\pgfpathrectangle{\pgfqpoint{7.105882in}{1.973684in}}{\pgfqpoint{4.376471in}{0.978947in}} %
\pgfusepath{clip}%
\pgfsetbuttcap%
\pgfsetroundjoin%
\definecolor{currentfill}{rgb}{1.000000,0.000000,0.000000}%
\pgfsetfillcolor{currentfill}%
\pgfsetlinewidth{2.007500pt}%
\definecolor{currentstroke}{rgb}{1.000000,0.000000,0.000000}%
\pgfsetstrokecolor{currentstroke}%
\pgfsetdash{}{0pt}%
\pgfpathmoveto{\pgfqpoint{10.190596in}{2.457212in}}%
\pgfpathlineto{\pgfqpoint{10.252709in}{2.457212in}}%
\pgfpathmoveto{\pgfqpoint{10.221653in}{2.426156in}}%
\pgfpathlineto{\pgfqpoint{10.221653in}{2.488269in}}%
\pgfusepath{stroke,fill}%
\end{pgfscope}%
\begin{pgfscope}%
\pgfpathrectangle{\pgfqpoint{7.105882in}{1.973684in}}{\pgfqpoint{4.376471in}{0.978947in}} %
\pgfusepath{clip}%
\pgfsetbuttcap%
\pgfsetroundjoin%
\definecolor{currentfill}{rgb}{1.000000,0.000000,0.000000}%
\pgfsetfillcolor{currentfill}%
\pgfsetlinewidth{2.007500pt}%
\definecolor{currentstroke}{rgb}{1.000000,0.000000,0.000000}%
\pgfsetstrokecolor{currentstroke}%
\pgfsetdash{}{0pt}%
\pgfpathmoveto{\pgfqpoint{8.452025in}{2.362218in}}%
\pgfpathlineto{\pgfqpoint{8.514138in}{2.362218in}}%
\pgfpathmoveto{\pgfqpoint{8.483082in}{2.331162in}}%
\pgfpathlineto{\pgfqpoint{8.483082in}{2.393275in}}%
\pgfusepath{stroke,fill}%
\end{pgfscope}%
\begin{pgfscope}%
\pgfpathrectangle{\pgfqpoint{7.105882in}{1.973684in}}{\pgfqpoint{4.376471in}{0.978947in}} %
\pgfusepath{clip}%
\pgfsetbuttcap%
\pgfsetroundjoin%
\definecolor{currentfill}{rgb}{1.000000,0.000000,0.000000}%
\pgfsetfillcolor{currentfill}%
\pgfsetlinewidth{2.007500pt}%
\definecolor{currentstroke}{rgb}{1.000000,0.000000,0.000000}%
\pgfsetstrokecolor{currentstroke}%
\pgfsetdash{}{0pt}%
\pgfpathmoveto{\pgfqpoint{11.257573in}{2.762412in}}%
\pgfpathlineto{\pgfqpoint{11.319686in}{2.762412in}}%
\pgfpathmoveto{\pgfqpoint{11.288629in}{2.731355in}}%
\pgfpathlineto{\pgfqpoint{11.288629in}{2.793468in}}%
\pgfusepath{stroke,fill}%
\end{pgfscope}%
\begin{pgfscope}%
\pgfpathrectangle{\pgfqpoint{7.105882in}{1.973684in}}{\pgfqpoint{4.376471in}{0.978947in}} %
\pgfusepath{clip}%
\pgfsetbuttcap%
\pgfsetroundjoin%
\definecolor{currentfill}{rgb}{1.000000,0.000000,0.000000}%
\pgfsetfillcolor{currentfill}%
\pgfsetlinewidth{2.007500pt}%
\definecolor{currentstroke}{rgb}{1.000000,0.000000,0.000000}%
\pgfsetstrokecolor{currentstroke}%
\pgfsetdash{}{0pt}%
\pgfpathmoveto{\pgfqpoint{9.777203in}{2.409615in}}%
\pgfpathlineto{\pgfqpoint{9.839316in}{2.409615in}}%
\pgfpathmoveto{\pgfqpoint{9.808260in}{2.378558in}}%
\pgfpathlineto{\pgfqpoint{9.808260in}{2.440671in}}%
\pgfusepath{stroke,fill}%
\end{pgfscope}%
\begin{pgfscope}%
\pgfpathrectangle{\pgfqpoint{7.105882in}{1.973684in}}{\pgfqpoint{4.376471in}{0.978947in}} %
\pgfusepath{clip}%
\pgfsetbuttcap%
\pgfsetroundjoin%
\definecolor{currentfill}{rgb}{1.000000,0.000000,0.000000}%
\pgfsetfillcolor{currentfill}%
\pgfsetlinewidth{2.007500pt}%
\definecolor{currentstroke}{rgb}{1.000000,0.000000,0.000000}%
\pgfsetstrokecolor{currentstroke}%
\pgfsetdash{}{0pt}%
\pgfpathmoveto{\pgfqpoint{9.401925in}{2.158497in}}%
\pgfpathlineto{\pgfqpoint{9.464038in}{2.158497in}}%
\pgfpathmoveto{\pgfqpoint{9.432981in}{2.127441in}}%
\pgfpathlineto{\pgfqpoint{9.432981in}{2.189554in}}%
\pgfusepath{stroke,fill}%
\end{pgfscope}%
\begin{pgfscope}%
\pgfpathrectangle{\pgfqpoint{7.105882in}{1.973684in}}{\pgfqpoint{4.376471in}{0.978947in}} %
\pgfusepath{clip}%
\pgfsetbuttcap%
\pgfsetroundjoin%
\definecolor{currentfill}{rgb}{0.000000,0.000000,0.000000}%
\pgfsetfillcolor{currentfill}%
\pgfsetlinewidth{0.301125pt}%
\definecolor{currentstroke}{rgb}{0.000000,0.000000,0.000000}%
\pgfsetstrokecolor{currentstroke}%
\pgfsetdash{}{0pt}%
\pgfsys@defobject{currentmarker}{\pgfqpoint{-0.015528in}{-0.015528in}}{\pgfqpoint{0.015528in}{0.015528in}}{%
\pgfpathmoveto{\pgfqpoint{0.000000in}{-0.015528in}}%
\pgfpathcurveto{\pgfqpoint{0.004118in}{-0.015528in}}{\pgfqpoint{0.008068in}{-0.013892in}}{\pgfqpoint{0.010980in}{-0.010980in}}%
\pgfpathcurveto{\pgfqpoint{0.013892in}{-0.008068in}}{\pgfqpoint{0.015528in}{-0.004118in}}{\pgfqpoint{0.015528in}{0.000000in}}%
\pgfpathcurveto{\pgfqpoint{0.015528in}{0.004118in}}{\pgfqpoint{0.013892in}{0.008068in}}{\pgfqpoint{0.010980in}{0.010980in}}%
\pgfpathcurveto{\pgfqpoint{0.008068in}{0.013892in}}{\pgfqpoint{0.004118in}{0.015528in}}{\pgfqpoint{0.000000in}{0.015528in}}%
\pgfpathcurveto{\pgfqpoint{-0.004118in}{0.015528in}}{\pgfqpoint{-0.008068in}{0.013892in}}{\pgfqpoint{-0.010980in}{0.010980in}}%
\pgfpathcurveto{\pgfqpoint{-0.013892in}{0.008068in}}{\pgfqpoint{-0.015528in}{0.004118in}}{\pgfqpoint{-0.015528in}{0.000000in}}%
\pgfpathcurveto{\pgfqpoint{-0.015528in}{-0.004118in}}{\pgfqpoint{-0.013892in}{-0.008068in}}{\pgfqpoint{-0.010980in}{-0.010980in}}%
\pgfpathcurveto{\pgfqpoint{-0.008068in}{-0.013892in}}{\pgfqpoint{-0.004118in}{-0.015528in}}{\pgfqpoint{0.000000in}{-0.015528in}}%
\pgfpathclose%
\pgfusepath{stroke,fill}%
}%
\begin{pgfscope}%
\pgfsys@transformshift{7.981176in}{2.807547in}%
\pgfsys@useobject{currentmarker}{}%
\end{pgfscope}%
\begin{pgfscope}%
\pgfsys@transformshift{7.998770in}{2.713355in}%
\pgfsys@useobject{currentmarker}{}%
\end{pgfscope}%
\begin{pgfscope}%
\pgfsys@transformshift{8.016364in}{2.804129in}%
\pgfsys@useobject{currentmarker}{}%
\end{pgfscope}%
\begin{pgfscope}%
\pgfsys@transformshift{8.033958in}{2.823266in}%
\pgfsys@useobject{currentmarker}{}%
\end{pgfscope}%
\begin{pgfscope}%
\pgfsys@transformshift{8.051552in}{2.758481in}%
\pgfsys@useobject{currentmarker}{}%
\end{pgfscope}%
\begin{pgfscope}%
\pgfsys@transformshift{8.069146in}{2.754380in}%
\pgfsys@useobject{currentmarker}{}%
\end{pgfscope}%
\begin{pgfscope}%
\pgfsys@transformshift{8.086740in}{2.630622in}%
\pgfsys@useobject{currentmarker}{}%
\end{pgfscope}%
\begin{pgfscope}%
\pgfsys@transformshift{8.104333in}{2.630227in}%
\pgfsys@useobject{currentmarker}{}%
\end{pgfscope}%
\begin{pgfscope}%
\pgfsys@transformshift{8.121927in}{2.570900in}%
\pgfsys@useobject{currentmarker}{}%
\end{pgfscope}%
\begin{pgfscope}%
\pgfsys@transformshift{8.139521in}{2.575614in}%
\pgfsys@useobject{currentmarker}{}%
\end{pgfscope}%
\begin{pgfscope}%
\pgfsys@transformshift{8.157115in}{2.502913in}%
\pgfsys@useobject{currentmarker}{}%
\end{pgfscope}%
\begin{pgfscope}%
\pgfsys@transformshift{8.174709in}{2.384290in}%
\pgfsys@useobject{currentmarker}{}%
\end{pgfscope}%
\begin{pgfscope}%
\pgfsys@transformshift{8.192303in}{2.554036in}%
\pgfsys@useobject{currentmarker}{}%
\end{pgfscope}%
\begin{pgfscope}%
\pgfsys@transformshift{8.209897in}{2.471885in}%
\pgfsys@useobject{currentmarker}{}%
\end{pgfscope}%
\begin{pgfscope}%
\pgfsys@transformshift{8.227490in}{2.324989in}%
\pgfsys@useobject{currentmarker}{}%
\end{pgfscope}%
\begin{pgfscope}%
\pgfsys@transformshift{8.245084in}{2.518392in}%
\pgfsys@useobject{currentmarker}{}%
\end{pgfscope}%
\begin{pgfscope}%
\pgfsys@transformshift{8.262678in}{2.360390in}%
\pgfsys@useobject{currentmarker}{}%
\end{pgfscope}%
\begin{pgfscope}%
\pgfsys@transformshift{8.280272in}{2.441767in}%
\pgfsys@useobject{currentmarker}{}%
\end{pgfscope}%
\begin{pgfscope}%
\pgfsys@transformshift{8.297866in}{2.496324in}%
\pgfsys@useobject{currentmarker}{}%
\end{pgfscope}%
\begin{pgfscope}%
\pgfsys@transformshift{8.315460in}{2.422660in}%
\pgfsys@useobject{currentmarker}{}%
\end{pgfscope}%
\begin{pgfscope}%
\pgfsys@transformshift{8.333054in}{2.515310in}%
\pgfsys@useobject{currentmarker}{}%
\end{pgfscope}%
\begin{pgfscope}%
\pgfsys@transformshift{8.350647in}{2.264938in}%
\pgfsys@useobject{currentmarker}{}%
\end{pgfscope}%
\begin{pgfscope}%
\pgfsys@transformshift{8.368241in}{2.425762in}%
\pgfsys@useobject{currentmarker}{}%
\end{pgfscope}%
\begin{pgfscope}%
\pgfsys@transformshift{8.385835in}{2.310913in}%
\pgfsys@useobject{currentmarker}{}%
\end{pgfscope}%
\begin{pgfscope}%
\pgfsys@transformshift{8.403429in}{2.290072in}%
\pgfsys@useobject{currentmarker}{}%
\end{pgfscope}%
\begin{pgfscope}%
\pgfsys@transformshift{8.421023in}{2.320035in}%
\pgfsys@useobject{currentmarker}{}%
\end{pgfscope}%
\begin{pgfscope}%
\pgfsys@transformshift{8.438617in}{2.349489in}%
\pgfsys@useobject{currentmarker}{}%
\end{pgfscope}%
\begin{pgfscope}%
\pgfsys@transformshift{8.456210in}{2.391104in}%
\pgfsys@useobject{currentmarker}{}%
\end{pgfscope}%
\begin{pgfscope}%
\pgfsys@transformshift{8.473804in}{2.272497in}%
\pgfsys@useobject{currentmarker}{}%
\end{pgfscope}%
\begin{pgfscope}%
\pgfsys@transformshift{8.491398in}{2.490806in}%
\pgfsys@useobject{currentmarker}{}%
\end{pgfscope}%
\begin{pgfscope}%
\pgfsys@transformshift{8.508992in}{2.455626in}%
\pgfsys@useobject{currentmarker}{}%
\end{pgfscope}%
\begin{pgfscope}%
\pgfsys@transformshift{8.526586in}{2.262090in}%
\pgfsys@useobject{currentmarker}{}%
\end{pgfscope}%
\begin{pgfscope}%
\pgfsys@transformshift{8.544180in}{2.582363in}%
\pgfsys@useobject{currentmarker}{}%
\end{pgfscope}%
\begin{pgfscope}%
\pgfsys@transformshift{8.561774in}{2.636855in}%
\pgfsys@useobject{currentmarker}{}%
\end{pgfscope}%
\begin{pgfscope}%
\pgfsys@transformshift{8.579367in}{2.577536in}%
\pgfsys@useobject{currentmarker}{}%
\end{pgfscope}%
\begin{pgfscope}%
\pgfsys@transformshift{8.596961in}{2.453483in}%
\pgfsys@useobject{currentmarker}{}%
\end{pgfscope}%
\begin{pgfscope}%
\pgfsys@transformshift{8.614555in}{2.377630in}%
\pgfsys@useobject{currentmarker}{}%
\end{pgfscope}%
\begin{pgfscope}%
\pgfsys@transformshift{8.632149in}{2.609618in}%
\pgfsys@useobject{currentmarker}{}%
\end{pgfscope}%
\begin{pgfscope}%
\pgfsys@transformshift{8.649743in}{2.476330in}%
\pgfsys@useobject{currentmarker}{}%
\end{pgfscope}%
\begin{pgfscope}%
\pgfsys@transformshift{8.667337in}{2.657325in}%
\pgfsys@useobject{currentmarker}{}%
\end{pgfscope}%
\begin{pgfscope}%
\pgfsys@transformshift{8.684931in}{2.568798in}%
\pgfsys@useobject{currentmarker}{}%
\end{pgfscope}%
\begin{pgfscope}%
\pgfsys@transformshift{8.702524in}{2.661511in}%
\pgfsys@useobject{currentmarker}{}%
\end{pgfscope}%
\begin{pgfscope}%
\pgfsys@transformshift{8.720118in}{2.611893in}%
\pgfsys@useobject{currentmarker}{}%
\end{pgfscope}%
\begin{pgfscope}%
\pgfsys@transformshift{8.737712in}{2.660326in}%
\pgfsys@useobject{currentmarker}{}%
\end{pgfscope}%
\begin{pgfscope}%
\pgfsys@transformshift{8.755306in}{2.600979in}%
\pgfsys@useobject{currentmarker}{}%
\end{pgfscope}%
\begin{pgfscope}%
\pgfsys@transformshift{8.772900in}{2.792401in}%
\pgfsys@useobject{currentmarker}{}%
\end{pgfscope}%
\begin{pgfscope}%
\pgfsys@transformshift{8.790494in}{2.632210in}%
\pgfsys@useobject{currentmarker}{}%
\end{pgfscope}%
\begin{pgfscope}%
\pgfsys@transformshift{8.808087in}{2.667702in}%
\pgfsys@useobject{currentmarker}{}%
\end{pgfscope}%
\begin{pgfscope}%
\pgfsys@transformshift{8.825681in}{2.824526in}%
\pgfsys@useobject{currentmarker}{}%
\end{pgfscope}%
\begin{pgfscope}%
\pgfsys@transformshift{8.843275in}{2.499083in}%
\pgfsys@useobject{currentmarker}{}%
\end{pgfscope}%
\begin{pgfscope}%
\pgfsys@transformshift{8.860869in}{2.509147in}%
\pgfsys@useobject{currentmarker}{}%
\end{pgfscope}%
\begin{pgfscope}%
\pgfsys@transformshift{8.878463in}{2.737834in}%
\pgfsys@useobject{currentmarker}{}%
\end{pgfscope}%
\begin{pgfscope}%
\pgfsys@transformshift{8.896057in}{2.517682in}%
\pgfsys@useobject{currentmarker}{}%
\end{pgfscope}%
\begin{pgfscope}%
\pgfsys@transformshift{8.913651in}{2.831865in}%
\pgfsys@useobject{currentmarker}{}%
\end{pgfscope}%
\begin{pgfscope}%
\pgfsys@transformshift{8.931244in}{2.585872in}%
\pgfsys@useobject{currentmarker}{}%
\end{pgfscope}%
\begin{pgfscope}%
\pgfsys@transformshift{8.948838in}{2.544256in}%
\pgfsys@useobject{currentmarker}{}%
\end{pgfscope}%
\begin{pgfscope}%
\pgfsys@transformshift{8.966432in}{2.807077in}%
\pgfsys@useobject{currentmarker}{}%
\end{pgfscope}%
\begin{pgfscope}%
\pgfsys@transformshift{8.984026in}{2.750613in}%
\pgfsys@useobject{currentmarker}{}%
\end{pgfscope}%
\begin{pgfscope}%
\pgfsys@transformshift{9.001620in}{2.776955in}%
\pgfsys@useobject{currentmarker}{}%
\end{pgfscope}%
\begin{pgfscope}%
\pgfsys@transformshift{9.019214in}{2.664098in}%
\pgfsys@useobject{currentmarker}{}%
\end{pgfscope}%
\begin{pgfscope}%
\pgfsys@transformshift{9.036808in}{2.467509in}%
\pgfsys@useobject{currentmarker}{}%
\end{pgfscope}%
\begin{pgfscope}%
\pgfsys@transformshift{9.054401in}{2.732300in}%
\pgfsys@useobject{currentmarker}{}%
\end{pgfscope}%
\begin{pgfscope}%
\pgfsys@transformshift{9.071995in}{2.491099in}%
\pgfsys@useobject{currentmarker}{}%
\end{pgfscope}%
\begin{pgfscope}%
\pgfsys@transformshift{9.089589in}{2.580038in}%
\pgfsys@useobject{currentmarker}{}%
\end{pgfscope}%
\begin{pgfscope}%
\pgfsys@transformshift{9.107183in}{2.573630in}%
\pgfsys@useobject{currentmarker}{}%
\end{pgfscope}%
\begin{pgfscope}%
\pgfsys@transformshift{9.124777in}{2.439287in}%
\pgfsys@useobject{currentmarker}{}%
\end{pgfscope}%
\begin{pgfscope}%
\pgfsys@transformshift{9.142371in}{2.495197in}%
\pgfsys@useobject{currentmarker}{}%
\end{pgfscope}%
\begin{pgfscope}%
\pgfsys@transformshift{9.159965in}{2.503723in}%
\pgfsys@useobject{currentmarker}{}%
\end{pgfscope}%
\begin{pgfscope}%
\pgfsys@transformshift{9.177558in}{2.424999in}%
\pgfsys@useobject{currentmarker}{}%
\end{pgfscope}%
\begin{pgfscope}%
\pgfsys@transformshift{9.195152in}{2.251467in}%
\pgfsys@useobject{currentmarker}{}%
\end{pgfscope}%
\begin{pgfscope}%
\pgfsys@transformshift{9.212746in}{2.371193in}%
\pgfsys@useobject{currentmarker}{}%
\end{pgfscope}%
\begin{pgfscope}%
\pgfsys@transformshift{9.230340in}{2.453684in}%
\pgfsys@useobject{currentmarker}{}%
\end{pgfscope}%
\begin{pgfscope}%
\pgfsys@transformshift{9.247934in}{2.225893in}%
\pgfsys@useobject{currentmarker}{}%
\end{pgfscope}%
\begin{pgfscope}%
\pgfsys@transformshift{9.265528in}{2.260597in}%
\pgfsys@useobject{currentmarker}{}%
\end{pgfscope}%
\begin{pgfscope}%
\pgfsys@transformshift{9.283121in}{2.211646in}%
\pgfsys@useobject{currentmarker}{}%
\end{pgfscope}%
\begin{pgfscope}%
\pgfsys@transformshift{9.300715in}{2.425970in}%
\pgfsys@useobject{currentmarker}{}%
\end{pgfscope}%
\begin{pgfscope}%
\pgfsys@transformshift{9.318309in}{2.288698in}%
\pgfsys@useobject{currentmarker}{}%
\end{pgfscope}%
\begin{pgfscope}%
\pgfsys@transformshift{9.335903in}{2.245970in}%
\pgfsys@useobject{currentmarker}{}%
\end{pgfscope}%
\begin{pgfscope}%
\pgfsys@transformshift{9.353497in}{2.111856in}%
\pgfsys@useobject{currentmarker}{}%
\end{pgfscope}%
\begin{pgfscope}%
\pgfsys@transformshift{9.371091in}{2.233096in}%
\pgfsys@useobject{currentmarker}{}%
\end{pgfscope}%
\begin{pgfscope}%
\pgfsys@transformshift{9.388685in}{2.098982in}%
\pgfsys@useobject{currentmarker}{}%
\end{pgfscope}%
\begin{pgfscope}%
\pgfsys@transformshift{9.406278in}{2.162618in}%
\pgfsys@useobject{currentmarker}{}%
\end{pgfscope}%
\begin{pgfscope}%
\pgfsys@transformshift{9.423872in}{2.088214in}%
\pgfsys@useobject{currentmarker}{}%
\end{pgfscope}%
\begin{pgfscope}%
\pgfsys@transformshift{9.441466in}{2.217817in}%
\pgfsys@useobject{currentmarker}{}%
\end{pgfscope}%
\begin{pgfscope}%
\pgfsys@transformshift{9.459060in}{2.205555in}%
\pgfsys@useobject{currentmarker}{}%
\end{pgfscope}%
\begin{pgfscope}%
\pgfsys@transformshift{9.476654in}{2.125576in}%
\pgfsys@useobject{currentmarker}{}%
\end{pgfscope}%
\begin{pgfscope}%
\pgfsys@transformshift{9.494248in}{2.189386in}%
\pgfsys@useobject{currentmarker}{}%
\end{pgfscope}%
\begin{pgfscope}%
\pgfsys@transformshift{9.511842in}{2.041820in}%
\pgfsys@useobject{currentmarker}{}%
\end{pgfscope}%
\begin{pgfscope}%
\pgfsys@transformshift{9.529435in}{2.007535in}%
\pgfsys@useobject{currentmarker}{}%
\end{pgfscope}%
\begin{pgfscope}%
\pgfsys@transformshift{9.547029in}{2.212697in}%
\pgfsys@useobject{currentmarker}{}%
\end{pgfscope}%
\begin{pgfscope}%
\pgfsys@transformshift{9.564623in}{2.195046in}%
\pgfsys@useobject{currentmarker}{}%
\end{pgfscope}%
\begin{pgfscope}%
\pgfsys@transformshift{9.582217in}{2.254728in}%
\pgfsys@useobject{currentmarker}{}%
\end{pgfscope}%
\begin{pgfscope}%
\pgfsys@transformshift{9.599811in}{2.446545in}%
\pgfsys@useobject{currentmarker}{}%
\end{pgfscope}%
\begin{pgfscope}%
\pgfsys@transformshift{9.617405in}{2.314886in}%
\pgfsys@useobject{currentmarker}{}%
\end{pgfscope}%
\begin{pgfscope}%
\pgfsys@transformshift{9.634999in}{2.141871in}%
\pgfsys@useobject{currentmarker}{}%
\end{pgfscope}%
\begin{pgfscope}%
\pgfsys@transformshift{9.652592in}{2.366405in}%
\pgfsys@useobject{currentmarker}{}%
\end{pgfscope}%
\begin{pgfscope}%
\pgfsys@transformshift{9.670186in}{2.136836in}%
\pgfsys@useobject{currentmarker}{}%
\end{pgfscope}%
\begin{pgfscope}%
\pgfsys@transformshift{9.687780in}{2.243272in}%
\pgfsys@useobject{currentmarker}{}%
\end{pgfscope}%
\begin{pgfscope}%
\pgfsys@transformshift{9.705374in}{2.303294in}%
\pgfsys@useobject{currentmarker}{}%
\end{pgfscope}%
\begin{pgfscope}%
\pgfsys@transformshift{9.722968in}{2.505271in}%
\pgfsys@useobject{currentmarker}{}%
\end{pgfscope}%
\begin{pgfscope}%
\pgfsys@transformshift{9.740562in}{2.275099in}%
\pgfsys@useobject{currentmarker}{}%
\end{pgfscope}%
\begin{pgfscope}%
\pgfsys@transformshift{9.758155in}{2.287237in}%
\pgfsys@useobject{currentmarker}{}%
\end{pgfscope}%
\begin{pgfscope}%
\pgfsys@transformshift{9.775749in}{2.381713in}%
\pgfsys@useobject{currentmarker}{}%
\end{pgfscope}%
\begin{pgfscope}%
\pgfsys@transformshift{9.793343in}{2.343907in}%
\pgfsys@useobject{currentmarker}{}%
\end{pgfscope}%
\begin{pgfscope}%
\pgfsys@transformshift{9.810937in}{2.545636in}%
\pgfsys@useobject{currentmarker}{}%
\end{pgfscope}%
\begin{pgfscope}%
\pgfsys@transformshift{9.828531in}{2.338992in}%
\pgfsys@useobject{currentmarker}{}%
\end{pgfscope}%
\begin{pgfscope}%
\pgfsys@transformshift{9.846125in}{2.349474in}%
\pgfsys@useobject{currentmarker}{}%
\end{pgfscope}%
\begin{pgfscope}%
\pgfsys@transformshift{9.863719in}{2.438041in}%
\pgfsys@useobject{currentmarker}{}%
\end{pgfscope}%
\begin{pgfscope}%
\pgfsys@transformshift{9.881312in}{2.446775in}%
\pgfsys@useobject{currentmarker}{}%
\end{pgfscope}%
\begin{pgfscope}%
\pgfsys@transformshift{9.898906in}{2.707728in}%
\pgfsys@useobject{currentmarker}{}%
\end{pgfscope}%
\begin{pgfscope}%
\pgfsys@transformshift{9.916500in}{2.619589in}%
\pgfsys@useobject{currentmarker}{}%
\end{pgfscope}%
\begin{pgfscope}%
\pgfsys@transformshift{9.934094in}{2.541800in}%
\pgfsys@useobject{currentmarker}{}%
\end{pgfscope}%
\begin{pgfscope}%
\pgfsys@transformshift{9.951688in}{2.416208in}%
\pgfsys@useobject{currentmarker}{}%
\end{pgfscope}%
\begin{pgfscope}%
\pgfsys@transformshift{9.969282in}{2.633696in}%
\pgfsys@useobject{currentmarker}{}%
\end{pgfscope}%
\begin{pgfscope}%
\pgfsys@transformshift{9.986876in}{2.450091in}%
\pgfsys@useobject{currentmarker}{}%
\end{pgfscope}%
\begin{pgfscope}%
\pgfsys@transformshift{10.004469in}{2.397092in}%
\pgfsys@useobject{currentmarker}{}%
\end{pgfscope}%
\begin{pgfscope}%
\pgfsys@transformshift{10.022063in}{2.676321in}%
\pgfsys@useobject{currentmarker}{}%
\end{pgfscope}%
\begin{pgfscope}%
\pgfsys@transformshift{10.039657in}{2.586081in}%
\pgfsys@useobject{currentmarker}{}%
\end{pgfscope}%
\begin{pgfscope}%
\pgfsys@transformshift{10.057251in}{2.644312in}%
\pgfsys@useobject{currentmarker}{}%
\end{pgfscope}%
\begin{pgfscope}%
\pgfsys@transformshift{10.074845in}{2.577668in}%
\pgfsys@useobject{currentmarker}{}%
\end{pgfscope}%
\begin{pgfscope}%
\pgfsys@transformshift{10.092439in}{2.625475in}%
\pgfsys@useobject{currentmarker}{}%
\end{pgfscope}%
\begin{pgfscope}%
\pgfsys@transformshift{10.110033in}{2.462914in}%
\pgfsys@useobject{currentmarker}{}%
\end{pgfscope}%
\begin{pgfscope}%
\pgfsys@transformshift{10.127626in}{2.413409in}%
\pgfsys@useobject{currentmarker}{}%
\end{pgfscope}%
\begin{pgfscope}%
\pgfsys@transformshift{10.145220in}{2.576448in}%
\pgfsys@useobject{currentmarker}{}%
\end{pgfscope}%
\begin{pgfscope}%
\pgfsys@transformshift{10.162814in}{2.411720in}%
\pgfsys@useobject{currentmarker}{}%
\end{pgfscope}%
\begin{pgfscope}%
\pgfsys@transformshift{10.180408in}{2.408824in}%
\pgfsys@useobject{currentmarker}{}%
\end{pgfscope}%
\begin{pgfscope}%
\pgfsys@transformshift{10.198002in}{2.417137in}%
\pgfsys@useobject{currentmarker}{}%
\end{pgfscope}%
\begin{pgfscope}%
\pgfsys@transformshift{10.215596in}{2.448956in}%
\pgfsys@useobject{currentmarker}{}%
\end{pgfscope}%
\begin{pgfscope}%
\pgfsys@transformshift{10.233189in}{2.393983in}%
\pgfsys@useobject{currentmarker}{}%
\end{pgfscope}%
\begin{pgfscope}%
\pgfsys@transformshift{10.250783in}{2.272299in}%
\pgfsys@useobject{currentmarker}{}%
\end{pgfscope}%
\begin{pgfscope}%
\pgfsys@transformshift{10.268377in}{2.329023in}%
\pgfsys@useobject{currentmarker}{}%
\end{pgfscope}%
\begin{pgfscope}%
\pgfsys@transformshift{10.285971in}{2.150000in}%
\pgfsys@useobject{currentmarker}{}%
\end{pgfscope}%
\begin{pgfscope}%
\pgfsys@transformshift{10.303565in}{2.422690in}%
\pgfsys@useobject{currentmarker}{}%
\end{pgfscope}%
\begin{pgfscope}%
\pgfsys@transformshift{10.321159in}{2.178109in}%
\pgfsys@useobject{currentmarker}{}%
\end{pgfscope}%
\begin{pgfscope}%
\pgfsys@transformshift{10.338753in}{2.211944in}%
\pgfsys@useobject{currentmarker}{}%
\end{pgfscope}%
\begin{pgfscope}%
\pgfsys@transformshift{10.356346in}{2.313709in}%
\pgfsys@useobject{currentmarker}{}%
\end{pgfscope}%
\begin{pgfscope}%
\pgfsys@transformshift{10.373940in}{2.217733in}%
\pgfsys@useobject{currentmarker}{}%
\end{pgfscope}%
\begin{pgfscope}%
\pgfsys@transformshift{10.391534in}{2.436373in}%
\pgfsys@useobject{currentmarker}{}%
\end{pgfscope}%
\begin{pgfscope}%
\pgfsys@transformshift{10.409128in}{2.134357in}%
\pgfsys@useobject{currentmarker}{}%
\end{pgfscope}%
\begin{pgfscope}%
\pgfsys@transformshift{10.426722in}{2.282046in}%
\pgfsys@useobject{currentmarker}{}%
\end{pgfscope}%
\begin{pgfscope}%
\pgfsys@transformshift{10.444316in}{2.241026in}%
\pgfsys@useobject{currentmarker}{}%
\end{pgfscope}%
\begin{pgfscope}%
\pgfsys@transformshift{10.461910in}{2.117866in}%
\pgfsys@useobject{currentmarker}{}%
\end{pgfscope}%
\begin{pgfscope}%
\pgfsys@transformshift{10.479503in}{2.284136in}%
\pgfsys@useobject{currentmarker}{}%
\end{pgfscope}%
\begin{pgfscope}%
\pgfsys@transformshift{10.497097in}{2.209022in}%
\pgfsys@useobject{currentmarker}{}%
\end{pgfscope}%
\begin{pgfscope}%
\pgfsys@transformshift{10.514691in}{2.302978in}%
\pgfsys@useobject{currentmarker}{}%
\end{pgfscope}%
\begin{pgfscope}%
\pgfsys@transformshift{10.532285in}{2.308095in}%
\pgfsys@useobject{currentmarker}{}%
\end{pgfscope}%
\begin{pgfscope}%
\pgfsys@transformshift{10.549879in}{2.446676in}%
\pgfsys@useobject{currentmarker}{}%
\end{pgfscope}%
\begin{pgfscope}%
\pgfsys@transformshift{10.567473in}{2.366477in}%
\pgfsys@useobject{currentmarker}{}%
\end{pgfscope}%
\begin{pgfscope}%
\pgfsys@transformshift{10.585067in}{2.198779in}%
\pgfsys@useobject{currentmarker}{}%
\end{pgfscope}%
\begin{pgfscope}%
\pgfsys@transformshift{10.602660in}{2.220369in}%
\pgfsys@useobject{currentmarker}{}%
\end{pgfscope}%
\begin{pgfscope}%
\pgfsys@transformshift{10.620254in}{2.367339in}%
\pgfsys@useobject{currentmarker}{}%
\end{pgfscope}%
\begin{pgfscope}%
\pgfsys@transformshift{10.637848in}{2.334453in}%
\pgfsys@useobject{currentmarker}{}%
\end{pgfscope}%
\begin{pgfscope}%
\pgfsys@transformshift{10.655442in}{2.347263in}%
\pgfsys@useobject{currentmarker}{}%
\end{pgfscope}%
\begin{pgfscope}%
\pgfsys@transformshift{10.673036in}{2.133275in}%
\pgfsys@useobject{currentmarker}{}%
\end{pgfscope}%
\begin{pgfscope}%
\pgfsys@transformshift{10.690630in}{2.313560in}%
\pgfsys@useobject{currentmarker}{}%
\end{pgfscope}%
\begin{pgfscope}%
\pgfsys@transformshift{10.708223in}{2.260227in}%
\pgfsys@useobject{currentmarker}{}%
\end{pgfscope}%
\begin{pgfscope}%
\pgfsys@transformshift{10.725817in}{2.384894in}%
\pgfsys@useobject{currentmarker}{}%
\end{pgfscope}%
\begin{pgfscope}%
\pgfsys@transformshift{10.743411in}{2.368455in}%
\pgfsys@useobject{currentmarker}{}%
\end{pgfscope}%
\begin{pgfscope}%
\pgfsys@transformshift{10.761005in}{2.494470in}%
\pgfsys@useobject{currentmarker}{}%
\end{pgfscope}%
\begin{pgfscope}%
\pgfsys@transformshift{10.778599in}{2.458087in}%
\pgfsys@useobject{currentmarker}{}%
\end{pgfscope}%
\begin{pgfscope}%
\pgfsys@transformshift{10.796193in}{2.530704in}%
\pgfsys@useobject{currentmarker}{}%
\end{pgfscope}%
\begin{pgfscope}%
\pgfsys@transformshift{10.813787in}{2.428234in}%
\pgfsys@useobject{currentmarker}{}%
\end{pgfscope}%
\begin{pgfscope}%
\pgfsys@transformshift{10.831380in}{2.405057in}%
\pgfsys@useobject{currentmarker}{}%
\end{pgfscope}%
\begin{pgfscope}%
\pgfsys@transformshift{10.848974in}{2.485208in}%
\pgfsys@useobject{currentmarker}{}%
\end{pgfscope}%
\begin{pgfscope}%
\pgfsys@transformshift{10.866568in}{2.550822in}%
\pgfsys@useobject{currentmarker}{}%
\end{pgfscope}%
\begin{pgfscope}%
\pgfsys@transformshift{10.884162in}{2.616431in}%
\pgfsys@useobject{currentmarker}{}%
\end{pgfscope}%
\begin{pgfscope}%
\pgfsys@transformshift{10.901756in}{2.832843in}%
\pgfsys@useobject{currentmarker}{}%
\end{pgfscope}%
\begin{pgfscope}%
\pgfsys@transformshift{10.919350in}{2.622109in}%
\pgfsys@useobject{currentmarker}{}%
\end{pgfscope}%
\begin{pgfscope}%
\pgfsys@transformshift{10.936944in}{2.551996in}%
\pgfsys@useobject{currentmarker}{}%
\end{pgfscope}%
\begin{pgfscope}%
\pgfsys@transformshift{10.954537in}{2.636175in}%
\pgfsys@useobject{currentmarker}{}%
\end{pgfscope}%
\begin{pgfscope}%
\pgfsys@transformshift{10.972131in}{2.644912in}%
\pgfsys@useobject{currentmarker}{}%
\end{pgfscope}%
\begin{pgfscope}%
\pgfsys@transformshift{10.989725in}{2.760620in}%
\pgfsys@useobject{currentmarker}{}%
\end{pgfscope}%
\begin{pgfscope}%
\pgfsys@transformshift{11.007319in}{2.572201in}%
\pgfsys@useobject{currentmarker}{}%
\end{pgfscope}%
\begin{pgfscope}%
\pgfsys@transformshift{11.024913in}{2.751921in}%
\pgfsys@useobject{currentmarker}{}%
\end{pgfscope}%
\begin{pgfscope}%
\pgfsys@transformshift{11.042507in}{2.775821in}%
\pgfsys@useobject{currentmarker}{}%
\end{pgfscope}%
\begin{pgfscope}%
\pgfsys@transformshift{11.060101in}{2.796091in}%
\pgfsys@useobject{currentmarker}{}%
\end{pgfscope}%
\begin{pgfscope}%
\pgfsys@transformshift{11.077694in}{2.722154in}%
\pgfsys@useobject{currentmarker}{}%
\end{pgfscope}%
\begin{pgfscope}%
\pgfsys@transformshift{11.095288in}{2.767504in}%
\pgfsys@useobject{currentmarker}{}%
\end{pgfscope}%
\begin{pgfscope}%
\pgfsys@transformshift{11.112882in}{2.653279in}%
\pgfsys@useobject{currentmarker}{}%
\end{pgfscope}%
\begin{pgfscope}%
\pgfsys@transformshift{11.130476in}{2.752946in}%
\pgfsys@useobject{currentmarker}{}%
\end{pgfscope}%
\begin{pgfscope}%
\pgfsys@transformshift{11.148070in}{2.750693in}%
\pgfsys@useobject{currentmarker}{}%
\end{pgfscope}%
\begin{pgfscope}%
\pgfsys@transformshift{11.165664in}{2.849357in}%
\pgfsys@useobject{currentmarker}{}%
\end{pgfscope}%
\begin{pgfscope}%
\pgfsys@transformshift{11.183257in}{2.688050in}%
\pgfsys@useobject{currentmarker}{}%
\end{pgfscope}%
\begin{pgfscope}%
\pgfsys@transformshift{11.200851in}{2.882850in}%
\pgfsys@useobject{currentmarker}{}%
\end{pgfscope}%
\begin{pgfscope}%
\pgfsys@transformshift{11.218445in}{2.951139in}%
\pgfsys@useobject{currentmarker}{}%
\end{pgfscope}%
\begin{pgfscope}%
\pgfsys@transformshift{11.236039in}{2.581640in}%
\pgfsys@useobject{currentmarker}{}%
\end{pgfscope}%
\begin{pgfscope}%
\pgfsys@transformshift{11.253633in}{2.828734in}%
\pgfsys@useobject{currentmarker}{}%
\end{pgfscope}%
\begin{pgfscope}%
\pgfsys@transformshift{11.271227in}{2.845545in}%
\pgfsys@useobject{currentmarker}{}%
\end{pgfscope}%
\begin{pgfscope}%
\pgfsys@transformshift{11.288821in}{2.701636in}%
\pgfsys@useobject{currentmarker}{}%
\end{pgfscope}%
\begin{pgfscope}%
\pgfsys@transformshift{11.306414in}{2.715287in}%
\pgfsys@useobject{currentmarker}{}%
\end{pgfscope}%
\begin{pgfscope}%
\pgfsys@transformshift{11.324008in}{2.730634in}%
\pgfsys@useobject{currentmarker}{}%
\end{pgfscope}%
\begin{pgfscope}%
\pgfsys@transformshift{11.341602in}{2.701620in}%
\pgfsys@useobject{currentmarker}{}%
\end{pgfscope}%
\begin{pgfscope}%
\pgfsys@transformshift{11.359196in}{2.687893in}%
\pgfsys@useobject{currentmarker}{}%
\end{pgfscope}%
\begin{pgfscope}%
\pgfsys@transformshift{11.376790in}{2.535733in}%
\pgfsys@useobject{currentmarker}{}%
\end{pgfscope}%
\begin{pgfscope}%
\pgfsys@transformshift{11.394384in}{2.811411in}%
\pgfsys@useobject{currentmarker}{}%
\end{pgfscope}%
\begin{pgfscope}%
\pgfsys@transformshift{11.411978in}{2.791491in}%
\pgfsys@useobject{currentmarker}{}%
\end{pgfscope}%
\begin{pgfscope}%
\pgfsys@transformshift{11.429571in}{2.586388in}%
\pgfsys@useobject{currentmarker}{}%
\end{pgfscope}%
\begin{pgfscope}%
\pgfsys@transformshift{11.447165in}{2.508363in}%
\pgfsys@useobject{currentmarker}{}%
\end{pgfscope}%
\begin{pgfscope}%
\pgfsys@transformshift{11.464759in}{2.700439in}%
\pgfsys@useobject{currentmarker}{}%
\end{pgfscope}%
\begin{pgfscope}%
\pgfsys@transformshift{11.482353in}{2.579017in}%
\pgfsys@useobject{currentmarker}{}%
\end{pgfscope}%
\end{pgfscope}%
\begin{pgfscope}%
\pgfpathrectangle{\pgfqpoint{7.105882in}{1.973684in}}{\pgfqpoint{4.376471in}{0.978947in}} %
\pgfusepath{clip}%
\pgfsetroundcap%
\pgfsetroundjoin%
\pgfsetlinewidth{1.756562pt}%
\definecolor{currentstroke}{rgb}{0.298039,0.447059,0.690196}%
\pgfsetstrokecolor{currentstroke}%
\pgfsetdash{}{0pt}%
\pgfpathmoveto{\pgfqpoint{7.981176in}{2.242334in}}%
\pgfpathlineto{\pgfqpoint{7.998770in}{2.281612in}}%
\pgfpathlineto{\pgfqpoint{8.016364in}{2.309812in}}%
\pgfpathlineto{\pgfqpoint{8.033958in}{2.329302in}}%
\pgfpathlineto{\pgfqpoint{8.051552in}{2.342068in}}%
\pgfpathlineto{\pgfqpoint{8.069146in}{2.349759in}}%
\pgfpathlineto{\pgfqpoint{8.086740in}{2.353723in}}%
\pgfpathlineto{\pgfqpoint{8.104333in}{2.355053in}}%
\pgfpathlineto{\pgfqpoint{8.139521in}{2.353083in}}%
\pgfpathlineto{\pgfqpoint{8.245084in}{2.341488in}}%
\pgfpathlineto{\pgfqpoint{8.297866in}{2.339883in}}%
\pgfpathlineto{\pgfqpoint{8.368241in}{2.340892in}}%
\pgfpathlineto{\pgfqpoint{8.456210in}{2.341862in}}%
\pgfpathlineto{\pgfqpoint{8.526586in}{2.340105in}}%
\pgfpathlineto{\pgfqpoint{8.755306in}{2.331073in}}%
\pgfpathlineto{\pgfqpoint{8.825681in}{2.332118in}}%
\pgfpathlineto{\pgfqpoint{8.913651in}{2.336017in}}%
\pgfpathlineto{\pgfqpoint{9.071995in}{2.343717in}}%
\pgfpathlineto{\pgfqpoint{9.159965in}{2.345365in}}%
\pgfpathlineto{\pgfqpoint{9.265528in}{2.344555in}}%
\pgfpathlineto{\pgfqpoint{9.547029in}{2.340491in}}%
\pgfpathlineto{\pgfqpoint{10.285971in}{2.341272in}}%
\pgfpathlineto{\pgfqpoint{10.479503in}{2.341862in}}%
\pgfpathlineto{\pgfqpoint{10.848974in}{2.340874in}}%
\pgfpathlineto{\pgfqpoint{11.112882in}{2.342525in}}%
\pgfpathlineto{\pgfqpoint{11.411978in}{2.341986in}}%
\pgfpathlineto{\pgfqpoint{11.447165in}{2.344013in}}%
\pgfpathlineto{\pgfqpoint{11.482353in}{2.347970in}}%
\pgfpathlineto{\pgfqpoint{11.482353in}{2.347970in}}%
\pgfusepath{stroke}%
\end{pgfscope}%
\begin{pgfscope}%
\pgfsetrectcap%
\pgfsetmiterjoin%
\pgfsetlinewidth{1.003750pt}%
\definecolor{currentstroke}{rgb}{0.800000,0.800000,0.800000}%
\pgfsetstrokecolor{currentstroke}%
\pgfsetdash{}{0pt}%
\pgfpathmoveto{\pgfqpoint{7.105882in}{1.973684in}}%
\pgfpathlineto{\pgfqpoint{7.105882in}{2.952632in}}%
\pgfusepath{stroke}%
\end{pgfscope}%
\begin{pgfscope}%
\pgfsetrectcap%
\pgfsetmiterjoin%
\pgfsetlinewidth{1.003750pt}%
\definecolor{currentstroke}{rgb}{0.800000,0.800000,0.800000}%
\pgfsetstrokecolor{currentstroke}%
\pgfsetdash{}{0pt}%
\pgfpathmoveto{\pgfqpoint{11.482353in}{1.973684in}}%
\pgfpathlineto{\pgfqpoint{11.482353in}{2.952632in}}%
\pgfusepath{stroke}%
\end{pgfscope}%
\begin{pgfscope}%
\pgfsetrectcap%
\pgfsetmiterjoin%
\pgfsetlinewidth{1.003750pt}%
\definecolor{currentstroke}{rgb}{0.800000,0.800000,0.800000}%
\pgfsetstrokecolor{currentstroke}%
\pgfsetdash{}{0pt}%
\pgfpathmoveto{\pgfqpoint{7.105882in}{2.952632in}}%
\pgfpathlineto{\pgfqpoint{11.482353in}{2.952632in}}%
\pgfusepath{stroke}%
\end{pgfscope}%
\begin{pgfscope}%
\pgfsetrectcap%
\pgfsetmiterjoin%
\pgfsetlinewidth{1.003750pt}%
\definecolor{currentstroke}{rgb}{0.800000,0.800000,0.800000}%
\pgfsetstrokecolor{currentstroke}%
\pgfsetdash{}{0pt}%
\pgfpathmoveto{\pgfqpoint{7.105882in}{1.973684in}}%
\pgfpathlineto{\pgfqpoint{11.482353in}{1.973684in}}%
\pgfusepath{stroke}%
\end{pgfscope}%
\begin{pgfscope}%
\pgfsetroundcap%
\pgfsetroundjoin%
\pgfsetlinewidth{1.756562pt}%
\definecolor{currentstroke}{rgb}{0.298039,0.447059,0.690196}%
\pgfsetstrokecolor{currentstroke}%
\pgfsetdash{}{0pt}%
\pgfpathmoveto{\pgfqpoint{7.230882in}{2.567827in}}%
\pgfpathlineto{\pgfqpoint{7.508660in}{2.567827in}}%
\pgfusepath{stroke}%
\end{pgfscope}%
\begin{pgfscope}%
\definecolor{textcolor}{rgb}{0.150000,0.150000,0.150000}%
\pgfsetstrokecolor{textcolor}%
\pgfsetfillcolor{textcolor}%
\pgftext[x=7.619771in,y=2.519216in,left,base]{\color{textcolor}\sffamily\fontsize{10.000000}{12.000000}\selectfont \(\displaystyle \widetilde{\Phi}^* \theta^{\parallel}\)}%
\end{pgfscope}%
\begin{pgfscope}%
\pgfsetbuttcap%
\pgfsetroundjoin%
\definecolor{currentfill}{rgb}{1.000000,0.000000,0.000000}%
\pgfsetfillcolor{currentfill}%
\pgfsetlinewidth{2.007500pt}%
\definecolor{currentstroke}{rgb}{1.000000,0.000000,0.000000}%
\pgfsetstrokecolor{currentstroke}%
\pgfsetdash{}{0pt}%
\pgfpathmoveto{\pgfqpoint{7.338715in}{2.359209in}}%
\pgfpathlineto{\pgfqpoint{7.400828in}{2.359209in}}%
\pgfpathmoveto{\pgfqpoint{7.369771in}{2.328152in}}%
\pgfpathlineto{\pgfqpoint{7.369771in}{2.390265in}}%
\pgfusepath{stroke,fill}%
\end{pgfscope}%
\begin{pgfscope}%
\pgfsetbuttcap%
\pgfsetroundjoin%
\definecolor{currentfill}{rgb}{1.000000,0.000000,0.000000}%
\pgfsetfillcolor{currentfill}%
\pgfsetlinewidth{2.007500pt}%
\definecolor{currentstroke}{rgb}{1.000000,0.000000,0.000000}%
\pgfsetstrokecolor{currentstroke}%
\pgfsetdash{}{0pt}%
\pgfpathmoveto{\pgfqpoint{7.338715in}{2.359209in}}%
\pgfpathlineto{\pgfqpoint{7.400828in}{2.359209in}}%
\pgfpathmoveto{\pgfqpoint{7.369771in}{2.328152in}}%
\pgfpathlineto{\pgfqpoint{7.369771in}{2.390265in}}%
\pgfusepath{stroke,fill}%
\end{pgfscope}%
\begin{pgfscope}%
\pgfsetbuttcap%
\pgfsetroundjoin%
\definecolor{currentfill}{rgb}{1.000000,0.000000,0.000000}%
\pgfsetfillcolor{currentfill}%
\pgfsetlinewidth{2.007500pt}%
\definecolor{currentstroke}{rgb}{1.000000,0.000000,0.000000}%
\pgfsetstrokecolor{currentstroke}%
\pgfsetdash{}{0pt}%
\pgfpathmoveto{\pgfqpoint{7.338715in}{2.359209in}}%
\pgfpathlineto{\pgfqpoint{7.400828in}{2.359209in}}%
\pgfpathmoveto{\pgfqpoint{7.369771in}{2.328152in}}%
\pgfpathlineto{\pgfqpoint{7.369771in}{2.390265in}}%
\pgfusepath{stroke,fill}%
\end{pgfscope}%
\begin{pgfscope}%
\definecolor{textcolor}{rgb}{0.150000,0.150000,0.150000}%
\pgfsetstrokecolor{textcolor}%
\pgfsetfillcolor{textcolor}%
\pgftext[x=7.619771in,y=2.322751in,left,base]{\color{textcolor}\sffamily\fontsize{10.000000}{12.000000}\selectfont train}%
\end{pgfscope}%
\begin{pgfscope}%
\pgfsetbuttcap%
\pgfsetroundjoin%
\definecolor{currentfill}{rgb}{0.000000,0.000000,0.000000}%
\pgfsetfillcolor{currentfill}%
\pgfsetlinewidth{0.301125pt}%
\definecolor{currentstroke}{rgb}{0.000000,0.000000,0.000000}%
\pgfsetstrokecolor{currentstroke}%
\pgfsetdash{}{0pt}%
\pgfpathmoveto{\pgfqpoint{7.369771in}{2.147216in}}%
\pgfpathcurveto{\pgfqpoint{7.373889in}{2.147216in}}{\pgfqpoint{7.377839in}{2.148852in}}{\pgfqpoint{7.380751in}{2.151764in}}%
\pgfpathcurveto{\pgfqpoint{7.383663in}{2.154676in}}{\pgfqpoint{7.385299in}{2.158626in}}{\pgfqpoint{7.385299in}{2.162744in}}%
\pgfpathcurveto{\pgfqpoint{7.385299in}{2.166862in}}{\pgfqpoint{7.383663in}{2.170812in}}{\pgfqpoint{7.380751in}{2.173724in}}%
\pgfpathcurveto{\pgfqpoint{7.377839in}{2.176636in}}{\pgfqpoint{7.373889in}{2.178272in}}{\pgfqpoint{7.369771in}{2.178272in}}%
\pgfpathcurveto{\pgfqpoint{7.365653in}{2.178272in}}{\pgfqpoint{7.361703in}{2.176636in}}{\pgfqpoint{7.358791in}{2.173724in}}%
\pgfpathcurveto{\pgfqpoint{7.355879in}{2.170812in}}{\pgfqpoint{7.354243in}{2.166862in}}{\pgfqpoint{7.354243in}{2.162744in}}%
\pgfpathcurveto{\pgfqpoint{7.354243in}{2.158626in}}{\pgfqpoint{7.355879in}{2.154676in}}{\pgfqpoint{7.358791in}{2.151764in}}%
\pgfpathcurveto{\pgfqpoint{7.361703in}{2.148852in}}{\pgfqpoint{7.365653in}{2.147216in}}{\pgfqpoint{7.369771in}{2.147216in}}%
\pgfpathclose%
\pgfusepath{stroke,fill}%
\end{pgfscope}%
\begin{pgfscope}%
\pgfsetbuttcap%
\pgfsetroundjoin%
\definecolor{currentfill}{rgb}{0.000000,0.000000,0.000000}%
\pgfsetfillcolor{currentfill}%
\pgfsetlinewidth{0.301125pt}%
\definecolor{currentstroke}{rgb}{0.000000,0.000000,0.000000}%
\pgfsetstrokecolor{currentstroke}%
\pgfsetdash{}{0pt}%
\pgfpathmoveto{\pgfqpoint{7.369771in}{2.147216in}}%
\pgfpathcurveto{\pgfqpoint{7.373889in}{2.147216in}}{\pgfqpoint{7.377839in}{2.148852in}}{\pgfqpoint{7.380751in}{2.151764in}}%
\pgfpathcurveto{\pgfqpoint{7.383663in}{2.154676in}}{\pgfqpoint{7.385299in}{2.158626in}}{\pgfqpoint{7.385299in}{2.162744in}}%
\pgfpathcurveto{\pgfqpoint{7.385299in}{2.166862in}}{\pgfqpoint{7.383663in}{2.170812in}}{\pgfqpoint{7.380751in}{2.173724in}}%
\pgfpathcurveto{\pgfqpoint{7.377839in}{2.176636in}}{\pgfqpoint{7.373889in}{2.178272in}}{\pgfqpoint{7.369771in}{2.178272in}}%
\pgfpathcurveto{\pgfqpoint{7.365653in}{2.178272in}}{\pgfqpoint{7.361703in}{2.176636in}}{\pgfqpoint{7.358791in}{2.173724in}}%
\pgfpathcurveto{\pgfqpoint{7.355879in}{2.170812in}}{\pgfqpoint{7.354243in}{2.166862in}}{\pgfqpoint{7.354243in}{2.162744in}}%
\pgfpathcurveto{\pgfqpoint{7.354243in}{2.158626in}}{\pgfqpoint{7.355879in}{2.154676in}}{\pgfqpoint{7.358791in}{2.151764in}}%
\pgfpathcurveto{\pgfqpoint{7.361703in}{2.148852in}}{\pgfqpoint{7.365653in}{2.147216in}}{\pgfqpoint{7.369771in}{2.147216in}}%
\pgfpathclose%
\pgfusepath{stroke,fill}%
\end{pgfscope}%
\begin{pgfscope}%
\pgfsetbuttcap%
\pgfsetroundjoin%
\definecolor{currentfill}{rgb}{0.000000,0.000000,0.000000}%
\pgfsetfillcolor{currentfill}%
\pgfsetlinewidth{0.301125pt}%
\definecolor{currentstroke}{rgb}{0.000000,0.000000,0.000000}%
\pgfsetstrokecolor{currentstroke}%
\pgfsetdash{}{0pt}%
\pgfpathmoveto{\pgfqpoint{7.369771in}{2.147216in}}%
\pgfpathcurveto{\pgfqpoint{7.373889in}{2.147216in}}{\pgfqpoint{7.377839in}{2.148852in}}{\pgfqpoint{7.380751in}{2.151764in}}%
\pgfpathcurveto{\pgfqpoint{7.383663in}{2.154676in}}{\pgfqpoint{7.385299in}{2.158626in}}{\pgfqpoint{7.385299in}{2.162744in}}%
\pgfpathcurveto{\pgfqpoint{7.385299in}{2.166862in}}{\pgfqpoint{7.383663in}{2.170812in}}{\pgfqpoint{7.380751in}{2.173724in}}%
\pgfpathcurveto{\pgfqpoint{7.377839in}{2.176636in}}{\pgfqpoint{7.373889in}{2.178272in}}{\pgfqpoint{7.369771in}{2.178272in}}%
\pgfpathcurveto{\pgfqpoint{7.365653in}{2.178272in}}{\pgfqpoint{7.361703in}{2.176636in}}{\pgfqpoint{7.358791in}{2.173724in}}%
\pgfpathcurveto{\pgfqpoint{7.355879in}{2.170812in}}{\pgfqpoint{7.354243in}{2.166862in}}{\pgfqpoint{7.354243in}{2.162744in}}%
\pgfpathcurveto{\pgfqpoint{7.354243in}{2.158626in}}{\pgfqpoint{7.355879in}{2.154676in}}{\pgfqpoint{7.358791in}{2.151764in}}%
\pgfpathcurveto{\pgfqpoint{7.361703in}{2.148852in}}{\pgfqpoint{7.365653in}{2.147216in}}{\pgfqpoint{7.369771in}{2.147216in}}%
\pgfpathclose%
\pgfusepath{stroke,fill}%
\end{pgfscope}%
\begin{pgfscope}%
\definecolor{textcolor}{rgb}{0.150000,0.150000,0.150000}%
\pgfsetstrokecolor{textcolor}%
\pgfsetfillcolor{textcolor}%
\pgftext[x=7.619771in,y=2.126285in,left,base]{\color{textcolor}\sffamily\fontsize{10.000000}{12.000000}\selectfont test}%
\end{pgfscope}%
\begin{pgfscope}%
\pgfsetbuttcap%
\pgfsetmiterjoin%
\definecolor{currentfill}{rgb}{1.000000,1.000000,1.000000}%
\pgfsetfillcolor{currentfill}%
\pgfsetlinewidth{0.000000pt}%
\definecolor{currentstroke}{rgb}{0.000000,0.000000,0.000000}%
\pgfsetstrokecolor{currentstroke}%
\pgfsetstrokeopacity{0.000000}%
\pgfsetdash{}{0pt}%
\pgfpathmoveto{\pgfqpoint{12.211765in}{1.973684in}}%
\pgfpathlineto{\pgfqpoint{14.400000in}{1.973684in}}%
\pgfpathlineto{\pgfqpoint{14.400000in}{2.952632in}}%
\pgfpathlineto{\pgfqpoint{12.211765in}{2.952632in}}%
\pgfpathclose%
\pgfusepath{fill}%
\end{pgfscope}%
\begin{pgfscope}%
\pgfpathrectangle{\pgfqpoint{12.211765in}{1.973684in}}{\pgfqpoint{2.188235in}{0.978947in}} %
\pgfusepath{clip}%
\pgfsetroundcap%
\pgfsetroundjoin%
\pgfsetlinewidth{1.003750pt}%
\definecolor{currentstroke}{rgb}{0.800000,0.800000,0.800000}%
\pgfsetstrokecolor{currentstroke}%
\pgfsetdash{}{0pt}%
\pgfpathmoveto{\pgfqpoint{12.211765in}{1.973684in}}%
\pgfpathlineto{\pgfqpoint{12.211765in}{2.952632in}}%
\pgfusepath{stroke}%
\end{pgfscope}%
\begin{pgfscope}%
\pgfpathrectangle{\pgfqpoint{12.211765in}{1.973684in}}{\pgfqpoint{2.188235in}{0.978947in}} %
\pgfusepath{clip}%
\pgfsetroundcap%
\pgfsetroundjoin%
\pgfsetlinewidth{1.003750pt}%
\definecolor{currentstroke}{rgb}{0.800000,0.800000,0.800000}%
\pgfsetstrokecolor{currentstroke}%
\pgfsetdash{}{0pt}%
\pgfpathmoveto{\pgfqpoint{12.485294in}{1.973684in}}%
\pgfpathlineto{\pgfqpoint{12.485294in}{2.952632in}}%
\pgfusepath{stroke}%
\end{pgfscope}%
\begin{pgfscope}%
\pgfpathrectangle{\pgfqpoint{12.211765in}{1.973684in}}{\pgfqpoint{2.188235in}{0.978947in}} %
\pgfusepath{clip}%
\pgfsetroundcap%
\pgfsetroundjoin%
\pgfsetlinewidth{1.003750pt}%
\definecolor{currentstroke}{rgb}{0.800000,0.800000,0.800000}%
\pgfsetstrokecolor{currentstroke}%
\pgfsetdash{}{0pt}%
\pgfpathmoveto{\pgfqpoint{12.758824in}{1.973684in}}%
\pgfpathlineto{\pgfqpoint{12.758824in}{2.952632in}}%
\pgfusepath{stroke}%
\end{pgfscope}%
\begin{pgfscope}%
\pgfpathrectangle{\pgfqpoint{12.211765in}{1.973684in}}{\pgfqpoint{2.188235in}{0.978947in}} %
\pgfusepath{clip}%
\pgfsetroundcap%
\pgfsetroundjoin%
\pgfsetlinewidth{1.003750pt}%
\definecolor{currentstroke}{rgb}{0.800000,0.800000,0.800000}%
\pgfsetstrokecolor{currentstroke}%
\pgfsetdash{}{0pt}%
\pgfpathmoveto{\pgfqpoint{13.032353in}{1.973684in}}%
\pgfpathlineto{\pgfqpoint{13.032353in}{2.952632in}}%
\pgfusepath{stroke}%
\end{pgfscope}%
\begin{pgfscope}%
\pgfpathrectangle{\pgfqpoint{12.211765in}{1.973684in}}{\pgfqpoint{2.188235in}{0.978947in}} %
\pgfusepath{clip}%
\pgfsetroundcap%
\pgfsetroundjoin%
\pgfsetlinewidth{1.003750pt}%
\definecolor{currentstroke}{rgb}{0.800000,0.800000,0.800000}%
\pgfsetstrokecolor{currentstroke}%
\pgfsetdash{}{0pt}%
\pgfpathmoveto{\pgfqpoint{13.305882in}{1.973684in}}%
\pgfpathlineto{\pgfqpoint{13.305882in}{2.952632in}}%
\pgfusepath{stroke}%
\end{pgfscope}%
\begin{pgfscope}%
\pgfpathrectangle{\pgfqpoint{12.211765in}{1.973684in}}{\pgfqpoint{2.188235in}{0.978947in}} %
\pgfusepath{clip}%
\pgfsetroundcap%
\pgfsetroundjoin%
\pgfsetlinewidth{1.003750pt}%
\definecolor{currentstroke}{rgb}{0.800000,0.800000,0.800000}%
\pgfsetstrokecolor{currentstroke}%
\pgfsetdash{}{0pt}%
\pgfpathmoveto{\pgfqpoint{13.579412in}{1.973684in}}%
\pgfpathlineto{\pgfqpoint{13.579412in}{2.952632in}}%
\pgfusepath{stroke}%
\end{pgfscope}%
\begin{pgfscope}%
\pgfpathrectangle{\pgfqpoint{12.211765in}{1.973684in}}{\pgfqpoint{2.188235in}{0.978947in}} %
\pgfusepath{clip}%
\pgfsetroundcap%
\pgfsetroundjoin%
\pgfsetlinewidth{1.003750pt}%
\definecolor{currentstroke}{rgb}{0.800000,0.800000,0.800000}%
\pgfsetstrokecolor{currentstroke}%
\pgfsetdash{}{0pt}%
\pgfpathmoveto{\pgfqpoint{13.852941in}{1.973684in}}%
\pgfpathlineto{\pgfqpoint{13.852941in}{2.952632in}}%
\pgfusepath{stroke}%
\end{pgfscope}%
\begin{pgfscope}%
\pgfpathrectangle{\pgfqpoint{12.211765in}{1.973684in}}{\pgfqpoint{2.188235in}{0.978947in}} %
\pgfusepath{clip}%
\pgfsetroundcap%
\pgfsetroundjoin%
\pgfsetlinewidth{1.003750pt}%
\definecolor{currentstroke}{rgb}{0.800000,0.800000,0.800000}%
\pgfsetstrokecolor{currentstroke}%
\pgfsetdash{}{0pt}%
\pgfpathmoveto{\pgfqpoint{14.126471in}{1.973684in}}%
\pgfpathlineto{\pgfqpoint{14.126471in}{2.952632in}}%
\pgfusepath{stroke}%
\end{pgfscope}%
\begin{pgfscope}%
\pgfpathrectangle{\pgfqpoint{12.211765in}{1.973684in}}{\pgfqpoint{2.188235in}{0.978947in}} %
\pgfusepath{clip}%
\pgfsetroundcap%
\pgfsetroundjoin%
\pgfsetlinewidth{1.003750pt}%
\definecolor{currentstroke}{rgb}{0.800000,0.800000,0.800000}%
\pgfsetstrokecolor{currentstroke}%
\pgfsetdash{}{0pt}%
\pgfpathmoveto{\pgfqpoint{14.400000in}{1.973684in}}%
\pgfpathlineto{\pgfqpoint{14.400000in}{2.952632in}}%
\pgfusepath{stroke}%
\end{pgfscope}%
\begin{pgfscope}%
\pgfpathrectangle{\pgfqpoint{12.211765in}{1.973684in}}{\pgfqpoint{2.188235in}{0.978947in}} %
\pgfusepath{clip}%
\pgfsetroundcap%
\pgfsetroundjoin%
\pgfsetlinewidth{1.003750pt}%
\definecolor{currentstroke}{rgb}{0.800000,0.800000,0.800000}%
\pgfsetstrokecolor{currentstroke}%
\pgfsetdash{}{0pt}%
\pgfpathmoveto{\pgfqpoint{12.211765in}{1.973684in}}%
\pgfpathlineto{\pgfqpoint{14.400000in}{1.973684in}}%
\pgfusepath{stroke}%
\end{pgfscope}%
\begin{pgfscope}%
\definecolor{textcolor}{rgb}{0.150000,0.150000,0.150000}%
\pgfsetstrokecolor{textcolor}%
\pgfsetfillcolor{textcolor}%
\pgftext[x=12.114542in,y=1.973684in,right,]{\color{textcolor}\sffamily\fontsize{10.000000}{12.000000}\selectfont \(\displaystyle 0\)}%
\end{pgfscope}%
\begin{pgfscope}%
\pgfpathrectangle{\pgfqpoint{12.211765in}{1.973684in}}{\pgfqpoint{2.188235in}{0.978947in}} %
\pgfusepath{clip}%
\pgfsetroundcap%
\pgfsetroundjoin%
\pgfsetlinewidth{1.003750pt}%
\definecolor{currentstroke}{rgb}{0.800000,0.800000,0.800000}%
\pgfsetstrokecolor{currentstroke}%
\pgfsetdash{}{0pt}%
\pgfpathmoveto{\pgfqpoint{12.211765in}{2.218421in}}%
\pgfpathlineto{\pgfqpoint{14.400000in}{2.218421in}}%
\pgfusepath{stroke}%
\end{pgfscope}%
\begin{pgfscope}%
\definecolor{textcolor}{rgb}{0.150000,0.150000,0.150000}%
\pgfsetstrokecolor{textcolor}%
\pgfsetfillcolor{textcolor}%
\pgftext[x=12.114542in,y=2.218421in,right,]{\color{textcolor}\sffamily\fontsize{10.000000}{12.000000}\selectfont \(\displaystyle 50\)}%
\end{pgfscope}%
\begin{pgfscope}%
\pgfpathrectangle{\pgfqpoint{12.211765in}{1.973684in}}{\pgfqpoint{2.188235in}{0.978947in}} %
\pgfusepath{clip}%
\pgfsetroundcap%
\pgfsetroundjoin%
\pgfsetlinewidth{1.003750pt}%
\definecolor{currentstroke}{rgb}{0.800000,0.800000,0.800000}%
\pgfsetstrokecolor{currentstroke}%
\pgfsetdash{}{0pt}%
\pgfpathmoveto{\pgfqpoint{12.211765in}{2.463158in}}%
\pgfpathlineto{\pgfqpoint{14.400000in}{2.463158in}}%
\pgfusepath{stroke}%
\end{pgfscope}%
\begin{pgfscope}%
\definecolor{textcolor}{rgb}{0.150000,0.150000,0.150000}%
\pgfsetstrokecolor{textcolor}%
\pgfsetfillcolor{textcolor}%
\pgftext[x=12.114542in,y=2.463158in,right,]{\color{textcolor}\sffamily\fontsize{10.000000}{12.000000}\selectfont \(\displaystyle 100\)}%
\end{pgfscope}%
\begin{pgfscope}%
\pgfpathrectangle{\pgfqpoint{12.211765in}{1.973684in}}{\pgfqpoint{2.188235in}{0.978947in}} %
\pgfusepath{clip}%
\pgfsetroundcap%
\pgfsetroundjoin%
\pgfsetlinewidth{1.003750pt}%
\definecolor{currentstroke}{rgb}{0.800000,0.800000,0.800000}%
\pgfsetstrokecolor{currentstroke}%
\pgfsetdash{}{0pt}%
\pgfpathmoveto{\pgfqpoint{12.211765in}{2.707895in}}%
\pgfpathlineto{\pgfqpoint{14.400000in}{2.707895in}}%
\pgfusepath{stroke}%
\end{pgfscope}%
\begin{pgfscope}%
\definecolor{textcolor}{rgb}{0.150000,0.150000,0.150000}%
\pgfsetstrokecolor{textcolor}%
\pgfsetfillcolor{textcolor}%
\pgftext[x=12.114542in,y=2.707895in,right,]{\color{textcolor}\sffamily\fontsize{10.000000}{12.000000}\selectfont \(\displaystyle 150\)}%
\end{pgfscope}%
\begin{pgfscope}%
\pgfpathrectangle{\pgfqpoint{12.211765in}{1.973684in}}{\pgfqpoint{2.188235in}{0.978947in}} %
\pgfusepath{clip}%
\pgfsetroundcap%
\pgfsetroundjoin%
\pgfsetlinewidth{1.003750pt}%
\definecolor{currentstroke}{rgb}{0.800000,0.800000,0.800000}%
\pgfsetstrokecolor{currentstroke}%
\pgfsetdash{}{0pt}%
\pgfpathmoveto{\pgfqpoint{12.211765in}{2.952632in}}%
\pgfpathlineto{\pgfqpoint{14.400000in}{2.952632in}}%
\pgfusepath{stroke}%
\end{pgfscope}%
\begin{pgfscope}%
\definecolor{textcolor}{rgb}{0.150000,0.150000,0.150000}%
\pgfsetstrokecolor{textcolor}%
\pgfsetfillcolor{textcolor}%
\pgftext[x=12.114542in,y=2.952632in,right,]{\color{textcolor}\sffamily\fontsize{10.000000}{12.000000}\selectfont \(\displaystyle 200\)}%
\end{pgfscope}%
\begin{pgfscope}%
\definecolor{textcolor}{rgb}{0.150000,0.150000,0.150000}%
\pgfsetstrokecolor{textcolor}%
\pgfsetfillcolor{textcolor}%
\pgftext[x=11.836764in,y=2.463158in,,bottom,rotate=90.000000]{\color{textcolor}\sffamily\fontsize{11.000000}{13.200000}\selectfont \(\displaystyle \theta^{\parallel}_j\)}%
\end{pgfscope}%
\begin{pgfscope}%
\pgfpathrectangle{\pgfqpoint{12.211765in}{1.973684in}}{\pgfqpoint{2.188235in}{0.978947in}} %
\pgfusepath{clip}%
\pgfsetroundcap%
\pgfsetroundjoin%
\pgfsetlinewidth{1.756562pt}%
\definecolor{currentstroke}{rgb}{0.298039,0.447059,0.690196}%
\pgfsetstrokecolor{currentstroke}%
\pgfsetdash{}{0pt}%
\pgfpathmoveto{\pgfqpoint{13.254504in}{1.973684in}}%
\pgfpathlineto{\pgfqpoint{13.189266in}{1.978579in}}%
\pgfpathlineto{\pgfqpoint{13.312350in}{1.983474in}}%
\pgfpathlineto{\pgfqpoint{13.313084in}{1.988368in}}%
\pgfpathlineto{\pgfqpoint{13.260418in}{1.993263in}}%
\pgfpathlineto{\pgfqpoint{13.318237in}{1.998158in}}%
\pgfpathlineto{\pgfqpoint{13.308900in}{2.003053in}}%
\pgfpathlineto{\pgfqpoint{13.295380in}{2.007947in}}%
\pgfpathlineto{\pgfqpoint{13.301259in}{2.012842in}}%
\pgfpathlineto{\pgfqpoint{13.324132in}{2.017737in}}%
\pgfpathlineto{\pgfqpoint{13.308200in}{2.022632in}}%
\pgfpathlineto{\pgfqpoint{13.240112in}{2.027526in}}%
\pgfpathlineto{\pgfqpoint{13.294799in}{2.032421in}}%
\pgfpathlineto{\pgfqpoint{13.294049in}{2.037316in}}%
\pgfpathlineto{\pgfqpoint{13.270829in}{2.042211in}}%
\pgfpathlineto{\pgfqpoint{13.179538in}{2.047105in}}%
\pgfpathlineto{\pgfqpoint{13.312272in}{2.052000in}}%
\pgfpathlineto{\pgfqpoint{13.403988in}{2.056895in}}%
\pgfpathlineto{\pgfqpoint{13.399700in}{2.061789in}}%
\pgfpathlineto{\pgfqpoint{13.311738in}{2.066684in}}%
\pgfpathlineto{\pgfqpoint{13.303752in}{2.071579in}}%
\pgfpathlineto{\pgfqpoint{13.306042in}{2.076474in}}%
\pgfpathlineto{\pgfqpoint{13.245475in}{2.086263in}}%
\pgfpathlineto{\pgfqpoint{13.321097in}{2.091158in}}%
\pgfpathlineto{\pgfqpoint{13.272993in}{2.096053in}}%
\pgfpathlineto{\pgfqpoint{13.331668in}{2.100947in}}%
\pgfpathlineto{\pgfqpoint{13.272623in}{2.105842in}}%
\pgfpathlineto{\pgfqpoint{13.296570in}{2.110737in}}%
\pgfpathlineto{\pgfqpoint{13.335995in}{2.115632in}}%
\pgfpathlineto{\pgfqpoint{13.342260in}{2.120526in}}%
\pgfpathlineto{\pgfqpoint{13.326158in}{2.125421in}}%
\pgfpathlineto{\pgfqpoint{13.438286in}{2.130316in}}%
\pgfpathlineto{\pgfqpoint{13.387100in}{2.135211in}}%
\pgfpathlineto{\pgfqpoint{13.294939in}{2.140105in}}%
\pgfpathlineto{\pgfqpoint{13.326103in}{2.145000in}}%
\pgfpathlineto{\pgfqpoint{13.283532in}{2.149895in}}%
\pgfpathlineto{\pgfqpoint{13.298797in}{2.154789in}}%
\pgfpathlineto{\pgfqpoint{13.368892in}{2.159684in}}%
\pgfpathlineto{\pgfqpoint{13.268380in}{2.164579in}}%
\pgfpathlineto{\pgfqpoint{13.330643in}{2.169474in}}%
\pgfpathlineto{\pgfqpoint{13.322829in}{2.174368in}}%
\pgfpathlineto{\pgfqpoint{13.420742in}{2.179263in}}%
\pgfpathlineto{\pgfqpoint{13.286721in}{2.184158in}}%
\pgfpathlineto{\pgfqpoint{13.288931in}{2.189053in}}%
\pgfpathlineto{\pgfqpoint{13.267586in}{2.193947in}}%
\pgfpathlineto{\pgfqpoint{13.304105in}{2.198842in}}%
\pgfpathlineto{\pgfqpoint{13.301963in}{2.203737in}}%
\pgfpathlineto{\pgfqpoint{13.313267in}{2.208632in}}%
\pgfpathlineto{\pgfqpoint{13.268193in}{2.213526in}}%
\pgfpathlineto{\pgfqpoint{13.334556in}{2.218421in}}%
\pgfpathlineto{\pgfqpoint{13.244578in}{2.223316in}}%
\pgfpathlineto{\pgfqpoint{13.328795in}{2.228211in}}%
\pgfpathlineto{\pgfqpoint{13.304382in}{2.233105in}}%
\pgfpathlineto{\pgfqpoint{13.324976in}{2.238000in}}%
\pgfpathlineto{\pgfqpoint{13.324958in}{2.242895in}}%
\pgfpathlineto{\pgfqpoint{13.303177in}{2.247789in}}%
\pgfpathlineto{\pgfqpoint{13.226056in}{2.252684in}}%
\pgfpathlineto{\pgfqpoint{13.299865in}{2.257579in}}%
\pgfpathlineto{\pgfqpoint{13.308175in}{2.262474in}}%
\pgfpathlineto{\pgfqpoint{13.278465in}{2.267368in}}%
\pgfpathlineto{\pgfqpoint{13.307135in}{2.272263in}}%
\pgfpathlineto{\pgfqpoint{13.346703in}{2.277158in}}%
\pgfpathlineto{\pgfqpoint{13.236512in}{2.282053in}}%
\pgfpathlineto{\pgfqpoint{13.311820in}{2.286947in}}%
\pgfpathlineto{\pgfqpoint{13.313234in}{2.291842in}}%
\pgfpathlineto{\pgfqpoint{13.346895in}{2.296737in}}%
\pgfpathlineto{\pgfqpoint{13.365053in}{2.301632in}}%
\pgfpathlineto{\pgfqpoint{13.273398in}{2.311421in}}%
\pgfpathlineto{\pgfqpoint{13.321052in}{2.316316in}}%
\pgfpathlineto{\pgfqpoint{13.282483in}{2.321211in}}%
\pgfpathlineto{\pgfqpoint{13.275354in}{2.326105in}}%
\pgfpathlineto{\pgfqpoint{13.288075in}{2.331000in}}%
\pgfpathlineto{\pgfqpoint{13.273938in}{2.335895in}}%
\pgfpathlineto{\pgfqpoint{13.300924in}{2.340789in}}%
\pgfpathlineto{\pgfqpoint{13.300143in}{2.345684in}}%
\pgfpathlineto{\pgfqpoint{13.301660in}{2.350579in}}%
\pgfpathlineto{\pgfqpoint{13.317251in}{2.355474in}}%
\pgfpathlineto{\pgfqpoint{13.289428in}{2.360368in}}%
\pgfpathlineto{\pgfqpoint{13.310841in}{2.365263in}}%
\pgfpathlineto{\pgfqpoint{13.308695in}{2.370158in}}%
\pgfpathlineto{\pgfqpoint{13.293611in}{2.375053in}}%
\pgfpathlineto{\pgfqpoint{13.330902in}{2.379947in}}%
\pgfpathlineto{\pgfqpoint{13.321641in}{2.384842in}}%
\pgfpathlineto{\pgfqpoint{13.259575in}{2.389737in}}%
\pgfpathlineto{\pgfqpoint{13.316445in}{2.394632in}}%
\pgfpathlineto{\pgfqpoint{13.299725in}{2.399526in}}%
\pgfpathlineto{\pgfqpoint{13.289737in}{2.404421in}}%
\pgfpathlineto{\pgfqpoint{13.358679in}{2.409316in}}%
\pgfpathlineto{\pgfqpoint{13.292349in}{2.414211in}}%
\pgfpathlineto{\pgfqpoint{13.327800in}{2.419105in}}%
\pgfpathlineto{\pgfqpoint{13.256841in}{2.424000in}}%
\pgfpathlineto{\pgfqpoint{13.309998in}{2.428895in}}%
\pgfpathlineto{\pgfqpoint{13.322215in}{2.433789in}}%
\pgfpathlineto{\pgfqpoint{13.376373in}{2.438684in}}%
\pgfpathlineto{\pgfqpoint{13.312457in}{2.443579in}}%
\pgfpathlineto{\pgfqpoint{13.328567in}{2.448474in}}%
\pgfpathlineto{\pgfqpoint{13.324321in}{2.453368in}}%
\pgfpathlineto{\pgfqpoint{13.307728in}{2.458263in}}%
\pgfpathlineto{\pgfqpoint{13.319797in}{2.463158in}}%
\pgfpathlineto{\pgfqpoint{13.297347in}{2.468053in}}%
\pgfpathlineto{\pgfqpoint{13.256251in}{2.472947in}}%
\pgfpathlineto{\pgfqpoint{13.301587in}{2.477842in}}%
\pgfpathlineto{\pgfqpoint{13.314788in}{2.482737in}}%
\pgfpathlineto{\pgfqpoint{13.310430in}{2.487632in}}%
\pgfpathlineto{\pgfqpoint{13.326177in}{2.492526in}}%
\pgfpathlineto{\pgfqpoint{13.307544in}{2.497421in}}%
\pgfpathlineto{\pgfqpoint{13.282440in}{2.502316in}}%
\pgfpathlineto{\pgfqpoint{13.311880in}{2.512105in}}%
\pgfpathlineto{\pgfqpoint{13.293385in}{2.521895in}}%
\pgfpathlineto{\pgfqpoint{13.310972in}{2.526789in}}%
\pgfpathlineto{\pgfqpoint{13.258616in}{2.531684in}}%
\pgfpathlineto{\pgfqpoint{13.297900in}{2.536579in}}%
\pgfpathlineto{\pgfqpoint{13.310701in}{2.546368in}}%
\pgfpathlineto{\pgfqpoint{13.313635in}{2.556158in}}%
\pgfpathlineto{\pgfqpoint{13.325844in}{2.561053in}}%
\pgfpathlineto{\pgfqpoint{13.340612in}{2.570842in}}%
\pgfpathlineto{\pgfqpoint{13.284579in}{2.575737in}}%
\pgfpathlineto{\pgfqpoint{13.293754in}{2.580632in}}%
\pgfpathlineto{\pgfqpoint{13.300288in}{2.585526in}}%
\pgfpathlineto{\pgfqpoint{13.268401in}{2.590421in}}%
\pgfpathlineto{\pgfqpoint{13.327032in}{2.595316in}}%
\pgfpathlineto{\pgfqpoint{13.300683in}{2.600211in}}%
\pgfpathlineto{\pgfqpoint{13.315683in}{2.605105in}}%
\pgfpathlineto{\pgfqpoint{13.354184in}{2.610000in}}%
\pgfpathlineto{\pgfqpoint{13.288173in}{2.614895in}}%
\pgfpathlineto{\pgfqpoint{13.321392in}{2.624684in}}%
\pgfpathlineto{\pgfqpoint{13.300307in}{2.629579in}}%
\pgfpathlineto{\pgfqpoint{13.334820in}{2.634474in}}%
\pgfpathlineto{\pgfqpoint{13.293272in}{2.639368in}}%
\pgfpathlineto{\pgfqpoint{13.326724in}{2.649158in}}%
\pgfpathlineto{\pgfqpoint{13.280095in}{2.654053in}}%
\pgfpathlineto{\pgfqpoint{13.268983in}{2.658947in}}%
\pgfpathlineto{\pgfqpoint{13.290118in}{2.663842in}}%
\pgfpathlineto{\pgfqpoint{13.298482in}{2.668737in}}%
\pgfpathlineto{\pgfqpoint{13.288537in}{2.673632in}}%
\pgfpathlineto{\pgfqpoint{13.313720in}{2.678526in}}%
\pgfpathlineto{\pgfqpoint{13.321468in}{2.683421in}}%
\pgfpathlineto{\pgfqpoint{13.315744in}{2.688316in}}%
\pgfpathlineto{\pgfqpoint{13.322744in}{2.693211in}}%
\pgfpathlineto{\pgfqpoint{13.286898in}{2.703000in}}%
\pgfpathlineto{\pgfqpoint{13.299970in}{2.707895in}}%
\pgfpathlineto{\pgfqpoint{13.297043in}{2.712789in}}%
\pgfpathlineto{\pgfqpoint{13.290688in}{2.717684in}}%
\pgfpathlineto{\pgfqpoint{13.314533in}{2.722579in}}%
\pgfpathlineto{\pgfqpoint{13.309145in}{2.727474in}}%
\pgfpathlineto{\pgfqpoint{13.290034in}{2.732368in}}%
\pgfpathlineto{\pgfqpoint{13.289893in}{2.737263in}}%
\pgfpathlineto{\pgfqpoint{13.310068in}{2.742158in}}%
\pgfpathlineto{\pgfqpoint{13.296589in}{2.747053in}}%
\pgfpathlineto{\pgfqpoint{13.320897in}{2.751947in}}%
\pgfpathlineto{\pgfqpoint{13.297609in}{2.756842in}}%
\pgfpathlineto{\pgfqpoint{13.310892in}{2.766632in}}%
\pgfpathlineto{\pgfqpoint{13.304803in}{2.776421in}}%
\pgfpathlineto{\pgfqpoint{13.296531in}{2.781316in}}%
\pgfpathlineto{\pgfqpoint{13.303112in}{2.786211in}}%
\pgfpathlineto{\pgfqpoint{13.304446in}{2.791105in}}%
\pgfpathlineto{\pgfqpoint{13.301414in}{2.796000in}}%
\pgfpathlineto{\pgfqpoint{13.295928in}{2.800895in}}%
\pgfpathlineto{\pgfqpoint{13.313364in}{2.805789in}}%
\pgfpathlineto{\pgfqpoint{13.281741in}{2.810684in}}%
\pgfpathlineto{\pgfqpoint{13.314561in}{2.815579in}}%
\pgfpathlineto{\pgfqpoint{13.292118in}{2.820474in}}%
\pgfpathlineto{\pgfqpoint{13.295705in}{2.825368in}}%
\pgfpathlineto{\pgfqpoint{13.294487in}{2.830263in}}%
\pgfpathlineto{\pgfqpoint{13.308584in}{2.835158in}}%
\pgfpathlineto{\pgfqpoint{13.297720in}{2.840053in}}%
\pgfpathlineto{\pgfqpoint{13.292433in}{2.844947in}}%
\pgfpathlineto{\pgfqpoint{13.297799in}{2.849842in}}%
\pgfpathlineto{\pgfqpoint{13.311125in}{2.854737in}}%
\pgfpathlineto{\pgfqpoint{13.293557in}{2.859632in}}%
\pgfpathlineto{\pgfqpoint{13.287877in}{2.864526in}}%
\pgfpathlineto{\pgfqpoint{13.301068in}{2.869421in}}%
\pgfpathlineto{\pgfqpoint{13.329634in}{2.874316in}}%
\pgfpathlineto{\pgfqpoint{13.315084in}{2.879211in}}%
\pgfpathlineto{\pgfqpoint{13.314210in}{2.884105in}}%
\pgfpathlineto{\pgfqpoint{13.283960in}{2.889000in}}%
\pgfpathlineto{\pgfqpoint{13.323897in}{2.893895in}}%
\pgfpathlineto{\pgfqpoint{13.293169in}{2.898789in}}%
\pgfpathlineto{\pgfqpoint{13.316325in}{2.903684in}}%
\pgfpathlineto{\pgfqpoint{13.304179in}{2.908579in}}%
\pgfpathlineto{\pgfqpoint{13.305212in}{2.913474in}}%
\pgfpathlineto{\pgfqpoint{13.281762in}{2.918368in}}%
\pgfpathlineto{\pgfqpoint{13.320330in}{2.923263in}}%
\pgfpathlineto{\pgfqpoint{13.314057in}{2.928158in}}%
\pgfpathlineto{\pgfqpoint{13.338443in}{2.933053in}}%
\pgfpathlineto{\pgfqpoint{13.318058in}{2.937947in}}%
\pgfpathlineto{\pgfqpoint{13.321239in}{2.942842in}}%
\pgfpathlineto{\pgfqpoint{13.316399in}{2.947737in}}%
\pgfpathlineto{\pgfqpoint{13.316399in}{2.947737in}}%
\pgfusepath{stroke}%
\end{pgfscope}%
\begin{pgfscope}%
\pgfsetrectcap%
\pgfsetmiterjoin%
\pgfsetlinewidth{1.003750pt}%
\definecolor{currentstroke}{rgb}{0.800000,0.800000,0.800000}%
\pgfsetstrokecolor{currentstroke}%
\pgfsetdash{}{0pt}%
\pgfpathmoveto{\pgfqpoint{12.211765in}{1.973684in}}%
\pgfpathlineto{\pgfqpoint{12.211765in}{2.952632in}}%
\pgfusepath{stroke}%
\end{pgfscope}%
\begin{pgfscope}%
\pgfsetrectcap%
\pgfsetmiterjoin%
\pgfsetlinewidth{1.003750pt}%
\definecolor{currentstroke}{rgb}{0.800000,0.800000,0.800000}%
\pgfsetstrokecolor{currentstroke}%
\pgfsetdash{}{0pt}%
\pgfpathmoveto{\pgfqpoint{14.400000in}{1.973684in}}%
\pgfpathlineto{\pgfqpoint{14.400000in}{2.952632in}}%
\pgfusepath{stroke}%
\end{pgfscope}%
\begin{pgfscope}%
\pgfsetrectcap%
\pgfsetmiterjoin%
\pgfsetlinewidth{1.003750pt}%
\definecolor{currentstroke}{rgb}{0.800000,0.800000,0.800000}%
\pgfsetstrokecolor{currentstroke}%
\pgfsetdash{}{0pt}%
\pgfpathmoveto{\pgfqpoint{12.211765in}{2.952632in}}%
\pgfpathlineto{\pgfqpoint{14.400000in}{2.952632in}}%
\pgfusepath{stroke}%
\end{pgfscope}%
\begin{pgfscope}%
\pgfsetrectcap%
\pgfsetmiterjoin%
\pgfsetlinewidth{1.003750pt}%
\definecolor{currentstroke}{rgb}{0.800000,0.800000,0.800000}%
\pgfsetstrokecolor{currentstroke}%
\pgfsetdash{}{0pt}%
\pgfpathmoveto{\pgfqpoint{12.211765in}{1.973684in}}%
\pgfpathlineto{\pgfqpoint{14.400000in}{1.973684in}}%
\pgfusepath{stroke}%
\end{pgfscope}%
\begin{pgfscope}%
\pgfsetbuttcap%
\pgfsetmiterjoin%
\definecolor{currentfill}{rgb}{1.000000,1.000000,1.000000}%
\pgfsetfillcolor{currentfill}%
\pgfsetlinewidth{0.000000pt}%
\definecolor{currentstroke}{rgb}{0.000000,0.000000,0.000000}%
\pgfsetstrokecolor{currentstroke}%
\pgfsetstrokeopacity{0.000000}%
\pgfsetdash{}{0pt}%
\pgfpathmoveto{\pgfqpoint{2.000000in}{0.750000in}}%
\pgfpathlineto{\pgfqpoint{6.376471in}{0.750000in}}%
\pgfpathlineto{\pgfqpoint{6.376471in}{1.728947in}}%
\pgfpathlineto{\pgfqpoint{2.000000in}{1.728947in}}%
\pgfpathclose%
\pgfusepath{fill}%
\end{pgfscope}%
\begin{pgfscope}%
\pgfpathrectangle{\pgfqpoint{2.000000in}{0.750000in}}{\pgfqpoint{4.376471in}{0.978947in}} %
\pgfusepath{clip}%
\pgfsetroundcap%
\pgfsetroundjoin%
\pgfsetlinewidth{1.003750pt}%
\definecolor{currentstroke}{rgb}{0.800000,0.800000,0.800000}%
\pgfsetstrokecolor{currentstroke}%
\pgfsetdash{}{0pt}%
\pgfpathmoveto{\pgfqpoint{2.000000in}{0.750000in}}%
\pgfpathlineto{\pgfqpoint{2.000000in}{1.728947in}}%
\pgfusepath{stroke}%
\end{pgfscope}%
\begin{pgfscope}%
\definecolor{textcolor}{rgb}{0.150000,0.150000,0.150000}%
\pgfsetstrokecolor{textcolor}%
\pgfsetfillcolor{textcolor}%
\pgftext[x=2.000000in,y=0.652778in,,top]{\color{textcolor}\sffamily\fontsize{10.000000}{12.000000}\selectfont \(\displaystyle -1.5\)}%
\end{pgfscope}%
\begin{pgfscope}%
\pgfpathrectangle{\pgfqpoint{2.000000in}{0.750000in}}{\pgfqpoint{4.376471in}{0.978947in}} %
\pgfusepath{clip}%
\pgfsetroundcap%
\pgfsetroundjoin%
\pgfsetlinewidth{1.003750pt}%
\definecolor{currentstroke}{rgb}{0.800000,0.800000,0.800000}%
\pgfsetstrokecolor{currentstroke}%
\pgfsetdash{}{0pt}%
\pgfpathmoveto{\pgfqpoint{2.875294in}{0.750000in}}%
\pgfpathlineto{\pgfqpoint{2.875294in}{1.728947in}}%
\pgfusepath{stroke}%
\end{pgfscope}%
\begin{pgfscope}%
\definecolor{textcolor}{rgb}{0.150000,0.150000,0.150000}%
\pgfsetstrokecolor{textcolor}%
\pgfsetfillcolor{textcolor}%
\pgftext[x=2.875294in,y=0.652778in,,top]{\color{textcolor}\sffamily\fontsize{10.000000}{12.000000}\selectfont \(\displaystyle -1.0\)}%
\end{pgfscope}%
\begin{pgfscope}%
\pgfpathrectangle{\pgfqpoint{2.000000in}{0.750000in}}{\pgfqpoint{4.376471in}{0.978947in}} %
\pgfusepath{clip}%
\pgfsetroundcap%
\pgfsetroundjoin%
\pgfsetlinewidth{1.003750pt}%
\definecolor{currentstroke}{rgb}{0.800000,0.800000,0.800000}%
\pgfsetstrokecolor{currentstroke}%
\pgfsetdash{}{0pt}%
\pgfpathmoveto{\pgfqpoint{3.750588in}{0.750000in}}%
\pgfpathlineto{\pgfqpoint{3.750588in}{1.728947in}}%
\pgfusepath{stroke}%
\end{pgfscope}%
\begin{pgfscope}%
\definecolor{textcolor}{rgb}{0.150000,0.150000,0.150000}%
\pgfsetstrokecolor{textcolor}%
\pgfsetfillcolor{textcolor}%
\pgftext[x=3.750588in,y=0.652778in,,top]{\color{textcolor}\sffamily\fontsize{10.000000}{12.000000}\selectfont \(\displaystyle -0.5\)}%
\end{pgfscope}%
\begin{pgfscope}%
\pgfpathrectangle{\pgfqpoint{2.000000in}{0.750000in}}{\pgfqpoint{4.376471in}{0.978947in}} %
\pgfusepath{clip}%
\pgfsetroundcap%
\pgfsetroundjoin%
\pgfsetlinewidth{1.003750pt}%
\definecolor{currentstroke}{rgb}{0.800000,0.800000,0.800000}%
\pgfsetstrokecolor{currentstroke}%
\pgfsetdash{}{0pt}%
\pgfpathmoveto{\pgfqpoint{4.625882in}{0.750000in}}%
\pgfpathlineto{\pgfqpoint{4.625882in}{1.728947in}}%
\pgfusepath{stroke}%
\end{pgfscope}%
\begin{pgfscope}%
\definecolor{textcolor}{rgb}{0.150000,0.150000,0.150000}%
\pgfsetstrokecolor{textcolor}%
\pgfsetfillcolor{textcolor}%
\pgftext[x=4.625882in,y=0.652778in,,top]{\color{textcolor}\sffamily\fontsize{10.000000}{12.000000}\selectfont \(\displaystyle 0.0\)}%
\end{pgfscope}%
\begin{pgfscope}%
\pgfpathrectangle{\pgfqpoint{2.000000in}{0.750000in}}{\pgfqpoint{4.376471in}{0.978947in}} %
\pgfusepath{clip}%
\pgfsetroundcap%
\pgfsetroundjoin%
\pgfsetlinewidth{1.003750pt}%
\definecolor{currentstroke}{rgb}{0.800000,0.800000,0.800000}%
\pgfsetstrokecolor{currentstroke}%
\pgfsetdash{}{0pt}%
\pgfpathmoveto{\pgfqpoint{5.501176in}{0.750000in}}%
\pgfpathlineto{\pgfqpoint{5.501176in}{1.728947in}}%
\pgfusepath{stroke}%
\end{pgfscope}%
\begin{pgfscope}%
\definecolor{textcolor}{rgb}{0.150000,0.150000,0.150000}%
\pgfsetstrokecolor{textcolor}%
\pgfsetfillcolor{textcolor}%
\pgftext[x=5.501176in,y=0.652778in,,top]{\color{textcolor}\sffamily\fontsize{10.000000}{12.000000}\selectfont \(\displaystyle 0.5\)}%
\end{pgfscope}%
\begin{pgfscope}%
\pgfpathrectangle{\pgfqpoint{2.000000in}{0.750000in}}{\pgfqpoint{4.376471in}{0.978947in}} %
\pgfusepath{clip}%
\pgfsetroundcap%
\pgfsetroundjoin%
\pgfsetlinewidth{1.003750pt}%
\definecolor{currentstroke}{rgb}{0.800000,0.800000,0.800000}%
\pgfsetstrokecolor{currentstroke}%
\pgfsetdash{}{0pt}%
\pgfpathmoveto{\pgfqpoint{6.376471in}{0.750000in}}%
\pgfpathlineto{\pgfqpoint{6.376471in}{1.728947in}}%
\pgfusepath{stroke}%
\end{pgfscope}%
\begin{pgfscope}%
\definecolor{textcolor}{rgb}{0.150000,0.150000,0.150000}%
\pgfsetstrokecolor{textcolor}%
\pgfsetfillcolor{textcolor}%
\pgftext[x=6.376471in,y=0.652778in,,top]{\color{textcolor}\sffamily\fontsize{10.000000}{12.000000}\selectfont \(\displaystyle 1.0\)}%
\end{pgfscope}%
\begin{pgfscope}%
\definecolor{textcolor}{rgb}{0.150000,0.150000,0.150000}%
\pgfsetstrokecolor{textcolor}%
\pgfsetfillcolor{textcolor}%
\pgftext[x=4.188235in,y=0.456313in,,top]{\color{textcolor}\sffamily\fontsize{11.000000}{13.200000}\selectfont x}%
\end{pgfscope}%
\begin{pgfscope}%
\pgfpathrectangle{\pgfqpoint{2.000000in}{0.750000in}}{\pgfqpoint{4.376471in}{0.978947in}} %
\pgfusepath{clip}%
\pgfsetroundcap%
\pgfsetroundjoin%
\pgfsetlinewidth{1.003750pt}%
\definecolor{currentstroke}{rgb}{0.800000,0.800000,0.800000}%
\pgfsetstrokecolor{currentstroke}%
\pgfsetdash{}{0pt}%
\pgfpathmoveto{\pgfqpoint{2.000000in}{0.913158in}}%
\pgfpathlineto{\pgfqpoint{6.376471in}{0.913158in}}%
\pgfusepath{stroke}%
\end{pgfscope}%
\begin{pgfscope}%
\definecolor{textcolor}{rgb}{0.150000,0.150000,0.150000}%
\pgfsetstrokecolor{textcolor}%
\pgfsetfillcolor{textcolor}%
\pgftext[x=1.902778in,y=0.913158in,right,]{\color{textcolor}\sffamily\fontsize{10.000000}{12.000000}\selectfont \(\displaystyle -1\)}%
\end{pgfscope}%
\begin{pgfscope}%
\pgfpathrectangle{\pgfqpoint{2.000000in}{0.750000in}}{\pgfqpoint{4.376471in}{0.978947in}} %
\pgfusepath{clip}%
\pgfsetroundcap%
\pgfsetroundjoin%
\pgfsetlinewidth{1.003750pt}%
\definecolor{currentstroke}{rgb}{0.800000,0.800000,0.800000}%
\pgfsetstrokecolor{currentstroke}%
\pgfsetdash{}{0pt}%
\pgfpathmoveto{\pgfqpoint{2.000000in}{1.117105in}}%
\pgfpathlineto{\pgfqpoint{6.376471in}{1.117105in}}%
\pgfusepath{stroke}%
\end{pgfscope}%
\begin{pgfscope}%
\definecolor{textcolor}{rgb}{0.150000,0.150000,0.150000}%
\pgfsetstrokecolor{textcolor}%
\pgfsetfillcolor{textcolor}%
\pgftext[x=1.902778in,y=1.117105in,right,]{\color{textcolor}\sffamily\fontsize{10.000000}{12.000000}\selectfont \(\displaystyle 0\)}%
\end{pgfscope}%
\begin{pgfscope}%
\pgfpathrectangle{\pgfqpoint{2.000000in}{0.750000in}}{\pgfqpoint{4.376471in}{0.978947in}} %
\pgfusepath{clip}%
\pgfsetroundcap%
\pgfsetroundjoin%
\pgfsetlinewidth{1.003750pt}%
\definecolor{currentstroke}{rgb}{0.800000,0.800000,0.800000}%
\pgfsetstrokecolor{currentstroke}%
\pgfsetdash{}{0pt}%
\pgfpathmoveto{\pgfqpoint{2.000000in}{1.321053in}}%
\pgfpathlineto{\pgfqpoint{6.376471in}{1.321053in}}%
\pgfusepath{stroke}%
\end{pgfscope}%
\begin{pgfscope}%
\definecolor{textcolor}{rgb}{0.150000,0.150000,0.150000}%
\pgfsetstrokecolor{textcolor}%
\pgfsetfillcolor{textcolor}%
\pgftext[x=1.902778in,y=1.321053in,right,]{\color{textcolor}\sffamily\fontsize{10.000000}{12.000000}\selectfont \(\displaystyle 1\)}%
\end{pgfscope}%
\begin{pgfscope}%
\pgfpathrectangle{\pgfqpoint{2.000000in}{0.750000in}}{\pgfqpoint{4.376471in}{0.978947in}} %
\pgfusepath{clip}%
\pgfsetroundcap%
\pgfsetroundjoin%
\pgfsetlinewidth{1.003750pt}%
\definecolor{currentstroke}{rgb}{0.800000,0.800000,0.800000}%
\pgfsetstrokecolor{currentstroke}%
\pgfsetdash{}{0pt}%
\pgfpathmoveto{\pgfqpoint{2.000000in}{1.525000in}}%
\pgfpathlineto{\pgfqpoint{6.376471in}{1.525000in}}%
\pgfusepath{stroke}%
\end{pgfscope}%
\begin{pgfscope}%
\definecolor{textcolor}{rgb}{0.150000,0.150000,0.150000}%
\pgfsetstrokecolor{textcolor}%
\pgfsetfillcolor{textcolor}%
\pgftext[x=1.902778in,y=1.525000in,right,]{\color{textcolor}\sffamily\fontsize{10.000000}{12.000000}\selectfont \(\displaystyle 2\)}%
\end{pgfscope}%
\begin{pgfscope}%
\pgfpathrectangle{\pgfqpoint{2.000000in}{0.750000in}}{\pgfqpoint{4.376471in}{0.978947in}} %
\pgfusepath{clip}%
\pgfsetroundcap%
\pgfsetroundjoin%
\pgfsetlinewidth{1.003750pt}%
\definecolor{currentstroke}{rgb}{0.800000,0.800000,0.800000}%
\pgfsetstrokecolor{currentstroke}%
\pgfsetdash{}{0pt}%
\pgfpathmoveto{\pgfqpoint{2.000000in}{1.728947in}}%
\pgfpathlineto{\pgfqpoint{6.376471in}{1.728947in}}%
\pgfusepath{stroke}%
\end{pgfscope}%
\begin{pgfscope}%
\definecolor{textcolor}{rgb}{0.150000,0.150000,0.150000}%
\pgfsetstrokecolor{textcolor}%
\pgfsetfillcolor{textcolor}%
\pgftext[x=1.902778in,y=1.728947in,right,]{\color{textcolor}\sffamily\fontsize{10.000000}{12.000000}\selectfont \(\displaystyle 3\)}%
\end{pgfscope}%
\begin{pgfscope}%
\definecolor{textcolor}{rgb}{0.150000,0.150000,0.150000}%
\pgfsetstrokecolor{textcolor}%
\pgfsetfillcolor{textcolor}%
\pgftext[x=1.655864in,y=1.239474in,,bottom,rotate=90.000000]{\color{textcolor}\sffamily\fontsize{11.000000}{13.200000}\selectfont y}%
\end{pgfscope}%
\begin{pgfscope}%
\pgfpathrectangle{\pgfqpoint{2.000000in}{0.750000in}}{\pgfqpoint{4.376471in}{0.978947in}} %
\pgfusepath{clip}%
\pgfsetbuttcap%
\pgfsetroundjoin%
\definecolor{currentfill}{rgb}{1.000000,0.000000,0.000000}%
\pgfsetfillcolor{currentfill}%
\pgfsetlinewidth{2.007500pt}%
\definecolor{currentstroke}{rgb}{1.000000,0.000000,0.000000}%
\pgfsetstrokecolor{currentstroke}%
\pgfsetdash{}{0pt}%
\pgfpathmoveto{\pgfqpoint{4.765731in}{1.307576in}}%
\pgfpathlineto{\pgfqpoint{4.827844in}{1.307576in}}%
\pgfpathmoveto{\pgfqpoint{4.796787in}{1.276519in}}%
\pgfpathlineto{\pgfqpoint{4.796787in}{1.338632in}}%
\pgfusepath{stroke,fill}%
\end{pgfscope}%
\begin{pgfscope}%
\pgfpathrectangle{\pgfqpoint{2.000000in}{0.750000in}}{\pgfqpoint{4.376471in}{0.978947in}} %
\pgfusepath{clip}%
\pgfsetbuttcap%
\pgfsetroundjoin%
\definecolor{currentfill}{rgb}{1.000000,0.000000,0.000000}%
\pgfsetfillcolor{currentfill}%
\pgfsetlinewidth{2.007500pt}%
\definecolor{currentstroke}{rgb}{1.000000,0.000000,0.000000}%
\pgfsetstrokecolor{currentstroke}%
\pgfsetdash{}{0pt}%
\pgfpathmoveto{\pgfqpoint{5.348242in}{0.990364in}}%
\pgfpathlineto{\pgfqpoint{5.410355in}{0.990364in}}%
\pgfpathmoveto{\pgfqpoint{5.379298in}{0.959307in}}%
\pgfpathlineto{\pgfqpoint{5.379298in}{1.021420in}}%
\pgfusepath{stroke,fill}%
\end{pgfscope}%
\begin{pgfscope}%
\pgfpathrectangle{\pgfqpoint{2.000000in}{0.750000in}}{\pgfqpoint{4.376471in}{0.978947in}} %
\pgfusepath{clip}%
\pgfsetbuttcap%
\pgfsetroundjoin%
\definecolor{currentfill}{rgb}{1.000000,0.000000,0.000000}%
\pgfsetfillcolor{currentfill}%
\pgfsetlinewidth{2.007500pt}%
\definecolor{currentstroke}{rgb}{1.000000,0.000000,0.000000}%
\pgfsetstrokecolor{currentstroke}%
\pgfsetdash{}{0pt}%
\pgfpathmoveto{\pgfqpoint{4.954619in}{1.361104in}}%
\pgfpathlineto{\pgfqpoint{5.016732in}{1.361104in}}%
\pgfpathmoveto{\pgfqpoint{4.985675in}{1.330047in}}%
\pgfpathlineto{\pgfqpoint{4.985675in}{1.392160in}}%
\pgfusepath{stroke,fill}%
\end{pgfscope}%
\begin{pgfscope}%
\pgfpathrectangle{\pgfqpoint{2.000000in}{0.750000in}}{\pgfqpoint{4.376471in}{0.978947in}} %
\pgfusepath{clip}%
\pgfsetbuttcap%
\pgfsetroundjoin%
\definecolor{currentfill}{rgb}{1.000000,0.000000,0.000000}%
\pgfsetfillcolor{currentfill}%
\pgfsetlinewidth{2.007500pt}%
\definecolor{currentstroke}{rgb}{1.000000,0.000000,0.000000}%
\pgfsetstrokecolor{currentstroke}%
\pgfsetdash{}{0pt}%
\pgfpathmoveto{\pgfqpoint{4.751970in}{1.250192in}}%
\pgfpathlineto{\pgfqpoint{4.814083in}{1.250192in}}%
\pgfpathmoveto{\pgfqpoint{4.783026in}{1.219135in}}%
\pgfpathlineto{\pgfqpoint{4.783026in}{1.281248in}}%
\pgfusepath{stroke,fill}%
\end{pgfscope}%
\begin{pgfscope}%
\pgfpathrectangle{\pgfqpoint{2.000000in}{0.750000in}}{\pgfqpoint{4.376471in}{0.978947in}} %
\pgfusepath{clip}%
\pgfsetbuttcap%
\pgfsetroundjoin%
\definecolor{currentfill}{rgb}{1.000000,0.000000,0.000000}%
\pgfsetfillcolor{currentfill}%
\pgfsetlinewidth{2.007500pt}%
\definecolor{currentstroke}{rgb}{1.000000,0.000000,0.000000}%
\pgfsetstrokecolor{currentstroke}%
\pgfsetdash{}{0pt}%
\pgfpathmoveto{\pgfqpoint{4.327528in}{0.923797in}}%
\pgfpathlineto{\pgfqpoint{4.389641in}{0.923797in}}%
\pgfpathmoveto{\pgfqpoint{4.358584in}{0.892741in}}%
\pgfpathlineto{\pgfqpoint{4.358584in}{0.954854in}}%
\pgfusepath{stroke,fill}%
\end{pgfscope}%
\begin{pgfscope}%
\pgfpathrectangle{\pgfqpoint{2.000000in}{0.750000in}}{\pgfqpoint{4.376471in}{0.978947in}} %
\pgfusepath{clip}%
\pgfsetbuttcap%
\pgfsetroundjoin%
\definecolor{currentfill}{rgb}{1.000000,0.000000,0.000000}%
\pgfsetfillcolor{currentfill}%
\pgfsetlinewidth{2.007500pt}%
\definecolor{currentstroke}{rgb}{1.000000,0.000000,0.000000}%
\pgfsetstrokecolor{currentstroke}%
\pgfsetdash{}{0pt}%
\pgfpathmoveto{\pgfqpoint{5.105627in}{1.193262in}}%
\pgfpathlineto{\pgfqpoint{5.167740in}{1.193262in}}%
\pgfpathmoveto{\pgfqpoint{5.136683in}{1.162205in}}%
\pgfpathlineto{\pgfqpoint{5.136683in}{1.224318in}}%
\pgfusepath{stroke,fill}%
\end{pgfscope}%
\begin{pgfscope}%
\pgfpathrectangle{\pgfqpoint{2.000000in}{0.750000in}}{\pgfqpoint{4.376471in}{0.978947in}} %
\pgfusepath{clip}%
\pgfsetbuttcap%
\pgfsetroundjoin%
\definecolor{currentfill}{rgb}{1.000000,0.000000,0.000000}%
\pgfsetfillcolor{currentfill}%
\pgfsetlinewidth{2.007500pt}%
\definecolor{currentstroke}{rgb}{1.000000,0.000000,0.000000}%
\pgfsetstrokecolor{currentstroke}%
\pgfsetdash{}{0pt}%
\pgfpathmoveto{\pgfqpoint{4.376308in}{0.961291in}}%
\pgfpathlineto{\pgfqpoint{4.438421in}{0.961291in}}%
\pgfpathmoveto{\pgfqpoint{4.407364in}{0.930235in}}%
\pgfpathlineto{\pgfqpoint{4.407364in}{0.992348in}}%
\pgfusepath{stroke,fill}%
\end{pgfscope}%
\begin{pgfscope}%
\pgfpathrectangle{\pgfqpoint{2.000000in}{0.750000in}}{\pgfqpoint{4.376471in}{0.978947in}} %
\pgfusepath{clip}%
\pgfsetbuttcap%
\pgfsetroundjoin%
\definecolor{currentfill}{rgb}{1.000000,0.000000,0.000000}%
\pgfsetfillcolor{currentfill}%
\pgfsetlinewidth{2.007500pt}%
\definecolor{currentstroke}{rgb}{1.000000,0.000000,0.000000}%
\pgfsetstrokecolor{currentstroke}%
\pgfsetdash{}{0pt}%
\pgfpathmoveto{\pgfqpoint{5.966492in}{1.601409in}}%
\pgfpathlineto{\pgfqpoint{6.028605in}{1.601409in}}%
\pgfpathmoveto{\pgfqpoint{5.997549in}{1.570353in}}%
\pgfpathlineto{\pgfqpoint{5.997549in}{1.632466in}}%
\pgfusepath{stroke,fill}%
\end{pgfscope}%
\begin{pgfscope}%
\pgfpathrectangle{\pgfqpoint{2.000000in}{0.750000in}}{\pgfqpoint{4.376471in}{0.978947in}} %
\pgfusepath{clip}%
\pgfsetbuttcap%
\pgfsetroundjoin%
\definecolor{currentfill}{rgb}{1.000000,0.000000,0.000000}%
\pgfsetfillcolor{currentfill}%
\pgfsetlinewidth{2.007500pt}%
\definecolor{currentstroke}{rgb}{1.000000,0.000000,0.000000}%
\pgfsetstrokecolor{currentstroke}%
\pgfsetdash{}{0pt}%
\pgfpathmoveto{\pgfqpoint{6.218191in}{1.501812in}}%
\pgfpathlineto{\pgfqpoint{6.280304in}{1.501812in}}%
\pgfpathmoveto{\pgfqpoint{6.249248in}{1.470755in}}%
\pgfpathlineto{\pgfqpoint{6.249248in}{1.532868in}}%
\pgfusepath{stroke,fill}%
\end{pgfscope}%
\begin{pgfscope}%
\pgfpathrectangle{\pgfqpoint{2.000000in}{0.750000in}}{\pgfqpoint{4.376471in}{0.978947in}} %
\pgfusepath{clip}%
\pgfsetbuttcap%
\pgfsetroundjoin%
\definecolor{currentfill}{rgb}{1.000000,0.000000,0.000000}%
\pgfsetfillcolor{currentfill}%
\pgfsetlinewidth{2.007500pt}%
\definecolor{currentstroke}{rgb}{1.000000,0.000000,0.000000}%
\pgfsetstrokecolor{currentstroke}%
\pgfsetdash{}{0pt}%
\pgfpathmoveto{\pgfqpoint{4.186734in}{0.999208in}}%
\pgfpathlineto{\pgfqpoint{4.248847in}{0.999208in}}%
\pgfpathmoveto{\pgfqpoint{4.217791in}{0.968152in}}%
\pgfpathlineto{\pgfqpoint{4.217791in}{1.030265in}}%
\pgfusepath{stroke,fill}%
\end{pgfscope}%
\begin{pgfscope}%
\pgfpathrectangle{\pgfqpoint{2.000000in}{0.750000in}}{\pgfqpoint{4.376471in}{0.978947in}} %
\pgfusepath{clip}%
\pgfsetbuttcap%
\pgfsetroundjoin%
\definecolor{currentfill}{rgb}{1.000000,0.000000,0.000000}%
\pgfsetfillcolor{currentfill}%
\pgfsetlinewidth{2.007500pt}%
\definecolor{currentstroke}{rgb}{1.000000,0.000000,0.000000}%
\pgfsetstrokecolor{currentstroke}%
\pgfsetdash{}{0pt}%
\pgfpathmoveto{\pgfqpoint{5.616207in}{1.149384in}}%
\pgfpathlineto{\pgfqpoint{5.678320in}{1.149384in}}%
\pgfpathmoveto{\pgfqpoint{5.647263in}{1.118327in}}%
\pgfpathlineto{\pgfqpoint{5.647263in}{1.180440in}}%
\pgfusepath{stroke,fill}%
\end{pgfscope}%
\begin{pgfscope}%
\pgfpathrectangle{\pgfqpoint{2.000000in}{0.750000in}}{\pgfqpoint{4.376471in}{0.978947in}} %
\pgfusepath{clip}%
\pgfsetbuttcap%
\pgfsetroundjoin%
\definecolor{currentfill}{rgb}{1.000000,0.000000,0.000000}%
\pgfsetfillcolor{currentfill}%
\pgfsetlinewidth{2.007500pt}%
\definecolor{currentstroke}{rgb}{1.000000,0.000000,0.000000}%
\pgfsetstrokecolor{currentstroke}%
\pgfsetdash{}{0pt}%
\pgfpathmoveto{\pgfqpoint{4.695992in}{1.189479in}}%
\pgfpathlineto{\pgfqpoint{4.758105in}{1.189479in}}%
\pgfpathmoveto{\pgfqpoint{4.727049in}{1.158422in}}%
\pgfpathlineto{\pgfqpoint{4.727049in}{1.220535in}}%
\pgfusepath{stroke,fill}%
\end{pgfscope}%
\begin{pgfscope}%
\pgfpathrectangle{\pgfqpoint{2.000000in}{0.750000in}}{\pgfqpoint{4.376471in}{0.978947in}} %
\pgfusepath{clip}%
\pgfsetbuttcap%
\pgfsetroundjoin%
\definecolor{currentfill}{rgb}{1.000000,0.000000,0.000000}%
\pgfsetfillcolor{currentfill}%
\pgfsetlinewidth{2.007500pt}%
\definecolor{currentstroke}{rgb}{1.000000,0.000000,0.000000}%
\pgfsetstrokecolor{currentstroke}%
\pgfsetdash{}{0pt}%
\pgfpathmoveto{\pgfqpoint{4.833062in}{1.317035in}}%
\pgfpathlineto{\pgfqpoint{4.895175in}{1.317035in}}%
\pgfpathmoveto{\pgfqpoint{4.864118in}{1.285978in}}%
\pgfpathlineto{\pgfqpoint{4.864118in}{1.348091in}}%
\pgfusepath{stroke,fill}%
\end{pgfscope}%
\begin{pgfscope}%
\pgfpathrectangle{\pgfqpoint{2.000000in}{0.750000in}}{\pgfqpoint{4.376471in}{0.978947in}} %
\pgfusepath{clip}%
\pgfsetbuttcap%
\pgfsetroundjoin%
\definecolor{currentfill}{rgb}{1.000000,0.000000,0.000000}%
\pgfsetfillcolor{currentfill}%
\pgfsetlinewidth{2.007500pt}%
\definecolor{currentstroke}{rgb}{1.000000,0.000000,0.000000}%
\pgfsetstrokecolor{currentstroke}%
\pgfsetdash{}{0pt}%
\pgfpathmoveto{\pgfqpoint{6.084915in}{1.577199in}}%
\pgfpathlineto{\pgfqpoint{6.147028in}{1.577199in}}%
\pgfpathmoveto{\pgfqpoint{6.115971in}{1.546142in}}%
\pgfpathlineto{\pgfqpoint{6.115971in}{1.608255in}}%
\pgfusepath{stroke,fill}%
\end{pgfscope}%
\begin{pgfscope}%
\pgfpathrectangle{\pgfqpoint{2.000000in}{0.750000in}}{\pgfqpoint{4.376471in}{0.978947in}} %
\pgfusepath{clip}%
\pgfsetbuttcap%
\pgfsetroundjoin%
\definecolor{currentfill}{rgb}{1.000000,0.000000,0.000000}%
\pgfsetfillcolor{currentfill}%
\pgfsetlinewidth{2.007500pt}%
\definecolor{currentstroke}{rgb}{1.000000,0.000000,0.000000}%
\pgfsetstrokecolor{currentstroke}%
\pgfsetdash{}{0pt}%
\pgfpathmoveto{\pgfqpoint{3.092947in}{1.289809in}}%
\pgfpathlineto{\pgfqpoint{3.155060in}{1.289809in}}%
\pgfpathmoveto{\pgfqpoint{3.124004in}{1.258753in}}%
\pgfpathlineto{\pgfqpoint{3.124004in}{1.320866in}}%
\pgfusepath{stroke,fill}%
\end{pgfscope}%
\begin{pgfscope}%
\pgfpathrectangle{\pgfqpoint{2.000000in}{0.750000in}}{\pgfqpoint{4.376471in}{0.978947in}} %
\pgfusepath{clip}%
\pgfsetbuttcap%
\pgfsetroundjoin%
\definecolor{currentfill}{rgb}{1.000000,0.000000,0.000000}%
\pgfsetfillcolor{currentfill}%
\pgfsetlinewidth{2.007500pt}%
\definecolor{currentstroke}{rgb}{1.000000,0.000000,0.000000}%
\pgfsetstrokecolor{currentstroke}%
\pgfsetdash{}{0pt}%
\pgfpathmoveto{\pgfqpoint{3.149293in}{1.232176in}}%
\pgfpathlineto{\pgfqpoint{3.211406in}{1.232176in}}%
\pgfpathmoveto{\pgfqpoint{3.180349in}{1.201119in}}%
\pgfpathlineto{\pgfqpoint{3.180349in}{1.263232in}}%
\pgfusepath{stroke,fill}%
\end{pgfscope}%
\begin{pgfscope}%
\pgfpathrectangle{\pgfqpoint{2.000000in}{0.750000in}}{\pgfqpoint{4.376471in}{0.978947in}} %
\pgfusepath{clip}%
\pgfsetbuttcap%
\pgfsetroundjoin%
\definecolor{currentfill}{rgb}{1.000000,0.000000,0.000000}%
\pgfsetfillcolor{currentfill}%
\pgfsetlinewidth{2.007500pt}%
\definecolor{currentstroke}{rgb}{1.000000,0.000000,0.000000}%
\pgfsetstrokecolor{currentstroke}%
\pgfsetdash{}{0pt}%
\pgfpathmoveto{\pgfqpoint{2.915026in}{1.519456in}}%
\pgfpathlineto{\pgfqpoint{2.977139in}{1.519456in}}%
\pgfpathmoveto{\pgfqpoint{2.946082in}{1.488399in}}%
\pgfpathlineto{\pgfqpoint{2.946082in}{1.550512in}}%
\pgfusepath{stroke,fill}%
\end{pgfscope}%
\begin{pgfscope}%
\pgfpathrectangle{\pgfqpoint{2.000000in}{0.750000in}}{\pgfqpoint{4.376471in}{0.978947in}} %
\pgfusepath{clip}%
\pgfsetbuttcap%
\pgfsetroundjoin%
\definecolor{currentfill}{rgb}{1.000000,0.000000,0.000000}%
\pgfsetfillcolor{currentfill}%
\pgfsetlinewidth{2.007500pt}%
\definecolor{currentstroke}{rgb}{1.000000,0.000000,0.000000}%
\pgfsetstrokecolor{currentstroke}%
\pgfsetdash{}{0pt}%
\pgfpathmoveto{\pgfqpoint{5.759387in}{1.365052in}}%
\pgfpathlineto{\pgfqpoint{5.821500in}{1.365052in}}%
\pgfpathmoveto{\pgfqpoint{5.790443in}{1.333995in}}%
\pgfpathlineto{\pgfqpoint{5.790443in}{1.396108in}}%
\pgfusepath{stroke,fill}%
\end{pgfscope}%
\begin{pgfscope}%
\pgfpathrectangle{\pgfqpoint{2.000000in}{0.750000in}}{\pgfqpoint{4.376471in}{0.978947in}} %
\pgfusepath{clip}%
\pgfsetbuttcap%
\pgfsetroundjoin%
\definecolor{currentfill}{rgb}{1.000000,0.000000,0.000000}%
\pgfsetfillcolor{currentfill}%
\pgfsetlinewidth{2.007500pt}%
\definecolor{currentstroke}{rgb}{1.000000,0.000000,0.000000}%
\pgfsetstrokecolor{currentstroke}%
\pgfsetdash{}{0pt}%
\pgfpathmoveto{\pgfqpoint{5.568702in}{1.087470in}}%
\pgfpathlineto{\pgfqpoint{5.630815in}{1.087470in}}%
\pgfpathmoveto{\pgfqpoint{5.599758in}{1.056414in}}%
\pgfpathlineto{\pgfqpoint{5.599758in}{1.118527in}}%
\pgfusepath{stroke,fill}%
\end{pgfscope}%
\begin{pgfscope}%
\pgfpathrectangle{\pgfqpoint{2.000000in}{0.750000in}}{\pgfqpoint{4.376471in}{0.978947in}} %
\pgfusepath{clip}%
\pgfsetbuttcap%
\pgfsetroundjoin%
\definecolor{currentfill}{rgb}{1.000000,0.000000,0.000000}%
\pgfsetfillcolor{currentfill}%
\pgfsetlinewidth{2.007500pt}%
\definecolor{currentstroke}{rgb}{1.000000,0.000000,0.000000}%
\pgfsetstrokecolor{currentstroke}%
\pgfsetdash{}{0pt}%
\pgfpathmoveto{\pgfqpoint{5.890304in}{1.494833in}}%
\pgfpathlineto{\pgfqpoint{5.952417in}{1.494833in}}%
\pgfpathmoveto{\pgfqpoint{5.921360in}{1.463777in}}%
\pgfpathlineto{\pgfqpoint{5.921360in}{1.525890in}}%
\pgfusepath{stroke,fill}%
\end{pgfscope}%
\begin{pgfscope}%
\pgfpathrectangle{\pgfqpoint{2.000000in}{0.750000in}}{\pgfqpoint{4.376471in}{0.978947in}} %
\pgfusepath{clip}%
\pgfsetbuttcap%
\pgfsetroundjoin%
\definecolor{currentfill}{rgb}{1.000000,0.000000,0.000000}%
\pgfsetfillcolor{currentfill}%
\pgfsetlinewidth{2.007500pt}%
\definecolor{currentstroke}{rgb}{1.000000,0.000000,0.000000}%
\pgfsetstrokecolor{currentstroke}%
\pgfsetdash{}{0pt}%
\pgfpathmoveto{\pgfqpoint{6.270553in}{1.426070in}}%
\pgfpathlineto{\pgfqpoint{6.332666in}{1.426070in}}%
\pgfpathmoveto{\pgfqpoint{6.301610in}{1.395013in}}%
\pgfpathlineto{\pgfqpoint{6.301610in}{1.457126in}}%
\pgfusepath{stroke,fill}%
\end{pgfscope}%
\begin{pgfscope}%
\pgfpathrectangle{\pgfqpoint{2.000000in}{0.750000in}}{\pgfqpoint{4.376471in}{0.978947in}} %
\pgfusepath{clip}%
\pgfsetbuttcap%
\pgfsetroundjoin%
\definecolor{currentfill}{rgb}{1.000000,0.000000,0.000000}%
\pgfsetfillcolor{currentfill}%
\pgfsetlinewidth{2.007500pt}%
\definecolor{currentstroke}{rgb}{1.000000,0.000000,0.000000}%
\pgfsetstrokecolor{currentstroke}%
\pgfsetdash{}{0pt}%
\pgfpathmoveto{\pgfqpoint{5.642233in}{1.242637in}}%
\pgfpathlineto{\pgfqpoint{5.704346in}{1.242637in}}%
\pgfpathmoveto{\pgfqpoint{5.673289in}{1.211581in}}%
\pgfpathlineto{\pgfqpoint{5.673289in}{1.273694in}}%
\pgfusepath{stroke,fill}%
\end{pgfscope}%
\begin{pgfscope}%
\pgfpathrectangle{\pgfqpoint{2.000000in}{0.750000in}}{\pgfqpoint{4.376471in}{0.978947in}} %
\pgfusepath{clip}%
\pgfsetbuttcap%
\pgfsetroundjoin%
\definecolor{currentfill}{rgb}{1.000000,0.000000,0.000000}%
\pgfsetfillcolor{currentfill}%
\pgfsetlinewidth{2.007500pt}%
\definecolor{currentstroke}{rgb}{1.000000,0.000000,0.000000}%
\pgfsetstrokecolor{currentstroke}%
\pgfsetdash{}{0pt}%
\pgfpathmoveto{\pgfqpoint{4.459958in}{0.967097in}}%
\pgfpathlineto{\pgfqpoint{4.522071in}{0.967097in}}%
\pgfpathmoveto{\pgfqpoint{4.491015in}{0.936040in}}%
\pgfpathlineto{\pgfqpoint{4.491015in}{0.998153in}}%
\pgfusepath{stroke,fill}%
\end{pgfscope}%
\begin{pgfscope}%
\pgfpathrectangle{\pgfqpoint{2.000000in}{0.750000in}}{\pgfqpoint{4.376471in}{0.978947in}} %
\pgfusepath{clip}%
\pgfsetbuttcap%
\pgfsetroundjoin%
\definecolor{currentfill}{rgb}{1.000000,0.000000,0.000000}%
\pgfsetfillcolor{currentfill}%
\pgfsetlinewidth{2.007500pt}%
\definecolor{currentstroke}{rgb}{1.000000,0.000000,0.000000}%
\pgfsetstrokecolor{currentstroke}%
\pgfsetdash{}{0pt}%
\pgfpathmoveto{\pgfqpoint{5.577008in}{1.109507in}}%
\pgfpathlineto{\pgfqpoint{5.639121in}{1.109507in}}%
\pgfpathmoveto{\pgfqpoint{5.608065in}{1.078450in}}%
\pgfpathlineto{\pgfqpoint{5.608065in}{1.140563in}}%
\pgfusepath{stroke,fill}%
\end{pgfscope}%
\begin{pgfscope}%
\pgfpathrectangle{\pgfqpoint{2.000000in}{0.750000in}}{\pgfqpoint{4.376471in}{0.978947in}} %
\pgfusepath{clip}%
\pgfsetbuttcap%
\pgfsetroundjoin%
\definecolor{currentfill}{rgb}{1.000000,0.000000,0.000000}%
\pgfsetfillcolor{currentfill}%
\pgfsetlinewidth{2.007500pt}%
\definecolor{currentstroke}{rgb}{1.000000,0.000000,0.000000}%
\pgfsetstrokecolor{currentstroke}%
\pgfsetdash{}{0pt}%
\pgfpathmoveto{\pgfqpoint{3.258337in}{1.130244in}}%
\pgfpathlineto{\pgfqpoint{3.320450in}{1.130244in}}%
\pgfpathmoveto{\pgfqpoint{3.289394in}{1.099188in}}%
\pgfpathlineto{\pgfqpoint{3.289394in}{1.161301in}}%
\pgfusepath{stroke,fill}%
\end{pgfscope}%
\begin{pgfscope}%
\pgfpathrectangle{\pgfqpoint{2.000000in}{0.750000in}}{\pgfqpoint{4.376471in}{0.978947in}} %
\pgfusepath{clip}%
\pgfsetbuttcap%
\pgfsetroundjoin%
\definecolor{currentfill}{rgb}{1.000000,0.000000,0.000000}%
\pgfsetfillcolor{currentfill}%
\pgfsetlinewidth{2.007500pt}%
\definecolor{currentstroke}{rgb}{1.000000,0.000000,0.000000}%
\pgfsetstrokecolor{currentstroke}%
\pgfsetdash{}{0pt}%
\pgfpathmoveto{\pgfqpoint{5.084714in}{1.233528in}}%
\pgfpathlineto{\pgfqpoint{5.146827in}{1.233528in}}%
\pgfpathmoveto{\pgfqpoint{5.115771in}{1.202472in}}%
\pgfpathlineto{\pgfqpoint{5.115771in}{1.264585in}}%
\pgfusepath{stroke,fill}%
\end{pgfscope}%
\begin{pgfscope}%
\pgfpathrectangle{\pgfqpoint{2.000000in}{0.750000in}}{\pgfqpoint{4.376471in}{0.978947in}} %
\pgfusepath{clip}%
\pgfsetbuttcap%
\pgfsetroundjoin%
\definecolor{currentfill}{rgb}{1.000000,0.000000,0.000000}%
\pgfsetfillcolor{currentfill}%
\pgfsetlinewidth{2.007500pt}%
\definecolor{currentstroke}{rgb}{1.000000,0.000000,0.000000}%
\pgfsetstrokecolor{currentstroke}%
\pgfsetdash{}{0pt}%
\pgfpathmoveto{\pgfqpoint{3.346143in}{1.138534in}}%
\pgfpathlineto{\pgfqpoint{3.408256in}{1.138534in}}%
\pgfpathmoveto{\pgfqpoint{3.377199in}{1.107478in}}%
\pgfpathlineto{\pgfqpoint{3.377199in}{1.169591in}}%
\pgfusepath{stroke,fill}%
\end{pgfscope}%
\begin{pgfscope}%
\pgfpathrectangle{\pgfqpoint{2.000000in}{0.750000in}}{\pgfqpoint{4.376471in}{0.978947in}} %
\pgfusepath{clip}%
\pgfsetbuttcap%
\pgfsetroundjoin%
\definecolor{currentfill}{rgb}{1.000000,0.000000,0.000000}%
\pgfsetfillcolor{currentfill}%
\pgfsetlinewidth{2.007500pt}%
\definecolor{currentstroke}{rgb}{1.000000,0.000000,0.000000}%
\pgfsetstrokecolor{currentstroke}%
\pgfsetdash{}{0pt}%
\pgfpathmoveto{\pgfqpoint{6.151690in}{1.538728in}}%
\pgfpathlineto{\pgfqpoint{6.213803in}{1.538728in}}%
\pgfpathmoveto{\pgfqpoint{6.182747in}{1.507671in}}%
\pgfpathlineto{\pgfqpoint{6.182747in}{1.569784in}}%
\pgfusepath{stroke,fill}%
\end{pgfscope}%
\begin{pgfscope}%
\pgfpathrectangle{\pgfqpoint{2.000000in}{0.750000in}}{\pgfqpoint{4.376471in}{0.978947in}} %
\pgfusepath{clip}%
\pgfsetbuttcap%
\pgfsetroundjoin%
\definecolor{currentfill}{rgb}{1.000000,0.000000,0.000000}%
\pgfsetfillcolor{currentfill}%
\pgfsetlinewidth{2.007500pt}%
\definecolor{currentstroke}{rgb}{1.000000,0.000000,0.000000}%
\pgfsetstrokecolor{currentstroke}%
\pgfsetdash{}{0pt}%
\pgfpathmoveto{\pgfqpoint{4.671321in}{1.185930in}}%
\pgfpathlineto{\pgfqpoint{4.733434in}{1.185930in}}%
\pgfpathmoveto{\pgfqpoint{4.702377in}{1.154874in}}%
\pgfpathlineto{\pgfqpoint{4.702377in}{1.216987in}}%
\pgfusepath{stroke,fill}%
\end{pgfscope}%
\begin{pgfscope}%
\pgfpathrectangle{\pgfqpoint{2.000000in}{0.750000in}}{\pgfqpoint{4.376471in}{0.978947in}} %
\pgfusepath{clip}%
\pgfsetbuttcap%
\pgfsetroundjoin%
\definecolor{currentfill}{rgb}{1.000000,0.000000,0.000000}%
\pgfsetfillcolor{currentfill}%
\pgfsetlinewidth{2.007500pt}%
\definecolor{currentstroke}{rgb}{1.000000,0.000000,0.000000}%
\pgfsetstrokecolor{currentstroke}%
\pgfsetdash{}{0pt}%
\pgfpathmoveto{\pgfqpoint{4.296042in}{0.934813in}}%
\pgfpathlineto{\pgfqpoint{4.358155in}{0.934813in}}%
\pgfpathmoveto{\pgfqpoint{4.327099in}{0.903757in}}%
\pgfpathlineto{\pgfqpoint{4.327099in}{0.965870in}}%
\pgfusepath{stroke,fill}%
\end{pgfscope}%
\begin{pgfscope}%
\pgfpathrectangle{\pgfqpoint{2.000000in}{0.750000in}}{\pgfqpoint{4.376471in}{0.978947in}} %
\pgfusepath{clip}%
\pgfsetbuttcap%
\pgfsetroundjoin%
\definecolor{currentfill}{rgb}{0.000000,0.000000,0.000000}%
\pgfsetfillcolor{currentfill}%
\pgfsetlinewidth{0.301125pt}%
\definecolor{currentstroke}{rgb}{0.000000,0.000000,0.000000}%
\pgfsetstrokecolor{currentstroke}%
\pgfsetdash{}{0pt}%
\pgfsys@defobject{currentmarker}{\pgfqpoint{-0.015528in}{-0.015528in}}{\pgfqpoint{0.015528in}{0.015528in}}{%
\pgfpathmoveto{\pgfqpoint{0.000000in}{-0.015528in}}%
\pgfpathcurveto{\pgfqpoint{0.004118in}{-0.015528in}}{\pgfqpoint{0.008068in}{-0.013892in}}{\pgfqpoint{0.010980in}{-0.010980in}}%
\pgfpathcurveto{\pgfqpoint{0.013892in}{-0.008068in}}{\pgfqpoint{0.015528in}{-0.004118in}}{\pgfqpoint{0.015528in}{0.000000in}}%
\pgfpathcurveto{\pgfqpoint{0.015528in}{0.004118in}}{\pgfqpoint{0.013892in}{0.008068in}}{\pgfqpoint{0.010980in}{0.010980in}}%
\pgfpathcurveto{\pgfqpoint{0.008068in}{0.013892in}}{\pgfqpoint{0.004118in}{0.015528in}}{\pgfqpoint{0.000000in}{0.015528in}}%
\pgfpathcurveto{\pgfqpoint{-0.004118in}{0.015528in}}{\pgfqpoint{-0.008068in}{0.013892in}}{\pgfqpoint{-0.010980in}{0.010980in}}%
\pgfpathcurveto{\pgfqpoint{-0.013892in}{0.008068in}}{\pgfqpoint{-0.015528in}{0.004118in}}{\pgfqpoint{-0.015528in}{0.000000in}}%
\pgfpathcurveto{\pgfqpoint{-0.015528in}{-0.004118in}}{\pgfqpoint{-0.013892in}{-0.008068in}}{\pgfqpoint{-0.010980in}{-0.010980in}}%
\pgfpathcurveto{\pgfqpoint{-0.008068in}{-0.013892in}}{\pgfqpoint{-0.004118in}{-0.015528in}}{\pgfqpoint{0.000000in}{-0.015528in}}%
\pgfpathclose%
\pgfusepath{stroke,fill}%
}%
\begin{pgfscope}%
\pgfsys@transformshift{2.875294in}{1.583863in}%
\pgfsys@useobject{currentmarker}{}%
\end{pgfscope}%
\begin{pgfscope}%
\pgfsys@transformshift{2.892888in}{1.489671in}%
\pgfsys@useobject{currentmarker}{}%
\end{pgfscope}%
\begin{pgfscope}%
\pgfsys@transformshift{2.910482in}{1.580445in}%
\pgfsys@useobject{currentmarker}{}%
\end{pgfscope}%
\begin{pgfscope}%
\pgfsys@transformshift{2.928076in}{1.599582in}%
\pgfsys@useobject{currentmarker}{}%
\end{pgfscope}%
\begin{pgfscope}%
\pgfsys@transformshift{2.945670in}{1.534796in}%
\pgfsys@useobject{currentmarker}{}%
\end{pgfscope}%
\begin{pgfscope}%
\pgfsys@transformshift{2.963263in}{1.530696in}%
\pgfsys@useobject{currentmarker}{}%
\end{pgfscope}%
\begin{pgfscope}%
\pgfsys@transformshift{2.980857in}{1.406938in}%
\pgfsys@useobject{currentmarker}{}%
\end{pgfscope}%
\begin{pgfscope}%
\pgfsys@transformshift{2.998451in}{1.406543in}%
\pgfsys@useobject{currentmarker}{}%
\end{pgfscope}%
\begin{pgfscope}%
\pgfsys@transformshift{3.016045in}{1.347216in}%
\pgfsys@useobject{currentmarker}{}%
\end{pgfscope}%
\begin{pgfscope}%
\pgfsys@transformshift{3.033639in}{1.351930in}%
\pgfsys@useobject{currentmarker}{}%
\end{pgfscope}%
\begin{pgfscope}%
\pgfsys@transformshift{3.051233in}{1.279228in}%
\pgfsys@useobject{currentmarker}{}%
\end{pgfscope}%
\begin{pgfscope}%
\pgfsys@transformshift{3.068826in}{1.160606in}%
\pgfsys@useobject{currentmarker}{}%
\end{pgfscope}%
\begin{pgfscope}%
\pgfsys@transformshift{3.086420in}{1.330351in}%
\pgfsys@useobject{currentmarker}{}%
\end{pgfscope}%
\begin{pgfscope}%
\pgfsys@transformshift{3.104014in}{1.248201in}%
\pgfsys@useobject{currentmarker}{}%
\end{pgfscope}%
\begin{pgfscope}%
\pgfsys@transformshift{3.121608in}{1.101305in}%
\pgfsys@useobject{currentmarker}{}%
\end{pgfscope}%
\begin{pgfscope}%
\pgfsys@transformshift{3.139202in}{1.294708in}%
\pgfsys@useobject{currentmarker}{}%
\end{pgfscope}%
\begin{pgfscope}%
\pgfsys@transformshift{3.156796in}{1.136705in}%
\pgfsys@useobject{currentmarker}{}%
\end{pgfscope}%
\begin{pgfscope}%
\pgfsys@transformshift{3.174390in}{1.218082in}%
\pgfsys@useobject{currentmarker}{}%
\end{pgfscope}%
\begin{pgfscope}%
\pgfsys@transformshift{3.191983in}{1.272640in}%
\pgfsys@useobject{currentmarker}{}%
\end{pgfscope}%
\begin{pgfscope}%
\pgfsys@transformshift{3.209577in}{1.198976in}%
\pgfsys@useobject{currentmarker}{}%
\end{pgfscope}%
\begin{pgfscope}%
\pgfsys@transformshift{3.227171in}{1.291626in}%
\pgfsys@useobject{currentmarker}{}%
\end{pgfscope}%
\begin{pgfscope}%
\pgfsys@transformshift{3.244765in}{1.041254in}%
\pgfsys@useobject{currentmarker}{}%
\end{pgfscope}%
\begin{pgfscope}%
\pgfsys@transformshift{3.262359in}{1.202078in}%
\pgfsys@useobject{currentmarker}{}%
\end{pgfscope}%
\begin{pgfscope}%
\pgfsys@transformshift{3.279953in}{1.087229in}%
\pgfsys@useobject{currentmarker}{}%
\end{pgfscope}%
\begin{pgfscope}%
\pgfsys@transformshift{3.297547in}{1.066387in}%
\pgfsys@useobject{currentmarker}{}%
\end{pgfscope}%
\begin{pgfscope}%
\pgfsys@transformshift{3.315140in}{1.096351in}%
\pgfsys@useobject{currentmarker}{}%
\end{pgfscope}%
\begin{pgfscope}%
\pgfsys@transformshift{3.332734in}{1.125805in}%
\pgfsys@useobject{currentmarker}{}%
\end{pgfscope}%
\begin{pgfscope}%
\pgfsys@transformshift{3.350328in}{1.167420in}%
\pgfsys@useobject{currentmarker}{}%
\end{pgfscope}%
\begin{pgfscope}%
\pgfsys@transformshift{3.367922in}{1.048813in}%
\pgfsys@useobject{currentmarker}{}%
\end{pgfscope}%
\begin{pgfscope}%
\pgfsys@transformshift{3.385516in}{1.267121in}%
\pgfsys@useobject{currentmarker}{}%
\end{pgfscope}%
\begin{pgfscope}%
\pgfsys@transformshift{3.403110in}{1.231941in}%
\pgfsys@useobject{currentmarker}{}%
\end{pgfscope}%
\begin{pgfscope}%
\pgfsys@transformshift{3.420704in}{1.038406in}%
\pgfsys@useobject{currentmarker}{}%
\end{pgfscope}%
\begin{pgfscope}%
\pgfsys@transformshift{3.438297in}{1.358678in}%
\pgfsys@useobject{currentmarker}{}%
\end{pgfscope}%
\begin{pgfscope}%
\pgfsys@transformshift{3.455891in}{1.413170in}%
\pgfsys@useobject{currentmarker}{}%
\end{pgfscope}%
\begin{pgfscope}%
\pgfsys@transformshift{3.473485in}{1.353852in}%
\pgfsys@useobject{currentmarker}{}%
\end{pgfscope}%
\begin{pgfscope}%
\pgfsys@transformshift{3.491079in}{1.229798in}%
\pgfsys@useobject{currentmarker}{}%
\end{pgfscope}%
\begin{pgfscope}%
\pgfsys@transformshift{3.508673in}{1.153945in}%
\pgfsys@useobject{currentmarker}{}%
\end{pgfscope}%
\begin{pgfscope}%
\pgfsys@transformshift{3.526267in}{1.385934in}%
\pgfsys@useobject{currentmarker}{}%
\end{pgfscope}%
\begin{pgfscope}%
\pgfsys@transformshift{3.543860in}{1.252646in}%
\pgfsys@useobject{currentmarker}{}%
\end{pgfscope}%
\begin{pgfscope}%
\pgfsys@transformshift{3.561454in}{1.433641in}%
\pgfsys@useobject{currentmarker}{}%
\end{pgfscope}%
\begin{pgfscope}%
\pgfsys@transformshift{3.579048in}{1.345114in}%
\pgfsys@useobject{currentmarker}{}%
\end{pgfscope}%
\begin{pgfscope}%
\pgfsys@transformshift{3.596642in}{1.437827in}%
\pgfsys@useobject{currentmarker}{}%
\end{pgfscope}%
\begin{pgfscope}%
\pgfsys@transformshift{3.614236in}{1.388209in}%
\pgfsys@useobject{currentmarker}{}%
\end{pgfscope}%
\begin{pgfscope}%
\pgfsys@transformshift{3.631830in}{1.436642in}%
\pgfsys@useobject{currentmarker}{}%
\end{pgfscope}%
\begin{pgfscope}%
\pgfsys@transformshift{3.649424in}{1.377294in}%
\pgfsys@useobject{currentmarker}{}%
\end{pgfscope}%
\begin{pgfscope}%
\pgfsys@transformshift{3.667017in}{1.568717in}%
\pgfsys@useobject{currentmarker}{}%
\end{pgfscope}%
\begin{pgfscope}%
\pgfsys@transformshift{3.684611in}{1.408526in}%
\pgfsys@useobject{currentmarker}{}%
\end{pgfscope}%
\begin{pgfscope}%
\pgfsys@transformshift{3.702205in}{1.444018in}%
\pgfsys@useobject{currentmarker}{}%
\end{pgfscope}%
\begin{pgfscope}%
\pgfsys@transformshift{3.719799in}{1.600841in}%
\pgfsys@useobject{currentmarker}{}%
\end{pgfscope}%
\begin{pgfscope}%
\pgfsys@transformshift{3.737393in}{1.275399in}%
\pgfsys@useobject{currentmarker}{}%
\end{pgfscope}%
\begin{pgfscope}%
\pgfsys@transformshift{3.754987in}{1.285462in}%
\pgfsys@useobject{currentmarker}{}%
\end{pgfscope}%
\begin{pgfscope}%
\pgfsys@transformshift{3.772581in}{1.514150in}%
\pgfsys@useobject{currentmarker}{}%
\end{pgfscope}%
\begin{pgfscope}%
\pgfsys@transformshift{3.790174in}{1.293998in}%
\pgfsys@useobject{currentmarker}{}%
\end{pgfscope}%
\begin{pgfscope}%
\pgfsys@transformshift{3.807768in}{1.608181in}%
\pgfsys@useobject{currentmarker}{}%
\end{pgfscope}%
\begin{pgfscope}%
\pgfsys@transformshift{3.825362in}{1.362188in}%
\pgfsys@useobject{currentmarker}{}%
\end{pgfscope}%
\begin{pgfscope}%
\pgfsys@transformshift{3.842956in}{1.320572in}%
\pgfsys@useobject{currentmarker}{}%
\end{pgfscope}%
\begin{pgfscope}%
\pgfsys@transformshift{3.860550in}{1.583393in}%
\pgfsys@useobject{currentmarker}{}%
\end{pgfscope}%
\begin{pgfscope}%
\pgfsys@transformshift{3.878144in}{1.526929in}%
\pgfsys@useobject{currentmarker}{}%
\end{pgfscope}%
\begin{pgfscope}%
\pgfsys@transformshift{3.895738in}{1.553271in}%
\pgfsys@useobject{currentmarker}{}%
\end{pgfscope}%
\begin{pgfscope}%
\pgfsys@transformshift{3.913331in}{1.440414in}%
\pgfsys@useobject{currentmarker}{}%
\end{pgfscope}%
\begin{pgfscope}%
\pgfsys@transformshift{3.930925in}{1.243825in}%
\pgfsys@useobject{currentmarker}{}%
\end{pgfscope}%
\begin{pgfscope}%
\pgfsys@transformshift{3.948519in}{1.508616in}%
\pgfsys@useobject{currentmarker}{}%
\end{pgfscope}%
\begin{pgfscope}%
\pgfsys@transformshift{3.966113in}{1.267415in}%
\pgfsys@useobject{currentmarker}{}%
\end{pgfscope}%
\begin{pgfscope}%
\pgfsys@transformshift{3.983707in}{1.356354in}%
\pgfsys@useobject{currentmarker}{}%
\end{pgfscope}%
\begin{pgfscope}%
\pgfsys@transformshift{4.001301in}{1.349946in}%
\pgfsys@useobject{currentmarker}{}%
\end{pgfscope}%
\begin{pgfscope}%
\pgfsys@transformshift{4.018894in}{1.215603in}%
\pgfsys@useobject{currentmarker}{}%
\end{pgfscope}%
\begin{pgfscope}%
\pgfsys@transformshift{4.036488in}{1.271512in}%
\pgfsys@useobject{currentmarker}{}%
\end{pgfscope}%
\begin{pgfscope}%
\pgfsys@transformshift{4.054082in}{1.280038in}%
\pgfsys@useobject{currentmarker}{}%
\end{pgfscope}%
\begin{pgfscope}%
\pgfsys@transformshift{4.071676in}{1.201314in}%
\pgfsys@useobject{currentmarker}{}%
\end{pgfscope}%
\begin{pgfscope}%
\pgfsys@transformshift{4.089270in}{1.027783in}%
\pgfsys@useobject{currentmarker}{}%
\end{pgfscope}%
\begin{pgfscope}%
\pgfsys@transformshift{4.106864in}{1.147508in}%
\pgfsys@useobject{currentmarker}{}%
\end{pgfscope}%
\begin{pgfscope}%
\pgfsys@transformshift{4.124458in}{1.230000in}%
\pgfsys@useobject{currentmarker}{}%
\end{pgfscope}%
\begin{pgfscope}%
\pgfsys@transformshift{4.142051in}{1.002209in}%
\pgfsys@useobject{currentmarker}{}%
\end{pgfscope}%
\begin{pgfscope}%
\pgfsys@transformshift{4.159645in}{1.036913in}%
\pgfsys@useobject{currentmarker}{}%
\end{pgfscope}%
\begin{pgfscope}%
\pgfsys@transformshift{4.177239in}{0.987962in}%
\pgfsys@useobject{currentmarker}{}%
\end{pgfscope}%
\begin{pgfscope}%
\pgfsys@transformshift{4.194833in}{1.202285in}%
\pgfsys@useobject{currentmarker}{}%
\end{pgfscope}%
\begin{pgfscope}%
\pgfsys@transformshift{4.212427in}{1.065014in}%
\pgfsys@useobject{currentmarker}{}%
\end{pgfscope}%
\begin{pgfscope}%
\pgfsys@transformshift{4.230021in}{1.022286in}%
\pgfsys@useobject{currentmarker}{}%
\end{pgfscope}%
\begin{pgfscope}%
\pgfsys@transformshift{4.247615in}{0.888172in}%
\pgfsys@useobject{currentmarker}{}%
\end{pgfscope}%
\begin{pgfscope}%
\pgfsys@transformshift{4.265208in}{1.009412in}%
\pgfsys@useobject{currentmarker}{}%
\end{pgfscope}%
\begin{pgfscope}%
\pgfsys@transformshift{4.282802in}{0.875298in}%
\pgfsys@useobject{currentmarker}{}%
\end{pgfscope}%
\begin{pgfscope}%
\pgfsys@transformshift{4.300396in}{0.938933in}%
\pgfsys@useobject{currentmarker}{}%
\end{pgfscope}%
\begin{pgfscope}%
\pgfsys@transformshift{4.317990in}{0.864529in}%
\pgfsys@useobject{currentmarker}{}%
\end{pgfscope}%
\begin{pgfscope}%
\pgfsys@transformshift{4.335584in}{0.994133in}%
\pgfsys@useobject{currentmarker}{}%
\end{pgfscope}%
\begin{pgfscope}%
\pgfsys@transformshift{4.353178in}{0.981871in}%
\pgfsys@useobject{currentmarker}{}%
\end{pgfscope}%
\begin{pgfscope}%
\pgfsys@transformshift{4.370772in}{0.901892in}%
\pgfsys@useobject{currentmarker}{}%
\end{pgfscope}%
\begin{pgfscope}%
\pgfsys@transformshift{4.388365in}{0.965702in}%
\pgfsys@useobject{currentmarker}{}%
\end{pgfscope}%
\begin{pgfscope}%
\pgfsys@transformshift{4.405959in}{0.818136in}%
\pgfsys@useobject{currentmarker}{}%
\end{pgfscope}%
\begin{pgfscope}%
\pgfsys@transformshift{4.423553in}{0.783851in}%
\pgfsys@useobject{currentmarker}{}%
\end{pgfscope}%
\begin{pgfscope}%
\pgfsys@transformshift{4.441147in}{0.989012in}%
\pgfsys@useobject{currentmarker}{}%
\end{pgfscope}%
\begin{pgfscope}%
\pgfsys@transformshift{4.458741in}{0.971362in}%
\pgfsys@useobject{currentmarker}{}%
\end{pgfscope}%
\begin{pgfscope}%
\pgfsys@transformshift{4.476335in}{1.031043in}%
\pgfsys@useobject{currentmarker}{}%
\end{pgfscope}%
\begin{pgfscope}%
\pgfsys@transformshift{4.493928in}{1.222861in}%
\pgfsys@useobject{currentmarker}{}%
\end{pgfscope}%
\begin{pgfscope}%
\pgfsys@transformshift{4.511522in}{1.091202in}%
\pgfsys@useobject{currentmarker}{}%
\end{pgfscope}%
\begin{pgfscope}%
\pgfsys@transformshift{4.529116in}{0.918187in}%
\pgfsys@useobject{currentmarker}{}%
\end{pgfscope}%
\begin{pgfscope}%
\pgfsys@transformshift{4.546710in}{1.142721in}%
\pgfsys@useobject{currentmarker}{}%
\end{pgfscope}%
\begin{pgfscope}%
\pgfsys@transformshift{4.564304in}{0.913152in}%
\pgfsys@useobject{currentmarker}{}%
\end{pgfscope}%
\begin{pgfscope}%
\pgfsys@transformshift{4.581898in}{1.019588in}%
\pgfsys@useobject{currentmarker}{}%
\end{pgfscope}%
\begin{pgfscope}%
\pgfsys@transformshift{4.599492in}{1.079610in}%
\pgfsys@useobject{currentmarker}{}%
\end{pgfscope}%
\begin{pgfscope}%
\pgfsys@transformshift{4.617085in}{1.281587in}%
\pgfsys@useobject{currentmarker}{}%
\end{pgfscope}%
\begin{pgfscope}%
\pgfsys@transformshift{4.634679in}{1.051414in}%
\pgfsys@useobject{currentmarker}{}%
\end{pgfscope}%
\begin{pgfscope}%
\pgfsys@transformshift{4.652273in}{1.063552in}%
\pgfsys@useobject{currentmarker}{}%
\end{pgfscope}%
\begin{pgfscope}%
\pgfsys@transformshift{4.669867in}{1.158029in}%
\pgfsys@useobject{currentmarker}{}%
\end{pgfscope}%
\begin{pgfscope}%
\pgfsys@transformshift{4.687461in}{1.120223in}%
\pgfsys@useobject{currentmarker}{}%
\end{pgfscope}%
\begin{pgfscope}%
\pgfsys@transformshift{4.705055in}{1.321952in}%
\pgfsys@useobject{currentmarker}{}%
\end{pgfscope}%
\begin{pgfscope}%
\pgfsys@transformshift{4.722649in}{1.115308in}%
\pgfsys@useobject{currentmarker}{}%
\end{pgfscope}%
\begin{pgfscope}%
\pgfsys@transformshift{4.740242in}{1.125790in}%
\pgfsys@useobject{currentmarker}{}%
\end{pgfscope}%
\begin{pgfscope}%
\pgfsys@transformshift{4.757836in}{1.214357in}%
\pgfsys@useobject{currentmarker}{}%
\end{pgfscope}%
\begin{pgfscope}%
\pgfsys@transformshift{4.775430in}{1.223091in}%
\pgfsys@useobject{currentmarker}{}%
\end{pgfscope}%
\begin{pgfscope}%
\pgfsys@transformshift{4.793024in}{1.484043in}%
\pgfsys@useobject{currentmarker}{}%
\end{pgfscope}%
\begin{pgfscope}%
\pgfsys@transformshift{4.810618in}{1.395905in}%
\pgfsys@useobject{currentmarker}{}%
\end{pgfscope}%
\begin{pgfscope}%
\pgfsys@transformshift{4.828212in}{1.318116in}%
\pgfsys@useobject{currentmarker}{}%
\end{pgfscope}%
\begin{pgfscope}%
\pgfsys@transformshift{4.845805in}{1.192524in}%
\pgfsys@useobject{currentmarker}{}%
\end{pgfscope}%
\begin{pgfscope}%
\pgfsys@transformshift{4.863399in}{1.410011in}%
\pgfsys@useobject{currentmarker}{}%
\end{pgfscope}%
\begin{pgfscope}%
\pgfsys@transformshift{4.880993in}{1.226407in}%
\pgfsys@useobject{currentmarker}{}%
\end{pgfscope}%
\begin{pgfscope}%
\pgfsys@transformshift{4.898587in}{1.173408in}%
\pgfsys@useobject{currentmarker}{}%
\end{pgfscope}%
\begin{pgfscope}%
\pgfsys@transformshift{4.916181in}{1.452637in}%
\pgfsys@useobject{currentmarker}{}%
\end{pgfscope}%
\begin{pgfscope}%
\pgfsys@transformshift{4.933775in}{1.362397in}%
\pgfsys@useobject{currentmarker}{}%
\end{pgfscope}%
\begin{pgfscope}%
\pgfsys@transformshift{4.951369in}{1.420628in}%
\pgfsys@useobject{currentmarker}{}%
\end{pgfscope}%
\begin{pgfscope}%
\pgfsys@transformshift{4.968962in}{1.353984in}%
\pgfsys@useobject{currentmarker}{}%
\end{pgfscope}%
\begin{pgfscope}%
\pgfsys@transformshift{4.986556in}{1.401791in}%
\pgfsys@useobject{currentmarker}{}%
\end{pgfscope}%
\begin{pgfscope}%
\pgfsys@transformshift{5.004150in}{1.239230in}%
\pgfsys@useobject{currentmarker}{}%
\end{pgfscope}%
\begin{pgfscope}%
\pgfsys@transformshift{5.021744in}{1.189725in}%
\pgfsys@useobject{currentmarker}{}%
\end{pgfscope}%
\begin{pgfscope}%
\pgfsys@transformshift{5.039338in}{1.352764in}%
\pgfsys@useobject{currentmarker}{}%
\end{pgfscope}%
\begin{pgfscope}%
\pgfsys@transformshift{5.056932in}{1.188036in}%
\pgfsys@useobject{currentmarker}{}%
\end{pgfscope}%
\begin{pgfscope}%
\pgfsys@transformshift{5.074526in}{1.185140in}%
\pgfsys@useobject{currentmarker}{}%
\end{pgfscope}%
\begin{pgfscope}%
\pgfsys@transformshift{5.092119in}{1.193453in}%
\pgfsys@useobject{currentmarker}{}%
\end{pgfscope}%
\begin{pgfscope}%
\pgfsys@transformshift{5.109713in}{1.225272in}%
\pgfsys@useobject{currentmarker}{}%
\end{pgfscope}%
\begin{pgfscope}%
\pgfsys@transformshift{5.127307in}{1.170299in}%
\pgfsys@useobject{currentmarker}{}%
\end{pgfscope}%
\begin{pgfscope}%
\pgfsys@transformshift{5.144901in}{1.048615in}%
\pgfsys@useobject{currentmarker}{}%
\end{pgfscope}%
\begin{pgfscope}%
\pgfsys@transformshift{5.162495in}{1.105339in}%
\pgfsys@useobject{currentmarker}{}%
\end{pgfscope}%
\begin{pgfscope}%
\pgfsys@transformshift{5.180089in}{0.926315in}%
\pgfsys@useobject{currentmarker}{}%
\end{pgfscope}%
\begin{pgfscope}%
\pgfsys@transformshift{5.197683in}{1.199006in}%
\pgfsys@useobject{currentmarker}{}%
\end{pgfscope}%
\begin{pgfscope}%
\pgfsys@transformshift{5.215276in}{0.954425in}%
\pgfsys@useobject{currentmarker}{}%
\end{pgfscope}%
\begin{pgfscope}%
\pgfsys@transformshift{5.232870in}{0.988260in}%
\pgfsys@useobject{currentmarker}{}%
\end{pgfscope}%
\begin{pgfscope}%
\pgfsys@transformshift{5.250464in}{1.090024in}%
\pgfsys@useobject{currentmarker}{}%
\end{pgfscope}%
\begin{pgfscope}%
\pgfsys@transformshift{5.268058in}{0.994049in}%
\pgfsys@useobject{currentmarker}{}%
\end{pgfscope}%
\begin{pgfscope}%
\pgfsys@transformshift{5.285652in}{1.212689in}%
\pgfsys@useobject{currentmarker}{}%
\end{pgfscope}%
\begin{pgfscope}%
\pgfsys@transformshift{5.303246in}{0.910673in}%
\pgfsys@useobject{currentmarker}{}%
\end{pgfscope}%
\begin{pgfscope}%
\pgfsys@transformshift{5.320839in}{1.058361in}%
\pgfsys@useobject{currentmarker}{}%
\end{pgfscope}%
\begin{pgfscope}%
\pgfsys@transformshift{5.338433in}{1.017342in}%
\pgfsys@useobject{currentmarker}{}%
\end{pgfscope}%
\begin{pgfscope}%
\pgfsys@transformshift{5.356027in}{0.894182in}%
\pgfsys@useobject{currentmarker}{}%
\end{pgfscope}%
\begin{pgfscope}%
\pgfsys@transformshift{5.373621in}{1.060452in}%
\pgfsys@useobject{currentmarker}{}%
\end{pgfscope}%
\begin{pgfscope}%
\pgfsys@transformshift{5.391215in}{0.985337in}%
\pgfsys@useobject{currentmarker}{}%
\end{pgfscope}%
\begin{pgfscope}%
\pgfsys@transformshift{5.408809in}{1.079294in}%
\pgfsys@useobject{currentmarker}{}%
\end{pgfscope}%
\begin{pgfscope}%
\pgfsys@transformshift{5.426403in}{1.084411in}%
\pgfsys@useobject{currentmarker}{}%
\end{pgfscope}%
\begin{pgfscope}%
\pgfsys@transformshift{5.443996in}{1.222992in}%
\pgfsys@useobject{currentmarker}{}%
\end{pgfscope}%
\begin{pgfscope}%
\pgfsys@transformshift{5.461590in}{1.142792in}%
\pgfsys@useobject{currentmarker}{}%
\end{pgfscope}%
\begin{pgfscope}%
\pgfsys@transformshift{5.479184in}{0.975095in}%
\pgfsys@useobject{currentmarker}{}%
\end{pgfscope}%
\begin{pgfscope}%
\pgfsys@transformshift{5.496778in}{0.996685in}%
\pgfsys@useobject{currentmarker}{}%
\end{pgfscope}%
\begin{pgfscope}%
\pgfsys@transformshift{5.514372in}{1.143655in}%
\pgfsys@useobject{currentmarker}{}%
\end{pgfscope}%
\begin{pgfscope}%
\pgfsys@transformshift{5.531966in}{1.110768in}%
\pgfsys@useobject{currentmarker}{}%
\end{pgfscope}%
\begin{pgfscope}%
\pgfsys@transformshift{5.549560in}{1.123579in}%
\pgfsys@useobject{currentmarker}{}%
\end{pgfscope}%
\begin{pgfscope}%
\pgfsys@transformshift{5.567153in}{0.909591in}%
\pgfsys@useobject{currentmarker}{}%
\end{pgfscope}%
\begin{pgfscope}%
\pgfsys@transformshift{5.584747in}{1.089876in}%
\pgfsys@useobject{currentmarker}{}%
\end{pgfscope}%
\begin{pgfscope}%
\pgfsys@transformshift{5.602341in}{1.036542in}%
\pgfsys@useobject{currentmarker}{}%
\end{pgfscope}%
\begin{pgfscope}%
\pgfsys@transformshift{5.619935in}{1.161209in}%
\pgfsys@useobject{currentmarker}{}%
\end{pgfscope}%
\begin{pgfscope}%
\pgfsys@transformshift{5.637529in}{1.144771in}%
\pgfsys@useobject{currentmarker}{}%
\end{pgfscope}%
\begin{pgfscope}%
\pgfsys@transformshift{5.655123in}{1.270786in}%
\pgfsys@useobject{currentmarker}{}%
\end{pgfscope}%
\begin{pgfscope}%
\pgfsys@transformshift{5.672717in}{1.234403in}%
\pgfsys@useobject{currentmarker}{}%
\end{pgfscope}%
\begin{pgfscope}%
\pgfsys@transformshift{5.690310in}{1.307020in}%
\pgfsys@useobject{currentmarker}{}%
\end{pgfscope}%
\begin{pgfscope}%
\pgfsys@transformshift{5.707904in}{1.204550in}%
\pgfsys@useobject{currentmarker}{}%
\end{pgfscope}%
\begin{pgfscope}%
\pgfsys@transformshift{5.725498in}{1.181373in}%
\pgfsys@useobject{currentmarker}{}%
\end{pgfscope}%
\begin{pgfscope}%
\pgfsys@transformshift{5.743092in}{1.261523in}%
\pgfsys@useobject{currentmarker}{}%
\end{pgfscope}%
\begin{pgfscope}%
\pgfsys@transformshift{5.760686in}{1.327137in}%
\pgfsys@useobject{currentmarker}{}%
\end{pgfscope}%
\begin{pgfscope}%
\pgfsys@transformshift{5.778280in}{1.392747in}%
\pgfsys@useobject{currentmarker}{}%
\end{pgfscope}%
\begin{pgfscope}%
\pgfsys@transformshift{5.795873in}{1.609158in}%
\pgfsys@useobject{currentmarker}{}%
\end{pgfscope}%
\begin{pgfscope}%
\pgfsys@transformshift{5.813467in}{1.398425in}%
\pgfsys@useobject{currentmarker}{}%
\end{pgfscope}%
\begin{pgfscope}%
\pgfsys@transformshift{5.831061in}{1.328312in}%
\pgfsys@useobject{currentmarker}{}%
\end{pgfscope}%
\begin{pgfscope}%
\pgfsys@transformshift{5.848655in}{1.412490in}%
\pgfsys@useobject{currentmarker}{}%
\end{pgfscope}%
\begin{pgfscope}%
\pgfsys@transformshift{5.866249in}{1.421228in}%
\pgfsys@useobject{currentmarker}{}%
\end{pgfscope}%
\begin{pgfscope}%
\pgfsys@transformshift{5.883843in}{1.536935in}%
\pgfsys@useobject{currentmarker}{}%
\end{pgfscope}%
\begin{pgfscope}%
\pgfsys@transformshift{5.901437in}{1.348517in}%
\pgfsys@useobject{currentmarker}{}%
\end{pgfscope}%
\begin{pgfscope}%
\pgfsys@transformshift{5.919030in}{1.528236in}%
\pgfsys@useobject{currentmarker}{}%
\end{pgfscope}%
\begin{pgfscope}%
\pgfsys@transformshift{5.936624in}{1.552137in}%
\pgfsys@useobject{currentmarker}{}%
\end{pgfscope}%
\begin{pgfscope}%
\pgfsys@transformshift{5.954218in}{1.572407in}%
\pgfsys@useobject{currentmarker}{}%
\end{pgfscope}%
\begin{pgfscope}%
\pgfsys@transformshift{5.971812in}{1.498469in}%
\pgfsys@useobject{currentmarker}{}%
\end{pgfscope}%
\begin{pgfscope}%
\pgfsys@transformshift{5.989406in}{1.543820in}%
\pgfsys@useobject{currentmarker}{}%
\end{pgfscope}%
\begin{pgfscope}%
\pgfsys@transformshift{6.007000in}{1.429595in}%
\pgfsys@useobject{currentmarker}{}%
\end{pgfscope}%
\begin{pgfscope}%
\pgfsys@transformshift{6.024594in}{1.529262in}%
\pgfsys@useobject{currentmarker}{}%
\end{pgfscope}%
\begin{pgfscope}%
\pgfsys@transformshift{6.042187in}{1.527009in}%
\pgfsys@useobject{currentmarker}{}%
\end{pgfscope}%
\begin{pgfscope}%
\pgfsys@transformshift{6.059781in}{1.625672in}%
\pgfsys@useobject{currentmarker}{}%
\end{pgfscope}%
\begin{pgfscope}%
\pgfsys@transformshift{6.077375in}{1.464366in}%
\pgfsys@useobject{currentmarker}{}%
\end{pgfscope}%
\begin{pgfscope}%
\pgfsys@transformshift{6.094969in}{1.659166in}%
\pgfsys@useobject{currentmarker}{}%
\end{pgfscope}%
\begin{pgfscope}%
\pgfsys@transformshift{6.112563in}{1.727455in}%
\pgfsys@useobject{currentmarker}{}%
\end{pgfscope}%
\begin{pgfscope}%
\pgfsys@transformshift{6.130157in}{1.357955in}%
\pgfsys@useobject{currentmarker}{}%
\end{pgfscope}%
\begin{pgfscope}%
\pgfsys@transformshift{6.147751in}{1.605050in}%
\pgfsys@useobject{currentmarker}{}%
\end{pgfscope}%
\begin{pgfscope}%
\pgfsys@transformshift{6.165344in}{1.621861in}%
\pgfsys@useobject{currentmarker}{}%
\end{pgfscope}%
\begin{pgfscope}%
\pgfsys@transformshift{6.182938in}{1.477952in}%
\pgfsys@useobject{currentmarker}{}%
\end{pgfscope}%
\begin{pgfscope}%
\pgfsys@transformshift{6.200532in}{1.491603in}%
\pgfsys@useobject{currentmarker}{}%
\end{pgfscope}%
\begin{pgfscope}%
\pgfsys@transformshift{6.218126in}{1.506950in}%
\pgfsys@useobject{currentmarker}{}%
\end{pgfscope}%
\begin{pgfscope}%
\pgfsys@transformshift{6.235720in}{1.477936in}%
\pgfsys@useobject{currentmarker}{}%
\end{pgfscope}%
\begin{pgfscope}%
\pgfsys@transformshift{6.253314in}{1.464209in}%
\pgfsys@useobject{currentmarker}{}%
\end{pgfscope}%
\begin{pgfscope}%
\pgfsys@transformshift{6.270907in}{1.312049in}%
\pgfsys@useobject{currentmarker}{}%
\end{pgfscope}%
\begin{pgfscope}%
\pgfsys@transformshift{6.288501in}{1.587727in}%
\pgfsys@useobject{currentmarker}{}%
\end{pgfscope}%
\begin{pgfscope}%
\pgfsys@transformshift{6.306095in}{1.567807in}%
\pgfsys@useobject{currentmarker}{}%
\end{pgfscope}%
\begin{pgfscope}%
\pgfsys@transformshift{6.323689in}{1.362703in}%
\pgfsys@useobject{currentmarker}{}%
\end{pgfscope}%
\begin{pgfscope}%
\pgfsys@transformshift{6.341283in}{1.284679in}%
\pgfsys@useobject{currentmarker}{}%
\end{pgfscope}%
\begin{pgfscope}%
\pgfsys@transformshift{6.358877in}{1.476755in}%
\pgfsys@useobject{currentmarker}{}%
\end{pgfscope}%
\begin{pgfscope}%
\pgfsys@transformshift{6.376471in}{1.355333in}%
\pgfsys@useobject{currentmarker}{}%
\end{pgfscope}%
\end{pgfscope}%
\begin{pgfscope}%
\pgfpathrectangle{\pgfqpoint{2.000000in}{0.750000in}}{\pgfqpoint{4.376471in}{0.978947in}} %
\pgfusepath{clip}%
\pgfsetroundcap%
\pgfsetroundjoin%
\pgfsetlinewidth{1.756562pt}%
\definecolor{currentstroke}{rgb}{0.298039,0.447059,0.690196}%
\pgfsetstrokecolor{currentstroke}%
\pgfsetdash{}{0pt}%
\pgfpathmoveto{\pgfqpoint{2.925406in}{1.738947in}}%
\pgfpathlineto{\pgfqpoint{2.928076in}{1.698945in}}%
\pgfpathlineto{\pgfqpoint{2.945670in}{1.522794in}}%
\pgfpathlineto{\pgfqpoint{2.963263in}{1.411159in}}%
\pgfpathlineto{\pgfqpoint{2.980857in}{1.345555in}}%
\pgfpathlineto{\pgfqpoint{2.998451in}{1.311325in}}%
\pgfpathlineto{\pgfqpoint{3.016045in}{1.297080in}}%
\pgfpathlineto{\pgfqpoint{3.033639in}{1.294202in}}%
\pgfpathlineto{\pgfqpoint{3.086420in}{1.299926in}}%
\pgfpathlineto{\pgfqpoint{3.104014in}{1.296816in}}%
\pgfpathlineto{\pgfqpoint{3.121608in}{1.289253in}}%
\pgfpathlineto{\pgfqpoint{3.139202in}{1.277291in}}%
\pgfpathlineto{\pgfqpoint{3.156796in}{1.261497in}}%
\pgfpathlineto{\pgfqpoint{3.174390in}{1.242776in}}%
\pgfpathlineto{\pgfqpoint{3.244765in}{1.161402in}}%
\pgfpathlineto{\pgfqpoint{3.262359in}{1.145009in}}%
\pgfpathlineto{\pgfqpoint{3.279953in}{1.132032in}}%
\pgfpathlineto{\pgfqpoint{3.297547in}{1.123093in}}%
\pgfpathlineto{\pgfqpoint{3.315140in}{1.118619in}}%
\pgfpathlineto{\pgfqpoint{3.332734in}{1.118844in}}%
\pgfpathlineto{\pgfqpoint{3.350328in}{1.123817in}}%
\pgfpathlineto{\pgfqpoint{3.367922in}{1.133415in}}%
\pgfpathlineto{\pgfqpoint{3.385516in}{1.147365in}}%
\pgfpathlineto{\pgfqpoint{3.403110in}{1.165264in}}%
\pgfpathlineto{\pgfqpoint{3.420704in}{1.186602in}}%
\pgfpathlineto{\pgfqpoint{3.438297in}{1.210788in}}%
\pgfpathlineto{\pgfqpoint{3.473485in}{1.265080in}}%
\pgfpathlineto{\pgfqpoint{3.543860in}{1.378329in}}%
\pgfpathlineto{\pgfqpoint{3.561454in}{1.403910in}}%
\pgfpathlineto{\pgfqpoint{3.579048in}{1.427326in}}%
\pgfpathlineto{\pgfqpoint{3.596642in}{1.448160in}}%
\pgfpathlineto{\pgfqpoint{3.614236in}{1.466071in}}%
\pgfpathlineto{\pgfqpoint{3.631830in}{1.480799in}}%
\pgfpathlineto{\pgfqpoint{3.649424in}{1.492161in}}%
\pgfpathlineto{\pgfqpoint{3.667017in}{1.500055in}}%
\pgfpathlineto{\pgfqpoint{3.684611in}{1.504449in}}%
\pgfpathlineto{\pgfqpoint{3.702205in}{1.505381in}}%
\pgfpathlineto{\pgfqpoint{3.719799in}{1.502950in}}%
\pgfpathlineto{\pgfqpoint{3.737393in}{1.497308in}}%
\pgfpathlineto{\pgfqpoint{3.754987in}{1.488656in}}%
\pgfpathlineto{\pgfqpoint{3.772581in}{1.477228in}}%
\pgfpathlineto{\pgfqpoint{3.790174in}{1.463292in}}%
\pgfpathlineto{\pgfqpoint{3.807768in}{1.447132in}}%
\pgfpathlineto{\pgfqpoint{3.825362in}{1.429047in}}%
\pgfpathlineto{\pgfqpoint{3.860550in}{1.388318in}}%
\pgfpathlineto{\pgfqpoint{3.895738in}{1.343475in}}%
\pgfpathlineto{\pgfqpoint{4.018894in}{1.182333in}}%
\pgfpathlineto{\pgfqpoint{4.054082in}{1.140668in}}%
\pgfpathlineto{\pgfqpoint{4.089270in}{1.102324in}}%
\pgfpathlineto{\pgfqpoint{4.124458in}{1.067591in}}%
\pgfpathlineto{\pgfqpoint{4.159645in}{1.036605in}}%
\pgfpathlineto{\pgfqpoint{4.194833in}{1.009450in}}%
\pgfpathlineto{\pgfqpoint{4.230021in}{0.986226in}}%
\pgfpathlineto{\pgfqpoint{4.265208in}{0.967117in}}%
\pgfpathlineto{\pgfqpoint{4.300396in}{0.952409in}}%
\pgfpathlineto{\pgfqpoint{4.317990in}{0.946825in}}%
\pgfpathlineto{\pgfqpoint{4.335584in}{0.942495in}}%
\pgfpathlineto{\pgfqpoint{4.353178in}{0.939478in}}%
\pgfpathlineto{\pgfqpoint{4.370772in}{0.937837in}}%
\pgfpathlineto{\pgfqpoint{4.388365in}{0.937630in}}%
\pgfpathlineto{\pgfqpoint{4.405959in}{0.938914in}}%
\pgfpathlineto{\pgfqpoint{4.423553in}{0.941738in}}%
\pgfpathlineto{\pgfqpoint{4.441147in}{0.946144in}}%
\pgfpathlineto{\pgfqpoint{4.458741in}{0.952161in}}%
\pgfpathlineto{\pgfqpoint{4.476335in}{0.959806in}}%
\pgfpathlineto{\pgfqpoint{4.493928in}{0.969079in}}%
\pgfpathlineto{\pgfqpoint{4.511522in}{0.979960in}}%
\pgfpathlineto{\pgfqpoint{4.529116in}{0.992412in}}%
\pgfpathlineto{\pgfqpoint{4.546710in}{1.006376in}}%
\pgfpathlineto{\pgfqpoint{4.581898in}{1.038488in}}%
\pgfpathlineto{\pgfqpoint{4.617085in}{1.075371in}}%
\pgfpathlineto{\pgfqpoint{4.652273in}{1.115750in}}%
\pgfpathlineto{\pgfqpoint{4.757836in}{1.240935in}}%
\pgfpathlineto{\pgfqpoint{4.793024in}{1.277582in}}%
\pgfpathlineto{\pgfqpoint{4.810618in}{1.293859in}}%
\pgfpathlineto{\pgfqpoint{4.828212in}{1.308476in}}%
\pgfpathlineto{\pgfqpoint{4.845805in}{1.321233in}}%
\pgfpathlineto{\pgfqpoint{4.863399in}{1.331953in}}%
\pgfpathlineto{\pgfqpoint{4.880993in}{1.340482in}}%
\pgfpathlineto{\pgfqpoint{4.898587in}{1.346695in}}%
\pgfpathlineto{\pgfqpoint{4.916181in}{1.350496in}}%
\pgfpathlineto{\pgfqpoint{4.933775in}{1.351823in}}%
\pgfpathlineto{\pgfqpoint{4.951369in}{1.350647in}}%
\pgfpathlineto{\pgfqpoint{4.968962in}{1.346973in}}%
\pgfpathlineto{\pgfqpoint{4.986556in}{1.340843in}}%
\pgfpathlineto{\pgfqpoint{5.004150in}{1.332333in}}%
\pgfpathlineto{\pgfqpoint{5.021744in}{1.321552in}}%
\pgfpathlineto{\pgfqpoint{5.039338in}{1.308642in}}%
\pgfpathlineto{\pgfqpoint{5.056932in}{1.293778in}}%
\pgfpathlineto{\pgfqpoint{5.092119in}{1.259014in}}%
\pgfpathlineto{\pgfqpoint{5.127307in}{1.219153in}}%
\pgfpathlineto{\pgfqpoint{5.250464in}{1.071997in}}%
\pgfpathlineto{\pgfqpoint{5.285652in}{1.037174in}}%
\pgfpathlineto{\pgfqpoint{5.303246in}{1.022232in}}%
\pgfpathlineto{\pgfqpoint{5.320839in}{1.009202in}}%
\pgfpathlineto{\pgfqpoint{5.338433in}{0.998248in}}%
\pgfpathlineto{\pgfqpoint{5.356027in}{0.989506in}}%
\pgfpathlineto{\pgfqpoint{5.373621in}{0.983084in}}%
\pgfpathlineto{\pgfqpoint{5.391215in}{0.979057in}}%
\pgfpathlineto{\pgfqpoint{5.408809in}{0.977470in}}%
\pgfpathlineto{\pgfqpoint{5.426403in}{0.978340in}}%
\pgfpathlineto{\pgfqpoint{5.443996in}{0.981652in}}%
\pgfpathlineto{\pgfqpoint{5.461590in}{0.987366in}}%
\pgfpathlineto{\pgfqpoint{5.479184in}{0.995414in}}%
\pgfpathlineto{\pgfqpoint{5.496778in}{1.005707in}}%
\pgfpathlineto{\pgfqpoint{5.514372in}{1.018132in}}%
\pgfpathlineto{\pgfqpoint{5.531966in}{1.032560in}}%
\pgfpathlineto{\pgfqpoint{5.549560in}{1.048845in}}%
\pgfpathlineto{\pgfqpoint{5.584747in}{1.086341in}}%
\pgfpathlineto{\pgfqpoint{5.619935in}{1.129250in}}%
\pgfpathlineto{\pgfqpoint{5.655123in}{1.176127in}}%
\pgfpathlineto{\pgfqpoint{5.707904in}{1.250726in}}%
\pgfpathlineto{\pgfqpoint{5.778280in}{1.351050in}}%
\pgfpathlineto{\pgfqpoint{5.813467in}{1.398775in}}%
\pgfpathlineto{\pgfqpoint{5.848655in}{1.443354in}}%
\pgfpathlineto{\pgfqpoint{5.883843in}{1.483722in}}%
\pgfpathlineto{\pgfqpoint{5.919030in}{1.518855in}}%
\pgfpathlineto{\pgfqpoint{5.936624in}{1.534155in}}%
\pgfpathlineto{\pgfqpoint{5.954218in}{1.547791in}}%
\pgfpathlineto{\pgfqpoint{5.971812in}{1.559659in}}%
\pgfpathlineto{\pgfqpoint{5.989406in}{1.569664in}}%
\pgfpathlineto{\pgfqpoint{6.007000in}{1.577723in}}%
\pgfpathlineto{\pgfqpoint{6.024594in}{1.583771in}}%
\pgfpathlineto{\pgfqpoint{6.042187in}{1.587762in}}%
\pgfpathlineto{\pgfqpoint{6.059781in}{1.589671in}}%
\pgfpathlineto{\pgfqpoint{6.077375in}{1.589499in}}%
\pgfpathlineto{\pgfqpoint{6.094969in}{1.587272in}}%
\pgfpathlineto{\pgfqpoint{6.112563in}{1.583046in}}%
\pgfpathlineto{\pgfqpoint{6.130157in}{1.576899in}}%
\pgfpathlineto{\pgfqpoint{6.147751in}{1.568930in}}%
\pgfpathlineto{\pgfqpoint{6.165344in}{1.559253in}}%
\pgfpathlineto{\pgfqpoint{6.182938in}{1.547978in}}%
\pgfpathlineto{\pgfqpoint{6.200532in}{1.535203in}}%
\pgfpathlineto{\pgfqpoint{6.218126in}{1.520980in}}%
\pgfpathlineto{\pgfqpoint{6.235720in}{1.505290in}}%
\pgfpathlineto{\pgfqpoint{6.253314in}{1.487997in}}%
\pgfpathlineto{\pgfqpoint{6.270907in}{1.468800in}}%
\pgfpathlineto{\pgfqpoint{6.288501in}{1.447167in}}%
\pgfpathlineto{\pgfqpoint{6.306095in}{1.422255in}}%
\pgfpathlineto{\pgfqpoint{6.323689in}{1.392821in}}%
\pgfpathlineto{\pgfqpoint{6.341283in}{1.357107in}}%
\pgfpathlineto{\pgfqpoint{6.358877in}{1.312716in}}%
\pgfpathlineto{\pgfqpoint{6.376471in}{1.256451in}}%
\pgfpathlineto{\pgfqpoint{6.376471in}{1.256451in}}%
\pgfusepath{stroke}%
\end{pgfscope}%
\begin{pgfscope}%
\pgfpathrectangle{\pgfqpoint{2.000000in}{0.750000in}}{\pgfqpoint{4.376471in}{0.978947in}} %
\pgfusepath{clip}%
\pgfsetbuttcap%
\pgfsetroundjoin%
\pgfsetlinewidth{1.756562pt}%
\definecolor{currentstroke}{rgb}{1.000000,0.647059,0.000000}%
\pgfsetstrokecolor{currentstroke}%
\pgfsetdash{{6.000000pt}{6.000000pt}}{0.000000pt}%
\pgfpathmoveto{\pgfqpoint{2.925450in}{1.738947in}}%
\pgfpathlineto{\pgfqpoint{2.928076in}{1.699495in}}%
\pgfpathlineto{\pgfqpoint{2.945670in}{1.522784in}}%
\pgfpathlineto{\pgfqpoint{2.963263in}{1.410765in}}%
\pgfpathlineto{\pgfqpoint{2.980857in}{1.344940in}}%
\pgfpathlineto{\pgfqpoint{2.998451in}{1.310559in}}%
\pgfpathlineto{\pgfqpoint{3.016045in}{1.296319in}}%
\pgfpathlineto{\pgfqpoint{3.033639in}{1.293505in}}%
\pgfpathlineto{\pgfqpoint{3.086420in}{1.299505in}}%
\pgfpathlineto{\pgfqpoint{3.104014in}{1.296518in}}%
\pgfpathlineto{\pgfqpoint{3.121608in}{1.289198in}}%
\pgfpathlineto{\pgfqpoint{3.139202in}{1.277223in}}%
\pgfpathlineto{\pgfqpoint{3.156796in}{1.261502in}}%
\pgfpathlineto{\pgfqpoint{3.174390in}{1.242867in}}%
\pgfpathlineto{\pgfqpoint{3.244765in}{1.161395in}}%
\pgfpathlineto{\pgfqpoint{3.262359in}{1.144877in}}%
\pgfpathlineto{\pgfqpoint{3.279953in}{1.131856in}}%
\pgfpathlineto{\pgfqpoint{3.297547in}{1.122993in}}%
\pgfpathlineto{\pgfqpoint{3.315140in}{1.118487in}}%
\pgfpathlineto{\pgfqpoint{3.332734in}{1.118723in}}%
\pgfpathlineto{\pgfqpoint{3.350328in}{1.123740in}}%
\pgfpathlineto{\pgfqpoint{3.367922in}{1.133337in}}%
\pgfpathlineto{\pgfqpoint{3.385516in}{1.147453in}}%
\pgfpathlineto{\pgfqpoint{3.403110in}{1.165478in}}%
\pgfpathlineto{\pgfqpoint{3.420704in}{1.187001in}}%
\pgfpathlineto{\pgfqpoint{3.438297in}{1.211349in}}%
\pgfpathlineto{\pgfqpoint{3.473485in}{1.266020in}}%
\pgfpathlineto{\pgfqpoint{3.543860in}{1.380442in}}%
\pgfpathlineto{\pgfqpoint{3.561454in}{1.406259in}}%
\pgfpathlineto{\pgfqpoint{3.579048in}{1.429947in}}%
\pgfpathlineto{\pgfqpoint{3.596642in}{1.451047in}}%
\pgfpathlineto{\pgfqpoint{3.614236in}{1.469109in}}%
\pgfpathlineto{\pgfqpoint{3.631830in}{1.484084in}}%
\pgfpathlineto{\pgfqpoint{3.649424in}{1.495635in}}%
\pgfpathlineto{\pgfqpoint{3.667017in}{1.503590in}}%
\pgfpathlineto{\pgfqpoint{3.684611in}{1.508158in}}%
\pgfpathlineto{\pgfqpoint{3.702205in}{1.509204in}}%
\pgfpathlineto{\pgfqpoint{3.719799in}{1.506807in}}%
\pgfpathlineto{\pgfqpoint{3.737393in}{1.501100in}}%
\pgfpathlineto{\pgfqpoint{3.754987in}{1.492430in}}%
\pgfpathlineto{\pgfqpoint{3.772581in}{1.480996in}}%
\pgfpathlineto{\pgfqpoint{3.790174in}{1.466905in}}%
\pgfpathlineto{\pgfqpoint{3.807768in}{1.450605in}}%
\pgfpathlineto{\pgfqpoint{3.842956in}{1.412564in}}%
\pgfpathlineto{\pgfqpoint{3.878144in}{1.369040in}}%
\pgfpathlineto{\pgfqpoint{3.930925in}{1.298821in}}%
\pgfpathlineto{\pgfqpoint{4.001301in}{1.205514in}}%
\pgfpathlineto{\pgfqpoint{4.036488in}{1.161971in}}%
\pgfpathlineto{\pgfqpoint{4.071676in}{1.121630in}}%
\pgfpathlineto{\pgfqpoint{4.106864in}{1.084828in}}%
\pgfpathlineto{\pgfqpoint{4.142051in}{1.051791in}}%
\pgfpathlineto{\pgfqpoint{4.177239in}{1.022569in}}%
\pgfpathlineto{\pgfqpoint{4.212427in}{0.997303in}}%
\pgfpathlineto{\pgfqpoint{4.247615in}{0.976095in}}%
\pgfpathlineto{\pgfqpoint{4.265208in}{0.967014in}}%
\pgfpathlineto{\pgfqpoint{4.300396in}{0.952353in}}%
\pgfpathlineto{\pgfqpoint{4.317990in}{0.946808in}}%
\pgfpathlineto{\pgfqpoint{4.335584in}{0.942482in}}%
\pgfpathlineto{\pgfqpoint{4.353178in}{0.939467in}}%
\pgfpathlineto{\pgfqpoint{4.370772in}{0.937886in}}%
\pgfpathlineto{\pgfqpoint{4.388365in}{0.937680in}}%
\pgfpathlineto{\pgfqpoint{4.405959in}{0.938956in}}%
\pgfpathlineto{\pgfqpoint{4.423553in}{0.941732in}}%
\pgfpathlineto{\pgfqpoint{4.441147in}{0.946189in}}%
\pgfpathlineto{\pgfqpoint{4.458741in}{0.952188in}}%
\pgfpathlineto{\pgfqpoint{4.476335in}{0.959794in}}%
\pgfpathlineto{\pgfqpoint{4.493928in}{0.969080in}}%
\pgfpathlineto{\pgfqpoint{4.511522in}{0.979960in}}%
\pgfpathlineto{\pgfqpoint{4.529116in}{0.992377in}}%
\pgfpathlineto{\pgfqpoint{4.564304in}{1.021698in}}%
\pgfpathlineto{\pgfqpoint{4.599492in}{1.056365in}}%
\pgfpathlineto{\pgfqpoint{4.634679in}{1.095072in}}%
\pgfpathlineto{\pgfqpoint{4.687461in}{1.157959in}}%
\pgfpathlineto{\pgfqpoint{4.740242in}{1.220971in}}%
\pgfpathlineto{\pgfqpoint{4.775430in}{1.259858in}}%
\pgfpathlineto{\pgfqpoint{4.810618in}{1.293929in}}%
\pgfpathlineto{\pgfqpoint{4.828212in}{1.308505in}}%
\pgfpathlineto{\pgfqpoint{4.845805in}{1.321264in}}%
\pgfpathlineto{\pgfqpoint{4.863399in}{1.331982in}}%
\pgfpathlineto{\pgfqpoint{4.880993in}{1.340608in}}%
\pgfpathlineto{\pgfqpoint{4.898587in}{1.346783in}}%
\pgfpathlineto{\pgfqpoint{4.916181in}{1.350604in}}%
\pgfpathlineto{\pgfqpoint{4.933775in}{1.351936in}}%
\pgfpathlineto{\pgfqpoint{4.951369in}{1.350766in}}%
\pgfpathlineto{\pgfqpoint{4.968962in}{1.347156in}}%
\pgfpathlineto{\pgfqpoint{4.986556in}{1.340907in}}%
\pgfpathlineto{\pgfqpoint{5.004150in}{1.332455in}}%
\pgfpathlineto{\pgfqpoint{5.021744in}{1.321575in}}%
\pgfpathlineto{\pgfqpoint{5.039338in}{1.308742in}}%
\pgfpathlineto{\pgfqpoint{5.056932in}{1.293779in}}%
\pgfpathlineto{\pgfqpoint{5.074526in}{1.277211in}}%
\pgfpathlineto{\pgfqpoint{5.109713in}{1.239568in}}%
\pgfpathlineto{\pgfqpoint{5.144901in}{1.197967in}}%
\pgfpathlineto{\pgfqpoint{5.232870in}{1.091238in}}%
\pgfpathlineto{\pgfqpoint{5.268058in}{1.053733in}}%
\pgfpathlineto{\pgfqpoint{5.285652in}{1.037090in}}%
\pgfpathlineto{\pgfqpoint{5.303246in}{1.022115in}}%
\pgfpathlineto{\pgfqpoint{5.320839in}{1.009107in}}%
\pgfpathlineto{\pgfqpoint{5.338433in}{0.998227in}}%
\pgfpathlineto{\pgfqpoint{5.356027in}{0.989439in}}%
\pgfpathlineto{\pgfqpoint{5.373621in}{0.983115in}}%
\pgfpathlineto{\pgfqpoint{5.391215in}{0.979045in}}%
\pgfpathlineto{\pgfqpoint{5.408809in}{0.977477in}}%
\pgfpathlineto{\pgfqpoint{5.426403in}{0.978385in}}%
\pgfpathlineto{\pgfqpoint{5.443996in}{0.981721in}}%
\pgfpathlineto{\pgfqpoint{5.461590in}{0.987398in}}%
\pgfpathlineto{\pgfqpoint{5.479184in}{0.995563in}}%
\pgfpathlineto{\pgfqpoint{5.496778in}{1.005771in}}%
\pgfpathlineto{\pgfqpoint{5.514372in}{1.018231in}}%
\pgfpathlineto{\pgfqpoint{5.531966in}{1.032621in}}%
\pgfpathlineto{\pgfqpoint{5.549560in}{1.048915in}}%
\pgfpathlineto{\pgfqpoint{5.584747in}{1.086396in}}%
\pgfpathlineto{\pgfqpoint{5.619935in}{1.129317in}}%
\pgfpathlineto{\pgfqpoint{5.655123in}{1.176109in}}%
\pgfpathlineto{\pgfqpoint{5.707904in}{1.250734in}}%
\pgfpathlineto{\pgfqpoint{5.795873in}{1.375127in}}%
\pgfpathlineto{\pgfqpoint{5.831061in}{1.421445in}}%
\pgfpathlineto{\pgfqpoint{5.866249in}{1.464055in}}%
\pgfpathlineto{\pgfqpoint{5.901437in}{1.502009in}}%
\pgfpathlineto{\pgfqpoint{5.919030in}{1.518888in}}%
\pgfpathlineto{\pgfqpoint{5.936624in}{1.534211in}}%
\pgfpathlineto{\pgfqpoint{5.954218in}{1.547904in}}%
\pgfpathlineto{\pgfqpoint{5.971812in}{1.559655in}}%
\pgfpathlineto{\pgfqpoint{5.989406in}{1.569726in}}%
\pgfpathlineto{\pgfqpoint{6.007000in}{1.577792in}}%
\pgfpathlineto{\pgfqpoint{6.024594in}{1.583742in}}%
\pgfpathlineto{\pgfqpoint{6.042187in}{1.587850in}}%
\pgfpathlineto{\pgfqpoint{6.059781in}{1.589729in}}%
\pgfpathlineto{\pgfqpoint{6.077375in}{1.589505in}}%
\pgfpathlineto{\pgfqpoint{6.094969in}{1.587215in}}%
\pgfpathlineto{\pgfqpoint{6.112563in}{1.583045in}}%
\pgfpathlineto{\pgfqpoint{6.130157in}{1.576821in}}%
\pgfpathlineto{\pgfqpoint{6.147751in}{1.568792in}}%
\pgfpathlineto{\pgfqpoint{6.165344in}{1.559157in}}%
\pgfpathlineto{\pgfqpoint{6.182938in}{1.547904in}}%
\pgfpathlineto{\pgfqpoint{6.200532in}{1.535158in}}%
\pgfpathlineto{\pgfqpoint{6.218126in}{1.520992in}}%
\pgfpathlineto{\pgfqpoint{6.235720in}{1.505320in}}%
\pgfpathlineto{\pgfqpoint{6.253314in}{1.488129in}}%
\pgfpathlineto{\pgfqpoint{6.270907in}{1.468909in}}%
\pgfpathlineto{\pgfqpoint{6.288501in}{1.447225in}}%
\pgfpathlineto{\pgfqpoint{6.306095in}{1.422155in}}%
\pgfpathlineto{\pgfqpoint{6.323689in}{1.392274in}}%
\pgfpathlineto{\pgfqpoint{6.341283in}{1.355782in}}%
\pgfpathlineto{\pgfqpoint{6.358877in}{1.310235in}}%
\pgfpathlineto{\pgfqpoint{6.376471in}{1.252066in}}%
\pgfpathlineto{\pgfqpoint{6.376471in}{1.252066in}}%
\pgfusepath{stroke}%
\end{pgfscope}%
\begin{pgfscope}%
\pgfsetrectcap%
\pgfsetmiterjoin%
\pgfsetlinewidth{1.003750pt}%
\definecolor{currentstroke}{rgb}{0.800000,0.800000,0.800000}%
\pgfsetstrokecolor{currentstroke}%
\pgfsetdash{}{0pt}%
\pgfpathmoveto{\pgfqpoint{2.000000in}{0.750000in}}%
\pgfpathlineto{\pgfqpoint{2.000000in}{1.728947in}}%
\pgfusepath{stroke}%
\end{pgfscope}%
\begin{pgfscope}%
\pgfsetrectcap%
\pgfsetmiterjoin%
\pgfsetlinewidth{1.003750pt}%
\definecolor{currentstroke}{rgb}{0.800000,0.800000,0.800000}%
\pgfsetstrokecolor{currentstroke}%
\pgfsetdash{}{0pt}%
\pgfpathmoveto{\pgfqpoint{6.376471in}{0.750000in}}%
\pgfpathlineto{\pgfqpoint{6.376471in}{1.728947in}}%
\pgfusepath{stroke}%
\end{pgfscope}%
\begin{pgfscope}%
\pgfsetrectcap%
\pgfsetmiterjoin%
\pgfsetlinewidth{1.003750pt}%
\definecolor{currentstroke}{rgb}{0.800000,0.800000,0.800000}%
\pgfsetstrokecolor{currentstroke}%
\pgfsetdash{}{0pt}%
\pgfpathmoveto{\pgfqpoint{2.000000in}{1.728947in}}%
\pgfpathlineto{\pgfqpoint{6.376471in}{1.728947in}}%
\pgfusepath{stroke}%
\end{pgfscope}%
\begin{pgfscope}%
\pgfsetrectcap%
\pgfsetmiterjoin%
\pgfsetlinewidth{1.003750pt}%
\definecolor{currentstroke}{rgb}{0.800000,0.800000,0.800000}%
\pgfsetstrokecolor{currentstroke}%
\pgfsetdash{}{0pt}%
\pgfpathmoveto{\pgfqpoint{2.000000in}{0.750000in}}%
\pgfpathlineto{\pgfqpoint{6.376471in}{0.750000in}}%
\pgfusepath{stroke}%
\end{pgfscope}%
\begin{pgfscope}%
\pgfsetroundcap%
\pgfsetroundjoin%
\pgfsetlinewidth{1.756562pt}%
\definecolor{currentstroke}{rgb}{0.298039,0.447059,0.690196}%
\pgfsetstrokecolor{currentstroke}%
\pgfsetdash{}{0pt}%
\pgfpathmoveto{\pgfqpoint{2.125000in}{1.549004in}}%
\pgfpathlineto{\pgfqpoint{2.402778in}{1.549004in}}%
\pgfusepath{stroke}%
\end{pgfscope}%
\begin{pgfscope}%
\definecolor{textcolor}{rgb}{0.150000,0.150000,0.150000}%
\pgfsetstrokecolor{textcolor}%
\pgfsetfillcolor{textcolor}%
\pgftext[x=2.513889in,y=1.500393in,left,base]{\color{textcolor}\sffamily\fontsize{10.000000}{12.000000}\selectfont \(\displaystyle \widetilde{\Phi}^* \theta\)}%
\end{pgfscope}%
\begin{pgfscope}%
\pgfsetbuttcap%
\pgfsetroundjoin%
\pgfsetlinewidth{1.756562pt}%
\definecolor{currentstroke}{rgb}{1.000000,0.647059,0.000000}%
\pgfsetstrokecolor{currentstroke}%
\pgfsetdash{{6.000000pt}{6.000000pt}}{0.000000pt}%
\pgfpathmoveto{\pgfqpoint{2.125000in}{1.344143in}}%
\pgfpathlineto{\pgfqpoint{2.402778in}{1.344143in}}%
\pgfusepath{stroke}%
\end{pgfscope}%
\begin{pgfscope}%
\definecolor{textcolor}{rgb}{0.150000,0.150000,0.150000}%
\pgfsetstrokecolor{textcolor}%
\pgfsetfillcolor{textcolor}%
\pgftext[x=2.513889in,y=1.295532in,left,base]{\color{textcolor}\sffamily\fontsize{10.000000}{12.000000}\selectfont \(\displaystyle \widetilde{K}u\)}%
\end{pgfscope}%
\begin{pgfscope}%
\pgfsetbuttcap%
\pgfsetroundjoin%
\definecolor{currentfill}{rgb}{1.000000,0.000000,0.000000}%
\pgfsetfillcolor{currentfill}%
\pgfsetlinewidth{2.007500pt}%
\definecolor{currentstroke}{rgb}{1.000000,0.000000,0.000000}%
\pgfsetstrokecolor{currentstroke}%
\pgfsetdash{}{0pt}%
\pgfpathmoveto{\pgfqpoint{2.232832in}{1.135525in}}%
\pgfpathlineto{\pgfqpoint{2.294945in}{1.135525in}}%
\pgfpathmoveto{\pgfqpoint{2.263889in}{1.104468in}}%
\pgfpathlineto{\pgfqpoint{2.263889in}{1.166581in}}%
\pgfusepath{stroke,fill}%
\end{pgfscope}%
\begin{pgfscope}%
\pgfsetbuttcap%
\pgfsetroundjoin%
\definecolor{currentfill}{rgb}{1.000000,0.000000,0.000000}%
\pgfsetfillcolor{currentfill}%
\pgfsetlinewidth{2.007500pt}%
\definecolor{currentstroke}{rgb}{1.000000,0.000000,0.000000}%
\pgfsetstrokecolor{currentstroke}%
\pgfsetdash{}{0pt}%
\pgfpathmoveto{\pgfqpoint{2.232832in}{1.135525in}}%
\pgfpathlineto{\pgfqpoint{2.294945in}{1.135525in}}%
\pgfpathmoveto{\pgfqpoint{2.263889in}{1.104468in}}%
\pgfpathlineto{\pgfqpoint{2.263889in}{1.166581in}}%
\pgfusepath{stroke,fill}%
\end{pgfscope}%
\begin{pgfscope}%
\pgfsetbuttcap%
\pgfsetroundjoin%
\definecolor{currentfill}{rgb}{1.000000,0.000000,0.000000}%
\pgfsetfillcolor{currentfill}%
\pgfsetlinewidth{2.007500pt}%
\definecolor{currentstroke}{rgb}{1.000000,0.000000,0.000000}%
\pgfsetstrokecolor{currentstroke}%
\pgfsetdash{}{0pt}%
\pgfpathmoveto{\pgfqpoint{2.232832in}{1.135525in}}%
\pgfpathlineto{\pgfqpoint{2.294945in}{1.135525in}}%
\pgfpathmoveto{\pgfqpoint{2.263889in}{1.104468in}}%
\pgfpathlineto{\pgfqpoint{2.263889in}{1.166581in}}%
\pgfusepath{stroke,fill}%
\end{pgfscope}%
\begin{pgfscope}%
\definecolor{textcolor}{rgb}{0.150000,0.150000,0.150000}%
\pgfsetstrokecolor{textcolor}%
\pgfsetfillcolor{textcolor}%
\pgftext[x=2.513889in,y=1.099066in,left,base]{\color{textcolor}\sffamily\fontsize{10.000000}{12.000000}\selectfont train}%
\end{pgfscope}%
\begin{pgfscope}%
\pgfsetbuttcap%
\pgfsetroundjoin%
\definecolor{currentfill}{rgb}{0.000000,0.000000,0.000000}%
\pgfsetfillcolor{currentfill}%
\pgfsetlinewidth{0.301125pt}%
\definecolor{currentstroke}{rgb}{0.000000,0.000000,0.000000}%
\pgfsetstrokecolor{currentstroke}%
\pgfsetdash{}{0pt}%
\pgfpathmoveto{\pgfqpoint{2.263889in}{0.923531in}}%
\pgfpathcurveto{\pgfqpoint{2.268007in}{0.923531in}}{\pgfqpoint{2.271957in}{0.925167in}}{\pgfqpoint{2.274869in}{0.928079in}}%
\pgfpathcurveto{\pgfqpoint{2.277781in}{0.930991in}}{\pgfqpoint{2.279417in}{0.934941in}}{\pgfqpoint{2.279417in}{0.939060in}}%
\pgfpathcurveto{\pgfqpoint{2.279417in}{0.943178in}}{\pgfqpoint{2.277781in}{0.947128in}}{\pgfqpoint{2.274869in}{0.950040in}}%
\pgfpathcurveto{\pgfqpoint{2.271957in}{0.952952in}}{\pgfqpoint{2.268007in}{0.954588in}}{\pgfqpoint{2.263889in}{0.954588in}}%
\pgfpathcurveto{\pgfqpoint{2.259771in}{0.954588in}}{\pgfqpoint{2.255821in}{0.952952in}}{\pgfqpoint{2.252909in}{0.950040in}}%
\pgfpathcurveto{\pgfqpoint{2.249997in}{0.947128in}}{\pgfqpoint{2.248361in}{0.943178in}}{\pgfqpoint{2.248361in}{0.939060in}}%
\pgfpathcurveto{\pgfqpoint{2.248361in}{0.934941in}}{\pgfqpoint{2.249997in}{0.930991in}}{\pgfqpoint{2.252909in}{0.928079in}}%
\pgfpathcurveto{\pgfqpoint{2.255821in}{0.925167in}}{\pgfqpoint{2.259771in}{0.923531in}}{\pgfqpoint{2.263889in}{0.923531in}}%
\pgfpathclose%
\pgfusepath{stroke,fill}%
\end{pgfscope}%
\begin{pgfscope}%
\pgfsetbuttcap%
\pgfsetroundjoin%
\definecolor{currentfill}{rgb}{0.000000,0.000000,0.000000}%
\pgfsetfillcolor{currentfill}%
\pgfsetlinewidth{0.301125pt}%
\definecolor{currentstroke}{rgb}{0.000000,0.000000,0.000000}%
\pgfsetstrokecolor{currentstroke}%
\pgfsetdash{}{0pt}%
\pgfpathmoveto{\pgfqpoint{2.263889in}{0.923531in}}%
\pgfpathcurveto{\pgfqpoint{2.268007in}{0.923531in}}{\pgfqpoint{2.271957in}{0.925167in}}{\pgfqpoint{2.274869in}{0.928079in}}%
\pgfpathcurveto{\pgfqpoint{2.277781in}{0.930991in}}{\pgfqpoint{2.279417in}{0.934941in}}{\pgfqpoint{2.279417in}{0.939060in}}%
\pgfpathcurveto{\pgfqpoint{2.279417in}{0.943178in}}{\pgfqpoint{2.277781in}{0.947128in}}{\pgfqpoint{2.274869in}{0.950040in}}%
\pgfpathcurveto{\pgfqpoint{2.271957in}{0.952952in}}{\pgfqpoint{2.268007in}{0.954588in}}{\pgfqpoint{2.263889in}{0.954588in}}%
\pgfpathcurveto{\pgfqpoint{2.259771in}{0.954588in}}{\pgfqpoint{2.255821in}{0.952952in}}{\pgfqpoint{2.252909in}{0.950040in}}%
\pgfpathcurveto{\pgfqpoint{2.249997in}{0.947128in}}{\pgfqpoint{2.248361in}{0.943178in}}{\pgfqpoint{2.248361in}{0.939060in}}%
\pgfpathcurveto{\pgfqpoint{2.248361in}{0.934941in}}{\pgfqpoint{2.249997in}{0.930991in}}{\pgfqpoint{2.252909in}{0.928079in}}%
\pgfpathcurveto{\pgfqpoint{2.255821in}{0.925167in}}{\pgfqpoint{2.259771in}{0.923531in}}{\pgfqpoint{2.263889in}{0.923531in}}%
\pgfpathclose%
\pgfusepath{stroke,fill}%
\end{pgfscope}%
\begin{pgfscope}%
\pgfsetbuttcap%
\pgfsetroundjoin%
\definecolor{currentfill}{rgb}{0.000000,0.000000,0.000000}%
\pgfsetfillcolor{currentfill}%
\pgfsetlinewidth{0.301125pt}%
\definecolor{currentstroke}{rgb}{0.000000,0.000000,0.000000}%
\pgfsetstrokecolor{currentstroke}%
\pgfsetdash{}{0pt}%
\pgfpathmoveto{\pgfqpoint{2.263889in}{0.923531in}}%
\pgfpathcurveto{\pgfqpoint{2.268007in}{0.923531in}}{\pgfqpoint{2.271957in}{0.925167in}}{\pgfqpoint{2.274869in}{0.928079in}}%
\pgfpathcurveto{\pgfqpoint{2.277781in}{0.930991in}}{\pgfqpoint{2.279417in}{0.934941in}}{\pgfqpoint{2.279417in}{0.939060in}}%
\pgfpathcurveto{\pgfqpoint{2.279417in}{0.943178in}}{\pgfqpoint{2.277781in}{0.947128in}}{\pgfqpoint{2.274869in}{0.950040in}}%
\pgfpathcurveto{\pgfqpoint{2.271957in}{0.952952in}}{\pgfqpoint{2.268007in}{0.954588in}}{\pgfqpoint{2.263889in}{0.954588in}}%
\pgfpathcurveto{\pgfqpoint{2.259771in}{0.954588in}}{\pgfqpoint{2.255821in}{0.952952in}}{\pgfqpoint{2.252909in}{0.950040in}}%
\pgfpathcurveto{\pgfqpoint{2.249997in}{0.947128in}}{\pgfqpoint{2.248361in}{0.943178in}}{\pgfqpoint{2.248361in}{0.939060in}}%
\pgfpathcurveto{\pgfqpoint{2.248361in}{0.934941in}}{\pgfqpoint{2.249997in}{0.930991in}}{\pgfqpoint{2.252909in}{0.928079in}}%
\pgfpathcurveto{\pgfqpoint{2.255821in}{0.925167in}}{\pgfqpoint{2.259771in}{0.923531in}}{\pgfqpoint{2.263889in}{0.923531in}}%
\pgfpathclose%
\pgfusepath{stroke,fill}%
\end{pgfscope}%
\begin{pgfscope}%
\definecolor{textcolor}{rgb}{0.150000,0.150000,0.150000}%
\pgfsetstrokecolor{textcolor}%
\pgfsetfillcolor{textcolor}%
\pgftext[x=2.513889in,y=0.902601in,left,base]{\color{textcolor}\sffamily\fontsize{10.000000}{12.000000}\selectfont test}%
\end{pgfscope}%
\begin{pgfscope}%
\pgfsetbuttcap%
\pgfsetmiterjoin%
\definecolor{currentfill}{rgb}{1.000000,1.000000,1.000000}%
\pgfsetfillcolor{currentfill}%
\pgfsetlinewidth{0.000000pt}%
\definecolor{currentstroke}{rgb}{0.000000,0.000000,0.000000}%
\pgfsetstrokecolor{currentstroke}%
\pgfsetstrokeopacity{0.000000}%
\pgfsetdash{}{0pt}%
\pgfpathmoveto{\pgfqpoint{7.105882in}{0.750000in}}%
\pgfpathlineto{\pgfqpoint{11.482353in}{0.750000in}}%
\pgfpathlineto{\pgfqpoint{11.482353in}{1.728947in}}%
\pgfpathlineto{\pgfqpoint{7.105882in}{1.728947in}}%
\pgfpathclose%
\pgfusepath{fill}%
\end{pgfscope}%
\begin{pgfscope}%
\pgfpathrectangle{\pgfqpoint{7.105882in}{0.750000in}}{\pgfqpoint{4.376471in}{0.978947in}} %
\pgfusepath{clip}%
\pgfsetroundcap%
\pgfsetroundjoin%
\pgfsetlinewidth{1.003750pt}%
\definecolor{currentstroke}{rgb}{0.800000,0.800000,0.800000}%
\pgfsetstrokecolor{currentstroke}%
\pgfsetdash{}{0pt}%
\pgfpathmoveto{\pgfqpoint{7.105882in}{0.750000in}}%
\pgfpathlineto{\pgfqpoint{7.105882in}{1.728947in}}%
\pgfusepath{stroke}%
\end{pgfscope}%
\begin{pgfscope}%
\definecolor{textcolor}{rgb}{0.150000,0.150000,0.150000}%
\pgfsetstrokecolor{textcolor}%
\pgfsetfillcolor{textcolor}%
\pgftext[x=7.105882in,y=0.652778in,,top]{\color{textcolor}\sffamily\fontsize{10.000000}{12.000000}\selectfont \(\displaystyle -1.5\)}%
\end{pgfscope}%
\begin{pgfscope}%
\pgfpathrectangle{\pgfqpoint{7.105882in}{0.750000in}}{\pgfqpoint{4.376471in}{0.978947in}} %
\pgfusepath{clip}%
\pgfsetroundcap%
\pgfsetroundjoin%
\pgfsetlinewidth{1.003750pt}%
\definecolor{currentstroke}{rgb}{0.800000,0.800000,0.800000}%
\pgfsetstrokecolor{currentstroke}%
\pgfsetdash{}{0pt}%
\pgfpathmoveto{\pgfqpoint{7.981176in}{0.750000in}}%
\pgfpathlineto{\pgfqpoint{7.981176in}{1.728947in}}%
\pgfusepath{stroke}%
\end{pgfscope}%
\begin{pgfscope}%
\definecolor{textcolor}{rgb}{0.150000,0.150000,0.150000}%
\pgfsetstrokecolor{textcolor}%
\pgfsetfillcolor{textcolor}%
\pgftext[x=7.981176in,y=0.652778in,,top]{\color{textcolor}\sffamily\fontsize{10.000000}{12.000000}\selectfont \(\displaystyle -1.0\)}%
\end{pgfscope}%
\begin{pgfscope}%
\pgfpathrectangle{\pgfqpoint{7.105882in}{0.750000in}}{\pgfqpoint{4.376471in}{0.978947in}} %
\pgfusepath{clip}%
\pgfsetroundcap%
\pgfsetroundjoin%
\pgfsetlinewidth{1.003750pt}%
\definecolor{currentstroke}{rgb}{0.800000,0.800000,0.800000}%
\pgfsetstrokecolor{currentstroke}%
\pgfsetdash{}{0pt}%
\pgfpathmoveto{\pgfqpoint{8.856471in}{0.750000in}}%
\pgfpathlineto{\pgfqpoint{8.856471in}{1.728947in}}%
\pgfusepath{stroke}%
\end{pgfscope}%
\begin{pgfscope}%
\definecolor{textcolor}{rgb}{0.150000,0.150000,0.150000}%
\pgfsetstrokecolor{textcolor}%
\pgfsetfillcolor{textcolor}%
\pgftext[x=8.856471in,y=0.652778in,,top]{\color{textcolor}\sffamily\fontsize{10.000000}{12.000000}\selectfont \(\displaystyle -0.5\)}%
\end{pgfscope}%
\begin{pgfscope}%
\pgfpathrectangle{\pgfqpoint{7.105882in}{0.750000in}}{\pgfqpoint{4.376471in}{0.978947in}} %
\pgfusepath{clip}%
\pgfsetroundcap%
\pgfsetroundjoin%
\pgfsetlinewidth{1.003750pt}%
\definecolor{currentstroke}{rgb}{0.800000,0.800000,0.800000}%
\pgfsetstrokecolor{currentstroke}%
\pgfsetdash{}{0pt}%
\pgfpathmoveto{\pgfqpoint{9.731765in}{0.750000in}}%
\pgfpathlineto{\pgfqpoint{9.731765in}{1.728947in}}%
\pgfusepath{stroke}%
\end{pgfscope}%
\begin{pgfscope}%
\definecolor{textcolor}{rgb}{0.150000,0.150000,0.150000}%
\pgfsetstrokecolor{textcolor}%
\pgfsetfillcolor{textcolor}%
\pgftext[x=9.731765in,y=0.652778in,,top]{\color{textcolor}\sffamily\fontsize{10.000000}{12.000000}\selectfont \(\displaystyle 0.0\)}%
\end{pgfscope}%
\begin{pgfscope}%
\pgfpathrectangle{\pgfqpoint{7.105882in}{0.750000in}}{\pgfqpoint{4.376471in}{0.978947in}} %
\pgfusepath{clip}%
\pgfsetroundcap%
\pgfsetroundjoin%
\pgfsetlinewidth{1.003750pt}%
\definecolor{currentstroke}{rgb}{0.800000,0.800000,0.800000}%
\pgfsetstrokecolor{currentstroke}%
\pgfsetdash{}{0pt}%
\pgfpathmoveto{\pgfqpoint{10.607059in}{0.750000in}}%
\pgfpathlineto{\pgfqpoint{10.607059in}{1.728947in}}%
\pgfusepath{stroke}%
\end{pgfscope}%
\begin{pgfscope}%
\definecolor{textcolor}{rgb}{0.150000,0.150000,0.150000}%
\pgfsetstrokecolor{textcolor}%
\pgfsetfillcolor{textcolor}%
\pgftext[x=10.607059in,y=0.652778in,,top]{\color{textcolor}\sffamily\fontsize{10.000000}{12.000000}\selectfont \(\displaystyle 0.5\)}%
\end{pgfscope}%
\begin{pgfscope}%
\pgfpathrectangle{\pgfqpoint{7.105882in}{0.750000in}}{\pgfqpoint{4.376471in}{0.978947in}} %
\pgfusepath{clip}%
\pgfsetroundcap%
\pgfsetroundjoin%
\pgfsetlinewidth{1.003750pt}%
\definecolor{currentstroke}{rgb}{0.800000,0.800000,0.800000}%
\pgfsetstrokecolor{currentstroke}%
\pgfsetdash{}{0pt}%
\pgfpathmoveto{\pgfqpoint{11.482353in}{0.750000in}}%
\pgfpathlineto{\pgfqpoint{11.482353in}{1.728947in}}%
\pgfusepath{stroke}%
\end{pgfscope}%
\begin{pgfscope}%
\definecolor{textcolor}{rgb}{0.150000,0.150000,0.150000}%
\pgfsetstrokecolor{textcolor}%
\pgfsetfillcolor{textcolor}%
\pgftext[x=11.482353in,y=0.652778in,,top]{\color{textcolor}\sffamily\fontsize{10.000000}{12.000000}\selectfont \(\displaystyle 1.0\)}%
\end{pgfscope}%
\begin{pgfscope}%
\definecolor{textcolor}{rgb}{0.150000,0.150000,0.150000}%
\pgfsetstrokecolor{textcolor}%
\pgfsetfillcolor{textcolor}%
\pgftext[x=9.294118in,y=0.456313in,,top]{\color{textcolor}\sffamily\fontsize{11.000000}{13.200000}\selectfont x}%
\end{pgfscope}%
\begin{pgfscope}%
\pgfpathrectangle{\pgfqpoint{7.105882in}{0.750000in}}{\pgfqpoint{4.376471in}{0.978947in}} %
\pgfusepath{clip}%
\pgfsetroundcap%
\pgfsetroundjoin%
\pgfsetlinewidth{1.003750pt}%
\definecolor{currentstroke}{rgb}{0.800000,0.800000,0.800000}%
\pgfsetstrokecolor{currentstroke}%
\pgfsetdash{}{0pt}%
\pgfpathmoveto{\pgfqpoint{7.105882in}{0.913158in}}%
\pgfpathlineto{\pgfqpoint{11.482353in}{0.913158in}}%
\pgfusepath{stroke}%
\end{pgfscope}%
\begin{pgfscope}%
\definecolor{textcolor}{rgb}{0.150000,0.150000,0.150000}%
\pgfsetstrokecolor{textcolor}%
\pgfsetfillcolor{textcolor}%
\pgftext[x=7.008660in,y=0.913158in,right,]{\color{textcolor}\sffamily\fontsize{10.000000}{12.000000}\selectfont \(\displaystyle -1\)}%
\end{pgfscope}%
\begin{pgfscope}%
\pgfpathrectangle{\pgfqpoint{7.105882in}{0.750000in}}{\pgfqpoint{4.376471in}{0.978947in}} %
\pgfusepath{clip}%
\pgfsetroundcap%
\pgfsetroundjoin%
\pgfsetlinewidth{1.003750pt}%
\definecolor{currentstroke}{rgb}{0.800000,0.800000,0.800000}%
\pgfsetstrokecolor{currentstroke}%
\pgfsetdash{}{0pt}%
\pgfpathmoveto{\pgfqpoint{7.105882in}{1.117105in}}%
\pgfpathlineto{\pgfqpoint{11.482353in}{1.117105in}}%
\pgfusepath{stroke}%
\end{pgfscope}%
\begin{pgfscope}%
\definecolor{textcolor}{rgb}{0.150000,0.150000,0.150000}%
\pgfsetstrokecolor{textcolor}%
\pgfsetfillcolor{textcolor}%
\pgftext[x=7.008660in,y=1.117105in,right,]{\color{textcolor}\sffamily\fontsize{10.000000}{12.000000}\selectfont \(\displaystyle 0\)}%
\end{pgfscope}%
\begin{pgfscope}%
\pgfpathrectangle{\pgfqpoint{7.105882in}{0.750000in}}{\pgfqpoint{4.376471in}{0.978947in}} %
\pgfusepath{clip}%
\pgfsetroundcap%
\pgfsetroundjoin%
\pgfsetlinewidth{1.003750pt}%
\definecolor{currentstroke}{rgb}{0.800000,0.800000,0.800000}%
\pgfsetstrokecolor{currentstroke}%
\pgfsetdash{}{0pt}%
\pgfpathmoveto{\pgfqpoint{7.105882in}{1.321053in}}%
\pgfpathlineto{\pgfqpoint{11.482353in}{1.321053in}}%
\pgfusepath{stroke}%
\end{pgfscope}%
\begin{pgfscope}%
\definecolor{textcolor}{rgb}{0.150000,0.150000,0.150000}%
\pgfsetstrokecolor{textcolor}%
\pgfsetfillcolor{textcolor}%
\pgftext[x=7.008660in,y=1.321053in,right,]{\color{textcolor}\sffamily\fontsize{10.000000}{12.000000}\selectfont \(\displaystyle 1\)}%
\end{pgfscope}%
\begin{pgfscope}%
\pgfpathrectangle{\pgfqpoint{7.105882in}{0.750000in}}{\pgfqpoint{4.376471in}{0.978947in}} %
\pgfusepath{clip}%
\pgfsetroundcap%
\pgfsetroundjoin%
\pgfsetlinewidth{1.003750pt}%
\definecolor{currentstroke}{rgb}{0.800000,0.800000,0.800000}%
\pgfsetstrokecolor{currentstroke}%
\pgfsetdash{}{0pt}%
\pgfpathmoveto{\pgfqpoint{7.105882in}{1.525000in}}%
\pgfpathlineto{\pgfqpoint{11.482353in}{1.525000in}}%
\pgfusepath{stroke}%
\end{pgfscope}%
\begin{pgfscope}%
\definecolor{textcolor}{rgb}{0.150000,0.150000,0.150000}%
\pgfsetstrokecolor{textcolor}%
\pgfsetfillcolor{textcolor}%
\pgftext[x=7.008660in,y=1.525000in,right,]{\color{textcolor}\sffamily\fontsize{10.000000}{12.000000}\selectfont \(\displaystyle 2\)}%
\end{pgfscope}%
\begin{pgfscope}%
\pgfpathrectangle{\pgfqpoint{7.105882in}{0.750000in}}{\pgfqpoint{4.376471in}{0.978947in}} %
\pgfusepath{clip}%
\pgfsetroundcap%
\pgfsetroundjoin%
\pgfsetlinewidth{1.003750pt}%
\definecolor{currentstroke}{rgb}{0.800000,0.800000,0.800000}%
\pgfsetstrokecolor{currentstroke}%
\pgfsetdash{}{0pt}%
\pgfpathmoveto{\pgfqpoint{7.105882in}{1.728947in}}%
\pgfpathlineto{\pgfqpoint{11.482353in}{1.728947in}}%
\pgfusepath{stroke}%
\end{pgfscope}%
\begin{pgfscope}%
\definecolor{textcolor}{rgb}{0.150000,0.150000,0.150000}%
\pgfsetstrokecolor{textcolor}%
\pgfsetfillcolor{textcolor}%
\pgftext[x=7.008660in,y=1.728947in,right,]{\color{textcolor}\sffamily\fontsize{10.000000}{12.000000}\selectfont \(\displaystyle 3\)}%
\end{pgfscope}%
\begin{pgfscope}%
\pgfpathrectangle{\pgfqpoint{7.105882in}{0.750000in}}{\pgfqpoint{4.376471in}{0.978947in}} %
\pgfusepath{clip}%
\pgfsetbuttcap%
\pgfsetroundjoin%
\definecolor{currentfill}{rgb}{1.000000,0.000000,0.000000}%
\pgfsetfillcolor{currentfill}%
\pgfsetlinewidth{2.007500pt}%
\definecolor{currentstroke}{rgb}{1.000000,0.000000,0.000000}%
\pgfsetstrokecolor{currentstroke}%
\pgfsetdash{}{0pt}%
\pgfpathmoveto{\pgfqpoint{9.871613in}{1.307576in}}%
\pgfpathlineto{\pgfqpoint{9.933726in}{1.307576in}}%
\pgfpathmoveto{\pgfqpoint{9.902669in}{1.276519in}}%
\pgfpathlineto{\pgfqpoint{9.902669in}{1.338632in}}%
\pgfusepath{stroke,fill}%
\end{pgfscope}%
\begin{pgfscope}%
\pgfpathrectangle{\pgfqpoint{7.105882in}{0.750000in}}{\pgfqpoint{4.376471in}{0.978947in}} %
\pgfusepath{clip}%
\pgfsetbuttcap%
\pgfsetroundjoin%
\definecolor{currentfill}{rgb}{1.000000,0.000000,0.000000}%
\pgfsetfillcolor{currentfill}%
\pgfsetlinewidth{2.007500pt}%
\definecolor{currentstroke}{rgb}{1.000000,0.000000,0.000000}%
\pgfsetstrokecolor{currentstroke}%
\pgfsetdash{}{0pt}%
\pgfpathmoveto{\pgfqpoint{10.454124in}{0.990364in}}%
\pgfpathlineto{\pgfqpoint{10.516237in}{0.990364in}}%
\pgfpathmoveto{\pgfqpoint{10.485181in}{0.959307in}}%
\pgfpathlineto{\pgfqpoint{10.485181in}{1.021420in}}%
\pgfusepath{stroke,fill}%
\end{pgfscope}%
\begin{pgfscope}%
\pgfpathrectangle{\pgfqpoint{7.105882in}{0.750000in}}{\pgfqpoint{4.376471in}{0.978947in}} %
\pgfusepath{clip}%
\pgfsetbuttcap%
\pgfsetroundjoin%
\definecolor{currentfill}{rgb}{1.000000,0.000000,0.000000}%
\pgfsetfillcolor{currentfill}%
\pgfsetlinewidth{2.007500pt}%
\definecolor{currentstroke}{rgb}{1.000000,0.000000,0.000000}%
\pgfsetstrokecolor{currentstroke}%
\pgfsetdash{}{0pt}%
\pgfpathmoveto{\pgfqpoint{10.060501in}{1.361104in}}%
\pgfpathlineto{\pgfqpoint{10.122614in}{1.361104in}}%
\pgfpathmoveto{\pgfqpoint{10.091557in}{1.330047in}}%
\pgfpathlineto{\pgfqpoint{10.091557in}{1.392160in}}%
\pgfusepath{stroke,fill}%
\end{pgfscope}%
\begin{pgfscope}%
\pgfpathrectangle{\pgfqpoint{7.105882in}{0.750000in}}{\pgfqpoint{4.376471in}{0.978947in}} %
\pgfusepath{clip}%
\pgfsetbuttcap%
\pgfsetroundjoin%
\definecolor{currentfill}{rgb}{1.000000,0.000000,0.000000}%
\pgfsetfillcolor{currentfill}%
\pgfsetlinewidth{2.007500pt}%
\definecolor{currentstroke}{rgb}{1.000000,0.000000,0.000000}%
\pgfsetstrokecolor{currentstroke}%
\pgfsetdash{}{0pt}%
\pgfpathmoveto{\pgfqpoint{9.857852in}{1.250192in}}%
\pgfpathlineto{\pgfqpoint{9.919965in}{1.250192in}}%
\pgfpathmoveto{\pgfqpoint{9.888909in}{1.219135in}}%
\pgfpathlineto{\pgfqpoint{9.888909in}{1.281248in}}%
\pgfusepath{stroke,fill}%
\end{pgfscope}%
\begin{pgfscope}%
\pgfpathrectangle{\pgfqpoint{7.105882in}{0.750000in}}{\pgfqpoint{4.376471in}{0.978947in}} %
\pgfusepath{clip}%
\pgfsetbuttcap%
\pgfsetroundjoin%
\definecolor{currentfill}{rgb}{1.000000,0.000000,0.000000}%
\pgfsetfillcolor{currentfill}%
\pgfsetlinewidth{2.007500pt}%
\definecolor{currentstroke}{rgb}{1.000000,0.000000,0.000000}%
\pgfsetstrokecolor{currentstroke}%
\pgfsetdash{}{0pt}%
\pgfpathmoveto{\pgfqpoint{9.433410in}{0.923797in}}%
\pgfpathlineto{\pgfqpoint{9.495523in}{0.923797in}}%
\pgfpathmoveto{\pgfqpoint{9.464467in}{0.892741in}}%
\pgfpathlineto{\pgfqpoint{9.464467in}{0.954854in}}%
\pgfusepath{stroke,fill}%
\end{pgfscope}%
\begin{pgfscope}%
\pgfpathrectangle{\pgfqpoint{7.105882in}{0.750000in}}{\pgfqpoint{4.376471in}{0.978947in}} %
\pgfusepath{clip}%
\pgfsetbuttcap%
\pgfsetroundjoin%
\definecolor{currentfill}{rgb}{1.000000,0.000000,0.000000}%
\pgfsetfillcolor{currentfill}%
\pgfsetlinewidth{2.007500pt}%
\definecolor{currentstroke}{rgb}{1.000000,0.000000,0.000000}%
\pgfsetstrokecolor{currentstroke}%
\pgfsetdash{}{0pt}%
\pgfpathmoveto{\pgfqpoint{10.211509in}{1.193262in}}%
\pgfpathlineto{\pgfqpoint{10.273622in}{1.193262in}}%
\pgfpathmoveto{\pgfqpoint{10.242566in}{1.162205in}}%
\pgfpathlineto{\pgfqpoint{10.242566in}{1.224318in}}%
\pgfusepath{stroke,fill}%
\end{pgfscope}%
\begin{pgfscope}%
\pgfpathrectangle{\pgfqpoint{7.105882in}{0.750000in}}{\pgfqpoint{4.376471in}{0.978947in}} %
\pgfusepath{clip}%
\pgfsetbuttcap%
\pgfsetroundjoin%
\definecolor{currentfill}{rgb}{1.000000,0.000000,0.000000}%
\pgfsetfillcolor{currentfill}%
\pgfsetlinewidth{2.007500pt}%
\definecolor{currentstroke}{rgb}{1.000000,0.000000,0.000000}%
\pgfsetstrokecolor{currentstroke}%
\pgfsetdash{}{0pt}%
\pgfpathmoveto{\pgfqpoint{9.482190in}{0.961291in}}%
\pgfpathlineto{\pgfqpoint{9.544303in}{0.961291in}}%
\pgfpathmoveto{\pgfqpoint{9.513247in}{0.930235in}}%
\pgfpathlineto{\pgfqpoint{9.513247in}{0.992348in}}%
\pgfusepath{stroke,fill}%
\end{pgfscope}%
\begin{pgfscope}%
\pgfpathrectangle{\pgfqpoint{7.105882in}{0.750000in}}{\pgfqpoint{4.376471in}{0.978947in}} %
\pgfusepath{clip}%
\pgfsetbuttcap%
\pgfsetroundjoin%
\definecolor{currentfill}{rgb}{1.000000,0.000000,0.000000}%
\pgfsetfillcolor{currentfill}%
\pgfsetlinewidth{2.007500pt}%
\definecolor{currentstroke}{rgb}{1.000000,0.000000,0.000000}%
\pgfsetstrokecolor{currentstroke}%
\pgfsetdash{}{0pt}%
\pgfpathmoveto{\pgfqpoint{11.072375in}{1.601409in}}%
\pgfpathlineto{\pgfqpoint{11.134488in}{1.601409in}}%
\pgfpathmoveto{\pgfqpoint{11.103431in}{1.570353in}}%
\pgfpathlineto{\pgfqpoint{11.103431in}{1.632466in}}%
\pgfusepath{stroke,fill}%
\end{pgfscope}%
\begin{pgfscope}%
\pgfpathrectangle{\pgfqpoint{7.105882in}{0.750000in}}{\pgfqpoint{4.376471in}{0.978947in}} %
\pgfusepath{clip}%
\pgfsetbuttcap%
\pgfsetroundjoin%
\definecolor{currentfill}{rgb}{1.000000,0.000000,0.000000}%
\pgfsetfillcolor{currentfill}%
\pgfsetlinewidth{2.007500pt}%
\definecolor{currentstroke}{rgb}{1.000000,0.000000,0.000000}%
\pgfsetstrokecolor{currentstroke}%
\pgfsetdash{}{0pt}%
\pgfpathmoveto{\pgfqpoint{11.324073in}{1.501812in}}%
\pgfpathlineto{\pgfqpoint{11.386186in}{1.501812in}}%
\pgfpathmoveto{\pgfqpoint{11.355130in}{1.470755in}}%
\pgfpathlineto{\pgfqpoint{11.355130in}{1.532868in}}%
\pgfusepath{stroke,fill}%
\end{pgfscope}%
\begin{pgfscope}%
\pgfpathrectangle{\pgfqpoint{7.105882in}{0.750000in}}{\pgfqpoint{4.376471in}{0.978947in}} %
\pgfusepath{clip}%
\pgfsetbuttcap%
\pgfsetroundjoin%
\definecolor{currentfill}{rgb}{1.000000,0.000000,0.000000}%
\pgfsetfillcolor{currentfill}%
\pgfsetlinewidth{2.007500pt}%
\definecolor{currentstroke}{rgb}{1.000000,0.000000,0.000000}%
\pgfsetstrokecolor{currentstroke}%
\pgfsetdash{}{0pt}%
\pgfpathmoveto{\pgfqpoint{9.292616in}{0.999208in}}%
\pgfpathlineto{\pgfqpoint{9.354729in}{0.999208in}}%
\pgfpathmoveto{\pgfqpoint{9.323673in}{0.968152in}}%
\pgfpathlineto{\pgfqpoint{9.323673in}{1.030265in}}%
\pgfusepath{stroke,fill}%
\end{pgfscope}%
\begin{pgfscope}%
\pgfpathrectangle{\pgfqpoint{7.105882in}{0.750000in}}{\pgfqpoint{4.376471in}{0.978947in}} %
\pgfusepath{clip}%
\pgfsetbuttcap%
\pgfsetroundjoin%
\definecolor{currentfill}{rgb}{1.000000,0.000000,0.000000}%
\pgfsetfillcolor{currentfill}%
\pgfsetlinewidth{2.007500pt}%
\definecolor{currentstroke}{rgb}{1.000000,0.000000,0.000000}%
\pgfsetstrokecolor{currentstroke}%
\pgfsetdash{}{0pt}%
\pgfpathmoveto{\pgfqpoint{10.722089in}{1.149384in}}%
\pgfpathlineto{\pgfqpoint{10.784202in}{1.149384in}}%
\pgfpathmoveto{\pgfqpoint{10.753146in}{1.118327in}}%
\pgfpathlineto{\pgfqpoint{10.753146in}{1.180440in}}%
\pgfusepath{stroke,fill}%
\end{pgfscope}%
\begin{pgfscope}%
\pgfpathrectangle{\pgfqpoint{7.105882in}{0.750000in}}{\pgfqpoint{4.376471in}{0.978947in}} %
\pgfusepath{clip}%
\pgfsetbuttcap%
\pgfsetroundjoin%
\definecolor{currentfill}{rgb}{1.000000,0.000000,0.000000}%
\pgfsetfillcolor{currentfill}%
\pgfsetlinewidth{2.007500pt}%
\definecolor{currentstroke}{rgb}{1.000000,0.000000,0.000000}%
\pgfsetstrokecolor{currentstroke}%
\pgfsetdash{}{0pt}%
\pgfpathmoveto{\pgfqpoint{9.801874in}{1.189479in}}%
\pgfpathlineto{\pgfqpoint{9.863987in}{1.189479in}}%
\pgfpathmoveto{\pgfqpoint{9.832931in}{1.158422in}}%
\pgfpathlineto{\pgfqpoint{9.832931in}{1.220535in}}%
\pgfusepath{stroke,fill}%
\end{pgfscope}%
\begin{pgfscope}%
\pgfpathrectangle{\pgfqpoint{7.105882in}{0.750000in}}{\pgfqpoint{4.376471in}{0.978947in}} %
\pgfusepath{clip}%
\pgfsetbuttcap%
\pgfsetroundjoin%
\definecolor{currentfill}{rgb}{1.000000,0.000000,0.000000}%
\pgfsetfillcolor{currentfill}%
\pgfsetlinewidth{2.007500pt}%
\definecolor{currentstroke}{rgb}{1.000000,0.000000,0.000000}%
\pgfsetstrokecolor{currentstroke}%
\pgfsetdash{}{0pt}%
\pgfpathmoveto{\pgfqpoint{9.938944in}{1.317035in}}%
\pgfpathlineto{\pgfqpoint{10.001057in}{1.317035in}}%
\pgfpathmoveto{\pgfqpoint{9.970001in}{1.285978in}}%
\pgfpathlineto{\pgfqpoint{9.970001in}{1.348091in}}%
\pgfusepath{stroke,fill}%
\end{pgfscope}%
\begin{pgfscope}%
\pgfpathrectangle{\pgfqpoint{7.105882in}{0.750000in}}{\pgfqpoint{4.376471in}{0.978947in}} %
\pgfusepath{clip}%
\pgfsetbuttcap%
\pgfsetroundjoin%
\definecolor{currentfill}{rgb}{1.000000,0.000000,0.000000}%
\pgfsetfillcolor{currentfill}%
\pgfsetlinewidth{2.007500pt}%
\definecolor{currentstroke}{rgb}{1.000000,0.000000,0.000000}%
\pgfsetstrokecolor{currentstroke}%
\pgfsetdash{}{0pt}%
\pgfpathmoveto{\pgfqpoint{11.190797in}{1.577199in}}%
\pgfpathlineto{\pgfqpoint{11.252910in}{1.577199in}}%
\pgfpathmoveto{\pgfqpoint{11.221854in}{1.546142in}}%
\pgfpathlineto{\pgfqpoint{11.221854in}{1.608255in}}%
\pgfusepath{stroke,fill}%
\end{pgfscope}%
\begin{pgfscope}%
\pgfpathrectangle{\pgfqpoint{7.105882in}{0.750000in}}{\pgfqpoint{4.376471in}{0.978947in}} %
\pgfusepath{clip}%
\pgfsetbuttcap%
\pgfsetroundjoin%
\definecolor{currentfill}{rgb}{1.000000,0.000000,0.000000}%
\pgfsetfillcolor{currentfill}%
\pgfsetlinewidth{2.007500pt}%
\definecolor{currentstroke}{rgb}{1.000000,0.000000,0.000000}%
\pgfsetstrokecolor{currentstroke}%
\pgfsetdash{}{0pt}%
\pgfpathmoveto{\pgfqpoint{8.198830in}{1.289809in}}%
\pgfpathlineto{\pgfqpoint{8.260943in}{1.289809in}}%
\pgfpathmoveto{\pgfqpoint{8.229886in}{1.258753in}}%
\pgfpathlineto{\pgfqpoint{8.229886in}{1.320866in}}%
\pgfusepath{stroke,fill}%
\end{pgfscope}%
\begin{pgfscope}%
\pgfpathrectangle{\pgfqpoint{7.105882in}{0.750000in}}{\pgfqpoint{4.376471in}{0.978947in}} %
\pgfusepath{clip}%
\pgfsetbuttcap%
\pgfsetroundjoin%
\definecolor{currentfill}{rgb}{1.000000,0.000000,0.000000}%
\pgfsetfillcolor{currentfill}%
\pgfsetlinewidth{2.007500pt}%
\definecolor{currentstroke}{rgb}{1.000000,0.000000,0.000000}%
\pgfsetstrokecolor{currentstroke}%
\pgfsetdash{}{0pt}%
\pgfpathmoveto{\pgfqpoint{8.255175in}{1.232176in}}%
\pgfpathlineto{\pgfqpoint{8.317288in}{1.232176in}}%
\pgfpathmoveto{\pgfqpoint{8.286232in}{1.201119in}}%
\pgfpathlineto{\pgfqpoint{8.286232in}{1.263232in}}%
\pgfusepath{stroke,fill}%
\end{pgfscope}%
\begin{pgfscope}%
\pgfpathrectangle{\pgfqpoint{7.105882in}{0.750000in}}{\pgfqpoint{4.376471in}{0.978947in}} %
\pgfusepath{clip}%
\pgfsetbuttcap%
\pgfsetroundjoin%
\definecolor{currentfill}{rgb}{1.000000,0.000000,0.000000}%
\pgfsetfillcolor{currentfill}%
\pgfsetlinewidth{2.007500pt}%
\definecolor{currentstroke}{rgb}{1.000000,0.000000,0.000000}%
\pgfsetstrokecolor{currentstroke}%
\pgfsetdash{}{0pt}%
\pgfpathmoveto{\pgfqpoint{8.020908in}{1.519456in}}%
\pgfpathlineto{\pgfqpoint{8.083021in}{1.519456in}}%
\pgfpathmoveto{\pgfqpoint{8.051965in}{1.488399in}}%
\pgfpathlineto{\pgfqpoint{8.051965in}{1.550512in}}%
\pgfusepath{stroke,fill}%
\end{pgfscope}%
\begin{pgfscope}%
\pgfpathrectangle{\pgfqpoint{7.105882in}{0.750000in}}{\pgfqpoint{4.376471in}{0.978947in}} %
\pgfusepath{clip}%
\pgfsetbuttcap%
\pgfsetroundjoin%
\definecolor{currentfill}{rgb}{1.000000,0.000000,0.000000}%
\pgfsetfillcolor{currentfill}%
\pgfsetlinewidth{2.007500pt}%
\definecolor{currentstroke}{rgb}{1.000000,0.000000,0.000000}%
\pgfsetstrokecolor{currentstroke}%
\pgfsetdash{}{0pt}%
\pgfpathmoveto{\pgfqpoint{10.865269in}{1.365052in}}%
\pgfpathlineto{\pgfqpoint{10.927382in}{1.365052in}}%
\pgfpathmoveto{\pgfqpoint{10.896325in}{1.333995in}}%
\pgfpathlineto{\pgfqpoint{10.896325in}{1.396108in}}%
\pgfusepath{stroke,fill}%
\end{pgfscope}%
\begin{pgfscope}%
\pgfpathrectangle{\pgfqpoint{7.105882in}{0.750000in}}{\pgfqpoint{4.376471in}{0.978947in}} %
\pgfusepath{clip}%
\pgfsetbuttcap%
\pgfsetroundjoin%
\definecolor{currentfill}{rgb}{1.000000,0.000000,0.000000}%
\pgfsetfillcolor{currentfill}%
\pgfsetlinewidth{2.007500pt}%
\definecolor{currentstroke}{rgb}{1.000000,0.000000,0.000000}%
\pgfsetstrokecolor{currentstroke}%
\pgfsetdash{}{0pt}%
\pgfpathmoveto{\pgfqpoint{10.674584in}{1.087470in}}%
\pgfpathlineto{\pgfqpoint{10.736697in}{1.087470in}}%
\pgfpathmoveto{\pgfqpoint{10.705641in}{1.056414in}}%
\pgfpathlineto{\pgfqpoint{10.705641in}{1.118527in}}%
\pgfusepath{stroke,fill}%
\end{pgfscope}%
\begin{pgfscope}%
\pgfpathrectangle{\pgfqpoint{7.105882in}{0.750000in}}{\pgfqpoint{4.376471in}{0.978947in}} %
\pgfusepath{clip}%
\pgfsetbuttcap%
\pgfsetroundjoin%
\definecolor{currentfill}{rgb}{1.000000,0.000000,0.000000}%
\pgfsetfillcolor{currentfill}%
\pgfsetlinewidth{2.007500pt}%
\definecolor{currentstroke}{rgb}{1.000000,0.000000,0.000000}%
\pgfsetstrokecolor{currentstroke}%
\pgfsetdash{}{0pt}%
\pgfpathmoveto{\pgfqpoint{10.996186in}{1.494833in}}%
\pgfpathlineto{\pgfqpoint{11.058299in}{1.494833in}}%
\pgfpathmoveto{\pgfqpoint{11.027243in}{1.463777in}}%
\pgfpathlineto{\pgfqpoint{11.027243in}{1.525890in}}%
\pgfusepath{stroke,fill}%
\end{pgfscope}%
\begin{pgfscope}%
\pgfpathrectangle{\pgfqpoint{7.105882in}{0.750000in}}{\pgfqpoint{4.376471in}{0.978947in}} %
\pgfusepath{clip}%
\pgfsetbuttcap%
\pgfsetroundjoin%
\definecolor{currentfill}{rgb}{1.000000,0.000000,0.000000}%
\pgfsetfillcolor{currentfill}%
\pgfsetlinewidth{2.007500pt}%
\definecolor{currentstroke}{rgb}{1.000000,0.000000,0.000000}%
\pgfsetstrokecolor{currentstroke}%
\pgfsetdash{}{0pt}%
\pgfpathmoveto{\pgfqpoint{11.376435in}{1.426070in}}%
\pgfpathlineto{\pgfqpoint{11.438548in}{1.426070in}}%
\pgfpathmoveto{\pgfqpoint{11.407492in}{1.395013in}}%
\pgfpathlineto{\pgfqpoint{11.407492in}{1.457126in}}%
\pgfusepath{stroke,fill}%
\end{pgfscope}%
\begin{pgfscope}%
\pgfpathrectangle{\pgfqpoint{7.105882in}{0.750000in}}{\pgfqpoint{4.376471in}{0.978947in}} %
\pgfusepath{clip}%
\pgfsetbuttcap%
\pgfsetroundjoin%
\definecolor{currentfill}{rgb}{1.000000,0.000000,0.000000}%
\pgfsetfillcolor{currentfill}%
\pgfsetlinewidth{2.007500pt}%
\definecolor{currentstroke}{rgb}{1.000000,0.000000,0.000000}%
\pgfsetstrokecolor{currentstroke}%
\pgfsetdash{}{0pt}%
\pgfpathmoveto{\pgfqpoint{10.748115in}{1.242637in}}%
\pgfpathlineto{\pgfqpoint{10.810228in}{1.242637in}}%
\pgfpathmoveto{\pgfqpoint{10.779172in}{1.211581in}}%
\pgfpathlineto{\pgfqpoint{10.779172in}{1.273694in}}%
\pgfusepath{stroke,fill}%
\end{pgfscope}%
\begin{pgfscope}%
\pgfpathrectangle{\pgfqpoint{7.105882in}{0.750000in}}{\pgfqpoint{4.376471in}{0.978947in}} %
\pgfusepath{clip}%
\pgfsetbuttcap%
\pgfsetroundjoin%
\definecolor{currentfill}{rgb}{1.000000,0.000000,0.000000}%
\pgfsetfillcolor{currentfill}%
\pgfsetlinewidth{2.007500pt}%
\definecolor{currentstroke}{rgb}{1.000000,0.000000,0.000000}%
\pgfsetstrokecolor{currentstroke}%
\pgfsetdash{}{0pt}%
\pgfpathmoveto{\pgfqpoint{9.565841in}{0.967097in}}%
\pgfpathlineto{\pgfqpoint{9.627954in}{0.967097in}}%
\pgfpathmoveto{\pgfqpoint{9.596897in}{0.936040in}}%
\pgfpathlineto{\pgfqpoint{9.596897in}{0.998153in}}%
\pgfusepath{stroke,fill}%
\end{pgfscope}%
\begin{pgfscope}%
\pgfpathrectangle{\pgfqpoint{7.105882in}{0.750000in}}{\pgfqpoint{4.376471in}{0.978947in}} %
\pgfusepath{clip}%
\pgfsetbuttcap%
\pgfsetroundjoin%
\definecolor{currentfill}{rgb}{1.000000,0.000000,0.000000}%
\pgfsetfillcolor{currentfill}%
\pgfsetlinewidth{2.007500pt}%
\definecolor{currentstroke}{rgb}{1.000000,0.000000,0.000000}%
\pgfsetstrokecolor{currentstroke}%
\pgfsetdash{}{0pt}%
\pgfpathmoveto{\pgfqpoint{10.682890in}{1.109507in}}%
\pgfpathlineto{\pgfqpoint{10.745003in}{1.109507in}}%
\pgfpathmoveto{\pgfqpoint{10.713947in}{1.078450in}}%
\pgfpathlineto{\pgfqpoint{10.713947in}{1.140563in}}%
\pgfusepath{stroke,fill}%
\end{pgfscope}%
\begin{pgfscope}%
\pgfpathrectangle{\pgfqpoint{7.105882in}{0.750000in}}{\pgfqpoint{4.376471in}{0.978947in}} %
\pgfusepath{clip}%
\pgfsetbuttcap%
\pgfsetroundjoin%
\definecolor{currentfill}{rgb}{1.000000,0.000000,0.000000}%
\pgfsetfillcolor{currentfill}%
\pgfsetlinewidth{2.007500pt}%
\definecolor{currentstroke}{rgb}{1.000000,0.000000,0.000000}%
\pgfsetstrokecolor{currentstroke}%
\pgfsetdash{}{0pt}%
\pgfpathmoveto{\pgfqpoint{8.364220in}{1.130244in}}%
\pgfpathlineto{\pgfqpoint{8.426333in}{1.130244in}}%
\pgfpathmoveto{\pgfqpoint{8.395276in}{1.099188in}}%
\pgfpathlineto{\pgfqpoint{8.395276in}{1.161301in}}%
\pgfusepath{stroke,fill}%
\end{pgfscope}%
\begin{pgfscope}%
\pgfpathrectangle{\pgfqpoint{7.105882in}{0.750000in}}{\pgfqpoint{4.376471in}{0.978947in}} %
\pgfusepath{clip}%
\pgfsetbuttcap%
\pgfsetroundjoin%
\definecolor{currentfill}{rgb}{1.000000,0.000000,0.000000}%
\pgfsetfillcolor{currentfill}%
\pgfsetlinewidth{2.007500pt}%
\definecolor{currentstroke}{rgb}{1.000000,0.000000,0.000000}%
\pgfsetstrokecolor{currentstroke}%
\pgfsetdash{}{0pt}%
\pgfpathmoveto{\pgfqpoint{10.190596in}{1.233528in}}%
\pgfpathlineto{\pgfqpoint{10.252709in}{1.233528in}}%
\pgfpathmoveto{\pgfqpoint{10.221653in}{1.202472in}}%
\pgfpathlineto{\pgfqpoint{10.221653in}{1.264585in}}%
\pgfusepath{stroke,fill}%
\end{pgfscope}%
\begin{pgfscope}%
\pgfpathrectangle{\pgfqpoint{7.105882in}{0.750000in}}{\pgfqpoint{4.376471in}{0.978947in}} %
\pgfusepath{clip}%
\pgfsetbuttcap%
\pgfsetroundjoin%
\definecolor{currentfill}{rgb}{1.000000,0.000000,0.000000}%
\pgfsetfillcolor{currentfill}%
\pgfsetlinewidth{2.007500pt}%
\definecolor{currentstroke}{rgb}{1.000000,0.000000,0.000000}%
\pgfsetstrokecolor{currentstroke}%
\pgfsetdash{}{0pt}%
\pgfpathmoveto{\pgfqpoint{8.452025in}{1.138534in}}%
\pgfpathlineto{\pgfqpoint{8.514138in}{1.138534in}}%
\pgfpathmoveto{\pgfqpoint{8.483082in}{1.107478in}}%
\pgfpathlineto{\pgfqpoint{8.483082in}{1.169591in}}%
\pgfusepath{stroke,fill}%
\end{pgfscope}%
\begin{pgfscope}%
\pgfpathrectangle{\pgfqpoint{7.105882in}{0.750000in}}{\pgfqpoint{4.376471in}{0.978947in}} %
\pgfusepath{clip}%
\pgfsetbuttcap%
\pgfsetroundjoin%
\definecolor{currentfill}{rgb}{1.000000,0.000000,0.000000}%
\pgfsetfillcolor{currentfill}%
\pgfsetlinewidth{2.007500pt}%
\definecolor{currentstroke}{rgb}{1.000000,0.000000,0.000000}%
\pgfsetstrokecolor{currentstroke}%
\pgfsetdash{}{0pt}%
\pgfpathmoveto{\pgfqpoint{11.257573in}{1.538728in}}%
\pgfpathlineto{\pgfqpoint{11.319686in}{1.538728in}}%
\pgfpathmoveto{\pgfqpoint{11.288629in}{1.507671in}}%
\pgfpathlineto{\pgfqpoint{11.288629in}{1.569784in}}%
\pgfusepath{stroke,fill}%
\end{pgfscope}%
\begin{pgfscope}%
\pgfpathrectangle{\pgfqpoint{7.105882in}{0.750000in}}{\pgfqpoint{4.376471in}{0.978947in}} %
\pgfusepath{clip}%
\pgfsetbuttcap%
\pgfsetroundjoin%
\definecolor{currentfill}{rgb}{1.000000,0.000000,0.000000}%
\pgfsetfillcolor{currentfill}%
\pgfsetlinewidth{2.007500pt}%
\definecolor{currentstroke}{rgb}{1.000000,0.000000,0.000000}%
\pgfsetstrokecolor{currentstroke}%
\pgfsetdash{}{0pt}%
\pgfpathmoveto{\pgfqpoint{9.777203in}{1.185930in}}%
\pgfpathlineto{\pgfqpoint{9.839316in}{1.185930in}}%
\pgfpathmoveto{\pgfqpoint{9.808260in}{1.154874in}}%
\pgfpathlineto{\pgfqpoint{9.808260in}{1.216987in}}%
\pgfusepath{stroke,fill}%
\end{pgfscope}%
\begin{pgfscope}%
\pgfpathrectangle{\pgfqpoint{7.105882in}{0.750000in}}{\pgfqpoint{4.376471in}{0.978947in}} %
\pgfusepath{clip}%
\pgfsetbuttcap%
\pgfsetroundjoin%
\definecolor{currentfill}{rgb}{1.000000,0.000000,0.000000}%
\pgfsetfillcolor{currentfill}%
\pgfsetlinewidth{2.007500pt}%
\definecolor{currentstroke}{rgb}{1.000000,0.000000,0.000000}%
\pgfsetstrokecolor{currentstroke}%
\pgfsetdash{}{0pt}%
\pgfpathmoveto{\pgfqpoint{9.401925in}{0.934813in}}%
\pgfpathlineto{\pgfqpoint{9.464038in}{0.934813in}}%
\pgfpathmoveto{\pgfqpoint{9.432981in}{0.903757in}}%
\pgfpathlineto{\pgfqpoint{9.432981in}{0.965870in}}%
\pgfusepath{stroke,fill}%
\end{pgfscope}%
\begin{pgfscope}%
\pgfpathrectangle{\pgfqpoint{7.105882in}{0.750000in}}{\pgfqpoint{4.376471in}{0.978947in}} %
\pgfusepath{clip}%
\pgfsetbuttcap%
\pgfsetroundjoin%
\definecolor{currentfill}{rgb}{0.000000,0.000000,0.000000}%
\pgfsetfillcolor{currentfill}%
\pgfsetlinewidth{0.301125pt}%
\definecolor{currentstroke}{rgb}{0.000000,0.000000,0.000000}%
\pgfsetstrokecolor{currentstroke}%
\pgfsetdash{}{0pt}%
\pgfsys@defobject{currentmarker}{\pgfqpoint{-0.015528in}{-0.015528in}}{\pgfqpoint{0.015528in}{0.015528in}}{%
\pgfpathmoveto{\pgfqpoint{0.000000in}{-0.015528in}}%
\pgfpathcurveto{\pgfqpoint{0.004118in}{-0.015528in}}{\pgfqpoint{0.008068in}{-0.013892in}}{\pgfqpoint{0.010980in}{-0.010980in}}%
\pgfpathcurveto{\pgfqpoint{0.013892in}{-0.008068in}}{\pgfqpoint{0.015528in}{-0.004118in}}{\pgfqpoint{0.015528in}{0.000000in}}%
\pgfpathcurveto{\pgfqpoint{0.015528in}{0.004118in}}{\pgfqpoint{0.013892in}{0.008068in}}{\pgfqpoint{0.010980in}{0.010980in}}%
\pgfpathcurveto{\pgfqpoint{0.008068in}{0.013892in}}{\pgfqpoint{0.004118in}{0.015528in}}{\pgfqpoint{0.000000in}{0.015528in}}%
\pgfpathcurveto{\pgfqpoint{-0.004118in}{0.015528in}}{\pgfqpoint{-0.008068in}{0.013892in}}{\pgfqpoint{-0.010980in}{0.010980in}}%
\pgfpathcurveto{\pgfqpoint{-0.013892in}{0.008068in}}{\pgfqpoint{-0.015528in}{0.004118in}}{\pgfqpoint{-0.015528in}{0.000000in}}%
\pgfpathcurveto{\pgfqpoint{-0.015528in}{-0.004118in}}{\pgfqpoint{-0.013892in}{-0.008068in}}{\pgfqpoint{-0.010980in}{-0.010980in}}%
\pgfpathcurveto{\pgfqpoint{-0.008068in}{-0.013892in}}{\pgfqpoint{-0.004118in}{-0.015528in}}{\pgfqpoint{0.000000in}{-0.015528in}}%
\pgfpathclose%
\pgfusepath{stroke,fill}%
}%
\begin{pgfscope}%
\pgfsys@transformshift{7.981176in}{1.583863in}%
\pgfsys@useobject{currentmarker}{}%
\end{pgfscope}%
\begin{pgfscope}%
\pgfsys@transformshift{7.998770in}{1.489671in}%
\pgfsys@useobject{currentmarker}{}%
\end{pgfscope}%
\begin{pgfscope}%
\pgfsys@transformshift{8.016364in}{1.580445in}%
\pgfsys@useobject{currentmarker}{}%
\end{pgfscope}%
\begin{pgfscope}%
\pgfsys@transformshift{8.033958in}{1.599582in}%
\pgfsys@useobject{currentmarker}{}%
\end{pgfscope}%
\begin{pgfscope}%
\pgfsys@transformshift{8.051552in}{1.534796in}%
\pgfsys@useobject{currentmarker}{}%
\end{pgfscope}%
\begin{pgfscope}%
\pgfsys@transformshift{8.069146in}{1.530696in}%
\pgfsys@useobject{currentmarker}{}%
\end{pgfscope}%
\begin{pgfscope}%
\pgfsys@transformshift{8.086740in}{1.406938in}%
\pgfsys@useobject{currentmarker}{}%
\end{pgfscope}%
\begin{pgfscope}%
\pgfsys@transformshift{8.104333in}{1.406543in}%
\pgfsys@useobject{currentmarker}{}%
\end{pgfscope}%
\begin{pgfscope}%
\pgfsys@transformshift{8.121927in}{1.347216in}%
\pgfsys@useobject{currentmarker}{}%
\end{pgfscope}%
\begin{pgfscope}%
\pgfsys@transformshift{8.139521in}{1.351930in}%
\pgfsys@useobject{currentmarker}{}%
\end{pgfscope}%
\begin{pgfscope}%
\pgfsys@transformshift{8.157115in}{1.279228in}%
\pgfsys@useobject{currentmarker}{}%
\end{pgfscope}%
\begin{pgfscope}%
\pgfsys@transformshift{8.174709in}{1.160606in}%
\pgfsys@useobject{currentmarker}{}%
\end{pgfscope}%
\begin{pgfscope}%
\pgfsys@transformshift{8.192303in}{1.330351in}%
\pgfsys@useobject{currentmarker}{}%
\end{pgfscope}%
\begin{pgfscope}%
\pgfsys@transformshift{8.209897in}{1.248201in}%
\pgfsys@useobject{currentmarker}{}%
\end{pgfscope}%
\begin{pgfscope}%
\pgfsys@transformshift{8.227490in}{1.101305in}%
\pgfsys@useobject{currentmarker}{}%
\end{pgfscope}%
\begin{pgfscope}%
\pgfsys@transformshift{8.245084in}{1.294708in}%
\pgfsys@useobject{currentmarker}{}%
\end{pgfscope}%
\begin{pgfscope}%
\pgfsys@transformshift{8.262678in}{1.136705in}%
\pgfsys@useobject{currentmarker}{}%
\end{pgfscope}%
\begin{pgfscope}%
\pgfsys@transformshift{8.280272in}{1.218082in}%
\pgfsys@useobject{currentmarker}{}%
\end{pgfscope}%
\begin{pgfscope}%
\pgfsys@transformshift{8.297866in}{1.272640in}%
\pgfsys@useobject{currentmarker}{}%
\end{pgfscope}%
\begin{pgfscope}%
\pgfsys@transformshift{8.315460in}{1.198976in}%
\pgfsys@useobject{currentmarker}{}%
\end{pgfscope}%
\begin{pgfscope}%
\pgfsys@transformshift{8.333054in}{1.291626in}%
\pgfsys@useobject{currentmarker}{}%
\end{pgfscope}%
\begin{pgfscope}%
\pgfsys@transformshift{8.350647in}{1.041254in}%
\pgfsys@useobject{currentmarker}{}%
\end{pgfscope}%
\begin{pgfscope}%
\pgfsys@transformshift{8.368241in}{1.202078in}%
\pgfsys@useobject{currentmarker}{}%
\end{pgfscope}%
\begin{pgfscope}%
\pgfsys@transformshift{8.385835in}{1.087229in}%
\pgfsys@useobject{currentmarker}{}%
\end{pgfscope}%
\begin{pgfscope}%
\pgfsys@transformshift{8.403429in}{1.066387in}%
\pgfsys@useobject{currentmarker}{}%
\end{pgfscope}%
\begin{pgfscope}%
\pgfsys@transformshift{8.421023in}{1.096351in}%
\pgfsys@useobject{currentmarker}{}%
\end{pgfscope}%
\begin{pgfscope}%
\pgfsys@transformshift{8.438617in}{1.125805in}%
\pgfsys@useobject{currentmarker}{}%
\end{pgfscope}%
\begin{pgfscope}%
\pgfsys@transformshift{8.456210in}{1.167420in}%
\pgfsys@useobject{currentmarker}{}%
\end{pgfscope}%
\begin{pgfscope}%
\pgfsys@transformshift{8.473804in}{1.048813in}%
\pgfsys@useobject{currentmarker}{}%
\end{pgfscope}%
\begin{pgfscope}%
\pgfsys@transformshift{8.491398in}{1.267121in}%
\pgfsys@useobject{currentmarker}{}%
\end{pgfscope}%
\begin{pgfscope}%
\pgfsys@transformshift{8.508992in}{1.231941in}%
\pgfsys@useobject{currentmarker}{}%
\end{pgfscope}%
\begin{pgfscope}%
\pgfsys@transformshift{8.526586in}{1.038406in}%
\pgfsys@useobject{currentmarker}{}%
\end{pgfscope}%
\begin{pgfscope}%
\pgfsys@transformshift{8.544180in}{1.358678in}%
\pgfsys@useobject{currentmarker}{}%
\end{pgfscope}%
\begin{pgfscope}%
\pgfsys@transformshift{8.561774in}{1.413170in}%
\pgfsys@useobject{currentmarker}{}%
\end{pgfscope}%
\begin{pgfscope}%
\pgfsys@transformshift{8.579367in}{1.353852in}%
\pgfsys@useobject{currentmarker}{}%
\end{pgfscope}%
\begin{pgfscope}%
\pgfsys@transformshift{8.596961in}{1.229798in}%
\pgfsys@useobject{currentmarker}{}%
\end{pgfscope}%
\begin{pgfscope}%
\pgfsys@transformshift{8.614555in}{1.153945in}%
\pgfsys@useobject{currentmarker}{}%
\end{pgfscope}%
\begin{pgfscope}%
\pgfsys@transformshift{8.632149in}{1.385934in}%
\pgfsys@useobject{currentmarker}{}%
\end{pgfscope}%
\begin{pgfscope}%
\pgfsys@transformshift{8.649743in}{1.252646in}%
\pgfsys@useobject{currentmarker}{}%
\end{pgfscope}%
\begin{pgfscope}%
\pgfsys@transformshift{8.667337in}{1.433641in}%
\pgfsys@useobject{currentmarker}{}%
\end{pgfscope}%
\begin{pgfscope}%
\pgfsys@transformshift{8.684931in}{1.345114in}%
\pgfsys@useobject{currentmarker}{}%
\end{pgfscope}%
\begin{pgfscope}%
\pgfsys@transformshift{8.702524in}{1.437827in}%
\pgfsys@useobject{currentmarker}{}%
\end{pgfscope}%
\begin{pgfscope}%
\pgfsys@transformshift{8.720118in}{1.388209in}%
\pgfsys@useobject{currentmarker}{}%
\end{pgfscope}%
\begin{pgfscope}%
\pgfsys@transformshift{8.737712in}{1.436642in}%
\pgfsys@useobject{currentmarker}{}%
\end{pgfscope}%
\begin{pgfscope}%
\pgfsys@transformshift{8.755306in}{1.377294in}%
\pgfsys@useobject{currentmarker}{}%
\end{pgfscope}%
\begin{pgfscope}%
\pgfsys@transformshift{8.772900in}{1.568717in}%
\pgfsys@useobject{currentmarker}{}%
\end{pgfscope}%
\begin{pgfscope}%
\pgfsys@transformshift{8.790494in}{1.408526in}%
\pgfsys@useobject{currentmarker}{}%
\end{pgfscope}%
\begin{pgfscope}%
\pgfsys@transformshift{8.808087in}{1.444018in}%
\pgfsys@useobject{currentmarker}{}%
\end{pgfscope}%
\begin{pgfscope}%
\pgfsys@transformshift{8.825681in}{1.600841in}%
\pgfsys@useobject{currentmarker}{}%
\end{pgfscope}%
\begin{pgfscope}%
\pgfsys@transformshift{8.843275in}{1.275399in}%
\pgfsys@useobject{currentmarker}{}%
\end{pgfscope}%
\begin{pgfscope}%
\pgfsys@transformshift{8.860869in}{1.285462in}%
\pgfsys@useobject{currentmarker}{}%
\end{pgfscope}%
\begin{pgfscope}%
\pgfsys@transformshift{8.878463in}{1.514150in}%
\pgfsys@useobject{currentmarker}{}%
\end{pgfscope}%
\begin{pgfscope}%
\pgfsys@transformshift{8.896057in}{1.293998in}%
\pgfsys@useobject{currentmarker}{}%
\end{pgfscope}%
\begin{pgfscope}%
\pgfsys@transformshift{8.913651in}{1.608181in}%
\pgfsys@useobject{currentmarker}{}%
\end{pgfscope}%
\begin{pgfscope}%
\pgfsys@transformshift{8.931244in}{1.362188in}%
\pgfsys@useobject{currentmarker}{}%
\end{pgfscope}%
\begin{pgfscope}%
\pgfsys@transformshift{8.948838in}{1.320572in}%
\pgfsys@useobject{currentmarker}{}%
\end{pgfscope}%
\begin{pgfscope}%
\pgfsys@transformshift{8.966432in}{1.583393in}%
\pgfsys@useobject{currentmarker}{}%
\end{pgfscope}%
\begin{pgfscope}%
\pgfsys@transformshift{8.984026in}{1.526929in}%
\pgfsys@useobject{currentmarker}{}%
\end{pgfscope}%
\begin{pgfscope}%
\pgfsys@transformshift{9.001620in}{1.553271in}%
\pgfsys@useobject{currentmarker}{}%
\end{pgfscope}%
\begin{pgfscope}%
\pgfsys@transformshift{9.019214in}{1.440414in}%
\pgfsys@useobject{currentmarker}{}%
\end{pgfscope}%
\begin{pgfscope}%
\pgfsys@transformshift{9.036808in}{1.243825in}%
\pgfsys@useobject{currentmarker}{}%
\end{pgfscope}%
\begin{pgfscope}%
\pgfsys@transformshift{9.054401in}{1.508616in}%
\pgfsys@useobject{currentmarker}{}%
\end{pgfscope}%
\begin{pgfscope}%
\pgfsys@transformshift{9.071995in}{1.267415in}%
\pgfsys@useobject{currentmarker}{}%
\end{pgfscope}%
\begin{pgfscope}%
\pgfsys@transformshift{9.089589in}{1.356354in}%
\pgfsys@useobject{currentmarker}{}%
\end{pgfscope}%
\begin{pgfscope}%
\pgfsys@transformshift{9.107183in}{1.349946in}%
\pgfsys@useobject{currentmarker}{}%
\end{pgfscope}%
\begin{pgfscope}%
\pgfsys@transformshift{9.124777in}{1.215603in}%
\pgfsys@useobject{currentmarker}{}%
\end{pgfscope}%
\begin{pgfscope}%
\pgfsys@transformshift{9.142371in}{1.271512in}%
\pgfsys@useobject{currentmarker}{}%
\end{pgfscope}%
\begin{pgfscope}%
\pgfsys@transformshift{9.159965in}{1.280038in}%
\pgfsys@useobject{currentmarker}{}%
\end{pgfscope}%
\begin{pgfscope}%
\pgfsys@transformshift{9.177558in}{1.201314in}%
\pgfsys@useobject{currentmarker}{}%
\end{pgfscope}%
\begin{pgfscope}%
\pgfsys@transformshift{9.195152in}{1.027783in}%
\pgfsys@useobject{currentmarker}{}%
\end{pgfscope}%
\begin{pgfscope}%
\pgfsys@transformshift{9.212746in}{1.147508in}%
\pgfsys@useobject{currentmarker}{}%
\end{pgfscope}%
\begin{pgfscope}%
\pgfsys@transformshift{9.230340in}{1.230000in}%
\pgfsys@useobject{currentmarker}{}%
\end{pgfscope}%
\begin{pgfscope}%
\pgfsys@transformshift{9.247934in}{1.002209in}%
\pgfsys@useobject{currentmarker}{}%
\end{pgfscope}%
\begin{pgfscope}%
\pgfsys@transformshift{9.265528in}{1.036913in}%
\pgfsys@useobject{currentmarker}{}%
\end{pgfscope}%
\begin{pgfscope}%
\pgfsys@transformshift{9.283121in}{0.987962in}%
\pgfsys@useobject{currentmarker}{}%
\end{pgfscope}%
\begin{pgfscope}%
\pgfsys@transformshift{9.300715in}{1.202285in}%
\pgfsys@useobject{currentmarker}{}%
\end{pgfscope}%
\begin{pgfscope}%
\pgfsys@transformshift{9.318309in}{1.065014in}%
\pgfsys@useobject{currentmarker}{}%
\end{pgfscope}%
\begin{pgfscope}%
\pgfsys@transformshift{9.335903in}{1.022286in}%
\pgfsys@useobject{currentmarker}{}%
\end{pgfscope}%
\begin{pgfscope}%
\pgfsys@transformshift{9.353497in}{0.888172in}%
\pgfsys@useobject{currentmarker}{}%
\end{pgfscope}%
\begin{pgfscope}%
\pgfsys@transformshift{9.371091in}{1.009412in}%
\pgfsys@useobject{currentmarker}{}%
\end{pgfscope}%
\begin{pgfscope}%
\pgfsys@transformshift{9.388685in}{0.875298in}%
\pgfsys@useobject{currentmarker}{}%
\end{pgfscope}%
\begin{pgfscope}%
\pgfsys@transformshift{9.406278in}{0.938933in}%
\pgfsys@useobject{currentmarker}{}%
\end{pgfscope}%
\begin{pgfscope}%
\pgfsys@transformshift{9.423872in}{0.864529in}%
\pgfsys@useobject{currentmarker}{}%
\end{pgfscope}%
\begin{pgfscope}%
\pgfsys@transformshift{9.441466in}{0.994133in}%
\pgfsys@useobject{currentmarker}{}%
\end{pgfscope}%
\begin{pgfscope}%
\pgfsys@transformshift{9.459060in}{0.981871in}%
\pgfsys@useobject{currentmarker}{}%
\end{pgfscope}%
\begin{pgfscope}%
\pgfsys@transformshift{9.476654in}{0.901892in}%
\pgfsys@useobject{currentmarker}{}%
\end{pgfscope}%
\begin{pgfscope}%
\pgfsys@transformshift{9.494248in}{0.965702in}%
\pgfsys@useobject{currentmarker}{}%
\end{pgfscope}%
\begin{pgfscope}%
\pgfsys@transformshift{9.511842in}{0.818136in}%
\pgfsys@useobject{currentmarker}{}%
\end{pgfscope}%
\begin{pgfscope}%
\pgfsys@transformshift{9.529435in}{0.783851in}%
\pgfsys@useobject{currentmarker}{}%
\end{pgfscope}%
\begin{pgfscope}%
\pgfsys@transformshift{9.547029in}{0.989012in}%
\pgfsys@useobject{currentmarker}{}%
\end{pgfscope}%
\begin{pgfscope}%
\pgfsys@transformshift{9.564623in}{0.971362in}%
\pgfsys@useobject{currentmarker}{}%
\end{pgfscope}%
\begin{pgfscope}%
\pgfsys@transformshift{9.582217in}{1.031043in}%
\pgfsys@useobject{currentmarker}{}%
\end{pgfscope}%
\begin{pgfscope}%
\pgfsys@transformshift{9.599811in}{1.222861in}%
\pgfsys@useobject{currentmarker}{}%
\end{pgfscope}%
\begin{pgfscope}%
\pgfsys@transformshift{9.617405in}{1.091202in}%
\pgfsys@useobject{currentmarker}{}%
\end{pgfscope}%
\begin{pgfscope}%
\pgfsys@transformshift{9.634999in}{0.918187in}%
\pgfsys@useobject{currentmarker}{}%
\end{pgfscope}%
\begin{pgfscope}%
\pgfsys@transformshift{9.652592in}{1.142721in}%
\pgfsys@useobject{currentmarker}{}%
\end{pgfscope}%
\begin{pgfscope}%
\pgfsys@transformshift{9.670186in}{0.913152in}%
\pgfsys@useobject{currentmarker}{}%
\end{pgfscope}%
\begin{pgfscope}%
\pgfsys@transformshift{9.687780in}{1.019588in}%
\pgfsys@useobject{currentmarker}{}%
\end{pgfscope}%
\begin{pgfscope}%
\pgfsys@transformshift{9.705374in}{1.079610in}%
\pgfsys@useobject{currentmarker}{}%
\end{pgfscope}%
\begin{pgfscope}%
\pgfsys@transformshift{9.722968in}{1.281587in}%
\pgfsys@useobject{currentmarker}{}%
\end{pgfscope}%
\begin{pgfscope}%
\pgfsys@transformshift{9.740562in}{1.051414in}%
\pgfsys@useobject{currentmarker}{}%
\end{pgfscope}%
\begin{pgfscope}%
\pgfsys@transformshift{9.758155in}{1.063552in}%
\pgfsys@useobject{currentmarker}{}%
\end{pgfscope}%
\begin{pgfscope}%
\pgfsys@transformshift{9.775749in}{1.158029in}%
\pgfsys@useobject{currentmarker}{}%
\end{pgfscope}%
\begin{pgfscope}%
\pgfsys@transformshift{9.793343in}{1.120223in}%
\pgfsys@useobject{currentmarker}{}%
\end{pgfscope}%
\begin{pgfscope}%
\pgfsys@transformshift{9.810937in}{1.321952in}%
\pgfsys@useobject{currentmarker}{}%
\end{pgfscope}%
\begin{pgfscope}%
\pgfsys@transformshift{9.828531in}{1.115308in}%
\pgfsys@useobject{currentmarker}{}%
\end{pgfscope}%
\begin{pgfscope}%
\pgfsys@transformshift{9.846125in}{1.125790in}%
\pgfsys@useobject{currentmarker}{}%
\end{pgfscope}%
\begin{pgfscope}%
\pgfsys@transformshift{9.863719in}{1.214357in}%
\pgfsys@useobject{currentmarker}{}%
\end{pgfscope}%
\begin{pgfscope}%
\pgfsys@transformshift{9.881312in}{1.223091in}%
\pgfsys@useobject{currentmarker}{}%
\end{pgfscope}%
\begin{pgfscope}%
\pgfsys@transformshift{9.898906in}{1.484043in}%
\pgfsys@useobject{currentmarker}{}%
\end{pgfscope}%
\begin{pgfscope}%
\pgfsys@transformshift{9.916500in}{1.395905in}%
\pgfsys@useobject{currentmarker}{}%
\end{pgfscope}%
\begin{pgfscope}%
\pgfsys@transformshift{9.934094in}{1.318116in}%
\pgfsys@useobject{currentmarker}{}%
\end{pgfscope}%
\begin{pgfscope}%
\pgfsys@transformshift{9.951688in}{1.192524in}%
\pgfsys@useobject{currentmarker}{}%
\end{pgfscope}%
\begin{pgfscope}%
\pgfsys@transformshift{9.969282in}{1.410011in}%
\pgfsys@useobject{currentmarker}{}%
\end{pgfscope}%
\begin{pgfscope}%
\pgfsys@transformshift{9.986876in}{1.226407in}%
\pgfsys@useobject{currentmarker}{}%
\end{pgfscope}%
\begin{pgfscope}%
\pgfsys@transformshift{10.004469in}{1.173408in}%
\pgfsys@useobject{currentmarker}{}%
\end{pgfscope}%
\begin{pgfscope}%
\pgfsys@transformshift{10.022063in}{1.452637in}%
\pgfsys@useobject{currentmarker}{}%
\end{pgfscope}%
\begin{pgfscope}%
\pgfsys@transformshift{10.039657in}{1.362397in}%
\pgfsys@useobject{currentmarker}{}%
\end{pgfscope}%
\begin{pgfscope}%
\pgfsys@transformshift{10.057251in}{1.420628in}%
\pgfsys@useobject{currentmarker}{}%
\end{pgfscope}%
\begin{pgfscope}%
\pgfsys@transformshift{10.074845in}{1.353984in}%
\pgfsys@useobject{currentmarker}{}%
\end{pgfscope}%
\begin{pgfscope}%
\pgfsys@transformshift{10.092439in}{1.401791in}%
\pgfsys@useobject{currentmarker}{}%
\end{pgfscope}%
\begin{pgfscope}%
\pgfsys@transformshift{10.110033in}{1.239230in}%
\pgfsys@useobject{currentmarker}{}%
\end{pgfscope}%
\begin{pgfscope}%
\pgfsys@transformshift{10.127626in}{1.189725in}%
\pgfsys@useobject{currentmarker}{}%
\end{pgfscope}%
\begin{pgfscope}%
\pgfsys@transformshift{10.145220in}{1.352764in}%
\pgfsys@useobject{currentmarker}{}%
\end{pgfscope}%
\begin{pgfscope}%
\pgfsys@transformshift{10.162814in}{1.188036in}%
\pgfsys@useobject{currentmarker}{}%
\end{pgfscope}%
\begin{pgfscope}%
\pgfsys@transformshift{10.180408in}{1.185140in}%
\pgfsys@useobject{currentmarker}{}%
\end{pgfscope}%
\begin{pgfscope}%
\pgfsys@transformshift{10.198002in}{1.193453in}%
\pgfsys@useobject{currentmarker}{}%
\end{pgfscope}%
\begin{pgfscope}%
\pgfsys@transformshift{10.215596in}{1.225272in}%
\pgfsys@useobject{currentmarker}{}%
\end{pgfscope}%
\begin{pgfscope}%
\pgfsys@transformshift{10.233189in}{1.170299in}%
\pgfsys@useobject{currentmarker}{}%
\end{pgfscope}%
\begin{pgfscope}%
\pgfsys@transformshift{10.250783in}{1.048615in}%
\pgfsys@useobject{currentmarker}{}%
\end{pgfscope}%
\begin{pgfscope}%
\pgfsys@transformshift{10.268377in}{1.105339in}%
\pgfsys@useobject{currentmarker}{}%
\end{pgfscope}%
\begin{pgfscope}%
\pgfsys@transformshift{10.285971in}{0.926315in}%
\pgfsys@useobject{currentmarker}{}%
\end{pgfscope}%
\begin{pgfscope}%
\pgfsys@transformshift{10.303565in}{1.199006in}%
\pgfsys@useobject{currentmarker}{}%
\end{pgfscope}%
\begin{pgfscope}%
\pgfsys@transformshift{10.321159in}{0.954425in}%
\pgfsys@useobject{currentmarker}{}%
\end{pgfscope}%
\begin{pgfscope}%
\pgfsys@transformshift{10.338753in}{0.988260in}%
\pgfsys@useobject{currentmarker}{}%
\end{pgfscope}%
\begin{pgfscope}%
\pgfsys@transformshift{10.356346in}{1.090024in}%
\pgfsys@useobject{currentmarker}{}%
\end{pgfscope}%
\begin{pgfscope}%
\pgfsys@transformshift{10.373940in}{0.994049in}%
\pgfsys@useobject{currentmarker}{}%
\end{pgfscope}%
\begin{pgfscope}%
\pgfsys@transformshift{10.391534in}{1.212689in}%
\pgfsys@useobject{currentmarker}{}%
\end{pgfscope}%
\begin{pgfscope}%
\pgfsys@transformshift{10.409128in}{0.910673in}%
\pgfsys@useobject{currentmarker}{}%
\end{pgfscope}%
\begin{pgfscope}%
\pgfsys@transformshift{10.426722in}{1.058361in}%
\pgfsys@useobject{currentmarker}{}%
\end{pgfscope}%
\begin{pgfscope}%
\pgfsys@transformshift{10.444316in}{1.017342in}%
\pgfsys@useobject{currentmarker}{}%
\end{pgfscope}%
\begin{pgfscope}%
\pgfsys@transformshift{10.461910in}{0.894182in}%
\pgfsys@useobject{currentmarker}{}%
\end{pgfscope}%
\begin{pgfscope}%
\pgfsys@transformshift{10.479503in}{1.060452in}%
\pgfsys@useobject{currentmarker}{}%
\end{pgfscope}%
\begin{pgfscope}%
\pgfsys@transformshift{10.497097in}{0.985337in}%
\pgfsys@useobject{currentmarker}{}%
\end{pgfscope}%
\begin{pgfscope}%
\pgfsys@transformshift{10.514691in}{1.079294in}%
\pgfsys@useobject{currentmarker}{}%
\end{pgfscope}%
\begin{pgfscope}%
\pgfsys@transformshift{10.532285in}{1.084411in}%
\pgfsys@useobject{currentmarker}{}%
\end{pgfscope}%
\begin{pgfscope}%
\pgfsys@transformshift{10.549879in}{1.222992in}%
\pgfsys@useobject{currentmarker}{}%
\end{pgfscope}%
\begin{pgfscope}%
\pgfsys@transformshift{10.567473in}{1.142792in}%
\pgfsys@useobject{currentmarker}{}%
\end{pgfscope}%
\begin{pgfscope}%
\pgfsys@transformshift{10.585067in}{0.975095in}%
\pgfsys@useobject{currentmarker}{}%
\end{pgfscope}%
\begin{pgfscope}%
\pgfsys@transformshift{10.602660in}{0.996685in}%
\pgfsys@useobject{currentmarker}{}%
\end{pgfscope}%
\begin{pgfscope}%
\pgfsys@transformshift{10.620254in}{1.143655in}%
\pgfsys@useobject{currentmarker}{}%
\end{pgfscope}%
\begin{pgfscope}%
\pgfsys@transformshift{10.637848in}{1.110768in}%
\pgfsys@useobject{currentmarker}{}%
\end{pgfscope}%
\begin{pgfscope}%
\pgfsys@transformshift{10.655442in}{1.123579in}%
\pgfsys@useobject{currentmarker}{}%
\end{pgfscope}%
\begin{pgfscope}%
\pgfsys@transformshift{10.673036in}{0.909591in}%
\pgfsys@useobject{currentmarker}{}%
\end{pgfscope}%
\begin{pgfscope}%
\pgfsys@transformshift{10.690630in}{1.089876in}%
\pgfsys@useobject{currentmarker}{}%
\end{pgfscope}%
\begin{pgfscope}%
\pgfsys@transformshift{10.708223in}{1.036542in}%
\pgfsys@useobject{currentmarker}{}%
\end{pgfscope}%
\begin{pgfscope}%
\pgfsys@transformshift{10.725817in}{1.161209in}%
\pgfsys@useobject{currentmarker}{}%
\end{pgfscope}%
\begin{pgfscope}%
\pgfsys@transformshift{10.743411in}{1.144771in}%
\pgfsys@useobject{currentmarker}{}%
\end{pgfscope}%
\begin{pgfscope}%
\pgfsys@transformshift{10.761005in}{1.270786in}%
\pgfsys@useobject{currentmarker}{}%
\end{pgfscope}%
\begin{pgfscope}%
\pgfsys@transformshift{10.778599in}{1.234403in}%
\pgfsys@useobject{currentmarker}{}%
\end{pgfscope}%
\begin{pgfscope}%
\pgfsys@transformshift{10.796193in}{1.307020in}%
\pgfsys@useobject{currentmarker}{}%
\end{pgfscope}%
\begin{pgfscope}%
\pgfsys@transformshift{10.813787in}{1.204550in}%
\pgfsys@useobject{currentmarker}{}%
\end{pgfscope}%
\begin{pgfscope}%
\pgfsys@transformshift{10.831380in}{1.181373in}%
\pgfsys@useobject{currentmarker}{}%
\end{pgfscope}%
\begin{pgfscope}%
\pgfsys@transformshift{10.848974in}{1.261523in}%
\pgfsys@useobject{currentmarker}{}%
\end{pgfscope}%
\begin{pgfscope}%
\pgfsys@transformshift{10.866568in}{1.327137in}%
\pgfsys@useobject{currentmarker}{}%
\end{pgfscope}%
\begin{pgfscope}%
\pgfsys@transformshift{10.884162in}{1.392747in}%
\pgfsys@useobject{currentmarker}{}%
\end{pgfscope}%
\begin{pgfscope}%
\pgfsys@transformshift{10.901756in}{1.609158in}%
\pgfsys@useobject{currentmarker}{}%
\end{pgfscope}%
\begin{pgfscope}%
\pgfsys@transformshift{10.919350in}{1.398425in}%
\pgfsys@useobject{currentmarker}{}%
\end{pgfscope}%
\begin{pgfscope}%
\pgfsys@transformshift{10.936944in}{1.328312in}%
\pgfsys@useobject{currentmarker}{}%
\end{pgfscope}%
\begin{pgfscope}%
\pgfsys@transformshift{10.954537in}{1.412490in}%
\pgfsys@useobject{currentmarker}{}%
\end{pgfscope}%
\begin{pgfscope}%
\pgfsys@transformshift{10.972131in}{1.421228in}%
\pgfsys@useobject{currentmarker}{}%
\end{pgfscope}%
\begin{pgfscope}%
\pgfsys@transformshift{10.989725in}{1.536935in}%
\pgfsys@useobject{currentmarker}{}%
\end{pgfscope}%
\begin{pgfscope}%
\pgfsys@transformshift{11.007319in}{1.348517in}%
\pgfsys@useobject{currentmarker}{}%
\end{pgfscope}%
\begin{pgfscope}%
\pgfsys@transformshift{11.024913in}{1.528236in}%
\pgfsys@useobject{currentmarker}{}%
\end{pgfscope}%
\begin{pgfscope}%
\pgfsys@transformshift{11.042507in}{1.552137in}%
\pgfsys@useobject{currentmarker}{}%
\end{pgfscope}%
\begin{pgfscope}%
\pgfsys@transformshift{11.060101in}{1.572407in}%
\pgfsys@useobject{currentmarker}{}%
\end{pgfscope}%
\begin{pgfscope}%
\pgfsys@transformshift{11.077694in}{1.498469in}%
\pgfsys@useobject{currentmarker}{}%
\end{pgfscope}%
\begin{pgfscope}%
\pgfsys@transformshift{11.095288in}{1.543820in}%
\pgfsys@useobject{currentmarker}{}%
\end{pgfscope}%
\begin{pgfscope}%
\pgfsys@transformshift{11.112882in}{1.429595in}%
\pgfsys@useobject{currentmarker}{}%
\end{pgfscope}%
\begin{pgfscope}%
\pgfsys@transformshift{11.130476in}{1.529262in}%
\pgfsys@useobject{currentmarker}{}%
\end{pgfscope}%
\begin{pgfscope}%
\pgfsys@transformshift{11.148070in}{1.527009in}%
\pgfsys@useobject{currentmarker}{}%
\end{pgfscope}%
\begin{pgfscope}%
\pgfsys@transformshift{11.165664in}{1.625672in}%
\pgfsys@useobject{currentmarker}{}%
\end{pgfscope}%
\begin{pgfscope}%
\pgfsys@transformshift{11.183257in}{1.464366in}%
\pgfsys@useobject{currentmarker}{}%
\end{pgfscope}%
\begin{pgfscope}%
\pgfsys@transformshift{11.200851in}{1.659166in}%
\pgfsys@useobject{currentmarker}{}%
\end{pgfscope}%
\begin{pgfscope}%
\pgfsys@transformshift{11.218445in}{1.727455in}%
\pgfsys@useobject{currentmarker}{}%
\end{pgfscope}%
\begin{pgfscope}%
\pgfsys@transformshift{11.236039in}{1.357955in}%
\pgfsys@useobject{currentmarker}{}%
\end{pgfscope}%
\begin{pgfscope}%
\pgfsys@transformshift{11.253633in}{1.605050in}%
\pgfsys@useobject{currentmarker}{}%
\end{pgfscope}%
\begin{pgfscope}%
\pgfsys@transformshift{11.271227in}{1.621861in}%
\pgfsys@useobject{currentmarker}{}%
\end{pgfscope}%
\begin{pgfscope}%
\pgfsys@transformshift{11.288821in}{1.477952in}%
\pgfsys@useobject{currentmarker}{}%
\end{pgfscope}%
\begin{pgfscope}%
\pgfsys@transformshift{11.306414in}{1.491603in}%
\pgfsys@useobject{currentmarker}{}%
\end{pgfscope}%
\begin{pgfscope}%
\pgfsys@transformshift{11.324008in}{1.506950in}%
\pgfsys@useobject{currentmarker}{}%
\end{pgfscope}%
\begin{pgfscope}%
\pgfsys@transformshift{11.341602in}{1.477936in}%
\pgfsys@useobject{currentmarker}{}%
\end{pgfscope}%
\begin{pgfscope}%
\pgfsys@transformshift{11.359196in}{1.464209in}%
\pgfsys@useobject{currentmarker}{}%
\end{pgfscope}%
\begin{pgfscope}%
\pgfsys@transformshift{11.376790in}{1.312049in}%
\pgfsys@useobject{currentmarker}{}%
\end{pgfscope}%
\begin{pgfscope}%
\pgfsys@transformshift{11.394384in}{1.587727in}%
\pgfsys@useobject{currentmarker}{}%
\end{pgfscope}%
\begin{pgfscope}%
\pgfsys@transformshift{11.411978in}{1.567807in}%
\pgfsys@useobject{currentmarker}{}%
\end{pgfscope}%
\begin{pgfscope}%
\pgfsys@transformshift{11.429571in}{1.362703in}%
\pgfsys@useobject{currentmarker}{}%
\end{pgfscope}%
\begin{pgfscope}%
\pgfsys@transformshift{11.447165in}{1.284679in}%
\pgfsys@useobject{currentmarker}{}%
\end{pgfscope}%
\begin{pgfscope}%
\pgfsys@transformshift{11.464759in}{1.476755in}%
\pgfsys@useobject{currentmarker}{}%
\end{pgfscope}%
\begin{pgfscope}%
\pgfsys@transformshift{11.482353in}{1.355333in}%
\pgfsys@useobject{currentmarker}{}%
\end{pgfscope}%
\end{pgfscope}%
\begin{pgfscope}%
\pgfpathrectangle{\pgfqpoint{7.105882in}{0.750000in}}{\pgfqpoint{4.376471in}{0.978947in}} %
\pgfusepath{clip}%
\pgfsetroundcap%
\pgfsetroundjoin%
\pgfsetlinewidth{1.756562pt}%
\definecolor{currentstroke}{rgb}{0.298039,0.447059,0.690196}%
\pgfsetstrokecolor{currentstroke}%
\pgfsetdash{}{0pt}%
\pgfpathmoveto{\pgfqpoint{7.981176in}{1.120622in}}%
\pgfpathlineto{\pgfqpoint{8.033958in}{1.117688in}}%
\pgfpathlineto{\pgfqpoint{8.104333in}{1.116343in}}%
\pgfpathlineto{\pgfqpoint{8.473804in}{1.117072in}}%
\pgfpathlineto{\pgfqpoint{8.649743in}{1.119198in}}%
\pgfpathlineto{\pgfqpoint{8.825681in}{1.120911in}}%
\pgfpathlineto{\pgfqpoint{8.984026in}{1.119870in}}%
\pgfpathlineto{\pgfqpoint{9.300715in}{1.117123in}}%
\pgfpathlineto{\pgfqpoint{9.969282in}{1.117178in}}%
\pgfpathlineto{\pgfqpoint{11.218445in}{1.117051in}}%
\pgfpathlineto{\pgfqpoint{11.447165in}{1.115828in}}%
\pgfpathlineto{\pgfqpoint{11.482353in}{1.112722in}}%
\pgfpathlineto{\pgfqpoint{11.482353in}{1.112722in}}%
\pgfusepath{stroke}%
\end{pgfscope}%
\begin{pgfscope}%
\pgfsetrectcap%
\pgfsetmiterjoin%
\pgfsetlinewidth{1.003750pt}%
\definecolor{currentstroke}{rgb}{0.800000,0.800000,0.800000}%
\pgfsetstrokecolor{currentstroke}%
\pgfsetdash{}{0pt}%
\pgfpathmoveto{\pgfqpoint{7.105882in}{0.750000in}}%
\pgfpathlineto{\pgfqpoint{7.105882in}{1.728947in}}%
\pgfusepath{stroke}%
\end{pgfscope}%
\begin{pgfscope}%
\pgfsetrectcap%
\pgfsetmiterjoin%
\pgfsetlinewidth{1.003750pt}%
\definecolor{currentstroke}{rgb}{0.800000,0.800000,0.800000}%
\pgfsetstrokecolor{currentstroke}%
\pgfsetdash{}{0pt}%
\pgfpathmoveto{\pgfqpoint{11.482353in}{0.750000in}}%
\pgfpathlineto{\pgfqpoint{11.482353in}{1.728947in}}%
\pgfusepath{stroke}%
\end{pgfscope}%
\begin{pgfscope}%
\pgfsetrectcap%
\pgfsetmiterjoin%
\pgfsetlinewidth{1.003750pt}%
\definecolor{currentstroke}{rgb}{0.800000,0.800000,0.800000}%
\pgfsetstrokecolor{currentstroke}%
\pgfsetdash{}{0pt}%
\pgfpathmoveto{\pgfqpoint{7.105882in}{1.728947in}}%
\pgfpathlineto{\pgfqpoint{11.482353in}{1.728947in}}%
\pgfusepath{stroke}%
\end{pgfscope}%
\begin{pgfscope}%
\pgfsetrectcap%
\pgfsetmiterjoin%
\pgfsetlinewidth{1.003750pt}%
\definecolor{currentstroke}{rgb}{0.800000,0.800000,0.800000}%
\pgfsetstrokecolor{currentstroke}%
\pgfsetdash{}{0pt}%
\pgfpathmoveto{\pgfqpoint{7.105882in}{0.750000in}}%
\pgfpathlineto{\pgfqpoint{11.482353in}{0.750000in}}%
\pgfusepath{stroke}%
\end{pgfscope}%
\begin{pgfscope}%
\pgfsetroundcap%
\pgfsetroundjoin%
\pgfsetlinewidth{1.756562pt}%
\definecolor{currentstroke}{rgb}{0.298039,0.447059,0.690196}%
\pgfsetstrokecolor{currentstroke}%
\pgfsetdash{}{0pt}%
\pgfpathmoveto{\pgfqpoint{7.230882in}{1.344143in}}%
\pgfpathlineto{\pgfqpoint{7.508660in}{1.344143in}}%
\pgfusepath{stroke}%
\end{pgfscope}%
\begin{pgfscope}%
\definecolor{textcolor}{rgb}{0.150000,0.150000,0.150000}%
\pgfsetstrokecolor{textcolor}%
\pgfsetfillcolor{textcolor}%
\pgftext[x=7.619771in,y=1.295532in,left,base]{\color{textcolor}\sffamily\fontsize{10.000000}{12.000000}\selectfont \(\displaystyle \widetilde{\Phi}^* \theta^{\parallel}\)}%
\end{pgfscope}%
\begin{pgfscope}%
\pgfsetbuttcap%
\pgfsetroundjoin%
\definecolor{currentfill}{rgb}{1.000000,0.000000,0.000000}%
\pgfsetfillcolor{currentfill}%
\pgfsetlinewidth{2.007500pt}%
\definecolor{currentstroke}{rgb}{1.000000,0.000000,0.000000}%
\pgfsetstrokecolor{currentstroke}%
\pgfsetdash{}{0pt}%
\pgfpathmoveto{\pgfqpoint{7.338715in}{1.135525in}}%
\pgfpathlineto{\pgfqpoint{7.400828in}{1.135525in}}%
\pgfpathmoveto{\pgfqpoint{7.369771in}{1.104468in}}%
\pgfpathlineto{\pgfqpoint{7.369771in}{1.166581in}}%
\pgfusepath{stroke,fill}%
\end{pgfscope}%
\begin{pgfscope}%
\pgfsetbuttcap%
\pgfsetroundjoin%
\definecolor{currentfill}{rgb}{1.000000,0.000000,0.000000}%
\pgfsetfillcolor{currentfill}%
\pgfsetlinewidth{2.007500pt}%
\definecolor{currentstroke}{rgb}{1.000000,0.000000,0.000000}%
\pgfsetstrokecolor{currentstroke}%
\pgfsetdash{}{0pt}%
\pgfpathmoveto{\pgfqpoint{7.338715in}{1.135525in}}%
\pgfpathlineto{\pgfqpoint{7.400828in}{1.135525in}}%
\pgfpathmoveto{\pgfqpoint{7.369771in}{1.104468in}}%
\pgfpathlineto{\pgfqpoint{7.369771in}{1.166581in}}%
\pgfusepath{stroke,fill}%
\end{pgfscope}%
\begin{pgfscope}%
\pgfsetbuttcap%
\pgfsetroundjoin%
\definecolor{currentfill}{rgb}{1.000000,0.000000,0.000000}%
\pgfsetfillcolor{currentfill}%
\pgfsetlinewidth{2.007500pt}%
\definecolor{currentstroke}{rgb}{1.000000,0.000000,0.000000}%
\pgfsetstrokecolor{currentstroke}%
\pgfsetdash{}{0pt}%
\pgfpathmoveto{\pgfqpoint{7.338715in}{1.135525in}}%
\pgfpathlineto{\pgfqpoint{7.400828in}{1.135525in}}%
\pgfpathmoveto{\pgfqpoint{7.369771in}{1.104468in}}%
\pgfpathlineto{\pgfqpoint{7.369771in}{1.166581in}}%
\pgfusepath{stroke,fill}%
\end{pgfscope}%
\begin{pgfscope}%
\definecolor{textcolor}{rgb}{0.150000,0.150000,0.150000}%
\pgfsetstrokecolor{textcolor}%
\pgfsetfillcolor{textcolor}%
\pgftext[x=7.619771in,y=1.099066in,left,base]{\color{textcolor}\sffamily\fontsize{10.000000}{12.000000}\selectfont train}%
\end{pgfscope}%
\begin{pgfscope}%
\pgfsetbuttcap%
\pgfsetroundjoin%
\definecolor{currentfill}{rgb}{0.000000,0.000000,0.000000}%
\pgfsetfillcolor{currentfill}%
\pgfsetlinewidth{0.301125pt}%
\definecolor{currentstroke}{rgb}{0.000000,0.000000,0.000000}%
\pgfsetstrokecolor{currentstroke}%
\pgfsetdash{}{0pt}%
\pgfpathmoveto{\pgfqpoint{7.369771in}{0.923531in}}%
\pgfpathcurveto{\pgfqpoint{7.373889in}{0.923531in}}{\pgfqpoint{7.377839in}{0.925167in}}{\pgfqpoint{7.380751in}{0.928079in}}%
\pgfpathcurveto{\pgfqpoint{7.383663in}{0.930991in}}{\pgfqpoint{7.385299in}{0.934941in}}{\pgfqpoint{7.385299in}{0.939060in}}%
\pgfpathcurveto{\pgfqpoint{7.385299in}{0.943178in}}{\pgfqpoint{7.383663in}{0.947128in}}{\pgfqpoint{7.380751in}{0.950040in}}%
\pgfpathcurveto{\pgfqpoint{7.377839in}{0.952952in}}{\pgfqpoint{7.373889in}{0.954588in}}{\pgfqpoint{7.369771in}{0.954588in}}%
\pgfpathcurveto{\pgfqpoint{7.365653in}{0.954588in}}{\pgfqpoint{7.361703in}{0.952952in}}{\pgfqpoint{7.358791in}{0.950040in}}%
\pgfpathcurveto{\pgfqpoint{7.355879in}{0.947128in}}{\pgfqpoint{7.354243in}{0.943178in}}{\pgfqpoint{7.354243in}{0.939060in}}%
\pgfpathcurveto{\pgfqpoint{7.354243in}{0.934941in}}{\pgfqpoint{7.355879in}{0.930991in}}{\pgfqpoint{7.358791in}{0.928079in}}%
\pgfpathcurveto{\pgfqpoint{7.361703in}{0.925167in}}{\pgfqpoint{7.365653in}{0.923531in}}{\pgfqpoint{7.369771in}{0.923531in}}%
\pgfpathclose%
\pgfusepath{stroke,fill}%
\end{pgfscope}%
\begin{pgfscope}%
\pgfsetbuttcap%
\pgfsetroundjoin%
\definecolor{currentfill}{rgb}{0.000000,0.000000,0.000000}%
\pgfsetfillcolor{currentfill}%
\pgfsetlinewidth{0.301125pt}%
\definecolor{currentstroke}{rgb}{0.000000,0.000000,0.000000}%
\pgfsetstrokecolor{currentstroke}%
\pgfsetdash{}{0pt}%
\pgfpathmoveto{\pgfqpoint{7.369771in}{0.923531in}}%
\pgfpathcurveto{\pgfqpoint{7.373889in}{0.923531in}}{\pgfqpoint{7.377839in}{0.925167in}}{\pgfqpoint{7.380751in}{0.928079in}}%
\pgfpathcurveto{\pgfqpoint{7.383663in}{0.930991in}}{\pgfqpoint{7.385299in}{0.934941in}}{\pgfqpoint{7.385299in}{0.939060in}}%
\pgfpathcurveto{\pgfqpoint{7.385299in}{0.943178in}}{\pgfqpoint{7.383663in}{0.947128in}}{\pgfqpoint{7.380751in}{0.950040in}}%
\pgfpathcurveto{\pgfqpoint{7.377839in}{0.952952in}}{\pgfqpoint{7.373889in}{0.954588in}}{\pgfqpoint{7.369771in}{0.954588in}}%
\pgfpathcurveto{\pgfqpoint{7.365653in}{0.954588in}}{\pgfqpoint{7.361703in}{0.952952in}}{\pgfqpoint{7.358791in}{0.950040in}}%
\pgfpathcurveto{\pgfqpoint{7.355879in}{0.947128in}}{\pgfqpoint{7.354243in}{0.943178in}}{\pgfqpoint{7.354243in}{0.939060in}}%
\pgfpathcurveto{\pgfqpoint{7.354243in}{0.934941in}}{\pgfqpoint{7.355879in}{0.930991in}}{\pgfqpoint{7.358791in}{0.928079in}}%
\pgfpathcurveto{\pgfqpoint{7.361703in}{0.925167in}}{\pgfqpoint{7.365653in}{0.923531in}}{\pgfqpoint{7.369771in}{0.923531in}}%
\pgfpathclose%
\pgfusepath{stroke,fill}%
\end{pgfscope}%
\begin{pgfscope}%
\pgfsetbuttcap%
\pgfsetroundjoin%
\definecolor{currentfill}{rgb}{0.000000,0.000000,0.000000}%
\pgfsetfillcolor{currentfill}%
\pgfsetlinewidth{0.301125pt}%
\definecolor{currentstroke}{rgb}{0.000000,0.000000,0.000000}%
\pgfsetstrokecolor{currentstroke}%
\pgfsetdash{}{0pt}%
\pgfpathmoveto{\pgfqpoint{7.369771in}{0.923531in}}%
\pgfpathcurveto{\pgfqpoint{7.373889in}{0.923531in}}{\pgfqpoint{7.377839in}{0.925167in}}{\pgfqpoint{7.380751in}{0.928079in}}%
\pgfpathcurveto{\pgfqpoint{7.383663in}{0.930991in}}{\pgfqpoint{7.385299in}{0.934941in}}{\pgfqpoint{7.385299in}{0.939060in}}%
\pgfpathcurveto{\pgfqpoint{7.385299in}{0.943178in}}{\pgfqpoint{7.383663in}{0.947128in}}{\pgfqpoint{7.380751in}{0.950040in}}%
\pgfpathcurveto{\pgfqpoint{7.377839in}{0.952952in}}{\pgfqpoint{7.373889in}{0.954588in}}{\pgfqpoint{7.369771in}{0.954588in}}%
\pgfpathcurveto{\pgfqpoint{7.365653in}{0.954588in}}{\pgfqpoint{7.361703in}{0.952952in}}{\pgfqpoint{7.358791in}{0.950040in}}%
\pgfpathcurveto{\pgfqpoint{7.355879in}{0.947128in}}{\pgfqpoint{7.354243in}{0.943178in}}{\pgfqpoint{7.354243in}{0.939060in}}%
\pgfpathcurveto{\pgfqpoint{7.354243in}{0.934941in}}{\pgfqpoint{7.355879in}{0.930991in}}{\pgfqpoint{7.358791in}{0.928079in}}%
\pgfpathcurveto{\pgfqpoint{7.361703in}{0.925167in}}{\pgfqpoint{7.365653in}{0.923531in}}{\pgfqpoint{7.369771in}{0.923531in}}%
\pgfpathclose%
\pgfusepath{stroke,fill}%
\end{pgfscope}%
\begin{pgfscope}%
\definecolor{textcolor}{rgb}{0.150000,0.150000,0.150000}%
\pgfsetstrokecolor{textcolor}%
\pgfsetfillcolor{textcolor}%
\pgftext[x=7.619771in,y=0.902601in,left,base]{\color{textcolor}\sffamily\fontsize{10.000000}{12.000000}\selectfont test}%
\end{pgfscope}%
\begin{pgfscope}%
\pgfsetbuttcap%
\pgfsetmiterjoin%
\definecolor{currentfill}{rgb}{1.000000,1.000000,1.000000}%
\pgfsetfillcolor{currentfill}%
\pgfsetlinewidth{0.000000pt}%
\definecolor{currentstroke}{rgb}{0.000000,0.000000,0.000000}%
\pgfsetstrokecolor{currentstroke}%
\pgfsetstrokeopacity{0.000000}%
\pgfsetdash{}{0pt}%
\pgfpathmoveto{\pgfqpoint{12.211765in}{0.750000in}}%
\pgfpathlineto{\pgfqpoint{14.400000in}{0.750000in}}%
\pgfpathlineto{\pgfqpoint{14.400000in}{1.728947in}}%
\pgfpathlineto{\pgfqpoint{12.211765in}{1.728947in}}%
\pgfpathclose%
\pgfusepath{fill}%
\end{pgfscope}%
\begin{pgfscope}%
\pgfpathrectangle{\pgfqpoint{12.211765in}{0.750000in}}{\pgfqpoint{2.188235in}{0.978947in}} %
\pgfusepath{clip}%
\pgfsetroundcap%
\pgfsetroundjoin%
\pgfsetlinewidth{1.003750pt}%
\definecolor{currentstroke}{rgb}{0.800000,0.800000,0.800000}%
\pgfsetstrokecolor{currentstroke}%
\pgfsetdash{}{0pt}%
\pgfpathmoveto{\pgfqpoint{12.211765in}{0.750000in}}%
\pgfpathlineto{\pgfqpoint{12.211765in}{1.728947in}}%
\pgfusepath{stroke}%
\end{pgfscope}%
\begin{pgfscope}%
\definecolor{textcolor}{rgb}{0.150000,0.150000,0.150000}%
\pgfsetstrokecolor{textcolor}%
\pgfsetfillcolor{textcolor}%
\pgftext[x=12.211765in,y=0.652778in,,top]{\color{textcolor}\sffamily\fontsize{10.000000}{12.000000}\selectfont \(\displaystyle -2.0\)}%
\end{pgfscope}%
\begin{pgfscope}%
\pgfpathrectangle{\pgfqpoint{12.211765in}{0.750000in}}{\pgfqpoint{2.188235in}{0.978947in}} %
\pgfusepath{clip}%
\pgfsetroundcap%
\pgfsetroundjoin%
\pgfsetlinewidth{1.003750pt}%
\definecolor{currentstroke}{rgb}{0.800000,0.800000,0.800000}%
\pgfsetstrokecolor{currentstroke}%
\pgfsetdash{}{0pt}%
\pgfpathmoveto{\pgfqpoint{12.485294in}{0.750000in}}%
\pgfpathlineto{\pgfqpoint{12.485294in}{1.728947in}}%
\pgfusepath{stroke}%
\end{pgfscope}%
\begin{pgfscope}%
\definecolor{textcolor}{rgb}{0.150000,0.150000,0.150000}%
\pgfsetstrokecolor{textcolor}%
\pgfsetfillcolor{textcolor}%
\pgftext[x=12.485294in,y=0.652778in,,top]{\color{textcolor}\sffamily\fontsize{10.000000}{12.000000}\selectfont \(\displaystyle -1.5\)}%
\end{pgfscope}%
\begin{pgfscope}%
\pgfpathrectangle{\pgfqpoint{12.211765in}{0.750000in}}{\pgfqpoint{2.188235in}{0.978947in}} %
\pgfusepath{clip}%
\pgfsetroundcap%
\pgfsetroundjoin%
\pgfsetlinewidth{1.003750pt}%
\definecolor{currentstroke}{rgb}{0.800000,0.800000,0.800000}%
\pgfsetstrokecolor{currentstroke}%
\pgfsetdash{}{0pt}%
\pgfpathmoveto{\pgfqpoint{12.758824in}{0.750000in}}%
\pgfpathlineto{\pgfqpoint{12.758824in}{1.728947in}}%
\pgfusepath{stroke}%
\end{pgfscope}%
\begin{pgfscope}%
\definecolor{textcolor}{rgb}{0.150000,0.150000,0.150000}%
\pgfsetstrokecolor{textcolor}%
\pgfsetfillcolor{textcolor}%
\pgftext[x=12.758824in,y=0.652778in,,top]{\color{textcolor}\sffamily\fontsize{10.000000}{12.000000}\selectfont \(\displaystyle -1.0\)}%
\end{pgfscope}%
\begin{pgfscope}%
\pgfpathrectangle{\pgfqpoint{12.211765in}{0.750000in}}{\pgfqpoint{2.188235in}{0.978947in}} %
\pgfusepath{clip}%
\pgfsetroundcap%
\pgfsetroundjoin%
\pgfsetlinewidth{1.003750pt}%
\definecolor{currentstroke}{rgb}{0.800000,0.800000,0.800000}%
\pgfsetstrokecolor{currentstroke}%
\pgfsetdash{}{0pt}%
\pgfpathmoveto{\pgfqpoint{13.032353in}{0.750000in}}%
\pgfpathlineto{\pgfqpoint{13.032353in}{1.728947in}}%
\pgfusepath{stroke}%
\end{pgfscope}%
\begin{pgfscope}%
\definecolor{textcolor}{rgb}{0.150000,0.150000,0.150000}%
\pgfsetstrokecolor{textcolor}%
\pgfsetfillcolor{textcolor}%
\pgftext[x=13.032353in,y=0.652778in,,top]{\color{textcolor}\sffamily\fontsize{10.000000}{12.000000}\selectfont \(\displaystyle -0.5\)}%
\end{pgfscope}%
\begin{pgfscope}%
\pgfpathrectangle{\pgfqpoint{12.211765in}{0.750000in}}{\pgfqpoint{2.188235in}{0.978947in}} %
\pgfusepath{clip}%
\pgfsetroundcap%
\pgfsetroundjoin%
\pgfsetlinewidth{1.003750pt}%
\definecolor{currentstroke}{rgb}{0.800000,0.800000,0.800000}%
\pgfsetstrokecolor{currentstroke}%
\pgfsetdash{}{0pt}%
\pgfpathmoveto{\pgfqpoint{13.305882in}{0.750000in}}%
\pgfpathlineto{\pgfqpoint{13.305882in}{1.728947in}}%
\pgfusepath{stroke}%
\end{pgfscope}%
\begin{pgfscope}%
\definecolor{textcolor}{rgb}{0.150000,0.150000,0.150000}%
\pgfsetstrokecolor{textcolor}%
\pgfsetfillcolor{textcolor}%
\pgftext[x=13.305882in,y=0.652778in,,top]{\color{textcolor}\sffamily\fontsize{10.000000}{12.000000}\selectfont \(\displaystyle 0.0\)}%
\end{pgfscope}%
\begin{pgfscope}%
\pgfpathrectangle{\pgfqpoint{12.211765in}{0.750000in}}{\pgfqpoint{2.188235in}{0.978947in}} %
\pgfusepath{clip}%
\pgfsetroundcap%
\pgfsetroundjoin%
\pgfsetlinewidth{1.003750pt}%
\definecolor{currentstroke}{rgb}{0.800000,0.800000,0.800000}%
\pgfsetstrokecolor{currentstroke}%
\pgfsetdash{}{0pt}%
\pgfpathmoveto{\pgfqpoint{13.579412in}{0.750000in}}%
\pgfpathlineto{\pgfqpoint{13.579412in}{1.728947in}}%
\pgfusepath{stroke}%
\end{pgfscope}%
\begin{pgfscope}%
\definecolor{textcolor}{rgb}{0.150000,0.150000,0.150000}%
\pgfsetstrokecolor{textcolor}%
\pgfsetfillcolor{textcolor}%
\pgftext[x=13.579412in,y=0.652778in,,top]{\color{textcolor}\sffamily\fontsize{10.000000}{12.000000}\selectfont \(\displaystyle 0.5\)}%
\end{pgfscope}%
\begin{pgfscope}%
\pgfpathrectangle{\pgfqpoint{12.211765in}{0.750000in}}{\pgfqpoint{2.188235in}{0.978947in}} %
\pgfusepath{clip}%
\pgfsetroundcap%
\pgfsetroundjoin%
\pgfsetlinewidth{1.003750pt}%
\definecolor{currentstroke}{rgb}{0.800000,0.800000,0.800000}%
\pgfsetstrokecolor{currentstroke}%
\pgfsetdash{}{0pt}%
\pgfpathmoveto{\pgfqpoint{13.852941in}{0.750000in}}%
\pgfpathlineto{\pgfqpoint{13.852941in}{1.728947in}}%
\pgfusepath{stroke}%
\end{pgfscope}%
\begin{pgfscope}%
\definecolor{textcolor}{rgb}{0.150000,0.150000,0.150000}%
\pgfsetstrokecolor{textcolor}%
\pgfsetfillcolor{textcolor}%
\pgftext[x=13.852941in,y=0.652778in,,top]{\color{textcolor}\sffamily\fontsize{10.000000}{12.000000}\selectfont \(\displaystyle 1.0\)}%
\end{pgfscope}%
\begin{pgfscope}%
\pgfpathrectangle{\pgfqpoint{12.211765in}{0.750000in}}{\pgfqpoint{2.188235in}{0.978947in}} %
\pgfusepath{clip}%
\pgfsetroundcap%
\pgfsetroundjoin%
\pgfsetlinewidth{1.003750pt}%
\definecolor{currentstroke}{rgb}{0.800000,0.800000,0.800000}%
\pgfsetstrokecolor{currentstroke}%
\pgfsetdash{}{0pt}%
\pgfpathmoveto{\pgfqpoint{14.126471in}{0.750000in}}%
\pgfpathlineto{\pgfqpoint{14.126471in}{1.728947in}}%
\pgfusepath{stroke}%
\end{pgfscope}%
\begin{pgfscope}%
\definecolor{textcolor}{rgb}{0.150000,0.150000,0.150000}%
\pgfsetstrokecolor{textcolor}%
\pgfsetfillcolor{textcolor}%
\pgftext[x=14.126471in,y=0.652778in,,top]{\color{textcolor}\sffamily\fontsize{10.000000}{12.000000}\selectfont \(\displaystyle 1.5\)}%
\end{pgfscope}%
\begin{pgfscope}%
\pgfpathrectangle{\pgfqpoint{12.211765in}{0.750000in}}{\pgfqpoint{2.188235in}{0.978947in}} %
\pgfusepath{clip}%
\pgfsetroundcap%
\pgfsetroundjoin%
\pgfsetlinewidth{1.003750pt}%
\definecolor{currentstroke}{rgb}{0.800000,0.800000,0.800000}%
\pgfsetstrokecolor{currentstroke}%
\pgfsetdash{}{0pt}%
\pgfpathmoveto{\pgfqpoint{14.400000in}{0.750000in}}%
\pgfpathlineto{\pgfqpoint{14.400000in}{1.728947in}}%
\pgfusepath{stroke}%
\end{pgfscope}%
\begin{pgfscope}%
\definecolor{textcolor}{rgb}{0.150000,0.150000,0.150000}%
\pgfsetstrokecolor{textcolor}%
\pgfsetfillcolor{textcolor}%
\pgftext[x=14.400000in,y=0.652778in,,top]{\color{textcolor}\sffamily\fontsize{10.000000}{12.000000}\selectfont \(\displaystyle 2.0\)}%
\end{pgfscope}%
\begin{pgfscope}%
\definecolor{textcolor}{rgb}{0.150000,0.150000,0.150000}%
\pgfsetstrokecolor{textcolor}%
\pgfsetfillcolor{textcolor}%
\pgftext[x=13.305882in,y=0.456313in,,top]{\color{textcolor}\sffamily\fontsize{11.000000}{13.200000}\selectfont \(\displaystyle j\)}%
\end{pgfscope}%
\begin{pgfscope}%
\definecolor{textcolor}{rgb}{0.150000,0.150000,0.150000}%
\pgfsetstrokecolor{textcolor}%
\pgfsetfillcolor{textcolor}%
\pgftext[x=14.400000in,y=0.484090in,right,top]{\color{textcolor}\sffamily\fontsize{10.000000}{12.000000}\selectfont \(\displaystyle \times10^{5}\)}%
\end{pgfscope}%
\begin{pgfscope}%
\pgfpathrectangle{\pgfqpoint{12.211765in}{0.750000in}}{\pgfqpoint{2.188235in}{0.978947in}} %
\pgfusepath{clip}%
\pgfsetroundcap%
\pgfsetroundjoin%
\pgfsetlinewidth{1.003750pt}%
\definecolor{currentstroke}{rgb}{0.800000,0.800000,0.800000}%
\pgfsetstrokecolor{currentstroke}%
\pgfsetdash{}{0pt}%
\pgfpathmoveto{\pgfqpoint{12.211765in}{0.750000in}}%
\pgfpathlineto{\pgfqpoint{14.400000in}{0.750000in}}%
\pgfusepath{stroke}%
\end{pgfscope}%
\begin{pgfscope}%
\definecolor{textcolor}{rgb}{0.150000,0.150000,0.150000}%
\pgfsetstrokecolor{textcolor}%
\pgfsetfillcolor{textcolor}%
\pgftext[x=12.114542in,y=0.750000in,right,]{\color{textcolor}\sffamily\fontsize{10.000000}{12.000000}\selectfont \(\displaystyle 0\)}%
\end{pgfscope}%
\begin{pgfscope}%
\pgfpathrectangle{\pgfqpoint{12.211765in}{0.750000in}}{\pgfqpoint{2.188235in}{0.978947in}} %
\pgfusepath{clip}%
\pgfsetroundcap%
\pgfsetroundjoin%
\pgfsetlinewidth{1.003750pt}%
\definecolor{currentstroke}{rgb}{0.800000,0.800000,0.800000}%
\pgfsetstrokecolor{currentstroke}%
\pgfsetdash{}{0pt}%
\pgfpathmoveto{\pgfqpoint{12.211765in}{0.994737in}}%
\pgfpathlineto{\pgfqpoint{14.400000in}{0.994737in}}%
\pgfusepath{stroke}%
\end{pgfscope}%
\begin{pgfscope}%
\definecolor{textcolor}{rgb}{0.150000,0.150000,0.150000}%
\pgfsetstrokecolor{textcolor}%
\pgfsetfillcolor{textcolor}%
\pgftext[x=12.114542in,y=0.994737in,right,]{\color{textcolor}\sffamily\fontsize{10.000000}{12.000000}\selectfont \(\displaystyle 50\)}%
\end{pgfscope}%
\begin{pgfscope}%
\pgfpathrectangle{\pgfqpoint{12.211765in}{0.750000in}}{\pgfqpoint{2.188235in}{0.978947in}} %
\pgfusepath{clip}%
\pgfsetroundcap%
\pgfsetroundjoin%
\pgfsetlinewidth{1.003750pt}%
\definecolor{currentstroke}{rgb}{0.800000,0.800000,0.800000}%
\pgfsetstrokecolor{currentstroke}%
\pgfsetdash{}{0pt}%
\pgfpathmoveto{\pgfqpoint{12.211765in}{1.239474in}}%
\pgfpathlineto{\pgfqpoint{14.400000in}{1.239474in}}%
\pgfusepath{stroke}%
\end{pgfscope}%
\begin{pgfscope}%
\definecolor{textcolor}{rgb}{0.150000,0.150000,0.150000}%
\pgfsetstrokecolor{textcolor}%
\pgfsetfillcolor{textcolor}%
\pgftext[x=12.114542in,y=1.239474in,right,]{\color{textcolor}\sffamily\fontsize{10.000000}{12.000000}\selectfont \(\displaystyle 100\)}%
\end{pgfscope}%
\begin{pgfscope}%
\pgfpathrectangle{\pgfqpoint{12.211765in}{0.750000in}}{\pgfqpoint{2.188235in}{0.978947in}} %
\pgfusepath{clip}%
\pgfsetroundcap%
\pgfsetroundjoin%
\pgfsetlinewidth{1.003750pt}%
\definecolor{currentstroke}{rgb}{0.800000,0.800000,0.800000}%
\pgfsetstrokecolor{currentstroke}%
\pgfsetdash{}{0pt}%
\pgfpathmoveto{\pgfqpoint{12.211765in}{1.484211in}}%
\pgfpathlineto{\pgfqpoint{14.400000in}{1.484211in}}%
\pgfusepath{stroke}%
\end{pgfscope}%
\begin{pgfscope}%
\definecolor{textcolor}{rgb}{0.150000,0.150000,0.150000}%
\pgfsetstrokecolor{textcolor}%
\pgfsetfillcolor{textcolor}%
\pgftext[x=12.114542in,y=1.484211in,right,]{\color{textcolor}\sffamily\fontsize{10.000000}{12.000000}\selectfont \(\displaystyle 150\)}%
\end{pgfscope}%
\begin{pgfscope}%
\pgfpathrectangle{\pgfqpoint{12.211765in}{0.750000in}}{\pgfqpoint{2.188235in}{0.978947in}} %
\pgfusepath{clip}%
\pgfsetroundcap%
\pgfsetroundjoin%
\pgfsetlinewidth{1.003750pt}%
\definecolor{currentstroke}{rgb}{0.800000,0.800000,0.800000}%
\pgfsetstrokecolor{currentstroke}%
\pgfsetdash{}{0pt}%
\pgfpathmoveto{\pgfqpoint{12.211765in}{1.728947in}}%
\pgfpathlineto{\pgfqpoint{14.400000in}{1.728947in}}%
\pgfusepath{stroke}%
\end{pgfscope}%
\begin{pgfscope}%
\definecolor{textcolor}{rgb}{0.150000,0.150000,0.150000}%
\pgfsetstrokecolor{textcolor}%
\pgfsetfillcolor{textcolor}%
\pgftext[x=12.114542in,y=1.728947in,right,]{\color{textcolor}\sffamily\fontsize{10.000000}{12.000000}\selectfont \(\displaystyle 200\)}%
\end{pgfscope}%
\begin{pgfscope}%
\definecolor{textcolor}{rgb}{0.150000,0.150000,0.150000}%
\pgfsetstrokecolor{textcolor}%
\pgfsetfillcolor{textcolor}%
\pgftext[x=11.836764in,y=1.239474in,,bottom,rotate=90.000000]{\color{textcolor}\sffamily\fontsize{11.000000}{13.200000}\selectfont \(\displaystyle \theta^{\parallel}_j\)}%
\end{pgfscope}%
\begin{pgfscope}%
\pgfpathrectangle{\pgfqpoint{12.211765in}{0.750000in}}{\pgfqpoint{2.188235in}{0.978947in}} %
\pgfusepath{clip}%
\pgfsetroundcap%
\pgfsetroundjoin%
\pgfsetlinewidth{1.756562pt}%
\definecolor{currentstroke}{rgb}{0.298039,0.447059,0.690196}%
\pgfsetstrokecolor{currentstroke}%
\pgfsetdash{}{0pt}%
\pgfpathmoveto{\pgfqpoint{13.318583in}{0.750000in}}%
\pgfpathlineto{\pgfqpoint{13.319716in}{0.754895in}}%
\pgfpathlineto{\pgfqpoint{13.505278in}{0.759789in}}%
\pgfpathlineto{\pgfqpoint{13.088970in}{0.764684in}}%
\pgfpathlineto{\pgfqpoint{13.009376in}{0.769579in}}%
\pgfpathlineto{\pgfqpoint{13.306678in}{0.774474in}}%
\pgfpathlineto{\pgfqpoint{12.835495in}{0.779368in}}%
\pgfpathlineto{\pgfqpoint{13.427294in}{0.789158in}}%
\pgfpathlineto{\pgfqpoint{12.873136in}{0.794053in}}%
\pgfpathlineto{\pgfqpoint{13.296390in}{0.798947in}}%
\pgfpathlineto{\pgfqpoint{13.125624in}{0.803842in}}%
\pgfpathlineto{\pgfqpoint{13.722766in}{0.813632in}}%
\pgfpathlineto{\pgfqpoint{13.222671in}{0.818526in}}%
\pgfpathlineto{\pgfqpoint{13.009751in}{0.823421in}}%
\pgfpathlineto{\pgfqpoint{14.262259in}{0.828316in}}%
\pgfpathlineto{\pgfqpoint{12.589164in}{0.833211in}}%
\pgfpathlineto{\pgfqpoint{12.347675in}{0.838105in}}%
\pgfpathlineto{\pgfqpoint{13.300604in}{0.843000in}}%
\pgfpathlineto{\pgfqpoint{13.283917in}{0.847895in}}%
\pgfpathlineto{\pgfqpoint{13.643233in}{0.852789in}}%
\pgfpathlineto{\pgfqpoint{13.609389in}{0.857684in}}%
\pgfpathlineto{\pgfqpoint{14.079650in}{0.862579in}}%
\pgfpathlineto{\pgfqpoint{13.331759in}{0.867474in}}%
\pgfpathlineto{\pgfqpoint{13.431527in}{0.872368in}}%
\pgfpathlineto{\pgfqpoint{13.283798in}{0.877263in}}%
\pgfpathlineto{\pgfqpoint{13.311260in}{0.882158in}}%
\pgfpathlineto{\pgfqpoint{13.028419in}{0.887053in}}%
\pgfpathlineto{\pgfqpoint{13.205100in}{0.891947in}}%
\pgfpathlineto{\pgfqpoint{13.464490in}{0.896842in}}%
\pgfpathlineto{\pgfqpoint{13.372658in}{0.901737in}}%
\pgfpathlineto{\pgfqpoint{13.379991in}{0.906632in}}%
\pgfpathlineto{\pgfqpoint{13.275185in}{0.911526in}}%
\pgfpathlineto{\pgfqpoint{13.307792in}{0.916421in}}%
\pgfpathlineto{\pgfqpoint{13.302084in}{0.921316in}}%
\pgfpathlineto{\pgfqpoint{13.321845in}{0.926211in}}%
\pgfpathlineto{\pgfqpoint{13.306216in}{0.931105in}}%
\pgfpathlineto{\pgfqpoint{13.320266in}{0.936000in}}%
\pgfpathlineto{\pgfqpoint{13.269674in}{0.945789in}}%
\pgfpathlineto{\pgfqpoint{13.293961in}{0.950684in}}%
\pgfpathlineto{\pgfqpoint{13.253555in}{0.955579in}}%
\pgfpathlineto{\pgfqpoint{13.284425in}{0.960474in}}%
\pgfpathlineto{\pgfqpoint{13.264993in}{0.965368in}}%
\pgfpathlineto{\pgfqpoint{13.326918in}{0.970263in}}%
\pgfpathlineto{\pgfqpoint{13.341694in}{0.975158in}}%
\pgfpathlineto{\pgfqpoint{13.375413in}{0.980053in}}%
\pgfpathlineto{\pgfqpoint{13.264599in}{0.984947in}}%
\pgfpathlineto{\pgfqpoint{13.368765in}{0.989842in}}%
\pgfpathlineto{\pgfqpoint{13.347897in}{0.994737in}}%
\pgfpathlineto{\pgfqpoint{13.239606in}{0.999632in}}%
\pgfpathlineto{\pgfqpoint{13.315040in}{1.004526in}}%
\pgfpathlineto{\pgfqpoint{13.198593in}{1.009421in}}%
\pgfpathlineto{\pgfqpoint{13.411475in}{1.014316in}}%
\pgfpathlineto{\pgfqpoint{13.348855in}{1.019211in}}%
\pgfpathlineto{\pgfqpoint{13.365114in}{1.024105in}}%
\pgfpathlineto{\pgfqpoint{13.250677in}{1.029000in}}%
\pgfpathlineto{\pgfqpoint{13.195021in}{1.033895in}}%
\pgfpathlineto{\pgfqpoint{13.350873in}{1.038789in}}%
\pgfpathlineto{\pgfqpoint{13.319205in}{1.043684in}}%
\pgfpathlineto{\pgfqpoint{13.250544in}{1.048579in}}%
\pgfpathlineto{\pgfqpoint{13.212046in}{1.053474in}}%
\pgfpathlineto{\pgfqpoint{13.578251in}{1.063263in}}%
\pgfpathlineto{\pgfqpoint{13.405082in}{1.068158in}}%
\pgfpathlineto{\pgfqpoint{13.336883in}{1.077947in}}%
\pgfpathlineto{\pgfqpoint{13.262717in}{1.082842in}}%
\pgfpathlineto{\pgfqpoint{13.292784in}{1.087737in}}%
\pgfpathlineto{\pgfqpoint{13.277665in}{1.092632in}}%
\pgfpathlineto{\pgfqpoint{13.279471in}{1.097526in}}%
\pgfpathlineto{\pgfqpoint{13.299348in}{1.102421in}}%
\pgfpathlineto{\pgfqpoint{13.426340in}{1.107316in}}%
\pgfpathlineto{\pgfqpoint{13.360732in}{1.112211in}}%
\pgfpathlineto{\pgfqpoint{13.319368in}{1.117105in}}%
\pgfpathlineto{\pgfqpoint{13.315122in}{1.122000in}}%
\pgfpathlineto{\pgfqpoint{13.347821in}{1.126895in}}%
\pgfpathlineto{\pgfqpoint{13.310284in}{1.131789in}}%
\pgfpathlineto{\pgfqpoint{13.394929in}{1.136684in}}%
\pgfpathlineto{\pgfqpoint{13.386711in}{1.141579in}}%
\pgfpathlineto{\pgfqpoint{13.312265in}{1.146474in}}%
\pgfpathlineto{\pgfqpoint{13.318126in}{1.151368in}}%
\pgfpathlineto{\pgfqpoint{13.321795in}{1.156263in}}%
\pgfpathlineto{\pgfqpoint{13.338412in}{1.161158in}}%
\pgfpathlineto{\pgfqpoint{13.289559in}{1.166053in}}%
\pgfpathlineto{\pgfqpoint{13.292515in}{1.170947in}}%
\pgfpathlineto{\pgfqpoint{13.373984in}{1.175842in}}%
\pgfpathlineto{\pgfqpoint{13.197679in}{1.180737in}}%
\pgfpathlineto{\pgfqpoint{13.194352in}{1.185632in}}%
\pgfpathlineto{\pgfqpoint{13.279855in}{1.190526in}}%
\pgfpathlineto{\pgfqpoint{13.304310in}{1.195421in}}%
\pgfpathlineto{\pgfqpoint{13.381147in}{1.200316in}}%
\pgfpathlineto{\pgfqpoint{13.309790in}{1.205211in}}%
\pgfpathlineto{\pgfqpoint{13.253022in}{1.215000in}}%
\pgfpathlineto{\pgfqpoint{13.337301in}{1.219895in}}%
\pgfpathlineto{\pgfqpoint{13.235095in}{1.224789in}}%
\pgfpathlineto{\pgfqpoint{13.346844in}{1.229684in}}%
\pgfpathlineto{\pgfqpoint{13.379070in}{1.234579in}}%
\pgfpathlineto{\pgfqpoint{13.232800in}{1.249263in}}%
\pgfpathlineto{\pgfqpoint{13.290115in}{1.254158in}}%
\pgfpathlineto{\pgfqpoint{13.262587in}{1.259053in}}%
\pgfpathlineto{\pgfqpoint{13.352596in}{1.263947in}}%
\pgfpathlineto{\pgfqpoint{13.323984in}{1.268842in}}%
\pgfpathlineto{\pgfqpoint{13.309931in}{1.273737in}}%
\pgfpathlineto{\pgfqpoint{13.329651in}{1.278632in}}%
\pgfpathlineto{\pgfqpoint{13.327334in}{1.283526in}}%
\pgfpathlineto{\pgfqpoint{13.295086in}{1.288421in}}%
\pgfpathlineto{\pgfqpoint{13.304888in}{1.293316in}}%
\pgfpathlineto{\pgfqpoint{13.325003in}{1.298211in}}%
\pgfpathlineto{\pgfqpoint{13.260809in}{1.303105in}}%
\pgfpathlineto{\pgfqpoint{13.319008in}{1.308000in}}%
\pgfpathlineto{\pgfqpoint{13.314798in}{1.312895in}}%
\pgfpathlineto{\pgfqpoint{13.259183in}{1.317789in}}%
\pgfpathlineto{\pgfqpoint{13.321789in}{1.322684in}}%
\pgfpathlineto{\pgfqpoint{13.287730in}{1.327579in}}%
\pgfpathlineto{\pgfqpoint{13.305615in}{1.332474in}}%
\pgfpathlineto{\pgfqpoint{13.288911in}{1.337368in}}%
\pgfpathlineto{\pgfqpoint{13.352404in}{1.342263in}}%
\pgfpathlineto{\pgfqpoint{13.267461in}{1.347158in}}%
\pgfpathlineto{\pgfqpoint{13.289447in}{1.352053in}}%
\pgfpathlineto{\pgfqpoint{13.273537in}{1.356947in}}%
\pgfpathlineto{\pgfqpoint{13.286382in}{1.361842in}}%
\pgfpathlineto{\pgfqpoint{13.284816in}{1.366737in}}%
\pgfpathlineto{\pgfqpoint{13.307654in}{1.371632in}}%
\pgfpathlineto{\pgfqpoint{13.313491in}{1.376526in}}%
\pgfpathlineto{\pgfqpoint{13.334709in}{1.381421in}}%
\pgfpathlineto{\pgfqpoint{13.376685in}{1.386316in}}%
\pgfpathlineto{\pgfqpoint{13.335922in}{1.391211in}}%
\pgfpathlineto{\pgfqpoint{13.311644in}{1.396105in}}%
\pgfpathlineto{\pgfqpoint{13.310785in}{1.401000in}}%
\pgfpathlineto{\pgfqpoint{13.299466in}{1.405895in}}%
\pgfpathlineto{\pgfqpoint{13.308094in}{1.410789in}}%
\pgfpathlineto{\pgfqpoint{13.351926in}{1.415684in}}%
\pgfpathlineto{\pgfqpoint{13.276298in}{1.420579in}}%
\pgfpathlineto{\pgfqpoint{13.301670in}{1.425474in}}%
\pgfpathlineto{\pgfqpoint{13.339814in}{1.430368in}}%
\pgfpathlineto{\pgfqpoint{13.308825in}{1.435263in}}%
\pgfpathlineto{\pgfqpoint{13.334439in}{1.440158in}}%
\pgfpathlineto{\pgfqpoint{13.316049in}{1.445053in}}%
\pgfpathlineto{\pgfqpoint{13.326517in}{1.449947in}}%
\pgfpathlineto{\pgfqpoint{13.312145in}{1.454842in}}%
\pgfpathlineto{\pgfqpoint{13.337528in}{1.459737in}}%
\pgfpathlineto{\pgfqpoint{13.338402in}{1.464632in}}%
\pgfpathlineto{\pgfqpoint{13.310576in}{1.469526in}}%
\pgfpathlineto{\pgfqpoint{13.309865in}{1.474421in}}%
\pgfpathlineto{\pgfqpoint{13.299039in}{1.479316in}}%
\pgfpathlineto{\pgfqpoint{13.302382in}{1.484211in}}%
\pgfpathlineto{\pgfqpoint{13.366004in}{1.489105in}}%
\pgfpathlineto{\pgfqpoint{13.380912in}{1.494000in}}%
\pgfpathlineto{\pgfqpoint{13.349933in}{1.498895in}}%
\pgfpathlineto{\pgfqpoint{13.296258in}{1.503789in}}%
\pgfpathlineto{\pgfqpoint{13.285974in}{1.508684in}}%
\pgfpathlineto{\pgfqpoint{13.299521in}{1.513579in}}%
\pgfpathlineto{\pgfqpoint{13.295012in}{1.518474in}}%
\pgfpathlineto{\pgfqpoint{13.313393in}{1.523368in}}%
\pgfpathlineto{\pgfqpoint{13.316415in}{1.528263in}}%
\pgfpathlineto{\pgfqpoint{13.317807in}{1.533158in}}%
\pgfpathlineto{\pgfqpoint{13.266680in}{1.538053in}}%
\pgfpathlineto{\pgfqpoint{13.289385in}{1.542947in}}%
\pgfpathlineto{\pgfqpoint{13.321376in}{1.547842in}}%
\pgfpathlineto{\pgfqpoint{13.274531in}{1.557632in}}%
\pgfpathlineto{\pgfqpoint{13.314857in}{1.562526in}}%
\pgfpathlineto{\pgfqpoint{13.303647in}{1.567421in}}%
\pgfpathlineto{\pgfqpoint{13.348105in}{1.572316in}}%
\pgfpathlineto{\pgfqpoint{13.273644in}{1.577211in}}%
\pgfpathlineto{\pgfqpoint{13.296569in}{1.582105in}}%
\pgfpathlineto{\pgfqpoint{13.366367in}{1.587000in}}%
\pgfpathlineto{\pgfqpoint{13.264048in}{1.591895in}}%
\pgfpathlineto{\pgfqpoint{13.323509in}{1.596789in}}%
\pgfpathlineto{\pgfqpoint{13.314195in}{1.601684in}}%
\pgfpathlineto{\pgfqpoint{13.302483in}{1.606579in}}%
\pgfpathlineto{\pgfqpoint{13.367083in}{1.611474in}}%
\pgfpathlineto{\pgfqpoint{13.357510in}{1.616368in}}%
\pgfpathlineto{\pgfqpoint{13.362290in}{1.621263in}}%
\pgfpathlineto{\pgfqpoint{13.294772in}{1.626158in}}%
\pgfpathlineto{\pgfqpoint{13.287740in}{1.631053in}}%
\pgfpathlineto{\pgfqpoint{13.321781in}{1.635947in}}%
\pgfpathlineto{\pgfqpoint{13.306540in}{1.640842in}}%
\pgfpathlineto{\pgfqpoint{13.297179in}{1.645737in}}%
\pgfpathlineto{\pgfqpoint{13.302243in}{1.650632in}}%
\pgfpathlineto{\pgfqpoint{13.378972in}{1.655526in}}%
\pgfpathlineto{\pgfqpoint{13.307493in}{1.660421in}}%
\pgfpathlineto{\pgfqpoint{13.332046in}{1.665316in}}%
\pgfpathlineto{\pgfqpoint{13.292913in}{1.670211in}}%
\pgfpathlineto{\pgfqpoint{13.306141in}{1.675105in}}%
\pgfpathlineto{\pgfqpoint{13.359208in}{1.680000in}}%
\pgfpathlineto{\pgfqpoint{13.333258in}{1.684895in}}%
\pgfpathlineto{\pgfqpoint{13.274190in}{1.689789in}}%
\pgfpathlineto{\pgfqpoint{13.281137in}{1.694684in}}%
\pgfpathlineto{\pgfqpoint{13.375858in}{1.699579in}}%
\pgfpathlineto{\pgfqpoint{13.297677in}{1.704474in}}%
\pgfpathlineto{\pgfqpoint{13.310534in}{1.709368in}}%
\pgfpathlineto{\pgfqpoint{13.240434in}{1.714263in}}%
\pgfpathlineto{\pgfqpoint{13.245646in}{1.719158in}}%
\pgfpathlineto{\pgfqpoint{13.285100in}{1.724053in}}%
\pgfpathlineto{\pgfqpoint{13.285100in}{1.724053in}}%
\pgfusepath{stroke}%
\end{pgfscope}%
\begin{pgfscope}%
\pgfsetrectcap%
\pgfsetmiterjoin%
\pgfsetlinewidth{1.003750pt}%
\definecolor{currentstroke}{rgb}{0.800000,0.800000,0.800000}%
\pgfsetstrokecolor{currentstroke}%
\pgfsetdash{}{0pt}%
\pgfpathmoveto{\pgfqpoint{12.211765in}{0.750000in}}%
\pgfpathlineto{\pgfqpoint{12.211765in}{1.728947in}}%
\pgfusepath{stroke}%
\end{pgfscope}%
\begin{pgfscope}%
\pgfsetrectcap%
\pgfsetmiterjoin%
\pgfsetlinewidth{1.003750pt}%
\definecolor{currentstroke}{rgb}{0.800000,0.800000,0.800000}%
\pgfsetstrokecolor{currentstroke}%
\pgfsetdash{}{0pt}%
\pgfpathmoveto{\pgfqpoint{14.400000in}{0.750000in}}%
\pgfpathlineto{\pgfqpoint{14.400000in}{1.728947in}}%
\pgfusepath{stroke}%
\end{pgfscope}%
\begin{pgfscope}%
\pgfsetrectcap%
\pgfsetmiterjoin%
\pgfsetlinewidth{1.003750pt}%
\definecolor{currentstroke}{rgb}{0.800000,0.800000,0.800000}%
\pgfsetstrokecolor{currentstroke}%
\pgfsetdash{}{0pt}%
\pgfpathmoveto{\pgfqpoint{12.211765in}{1.728947in}}%
\pgfpathlineto{\pgfqpoint{14.400000in}{1.728947in}}%
\pgfusepath{stroke}%
\end{pgfscope}%
\begin{pgfscope}%
\pgfsetrectcap%
\pgfsetmiterjoin%
\pgfsetlinewidth{1.003750pt}%
\definecolor{currentstroke}{rgb}{0.800000,0.800000,0.800000}%
\pgfsetstrokecolor{currentstroke}%
\pgfsetdash{}{0pt}%
\pgfpathmoveto{\pgfqpoint{12.211765in}{0.750000in}}%
\pgfpathlineto{\pgfqpoint{14.400000in}{0.750000in}}%
\pgfusepath{stroke}%
\end{pgfscope}%
\end{pgfpicture}%
\makeatother%
\endgroup%
}')}
\caption[ORFF Representer theorem]{ORFF Representer theorem. We trained a first model named $\tildeK{\omega}$ following}
\label{fig:representer}
\end{figure}
\end{landscape}}

\subsection{Efficient linear operators}
When developping \cref{alg:close_form} we considered that the feature map $\tildePhi{\omega}(x)$ was a matrix from $\mathbb{R}^u$ to $\mathbb{R}^{r}$ for all $x\in\mathcal{X}$, and therefore that computing $\tildePhi{\omega}(x)^T \theta$ has a time complexity of $O(r^2u)$. While this holds true in the most generic senario, in many cases the feature maps presents some structure or sparsity allowing to reduce the computational cost of evaluating the feature map. We focus on the \acl{ORFF} given by \cref{alg:ORFF_construction}, developped in \cref{subsec:building_ORFF} and \cref{subsec:examples_ORFF} and treat the decomposable kernel, the curl-free kernel and the divergence-free kernel as an example. We recall that if $\mathcal{U}'=\mathbb{R}^{u'}$ and $\mathcal{U}=\mathbb{R}^u$, then $\tildeH{\omega}=\mathbb{R}^{2Du'}$ thus the \acl{ORFF}s given in \cref{ch:operator-valued_random_fourier_features} have the form
\begin{dmath*}
\begin{cases}\tildePhi{\omega}(x) \in\mathcal{L}\left(\mathbb{R}^u, \mathbb{R}^{2Du'}\right) &:  y \mapsto \frac{1}{\sqrt{D}}\Vect_{j=1}^D\pairing{x, \omega_j}B(\omega_j)^T y \\ \tildePhi{\omega}(x)^T \in\mathcal{L}\left(\mathbb{R}^{2Du'}, \mathbb{R}^u\right) &: \theta \mapsto \frac{1}{\sqrt{D}} \sum_{j=1}^D \pairing{x, \omega_j}B(\omega_j)\theta_j \end{cases},
\end{dmath*}
where $\omega_j\sim\probability_{\dual{\Haar}, \rho}$ \iid~and $B(\omega_j)\in\mathcal{L}\left(\mathbb{R}^u,\mathbb{R}^{u'}\right)$ for all $\omega_j\in\dual{\mathcal{X}}$. Hence the \acl{ORFF} can be seen as the block matrix
\begin{dmath}
\label{eq:matrix_orff}
\tildePhi{\omega}(x) = \begin{pmatrix} \cos\inner{x,\omega_1}B(\omega_1)^T \\
\sin\inner{x,\omega_1}B(\omega_1)^T \\
\vdots \\
\cos\inner{x,\omega_D}B(\omega_D)^T \\
\sin\inner{x,\omega_D}B(\omega_D)^T
\end{pmatrix}\hiderel{\in}\mathcal{L}\left(\mathbb{R}^u, \mathbb{R}^{2Du'}\right),
\end{dmath}
$\omega_j\sim\probability_{\dual{\Haar}, \rho}$ \iid.
\subsection{Decomposable kernel}
Following \cref{eq:matrix_orff} the feature map associated to the decomposable kernel would be
\begin{dmath*}
\tildePhi{\omega}(x) = \begin{pmatrix} \cos\inner{x,\omega_1}B^T \\
\sin\inner{x,\omega_1}B^T \\
\vdots \\
\cos\inner{x,\omega_D}B^T \\
\sin\inner{x,\omega_D}B^T
\end{pmatrix}\hiderel{\in}\mathcal{L}\left(\mathbb{R}^u, \mathbb{R}^{2Du'}\right),
\end{dmath*}
$\omega_j\sim\probability_{\dual{\Haar}, \rho}$ \iid. Thus evaluating a matrix vector product such as $\tildePhi{\omega}(x)^T\theta$ or $\tildePhi{\omega}(x)y$ have $O_t((2Du')^2u)$ time complexity, which is utterly inefficient\ldots~Indeed, recall that if $B\in\mathcal{L}\left(\mathbb{R}^{u'}, \mathbb{R}^u\right)$ is matrix, the operator $\tildePhi{\omega}(x)$ corresponding to the decomposable kernel is
\begin{dgroup*}
\begin{dmath*}
\tildePhi{\omega}(x)y \hiderel{=} \frac{1}{\sqrt{D}}\Vect_{j=1}^D\pairing{x, \omega_j}B^T y \hiderel{=} \left(\frac{1}{\sqrt{D}}\Vect_{j=1}^D\pairing{x, \omega_j}\right)B^T y
\end{dmath*}
\begin{dmath*}
\tildePhi{\omega}(x)^T\theta \hiderel{=} \frac{1}{\sqrt{D}} \sum_{j=1}^D \pairing{x, \omega_j}B\theta_j \hiderel{=} B\left(\frac{1}{\sqrt{D}} \sum_{j=1}^D \pairing{x, \omega_j}\theta_j\right)
\end{dmath*}
\end{dgroup*}
\subsection{Curl-free kernel}
\subsection{Divergence-free kernel}

\clearpage
%----------------------------------------------------------------------------------------
\section{Conclusions}
\label{sec:conclusions}

\chapterend
